\ProblemFileHeader{XTL_SV_QUESTIONCOUNT}% Process how many problems are in this file and how to detect if it has a desirable problem
\ifproblemToFind% If it has a desirable problem search the file.
%%\tagged{Ans@ShortAns, Type@Compute, Topic@TestBank, Sub@Rational, Sub@VerticalAsymptotes, Sub@HorizontalAsymptotes, File@0048}{
\latexProblemContent{
\ifVerboseLocation This is Precalc Compute Question 0048. \\ \fi
\begin{problem}

Consider the rational function $f(x) = \sage{f}$.  Identify any vertical asymptotes.

\input{Precalc-Compute-0048.HELP.tex}

\begin{feedback}[attempt]
Remember that vertical asymptotes come from the denominator being $0$.
\end{feedback}

%\begin{feedback}[VA=$\sage{b}$]
%That factor canceled out, which would lead to a hole.
%\end{feedback}

\begin{feedback}[correct]
Remember that vertical asymptotes are barriers that the graph cannot cross.
\end{feedback}

\begin{problem}

Are there any holes?  If so, give the coordinates of the hole.  If not, enter ``NONE''.

\begin{feedback}[attempt]
Remember that holes occur from cancellation of factors in the numerator and the denominator.
\end{feedback}

%\begin{feedback}[VA=$\sage{c}$]
%That factor did not cancel out, which would lead to a vertical asymptote.
%\end{feedback}

\begin{feedback}[correct]
Remember that holes are removable discontinuities in the graph.
\end{feedback}

\begin{problem}

Identify any horizontal asymptotes.  If none, enter ``NONE''.

\[y=\answer[format=integer,id=HA]{0}\]

\begin{feedback}[attempt]
Remember that horizontal asymptotes you need to compare the degree of the numerator and the denominator.
\end{feedback}

\begin{feedback}[HA>1]
This would happen if the degree of the numerator was higher.
\end{feedback}

\begin{feedback}[HA=1]
This would happen if the degree of the numerator equaled the degree of the denominator.
\end{feedback}

\begin{feedback}[correct]
The horizontal asymptote is describing the long-run behavior of the function.
\end{feedback}


\end{problem}

\[(\answer[format=string]{NONE}, \answer[format=string]{NONE})\]

\end{problem}

  \begin{validator}[a^2+b^2==$\sage{square}$]
\[x=\answer[format=integer,id=a]{\sage{c}}\qquad\qquad x=\answer[format=integer,id=b]{\sage{b}}\]
  \end{validator}

\[\sage{c}  \qquad\qquad  \sage{b}\]

\end{problem}}%}

