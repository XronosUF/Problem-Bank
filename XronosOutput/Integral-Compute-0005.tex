\ProblemFileHeader{XTL_SV_QUESTIONCOUNT}% Process how many problems are in this file and how to detect if it has a desirable problem
\ifproblemToFind% If it has a desirable problem search the file.
%%\tagged{Ans@ShortAns, Type@Compute, Topic@Integral, Sub@Definite, \Func@Piecewise, File@0005}{
\latexProblemContent{
\ifVerboseLocation This is Integration Compute Question 0005. \\ \fi
\begin{problem}

Given the piecewise function 
\[f(x)=\left\{\begin{array}{ll}
{2 \, x + 5}\; , & x\leq{-2}\\[3pt]
{-\frac{2}{3} \, x - \frac{1}{3}}\; , & {-2}< x< {1}\\[3pt]
{4 \, x - 5}\; , & x\geq{1}
\end{array} \right.\]
evaluate the following definite integral by interpreting it in terms of areas.  

\input{Integral-Compute-0005.HELP.tex}

\[
\int_{-7}^{0} f(x)\;dx= \answer{-\frac{58}{3}}
\]  
\end{problem}}%}

\latexProblemContent{
\ifVerboseLocation This is Integration Compute Question 0005. \\ \fi
\begin{problem}

Given the piecewise function 
\[f(x)=\left\{\begin{array}{ll}
{4 \, x + 1}\; , & x\leq{3}\\[3pt]
{-\frac{13}{2} \, x + \frac{65}{2}}\; , & {3}< x< {7}\\[3pt]
{x - 20}\; , & x\geq{7}
\end{array} \right.\]
evaluate the following definite integral by interpreting it in terms of areas.  

\input{Integral-Compute-0005.HELP.tex}

\[
\int_{0}^{5} f(x)\;dx= \answer{34}
\]  
\end{problem}}%}

\latexProblemContent{
\ifVerboseLocation This is Integration Compute Question 0005. \\ \fi
\begin{problem}

Given the piecewise function 
\[f(x)=\left\{\begin{array}{ll}
{2 \, x + 4}\; , & x\leq{1}\\[3pt]
{-2 \, x + 8}\; , & {1}< x< {7}\\[3pt]
{x - 13}\; , & x\geq{7}
\end{array} \right.\]
evaluate the following definite integral by interpreting it in terms of areas.  

\input{Integral-Compute-0005.HELP.tex}

\[
\int_{1}^{5} f(x)\;dx= \answer{8}
\]  
\end{problem}}%}

\latexProblemContent{
\ifVerboseLocation This is Integration Compute Question 0005. \\ \fi
\begin{problem}

Given the piecewise function 
\[f(x)=\left\{\begin{array}{ll}
{3 \, x + 3}\; , & x\leq{-2}\\[3pt]
{2 \, x + 1}\; , & {-2}< x< {1}\\[3pt]
{4 \, x - 1}\; , & x\geq{1}
\end{array} \right.\]
evaluate the following definite integral by interpreting it in terms of areas.  

\input{Integral-Compute-0005.HELP.tex}

\[
\int_{-3}^{3} f(x)\;dx= \answer{\frac{19}{2}}
\]  
\end{problem}}%}

\latexProblemContent{
\ifVerboseLocation This is Integration Compute Question 0005. \\ \fi
\begin{problem}

Given the piecewise function 
\[f(x)=\left\{\begin{array}{ll}
{x + 3}\; , & x\leq{2}\\[3pt]
{-\frac{5}{3} \, x + \frac{25}{3}}\; , & {2}< x< {8}\\[3pt]
{x - 13}\; , & x\geq{8}
\end{array} \right.\]
evaluate the following definite integral by interpreting it in terms of areas.  

\input{Integral-Compute-0005.HELP.tex}

\[
\int_{-1}^{10} f(x)\;dx= \answer{\frac{5}{2}}
\]  
\end{problem}}%}

\latexProblemContent{
\ifVerboseLocation This is Integration Compute Question 0005. \\ \fi
\begin{problem}

Given the piecewise function 
\[f(x)=\left\{\begin{array}{ll}
{3 \, x + 2}\; , & x\leq{3}\\[3pt]
{-\frac{22}{3} \, x + 33}\; , & {3}< x< {6}\\[3pt]
{2 \, x - 23}\; , & x\geq{6}
\end{array} \right.\]
evaluate the following definite integral by interpreting it in terms of areas.  

\input{Integral-Compute-0005.HELP.tex}

\[
\int_{-2}^{6} f(x)\;dx= \answer{\frac{35}{2}}
\]  
\end{problem}}%}

\latexProblemContent{
\ifVerboseLocation This is Integration Compute Question 0005. \\ \fi
\begin{problem}

Given the piecewise function 
\[f(x)=\left\{\begin{array}{ll}
{3 \, x + 4}\; , & x\leq{4}\\[3pt]
{-4 \, x + 32}\; , & {4}< x< {12}\\[3pt]
{x - 28}\; , & x\geq{12}
\end{array} \right.\]
evaluate the following definite integral by interpreting it in terms of areas.  

\input{Integral-Compute-0005.HELP.tex}

\[
\int_{0}^{5} f(x)\;dx= \answer{54}
\]  
\end{problem}}%}

\latexProblemContent{
\ifVerboseLocation This is Integration Compute Question 0005. \\ \fi
\begin{problem}

Given the piecewise function 
\[f(x)=\left\{\begin{array}{ll}
{4 \, x + 3}\; , & x\leq{-4}\\[3pt]
{\frac{13}{4} \, x}\; , & {-4}< x< {4}\\[3pt]
{2 \, x + 5}\; , & x\geq{4}
\end{array} \right.\]
evaluate the following definite integral by interpreting it in terms of areas.  

\input{Integral-Compute-0005.HELP.tex}

\[
\int_{-6}^{4} f(x)\;dx= \answer{-34}
\]  
\end{problem}}%}

\latexProblemContent{
\ifVerboseLocation This is Integration Compute Question 0005. \\ \fi
\begin{problem}

Given the piecewise function 
\[f(x)=\left\{\begin{array}{ll}
{3 \, x + 2}\; , & x\leq{4}\\[3pt]
{-7 \, x + 42}\; , & {4}< x< {8}\\[3pt]
{x - 22}\; , & x\geq{8}
\end{array} \right.\]
evaluate the following definite integral by interpreting it in terms of areas.  

\input{Integral-Compute-0005.HELP.tex}

\[
\int_{2}^{7} f(x)\;dx= \answer{\frac{65}{2}}
\]  
\end{problem}}%}

\latexProblemContent{
\ifVerboseLocation This is Integration Compute Question 0005. \\ \fi
\begin{problem}

Given the piecewise function 
\[f(x)=\left\{\begin{array}{ll}
{x + 3}\; , & x\leq{-5}\\[3pt]
{\frac{1}{2} \, x + \frac{1}{2}}\; , & {-5}< x< {3}\\[3pt]
{2 \, x - 4}\; , & x\geq{3}
\end{array} \right.\]
evaluate the following definite integral by interpreting it in terms of areas.  

\input{Integral-Compute-0005.HELP.tex}

\[
\int_{-10}^{-2} f(x)\;dx= \answer{-\frac{105}{4}}
\]  
\end{problem}}%}

\latexProblemContent{
\ifVerboseLocation This is Integration Compute Question 0005. \\ \fi
\begin{problem}

Given the piecewise function 
\[f(x)=\left\{\begin{array}{ll}
{2 \, x + 3}\; , & x\leq{-1}\\[3pt]
{-\frac{2}{3} \, x + \frac{1}{3}}\; , & {-1}< x< {2}\\[3pt]
{x - 3}\; , & x\geq{2}
\end{array} \right.\]
evaluate the following definite integral by interpreting it in terms of areas.  

\input{Integral-Compute-0005.HELP.tex}

\[
\int_{-4}^{6} f(x)\;dx= \answer{-2}
\]  
\end{problem}}%}

\latexProblemContent{
\ifVerboseLocation This is Integration Compute Question 0005. \\ \fi
\begin{problem}

Given the piecewise function 
\[f(x)=\left\{\begin{array}{ll}
{4 \, x + 5}\; , & x\leq{3}\\[3pt]
{-\frac{34}{3} \, x + 51}\; , & {3}< x< {6}\\[3pt]
{3 \, x - 35}\; , & x\geq{6}
\end{array} \right.\]
evaluate the following definite integral by interpreting it in terms of areas.  

\input{Integral-Compute-0005.HELP.tex}

\[
\int_{0}^{10} f(x)\;dx= \answer{-11}
\]  
\end{problem}}%}

\latexProblemContent{
\ifVerboseLocation This is Integration Compute Question 0005. \\ \fi
\begin{problem}

Given the piecewise function 
\[f(x)=\left\{\begin{array}{ll}
{2 \, x + 2}\; , & x\leq{2}\\[3pt]
{-\frac{12}{5} \, x + \frac{54}{5}}\; , & {2}< x< {7}\\[3pt]
{2 \, x - 20}\; , & x\geq{7}
\end{array} \right.\]
evaluate the following definite integral by interpreting it in terms of areas.  

\input{Integral-Compute-0005.HELP.tex}

\[
\int_{2}^{4} f(x)\;dx= \answer{\frac{36}{5}}
\]  
\end{problem}}%}

\latexProblemContent{
\ifVerboseLocation This is Integration Compute Question 0005. \\ \fi
\begin{problem}

Given the piecewise function 
\[f(x)=\left\{\begin{array}{ll}
{x + 5}\; , & x\leq{3}\\[3pt]
{-\frac{16}{5} \, x + \frac{88}{5}}\; , & {3}< x< {8}\\[3pt]
{4 \, x - 40}\; , & x\geq{8}
\end{array} \right.\]
evaluate the following definite integral by interpreting it in terms of areas.  

\input{Integral-Compute-0005.HELP.tex}

\[
\int_{-1}^{10} f(x)\;dx= \answer{16}
\]  
\end{problem}}%}

\latexProblemContent{
\ifVerboseLocation This is Integration Compute Question 0005. \\ \fi
\begin{problem}

Given the piecewise function 
\[f(x)=\left\{\begin{array}{ll}
{4 \, x + 1}\; , & x\leq{-2}\\[3pt]
{\frac{7}{2} \, x}\; , & {-2}< x< {2}\\[3pt]
{x + 5}\; , & x\geq{2}
\end{array} \right.\]
evaluate the following definite integral by interpreting it in terms of areas.  

\input{Integral-Compute-0005.HELP.tex}

\[
\int_{-4}^{2} f(x)\;dx= \answer{-22}
\]  
\end{problem}}%}

\latexProblemContent{
\ifVerboseLocation This is Integration Compute Question 0005. \\ \fi
\begin{problem}

Given the piecewise function 
\[f(x)=\left\{\begin{array}{ll}
{2 \, x + 3}\; , & x\leq{5}\\[3pt]
{-\frac{26}{7} \, x + \frac{221}{7}}\; , & {5}< x< {12}\\[3pt]
{2 \, x - 37}\; , & x\geq{12}
\end{array} \right.\]
evaluate the following definite integral by interpreting it in terms of areas.  

\input{Integral-Compute-0005.HELP.tex}

\[
\int_{4}^{14} f(x)\;dx= \answer{-10}
\]  
\end{problem}}%}

\latexProblemContent{
\ifVerboseLocation This is Integration Compute Question 0005. \\ \fi
\begin{problem}

Given the piecewise function 
\[f(x)=\left\{\begin{array}{ll}
{x + 1}\; , & x\leq{-5}\\[3pt]
{\frac{8}{3} \, x + \frac{28}{3}}\; , & {-5}< x< {-2}\\[3pt]
{2 \, x + 8}\; , & x\geq{-2}
\end{array} \right.\]
evaluate the following definite integral by interpreting it in terms of areas.  

\input{Integral-Compute-0005.HELP.tex}

\[
\int_{-5}^{-4} f(x)\;dx= \answer{-\frac{8}{3}}
\]  
\end{problem}}%}

\latexProblemContent{
\ifVerboseLocation This is Integration Compute Question 0005. \\ \fi
\begin{problem}

Given the piecewise function 
\[f(x)=\left\{\begin{array}{ll}
{4 \, x + 2}\; , & x\leq{1}\\[3pt]
{-3 \, x + 9}\; , & {1}< x< {5}\\[3pt]
{2 \, x - 16}\; , & x\geq{5}
\end{array} \right.\]
evaluate the following definite integral by interpreting it in terms of areas.  

\input{Integral-Compute-0005.HELP.tex}

\[
\int_{-1}^{2} f(x)\;dx= \answer{\frac{17}{2}}
\]  
\end{problem}}%}

\latexProblemContent{
\ifVerboseLocation This is Integration Compute Question 0005. \\ \fi
\begin{problem}

Given the piecewise function 
\[f(x)=\left\{\begin{array}{ll}
{2 \, x + 2}\; , & x\leq{-3}\\[3pt]
{x - 1}\; , & {-3}< x< {5}\\[3pt]
{2 \, x - 6}\; , & x\geq{5}
\end{array} \right.\]
evaluate the following definite integral by interpreting it in terms of areas.  

\input{Integral-Compute-0005.HELP.tex}

\[
\int_{-8}^{4} f(x)\;dx= \answer{-\frac{97}{2}}
\]  
\end{problem}}%}

\latexProblemContent{
\ifVerboseLocation This is Integration Compute Question 0005. \\ \fi
\begin{problem}

Given the piecewise function 
\[f(x)=\left\{\begin{array}{ll}
{3 \, x + 1}\; , & x\leq{2}\\[3pt]
{-\frac{7}{2} \, x + 14}\; , & {2}< x< {6}\\[3pt]
{2 \, x - 19}\; , & x\geq{6}
\end{array} \right.\]
evaluate the following definite integral by interpreting it in terms of areas.  

\input{Integral-Compute-0005.HELP.tex}

\[
\int_{1}^{8} f(x)\;dx= \answer{-\frac{9}{2}}
\]  
\end{problem}}%}

\latexProblemContent{
\ifVerboseLocation This is Integration Compute Question 0005. \\ \fi
\begin{problem}

Given the piecewise function 
\[f(x)=\left\{\begin{array}{ll}
{3 \, x + 3}\; , & x\leq{-4}\\[3pt]
{\frac{18}{7} \, x + \frac{9}{7}}\; , & {-4}< x< {3}\\[3pt]
{3 \, x}\; , & x\geq{3}
\end{array} \right.\]
evaluate the following definite integral by interpreting it in terms of areas.  

\input{Integral-Compute-0005.HELP.tex}

\[
\int_{-5}^{7} f(x)\;dx= \answer{\frac{99}{2}}
\]  
\end{problem}}%}

\latexProblemContent{
\ifVerboseLocation This is Integration Compute Question 0005. \\ \fi
\begin{problem}

Given the piecewise function 
\[f(x)=\left\{\begin{array}{ll}
{2 \, x + 2}\; , & x\leq{1}\\[3pt]
{-2 \, x + 6}\; , & {1}< x< {5}\\[3pt]
{3 \, x - 19}\; , & x\geq{5}
\end{array} \right.\]
evaluate the following definite integral by interpreting it in terms of areas.  

\input{Integral-Compute-0005.HELP.tex}

\[
\int_{-3}^{7} f(x)\;dx= \answer{-2}
\]  
\end{problem}}%}

\latexProblemContent{
\ifVerboseLocation This is Integration Compute Question 0005. \\ \fi
\begin{problem}

Given the piecewise function 
\[f(x)=\left\{\begin{array}{ll}
{x + 4}\; , & x\leq{5}\\[3pt]
{-\frac{9}{4} \, x + \frac{81}{4}}\; , & {5}< x< {13}\\[3pt]
{3 \, x - 48}\; , & x\geq{13}
\end{array} \right.\]
evaluate the following definite integral by interpreting it in terms of areas.  

\input{Integral-Compute-0005.HELP.tex}

\[
\int_{1}^{11} f(x)\;dx= \answer{\frac{83}{2}}
\]  
\end{problem}}%}

\latexProblemContent{
\ifVerboseLocation This is Integration Compute Question 0005. \\ \fi
\begin{problem}

Given the piecewise function 
\[f(x)=\left\{\begin{array}{ll}
{4 \, x + 2}\; , & x\leq{-2}\\[3pt]
{3 \, x}\; , & {-2}< x< {2}\\[3pt]
{3 \, x}\; , & x\geq{2}
\end{array} \right.\]
evaluate the following definite integral by interpreting it in terms of areas.  

\input{Integral-Compute-0005.HELP.tex}

\[
\int_{-3}^{2} f(x)\;dx= \answer{-8}
\]  
\end{problem}}%}

\latexProblemContent{
\ifVerboseLocation This is Integration Compute Question 0005. \\ \fi
\begin{problem}

Given the piecewise function 
\[f(x)=\left\{\begin{array}{ll}
{3 \, x + 2}\; , & x\leq{1}\\[3pt]
{-\frac{5}{4} \, x + \frac{25}{4}}\; , & {1}< x< {9}\\[3pt]
{x - 14}\; , & x\geq{9}
\end{array} \right.\]
evaluate the following definite integral by interpreting it in terms of areas.  

\input{Integral-Compute-0005.HELP.tex}

\[
\int_{-1}^{11} f(x)\;dx= \answer{-4}
\]  
\end{problem}}%}

\latexProblemContent{
\ifVerboseLocation This is Integration Compute Question 0005. \\ \fi
\begin{problem}

Given the piecewise function 
\[f(x)=\left\{\begin{array}{ll}
{3 \, x + 1}\; , & x\leq{-1}\\[3pt]
{\frac{1}{2} \, x - \frac{3}{2}}\; , & {-1}< x< {7}\\[3pt]
{x - 5}\; , & x\geq{7}
\end{array} \right.\]
evaluate the following definite integral by interpreting it in terms of areas.  

\input{Integral-Compute-0005.HELP.tex}

\[
\int_{-1}^{7} f(x)\;dx= \answer{0}
\]  
\end{problem}}%}

\latexProblemContent{
\ifVerboseLocation This is Integration Compute Question 0005. \\ \fi
\begin{problem}

Given the piecewise function 
\[f(x)=\left\{\begin{array}{ll}
{2 \, x + 1}\; , & x\leq{4}\\[3pt]
{-\frac{9}{2} \, x + 27}\; , & {4}< x< {8}\\[3pt]
{4 \, x - 41}\; , & x\geq{8}
\end{array} \right.\]
evaluate the following definite integral by interpreting it in terms of areas.  

\input{Integral-Compute-0005.HELP.tex}

\[
\int_{-1}^{8} f(x)\;dx= \answer{20}
\]  
\end{problem}}%}

\latexProblemContent{
\ifVerboseLocation This is Integration Compute Question 0005. \\ \fi
\begin{problem}

Given the piecewise function 
\[f(x)=\left\{\begin{array}{ll}
{4 \, x + 5}\; , & x\leq{5}\\[3pt]
{-\frac{25}{2} \, x + \frac{175}{2}}\; , & {5}< x< {9}\\[3pt]
{x - 34}\; , & x\geq{9}
\end{array} \right.\]
evaluate the following definite integral by interpreting it in terms of areas.  

\input{Integral-Compute-0005.HELP.tex}

\[
\int_{0}^{7} f(x)\;dx= \answer{100}
\]  
\end{problem}}%}

\latexProblemContent{
\ifVerboseLocation This is Integration Compute Question 0005. \\ \fi
\begin{problem}

Given the piecewise function 
\[f(x)=\left\{\begin{array}{ll}
{4 \, x + 4}\; , & x\leq{1}\\[3pt]
{-\frac{8}{3} \, x + \frac{32}{3}}\; , & {1}< x< {7}\\[3pt]
{3 \, x - 29}\; , & x\geq{7}
\end{array} \right.\]
evaluate the following definite integral by interpreting it in terms of areas.  

\input{Integral-Compute-0005.HELP.tex}

\[
\int_{-2}^{10} f(x)\;dx= \answer{-\frac{9}{2}}
\]  
\end{problem}}%}

\latexProblemContent{
\ifVerboseLocation This is Integration Compute Question 0005. \\ \fi
\begin{problem}

Given the piecewise function 
\[f(x)=\left\{\begin{array}{ll}
{x + 1}\; , & x\leq{1}\\[3pt]
{-\frac{2}{3} \, x + \frac{8}{3}}\; , & {1}< x< {7}\\[3pt]
{2 \, x - 16}\; , & x\geq{7}
\end{array} \right.\]
evaluate the following definite integral by interpreting it in terms of areas.  

\input{Integral-Compute-0005.HELP.tex}

\[
\int_{-1}^{9} f(x)\;dx= \answer{2}
\]  
\end{problem}}%}

\latexProblemContent{
\ifVerboseLocation This is Integration Compute Question 0005. \\ \fi
\begin{problem}

Given the piecewise function 
\[f(x)=\left\{\begin{array}{ll}
{4 \, x + 4}\; , & x\leq{4}\\[3pt]
{-\frac{40}{7} \, x + \frac{300}{7}}\; , & {4}< x< {11}\\[3pt]
{4 \, x - 64}\; , & x\geq{11}
\end{array} \right.\]
evaluate the following definite integral by interpreting it in terms of areas.  

\input{Integral-Compute-0005.HELP.tex}

\[
\int_{1}^{15} f(x)\;dx= \answer{-6}
\]  
\end{problem}}%}

\latexProblemContent{
\ifVerboseLocation This is Integration Compute Question 0005. \\ \fi
\begin{problem}

Given the piecewise function 
\[f(x)=\left\{\begin{array}{ll}
{x + 1}\; , & x\leq{1}\\[3pt]
{-x + 3}\; , & {1}< x< {5}\\[3pt]
{4 \, x - 22}\; , & x\geq{5}
\end{array} \right.\]
evaluate the following definite integral by interpreting it in terms of areas.  

\input{Integral-Compute-0005.HELP.tex}

\[
\int_{0}^{5} f(x)\;dx= \answer{\frac{3}{2}}
\]  
\end{problem}}%}

\latexProblemContent{
\ifVerboseLocation This is Integration Compute Question 0005. \\ \fi
\begin{problem}

Given the piecewise function 
\[f(x)=\left\{\begin{array}{ll}
{x + 2}\; , & x\leq{-4}\\[3pt]
{\frac{4}{5} \, x + \frac{6}{5}}\; , & {-4}< x< {1}\\[3pt]
{3 \, x - 1}\; , & x\geq{1}
\end{array} \right.\]
evaluate the following definite integral by interpreting it in terms of areas.  

\input{Integral-Compute-0005.HELP.tex}

\[
\int_{-4}^{1} f(x)\;dx= \answer{0}
\]  
\end{problem}}%}

\latexProblemContent{
\ifVerboseLocation This is Integration Compute Question 0005. \\ \fi
\begin{problem}

Given the piecewise function 
\[f(x)=\left\{\begin{array}{ll}
{x + 5}\; , & x\leq{-3}\\[3pt]
{-x - 1}\; , & {-3}< x< {1}\\[3pt]
{3 \, x - 5}\; , & x\geq{1}
\end{array} \right.\]
evaluate the following definite integral by interpreting it in terms of areas.  

\input{Integral-Compute-0005.HELP.tex}

\[
\int_{-4}^{-2} f(x)\;dx= \answer{3}
\]  
\end{problem}}%}

\latexProblemContent{
\ifVerboseLocation This is Integration Compute Question 0005. \\ \fi
\begin{problem}

Given the piecewise function 
\[f(x)=\left\{\begin{array}{ll}
{4 \, x + 4}\; , & x\leq{-5}\\[3pt]
{8 \, x + 24}\; , & {-5}< x< {-1}\\[3pt]
{x + 17}\; , & x\geq{-1}
\end{array} \right.\]
evaluate the following definite integral by interpreting it in terms of areas.  

\input{Integral-Compute-0005.HELP.tex}

\[
\int_{-10}^{-2} f(x)\;dx= \answer{-142}
\]  
\end{problem}}%}

\latexProblemContent{
\ifVerboseLocation This is Integration Compute Question 0005. \\ \fi
\begin{problem}

Given the piecewise function 
\[f(x)=\left\{\begin{array}{ll}
{2 \, x + 3}\; , & x\leq{2}\\[3pt]
{-\frac{14}{3} \, x + \frac{49}{3}}\; , & {2}< x< {5}\\[3pt]
{2 \, x - 17}\; , & x\geq{5}
\end{array} \right.\]
evaluate the following definite integral by interpreting it in terms of areas.  

\input{Integral-Compute-0005.HELP.tex}

\[
\int_{2}^{6} f(x)\;dx= \answer{-6}
\]  
\end{problem}}%}

\latexProblemContent{
\ifVerboseLocation This is Integration Compute Question 0005. \\ \fi
\begin{problem}

Given the piecewise function 
\[f(x)=\left\{\begin{array}{ll}
{x + 5}\; , & x\leq{-2}\\[3pt]
{-\frac{6}{5} \, x + \frac{3}{5}}\; , & {-2}< x< {3}\\[3pt]
{4 \, x - 15}\; , & x\geq{3}
\end{array} \right.\]
evaluate the following definite integral by interpreting it in terms of areas.  

\input{Integral-Compute-0005.HELP.tex}

\[
\int_{-3}^{1} f(x)\;dx= \answer{\frac{61}{10}}
\]  
\end{problem}}%}

\latexProblemContent{
\ifVerboseLocation This is Integration Compute Question 0005. \\ \fi
\begin{problem}

Given the piecewise function 
\[f(x)=\left\{\begin{array}{ll}
{2 \, x + 1}\; , & x\leq{2}\\[3pt]
{-\frac{5}{4} \, x + \frac{15}{2}}\; , & {2}< x< {10}\\[3pt]
{2 \, x - 25}\; , & x\geq{10}
\end{array} \right.\]
evaluate the following definite integral by interpreting it in terms of areas.  

\input{Integral-Compute-0005.HELP.tex}

\[
\int_{-3}^{7} f(x)\;dx= \answer{\frac{75}{8}}
\]  
\end{problem}}%}

\latexProblemContent{
\ifVerboseLocation This is Integration Compute Question 0005. \\ \fi
\begin{problem}

Given the piecewise function 
\[f(x)=\left\{\begin{array}{ll}
{2 \, x + 4}\; , & x\leq{5}\\[3pt]
{-\frac{7}{2} \, x + \frac{63}{2}}\; , & {5}< x< {13}\\[3pt]
{x - 27}\; , & x\geq{13}
\end{array} \right.\]
evaluate the following definite integral by interpreting it in terms of areas.  

\input{Integral-Compute-0005.HELP.tex}

\[
\int_{1}^{12} f(x)\;dx= \answer{\frac{209}{4}}
\]  
\end{problem}}%}

\latexProblemContent{
\ifVerboseLocation This is Integration Compute Question 0005. \\ \fi
\begin{problem}

Given the piecewise function 
\[f(x)=\left\{\begin{array}{ll}
{4 \, x + 2}\; , & x\leq{3}\\[3pt]
{-\frac{28}{5} \, x + \frac{154}{5}}\; , & {3}< x< {8}\\[3pt]
{4 \, x - 46}\; , & x\geq{8}
\end{array} \right.\]
evaluate the following definite integral by interpreting it in terms of areas.  

\input{Integral-Compute-0005.HELP.tex}

\[
\int_{3}^{11} f(x)\;dx= \answer{-24}
\]  
\end{problem}}%}

\latexProblemContent{
\ifVerboseLocation This is Integration Compute Question 0005. \\ \fi
\begin{problem}

Given the piecewise function 
\[f(x)=\left\{\begin{array}{ll}
{x + 2}\; , & x\leq{2}\\[3pt]
{-2 \, x + 8}\; , & {2}< x< {6}\\[3pt]
{3 \, x - 22}\; , & x\geq{6}
\end{array} \right.\]
evaluate the following definite integral by interpreting it in terms of areas.  

\input{Integral-Compute-0005.HELP.tex}

\[
\int_{1}^{9} f(x)\;dx= \answer{5}
\]  
\end{problem}}%}

\latexProblemContent{
\ifVerboseLocation This is Integration Compute Question 0005. \\ \fi
\begin{problem}

Given the piecewise function 
\[f(x)=\left\{\begin{array}{ll}
{4 \, x + 2}\; , & x\leq{1}\\[3pt]
{-3 \, x + 9}\; , & {1}< x< {5}\\[3pt]
{4 \, x - 26}\; , & x\geq{5}
\end{array} \right.\]
evaluate the following definite integral by interpreting it in terms of areas.  

\input{Integral-Compute-0005.HELP.tex}

\[
\int_{-2}^{4} f(x)\;dx= \answer{\frac{9}{2}}
\]  
\end{problem}}%}

\latexProblemContent{
\ifVerboseLocation This is Integration Compute Question 0005. \\ \fi
\begin{problem}

Given the piecewise function 
\[f(x)=\left\{\begin{array}{ll}
{x + 4}\; , & x\leq{-5}\\[3pt]
{\frac{1}{3} \, x + \frac{2}{3}}\; , & {-5}< x< {1}\\[3pt]
{2 \, x - 1}\; , & x\geq{1}
\end{array} \right.\]
evaluate the following definite integral by interpreting it in terms of areas.  

\input{Integral-Compute-0005.HELP.tex}

\[
\int_{-7}^{3} f(x)\;dx= \answer{2}
\]  
\end{problem}}%}

\latexProblemContent{
\ifVerboseLocation This is Integration Compute Question 0005. \\ \fi
\begin{problem}

Given the piecewise function 
\[f(x)=\left\{\begin{array}{ll}
{x + 3}\; , & x\leq{3}\\[3pt]
{-3 \, x + 15}\; , & {3}< x< {7}\\[3pt]
{x - 13}\; , & x\geq{7}
\end{array} \right.\]
evaluate the following definite integral by interpreting it in terms of areas.  

\input{Integral-Compute-0005.HELP.tex}

\[
\int_{3}^{6} f(x)\;dx= \answer{\frac{9}{2}}
\]  
\end{problem}}%}

\latexProblemContent{
\ifVerboseLocation This is Integration Compute Question 0005. \\ \fi
\begin{problem}

Given the piecewise function 
\[f(x)=\left\{\begin{array}{ll}
{3 \, x + 5}\; , & x\leq{-5}\\[3pt]
{\frac{5}{2} \, x + \frac{5}{2}}\; , & {-5}< x< {3}\\[3pt]
{x + 7}\; , & x\geq{3}
\end{array} \right.\]
evaluate the following definite integral by interpreting it in terms of areas.  

\input{Integral-Compute-0005.HELP.tex}

\[
\int_{-10}^{8} f(x)\;dx= \answer{-25}
\]  
\end{problem}}%}

\latexProblemContent{
\ifVerboseLocation This is Integration Compute Question 0005. \\ \fi
\begin{problem}

Given the piecewise function 
\[f(x)=\left\{\begin{array}{ll}
{2 \, x + 4}\; , & x\leq{3}\\[3pt]
{-4 \, x + 22}\; , & {3}< x< {8}\\[3pt]
{3 \, x - 34}\; , & x\geq{8}
\end{array} \right.\]
evaluate the following definite integral by interpreting it in terms of areas.  

\input{Integral-Compute-0005.HELP.tex}

\[
\int_{1}^{11} f(x)\;dx= \answer{-\frac{1}{2}}
\]  
\end{problem}}%}

\latexProblemContent{
\ifVerboseLocation This is Integration Compute Question 0005. \\ \fi
\begin{problem}

Given the piecewise function 
\[f(x)=\left\{\begin{array}{ll}
{x + 2}\; , & x\leq{4}\\[3pt]
{-3 \, x + 18}\; , & {4}< x< {8}\\[3pt]
{2 \, x - 22}\; , & x\geq{8}
\end{array} \right.\]
evaluate the following definite integral by interpreting it in terms of areas.  

\input{Integral-Compute-0005.HELP.tex}

\[
\int_{-1}^{9} f(x)\;dx= \answer{\frac{25}{2}}
\]  
\end{problem}}%}

\latexProblemContent{
\ifVerboseLocation This is Integration Compute Question 0005. \\ \fi
\begin{problem}

Given the piecewise function 
\[f(x)=\left\{\begin{array}{ll}
{4 \, x + 1}\; , & x\leq{4}\\[3pt]
{-\frac{34}{3} \, x + \frac{187}{3}}\; , & {4}< x< {7}\\[3pt]
{3 \, x - 38}\; , & x\geq{7}
\end{array} \right.\]
evaluate the following definite integral by interpreting it in terms of areas.  

\input{Integral-Compute-0005.HELP.tex}

\[
\int_{2}^{6} f(x)\;dx= \answer{\frac{112}{3}}
\]  
\end{problem}}%}

\latexProblemContent{
\ifVerboseLocation This is Integration Compute Question 0005. \\ \fi
\begin{problem}

Given the piecewise function 
\[f(x)=\left\{\begin{array}{ll}
{3 \, x + 2}\; , & x\leq{-2}\\[3pt]
{\frac{4}{3} \, x - \frac{4}{3}}\; , & {-2}< x< {4}\\[3pt]
{4 \, x - 12}\; , & x\geq{4}
\end{array} \right.\]
evaluate the following definite integral by interpreting it in terms of areas.  

\input{Integral-Compute-0005.HELP.tex}

\[
\int_{-7}^{8} f(x)\;dx= \answer{-\frac{19}{2}}
\]  
\end{problem}}%}

\latexProblemContent{
\ifVerboseLocation This is Integration Compute Question 0005. \\ \fi
\begin{problem}

Given the piecewise function 
\[f(x)=\left\{\begin{array}{ll}
{2 \, x + 4}\; , & x\leq{3}\\[3pt]
{-4 \, x + 22}\; , & {3}< x< {8}\\[3pt]
{2 \, x - 26}\; , & x\geq{8}
\end{array} \right.\]
evaluate the following definite integral by interpreting it in terms of areas.  

\input{Integral-Compute-0005.HELP.tex}

\[
\int_{0}^{11} f(x)\;dx= \answer{0}
\]  
\end{problem}}%}

\latexProblemContent{
\ifVerboseLocation This is Integration Compute Question 0005. \\ \fi
\begin{problem}

Given the piecewise function 
\[f(x)=\left\{\begin{array}{ll}
{3 \, x + 5}\; , & x\leq{3}\\[3pt]
{-\frac{28}{5} \, x + \frac{154}{5}}\; , & {3}< x< {8}\\[3pt]
{3 \, x - 38}\; , & x\geq{8}
\end{array} \right.\]
evaluate the following definite integral by interpreting it in terms of areas.  

\input{Integral-Compute-0005.HELP.tex}

\[
\int_{2}^{8} f(x)\;dx= \answer{\frac{25}{2}}
\]  
\end{problem}}%}

\latexProblemContent{
\ifVerboseLocation This is Integration Compute Question 0005. \\ \fi
\begin{problem}

Given the piecewise function 
\[f(x)=\left\{\begin{array}{ll}
{2 \, x + 4}\; , & x\leq{2}\\[3pt]
{-\frac{8}{3} \, x + \frac{40}{3}}\; , & {2}< x< {8}\\[3pt]
{x - 16}\; , & x\geq{8}
\end{array} \right.\]
evaluate the following definite integral by interpreting it in terms of areas.  

\input{Integral-Compute-0005.HELP.tex}

\[
\int_{0}^{5} f(x)\;dx= \answer{24}
\]  
\end{problem}}%}

\latexProblemContent{
\ifVerboseLocation This is Integration Compute Question 0005. \\ \fi
\begin{problem}

Given the piecewise function 
\[f(x)=\left\{\begin{array}{ll}
{2 \, x + 4}\; , & x\leq{2}\\[3pt]
{-\frac{16}{3} \, x + \frac{56}{3}}\; , & {2}< x< {5}\\[3pt]
{x - 13}\; , & x\geq{5}
\end{array} \right.\]
evaluate the following definite integral by interpreting it in terms of areas.  

\input{Integral-Compute-0005.HELP.tex}

\[
\int_{-2}^{6} f(x)\;dx= \answer{\frac{17}{2}}
\]  
\end{problem}}%}

\latexProblemContent{
\ifVerboseLocation This is Integration Compute Question 0005. \\ \fi
\begin{problem}

Given the piecewise function 
\[f(x)=\left\{\begin{array}{ll}
{x + 1}\; , & x\leq{5}\\[3pt]
{-2 \, x + 16}\; , & {5}< x< {11}\\[3pt]
{4 \, x - 50}\; , & x\geq{11}
\end{array} \right.\]
evaluate the following definite integral by interpreting it in terms of areas.  

\input{Integral-Compute-0005.HELP.tex}

\[
\int_{5}^{16} f(x)\;dx= \answer{20}
\]  
\end{problem}}%}

\latexProblemContent{
\ifVerboseLocation This is Integration Compute Question 0005. \\ \fi
\begin{problem}

Given the piecewise function 
\[f(x)=\left\{\begin{array}{ll}
{x + 4}\; , & x\leq{2}\\[3pt]
{-\frac{12}{7} \, x + \frac{66}{7}}\; , & {2}< x< {9}\\[3pt]
{3 \, x - 33}\; , & x\geq{9}
\end{array} \right.\]
evaluate the following definite integral by interpreting it in terms of areas.  

\input{Integral-Compute-0005.HELP.tex}

\[
\int_{-3}^{5} f(x)\;dx= \answer{\frac{389}{14}}
\]  
\end{problem}}%}

\latexProblemContent{
\ifVerboseLocation This is Integration Compute Question 0005. \\ \fi
\begin{problem}

Given the piecewise function 
\[f(x)=\left\{\begin{array}{ll}
{2 \, x + 1}\; , & x\leq{4}\\[3pt]
{-6 \, x + 33}\; , & {4}< x< {7}\\[3pt]
{2 \, x - 23}\; , & x\geq{7}
\end{array} \right.\]
evaluate the following definite integral by interpreting it in terms of areas.  

\input{Integral-Compute-0005.HELP.tex}

\[
\int_{2}^{5} f(x)\;dx= \answer{20}
\]  
\end{problem}}%}

\latexProblemContent{
\ifVerboseLocation This is Integration Compute Question 0005. \\ \fi
\begin{problem}

Given the piecewise function 
\[f(x)=\left\{\begin{array}{ll}
{x + 5}\; , & x\leq{1}\\[3pt]
{-\frac{3}{2} \, x + \frac{15}{2}}\; , & {1}< x< {9}\\[3pt]
{3 \, x - 33}\; , & x\geq{9}
\end{array} \right.\]
evaluate the following definite integral by interpreting it in terms of areas.  

\input{Integral-Compute-0005.HELP.tex}

\[
\int_{0}^{5} f(x)\;dx= \answer{\frac{35}{2}}
\]  
\end{problem}}%}

\latexProblemContent{
\ifVerboseLocation This is Integration Compute Question 0005. \\ \fi
\begin{problem}

Given the piecewise function 
\[f(x)=\left\{\begin{array}{ll}
{4 \, x + 1}\; , & x\leq{-3}\\[3pt]
{\frac{22}{7} \, x - \frac{11}{7}}\; , & {-3}< x< {4}\\[3pt]
{2 \, x + 3}\; , & x\geq{4}
\end{array} \right.\]
evaluate the following definite integral by interpreting it in terms of areas.  

\input{Integral-Compute-0005.HELP.tex}

\[
\int_{-7}^{5} f(x)\;dx= \answer{-64}
\]  
\end{problem}}%}

\latexProblemContent{
\ifVerboseLocation This is Integration Compute Question 0005. \\ \fi
\begin{problem}

Given the piecewise function 
\[f(x)=\left\{\begin{array}{ll}
{3 \, x + 4}\; , & x\leq{-2}\\[3pt]
{\frac{4}{5} \, x - \frac{2}{5}}\; , & {-2}< x< {3}\\[3pt]
{x - 1}\; , & x\geq{3}
\end{array} \right.\]
evaluate the following definite integral by interpreting it in terms of areas.  

\input{Integral-Compute-0005.HELP.tex}

\[
\int_{-5}^{4} f(x)\;dx= \answer{-17}
\]  
\end{problem}}%}

\latexProblemContent{
\ifVerboseLocation This is Integration Compute Question 0005. \\ \fi
\begin{problem}

Given the piecewise function 
\[f(x)=\left\{\begin{array}{ll}
{3 \, x + 5}\; , & x\leq{1}\\[3pt]
{-\frac{16}{5} \, x + \frac{56}{5}}\; , & {1}< x< {6}\\[3pt]
{x - 14}\; , & x\geq{6}
\end{array} \right.\]
evaluate the following definite integral by interpreting it in terms of areas.  

\input{Integral-Compute-0005.HELP.tex}

\[
\int_{-3}^{2} f(x)\;dx= \answer{\frac{72}{5}}
\]  
\end{problem}}%}

\latexProblemContent{
\ifVerboseLocation This is Integration Compute Question 0005. \\ \fi
\begin{problem}

Given the piecewise function 
\[f(x)=\left\{\begin{array}{ll}
{2 \, x + 4}\; , & x\leq{-5}\\[3pt]
{3 \, x + 9}\; , & {-5}< x< {-1}\\[3pt]
{x + 7}\; , & x\geq{-1}
\end{array} \right.\]
evaluate the following definite integral by interpreting it in terms of areas.  

\input{Integral-Compute-0005.HELP.tex}

\[
\int_{-9}^{-1} f(x)\;dx= \answer{-40}
\]  
\end{problem}}%}

\latexProblemContent{
\ifVerboseLocation This is Integration Compute Question 0005. \\ \fi
\begin{problem}

Given the piecewise function 
\[f(x)=\left\{\begin{array}{ll}
{x + 4}\; , & x\leq{2}\\[3pt]
{-\frac{3}{2} \, x + 9}\; , & {2}< x< {10}\\[3pt]
{4 \, x - 46}\; , & x\geq{10}
\end{array} \right.\]
evaluate the following definite integral by interpreting it in terms of areas.  

\input{Integral-Compute-0005.HELP.tex}

\[
\int_{1}^{12} f(x)\;dx= \answer{\frac{3}{2}}
\]  
\end{problem}}%}

\latexProblemContent{
\ifVerboseLocation This is Integration Compute Question 0005. \\ \fi
\begin{problem}

Given the piecewise function 
\[f(x)=\left\{\begin{array}{ll}
{x + 2}\; , & x\leq{-5}\\[3pt]
{\frac{6}{7} \, x + \frac{9}{7}}\; , & {-5}< x< {2}\\[3pt]
{2 \, x - 1}\; , & x\geq{2}
\end{array} \right.\]
evaluate the following definite integral by interpreting it in terms of areas.  

\input{Integral-Compute-0005.HELP.tex}

\[
\int_{-8}^{0} f(x)\;dx= \answer{-\frac{249}{14}}
\]  
\end{problem}}%}

\latexProblemContent{
\ifVerboseLocation This is Integration Compute Question 0005. \\ \fi
\begin{problem}

Given the piecewise function 
\[f(x)=\left\{\begin{array}{ll}
{2 \, x + 1}\; , & x\leq{2}\\[3pt]
{-\frac{5}{2} \, x + 10}\; , & {2}< x< {6}\\[3pt]
{x - 11}\; , & x\geq{6}
\end{array} \right.\]
evaluate the following definite integral by interpreting it in terms of areas.  

\input{Integral-Compute-0005.HELP.tex}

\[
\int_{2}^{5} f(x)\;dx= \answer{\frac{15}{4}}
\]  
\end{problem}}%}

\latexProblemContent{
\ifVerboseLocation This is Integration Compute Question 0005. \\ \fi
\begin{problem}

Given the piecewise function 
\[f(x)=\left\{\begin{array}{ll}
{3 \, x + 1}\; , & x\leq{-1}\\[3pt]
{\frac{1}{2} \, x - \frac{3}{2}}\; , & {-1}< x< {7}\\[3pt]
{2 \, x - 12}\; , & x\geq{7}
\end{array} \right.\]
evaluate the following definite integral by interpreting it in terms of areas.  

\input{Integral-Compute-0005.HELP.tex}

\[
\int_{-6}^{7} f(x)\;dx= \answer{-\frac{95}{2}}
\]  
\end{problem}}%}

\latexProblemContent{
\ifVerboseLocation This is Integration Compute Question 0005. \\ \fi
\begin{problem}

Given the piecewise function 
\[f(x)=\left\{\begin{array}{ll}
{3 \, x + 1}\; , & x\leq{2}\\[3pt]
{-\frac{14}{3} \, x + \frac{49}{3}}\; , & {2}< x< {5}\\[3pt]
{2 \, x - 17}\; , & x\geq{5}
\end{array} \right.\]
evaluate the following definite integral by interpreting it in terms of areas.  

\input{Integral-Compute-0005.HELP.tex}

\[
\int_{-1}^{10} f(x)\;dx= \answer{-\frac{5}{2}}
\]  
\end{problem}}%}

\latexProblemContent{
\ifVerboseLocation This is Integration Compute Question 0005. \\ \fi
\begin{problem}

Given the piecewise function 
\[f(x)=\left\{\begin{array}{ll}
{3 \, x + 2}\; , & x\leq{-2}\\[3pt]
{\frac{4}{3} \, x - \frac{4}{3}}\; , & {-2}< x< {4}\\[3pt]
{3 \, x - 8}\; , & x\geq{4}
\end{array} \right.\]
evaluate the following definite integral by interpreting it in terms of areas.  

\input{Integral-Compute-0005.HELP.tex}

\[
\int_{-4}^{7} f(x)\;dx= \answer{\frac{23}{2}}
\]  
\end{problem}}%}

\latexProblemContent{
\ifVerboseLocation This is Integration Compute Question 0005. \\ \fi
\begin{problem}

Given the piecewise function 
\[f(x)=\left\{\begin{array}{ll}
{4 \, x + 3}\; , & x\leq{4}\\[3pt]
{-\frac{38}{7} \, x + \frac{285}{7}}\; , & {4}< x< {11}\\[3pt]
{4 \, x - 63}\; , & x\geq{11}
\end{array} \right.\]
evaluate the following definite integral by interpreting it in terms of areas.  

\input{Integral-Compute-0005.HELP.tex}

\[
\int_{4}^{11} f(x)\;dx= \answer{0}
\]  
\end{problem}}%}

\latexProblemContent{
\ifVerboseLocation This is Integration Compute Question 0005. \\ \fi
\begin{problem}

Given the piecewise function 
\[f(x)=\left\{\begin{array}{ll}
{4 \, x + 2}\; , & x\leq{5}\\[3pt]
{-\frac{44}{3} \, x + \frac{286}{3}}\; , & {5}< x< {8}\\[3pt]
{2 \, x - 38}\; , & x\geq{8}
\end{array} \right.\]
evaluate the following definite integral by interpreting it in terms of areas.  

\input{Integral-Compute-0005.HELP.tex}

\[
\int_{5}^{7} f(x)\;dx= \answer{\frac{44}{3}}
\]  
\end{problem}}%}

\latexProblemContent{
\ifVerboseLocation This is Integration Compute Question 0005. \\ \fi
\begin{problem}

Given the piecewise function 
\[f(x)=\left\{\begin{array}{ll}
{3 \, x + 5}\; , & x\leq{-4}\\[3pt]
{2 \, x + 1}\; , & {-4}< x< {3}\\[3pt]
{4 \, x - 5}\; , & x\geq{3}
\end{array} \right.\]
evaluate the following definite integral by interpreting it in terms of areas.  

\input{Integral-Compute-0005.HELP.tex}

\[
\int_{-6}^{-2} f(x)\;dx= \answer{-30}
\]  
\end{problem}}%}

\latexProblemContent{
\ifVerboseLocation This is Integration Compute Question 0005. \\ \fi
\begin{problem}

Given the piecewise function 
\[f(x)=\left\{\begin{array}{ll}
{4 \, x + 3}\; , & x\leq{-3}\\[3pt]
{\frac{9}{4} \, x - \frac{9}{4}}\; , & {-3}< x< {5}\\[3pt]
{4 \, x - 11}\; , & x\geq{5}
\end{array} \right.\]
evaluate the following definite integral by interpreting it in terms of areas.  

\input{Integral-Compute-0005.HELP.tex}

\[
\int_{-7}^{0} f(x)\;dx= \answer{-\frac{679}{8}}
\]  
\end{problem}}%}

\latexProblemContent{
\ifVerboseLocation This is Integration Compute Question 0005. \\ \fi
\begin{problem}

Given the piecewise function 
\[f(x)=\left\{\begin{array}{ll}
{2 \, x + 2}\; , & x\leq{-4}\\[3pt]
{\frac{12}{5} \, x + \frac{18}{5}}\; , & {-4}< x< {1}\\[3pt]
{x + 5}\; , & x\geq{1}
\end{array} \right.\]
evaluate the following definite integral by interpreting it in terms of areas.  

\input{Integral-Compute-0005.HELP.tex}

\[
\int_{-4}^{6} f(x)\;dx= \answer{\frac{85}{2}}
\]  
\end{problem}}%}

\latexProblemContent{
\ifVerboseLocation This is Integration Compute Question 0005. \\ \fi
\begin{problem}

Given the piecewise function 
\[f(x)=\left\{\begin{array}{ll}
{2 \, x + 4}\; , & x\leq{1}\\[3pt]
{-2 \, x + 8}\; , & {1}< x< {7}\\[3pt]
{x - 13}\; , & x\geq{7}
\end{array} \right.\]
evaluate the following definite integral by interpreting it in terms of areas.  

\input{Integral-Compute-0005.HELP.tex}

\[
\int_{1}^{6} f(x)\;dx= \answer{5}
\]  
\end{problem}}%}

\latexProblemContent{
\ifVerboseLocation This is Integration Compute Question 0005. \\ \fi
\begin{problem}

Given the piecewise function 
\[f(x)=\left\{\begin{array}{ll}
{3 \, x + 5}\; , & x\leq{2}\\[3pt]
{-\frac{11}{3} \, x + \frac{55}{3}}\; , & {2}< x< {8}\\[3pt]
{4 \, x - 43}\; , & x\geq{8}
\end{array} \right.\]
evaluate the following definite integral by interpreting it in terms of areas.  

\input{Integral-Compute-0005.HELP.tex}

\[
\int_{-3}^{10} f(x)\;dx= \answer{\frac{7}{2}}
\]  
\end{problem}}%}

\latexProblemContent{
\ifVerboseLocation This is Integration Compute Question 0005. \\ \fi
\begin{problem}

Given the piecewise function 
\[f(x)=\left\{\begin{array}{ll}
{3 \, x + 1}\; , & x\leq{-4}\\[3pt]
{\frac{22}{5} \, x + \frac{33}{5}}\; , & {-4}< x< {1}\\[3pt]
{3 \, x + 8}\; , & x\geq{1}
\end{array} \right.\]
evaluate the following definite integral by interpreting it in terms of areas.  

\input{Integral-Compute-0005.HELP.tex}

\[
\int_{-9}^{1} f(x)\;dx= \answer{-\frac{185}{2}}
\]  
\end{problem}}%}

\latexProblemContent{
\ifVerboseLocation This is Integration Compute Question 0005. \\ \fi
\begin{problem}

Given the piecewise function 
\[f(x)=\left\{\begin{array}{ll}
{4 \, x + 5}\; , & x\leq{-5}\\[3pt]
{5 \, x + 10}\; , & {-5}< x< {1}\\[3pt]
{x + 14}\; , & x\geq{1}
\end{array} \right.\]
evaluate the following definite integral by interpreting it in terms of areas.  

\input{Integral-Compute-0005.HELP.tex}

\[
\int_{-6}^{6} f(x)\;dx= \answer{\frac{141}{2}}
\]  
\end{problem}}%}

\latexProblemContent{
\ifVerboseLocation This is Integration Compute Question 0005. \\ \fi
\begin{problem}

Given the piecewise function 
\[f(x)=\left\{\begin{array}{ll}
{2 \, x + 1}\; , & x\leq{4}\\[3pt]
{-\frac{18}{5} \, x + \frac{117}{5}}\; , & {4}< x< {9}\\[3pt]
{3 \, x - 36}\; , & x\geq{9}
\end{array} \right.\]
evaluate the following definite integral by interpreting it in terms of areas.  

\input{Integral-Compute-0005.HELP.tex}

\[
\int_{1}^{11} f(x)\;dx= \answer{6}
\]  
\end{problem}}%}

\latexProblemContent{
\ifVerboseLocation This is Integration Compute Question 0005. \\ \fi
\begin{problem}

Given the piecewise function 
\[f(x)=\left\{\begin{array}{ll}
{3 \, x + 4}\; , & x\leq{1}\\[3pt]
{-\frac{7}{3} \, x + \frac{28}{3}}\; , & {1}< x< {7}\\[3pt]
{3 \, x - 28}\; , & x\geq{7}
\end{array} \right.\]
evaluate the following definite integral by interpreting it in terms of areas.  

\input{Integral-Compute-0005.HELP.tex}

\[
\int_{-1}^{7} f(x)\;dx= \answer{8}
\]  
\end{problem}}%}

\latexProblemContent{
\ifVerboseLocation This is Integration Compute Question 0005. \\ \fi
\begin{problem}

Given the piecewise function 
\[f(x)=\left\{\begin{array}{ll}
{3 \, x + 1}\; , & x\leq{2}\\[3pt]
{-\frac{7}{4} \, x + \frac{21}{2}}\; , & {2}< x< {10}\\[3pt]
{4 \, x - 47}\; , & x\geq{10}
\end{array} \right.\]
evaluate the following definite integral by interpreting it in terms of areas.  

\input{Integral-Compute-0005.HELP.tex}

\[
\int_{-1}^{4} f(x)\;dx= \answer{18}
\]  
\end{problem}}%}

\latexProblemContent{
\ifVerboseLocation This is Integration Compute Question 0005. \\ \fi
\begin{problem}

Given the piecewise function 
\[f(x)=\left\{\begin{array}{ll}
{4 \, x + 2}\; , & x\leq{1}\\[3pt]
{-\frac{3}{2} \, x + \frac{15}{2}}\; , & {1}< x< {9}\\[3pt]
{3 \, x - 33}\; , & x\geq{9}
\end{array} \right.\]
evaluate the following definite integral by interpreting it in terms of areas.  

\input{Integral-Compute-0005.HELP.tex}

\[
\int_{-1}^{7} f(x)\;dx= \answer{13}
\]  
\end{problem}}%}

\latexProblemContent{
\ifVerboseLocation This is Integration Compute Question 0005. \\ \fi
\begin{problem}

Given the piecewise function 
\[f(x)=\left\{\begin{array}{ll}
{3 \, x + 4}\; , & x\leq{3}\\[3pt]
{-\frac{13}{2} \, x + \frac{65}{2}}\; , & {3}< x< {7}\\[3pt]
{2 \, x - 27}\; , & x\geq{7}
\end{array} \right.\]
evaluate the following definite integral by interpreting it in terms of areas.  

\input{Integral-Compute-0005.HELP.tex}

\[
\int_{2}^{8} f(x)\;dx= \answer{-\frac{1}{2}}
\]  
\end{problem}}%}

\latexProblemContent{
\ifVerboseLocation This is Integration Compute Question 0005. \\ \fi
\begin{problem}

Given the piecewise function 
\[f(x)=\left\{\begin{array}{ll}
{x + 5}\; , & x\leq{4}\\[3pt]
{-\frac{18}{5} \, x + \frac{117}{5}}\; , & {4}< x< {9}\\[3pt]
{4 \, x - 45}\; , & x\geq{9}
\end{array} \right.\]
evaluate the following definite integral by interpreting it in terms of areas.  

\input{Integral-Compute-0005.HELP.tex}

\[
\int_{0}^{10} f(x)\;dx= \answer{21}
\]  
\end{problem}}%}

\latexProblemContent{
\ifVerboseLocation This is Integration Compute Question 0005. \\ \fi
\begin{problem}

Given the piecewise function 
\[f(x)=\left\{\begin{array}{ll}
{3 \, x + 4}\; , & x\leq{-2}\\[3pt]
{x}\; , & {-2}< x< {2}\\[3pt]
{3 \, x - 4}\; , & x\geq{2}
\end{array} \right.\]
evaluate the following definite integral by interpreting it in terms of areas.  

\input{Integral-Compute-0005.HELP.tex}

\[
\int_{-5}^{3} f(x)\;dx= \answer{-16}
\]  
\end{problem}}%}

\latexProblemContent{
\ifVerboseLocation This is Integration Compute Question 0005. \\ \fi
\begin{problem}

Given the piecewise function 
\[f(x)=\left\{\begin{array}{ll}
{3 \, x + 2}\; , & x\leq{-1}\\[3pt]
{\frac{1}{2} \, x - \frac{1}{2}}\; , & {-1}< x< {3}\\[3pt]
{4 \, x - 11}\; , & x\geq{3}
\end{array} \right.\]
evaluate the following definite integral by interpreting it in terms of areas.  

\input{Integral-Compute-0005.HELP.tex}

\[
\int_{-2}^{4} f(x)\;dx= \answer{\frac{1}{2}}
\]  
\end{problem}}%}

\latexProblemContent{
\ifVerboseLocation This is Integration Compute Question 0005. \\ \fi
\begin{problem}

Given the piecewise function 
\[f(x)=\left\{\begin{array}{ll}
{4 \, x + 3}\; , & x\leq{5}\\[3pt]
{-\frac{46}{7} \, x + \frac{391}{7}}\; , & {5}< x< {12}\\[3pt]
{2 \, x - 47}\; , & x\geq{12}
\end{array} \right.\]
evaluate the following definite integral by interpreting it in terms of areas.  

\input{Integral-Compute-0005.HELP.tex}

\[
\int_{0}^{16} f(x)\;dx= \answer{-11}
\]  
\end{problem}}%}

\latexProblemContent{
\ifVerboseLocation This is Integration Compute Question 0005. \\ \fi
\begin{problem}

Given the piecewise function 
\[f(x)=\left\{\begin{array}{ll}
{x + 1}\; , & x\leq{2}\\[3pt]
{-\frac{3}{2} \, x + 6}\; , & {2}< x< {6}\\[3pt]
{x - 9}\; , & x\geq{6}
\end{array} \right.\]
evaluate the following definite integral by interpreting it in terms of areas.  

\input{Integral-Compute-0005.HELP.tex}

\[
\int_{0}^{5} f(x)\;dx= \answer{\frac{25}{4}}
\]  
\end{problem}}%}

\latexProblemContent{
\ifVerboseLocation This is Integration Compute Question 0005. \\ \fi
\begin{problem}

Given the piecewise function 
\[f(x)=\left\{\begin{array}{ll}
{3 \, x + 3}\; , & x\leq{-1}\\[3pt]
{0}\; , & {-1}< x< {6}\\[3pt]
{3 \, x - 18}\; , & x\geq{6}
\end{array} \right.\]
evaluate the following definite integral by interpreting it in terms of areas.  

\input{Integral-Compute-0005.HELP.tex}

\[
\int_{-5}^{5} f(x)\;dx= \answer{-24}
\]  
\end{problem}}%}

\latexProblemContent{
\ifVerboseLocation This is Integration Compute Question 0005. \\ \fi
\begin{problem}

Given the piecewise function 
\[f(x)=\left\{\begin{array}{ll}
{3 \, x + 3}\; , & x\leq{3}\\[3pt]
{-\frac{24}{7} \, x + \frac{156}{7}}\; , & {3}< x< {10}\\[3pt]
{3 \, x - 42}\; , & x\geq{10}
\end{array} \right.\]
evaluate the following definite integral by interpreting it in terms of areas.  

\input{Integral-Compute-0005.HELP.tex}

\[
\int_{0}^{15} f(x)\;dx= \answer{0}
\]  
\end{problem}}%}

\latexProblemContent{
\ifVerboseLocation This is Integration Compute Question 0005. \\ \fi
\begin{problem}

Given the piecewise function 
\[f(x)=\left\{\begin{array}{ll}
{4 \, x + 3}\; , & x\leq{-4}\\[3pt]
{\frac{26}{5} \, x + \frac{39}{5}}\; , & {-4}< x< {1}\\[3pt]
{4 \, x + 9}\; , & x\geq{1}
\end{array} \right.\]
evaluate the following definite integral by interpreting it in terms of areas.  

\input{Integral-Compute-0005.HELP.tex}

\[
\int_{-9}^{-2} f(x)\;dx= \answer{-\frac{653}{5}}
\]  
\end{problem}}%}

\latexProblemContent{
\ifVerboseLocation This is Integration Compute Question 0005. \\ \fi
\begin{problem}

Given the piecewise function 
\[f(x)=\left\{\begin{array}{ll}
{4 \, x + 3}\; , & x\leq{4}\\[3pt]
{-\frac{38}{5} \, x + \frac{247}{5}}\; , & {4}< x< {9}\\[3pt]
{x - 28}\; , & x\geq{9}
\end{array} \right.\]
evaluate the following definite integral by interpreting it in terms of areas.  

\input{Integral-Compute-0005.HELP.tex}

\[
\int_{1}^{11} f(x)\;dx= \answer{3}
\]  
\end{problem}}%}

\latexProblemContent{
\ifVerboseLocation This is Integration Compute Question 0005. \\ \fi
\begin{problem}

Given the piecewise function 
\[f(x)=\left\{\begin{array}{ll}
{4 \, x + 5}\; , & x\leq{1}\\[3pt]
{-3 \, x + 12}\; , & {1}< x< {7}\\[3pt]
{4 \, x - 37}\; , & x\geq{7}
\end{array} \right.\]
evaluate the following definite integral by interpreting it in terms of areas.  

\input{Integral-Compute-0005.HELP.tex}

\[
\int_{-1}^{3} f(x)\;dx= \answer{22}
\]  
\end{problem}}%}

\latexProblemContent{
\ifVerboseLocation This is Integration Compute Question 0005. \\ \fi
\begin{problem}

Given the piecewise function 
\[f(x)=\left\{\begin{array}{ll}
{4 \, x + 1}\; , & x\leq{-2}\\[3pt]
{\frac{7}{3} \, x - \frac{7}{3}}\; , & {-2}< x< {4}\\[3pt]
{x + 3}\; , & x\geq{4}
\end{array} \right.\]
evaluate the following definite integral by interpreting it in terms of areas.  

\input{Integral-Compute-0005.HELP.tex}

\[
\int_{-3}^{1} f(x)\;dx= \answer{-\frac{39}{2}}
\]  
\end{problem}}%}

\latexProblemContent{
\ifVerboseLocation This is Integration Compute Question 0005. \\ \fi
\begin{problem}

Given the piecewise function 
\[f(x)=\left\{\begin{array}{ll}
{x + 5}\; , & x\leq{-4}\\[3pt]
{-\frac{2}{5} \, x - \frac{3}{5}}\; , & {-4}< x< {1}\\[3pt]
{4 \, x - 5}\; , & x\geq{1}
\end{array} \right.\]
evaluate the following definite integral by interpreting it in terms of areas.  

\input{Integral-Compute-0005.HELP.tex}

\[
\int_{-7}^{-1} f(x)\;dx= \answer{-\frac{3}{10}}
\]  
\end{problem}}%}

\latexProblemContent{
\ifVerboseLocation This is Integration Compute Question 0005. \\ \fi
\begin{problem}

Given the piecewise function 
\[f(x)=\left\{\begin{array}{ll}
{x + 3}\; , & x\leq{-2}\\[3pt]
{-\frac{2}{7} \, x + \frac{3}{7}}\; , & {-2}< x< {5}\\[3pt]
{4 \, x - 21}\; , & x\geq{5}
\end{array} \right.\]
evaluate the following definite integral by interpreting it in terms of areas.  

\input{Integral-Compute-0005.HELP.tex}

\[
\int_{-5}^{9} f(x)\;dx= \answer{\frac{53}{2}}
\]  
\end{problem}}%}

\latexProblemContent{
\ifVerboseLocation This is Integration Compute Question 0005. \\ \fi
\begin{problem}

Given the piecewise function 
\[f(x)=\left\{\begin{array}{ll}
{2 \, x + 3}\; , & x\leq{-2}\\[3pt]
{\frac{1}{2} \, x}\; , & {-2}< x< {2}\\[3pt]
{2 \, x - 3}\; , & x\geq{2}
\end{array} \right.\]
evaluate the following definite integral by interpreting it in terms of areas.  

\input{Integral-Compute-0005.HELP.tex}

\[
\int_{-3}^{1} f(x)\;dx= \answer{-\frac{11}{4}}
\]  
\end{problem}}%}

\latexProblemContent{
\ifVerboseLocation This is Integration Compute Question 0005. \\ \fi
\begin{problem}

Given the piecewise function 
\[f(x)=\left\{\begin{array}{ll}
{4 \, x + 5}\; , & x\leq{3}\\[3pt]
{-\frac{17}{3} \, x + 34}\; , & {3}< x< {9}\\[3pt]
{x - 26}\; , & x\geq{9}
\end{array} \right.\]
evaluate the following definite integral by interpreting it in terms of areas.  

\input{Integral-Compute-0005.HELP.tex}

\[
\int_{1}^{13} f(x)\;dx= \answer{-34}
\]  
\end{problem}}%}

\latexProblemContent{
\ifVerboseLocation This is Integration Compute Question 0005. \\ \fi
\begin{problem}

Given the piecewise function 
\[f(x)=\left\{\begin{array}{ll}
{2 \, x + 4}\; , & x\leq{4}\\[3pt]
{-8 \, x + 44}\; , & {4}< x< {7}\\[3pt]
{4 \, x - 40}\; , & x\geq{7}
\end{array} \right.\]
evaluate the following definite integral by interpreting it in terms of areas.  

\input{Integral-Compute-0005.HELP.tex}

\[
\int_{-1}^{5} f(x)\;dx= \answer{43}
\]  
\end{problem}}%}

\latexProblemContent{
\ifVerboseLocation This is Integration Compute Question 0005. \\ \fi
\begin{problem}

Given the piecewise function 
\[f(x)=\left\{\begin{array}{ll}
{4 \, x + 2}\; , & x\leq{1}\\[3pt]
{-3 \, x + 9}\; , & {1}< x< {5}\\[3pt]
{3 \, x - 21}\; , & x\geq{5}
\end{array} \right.\]
evaluate the following definite integral by interpreting it in terms of areas.  

\input{Integral-Compute-0005.HELP.tex}

\[
\int_{-1}^{9} f(x)\;dx= \answer{4}
\]  
\end{problem}}%}

\latexProblemContent{
\ifVerboseLocation This is Integration Compute Question 0005. \\ \fi
\begin{problem}

Given the piecewise function 
\[f(x)=\left\{\begin{array}{ll}
{2 \, x + 2}\; , & x\leq{2}\\[3pt]
{-4 \, x + 14}\; , & {2}< x< {5}\\[3pt]
{4 \, x - 26}\; , & x\geq{5}
\end{array} \right.\]
evaluate the following definite integral by interpreting it in terms of areas.  

\input{Integral-Compute-0005.HELP.tex}

\[
\int_{1}^{4} f(x)\;dx= \answer{9}
\]  
\end{problem}}%}

\latexProblemContent{
\ifVerboseLocation This is Integration Compute Question 0005. \\ \fi
\begin{problem}

Given the piecewise function 
\[f(x)=\left\{\begin{array}{ll}
{3 \, x + 5}\; , & x\leq{-3}\\[3pt]
{\frac{4}{3} \, x}\; , & {-3}< x< {3}\\[3pt]
{4 \, x - 8}\; , & x\geq{3}
\end{array} \right.\]
evaluate the following definite integral by interpreting it in terms of areas.  

\input{Integral-Compute-0005.HELP.tex}

\[
\int_{-8}^{-1} f(x)\;dx= \answer{-\frac{377}{6}}
\]  
\end{problem}}%}

\latexProblemContent{
\ifVerboseLocation This is Integration Compute Question 0005. \\ \fi
\begin{problem}

Given the piecewise function 
\[f(x)=\left\{\begin{array}{ll}
{4 \, x + 1}\; , & x\leq{3}\\[3pt]
{-\frac{13}{2} \, x + \frac{65}{2}}\; , & {3}< x< {7}\\[3pt]
{2 \, x - 27}\; , & x\geq{7}
\end{array} \right.\]
evaluate the following definite integral by interpreting it in terms of areas.  

\input{Integral-Compute-0005.HELP.tex}

\[
\int_{-1}^{9} f(x)\;dx= \answer{-2}
\]  
\end{problem}}%}

\latexProblemContent{
\ifVerboseLocation This is Integration Compute Question 0005. \\ \fi
\begin{problem}

Given the piecewise function 
\[f(x)=\left\{\begin{array}{ll}
{4 \, x + 4}\; , & x\leq{1}\\[3pt]
{-\frac{16}{7} \, x + \frac{72}{7}}\; , & {1}< x< {8}\\[3pt]
{3 \, x - 32}\; , & x\geq{8}
\end{array} \right.\]
evaluate the following definite integral by interpreting it in terms of areas.  

\input{Integral-Compute-0005.HELP.tex}

\[
\int_{-3}^{8} f(x)\;dx= \answer{0}
\]  
\end{problem}}%}

\latexProblemContent{
\ifVerboseLocation This is Integration Compute Question 0005. \\ \fi
\begin{problem}

Given the piecewise function 
\[f(x)=\left\{\begin{array}{ll}
{x + 5}\; , & x\leq{-5}\\[3pt]
{0}\; , & {-5}< x< {3}\\[3pt]
{3 \, x - 9}\; , & x\geq{3}
\end{array} \right.\]
evaluate the following definite integral by interpreting it in terms of areas.  

\input{Integral-Compute-0005.HELP.tex}

\[
\int_{-5}^{-2} f(x)\;dx= \answer{0}
\]  
\end{problem}}%}

\latexProblemContent{
\ifVerboseLocation This is Integration Compute Question 0005. \\ \fi
\begin{problem}

Given the piecewise function 
\[f(x)=\left\{\begin{array}{ll}
{2 \, x + 2}\; , & x\leq{-1}\\[3pt]
{0}\; , & {-1}< x< {7}\\[3pt]
{x - 7}\; , & x\geq{7}
\end{array} \right.\]
evaluate the following definite integral by interpreting it in terms of areas.  

\input{Integral-Compute-0005.HELP.tex}

\[
\int_{-4}^{7} f(x)\;dx= \answer{-9}
\]  
\end{problem}}%}

\latexProblemContent{
\ifVerboseLocation This is Integration Compute Question 0005. \\ \fi
\begin{problem}

Given the piecewise function 
\[f(x)=\left\{\begin{array}{ll}
{3 \, x + 1}\; , & x\leq{-4}\\[3pt]
{\frac{22}{3} \, x + \frac{55}{3}}\; , & {-4}< x< {-1}\\[3pt]
{x + 12}\; , & x\geq{-1}
\end{array} \right.\]
evaluate the following definite integral by interpreting it in terms of areas.  

\input{Integral-Compute-0005.HELP.tex}

\[
\int_{-6}^{3} f(x)\;dx= \answer{24}
\]  
\end{problem}}%}

\latexProblemContent{
\ifVerboseLocation This is Integration Compute Question 0005. \\ \fi
\begin{problem}

Given the piecewise function 
\[f(x)=\left\{\begin{array}{ll}
{x + 4}\; , & x\leq{-5}\\[3pt]
{\frac{1}{4} \, x + \frac{1}{4}}\; , & {-5}< x< {3}\\[3pt]
{4 \, x - 11}\; , & x\geq{3}
\end{array} \right.\]
evaluate the following definite integral by interpreting it in terms of areas.  

\input{Integral-Compute-0005.HELP.tex}

\[
\int_{-9}^{6} f(x)\;dx= \answer{9}
\]  
\end{problem}}%}

\latexProblemContent{
\ifVerboseLocation This is Integration Compute Question 0005. \\ \fi
\begin{problem}

Given the piecewise function 
\[f(x)=\left\{\begin{array}{ll}
{x + 5}\; , & x\leq{2}\\[3pt]
{-\frac{7}{2} \, x + 14}\; , & {2}< x< {6}\\[3pt]
{x - 13}\; , & x\geq{6}
\end{array} \right.\]
evaluate the following definite integral by interpreting it in terms of areas.  

\input{Integral-Compute-0005.HELP.tex}

\[
\int_{-3}^{11} f(x)\;dx= \answer{0}
\]  
\end{problem}}%}

\latexProblemContent{
\ifVerboseLocation This is Integration Compute Question 0005. \\ \fi
\begin{problem}

Given the piecewise function 
\[f(x)=\left\{\begin{array}{ll}
{2 \, x + 5}\; , & x\leq{-4}\\[3pt]
{\frac{6}{5} \, x + \frac{9}{5}}\; , & {-4}< x< {1}\\[3pt]
{3 \, x}\; , & x\geq{1}
\end{array} \right.\]
evaluate the following definite integral by interpreting it in terms of areas.  

\input{Integral-Compute-0005.HELP.tex}

\[
\int_{-8}^{0} f(x)\;dx= \answer{-\frac{152}{5}}
\]  
\end{problem}}%}

\latexProblemContent{
\ifVerboseLocation This is Integration Compute Question 0005. \\ \fi
\begin{problem}

Given the piecewise function 
\[f(x)=\left\{\begin{array}{ll}
{2 \, x + 1}\; , & x\leq{2}\\[3pt]
{-\frac{10}{3} \, x + \frac{35}{3}}\; , & {2}< x< {5}\\[3pt]
{2 \, x - 15}\; , & x\geq{5}
\end{array} \right.\]
evaluate the following definite integral by interpreting it in terms of areas.  

\input{Integral-Compute-0005.HELP.tex}

\[
\int_{1}^{6} f(x)\;dx= \answer{0}
\]  
\end{problem}}%}

\latexProblemContent{
\ifVerboseLocation This is Integration Compute Question 0005. \\ \fi
\begin{problem}

Given the piecewise function 
\[f(x)=\left\{\begin{array}{ll}
{2 \, x + 1}\; , & x\leq{-4}\\[3pt]
{\frac{14}{3} \, x + \frac{35}{3}}\; , & {-4}< x< {-1}\\[3pt]
{x + 8}\; , & x\geq{-1}
\end{array} \right.\]
evaluate the following definite integral by interpreting it in terms of areas.  

\input{Integral-Compute-0005.HELP.tex}

\[
\int_{-8}^{3} f(x)\;dx= \answer{-8}
\]  
\end{problem}}%}

\latexProblemContent{
\ifVerboseLocation This is Integration Compute Question 0005. \\ \fi
\begin{problem}

Given the piecewise function 
\[f(x)=\left\{\begin{array}{ll}
{4 \, x + 4}\; , & x\leq{4}\\[3pt]
{-8 \, x + 52}\; , & {4}< x< {9}\\[3pt]
{3 \, x - 47}\; , & x\geq{9}
\end{array} \right.\]
evaluate the following definite integral by interpreting it in terms of areas.  

\input{Integral-Compute-0005.HELP.tex}

\[
\int_{0}^{10} f(x)\;dx= \answer{\frac{59}{2}}
\]  
\end{problem}}%}

\latexProblemContent{
\ifVerboseLocation This is Integration Compute Question 0005. \\ \fi
\begin{problem}

Given the piecewise function 
\[f(x)=\left\{\begin{array}{ll}
{x + 1}\; , & x\leq{3}\\[3pt]
{-2 \, x + 10}\; , & {3}< x< {7}\\[3pt]
{4 \, x - 32}\; , & x\geq{7}
\end{array} \right.\]
evaluate the following definite integral by interpreting it in terms of areas.  

\input{Integral-Compute-0005.HELP.tex}

\[
\int_{-1}^{8} f(x)\;dx= \answer{6}
\]  
\end{problem}}%}

\latexProblemContent{
\ifVerboseLocation This is Integration Compute Question 0005. \\ \fi
\begin{problem}

Given the piecewise function 
\[f(x)=\left\{\begin{array}{ll}
{2 \, x + 5}\; , & x\leq{-4}\\[3pt]
{\frac{3}{2} \, x + 3}\; , & {-4}< x< {0}\\[3pt]
{2 \, x + 3}\; , & x\geq{0}
\end{array} \right.\]
evaluate the following definite integral by interpreting it in terms of areas.  

\input{Integral-Compute-0005.HELP.tex}

\[
\int_{-8}^{-3} f(x)\;dx= \answer{-\frac{121}{4}}
\]  
\end{problem}}%}

\latexProblemContent{
\ifVerboseLocation This is Integration Compute Question 0005. \\ \fi
\begin{problem}

Given the piecewise function 
\[f(x)=\left\{\begin{array}{ll}
{x + 5}\; , & x\leq{1}\\[3pt]
{-\frac{12}{5} \, x + \frac{42}{5}}\; , & {1}< x< {6}\\[3pt]
{2 \, x - 18}\; , & x\geq{6}
\end{array} \right.\]
evaluate the following definite integral by interpreting it in terms of areas.  

\input{Integral-Compute-0005.HELP.tex}

\[
\int_{-3}^{8} f(x)\;dx= \answer{8}
\]  
\end{problem}}%}

\latexProblemContent{
\ifVerboseLocation This is Integration Compute Question 0005. \\ \fi
\begin{problem}

Given the piecewise function 
\[f(x)=\left\{\begin{array}{ll}
{x + 5}\; , & x\leq{-3}\\[3pt]
{-\frac{2}{3} \, x}\; , & {-3}< x< {3}\\[3pt]
{3 \, x - 11}\; , & x\geq{3}
\end{array} \right.\]
evaluate the following definite integral by interpreting it in terms of areas.  

\input{Integral-Compute-0005.HELP.tex}

\[
\int_{-7}^{0} f(x)\;dx= \answer{3}
\]  
\end{problem}}%}

\latexProblemContent{
\ifVerboseLocation This is Integration Compute Question 0005. \\ \fi
\begin{problem}

Given the piecewise function 
\[f(x)=\left\{\begin{array}{ll}
{2 \, x + 4}\; , & x\leq{4}\\[3pt]
{-6 \, x + 36}\; , & {4}< x< {8}\\[3pt]
{4 \, x - 44}\; , & x\geq{8}
\end{array} \right.\]
evaluate the following definite integral by interpreting it in terms of areas.  

\input{Integral-Compute-0005.HELP.tex}

\[
\int_{3}^{11} f(x)\;dx= \answer{-7}
\]  
\end{problem}}%}

\latexProblemContent{
\ifVerboseLocation This is Integration Compute Question 0005. \\ \fi
\begin{problem}

Given the piecewise function 
\[f(x)=\left\{\begin{array}{ll}
{4 \, x + 4}\; , & x\leq{-3}\\[3pt]
{\frac{16}{3} \, x + 8}\; , & {-3}< x< {0}\\[3pt]
{x + 8}\; , & x\geq{0}
\end{array} \right.\]
evaluate the following definite integral by interpreting it in terms of areas.  

\input{Integral-Compute-0005.HELP.tex}

\[
\int_{-8}^{-1} f(x)\;dx= \answer{-\frac{286}{3}}
\]  
\end{problem}}%}

\latexProblemContent{
\ifVerboseLocation This is Integration Compute Question 0005. \\ \fi
\begin{problem}

Given the piecewise function 
\[f(x)=\left\{\begin{array}{ll}
{3 \, x + 3}\; , & x\leq{-2}\\[3pt]
{\frac{3}{4} \, x - \frac{3}{2}}\; , & {-2}< x< {6}\\[3pt]
{4 \, x - 21}\; , & x\geq{6}
\end{array} \right.\]
evaluate the following definite integral by interpreting it in terms of areas.  

\input{Integral-Compute-0005.HELP.tex}

\[
\int_{-2}^{8} f(x)\;dx= \answer{14}
\]  
\end{problem}}%}

\latexProblemContent{
\ifVerboseLocation This is Integration Compute Question 0005. \\ \fi
\begin{problem}

Given the piecewise function 
\[f(x)=\left\{\begin{array}{ll}
{4 \, x + 3}\; , & x\leq{1}\\[3pt]
{-\frac{7}{3} \, x + \frac{28}{3}}\; , & {1}< x< {7}\\[3pt]
{2 \, x - 21}\; , & x\geq{7}
\end{array} \right.\]
evaluate the following definite integral by interpreting it in terms of areas.  

\input{Integral-Compute-0005.HELP.tex}

\[
\int_{0}^{10} f(x)\;dx= \answer{-7}
\]  
\end{problem}}%}

\latexProblemContent{
\ifVerboseLocation This is Integration Compute Question 0005. \\ \fi
\begin{problem}

Given the piecewise function 
\[f(x)=\left\{\begin{array}{ll}
{2 \, x + 2}\; , & x\leq{3}\\[3pt]
{-\frac{8}{3} \, x + 16}\; , & {3}< x< {9}\\[3pt]
{4 \, x - 44}\; , & x\geq{9}
\end{array} \right.\]
evaluate the following definite integral by interpreting it in terms of areas.  

\input{Integral-Compute-0005.HELP.tex}

\[
\int_{-1}^{6} f(x)\;dx= \answer{28}
\]  
\end{problem}}%}

\latexProblemContent{
\ifVerboseLocation This is Integration Compute Question 0005. \\ \fi
\begin{problem}

Given the piecewise function 
\[f(x)=\left\{\begin{array}{ll}
{2 \, x + 5}\; , & x\leq{-1}\\[3pt]
{-2 \, x + 1}\; , & {-1}< x< {2}\\[3pt]
{3 \, x - 9}\; , & x\geq{2}
\end{array} \right.\]
evaluate the following definite integral by interpreting it in terms of areas.  

\input{Integral-Compute-0005.HELP.tex}

\[
\int_{-1}^{4} f(x)\;dx= \answer{0}
\]  
\end{problem}}%}

\latexProblemContent{
\ifVerboseLocation This is Integration Compute Question 0005. \\ \fi
\begin{problem}

Given the piecewise function 
\[f(x)=\left\{\begin{array}{ll}
{3 \, x + 2}\; , & x\leq{-2}\\[3pt]
{\frac{8}{5} \, x - \frac{4}{5}}\; , & {-2}< x< {3}\\[3pt]
{4 \, x - 8}\; , & x\geq{3}
\end{array} \right.\]
evaluate the following definite integral by interpreting it in terms of areas.  

\input{Integral-Compute-0005.HELP.tex}

\[
\int_{-7}^{4} f(x)\;dx= \answer{-\frac{103}{2}}
\]  
\end{problem}}%}

\latexProblemContent{
\ifVerboseLocation This is Integration Compute Question 0005. \\ \fi
\begin{problem}

Given the piecewise function 
\[f(x)=\left\{\begin{array}{ll}
{x + 1}\; , & x\leq{-2}\\[3pt]
{\frac{2}{5} \, x - \frac{1}{5}}\; , & {-2}< x< {3}\\[3pt]
{x - 2}\; , & x\geq{3}
\end{array} \right.\]
evaluate the following definite integral by interpreting it in terms of areas.  

\input{Integral-Compute-0005.HELP.tex}

\[
\int_{-4}^{3} f(x)\;dx= \answer{-4}
\]  
\end{problem}}%}

\latexProblemContent{
\ifVerboseLocation This is Integration Compute Question 0005. \\ \fi
\begin{problem}

Given the piecewise function 
\[f(x)=\left\{\begin{array}{ll}
{x + 4}\; , & x\leq{1}\\[3pt]
{-2 \, x + 7}\; , & {1}< x< {6}\\[3pt]
{3 \, x - 23}\; , & x\geq{6}
\end{array} \right.\]
evaluate the following definite integral by interpreting it in terms of areas.  

\input{Integral-Compute-0005.HELP.tex}

\[
\int_{-4}^{3} f(x)\;dx= \answer{\frac{37}{2}}
\]  
\end{problem}}%}

\latexProblemContent{
\ifVerboseLocation This is Integration Compute Question 0005. \\ \fi
\begin{problem}

Given the piecewise function 
\[f(x)=\left\{\begin{array}{ll}
{4 \, x + 4}\; , & x\leq{5}\\[3pt]
{-8 \, x + 64}\; , & {5}< x< {11}\\[3pt]
{2 \, x - 46}\; , & x\geq{11}
\end{array} \right.\]
evaluate the following definite integral by interpreting it in terms of areas.  

\input{Integral-Compute-0005.HELP.tex}

\[
\int_{1}^{7} f(x)\;dx= \answer{96}
\]  
\end{problem}}%}

\latexProblemContent{
\ifVerboseLocation This is Integration Compute Question 0005. \\ \fi
\begin{problem}

Given the piecewise function 
\[f(x)=\left\{\begin{array}{ll}
{3 \, x + 2}\; , & x\leq{1}\\[3pt]
{-\frac{5}{3} \, x + \frac{20}{3}}\; , & {1}< x< {7}\\[3pt]
{x - 12}\; , & x\geq{7}
\end{array} \right.\]
evaluate the following definite integral by interpreting it in terms of areas.  

\input{Integral-Compute-0005.HELP.tex}

\[
\int_{0}^{12} f(x)\;dx= \answer{-9}
\]  
\end{problem}}%}

\latexProblemContent{
\ifVerboseLocation This is Integration Compute Question 0005. \\ \fi
\begin{problem}

Given the piecewise function 
\[f(x)=\left\{\begin{array}{ll}
{4 \, x + 1}\; , & x\leq{-3}\\[3pt]
{\frac{22}{7} \, x - \frac{11}{7}}\; , & {-3}< x< {4}\\[3pt]
{x + 7}\; , & x\geq{4}
\end{array} \right.\]
evaluate the following definite integral by interpreting it in terms of areas.  

\input{Integral-Compute-0005.HELP.tex}

\[
\int_{-4}^{2} f(x)\;dx= \answer{-\frac{201}{7}}
\]  
\end{problem}}%}

\latexProblemContent{
\ifVerboseLocation This is Integration Compute Question 0005. \\ \fi
\begin{problem}

Given the piecewise function 
\[f(x)=\left\{\begin{array}{ll}
{4 \, x + 5}\; , & x\leq{4}\\[3pt]
{-\frac{21}{2} \, x + 63}\; , & {4}< x< {8}\\[3pt]
{x - 29}\; , & x\geq{8}
\end{array} \right.\]
evaluate the following definite integral by interpreting it in terms of areas.  

\input{Integral-Compute-0005.HELP.tex}

\[
\int_{1}^{6} f(x)\;dx= \answer{66}
\]  
\end{problem}}%}

\latexProblemContent{
\ifVerboseLocation This is Integration Compute Question 0005. \\ \fi
\begin{problem}

Given the piecewise function 
\[f(x)=\left\{\begin{array}{ll}
{4 \, x + 5}\; , & x\leq{3}\\[3pt]
{-\frac{17}{4} \, x + \frac{119}{4}}\; , & {3}< x< {11}\\[3pt]
{3 \, x - 50}\; , & x\geq{11}
\end{array} \right.\]
evaluate the following definite integral by interpreting it in terms of areas.  

\input{Integral-Compute-0005.HELP.tex}

\[
\int_{0}^{7} f(x)\;dx= \answer{67}
\]  
\end{problem}}%}

\latexProblemContent{
\ifVerboseLocation This is Integration Compute Question 0005. \\ \fi
\begin{problem}

Given the piecewise function 
\[f(x)=\left\{\begin{array}{ll}
{3 \, x + 5}\; , & x\leq{5}\\[3pt]
{-\frac{40}{3} \, x + \frac{260}{3}}\; , & {5}< x< {8}\\[3pt]
{3 \, x - 44}\; , & x\geq{8}
\end{array} \right.\]
evaluate the following definite integral by interpreting it in terms of areas.  

\input{Integral-Compute-0005.HELP.tex}

\[
\int_{0}^{13} f(x)\;dx= \answer{0}
\]  
\end{problem}}%}

\latexProblemContent{
\ifVerboseLocation This is Integration Compute Question 0005. \\ \fi
\begin{problem}

Given the piecewise function 
\[f(x)=\left\{\begin{array}{ll}
{4 \, x + 3}\; , & x\leq{-4}\\[3pt]
{\frac{26}{7} \, x + \frac{13}{7}}\; , & {-4}< x< {3}\\[3pt]
{x + 10}\; , & x\geq{3}
\end{array} \right.\]
evaluate the following definite integral by interpreting it in terms of areas.  

\input{Integral-Compute-0005.HELP.tex}

\[
\int_{-5}^{1} f(x)\;dx= \answer{-\frac{235}{7}}
\]  
\end{problem}}%}

\latexProblemContent{
\ifVerboseLocation This is Integration Compute Question 0005. \\ \fi
\begin{problem}

Given the piecewise function 
\[f(x)=\left\{\begin{array}{ll}
{x + 5}\; , & x\leq{-3}\\[3pt]
{-x - 1}\; , & {-3}< x< {1}\\[3pt]
{4 \, x - 6}\; , & x\geq{1}
\end{array} \right.\]
evaluate the following definite integral by interpreting it in terms of areas.  

\input{Integral-Compute-0005.HELP.tex}

\[
\int_{-4}^{4} f(x)\;dx= \answer{\frac{27}{2}}
\]  
\end{problem}}%}

\latexProblemContent{
\ifVerboseLocation This is Integration Compute Question 0005. \\ \fi
\begin{problem}

Given the piecewise function 
\[f(x)=\left\{\begin{array}{ll}
{x + 3}\; , & x\leq{5}\\[3pt]
{-\frac{16}{5} \, x + 24}\; , & {5}< x< {10}\\[3pt]
{2 \, x - 28}\; , & x\geq{10}
\end{array} \right.\]
evaluate the following definite integral by interpreting it in terms of areas.  

\input{Integral-Compute-0005.HELP.tex}

\[
\int_{1}^{8} f(x)\;dx= \answer{\frac{168}{5}}
\]  
\end{problem}}%}

\latexProblemContent{
\ifVerboseLocation This is Integration Compute Question 0005. \\ \fi
\begin{problem}

Given the piecewise function 
\[f(x)=\left\{\begin{array}{ll}
{2 \, x + 1}\; , & x\leq{5}\\[3pt]
{-\frac{22}{7} \, x + \frac{187}{7}}\; , & {5}< x< {12}\\[3pt]
{x - 23}\; , & x\geq{12}
\end{array} \right.\]
evaluate the following definite integral by interpreting it in terms of areas.  

\input{Integral-Compute-0005.HELP.tex}

\[
\int_{1}^{12} f(x)\;dx= \answer{28}
\]  
\end{problem}}%}

\latexProblemContent{
\ifVerboseLocation This is Integration Compute Question 0005. \\ \fi
\begin{problem}

Given the piecewise function 
\[f(x)=\left\{\begin{array}{ll}
{3 \, x + 1}\; , & x\leq{-2}\\[3pt]
{\frac{10}{3} \, x + \frac{5}{3}}\; , & {-2}< x< {1}\\[3pt]
{3 \, x + 2}\; , & x\geq{1}
\end{array} \right.\]
evaluate the following definite integral by interpreting it in terms of areas.  

\input{Integral-Compute-0005.HELP.tex}

\[
\int_{-2}^{1} f(x)\;dx= \answer{0}
\]  
\end{problem}}%}

\latexProblemContent{
\ifVerboseLocation This is Integration Compute Question 0005. \\ \fi
\begin{problem}

Given the piecewise function 
\[f(x)=\left\{\begin{array}{ll}
{2 \, x + 2}\; , & x\leq{-2}\\[3pt]
{\frac{4}{5} \, x - \frac{2}{5}}\; , & {-2}< x< {3}\\[3pt]
{4 \, x - 10}\; , & x\geq{3}
\end{array} \right.\]
evaluate the following definite integral by interpreting it in terms of areas.  

\input{Integral-Compute-0005.HELP.tex}

\[
\int_{-3}^{6} f(x)\;dx= \answer{21}
\]  
\end{problem}}%}

\latexProblemContent{
\ifVerboseLocation This is Integration Compute Question 0005. \\ \fi
\begin{problem}

Given the piecewise function 
\[f(x)=\left\{\begin{array}{ll}
{2 \, x + 5}\; , & x\leq{-3}\\[3pt]
{\frac{2}{5} \, x + \frac{1}{5}}\; , & {-3}< x< {2}\\[3pt]
{x - 1}\; , & x\geq{2}
\end{array} \right.\]
evaluate the following definite integral by interpreting it in terms of areas.  

\input{Integral-Compute-0005.HELP.tex}

\[
\int_{-3}^{3} f(x)\;dx= \answer{\frac{3}{2}}
\]  
\end{problem}}%}

\latexProblemContent{
\ifVerboseLocation This is Integration Compute Question 0005. \\ \fi
\begin{problem}

Given the piecewise function 
\[f(x)=\left\{\begin{array}{ll}
{2 \, x + 1}\; , & x\leq{2}\\[3pt]
{-\frac{10}{3} \, x + \frac{35}{3}}\; , & {2}< x< {5}\\[3pt]
{3 \, x - 20}\; , & x\geq{5}
\end{array} \right.\]
evaluate the following definite integral by interpreting it in terms of areas.  

\input{Integral-Compute-0005.HELP.tex}

\[
\int_{-3}^{6} f(x)\;dx= \answer{-\frac{7}{2}}
\]  
\end{problem}}%}

\latexProblemContent{
\ifVerboseLocation This is Integration Compute Question 0005. \\ \fi
\begin{problem}

Given the piecewise function 
\[f(x)=\left\{\begin{array}{ll}
{3 \, x + 1}\; , & x\leq{1}\\[3pt]
{-x + 5}\; , & {1}< x< {9}\\[3pt]
{4 \, x - 40}\; , & x\geq{9}
\end{array} \right.\]
evaluate the following definite integral by interpreting it in terms of areas.  

\input{Integral-Compute-0005.HELP.tex}

\[
\int_{-3}^{9} f(x)\;dx= \answer{-8}
\]  
\end{problem}}%}

\latexProblemContent{
\ifVerboseLocation This is Integration Compute Question 0005. \\ \fi
\begin{problem}

Given the piecewise function 
\[f(x)=\left\{\begin{array}{ll}
{x + 4}\; , & x\leq{2}\\[3pt]
{-\frac{12}{7} \, x + \frac{66}{7}}\; , & {2}< x< {9}\\[3pt]
{3 \, x - 33}\; , & x\geq{9}
\end{array} \right.\]
evaluate the following definite integral by interpreting it in terms of areas.  

\input{Integral-Compute-0005.HELP.tex}

\[
\int_{0}^{12} f(x)\;dx= \answer{\frac{11}{2}}
\]  
\end{problem}}%}

\latexProblemContent{
\ifVerboseLocation This is Integration Compute Question 0005. \\ \fi
\begin{problem}

Given the piecewise function 
\[f(x)=\left\{\begin{array}{ll}
{4 \, x + 5}\; , & x\leq{-1}\\[3pt]
{-\frac{2}{7} \, x + \frac{5}{7}}\; , & {-1}< x< {6}\\[3pt]
{4 \, x - 25}\; , & x\geq{6}
\end{array} \right.\]
evaluate the following definite integral by interpreting it in terms of areas.  

\input{Integral-Compute-0005.HELP.tex}

\[
\int_{-6}^{11} f(x)\;dx= \answer{0}
\]  
\end{problem}}%}

\latexProblemContent{
\ifVerboseLocation This is Integration Compute Question 0005. \\ \fi
\begin{problem}

Given the piecewise function 
\[f(x)=\left\{\begin{array}{ll}
{x + 1}\; , & x\leq{5}\\[3pt]
{-4 \, x + 26}\; , & {5}< x< {8}\\[3pt]
{2 \, x - 22}\; , & x\geq{8}
\end{array} \right.\]
evaluate the following definite integral by interpreting it in terms of areas.  

\input{Integral-Compute-0005.HELP.tex}

\[
\int_{3}^{11} f(x)\;dx= \answer{1}
\]  
\end{problem}}%}

\latexProblemContent{
\ifVerboseLocation This is Integration Compute Question 0005. \\ \fi
\begin{problem}

Given the piecewise function 
\[f(x)=\left\{\begin{array}{ll}
{x + 3}\; , & x\leq{-5}\\[3pt]
{\frac{1}{2} \, x + \frac{1}{2}}\; , & {-5}< x< {3}\\[3pt]
{4 \, x - 10}\; , & x\geq{3}
\end{array} \right.\]
evaluate the following definite integral by interpreting it in terms of areas.  

\input{Integral-Compute-0005.HELP.tex}

\[
\int_{-8}^{3} f(x)\;dx= \answer{-\frac{21}{2}}
\]  
\end{problem}}%}

\latexProblemContent{
\ifVerboseLocation This is Integration Compute Question 0005. \\ \fi
\begin{problem}

Given the piecewise function 
\[f(x)=\left\{\begin{array}{ll}
{x + 1}\; , & x\leq{2}\\[3pt]
{-\frac{6}{7} \, x + \frac{33}{7}}\; , & {2}< x< {9}\\[3pt]
{4 \, x - 39}\; , & x\geq{9}
\end{array} \right.\]
evaluate the following definite integral by interpreting it in terms of areas.  

\input{Integral-Compute-0005.HELP.tex}

\[
\int_{2}^{7} f(x)\;dx= \answer{\frac{30}{7}}
\]  
\end{problem}}%}

\latexProblemContent{
\ifVerboseLocation This is Integration Compute Question 0005. \\ \fi
\begin{problem}

Given the piecewise function 
\[f(x)=\left\{\begin{array}{ll}
{4 \, x + 3}\; , & x\leq{2}\\[3pt]
{-\frac{22}{3} \, x + \frac{77}{3}}\; , & {2}< x< {5}\\[3pt]
{2 \, x - 21}\; , & x\geq{5}
\end{array} \right.\]
evaluate the following definite integral by interpreting it in terms of areas.  

\input{Integral-Compute-0005.HELP.tex}

\[
\int_{0}^{7} f(x)\;dx= \answer{-4}
\]  
\end{problem}}%}

\latexProblemContent{
\ifVerboseLocation This is Integration Compute Question 0005. \\ \fi
\begin{problem}

Given the piecewise function 
\[f(x)=\left\{\begin{array}{ll}
{4 \, x + 1}\; , & x\leq{2}\\[3pt]
{-\frac{18}{5} \, x + \frac{81}{5}}\; , & {2}< x< {7}\\[3pt]
{4 \, x - 37}\; , & x\geq{7}
\end{array} \right.\]
evaluate the following definite integral by interpreting it in terms of areas.  

\input{Integral-Compute-0005.HELP.tex}

\[
\int_{0}^{11} f(x)\;dx= \answer{6}
\]  
\end{problem}}%}

\latexProblemContent{
\ifVerboseLocation This is Integration Compute Question 0005. \\ \fi
\begin{problem}

Given the piecewise function 
\[f(x)=\left\{\begin{array}{ll}
{2 \, x + 2}\; , & x\leq{-2}\\[3pt]
{\frac{1}{2} \, x - 1}\; , & {-2}< x< {6}\\[3pt]
{2 \, x - 10}\; , & x\geq{6}
\end{array} \right.\]
evaluate the following definite integral by interpreting it in terms of areas.  

\input{Integral-Compute-0005.HELP.tex}

\[
\int_{-6}^{11} f(x)\;dx= \answer{11}
\]  
\end{problem}}%}

\latexProblemContent{
\ifVerboseLocation This is Integration Compute Question 0005. \\ \fi
\begin{problem}

Given the piecewise function 
\[f(x)=\left\{\begin{array}{ll}
{x + 5}\; , & x\leq{2}\\[3pt]
{-\frac{14}{5} \, x + \frac{63}{5}}\; , & {2}< x< {7}\\[3pt]
{2 \, x - 21}\; , & x\geq{7}
\end{array} \right.\]
evaluate the following definite integral by interpreting it in terms of areas.  

\input{Integral-Compute-0005.HELP.tex}

\[
\int_{0}^{12} f(x)\;dx= \answer{2}
\]  
\end{problem}}%}

\latexProblemContent{
\ifVerboseLocation This is Integration Compute Question 0005. \\ \fi
\begin{problem}

Given the piecewise function 
\[f(x)=\left\{\begin{array}{ll}
{3 \, x + 1}\; , & x\leq{-2}\\[3pt]
{\frac{5}{2} \, x}\; , & {-2}< x< {2}\\[3pt]
{x + 3}\; , & x\geq{2}
\end{array} \right.\]
evaluate the following definite integral by interpreting it in terms of areas.  

\input{Integral-Compute-0005.HELP.tex}

\[
\int_{-7}^{6} f(x)\;dx= \answer{-\frac{69}{2}}
\]  
\end{problem}}%}

\latexProblemContent{
\ifVerboseLocation This is Integration Compute Question 0005. \\ \fi
\begin{problem}

Given the piecewise function 
\[f(x)=\left\{\begin{array}{ll}
{3 \, x + 5}\; , & x\leq{2}\\[3pt]
{-\frac{22}{3} \, x + \frac{77}{3}}\; , & {2}< x< {5}\\[3pt]
{2 \, x - 21}\; , & x\geq{5}
\end{array} \right.\]
evaluate the following definite integral by interpreting it in terms of areas.  

\input{Integral-Compute-0005.HELP.tex}

\[
\int_{-1}^{5} f(x)\;dx= \answer{\frac{39}{2}}
\]  
\end{problem}}%}

\latexProblemContent{
\ifVerboseLocation This is Integration Compute Question 0005. \\ \fi
\begin{problem}

Given the piecewise function 
\[f(x)=\left\{\begin{array}{ll}
{2 \, x + 2}\; , & x\leq{3}\\[3pt]
{-\frac{16}{3} \, x + 24}\; , & {3}< x< {6}\\[3pt]
{x - 14}\; , & x\geq{6}
\end{array} \right.\]
evaluate the following definite integral by interpreting it in terms of areas.  

\input{Integral-Compute-0005.HELP.tex}

\[
\int_{3}^{4} f(x)\;dx= \answer{\frac{16}{3}}
\]  
\end{problem}}%}

\latexProblemContent{
\ifVerboseLocation This is Integration Compute Question 0005. \\ \fi
\begin{problem}

Given the piecewise function 
\[f(x)=\left\{\begin{array}{ll}
{2 \, x + 2}\; , & x\leq{-1}\\[3pt]
{0}\; , & {-1}< x< {7}\\[3pt]
{x - 7}\; , & x\geq{7}
\end{array} \right.\]
evaluate the following definite integral by interpreting it in terms of areas.  

\input{Integral-Compute-0005.HELP.tex}

\[
\int_{-2}^{1} f(x)\;dx= \answer{-1}
\]  
\end{problem}}%}

\latexProblemContent{
\ifVerboseLocation This is Integration Compute Question 0005. \\ \fi
\begin{problem}

Given the piecewise function 
\[f(x)=\left\{\begin{array}{ll}
{4 \, x + 3}\; , & x\leq{-2}\\[3pt]
{\frac{10}{3} \, x + \frac{5}{3}}\; , & {-2}< x< {1}\\[3pt]
{2 \, x + 3}\; , & x\geq{1}
\end{array} \right.\]
evaluate the following definite integral by interpreting it in terms of areas.  

\input{Integral-Compute-0005.HELP.tex}

\[
\int_{-4}^{-1} f(x)\;dx= \answer{-\frac{64}{3}}
\]  
\end{problem}}%}

\latexProblemContent{
\ifVerboseLocation This is Integration Compute Question 0005. \\ \fi
\begin{problem}

Given the piecewise function 
\[f(x)=\left\{\begin{array}{ll}
{2 \, x + 5}\; , & x\leq{-4}\\[3pt]
{\frac{6}{5} \, x + \frac{9}{5}}\; , & {-4}< x< {1}\\[3pt]
{4 \, x - 1}\; , & x\geq{1}
\end{array} \right.\]
evaluate the following definite integral by interpreting it in terms of areas.  

\input{Integral-Compute-0005.HELP.tex}

\[
\int_{-9}^{-1} f(x)\;dx= \answer{-\frac{218}{5}}
\]  
\end{problem}}%}

\latexProblemContent{
\ifVerboseLocation This is Integration Compute Question 0005. \\ \fi
\begin{problem}

Given the piecewise function 
\[f(x)=\left\{\begin{array}{ll}
{2 \, x + 5}\; , & x\leq{5}\\[3pt]
{-6 \, x + 45}\; , & {5}< x< {10}\\[3pt]
{3 \, x - 45}\; , & x\geq{10}
\end{array} \right.\]
evaluate the following definite integral by interpreting it in terms of areas.  

\input{Integral-Compute-0005.HELP.tex}

\[
\int_{3}^{6} f(x)\;dx= \answer{38}
\]  
\end{problem}}%}

\latexProblemContent{
\ifVerboseLocation This is Integration Compute Question 0005. \\ \fi
\begin{problem}

Given the piecewise function 
\[f(x)=\left\{\begin{array}{ll}
{3 \, x + 5}\; , & x\leq{3}\\[3pt]
{-\frac{7}{2} \, x + \frac{49}{2}}\; , & {3}< x< {11}\\[3pt]
{2 \, x - 36}\; , & x\geq{11}
\end{array} \right.\]
evaluate the following definite integral by interpreting it in terms of areas.  

\input{Integral-Compute-0005.HELP.tex}

\[
\int_{1}^{5} f(x)\;dx= \answer{43}
\]  
\end{problem}}%}

\latexProblemContent{
\ifVerboseLocation This is Integration Compute Question 0005. \\ \fi
\begin{problem}

Given the piecewise function 
\[f(x)=\left\{\begin{array}{ll}
{3 \, x + 2}\; , & x\leq{1}\\[3pt]
{-\frac{10}{3} \, x + \frac{25}{3}}\; , & {1}< x< {4}\\[3pt]
{3 \, x - 17}\; , & x\geq{4}
\end{array} \right.\]
evaluate the following definite integral by interpreting it in terms of areas.  

\input{Integral-Compute-0005.HELP.tex}

\[
\int_{-3}^{2} f(x)\;dx= \answer{-\frac{2}{3}}
\]  
\end{problem}}%}

\latexProblemContent{
\ifVerboseLocation This is Integration Compute Question 0005. \\ \fi
\begin{problem}

Given the piecewise function 
\[f(x)=\left\{\begin{array}{ll}
{3 \, x + 2}\; , & x\leq{-2}\\[3pt]
{\frac{4}{3} \, x - \frac{4}{3}}\; , & {-2}< x< {4}\\[3pt]
{x}\; , & x\geq{4}
\end{array} \right.\]
evaluate the following definite integral by interpreting it in terms of areas.  

\input{Integral-Compute-0005.HELP.tex}

\[
\int_{-4}^{0} f(x)\;dx= \answer{-\frac{58}{3}}
\]  
\end{problem}}%}

\latexProblemContent{
\ifVerboseLocation This is Integration Compute Question 0005. \\ \fi
\begin{problem}

Given the piecewise function 
\[f(x)=\left\{\begin{array}{ll}
{2 \, x + 4}\; , & x\leq{-2}\\[3pt]
{0}\; , & {-2}< x< {2}\\[3pt]
{3 \, x - 6}\; , & x\geq{2}
\end{array} \right.\]
evaluate the following definite integral by interpreting it in terms of areas.  

\input{Integral-Compute-0005.HELP.tex}

\[
\int_{-3}^{5} f(x)\;dx= \answer{\frac{25}{2}}
\]  
\end{problem}}%}

\latexProblemContent{
\ifVerboseLocation This is Integration Compute Question 0005. \\ \fi
\begin{problem}

Given the piecewise function 
\[f(x)=\left\{\begin{array}{ll}
{2 \, x + 5}\; , & x\leq{-3}\\[3pt]
{\frac{1}{4} \, x - \frac{1}{4}}\; , & {-3}< x< {5}\\[3pt]
{2 \, x - 9}\; , & x\geq{5}
\end{array} \right.\]
evaluate the following definite integral by interpreting it in terms of areas.  

\input{Integral-Compute-0005.HELP.tex}

\[
\int_{-5}^{8} f(x)\;dx= \answer{6}
\]  
\end{problem}}%}

\latexProblemContent{
\ifVerboseLocation This is Integration Compute Question 0005. \\ \fi
\begin{problem}

Given the piecewise function 
\[f(x)=\left\{\begin{array}{ll}
{4 \, x + 4}\; , & x\leq{4}\\[3pt]
{-\frac{40}{7} \, x + \frac{300}{7}}\; , & {4}< x< {11}\\[3pt]
{3 \, x - 53}\; , & x\geq{11}
\end{array} \right.\]
evaluate the following definite integral by interpreting it in terms of areas.  

\input{Integral-Compute-0005.HELP.tex}

\[
\int_{3}^{12} f(x)\;dx= \answer{-\frac{1}{2}}
\]  
\end{problem}}%}

\latexProblemContent{
\ifVerboseLocation This is Integration Compute Question 0005. \\ \fi
\begin{problem}

Given the piecewise function 
\[f(x)=\left\{\begin{array}{ll}
{x + 4}\; , & x\leq{-1}\\[3pt]
{-x + 2}\; , & {-1}< x< {5}\\[3pt]
{4 \, x - 23}\; , & x\geq{5}
\end{array} \right.\]
evaluate the following definite integral by interpreting it in terms of areas.  

\input{Integral-Compute-0005.HELP.tex}

\[
\int_{-2}^{10} f(x)\;dx= \answer{\frac{75}{2}}
\]  
\end{problem}}%}

\latexProblemContent{
\ifVerboseLocation This is Integration Compute Question 0005. \\ \fi
\begin{problem}

Given the piecewise function 
\[f(x)=\left\{\begin{array}{ll}
{3 \, x + 4}\; , & x\leq{-2}\\[3pt]
{x}\; , & {-2}< x< {2}\\[3pt]
{4 \, x - 6}\; , & x\geq{2}
\end{array} \right.\]
evaluate the following definite integral by interpreting it in terms of areas.  

\input{Integral-Compute-0005.HELP.tex}

\[
\int_{-7}^{0} f(x)\;dx= \answer{-\frac{99}{2}}
\]  
\end{problem}}%}

\latexProblemContent{
\ifVerboseLocation This is Integration Compute Question 0005. \\ \fi
\begin{problem}

Given the piecewise function 
\[f(x)=\left\{\begin{array}{ll}
{x + 1}\; , & x\leq{-1}\\[3pt]
{0}\; , & {-1}< x< {5}\\[3pt]
{4 \, x - 20}\; , & x\geq{5}
\end{array} \right.\]
evaluate the following definite integral by interpreting it in terms of areas.  

\input{Integral-Compute-0005.HELP.tex}

\[
\int_{-4}^{8} f(x)\;dx= \answer{\frac{27}{2}}
\]  
\end{problem}}%}

\latexProblemContent{
\ifVerboseLocation This is Integration Compute Question 0005. \\ \fi
\begin{problem}

Given the piecewise function 
\[f(x)=\left\{\begin{array}{ll}
{2 \, x + 3}\; , & x\leq{4}\\[3pt]
{-\frac{11}{4} \, x + 22}\; , & {4}< x< {12}\\[3pt]
{x - 23}\; , & x\geq{12}
\end{array} \right.\]
evaluate the following definite integral by interpreting it in terms of areas.  

\input{Integral-Compute-0005.HELP.tex}

\[
\int_{0}^{16} f(x)\;dx= \answer{-8}
\]  
\end{problem}}%}

\latexProblemContent{
\ifVerboseLocation This is Integration Compute Question 0005. \\ \fi
\begin{problem}

Given the piecewise function 
\[f(x)=\left\{\begin{array}{ll}
{2 \, x + 2}\; , & x\leq{2}\\[3pt]
{-4 \, x + 14}\; , & {2}< x< {5}\\[3pt]
{3 \, x - 21}\; , & x\geq{5}
\end{array} \right.\]
evaluate the following definite integral by interpreting it in terms of areas.  

\input{Integral-Compute-0005.HELP.tex}

\[
\int_{-3}^{6} f(x)\;dx= \answer{\frac{1}{2}}
\]  
\end{problem}}%}

\latexProblemContent{
\ifVerboseLocation This is Integration Compute Question 0005. \\ \fi
\begin{problem}

Given the piecewise function 
\[f(x)=\left\{\begin{array}{ll}
{x + 4}\; , & x\leq{5}\\[3pt]
{-\frac{18}{7} \, x + \frac{153}{7}}\; , & {5}< x< {12}\\[3pt]
{x - 21}\; , & x\geq{12}
\end{array} \right.\]
evaluate the following definite integral by interpreting it in terms of areas.  

\input{Integral-Compute-0005.HELP.tex}

\[
\int_{4}^{7} f(x)\;dx= \answer{\frac{299}{14}}
\]  
\end{problem}}%}

\latexProblemContent{
\ifVerboseLocation This is Integration Compute Question 0005. \\ \fi
\begin{problem}

Given the piecewise function 
\[f(x)=\left\{\begin{array}{ll}
{4 \, x + 4}\; , & x\leq{-4}\\[3pt]
{\frac{24}{7} \, x + \frac{12}{7}}\; , & {-4}< x< {3}\\[3pt]
{4 \, x}\; , & x\geq{3}
\end{array} \right.\]
evaluate the following definite integral by interpreting it in terms of areas.  

\input{Integral-Compute-0005.HELP.tex}

\[
\int_{-9}^{3} f(x)\;dx= \answer{-110}
\]  
\end{problem}}%}

\latexProblemContent{
\ifVerboseLocation This is Integration Compute Question 0005. \\ \fi
\begin{problem}

Given the piecewise function 
\[f(x)=\left\{\begin{array}{ll}
{x + 3}\; , & x\leq{2}\\[3pt]
{-\frac{10}{3} \, x + \frac{35}{3}}\; , & {2}< x< {5}\\[3pt]
{4 \, x - 25}\; , & x\geq{5}
\end{array} \right.\]
evaluate the following definite integral by interpreting it in terms of areas.  

\input{Integral-Compute-0005.HELP.tex}

\[
\int_{2}^{5} f(x)\;dx= \answer{0}
\]  
\end{problem}}%}

\latexProblemContent{
\ifVerboseLocation This is Integration Compute Question 0005. \\ \fi
\begin{problem}

Given the piecewise function 
\[f(x)=\left\{\begin{array}{ll}
{2 \, x + 4}\; , & x\leq{-5}\\[3pt]
{\frac{3}{2} \, x + \frac{3}{2}}\; , & {-5}< x< {3}\\[3pt]
{x + 3}\; , & x\geq{3}
\end{array} \right.\]
evaluate the following definite integral by interpreting it in terms of areas.  

\input{Integral-Compute-0005.HELP.tex}

\[
\int_{-5}^{7} f(x)\;dx= \answer{32}
\]  
\end{problem}}%}

\latexProblemContent{
\ifVerboseLocation This is Integration Compute Question 0005. \\ \fi
\begin{problem}

Given the piecewise function 
\[f(x)=\left\{\begin{array}{ll}
{4 \, x + 3}\; , & x\leq{-2}\\[3pt]
{\frac{10}{7} \, x - \frac{15}{7}}\; , & {-2}< x< {5}\\[3pt]
{4 \, x - 15}\; , & x\geq{5}
\end{array} \right.\]
evaluate the following definite integral by interpreting it in terms of areas.  

\input{Integral-Compute-0005.HELP.tex}

\[
\int_{-4}^{10} f(x)\;dx= \answer{57}
\]  
\end{problem}}%}

\latexProblemContent{
\ifVerboseLocation This is Integration Compute Question 0005. \\ \fi
\begin{problem}

Given the piecewise function 
\[f(x)=\left\{\begin{array}{ll}
{4 \, x + 2}\; , & x\leq{-3}\\[3pt]
{\frac{5}{2} \, x - \frac{5}{2}}\; , & {-3}< x< {5}\\[3pt]
{3 \, x - 5}\; , & x\geq{5}
\end{array} \right.\]
evaluate the following definite integral by interpreting it in terms of areas.  

\input{Integral-Compute-0005.HELP.tex}

\[
\int_{-3}^{4} f(x)\;dx= \answer{-\frac{35}{4}}
\]  
\end{problem}}%}

\latexProblemContent{
\ifVerboseLocation This is Integration Compute Question 0005. \\ \fi
\begin{problem}

Given the piecewise function 
\[f(x)=\left\{\begin{array}{ll}
{3 \, x + 4}\; , & x\leq{-2}\\[3pt]
{\frac{2}{3} \, x - \frac{2}{3}}\; , & {-2}< x< {4}\\[3pt]
{4 \, x - 14}\; , & x\geq{4}
\end{array} \right.\]
evaluate the following definite integral by interpreting it in terms of areas.  

\input{Integral-Compute-0005.HELP.tex}

\[
\int_{-2}^{-1} f(x)\;dx= \answer{-\frac{5}{3}}
\]  
\end{problem}}%}

\latexProblemContent{
\ifVerboseLocation This is Integration Compute Question 0005. \\ \fi
\begin{problem}

Given the piecewise function 
\[f(x)=\left\{\begin{array}{ll}
{x + 1}\; , & x\leq{1}\\[3pt]
{-\frac{1}{2} \, x + \frac{5}{2}}\; , & {1}< x< {9}\\[3pt]
{3 \, x - 29}\; , & x\geq{9}
\end{array} \right.\]
evaluate the following definite integral by interpreting it in terms of areas.  

\input{Integral-Compute-0005.HELP.tex}

\[
\int_{0}^{3} f(x)\;dx= \answer{\frac{9}{2}}
\]  
\end{problem}}%}

\latexProblemContent{
\ifVerboseLocation This is Integration Compute Question 0005. \\ \fi
\begin{problem}

Given the piecewise function 
\[f(x)=\left\{\begin{array}{ll}
{4 \, x + 4}\; , & x\leq{-2}\\[3pt]
{\frac{8}{5} \, x - \frac{4}{5}}\; , & {-2}< x< {3}\\[3pt]
{3 \, x - 5}\; , & x\geq{3}
\end{array} \right.\]
evaluate the following definite integral by interpreting it in terms of areas.  

\input{Integral-Compute-0005.HELP.tex}

\[
\int_{-7}^{4} f(x)\;dx= \answer{-\frac{129}{2}}
\]  
\end{problem}}%}

\latexProblemContent{
\ifVerboseLocation This is Integration Compute Question 0005. \\ \fi
\begin{problem}

Given the piecewise function 
\[f(x)=\left\{\begin{array}{ll}
{3 \, x + 4}\; , & x\leq{-2}\\[3pt]
{\frac{4}{5} \, x - \frac{2}{5}}\; , & {-2}< x< {3}\\[3pt]
{2 \, x - 4}\; , & x\geq{3}
\end{array} \right.\]
evaluate the following definite integral by interpreting it in terms of areas.  

\input{Integral-Compute-0005.HELP.tex}

\[
\int_{-6}^{1} f(x)\;dx= \answer{-\frac{172}{5}}
\]  
\end{problem}}%}

\latexProblemContent{
\ifVerboseLocation This is Integration Compute Question 0005. \\ \fi
\begin{problem}

Given the piecewise function 
\[f(x)=\left\{\begin{array}{ll}
{4 \, x + 1}\; , & x\leq{3}\\[3pt]
{-\frac{26}{3} \, x + 39}\; , & {3}< x< {6}\\[3pt]
{x - 19}\; , & x\geq{6}
\end{array} \right.\]
evaluate the following definite integral by interpreting it in terms of areas.  

\input{Integral-Compute-0005.HELP.tex}

\[
\int_{1}^{5} f(x)\;dx= \answer{\frac{80}{3}}
\]  
\end{problem}}%}

\latexProblemContent{
\ifVerboseLocation This is Integration Compute Question 0005. \\ \fi
\begin{problem}

Given the piecewise function 
\[f(x)=\left\{\begin{array}{ll}
{3 \, x + 3}\; , & x\leq{5}\\[3pt]
{-9 \, x + 63}\; , & {5}< x< {9}\\[3pt]
{4 \, x - 54}\; , & x\geq{9}
\end{array} \right.\]
evaluate the following definite integral by interpreting it in terms of areas.  

\input{Integral-Compute-0005.HELP.tex}

\[
\int_{0}^{6} f(x)\;dx= \answer{66}
\]  
\end{problem}}%}

\latexProblemContent{
\ifVerboseLocation This is Integration Compute Question 0005. \\ \fi
\begin{problem}

Given the piecewise function 
\[f(x)=\left\{\begin{array}{ll}
{2 \, x + 4}\; , & x\leq{1}\\[3pt]
{-2 \, x + 8}\; , & {1}< x< {7}\\[3pt]
{x - 13}\; , & x\geq{7}
\end{array} \right.\]
evaluate the following definite integral by interpreting it in terms of areas.  

\input{Integral-Compute-0005.HELP.tex}

\[
\int_{0}^{2} f(x)\;dx= \answer{10}
\]  
\end{problem}}%}

\latexProblemContent{
\ifVerboseLocation This is Integration Compute Question 0005. \\ \fi
\begin{problem}

Given the piecewise function 
\[f(x)=\left\{\begin{array}{ll}
{x + 3}\; , & x\leq{-1}\\[3pt]
{-\frac{2}{3} \, x + \frac{4}{3}}\; , & {-1}< x< {5}\\[3pt]
{2 \, x - 12}\; , & x\geq{5}
\end{array} \right.\]
evaluate the following definite integral by interpreting it in terms of areas.  

\input{Integral-Compute-0005.HELP.tex}

\[
\int_{-2}^{4} f(x)\;dx= \answer{\frac{19}{6}}
\]  
\end{problem}}%}

\latexProblemContent{
\ifVerboseLocation This is Integration Compute Question 0005. \\ \fi
\begin{problem}

Given the piecewise function 
\[f(x)=\left\{\begin{array}{ll}
{x + 4}\; , & x\leq{5}\\[3pt]
{-\frac{9}{4} \, x + \frac{81}{4}}\; , & {5}< x< {13}\\[3pt]
{4 \, x - 61}\; , & x\geq{13}
\end{array} \right.\]
evaluate the following definite integral by interpreting it in terms of areas.  

\input{Integral-Compute-0005.HELP.tex}

\[
\int_{4}^{18} f(x)\;dx= \answer{\frac{27}{2}}
\]  
\end{problem}}%}

\latexProblemContent{
\ifVerboseLocation This is Integration Compute Question 0005. \\ \fi
\begin{problem}

Given the piecewise function 
\[f(x)=\left\{\begin{array}{ll}
{3 \, x + 1}\; , & x\leq{1}\\[3pt]
{-\frac{8}{5} \, x + \frac{28}{5}}\; , & {1}< x< {6}\\[3pt]
{2 \, x - 16}\; , & x\geq{6}
\end{array} \right.\]
evaluate the following definite integral by interpreting it in terms of areas.  

\input{Integral-Compute-0005.HELP.tex}

\[
\int_{1}^{10} f(x)\;dx= \answer{0}
\]  
\end{problem}}%}

\latexProblemContent{
\ifVerboseLocation This is Integration Compute Question 0005. \\ \fi
\begin{problem}

Given the piecewise function 
\[f(x)=\left\{\begin{array}{ll}
{4 \, x + 2}\; , & x\leq{1}\\[3pt]
{-\frac{3}{2} \, x + \frac{15}{2}}\; , & {1}< x< {9}\\[3pt]
{4 \, x - 42}\; , & x\geq{9}
\end{array} \right.\]
evaluate the following definite integral by interpreting it in terms of areas.  

\input{Integral-Compute-0005.HELP.tex}

\[
\int_{-4}^{7} f(x)\;dx= \answer{-11}
\]  
\end{problem}}%}

\latexProblemContent{
\ifVerboseLocation This is Integration Compute Question 0005. \\ \fi
\begin{problem}

Given the piecewise function 
\[f(x)=\left\{\begin{array}{ll}
{3 \, x + 2}\; , & x\leq{-4}\\[3pt]
{4 \, x + 6}\; , & {-4}< x< {1}\\[3pt]
{2 \, x + 8}\; , & x\geq{1}
\end{array} \right.\]
evaluate the following definite integral by interpreting it in terms of areas.  

\input{Integral-Compute-0005.HELP.tex}

\[
\int_{-6}^{6} f(x)\;dx= \answer{49}
\]  
\end{problem}}%}

\latexProblemContent{
\ifVerboseLocation This is Integration Compute Question 0005. \\ \fi
\begin{problem}

Given the piecewise function 
\[f(x)=\left\{\begin{array}{ll}
{x + 5}\; , & x\leq{-2}\\[3pt]
{-2 \, x - 1}\; , & {-2}< x< {1}\\[3pt]
{x - 4}\; , & x\geq{1}
\end{array} \right.\]
evaluate the following definite integral by interpreting it in terms of areas.  

\input{Integral-Compute-0005.HELP.tex}

\[
\int_{-6}^{2} f(x)\;dx= \answer{\frac{3}{2}}
\]  
\end{problem}}%}

\latexProblemContent{
\ifVerboseLocation This is Integration Compute Question 0005. \\ \fi
\begin{problem}

Given the piecewise function 
\[f(x)=\left\{\begin{array}{ll}
{4 \, x + 1}\; , & x\leq{-3}\\[3pt]
{\frac{11}{4} \, x - \frac{11}{4}}\; , & {-3}< x< {5}\\[3pt]
{3 \, x - 4}\; , & x\geq{5}
\end{array} \right.\]
evaluate the following definite integral by interpreting it in terms of areas.  

\input{Integral-Compute-0005.HELP.tex}

\[
\int_{-5}^{9} f(x)\;dx= \answer{38}
\]  
\end{problem}}%}

\latexProblemContent{
\ifVerboseLocation This is Integration Compute Question 0005. \\ \fi
\begin{problem}

Given the piecewise function 
\[f(x)=\left\{\begin{array}{ll}
{4 \, x + 4}\; , & x\leq{1}\\[3pt]
{-\frac{16}{3} \, x + \frac{40}{3}}\; , & {1}< x< {4}\\[3pt]
{4 \, x - 24}\; , & x\geq{4}
\end{array} \right.\]
evaluate the following definite integral by interpreting it in terms of areas.  

\input{Integral-Compute-0005.HELP.tex}

\[
\int_{-1}^{9} f(x)\;dx= \answer{18}
\]  
\end{problem}}%}

\latexProblemContent{
\ifVerboseLocation This is Integration Compute Question 0005. \\ \fi
\begin{problem}

Given the piecewise function 
\[f(x)=\left\{\begin{array}{ll}
{4 \, x + 5}\; , & x\leq{-3}\\[3pt]
{\frac{14}{5} \, x + \frac{7}{5}}\; , & {-3}< x< {2}\\[3pt]
{x + 5}\; , & x\geq{2}
\end{array} \right.\]
evaluate the following definite integral by interpreting it in terms of areas.  

\input{Integral-Compute-0005.HELP.tex}

\[
\int_{-6}^{4} f(x)\;dx= \answer{-23}
\]  
\end{problem}}%}

\latexProblemContent{
\ifVerboseLocation This is Integration Compute Question 0005. \\ \fi
\begin{problem}

Given the piecewise function 
\[f(x)=\left\{\begin{array}{ll}
{3 \, x + 5}\; , & x\leq{1}\\[3pt]
{-\frac{16}{5} \, x + \frac{56}{5}}\; , & {1}< x< {6}\\[3pt]
{4 \, x - 32}\; , & x\geq{6}
\end{array} \right.\]
evaluate the following definite integral by interpreting it in terms of areas.  

\input{Integral-Compute-0005.HELP.tex}

\[
\int_{-3}^{11} f(x)\;dx= \answer{18}
\]  
\end{problem}}%}

\latexProblemContent{
\ifVerboseLocation This is Integration Compute Question 0005. \\ \fi
\begin{problem}

Given the piecewise function 
\[f(x)=\left\{\begin{array}{ll}
{2 \, x + 1}\; , & x\leq{-3}\\[3pt]
{\frac{5}{4} \, x - \frac{5}{4}}\; , & {-3}< x< {5}\\[3pt]
{4 \, x - 15}\; , & x\geq{5}
\end{array} \right.\]
evaluate the following definite integral by interpreting it in terms of areas.  

\input{Integral-Compute-0005.HELP.tex}

\[
\int_{-7}^{9} f(x)\;dx= \answer{16}
\]  
\end{problem}}%}

\latexProblemContent{
\ifVerboseLocation This is Integration Compute Question 0005. \\ \fi
\begin{problem}

Given the piecewise function 
\[f(x)=\left\{\begin{array}{ll}
{x + 4}\; , & x\leq{3}\\[3pt]
{-\frac{14}{3} \, x + 21}\; , & {3}< x< {6}\\[3pt]
{3 \, x - 25}\; , & x\geq{6}
\end{array} \right.\]
evaluate the following definite integral by interpreting it in terms of areas.  

\input{Integral-Compute-0005.HELP.tex}

\[
\int_{2}^{5} f(x)\;dx= \answer{\frac{67}{6}}
\]  
\end{problem}}%}

\latexProblemContent{
\ifVerboseLocation This is Integration Compute Question 0005. \\ \fi
\begin{problem}

Given the piecewise function 
\[f(x)=\left\{\begin{array}{ll}
{3 \, x + 1}\; , & x\leq{4}\\[3pt]
{-\frac{13}{4} \, x + 26}\; , & {4}< x< {12}\\[3pt]
{2 \, x - 37}\; , & x\geq{12}
\end{array} \right.\]
evaluate the following definite integral by interpreting it in terms of areas.  

\input{Integral-Compute-0005.HELP.tex}

\[
\int_{1}^{15} f(x)\;dx= \answer{-\frac{9}{2}}
\]  
\end{problem}}%}

\latexProblemContent{
\ifVerboseLocation This is Integration Compute Question 0005. \\ \fi
\begin{problem}

Given the piecewise function 
\[f(x)=\left\{\begin{array}{ll}
{3 \, x + 4}\; , & x\leq{-3}\\[3pt]
{\frac{5}{4} \, x - \frac{5}{4}}\; , & {-3}< x< {5}\\[3pt]
{3 \, x - 10}\; , & x\geq{5}
\end{array} \right.\]
evaluate the following definite integral by interpreting it in terms of areas.  

\input{Integral-Compute-0005.HELP.tex}

\[
\int_{-6}^{4} f(x)\;dx= \answer{-\frac{263}{8}}
\]  
\end{problem}}%}

\latexProblemContent{
\ifVerboseLocation This is Integration Compute Question 0005. \\ \fi
\begin{problem}

Given the piecewise function 
\[f(x)=\left\{\begin{array}{ll}
{x + 2}\; , & x\leq{-3}\\[3pt]
{\frac{2}{5} \, x + \frac{1}{5}}\; , & {-3}< x< {2}\\[3pt]
{2 \, x - 3}\; , & x\geq{2}
\end{array} \right.\]
evaluate the following definite integral by interpreting it in terms of areas.  

\input{Integral-Compute-0005.HELP.tex}

\[
\int_{-7}^{-2} f(x)\;dx= \answer{-\frac{64}{5}}
\]  
\end{problem}}%}

\latexProblemContent{
\ifVerboseLocation This is Integration Compute Question 0005. \\ \fi
\begin{problem}

Given the piecewise function 
\[f(x)=\left\{\begin{array}{ll}
{4 \, x + 4}\; , & x\leq{4}\\[3pt]
{-\frac{40}{7} \, x + \frac{300}{7}}\; , & {4}< x< {11}\\[3pt]
{3 \, x - 53}\; , & x\geq{11}
\end{array} \right.\]
evaluate the following definite integral by interpreting it in terms of areas.  

\input{Integral-Compute-0005.HELP.tex}

\[
\int_{2}^{13} f(x)\;dx= \answer{-2}
\]  
\end{problem}}%}

\latexProblemContent{
\ifVerboseLocation This is Integration Compute Question 0005. \\ \fi
\begin{problem}

Given the piecewise function 
\[f(x)=\left\{\begin{array}{ll}
{x + 5}\; , & x\leq{5}\\[3pt]
{-\frac{20}{7} \, x + \frac{170}{7}}\; , & {5}< x< {12}\\[3pt]
{4 \, x - 58}\; , & x\geq{12}
\end{array} \right.\]
evaluate the following definite integral by interpreting it in terms of areas.  

\input{Integral-Compute-0005.HELP.tex}

\[
\int_{5}^{12} f(x)\;dx= \answer{0}
\]  
\end{problem}}%}

\latexProblemContent{
\ifVerboseLocation This is Integration Compute Question 0005. \\ \fi
\begin{problem}

Given the piecewise function 
\[f(x)=\left\{\begin{array}{ll}
{3 \, x + 5}\; , & x\leq{-3}\\[3pt]
{\frac{8}{3} \, x + 4}\; , & {-3}< x< {0}\\[3pt]
{3 \, x + 4}\; , & x\geq{0}
\end{array} \right.\]
evaluate the following definite integral by interpreting it in terms of areas.  

\input{Integral-Compute-0005.HELP.tex}

\[
\int_{-5}^{0} f(x)\;dx= \answer{-14}
\]  
\end{problem}}%}

\latexProblemContent{
\ifVerboseLocation This is Integration Compute Question 0005. \\ \fi
\begin{problem}

Given the piecewise function 
\[f(x)=\left\{\begin{array}{ll}
{4 \, x + 1}\; , & x\leq{-1}\\[3pt]
{2 \, x - 1}\; , & {-1}< x< {2}\\[3pt]
{x + 1}\; , & x\geq{2}
\end{array} \right.\]
evaluate the following definite integral by interpreting it in terms of areas.  

\input{Integral-Compute-0005.HELP.tex}

\[
\int_{-1}^{5} f(x)\;dx= \answer{\frac{27}{2}}
\]  
\end{problem}}%}

\latexProblemContent{
\ifVerboseLocation This is Integration Compute Question 0005. \\ \fi
\begin{problem}

Given the piecewise function 
\[f(x)=\left\{\begin{array}{ll}
{x + 3}\; , & x\leq{4}\\[3pt]
{-\frac{7}{3} \, x + \frac{49}{3}}\; , & {4}< x< {10}\\[3pt]
{3 \, x - 37}\; , & x\geq{10}
\end{array} \right.\]
evaluate the following definite integral by interpreting it in terms of areas.  

\input{Integral-Compute-0005.HELP.tex}

\[
\int_{-1}^{9} f(x)\;dx= \answer{\frac{85}{3}}
\]  
\end{problem}}%}

\latexProblemContent{
\ifVerboseLocation This is Integration Compute Question 0005. \\ \fi
\begin{problem}

Given the piecewise function 
\[f(x)=\left\{\begin{array}{ll}
{2 \, x + 1}\; , & x\leq{-3}\\[3pt]
{\frac{5}{3} \, x}\; , & {-3}< x< {3}\\[3pt]
{3 \, x - 4}\; , & x\geq{3}
\end{array} \right.\]
evaluate the following definite integral by interpreting it in terms of areas.  

\input{Integral-Compute-0005.HELP.tex}

\[
\int_{-7}^{-1} f(x)\;dx= \answer{-\frac{128}{3}}
\]  
\end{problem}}%}

\latexProblemContent{
\ifVerboseLocation This is Integration Compute Question 0005. \\ \fi
\begin{problem}

Given the piecewise function 
\[f(x)=\left\{\begin{array}{ll}
{2 \, x + 3}\; , & x\leq{-2}\\[3pt]
{\frac{1}{2} \, x}\; , & {-2}< x< {2}\\[3pt]
{4 \, x - 7}\; , & x\geq{2}
\end{array} \right.\]
evaluate the following definite integral by interpreting it in terms of areas.  

\input{Integral-Compute-0005.HELP.tex}

\[
\int_{-4}^{6} f(x)\;dx= \answer{30}
\]  
\end{problem}}%}

\latexProblemContent{
\ifVerboseLocation This is Integration Compute Question 0005. \\ \fi
\begin{problem}

Given the piecewise function 
\[f(x)=\left\{\begin{array}{ll}
{x + 3}\; , & x\leq{-5}\\[3pt]
{\frac{4}{5} \, x + 2}\; , & {-5}< x< {0}\\[3pt]
{x + 2}\; , & x\geq{0}
\end{array} \right.\]
evaluate the following definite integral by interpreting it in terms of areas.  

\input{Integral-Compute-0005.HELP.tex}

\[
\int_{-9}^{4} f(x)\;dx= \answer{0}
\]  
\end{problem}}%}

\latexProblemContent{
\ifVerboseLocation This is Integration Compute Question 0005. \\ \fi
\begin{problem}

Given the piecewise function 
\[f(x)=\left\{\begin{array}{ll}
{3 \, x + 4}\; , & x\leq{3}\\[3pt]
{-\frac{13}{4} \, x + \frac{91}{4}}\; , & {3}< x< {11}\\[3pt]
{2 \, x - 35}\; , & x\geq{11}
\end{array} \right.\]
evaluate the following definite integral by interpreting it in terms of areas.  

\input{Integral-Compute-0005.HELP.tex}

\[
\int_{3}^{8} f(x)\;dx= \answer{\frac{195}{8}}
\]  
\end{problem}}%}

\latexProblemContent{
\ifVerboseLocation This is Integration Compute Question 0005. \\ \fi
\begin{problem}

Given the piecewise function 
\[f(x)=\left\{\begin{array}{ll}
{3 \, x + 3}\; , & x\leq{4}\\[3pt]
{-6 \, x + 39}\; , & {4}< x< {9}\\[3pt]
{3 \, x - 42}\; , & x\geq{9}
\end{array} \right.\]
evaluate the following definite integral by interpreting it in terms of areas.  

\input{Integral-Compute-0005.HELP.tex}

\[
\int_{0}^{12} f(x)\;dx= \answer{\frac{9}{2}}
\]  
\end{problem}}%}

\latexProblemContent{
\ifVerboseLocation This is Integration Compute Question 0005. \\ \fi
\begin{problem}

Given the piecewise function 
\[f(x)=\left\{\begin{array}{ll}
{4 \, x + 5}\; , & x\leq{1}\\[3pt]
{-6 \, x + 15}\; , & {1}< x< {4}\\[3pt]
{x - 13}\; , & x\geq{4}
\end{array} \right.\]
evaluate the following definite integral by interpreting it in terms of areas.  

\input{Integral-Compute-0005.HELP.tex}

\[
\int_{1}^{7} f(x)\;dx= \answer{-\frac{45}{2}}
\]  
\end{problem}}%}

\latexProblemContent{
\ifVerboseLocation This is Integration Compute Question 0005. \\ \fi
\begin{problem}

Given the piecewise function 
\[f(x)=\left\{\begin{array}{ll}
{4 \, x + 1}\; , & x\leq{-2}\\[3pt]
{\frac{7}{4} \, x - \frac{7}{2}}\; , & {-2}< x< {6}\\[3pt]
{3 \, x - 11}\; , & x\geq{6}
\end{array} \right.\]
evaluate the following definite integral by interpreting it in terms of areas.  

\input{Integral-Compute-0005.HELP.tex}

\[
\int_{-3}^{9} f(x)\;dx= \answer{\frac{51}{2}}
\]  
\end{problem}}%}

\latexProblemContent{
\ifVerboseLocation This is Integration Compute Question 0005. \\ \fi
\begin{problem}

Given the piecewise function 
\[f(x)=\left\{\begin{array}{ll}
{x + 5}\; , & x\leq{3}\\[3pt]
{-2 \, x + 14}\; , & {3}< x< {11}\\[3pt]
{4 \, x - 52}\; , & x\geq{11}
\end{array} \right.\]
evaluate the following definite integral by interpreting it in terms of areas.  

\input{Integral-Compute-0005.HELP.tex}

\[
\int_{-1}^{8} f(x)\;dx= \answer{39}
\]  
\end{problem}}%}

\latexProblemContent{
\ifVerboseLocation This is Integration Compute Question 0005. \\ \fi
\begin{problem}

Given the piecewise function 
\[f(x)=\left\{\begin{array}{ll}
{3 \, x + 2}\; , & x\leq{-2}\\[3pt]
{\frac{4}{3} \, x - \frac{4}{3}}\; , & {-2}< x< {4}\\[3pt]
{2 \, x - 4}\; , & x\geq{4}
\end{array} \right.\]
evaluate the following definite integral by interpreting it in terms of areas.  

\input{Integral-Compute-0005.HELP.tex}

\[
\int_{-6}^{2} f(x)\;dx= \answer{-\frac{136}{3}}
\]  
\end{problem}}%}

\latexProblemContent{
\ifVerboseLocation This is Integration Compute Question 0005. \\ \fi
\begin{problem}

Given the piecewise function 
\[f(x)=\left\{\begin{array}{ll}
{3 \, x + 1}\; , & x\leq{1}\\[3pt]
{-\frac{8}{3} \, x + \frac{20}{3}}\; , & {1}< x< {4}\\[3pt]
{x - 8}\; , & x\geq{4}
\end{array} \right.\]
evaluate the following definite integral by interpreting it in terms of areas.  

\input{Integral-Compute-0005.HELP.tex}

\[
\int_{1}^{3} f(x)\;dx= \answer{\frac{8}{3}}
\]  
\end{problem}}%}

\latexProblemContent{
\ifVerboseLocation This is Integration Compute Question 0005. \\ \fi
\begin{problem}

Given the piecewise function 
\[f(x)=\left\{\begin{array}{ll}
{x + 5}\; , & x\leq{-3}\\[3pt]
{-\frac{4}{7} \, x + \frac{2}{7}}\; , & {-3}< x< {4}\\[3pt]
{3 \, x - 14}\; , & x\geq{4}
\end{array} \right.\]
evaluate the following definite integral by interpreting it in terms of areas.  

\input{Integral-Compute-0005.HELP.tex}

\[
\int_{-4}^{4} f(x)\;dx= \answer{\frac{3}{2}}
\]  
\end{problem}}%}

\latexProblemContent{
\ifVerboseLocation This is Integration Compute Question 0005. \\ \fi
\begin{problem}

Given the piecewise function 
\[f(x)=\left\{\begin{array}{ll}
{3 \, x + 1}\; , & x\leq{-2}\\[3pt]
{\frac{10}{7} \, x - \frac{15}{7}}\; , & {-2}< x< {5}\\[3pt]
{3 \, x - 10}\; , & x\geq{5}
\end{array} \right.\]
evaluate the following definite integral by interpreting it in terms of areas.  

\input{Integral-Compute-0005.HELP.tex}

\[
\int_{-5}^{1} f(x)\;dx= \answer{-\frac{519}{14}}
\]  
\end{problem}}%}

\latexProblemContent{
\ifVerboseLocation This is Integration Compute Question 0005. \\ \fi
\begin{problem}

Given the piecewise function 
\[f(x)=\left\{\begin{array}{ll}
{3 \, x + 5}\; , & x\leq{2}\\[3pt]
{-\frac{11}{2} \, x + 22}\; , & {2}< x< {6}\\[3pt]
{2 \, x - 23}\; , & x\geq{6}
\end{array} \right.\]
evaluate the following definite integral by interpreting it in terms of areas.  

\input{Integral-Compute-0005.HELP.tex}

\[
\int_{0}^{5} f(x)\;dx= \answer{\frac{97}{4}}
\]  
\end{problem}}%}

\latexProblemContent{
\ifVerboseLocation This is Integration Compute Question 0005. \\ \fi
\begin{problem}

Given the piecewise function 
\[f(x)=\left\{\begin{array}{ll}
{4 \, x + 4}\; , & x\leq{-2}\\[3pt]
{\frac{8}{3} \, x + \frac{4}{3}}\; , & {-2}< x< {1}\\[3pt]
{x + 3}\; , & x\geq{1}
\end{array} \right.\]
evaluate the following definite integral by interpreting it in terms of areas.  

\input{Integral-Compute-0005.HELP.tex}

\[
\int_{-5}^{-1} f(x)\;dx= \answer{-\frac{98}{3}}
\]  
\end{problem}}%}

\latexProblemContent{
\ifVerboseLocation This is Integration Compute Question 0005. \\ \fi
\begin{problem}

Given the piecewise function 
\[f(x)=\left\{\begin{array}{ll}
{x + 2}\; , & x\leq{-3}\\[3pt]
{\frac{2}{5} \, x + \frac{1}{5}}\; , & {-3}< x< {2}\\[3pt]
{x - 1}\; , & x\geq{2}
\end{array} \right.\]
evaluate the following definite integral by interpreting it in terms of areas.  

\input{Integral-Compute-0005.HELP.tex}

\[
\int_{-3}^{-1} f(x)\;dx= \answer{-\frac{6}{5}}
\]  
\end{problem}}%}

\latexProblemContent{
\ifVerboseLocation This is Integration Compute Question 0005. \\ \fi
\begin{problem}

Given the piecewise function 
\[f(x)=\left\{\begin{array}{ll}
{4 \, x + 5}\; , & x\leq{5}\\[3pt]
{-10 \, x + 75}\; , & {5}< x< {10}\\[3pt]
{x - 35}\; , & x\geq{10}
\end{array} \right.\]
evaluate the following definite integral by interpreting it in terms of areas.  

\input{Integral-Compute-0005.HELP.tex}

\[
\int_{2}^{8} f(x)\;dx= \answer{87}
\]  
\end{problem}}%}

\latexProblemContent{
\ifVerboseLocation This is Integration Compute Question 0005. \\ \fi
\begin{problem}

Given the piecewise function 
\[f(x)=\left\{\begin{array}{ll}
{2 \, x + 2}\; , & x\leq{-2}\\[3pt]
{\frac{2}{3} \, x - \frac{2}{3}}\; , & {-2}< x< {4}\\[3pt]
{x - 2}\; , & x\geq{4}
\end{array} \right.\]
evaluate the following definite integral by interpreting it in terms of areas.  

\input{Integral-Compute-0005.HELP.tex}

\[
\int_{-5}^{0} f(x)\;dx= \answer{-\frac{53}{3}}
\]  
\end{problem}}%}

\latexProblemContent{
\ifVerboseLocation This is Integration Compute Question 0005. \\ \fi
\begin{problem}

Given the piecewise function 
\[f(x)=\left\{\begin{array}{ll}
{x + 1}\; , & x\leq{2}\\[3pt]
{-\frac{6}{5} \, x + \frac{27}{5}}\; , & {2}< x< {7}\\[3pt]
{x - 10}\; , & x\geq{7}
\end{array} \right.\]
evaluate the following definite integral by interpreting it in terms of areas.  

\input{Integral-Compute-0005.HELP.tex}

\[
\int_{2}^{11} f(x)\;dx= \answer{-4}
\]  
\end{problem}}%}

\latexProblemContent{
\ifVerboseLocation This is Integration Compute Question 0005. \\ \fi
\begin{problem}

Given the piecewise function 
\[f(x)=\left\{\begin{array}{ll}
{2 \, x + 1}\; , & x\leq{1}\\[3pt]
{-\frac{6}{5} \, x + \frac{21}{5}}\; , & {1}< x< {6}\\[3pt]
{3 \, x - 21}\; , & x\geq{6}
\end{array} \right.\]
evaluate the following definite integral by interpreting it in terms of areas.  

\input{Integral-Compute-0005.HELP.tex}

\[
\int_{-2}^{4} f(x)\;dx= \answer{\frac{18}{5}}
\]  
\end{problem}}%}

\latexProblemContent{
\ifVerboseLocation This is Integration Compute Question 0005. \\ \fi
\begin{problem}

Given the piecewise function 
\[f(x)=\left\{\begin{array}{ll}
{4 \, x + 1}\; , & x\leq{2}\\[3pt]
{-3 \, x + 15}\; , & {2}< x< {8}\\[3pt]
{x - 17}\; , & x\geq{8}
\end{array} \right.\]
evaluate the following definite integral by interpreting it in terms of areas.  

\input{Integral-Compute-0005.HELP.tex}

\[
\int_{1}^{11} f(x)\;dx= \answer{-\frac{31}{2}}
\]  
\end{problem}}%}

\latexProblemContent{
\ifVerboseLocation This is Integration Compute Question 0005. \\ \fi
\begin{problem}

Given the piecewise function 
\[f(x)=\left\{\begin{array}{ll}
{2 \, x + 1}\; , & x\leq{4}\\[3pt]
{-6 \, x + 33}\; , & {4}< x< {7}\\[3pt]
{4 \, x - 37}\; , & x\geq{7}
\end{array} \right.\]
evaluate the following definite integral by interpreting it in terms of areas.  

\input{Integral-Compute-0005.HELP.tex}

\[
\int_{2}^{8} f(x)\;dx= \answer{7}
\]  
\end{problem}}%}

\latexProblemContent{
\ifVerboseLocation This is Integration Compute Question 0005. \\ \fi
\begin{problem}

Given the piecewise function 
\[f(x)=\left\{\begin{array}{ll}
{4 \, x + 1}\; , & x\leq{-2}\\[3pt]
{\frac{14}{3} \, x + \frac{7}{3}}\; , & {-2}< x< {1}\\[3pt]
{4 \, x + 3}\; , & x\geq{1}
\end{array} \right.\]
evaluate the following definite integral by interpreting it in terms of areas.  

\input{Integral-Compute-0005.HELP.tex}

\[
\int_{-7}^{2} f(x)\;dx= \answer{-76}
\]  
\end{problem}}%}

\latexProblemContent{
\ifVerboseLocation This is Integration Compute Question 0005. \\ \fi
\begin{problem}

Given the piecewise function 
\[f(x)=\left\{\begin{array}{ll}
{2 \, x + 4}\; , & x\leq{1}\\[3pt]
{-\frac{12}{5} \, x + \frac{42}{5}}\; , & {1}< x< {6}\\[3pt]
{4 \, x - 30}\; , & x\geq{6}
\end{array} \right.\]
evaluate the following definite integral by interpreting it in terms of areas.  

\input{Integral-Compute-0005.HELP.tex}

\[
\int_{-3}^{9} f(x)\;dx= \answer{8}
\]  
\end{problem}}%}

\latexProblemContent{
\ifVerboseLocation This is Integration Compute Question 0005. \\ \fi
\begin{problem}

Given the piecewise function 
\[f(x)=\left\{\begin{array}{ll}
{3 \, x + 3}\; , & x\leq{1}\\[3pt]
{-4 \, x + 10}\; , & {1}< x< {4}\\[3pt]
{2 \, x - 14}\; , & x\geq{4}
\end{array} \right.\]
evaluate the following definite integral by interpreting it in terms of areas.  

\input{Integral-Compute-0005.HELP.tex}

\[
\int_{0}^{8} f(x)\;dx= \answer{-\frac{7}{2}}
\]  
\end{problem}}%}

\latexProblemContent{
\ifVerboseLocation This is Integration Compute Question 0005. \\ \fi
\begin{problem}

Given the piecewise function 
\[f(x)=\left\{\begin{array}{ll}
{3 \, x + 1}\; , & x\leq{4}\\[3pt]
{-\frac{13}{2} \, x + 39}\; , & {4}< x< {8}\\[3pt]
{3 \, x - 37}\; , & x\geq{8}
\end{array} \right.\]
evaluate the following definite integral by interpreting it in terms of areas.  

\input{Integral-Compute-0005.HELP.tex}

\[
\int_{1}^{6} f(x)\;dx= \answer{\frac{77}{2}}
\]  
\end{problem}}%}

\latexProblemContent{
\ifVerboseLocation This is Integration Compute Question 0005. \\ \fi
\begin{problem}

Given the piecewise function 
\[f(x)=\left\{\begin{array}{ll}
{x + 4}\; , & x\leq{5}\\[3pt]
{-\frac{18}{5} \, x + 27}\; , & {5}< x< {10}\\[3pt]
{3 \, x - 39}\; , & x\geq{10}
\end{array} \right.\]
evaluate the following definite integral by interpreting it in terms of areas.  

\input{Integral-Compute-0005.HELP.tex}

\[
\int_{3}^{8} f(x)\;dx= \answer{\frac{134}{5}}
\]  
\end{problem}}%}

\latexProblemContent{
\ifVerboseLocation This is Integration Compute Question 0005. \\ \fi
\begin{problem}

Given the piecewise function 
\[f(x)=\left\{\begin{array}{ll}
{3 \, x + 5}\; , & x\leq{-5}\\[3pt]
{\frac{20}{7} \, x + \frac{30}{7}}\; , & {-5}< x< {2}\\[3pt]
{x + 8}\; , & x\geq{2}
\end{array} \right.\]
evaluate the following definite integral by interpreting it in terms of areas.  

\input{Integral-Compute-0005.HELP.tex}

\[
\int_{-10}^{4} f(x)\;dx= \answer{-\frac{131}{2}}
\]  
\end{problem}}%}

\latexProblemContent{
\ifVerboseLocation This is Integration Compute Question 0005. \\ \fi
\begin{problem}

Given the piecewise function 
\[f(x)=\left\{\begin{array}{ll}
{2 \, x + 3}\; , & x\leq{-1}\\[3pt]
{-\frac{2}{5} \, x + \frac{3}{5}}\; , & {-1}< x< {4}\\[3pt]
{x - 5}\; , & x\geq{4}
\end{array} \right.\]
evaluate the following definite integral by interpreting it in terms of areas.  

\input{Integral-Compute-0005.HELP.tex}

\[
\int_{-4}^{0} f(x)\;dx= \answer{-\frac{26}{5}}
\]  
\end{problem}}%}

\latexProblemContent{
\ifVerboseLocation This is Integration Compute Question 0005. \\ \fi
\begin{problem}

Given the piecewise function 
\[f(x)=\left\{\begin{array}{ll}
{x + 2}\; , & x\leq{3}\\[3pt]
{-\frac{5}{2} \, x + \frac{25}{2}}\; , & {3}< x< {7}\\[3pt]
{x - 12}\; , & x\geq{7}
\end{array} \right.\]
evaluate the following definite integral by interpreting it in terms of areas.  

\input{Integral-Compute-0005.HELP.tex}

\[
\int_{-2}^{7} f(x)\;dx= \answer{\frac{25}{2}}
\]  
\end{problem}}%}

\latexProblemContent{
\ifVerboseLocation This is Integration Compute Question 0005. \\ \fi
\begin{problem}

Given the piecewise function 
\[f(x)=\left\{\begin{array}{ll}
{x + 1}\; , & x\leq{-2}\\[3pt]
{\frac{2}{7} \, x - \frac{3}{7}}\; , & {-2}< x< {5}\\[3pt]
{3 \, x - 14}\; , & x\geq{5}
\end{array} \right.\]
evaluate the following definite integral by interpreting it in terms of areas.  

\input{Integral-Compute-0005.HELP.tex}

\[
\int_{-3}^{2} f(x)\;dx= \answer{-\frac{45}{14}}
\]  
\end{problem}}%}

\latexProblemContent{
\ifVerboseLocation This is Integration Compute Question 0005. \\ \fi
\begin{problem}

Given the piecewise function 
\[f(x)=\left\{\begin{array}{ll}
{4 \, x + 4}\; , & x\leq{1}\\[3pt]
{-\frac{16}{5} \, x + \frac{56}{5}}\; , & {1}< x< {6}\\[3pt]
{4 \, x - 32}\; , & x\geq{6}
\end{array} \right.\]
evaluate the following definite integral by interpreting it in terms of areas.  

\input{Integral-Compute-0005.HELP.tex}

\[
\int_{0}^{6} f(x)\;dx= \answer{6}
\]  
\end{problem}}%}

\latexProblemContent{
\ifVerboseLocation This is Integration Compute Question 0005. \\ \fi
\begin{problem}

Given the piecewise function 
\[f(x)=\left\{\begin{array}{ll}
{x + 4}\; , & x\leq{3}\\[3pt]
{-\frac{14}{5} \, x + \frac{77}{5}}\; , & {3}< x< {8}\\[3pt]
{3 \, x - 31}\; , & x\geq{8}
\end{array} \right.\]
evaluate the following definite integral by interpreting it in terms of areas.  

\input{Integral-Compute-0005.HELP.tex}

\[
\int_{3}^{11} f(x)\;dx= \answer{-\frac{15}{2}}
\]  
\end{problem}}%}

\latexProblemContent{
\ifVerboseLocation This is Integration Compute Question 0005. \\ \fi
\begin{problem}

Given the piecewise function 
\[f(x)=\left\{\begin{array}{ll}
{4 \, x + 5}\; , & x\leq{-5}\\[3pt]
{\frac{15}{2} \, x + \frac{45}{2}}\; , & {-5}< x< {-1}\\[3pt]
{3 \, x + 18}\; , & x\geq{-1}
\end{array} \right.\]
evaluate the following definite integral by interpreting it in terms of areas.  

\input{Integral-Compute-0005.HELP.tex}

\[
\int_{-7}^{0} f(x)\;dx= \answer{-\frac{43}{2}}
\]  
\end{problem}}%}

\latexProblemContent{
\ifVerboseLocation This is Integration Compute Question 0005. \\ \fi
\begin{problem}

Given the piecewise function 
\[f(x)=\left\{\begin{array}{ll}
{2 \, x + 1}\; , & x\leq{2}\\[3pt]
{-\frac{5}{2} \, x + 10}\; , & {2}< x< {6}\\[3pt]
{x - 11}\; , & x\geq{6}
\end{array} \right.\]
evaluate the following definite integral by interpreting it in terms of areas.  

\input{Integral-Compute-0005.HELP.tex}

\[
\int_{1}^{5} f(x)\;dx= \answer{\frac{31}{4}}
\]  
\end{problem}}%}

\latexProblemContent{
\ifVerboseLocation This is Integration Compute Question 0005. \\ \fi
\begin{problem}

Given the piecewise function 
\[f(x)=\left\{\begin{array}{ll}
{2 \, x + 4}\; , & x\leq{4}\\[3pt]
{-3 \, x + 24}\; , & {4}< x< {12}\\[3pt]
{3 \, x - 48}\; , & x\geq{12}
\end{array} \right.\]
evaluate the following definite integral by interpreting it in terms of areas.  

\input{Integral-Compute-0005.HELP.tex}

\[
\int_{4}^{14} f(x)\;dx= \answer{-18}
\]  
\end{problem}}%}

\latexProblemContent{
\ifVerboseLocation This is Integration Compute Question 0005. \\ \fi
\begin{problem}

Given the piecewise function 
\[f(x)=\left\{\begin{array}{ll}
{x + 1}\; , & x\leq{1}\\[3pt]
{-x + 3}\; , & {1}< x< {5}\\[3pt]
{x - 7}\; , & x\geq{5}
\end{array} \right.\]
evaluate the following definite integral by interpreting it in terms of areas.  

\input{Integral-Compute-0005.HELP.tex}

\[
\int_{-1}^{9} f(x)\;dx= \answer{2}
\]  
\end{problem}}%}

\latexProblemContent{
\ifVerboseLocation This is Integration Compute Question 0005. \\ \fi
\begin{problem}

Given the piecewise function 
\[f(x)=\left\{\begin{array}{ll}
{2 \, x + 5}\; , & x\leq{-2}\\[3pt]
{-\frac{1}{3} \, x + \frac{1}{3}}\; , & {-2}< x< {4}\\[3pt]
{2 \, x - 9}\; , & x\geq{4}
\end{array} \right.\]
evaluate the following definite integral by interpreting it in terms of areas.  

\input{Integral-Compute-0005.HELP.tex}

\[
\int_{-5}^{4} f(x)\;dx= \answer{-6}
\]  
\end{problem}}%}

\latexProblemContent{
\ifVerboseLocation This is Integration Compute Question 0005. \\ \fi
\begin{problem}

Given the piecewise function 
\[f(x)=\left\{\begin{array}{ll}
{3 \, x + 3}\; , & x\leq{3}\\[3pt]
{-4 \, x + 24}\; , & {3}< x< {9}\\[3pt]
{x - 21}\; , & x\geq{9}
\end{array} \right.\]
evaluate the following definite integral by interpreting it in terms of areas.  

\input{Integral-Compute-0005.HELP.tex}

\[
\int_{1}^{14} f(x)\;dx= \answer{-\frac{59}{2}}
\]  
\end{problem}}%}

\latexProblemContent{
\ifVerboseLocation This is Integration Compute Question 0005. \\ \fi
\begin{problem}

Given the piecewise function 
\[f(x)=\left\{\begin{array}{ll}
{x + 1}\; , & x\leq{3}\\[3pt]
{-2 \, x + 10}\; , & {3}< x< {7}\\[3pt]
{3 \, x - 25}\; , & x\geq{7}
\end{array} \right.\]
evaluate the following definite integral by interpreting it in terms of areas.  

\input{Integral-Compute-0005.HELP.tex}

\[
\int_{-1}^{8} f(x)\;dx= \answer{\frac{11}{2}}
\]  
\end{problem}}%}

\latexProblemContent{
\ifVerboseLocation This is Integration Compute Question 0005. \\ \fi
\begin{problem}

Given the piecewise function 
\[f(x)=\left\{\begin{array}{ll}
{x + 3}\; , & x\leq{-1}\\[3pt]
{-\frac{4}{7} \, x + \frac{10}{7}}\; , & {-1}< x< {6}\\[3pt]
{4 \, x - 26}\; , & x\geq{6}
\end{array} \right.\]
evaluate the following definite integral by interpreting it in terms of areas.  

\input{Integral-Compute-0005.HELP.tex}

\[
\int_{-2}^{5} f(x)\;dx= \answer{\frac{45}{14}}
\]  
\end{problem}}%}

\latexProblemContent{
\ifVerboseLocation This is Integration Compute Question 0005. \\ \fi
\begin{problem}

Given the piecewise function 
\[f(x)=\left\{\begin{array}{ll}
{4 \, x + 4}\; , & x\leq{-5}\\[3pt]
{8 \, x + 24}\; , & {-5}< x< {-1}\\[3pt]
{x + 17}\; , & x\geq{-1}
\end{array} \right.\]
evaluate the following definite integral by interpreting it in terms of areas.  

\input{Integral-Compute-0005.HELP.tex}

\[
\int_{-8}^{4} f(x)\;dx= \answer{\frac{53}{2}}
\]  
\end{problem}}%}

\latexProblemContent{
\ifVerboseLocation This is Integration Compute Question 0005. \\ \fi
\begin{problem}

Given the piecewise function 
\[f(x)=\left\{\begin{array}{ll}
{3 \, x + 1}\; , & x\leq{-1}\\[3pt]
{\frac{2}{3} \, x - \frac{4}{3}}\; , & {-1}< x< {5}\\[3pt]
{x - 3}\; , & x\geq{5}
\end{array} \right.\]
evaluate the following definite integral by interpreting it in terms of areas.  

\input{Integral-Compute-0005.HELP.tex}

\[
\int_{-2}^{5} f(x)\;dx= \answer{-\frac{7}{2}}
\]  
\end{problem}}%}

\latexProblemContent{
\ifVerboseLocation This is Integration Compute Question 0005. \\ \fi
\begin{problem}

Given the piecewise function 
\[f(x)=\left\{\begin{array}{ll}
{2 \, x + 5}\; , & x\leq{5}\\[3pt]
{-10 \, x + 65}\; , & {5}< x< {8}\\[3pt]
{4 \, x - 47}\; , & x\geq{8}
\end{array} \right.\]
evaluate the following definite integral by interpreting it in terms of areas.  

\input{Integral-Compute-0005.HELP.tex}

\[
\int_{4}^{11} f(x)\;dx= \answer{-13}
\]  
\end{problem}}%}

\latexProblemContent{
\ifVerboseLocation This is Integration Compute Question 0005. \\ \fi
\begin{problem}

Given the piecewise function 
\[f(x)=\left\{\begin{array}{ll}
{3 \, x + 2}\; , & x\leq{5}\\[3pt]
{-\frac{34}{3} \, x + \frac{221}{3}}\; , & {5}< x< {8}\\[3pt]
{4 \, x - 49}\; , & x\geq{8}
\end{array} \right.\]
evaluate the following definite integral by interpreting it in terms of areas.  

\input{Integral-Compute-0005.HELP.tex}

\[
\int_{4}^{6} f(x)\;dx= \answer{\frac{161}{6}}
\]  
\end{problem}}%}

\latexProblemContent{
\ifVerboseLocation This is Integration Compute Question 0005. \\ \fi
\begin{problem}

Given the piecewise function 
\[f(x)=\left\{\begin{array}{ll}
{2 \, x + 5}\; , & x\leq{2}\\[3pt]
{-6 \, x + 21}\; , & {2}< x< {5}\\[3pt]
{3 \, x - 24}\; , & x\geq{5}
\end{array} \right.\]
evaluate the following definite integral by interpreting it in terms of areas.  

\input{Integral-Compute-0005.HELP.tex}

\[
\int_{-3}^{9} f(x)\;dx= \answer{8}
\]  
\end{problem}}%}

\latexProblemContent{
\ifVerboseLocation This is Integration Compute Question 0005. \\ \fi
\begin{problem}

Given the piecewise function 
\[f(x)=\left\{\begin{array}{ll}
{2 \, x + 1}\; , & x\leq{-3}\\[3pt]
{\frac{10}{3} \, x + 5}\; , & {-3}< x< {0}\\[3pt]
{4 \, x + 5}\; , & x\geq{0}
\end{array} \right.\]
evaluate the following definite integral by interpreting it in terms of areas.  

\input{Integral-Compute-0005.HELP.tex}

\[
\int_{-8}^{-2} f(x)\;dx= \answer{-\frac{160}{3}}
\]  
\end{problem}}%}

\latexProblemContent{
\ifVerboseLocation This is Integration Compute Question 0005. \\ \fi
\begin{problem}

Given the piecewise function 
\[f(x)=\left\{\begin{array}{ll}
{3 \, x + 5}\; , & x\leq{-4}\\[3pt]
{\frac{7}{3} \, x + \frac{7}{3}}\; , & {-4}< x< {2}\\[3pt]
{2 \, x + 3}\; , & x\geq{2}
\end{array} \right.\]
evaluate the following definite integral by interpreting it in terms of areas.  

\input{Integral-Compute-0005.HELP.tex}

\[
\int_{-6}^{4} f(x)\;dx= \answer{-2}
\]  
\end{problem}}%}

\latexProblemContent{
\ifVerboseLocation This is Integration Compute Question 0005. \\ \fi
\begin{problem}

Given the piecewise function 
\[f(x)=\left\{\begin{array}{ll}
{x + 1}\; , & x\leq{2}\\[3pt]
{-2 \, x + 7}\; , & {2}< x< {5}\\[3pt]
{3 \, x - 18}\; , & x\geq{5}
\end{array} \right.\]
evaluate the following definite integral by interpreting it in terms of areas.  

\input{Integral-Compute-0005.HELP.tex}

\[
\int_{-3}^{7} f(x)\;dx= \answer{\frac{5}{2}}
\]  
\end{problem}}%}

\latexProblemContent{
\ifVerboseLocation This is Integration Compute Question 0005. \\ \fi
\begin{problem}

Given the piecewise function 
\[f(x)=\left\{\begin{array}{ll}
{3 \, x + 5}\; , & x\leq{-2}\\[3pt]
{\frac{1}{2} \, x}\; , & {-2}< x< {2}\\[3pt]
{3 \, x - 5}\; , & x\geq{2}
\end{array} \right.\]
evaluate the following definite integral by interpreting it in terms of areas.  

\input{Integral-Compute-0005.HELP.tex}

\[
\int_{-4}^{2} f(x)\;dx= \answer{-8}
\]  
\end{problem}}%}

\latexProblemContent{
\ifVerboseLocation This is Integration Compute Question 0005. \\ \fi
\begin{problem}

Given the piecewise function 
\[f(x)=\left\{\begin{array}{ll}
{2 \, x + 2}\; , & x\leq{-2}\\[3pt]
{x}\; , & {-2}< x< {2}\\[3pt]
{3 \, x - 4}\; , & x\geq{2}
\end{array} \right.\]
evaluate the following definite integral by interpreting it in terms of areas.  

\input{Integral-Compute-0005.HELP.tex}

\[
\int_{-2}^{2} f(x)\;dx= \answer{0}
\]  
\end{problem}}%}

\latexProblemContent{
\ifVerboseLocation This is Integration Compute Question 0005. \\ \fi
\begin{problem}

Given the piecewise function 
\[f(x)=\left\{\begin{array}{ll}
{2 \, x + 5}\; , & x\leq{5}\\[3pt]
{-\frac{30}{7} \, x + \frac{255}{7}}\; , & {5}< x< {12}\\[3pt]
{4 \, x - 63}\; , & x\geq{12}
\end{array} \right.\]
evaluate the following definite integral by interpreting it in terms of areas.  

\input{Integral-Compute-0005.HELP.tex}

\[
\int_{4}^{9} f(x)\;dx= \answer{\frac{278}{7}}
\]  
\end{problem}}%}

\latexProblemContent{
\ifVerboseLocation This is Integration Compute Question 0005. \\ \fi
\begin{problem}

Given the piecewise function 
\[f(x)=\left\{\begin{array}{ll}
{3 \, x + 3}\; , & x\leq{-2}\\[3pt]
{\frac{6}{7} \, x - \frac{9}{7}}\; , & {-2}< x< {5}\\[3pt]
{3 \, x - 12}\; , & x\geq{5}
\end{array} \right.\]
evaluate the following definite integral by interpreting it in terms of areas.  

\input{Integral-Compute-0005.HELP.tex}

\[
\int_{-7}^{1} f(x)\;dx= \answer{-\frac{807}{14}}
\]  
\end{problem}}%}

\latexProblemContent{
\ifVerboseLocation This is Integration Compute Question 0005. \\ \fi
\begin{problem}

Given the piecewise function 
\[f(x)=\left\{\begin{array}{ll}
{4 \, x + 1}\; , & x\leq{-5}\\[3pt]
{\frac{19}{4} \, x + \frac{19}{4}}\; , & {-5}< x< {3}\\[3pt]
{x + 16}\; , & x\geq{3}
\end{array} \right.\]
evaluate the following definite integral by interpreting it in terms of areas.  

\input{Integral-Compute-0005.HELP.tex}

\[
\int_{-7}^{5} f(x)\;dx= \answer{-6}
\]  
\end{problem}}%}

\latexProblemContent{
\ifVerboseLocation This is Integration Compute Question 0005. \\ \fi
\begin{problem}

Given the piecewise function 
\[f(x)=\left\{\begin{array}{ll}
{x + 2}\; , & x\leq{-2}\\[3pt]
{0}\; , & {-2}< x< {5}\\[3pt]
{4 \, x - 20}\; , & x\geq{5}
\end{array} \right.\]
evaluate the following definite integral by interpreting it in terms of areas.  

\input{Integral-Compute-0005.HELP.tex}

\[
\int_{-6}^{8} f(x)\;dx= \answer{10}
\]  
\end{problem}}%}

\latexProblemContent{
\ifVerboseLocation This is Integration Compute Question 0005. \\ \fi
\begin{problem}

Given the piecewise function 
\[f(x)=\left\{\begin{array}{ll}
{x + 3}\; , & x\leq{-2}\\[3pt]
{-\frac{2}{3} \, x - \frac{1}{3}}\; , & {-2}< x< {1}\\[3pt]
{4 \, x - 5}\; , & x\geq{1}
\end{array} \right.\]
evaluate the following definite integral by interpreting it in terms of areas.  

\input{Integral-Compute-0005.HELP.tex}

\[
\int_{-3}^{2} f(x)\;dx= \answer{\frac{3}{2}}
\]  
\end{problem}}%}

\latexProblemContent{
\ifVerboseLocation This is Integration Compute Question 0005. \\ \fi
\begin{problem}

Given the piecewise function 
\[f(x)=\left\{\begin{array}{ll}
{2 \, x + 5}\; , & x\leq{-3}\\[3pt]
{\frac{1}{3} \, x}\; , & {-3}< x< {3}\\[3pt]
{x - 2}\; , & x\geq{3}
\end{array} \right.\]
evaluate the following definite integral by interpreting it in terms of areas.  

\input{Integral-Compute-0005.HELP.tex}

\[
\int_{-8}^{-1} f(x)\;dx= \answer{-\frac{94}{3}}
\]  
\end{problem}}%}

\latexProblemContent{
\ifVerboseLocation This is Integration Compute Question 0005. \\ \fi
\begin{problem}

Given the piecewise function 
\[f(x)=\left\{\begin{array}{ll}
{3 \, x + 5}\; , & x\leq{5}\\[3pt]
{-5 \, x + 45}\; , & {5}< x< {13}\\[3pt]
{2 \, x - 46}\; , & x\geq{13}
\end{array} \right.\]
evaluate the following definite integral by interpreting it in terms of areas.  

\input{Integral-Compute-0005.HELP.tex}

\[
\int_{3}^{16} f(x)\;dx= \answer{-17}
\]  
\end{problem}}%}

\latexProblemContent{
\ifVerboseLocation This is Integration Compute Question 0005. \\ \fi
\begin{problem}

Given the piecewise function 
\[f(x)=\left\{\begin{array}{ll}
{4 \, x + 1}\; , & x\leq{5}\\[3pt]
{-\frac{42}{5} \, x + 63}\; , & {5}< x< {10}\\[3pt]
{3 \, x - 51}\; , & x\geq{10}
\end{array} \right.\]
evaluate the following definite integral by interpreting it in terms of areas.  

\input{Integral-Compute-0005.HELP.tex}

\[
\int_{3}^{8} f(x)\;dx= \answer{\frac{296}{5}}
\]  
\end{problem}}%}

\latexProblemContent{
\ifVerboseLocation This is Integration Compute Question 0005. \\ \fi
\begin{problem}

Given the piecewise function 
\[f(x)=\left\{\begin{array}{ll}
{x + 4}\; , & x\leq{3}\\[3pt]
{-\frac{7}{2} \, x + \frac{35}{2}}\; , & {3}< x< {7}\\[3pt]
{x - 14}\; , & x\geq{7}
\end{array} \right.\]
evaluate the following definite integral by interpreting it in terms of areas.  

\input{Integral-Compute-0005.HELP.tex}

\[
\int_{2}^{6} f(x)\;dx= \answer{\frac{47}{4}}
\]  
\end{problem}}%}

\latexProblemContent{
\ifVerboseLocation This is Integration Compute Question 0005. \\ \fi
\begin{problem}

Given the piecewise function 
\[f(x)=\left\{\begin{array}{ll}
{3 \, x + 3}\; , & x\leq{5}\\[3pt]
{-\frac{36}{5} \, x + 54}\; , & {5}< x< {10}\\[3pt]
{x - 28}\; , & x\geq{10}
\end{array} \right.\]
evaluate the following definite integral by interpreting it in terms of areas.  

\input{Integral-Compute-0005.HELP.tex}

\[
\int_{4}^{10} f(x)\;dx= \answer{\frac{33}{2}}
\]  
\end{problem}}%}

\latexProblemContent{
\ifVerboseLocation This is Integration Compute Question 0005. \\ \fi
\begin{problem}

Given the piecewise function 
\[f(x)=\left\{\begin{array}{ll}
{x + 1}\; , & x\leq{3}\\[3pt]
{-\frac{8}{3} \, x + 12}\; , & {3}< x< {6}\\[3pt]
{x - 10}\; , & x\geq{6}
\end{array} \right.\]
evaluate the following definite integral by interpreting it in terms of areas.  

\input{Integral-Compute-0005.HELP.tex}

\[
\int_{2}^{9} f(x)\;dx= \answer{-4}
\]  
\end{problem}}%}

\latexProblemContent{
\ifVerboseLocation This is Integration Compute Question 0005. \\ \fi
\begin{problem}

Given the piecewise function 
\[f(x)=\left\{\begin{array}{ll}
{2 \, x + 4}\; , & x\leq{-5}\\[3pt]
{\frac{12}{5} \, x + 6}\; , & {-5}< x< {0}\\[3pt]
{x + 6}\; , & x\geq{0}
\end{array} \right.\]
evaluate the following definite integral by interpreting it in terms of areas.  

\input{Integral-Compute-0005.HELP.tex}

\[
\int_{-8}^{1} f(x)\;dx= \answer{-\frac{41}{2}}
\]  
\end{problem}}%}

\latexProblemContent{
\ifVerboseLocation This is Integration Compute Question 0005. \\ \fi
\begin{problem}

Given the piecewise function 
\[f(x)=\left\{\begin{array}{ll}
{3 \, x + 1}\; , & x\leq{2}\\[3pt]
{-\frac{7}{2} \, x + 14}\; , & {2}< x< {6}\\[3pt]
{3 \, x - 25}\; , & x\geq{6}
\end{array} \right.\]
evaluate the following definite integral by interpreting it in terms of areas.  

\input{Integral-Compute-0005.HELP.tex}

\[
\int_{1}^{9} f(x)\;dx= \answer{-2}
\]  
\end{problem}}%}

\latexProblemContent{
\ifVerboseLocation This is Integration Compute Question 0005. \\ \fi
\begin{problem}

Given the piecewise function 
\[f(x)=\left\{\begin{array}{ll}
{x + 4}\; , & x\leq{5}\\[3pt]
{-\frac{18}{7} \, x + \frac{153}{7}}\; , & {5}< x< {12}\\[3pt]
{x - 21}\; , & x\geq{12}
\end{array} \right.\]
evaluate the following definite integral by interpreting it in terms of areas.  

\input{Integral-Compute-0005.HELP.tex}

\[
\int_{4}^{14} f(x)\;dx= \answer{-\frac{15}{2}}
\]  
\end{problem}}%}

\latexProblemContent{
\ifVerboseLocation This is Integration Compute Question 0005. \\ \fi
\begin{problem}

Given the piecewise function 
\[f(x)=\left\{\begin{array}{ll}
{2 \, x + 3}\; , & x\leq{2}\\[3pt]
{-\frac{7}{4} \, x + \frac{21}{2}}\; , & {2}< x< {10}\\[3pt]
{3 \, x - 37}\; , & x\geq{10}
\end{array} \right.\]
evaluate the following definite integral by interpreting it in terms of areas.  

\input{Integral-Compute-0005.HELP.tex}

\[
\int_{-1}^{3} f(x)\;dx= \answer{\frac{145}{8}}
\]  
\end{problem}}%}

\latexProblemContent{
\ifVerboseLocation This is Integration Compute Question 0005. \\ \fi
\begin{problem}

Given the piecewise function 
\[f(x)=\left\{\begin{array}{ll}
{x + 2}\; , & x\leq{-4}\\[3pt]
{\frac{4}{5} \, x + \frac{6}{5}}\; , & {-4}< x< {1}\\[3pt]
{4 \, x - 2}\; , & x\geq{1}
\end{array} \right.\]
evaluate the following definite integral by interpreting it in terms of areas.  

\input{Integral-Compute-0005.HELP.tex}

\[
\int_{-5}^{4} f(x)\;dx= \answer{\frac{43}{2}}
\]  
\end{problem}}%}

\latexProblemContent{
\ifVerboseLocation This is Integration Compute Question 0005. \\ \fi
\begin{problem}

Given the piecewise function 
\[f(x)=\left\{\begin{array}{ll}
{x + 2}\; , & x\leq{-1}\\[3pt]
{-\frac{2}{3} \, x + \frac{1}{3}}\; , & {-1}< x< {2}\\[3pt]
{3 \, x - 7}\; , & x\geq{2}
\end{array} \right.\]
evaluate the following definite integral by interpreting it in terms of areas.  

\input{Integral-Compute-0005.HELP.tex}

\[
\int_{-3}^{5} f(x)\;dx= \answer{\frac{21}{2}}
\]  
\end{problem}}%}

\latexProblemContent{
\ifVerboseLocation This is Integration Compute Question 0005. \\ \fi
\begin{problem}

Given the piecewise function 
\[f(x)=\left\{\begin{array}{ll}
{4 \, x + 2}\; , & x\leq{-1}\\[3pt]
{\frac{1}{2} \, x - \frac{3}{2}}\; , & {-1}< x< {7}\\[3pt]
{2 \, x - 12}\; , & x\geq{7}
\end{array} \right.\]
evaluate the following definite integral by interpreting it in terms of areas.  

\input{Integral-Compute-0005.HELP.tex}

\[
\int_{-4}^{8} f(x)\;dx= \answer{-21}
\]  
\end{problem}}%}

\latexProblemContent{
\ifVerboseLocation This is Integration Compute Question 0005. \\ \fi
\begin{problem}

Given the piecewise function 
\[f(x)=\left\{\begin{array}{ll}
{3 \, x + 2}\; , & x\leq{-1}\\[3pt]
{\frac{2}{3} \, x - \frac{1}{3}}\; , & {-1}< x< {2}\\[3pt]
{4 \, x - 7}\; , & x\geq{2}
\end{array} \right.\]
evaluate the following definite integral by interpreting it in terms of areas.  

\input{Integral-Compute-0005.HELP.tex}

\[
\int_{-2}^{1} f(x)\;dx= \answer{-\frac{19}{6}}
\]  
\end{problem}}%}

\latexProblemContent{
\ifVerboseLocation This is Integration Compute Question 0005. \\ \fi
\begin{problem}

Given the piecewise function 
\[f(x)=\left\{\begin{array}{ll}
{4 \, x + 3}\; , & x\leq{5}\\[3pt]
{-\frac{23}{4} \, x + \frac{207}{4}}\; , & {5}< x< {13}\\[3pt]
{4 \, x - 75}\; , & x\geq{13}
\end{array} \right.\]
evaluate the following definite integral by interpreting it in terms of areas.  

\input{Integral-Compute-0005.HELP.tex}

\[
\int_{1}^{7} f(x)\;dx= \answer{\frac{189}{2}}
\]  
\end{problem}}%}

\latexProblemContent{
\ifVerboseLocation This is Integration Compute Question 0005. \\ \fi
\begin{problem}

Given the piecewise function 
\[f(x)=\left\{\begin{array}{ll}
{x + 2}\; , & x\leq{3}\\[3pt]
{-\frac{10}{3} \, x + 15}\; , & {3}< x< {6}\\[3pt]
{x - 11}\; , & x\geq{6}
\end{array} \right.\]
evaluate the following definite integral by interpreting it in terms of areas.  

\input{Integral-Compute-0005.HELP.tex}

\[
\int_{0}^{10} f(x)\;dx= \answer{-\frac{3}{2}}
\]  
\end{problem}}%}

\latexProblemContent{
\ifVerboseLocation This is Integration Compute Question 0005. \\ \fi
\begin{problem}

Given the piecewise function 
\[f(x)=\left\{\begin{array}{ll}
{4 \, x + 3}\; , & x\leq{3}\\[3pt]
{-\frac{15}{2} \, x + \frac{75}{2}}\; , & {3}< x< {7}\\[3pt]
{x - 22}\; , & x\geq{7}
\end{array} \right.\]
evaluate the following definite integral by interpreting it in terms of areas.  

\input{Integral-Compute-0005.HELP.tex}

\[
\int_{2}^{4} f(x)\;dx= \answer{\frac{97}{4}}
\]  
\end{problem}}%}

\latexProblemContent{
\ifVerboseLocation This is Integration Compute Question 0005. \\ \fi
\begin{problem}

Given the piecewise function 
\[f(x)=\left\{\begin{array}{ll}
{4 \, x + 3}\; , & x\leq{1}\\[3pt]
{-\frac{7}{3} \, x + \frac{28}{3}}\; , & {1}< x< {7}\\[3pt]
{2 \, x - 21}\; , & x\geq{7}
\end{array} \right.\]
evaluate the following definite integral by interpreting it in terms of areas.  

\input{Integral-Compute-0005.HELP.tex}

\[
\int_{1}^{6} f(x)\;dx= \answer{\frac{35}{6}}
\]  
\end{problem}}%}

\latexProblemContent{
\ifVerboseLocation This is Integration Compute Question 0005. \\ \fi
\begin{problem}

Given the piecewise function 
\[f(x)=\left\{\begin{array}{ll}
{3 \, x + 4}\; , & x\leq{-1}\\[3pt]
{-\frac{1}{4} \, x + \frac{3}{4}}\; , & {-1}< x< {7}\\[3pt]
{x - 8}\; , & x\geq{7}
\end{array} \right.\]
evaluate the following definite integral by interpreting it in terms of areas.  

\input{Integral-Compute-0005.HELP.tex}

\[
\int_{-6}^{0} f(x)\;dx= \answer{-\frac{253}{8}}
\]  
\end{problem}}%}

\latexProblemContent{
\ifVerboseLocation This is Integration Compute Question 0005. \\ \fi
\begin{problem}

Given the piecewise function 
\[f(x)=\left\{\begin{array}{ll}
{x + 4}\; , & x\leq{-5}\\[3pt]
{\frac{2}{5} \, x + 1}\; , & {-5}< x< {0}\\[3pt]
{4 \, x + 1}\; , & x\geq{0}
\end{array} \right.\]
evaluate the following definite integral by interpreting it in terms of areas.  

\input{Integral-Compute-0005.HELP.tex}

\[
\int_{-9}^{1} f(x)\;dx= \answer{-9}
\]  
\end{problem}}%}

\latexProblemContent{
\ifVerboseLocation This is Integration Compute Question 0005. \\ \fi
\begin{problem}

Given the piecewise function 
\[f(x)=\left\{\begin{array}{ll}
{4 \, x + 4}\; , & x\leq{5}\\[3pt]
{-16 \, x + 104}\; , & {5}< x< {8}\\[3pt]
{2 \, x - 40}\; , & x\geq{8}
\end{array} \right.\]
evaluate the following definite integral by interpreting it in terms of areas.  

\input{Integral-Compute-0005.HELP.tex}

\[
\int_{1}^{13} f(x)\;dx= \answer{-31}
\]  
\end{problem}}%}

\latexProblemContent{
\ifVerboseLocation This is Integration Compute Question 0005. \\ \fi
\begin{problem}

Given the piecewise function 
\[f(x)=\left\{\begin{array}{ll}
{2 \, x + 4}\; , & x\leq{-2}\\[3pt]
{0}\; , & {-2}< x< {6}\\[3pt]
{2 \, x - 12}\; , & x\geq{6}
\end{array} \right.\]
evaluate the following definite integral by interpreting it in terms of areas.  

\input{Integral-Compute-0005.HELP.tex}

\[
\int_{-3}^{7} f(x)\;dx= \answer{0}
\]  
\end{problem}}%}

\latexProblemContent{
\ifVerboseLocation This is Integration Compute Question 0005. \\ \fi
\begin{problem}

Given the piecewise function 
\[f(x)=\left\{\begin{array}{ll}
{3 \, x + 3}\; , & x\leq{-4}\\[3pt]
{\frac{9}{2} \, x + 9}\; , & {-4}< x< {0}\\[3pt]
{3 \, x + 9}\; , & x\geq{0}
\end{array} \right.\]
evaluate the following definite integral by interpreting it in terms of areas.  

\input{Integral-Compute-0005.HELP.tex}

\[
\int_{-5}^{0} f(x)\;dx= \answer{-\frac{21}{2}}
\]  
\end{problem}}%}

\latexProblemContent{
\ifVerboseLocation This is Integration Compute Question 0005. \\ \fi
\begin{problem}

Given the piecewise function 
\[f(x)=\left\{\begin{array}{ll}
{2 \, x + 5}\; , & x\leq{-3}\\[3pt]
{\frac{2}{3} \, x + 1}\; , & {-3}< x< {0}\\[3pt]
{4 \, x + 1}\; , & x\geq{0}
\end{array} \right.\]
evaluate the following definite integral by interpreting it in terms of areas.  

\input{Integral-Compute-0005.HELP.tex}

\[
\int_{-4}^{4} f(x)\;dx= \answer{34}
\]  
\end{problem}}%}

\latexProblemContent{
\ifVerboseLocation This is Integration Compute Question 0005. \\ \fi
\begin{problem}

Given the piecewise function 
\[f(x)=\left\{\begin{array}{ll}
{4 \, x + 2}\; , & x\leq{3}\\[3pt]
{-\frac{28}{5} \, x + \frac{154}{5}}\; , & {3}< x< {8}\\[3pt]
{2 \, x - 30}\; , & x\geq{8}
\end{array} \right.\]
evaluate the following definite integral by interpreting it in terms of areas.  

\input{Integral-Compute-0005.HELP.tex}

\[
\int_{2}^{10} f(x)\;dx= \answer{-12}
\]  
\end{problem}}%}

\latexProblemContent{
\ifVerboseLocation This is Integration Compute Question 0005. \\ \fi
\begin{problem}

Given the piecewise function 
\[f(x)=\left\{\begin{array}{ll}
{3 \, x + 2}\; , & x\leq{3}\\[3pt]
{-\frac{22}{5} \, x + \frac{121}{5}}\; , & {3}< x< {8}\\[3pt]
{x - 19}\; , & x\geq{8}
\end{array} \right.\]
evaluate the following definite integral by interpreting it in terms of areas.  

\input{Integral-Compute-0005.HELP.tex}

\[
\int_{0}^{9} f(x)\;dx= \answer{9}
\]  
\end{problem}}%}

\latexProblemContent{
\ifVerboseLocation This is Integration Compute Question 0005. \\ \fi
\begin{problem}

Given the piecewise function 
\[f(x)=\left\{\begin{array}{ll}
{x + 1}\; , & x\leq{5}\\[3pt]
{-3 \, x + 21}\; , & {5}< x< {9}\\[3pt]
{4 \, x - 42}\; , & x\geq{9}
\end{array} \right.\]
evaluate the following definite integral by interpreting it in terms of areas.  

\input{Integral-Compute-0005.HELP.tex}

\[
\int_{5}^{13} f(x)\;dx= \answer{8}
\]  
\end{problem}}%}

\latexProblemContent{
\ifVerboseLocation This is Integration Compute Question 0005. \\ \fi
\begin{problem}

Given the piecewise function 
\[f(x)=\left\{\begin{array}{ll}
{x + 2}\; , & x\leq{5}\\[3pt]
{-\frac{14}{5} \, x + 21}\; , & {5}< x< {10}\\[3pt]
{2 \, x - 27}\; , & x\geq{10}
\end{array} \right.\]
evaluate the following definite integral by interpreting it in terms of areas.  

\input{Integral-Compute-0005.HELP.tex}

\[
\int_{5}^{7} f(x)\;dx= \answer{\frac{42}{5}}
\]  
\end{problem}}%}

\latexProblemContent{
\ifVerboseLocation This is Integration Compute Question 0005. \\ \fi
\begin{problem}

Given the piecewise function 
\[f(x)=\left\{\begin{array}{ll}
{2 \, x + 4}\; , & x\leq{-4}\\[3pt]
{\frac{8}{3} \, x + \frac{20}{3}}\; , & {-4}< x< {-1}\\[3pt]
{4 \, x + 8}\; , & x\geq{-1}
\end{array} \right.\]
evaluate the following definite integral by interpreting it in terms of areas.  

\input{Integral-Compute-0005.HELP.tex}

\[
\int_{-4}^{2} f(x)\;dx= \answer{30}
\]  
\end{problem}}%}

\latexProblemContent{
\ifVerboseLocation This is Integration Compute Question 0005. \\ \fi
\begin{problem}

Given the piecewise function 
\[f(x)=\left\{\begin{array}{ll}
{4 \, x + 1}\; , & x\leq{5}\\[3pt]
{-7 \, x + 56}\; , & {5}< x< {11}\\[3pt]
{x - 32}\; , & x\geq{11}
\end{array} \right.\]
evaluate the following definite integral by interpreting it in terms of areas.  

\input{Integral-Compute-0005.HELP.tex}

\[
\int_{5}^{14} f(x)\;dx= \answer{-\frac{117}{2}}
\]  
\end{problem}}%}

\latexProblemContent{
\ifVerboseLocation This is Integration Compute Question 0005. \\ \fi
\begin{problem}

Given the piecewise function 
\[f(x)=\left\{\begin{array}{ll}
{4 \, x + 1}\; , & x\leq{-5}\\[3pt]
{\frac{19}{2} \, x + \frac{57}{2}}\; , & {-5}< x< {-1}\\[3pt]
{x + 20}\; , & x\geq{-1}
\end{array} \right.\]
evaluate the following definite integral by interpreting it in terms of areas.  

\input{Integral-Compute-0005.HELP.tex}

\[
\int_{-6}^{-3} f(x)\;dx= \answer{-40}
\]  
\end{problem}}%}

\latexProblemContent{
\ifVerboseLocation This is Integration Compute Question 0005. \\ \fi
\begin{problem}

Given the piecewise function 
\[f(x)=\left\{\begin{array}{ll}
{3 \, x + 4}\; , & x\leq{-4}\\[3pt]
{\frac{16}{3} \, x + \frac{40}{3}}\; , & {-4}< x< {-1}\\[3pt]
{x + 9}\; , & x\geq{-1}
\end{array} \right.\]
evaluate the following definite integral by interpreting it in terms of areas.  

\input{Integral-Compute-0005.HELP.tex}

\[
\int_{-7}^{-1} f(x)\;dx= \answer{-\frac{75}{2}}
\]  
\end{problem}}%}

\latexProblemContent{
\ifVerboseLocation This is Integration Compute Question 0005. \\ \fi
\begin{problem}

Given the piecewise function 
\[f(x)=\left\{\begin{array}{ll}
{3 \, x + 2}\; , & x\leq{1}\\[3pt]
{-\frac{10}{7} \, x + \frac{45}{7}}\; , & {1}< x< {8}\\[3pt]
{4 \, x - 37}\; , & x\geq{8}
\end{array} \right.\]
evaluate the following definite integral by interpreting it in terms of areas.  

\input{Integral-Compute-0005.HELP.tex}

\[
\int_{1}^{8} f(x)\;dx= \answer{0}
\]  
\end{problem}}%}

\latexProblemContent{
\ifVerboseLocation This is Integration Compute Question 0005. \\ \fi
\begin{problem}

Given the piecewise function 
\[f(x)=\left\{\begin{array}{ll}
{2 \, x + 4}\; , & x\leq{-3}\\[3pt]
{\frac{1}{2} \, x - \frac{1}{2}}\; , & {-3}< x< {5}\\[3pt]
{4 \, x - 18}\; , & x\geq{5}
\end{array} \right.\]
evaluate the following definite integral by interpreting it in terms of areas.  

\input{Integral-Compute-0005.HELP.tex}

\[
\int_{-8}^{5} f(x)\;dx= \answer{-35}
\]  
\end{problem}}%}

\latexProblemContent{
\ifVerboseLocation This is Integration Compute Question 0005. \\ \fi
\begin{problem}

Given the piecewise function 
\[f(x)=\left\{\begin{array}{ll}
{2 \, x + 1}\; , & x\leq{3}\\[3pt]
{-\frac{14}{5} \, x + \frac{77}{5}}\; , & {3}< x< {8}\\[3pt]
{x - 15}\; , & x\geq{8}
\end{array} \right.\]
evaluate the following definite integral by interpreting it in terms of areas.  

\input{Integral-Compute-0005.HELP.tex}

\[
\int_{1}^{6} f(x)\;dx= \answer{\frac{92}{5}}
\]  
\end{problem}}%}

\latexProblemContent{
\ifVerboseLocation This is Integration Compute Question 0005. \\ \fi
\begin{problem}

Given the piecewise function 
\[f(x)=\left\{\begin{array}{ll}
{x + 2}\; , & x\leq{4}\\[3pt]
{-\frac{3}{2} \, x + 12}\; , & {4}< x< {12}\\[3pt]
{4 \, x - 54}\; , & x\geq{12}
\end{array} \right.\]
evaluate the following definite integral by interpreting it in terms of areas.  

\input{Integral-Compute-0005.HELP.tex}

\[
\int_{2}^{15} f(x)\;dx= \answer{10}
\]  
\end{problem}}%}

\latexProblemContent{
\ifVerboseLocation This is Integration Compute Question 0005. \\ \fi
\begin{problem}

Given the piecewise function 
\[f(x)=\left\{\begin{array}{ll}
{x + 5}\; , & x\leq{-2}\\[3pt]
{-\frac{3}{2} \, x}\; , & {-2}< x< {2}\\[3pt]
{2 \, x - 7}\; , & x\geq{2}
\end{array} \right.\]
evaluate the following definite integral by interpreting it in terms of areas.  

\input{Integral-Compute-0005.HELP.tex}

\[
\int_{-5}^{6} f(x)\;dx= \answer{\frac{17}{2}}
\]  
\end{problem}}%}

\latexProblemContent{
\ifVerboseLocation This is Integration Compute Question 0005. \\ \fi
\begin{problem}

Given the piecewise function 
\[f(x)=\left\{\begin{array}{ll}
{x + 3}\; , & x\leq{-1}\\[3pt]
{-\frac{4}{3} \, x + \frac{2}{3}}\; , & {-1}< x< {2}\\[3pt]
{2 \, x - 6}\; , & x\geq{2}
\end{array} \right.\]
evaluate the following definite integral by interpreting it in terms of areas.  

\input{Integral-Compute-0005.HELP.tex}

\[
\int_{-1}^{3} f(x)\;dx= \answer{-1}
\]  
\end{problem}}%}

\latexProblemContent{
\ifVerboseLocation This is Integration Compute Question 0005. \\ \fi
\begin{problem}

Given the piecewise function 
\[f(x)=\left\{\begin{array}{ll}
{x + 2}\; , & x\leq{5}\\[3pt]
{-\frac{7}{4} \, x + \frac{63}{4}}\; , & {5}< x< {13}\\[3pt]
{x - 20}\; , & x\geq{13}
\end{array} \right.\]
evaluate the following definite integral by interpreting it in terms of areas.  

\input{Integral-Compute-0005.HELP.tex}

\[
\int_{3}^{10} f(x)\;dx= \answer{\frac{201}{8}}
\]  
\end{problem}}%}

\latexProblemContent{
\ifVerboseLocation This is Integration Compute Question 0005. \\ \fi
\begin{problem}

Given the piecewise function 
\[f(x)=\left\{\begin{array}{ll}
{x + 4}\; , & x\leq{3}\\[3pt]
{-\frac{7}{3} \, x + 14}\; , & {3}< x< {9}\\[3pt]
{2 \, x - 25}\; , & x\geq{9}
\end{array} \right.\]
evaluate the following definite integral by interpreting it in terms of areas.  

\input{Integral-Compute-0005.HELP.tex}

\[
\int_{-2}^{7} f(x)\;dx= \answer{\frac{191}{6}}
\]  
\end{problem}}%}

\latexProblemContent{
\ifVerboseLocation This is Integration Compute Question 0005. \\ \fi
\begin{problem}

Given the piecewise function 
\[f(x)=\left\{\begin{array}{ll}
{3 \, x + 1}\; , & x\leq{-4}\\[3pt]
{\frac{22}{5} \, x + \frac{33}{5}}\; , & {-4}< x< {1}\\[3pt]
{3 \, x + 8}\; , & x\geq{1}
\end{array} \right.\]
evaluate the following definite integral by interpreting it in terms of areas.  

\input{Integral-Compute-0005.HELP.tex}

\[
\int_{-4}^{2} f(x)\;dx= \answer{\frac{25}{2}}
\]  
\end{problem}}%}

\latexProblemContent{
\ifVerboseLocation This is Integration Compute Question 0005. \\ \fi
\begin{problem}

Given the piecewise function 
\[f(x)=\left\{\begin{array}{ll}
{x + 5}\; , & x\leq{-1}\\[3pt]
{-2 \, x + 2}\; , & {-1}< x< {3}\\[3pt]
{4 \, x - 16}\; , & x\geq{3}
\end{array} \right.\]
evaluate the following definite integral by interpreting it in terms of areas.  

\input{Integral-Compute-0005.HELP.tex}

\[
\int_{-3}^{1} f(x)\;dx= \answer{10}
\]  
\end{problem}}%}

\latexProblemContent{
\ifVerboseLocation This is Integration Compute Question 0005. \\ \fi
\begin{problem}

Given the piecewise function 
\[f(x)=\left\{\begin{array}{ll}
{3 \, x + 1}\; , & x\leq{-5}\\[3pt]
{4 \, x + 6}\; , & {-5}< x< {2}\\[3pt]
{3 \, x + 8}\; , & x\geq{2}
\end{array} \right.\]
evaluate the following definite integral by interpreting it in terms of areas.  

\input{Integral-Compute-0005.HELP.tex}

\[
\int_{-7}^{6} f(x)\;dx= \answer{46}
\]  
\end{problem}}%}

\latexProblemContent{
\ifVerboseLocation This is Integration Compute Question 0005. \\ \fi
\begin{problem}

Given the piecewise function 
\[f(x)=\left\{\begin{array}{ll}
{4 \, x + 4}\; , & x\leq{-1}\\[3pt]
{0}\; , & {-1}< x< {5}\\[3pt]
{x - 5}\; , & x\geq{5}
\end{array} \right.\]
evaluate the following definite integral by interpreting it in terms of areas.  

\input{Integral-Compute-0005.HELP.tex}

\[
\int_{-3}^{7} f(x)\;dx= \answer{-6}
\]  
\end{problem}}%}

\latexProblemContent{
\ifVerboseLocation This is Integration Compute Question 0005. \\ \fi
\begin{problem}

Given the piecewise function 
\[f(x)=\left\{\begin{array}{ll}
{3 \, x + 4}\; , & x\leq{-4}\\[3pt]
{\frac{8}{3} \, x + \frac{8}{3}}\; , & {-4}< x< {2}\\[3pt]
{x + 6}\; , & x\geq{2}
\end{array} \right.\]
evaluate the following definite integral by interpreting it in terms of areas.  

\input{Integral-Compute-0005.HELP.tex}

\[
\int_{-4}^{1} f(x)\;dx= \answer{-\frac{20}{3}}
\]  
\end{problem}}%}

\latexProblemContent{
\ifVerboseLocation This is Integration Compute Question 0005. \\ \fi
\begin{problem}

Given the piecewise function 
\[f(x)=\left\{\begin{array}{ll}
{2 \, x + 3}\; , & x\leq{-1}\\[3pt]
{-\frac{1}{3} \, x + \frac{2}{3}}\; , & {-1}< x< {5}\\[3pt]
{3 \, x - 16}\; , & x\geq{5}
\end{array} \right.\]
evaluate the following definite integral by interpreting it in terms of areas.  

\input{Integral-Compute-0005.HELP.tex}

\[
\int_{-1}^{4} f(x)\;dx= \answer{\frac{5}{6}}
\]  
\end{problem}}%}

\latexProblemContent{
\ifVerboseLocation This is Integration Compute Question 0005. \\ \fi
\begin{problem}

Given the piecewise function 
\[f(x)=\left\{\begin{array}{ll}
{2 \, x + 3}\; , & x\leq{5}\\[3pt]
{-\frac{13}{3} \, x + \frac{104}{3}}\; , & {5}< x< {11}\\[3pt]
{x - 24}\; , & x\geq{11}
\end{array} \right.\]
evaluate the following definite integral by interpreting it in terms of areas.  

\input{Integral-Compute-0005.HELP.tex}

\[
\int_{1}^{7} f(x)\;dx= \answer{\frac{160}{3}}
\]  
\end{problem}}%}

\latexProblemContent{
\ifVerboseLocation This is Integration Compute Question 0005. \\ \fi
\begin{problem}

Given the piecewise function 
\[f(x)=\left\{\begin{array}{ll}
{3 \, x + 3}\; , & x\leq{4}\\[3pt]
{-6 \, x + 39}\; , & {4}< x< {9}\\[3pt]
{2 \, x - 33}\; , & x\geq{9}
\end{array} \right.\]
evaluate the following definite integral by interpreting it in terms of areas.  

\input{Integral-Compute-0005.HELP.tex}

\[
\int_{3}^{5} f(x)\;dx= \answer{\frac{51}{2}}
\]  
\end{problem}}%}

\latexProblemContent{
\ifVerboseLocation This is Integration Compute Question 0005. \\ \fi
\begin{problem}

Given the piecewise function 
\[f(x)=\left\{\begin{array}{ll}
{3 \, x + 4}\; , & x\leq{5}\\[3pt]
{-\frac{19}{2} \, x + \frac{133}{2}}\; , & {5}< x< {9}\\[3pt]
{3 \, x - 46}\; , & x\geq{9}
\end{array} \right.\]
evaluate the following definite integral by interpreting it in terms of areas.  

\input{Integral-Compute-0005.HELP.tex}

\[
\int_{0}^{10} f(x)\;dx= \answer{40}
\]  
\end{problem}}%}

\latexProblemContent{
\ifVerboseLocation This is Integration Compute Question 0005. \\ \fi
\begin{problem}

Given the piecewise function 
\[f(x)=\left\{\begin{array}{ll}
{2 \, x + 3}\; , & x\leq{-1}\\[3pt]
{-\frac{2}{7} \, x + \frac{5}{7}}\; , & {-1}< x< {6}\\[3pt]
{4 \, x - 25}\; , & x\geq{6}
\end{array} \right.\]
evaluate the following definite integral by interpreting it in terms of areas.  

\input{Integral-Compute-0005.HELP.tex}

\[
\int_{-2}^{10} f(x)\;dx= \answer{28}
\]  
\end{problem}}%}

\latexProblemContent{
\ifVerboseLocation This is Integration Compute Question 0005. \\ \fi
\begin{problem}

Given the piecewise function 
\[f(x)=\left\{\begin{array}{ll}
{3 \, x + 1}\; , & x\leq{-5}\\[3pt]
{\frac{28}{3} \, x + \frac{98}{3}}\; , & {-5}< x< {-2}\\[3pt]
{4 \, x + 22}\; , & x\geq{-2}
\end{array} \right.\]
evaluate the following definite integral by interpreting it in terms of areas.  

\input{Integral-Compute-0005.HELP.tex}

\[
\int_{-7}^{-2} f(x)\;dx= \answer{-34}
\]  
\end{problem}}%}

\latexProblemContent{
\ifVerboseLocation This is Integration Compute Question 0005. \\ \fi
\begin{problem}

Given the piecewise function 
\[f(x)=\left\{\begin{array}{ll}
{2 \, x + 3}\; , & x\leq{-4}\\[3pt]
{2 \, x + 3}\; , & {-4}< x< {1}\\[3pt]
{4 \, x + 1}\; , & x\geq{1}
\end{array} \right.\]
evaluate the following definite integral by interpreting it in terms of areas.  

\input{Integral-Compute-0005.HELP.tex}

\[
\int_{-9}^{0} f(x)\;dx= \answer{-54}
\]  
\end{problem}}%}

\latexProblemContent{
\ifVerboseLocation This is Integration Compute Question 0005. \\ \fi
\begin{problem}

Given the piecewise function 
\[f(x)=\left\{\begin{array}{ll}
{2 \, x + 5}\; , & x\leq{-4}\\[3pt]
{\frac{3}{2} \, x + 3}\; , & {-4}< x< {0}\\[3pt]
{2 \, x + 3}\; , & x\geq{0}
\end{array} \right.\]
evaluate the following definite integral by interpreting it in terms of areas.  

\input{Integral-Compute-0005.HELP.tex}

\[
\int_{-9}^{5} f(x)\;dx= \answer{0}
\]  
\end{problem}}%}

\latexProblemContent{
\ifVerboseLocation This is Integration Compute Question 0005. \\ \fi
\begin{problem}

Given the piecewise function 
\[f(x)=\left\{\begin{array}{ll}
{3 \, x + 5}\; , & x\leq{4}\\[3pt]
{-\frac{17}{2} \, x + 51}\; , & {4}< x< {8}\\[3pt]
{4 \, x - 49}\; , & x\geq{8}
\end{array} \right.\]
evaluate the following definite integral by interpreting it in terms of areas.  

\input{Integral-Compute-0005.HELP.tex}

\[
\int_{0}^{6} f(x)\;dx= \answer{61}
\]  
\end{problem}}%}

\latexProblemContent{
\ifVerboseLocation This is Integration Compute Question 0005. \\ \fi
\begin{problem}

Given the piecewise function 
\[f(x)=\left\{\begin{array}{ll}
{x + 5}\; , & x\leq{-3}\\[3pt]
{-\frac{4}{3} \, x - 2}\; , & {-3}< x< {0}\\[3pt]
{4 \, x - 2}\; , & x\geq{0}
\end{array} \right.\]
evaluate the following definite integral by interpreting it in terms of areas.  

\input{Integral-Compute-0005.HELP.tex}

\[
\int_{-3}^{1} f(x)\;dx= \answer{0}
\]  
\end{problem}}%}

\latexProblemContent{
\ifVerboseLocation This is Integration Compute Question 0005. \\ \fi
\begin{problem}

Given the piecewise function 
\[f(x)=\left\{\begin{array}{ll}
{3 \, x + 3}\; , & x\leq{3}\\[3pt]
{-3 \, x + 21}\; , & {3}< x< {11}\\[3pt]
{x - 23}\; , & x\geq{11}
\end{array} \right.\]
evaluate the following definite integral by interpreting it in terms of areas.  

\input{Integral-Compute-0005.HELP.tex}

\[
\int_{0}^{16} f(x)\;dx= \answer{-25}
\]  
\end{problem}}%}

\latexProblemContent{
\ifVerboseLocation This is Integration Compute Question 0005. \\ \fi
\begin{problem}

Given the piecewise function 
\[f(x)=\left\{\begin{array}{ll}
{3 \, x + 3}\; , & x\leq{5}\\[3pt]
{-9 \, x + 63}\; , & {5}< x< {9}\\[3pt]
{4 \, x - 54}\; , & x\geq{9}
\end{array} \right.\]
evaluate the following definite integral by interpreting it in terms of areas.  

\input{Integral-Compute-0005.HELP.tex}

\[
\int_{2}^{13} f(x)\;dx= \answer{\frac{1}{2}}
\]  
\end{problem}}%}

\latexProblemContent{
\ifVerboseLocation This is Integration Compute Question 0005. \\ \fi
\begin{problem}

Given the piecewise function 
\[f(x)=\left\{\begin{array}{ll}
{2 \, x + 2}\; , & x\leq{-3}\\[3pt]
{\frac{8}{5} \, x + \frac{4}{5}}\; , & {-3}< x< {2}\\[3pt]
{4 \, x - 4}\; , & x\geq{2}
\end{array} \right.\]
evaluate the following definite integral by interpreting it in terms of areas.  

\input{Integral-Compute-0005.HELP.tex}

\[
\int_{-5}^{-1} f(x)\;dx= \answer{-\frac{84}{5}}
\]  
\end{problem}}%}

\latexProblemContent{
\ifVerboseLocation This is Integration Compute Question 0005. \\ \fi
\begin{problem}

Given the piecewise function 
\[f(x)=\left\{\begin{array}{ll}
{2 \, x + 4}\; , & x\leq{-1}\\[3pt]
{-\frac{2}{3} \, x + \frac{4}{3}}\; , & {-1}< x< {5}\\[3pt]
{x - 7}\; , & x\geq{5}
\end{array} \right.\]
evaluate the following definite integral by interpreting it in terms of areas.  

\input{Integral-Compute-0005.HELP.tex}

\[
\int_{-4}^{6} f(x)\;dx= \answer{-\frac{9}{2}}
\]  
\end{problem}}%}

\latexProblemContent{
\ifVerboseLocation This is Integration Compute Question 0005. \\ \fi
\begin{problem}

Given the piecewise function 
\[f(x)=\left\{\begin{array}{ll}
{x + 5}\; , & x\leq{5}\\[3pt]
{-\frac{5}{2} \, x + \frac{45}{2}}\; , & {5}< x< {13}\\[3pt]
{3 \, x - 49}\; , & x\geq{13}
\end{array} \right.\]
evaluate the following definite integral by interpreting it in terms of areas.  

\input{Integral-Compute-0005.HELP.tex}

\[
\int_{1}^{17} f(x)\;dx= \answer{16}
\]  
\end{problem}}%}

\latexProblemContent{
\ifVerboseLocation This is Integration Compute Question 0005. \\ \fi
\begin{problem}

Given the piecewise function 
\[f(x)=\left\{\begin{array}{ll}
{3 \, x + 3}\; , & x\leq{1}\\[3pt]
{-\frac{12}{7} \, x + \frac{54}{7}}\; , & {1}< x< {8}\\[3pt]
{2 \, x - 22}\; , & x\geq{8}
\end{array} \right.\]
evaluate the following definite integral by interpreting it in terms of areas.  

\input{Integral-Compute-0005.HELP.tex}

\[
\int_{-2}^{5} f(x)\;dx= \answer{\frac{207}{14}}
\]  
\end{problem}}%}

\latexProblemContent{
\ifVerboseLocation This is Integration Compute Question 0005. \\ \fi
\begin{problem}

Given the piecewise function 
\[f(x)=\left\{\begin{array}{ll}
{2 \, x + 1}\; , & x\leq{3}\\[3pt]
{-\frac{14}{3} \, x + 21}\; , & {3}< x< {6}\\[3pt]
{x - 13}\; , & x\geq{6}
\end{array} \right.\]
evaluate the following definite integral by interpreting it in terms of areas.  

\input{Integral-Compute-0005.HELP.tex}

\[
\int_{0}^{8} f(x)\;dx= \answer{0}
\]  
\end{problem}}%}

\latexProblemContent{
\ifVerboseLocation This is Integration Compute Question 0005. \\ \fi
\begin{problem}

Given the piecewise function 
\[f(x)=\left\{\begin{array}{ll}
{3 \, x + 2}\; , & x\leq{3}\\[3pt]
{-\frac{22}{5} \, x + \frac{121}{5}}\; , & {3}< x< {8}\\[3pt]
{x - 19}\; , & x\geq{8}
\end{array} \right.\]
evaluate the following definite integral by interpreting it in terms of areas.  

\input{Integral-Compute-0005.HELP.tex}

\[
\int_{-2}^{10} f(x)\;dx= \answer{-\frac{5}{2}}
\]  
\end{problem}}%}

\latexProblemContent{
\ifVerboseLocation This is Integration Compute Question 0005. \\ \fi
\begin{problem}

Given the piecewise function 
\[f(x)=\left\{\begin{array}{ll}
{x + 1}\; , & x\leq{5}\\[3pt]
{-3 \, x + 21}\; , & {5}< x< {9}\\[3pt]
{4 \, x - 42}\; , & x\geq{9}
\end{array} \right.\]
evaluate the following definite integral by interpreting it in terms of areas.  

\input{Integral-Compute-0005.HELP.tex}

\[
\int_{1}^{14} f(x)\;dx= \answer{36}
\]  
\end{problem}}%}

\latexProblemContent{
\ifVerboseLocation This is Integration Compute Question 0005. \\ \fi
\begin{problem}

Given the piecewise function 
\[f(x)=\left\{\begin{array}{ll}
{4 \, x + 4}\; , & x\leq{-1}\\[3pt]
{0}\; , & {-1}< x< {6}\\[3pt]
{2 \, x - 12}\; , & x\geq{6}
\end{array} \right.\]
evaluate the following definite integral by interpreting it in terms of areas.  

\input{Integral-Compute-0005.HELP.tex}

\[
\int_{-3}^{2} f(x)\;dx= \answer{-8}
\]  
\end{problem}}%}

\latexProblemContent{
\ifVerboseLocation This is Integration Compute Question 0005. \\ \fi
\begin{problem}

Given the piecewise function 
\[f(x)=\left\{\begin{array}{ll}
{x + 3}\; , & x\leq{3}\\[3pt]
{-\frac{12}{7} \, x + \frac{78}{7}}\; , & {3}< x< {10}\\[3pt]
{4 \, x - 46}\; , & x\geq{10}
\end{array} \right.\]
evaluate the following definite integral by interpreting it in terms of areas.  

\input{Integral-Compute-0005.HELP.tex}

\[
\int_{2}^{11} f(x)\;dx= \answer{\frac{3}{2}}
\]  
\end{problem}}%}

\latexProblemContent{
\ifVerboseLocation This is Integration Compute Question 0005. \\ \fi
\begin{problem}

Given the piecewise function 
\[f(x)=\left\{\begin{array}{ll}
{x + 1}\; , & x\leq{1}\\[3pt]
{-\frac{2}{3} \, x + \frac{8}{3}}\; , & {1}< x< {7}\\[3pt]
{x - 9}\; , & x\geq{7}
\end{array} \right.\]
evaluate the following definite integral by interpreting it in terms of areas.  

\input{Integral-Compute-0005.HELP.tex}

\[
\int_{-2}^{9} f(x)\;dx= \answer{-\frac{1}{2}}
\]  
\end{problem}}%}

\latexProblemContent{
\ifVerboseLocation This is Integration Compute Question 0005. \\ \fi
\begin{problem}

Given the piecewise function 
\[f(x)=\left\{\begin{array}{ll}
{x + 3}\; , & x\leq{3}\\[3pt]
{-3 \, x + 15}\; , & {3}< x< {7}\\[3pt]
{x - 13}\; , & x\geq{7}
\end{array} \right.\]
evaluate the following definite integral by interpreting it in terms of areas.  

\input{Integral-Compute-0005.HELP.tex}

\[
\int_{-1}^{10} f(x)\;dx= \answer{\frac{5}{2}}
\]  
\end{problem}}%}

\latexProblemContent{
\ifVerboseLocation This is Integration Compute Question 0005. \\ \fi
\begin{problem}

Given the piecewise function 
\[f(x)=\left\{\begin{array}{ll}
{2 \, x + 4}\; , & x\leq{-4}\\[3pt]
{\frac{8}{7} \, x + \frac{4}{7}}\; , & {-4}< x< {3}\\[3pt]
{2 \, x - 2}\; , & x\geq{3}
\end{array} \right.\]
evaluate the following definite integral by interpreting it in terms of areas.  

\input{Integral-Compute-0005.HELP.tex}

\[
\int_{-6}^{0} f(x)\;dx= \answer{-\frac{132}{7}}
\]  
\end{problem}}%}

\latexProblemContent{
\ifVerboseLocation This is Integration Compute Question 0005. \\ \fi
\begin{problem}

Given the piecewise function 
\[f(x)=\left\{\begin{array}{ll}
{3 \, x + 3}\; , & x\leq{-2}\\[3pt]
{x - 1}\; , & {-2}< x< {4}\\[3pt]
{2 \, x - 5}\; , & x\geq{4}
\end{array} \right.\]
evaluate the following definite integral by interpreting it in terms of areas.  

\input{Integral-Compute-0005.HELP.tex}

\[
\int_{-6}^{9} f(x)\;dx= \answer{4}
\]  
\end{problem}}%}

\latexProblemContent{
\ifVerboseLocation This is Integration Compute Question 0005. \\ \fi
\begin{problem}

Given the piecewise function 
\[f(x)=\left\{\begin{array}{ll}
{2 \, x + 2}\; , & x\leq{3}\\[3pt]
{-4 \, x + 20}\; , & {3}< x< {7}\\[3pt]
{x - 15}\; , & x\geq{7}
\end{array} \right.\]
evaluate the following definite integral by interpreting it in terms of areas.  

\input{Integral-Compute-0005.HELP.tex}

\[
\int_{2}^{7} f(x)\;dx= \answer{7}
\]  
\end{problem}}%}

\latexProblemContent{
\ifVerboseLocation This is Integration Compute Question 0005. \\ \fi
\begin{problem}

Given the piecewise function 
\[f(x)=\left\{\begin{array}{ll}
{4 \, x + 2}\; , & x\leq{4}\\[3pt]
{-12 \, x + 66}\; , & {4}< x< {7}\\[3pt]
{x - 25}\; , & x\geq{7}
\end{array} \right.\]
evaluate the following definite integral by interpreting it in terms of areas.  

\input{Integral-Compute-0005.HELP.tex}

\[
\int_{3}^{12} f(x)\;dx= \answer{-\frac{123}{2}}
\]  
\end{problem}}%}

\latexProblemContent{
\ifVerboseLocation This is Integration Compute Question 0005. \\ \fi
\begin{problem}

Given the piecewise function 
\[f(x)=\left\{\begin{array}{ll}
{3 \, x + 5}\; , & x\leq{2}\\[3pt]
{-\frac{22}{3} \, x + \frac{77}{3}}\; , & {2}< x< {5}\\[3pt]
{x - 16}\; , & x\geq{5}
\end{array} \right.\]
evaluate the following definite integral by interpreting it in terms of areas.  

\input{Integral-Compute-0005.HELP.tex}

\[
\int_{2}^{8} f(x)\;dx= \answer{-\frac{57}{2}}
\]  
\end{problem}}%}

\latexProblemContent{
\ifVerboseLocation This is Integration Compute Question 0005. \\ \fi
\begin{problem}

Given the piecewise function 
\[f(x)=\left\{\begin{array}{ll}
{3 \, x + 1}\; , & x\leq{-5}\\[3pt]
{7 \, x + 21}\; , & {-5}< x< {-1}\\[3pt]
{2 \, x + 16}\; , & x\geq{-1}
\end{array} \right.\]
evaluate the following definite integral by interpreting it in terms of areas.  

\input{Integral-Compute-0005.HELP.tex}

\[
\int_{-7}^{1} f(x)\;dx= \answer{-2}
\]  
\end{problem}}%}

\latexProblemContent{
\ifVerboseLocation This is Integration Compute Question 0005. \\ \fi
\begin{problem}

Given the piecewise function 
\[f(x)=\left\{\begin{array}{ll}
{x + 3}\; , & x\leq{2}\\[3pt]
{-\frac{10}{3} \, x + \frac{35}{3}}\; , & {2}< x< {5}\\[3pt]
{x - 10}\; , & x\geq{5}
\end{array} \right.\]
evaluate the following definite integral by interpreting it in terms of areas.  

\input{Integral-Compute-0005.HELP.tex}

\[
\int_{1}^{4} f(x)\;dx= \answer{\frac{47}{6}}
\]  
\end{problem}}%}

\latexProblemContent{
\ifVerboseLocation This is Integration Compute Question 0005. \\ \fi
\begin{problem}

Given the piecewise function 
\[f(x)=\left\{\begin{array}{ll}
{x + 2}\; , & x\leq{-3}\\[3pt]
{\frac{2}{3} \, x + 1}\; , & {-3}< x< {0}\\[3pt]
{3 \, x + 1}\; , & x\geq{0}
\end{array} \right.\]
evaluate the following definite integral by interpreting it in terms of areas.  

\input{Integral-Compute-0005.HELP.tex}

\[
\int_{-7}^{0} f(x)\;dx= \answer{-12}
\]  
\end{problem}}%}

\latexProblemContent{
\ifVerboseLocation This is Integration Compute Question 0005. \\ \fi
\begin{problem}

Given the piecewise function 
\[f(x)=\left\{\begin{array}{ll}
{2 \, x + 4}\; , & x\leq{3}\\[3pt]
{-4 \, x + 22}\; , & {3}< x< {8}\\[3pt]
{4 \, x - 42}\; , & x\geq{8}
\end{array} \right.\]
evaluate the following definite integral by interpreting it in terms of areas.  

\input{Integral-Compute-0005.HELP.tex}

\[
\int_{2}^{6} f(x)\;dx= \answer{21}
\]  
\end{problem}}%}

\latexProblemContent{
\ifVerboseLocation This is Integration Compute Question 0005. \\ \fi
\begin{problem}

Given the piecewise function 
\[f(x)=\left\{\begin{array}{ll}
{2 \, x + 3}\; , & x\leq{-3}\\[3pt]
{\frac{3}{4} \, x - \frac{3}{4}}\; , & {-3}< x< {5}\\[3pt]
{4 \, x - 17}\; , & x\geq{5}
\end{array} \right.\]
evaluate the following definite integral by interpreting it in terms of areas.  

\input{Integral-Compute-0005.HELP.tex}

\[
\int_{-8}^{6} f(x)\;dx= \answer{-35}
\]  
\end{problem}}%}

\latexProblemContent{
\ifVerboseLocation This is Integration Compute Question 0005. \\ \fi
\begin{problem}

Given the piecewise function 
\[f(x)=\left\{\begin{array}{ll}
{4 \, x + 2}\; , & x\leq{-5}\\[3pt]
{\frac{36}{5} \, x + 18}\; , & {-5}< x< {0}\\[3pt]
{2 \, x + 18}\; , & x\geq{0}
\end{array} \right.\]
evaluate the following definite integral by interpreting it in terms of areas.  

\input{Integral-Compute-0005.HELP.tex}

\[
\int_{-8}^{0} f(x)\;dx= \answer{-72}
\]  
\end{problem}}%}

\latexProblemContent{
\ifVerboseLocation This is Integration Compute Question 0005. \\ \fi
\begin{problem}

Given the piecewise function 
\[f(x)=\left\{\begin{array}{ll}
{x + 2}\; , & x\leq{1}\\[3pt]
{-x + 4}\; , & {1}< x< {7}\\[3pt]
{3 \, x - 24}\; , & x\geq{7}
\end{array} \right.\]
evaluate the following definite integral by interpreting it in terms of areas.  

\input{Integral-Compute-0005.HELP.tex}

\[
\int_{1}^{12} f(x)\;dx= \answer{\frac{45}{2}}
\]  
\end{problem}}%}

\latexProblemContent{
\ifVerboseLocation This is Integration Compute Question 0005. \\ \fi
\begin{problem}

Given the piecewise function 
\[f(x)=\left\{\begin{array}{ll}
{2 \, x + 2}\; , & x\leq{-2}\\[3pt]
{x}\; , & {-2}< x< {2}\\[3pt]
{2 \, x - 2}\; , & x\geq{2}
\end{array} \right.\]
evaluate the following definite integral by interpreting it in terms of areas.  

\input{Integral-Compute-0005.HELP.tex}

\[
\int_{-3}^{-1} f(x)\;dx= \answer{-\frac{9}{2}}
\]  
\end{problem}}%}

\latexProblemContent{
\ifVerboseLocation This is Integration Compute Question 0005. \\ \fi
\begin{problem}

Given the piecewise function 
\[f(x)=\left\{\begin{array}{ll}
{4 \, x + 1}\; , & x\leq{2}\\[3pt]
{-3 \, x + 15}\; , & {2}< x< {8}\\[3pt]
{3 \, x - 33}\; , & x\geq{8}
\end{array} \right.\]
evaluate the following definite integral by interpreting it in terms of areas.  

\input{Integral-Compute-0005.HELP.tex}

\[
\int_{-2}^{12} f(x)\;dx= \answer{-8}
\]  
\end{problem}}%}

\latexProblemContent{
\ifVerboseLocation This is Integration Compute Question 0005. \\ \fi
\begin{problem}

Given the piecewise function 
\[f(x)=\left\{\begin{array}{ll}
{4 \, x + 2}\; , & x\leq{-3}\\[3pt]
{5 \, x + 5}\; , & {-3}< x< {1}\\[3pt]
{2 \, x + 8}\; , & x\geq{1}
\end{array} \right.\]
evaluate the following definite integral by interpreting it in terms of areas.  

\input{Integral-Compute-0005.HELP.tex}

\[
\int_{-3}^{1} f(x)\;dx= \answer{0}
\]  
\end{problem}}%}

\latexProblemContent{
\ifVerboseLocation This is Integration Compute Question 0005. \\ \fi
\begin{problem}

Given the piecewise function 
\[f(x)=\left\{\begin{array}{ll}
{2 \, x + 1}\; , & x\leq{-4}\\[3pt]
{\frac{14}{5} \, x + \frac{21}{5}}\; , & {-4}< x< {1}\\[3pt]
{x + 6}\; , & x\geq{1}
\end{array} \right.\]
evaluate the following definite integral by interpreting it in terms of areas.  

\input{Integral-Compute-0005.HELP.tex}

\[
\int_{-7}^{3} f(x)\;dx= \answer{-14}
\]  
\end{problem}}%}

\latexProblemContent{
\ifVerboseLocation This is Integration Compute Question 0005. \\ \fi
\begin{problem}

Given the piecewise function 
\[f(x)=\left\{\begin{array}{ll}
{x + 5}\; , & x\leq{-2}\\[3pt]
{-\frac{3}{4} \, x + \frac{3}{2}}\; , & {-2}< x< {6}\\[3pt]
{x - 9}\; , & x\geq{6}
\end{array} \right.\]
evaluate the following definite integral by interpreting it in terms of areas.  

\input{Integral-Compute-0005.HELP.tex}

\[
\int_{-7}^{6} f(x)\;dx= \answer{\frac{5}{2}}
\]  
\end{problem}}%}

\latexProblemContent{
\ifVerboseLocation This is Integration Compute Question 0005. \\ \fi
\begin{problem}

Given the piecewise function 
\[f(x)=\left\{\begin{array}{ll}
{2 \, x + 3}\; , & x\leq{3}\\[3pt]
{-\frac{9}{2} \, x + \frac{45}{2}}\; , & {3}< x< {7}\\[3pt]
{2 \, x - 23}\; , & x\geq{7}
\end{array} \right.\]
evaluate the following definite integral by interpreting it in terms of areas.  

\input{Integral-Compute-0005.HELP.tex}

\[
\int_{-1}^{12} f(x)\;dx= \answer{0}
\]  
\end{problem}}%}

\latexProblemContent{
\ifVerboseLocation This is Integration Compute Question 0005. \\ \fi
\begin{problem}

Given the piecewise function 
\[f(x)=\left\{\begin{array}{ll}
{x + 3}\; , & x\leq{4}\\[3pt]
{-\frac{7}{2} \, x + 21}\; , & {4}< x< {8}\\[3pt]
{3 \, x - 31}\; , & x\geq{8}
\end{array} \right.\]
evaluate the following definite integral by interpreting it in terms of areas.  

\input{Integral-Compute-0005.HELP.tex}

\[
\int_{1}^{10} f(x)\;dx= \answer{\frac{17}{2}}
\]  
\end{problem}}%}

\latexProblemContent{
\ifVerboseLocation This is Integration Compute Question 0005. \\ \fi
\begin{problem}

Given the piecewise function 
\[f(x)=\left\{\begin{array}{ll}
{4 \, x + 2}\; , & x\leq{-1}\\[3pt]
{\frac{4}{5} \, x - \frac{6}{5}}\; , & {-1}< x< {4}\\[3pt]
{x - 2}\; , & x\geq{4}
\end{array} \right.\]
evaluate the following definite integral by interpreting it in terms of areas.  

\input{Integral-Compute-0005.HELP.tex}

\[
\int_{-6}^{1} f(x)\;dx= \answer{-\frac{312}{5}}
\]  
\end{problem}}%}

\latexProblemContent{
\ifVerboseLocation This is Integration Compute Question 0005. \\ \fi
\begin{problem}

Given the piecewise function 
\[f(x)=\left\{\begin{array}{ll}
{4 \, x + 1}\; , & x\leq{5}\\[3pt]
{-6 \, x + 51}\; , & {5}< x< {12}\\[3pt]
{2 \, x - 45}\; , & x\geq{12}
\end{array} \right.\]
evaluate the following definite integral by interpreting it in terms of areas.  

\input{Integral-Compute-0005.HELP.tex}

\[
\int_{2}^{11} f(x)\;dx= \answer{63}
\]  
\end{problem}}%}

\latexProblemContent{
\ifVerboseLocation This is Integration Compute Question 0005. \\ \fi
\begin{problem}

Given the piecewise function 
\[f(x)=\left\{\begin{array}{ll}
{4 \, x + 3}\; , & x\leq{-4}\\[3pt]
{\frac{26}{3} \, x + \frac{65}{3}}\; , & {-4}< x< {-1}\\[3pt]
{3 \, x + 16}\; , & x\geq{-1}
\end{array} \right.\]
evaluate the following definite integral by interpreting it in terms of areas.  

\input{Integral-Compute-0005.HELP.tex}

\[
\int_{-6}^{1} f(x)\;dx= \answer{-2}
\]  
\end{problem}}%}

\latexProblemContent{
\ifVerboseLocation This is Integration Compute Question 0005. \\ \fi
\begin{problem}

Given the piecewise function 
\[f(x)=\left\{\begin{array}{ll}
{3 \, x + 3}\; , & x\leq{-5}\\[3pt]
{\frac{24}{5} \, x + 12}\; , & {-5}< x< {0}\\[3pt]
{2 \, x + 12}\; , & x\geq{0}
\end{array} \right.\]
evaluate the following definite integral by interpreting it in terms of areas.  

\input{Integral-Compute-0005.HELP.tex}

\[
\int_{-5}^{5} f(x)\;dx= \answer{85}
\]  
\end{problem}}%}

\latexProblemContent{
\ifVerboseLocation This is Integration Compute Question 0005. \\ \fi
\begin{problem}

Given the piecewise function 
\[f(x)=\left\{\begin{array}{ll}
{2 \, x + 3}\; , & x\leq{-4}\\[3pt]
{2 \, x + 3}\; , & {-4}< x< {1}\\[3pt]
{4 \, x + 1}\; , & x\geq{1}
\end{array} \right.\]
evaluate the following definite integral by interpreting it in terms of areas.  

\input{Integral-Compute-0005.HELP.tex}

\[
\int_{-4}^{-2} f(x)\;dx= \answer{-6}
\]  
\end{problem}}%}

\latexProblemContent{
\ifVerboseLocation This is Integration Compute Question 0005. \\ \fi
\begin{problem}

Given the piecewise function 
\[f(x)=\left\{\begin{array}{ll}
{4 \, x + 5}\; , & x\leq{-5}\\[3pt]
{\frac{15}{2} \, x + \frac{45}{2}}\; , & {-5}< x< {-1}\\[3pt]
{2 \, x + 17}\; , & x\geq{-1}
\end{array} \right.\]
evaluate the following definite integral by interpreting it in terms of areas.  

\input{Integral-Compute-0005.HELP.tex}

\[
\int_{-7}^{4} f(x)\;dx= \answer{62}
\]  
\end{problem}}%}

\latexProblemContent{
\ifVerboseLocation This is Integration Compute Question 0005. \\ \fi
\begin{problem}

Given the piecewise function 
\[f(x)=\left\{\begin{array}{ll}
{3 \, x + 1}\; , & x\leq{2}\\[3pt]
{-2 \, x + 11}\; , & {2}< x< {9}\\[3pt]
{2 \, x - 25}\; , & x\geq{9}
\end{array} \right.\]
evaluate the following definite integral by interpreting it in terms of areas.  

\input{Integral-Compute-0005.HELP.tex}

\[
\int_{-3}^{8} f(x)\;dx= \answer{\frac{7}{2}}
\]  
\end{problem}}%}

\latexProblemContent{
\ifVerboseLocation This is Integration Compute Question 0005. \\ \fi
\begin{problem}

Given the piecewise function 
\[f(x)=\left\{\begin{array}{ll}
{2 \, x + 2}\; , & x\leq{-1}\\[3pt]
{0}\; , & {-1}< x< {2}\\[3pt]
{2 \, x - 4}\; , & x\geq{2}
\end{array} \right.\]
evaluate the following definite integral by interpreting it in terms of areas.  

\input{Integral-Compute-0005.HELP.tex}

\[
\int_{-1}^{2} f(x)\;dx= \answer{0}
\]  
\end{problem}}%}

\latexProblemContent{
\ifVerboseLocation This is Integration Compute Question 0005. \\ \fi
\begin{problem}

Given the piecewise function 
\[f(x)=\left\{\begin{array}{ll}
{4 \, x + 4}\; , & x\leq{-1}\\[3pt]
{0}\; , & {-1}< x< {3}\\[3pt]
{4 \, x - 12}\; , & x\geq{3}
\end{array} \right.\]
evaluate the following definite integral by interpreting it in terms of areas.  

\input{Integral-Compute-0005.HELP.tex}

\[
\int_{-3}^{2} f(x)\;dx= \answer{-8}
\]  
\end{problem}}%}

\latexProblemContent{
\ifVerboseLocation This is Integration Compute Question 0005. \\ \fi
\begin{problem}

Given the piecewise function 
\[f(x)=\left\{\begin{array}{ll}
{x + 4}\; , & x\leq{2}\\[3pt]
{-3 \, x + 12}\; , & {2}< x< {6}\\[3pt]
{2 \, x - 18}\; , & x\geq{6}
\end{array} \right.\]
evaluate the following definite integral by interpreting it in terms of areas.  

\input{Integral-Compute-0005.HELP.tex}

\[
\int_{-3}^{6} f(x)\;dx= \answer{\frac{35}{2}}
\]  
\end{problem}}%}

\latexProblemContent{
\ifVerboseLocation This is Integration Compute Question 0005. \\ \fi
\begin{problem}

Given the piecewise function 
\[f(x)=\left\{\begin{array}{ll}
{4 \, x + 3}\; , & x\leq{-3}\\[3pt]
{\frac{9}{4} \, x - \frac{9}{4}}\; , & {-3}< x< {5}\\[3pt]
{4 \, x - 11}\; , & x\geq{5}
\end{array} \right.\]
evaluate the following definite integral by interpreting it in terms of areas.  

\input{Integral-Compute-0005.HELP.tex}

\[
\int_{-4}^{6} f(x)\;dx= \answer{0}
\]  
\end{problem}}%}

\latexProblemContent{
\ifVerboseLocation This is Integration Compute Question 0005. \\ \fi
\begin{problem}

Given the piecewise function 
\[f(x)=\left\{\begin{array}{ll}
{3 \, x + 2}\; , & x\leq{-3}\\[3pt]
{\frac{7}{2} \, x + \frac{7}{2}}\; , & {-3}< x< {1}\\[3pt]
{2 \, x + 5}\; , & x\geq{1}
\end{array} \right.\]
evaluate the following definite integral by interpreting it in terms of areas.  

\input{Integral-Compute-0005.HELP.tex}

\[
\int_{-5}^{-2} f(x)\;dx= \answer{-\frac{101}{4}}
\]  
\end{problem}}%}

\latexProblemContent{
\ifVerboseLocation This is Integration Compute Question 0005. \\ \fi
\begin{problem}

Given the piecewise function 
\[f(x)=\left\{\begin{array}{ll}
{4 \, x + 2}\; , & x\leq{3}\\[3pt]
{-\frac{28}{3} \, x + 42}\; , & {3}< x< {6}\\[3pt]
{2 \, x - 26}\; , & x\geq{6}
\end{array} \right.\]
evaluate the following definite integral by interpreting it in terms of areas.  

\input{Integral-Compute-0005.HELP.tex}

\[
\int_{-1}^{11} f(x)\;dx= \answer{-21}
\]  
\end{problem}}%}

\latexProblemContent{
\ifVerboseLocation This is Integration Compute Question 0005. \\ \fi
\begin{problem}

Given the piecewise function 
\[f(x)=\left\{\begin{array}{ll}
{2 \, x + 2}\; , & x\leq{1}\\[3pt]
{-x + 5}\; , & {1}< x< {9}\\[3pt]
{3 \, x - 31}\; , & x\geq{9}
\end{array} \right.\]
evaluate the following definite integral by interpreting it in terms of areas.  

\input{Integral-Compute-0005.HELP.tex}

\[
\int_{-3}^{8} f(x)\;dx= \answer{\frac{7}{2}}
\]  
\end{problem}}%}

\latexProblemContent{
\ifVerboseLocation This is Integration Compute Question 0005. \\ \fi
\begin{problem}

Given the piecewise function 
\[f(x)=\left\{\begin{array}{ll}
{3 \, x + 5}\; , & x\leq{4}\\[3pt]
{-\frac{17}{3} \, x + \frac{119}{3}}\; , & {4}< x< {10}\\[3pt]
{2 \, x - 37}\; , & x\geq{10}
\end{array} \right.\]
evaluate the following definite integral by interpreting it in terms of areas.  

\input{Integral-Compute-0005.HELP.tex}

\[
\int_{4}^{6} f(x)\;dx= \answer{\frac{68}{3}}
\]  
\end{problem}}%}

\latexProblemContent{
\ifVerboseLocation This is Integration Compute Question 0005. \\ \fi
\begin{problem}

Given the piecewise function 
\[f(x)=\left\{\begin{array}{ll}
{x + 1}\; , & x\leq{-4}\\[3pt]
{2 \, x + 5}\; , & {-4}< x< {-1}\\[3pt]
{4 \, x + 7}\; , & x\geq{-1}
\end{array} \right.\]
evaluate the following definite integral by interpreting it in terms of areas.  

\input{Integral-Compute-0005.HELP.tex}

\[
\int_{-8}^{3} f(x)\;dx= \answer{24}
\]  
\end{problem}}%}

\latexProblemContent{
\ifVerboseLocation This is Integration Compute Question 0005. \\ \fi
\begin{problem}

Given the piecewise function 
\[f(x)=\left\{\begin{array}{ll}
{4 \, x + 5}\; , & x\leq{4}\\[3pt]
{-\frac{21}{4} \, x + 42}\; , & {4}< x< {12}\\[3pt]
{4 \, x - 69}\; , & x\geq{12}
\end{array} \right.\]
evaluate the following definite integral by interpreting it in terms of areas.  

\input{Integral-Compute-0005.HELP.tex}

\[
\int_{-1}^{12} f(x)\;dx= \answer{55}
\]  
\end{problem}}%}

\latexProblemContent{
\ifVerboseLocation This is Integration Compute Question 0005. \\ \fi
\begin{problem}

Given the piecewise function 
\[f(x)=\left\{\begin{array}{ll}
{x + 3}\; , & x\leq{1}\\[3pt]
{-2 \, x + 6}\; , & {1}< x< {5}\\[3pt]
{4 \, x - 24}\; , & x\geq{5}
\end{array} \right.\]
evaluate the following definite integral by interpreting it in terms of areas.  

\input{Integral-Compute-0005.HELP.tex}

\[
\int_{-4}^{7} f(x)\;dx= \answer{\frac{15}{2}}
\]  
\end{problem}}%}

\latexProblemContent{
\ifVerboseLocation This is Integration Compute Question 0005. \\ \fi
\begin{problem}

Given the piecewise function 
\[f(x)=\left\{\begin{array}{ll}
{3 \, x + 5}\; , & x\leq{4}\\[3pt]
{-\frac{34}{7} \, x + \frac{255}{7}}\; , & {4}< x< {11}\\[3pt]
{3 \, x - 50}\; , & x\geq{11}
\end{array} \right.\]
evaluate the following definite integral by interpreting it in terms of areas.  

\input{Integral-Compute-0005.HELP.tex}

\[
\int_{-1}^{13} f(x)\;dx= \answer{\frac{39}{2}}
\]  
\end{problem}}%}

\latexProblemContent{
\ifVerboseLocation This is Integration Compute Question 0005. \\ \fi
\begin{problem}

Given the piecewise function 
\[f(x)=\left\{\begin{array}{ll}
{2 \, x + 3}\; , & x\leq{5}\\[3pt]
{-\frac{13}{4} \, x + \frac{117}{4}}\; , & {5}< x< {13}\\[3pt]
{2 \, x - 39}\; , & x\geq{13}
\end{array} \right.\]
evaluate the following definite integral by interpreting it in terms of areas.  

\input{Integral-Compute-0005.HELP.tex}

\[
\int_{4}^{9} f(x)\;dx= \answer{38}
\]  
\end{problem}}%}

\latexProblemContent{
\ifVerboseLocation This is Integration Compute Question 0005. \\ \fi
\begin{problem}

Given the piecewise function 
\[f(x)=\left\{\begin{array}{ll}
{3 \, x + 4}\; , & x\leq{1}\\[3pt]
{-\frac{7}{2} \, x + \frac{21}{2}}\; , & {1}< x< {5}\\[3pt]
{3 \, x - 22}\; , & x\geq{5}
\end{array} \right.\]
evaluate the following definite integral by interpreting it in terms of areas.  

\input{Integral-Compute-0005.HELP.tex}

\[
\int_{-3}^{6} f(x)\;dx= \answer{-\frac{3}{2}}
\]  
\end{problem}}%}

\latexProblemContent{
\ifVerboseLocation This is Integration Compute Question 0005. \\ \fi
\begin{problem}

Given the piecewise function 
\[f(x)=\left\{\begin{array}{ll}
{3 \, x + 1}\; , & x\leq{-3}\\[3pt]
{4 \, x + 4}\; , & {-3}< x< {1}\\[3pt]
{3 \, x + 5}\; , & x\geq{1}
\end{array} \right.\]
evaluate the following definite integral by interpreting it in terms of areas.  

\input{Integral-Compute-0005.HELP.tex}

\[
\int_{-3}^{-1} f(x)\;dx= \answer{-8}
\]  
\end{problem}}%}

\latexProblemContent{
\ifVerboseLocation This is Integration Compute Question 0005. \\ \fi
\begin{problem}

Given the piecewise function 
\[f(x)=\left\{\begin{array}{ll}
{2 \, x + 5}\; , & x\leq{-4}\\[3pt]
{2 \, x + 5}\; , & {-4}< x< {-1}\\[3pt]
{4 \, x + 7}\; , & x\geq{-1}
\end{array} \right.\]
evaluate the following definite integral by interpreting it in terms of areas.  

\input{Integral-Compute-0005.HELP.tex}

\[
\int_{-7}^{4} f(x)\;dx= \answer{47}
\]  
\end{problem}}%}

\latexProblemContent{
\ifVerboseLocation This is Integration Compute Question 0005. \\ \fi
\begin{problem}

Given the piecewise function 
\[f(x)=\left\{\begin{array}{ll}
{3 \, x + 3}\; , & x\leq{4}\\[3pt]
{-\frac{15}{4} \, x + 30}\; , & {4}< x< {12}\\[3pt]
{3 \, x - 51}\; , & x\geq{12}
\end{array} \right.\]
evaluate the following definite integral by interpreting it in terms of areas.  

\input{Integral-Compute-0005.HELP.tex}

\[
\int_{0}^{15} f(x)\;dx= \answer{\frac{9}{2}}
\]  
\end{problem}}%}

\latexProblemContent{
\ifVerboseLocation This is Integration Compute Question 0005. \\ \fi
\begin{problem}

Given the piecewise function 
\[f(x)=\left\{\begin{array}{ll}
{2 \, x + 3}\; , & x\leq{-2}\\[3pt]
{\frac{1}{4} \, x - \frac{1}{2}}\; , & {-2}< x< {6}\\[3pt]
{x - 5}\; , & x\geq{6}
\end{array} \right.\]
evaluate the following definite integral by interpreting it in terms of areas.  

\input{Integral-Compute-0005.HELP.tex}

\[
\int_{-7}^{-1} f(x)\;dx= \answer{-\frac{247}{8}}
\]  
\end{problem}}%}

\latexProblemContent{
\ifVerboseLocation This is Integration Compute Question 0005. \\ \fi
\begin{problem}

Given the piecewise function 
\[f(x)=\left\{\begin{array}{ll}
{x + 4}\; , & x\leq{-4}\\[3pt]
{0}\; , & {-4}< x< {2}\\[3pt]
{3 \, x - 6}\; , & x\geq{2}
\end{array} \right.\]
evaluate the following definite integral by interpreting it in terms of areas.  

\input{Integral-Compute-0005.HELP.tex}

\[
\int_{-6}^{4} f(x)\;dx= \answer{4}
\]  
\end{problem}}%}

\latexProblemContent{
\ifVerboseLocation This is Integration Compute Question 0005. \\ \fi
\begin{problem}

Given the piecewise function 
\[f(x)=\left\{\begin{array}{ll}
{3 \, x + 2}\; , & x\leq{4}\\[3pt]
{-\frac{7}{2} \, x + 28}\; , & {4}< x< {12}\\[3pt]
{4 \, x - 62}\; , & x\geq{12}
\end{array} \right.\]
evaluate the following definite integral by interpreting it in terms of areas.  

\input{Integral-Compute-0005.HELP.tex}

\[
\int_{2}^{7} f(x)\;dx= \answer{\frac{193}{4}}
\]  
\end{problem}}%}

\latexProblemContent{
\ifVerboseLocation This is Integration Compute Question 0005. \\ \fi
\begin{problem}

Given the piecewise function 
\[f(x)=\left\{\begin{array}{ll}
{3 \, x + 4}\; , & x\leq{-1}\\[3pt]
{-\frac{2}{7} \, x + \frac{5}{7}}\; , & {-1}< x< {6}\\[3pt]
{3 \, x - 19}\; , & x\geq{6}
\end{array} \right.\]
evaluate the following definite integral by interpreting it in terms of areas.  

\input{Integral-Compute-0005.HELP.tex}

\[
\int_{-3}^{2} f(x)\;dx= \answer{-\frac{16}{7}}
\]  
\end{problem}}%}

\latexProblemContent{
\ifVerboseLocation This is Integration Compute Question 0005. \\ \fi
\begin{problem}

Given the piecewise function 
\[f(x)=\left\{\begin{array}{ll}
{3 \, x + 3}\; , & x\leq{4}\\[3pt]
{-\frac{30}{7} \, x + \frac{225}{7}}\; , & {4}< x< {11}\\[3pt]
{4 \, x - 59}\; , & x\geq{11}
\end{array} \right.\]
evaluate the following definite integral by interpreting it in terms of areas.  

\input{Integral-Compute-0005.HELP.tex}

\[
\int_{4}^{9} f(x)\;dx= \answer{\frac{150}{7}}
\]  
\end{problem}}%}

\latexProblemContent{
\ifVerboseLocation This is Integration Compute Question 0005. \\ \fi
\begin{problem}

Given the piecewise function 
\[f(x)=\left\{\begin{array}{ll}
{2 \, x + 3}\; , & x\leq{2}\\[3pt]
{-2 \, x + 11}\; , & {2}< x< {9}\\[3pt]
{3 \, x - 34}\; , & x\geq{9}
\end{array} \right.\]
evaluate the following definite integral by interpreting it in terms of areas.  

\input{Integral-Compute-0005.HELP.tex}

\[
\int_{1}^{9} f(x)\;dx= \answer{6}
\]  
\end{problem}}%}

\latexProblemContent{
\ifVerboseLocation This is Integration Compute Question 0005. \\ \fi
\begin{problem}

Given the piecewise function 
\[f(x)=\left\{\begin{array}{ll}
{x + 4}\; , & x\leq{-1}\\[3pt]
{-\frac{3}{4} \, x + \frac{9}{4}}\; , & {-1}< x< {7}\\[3pt]
{2 \, x - 17}\; , & x\geq{7}
\end{array} \right.\]
evaluate the following definite integral by interpreting it in terms of areas.  

\input{Integral-Compute-0005.HELP.tex}

\[
\int_{-4}^{4} f(x)\;dx= \answer{\frac{81}{8}}
\]  
\end{problem}}%}

\latexProblemContent{
\ifVerboseLocation This is Integration Compute Question 0005. \\ \fi
\begin{problem}

Given the piecewise function 
\[f(x)=\left\{\begin{array}{ll}
{3 \, x + 5}\; , & x\leq{-5}\\[3pt]
{\frac{5}{2} \, x + \frac{5}{2}}\; , & {-5}< x< {3}\\[3pt]
{2 \, x + 4}\; , & x\geq{3}
\end{array} \right.\]
evaluate the following definite integral by interpreting it in terms of areas.  

\input{Integral-Compute-0005.HELP.tex}

\[
\int_{-9}^{3} f(x)\;dx= \answer{-64}
\]  
\end{problem}}%}

\latexProblemContent{
\ifVerboseLocation This is Integration Compute Question 0005. \\ \fi
\begin{problem}

Given the piecewise function 
\[f(x)=\left\{\begin{array}{ll}
{2 \, x + 5}\; , & x\leq{5}\\[3pt]
{-5 \, x + 40}\; , & {5}< x< {11}\\[3pt]
{3 \, x - 48}\; , & x\geq{11}
\end{array} \right.\]
evaluate the following definite integral by interpreting it in terms of areas.  

\input{Integral-Compute-0005.HELP.tex}

\[
\int_{1}^{14} f(x)\;dx= \answer{\frac{25}{2}}
\]  
\end{problem}}%}

\latexProblemContent{
\ifVerboseLocation This is Integration Compute Question 0005. \\ \fi
\begin{problem}

Given the piecewise function 
\[f(x)=\left\{\begin{array}{ll}
{x + 3}\; , & x\leq{2}\\[3pt]
{-\frac{5}{3} \, x + \frac{25}{3}}\; , & {2}< x< {8}\\[3pt]
{4 \, x - 37}\; , & x\geq{8}
\end{array} \right.\]
evaluate the following definite integral by interpreting it in terms of areas.  

\input{Integral-Compute-0005.HELP.tex}

\[
\int_{1}^{3} f(x)\;dx= \answer{\frac{26}{3}}
\]  
\end{problem}}%}

\latexProblemContent{
\ifVerboseLocation This is Integration Compute Question 0005. \\ \fi
\begin{problem}

Given the piecewise function 
\[f(x)=\left\{\begin{array}{ll}
{4 \, x + 3}\; , & x\leq{-1}\\[3pt]
{\frac{1}{3} \, x - \frac{2}{3}}\; , & {-1}< x< {5}\\[3pt]
{2 \, x - 9}\; , & x\geq{5}
\end{array} \right.\]
evaluate the following definite integral by interpreting it in terms of areas.  

\input{Integral-Compute-0005.HELP.tex}

\[
\int_{-2}^{1} f(x)\;dx= \answer{-\frac{13}{3}}
\]  
\end{problem}}%}

\latexProblemContent{
\ifVerboseLocation This is Integration Compute Question 0005. \\ \fi
\begin{problem}

Given the piecewise function 
\[f(x)=\left\{\begin{array}{ll}
{4 \, x + 4}\; , & x\leq{1}\\[3pt]
{-\frac{8}{3} \, x + \frac{32}{3}}\; , & {1}< x< {7}\\[3pt]
{x - 15}\; , & x\geq{7}
\end{array} \right.\]
evaluate the following definite integral by interpreting it in terms of areas.  

\input{Integral-Compute-0005.HELP.tex}

\[
\int_{1}^{4} f(x)\;dx= \answer{12}
\]  
\end{problem}}%}

\latexProblemContent{
\ifVerboseLocation This is Integration Compute Question 0005. \\ \fi
\begin{problem}

Given the piecewise function 
\[f(x)=\left\{\begin{array}{ll}
{x + 4}\; , & x\leq{5}\\[3pt]
{-\frac{9}{4} \, x + \frac{81}{4}}\; , & {5}< x< {13}\\[3pt]
{4 \, x - 61}\; , & x\geq{13}
\end{array} \right.\]
evaluate the following definite integral by interpreting it in terms of areas.  

\input{Integral-Compute-0005.HELP.tex}

\[
\int_{1}^{11} f(x)\;dx= \answer{\frac{83}{2}}
\]  
\end{problem}}%}

\latexProblemContent{
\ifVerboseLocation This is Integration Compute Question 0005. \\ \fi
\begin{problem}

Given the piecewise function 
\[f(x)=\left\{\begin{array}{ll}
{4 \, x + 4}\; , & x\leq{4}\\[3pt]
{-10 \, x + 60}\; , & {4}< x< {8}\\[3pt]
{2 \, x - 36}\; , & x\geq{8}
\end{array} \right.\]
evaluate the following definite integral by interpreting it in terms of areas.  

\input{Integral-Compute-0005.HELP.tex}

\[
\int_{0}^{7} f(x)\;dx= \answer{63}
\]  
\end{problem}}%}

\latexProblemContent{
\ifVerboseLocation This is Integration Compute Question 0005. \\ \fi
\begin{problem}

Given the piecewise function 
\[f(x)=\left\{\begin{array}{ll}
{3 \, x + 1}\; , & x\leq{1}\\[3pt]
{-x + 5}\; , & {1}< x< {9}\\[3pt]
{2 \, x - 22}\; , & x\geq{9}
\end{array} \right.\]
evaluate the following definite integral by interpreting it in terms of areas.  

\input{Integral-Compute-0005.HELP.tex}

\[
\int_{-1}^{2} f(x)\;dx= \answer{\frac{11}{2}}
\]  
\end{problem}}%}

\latexProblemContent{
\ifVerboseLocation This is Integration Compute Question 0005. \\ \fi
\begin{problem}

Given the piecewise function 
\[f(x)=\left\{\begin{array}{ll}
{2 \, x + 5}\; , & x\leq{-4}\\[3pt]
{\frac{6}{7} \, x + \frac{3}{7}}\; , & {-4}< x< {3}\\[3pt]
{2 \, x - 3}\; , & x\geq{3}
\end{array} \right.\]
evaluate the following definite integral by interpreting it in terms of areas.  

\input{Integral-Compute-0005.HELP.tex}

\[
\int_{-9}^{4} f(x)\;dx= \answer{-36}
\]  
\end{problem}}%}

\latexProblemContent{
\ifVerboseLocation This is Integration Compute Question 0005. \\ \fi
\begin{problem}

Given the piecewise function 
\[f(x)=\left\{\begin{array}{ll}
{2 \, x + 5}\; , & x\leq{1}\\[3pt]
{-\frac{14}{5} \, x + \frac{49}{5}}\; , & {1}< x< {6}\\[3pt]
{3 \, x - 25}\; , & x\geq{6}
\end{array} \right.\]
evaluate the following definite integral by interpreting it in terms of areas.  

\input{Integral-Compute-0005.HELP.tex}

\[
\int_{0}^{8} f(x)\;dx= \answer{-2}
\]  
\end{problem}}%}

\latexProblemContent{
\ifVerboseLocation This is Integration Compute Question 0005. \\ \fi
\begin{problem}

Given the piecewise function 
\[f(x)=\left\{\begin{array}{ll}
{4 \, x + 1}\; , & x\leq{-2}\\[3pt]
{2 \, x - 3}\; , & {-2}< x< {5}\\[3pt]
{3 \, x - 8}\; , & x\geq{5}
\end{array} \right.\]
evaluate the following definite integral by interpreting it in terms of areas.  

\input{Integral-Compute-0005.HELP.tex}

\[
\int_{-2}^{2} f(x)\;dx= \answer{-12}
\]  
\end{problem}}%}

\latexProblemContent{
\ifVerboseLocation This is Integration Compute Question 0005. \\ \fi
\begin{problem}

Given the piecewise function 
\[f(x)=\left\{\begin{array}{ll}
{2 \, x + 2}\; , & x\leq{-2}\\[3pt]
{\frac{2}{3} \, x - \frac{2}{3}}\; , & {-2}< x< {4}\\[3pt]
{4 \, x - 14}\; , & x\geq{4}
\end{array} \right.\]
evaluate the following definite integral by interpreting it in terms of areas.  

\input{Integral-Compute-0005.HELP.tex}

\[
\int_{-5}^{0} f(x)\;dx= \answer{-\frac{53}{3}}
\]  
\end{problem}}%}

\latexProblemContent{
\ifVerboseLocation This is Integration Compute Question 0005. \\ \fi
\begin{problem}

Given the piecewise function 
\[f(x)=\left\{\begin{array}{ll}
{3 \, x + 1}\; , & x\leq{1}\\[3pt]
{-2 \, x + 6}\; , & {1}< x< {5}\\[3pt]
{4 \, x - 24}\; , & x\geq{5}
\end{array} \right.\]
evaluate the following definite integral by interpreting it in terms of areas.  

\input{Integral-Compute-0005.HELP.tex}

\[
\int_{-2}^{6} f(x)\;dx= \answer{-\frac{7}{2}}
\]  
\end{problem}}%}

\latexProblemContent{
\ifVerboseLocation This is Integration Compute Question 0005. \\ \fi
\begin{problem}

Given the piecewise function 
\[f(x)=\left\{\begin{array}{ll}
{2 \, x + 1}\; , & x\leq{4}\\[3pt]
{-\frac{18}{5} \, x + \frac{117}{5}}\; , & {4}< x< {9}\\[3pt]
{4 \, x - 45}\; , & x\geq{9}
\end{array} \right.\]
evaluate the following definite integral by interpreting it in terms of areas.  

\input{Integral-Compute-0005.HELP.tex}

\[
\int_{2}^{5} f(x)\;dx= \answer{\frac{106}{5}}
\]  
\end{problem}}%}

\latexProblemContent{
\ifVerboseLocation This is Integration Compute Question 0005. \\ \fi
\begin{problem}

Given the piecewise function 
\[f(x)=\left\{\begin{array}{ll}
{3 \, x + 4}\; , & x\leq{-1}\\[3pt]
{-\frac{1}{3} \, x + \frac{2}{3}}\; , & {-1}< x< {5}\\[3pt]
{x - 6}\; , & x\geq{5}
\end{array} \right.\]
evaluate the following definite integral by interpreting it in terms of areas.  

\input{Integral-Compute-0005.HELP.tex}

\[
\int_{-2}^{1} f(x)\;dx= \answer{\frac{5}{6}}
\]  
\end{problem}}%}

\latexProblemContent{
\ifVerboseLocation This is Integration Compute Question 0005. \\ \fi
\begin{problem}

Given the piecewise function 
\[f(x)=\left\{\begin{array}{ll}
{2 \, x + 2}\; , & x\leq{-2}\\[3pt]
{x}\; , & {-2}< x< {2}\\[3pt]
{2 \, x - 2}\; , & x\geq{2}
\end{array} \right.\]
evaluate the following definite integral by interpreting it in terms of areas.  

\input{Integral-Compute-0005.HELP.tex}

\[
\int_{-7}^{6} f(x)\;dx= \answer{-11}
\]  
\end{problem}}%}

\latexProblemContent{
\ifVerboseLocation This is Integration Compute Question 0005. \\ \fi
\begin{problem}

Given the piecewise function 
\[f(x)=\left\{\begin{array}{ll}
{3 \, x + 5}\; , & x\leq{5}\\[3pt]
{-\frac{40}{7} \, x + \frac{340}{7}}\; , & {5}< x< {12}\\[3pt]
{4 \, x - 68}\; , & x\geq{12}
\end{array} \right.\]
evaluate the following definite integral by interpreting it in terms of areas.  

\input{Integral-Compute-0005.HELP.tex}

\[
\int_{2}^{17} f(x)\;dx= \answer{-\frac{7}{2}}
\]  
\end{problem}}%}

\latexProblemContent{
\ifVerboseLocation This is Integration Compute Question 0005. \\ \fi
\begin{problem}

Given the piecewise function 
\[f(x)=\left\{\begin{array}{ll}
{2 \, x + 2}\; , & x\leq{4}\\[3pt]
{-\frac{10}{3} \, x + \frac{70}{3}}\; , & {4}< x< {10}\\[3pt]
{3 \, x - 40}\; , & x\geq{10}
\end{array} \right.\]
evaluate the following definite integral by interpreting it in terms of areas.  

\input{Integral-Compute-0005.HELP.tex}

\[
\int_{-1}^{6} f(x)\;dx= \answer{\frac{115}{3}}
\]  
\end{problem}}%}

\latexProblemContent{
\ifVerboseLocation This is Integration Compute Question 0005. \\ \fi
\begin{problem}

Given the piecewise function 
\[f(x)=\left\{\begin{array}{ll}
{4 \, x + 3}\; , & x\leq{-3}\\[3pt]
{\frac{9}{4} \, x - \frac{9}{4}}\; , & {-3}< x< {5}\\[3pt]
{4 \, x - 11}\; , & x\geq{5}
\end{array} \right.\]
evaluate the following definite integral by interpreting it in terms of areas.  

\input{Integral-Compute-0005.HELP.tex}

\[
\int_{-8}^{10} f(x)\;dx= \answer{0}
\]  
\end{problem}}%}

\latexProblemContent{
\ifVerboseLocation This is Integration Compute Question 0005. \\ \fi
\begin{problem}

Given the piecewise function 
\[f(x)=\left\{\begin{array}{ll}
{4 \, x + 4}\; , & x\leq{5}\\[3pt]
{-12 \, x + 84}\; , & {5}< x< {9}\\[3pt]
{4 \, x - 60}\; , & x\geq{9}
\end{array} \right.\]
evaluate the following definite integral by interpreting it in terms of areas.  

\input{Integral-Compute-0005.HELP.tex}

\[
\int_{4}^{13} f(x)\;dx= \answer{-42}
\]  
\end{problem}}%}

\latexProblemContent{
\ifVerboseLocation This is Integration Compute Question 0005. \\ \fi
\begin{problem}

Given the piecewise function 
\[f(x)=\left\{\begin{array}{ll}
{3 \, x + 5}\; , & x\leq{-1}\\[3pt]
{-x + 1}\; , & {-1}< x< {3}\\[3pt]
{3 \, x - 11}\; , & x\geq{3}
\end{array} \right.\]
evaluate the following definite integral by interpreting it in terms of areas.  

\input{Integral-Compute-0005.HELP.tex}

\[
\int_{-3}^{5} f(x)\;dx= \answer{0}
\]  
\end{problem}}%}

\latexProblemContent{
\ifVerboseLocation This is Integration Compute Question 0005. \\ \fi
\begin{problem}

Given the piecewise function 
\[f(x)=\left\{\begin{array}{ll}
{x + 3}\; , & x\leq{3}\\[3pt]
{-\frac{3}{2} \, x + \frac{21}{2}}\; , & {3}< x< {11}\\[3pt]
{4 \, x - 50}\; , & x\geq{11}
\end{array} \right.\]
evaluate the following definite integral by interpreting it in terms of areas.  

\input{Integral-Compute-0005.HELP.tex}

\[
\int_{-1}^{10} f(x)\;dx= \answer{\frac{85}{4}}
\]  
\end{problem}}%}

\latexProblemContent{
\ifVerboseLocation This is Integration Compute Question 0005. \\ \fi
\begin{problem}

Given the piecewise function 
\[f(x)=\left\{\begin{array}{ll}
{x + 5}\; , & x\leq{-2}\\[3pt]
{-x + 1}\; , & {-2}< x< {4}\\[3pt]
{3 \, x - 15}\; , & x\geq{4}
\end{array} \right.\]
evaluate the following definite integral by interpreting it in terms of areas.  

\input{Integral-Compute-0005.HELP.tex}

\[
\int_{-3}^{5} f(x)\;dx= \answer{1}
\]  
\end{problem}}%}

\latexProblemContent{
\ifVerboseLocation This is Integration Compute Question 0005. \\ \fi
\begin{problem}

Given the piecewise function 
\[f(x)=\left\{\begin{array}{ll}
{3 \, x + 5}\; , & x\leq{-5}\\[3pt]
{\frac{10}{3} \, x + \frac{20}{3}}\; , & {-5}< x< {1}\\[3pt]
{4 \, x + 6}\; , & x\geq{1}
\end{array} \right.\]
evaluate the following definite integral by interpreting it in terms of areas.  

\input{Integral-Compute-0005.HELP.tex}

\[
\int_{-8}^{3} f(x)\;dx= \answer{-\frac{31}{2}}
\]  
\end{problem}}%}

\latexProblemContent{
\ifVerboseLocation This is Integration Compute Question 0005. \\ \fi
\begin{problem}

Given the piecewise function 
\[f(x)=\left\{\begin{array}{ll}
{3 \, x + 3}\; , & x\leq{-2}\\[3pt]
{2 \, x + 1}\; , & {-2}< x< {1}\\[3pt]
{x + 2}\; , & x\geq{1}
\end{array} \right.\]
evaluate the following definite integral by interpreting it in terms of areas.  

\input{Integral-Compute-0005.HELP.tex}

\[
\int_{-6}^{2} f(x)\;dx= \answer{-\frac{65}{2}}
\]  
\end{problem}}%}

\latexProblemContent{
\ifVerboseLocation This is Integration Compute Question 0005. \\ \fi
\begin{problem}

Given the piecewise function 
\[f(x)=\left\{\begin{array}{ll}
{x + 3}\; , & x\leq{1}\\[3pt]
{-\frac{8}{3} \, x + \frac{20}{3}}\; , & {1}< x< {4}\\[3pt]
{x - 8}\; , & x\geq{4}
\end{array} \right.\]
evaluate the following definite integral by interpreting it in terms of areas.  

\input{Integral-Compute-0005.HELP.tex}

\[
\int_{0}^{4} f(x)\;dx= \answer{\frac{7}{2}}
\]  
\end{problem}}%}

\latexProblemContent{
\ifVerboseLocation This is Integration Compute Question 0005. \\ \fi
\begin{problem}

Given the piecewise function 
\[f(x)=\left\{\begin{array}{ll}
{3 \, x + 3}\; , & x\leq{1}\\[3pt]
{-2 \, x + 8}\; , & {1}< x< {7}\\[3pt]
{2 \, x - 20}\; , & x\geq{7}
\end{array} \right.\]
evaluate the following definite integral by interpreting it in terms of areas.  

\input{Integral-Compute-0005.HELP.tex}

\[
\int_{-4}^{12} f(x)\;dx= \answer{-\frac{25}{2}}
\]  
\end{problem}}%}

\latexProblemContent{
\ifVerboseLocation This is Integration Compute Question 0005. \\ \fi
\begin{problem}

Given the piecewise function 
\[f(x)=\left\{\begin{array}{ll}
{4 \, x + 3}\; , & x\leq{-2}\\[3pt]
{2 \, x - 1}\; , & {-2}< x< {3}\\[3pt]
{x + 2}\; , & x\geq{3}
\end{array} \right.\]
evaluate the following definite integral by interpreting it in terms of areas.  

\input{Integral-Compute-0005.HELP.tex}

\[
\int_{-3}^{6} f(x)\;dx= \answer{\frac{25}{2}}
\]  
\end{problem}}%}

\latexProblemContent{
\ifVerboseLocation This is Integration Compute Question 0005. \\ \fi
\begin{problem}

Given the piecewise function 
\[f(x)=\left\{\begin{array}{ll}
{4 \, x + 1}\; , & x\leq{-3}\\[3pt]
{\frac{11}{2} \, x + \frac{11}{2}}\; , & {-3}< x< {1}\\[3pt]
{3 \, x + 8}\; , & x\geq{1}
\end{array} \right.\]
evaluate the following definite integral by interpreting it in terms of areas.  

\input{Integral-Compute-0005.HELP.tex}

\[
\int_{-6}^{6} f(x)\;dx= \answer{\frac{83}{2}}
\]  
\end{problem}}%}

\latexProblemContent{
\ifVerboseLocation This is Integration Compute Question 0005. \\ \fi
\begin{problem}

Given the piecewise function 
\[f(x)=\left\{\begin{array}{ll}
{4 \, x + 2}\; , & x\leq{5}\\[3pt]
{-\frac{44}{5} \, x + 66}\; , & {5}< x< {10}\\[3pt]
{2 \, x - 42}\; , & x\geq{10}
\end{array} \right.\]
evaluate the following definite integral by interpreting it in terms of areas.  

\input{Integral-Compute-0005.HELP.tex}

\[
\int_{2}^{6} f(x)\;dx= \answer{\frac{328}{5}}
\]  
\end{problem}}%}

\latexProblemContent{
\ifVerboseLocation This is Integration Compute Question 0005. \\ \fi
\begin{problem}

Given the piecewise function 
\[f(x)=\left\{\begin{array}{ll}
{x + 3}\; , & x\leq{-4}\\[3pt]
{\frac{2}{3} \, x + \frac{5}{3}}\; , & {-4}< x< {-1}\\[3pt]
{4 \, x + 5}\; , & x\geq{-1}
\end{array} \right.\]
evaluate the following definite integral by interpreting it in terms of areas.  

\input{Integral-Compute-0005.HELP.tex}

\[
\int_{-6}^{-2} f(x)\;dx= \answer{-\frac{14}{3}}
\]  
\end{problem}}%}

\latexProblemContent{
\ifVerboseLocation This is Integration Compute Question 0005. \\ \fi
\begin{problem}

Given the piecewise function 
\[f(x)=\left\{\begin{array}{ll}
{3 \, x + 4}\; , & x\leq{-5}\\[3pt]
{\frac{22}{7} \, x + \frac{33}{7}}\; , & {-5}< x< {2}\\[3pt]
{x + 9}\; , & x\geq{2}
\end{array} \right.\]
evaluate the following definite integral by interpreting it in terms of areas.  

\input{Integral-Compute-0005.HELP.tex}

\[
\int_{-9}^{7} f(x)\;dx= \answer{-\frac{1}{2}}
\]  
\end{problem}}%}

\latexProblemContent{
\ifVerboseLocation This is Integration Compute Question 0005. \\ \fi
\begin{problem}

Given the piecewise function 
\[f(x)=\left\{\begin{array}{ll}
{4 \, x + 2}\; , & x\leq{1}\\[3pt]
{-\frac{12}{5} \, x + \frac{42}{5}}\; , & {1}< x< {6}\\[3pt]
{x - 12}\; , & x\geq{6}
\end{array} \right.\]
evaluate the following definite integral by interpreting it in terms of areas.  

\input{Integral-Compute-0005.HELP.tex}

\[
\int_{-1}^{10} f(x)\;dx= \answer{-12}
\]  
\end{problem}}%}

\latexProblemContent{
\ifVerboseLocation This is Integration Compute Question 0005. \\ \fi
\begin{problem}

Given the piecewise function 
\[f(x)=\left\{\begin{array}{ll}
{x + 4}\; , & x\leq{-1}\\[3pt]
{-x + 2}\; , & {-1}< x< {5}\\[3pt]
{3 \, x - 18}\; , & x\geq{5}
\end{array} \right.\]
evaluate the following definite integral by interpreting it in terms of areas.  

\input{Integral-Compute-0005.HELP.tex}

\[
\int_{-2}^{3} f(x)\;dx= \answer{\frac{13}{2}}
\]  
\end{problem}}%}

\latexProblemContent{
\ifVerboseLocation This is Integration Compute Question 0005. \\ \fi
\begin{problem}

Given the piecewise function 
\[f(x)=\left\{\begin{array}{ll}
{3 \, x + 3}\; , & x\leq{4}\\[3pt]
{-\frac{30}{7} \, x + \frac{225}{7}}\; , & {4}< x< {11}\\[3pt]
{2 \, x - 37}\; , & x\geq{11}
\end{array} \right.\]
evaluate the following definite integral by interpreting it in terms of areas.  

\input{Integral-Compute-0005.HELP.tex}

\[
\int_{-1}^{15} f(x)\;dx= \answer{-\frac{13}{2}}
\]  
\end{problem}}%}

\latexProblemContent{
\ifVerboseLocation This is Integration Compute Question 0005. \\ \fi
\begin{problem}

Given the piecewise function 
\[f(x)=\left\{\begin{array}{ll}
{3 \, x + 2}\; , & x\leq{2}\\[3pt]
{-\frac{16}{7} \, x + \frac{88}{7}}\; , & {2}< x< {9}\\[3pt]
{x - 17}\; , & x\geq{9}
\end{array} \right.\]
evaluate the following definite integral by interpreting it in terms of areas.  

\input{Integral-Compute-0005.HELP.tex}

\[
\int_{2}^{7} f(x)\;dx= \answer{\frac{80}{7}}
\]  
\end{problem}}%}

\latexProblemContent{
\ifVerboseLocation This is Integration Compute Question 0005. \\ \fi
\begin{problem}

Given the piecewise function 
\[f(x)=\left\{\begin{array}{ll}
{2 \, x + 5}\; , & x\leq{2}\\[3pt]
{-6 \, x + 21}\; , & {2}< x< {5}\\[3pt]
{2 \, x - 19}\; , & x\geq{5}
\end{array} \right.\]
evaluate the following definite integral by interpreting it in terms of areas.  

\input{Integral-Compute-0005.HELP.tex}

\[
\int_{0}^{7} f(x)\;dx= \answer{0}
\]  
\end{problem}}%}

\latexProblemContent{
\ifVerboseLocation This is Integration Compute Question 0005. \\ \fi
\begin{problem}

Given the piecewise function 
\[f(x)=\left\{\begin{array}{ll}
{4 \, x + 2}\; , & x\leq{2}\\[3pt]
{-\frac{5}{2} \, x + 15}\; , & {2}< x< {10}\\[3pt]
{x - 20}\; , & x\geq{10}
\end{array} \right.\]
evaluate the following definite integral by interpreting it in terms of areas.  

\input{Integral-Compute-0005.HELP.tex}

\[
\int_{1}^{11} f(x)\;dx= \answer{-\frac{3}{2}}
\]  
\end{problem}}%}

\latexProblemContent{
\ifVerboseLocation This is Integration Compute Question 0005. \\ \fi
\begin{problem}

Given the piecewise function 
\[f(x)=\left\{\begin{array}{ll}
{2 \, x + 3}\; , & x\leq{-5}\\[3pt]
{2 \, x + 3}\; , & {-5}< x< {2}\\[3pt]
{3 \, x + 1}\; , & x\geq{2}
\end{array} \right.\]
evaluate the following definite integral by interpreting it in terms of areas.  

\input{Integral-Compute-0005.HELP.tex}

\[
\int_{-10}^{1} f(x)\;dx= \answer{-66}
\]  
\end{problem}}%}

\latexProblemContent{
\ifVerboseLocation This is Integration Compute Question 0005. \\ \fi
\begin{problem}

Given the piecewise function 
\[f(x)=\left\{\begin{array}{ll}
{2 \, x + 5}\; , & x\leq{-3}\\[3pt]
{\frac{2}{5} \, x + \frac{1}{5}}\; , & {-3}< x< {2}\\[3pt]
{3 \, x - 5}\; , & x\geq{2}
\end{array} \right.\]
evaluate the following definite integral by interpreting it in terms of areas.  

\input{Integral-Compute-0005.HELP.tex}

\[
\int_{-8}^{4} f(x)\;dx= \answer{-22}
\]  
\end{problem}}%}

\latexProblemContent{
\ifVerboseLocation This is Integration Compute Question 0005. \\ \fi
\begin{problem}

Given the piecewise function 
\[f(x)=\left\{\begin{array}{ll}
{4 \, x + 1}\; , & x\leq{2}\\[3pt]
{-\frac{18}{5} \, x + \frac{81}{5}}\; , & {2}< x< {7}\\[3pt]
{2 \, x - 23}\; , & x\geq{7}
\end{array} \right.\]
evaluate the following definite integral by interpreting it in terms of areas.  

\input{Integral-Compute-0005.HELP.tex}

\[
\int_{-2}^{10} f(x)\;dx= \answer{-14}
\]  
\end{problem}}%}

\latexProblemContent{
\ifVerboseLocation This is Integration Compute Question 0005. \\ \fi
\begin{problem}

Given the piecewise function 
\[f(x)=\left\{\begin{array}{ll}
{3 \, x + 2}\; , & x\leq{-4}\\[3pt]
{\frac{20}{3} \, x + \frac{50}{3}}\; , & {-4}< x< {-1}\\[3pt]
{x + 11}\; , & x\geq{-1}
\end{array} \right.\]
evaluate the following definite integral by interpreting it in terms of areas.  

\input{Integral-Compute-0005.HELP.tex}

\[
\int_{-5}^{1} f(x)\;dx= \answer{\frac{21}{2}}
\]  
\end{problem}}%}

\latexProblemContent{
\ifVerboseLocation This is Integration Compute Question 0005. \\ \fi
\begin{problem}

Given the piecewise function 
\[f(x)=\left\{\begin{array}{ll}
{x + 4}\; , & x\leq{-2}\\[3pt]
{-\frac{4}{7} \, x + \frac{6}{7}}\; , & {-2}< x< {5}\\[3pt]
{3 \, x - 17}\; , & x\geq{5}
\end{array} \right.\]
evaluate the following definite integral by interpreting it in terms of areas.  

\input{Integral-Compute-0005.HELP.tex}

\[
\int_{-7}^{4} f(x)\;dx= \answer{-\frac{11}{14}}
\]  
\end{problem}}%}

\latexProblemContent{
\ifVerboseLocation This is Integration Compute Question 0005. \\ \fi
\begin{problem}

Given the piecewise function 
\[f(x)=\left\{\begin{array}{ll}
{x + 2}\; , & x\leq{4}\\[3pt]
{-2 \, x + 14}\; , & {4}< x< {10}\\[3pt]
{4 \, x - 46}\; , & x\geq{10}
\end{array} \right.\]
evaluate the following definite integral by interpreting it in terms of areas.  

\input{Integral-Compute-0005.HELP.tex}

\[
\int_{2}^{10} f(x)\;dx= \answer{10}
\]  
\end{problem}}%}

\latexProblemContent{
\ifVerboseLocation This is Integration Compute Question 0005. \\ \fi
\begin{problem}

Given the piecewise function 
\[f(x)=\left\{\begin{array}{ll}
{4 \, x + 2}\; , & x\leq{-5}\\[3pt]
{12 \, x + 42}\; , & {-5}< x< {-2}\\[3pt]
{2 \, x + 22}\; , & x\geq{-2}
\end{array} \right.\]
evaluate the following definite integral by interpreting it in terms of areas.  

\input{Integral-Compute-0005.HELP.tex}

\[
\int_{-7}^{-1} f(x)\;dx= \answer{-25}
\]  
\end{problem}}%}

\latexProblemContent{
\ifVerboseLocation This is Integration Compute Question 0005. \\ \fi
\begin{problem}

Given the piecewise function 
\[f(x)=\left\{\begin{array}{ll}
{2 \, x + 1}\; , & x\leq{-5}\\[3pt]
{3 \, x + 6}\; , & {-5}< x< {1}\\[3pt]
{x + 8}\; , & x\geq{1}
\end{array} \right.\]
evaluate the following definite integral by interpreting it in terms of areas.  

\input{Integral-Compute-0005.HELP.tex}

\[
\int_{-10}^{-4} f(x)\;dx= \answer{-\frac{155}{2}}
\]  
\end{problem}}%}

\latexProblemContent{
\ifVerboseLocation This is Integration Compute Question 0005. \\ \fi
\begin{problem}

Given the piecewise function 
\[f(x)=\left\{\begin{array}{ll}
{x + 3}\; , & x\leq{-5}\\[3pt]
{\frac{1}{2} \, x + \frac{1}{2}}\; , & {-5}< x< {3}\\[3pt]
{2 \, x - 4}\; , & x\geq{3}
\end{array} \right.\]
evaluate the following definite integral by interpreting it in terms of areas.  

\input{Integral-Compute-0005.HELP.tex}

\[
\int_{-9}^{2} f(x)\;dx= \answer{-\frac{71}{4}}
\]  
\end{problem}}%}

\latexProblemContent{
\ifVerboseLocation This is Integration Compute Question 0005. \\ \fi
\begin{problem}

Given the piecewise function 
\[f(x)=\left\{\begin{array}{ll}
{4 \, x + 3}\; , & x\leq{3}\\[3pt]
{-\frac{30}{7} \, x + \frac{195}{7}}\; , & {3}< x< {10}\\[3pt]
{2 \, x - 35}\; , & x\geq{10}
\end{array} \right.\]
evaluate the following definite integral by interpreting it in terms of areas.  

\input{Integral-Compute-0005.HELP.tex}

\[
\int_{1}^{4} f(x)\;dx= \answer{\frac{244}{7}}
\]  
\end{problem}}%}

\latexProblemContent{
\ifVerboseLocation This is Integration Compute Question 0005. \\ \fi
\begin{problem}

Given the piecewise function 
\[f(x)=\left\{\begin{array}{ll}
{4 \, x + 1}\; , & x\leq{1}\\[3pt]
{-\frac{10}{3} \, x + \frac{25}{3}}\; , & {1}< x< {4}\\[3pt]
{2 \, x - 13}\; , & x\geq{4}
\end{array} \right.\]
evaluate the following definite integral by interpreting it in terms of areas.  

\input{Integral-Compute-0005.HELP.tex}

\[
\int_{-1}^{5} f(x)\;dx= \answer{-2}
\]  
\end{problem}}%}

\latexProblemContent{
\ifVerboseLocation This is Integration Compute Question 0005. \\ \fi
\begin{problem}

Given the piecewise function 
\[f(x)=\left\{\begin{array}{ll}
{2 \, x + 3}\; , & x\leq{-4}\\[3pt]
{\frac{10}{7} \, x + \frac{5}{7}}\; , & {-4}< x< {3}\\[3pt]
{x + 2}\; , & x\geq{3}
\end{array} \right.\]
evaluate the following definite integral by interpreting it in terms of areas.  

\input{Integral-Compute-0005.HELP.tex}

\[
\int_{-9}^{2} f(x)\;dx= \answer{-\frac{380}{7}}
\]  
\end{problem}}%}

\latexProblemContent{
\ifVerboseLocation This is Integration Compute Question 0005. \\ \fi
\begin{problem}

Given the piecewise function 
\[f(x)=\left\{\begin{array}{ll}
{2 \, x + 3}\; , & x\leq{-4}\\[3pt]
{\frac{10}{3} \, x + \frac{25}{3}}\; , & {-4}< x< {-1}\\[3pt]
{x + 6}\; , & x\geq{-1}
\end{array} \right.\]
evaluate the following definite integral by interpreting it in terms of areas.  

\input{Integral-Compute-0005.HELP.tex}

\[
\int_{-6}^{3} f(x)\;dx= \answer{14}
\]  
\end{problem}}%}

\latexProblemContent{
\ifVerboseLocation This is Integration Compute Question 0005. \\ \fi
\begin{problem}

Given the piecewise function 
\[f(x)=\left\{\begin{array}{ll}
{3 \, x + 4}\; , & x\leq{-4}\\[3pt]
{2 \, x}\; , & {-4}< x< {4}\\[3pt]
{4 \, x - 8}\; , & x\geq{4}
\end{array} \right.\]
evaluate the following definite integral by interpreting it in terms of areas.  

\input{Integral-Compute-0005.HELP.tex}

\[
\int_{-8}^{7} f(x)\;dx= \answer{-14}
\]  
\end{problem}}%}

\latexProblemContent{
\ifVerboseLocation This is Integration Compute Question 0005. \\ \fi
\begin{problem}

Given the piecewise function 
\[f(x)=\left\{\begin{array}{ll}
{2 \, x + 2}\; , & x\leq{2}\\[3pt]
{-3 \, x + 12}\; , & {2}< x< {6}\\[3pt]
{3 \, x - 24}\; , & x\geq{6}
\end{array} \right.\]
evaluate the following definite integral by interpreting it in terms of areas.  

\input{Integral-Compute-0005.HELP.tex}

\[
\int_{0}^{5} f(x)\;dx= \answer{\frac{25}{2}}
\]  
\end{problem}}%}

\latexProblemContent{
\ifVerboseLocation This is Integration Compute Question 0005. \\ \fi
\begin{problem}

Given the piecewise function 
\[f(x)=\left\{\begin{array}{ll}
{4 \, x + 1}\; , & x\leq{-2}\\[3pt]
{\frac{14}{3} \, x + \frac{7}{3}}\; , & {-2}< x< {1}\\[3pt]
{4 \, x + 3}\; , & x\geq{1}
\end{array} \right.\]
evaluate the following definite integral by interpreting it in terms of areas.  

\input{Integral-Compute-0005.HELP.tex}

\[
\int_{-2}^{1} f(x)\;dx= \answer{0}
\]  
\end{problem}}%}

\latexProblemContent{
\ifVerboseLocation This is Integration Compute Question 0005. \\ \fi
\begin{problem}

Given the piecewise function 
\[f(x)=\left\{\begin{array}{ll}
{3 \, x + 5}\; , & x\leq{4}\\[3pt]
{-\frac{34}{3} \, x + \frac{187}{3}}\; , & {4}< x< {7}\\[3pt]
{2 \, x - 31}\; , & x\geq{7}
\end{array} \right.\]
evaluate the following definite integral by interpreting it in terms of areas.  

\input{Integral-Compute-0005.HELP.tex}

\[
\int_{3}^{6} f(x)\;dx= \answer{\frac{161}{6}}
\]  
\end{problem}}%}

\latexProblemContent{
\ifVerboseLocation This is Integration Compute Question 0005. \\ \fi
\begin{problem}

Given the piecewise function 
\[f(x)=\left\{\begin{array}{ll}
{x + 3}\; , & x\leq{-2}\\[3pt]
{-\frac{2}{5} \, x + \frac{1}{5}}\; , & {-2}< x< {3}\\[3pt]
{2 \, x - 7}\; , & x\geq{3}
\end{array} \right.\]
evaluate the following definite integral by interpreting it in terms of areas.  

\input{Integral-Compute-0005.HELP.tex}

\[
\int_{-5}^{6} f(x)\;dx= \answer{\frac{9}{2}}
\]  
\end{problem}}%}

\latexProblemContent{
\ifVerboseLocation This is Integration Compute Question 0005. \\ \fi
\begin{problem}

Given the piecewise function 
\[f(x)=\left\{\begin{array}{ll}
{2 \, x + 1}\; , & x\leq{-2}\\[3pt]
{\frac{3}{2} \, x}\; , & {-2}< x< {2}\\[3pt]
{4 \, x - 5}\; , & x\geq{2}
\end{array} \right.\]
evaluate the following definite integral by interpreting it in terms of areas.  

\input{Integral-Compute-0005.HELP.tex}

\[
\int_{-4}^{6} f(x)\;dx= \answer{34}
\]  
\end{problem}}%}

\latexProblemContent{
\ifVerboseLocation This is Integration Compute Question 0005. \\ \fi
\begin{problem}

Given the piecewise function 
\[f(x)=\left\{\begin{array}{ll}
{2 \, x + 5}\; , & x\leq{-1}\\[3pt]
{-\frac{3}{2} \, x + \frac{3}{2}}\; , & {-1}< x< {3}\\[3pt]
{4 \, x - 15}\; , & x\geq{3}
\end{array} \right.\]
evaluate the following definite integral by interpreting it in terms of areas.  

\input{Integral-Compute-0005.HELP.tex}

\[
\int_{-6}^{0} f(x)\;dx= \answer{-\frac{31}{4}}
\]  
\end{problem}}%}

\latexProblemContent{
\ifVerboseLocation This is Integration Compute Question 0005. \\ \fi
\begin{problem}

Given the piecewise function 
\[f(x)=\left\{\begin{array}{ll}
{4 \, x + 5}\; , & x\leq{3}\\[3pt]
{-\frac{34}{5} \, x + \frac{187}{5}}\; , & {3}< x< {8}\\[3pt]
{3 \, x - 41}\; , & x\geq{8}
\end{array} \right.\]
evaluate the following definite integral by interpreting it in terms of areas.  

\input{Integral-Compute-0005.HELP.tex}

\[
\int_{1}^{6} f(x)\;dx= \answer{\frac{232}{5}}
\]  
\end{problem}}%}

\latexProblemContent{
\ifVerboseLocation This is Integration Compute Question 0005. \\ \fi
\begin{problem}

Given the piecewise function 
\[f(x)=\left\{\begin{array}{ll}
{x + 4}\; , & x\leq{1}\\[3pt]
{-\frac{5}{3} \, x + \frac{20}{3}}\; , & {1}< x< {7}\\[3pt]
{x - 12}\; , & x\geq{7}
\end{array} \right.\]
evaluate the following definite integral by interpreting it in terms of areas.  

\input{Integral-Compute-0005.HELP.tex}

\[
\int_{0}^{4} f(x)\;dx= \answer{12}
\]  
\end{problem}}%}

\latexProblemContent{
\ifVerboseLocation This is Integration Compute Question 0005. \\ \fi
\begin{problem}

Given the piecewise function 
\[f(x)=\left\{\begin{array}{ll}
{2 \, x + 1}\; , & x\leq{1}\\[3pt]
{-\frac{6}{5} \, x + \frac{21}{5}}\; , & {1}< x< {6}\\[3pt]
{4 \, x - 27}\; , & x\geq{6}
\end{array} \right.\]
evaluate the following definite integral by interpreting it in terms of areas.  

\input{Integral-Compute-0005.HELP.tex}

\[
\int_{0}^{10} f(x)\;dx= \answer{22}
\]  
\end{problem}}%}

\latexProblemContent{
\ifVerboseLocation This is Integration Compute Question 0005. \\ \fi
\begin{problem}

Given the piecewise function 
\[f(x)=\left\{\begin{array}{ll}
{x + 2}\; , & x\leq{-3}\\[3pt]
{\frac{1}{2} \, x + \frac{1}{2}}\; , & {-3}< x< {1}\\[3pt]
{4 \, x - 3}\; , & x\geq{1}
\end{array} \right.\]
evaluate the following definite integral by interpreting it in terms of areas.  

\input{Integral-Compute-0005.HELP.tex}

\[
\int_{-7}^{3} f(x)\;dx= \answer{-2}
\]  
\end{problem}}%}

\latexProblemContent{
\ifVerboseLocation This is Integration Compute Question 0005. \\ \fi
\begin{problem}

Given the piecewise function 
\[f(x)=\left\{\begin{array}{ll}
{3 \, x + 5}\; , & x\leq{-1}\\[3pt]
{-x + 1}\; , & {-1}< x< {3}\\[3pt]
{2 \, x - 8}\; , & x\geq{3}
\end{array} \right.\]
evaluate the following definite integral by interpreting it in terms of areas.  

\input{Integral-Compute-0005.HELP.tex}

\[
\int_{-6}^{7} f(x)\;dx= \answer{-\frac{39}{2}}
\]  
\end{problem}}%}

\latexProblemContent{
\ifVerboseLocation This is Integration Compute Question 0005. \\ \fi
\begin{problem}

Given the piecewise function 
\[f(x)=\left\{\begin{array}{ll}
{x + 1}\; , & x\leq{3}\\[3pt]
{-\frac{8}{3} \, x + 12}\; , & {3}< x< {6}\\[3pt]
{2 \, x - 16}\; , & x\geq{6}
\end{array} \right.\]
evaluate the following definite integral by interpreting it in terms of areas.  

\input{Integral-Compute-0005.HELP.tex}

\[
\int_{2}^{11} f(x)\;dx= \answer{\frac{17}{2}}
\]  
\end{problem}}%}

\latexProblemContent{
\ifVerboseLocation This is Integration Compute Question 0005. \\ \fi
\begin{problem}

Given the piecewise function 
\[f(x)=\left\{\begin{array}{ll}
{2 \, x + 4}\; , & x\leq{5}\\[3pt]
{-7 \, x + 49}\; , & {5}< x< {9}\\[3pt]
{2 \, x - 32}\; , & x\geq{9}
\end{array} \right.\]
evaluate the following definite integral by interpreting it in terms of areas.  

\input{Integral-Compute-0005.HELP.tex}

\[
\int_{0}^{14} f(x)\;dx= \answer{0}
\]  
\end{problem}}%}

\latexProblemContent{
\ifVerboseLocation This is Integration Compute Question 0005. \\ \fi
\begin{problem}

Given the piecewise function 
\[f(x)=\left\{\begin{array}{ll}
{2 \, x + 1}\; , & x\leq{-4}\\[3pt]
{\frac{7}{3} \, x + \frac{7}{3}}\; , & {-4}< x< {2}\\[3pt]
{4 \, x - 1}\; , & x\geq{2}
\end{array} \right.\]
evaluate the following definite integral by interpreting it in terms of areas.  

\input{Integral-Compute-0005.HELP.tex}

\[
\int_{-6}^{1} f(x)\;dx= \answer{-\frac{143}{6}}
\]  
\end{problem}}%}

\latexProblemContent{
\ifVerboseLocation This is Integration Compute Question 0005. \\ \fi
\begin{problem}

Given the piecewise function 
\[f(x)=\left\{\begin{array}{ll}
{3 \, x + 1}\; , & x\leq{3}\\[3pt]
{-5 \, x + 25}\; , & {3}< x< {7}\\[3pt]
{x - 17}\; , & x\geq{7}
\end{array} \right.\]
evaluate the following definite integral by interpreting it in terms of areas.  

\input{Integral-Compute-0005.HELP.tex}

\[
\int_{-1}^{10} f(x)\;dx= \answer{-\frac{19}{2}}
\]  
\end{problem}}%}

\latexProblemContent{
\ifVerboseLocation This is Integration Compute Question 0005. \\ \fi
\begin{problem}

Given the piecewise function 
\[f(x)=\left\{\begin{array}{ll}
{4 \, x + 5}\; , & x\leq{4}\\[3pt]
{-14 \, x + 77}\; , & {4}< x< {7}\\[3pt]
{3 \, x - 42}\; , & x\geq{7}
\end{array} \right.\]
evaluate the following definite integral by interpreting it in terms of areas.  

\input{Integral-Compute-0005.HELP.tex}

\[
\int_{0}^{11} f(x)\;dx= \answer{-8}
\]  
\end{problem}}%}

\latexProblemContent{
\ifVerboseLocation This is Integration Compute Question 0005. \\ \fi
\begin{problem}

Given the piecewise function 
\[f(x)=\left\{\begin{array}{ll}
{2 \, x + 1}\; , & x\leq{1}\\[3pt]
{-\frac{3}{2} \, x + \frac{9}{2}}\; , & {1}< x< {5}\\[3pt]
{x - 8}\; , & x\geq{5}
\end{array} \right.\]
evaluate the following definite integral by interpreting it in terms of areas.  

\input{Integral-Compute-0005.HELP.tex}

\[
\int_{-4}^{10} f(x)\;dx= \answer{-\frac{25}{2}}
\]  
\end{problem}}%}

\latexProblemContent{
\ifVerboseLocation This is Integration Compute Question 0005. \\ \fi
\begin{problem}

Given the piecewise function 
\[f(x)=\left\{\begin{array}{ll}
{3 \, x + 3}\; , & x\leq{4}\\[3pt]
{-10 \, x + 55}\; , & {4}< x< {7}\\[3pt]
{4 \, x - 43}\; , & x\geq{7}
\end{array} \right.\]
evaluate the following definite integral by interpreting it in terms of areas.  

\input{Integral-Compute-0005.HELP.tex}

\[
\int_{0}^{9} f(x)\;dx= \answer{14}
\]  
\end{problem}}%}

\latexProblemContent{
\ifVerboseLocation This is Integration Compute Question 0005. \\ \fi
\begin{problem}

Given the piecewise function 
\[f(x)=\left\{\begin{array}{ll}
{2 \, x + 3}\; , & x\leq{-5}\\[3pt]
{\frac{14}{3} \, x + \frac{49}{3}}\; , & {-5}< x< {-2}\\[3pt]
{3 \, x + 13}\; , & x\geq{-2}
\end{array} \right.\]
evaluate the following definite integral by interpreting it in terms of areas.  

\input{Integral-Compute-0005.HELP.tex}

\[
\int_{-7}^{-2} f(x)\;dx= \answer{-18}
\]  
\end{problem}}%}

\latexProblemContent{
\ifVerboseLocation This is Integration Compute Question 0005. \\ \fi
\begin{problem}

Given the piecewise function 
\[f(x)=\left\{\begin{array}{ll}
{4 \, x + 3}\; , & x\leq{5}\\[3pt]
{-\frac{23}{2} \, x + \frac{161}{2}}\; , & {5}< x< {9}\\[3pt]
{2 \, x - 41}\; , & x\geq{9}
\end{array} \right.\]
evaluate the following definite integral by interpreting it in terms of areas.  

\input{Integral-Compute-0005.HELP.tex}

\[
\int_{0}^{12} f(x)\;dx= \answer{5}
\]  
\end{problem}}%}

\latexProblemContent{
\ifVerboseLocation This is Integration Compute Question 0005. \\ \fi
\begin{problem}

Given the piecewise function 
\[f(x)=\left\{\begin{array}{ll}
{2 \, x + 4}\; , & x\leq{-2}\\[3pt]
{0}\; , & {-2}< x< {3}\\[3pt]
{4 \, x - 12}\; , & x\geq{3}
\end{array} \right.\]
evaluate the following definite integral by interpreting it in terms of areas.  

\input{Integral-Compute-0005.HELP.tex}

\[
\int_{-4}^{6} f(x)\;dx= \answer{14}
\]  
\end{problem}}%}

\latexProblemContent{
\ifVerboseLocation This is Integration Compute Question 0005. \\ \fi
\begin{problem}

Given the piecewise function 
\[f(x)=\left\{\begin{array}{ll}
{2 \, x + 4}\; , & x\leq{-1}\\[3pt]
{-\frac{4}{5} \, x + \frac{6}{5}}\; , & {-1}< x< {4}\\[3pt]
{4 \, x - 18}\; , & x\geq{4}
\end{array} \right.\]
evaluate the following definite integral by interpreting it in terms of areas.  

\input{Integral-Compute-0005.HELP.tex}

\[
\int_{-6}^{6} f(x)\;dx= \answer{-11}
\]  
\end{problem}}%}

\latexProblemContent{
\ifVerboseLocation This is Integration Compute Question 0005. \\ \fi
\begin{problem}

Given the piecewise function 
\[f(x)=\left\{\begin{array}{ll}
{4 \, x + 4}\; , & x\leq{-5}\\[3pt]
{\frac{32}{3} \, x + \frac{112}{3}}\; , & {-5}< x< {-2}\\[3pt]
{2 \, x + 20}\; , & x\geq{-2}
\end{array} \right.\]
evaluate the following definite integral by interpreting it in terms of areas.  

\input{Integral-Compute-0005.HELP.tex}

\[
\int_{-6}^{0} f(x)\;dx= \answer{18}
\]  
\end{problem}}%}

\latexProblemContent{
\ifVerboseLocation This is Integration Compute Question 0005. \\ \fi
\begin{problem}

Given the piecewise function 
\[f(x)=\left\{\begin{array}{ll}
{2 \, x + 1}\; , & x\leq{-3}\\[3pt]
{2 \, x + 1}\; , & {-3}< x< {2}\\[3pt]
{2 \, x + 1}\; , & x\geq{2}
\end{array} \right.\]
evaluate the following definite integral by interpreting it in terms of areas.  

\input{Integral-Compute-0005.HELP.tex}

\[
\int_{-3}^{-2} f(x)\;dx= \answer{-4}
\]  
\end{problem}}%}

\latexProblemContent{
\ifVerboseLocation This is Integration Compute Question 0005. \\ \fi
\begin{problem}

Given the piecewise function 
\[f(x)=\left\{\begin{array}{ll}
{x + 2}\; , & x\leq{-1}\\[3pt]
{-\frac{2}{5} \, x + \frac{3}{5}}\; , & {-1}< x< {4}\\[3pt]
{4 \, x - 17}\; , & x\geq{4}
\end{array} \right.\]
evaluate the following definite integral by interpreting it in terms of areas.  

\input{Integral-Compute-0005.HELP.tex}

\[
\int_{-4}^{6} f(x)\;dx= \answer{\frac{9}{2}}
\]  
\end{problem}}%}

\latexProblemContent{
\ifVerboseLocation This is Integration Compute Question 0005. \\ \fi
\begin{problem}

Given the piecewise function 
\[f(x)=\left\{\begin{array}{ll}
{2 \, x + 4}\; , & x\leq{2}\\[3pt]
{-\frac{16}{7} \, x + \frac{88}{7}}\; , & {2}< x< {9}\\[3pt]
{4 \, x - 44}\; , & x\geq{9}
\end{array} \right.\]
evaluate the following definite integral by interpreting it in terms of areas.  

\input{Integral-Compute-0005.HELP.tex}

\[
\int_{-1}^{5} f(x)\;dx= \answer{\frac{201}{7}}
\]  
\end{problem}}%}

\latexProblemContent{
\ifVerboseLocation This is Integration Compute Question 0005. \\ \fi
\begin{problem}

Given the piecewise function 
\[f(x)=\left\{\begin{array}{ll}
{x + 1}\; , & x\leq{3}\\[3pt]
{-2 \, x + 10}\; , & {3}< x< {7}\\[3pt]
{4 \, x - 32}\; , & x\geq{7}
\end{array} \right.\]
evaluate the following definite integral by interpreting it in terms of areas.  

\input{Integral-Compute-0005.HELP.tex}

\[
\int_{2}^{8} f(x)\;dx= \answer{\frac{3}{2}}
\]  
\end{problem}}%}

\latexProblemContent{
\ifVerboseLocation This is Integration Compute Question 0005. \\ \fi
\begin{problem}

Given the piecewise function 
\[f(x)=\left\{\begin{array}{ll}
{4 \, x + 1}\; , & x\leq{-5}\\[3pt]
{\frac{38}{3} \, x + \frac{133}{3}}\; , & {-5}< x< {-2}\\[3pt]
{4 \, x + 27}\; , & x\geq{-2}
\end{array} \right.\]
evaluate the following definite integral by interpreting it in terms of areas.  

\input{Integral-Compute-0005.HELP.tex}

\[
\int_{-10}^{1} f(x)\;dx= \answer{-70}
\]  
\end{problem}}%}

\latexProblemContent{
\ifVerboseLocation This is Integration Compute Question 0005. \\ \fi
\begin{problem}

Given the piecewise function 
\[f(x)=\left\{\begin{array}{ll}
{2 \, x + 1}\; , & x\leq{-1}\\[3pt]
{\frac{1}{3} \, x - \frac{2}{3}}\; , & {-1}< x< {5}\\[3pt]
{x - 4}\; , & x\geq{5}
\end{array} \right.\]
evaluate the following definite integral by interpreting it in terms of areas.  

\input{Integral-Compute-0005.HELP.tex}

\[
\int_{-2}^{8} f(x)\;dx= \answer{\frac{11}{2}}
\]  
\end{problem}}%}

\latexProblemContent{
\ifVerboseLocation This is Integration Compute Question 0005. \\ \fi
\begin{problem}

Given the piecewise function 
\[f(x)=\left\{\begin{array}{ll}
{4 \, x + 1}\; , & x\leq{-3}\\[3pt]
{\frac{11}{2} \, x + \frac{11}{2}}\; , & {-3}< x< {1}\\[3pt]
{4 \, x + 7}\; , & x\geq{1}
\end{array} \right.\]
evaluate the following definite integral by interpreting it in terms of areas.  

\input{Integral-Compute-0005.HELP.tex}

\[
\int_{-6}^{2} f(x)\;dx= \answer{-38}
\]  
\end{problem}}%}

\latexProblemContent{
\ifVerboseLocation This is Integration Compute Question 0005. \\ \fi
\begin{problem}

Given the piecewise function 
\[f(x)=\left\{\begin{array}{ll}
{x + 3}\; , & x\leq{5}\\[3pt]
{-\frac{16}{7} \, x + \frac{136}{7}}\; , & {5}< x< {12}\\[3pt]
{x - 20}\; , & x\geq{12}
\end{array} \right.\]
evaluate the following definite integral by interpreting it in terms of areas.  

\input{Integral-Compute-0005.HELP.tex}

\[
\int_{1}^{6} f(x)\;dx= \answer{\frac{216}{7}}
\]  
\end{problem}}%}

\latexProblemContent{
\ifVerboseLocation This is Integration Compute Question 0005. \\ \fi
\begin{problem}

Given the piecewise function 
\[f(x)=\left\{\begin{array}{ll}
{3 \, x + 3}\; , & x\leq{4}\\[3pt]
{-10 \, x + 55}\; , & {4}< x< {7}\\[3pt]
{x - 22}\; , & x\geq{7}
\end{array} \right.\]
evaluate the following definite integral by interpreting it in terms of areas.  

\input{Integral-Compute-0005.HELP.tex}

\[
\int_{2}^{10} f(x)\;dx= \answer{-\frac{33}{2}}
\]  
\end{problem}}%}

\latexProblemContent{
\ifVerboseLocation This is Integration Compute Question 0005. \\ \fi
\begin{problem}

Given the piecewise function 
\[f(x)=\left\{\begin{array}{ll}
{3 \, x + 2}\; , & x\leq{4}\\[3pt]
{-\frac{14}{3} \, x + \frac{98}{3}}\; , & {4}< x< {10}\\[3pt]
{3 \, x - 44}\; , & x\geq{10}
\end{array} \right.\]
evaluate the following definite integral by interpreting it in terms of areas.  

\input{Integral-Compute-0005.HELP.tex}

\[
\int_{2}^{9} f(x)\;dx= \answer{\frac{101}{3}}
\]  
\end{problem}}%}

\latexProblemContent{
\ifVerboseLocation This is Integration Compute Question 0005. \\ \fi
\begin{problem}

Given the piecewise function 
\[f(x)=\left\{\begin{array}{ll}
{x + 4}\; , & x\leq{1}\\[3pt]
{-\frac{5}{3} \, x + \frac{20}{3}}\; , & {1}< x< {7}\\[3pt]
{4 \, x - 33}\; , & x\geq{7}
\end{array} \right.\]
evaluate the following definite integral by interpreting it in terms of areas.  

\input{Integral-Compute-0005.HELP.tex}

\[
\int_{-2}^{9} f(x)\;dx= \answer{\frac{17}{2}}
\]  
\end{problem}}%}

\latexProblemContent{
\ifVerboseLocation This is Integration Compute Question 0005. \\ \fi
\begin{problem}

Given the piecewise function 
\[f(x)=\left\{\begin{array}{ll}
{4 \, x + 5}\; , & x\leq{5}\\[3pt]
{-\frac{25}{3} \, x + \frac{200}{3}}\; , & {5}< x< {11}\\[3pt]
{3 \, x - 58}\; , & x\geq{11}
\end{array} \right.\]
evaluate the following definite integral by interpreting it in terms of areas.  

\input{Integral-Compute-0005.HELP.tex}

\[
\int_{0}^{11} f(x)\;dx= \answer{75}
\]  
\end{problem}}%}

\latexProblemContent{
\ifVerboseLocation This is Integration Compute Question 0005. \\ \fi
\begin{problem}

Given the piecewise function 
\[f(x)=\left\{\begin{array}{ll}
{4 \, x + 4}\; , & x\leq{-5}\\[3pt]
{\frac{16}{3} \, x + \frac{32}{3}}\; , & {-5}< x< {1}\\[3pt]
{3 \, x + 13}\; , & x\geq{1}
\end{array} \right.\]
evaluate the following definite integral by interpreting it in terms of areas.  

\input{Integral-Compute-0005.HELP.tex}

\[
\int_{-8}^{-3} f(x)\;dx= \answer{-\frac{262}{3}}
\]  
\end{problem}}%}

\latexProblemContent{
\ifVerboseLocation This is Integration Compute Question 0005. \\ \fi
\begin{problem}

Given the piecewise function 
\[f(x)=\left\{\begin{array}{ll}
{2 \, x + 2}\; , & x\leq{2}\\[3pt]
{-\frac{12}{5} \, x + \frac{54}{5}}\; , & {2}< x< {7}\\[3pt]
{4 \, x - 34}\; , & x\geq{7}
\end{array} \right.\]
evaluate the following definite integral by interpreting it in terms of areas.  

\input{Integral-Compute-0005.HELP.tex}

\[
\int_{2}^{7} f(x)\;dx= \answer{0}
\]  
\end{problem}}%}

\latexProblemContent{
\ifVerboseLocation This is Integration Compute Question 0005. \\ \fi
\begin{problem}

Given the piecewise function 
\[f(x)=\left\{\begin{array}{ll}
{4 \, x + 5}\; , & x\leq{4}\\[3pt]
{-\frac{21}{2} \, x + 63}\; , & {4}< x< {8}\\[3pt]
{x - 29}\; , & x\geq{8}
\end{array} \right.\]
evaluate the following definite integral by interpreting it in terms of areas.  

\input{Integral-Compute-0005.HELP.tex}

\[
\int_{1}^{12} f(x)\;dx= \answer{-31}
\]  
\end{problem}}%}

\latexProblemContent{
\ifVerboseLocation This is Integration Compute Question 0005. \\ \fi
\begin{problem}

Given the piecewise function 
\[f(x)=\left\{\begin{array}{ll}
{4 \, x + 5}\; , & x\leq{-3}\\[3pt]
{\frac{14}{3} \, x + 7}\; , & {-3}< x< {0}\\[3pt]
{3 \, x + 7}\; , & x\geq{0}
\end{array} \right.\]
evaluate the following definite integral by interpreting it in terms of areas.  

\input{Integral-Compute-0005.HELP.tex}

\[
\int_{-3}^{4} f(x)\;dx= \answer{52}
\]  
\end{problem}}%}

\latexProblemContent{
\ifVerboseLocation This is Integration Compute Question 0005. \\ \fi
\begin{problem}

Given the piecewise function 
\[f(x)=\left\{\begin{array}{ll}
{3 \, x + 3}\; , & x\leq{4}\\[3pt]
{-10 \, x + 55}\; , & {4}< x< {7}\\[3pt]
{4 \, x - 43}\; , & x\geq{7}
\end{array} \right.\]
evaluate the following definite integral by interpreting it in terms of areas.  

\input{Integral-Compute-0005.HELP.tex}

\[
\int_{1}^{9} f(x)\;dx= \answer{\frac{19}{2}}
\]  
\end{problem}}%}

\latexProblemContent{
\ifVerboseLocation This is Integration Compute Question 0005. \\ \fi
\begin{problem}

Given the piecewise function 
\[f(x)=\left\{\begin{array}{ll}
{4 \, x + 4}\; , & x\leq{-1}\\[3pt]
{0}\; , & {-1}< x< {3}\\[3pt]
{x - 3}\; , & x\geq{3}
\end{array} \right.\]
evaluate the following definite integral by interpreting it in terms of areas.  

\input{Integral-Compute-0005.HELP.tex}

\[
\int_{-4}^{3} f(x)\;dx= \answer{-18}
\]  
\end{problem}}%}

\latexProblemContent{
\ifVerboseLocation This is Integration Compute Question 0005. \\ \fi
\begin{problem}

Given the piecewise function 
\[f(x)=\left\{\begin{array}{ll}
{3 \, x + 2}\; , & x\leq{1}\\[3pt]
{-\frac{5}{3} \, x + \frac{20}{3}}\; , & {1}< x< {7}\\[3pt]
{x - 12}\; , & x\geq{7}
\end{array} \right.\]
evaluate the following definite integral by interpreting it in terms of areas.  

\input{Integral-Compute-0005.HELP.tex}

\[
\int_{-2}^{12} f(x)\;dx= \answer{-11}
\]  
\end{problem}}%}

\latexProblemContent{
\ifVerboseLocation This is Integration Compute Question 0005. \\ \fi
\begin{problem}

Given the piecewise function 
\[f(x)=\left\{\begin{array}{ll}
{x + 3}\; , & x\leq{-2}\\[3pt]
{-\frac{1}{2} \, x}\; , & {-2}< x< {2}\\[3pt]
{x - 3}\; , & x\geq{2}
\end{array} \right.\]
evaluate the following definite integral by interpreting it in terms of areas.  

\input{Integral-Compute-0005.HELP.tex}

\[
\int_{-3}^{5} f(x)\;dx= \answer{2}
\]  
\end{problem}}%}

\latexProblemContent{
\ifVerboseLocation This is Integration Compute Question 0005. \\ \fi
\begin{problem}

Given the piecewise function 
\[f(x)=\left\{\begin{array}{ll}
{x + 2}\; , & x\leq{-3}\\[3pt]
{\frac{2}{7} \, x - \frac{1}{7}}\; , & {-3}< x< {4}\\[3pt]
{x - 3}\; , & x\geq{4}
\end{array} \right.\]
evaluate the following definite integral by interpreting it in terms of areas.  

\input{Integral-Compute-0005.HELP.tex}

\[
\int_{-4}^{9} f(x)\;dx= \answer{16}
\]  
\end{problem}}%}

\latexProblemContent{
\ifVerboseLocation This is Integration Compute Question 0005. \\ \fi
\begin{problem}

Given the piecewise function 
\[f(x)=\left\{\begin{array}{ll}
{2 \, x + 4}\; , & x\leq{-5}\\[3pt]
{4 \, x + 14}\; , & {-5}< x< {-2}\\[3pt]
{3 \, x + 12}\; , & x\geq{-2}
\end{array} \right.\]
evaluate the following definite integral by interpreting it in terms of areas.  

\input{Integral-Compute-0005.HELP.tex}

\[
\int_{-7}^{1} f(x)\;dx= \answer{\frac{31}{2}}
\]  
\end{problem}}%}

\latexProblemContent{
\ifVerboseLocation This is Integration Compute Question 0005. \\ \fi
\begin{problem}

Given the piecewise function 
\[f(x)=\left\{\begin{array}{ll}
{3 \, x + 4}\; , & x\leq{-1}\\[3pt]
{-\frac{1}{4} \, x + \frac{3}{4}}\; , & {-1}< x< {7}\\[3pt]
{4 \, x - 29}\; , & x\geq{7}
\end{array} \right.\]
evaluate the following definite integral by interpreting it in terms of areas.  

\input{Integral-Compute-0005.HELP.tex}

\[
\int_{-4}^{5} f(x)\;dx= \answer{-9}
\]  
\end{problem}}%}

\latexProblemContent{
\ifVerboseLocation This is Integration Compute Question 0005. \\ \fi
\begin{problem}

Given the piecewise function 
\[f(x)=\left\{\begin{array}{ll}
{4 \, x + 4}\; , & x\leq{3}\\[3pt]
{-\frac{32}{7} \, x + \frac{208}{7}}\; , & {3}< x< {10}\\[3pt]
{4 \, x - 56}\; , & x\geq{10}
\end{array} \right.\]
evaluate the following definite integral by interpreting it in terms of areas.  

\input{Integral-Compute-0005.HELP.tex}

\[
\int_{3}^{10} f(x)\;dx= \answer{0}
\]  
\end{problem}}%}

\latexProblemContent{
\ifVerboseLocation This is Integration Compute Question 0005. \\ \fi
\begin{problem}

Given the piecewise function 
\[f(x)=\left\{\begin{array}{ll}
{3 \, x + 1}\; , & x\leq{-3}\\[3pt]
{\frac{16}{3} \, x + 8}\; , & {-3}< x< {0}\\[3pt]
{2 \, x + 8}\; , & x\geq{0}
\end{array} \right.\]
evaluate the following definite integral by interpreting it in terms of areas.  

\input{Integral-Compute-0005.HELP.tex}

\[
\int_{-6}^{4} f(x)\;dx= \answer{\frac{21}{2}}
\]  
\end{problem}}%}

\latexProblemContent{
\ifVerboseLocation This is Integration Compute Question 0005. \\ \fi
\begin{problem}

Given the piecewise function 
\[f(x)=\left\{\begin{array}{ll}
{3 \, x + 5}\; , & x\leq{-2}\\[3pt]
{\frac{2}{5} \, x - \frac{1}{5}}\; , & {-2}< x< {3}\\[3pt]
{3 \, x - 8}\; , & x\geq{3}
\end{array} \right.\]
evaluate the following definite integral by interpreting it in terms of areas.  

\input{Integral-Compute-0005.HELP.tex}

\[
\int_{-7}^{6} f(x)\;dx= \answer{-26}
\]  
\end{problem}}%}

\latexProblemContent{
\ifVerboseLocation This is Integration Compute Question 0005. \\ \fi
\begin{problem}

Given the piecewise function 
\[f(x)=\left\{\begin{array}{ll}
{3 \, x + 1}\; , & x\leq{5}\\[3pt]
{-\frac{32}{7} \, x + \frac{272}{7}}\; , & {5}< x< {12}\\[3pt]
{2 \, x - 40}\; , & x\geq{12}
\end{array} \right.\]
evaluate the following definite integral by interpreting it in terms of areas.  

\input{Integral-Compute-0005.HELP.tex}

\[
\int_{3}^{8} f(x)\;dx= \answer{\frac{374}{7}}
\]  
\end{problem}}%}

\latexProblemContent{
\ifVerboseLocation This is Integration Compute Question 0005. \\ \fi
\begin{problem}

Given the piecewise function 
\[f(x)=\left\{\begin{array}{ll}
{x + 4}\; , & x\leq{5}\\[3pt]
{-\frac{18}{7} \, x + \frac{153}{7}}\; , & {5}< x< {12}\\[3pt]
{3 \, x - 45}\; , & x\geq{12}
\end{array} \right.\]
evaluate the following definite integral by interpreting it in terms of areas.  

\input{Integral-Compute-0005.HELP.tex}

\[
\int_{4}^{11} f(x)\;dx= \answer{\frac{227}{14}}
\]  
\end{problem}}%}

\latexProblemContent{
\ifVerboseLocation This is Integration Compute Question 0005. \\ \fi
\begin{problem}

Given the piecewise function 
\[f(x)=\left\{\begin{array}{ll}
{4 \, x + 2}\; , & x\leq{-5}\\[3pt]
{6 \, x + 12}\; , & {-5}< x< {1}\\[3pt]
{x + 17}\; , & x\geq{1}
\end{array} \right.\]
evaluate the following definite integral by interpreting it in terms of areas.  

\input{Integral-Compute-0005.HELP.tex}

\[
\int_{-10}^{2} f(x)\;dx= \answer{-\frac{243}{2}}
\]  
\end{problem}}%}

\latexProblemContent{
\ifVerboseLocation This is Integration Compute Question 0005. \\ \fi
\begin{problem}

Given the piecewise function 
\[f(x)=\left\{\begin{array}{ll}
{3 \, x + 2}\; , & x\leq{-3}\\[3pt]
{\frac{7}{2} \, x + \frac{7}{2}}\; , & {-3}< x< {1}\\[3pt]
{x + 6}\; , & x\geq{1}
\end{array} \right.\]
evaluate the following definite integral by interpreting it in terms of areas.  

\input{Integral-Compute-0005.HELP.tex}

\[
\int_{-3}^{3} f(x)\;dx= \answer{16}
\]  
\end{problem}}%}

\latexProblemContent{
\ifVerboseLocation This is Integration Compute Question 0005. \\ \fi
\begin{problem}

Given the piecewise function 
\[f(x)=\left\{\begin{array}{ll}
{3 \, x + 5}\; , & x\leq{-3}\\[3pt]
{\frac{8}{3} \, x + 4}\; , & {-3}< x< {0}\\[3pt]
{2 \, x + 4}\; , & x\geq{0}
\end{array} \right.\]
evaluate the following definite integral by interpreting it in terms of areas.  

\input{Integral-Compute-0005.HELP.tex}

\[
\int_{-5}^{5} f(x)\;dx= \answer{31}
\]  
\end{problem}}%}

\latexProblemContent{
\ifVerboseLocation This is Integration Compute Question 0005. \\ \fi
\begin{problem}

Given the piecewise function 
\[f(x)=\left\{\begin{array}{ll}
{4 \, x + 5}\; , & x\leq{1}\\[3pt]
{-3 \, x + 12}\; , & {1}< x< {7}\\[3pt]
{2 \, x - 23}\; , & x\geq{7}
\end{array} \right.\]
evaluate the following definite integral by interpreting it in terms of areas.  

\input{Integral-Compute-0005.HELP.tex}

\[
\int_{0}^{10} f(x)\;dx= \answer{-11}
\]  
\end{problem}}%}

\latexProblemContent{
\ifVerboseLocation This is Integration Compute Question 0005. \\ \fi
\begin{problem}

Given the piecewise function 
\[f(x)=\left\{\begin{array}{ll}
{x + 1}\; , & x\leq{-1}\\[3pt]
{0}\; , & {-1}< x< {7}\\[3pt]
{4 \, x - 28}\; , & x\geq{7}
\end{array} \right.\]
evaluate the following definite integral by interpreting it in terms of areas.  

\input{Integral-Compute-0005.HELP.tex}

\[
\int_{-6}^{4} f(x)\;dx= \answer{-\frac{25}{2}}
\]  
\end{problem}}%}

\latexProblemContent{
\ifVerboseLocation This is Integration Compute Question 0005. \\ \fi
\begin{problem}

Given the piecewise function 
\[f(x)=\left\{\begin{array}{ll}
{x + 3}\; , & x\leq{-5}\\[3pt]
{\frac{1}{2} \, x + \frac{1}{2}}\; , & {-5}< x< {3}\\[3pt]
{x - 1}\; , & x\geq{3}
\end{array} \right.\]
evaluate the following definite integral by interpreting it in terms of areas.  

\input{Integral-Compute-0005.HELP.tex}

\[
\int_{-8}^{-2} f(x)\;dx= \answer{-\frac{57}{4}}
\]  
\end{problem}}%}

\latexProblemContent{
\ifVerboseLocation This is Integration Compute Question 0005. \\ \fi
\begin{problem}

Given the piecewise function 
\[f(x)=\left\{\begin{array}{ll}
{3 \, x + 2}\; , & x\leq{5}\\[3pt]
{-\frac{17}{3} \, x + \frac{136}{3}}\; , & {5}< x< {11}\\[3pt]
{3 \, x - 50}\; , & x\geq{11}
\end{array} \right.\]
evaluate the following definite integral by interpreting it in terms of areas.  

\input{Integral-Compute-0005.HELP.tex}

\[
\int_{0}^{10} f(x)\;dx= \answer{\frac{185}{3}}
\]  
\end{problem}}%}

\latexProblemContent{
\ifVerboseLocation This is Integration Compute Question 0005. \\ \fi
\begin{problem}

Given the piecewise function 
\[f(x)=\left\{\begin{array}{ll}
{x + 1}\; , & x\leq{-2}\\[3pt]
{\frac{2}{5} \, x - \frac{1}{5}}\; , & {-2}< x< {3}\\[3pt]
{3 \, x - 8}\; , & x\geq{3}
\end{array} \right.\]
evaluate the following definite integral by interpreting it in terms of areas.  

\input{Integral-Compute-0005.HELP.tex}

\[
\int_{-6}^{5} f(x)\;dx= \answer{-4}
\]  
\end{problem}}%}

\latexProblemContent{
\ifVerboseLocation This is Integration Compute Question 0005. \\ \fi
\begin{problem}

Given the piecewise function 
\[f(x)=\left\{\begin{array}{ll}
{x + 4}\; , & x\leq{5}\\[3pt]
{-3 \, x + 24}\; , & {5}< x< {11}\\[3pt]
{3 \, x - 42}\; , & x\geq{11}
\end{array} \right.\]
evaluate the following definite integral by interpreting it in terms of areas.  

\input{Integral-Compute-0005.HELP.tex}

\[
\int_{0}^{15} f(x)\;dx= \answer{\frac{41}{2}}
\]  
\end{problem}}%}

\latexProblemContent{
\ifVerboseLocation This is Integration Compute Question 0005. \\ \fi
\begin{problem}

Given the piecewise function 
\[f(x)=\left\{\begin{array}{ll}
{x + 3}\; , & x\leq{1}\\[3pt]
{-\frac{8}{7} \, x + \frac{36}{7}}\; , & {1}< x< {8}\\[3pt]
{3 \, x - 28}\; , & x\geq{8}
\end{array} \right.\]
evaluate the following definite integral by interpreting it in terms of areas.  

\input{Integral-Compute-0005.HELP.tex}

\[
\int_{1}^{5} f(x)\;dx= \answer{\frac{48}{7}}
\]  
\end{problem}}%}

\latexProblemContent{
\ifVerboseLocation This is Integration Compute Question 0005. \\ \fi
\begin{problem}

Given the piecewise function 
\[f(x)=\left\{\begin{array}{ll}
{x + 1}\; , & x\leq{3}\\[3pt]
{-2 \, x + 10}\; , & {3}< x< {7}\\[3pt]
{2 \, x - 18}\; , & x\geq{7}
\end{array} \right.\]
evaluate the following definite integral by interpreting it in terms of areas.  

\input{Integral-Compute-0005.HELP.tex}

\[
\int_{1}^{5} f(x)\;dx= \answer{10}
\]  
\end{problem}}%}

\latexProblemContent{
\ifVerboseLocation This is Integration Compute Question 0005. \\ \fi
\begin{problem}

Given the piecewise function 
\[f(x)=\left\{\begin{array}{ll}
{x + 2}\; , & x\leq{-1}\\[3pt]
{-\frac{1}{2} \, x + \frac{1}{2}}\; , & {-1}< x< {3}\\[3pt]
{3 \, x - 10}\; , & x\geq{3}
\end{array} \right.\]
evaluate the following definite integral by interpreting it in terms of areas.  

\input{Integral-Compute-0005.HELP.tex}

\[
\int_{-4}^{4} f(x)\;dx= \answer{-1}
\]  
\end{problem}}%}

\latexProblemContent{
\ifVerboseLocation This is Integration Compute Question 0005. \\ \fi
\begin{problem}

Given the piecewise function 
\[f(x)=\left\{\begin{array}{ll}
{4 \, x + 5}\; , & x\leq{-4}\\[3pt]
{\frac{22}{3} \, x + \frac{55}{3}}\; , & {-4}< x< {-1}\\[3pt]
{4 \, x + 15}\; , & x\geq{-1}
\end{array} \right.\]
evaluate the following definite integral by interpreting it in terms of areas.  

\input{Integral-Compute-0005.HELP.tex}

\[
\int_{-5}^{4} f(x)\;dx= \answer{92}
\]  
\end{problem}}%}

\latexProblemContent{
\ifVerboseLocation This is Integration Compute Question 0005. \\ \fi
\begin{problem}

Given the piecewise function 
\[f(x)=\left\{\begin{array}{ll}
{x + 3}\; , & x\leq{3}\\[3pt]
{-\frac{12}{7} \, x + \frac{78}{7}}\; , & {3}< x< {10}\\[3pt]
{3 \, x - 36}\; , & x\geq{10}
\end{array} \right.\]
evaluate the following definite integral by interpreting it in terms of areas.  

\input{Integral-Compute-0005.HELP.tex}

\[
\int_{-1}^{7} f(x)\;dx= \answer{\frac{184}{7}}
\]  
\end{problem}}%}

\latexProblemContent{
\ifVerboseLocation This is Integration Compute Question 0005. \\ \fi
\begin{problem}

Given the piecewise function 
\[f(x)=\left\{\begin{array}{ll}
{3 \, x + 5}\; , & x\leq{-4}\\[3pt]
{\frac{14}{5} \, x + \frac{21}{5}}\; , & {-4}< x< {1}\\[3pt]
{4 \, x + 3}\; , & x\geq{1}
\end{array} \right.\]
evaluate the following definite integral by interpreting it in terms of areas.  

\input{Integral-Compute-0005.HELP.tex}

\[
\int_{-9}^{-3} f(x)\;dx= \answer{-\frac{781}{10}}
\]  
\end{problem}}%}

\latexProblemContent{
\ifVerboseLocation This is Integration Compute Question 0005. \\ \fi
\begin{problem}

Given the piecewise function 
\[f(x)=\left\{\begin{array}{ll}
{2 \, x + 1}\; , & x\leq{-4}\\[3pt]
{\frac{7}{4} \, x}\; , & {-4}< x< {4}\\[3pt]
{3 \, x - 5}\; , & x\geq{4}
\end{array} \right.\]
evaluate the following definite integral by interpreting it in terms of areas.  

\input{Integral-Compute-0005.HELP.tex}

\[
\int_{-5}^{-1} f(x)\;dx= \answer{-\frac{169}{8}}
\]  
\end{problem}}%}

\latexProblemContent{
\ifVerboseLocation This is Integration Compute Question 0005. \\ \fi
\begin{problem}

Given the piecewise function 
\[f(x)=\left\{\begin{array}{ll}
{4 \, x + 1}\; , & x\leq{2}\\[3pt]
{-\frac{9}{2} \, x + 18}\; , & {2}< x< {6}\\[3pt]
{3 \, x - 27}\; , & x\geq{6}
\end{array} \right.\]
evaluate the following definite integral by interpreting it in terms of areas.  

\input{Integral-Compute-0005.HELP.tex}

\[
\int_{-3}^{3} f(x)\;dx= \answer{\frac{7}{4}}
\]  
\end{problem}}%}

\latexProblemContent{
\ifVerboseLocation This is Integration Compute Question 0005. \\ \fi
\begin{problem}

Given the piecewise function 
\[f(x)=\left\{\begin{array}{ll}
{3 \, x + 1}\; , & x\leq{3}\\[3pt]
{-5 \, x + 25}\; , & {3}< x< {7}\\[3pt]
{3 \, x - 31}\; , & x\geq{7}
\end{array} \right.\]
evaluate the following definite integral by interpreting it in terms of areas.  

\input{Integral-Compute-0005.HELP.tex}

\[
\int_{0}^{12} f(x)\;dx= \answer{4}
\]  
\end{problem}}%}

\latexProblemContent{
\ifVerboseLocation This is Integration Compute Question 0005. \\ \fi
\begin{problem}

Given the piecewise function 
\[f(x)=\left\{\begin{array}{ll}
{x + 3}\; , & x\leq{4}\\[3pt]
{-\frac{7}{2} \, x + 21}\; , & {4}< x< {8}\\[3pt]
{2 \, x - 23}\; , & x\geq{8}
\end{array} \right.\]
evaluate the following definite integral by interpreting it in terms of areas.  

\input{Integral-Compute-0005.HELP.tex}

\[
\int_{0}^{6} f(x)\;dx= \answer{27}
\]  
\end{problem}}%}

\latexProblemContent{
\ifVerboseLocation This is Integration Compute Question 0005. \\ \fi
\begin{problem}

Given the piecewise function 
\[f(x)=\left\{\begin{array}{ll}
{2 \, x + 3}\; , & x\leq{-1}\\[3pt]
{-\frac{2}{7} \, x + \frac{5}{7}}\; , & {-1}< x< {6}\\[3pt]
{2 \, x - 13}\; , & x\geq{6}
\end{array} \right.\]
evaluate the following definite integral by interpreting it in terms of areas.  

\input{Integral-Compute-0005.HELP.tex}

\[
\int_{-3}^{11} f(x)\;dx= \answer{18}
\]  
\end{problem}}%}

\latexProblemContent{
\ifVerboseLocation This is Integration Compute Question 0005. \\ \fi
\begin{problem}

Given the piecewise function 
\[f(x)=\left\{\begin{array}{ll}
{4 \, x + 4}\; , & x\leq{1}\\[3pt]
{-\frac{16}{5} \, x + \frac{56}{5}}\; , & {1}< x< {6}\\[3pt]
{2 \, x - 20}\; , & x\geq{6}
\end{array} \right.\]
evaluate the following definite integral by interpreting it in terms of areas.  

\input{Integral-Compute-0005.HELP.tex}

\[
\int_{-4}^{4} f(x)\;dx= \answer{-\frac{2}{5}}
\]  
\end{problem}}%}

\latexProblemContent{
\ifVerboseLocation This is Integration Compute Question 0005. \\ \fi
\begin{problem}

Given the piecewise function 
\[f(x)=\left\{\begin{array}{ll}
{x + 2}\; , & x\leq{-2}\\[3pt]
{0}\; , & {-2}< x< {2}\\[3pt]
{4 \, x - 8}\; , & x\geq{2}
\end{array} \right.\]
evaluate the following definite integral by interpreting it in terms of areas.  

\input{Integral-Compute-0005.HELP.tex}

\[
\int_{-2}^{5} f(x)\;dx= \answer{18}
\]  
\end{problem}}%}

\latexProblemContent{
\ifVerboseLocation This is Integration Compute Question 0005. \\ \fi
\begin{problem}

Given the piecewise function 
\[f(x)=\left\{\begin{array}{ll}
{3 \, x + 1}\; , & x\leq{1}\\[3pt]
{-x + 5}\; , & {1}< x< {9}\\[3pt]
{x - 13}\; , & x\geq{9}
\end{array} \right.\]
evaluate the following definite integral by interpreting it in terms of areas.  

\input{Integral-Compute-0005.HELP.tex}

\[
\int_{-2}^{7} f(x)\;dx= \answer{\frac{9}{2}}
\]  
\end{problem}}%}

\latexProblemContent{
\ifVerboseLocation This is Integration Compute Question 0005. \\ \fi
\begin{problem}

Given the piecewise function 
\[f(x)=\left\{\begin{array}{ll}
{2 \, x + 4}\; , & x\leq{-1}\\[3pt]
{-\frac{1}{2} \, x + \frac{3}{2}}\; , & {-1}< x< {7}\\[3pt]
{4 \, x - 30}\; , & x\geq{7}
\end{array} \right.\]
evaluate the following definite integral by interpreting it in terms of areas.  

\input{Integral-Compute-0005.HELP.tex}

\[
\int_{-2}^{12} f(x)\;dx= \answer{41}
\]  
\end{problem}}%}

\latexProblemContent{
\ifVerboseLocation This is Integration Compute Question 0005. \\ \fi
\begin{problem}

Given the piecewise function 
\[f(x)=\left\{\begin{array}{ll}
{4 \, x + 1}\; , & x\leq{3}\\[3pt]
{-\frac{26}{5} \, x + \frac{143}{5}}\; , & {3}< x< {8}\\[3pt]
{x - 21}\; , & x\geq{8}
\end{array} \right.\]
evaluate the following definite integral by interpreting it in terms of areas.  

\input{Integral-Compute-0005.HELP.tex}

\[
\int_{1}^{10} f(x)\;dx= \answer{-6}
\]  
\end{problem}}%}

