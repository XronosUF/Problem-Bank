\ProblemFileHeader{XTL_SV_QUESTIONCOUNT}% Process how many problems are in this file and how to detect if it has a desirable problem
\ifproblemToFind% If it has a desirable problem search the file.
%%\tagged{Ans@ShortAns, Type@Compute, Topic@Limit, Sub@DifferenceQuotient, Sub@Poly, File@0013}{
\latexProblemContent{
\ifVerboseLocation This is Derivative Compute Question 0013. \\ \fi
\begin{problem}

Determine if the limit approaches a finite number, $\pm\infty$, or does not exist. (If the limit does not exist, write DNE)

\input{Limit-Compute-0013.HELP.tex}

\[\lim_{h\to0}\frac{-2 \, \sqrt{h + 4} + 4}{h}=\answer{-\frac{1}{2}}\]

\end{problem}}%}

\latexProblemContent{
\ifVerboseLocation This is Derivative Compute Question 0013. \\ \fi
\begin{problem}

Determine if the limit approaches a finite number, $\pm\infty$, or does not exist. (If the limit does not exist, write DNE)

\input{Limit-Compute-0013.HELP.tex}

\[\lim_{h\to0}\frac{4 \, {\left(h + 3\right)}^{3} - 108}{h}=\answer{108}\]

\end{problem}}%}

\latexProblemContent{
\ifVerboseLocation This is Derivative Compute Question 0013. \\ \fi
\begin{problem}

Determine if the limit approaches a finite number, $\pm\infty$, or does not exist. (If the limit does not exist, write DNE)

\input{Limit-Compute-0013.HELP.tex}

\[\lim_{h\to0}\frac{5 \, {\left(h - 3\right)}^{2} - 45}{h}=\answer{-30}\]

\end{problem}}%}

\latexProblemContent{
\ifVerboseLocation This is Derivative Compute Question 0013. \\ \fi
\begin{problem}

Determine if the limit approaches a finite number, $\pm\infty$, or does not exist. (If the limit does not exist, write DNE)

\input{Limit-Compute-0013.HELP.tex}

\[\lim_{h\to0}\frac{3 \, \sqrt{3} - 3 \, \sqrt{h + 3}}{h}=\answer{-\frac{1}{2} \, \sqrt{3}}\]

\end{problem}}%}

\latexProblemContent{
\ifVerboseLocation This is Derivative Compute Question 0013. \\ \fi
\begin{problem}

Determine if the limit approaches a finite number, $\pm\infty$, or does not exist. (If the limit does not exist, write DNE)

\input{Limit-Compute-0013.HELP.tex}

\[\lim_{h\to0}\frac{-\frac{5}{h + 3} + \frac{5}{3}}{h}=\answer{\frac{5}{9}}\]

\end{problem}}%}

\latexProblemContent{
\ifVerboseLocation This is Derivative Compute Question 0013. \\ \fi
\begin{problem}

Determine if the limit approaches a finite number, $\pm\infty$, or does not exist. (If the limit does not exist, write DNE)

\input{Limit-Compute-0013.HELP.tex}

\[\lim_{h\to0}\frac{3 \, \sqrt{h + 1} - 3}{h}=\answer{\frac{3}{2}}\]

\end{problem}}%}

\latexProblemContent{
\ifVerboseLocation This is Derivative Compute Question 0013. \\ \fi
\begin{problem}

Determine if the limit approaches a finite number, $\pm\infty$, or does not exist. (If the limit does not exist, write DNE)

\input{Limit-Compute-0013.HELP.tex}

\[\lim_{h\to0}\frac{2 \, {\left(h - 4\right)}^{3} + 128}{h}=\answer{96}\]

\end{problem}}%}

\latexProblemContent{
\ifVerboseLocation This is Derivative Compute Question 0013. \\ \fi
\begin{problem}

Determine if the limit approaches a finite number, $\pm\infty$, or does not exist. (If the limit does not exist, write DNE)

\input{Limit-Compute-0013.HELP.tex}

\[\lim_{h\to0}\frac{-4 \, {\left(h - 2\right)}^{2} + 16}{h}=\answer{16}\]

\end{problem}}%}

\latexProblemContent{
\ifVerboseLocation This is Derivative Compute Question 0013. \\ \fi
\begin{problem}

Determine if the limit approaches a finite number, $\pm\infty$, or does not exist. (If the limit does not exist, write DNE)

\input{Limit-Compute-0013.HELP.tex}

\[\lim_{h\to0}\frac{\frac{3}{h - 3} + 1}{h}=\answer{-\frac{1}{3}}\]

\end{problem}}%}

\latexProblemContent{
\ifVerboseLocation This is Derivative Compute Question 0013. \\ \fi
\begin{problem}

Determine if the limit approaches a finite number, $\pm\infty$, or does not exist. (If the limit does not exist, write DNE)

\input{Limit-Compute-0013.HELP.tex}

\[\lim_{h\to0}\frac{\sqrt{h + 1} - 1}{h}=\answer{\frac{1}{2}}\]

\end{problem}}%}

\latexProblemContent{
\ifVerboseLocation This is Derivative Compute Question 0013. \\ \fi
\begin{problem}

Determine if the limit approaches a finite number, $\pm\infty$, or does not exist. (If the limit does not exist, write DNE)

\input{Limit-Compute-0013.HELP.tex}

\[\lim_{h\to0}\frac{\frac{3}{h + 3} - 1}{h}=\answer{-\frac{1}{3}}\]

\end{problem}}%}

\latexProblemContent{
\ifVerboseLocation This is Derivative Compute Question 0013. \\ \fi
\begin{problem}

Determine if the limit approaches a finite number, $\pm\infty$, or does not exist. (If the limit does not exist, write DNE)

\input{Limit-Compute-0013.HELP.tex}

\[\lim_{h\to0}\frac{-5 \, {\left(h + 2\right)}^{2} + 20}{h}=\answer{-20}\]

\end{problem}}%}

\latexProblemContent{
\ifVerboseLocation This is Derivative Compute Question 0013. \\ \fi
\begin{problem}

Determine if the limit approaches a finite number, $\pm\infty$, or does not exist. (If the limit does not exist, write DNE)

\input{Limit-Compute-0013.HELP.tex}

\[\lim_{h\to0}\frac{-4 \, {\left(h + 2\right)}^{3} + 32}{h}=\answer{-48}\]

\end{problem}}%}

\latexProblemContent{
\ifVerboseLocation This is Derivative Compute Question 0013. \\ \fi
\begin{problem}

Determine if the limit approaches a finite number, $\pm\infty$, or does not exist. (If the limit does not exist, write DNE)

\input{Limit-Compute-0013.HELP.tex}

\[\lim_{h\to0}\frac{4 \, \sqrt{h + 1} - 4}{h}=\answer{2}\]

\end{problem}}%}

\latexProblemContent{
\ifVerboseLocation This is Derivative Compute Question 0013. \\ \fi
\begin{problem}

Determine if the limit approaches a finite number, $\pm\infty$, or does not exist. (If the limit does not exist, write DNE)

\input{Limit-Compute-0013.HELP.tex}

\[\lim_{h\to0}\frac{5 \, \sqrt{h + 4} - 10}{h}=\answer{\frac{5}{4}}\]

\end{problem}}%}

\latexProblemContent{
\ifVerboseLocation This is Derivative Compute Question 0013. \\ \fi
\begin{problem}

Determine if the limit approaches a finite number, $\pm\infty$, or does not exist. (If the limit does not exist, write DNE)

\input{Limit-Compute-0013.HELP.tex}

\[\lim_{h\to0}\frac{4 \, {\left(h - 4\right)}^{2} - 64}{h}=\answer{-32}\]

\end{problem}}%}

\latexProblemContent{
\ifVerboseLocation This is Derivative Compute Question 0013. \\ \fi
\begin{problem}

Determine if the limit approaches a finite number, $\pm\infty$, or does not exist. (If the limit does not exist, write DNE)

\input{Limit-Compute-0013.HELP.tex}

\[\lim_{h\to0}\frac{\frac{2}{h - 4} + \frac{1}{2}}{h}=\answer{-\frac{1}{8}}\]

\end{problem}}%}

\latexProblemContent{
\ifVerboseLocation This is Derivative Compute Question 0013. \\ \fi
\begin{problem}

Determine if the limit approaches a finite number, $\pm\infty$, or does not exist. (If the limit does not exist, write DNE)

\input{Limit-Compute-0013.HELP.tex}

\[\lim_{h\to0}\frac{-\frac{1}{h + 2} + \frac{1}{2}}{h}=\answer{\frac{1}{4}}\]

\end{problem}}%}

\latexProblemContent{
\ifVerboseLocation This is Derivative Compute Question 0013. \\ \fi
\begin{problem}

Determine if the limit approaches a finite number, $\pm\infty$, or does not exist. (If the limit does not exist, write DNE)

\input{Limit-Compute-0013.HELP.tex}

\[\lim_{h\to0}\frac{-\frac{4}{h - 2} - 2}{h}=\answer{1}\]

\end{problem}}%}

\latexProblemContent{
\ifVerboseLocation This is Derivative Compute Question 0013. \\ \fi
\begin{problem}

Determine if the limit approaches a finite number, $\pm\infty$, or does not exist. (If the limit does not exist, write DNE)

\input{Limit-Compute-0013.HELP.tex}

\[\lim_{h\to0}\frac{-5 \, {\left(h - 2\right)}^{3} - 40}{h}=\answer{-60}\]

\end{problem}}%}

\latexProblemContent{
\ifVerboseLocation This is Derivative Compute Question 0013. \\ \fi
\begin{problem}

Determine if the limit approaches a finite number, $\pm\infty$, or does not exist. (If the limit does not exist, write DNE)

\input{Limit-Compute-0013.HELP.tex}

\[\lim_{h\to0}\frac{\frac{3}{h + 1} - 3}{h}=\answer{-3}\]

\end{problem}}%}

\latexProblemContent{
\ifVerboseLocation This is Derivative Compute Question 0013. \\ \fi
\begin{problem}

Determine if the limit approaches a finite number, $\pm\infty$, or does not exist. (If the limit does not exist, write DNE)

\input{Limit-Compute-0013.HELP.tex}

\[\lim_{h\to0}\frac{\sqrt{2} - \sqrt{h + 2}}{h}=\answer{-\frac{1}{4} \, \sqrt{2}}\]

\end{problem}}%}

\latexProblemContent{
\ifVerboseLocation This is Derivative Compute Question 0013. \\ \fi
\begin{problem}

Determine if the limit approaches a finite number, $\pm\infty$, or does not exist. (If the limit does not exist, write DNE)

\input{Limit-Compute-0013.HELP.tex}

\[\lim_{h\to0}\frac{-4 \, {\left(h + 1\right)}^{3} + 4}{h}=\answer{-12}\]

\end{problem}}%}

\latexProblemContent{
\ifVerboseLocation This is Derivative Compute Question 0013. \\ \fi
\begin{problem}

Determine if the limit approaches a finite number, $\pm\infty$, or does not exist. (If the limit does not exist, write DNE)

\input{Limit-Compute-0013.HELP.tex}

\[\lim_{h\to0}\frac{4 \, {\left(h - 4\right)}^{3} + 256}{h}=\answer{192}\]

\end{problem}}%}

\latexProblemContent{
\ifVerboseLocation This is Derivative Compute Question 0013. \\ \fi
\begin{problem}

Determine if the limit approaches a finite number, $\pm\infty$, or does not exist. (If the limit does not exist, write DNE)

\input{Limit-Compute-0013.HELP.tex}

\[\lim_{h\to0}\frac{5 \, \sqrt{2} - 5 \, \sqrt{h + 2}}{h}=\answer{-\frac{5}{4} \, \sqrt{2}}\]

\end{problem}}%}

\latexProblemContent{
\ifVerboseLocation This is Derivative Compute Question 0013. \\ \fi
\begin{problem}

Determine if the limit approaches a finite number, $\pm\infty$, or does not exist. (If the limit does not exist, write DNE)

\input{Limit-Compute-0013.HELP.tex}

\[\lim_{h\to0}\frac{-\frac{5}{h - 2} - \frac{5}{2}}{h}=\answer{\frac{5}{4}}\]

\end{problem}}%}

\latexProblemContent{
\ifVerboseLocation This is Derivative Compute Question 0013. \\ \fi
\begin{problem}

Determine if the limit approaches a finite number, $\pm\infty$, or does not exist. (If the limit does not exist, write DNE)

\input{Limit-Compute-0013.HELP.tex}

\[\lim_{h\to0}\frac{{\left(h + 2\right)}^{2} - 4}{h}=\answer{4}\]

\end{problem}}%}

\latexProblemContent{
\ifVerboseLocation This is Derivative Compute Question 0013. \\ \fi
\begin{problem}

Determine if the limit approaches a finite number, $\pm\infty$, or does not exist. (If the limit does not exist, write DNE)

\input{Limit-Compute-0013.HELP.tex}

\[\lim_{h\to0}\frac{-2 \, \sqrt{h + 1} + 2}{h}=\answer{-1}\]

\end{problem}}%}

\latexProblemContent{
\ifVerboseLocation This is Derivative Compute Question 0013. \\ \fi
\begin{problem}

Determine if the limit approaches a finite number, $\pm\infty$, or does not exist. (If the limit does not exist, write DNE)

\input{Limit-Compute-0013.HELP.tex}

\[\lim_{h\to0}\frac{{\left(h - 4\right)}^{2} - 16}{h}=\answer{-8}\]

\end{problem}}%}

\latexProblemContent{
\ifVerboseLocation This is Derivative Compute Question 0013. \\ \fi
\begin{problem}

Determine if the limit approaches a finite number, $\pm\infty$, or does not exist. (If the limit does not exist, write DNE)

\input{Limit-Compute-0013.HELP.tex}

\[\lim_{h\to0}\frac{4 \, \sqrt{3} - 4 \, \sqrt{h + 3}}{h}=\answer{-\frac{2}{3} \, \sqrt{3}}\]

\end{problem}}%}

\latexProblemContent{
\ifVerboseLocation This is Derivative Compute Question 0013. \\ \fi
\begin{problem}

Determine if the limit approaches a finite number, $\pm\infty$, or does not exist. (If the limit does not exist, write DNE)

\input{Limit-Compute-0013.HELP.tex}

\[\lim_{h\to0}\frac{-4 \, {\left(h - 4\right)}^{2} + 64}{h}=\answer{32}\]

\end{problem}}%}

\latexProblemContent{
\ifVerboseLocation This is Derivative Compute Question 0013. \\ \fi
\begin{problem}

Determine if the limit approaches a finite number, $\pm\infty$, or does not exist. (If the limit does not exist, write DNE)

\input{Limit-Compute-0013.HELP.tex}

\[\lim_{h\to0}\frac{-2 \, {\left(h - 3\right)}^{2} + 18}{h}=\answer{12}\]

\end{problem}}%}

\latexProblemContent{
\ifVerboseLocation This is Derivative Compute Question 0013. \\ \fi
\begin{problem}

Determine if the limit approaches a finite number, $\pm\infty$, or does not exist. (If the limit does not exist, write DNE)

\input{Limit-Compute-0013.HELP.tex}

\[\lim_{h\to0}\frac{-\frac{1}{h + 4} + \frac{1}{4}}{h}=\answer{\frac{1}{16}}\]

\end{problem}}%}

\latexProblemContent{
\ifVerboseLocation This is Derivative Compute Question 0013. \\ \fi
\begin{problem}

Determine if the limit approaches a finite number, $\pm\infty$, or does not exist. (If the limit does not exist, write DNE)

\input{Limit-Compute-0013.HELP.tex}

\[\lim_{h\to0}\frac{\frac{2}{h + 3} - \frac{2}{3}}{h}=\answer{-\frac{2}{9}}\]

\end{problem}}%}

\latexProblemContent{
\ifVerboseLocation This is Derivative Compute Question 0013. \\ \fi
\begin{problem}

Determine if the limit approaches a finite number, $\pm\infty$, or does not exist. (If the limit does not exist, write DNE)

\input{Limit-Compute-0013.HELP.tex}

\[\lim_{h\to0}\frac{-3 \, {\left(h - 4\right)}^{3} - 192}{h}=\answer{-144}\]

\end{problem}}%}

\latexProblemContent{
\ifVerboseLocation This is Derivative Compute Question 0013. \\ \fi
\begin{problem}

Determine if the limit approaches a finite number, $\pm\infty$, or does not exist. (If the limit does not exist, write DNE)

\input{Limit-Compute-0013.HELP.tex}

\[\lim_{h\to0}\frac{-5 \, {\left(h - 4\right)}^{2} + 80}{h}=\answer{40}\]

\end{problem}}%}

\latexProblemContent{
\ifVerboseLocation This is Derivative Compute Question 0013. \\ \fi
\begin{problem}

Determine if the limit approaches a finite number, $\pm\infty$, or does not exist. (If the limit does not exist, write DNE)

\input{Limit-Compute-0013.HELP.tex}

\[\lim_{h\to0}\frac{-\frac{5}{h - 4} - \frac{5}{4}}{h}=\answer{\frac{5}{16}}\]

\end{problem}}%}

\latexProblemContent{
\ifVerboseLocation This is Derivative Compute Question 0013. \\ \fi
\begin{problem}

Determine if the limit approaches a finite number, $\pm\infty$, or does not exist. (If the limit does not exist, write DNE)

\input{Limit-Compute-0013.HELP.tex}

\[\lim_{h\to0}\frac{2 \, {\left(h + 3\right)}^{2} - 18}{h}=\answer{12}\]

\end{problem}}%}

\latexProblemContent{
\ifVerboseLocation This is Derivative Compute Question 0013. \\ \fi
\begin{problem}

Determine if the limit approaches a finite number, $\pm\infty$, or does not exist. (If the limit does not exist, write DNE)

\input{Limit-Compute-0013.HELP.tex}

\[\lim_{h\to0}\frac{\frac{2}{h + 1} - 2}{h}=\answer{-2}\]

\end{problem}}%}

\latexProblemContent{
\ifVerboseLocation This is Derivative Compute Question 0013. \\ \fi
\begin{problem}

Determine if the limit approaches a finite number, $\pm\infty$, or does not exist. (If the limit does not exist, write DNE)

\input{Limit-Compute-0013.HELP.tex}

\[\lim_{h\to0}\frac{{\left(h - 3\right)}^{2} - 9}{h}=\answer{-6}\]

\end{problem}}%}

\latexProblemContent{
\ifVerboseLocation This is Derivative Compute Question 0013. \\ \fi
\begin{problem}

Determine if the limit approaches a finite number, $\pm\infty$, or does not exist. (If the limit does not exist, write DNE)

\input{Limit-Compute-0013.HELP.tex}

\[\lim_{h\to0}\frac{-\frac{2}{h - 3} - \frac{2}{3}}{h}=\answer{\frac{2}{9}}\]

\end{problem}}%}

\latexProblemContent{
\ifVerboseLocation This is Derivative Compute Question 0013. \\ \fi
\begin{problem}

Determine if the limit approaches a finite number, $\pm\infty$, or does not exist. (If the limit does not exist, write DNE)

\input{Limit-Compute-0013.HELP.tex}

\[\lim_{h\to0}\frac{-4 \, {\left(h + 3\right)}^{3} + 108}{h}=\answer{-108}\]

\end{problem}}%}

\latexProblemContent{
\ifVerboseLocation This is Derivative Compute Question 0013. \\ \fi
\begin{problem}

Determine if the limit approaches a finite number, $\pm\infty$, or does not exist. (If the limit does not exist, write DNE)

\input{Limit-Compute-0013.HELP.tex}

\[\lim_{h\to0}\frac{-\frac{4}{h - 4} - 1}{h}=\answer{\frac{1}{4}}\]

\end{problem}}%}

\latexProblemContent{
\ifVerboseLocation This is Derivative Compute Question 0013. \\ \fi
\begin{problem}

Determine if the limit approaches a finite number, $\pm\infty$, or does not exist. (If the limit does not exist, write DNE)

\input{Limit-Compute-0013.HELP.tex}

\[\lim_{h\to0}\frac{\frac{3}{h + 2} - \frac{3}{2}}{h}=\answer{-\frac{3}{4}}\]

\end{problem}}%}

\latexProblemContent{
\ifVerboseLocation This is Derivative Compute Question 0013. \\ \fi
\begin{problem}

Determine if the limit approaches a finite number, $\pm\infty$, or does not exist. (If the limit does not exist, write DNE)

\input{Limit-Compute-0013.HELP.tex}

\[\lim_{h\to0}\frac{-3 \, \sqrt{2} + 3 \, \sqrt{h + 2}}{h}=\answer{\frac{3}{4} \, \sqrt{2}}\]

\end{problem}}%}

\latexProblemContent{
\ifVerboseLocation This is Derivative Compute Question 0013. \\ \fi
\begin{problem}

Determine if the limit approaches a finite number, $\pm\infty$, or does not exist. (If the limit does not exist, write DNE)

\input{Limit-Compute-0013.HELP.tex}

\[\lim_{h\to0}\frac{5 \, {\left(h - 2\right)}^{2} - 20}{h}=\answer{-20}\]

\end{problem}}%}

\latexProblemContent{
\ifVerboseLocation This is Derivative Compute Question 0013. \\ \fi
\begin{problem}

Determine if the limit approaches a finite number, $\pm\infty$, or does not exist. (If the limit does not exist, write DNE)

\input{Limit-Compute-0013.HELP.tex}

\[\lim_{h\to0}\frac{4 \, {\left(h - 3\right)}^{2} - 36}{h}=\answer{-24}\]

\end{problem}}%}

\latexProblemContent{
\ifVerboseLocation This is Derivative Compute Question 0013. \\ \fi
\begin{problem}

Determine if the limit approaches a finite number, $\pm\infty$, or does not exist. (If the limit does not exist, write DNE)

\input{Limit-Compute-0013.HELP.tex}

\[\lim_{h\to0}\frac{-\frac{4}{h - 1} - 4}{h}=\answer{4}\]

\end{problem}}%}

\latexProblemContent{
\ifVerboseLocation This is Derivative Compute Question 0013. \\ \fi
\begin{problem}

Determine if the limit approaches a finite number, $\pm\infty$, or does not exist. (If the limit does not exist, write DNE)

\input{Limit-Compute-0013.HELP.tex}

\[\lim_{h\to0}\frac{-2 \, {\left(h + 2\right)}^{3} + 16}{h}=\answer{-24}\]

\end{problem}}%}

\latexProblemContent{
\ifVerboseLocation This is Derivative Compute Question 0013. \\ \fi
\begin{problem}

Determine if the limit approaches a finite number, $\pm\infty$, or does not exist. (If the limit does not exist, write DNE)

\input{Limit-Compute-0013.HELP.tex}

\[\lim_{h\to0}\frac{\frac{4}{h + 4} - 1}{h}=\answer{-\frac{1}{4}}\]

\end{problem}}%}

\latexProblemContent{
\ifVerboseLocation This is Derivative Compute Question 0013. \\ \fi
\begin{problem}

Determine if the limit approaches a finite number, $\pm\infty$, or does not exist. (If the limit does not exist, write DNE)

\input{Limit-Compute-0013.HELP.tex}

\[\lim_{h\to0}\frac{-4 \, {\left(h + 4\right)}^{3} + 256}{h}=\answer{-192}\]

\end{problem}}%}

\latexProblemContent{
\ifVerboseLocation This is Derivative Compute Question 0013. \\ \fi
\begin{problem}

Determine if the limit approaches a finite number, $\pm\infty$, or does not exist. (If the limit does not exist, write DNE)

\input{Limit-Compute-0013.HELP.tex}

\[\lim_{h\to0}\frac{-4 \, {\left(h + 1\right)}^{2} + 4}{h}=\answer{-8}\]

\end{problem}}%}

\latexProblemContent{
\ifVerboseLocation This is Derivative Compute Question 0013. \\ \fi
\begin{problem}

Determine if the limit approaches a finite number, $\pm\infty$, or does not exist. (If the limit does not exist, write DNE)

\input{Limit-Compute-0013.HELP.tex}

\[\lim_{h\to0}\frac{-3 \, {\left(h - 4\right)}^{2} + 48}{h}=\answer{24}\]

\end{problem}}%}

\latexProblemContent{
\ifVerboseLocation This is Derivative Compute Question 0013. \\ \fi
\begin{problem}

Determine if the limit approaches a finite number, $\pm\infty$, or does not exist. (If the limit does not exist, write DNE)

\input{Limit-Compute-0013.HELP.tex}

\[\lim_{h\to0}\frac{-\frac{4}{h + 3} + \frac{4}{3}}{h}=\answer{\frac{4}{9}}\]

\end{problem}}%}

\latexProblemContent{
\ifVerboseLocation This is Derivative Compute Question 0013. \\ \fi
\begin{problem}

Determine if the limit approaches a finite number, $\pm\infty$, or does not exist. (If the limit does not exist, write DNE)

\input{Limit-Compute-0013.HELP.tex}

\[\lim_{h\to0}\frac{-3 \, {\left(h + 1\right)}^{3} + 3}{h}=\answer{-9}\]

\end{problem}}%}

\latexProblemContent{
\ifVerboseLocation This is Derivative Compute Question 0013. \\ \fi
\begin{problem}

Determine if the limit approaches a finite number, $\pm\infty$, or does not exist. (If the limit does not exist, write DNE)

\input{Limit-Compute-0013.HELP.tex}

\[\lim_{h\to0}\frac{-\frac{1}{h + 3} + \frac{1}{3}}{h}=\answer{\frac{1}{9}}\]

\end{problem}}%}

\latexProblemContent{
\ifVerboseLocation This is Derivative Compute Question 0013. \\ \fi
\begin{problem}

Determine if the limit approaches a finite number, $\pm\infty$, or does not exist. (If the limit does not exist, write DNE)

\input{Limit-Compute-0013.HELP.tex}

\[\lim_{h\to0}\frac{-5 \, {\left(h + 3\right)}^{3} + 135}{h}=\answer{-135}\]

\end{problem}}%}

\latexProblemContent{
\ifVerboseLocation This is Derivative Compute Question 0013. \\ \fi
\begin{problem}

Determine if the limit approaches a finite number, $\pm\infty$, or does not exist. (If the limit does not exist, write DNE)

\input{Limit-Compute-0013.HELP.tex}

\[\lim_{h\to0}\frac{-\frac{3}{h - 1} - 3}{h}=\answer{3}\]

\end{problem}}%}

\latexProblemContent{
\ifVerboseLocation This is Derivative Compute Question 0013. \\ \fi
\begin{problem}

Determine if the limit approaches a finite number, $\pm\infty$, or does not exist. (If the limit does not exist, write DNE)

\input{Limit-Compute-0013.HELP.tex}

\[\lim_{h\to0}\frac{-\frac{5}{h + 1} + 5}{h}=\answer{5}\]

\end{problem}}%}

\latexProblemContent{
\ifVerboseLocation This is Derivative Compute Question 0013. \\ \fi
\begin{problem}

Determine if the limit approaches a finite number, $\pm\infty$, or does not exist. (If the limit does not exist, write DNE)

\input{Limit-Compute-0013.HELP.tex}

\[\lim_{h\to0}\frac{-4 \, {\left(h + 2\right)}^{2} + 16}{h}=\answer{-16}\]

\end{problem}}%}

\latexProblemContent{
\ifVerboseLocation This is Derivative Compute Question 0013. \\ \fi
\begin{problem}

Determine if the limit approaches a finite number, $\pm\infty$, or does not exist. (If the limit does not exist, write DNE)

\input{Limit-Compute-0013.HELP.tex}

\[\lim_{h\to0}\frac{-\frac{1}{h + 1} + 1}{h}=\answer{1}\]

\end{problem}}%}

\latexProblemContent{
\ifVerboseLocation This is Derivative Compute Question 0013. \\ \fi
\begin{problem}

Determine if the limit approaches a finite number, $\pm\infty$, or does not exist. (If the limit does not exist, write DNE)

\input{Limit-Compute-0013.HELP.tex}

\[\lim_{h\to0}\frac{3 \, {\left(h + 2\right)}^{3} - 24}{h}=\answer{36}\]

\end{problem}}%}

\latexProblemContent{
\ifVerboseLocation This is Derivative Compute Question 0013. \\ \fi
\begin{problem}

Determine if the limit approaches a finite number, $\pm\infty$, or does not exist. (If the limit does not exist, write DNE)

\input{Limit-Compute-0013.HELP.tex}

\[\lim_{h\to0}\frac{4 \, \sqrt{h + 4} - 8}{h}=\answer{1}\]

\end{problem}}%}

\latexProblemContent{
\ifVerboseLocation This is Derivative Compute Question 0013. \\ \fi
\begin{problem}

Determine if the limit approaches a finite number, $\pm\infty$, or does not exist. (If the limit does not exist, write DNE)

\input{Limit-Compute-0013.HELP.tex}

\[\lim_{h\to0}\frac{-5 \, {\left(h - 1\right)}^{2} + 5}{h}=\answer{10}\]

\end{problem}}%}

\latexProblemContent{
\ifVerboseLocation This is Derivative Compute Question 0013. \\ \fi
\begin{problem}

Determine if the limit approaches a finite number, $\pm\infty$, or does not exist. (If the limit does not exist, write DNE)

\input{Limit-Compute-0013.HELP.tex}

\[\lim_{h\to0}\frac{\frac{4}{h + 1} - 4}{h}=\answer{-4}\]

\end{problem}}%}

\latexProblemContent{
\ifVerboseLocation This is Derivative Compute Question 0013. \\ \fi
\begin{problem}

Determine if the limit approaches a finite number, $\pm\infty$, or does not exist. (If the limit does not exist, write DNE)

\input{Limit-Compute-0013.HELP.tex}

\[\lim_{h\to0}\frac{\frac{5}{h + 3} - \frac{5}{3}}{h}=\answer{-\frac{5}{9}}\]

\end{problem}}%}

\latexProblemContent{
\ifVerboseLocation This is Derivative Compute Question 0013. \\ \fi
\begin{problem}

Determine if the limit approaches a finite number, $\pm\infty$, or does not exist. (If the limit does not exist, write DNE)

\input{Limit-Compute-0013.HELP.tex}

\[\lim_{h\to0}\frac{\frac{1}{h - 4} + \frac{1}{4}}{h}=\answer{-\frac{1}{16}}\]

\end{problem}}%}

\latexProblemContent{
\ifVerboseLocation This is Derivative Compute Question 0013. \\ \fi
\begin{problem}

Determine if the limit approaches a finite number, $\pm\infty$, or does not exist. (If the limit does not exist, write DNE)

\input{Limit-Compute-0013.HELP.tex}

\[\lim_{h\to0}\frac{\frac{5}{h + 4} - \frac{5}{4}}{h}=\answer{-\frac{5}{16}}\]

\end{problem}}%}

\latexProblemContent{
\ifVerboseLocation This is Derivative Compute Question 0013. \\ \fi
\begin{problem}

Determine if the limit approaches a finite number, $\pm\infty$, or does not exist. (If the limit does not exist, write DNE)

\input{Limit-Compute-0013.HELP.tex}

\[\lim_{h\to0}\frac{-\frac{2}{h + 2} + 1}{h}=\answer{\frac{1}{2}}\]

\end{problem}}%}

\latexProblemContent{
\ifVerboseLocation This is Derivative Compute Question 0013. \\ \fi
\begin{problem}

Determine if the limit approaches a finite number, $\pm\infty$, or does not exist. (If the limit does not exist, write DNE)

\input{Limit-Compute-0013.HELP.tex}

\[\lim_{h\to0}\frac{-4 \, \sqrt{3} + 4 \, \sqrt{h + 3}}{h}=\answer{\frac{2}{3} \, \sqrt{3}}\]

\end{problem}}%}

\latexProblemContent{
\ifVerboseLocation This is Derivative Compute Question 0013. \\ \fi
\begin{problem}

Determine if the limit approaches a finite number, $\pm\infty$, or does not exist. (If the limit does not exist, write DNE)

\input{Limit-Compute-0013.HELP.tex}

\[\lim_{h\to0}\frac{-3 \, \sqrt{h + 1} + 3}{h}=\answer{-\frac{3}{2}}\]

\end{problem}}%}

\latexProblemContent{
\ifVerboseLocation This is Derivative Compute Question 0013. \\ \fi
\begin{problem}

Determine if the limit approaches a finite number, $\pm\infty$, or does not exist. (If the limit does not exist, write DNE)

\input{Limit-Compute-0013.HELP.tex}

\[\lim_{h\to0}\frac{\frac{4}{h - 3} + \frac{4}{3}}{h}=\answer{-\frac{4}{9}}\]

\end{problem}}%}

\latexProblemContent{
\ifVerboseLocation This is Derivative Compute Question 0013. \\ \fi
\begin{problem}

Determine if the limit approaches a finite number, $\pm\infty$, or does not exist. (If the limit does not exist, write DNE)

\input{Limit-Compute-0013.HELP.tex}

\[\lim_{h\to0}\frac{-\frac{4}{h + 2} + 2}{h}=\answer{1}\]

\end{problem}}%}

\latexProblemContent{
\ifVerboseLocation This is Derivative Compute Question 0013. \\ \fi
\begin{problem}

Determine if the limit approaches a finite number, $\pm\infty$, or does not exist. (If the limit does not exist, write DNE)

\input{Limit-Compute-0013.HELP.tex}

\[\lim_{h\to0}\frac{\frac{2}{h + 2} - 1}{h}=\answer{-\frac{1}{2}}\]

\end{problem}}%}

\latexProblemContent{
\ifVerboseLocation This is Derivative Compute Question 0013. \\ \fi
\begin{problem}

Determine if the limit approaches a finite number, $\pm\infty$, or does not exist. (If the limit does not exist, write DNE)

\input{Limit-Compute-0013.HELP.tex}

\[\lim_{h\to0}\frac{4 \, {\left(h + 2\right)}^{3} - 32}{h}=\answer{48}\]

\end{problem}}%}

\latexProblemContent{
\ifVerboseLocation This is Derivative Compute Question 0013. \\ \fi
\begin{problem}

Determine if the limit approaches a finite number, $\pm\infty$, or does not exist. (If the limit does not exist, write DNE)

\input{Limit-Compute-0013.HELP.tex}

\[\lim_{h\to0}\frac{-2 \, {\left(h - 2\right)}^{3} - 16}{h}=\answer{-24}\]

\end{problem}}%}

\latexProblemContent{
\ifVerboseLocation This is Derivative Compute Question 0013. \\ \fi
\begin{problem}

Determine if the limit approaches a finite number, $\pm\infty$, or does not exist. (If the limit does not exist, write DNE)

\input{Limit-Compute-0013.HELP.tex}

\[\lim_{h\to0}\frac{2 \, \sqrt{2} - 2 \, \sqrt{h + 2}}{h}=\answer{-\frac{1}{2} \, \sqrt{2}}\]

\end{problem}}%}

\latexProblemContent{
\ifVerboseLocation This is Derivative Compute Question 0013. \\ \fi
\begin{problem}

Determine if the limit approaches a finite number, $\pm\infty$, or does not exist. (If the limit does not exist, write DNE)

\input{Limit-Compute-0013.HELP.tex}

\[\lim_{h\to0}\frac{-{\left(h + 2\right)}^{2} + 4}{h}=\answer{-4}\]

\end{problem}}%}

\latexProblemContent{
\ifVerboseLocation This is Derivative Compute Question 0013. \\ \fi
\begin{problem}

Determine if the limit approaches a finite number, $\pm\infty$, or does not exist. (If the limit does not exist, write DNE)

\input{Limit-Compute-0013.HELP.tex}

\[\lim_{h\to0}\frac{2 \, {\left(h - 2\right)}^{3} + 16}{h}=\answer{24}\]

\end{problem}}%}

\latexProblemContent{
\ifVerboseLocation This is Derivative Compute Question 0013. \\ \fi
\begin{problem}

Determine if the limit approaches a finite number, $\pm\infty$, or does not exist. (If the limit does not exist, write DNE)

\input{Limit-Compute-0013.HELP.tex}

\[\lim_{h\to0}\frac{3 \, {\left(h + 1\right)}^{3} - 3}{h}=\answer{9}\]

\end{problem}}%}

\latexProblemContent{
\ifVerboseLocation This is Derivative Compute Question 0013. \\ \fi
\begin{problem}

Determine if the limit approaches a finite number, $\pm\infty$, or does not exist. (If the limit does not exist, write DNE)

\input{Limit-Compute-0013.HELP.tex}

\[\lim_{h\to0}\frac{2 \, {\left(h + 4\right)}^{2} - 32}{h}=\answer{16}\]

\end{problem}}%}

\latexProblemContent{
\ifVerboseLocation This is Derivative Compute Question 0013. \\ \fi
\begin{problem}

Determine if the limit approaches a finite number, $\pm\infty$, or does not exist. (If the limit does not exist, write DNE)

\input{Limit-Compute-0013.HELP.tex}

\[\lim_{h\to0}\frac{3 \, {\left(h - 1\right)}^{3} + 3}{h}=\answer{9}\]

\end{problem}}%}

\latexProblemContent{
\ifVerboseLocation This is Derivative Compute Question 0013. \\ \fi
\begin{problem}

Determine if the limit approaches a finite number, $\pm\infty$, or does not exist. (If the limit does not exist, write DNE)

\input{Limit-Compute-0013.HELP.tex}

\[\lim_{h\to0}\frac{\frac{2}{h - 2} + 1}{h}=\answer{-\frac{1}{2}}\]

\end{problem}}%}

\latexProblemContent{
\ifVerboseLocation This is Derivative Compute Question 0013. \\ \fi
\begin{problem}

Determine if the limit approaches a finite number, $\pm\infty$, or does not exist. (If the limit does not exist, write DNE)

\input{Limit-Compute-0013.HELP.tex}

\[\lim_{h\to0}\frac{2 \, {\left(h + 2\right)}^{2} - 8}{h}=\answer{8}\]

\end{problem}}%}

\latexProblemContent{
\ifVerboseLocation This is Derivative Compute Question 0013. \\ \fi
\begin{problem}

Determine if the limit approaches a finite number, $\pm\infty$, or does not exist. (If the limit does not exist, write DNE)

\input{Limit-Compute-0013.HELP.tex}

\[\lim_{h\to0}\frac{2 \, \sqrt{3} - 2 \, \sqrt{h + 3}}{h}=\answer{-\frac{1}{3} \, \sqrt{3}}\]

\end{problem}}%}

\latexProblemContent{
\ifVerboseLocation This is Derivative Compute Question 0013. \\ \fi
\begin{problem}

Determine if the limit approaches a finite number, $\pm\infty$, or does not exist. (If the limit does not exist, write DNE)

\input{Limit-Compute-0013.HELP.tex}

\[\lim_{h\to0}\frac{3 \, {\left(h - 3\right)}^{3} + 81}{h}=\answer{81}\]

\end{problem}}%}

\latexProblemContent{
\ifVerboseLocation This is Derivative Compute Question 0013. \\ \fi
\begin{problem}

Determine if the limit approaches a finite number, $\pm\infty$, or does not exist. (If the limit does not exist, write DNE)

\input{Limit-Compute-0013.HELP.tex}

\[\lim_{h\to0}\frac{\frac{5}{h - 2} + \frac{5}{2}}{h}=\answer{-\frac{5}{4}}\]

\end{problem}}%}

\latexProblemContent{
\ifVerboseLocation This is Derivative Compute Question 0013. \\ \fi
\begin{problem}

Determine if the limit approaches a finite number, $\pm\infty$, or does not exist. (If the limit does not exist, write DNE)

\input{Limit-Compute-0013.HELP.tex}

\[\lim_{h\to0}\frac{-\sqrt{h + 1} + 1}{h}=\answer{-\frac{1}{2}}\]

\end{problem}}%}

\latexProblemContent{
\ifVerboseLocation This is Derivative Compute Question 0013. \\ \fi
\begin{problem}

Determine if the limit approaches a finite number, $\pm\infty$, or does not exist. (If the limit does not exist, write DNE)

\input{Limit-Compute-0013.HELP.tex}

\[\lim_{h\to0}\frac{5 \, {\left(h - 4\right)}^{3} + 320}{h}=\answer{240}\]

\end{problem}}%}

\latexProblemContent{
\ifVerboseLocation This is Derivative Compute Question 0013. \\ \fi
\begin{problem}

Determine if the limit approaches a finite number, $\pm\infty$, or does not exist. (If the limit does not exist, write DNE)

\input{Limit-Compute-0013.HELP.tex}

\[\lim_{h\to0}\frac{-2 \, {\left(h - 1\right)}^{2} + 2}{h}=\answer{4}\]

\end{problem}}%}

\latexProblemContent{
\ifVerboseLocation This is Derivative Compute Question 0013. \\ \fi
\begin{problem}

Determine if the limit approaches a finite number, $\pm\infty$, or does not exist. (If the limit does not exist, write DNE)

\input{Limit-Compute-0013.HELP.tex}

\[\lim_{h\to0}\frac{\sqrt{h + 4} - 2}{h}=\answer{\frac{1}{4}}\]

\end{problem}}%}

\latexProblemContent{
\ifVerboseLocation This is Derivative Compute Question 0013. \\ \fi
\begin{problem}

Determine if the limit approaches a finite number, $\pm\infty$, or does not exist. (If the limit does not exist, write DNE)

\input{Limit-Compute-0013.HELP.tex}

\[\lim_{h\to0}\frac{4 \, {\left(h + 3\right)}^{2} - 36}{h}=\answer{24}\]

\end{problem}}%}

\latexProblemContent{
\ifVerboseLocation This is Derivative Compute Question 0013. \\ \fi
\begin{problem}

Determine if the limit approaches a finite number, $\pm\infty$, or does not exist. (If the limit does not exist, write DNE)

\input{Limit-Compute-0013.HELP.tex}

\[\lim_{h\to0}\frac{-3 \, {\left(h + 4\right)}^{2} + 48}{h}=\answer{-24}\]

\end{problem}}%}

\latexProblemContent{
\ifVerboseLocation This is Derivative Compute Question 0013. \\ \fi
\begin{problem}

Determine if the limit approaches a finite number, $\pm\infty$, or does not exist. (If the limit does not exist, write DNE)

\input{Limit-Compute-0013.HELP.tex}

\[\lim_{h\to0}\frac{-2 \, {\left(h + 3\right)}^{2} + 18}{h}=\answer{-12}\]

\end{problem}}%}

\latexProblemContent{
\ifVerboseLocation This is Derivative Compute Question 0013. \\ \fi
\begin{problem}

Determine if the limit approaches a finite number, $\pm\infty$, or does not exist. (If the limit does not exist, write DNE)

\input{Limit-Compute-0013.HELP.tex}

\[\lim_{h\to0}\frac{-3 \, {\left(h - 3\right)}^{2} + 27}{h}=\answer{18}\]

\end{problem}}%}

\latexProblemContent{
\ifVerboseLocation This is Derivative Compute Question 0013. \\ \fi
\begin{problem}

Determine if the limit approaches a finite number, $\pm\infty$, or does not exist. (If the limit does not exist, write DNE)

\input{Limit-Compute-0013.HELP.tex}

\[\lim_{h\to0}\frac{-5 \, \sqrt{h + 1} + 5}{h}=\answer{-\frac{5}{2}}\]

\end{problem}}%}

\latexProblemContent{
\ifVerboseLocation This is Derivative Compute Question 0013. \\ \fi
\begin{problem}

Determine if the limit approaches a finite number, $\pm\infty$, or does not exist. (If the limit does not exist, write DNE)

\input{Limit-Compute-0013.HELP.tex}

\[\lim_{h\to0}\frac{{\left(h + 2\right)}^{3} - 8}{h}=\answer{12}\]

\end{problem}}%}

\latexProblemContent{
\ifVerboseLocation This is Derivative Compute Question 0013. \\ \fi
\begin{problem}

Determine if the limit approaches a finite number, $\pm\infty$, or does not exist. (If the limit does not exist, write DNE)

\input{Limit-Compute-0013.HELP.tex}

\[\lim_{h\to0}\frac{-5 \, {\left(h + 1\right)}^{3} + 5}{h}=\answer{-15}\]

\end{problem}}%}

\latexProblemContent{
\ifVerboseLocation This is Derivative Compute Question 0013. \\ \fi
\begin{problem}

Determine if the limit approaches a finite number, $\pm\infty$, or does not exist. (If the limit does not exist, write DNE)

\input{Limit-Compute-0013.HELP.tex}

\[\lim_{h\to0}\frac{-\frac{4}{h - 3} - \frac{4}{3}}{h}=\answer{\frac{4}{9}}\]

\end{problem}}%}

\latexProblemContent{
\ifVerboseLocation This is Derivative Compute Question 0013. \\ \fi
\begin{problem}

Determine if the limit approaches a finite number, $\pm\infty$, or does not exist. (If the limit does not exist, write DNE)

\input{Limit-Compute-0013.HELP.tex}

\[\lim_{h\to0}\frac{-\frac{1}{h - 2} - \frac{1}{2}}{h}=\answer{\frac{1}{4}}\]

\end{problem}}%}

\latexProblemContent{
\ifVerboseLocation This is Derivative Compute Question 0013. \\ \fi
\begin{problem}

Determine if the limit approaches a finite number, $\pm\infty$, or does not exist. (If the limit does not exist, write DNE)

\input{Limit-Compute-0013.HELP.tex}

\[\lim_{h\to0}\frac{-{\left(h + 2\right)}^{3} + 8}{h}=\answer{-12}\]

\end{problem}}%}

\latexProblemContent{
\ifVerboseLocation This is Derivative Compute Question 0013. \\ \fi
\begin{problem}

Determine if the limit approaches a finite number, $\pm\infty$, or does not exist. (If the limit does not exist, write DNE)

\input{Limit-Compute-0013.HELP.tex}

\[\lim_{h\to0}\frac{-3 \, \sqrt{h + 4} + 6}{h}=\answer{-\frac{3}{4}}\]

\end{problem}}%}

\latexProblemContent{
\ifVerboseLocation This is Derivative Compute Question 0013. \\ \fi
\begin{problem}

Determine if the limit approaches a finite number, $\pm\infty$, or does not exist. (If the limit does not exist, write DNE)

\input{Limit-Compute-0013.HELP.tex}

\[\lim_{h\to0}\frac{-2 \, {\left(h + 4\right)}^{2} + 32}{h}=\answer{-16}\]

\end{problem}}%}

\latexProblemContent{
\ifVerboseLocation This is Derivative Compute Question 0013. \\ \fi
\begin{problem}

Determine if the limit approaches a finite number, $\pm\infty$, or does not exist. (If the limit does not exist, write DNE)

\input{Limit-Compute-0013.HELP.tex}

\[\lim_{h\to0}\frac{-5 \, {\left(h - 3\right)}^{2} + 45}{h}=\answer{30}\]

\end{problem}}%}

\latexProblemContent{
\ifVerboseLocation This is Derivative Compute Question 0013. \\ \fi
\begin{problem}

Determine if the limit approaches a finite number, $\pm\infty$, or does not exist. (If the limit does not exist, write DNE)

\input{Limit-Compute-0013.HELP.tex}

\[\lim_{h\to0}\frac{-3 \, {\left(h - 2\right)}^{3} - 24}{h}=\answer{-36}\]

\end{problem}}%}

\latexProblemContent{
\ifVerboseLocation This is Derivative Compute Question 0013. \\ \fi
\begin{problem}

Determine if the limit approaches a finite number, $\pm\infty$, or does not exist. (If the limit does not exist, write DNE)

\input{Limit-Compute-0013.HELP.tex}

\[\lim_{h\to0}\frac{-5 \, \sqrt{h + 4} + 10}{h}=\answer{-\frac{5}{4}}\]

\end{problem}}%}

\latexProblemContent{
\ifVerboseLocation This is Derivative Compute Question 0013. \\ \fi
\begin{problem}

Determine if the limit approaches a finite number, $\pm\infty$, or does not exist. (If the limit does not exist, write DNE)

\input{Limit-Compute-0013.HELP.tex}

\[\lim_{h\to0}\frac{\frac{1}{h - 1} + 1}{h}=\answer{-1}\]

\end{problem}}%}

\latexProblemContent{
\ifVerboseLocation This is Derivative Compute Question 0013. \\ \fi
\begin{problem}

Determine if the limit approaches a finite number, $\pm\infty$, or does not exist. (If the limit does not exist, write DNE)

\input{Limit-Compute-0013.HELP.tex}

\[\lim_{h\to0}\frac{-5 \, \sqrt{2} + 5 \, \sqrt{h + 2}}{h}=\answer{\frac{5}{4} \, \sqrt{2}}\]

\end{problem}}%}

\latexProblemContent{
\ifVerboseLocation This is Derivative Compute Question 0013. \\ \fi
\begin{problem}

Determine if the limit approaches a finite number, $\pm\infty$, or does not exist. (If the limit does not exist, write DNE)

\input{Limit-Compute-0013.HELP.tex}

\[\lim_{h\to0}\frac{5 \, \sqrt{h + 1} - 5}{h}=\answer{\frac{5}{2}}\]

\end{problem}}%}

\latexProblemContent{
\ifVerboseLocation This is Derivative Compute Question 0013. \\ \fi
\begin{problem}

Determine if the limit approaches a finite number, $\pm\infty$, or does not exist. (If the limit does not exist, write DNE)

\input{Limit-Compute-0013.HELP.tex}

\[\lim_{h\to0}\frac{5 \, \sqrt{3} - 5 \, \sqrt{h + 3}}{h}=\answer{-\frac{5}{6} \, \sqrt{3}}\]

\end{problem}}%}

\latexProblemContent{
\ifVerboseLocation This is Derivative Compute Question 0013. \\ \fi
\begin{problem}

Determine if the limit approaches a finite number, $\pm\infty$, or does not exist. (If the limit does not exist, write DNE)

\input{Limit-Compute-0013.HELP.tex}

\[\lim_{h\to0}\frac{3 \, {\left(h - 2\right)}^{3} + 24}{h}=\answer{36}\]

\end{problem}}%}

\latexProblemContent{
\ifVerboseLocation This is Derivative Compute Question 0013. \\ \fi
\begin{problem}

Determine if the limit approaches a finite number, $\pm\infty$, or does not exist. (If the limit does not exist, write DNE)

\input{Limit-Compute-0013.HELP.tex}

\[\lim_{h\to0}\frac{-\frac{4}{h + 1} + 4}{h}=\answer{4}\]

\end{problem}}%}

\latexProblemContent{
\ifVerboseLocation This is Derivative Compute Question 0013. \\ \fi
\begin{problem}

Determine if the limit approaches a finite number, $\pm\infty$, or does not exist. (If the limit does not exist, write DNE)

\input{Limit-Compute-0013.HELP.tex}

\[\lim_{h\to0}\frac{2 \, \sqrt{h + 4} - 4}{h}=\answer{\frac{1}{2}}\]

\end{problem}}%}

\latexProblemContent{
\ifVerboseLocation This is Derivative Compute Question 0013. \\ \fi
\begin{problem}

Determine if the limit approaches a finite number, $\pm\infty$, or does not exist. (If the limit does not exist, write DNE)

\input{Limit-Compute-0013.HELP.tex}

\[\lim_{h\to0}\frac{4 \, {\left(h + 2\right)}^{2} - 16}{h}=\answer{16}\]

\end{problem}}%}

\latexProblemContent{
\ifVerboseLocation This is Derivative Compute Question 0013. \\ \fi
\begin{problem}

Determine if the limit approaches a finite number, $\pm\infty$, or does not exist. (If the limit does not exist, write DNE)

\input{Limit-Compute-0013.HELP.tex}

\[\lim_{h\to0}\frac{2 \, {\left(h - 3\right)}^{3} + 54}{h}=\answer{54}\]

\end{problem}}%}

\latexProblemContent{
\ifVerboseLocation This is Derivative Compute Question 0013. \\ \fi
\begin{problem}

Determine if the limit approaches a finite number, $\pm\infty$, or does not exist. (If the limit does not exist, write DNE)

\input{Limit-Compute-0013.HELP.tex}

\[\lim_{h\to0}\frac{-3 \, {\left(h + 3\right)}^{2} + 27}{h}=\answer{-18}\]

\end{problem}}%}

\latexProblemContent{
\ifVerboseLocation This is Derivative Compute Question 0013. \\ \fi
\begin{problem}

Determine if the limit approaches a finite number, $\pm\infty$, or does not exist. (If the limit does not exist, write DNE)

\input{Limit-Compute-0013.HELP.tex}

\[\lim_{h\to0}\frac{-2 \, {\left(h - 2\right)}^{2} + 8}{h}=\answer{8}\]

\end{problem}}%}

\latexProblemContent{
\ifVerboseLocation This is Derivative Compute Question 0013. \\ \fi
\begin{problem}

Determine if the limit approaches a finite number, $\pm\infty$, or does not exist. (If the limit does not exist, write DNE)

\input{Limit-Compute-0013.HELP.tex}

\[\lim_{h\to0}\frac{4 \, \sqrt{2} - 4 \, \sqrt{h + 2}}{h}=\answer{-\sqrt{2}}\]

\end{problem}}%}

\latexProblemContent{
\ifVerboseLocation This is Derivative Compute Question 0013. \\ \fi
\begin{problem}

Determine if the limit approaches a finite number, $\pm\infty$, or does not exist. (If the limit does not exist, write DNE)

\input{Limit-Compute-0013.HELP.tex}

\[\lim_{h\to0}\frac{-4 \, {\left(h - 1\right)}^{2} + 4}{h}=\answer{8}\]

\end{problem}}%}

\latexProblemContent{
\ifVerboseLocation This is Derivative Compute Question 0013. \\ \fi
\begin{problem}

Determine if the limit approaches a finite number, $\pm\infty$, or does not exist. (If the limit does not exist, write DNE)

\input{Limit-Compute-0013.HELP.tex}

\[\lim_{h\to0}\frac{-3 \, {\left(h + 1\right)}^{2} + 3}{h}=\answer{-6}\]

\end{problem}}%}

\latexProblemContent{
\ifVerboseLocation This is Derivative Compute Question 0013. \\ \fi
\begin{problem}

Determine if the limit approaches a finite number, $\pm\infty$, or does not exist. (If the limit does not exist, write DNE)

\input{Limit-Compute-0013.HELP.tex}

\[\lim_{h\to0}\frac{-3 \, {\left(h + 2\right)}^{3} + 24}{h}=\answer{-36}\]

\end{problem}}%}

\latexProblemContent{
\ifVerboseLocation This is Derivative Compute Question 0013. \\ \fi
\begin{problem}

Determine if the limit approaches a finite number, $\pm\infty$, or does not exist. (If the limit does not exist, write DNE)

\input{Limit-Compute-0013.HELP.tex}

\[\lim_{h\to0}\frac{4 \, {\left(h - 1\right)}^{3} + 4}{h}=\answer{12}\]

\end{problem}}%}

\latexProblemContent{
\ifVerboseLocation This is Derivative Compute Question 0013. \\ \fi
\begin{problem}

Determine if the limit approaches a finite number, $\pm\infty$, or does not exist. (If the limit does not exist, write DNE)

\input{Limit-Compute-0013.HELP.tex}

\[\lim_{h\to0}\frac{-4 \, {\left(h - 2\right)}^{3} - 32}{h}=\answer{-48}\]

\end{problem}}%}

\latexProblemContent{
\ifVerboseLocation This is Derivative Compute Question 0013. \\ \fi
\begin{problem}

Determine if the limit approaches a finite number, $\pm\infty$, or does not exist. (If the limit does not exist, write DNE)

\input{Limit-Compute-0013.HELP.tex}

\[\lim_{h\to0}\frac{-2 \, {\left(h - 1\right)}^{3} - 2}{h}=\answer{-6}\]

\end{problem}}%}

\latexProblemContent{
\ifVerboseLocation This is Derivative Compute Question 0013. \\ \fi
\begin{problem}

Determine if the limit approaches a finite number, $\pm\infty$, or does not exist. (If the limit does not exist, write DNE)

\input{Limit-Compute-0013.HELP.tex}

\[\lim_{h\to0}\frac{\frac{3}{h - 4} + \frac{3}{4}}{h}=\answer{-\frac{3}{16}}\]

\end{problem}}%}

\latexProblemContent{
\ifVerboseLocation This is Derivative Compute Question 0013. \\ \fi
\begin{problem}

Determine if the limit approaches a finite number, $\pm\infty$, or does not exist. (If the limit does not exist, write DNE)

\input{Limit-Compute-0013.HELP.tex}

\[\lim_{h\to0}\frac{-2 \, {\left(h - 3\right)}^{3} - 54}{h}=\answer{-54}\]

\end{problem}}%}

\latexProblemContent{
\ifVerboseLocation This is Derivative Compute Question 0013. \\ \fi
\begin{problem}

Determine if the limit approaches a finite number, $\pm\infty$, or does not exist. (If the limit does not exist, write DNE)

\input{Limit-Compute-0013.HELP.tex}

\[\lim_{h\to0}\frac{2 \, {\left(h + 3\right)}^{3} - 54}{h}=\answer{54}\]

\end{problem}}%}

\latexProblemContent{
\ifVerboseLocation This is Derivative Compute Question 0013. \\ \fi
\begin{problem}

Determine if the limit approaches a finite number, $\pm\infty$, or does not exist. (If the limit does not exist, write DNE)

\input{Limit-Compute-0013.HELP.tex}

\[\lim_{h\to0}\frac{2 \, {\left(h + 1\right)}^{2} - 2}{h}=\answer{4}\]

\end{problem}}%}

\latexProblemContent{
\ifVerboseLocation This is Derivative Compute Question 0013. \\ \fi
\begin{problem}

Determine if the limit approaches a finite number, $\pm\infty$, or does not exist. (If the limit does not exist, write DNE)

\input{Limit-Compute-0013.HELP.tex}

\[\lim_{h\to0}\frac{3 \, {\left(h - 1\right)}^{2} - 3}{h}=\answer{-6}\]

\end{problem}}%}

\latexProblemContent{
\ifVerboseLocation This is Derivative Compute Question 0013. \\ \fi
\begin{problem}

Determine if the limit approaches a finite number, $\pm\infty$, or does not exist. (If the limit does not exist, write DNE)

\input{Limit-Compute-0013.HELP.tex}

\[\lim_{h\to0}\frac{-{\left(h + 1\right)}^{3} + 1}{h}=\answer{-3}\]

\end{problem}}%}

\latexProblemContent{
\ifVerboseLocation This is Derivative Compute Question 0013. \\ \fi
\begin{problem}

Determine if the limit approaches a finite number, $\pm\infty$, or does not exist. (If the limit does not exist, write DNE)

\input{Limit-Compute-0013.HELP.tex}

\[\lim_{h\to0}\frac{4 \, {\left(h + 4\right)}^{2} - 64}{h}=\answer{32}\]

\end{problem}}%}

\latexProblemContent{
\ifVerboseLocation This is Derivative Compute Question 0013. \\ \fi
\begin{problem}

Determine if the limit approaches a finite number, $\pm\infty$, or does not exist. (If the limit does not exist, write DNE)

\input{Limit-Compute-0013.HELP.tex}

\[\lim_{h\to0}\frac{-2 \, {\left(h - 4\right)}^{2} + 32}{h}=\answer{16}\]

\end{problem}}%}

\latexProblemContent{
\ifVerboseLocation This is Derivative Compute Question 0013. \\ \fi
\begin{problem}

Determine if the limit approaches a finite number, $\pm\infty$, or does not exist. (If the limit does not exist, write DNE)

\input{Limit-Compute-0013.HELP.tex}

\[\lim_{h\to0}\frac{-{\left(h - 1\right)}^{2} + 1}{h}=\answer{2}\]

\end{problem}}%}

\latexProblemContent{
\ifVerboseLocation This is Derivative Compute Question 0013. \\ \fi
\begin{problem}

Determine if the limit approaches a finite number, $\pm\infty$, or does not exist. (If the limit does not exist, write DNE)

\input{Limit-Compute-0013.HELP.tex}

\[\lim_{h\to0}\frac{-\frac{3}{h - 3} - 1}{h}=\answer{\frac{1}{3}}\]

\end{problem}}%}

\latexProblemContent{
\ifVerboseLocation This is Derivative Compute Question 0013. \\ \fi
\begin{problem}

Determine if the limit approaches a finite number, $\pm\infty$, or does not exist. (If the limit does not exist, write DNE)

\input{Limit-Compute-0013.HELP.tex}

\[\lim_{h\to0}\frac{\frac{3}{h + 4} - \frac{3}{4}}{h}=\answer{-\frac{3}{16}}\]

\end{problem}}%}

\latexProblemContent{
\ifVerboseLocation This is Derivative Compute Question 0013. \\ \fi
\begin{problem}

Determine if the limit approaches a finite number, $\pm\infty$, or does not exist. (If the limit does not exist, write DNE)

\input{Limit-Compute-0013.HELP.tex}

\[\lim_{h\to0}\frac{{\left(h - 3\right)}^{3} + 27}{h}=\answer{27}\]

\end{problem}}%}

\latexProblemContent{
\ifVerboseLocation This is Derivative Compute Question 0013. \\ \fi
\begin{problem}

Determine if the limit approaches a finite number, $\pm\infty$, or does not exist. (If the limit does not exist, write DNE)

\input{Limit-Compute-0013.HELP.tex}

\[\lim_{h\to0}\frac{-3 \, {\left(h - 3\right)}^{3} - 81}{h}=\answer{-81}\]

\end{problem}}%}

\latexProblemContent{
\ifVerboseLocation This is Derivative Compute Question 0013. \\ \fi
\begin{problem}

Determine if the limit approaches a finite number, $\pm\infty$, or does not exist. (If the limit does not exist, write DNE)

\input{Limit-Compute-0013.HELP.tex}

\[\lim_{h\to0}\frac{\frac{2}{h + 4} - \frac{1}{2}}{h}=\answer{-\frac{1}{8}}\]

\end{problem}}%}

\latexProblemContent{
\ifVerboseLocation This is Derivative Compute Question 0013. \\ \fi
\begin{problem}

Determine if the limit approaches a finite number, $\pm\infty$, or does not exist. (If the limit does not exist, write DNE)

\input{Limit-Compute-0013.HELP.tex}

\[\lim_{h\to0}\frac{-\frac{2}{h - 2} - 1}{h}=\answer{\frac{1}{2}}\]

\end{problem}}%}

\latexProblemContent{
\ifVerboseLocation This is Derivative Compute Question 0013. \\ \fi
\begin{problem}

Determine if the limit approaches a finite number, $\pm\infty$, or does not exist. (If the limit does not exist, write DNE)

\input{Limit-Compute-0013.HELP.tex}

\[\lim_{h\to0}\frac{3 \, \sqrt{h + 4} - 6}{h}=\answer{\frac{3}{4}}\]

\end{problem}}%}

\latexProblemContent{
\ifVerboseLocation This is Derivative Compute Question 0013. \\ \fi
\begin{problem}

Determine if the limit approaches a finite number, $\pm\infty$, or does not exist. (If the limit does not exist, write DNE)

\input{Limit-Compute-0013.HELP.tex}

\[\lim_{h\to0}\frac{2 \, {\left(h + 2\right)}^{3} - 16}{h}=\answer{24}\]

\end{problem}}%}

\latexProblemContent{
\ifVerboseLocation This is Derivative Compute Question 0013. \\ \fi
\begin{problem}

Determine if the limit approaches a finite number, $\pm\infty$, or does not exist. (If the limit does not exist, write DNE)

\input{Limit-Compute-0013.HELP.tex}

\[\lim_{h\to0}\frac{-{\left(h + 3\right)}^{3} + 27}{h}=\answer{-27}\]

\end{problem}}%}

\latexProblemContent{
\ifVerboseLocation This is Derivative Compute Question 0013. \\ \fi
\begin{problem}

Determine if the limit approaches a finite number, $\pm\infty$, or does not exist. (If the limit does not exist, write DNE)

\input{Limit-Compute-0013.HELP.tex}

\[\lim_{h\to0}\frac{5 \, {\left(h + 3\right)}^{3} - 135}{h}=\answer{135}\]

\end{problem}}%}

\latexProblemContent{
\ifVerboseLocation This is Derivative Compute Question 0013. \\ \fi
\begin{problem}

Determine if the limit approaches a finite number, $\pm\infty$, or does not exist. (If the limit does not exist, write DNE)

\input{Limit-Compute-0013.HELP.tex}

\[\lim_{h\to0}\frac{-4 \, {\left(h - 3\right)}^{2} + 36}{h}=\answer{24}\]

\end{problem}}%}

\latexProblemContent{
\ifVerboseLocation This is Derivative Compute Question 0013. \\ \fi
\begin{problem}

Determine if the limit approaches a finite number, $\pm\infty$, or does not exist. (If the limit does not exist, write DNE)

\input{Limit-Compute-0013.HELP.tex}

\[\lim_{h\to0}\frac{\frac{5}{h + 2} - \frac{5}{2}}{h}=\answer{-\frac{5}{4}}\]

\end{problem}}%}

\latexProblemContent{
\ifVerboseLocation This is Derivative Compute Question 0013. \\ \fi
\begin{problem}

Determine if the limit approaches a finite number, $\pm\infty$, or does not exist. (If the limit does not exist, write DNE)

\input{Limit-Compute-0013.HELP.tex}

\[\lim_{h\to0}\frac{2 \, {\left(h - 1\right)}^{3} + 2}{h}=\answer{6}\]

\end{problem}}%}

\latexProblemContent{
\ifVerboseLocation This is Derivative Compute Question 0013. \\ \fi
\begin{problem}

Determine if the limit approaches a finite number, $\pm\infty$, or does not exist. (If the limit does not exist, write DNE)

\input{Limit-Compute-0013.HELP.tex}

\[\lim_{h\to0}\frac{-\frac{2}{h - 4} - \frac{1}{2}}{h}=\answer{\frac{1}{8}}\]

\end{problem}}%}

\latexProblemContent{
\ifVerboseLocation This is Derivative Compute Question 0013. \\ \fi
\begin{problem}

Determine if the limit approaches a finite number, $\pm\infty$, or does not exist. (If the limit does not exist, write DNE)

\input{Limit-Compute-0013.HELP.tex}

\[\lim_{h\to0}\frac{{\left(h + 4\right)}^{2} - 16}{h}=\answer{8}\]

\end{problem}}%}

\latexProblemContent{
\ifVerboseLocation This is Derivative Compute Question 0013. \\ \fi
\begin{problem}

Determine if the limit approaches a finite number, $\pm\infty$, or does not exist. (If the limit does not exist, write DNE)

\input{Limit-Compute-0013.HELP.tex}

\[\lim_{h\to0}\frac{{\left(h - 1\right)}^{3} + 1}{h}=\answer{3}\]

\end{problem}}%}

\latexProblemContent{
\ifVerboseLocation This is Derivative Compute Question 0013. \\ \fi
\begin{problem}

Determine if the limit approaches a finite number, $\pm\infty$, or does not exist. (If the limit does not exist, write DNE)

\input{Limit-Compute-0013.HELP.tex}

\[\lim_{h\to0}\frac{-4 \, \sqrt{h + 4} + 8}{h}=\answer{-1}\]

\end{problem}}%}

\latexProblemContent{
\ifVerboseLocation This is Derivative Compute Question 0013. \\ \fi
\begin{problem}

Determine if the limit approaches a finite number, $\pm\infty$, or does not exist. (If the limit does not exist, write DNE)

\input{Limit-Compute-0013.HELP.tex}

\[\lim_{h\to0}\frac{-{\left(h - 4\right)}^{3} - 64}{h}=\answer{-48}\]

\end{problem}}%}

\latexProblemContent{
\ifVerboseLocation This is Derivative Compute Question 0013. \\ \fi
\begin{problem}

Determine if the limit approaches a finite number, $\pm\infty$, or does not exist. (If the limit does not exist, write DNE)

\input{Limit-Compute-0013.HELP.tex}

\[\lim_{h\to0}\frac{-\frac{3}{h - 4} - \frac{3}{4}}{h}=\answer{\frac{3}{16}}\]

\end{problem}}%}

\latexProblemContent{
\ifVerboseLocation This is Derivative Compute Question 0013. \\ \fi
\begin{problem}

Determine if the limit approaches a finite number, $\pm\infty$, or does not exist. (If the limit does not exist, write DNE)

\input{Limit-Compute-0013.HELP.tex}

\[\lim_{h\to0}\frac{\frac{5}{h - 1} + 5}{h}=\answer{-5}\]

\end{problem}}%}

\latexProblemContent{
\ifVerboseLocation This is Derivative Compute Question 0013. \\ \fi
\begin{problem}

Determine if the limit approaches a finite number, $\pm\infty$, or does not exist. (If the limit does not exist, write DNE)

\input{Limit-Compute-0013.HELP.tex}

\[\lim_{h\to0}\frac{-5 \, {\left(h + 3\right)}^{2} + 45}{h}=\answer{-30}\]

\end{problem}}%}

\latexProblemContent{
\ifVerboseLocation This is Derivative Compute Question 0013. \\ \fi
\begin{problem}

Determine if the limit approaches a finite number, $\pm\infty$, or does not exist. (If the limit does not exist, write DNE)

\input{Limit-Compute-0013.HELP.tex}

\[\lim_{h\to0}\frac{-\frac{2}{h + 1} + 2}{h}=\answer{2}\]

\end{problem}}%}

\latexProblemContent{
\ifVerboseLocation This is Derivative Compute Question 0013. \\ \fi
\begin{problem}

Determine if the limit approaches a finite number, $\pm\infty$, or does not exist. (If the limit does not exist, write DNE)

\input{Limit-Compute-0013.HELP.tex}

\[\lim_{h\to0}\frac{-3 \, {\left(h + 4\right)}^{3} + 192}{h}=\answer{-144}\]

\end{problem}}%}

\latexProblemContent{
\ifVerboseLocation This is Derivative Compute Question 0013. \\ \fi
\begin{problem}

Determine if the limit approaches a finite number, $\pm\infty$, or does not exist. (If the limit does not exist, write DNE)

\input{Limit-Compute-0013.HELP.tex}

\[\lim_{h\to0}\frac{-4 \, {\left(h - 1\right)}^{3} - 4}{h}=\answer{-12}\]

\end{problem}}%}

\latexProblemContent{
\ifVerboseLocation This is Derivative Compute Question 0013. \\ \fi
\begin{problem}

Determine if the limit approaches a finite number, $\pm\infty$, or does not exist. (If the limit does not exist, write DNE)

\input{Limit-Compute-0013.HELP.tex}

\[\lim_{h\to0}\frac{-{\left(h + 4\right)}^{3} + 64}{h}=\answer{-48}\]

\end{problem}}%}

\latexProblemContent{
\ifVerboseLocation This is Derivative Compute Question 0013. \\ \fi
\begin{problem}

Determine if the limit approaches a finite number, $\pm\infty$, or does not exist. (If the limit does not exist, write DNE)

\input{Limit-Compute-0013.HELP.tex}

\[\lim_{h\to0}\frac{-2 \, {\left(h + 2\right)}^{2} + 8}{h}=\answer{-8}\]

\end{problem}}%}

\latexProblemContent{
\ifVerboseLocation This is Derivative Compute Question 0013. \\ \fi
\begin{problem}

Determine if the limit approaches a finite number, $\pm\infty$, or does not exist. (If the limit does not exist, write DNE)

\input{Limit-Compute-0013.HELP.tex}

\[\lim_{h\to0}\frac{\frac{5}{h - 4} + \frac{5}{4}}{h}=\answer{-\frac{5}{16}}\]

\end{problem}}%}

\latexProblemContent{
\ifVerboseLocation This is Derivative Compute Question 0013. \\ \fi
\begin{problem}

Determine if the limit approaches a finite number, $\pm\infty$, or does not exist. (If the limit does not exist, write DNE)

\input{Limit-Compute-0013.HELP.tex}

\[\lim_{h\to0}\frac{-{\left(h - 3\right)}^{3} - 27}{h}=\answer{-27}\]

\end{problem}}%}

\latexProblemContent{
\ifVerboseLocation This is Derivative Compute Question 0013. \\ \fi
\begin{problem}

Determine if the limit approaches a finite number, $\pm\infty$, or does not exist. (If the limit does not exist, write DNE)

\input{Limit-Compute-0013.HELP.tex}

\[\lim_{h\to0}\frac{-\frac{1}{h - 3} - \frac{1}{3}}{h}=\answer{\frac{1}{9}}\]

\end{problem}}%}

\latexProblemContent{
\ifVerboseLocation This is Derivative Compute Question 0013. \\ \fi
\begin{problem}

Determine if the limit approaches a finite number, $\pm\infty$, or does not exist. (If the limit does not exist, write DNE)

\input{Limit-Compute-0013.HELP.tex}

\[\lim_{h\to0}\frac{-\frac{3}{h + 3} + 1}{h}=\answer{\frac{1}{3}}\]

\end{problem}}%}

\latexProblemContent{
\ifVerboseLocation This is Derivative Compute Question 0013. \\ \fi
\begin{problem}

Determine if the limit approaches a finite number, $\pm\infty$, or does not exist. (If the limit does not exist, write DNE)

\input{Limit-Compute-0013.HELP.tex}

\[\lim_{h\to0}\frac{-\sqrt{h + 4} + 2}{h}=\answer{-\frac{1}{4}}\]

\end{problem}}%}

\latexProblemContent{
\ifVerboseLocation This is Derivative Compute Question 0013. \\ \fi
\begin{problem}

Determine if the limit approaches a finite number, $\pm\infty$, or does not exist. (If the limit does not exist, write DNE)

\input{Limit-Compute-0013.HELP.tex}

\[\lim_{h\to0}\frac{-{\left(h + 4\right)}^{2} + 16}{h}=\answer{-8}\]

\end{problem}}%}

\latexProblemContent{
\ifVerboseLocation This is Derivative Compute Question 0013. \\ \fi
\begin{problem}

Determine if the limit approaches a finite number, $\pm\infty$, or does not exist. (If the limit does not exist, write DNE)

\input{Limit-Compute-0013.HELP.tex}

\[\lim_{h\to0}\frac{-\frac{4}{h + 4} + 1}{h}=\answer{\frac{1}{4}}\]

\end{problem}}%}

\latexProblemContent{
\ifVerboseLocation This is Derivative Compute Question 0013. \\ \fi
\begin{problem}

Determine if the limit approaches a finite number, $\pm\infty$, or does not exist. (If the limit does not exist, write DNE)

\input{Limit-Compute-0013.HELP.tex}

\[\lim_{h\to0}\frac{-\frac{2}{h + 3} + \frac{2}{3}}{h}=\answer{\frac{2}{9}}\]

\end{problem}}%}

\latexProblemContent{
\ifVerboseLocation This is Derivative Compute Question 0013. \\ \fi
\begin{problem}

Determine if the limit approaches a finite number, $\pm\infty$, or does not exist. (If the limit does not exist, write DNE)

\input{Limit-Compute-0013.HELP.tex}

\[\lim_{h\to0}\frac{-4 \, \sqrt{2} + 4 \, \sqrt{h + 2}}{h}=\answer{\sqrt{2}}\]

\end{problem}}%}

\latexProblemContent{
\ifVerboseLocation This is Derivative Compute Question 0013. \\ \fi
\begin{problem}

Determine if the limit approaches a finite number, $\pm\infty$, or does not exist. (If the limit does not exist, write DNE)

\input{Limit-Compute-0013.HELP.tex}

\[\lim_{h\to0}\frac{{\left(h - 2\right)}^{3} + 8}{h}=\answer{12}\]

\end{problem}}%}

\latexProblemContent{
\ifVerboseLocation This is Derivative Compute Question 0013. \\ \fi
\begin{problem}

Determine if the limit approaches a finite number, $\pm\infty$, or does not exist. (If the limit does not exist, write DNE)

\input{Limit-Compute-0013.HELP.tex}

\[\lim_{h\to0}\frac{-4 \, {\left(h + 3\right)}^{2} + 36}{h}=\answer{-24}\]

\end{problem}}%}

\latexProblemContent{
\ifVerboseLocation This is Derivative Compute Question 0013. \\ \fi
\begin{problem}

Determine if the limit approaches a finite number, $\pm\infty$, or does not exist. (If the limit does not exist, write DNE)

\input{Limit-Compute-0013.HELP.tex}

\[\lim_{h\to0}\frac{\frac{3}{h - 2} + \frac{3}{2}}{h}=\answer{-\frac{3}{4}}\]

\end{problem}}%}

\latexProblemContent{
\ifVerboseLocation This is Derivative Compute Question 0013. \\ \fi
\begin{problem}

Determine if the limit approaches a finite number, $\pm\infty$, or does not exist. (If the limit does not exist, write DNE)

\input{Limit-Compute-0013.HELP.tex}

\[\lim_{h\to0}\frac{-\frac{5}{h + 2} + \frac{5}{2}}{h}=\answer{\frac{5}{4}}\]

\end{problem}}%}

\latexProblemContent{
\ifVerboseLocation This is Derivative Compute Question 0013. \\ \fi
\begin{problem}

Determine if the limit approaches a finite number, $\pm\infty$, or does not exist. (If the limit does not exist, write DNE)

\input{Limit-Compute-0013.HELP.tex}

\[\lim_{h\to0}\frac{-2 \, \sqrt{2} + 2 \, \sqrt{h + 2}}{h}=\answer{\frac{1}{2} \, \sqrt{2}}\]

\end{problem}}%}

\latexProblemContent{
\ifVerboseLocation This is Derivative Compute Question 0013. \\ \fi
\begin{problem}

Determine if the limit approaches a finite number, $\pm\infty$, or does not exist. (If the limit does not exist, write DNE)

\input{Limit-Compute-0013.HELP.tex}

\[\lim_{h\to0}\frac{-3 \, {\left(h - 2\right)}^{2} + 12}{h}=\answer{12}\]

\end{problem}}%}

\latexProblemContent{
\ifVerboseLocation This is Derivative Compute Question 0013. \\ \fi
\begin{problem}

Determine if the limit approaches a finite number, $\pm\infty$, or does not exist. (If the limit does not exist, write DNE)

\input{Limit-Compute-0013.HELP.tex}

\[\lim_{h\to0}\frac{\frac{2}{h - 3} + \frac{2}{3}}{h}=\answer{-\frac{2}{9}}\]

\end{problem}}%}

\latexProblemContent{
\ifVerboseLocation This is Derivative Compute Question 0013. \\ \fi
\begin{problem}

Determine if the limit approaches a finite number, $\pm\infty$, or does not exist. (If the limit does not exist, write DNE)

\input{Limit-Compute-0013.HELP.tex}

\[\lim_{h\to0}\frac{-4 \, \sqrt{h + 1} + 4}{h}=\answer{-2}\]

\end{problem}}%}

\latexProblemContent{
\ifVerboseLocation This is Derivative Compute Question 0013. \\ \fi
\begin{problem}

Determine if the limit approaches a finite number, $\pm\infty$, or does not exist. (If the limit does not exist, write DNE)

\input{Limit-Compute-0013.HELP.tex}

\[\lim_{h\to0}\frac{2 \, {\left(h + 1\right)}^{3} - 2}{h}=\answer{6}\]

\end{problem}}%}

\latexProblemContent{
\ifVerboseLocation This is Derivative Compute Question 0013. \\ \fi
\begin{problem}

Determine if the limit approaches a finite number, $\pm\infty$, or does not exist. (If the limit does not exist, write DNE)

\input{Limit-Compute-0013.HELP.tex}

\[\lim_{h\to0}\frac{-{\left(h - 2\right)}^{3} - 8}{h}=\answer{-12}\]

\end{problem}}%}

\latexProblemContent{
\ifVerboseLocation This is Derivative Compute Question 0013. \\ \fi
\begin{problem}

Determine if the limit approaches a finite number, $\pm\infty$, or does not exist. (If the limit does not exist, write DNE)

\input{Limit-Compute-0013.HELP.tex}

\[\lim_{h\to0}\frac{\frac{4}{h - 4} + 1}{h}=\answer{-\frac{1}{4}}\]

\end{problem}}%}

\latexProblemContent{
\ifVerboseLocation This is Derivative Compute Question 0013. \\ \fi
\begin{problem}

Determine if the limit approaches a finite number, $\pm\infty$, or does not exist. (If the limit does not exist, write DNE)

\input{Limit-Compute-0013.HELP.tex}

\[\lim_{h\to0}\frac{{\left(h + 3\right)}^{3} - 27}{h}=\answer{27}\]

\end{problem}}%}

\latexProblemContent{
\ifVerboseLocation This is Derivative Compute Question 0013. \\ \fi
\begin{problem}

Determine if the limit approaches a finite number, $\pm\infty$, or does not exist. (If the limit does not exist, write DNE)

\input{Limit-Compute-0013.HELP.tex}

\[\lim_{h\to0}\frac{-5 \, {\left(h + 2\right)}^{3} + 40}{h}=\answer{-60}\]

\end{problem}}%}

\latexProblemContent{
\ifVerboseLocation This is Derivative Compute Question 0013. \\ \fi
\begin{problem}

Determine if the limit approaches a finite number, $\pm\infty$, or does not exist. (If the limit does not exist, write DNE)

\input{Limit-Compute-0013.HELP.tex}

\[\lim_{h\to0}\frac{\sqrt{3} - \sqrt{h + 3}}{h}=\answer{-\frac{1}{6} \, \sqrt{3}}\]

\end{problem}}%}

\latexProblemContent{
\ifVerboseLocation This is Derivative Compute Question 0013. \\ \fi
\begin{problem}

Determine if the limit approaches a finite number, $\pm\infty$, or does not exist. (If the limit does not exist, write DNE)

\input{Limit-Compute-0013.HELP.tex}

\[\lim_{h\to0}\frac{2 \, {\left(h - 2\right)}^{2} - 8}{h}=\answer{-8}\]

\end{problem}}%}

\latexProblemContent{
\ifVerboseLocation This is Derivative Compute Question 0013. \\ \fi
\begin{problem}

Determine if the limit approaches a finite number, $\pm\infty$, or does not exist. (If the limit does not exist, write DNE)

\input{Limit-Compute-0013.HELP.tex}

\[\lim_{h\to0}\frac{4 \, {\left(h + 4\right)}^{3} - 256}{h}=\answer{192}\]

\end{problem}}%}

\latexProblemContent{
\ifVerboseLocation This is Derivative Compute Question 0013. \\ \fi
\begin{problem}

Determine if the limit approaches a finite number, $\pm\infty$, or does not exist. (If the limit does not exist, write DNE)

\input{Limit-Compute-0013.HELP.tex}

\[\lim_{h\to0}\frac{-{\left(h + 3\right)}^{2} + 9}{h}=\answer{-6}\]

\end{problem}}%}

\latexProblemContent{
\ifVerboseLocation This is Derivative Compute Question 0013. \\ \fi
\begin{problem}

Determine if the limit approaches a finite number, $\pm\infty$, or does not exist. (If the limit does not exist, write DNE)

\input{Limit-Compute-0013.HELP.tex}

\[\lim_{h\to0}\frac{2 \, {\left(h - 4\right)}^{2} - 32}{h}=\answer{-16}\]

\end{problem}}%}

\latexProblemContent{
\ifVerboseLocation This is Derivative Compute Question 0013. \\ \fi
\begin{problem}

Determine if the limit approaches a finite number, $\pm\infty$, or does not exist. (If the limit does not exist, write DNE)

\input{Limit-Compute-0013.HELP.tex}

\[\lim_{h\to0}\frac{4 \, {\left(h + 1\right)}^{2} - 4}{h}=\answer{8}\]

\end{problem}}%}

\latexProblemContent{
\ifVerboseLocation This is Derivative Compute Question 0013. \\ \fi
\begin{problem}

Determine if the limit approaches a finite number, $\pm\infty$, or does not exist. (If the limit does not exist, write DNE)

\input{Limit-Compute-0013.HELP.tex}

\[\lim_{h\to0}\frac{3 \, \sqrt{2} - 3 \, \sqrt{h + 2}}{h}=\answer{-\frac{3}{4} \, \sqrt{2}}\]

\end{problem}}%}

\latexProblemContent{
\ifVerboseLocation This is Derivative Compute Question 0013. \\ \fi
\begin{problem}

Determine if the limit approaches a finite number, $\pm\infty$, or does not exist. (If the limit does not exist, write DNE)

\input{Limit-Compute-0013.HELP.tex}

\[\lim_{h\to0}\frac{5 \, {\left(h + 4\right)}^{2} - 80}{h}=\answer{40}\]

\end{problem}}%}

\latexProblemContent{
\ifVerboseLocation This is Derivative Compute Question 0013. \\ \fi
\begin{problem}

Determine if the limit approaches a finite number, $\pm\infty$, or does not exist. (If the limit does not exist, write DNE)

\input{Limit-Compute-0013.HELP.tex}

\[\lim_{h\to0}\frac{-\frac{2}{h - 1} - 2}{h}=\answer{2}\]

\end{problem}}%}

\latexProblemContent{
\ifVerboseLocation This is Derivative Compute Question 0013. \\ \fi
\begin{problem}

Determine if the limit approaches a finite number, $\pm\infty$, or does not exist. (If the limit does not exist, write DNE)

\input{Limit-Compute-0013.HELP.tex}

\[\lim_{h\to0}\frac{2 \, \sqrt{h + 1} - 2}{h}=\answer{1}\]

\end{problem}}%}

\latexProblemContent{
\ifVerboseLocation This is Derivative Compute Question 0013. \\ \fi
\begin{problem}

Determine if the limit approaches a finite number, $\pm\infty$, or does not exist. (If the limit does not exist, write DNE)

\input{Limit-Compute-0013.HELP.tex}

\[\lim_{h\to0}\frac{3 \, {\left(h + 4\right)}^{3} - 192}{h}=\answer{144}\]

\end{problem}}%}

\latexProblemContent{
\ifVerboseLocation This is Derivative Compute Question 0013. \\ \fi
\begin{problem}

Determine if the limit approaches a finite number, $\pm\infty$, or does not exist. (If the limit does not exist, write DNE)

\input{Limit-Compute-0013.HELP.tex}

\[\lim_{h\to0}\frac{3 \, {\left(h - 4\right)}^{2} - 48}{h}=\answer{-24}\]

\end{problem}}%}

\latexProblemContent{
\ifVerboseLocation This is Derivative Compute Question 0013. \\ \fi
\begin{problem}

Determine if the limit approaches a finite number, $\pm\infty$, or does not exist. (If the limit does not exist, write DNE)

\input{Limit-Compute-0013.HELP.tex}

\[\lim_{h\to0}\frac{-5 \, {\left(h + 4\right)}^{3} + 320}{h}=\answer{-240}\]

\end{problem}}%}

\latexProblemContent{
\ifVerboseLocation This is Derivative Compute Question 0013. \\ \fi
\begin{problem}

Determine if the limit approaches a finite number, $\pm\infty$, or does not exist. (If the limit does not exist, write DNE)

\input{Limit-Compute-0013.HELP.tex}

\[\lim_{h\to0}\frac{4 \, {\left(h - 1\right)}^{2} - 4}{h}=\answer{-8}\]

\end{problem}}%}

\latexProblemContent{
\ifVerboseLocation This is Derivative Compute Question 0013. \\ \fi
\begin{problem}

Determine if the limit approaches a finite number, $\pm\infty$, or does not exist. (If the limit does not exist, write DNE)

\input{Limit-Compute-0013.HELP.tex}

\[\lim_{h\to0}\frac{3 \, {\left(h + 1\right)}^{2} - 3}{h}=\answer{6}\]

\end{problem}}%}

\latexProblemContent{
\ifVerboseLocation This is Derivative Compute Question 0013. \\ \fi
\begin{problem}

Determine if the limit approaches a finite number, $\pm\infty$, or does not exist. (If the limit does not exist, write DNE)

\input{Limit-Compute-0013.HELP.tex}

\[\lim_{h\to0}\frac{{\left(h - 1\right)}^{2} - 1}{h}=\answer{-2}\]

\end{problem}}%}

\latexProblemContent{
\ifVerboseLocation This is Derivative Compute Question 0013. \\ \fi
\begin{problem}

Determine if the limit approaches a finite number, $\pm\infty$, or does not exist. (If the limit does not exist, write DNE)

\input{Limit-Compute-0013.HELP.tex}

\[\lim_{h\to0}\frac{-{\left(h - 3\right)}^{2} + 9}{h}=\answer{6}\]

\end{problem}}%}

\latexProblemContent{
\ifVerboseLocation This is Derivative Compute Question 0013. \\ \fi
\begin{problem}

Determine if the limit approaches a finite number, $\pm\infty$, or does not exist. (If the limit does not exist, write DNE)

\input{Limit-Compute-0013.HELP.tex}

\[\lim_{h\to0}\frac{4 \, {\left(h + 1\right)}^{3} - 4}{h}=\answer{12}\]

\end{problem}}%}

\latexProblemContent{
\ifVerboseLocation This is Derivative Compute Question 0013. \\ \fi
\begin{problem}

Determine if the limit approaches a finite number, $\pm\infty$, or does not exist. (If the limit does not exist, write DNE)

\input{Limit-Compute-0013.HELP.tex}

\[\lim_{h\to0}\frac{-2 \, {\left(h - 4\right)}^{3} - 128}{h}=\answer{-96}\]

\end{problem}}%}

\latexProblemContent{
\ifVerboseLocation This is Derivative Compute Question 0013. \\ \fi
\begin{problem}

Determine if the limit approaches a finite number, $\pm\infty$, or does not exist. (If the limit does not exist, write DNE)

\input{Limit-Compute-0013.HELP.tex}

\[\lim_{h\to0}\frac{3 \, {\left(h + 3\right)}^{2} - 27}{h}=\answer{18}\]

\end{problem}}%}

\latexProblemContent{
\ifVerboseLocation This is Derivative Compute Question 0013. \\ \fi
\begin{problem}

Determine if the limit approaches a finite number, $\pm\infty$, or does not exist. (If the limit does not exist, write DNE)

\input{Limit-Compute-0013.HELP.tex}

\[\lim_{h\to0}\frac{-5 \, {\left(h + 1\right)}^{2} + 5}{h}=\answer{-10}\]

\end{problem}}%}

\latexProblemContent{
\ifVerboseLocation This is Derivative Compute Question 0013. \\ \fi
\begin{problem}

Determine if the limit approaches a finite number, $\pm\infty$, or does not exist. (If the limit does not exist, write DNE)

\input{Limit-Compute-0013.HELP.tex}

\[\lim_{h\to0}\frac{-\sqrt{3} + \sqrt{h + 3}}{h}=\answer{\frac{1}{6} \, \sqrt{3}}\]

\end{problem}}%}

\latexProblemContent{
\ifVerboseLocation This is Derivative Compute Question 0013. \\ \fi
\begin{problem}

Determine if the limit approaches a finite number, $\pm\infty$, or does not exist. (If the limit does not exist, write DNE)

\input{Limit-Compute-0013.HELP.tex}

\[\lim_{h\to0}\frac{2 \, {\left(h + 4\right)}^{3} - 128}{h}=\answer{96}\]

\end{problem}}%}

\latexProblemContent{
\ifVerboseLocation This is Derivative Compute Question 0013. \\ \fi
\begin{problem}

Determine if the limit approaches a finite number, $\pm\infty$, or does not exist. (If the limit does not exist, write DNE)

\input{Limit-Compute-0013.HELP.tex}

\[\lim_{h\to0}\frac{\frac{1}{h + 3} - \frac{1}{3}}{h}=\answer{-\frac{1}{9}}\]

\end{problem}}%}

\latexProblemContent{
\ifVerboseLocation This is Derivative Compute Question 0013. \\ \fi
\begin{problem}

Determine if the limit approaches a finite number, $\pm\infty$, or does not exist. (If the limit does not exist, write DNE)

\input{Limit-Compute-0013.HELP.tex}

\[\lim_{h\to0}\frac{\frac{1}{h - 2} + \frac{1}{2}}{h}=\answer{-\frac{1}{4}}\]

\end{problem}}%}

\latexProblemContent{
\ifVerboseLocation This is Derivative Compute Question 0013. \\ \fi
\begin{problem}

Determine if the limit approaches a finite number, $\pm\infty$, or does not exist. (If the limit does not exist, write DNE)

\input{Limit-Compute-0013.HELP.tex}

\[\lim_{h\to0}\frac{\frac{1}{h - 3} + \frac{1}{3}}{h}=\answer{-\frac{1}{9}}\]

\end{problem}}%}

\latexProblemContent{
\ifVerboseLocation This is Derivative Compute Question 0013. \\ \fi
\begin{problem}

Determine if the limit approaches a finite number, $\pm\infty$, or does not exist. (If the limit does not exist, write DNE)

\input{Limit-Compute-0013.HELP.tex}

\[\lim_{h\to0}\frac{-2 \, {\left(h + 1\right)}^{2} + 2}{h}=\answer{-4}\]

\end{problem}}%}

\latexProblemContent{
\ifVerboseLocation This is Derivative Compute Question 0013. \\ \fi
\begin{problem}

Determine if the limit approaches a finite number, $\pm\infty$, or does not exist. (If the limit does not exist, write DNE)

\input{Limit-Compute-0013.HELP.tex}

\[\lim_{h\to0}\frac{-2 \, {\left(h + 1\right)}^{3} + 2}{h}=\answer{-6}\]

\end{problem}}%}

\latexProblemContent{
\ifVerboseLocation This is Derivative Compute Question 0013. \\ \fi
\begin{problem}

Determine if the limit approaches a finite number, $\pm\infty$, or does not exist. (If the limit does not exist, write DNE)

\input{Limit-Compute-0013.HELP.tex}

\[\lim_{h\to0}\frac{3 \, {\left(h + 4\right)}^{2} - 48}{h}=\answer{24}\]

\end{problem}}%}

\latexProblemContent{
\ifVerboseLocation This is Derivative Compute Question 0013. \\ \fi
\begin{problem}

Determine if the limit approaches a finite number, $\pm\infty$, or does not exist. (If the limit does not exist, write DNE)

\input{Limit-Compute-0013.HELP.tex}

\[\lim_{h\to0}\frac{{\left(h + 3\right)}^{2} - 9}{h}=\answer{6}\]

\end{problem}}%}

\latexProblemContent{
\ifVerboseLocation This is Derivative Compute Question 0013. \\ \fi
\begin{problem}

Determine if the limit approaches a finite number, $\pm\infty$, or does not exist. (If the limit does not exist, write DNE)

\input{Limit-Compute-0013.HELP.tex}

\[\lim_{h\to0}\frac{5 \, {\left(h - 3\right)}^{3} + 135}{h}=\answer{135}\]

\end{problem}}%}

\latexProblemContent{
\ifVerboseLocation This is Derivative Compute Question 0013. \\ \fi
\begin{problem}

Determine if the limit approaches a finite number, $\pm\infty$, or does not exist. (If the limit does not exist, write DNE)

\input{Limit-Compute-0013.HELP.tex}

\[\lim_{h\to0}\frac{-5 \, {\left(h - 1\right)}^{3} - 5}{h}=\answer{-15}\]

\end{problem}}%}

\latexProblemContent{
\ifVerboseLocation This is Derivative Compute Question 0013. \\ \fi
\begin{problem}

Determine if the limit approaches a finite number, $\pm\infty$, or does not exist. (If the limit does not exist, write DNE)

\input{Limit-Compute-0013.HELP.tex}

\[\lim_{h\to0}\frac{-5 \, {\left(h - 4\right)}^{3} - 320}{h}=\answer{-240}\]

\end{problem}}%}

\latexProblemContent{
\ifVerboseLocation This is Derivative Compute Question 0013. \\ \fi
\begin{problem}

Determine if the limit approaches a finite number, $\pm\infty$, or does not exist. (If the limit does not exist, write DNE)

\input{Limit-Compute-0013.HELP.tex}

\[\lim_{h\to0}\frac{{\left(h + 1\right)}^{2} - 1}{h}=\answer{2}\]

\end{problem}}%}

\latexProblemContent{
\ifVerboseLocation This is Derivative Compute Question 0013. \\ \fi
\begin{problem}

Determine if the limit approaches a finite number, $\pm\infty$, or does not exist. (If the limit does not exist, write DNE)

\input{Limit-Compute-0013.HELP.tex}

\[\lim_{h\to0}\frac{5 \, {\left(h + 4\right)}^{3} - 320}{h}=\answer{240}\]

\end{problem}}%}

\latexProblemContent{
\ifVerboseLocation This is Derivative Compute Question 0013. \\ \fi
\begin{problem}

Determine if the limit approaches a finite number, $\pm\infty$, or does not exist. (If the limit does not exist, write DNE)

\input{Limit-Compute-0013.HELP.tex}

\[\lim_{h\to0}\frac{-4 \, {\left(h - 3\right)}^{3} - 108}{h}=\answer{-108}\]

\end{problem}}%}

\latexProblemContent{
\ifVerboseLocation This is Derivative Compute Question 0013. \\ \fi
\begin{problem}

Determine if the limit approaches a finite number, $\pm\infty$, or does not exist. (If the limit does not exist, write DNE)

\input{Limit-Compute-0013.HELP.tex}

\[\lim_{h\to0}\frac{-\frac{5}{h - 3} - \frac{5}{3}}{h}=\answer{\frac{5}{9}}\]

\end{problem}}%}

\latexProblemContent{
\ifVerboseLocation This is Derivative Compute Question 0013. \\ \fi
\begin{problem}

Determine if the limit approaches a finite number, $\pm\infty$, or does not exist. (If the limit does not exist, write DNE)

\input{Limit-Compute-0013.HELP.tex}

\[\lim_{h\to0}\frac{{\left(h - 2\right)}^{2} - 4}{h}=\answer{-4}\]

\end{problem}}%}

\latexProblemContent{
\ifVerboseLocation This is Derivative Compute Question 0013. \\ \fi
\begin{problem}

Determine if the limit approaches a finite number, $\pm\infty$, or does not exist. (If the limit does not exist, write DNE)

\input{Limit-Compute-0013.HELP.tex}

\[\lim_{h\to0}\frac{2 \, {\left(h - 3\right)}^{2} - 18}{h}=\answer{-12}\]

\end{problem}}%}

\latexProblemContent{
\ifVerboseLocation This is Derivative Compute Question 0013. \\ \fi
\begin{problem}

Determine if the limit approaches a finite number, $\pm\infty$, or does not exist. (If the limit does not exist, write DNE)

\input{Limit-Compute-0013.HELP.tex}

\[\lim_{h\to0}\frac{-2 \, \sqrt{3} + 2 \, \sqrt{h + 3}}{h}=\answer{\frac{1}{3} \, \sqrt{3}}\]

\end{problem}}%}

\latexProblemContent{
\ifVerboseLocation This is Derivative Compute Question 0013. \\ \fi
\begin{problem}

Determine if the limit approaches a finite number, $\pm\infty$, or does not exist. (If the limit does not exist, write DNE)

\input{Limit-Compute-0013.HELP.tex}

\[\lim_{h\to0}\frac{-5 \, \sqrt{3} + 5 \, \sqrt{h + 3}}{h}=\answer{\frac{5}{6} \, \sqrt{3}}\]

\end{problem}}%}

\latexProblemContent{
\ifVerboseLocation This is Derivative Compute Question 0013. \\ \fi
\begin{problem}

Determine if the limit approaches a finite number, $\pm\infty$, or does not exist. (If the limit does not exist, write DNE)

\input{Limit-Compute-0013.HELP.tex}

\[\lim_{h\to0}\frac{-\sqrt{2} + \sqrt{h + 2}}{h}=\answer{\frac{1}{4} \, \sqrt{2}}\]

\end{problem}}%}

\latexProblemContent{
\ifVerboseLocation This is Derivative Compute Question 0013. \\ \fi
\begin{problem}

Determine if the limit approaches a finite number, $\pm\infty$, or does not exist. (If the limit does not exist, write DNE)

\input{Limit-Compute-0013.HELP.tex}

\[\lim_{h\to0}\frac{-3 \, {\left(h - 1\right)}^{2} + 3}{h}=\answer{6}\]

\end{problem}}%}

\latexProblemContent{
\ifVerboseLocation This is Derivative Compute Question 0013. \\ \fi
\begin{problem}

Determine if the limit approaches a finite number, $\pm\infty$, or does not exist. (If the limit does not exist, write DNE)

\input{Limit-Compute-0013.HELP.tex}

\[\lim_{h\to0}\frac{\frac{4}{h - 1} + 4}{h}=\answer{-4}\]

\end{problem}}%}

\latexProblemContent{
\ifVerboseLocation This is Derivative Compute Question 0013. \\ \fi
\begin{problem}

Determine if the limit approaches a finite number, $\pm\infty$, or does not exist. (If the limit does not exist, write DNE)

\input{Limit-Compute-0013.HELP.tex}

\[\lim_{h\to0}\frac{5 \, {\left(h + 1\right)}^{3} - 5}{h}=\answer{15}\]

\end{problem}}%}

\latexProblemContent{
\ifVerboseLocation This is Derivative Compute Question 0013. \\ \fi
\begin{problem}

Determine if the limit approaches a finite number, $\pm\infty$, or does not exist. (If the limit does not exist, write DNE)

\input{Limit-Compute-0013.HELP.tex}

\[\lim_{h\to0}\frac{-5 \, {\left(h - 2\right)}^{2} + 20}{h}=\answer{20}\]

\end{problem}}%}

\latexProblemContent{
\ifVerboseLocation This is Derivative Compute Question 0013. \\ \fi
\begin{problem}

Determine if the limit approaches a finite number, $\pm\infty$, or does not exist. (If the limit does not exist, write DNE)

\input{Limit-Compute-0013.HELP.tex}

\[\lim_{h\to0}\frac{\frac{1}{h + 4} - \frac{1}{4}}{h}=\answer{-\frac{1}{16}}\]

\end{problem}}%}

\latexProblemContent{
\ifVerboseLocation This is Derivative Compute Question 0013. \\ \fi
\begin{problem}

Determine if the limit approaches a finite number, $\pm\infty$, or does not exist. (If the limit does not exist, write DNE)

\input{Limit-Compute-0013.HELP.tex}

\[\lim_{h\to0}\frac{-\frac{3}{h + 1} + 3}{h}=\answer{3}\]

\end{problem}}%}

\latexProblemContent{
\ifVerboseLocation This is Derivative Compute Question 0013. \\ \fi
\begin{problem}

Determine if the limit approaches a finite number, $\pm\infty$, or does not exist. (If the limit does not exist, write DNE)

\input{Limit-Compute-0013.HELP.tex}

\[\lim_{h\to0}\frac{-3 \, \sqrt{3} + 3 \, \sqrt{h + 3}}{h}=\answer{\frac{1}{2} \, \sqrt{3}}\]

\end{problem}}%}

\latexProblemContent{
\ifVerboseLocation This is Derivative Compute Question 0013. \\ \fi
\begin{problem}

Determine if the limit approaches a finite number, $\pm\infty$, or does not exist. (If the limit does not exist, write DNE)

\input{Limit-Compute-0013.HELP.tex}

\[\lim_{h\to0}\frac{5 \, {\left(h + 2\right)}^{2} - 20}{h}=\answer{20}\]

\end{problem}}%}

\latexProblemContent{
\ifVerboseLocation This is Derivative Compute Question 0013. \\ \fi
\begin{problem}

Determine if the limit approaches a finite number, $\pm\infty$, or does not exist. (If the limit does not exist, write DNE)

\input{Limit-Compute-0013.HELP.tex}

\[\lim_{h\to0}\frac{\frac{1}{h + 1} - 1}{h}=\answer{-1}\]

\end{problem}}%}

\latexProblemContent{
\ifVerboseLocation This is Derivative Compute Question 0013. \\ \fi
\begin{problem}

Determine if the limit approaches a finite number, $\pm\infty$, or does not exist. (If the limit does not exist, write DNE)

\input{Limit-Compute-0013.HELP.tex}

\[\lim_{h\to0}\frac{-\frac{3}{h + 4} + \frac{3}{4}}{h}=\answer{\frac{3}{16}}\]

\end{problem}}%}

\latexProblemContent{
\ifVerboseLocation This is Derivative Compute Question 0013. \\ \fi
\begin{problem}

Determine if the limit approaches a finite number, $\pm\infty$, or does not exist. (If the limit does not exist, write DNE)

\input{Limit-Compute-0013.HELP.tex}

\[\lim_{h\to0}\frac{-\frac{5}{h + 4} + \frac{5}{4}}{h}=\answer{\frac{5}{16}}\]

\end{problem}}%}

\latexProblemContent{
\ifVerboseLocation This is Derivative Compute Question 0013. \\ \fi
\begin{problem}

Determine if the limit approaches a finite number, $\pm\infty$, or does not exist. (If the limit does not exist, write DNE)

\input{Limit-Compute-0013.HELP.tex}

\[\lim_{h\to0}\frac{3 \, {\left(h - 2\right)}^{2} - 12}{h}=\answer{-12}\]

\end{problem}}%}

\latexProblemContent{
\ifVerboseLocation This is Derivative Compute Question 0013. \\ \fi
\begin{problem}

Determine if the limit approaches a finite number, $\pm\infty$, or does not exist. (If the limit does not exist, write DNE)

\input{Limit-Compute-0013.HELP.tex}

\[\lim_{h\to0}\frac{{\left(h - 4\right)}^{3} + 64}{h}=\answer{48}\]

\end{problem}}%}

\latexProblemContent{
\ifVerboseLocation This is Derivative Compute Question 0013. \\ \fi
\begin{problem}

Determine if the limit approaches a finite number, $\pm\infty$, or does not exist. (If the limit does not exist, write DNE)

\input{Limit-Compute-0013.HELP.tex}

\[\lim_{h\to0}\frac{5 \, {\left(h - 2\right)}^{3} + 40}{h}=\answer{60}\]

\end{problem}}%}

\latexProblemContent{
\ifVerboseLocation This is Derivative Compute Question 0013. \\ \fi
\begin{problem}

Determine if the limit approaches a finite number, $\pm\infty$, or does not exist. (If the limit does not exist, write DNE)

\input{Limit-Compute-0013.HELP.tex}

\[\lim_{h\to0}\frac{-4 \, {\left(h + 4\right)}^{2} + 64}{h}=\answer{-32}\]

\end{problem}}%}

\latexProblemContent{
\ifVerboseLocation This is Derivative Compute Question 0013. \\ \fi
\begin{problem}

Determine if the limit approaches a finite number, $\pm\infty$, or does not exist. (If the limit does not exist, write DNE)

\input{Limit-Compute-0013.HELP.tex}

\[\lim_{h\to0}\frac{-5 \, {\left(h - 3\right)}^{3} - 135}{h}=\answer{-135}\]

\end{problem}}%}

\latexProblemContent{
\ifVerboseLocation This is Derivative Compute Question 0013. \\ \fi
\begin{problem}

Determine if the limit approaches a finite number, $\pm\infty$, or does not exist. (If the limit does not exist, write DNE)

\input{Limit-Compute-0013.HELP.tex}

\[\lim_{h\to0}\frac{-\frac{1}{h - 1} - 1}{h}=\answer{1}\]

\end{problem}}%}

\latexProblemContent{
\ifVerboseLocation This is Derivative Compute Question 0013. \\ \fi
\begin{problem}

Determine if the limit approaches a finite number, $\pm\infty$, or does not exist. (If the limit does not exist, write DNE)

\input{Limit-Compute-0013.HELP.tex}

\[\lim_{h\to0}\frac{3 \, {\left(h + 2\right)}^{2} - 12}{h}=\answer{12}\]

\end{problem}}%}

\latexProblemContent{
\ifVerboseLocation This is Derivative Compute Question 0013. \\ \fi
\begin{problem}

Determine if the limit approaches a finite number, $\pm\infty$, or does not exist. (If the limit does not exist, write DNE)

\input{Limit-Compute-0013.HELP.tex}

\[\lim_{h\to0}\frac{5 \, {\left(h - 1\right)}^{3} + 5}{h}=\answer{15}\]

\end{problem}}%}

\latexProblemContent{
\ifVerboseLocation This is Derivative Compute Question 0013. \\ \fi
\begin{problem}

Determine if the limit approaches a finite number, $\pm\infty$, or does not exist. (If the limit does not exist, write DNE)

\input{Limit-Compute-0013.HELP.tex}

\[\lim_{h\to0}\frac{4 \, {\left(h - 2\right)}^{2} - 16}{h}=\answer{-16}\]

\end{problem}}%}

\latexProblemContent{
\ifVerboseLocation This is Derivative Compute Question 0013. \\ \fi
\begin{problem}

Determine if the limit approaches a finite number, $\pm\infty$, or does not exist. (If the limit does not exist, write DNE)

\input{Limit-Compute-0013.HELP.tex}

\[\lim_{h\to0}\frac{-2 \, {\left(h + 3\right)}^{3} + 54}{h}=\answer{-54}\]

\end{problem}}%}

\latexProblemContent{
\ifVerboseLocation This is Derivative Compute Question 0013. \\ \fi
\begin{problem}

Determine if the limit approaches a finite number, $\pm\infty$, or does not exist. (If the limit does not exist, write DNE)

\input{Limit-Compute-0013.HELP.tex}

\[\lim_{h\to0}\frac{{\left(h + 4\right)}^{3} - 64}{h}=\answer{48}\]

\end{problem}}%}

\latexProblemContent{
\ifVerboseLocation This is Derivative Compute Question 0013. \\ \fi
\begin{problem}

Determine if the limit approaches a finite number, $\pm\infty$, or does not exist. (If the limit does not exist, write DNE)

\input{Limit-Compute-0013.HELP.tex}

\[\lim_{h\to0}\frac{4 \, {\left(h - 2\right)}^{3} + 32}{h}=\answer{48}\]

\end{problem}}%}

\latexProblemContent{
\ifVerboseLocation This is Derivative Compute Question 0013. \\ \fi
\begin{problem}

Determine if the limit approaches a finite number, $\pm\infty$, or does not exist. (If the limit does not exist, write DNE)

\input{Limit-Compute-0013.HELP.tex}

\[\lim_{h\to0}\frac{2 \, {\left(h - 1\right)}^{2} - 2}{h}=\answer{-4}\]

\end{problem}}%}

\latexProblemContent{
\ifVerboseLocation This is Derivative Compute Question 0013. \\ \fi
\begin{problem}

Determine if the limit approaches a finite number, $\pm\infty$, or does not exist. (If the limit does not exist, write DNE)

\input{Limit-Compute-0013.HELP.tex}

\[\lim_{h\to0}\frac{5 \, {\left(h - 1\right)}^{2} - 5}{h}=\answer{-10}\]

\end{problem}}%}

\latexProblemContent{
\ifVerboseLocation This is Derivative Compute Question 0013. \\ \fi
\begin{problem}

Determine if the limit approaches a finite number, $\pm\infty$, or does not exist. (If the limit does not exist, write DNE)

\input{Limit-Compute-0013.HELP.tex}

\[\lim_{h\to0}\frac{-{\left(h - 2\right)}^{2} + 4}{h}=\answer{4}\]

\end{problem}}%}

\latexProblemContent{
\ifVerboseLocation This is Derivative Compute Question 0013. \\ \fi
\begin{problem}

Determine if the limit approaches a finite number, $\pm\infty$, or does not exist. (If the limit does not exist, write DNE)

\input{Limit-Compute-0013.HELP.tex}

\[\lim_{h\to0}\frac{-3 \, {\left(h - 1\right)}^{3} - 3}{h}=\answer{-9}\]

\end{problem}}%}

\latexProblemContent{
\ifVerboseLocation This is Derivative Compute Question 0013. \\ \fi
\begin{problem}

Determine if the limit approaches a finite number, $\pm\infty$, or does not exist. (If the limit does not exist, write DNE)

\input{Limit-Compute-0013.HELP.tex}

\[\lim_{h\to0}\frac{5 \, {\left(h + 1\right)}^{2} - 5}{h}=\answer{10}\]

\end{problem}}%}

\latexProblemContent{
\ifVerboseLocation This is Derivative Compute Question 0013. \\ \fi
\begin{problem}

Determine if the limit approaches a finite number, $\pm\infty$, or does not exist. (If the limit does not exist, write DNE)

\input{Limit-Compute-0013.HELP.tex}

\[\lim_{h\to0}\frac{-3 \, {\left(h + 3\right)}^{3} + 81}{h}=\answer{-81}\]

\end{problem}}%}

