\ProblemFileHeader{XTL_SV_QUESTIONCOUNT}% Process how many problems are in this file and how to detect if it has a desirable problem
\ifproblemToFind% If it has a desirable problem search the file.
%%\tagged{Ans@ShortAns, Type@Concept, Topic@Limit, Sub@LimitLaws, File@0001}{
\latexProblemContent{
\ifVerboseLocation This is Derivative Concept Question 0001. \\ \fi
\begin{problem}

If you know that $\lim\limits_{x\to{-3}}f(x)={5}$ and $\lim\limits_{x\to0}g(x)={1}$, then evaluate the following limit:

\input{Limit-Concept-0001.HELP.tex}

\[\lim_{x\to0}f({x - 3})g(x)=\answer{5}\]
\end{problem}}%}

\latexProblemContent{
\ifVerboseLocation This is Derivative Concept Question 0001. \\ \fi
\begin{problem}

If you know that $\lim\limits_{x\to{2}}f(x)={-4}$ and $\lim\limits_{x\to0}g(x)={-1}$, then evaluate the following limit:

\input{Limit-Concept-0001.HELP.tex}

\[\lim_{x\to0}f({x + 2})g(x)=\answer{4}\]
\end{problem}}%}

\latexProblemContent{
\ifVerboseLocation This is Derivative Concept Question 0001. \\ \fi
\begin{problem}

If you know that $\lim\limits_{x\to{1}}f(x)={2}$ and $\lim\limits_{x\to0}g(x)={-2}$, then evaluate the following limit:

\input{Limit-Concept-0001.HELP.tex}

\[\lim_{x\to0}f({x + 1})g(x)=\answer{-4}\]
\end{problem}}%}

\latexProblemContent{
\ifVerboseLocation This is Derivative Concept Question 0001. \\ \fi
\begin{problem}

If you know that $\lim\limits_{x\to{-1}}f(x)={-3}$ and $\lim\limits_{x\to0}g(x)={5}$, then evaluate the following limit:

\input{Limit-Concept-0001.HELP.tex}

\[\lim_{x\to0}f({x - 1})g(x)=\answer{-15}\]
\end{problem}}%}

\latexProblemContent{
\ifVerboseLocation This is Derivative Concept Question 0001. \\ \fi
\begin{problem}

If you know that $\lim\limits_{x\to{-1}}f(x)={-5}$ and $\lim\limits_{x\to0}g(x)={-1}$, then evaluate the following limit:

\input{Limit-Concept-0001.HELP.tex}

\[\lim_{x\to0}f({x - 1})g(x)=\answer{5}\]
\end{problem}}%}

\latexProblemContent{
\ifVerboseLocation This is Derivative Concept Question 0001. \\ \fi
\begin{problem}

If you know that $\lim\limits_{x\to{-2}}f(x)={-5}$ and $\lim\limits_{x\to0}g(x)={2}$, then evaluate the following limit:

\input{Limit-Concept-0001.HELP.tex}

\[\lim_{x\to0}f({x - 2})g(x)=\answer{-10}\]
\end{problem}}%}

\latexProblemContent{
\ifVerboseLocation This is Derivative Concept Question 0001. \\ \fi
\begin{problem}

If you know that $\lim\limits_{x\to{2}}f(x)={1}$ and $\lim\limits_{x\to0}g(x)={1}$, then evaluate the following limit:

\input{Limit-Concept-0001.HELP.tex}

\[\lim_{x\to0}f({x + 2})g(x)=\answer{1}\]
\end{problem}}%}

\latexProblemContent{
\ifVerboseLocation This is Derivative Concept Question 0001. \\ \fi
\begin{problem}

If you know that $\lim\limits_{x\to{2}}f(x)={1}$ and $\lim\limits_{x\to0}g(x)={-4}$, then evaluate the following limit:

\input{Limit-Concept-0001.HELP.tex}

\[\lim_{x\to0}f({x + 2})g(x)=\answer{-4}\]
\end{problem}}%}

\latexProblemContent{
\ifVerboseLocation This is Derivative Concept Question 0001. \\ \fi
\begin{problem}

If you know that $\lim\limits_{x\to{2}}f(x)={-3}$ and $\lim\limits_{x\to0}g(x)={-4}$, then evaluate the following limit:

\input{Limit-Concept-0001.HELP.tex}

\[\lim_{x\to0}f({x + 2})g(x)=\answer{12}\]
\end{problem}}%}

\latexProblemContent{
\ifVerboseLocation This is Derivative Concept Question 0001. \\ \fi
\begin{problem}

If you know that $\lim\limits_{x\to{4}}f(x)={-5}$ and $\lim\limits_{x\to0}g(x)={1}$, then evaluate the following limit:

\input{Limit-Concept-0001.HELP.tex}

\[\lim_{x\to0}f({x + 4})g(x)=\answer{-5}\]
\end{problem}}%}

\latexProblemContent{
\ifVerboseLocation This is Derivative Concept Question 0001. \\ \fi
\begin{problem}

If you know that $\lim\limits_{x\to{-2}}f(x)={4}$ and $\lim\limits_{x\to0}g(x)={4}$, then evaluate the following limit:

\input{Limit-Concept-0001.HELP.tex}

\[\lim_{x\to0}f({x - 2})g(x)=\answer{16}\]
\end{problem}}%}

\latexProblemContent{
\ifVerboseLocation This is Derivative Concept Question 0001. \\ \fi
\begin{problem}

If you know that $\lim\limits_{x\to{3}}f(x)={4}$ and $\lim\limits_{x\to0}g(x)={-3}$, then evaluate the following limit:

\input{Limit-Concept-0001.HELP.tex}

\[\lim_{x\to0}f({x + 3})g(x)=\answer{-12}\]
\end{problem}}%}

\latexProblemContent{
\ifVerboseLocation This is Derivative Concept Question 0001. \\ \fi
\begin{problem}

If you know that $\lim\limits_{x\to{5}}f(x)={3}$ and $\lim\limits_{x\to0}g(x)={-4}$, then evaluate the following limit:

\input{Limit-Concept-0001.HELP.tex}

\[\lim_{x\to0}f({x + 5})g(x)=\answer{-12}\]
\end{problem}}%}

\latexProblemContent{
\ifVerboseLocation This is Derivative Concept Question 0001. \\ \fi
\begin{problem}

If you know that $\lim\limits_{x\to{4}}f(x)={1}$ and $\lim\limits_{x\to0}g(x)={2}$, then evaluate the following limit:

\input{Limit-Concept-0001.HELP.tex}

\[\lim_{x\to0}f({x + 4})g(x)=\answer{2}\]
\end{problem}}%}

\latexProblemContent{
\ifVerboseLocation This is Derivative Concept Question 0001. \\ \fi
\begin{problem}

If you know that $\lim\limits_{x\to{-4}}f(x)={2}$ and $\lim\limits_{x\to0}g(x)={2}$, then evaluate the following limit:

\input{Limit-Concept-0001.HELP.tex}

\[\lim_{x\to0}f({x - 4})g(x)=\answer{4}\]
\end{problem}}%}

\latexProblemContent{
\ifVerboseLocation This is Derivative Concept Question 0001. \\ \fi
\begin{problem}

If you know that $\lim\limits_{x\to{-5}}f(x)={-5}$ and $\lim\limits_{x\to0}g(x)={-5}$, then evaluate the following limit:

\input{Limit-Concept-0001.HELP.tex}

\[\lim_{x\to0}f({x - 5})g(x)=\answer{25}\]
\end{problem}}%}

\latexProblemContent{
\ifVerboseLocation This is Derivative Concept Question 0001. \\ \fi
\begin{problem}

If you know that $\lim\limits_{x\to{3}}f(x)={-5}$ and $\lim\limits_{x\to0}g(x)={-5}$, then evaluate the following limit:

\input{Limit-Concept-0001.HELP.tex}

\[\lim_{x\to0}f({x + 3})g(x)=\answer{25}\]
\end{problem}}%}

\latexProblemContent{
\ifVerboseLocation This is Derivative Concept Question 0001. \\ \fi
\begin{problem}

If you know that $\lim\limits_{x\to{4}}f(x)={4}$ and $\lim\limits_{x\to0}g(x)={-1}$, then evaluate the following limit:

\input{Limit-Concept-0001.HELP.tex}

\[\lim_{x\to0}f({x + 4})g(x)=\answer{-4}\]
\end{problem}}%}

\latexProblemContent{
\ifVerboseLocation This is Derivative Concept Question 0001. \\ \fi
\begin{problem}

If you know that $\lim\limits_{x\to{1}}f(x)={1}$ and $\lim\limits_{x\to0}g(x)={4}$, then evaluate the following limit:

\input{Limit-Concept-0001.HELP.tex}

\[\lim_{x\to0}f({x + 1})g(x)=\answer{4}\]
\end{problem}}%}

\latexProblemContent{
\ifVerboseLocation This is Derivative Concept Question 0001. \\ \fi
\begin{problem}

If you know that $\lim\limits_{x\to{5}}f(x)={-3}$ and $\lim\limits_{x\to0}g(x)={1}$, then evaluate the following limit:

\input{Limit-Concept-0001.HELP.tex}

\[\lim_{x\to0}f({x + 5})g(x)=\answer{-3}\]
\end{problem}}%}

\latexProblemContent{
\ifVerboseLocation This is Derivative Concept Question 0001. \\ \fi
\begin{problem}

If you know that $\lim\limits_{x\to{-1}}f(x)={-3}$ and $\lim\limits_{x\to0}g(x)={-2}$, then evaluate the following limit:

\input{Limit-Concept-0001.HELP.tex}

\[\lim_{x\to0}f({x - 1})g(x)=\answer{6}\]
\end{problem}}%}

\latexProblemContent{
\ifVerboseLocation This is Derivative Concept Question 0001. \\ \fi
\begin{problem}

If you know that $\lim\limits_{x\to{3}}f(x)={4}$ and $\lim\limits_{x\to0}g(x)={2}$, then evaluate the following limit:

\input{Limit-Concept-0001.HELP.tex}

\[\lim_{x\to0}f({x + 3})g(x)=\answer{8}\]
\end{problem}}%}

\latexProblemContent{
\ifVerboseLocation This is Derivative Concept Question 0001. \\ \fi
\begin{problem}

If you know that $\lim\limits_{x\to{3}}f(x)={1}$ and $\lim\limits_{x\to0}g(x)={-5}$, then evaluate the following limit:

\input{Limit-Concept-0001.HELP.tex}

\[\lim_{x\to0}f({x + 3})g(x)=\answer{-5}\]
\end{problem}}%}

\latexProblemContent{
\ifVerboseLocation This is Derivative Concept Question 0001. \\ \fi
\begin{problem}

If you know that $\lim\limits_{x\to{2}}f(x)={1}$ and $\lim\limits_{x\to0}g(x)={-1}$, then evaluate the following limit:

\input{Limit-Concept-0001.HELP.tex}

\[\lim_{x\to0}f({x + 2})g(x)=\answer{-1}\]
\end{problem}}%}

\latexProblemContent{
\ifVerboseLocation This is Derivative Concept Question 0001. \\ \fi
\begin{problem}

If you know that $\lim\limits_{x\to{4}}f(x)={-2}$ and $\lim\limits_{x\to0}g(x)={-1}$, then evaluate the following limit:

\input{Limit-Concept-0001.HELP.tex}

\[\lim_{x\to0}f({x + 4})g(x)=\answer{2}\]
\end{problem}}%}

\latexProblemContent{
\ifVerboseLocation This is Derivative Concept Question 0001. \\ \fi
\begin{problem}

If you know that $\lim\limits_{x\to{5}}f(x)={-2}$ and $\lim\limits_{x\to0}g(x)={-5}$, then evaluate the following limit:

\input{Limit-Concept-0001.HELP.tex}

\[\lim_{x\to0}f({x + 5})g(x)=\answer{10}\]
\end{problem}}%}

\latexProblemContent{
\ifVerboseLocation This is Derivative Concept Question 0001. \\ \fi
\begin{problem}

If you know that $\lim\limits_{x\to{3}}f(x)={-5}$ and $\lim\limits_{x\to0}g(x)={4}$, then evaluate the following limit:

\input{Limit-Concept-0001.HELP.tex}

\[\lim_{x\to0}f({x + 3})g(x)=\answer{-20}\]
\end{problem}}%}

\latexProblemContent{
\ifVerboseLocation This is Derivative Concept Question 0001. \\ \fi
\begin{problem}

If you know that $\lim\limits_{x\to{-2}}f(x)={2}$ and $\lim\limits_{x\to0}g(x)={4}$, then evaluate the following limit:

\input{Limit-Concept-0001.HELP.tex}

\[\lim_{x\to0}f({x - 2})g(x)=\answer{8}\]
\end{problem}}%}

\latexProblemContent{
\ifVerboseLocation This is Derivative Concept Question 0001. \\ \fi
\begin{problem}

If you know that $\lim\limits_{x\to{-4}}f(x)={-1}$ and $\lim\limits_{x\to0}g(x)={-5}$, then evaluate the following limit:

\input{Limit-Concept-0001.HELP.tex}

\[\lim_{x\to0}f({x - 4})g(x)=\answer{5}\]
\end{problem}}%}

\latexProblemContent{
\ifVerboseLocation This is Derivative Concept Question 0001. \\ \fi
\begin{problem}

If you know that $\lim\limits_{x\to{2}}f(x)={4}$ and $\lim\limits_{x\to0}g(x)={4}$, then evaluate the following limit:

\input{Limit-Concept-0001.HELP.tex}

\[\lim_{x\to0}f({x + 2})g(x)=\answer{16}\]
\end{problem}}%}

\latexProblemContent{
\ifVerboseLocation This is Derivative Concept Question 0001. \\ \fi
\begin{problem}

If you know that $\lim\limits_{x\to{-1}}f(x)={-3}$ and $\lim\limits_{x\to0}g(x)={-3}$, then evaluate the following limit:

\input{Limit-Concept-0001.HELP.tex}

\[\lim_{x\to0}f({x - 1})g(x)=\answer{9}\]
\end{problem}}%}

\latexProblemContent{
\ifVerboseLocation This is Derivative Concept Question 0001. \\ \fi
\begin{problem}

If you know that $\lim\limits_{x\to{1}}f(x)={3}$ and $\lim\limits_{x\to0}g(x)={2}$, then evaluate the following limit:

\input{Limit-Concept-0001.HELP.tex}

\[\lim_{x\to0}f({x + 1})g(x)=\answer{6}\]
\end{problem}}%}

\latexProblemContent{
\ifVerboseLocation This is Derivative Concept Question 0001. \\ \fi
\begin{problem}

If you know that $\lim\limits_{x\to{-1}}f(x)={-3}$ and $\lim\limits_{x\to0}g(x)={1}$, then evaluate the following limit:

\input{Limit-Concept-0001.HELP.tex}

\[\lim_{x\to0}f({x - 1})g(x)=\answer{-3}\]
\end{problem}}%}

\latexProblemContent{
\ifVerboseLocation This is Derivative Concept Question 0001. \\ \fi
\begin{problem}

If you know that $\lim\limits_{x\to{1}}f(x)={-3}$ and $\lim\limits_{x\to0}g(x)={5}$, then evaluate the following limit:

\input{Limit-Concept-0001.HELP.tex}

\[\lim_{x\to0}f({x + 1})g(x)=\answer{-15}\]
\end{problem}}%}

\latexProblemContent{
\ifVerboseLocation This is Derivative Concept Question 0001. \\ \fi
\begin{problem}

If you know that $\lim\limits_{x\to{-4}}f(x)={-2}$ and $\lim\limits_{x\to0}g(x)={3}$, then evaluate the following limit:

\input{Limit-Concept-0001.HELP.tex}

\[\lim_{x\to0}f({x - 4})g(x)=\answer{-6}\]
\end{problem}}%}

\latexProblemContent{
\ifVerboseLocation This is Derivative Concept Question 0001. \\ \fi
\begin{problem}

If you know that $\lim\limits_{x\to{-3}}f(x)={-4}$ and $\lim\limits_{x\to0}g(x)={4}$, then evaluate the following limit:

\input{Limit-Concept-0001.HELP.tex}

\[\lim_{x\to0}f({x - 3})g(x)=\answer{-16}\]
\end{problem}}%}

\latexProblemContent{
\ifVerboseLocation This is Derivative Concept Question 0001. \\ \fi
\begin{problem}

If you know that $\lim\limits_{x\to{-4}}f(x)={-3}$ and $\lim\limits_{x\to0}g(x)={-5}$, then evaluate the following limit:

\input{Limit-Concept-0001.HELP.tex}

\[\lim_{x\to0}f({x - 4})g(x)=\answer{15}\]
\end{problem}}%}

\latexProblemContent{
\ifVerboseLocation This is Derivative Concept Question 0001. \\ \fi
\begin{problem}

If you know that $\lim\limits_{x\to{-2}}f(x)={1}$ and $\lim\limits_{x\to0}g(x)={3}$, then evaluate the following limit:

\input{Limit-Concept-0001.HELP.tex}

\[\lim_{x\to0}f({x - 2})g(x)=\answer{3}\]
\end{problem}}%}

\latexProblemContent{
\ifVerboseLocation This is Derivative Concept Question 0001. \\ \fi
\begin{problem}

If you know that $\lim\limits_{x\to{1}}f(x)={-1}$ and $\lim\limits_{x\to0}g(x)={-5}$, then evaluate the following limit:

\input{Limit-Concept-0001.HELP.tex}

\[\lim_{x\to0}f({x + 1})g(x)=\answer{5}\]
\end{problem}}%}

\latexProblemContent{
\ifVerboseLocation This is Derivative Concept Question 0001. \\ \fi
\begin{problem}

If you know that $\lim\limits_{x\to{-2}}f(x)={-1}$ and $\lim\limits_{x\to0}g(x)={-2}$, then evaluate the following limit:

\input{Limit-Concept-0001.HELP.tex}

\[\lim_{x\to0}f({x - 2})g(x)=\answer{2}\]
\end{problem}}%}

\latexProblemContent{
\ifVerboseLocation This is Derivative Concept Question 0001. \\ \fi
\begin{problem}

If you know that $\lim\limits_{x\to{1}}f(x)={-5}$ and $\lim\limits_{x\to0}g(x)={1}$, then evaluate the following limit:

\input{Limit-Concept-0001.HELP.tex}

\[\lim_{x\to0}f({x + 1})g(x)=\answer{-5}\]
\end{problem}}%}

\latexProblemContent{
\ifVerboseLocation This is Derivative Concept Question 0001. \\ \fi
\begin{problem}

If you know that $\lim\limits_{x\to{-4}}f(x)={1}$ and $\lim\limits_{x\to0}g(x)={1}$, then evaluate the following limit:

\input{Limit-Concept-0001.HELP.tex}

\[\lim_{x\to0}f({x - 4})g(x)=\answer{1}\]
\end{problem}}%}

\latexProblemContent{
\ifVerboseLocation This is Derivative Concept Question 0001. \\ \fi
\begin{problem}

If you know that $\lim\limits_{x\to{2}}f(x)={1}$ and $\lim\limits_{x\to0}g(x)={2}$, then evaluate the following limit:

\input{Limit-Concept-0001.HELP.tex}

\[\lim_{x\to0}f({x + 2})g(x)=\answer{2}\]
\end{problem}}%}

\latexProblemContent{
\ifVerboseLocation This is Derivative Concept Question 0001. \\ \fi
\begin{problem}

If you know that $\lim\limits_{x\to{2}}f(x)={-1}$ and $\lim\limits_{x\to0}g(x)={5}$, then evaluate the following limit:

\input{Limit-Concept-0001.HELP.tex}

\[\lim_{x\to0}f({x + 2})g(x)=\answer{-5}\]
\end{problem}}%}

\latexProblemContent{
\ifVerboseLocation This is Derivative Concept Question 0001. \\ \fi
\begin{problem}

If you know that $\lim\limits_{x\to{-4}}f(x)={-2}$ and $\lim\limits_{x\to0}g(x)={-4}$, then evaluate the following limit:

\input{Limit-Concept-0001.HELP.tex}

\[\lim_{x\to0}f({x - 4})g(x)=\answer{8}\]
\end{problem}}%}

\latexProblemContent{
\ifVerboseLocation This is Derivative Concept Question 0001. \\ \fi
\begin{problem}

If you know that $\lim\limits_{x\to{5}}f(x)={-3}$ and $\lim\limits_{x\to0}g(x)={-2}$, then evaluate the following limit:

\input{Limit-Concept-0001.HELP.tex}

\[\lim_{x\to0}f({x + 5})g(x)=\answer{6}\]
\end{problem}}%}

\latexProblemContent{
\ifVerboseLocation This is Derivative Concept Question 0001. \\ \fi
\begin{problem}

If you know that $\lim\limits_{x\to{-2}}f(x)={-2}$ and $\lim\limits_{x\to0}g(x)={4}$, then evaluate the following limit:

\input{Limit-Concept-0001.HELP.tex}

\[\lim_{x\to0}f({x - 2})g(x)=\answer{-8}\]
\end{problem}}%}

\latexProblemContent{
\ifVerboseLocation This is Derivative Concept Question 0001. \\ \fi
\begin{problem}

If you know that $\lim\limits_{x\to{4}}f(x)={-3}$ and $\lim\limits_{x\to0}g(x)={-3}$, then evaluate the following limit:

\input{Limit-Concept-0001.HELP.tex}

\[\lim_{x\to0}f({x + 4})g(x)=\answer{9}\]
\end{problem}}%}

\latexProblemContent{
\ifVerboseLocation This is Derivative Concept Question 0001. \\ \fi
\begin{problem}

If you know that $\lim\limits_{x\to{-2}}f(x)={3}$ and $\lim\limits_{x\to0}g(x)={-2}$, then evaluate the following limit:

\input{Limit-Concept-0001.HELP.tex}

\[\lim_{x\to0}f({x - 2})g(x)=\answer{-6}\]
\end{problem}}%}

\latexProblemContent{
\ifVerboseLocation This is Derivative Concept Question 0001. \\ \fi
\begin{problem}

If you know that $\lim\limits_{x\to{4}}f(x)={-2}$ and $\lim\limits_{x\to0}g(x)={-4}$, then evaluate the following limit:

\input{Limit-Concept-0001.HELP.tex}

\[\lim_{x\to0}f({x + 4})g(x)=\answer{8}\]
\end{problem}}%}

\latexProblemContent{
\ifVerboseLocation This is Derivative Concept Question 0001. \\ \fi
\begin{problem}

If you know that $\lim\limits_{x\to{3}}f(x)={-2}$ and $\lim\limits_{x\to0}g(x)={-3}$, then evaluate the following limit:

\input{Limit-Concept-0001.HELP.tex}

\[\lim_{x\to0}f({x + 3})g(x)=\answer{6}\]
\end{problem}}%}

\latexProblemContent{
\ifVerboseLocation This is Derivative Concept Question 0001. \\ \fi
\begin{problem}

If you know that $\lim\limits_{x\to{4}}f(x)={5}$ and $\lim\limits_{x\to0}g(x)={4}$, then evaluate the following limit:

\input{Limit-Concept-0001.HELP.tex}

\[\lim_{x\to0}f({x + 4})g(x)=\answer{20}\]
\end{problem}}%}

\latexProblemContent{
\ifVerboseLocation This is Derivative Concept Question 0001. \\ \fi
\begin{problem}

If you know that $\lim\limits_{x\to{-3}}f(x)={-1}$ and $\lim\limits_{x\to0}g(x)={2}$, then evaluate the following limit:

\input{Limit-Concept-0001.HELP.tex}

\[\lim_{x\to0}f({x - 3})g(x)=\answer{-2}\]
\end{problem}}%}

\latexProblemContent{
\ifVerboseLocation This is Derivative Concept Question 0001. \\ \fi
\begin{problem}

If you know that $\lim\limits_{x\to{1}}f(x)={-1}$ and $\lim\limits_{x\to0}g(x)={2}$, then evaluate the following limit:

\input{Limit-Concept-0001.HELP.tex}

\[\lim_{x\to0}f({x + 1})g(x)=\answer{-2}\]
\end{problem}}%}

\latexProblemContent{
\ifVerboseLocation This is Derivative Concept Question 0001. \\ \fi
\begin{problem}

If you know that $\lim\limits_{x\to{4}}f(x)={-3}$ and $\lim\limits_{x\to0}g(x)={-1}$, then evaluate the following limit:

\input{Limit-Concept-0001.HELP.tex}

\[\lim_{x\to0}f({x + 4})g(x)=\answer{3}\]
\end{problem}}%}

\latexProblemContent{
\ifVerboseLocation This is Derivative Concept Question 0001. \\ \fi
\begin{problem}

If you know that $\lim\limits_{x\to{3}}f(x)={-2}$ and $\lim\limits_{x\to0}g(x)={-1}$, then evaluate the following limit:

\input{Limit-Concept-0001.HELP.tex}

\[\lim_{x\to0}f({x + 3})g(x)=\answer{2}\]
\end{problem}}%}

\latexProblemContent{
\ifVerboseLocation This is Derivative Concept Question 0001. \\ \fi
\begin{problem}

If you know that $\lim\limits_{x\to{-2}}f(x)={-1}$ and $\lim\limits_{x\to0}g(x)={1}$, then evaluate the following limit:

\input{Limit-Concept-0001.HELP.tex}

\[\lim_{x\to0}f({x - 2})g(x)=\answer{-1}\]
\end{problem}}%}

\latexProblemContent{
\ifVerboseLocation This is Derivative Concept Question 0001. \\ \fi
\begin{problem}

If you know that $\lim\limits_{x\to{5}}f(x)={4}$ and $\lim\limits_{x\to0}g(x)={4}$, then evaluate the following limit:

\input{Limit-Concept-0001.HELP.tex}

\[\lim_{x\to0}f({x + 5})g(x)=\answer{16}\]
\end{problem}}%}

\latexProblemContent{
\ifVerboseLocation This is Derivative Concept Question 0001. \\ \fi
\begin{problem}

If you know that $\lim\limits_{x\to{1}}f(x)={-1}$ and $\lim\limits_{x\to0}g(x)={-4}$, then evaluate the following limit:

\input{Limit-Concept-0001.HELP.tex}

\[\lim_{x\to0}f({x + 1})g(x)=\answer{4}\]
\end{problem}}%}

\latexProblemContent{
\ifVerboseLocation This is Derivative Concept Question 0001. \\ \fi
\begin{problem}

If you know that $\lim\limits_{x\to{-2}}f(x)={-2}$ and $\lim\limits_{x\to0}g(x)={3}$, then evaluate the following limit:

\input{Limit-Concept-0001.HELP.tex}

\[\lim_{x\to0}f({x - 2})g(x)=\answer{-6}\]
\end{problem}}%}

\latexProblemContent{
\ifVerboseLocation This is Derivative Concept Question 0001. \\ \fi
\begin{problem}

If you know that $\lim\limits_{x\to{-1}}f(x)={1}$ and $\lim\limits_{x\to0}g(x)={-3}$, then evaluate the following limit:

\input{Limit-Concept-0001.HELP.tex}

\[\lim_{x\to0}f({x - 1})g(x)=\answer{-3}\]
\end{problem}}%}

\latexProblemContent{
\ifVerboseLocation This is Derivative Concept Question 0001. \\ \fi
\begin{problem}

If you know that $\lim\limits_{x\to{1}}f(x)={5}$ and $\lim\limits_{x\to0}g(x)={-4}$, then evaluate the following limit:

\input{Limit-Concept-0001.HELP.tex}

\[\lim_{x\to0}f({x + 1})g(x)=\answer{-20}\]
\end{problem}}%}

\latexProblemContent{
\ifVerboseLocation This is Derivative Concept Question 0001. \\ \fi
\begin{problem}

If you know that $\lim\limits_{x\to{-1}}f(x)={-3}$ and $\lim\limits_{x\to0}g(x)={-5}$, then evaluate the following limit:

\input{Limit-Concept-0001.HELP.tex}

\[\lim_{x\to0}f({x - 1})g(x)=\answer{15}\]
\end{problem}}%}

\latexProblemContent{
\ifVerboseLocation This is Derivative Concept Question 0001. \\ \fi
\begin{problem}

If you know that $\lim\limits_{x\to{-4}}f(x)={2}$ and $\lim\limits_{x\to0}g(x)={-1}$, then evaluate the following limit:

\input{Limit-Concept-0001.HELP.tex}

\[\lim_{x\to0}f({x - 4})g(x)=\answer{-2}\]
\end{problem}}%}

\latexProblemContent{
\ifVerboseLocation This is Derivative Concept Question 0001. \\ \fi
\begin{problem}

If you know that $\lim\limits_{x\to{-1}}f(x)={-1}$ and $\lim\limits_{x\to0}g(x)={4}$, then evaluate the following limit:

\input{Limit-Concept-0001.HELP.tex}

\[\lim_{x\to0}f({x - 1})g(x)=\answer{-4}\]
\end{problem}}%}

\latexProblemContent{
\ifVerboseLocation This is Derivative Concept Question 0001. \\ \fi
\begin{problem}

If you know that $\lim\limits_{x\to{-2}}f(x)={2}$ and $\lim\limits_{x\to0}g(x)={-4}$, then evaluate the following limit:

\input{Limit-Concept-0001.HELP.tex}

\[\lim_{x\to0}f({x - 2})g(x)=\answer{-8}\]
\end{problem}}%}

\latexProblemContent{
\ifVerboseLocation This is Derivative Concept Question 0001. \\ \fi
\begin{problem}

If you know that $\lim\limits_{x\to{2}}f(x)={2}$ and $\lim\limits_{x\to0}g(x)={2}$, then evaluate the following limit:

\input{Limit-Concept-0001.HELP.tex}

\[\lim_{x\to0}f({x + 2})g(x)=\answer{4}\]
\end{problem}}%}

\latexProblemContent{
\ifVerboseLocation This is Derivative Concept Question 0001. \\ \fi
\begin{problem}

If you know that $\lim\limits_{x\to{-3}}f(x)={-5}$ and $\lim\limits_{x\to0}g(x)={-2}$, then evaluate the following limit:

\input{Limit-Concept-0001.HELP.tex}

\[\lim_{x\to0}f({x - 3})g(x)=\answer{10}\]
\end{problem}}%}

\latexProblemContent{
\ifVerboseLocation This is Derivative Concept Question 0001. \\ \fi
\begin{problem}

If you know that $\lim\limits_{x\to{-3}}f(x)={2}$ and $\lim\limits_{x\to0}g(x)={-4}$, then evaluate the following limit:

\input{Limit-Concept-0001.HELP.tex}

\[\lim_{x\to0}f({x - 3})g(x)=\answer{-8}\]
\end{problem}}%}

\latexProblemContent{
\ifVerboseLocation This is Derivative Concept Question 0001. \\ \fi
\begin{problem}

If you know that $\lim\limits_{x\to{-2}}f(x)={5}$ and $\lim\limits_{x\to0}g(x)={-2}$, then evaluate the following limit:

\input{Limit-Concept-0001.HELP.tex}

\[\lim_{x\to0}f({x - 2})g(x)=\answer{-10}\]
\end{problem}}%}

\latexProblemContent{
\ifVerboseLocation This is Derivative Concept Question 0001. \\ \fi
\begin{problem}

If you know that $\lim\limits_{x\to{-5}}f(x)={-4}$ and $\lim\limits_{x\to0}g(x)={-3}$, then evaluate the following limit:

\input{Limit-Concept-0001.HELP.tex}

\[\lim_{x\to0}f({x - 5})g(x)=\answer{12}\]
\end{problem}}%}

\latexProblemContent{
\ifVerboseLocation This is Derivative Concept Question 0001. \\ \fi
\begin{problem}

If you know that $\lim\limits_{x\to{-2}}f(x)={5}$ and $\lim\limits_{x\to0}g(x)={2}$, then evaluate the following limit:

\input{Limit-Concept-0001.HELP.tex}

\[\lim_{x\to0}f({x - 2})g(x)=\answer{10}\]
\end{problem}}%}

\latexProblemContent{
\ifVerboseLocation This is Derivative Concept Question 0001. \\ \fi
\begin{problem}

If you know that $\lim\limits_{x\to{4}}f(x)={-3}$ and $\lim\limits_{x\to0}g(x)={1}$, then evaluate the following limit:

\input{Limit-Concept-0001.HELP.tex}

\[\lim_{x\to0}f({x + 4})g(x)=\answer{-3}\]
\end{problem}}%}

\latexProblemContent{
\ifVerboseLocation This is Derivative Concept Question 0001. \\ \fi
\begin{problem}

If you know that $\lim\limits_{x\to{4}}f(x)={2}$ and $\lim\limits_{x\to0}g(x)={-5}$, then evaluate the following limit:

\input{Limit-Concept-0001.HELP.tex}

\[\lim_{x\to0}f({x + 4})g(x)=\answer{-10}\]
\end{problem}}%}

\latexProblemContent{
\ifVerboseLocation This is Derivative Concept Question 0001. \\ \fi
\begin{problem}

If you know that $\lim\limits_{x\to{-2}}f(x)={4}$ and $\lim\limits_{x\to0}g(x)={-5}$, then evaluate the following limit:

\input{Limit-Concept-0001.HELP.tex}

\[\lim_{x\to0}f({x - 2})g(x)=\answer{-20}\]
\end{problem}}%}

\latexProblemContent{
\ifVerboseLocation This is Derivative Concept Question 0001. \\ \fi
\begin{problem}

If you know that $\lim\limits_{x\to{-1}}f(x)={3}$ and $\lim\limits_{x\to0}g(x)={-2}$, then evaluate the following limit:

\input{Limit-Concept-0001.HELP.tex}

\[\lim_{x\to0}f({x - 1})g(x)=\answer{-6}\]
\end{problem}}%}

\latexProblemContent{
\ifVerboseLocation This is Derivative Concept Question 0001. \\ \fi
\begin{problem}

If you know that $\lim\limits_{x\to{-4}}f(x)={-3}$ and $\lim\limits_{x\to0}g(x)={-3}$, then evaluate the following limit:

\input{Limit-Concept-0001.HELP.tex}

\[\lim_{x\to0}f({x - 4})g(x)=\answer{9}\]
\end{problem}}%}

\latexProblemContent{
\ifVerboseLocation This is Derivative Concept Question 0001. \\ \fi
\begin{problem}

If you know that $\lim\limits_{x\to{-5}}f(x)={2}$ and $\lim\limits_{x\to0}g(x)={2}$, then evaluate the following limit:

\input{Limit-Concept-0001.HELP.tex}

\[\lim_{x\to0}f({x - 5})g(x)=\answer{4}\]
\end{problem}}%}

\latexProblemContent{
\ifVerboseLocation This is Derivative Concept Question 0001. \\ \fi
\begin{problem}

If you know that $\lim\limits_{x\to{-1}}f(x)={-4}$ and $\lim\limits_{x\to0}g(x)={-3}$, then evaluate the following limit:

\input{Limit-Concept-0001.HELP.tex}

\[\lim_{x\to0}f({x - 1})g(x)=\answer{12}\]
\end{problem}}%}

\latexProblemContent{
\ifVerboseLocation This is Derivative Concept Question 0001. \\ \fi
\begin{problem}

If you know that $\lim\limits_{x\to{5}}f(x)={3}$ and $\lim\limits_{x\to0}g(x)={-5}$, then evaluate the following limit:

\input{Limit-Concept-0001.HELP.tex}

\[\lim_{x\to0}f({x + 5})g(x)=\answer{-15}\]
\end{problem}}%}

\latexProblemContent{
\ifVerboseLocation This is Derivative Concept Question 0001. \\ \fi
\begin{problem}

If you know that $\lim\limits_{x\to{-2}}f(x)={2}$ and $\lim\limits_{x\to0}g(x)={3}$, then evaluate the following limit:

\input{Limit-Concept-0001.HELP.tex}

\[\lim_{x\to0}f({x - 2})g(x)=\answer{6}\]
\end{problem}}%}

\latexProblemContent{
\ifVerboseLocation This is Derivative Concept Question 0001. \\ \fi
\begin{problem}

If you know that $\lim\limits_{x\to{-4}}f(x)={4}$ and $\lim\limits_{x\to0}g(x)={3}$, then evaluate the following limit:

\input{Limit-Concept-0001.HELP.tex}

\[\lim_{x\to0}f({x - 4})g(x)=\answer{12}\]
\end{problem}}%}

\latexProblemContent{
\ifVerboseLocation This is Derivative Concept Question 0001. \\ \fi
\begin{problem}

If you know that $\lim\limits_{x\to{4}}f(x)={4}$ and $\lim\limits_{x\to0}g(x)={-3}$, then evaluate the following limit:

\input{Limit-Concept-0001.HELP.tex}

\[\lim_{x\to0}f({x + 4})g(x)=\answer{-12}\]
\end{problem}}%}

\latexProblemContent{
\ifVerboseLocation This is Derivative Concept Question 0001. \\ \fi
\begin{problem}

If you know that $\lim\limits_{x\to{3}}f(x)={-4}$ and $\lim\limits_{x\to0}g(x)={3}$, then evaluate the following limit:

\input{Limit-Concept-0001.HELP.tex}

\[\lim_{x\to0}f({x + 3})g(x)=\answer{-12}\]
\end{problem}}%}

\latexProblemContent{
\ifVerboseLocation This is Derivative Concept Question 0001. \\ \fi
\begin{problem}

If you know that $\lim\limits_{x\to{-3}}f(x)={3}$ and $\lim\limits_{x\to0}g(x)={-4}$, then evaluate the following limit:

\input{Limit-Concept-0001.HELP.tex}

\[\lim_{x\to0}f({x - 3})g(x)=\answer{-12}\]
\end{problem}}%}

\latexProblemContent{
\ifVerboseLocation This is Derivative Concept Question 0001. \\ \fi
\begin{problem}

If you know that $\lim\limits_{x\to{1}}f(x)={5}$ and $\lim\limits_{x\to0}g(x)={-5}$, then evaluate the following limit:

\input{Limit-Concept-0001.HELP.tex}

\[\lim_{x\to0}f({x + 1})g(x)=\answer{-25}\]
\end{problem}}%}

\latexProblemContent{
\ifVerboseLocation This is Derivative Concept Question 0001. \\ \fi
\begin{problem}

If you know that $\lim\limits_{x\to{-5}}f(x)={-3}$ and $\lim\limits_{x\to0}g(x)={4}$, then evaluate the following limit:

\input{Limit-Concept-0001.HELP.tex}

\[\lim_{x\to0}f({x - 5})g(x)=\answer{-12}\]
\end{problem}}%}

\latexProblemContent{
\ifVerboseLocation This is Derivative Concept Question 0001. \\ \fi
\begin{problem}

If you know that $\lim\limits_{x\to{1}}f(x)={-4}$ and $\lim\limits_{x\to0}g(x)={1}$, then evaluate the following limit:

\input{Limit-Concept-0001.HELP.tex}

\[\lim_{x\to0}f({x + 1})g(x)=\answer{-4}\]
\end{problem}}%}

\latexProblemContent{
\ifVerboseLocation This is Derivative Concept Question 0001. \\ \fi
\begin{problem}

If you know that $\lim\limits_{x\to{4}}f(x)={4}$ and $\lim\limits_{x\to0}g(x)={-5}$, then evaluate the following limit:

\input{Limit-Concept-0001.HELP.tex}

\[\lim_{x\to0}f({x + 4})g(x)=\answer{-20}\]
\end{problem}}%}

\latexProblemContent{
\ifVerboseLocation This is Derivative Concept Question 0001. \\ \fi
\begin{problem}

If you know that $\lim\limits_{x\to{-1}}f(x)={4}$ and $\lim\limits_{x\to0}g(x)={-2}$, then evaluate the following limit:

\input{Limit-Concept-0001.HELP.tex}

\[\lim_{x\to0}f({x - 1})g(x)=\answer{-8}\]
\end{problem}}%}

\latexProblemContent{
\ifVerboseLocation This is Derivative Concept Question 0001. \\ \fi
\begin{problem}

If you know that $\lim\limits_{x\to{4}}f(x)={5}$ and $\lim\limits_{x\to0}g(x)={-2}$, then evaluate the following limit:

\input{Limit-Concept-0001.HELP.tex}

\[\lim_{x\to0}f({x + 4})g(x)=\answer{-10}\]
\end{problem}}%}

\latexProblemContent{
\ifVerboseLocation This is Derivative Concept Question 0001. \\ \fi
\begin{problem}

If you know that $\lim\limits_{x\to{-2}}f(x)={1}$ and $\lim\limits_{x\to0}g(x)={-2}$, then evaluate the following limit:

\input{Limit-Concept-0001.HELP.tex}

\[\lim_{x\to0}f({x - 2})g(x)=\answer{-2}\]
\end{problem}}%}

\latexProblemContent{
\ifVerboseLocation This is Derivative Concept Question 0001. \\ \fi
\begin{problem}

If you know that $\lim\limits_{x\to{-4}}f(x)={-4}$ and $\lim\limits_{x\to0}g(x)={2}$, then evaluate the following limit:

\input{Limit-Concept-0001.HELP.tex}

\[\lim_{x\to0}f({x - 4})g(x)=\answer{-8}\]
\end{problem}}%}

\latexProblemContent{
\ifVerboseLocation This is Derivative Concept Question 0001. \\ \fi
\begin{problem}

If you know that $\lim\limits_{x\to{-1}}f(x)={1}$ and $\lim\limits_{x\to0}g(x)={-2}$, then evaluate the following limit:

\input{Limit-Concept-0001.HELP.tex}

\[\lim_{x\to0}f({x - 1})g(x)=\answer{-2}\]
\end{problem}}%}

\latexProblemContent{
\ifVerboseLocation This is Derivative Concept Question 0001. \\ \fi
\begin{problem}

If you know that $\lim\limits_{x\to{4}}f(x)={-4}$ and $\lim\limits_{x\to0}g(x)={-2}$, then evaluate the following limit:

\input{Limit-Concept-0001.HELP.tex}

\[\lim_{x\to0}f({x + 4})g(x)=\answer{8}\]
\end{problem}}%}

\latexProblemContent{
\ifVerboseLocation This is Derivative Concept Question 0001. \\ \fi
\begin{problem}

If you know that $\lim\limits_{x\to{-1}}f(x)={4}$ and $\lim\limits_{x\to0}g(x)={-1}$, then evaluate the following limit:

\input{Limit-Concept-0001.HELP.tex}

\[\lim_{x\to0}f({x - 1})g(x)=\answer{-4}\]
\end{problem}}%}

\latexProblemContent{
\ifVerboseLocation This is Derivative Concept Question 0001. \\ \fi
\begin{problem}

If you know that $\lim\limits_{x\to{-4}}f(x)={5}$ and $\lim\limits_{x\to0}g(x)={5}$, then evaluate the following limit:

\input{Limit-Concept-0001.HELP.tex}

\[\lim_{x\to0}f({x - 4})g(x)=\answer{25}\]
\end{problem}}%}

\latexProblemContent{
\ifVerboseLocation This is Derivative Concept Question 0001. \\ \fi
\begin{problem}

If you know that $\lim\limits_{x\to{-5}}f(x)={5}$ and $\lim\limits_{x\to0}g(x)={-3}$, then evaluate the following limit:

\input{Limit-Concept-0001.HELP.tex}

\[\lim_{x\to0}f({x - 5})g(x)=\answer{-15}\]
\end{problem}}%}

\latexProblemContent{
\ifVerboseLocation This is Derivative Concept Question 0001. \\ \fi
\begin{problem}

If you know that $\lim\limits_{x\to{3}}f(x)={5}$ and $\lim\limits_{x\to0}g(x)={3}$, then evaluate the following limit:

\input{Limit-Concept-0001.HELP.tex}

\[\lim_{x\to0}f({x + 3})g(x)=\answer{15}\]
\end{problem}}%}

\latexProblemContent{
\ifVerboseLocation This is Derivative Concept Question 0001. \\ \fi
\begin{problem}

If you know that $\lim\limits_{x\to{1}}f(x)={4}$ and $\lim\limits_{x\to0}g(x)={-5}$, then evaluate the following limit:

\input{Limit-Concept-0001.HELP.tex}

\[\lim_{x\to0}f({x + 1})g(x)=\answer{-20}\]
\end{problem}}%}

\latexProblemContent{
\ifVerboseLocation This is Derivative Concept Question 0001. \\ \fi
\begin{problem}

If you know that $\lim\limits_{x\to{-1}}f(x)={-4}$ and $\lim\limits_{x\to0}g(x)={3}$, then evaluate the following limit:

\input{Limit-Concept-0001.HELP.tex}

\[\lim_{x\to0}f({x - 1})g(x)=\answer{-12}\]
\end{problem}}%}

\latexProblemContent{
\ifVerboseLocation This is Derivative Concept Question 0001. \\ \fi
\begin{problem}

If you know that $\lim\limits_{x\to{-1}}f(x)={4}$ and $\lim\limits_{x\to0}g(x)={2}$, then evaluate the following limit:

\input{Limit-Concept-0001.HELP.tex}

\[\lim_{x\to0}f({x - 1})g(x)=\answer{8}\]
\end{problem}}%}

\latexProblemContent{
\ifVerboseLocation This is Derivative Concept Question 0001. \\ \fi
\begin{problem}

If you know that $\lim\limits_{x\to{3}}f(x)={3}$ and $\lim\limits_{x\to0}g(x)={4}$, then evaluate the following limit:

\input{Limit-Concept-0001.HELP.tex}

\[\lim_{x\to0}f({x + 3})g(x)=\answer{12}\]
\end{problem}}%}

\latexProblemContent{
\ifVerboseLocation This is Derivative Concept Question 0001. \\ \fi
\begin{problem}

If you know that $\lim\limits_{x\to{4}}f(x)={4}$ and $\lim\limits_{x\to0}g(x)={-4}$, then evaluate the following limit:

\input{Limit-Concept-0001.HELP.tex}

\[\lim_{x\to0}f({x + 4})g(x)=\answer{-16}\]
\end{problem}}%}

\latexProblemContent{
\ifVerboseLocation This is Derivative Concept Question 0001. \\ \fi
\begin{problem}

If you know that $\lim\limits_{x\to{5}}f(x)={-2}$ and $\lim\limits_{x\to0}g(x)={-4}$, then evaluate the following limit:

\input{Limit-Concept-0001.HELP.tex}

\[\lim_{x\to0}f({x + 5})g(x)=\answer{8}\]
\end{problem}}%}

\latexProblemContent{
\ifVerboseLocation This is Derivative Concept Question 0001. \\ \fi
\begin{problem}

If you know that $\lim\limits_{x\to{-2}}f(x)={-4}$ and $\lim\limits_{x\to0}g(x)={-5}$, then evaluate the following limit:

\input{Limit-Concept-0001.HELP.tex}

\[\lim_{x\to0}f({x - 2})g(x)=\answer{20}\]
\end{problem}}%}

\latexProblemContent{
\ifVerboseLocation This is Derivative Concept Question 0001. \\ \fi
\begin{problem}

If you know that $\lim\limits_{x\to{-5}}f(x)={1}$ and $\lim\limits_{x\to0}g(x)={1}$, then evaluate the following limit:

\input{Limit-Concept-0001.HELP.tex}

\[\lim_{x\to0}f({x - 5})g(x)=\answer{1}\]
\end{problem}}%}

\latexProblemContent{
\ifVerboseLocation This is Derivative Concept Question 0001. \\ \fi
\begin{problem}

If you know that $\lim\limits_{x\to{-5}}f(x)={-2}$ and $\lim\limits_{x\to0}g(x)={-2}$, then evaluate the following limit:

\input{Limit-Concept-0001.HELP.tex}

\[\lim_{x\to0}f({x - 5})g(x)=\answer{4}\]
\end{problem}}%}

\latexProblemContent{
\ifVerboseLocation This is Derivative Concept Question 0001. \\ \fi
\begin{problem}

If you know that $\lim\limits_{x\to{2}}f(x)={1}$ and $\lim\limits_{x\to0}g(x)={-5}$, then evaluate the following limit:

\input{Limit-Concept-0001.HELP.tex}

\[\lim_{x\to0}f({x + 2})g(x)=\answer{-5}\]
\end{problem}}%}

\latexProblemContent{
\ifVerboseLocation This is Derivative Concept Question 0001. \\ \fi
\begin{problem}

If you know that $\lim\limits_{x\to{-4}}f(x)={-2}$ and $\lim\limits_{x\to0}g(x)={-3}$, then evaluate the following limit:

\input{Limit-Concept-0001.HELP.tex}

\[\lim_{x\to0}f({x - 4})g(x)=\answer{6}\]
\end{problem}}%}

\latexProblemContent{
\ifVerboseLocation This is Derivative Concept Question 0001. \\ \fi
\begin{problem}

If you know that $\lim\limits_{x\to{-1}}f(x)={-2}$ and $\lim\limits_{x\to0}g(x)={-4}$, then evaluate the following limit:

\input{Limit-Concept-0001.HELP.tex}

\[\lim_{x\to0}f({x - 1})g(x)=\answer{8}\]
\end{problem}}%}

\latexProblemContent{
\ifVerboseLocation This is Derivative Concept Question 0001. \\ \fi
\begin{problem}

If you know that $\lim\limits_{x\to{-5}}f(x)={-4}$ and $\lim\limits_{x\to0}g(x)={-4}$, then evaluate the following limit:

\input{Limit-Concept-0001.HELP.tex}

\[\lim_{x\to0}f({x - 5})g(x)=\answer{16}\]
\end{problem}}%}

\latexProblemContent{
\ifVerboseLocation This is Derivative Concept Question 0001. \\ \fi
\begin{problem}

If you know that $\lim\limits_{x\to{-5}}f(x)={-5}$ and $\lim\limits_{x\to0}g(x)={-2}$, then evaluate the following limit:

\input{Limit-Concept-0001.HELP.tex}

\[\lim_{x\to0}f({x - 5})g(x)=\answer{10}\]
\end{problem}}%}

\latexProblemContent{
\ifVerboseLocation This is Derivative Concept Question 0001. \\ \fi
\begin{problem}

If you know that $\lim\limits_{x\to{-1}}f(x)={-5}$ and $\lim\limits_{x\to0}g(x)={-4}$, then evaluate the following limit:

\input{Limit-Concept-0001.HELP.tex}

\[\lim_{x\to0}f({x - 1})g(x)=\answer{20}\]
\end{problem}}%}

\latexProblemContent{
\ifVerboseLocation This is Derivative Concept Question 0001. \\ \fi
\begin{problem}

If you know that $\lim\limits_{x\to{-4}}f(x)={1}$ and $\lim\limits_{x\to0}g(x)={-1}$, then evaluate the following limit:

\input{Limit-Concept-0001.HELP.tex}

\[\lim_{x\to0}f({x - 4})g(x)=\answer{-1}\]
\end{problem}}%}

\latexProblemContent{
\ifVerboseLocation This is Derivative Concept Question 0001. \\ \fi
\begin{problem}

If you know that $\lim\limits_{x\to{-4}}f(x)={1}$ and $\lim\limits_{x\to0}g(x)={5}$, then evaluate the following limit:

\input{Limit-Concept-0001.HELP.tex}

\[\lim_{x\to0}f({x - 4})g(x)=\answer{5}\]
\end{problem}}%}

\latexProblemContent{
\ifVerboseLocation This is Derivative Concept Question 0001. \\ \fi
\begin{problem}

If you know that $\lim\limits_{x\to{-5}}f(x)={2}$ and $\lim\limits_{x\to0}g(x)={4}$, then evaluate the following limit:

\input{Limit-Concept-0001.HELP.tex}

\[\lim_{x\to0}f({x - 5})g(x)=\answer{8}\]
\end{problem}}%}

\latexProblemContent{
\ifVerboseLocation This is Derivative Concept Question 0001. \\ \fi
\begin{problem}

If you know that $\lim\limits_{x\to{5}}f(x)={2}$ and $\lim\limits_{x\to0}g(x)={3}$, then evaluate the following limit:

\input{Limit-Concept-0001.HELP.tex}

\[\lim_{x\to0}f({x + 5})g(x)=\answer{6}\]
\end{problem}}%}

\latexProblemContent{
\ifVerboseLocation This is Derivative Concept Question 0001. \\ \fi
\begin{problem}

If you know that $\lim\limits_{x\to{-1}}f(x)={-4}$ and $\lim\limits_{x\to0}g(x)={-2}$, then evaluate the following limit:

\input{Limit-Concept-0001.HELP.tex}

\[\lim_{x\to0}f({x - 1})g(x)=\answer{8}\]
\end{problem}}%}

\latexProblemContent{
\ifVerboseLocation This is Derivative Concept Question 0001. \\ \fi
\begin{problem}

If you know that $\lim\limits_{x\to{-5}}f(x)={3}$ and $\lim\limits_{x\to0}g(x)={-5}$, then evaluate the following limit:

\input{Limit-Concept-0001.HELP.tex}

\[\lim_{x\to0}f({x - 5})g(x)=\answer{-15}\]
\end{problem}}%}

\latexProblemContent{
\ifVerboseLocation This is Derivative Concept Question 0001. \\ \fi
\begin{problem}

If you know that $\lim\limits_{x\to{2}}f(x)={-5}$ and $\lim\limits_{x\to0}g(x)={4}$, then evaluate the following limit:

\input{Limit-Concept-0001.HELP.tex}

\[\lim_{x\to0}f({x + 2})g(x)=\answer{-20}\]
\end{problem}}%}

\latexProblemContent{
\ifVerboseLocation This is Derivative Concept Question 0001. \\ \fi
\begin{problem}

If you know that $\lim\limits_{x\to{2}}f(x)={-5}$ and $\lim\limits_{x\to0}g(x)={-1}$, then evaluate the following limit:

\input{Limit-Concept-0001.HELP.tex}

\[\lim_{x\to0}f({x + 2})g(x)=\answer{5}\]
\end{problem}}%}

\latexProblemContent{
\ifVerboseLocation This is Derivative Concept Question 0001. \\ \fi
\begin{problem}

If you know that $\lim\limits_{x\to{-5}}f(x)={4}$ and $\lim\limits_{x\to0}g(x)={-4}$, then evaluate the following limit:

\input{Limit-Concept-0001.HELP.tex}

\[\lim_{x\to0}f({x - 5})g(x)=\answer{-16}\]
\end{problem}}%}

\latexProblemContent{
\ifVerboseLocation This is Derivative Concept Question 0001. \\ \fi
\begin{problem}

If you know that $\lim\limits_{x\to{-4}}f(x)={2}$ and $\lim\limits_{x\to0}g(x)={5}$, then evaluate the following limit:

\input{Limit-Concept-0001.HELP.tex}

\[\lim_{x\to0}f({x - 4})g(x)=\answer{10}\]
\end{problem}}%}

\latexProblemContent{
\ifVerboseLocation This is Derivative Concept Question 0001. \\ \fi
\begin{problem}

If you know that $\lim\limits_{x\to{3}}f(x)={-2}$ and $\lim\limits_{x\to0}g(x)={3}$, then evaluate the following limit:

\input{Limit-Concept-0001.HELP.tex}

\[\lim_{x\to0}f({x + 3})g(x)=\answer{-6}\]
\end{problem}}%}

\latexProblemContent{
\ifVerboseLocation This is Derivative Concept Question 0001. \\ \fi
\begin{problem}

If you know that $\lim\limits_{x\to{5}}f(x)={-1}$ and $\lim\limits_{x\to0}g(x)={1}$, then evaluate the following limit:

\input{Limit-Concept-0001.HELP.tex}

\[\lim_{x\to0}f({x + 5})g(x)=\answer{-1}\]
\end{problem}}%}

\latexProblemContent{
\ifVerboseLocation This is Derivative Concept Question 0001. \\ \fi
\begin{problem}

If you know that $\lim\limits_{x\to{3}}f(x)={1}$ and $\lim\limits_{x\to0}g(x)={4}$, then evaluate the following limit:

\input{Limit-Concept-0001.HELP.tex}

\[\lim_{x\to0}f({x + 3})g(x)=\answer{4}\]
\end{problem}}%}

\latexProblemContent{
\ifVerboseLocation This is Derivative Concept Question 0001. \\ \fi
\begin{problem}

If you know that $\lim\limits_{x\to{-1}}f(x)={-3}$ and $\lim\limits_{x\to0}g(x)={3}$, then evaluate the following limit:

\input{Limit-Concept-0001.HELP.tex}

\[\lim_{x\to0}f({x - 1})g(x)=\answer{-9}\]
\end{problem}}%}

\latexProblemContent{
\ifVerboseLocation This is Derivative Concept Question 0001. \\ \fi
\begin{problem}

If you know that $\lim\limits_{x\to{4}}f(x)={2}$ and $\lim\limits_{x\to0}g(x)={5}$, then evaluate the following limit:

\input{Limit-Concept-0001.HELP.tex}

\[\lim_{x\to0}f({x + 4})g(x)=\answer{10}\]
\end{problem}}%}

\latexProblemContent{
\ifVerboseLocation This is Derivative Concept Question 0001. \\ \fi
\begin{problem}

If you know that $\lim\limits_{x\to{-5}}f(x)={3}$ and $\lim\limits_{x\to0}g(x)={4}$, then evaluate the following limit:

\input{Limit-Concept-0001.HELP.tex}

\[\lim_{x\to0}f({x - 5})g(x)=\answer{12}\]
\end{problem}}%}

\latexProblemContent{
\ifVerboseLocation This is Derivative Concept Question 0001. \\ \fi
\begin{problem}

If you know that $\lim\limits_{x\to{-1}}f(x)={-5}$ and $\lim\limits_{x\to0}g(x)={3}$, then evaluate the following limit:

\input{Limit-Concept-0001.HELP.tex}

\[\lim_{x\to0}f({x - 1})g(x)=\answer{-15}\]
\end{problem}}%}

\latexProblemContent{
\ifVerboseLocation This is Derivative Concept Question 0001. \\ \fi
\begin{problem}

If you know that $\lim\limits_{x\to{4}}f(x)={-5}$ and $\lim\limits_{x\to0}g(x)={-5}$, then evaluate the following limit:

\input{Limit-Concept-0001.HELP.tex}

\[\lim_{x\to0}f({x + 4})g(x)=\answer{25}\]
\end{problem}}%}

\latexProblemContent{
\ifVerboseLocation This is Derivative Concept Question 0001. \\ \fi
\begin{problem}

If you know that $\lim\limits_{x\to{-1}}f(x)={-1}$ and $\lim\limits_{x\to0}g(x)={5}$, then evaluate the following limit:

\input{Limit-Concept-0001.HELP.tex}

\[\lim_{x\to0}f({x - 1})g(x)=\answer{-5}\]
\end{problem}}%}

\latexProblemContent{
\ifVerboseLocation This is Derivative Concept Question 0001. \\ \fi
\begin{problem}

If you know that $\lim\limits_{x\to{-2}}f(x)={5}$ and $\lim\limits_{x\to0}g(x)={-1}$, then evaluate the following limit:

\input{Limit-Concept-0001.HELP.tex}

\[\lim_{x\to0}f({x - 2})g(x)=\answer{-5}\]
\end{problem}}%}

\latexProblemContent{
\ifVerboseLocation This is Derivative Concept Question 0001. \\ \fi
\begin{problem}

If you know that $\lim\limits_{x\to{5}}f(x)={2}$ and $\lim\limits_{x\to0}g(x)={2}$, then evaluate the following limit:

\input{Limit-Concept-0001.HELP.tex}

\[\lim_{x\to0}f({x + 5})g(x)=\answer{4}\]
\end{problem}}%}

\latexProblemContent{
\ifVerboseLocation This is Derivative Concept Question 0001. \\ \fi
\begin{problem}

If you know that $\lim\limits_{x\to{-3}}f(x)={1}$ and $\lim\limits_{x\to0}g(x)={-4}$, then evaluate the following limit:

\input{Limit-Concept-0001.HELP.tex}

\[\lim_{x\to0}f({x - 3})g(x)=\answer{-4}\]
\end{problem}}%}

\latexProblemContent{
\ifVerboseLocation This is Derivative Concept Question 0001. \\ \fi
\begin{problem}

If you know that $\lim\limits_{x\to{-3}}f(x)={-2}$ and $\lim\limits_{x\to0}g(x)={-2}$, then evaluate the following limit:

\input{Limit-Concept-0001.HELP.tex}

\[\lim_{x\to0}f({x - 3})g(x)=\answer{4}\]
\end{problem}}%}

\latexProblemContent{
\ifVerboseLocation This is Derivative Concept Question 0001. \\ \fi
\begin{problem}

If you know that $\lim\limits_{x\to{2}}f(x)={-5}$ and $\lim\limits_{x\to0}g(x)={5}$, then evaluate the following limit:

\input{Limit-Concept-0001.HELP.tex}

\[\lim_{x\to0}f({x + 2})g(x)=\answer{-25}\]
\end{problem}}%}

\latexProblemContent{
\ifVerboseLocation This is Derivative Concept Question 0001. \\ \fi
\begin{problem}

If you know that $\lim\limits_{x\to{-5}}f(x)={-3}$ and $\lim\limits_{x\to0}g(x)={1}$, then evaluate the following limit:

\input{Limit-Concept-0001.HELP.tex}

\[\lim_{x\to0}f({x - 5})g(x)=\answer{-3}\]
\end{problem}}%}

\latexProblemContent{
\ifVerboseLocation This is Derivative Concept Question 0001. \\ \fi
\begin{problem}

If you know that $\lim\limits_{x\to{2}}f(x)={2}$ and $\lim\limits_{x\to0}g(x)={3}$, then evaluate the following limit:

\input{Limit-Concept-0001.HELP.tex}

\[\lim_{x\to0}f({x + 2})g(x)=\answer{6}\]
\end{problem}}%}

\latexProblemContent{
\ifVerboseLocation This is Derivative Concept Question 0001. \\ \fi
\begin{problem}

If you know that $\lim\limits_{x\to{-5}}f(x)={-2}$ and $\lim\limits_{x\to0}g(x)={-4}$, then evaluate the following limit:

\input{Limit-Concept-0001.HELP.tex}

\[\lim_{x\to0}f({x - 5})g(x)=\answer{8}\]
\end{problem}}%}

\latexProblemContent{
\ifVerboseLocation This is Derivative Concept Question 0001. \\ \fi
\begin{problem}

If you know that $\lim\limits_{x\to{-4}}f(x)={4}$ and $\lim\limits_{x\to0}g(x)={-3}$, then evaluate the following limit:

\input{Limit-Concept-0001.HELP.tex}

\[\lim_{x\to0}f({x - 4})g(x)=\answer{-12}\]
\end{problem}}%}

\latexProblemContent{
\ifVerboseLocation This is Derivative Concept Question 0001. \\ \fi
\begin{problem}

If you know that $\lim\limits_{x\to{3}}f(x)={5}$ and $\lim\limits_{x\to0}g(x)={4}$, then evaluate the following limit:

\input{Limit-Concept-0001.HELP.tex}

\[\lim_{x\to0}f({x + 3})g(x)=\answer{20}\]
\end{problem}}%}

\latexProblemContent{
\ifVerboseLocation This is Derivative Concept Question 0001. \\ \fi
\begin{problem}

If you know that $\lim\limits_{x\to{-5}}f(x)={-2}$ and $\lim\limits_{x\to0}g(x)={5}$, then evaluate the following limit:

\input{Limit-Concept-0001.HELP.tex}

\[\lim_{x\to0}f({x - 5})g(x)=\answer{-10}\]
\end{problem}}%}

\latexProblemContent{
\ifVerboseLocation This is Derivative Concept Question 0001. \\ \fi
\begin{problem}

If you know that $\lim\limits_{x\to{1}}f(x)={5}$ and $\lim\limits_{x\to0}g(x)={5}$, then evaluate the following limit:

\input{Limit-Concept-0001.HELP.tex}

\[\lim_{x\to0}f({x + 1})g(x)=\answer{25}\]
\end{problem}}%}

\latexProblemContent{
\ifVerboseLocation This is Derivative Concept Question 0001. \\ \fi
\begin{problem}

If you know that $\lim\limits_{x\to{5}}f(x)={2}$ and $\lim\limits_{x\to0}g(x)={-2}$, then evaluate the following limit:

\input{Limit-Concept-0001.HELP.tex}

\[\lim_{x\to0}f({x + 5})g(x)=\answer{-4}\]
\end{problem}}%}

\latexProblemContent{
\ifVerboseLocation This is Derivative Concept Question 0001. \\ \fi
\begin{problem}

If you know that $\lim\limits_{x\to{1}}f(x)={3}$ and $\lim\limits_{x\to0}g(x)={1}$, then evaluate the following limit:

\input{Limit-Concept-0001.HELP.tex}

\[\lim_{x\to0}f({x + 1})g(x)=\answer{3}\]
\end{problem}}%}

\latexProblemContent{
\ifVerboseLocation This is Derivative Concept Question 0001. \\ \fi
\begin{problem}

If you know that $\lim\limits_{x\to{1}}f(x)={-4}$ and $\lim\limits_{x\to0}g(x)={5}$, then evaluate the following limit:

\input{Limit-Concept-0001.HELP.tex}

\[\lim_{x\to0}f({x + 1})g(x)=\answer{-20}\]
\end{problem}}%}

\latexProblemContent{
\ifVerboseLocation This is Derivative Concept Question 0001. \\ \fi
\begin{problem}

If you know that $\lim\limits_{x\to{4}}f(x)={3}$ and $\lim\limits_{x\to0}g(x)={3}$, then evaluate the following limit:

\input{Limit-Concept-0001.HELP.tex}

\[\lim_{x\to0}f({x + 4})g(x)=\answer{9}\]
\end{problem}}%}

\latexProblemContent{
\ifVerboseLocation This is Derivative Concept Question 0001. \\ \fi
\begin{problem}

If you know that $\lim\limits_{x\to{3}}f(x)={4}$ and $\lim\limits_{x\to0}g(x)={-1}$, then evaluate the following limit:

\input{Limit-Concept-0001.HELP.tex}

\[\lim_{x\to0}f({x + 3})g(x)=\answer{-4}\]
\end{problem}}%}

\latexProblemContent{
\ifVerboseLocation This is Derivative Concept Question 0001. \\ \fi
\begin{problem}

If you know that $\lim\limits_{x\to{-2}}f(x)={-1}$ and $\lim\limits_{x\to0}g(x)={4}$, then evaluate the following limit:

\input{Limit-Concept-0001.HELP.tex}

\[\lim_{x\to0}f({x - 2})g(x)=\answer{-4}\]
\end{problem}}%}

\latexProblemContent{
\ifVerboseLocation This is Derivative Concept Question 0001. \\ \fi
\begin{problem}

If you know that $\lim\limits_{x\to{2}}f(x)={3}$ and $\lim\limits_{x\to0}g(x)={-5}$, then evaluate the following limit:

\input{Limit-Concept-0001.HELP.tex}

\[\lim_{x\to0}f({x + 2})g(x)=\answer{-15}\]
\end{problem}}%}

\latexProblemContent{
\ifVerboseLocation This is Derivative Concept Question 0001. \\ \fi
\begin{problem}

If you know that $\lim\limits_{x\to{3}}f(x)={-1}$ and $\lim\limits_{x\to0}g(x)={3}$, then evaluate the following limit:

\input{Limit-Concept-0001.HELP.tex}

\[\lim_{x\to0}f({x + 3})g(x)=\answer{-3}\]
\end{problem}}%}

\latexProblemContent{
\ifVerboseLocation This is Derivative Concept Question 0001. \\ \fi
\begin{problem}

If you know that $\lim\limits_{x\to{4}}f(x)={5}$ and $\lim\limits_{x\to0}g(x)={-5}$, then evaluate the following limit:

\input{Limit-Concept-0001.HELP.tex}

\[\lim_{x\to0}f({x + 4})g(x)=\answer{-25}\]
\end{problem}}%}

\latexProblemContent{
\ifVerboseLocation This is Derivative Concept Question 0001. \\ \fi
\begin{problem}

If you know that $\lim\limits_{x\to{-1}}f(x)={-4}$ and $\lim\limits_{x\to0}g(x)={1}$, then evaluate the following limit:

\input{Limit-Concept-0001.HELP.tex}

\[\lim_{x\to0}f({x - 1})g(x)=\answer{-4}\]
\end{problem}}%}

\latexProblemContent{
\ifVerboseLocation This is Derivative Concept Question 0001. \\ \fi
\begin{problem}

If you know that $\lim\limits_{x\to{4}}f(x)={3}$ and $\lim\limits_{x\to0}g(x)={-3}$, then evaluate the following limit:

\input{Limit-Concept-0001.HELP.tex}

\[\lim_{x\to0}f({x + 4})g(x)=\answer{-9}\]
\end{problem}}%}

\latexProblemContent{
\ifVerboseLocation This is Derivative Concept Question 0001. \\ \fi
\begin{problem}

If you know that $\lim\limits_{x\to{-1}}f(x)={2}$ and $\lim\limits_{x\to0}g(x)={2}$, then evaluate the following limit:

\input{Limit-Concept-0001.HELP.tex}

\[\lim_{x\to0}f({x - 1})g(x)=\answer{4}\]
\end{problem}}%}

\latexProblemContent{
\ifVerboseLocation This is Derivative Concept Question 0001. \\ \fi
\begin{problem}

If you know that $\lim\limits_{x\to{-5}}f(x)={-3}$ and $\lim\limits_{x\to0}g(x)={3}$, then evaluate the following limit:

\input{Limit-Concept-0001.HELP.tex}

\[\lim_{x\to0}f({x - 5})g(x)=\answer{-9}\]
\end{problem}}%}

\latexProblemContent{
\ifVerboseLocation This is Derivative Concept Question 0001. \\ \fi
\begin{problem}

If you know that $\lim\limits_{x\to{-5}}f(x)={-4}$ and $\lim\limits_{x\to0}g(x)={4}$, then evaluate the following limit:

\input{Limit-Concept-0001.HELP.tex}

\[\lim_{x\to0}f({x - 5})g(x)=\answer{-16}\]
\end{problem}}%}

\latexProblemContent{
\ifVerboseLocation This is Derivative Concept Question 0001. \\ \fi
\begin{problem}

If you know that $\lim\limits_{x\to{-2}}f(x)={-5}$ and $\lim\limits_{x\to0}g(x)={3}$, then evaluate the following limit:

\input{Limit-Concept-0001.HELP.tex}

\[\lim_{x\to0}f({x - 2})g(x)=\answer{-15}\]
\end{problem}}%}

\latexProblemContent{
\ifVerboseLocation This is Derivative Concept Question 0001. \\ \fi
\begin{problem}

If you know that $\lim\limits_{x\to{-3}}f(x)={-2}$ and $\lim\limits_{x\to0}g(x)={4}$, then evaluate the following limit:

\input{Limit-Concept-0001.HELP.tex}

\[\lim_{x\to0}f({x - 3})g(x)=\answer{-8}\]
\end{problem}}%}

\latexProblemContent{
\ifVerboseLocation This is Derivative Concept Question 0001. \\ \fi
\begin{problem}

If you know that $\lim\limits_{x\to{1}}f(x)={-3}$ and $\lim\limits_{x\to0}g(x)={-1}$, then evaluate the following limit:

\input{Limit-Concept-0001.HELP.tex}

\[\lim_{x\to0}f({x + 1})g(x)=\answer{3}\]
\end{problem}}%}

\latexProblemContent{
\ifVerboseLocation This is Derivative Concept Question 0001. \\ \fi
\begin{problem}

If you know that $\lim\limits_{x\to{-1}}f(x)={-1}$ and $\lim\limits_{x\to0}g(x)={2}$, then evaluate the following limit:

\input{Limit-Concept-0001.HELP.tex}

\[\lim_{x\to0}f({x - 1})g(x)=\answer{-2}\]
\end{problem}}%}

\latexProblemContent{
\ifVerboseLocation This is Derivative Concept Question 0001. \\ \fi
\begin{problem}

If you know that $\lim\limits_{x\to{1}}f(x)={5}$ and $\lim\limits_{x\to0}g(x)={3}$, then evaluate the following limit:

\input{Limit-Concept-0001.HELP.tex}

\[\lim_{x\to0}f({x + 1})g(x)=\answer{15}\]
\end{problem}}%}

\latexProblemContent{
\ifVerboseLocation This is Derivative Concept Question 0001. \\ \fi
\begin{problem}

If you know that $\lim\limits_{x\to{-4}}f(x)={-4}$ and $\lim\limits_{x\to0}g(x)={-3}$, then evaluate the following limit:

\input{Limit-Concept-0001.HELP.tex}

\[\lim_{x\to0}f({x - 4})g(x)=\answer{12}\]
\end{problem}}%}

\latexProblemContent{
\ifVerboseLocation This is Derivative Concept Question 0001. \\ \fi
\begin{problem}

If you know that $\lim\limits_{x\to{5}}f(x)={-5}$ and $\lim\limits_{x\to0}g(x)={-2}$, then evaluate the following limit:

\input{Limit-Concept-0001.HELP.tex}

\[\lim_{x\to0}f({x + 5})g(x)=\answer{10}\]
\end{problem}}%}

\latexProblemContent{
\ifVerboseLocation This is Derivative Concept Question 0001. \\ \fi
\begin{problem}

If you know that $\lim\limits_{x\to{-2}}f(x)={2}$ and $\lim\limits_{x\to0}g(x)={5}$, then evaluate the following limit:

\input{Limit-Concept-0001.HELP.tex}

\[\lim_{x\to0}f({x - 2})g(x)=\answer{10}\]
\end{problem}}%}

\latexProblemContent{
\ifVerboseLocation This is Derivative Concept Question 0001. \\ \fi
\begin{problem}

If you know that $\lim\limits_{x\to{4}}f(x)={-1}$ and $\lim\limits_{x\to0}g(x)={-5}$, then evaluate the following limit:

\input{Limit-Concept-0001.HELP.tex}

\[\lim_{x\to0}f({x + 4})g(x)=\answer{5}\]
\end{problem}}%}

\latexProblemContent{
\ifVerboseLocation This is Derivative Concept Question 0001. \\ \fi
\begin{problem}

If you know that $\lim\limits_{x\to{-1}}f(x)={5}$ and $\lim\limits_{x\to0}g(x)={-5}$, then evaluate the following limit:

\input{Limit-Concept-0001.HELP.tex}

\[\lim_{x\to0}f({x - 1})g(x)=\answer{-25}\]
\end{problem}}%}

\latexProblemContent{
\ifVerboseLocation This is Derivative Concept Question 0001. \\ \fi
\begin{problem}

If you know that $\lim\limits_{x\to{1}}f(x)={-4}$ and $\lim\limits_{x\to0}g(x)={-1}$, then evaluate the following limit:

\input{Limit-Concept-0001.HELP.tex}

\[\lim_{x\to0}f({x + 1})g(x)=\answer{4}\]
\end{problem}}%}

\latexProblemContent{
\ifVerboseLocation This is Derivative Concept Question 0001. \\ \fi
\begin{problem}

If you know that $\lim\limits_{x\to{5}}f(x)={5}$ and $\lim\limits_{x\to0}g(x)={2}$, then evaluate the following limit:

\input{Limit-Concept-0001.HELP.tex}

\[\lim_{x\to0}f({x + 5})g(x)=\answer{10}\]
\end{problem}}%}

\latexProblemContent{
\ifVerboseLocation This is Derivative Concept Question 0001. \\ \fi
\begin{problem}

If you know that $\lim\limits_{x\to{-3}}f(x)={5}$ and $\lim\limits_{x\to0}g(x)={-3}$, then evaluate the following limit:

\input{Limit-Concept-0001.HELP.tex}

\[\lim_{x\to0}f({x - 3})g(x)=\answer{-15}\]
\end{problem}}%}

\latexProblemContent{
\ifVerboseLocation This is Derivative Concept Question 0001. \\ \fi
\begin{problem}

If you know that $\lim\limits_{x\to{-3}}f(x)={1}$ and $\lim\limits_{x\to0}g(x)={-5}$, then evaluate the following limit:

\input{Limit-Concept-0001.HELP.tex}

\[\lim_{x\to0}f({x - 3})g(x)=\answer{-5}\]
\end{problem}}%}

\latexProblemContent{
\ifVerboseLocation This is Derivative Concept Question 0001. \\ \fi
\begin{problem}

If you know that $\lim\limits_{x\to{4}}f(x)={-2}$ and $\lim\limits_{x\to0}g(x)={1}$, then evaluate the following limit:

\input{Limit-Concept-0001.HELP.tex}

\[\lim_{x\to0}f({x + 4})g(x)=\answer{-2}\]
\end{problem}}%}

\latexProblemContent{
\ifVerboseLocation This is Derivative Concept Question 0001. \\ \fi
\begin{problem}

If you know that $\lim\limits_{x\to{5}}f(x)={3}$ and $\lim\limits_{x\to0}g(x)={1}$, then evaluate the following limit:

\input{Limit-Concept-0001.HELP.tex}

\[\lim_{x\to0}f({x + 5})g(x)=\answer{3}\]
\end{problem}}%}

\latexProblemContent{
\ifVerboseLocation This is Derivative Concept Question 0001. \\ \fi
\begin{problem}

If you know that $\lim\limits_{x\to{-4}}f(x)={-4}$ and $\lim\limits_{x\to0}g(x)={4}$, then evaluate the following limit:

\input{Limit-Concept-0001.HELP.tex}

\[\lim_{x\to0}f({x - 4})g(x)=\answer{-16}\]
\end{problem}}%}

\latexProblemContent{
\ifVerboseLocation This is Derivative Concept Question 0001. \\ \fi
\begin{problem}

If you know that $\lim\limits_{x\to{4}}f(x)={4}$ and $\lim\limits_{x\to0}g(x)={5}$, then evaluate the following limit:

\input{Limit-Concept-0001.HELP.tex}

\[\lim_{x\to0}f({x + 4})g(x)=\answer{20}\]
\end{problem}}%}

\latexProblemContent{
\ifVerboseLocation This is Derivative Concept Question 0001. \\ \fi
\begin{problem}

If you know that $\lim\limits_{x\to{5}}f(x)={4}$ and $\lim\limits_{x\to0}g(x)={2}$, then evaluate the following limit:

\input{Limit-Concept-0001.HELP.tex}

\[\lim_{x\to0}f({x + 5})g(x)=\answer{8}\]
\end{problem}}%}

\latexProblemContent{
\ifVerboseLocation This is Derivative Concept Question 0001. \\ \fi
\begin{problem}

If you know that $\lim\limits_{x\to{4}}f(x)={-3}$ and $\lim\limits_{x\to0}g(x)={3}$, then evaluate the following limit:

\input{Limit-Concept-0001.HELP.tex}

\[\lim_{x\to0}f({x + 4})g(x)=\answer{-9}\]
\end{problem}}%}

\latexProblemContent{
\ifVerboseLocation This is Derivative Concept Question 0001. \\ \fi
\begin{problem}

If you know that $\lim\limits_{x\to{3}}f(x)={3}$ and $\lim\limits_{x\to0}g(x)={3}$, then evaluate the following limit:

\input{Limit-Concept-0001.HELP.tex}

\[\lim_{x\to0}f({x + 3})g(x)=\answer{9}\]
\end{problem}}%}

\latexProblemContent{
\ifVerboseLocation This is Derivative Concept Question 0001. \\ \fi
\begin{problem}

If you know that $\lim\limits_{x\to{-1}}f(x)={2}$ and $\lim\limits_{x\to0}g(x)={-4}$, then evaluate the following limit:

\input{Limit-Concept-0001.HELP.tex}

\[\lim_{x\to0}f({x - 1})g(x)=\answer{-8}\]
\end{problem}}%}

\latexProblemContent{
\ifVerboseLocation This is Derivative Concept Question 0001. \\ \fi
\begin{problem}

If you know that $\lim\limits_{x\to{3}}f(x)={3}$ and $\lim\limits_{x\to0}g(x)={-4}$, then evaluate the following limit:

\input{Limit-Concept-0001.HELP.tex}

\[\lim_{x\to0}f({x + 3})g(x)=\answer{-12}\]
\end{problem}}%}

\latexProblemContent{
\ifVerboseLocation This is Derivative Concept Question 0001. \\ \fi
\begin{problem}

If you know that $\lim\limits_{x\to{-5}}f(x)={5}$ and $\lim\limits_{x\to0}g(x)={2}$, then evaluate the following limit:

\input{Limit-Concept-0001.HELP.tex}

\[\lim_{x\to0}f({x - 5})g(x)=\answer{10}\]
\end{problem}}%}

\latexProblemContent{
\ifVerboseLocation This is Derivative Concept Question 0001. \\ \fi
\begin{problem}

If you know that $\lim\limits_{x\to{-2}}f(x)={-3}$ and $\lim\limits_{x\to0}g(x)={-3}$, then evaluate the following limit:

\input{Limit-Concept-0001.HELP.tex}

\[\lim_{x\to0}f({x - 2})g(x)=\answer{9}\]
\end{problem}}%}

\latexProblemContent{
\ifVerboseLocation This is Derivative Concept Question 0001. \\ \fi
\begin{problem}

If you know that $\lim\limits_{x\to{2}}f(x)={-1}$ and $\lim\limits_{x\to0}g(x)={2}$, then evaluate the following limit:

\input{Limit-Concept-0001.HELP.tex}

\[\lim_{x\to0}f({x + 2})g(x)=\answer{-2}\]
\end{problem}}%}

\latexProblemContent{
\ifVerboseLocation This is Derivative Concept Question 0001. \\ \fi
\begin{problem}

If you know that $\lim\limits_{x\to{3}}f(x)={1}$ and $\lim\limits_{x\to0}g(x)={3}$, then evaluate the following limit:

\input{Limit-Concept-0001.HELP.tex}

\[\lim_{x\to0}f({x + 3})g(x)=\answer{3}\]
\end{problem}}%}

\latexProblemContent{
\ifVerboseLocation This is Derivative Concept Question 0001. \\ \fi
\begin{problem}

If you know that $\lim\limits_{x\to{1}}f(x)={4}$ and $\lim\limits_{x\to0}g(x)={1}$, then evaluate the following limit:

\input{Limit-Concept-0001.HELP.tex}

\[\lim_{x\to0}f({x + 1})g(x)=\answer{4}\]
\end{problem}}%}

\latexProblemContent{
\ifVerboseLocation This is Derivative Concept Question 0001. \\ \fi
\begin{problem}

If you know that $\lim\limits_{x\to{3}}f(x)={-1}$ and $\lim\limits_{x\to0}g(x)={5}$, then evaluate the following limit:

\input{Limit-Concept-0001.HELP.tex}

\[\lim_{x\to0}f({x + 3})g(x)=\answer{-5}\]
\end{problem}}%}

\latexProblemContent{
\ifVerboseLocation This is Derivative Concept Question 0001. \\ \fi
\begin{problem}

If you know that $\lim\limits_{x\to{3}}f(x)={-1}$ and $\lim\limits_{x\to0}g(x)={4}$, then evaluate the following limit:

\input{Limit-Concept-0001.HELP.tex}

\[\lim_{x\to0}f({x + 3})g(x)=\answer{-4}\]
\end{problem}}%}

\latexProblemContent{
\ifVerboseLocation This is Derivative Concept Question 0001. \\ \fi
\begin{problem}

If you know that $\lim\limits_{x\to{2}}f(x)={-4}$ and $\lim\limits_{x\to0}g(x)={4}$, then evaluate the following limit:

\input{Limit-Concept-0001.HELP.tex}

\[\lim_{x\to0}f({x + 2})g(x)=\answer{-16}\]
\end{problem}}%}

\latexProblemContent{
\ifVerboseLocation This is Derivative Concept Question 0001. \\ \fi
\begin{problem}

If you know that $\lim\limits_{x\to{3}}f(x)={2}$ and $\lim\limits_{x\to0}g(x)={5}$, then evaluate the following limit:

\input{Limit-Concept-0001.HELP.tex}

\[\lim_{x\to0}f({x + 3})g(x)=\answer{10}\]
\end{problem}}%}

\latexProblemContent{
\ifVerboseLocation This is Derivative Concept Question 0001. \\ \fi
\begin{problem}

If you know that $\lim\limits_{x\to{2}}f(x)={-1}$ and $\lim\limits_{x\to0}g(x)={1}$, then evaluate the following limit:

\input{Limit-Concept-0001.HELP.tex}

\[\lim_{x\to0}f({x + 2})g(x)=\answer{-1}\]
\end{problem}}%}

\latexProblemContent{
\ifVerboseLocation This is Derivative Concept Question 0001. \\ \fi
\begin{problem}

If you know that $\lim\limits_{x\to{5}}f(x)={-2}$ and $\lim\limits_{x\to0}g(x)={2}$, then evaluate the following limit:

\input{Limit-Concept-0001.HELP.tex}

\[\lim_{x\to0}f({x + 5})g(x)=\answer{-4}\]
\end{problem}}%}

\latexProblemContent{
\ifVerboseLocation This is Derivative Concept Question 0001. \\ \fi
\begin{problem}

If you know that $\lim\limits_{x\to{-3}}f(x)={1}$ and $\lim\limits_{x\to0}g(x)={-2}$, then evaluate the following limit:

\input{Limit-Concept-0001.HELP.tex}

\[\lim_{x\to0}f({x - 3})g(x)=\answer{-2}\]
\end{problem}}%}

\latexProblemContent{
\ifVerboseLocation This is Derivative Concept Question 0001. \\ \fi
\begin{problem}

If you know that $\lim\limits_{x\to{1}}f(x)={3}$ and $\lim\limits_{x\to0}g(x)={-4}$, then evaluate the following limit:

\input{Limit-Concept-0001.HELP.tex}

\[\lim_{x\to0}f({x + 1})g(x)=\answer{-12}\]
\end{problem}}%}

\latexProblemContent{
\ifVerboseLocation This is Derivative Concept Question 0001. \\ \fi
\begin{problem}

If you know that $\lim\limits_{x\to{-5}}f(x)={1}$ and $\lim\limits_{x\to0}g(x)={5}$, then evaluate the following limit:

\input{Limit-Concept-0001.HELP.tex}

\[\lim_{x\to0}f({x - 5})g(x)=\answer{5}\]
\end{problem}}%}

\latexProblemContent{
\ifVerboseLocation This is Derivative Concept Question 0001. \\ \fi
\begin{problem}

If you know that $\lim\limits_{x\to{3}}f(x)={-1}$ and $\lim\limits_{x\to0}g(x)={-4}$, then evaluate the following limit:

\input{Limit-Concept-0001.HELP.tex}

\[\lim_{x\to0}f({x + 3})g(x)=\answer{4}\]
\end{problem}}%}

\latexProblemContent{
\ifVerboseLocation This is Derivative Concept Question 0001. \\ \fi
\begin{problem}

If you know that $\lim\limits_{x\to{4}}f(x)={-1}$ and $\lim\limits_{x\to0}g(x)={-1}$, then evaluate the following limit:

\input{Limit-Concept-0001.HELP.tex}

\[\lim_{x\to0}f({x + 4})g(x)=\answer{1}\]
\end{problem}}%}

\latexProblemContent{
\ifVerboseLocation This is Derivative Concept Question 0001. \\ \fi
\begin{problem}

If you know that $\lim\limits_{x\to{-2}}f(x)={-5}$ and $\lim\limits_{x\to0}g(x)={4}$, then evaluate the following limit:

\input{Limit-Concept-0001.HELP.tex}

\[\lim_{x\to0}f({x - 2})g(x)=\answer{-20}\]
\end{problem}}%}

\latexProblemContent{
\ifVerboseLocation This is Derivative Concept Question 0001. \\ \fi
\begin{problem}

If you know that $\lim\limits_{x\to{5}}f(x)={5}$ and $\lim\limits_{x\to0}g(x)={-4}$, then evaluate the following limit:

\input{Limit-Concept-0001.HELP.tex}

\[\lim_{x\to0}f({x + 5})g(x)=\answer{-20}\]
\end{problem}}%}

\latexProblemContent{
\ifVerboseLocation This is Derivative Concept Question 0001. \\ \fi
\begin{problem}

If you know that $\lim\limits_{x\to{2}}f(x)={-1}$ and $\lim\limits_{x\to0}g(x)={3}$, then evaluate the following limit:

\input{Limit-Concept-0001.HELP.tex}

\[\lim_{x\to0}f({x + 2})g(x)=\answer{-3}\]
\end{problem}}%}

\latexProblemContent{
\ifVerboseLocation This is Derivative Concept Question 0001. \\ \fi
\begin{problem}

If you know that $\lim\limits_{x\to{-2}}f(x)={5}$ and $\lim\limits_{x\to0}g(x)={4}$, then evaluate the following limit:

\input{Limit-Concept-0001.HELP.tex}

\[\lim_{x\to0}f({x - 2})g(x)=\answer{20}\]
\end{problem}}%}

\latexProblemContent{
\ifVerboseLocation This is Derivative Concept Question 0001. \\ \fi
\begin{problem}

If you know that $\lim\limits_{x\to{-5}}f(x)={2}$ and $\lim\limits_{x\to0}g(x)={3}$, then evaluate the following limit:

\input{Limit-Concept-0001.HELP.tex}

\[\lim_{x\to0}f({x - 5})g(x)=\answer{6}\]
\end{problem}}%}

\latexProblemContent{
\ifVerboseLocation This is Derivative Concept Question 0001. \\ \fi
\begin{problem}

If you know that $\lim\limits_{x\to{3}}f(x)={3}$ and $\lim\limits_{x\to0}g(x)={2}$, then evaluate the following limit:

\input{Limit-Concept-0001.HELP.tex}

\[\lim_{x\to0}f({x + 3})g(x)=\answer{6}\]
\end{problem}}%}

\latexProblemContent{
\ifVerboseLocation This is Derivative Concept Question 0001. \\ \fi
\begin{problem}

If you know that $\lim\limits_{x\to{-3}}f(x)={-4}$ and $\lim\limits_{x\to0}g(x)={1}$, then evaluate the following limit:

\input{Limit-Concept-0001.HELP.tex}

\[\lim_{x\to0}f({x - 3})g(x)=\answer{-4}\]
\end{problem}}%}

\latexProblemContent{
\ifVerboseLocation This is Derivative Concept Question 0001. \\ \fi
\begin{problem}

If you know that $\lim\limits_{x\to{-5}}f(x)={1}$ and $\lim\limits_{x\to0}g(x)={-5}$, then evaluate the following limit:

\input{Limit-Concept-0001.HELP.tex}

\[\lim_{x\to0}f({x - 5})g(x)=\answer{-5}\]
\end{problem}}%}

\latexProblemContent{
\ifVerboseLocation This is Derivative Concept Question 0001. \\ \fi
\begin{problem}

If you know that $\lim\limits_{x\to{-3}}f(x)={3}$ and $\lim\limits_{x\to0}g(x)={2}$, then evaluate the following limit:

\input{Limit-Concept-0001.HELP.tex}

\[\lim_{x\to0}f({x - 3})g(x)=\answer{6}\]
\end{problem}}%}

\latexProblemContent{
\ifVerboseLocation This is Derivative Concept Question 0001. \\ \fi
\begin{problem}

If you know that $\lim\limits_{x\to{-3}}f(x)={5}$ and $\lim\limits_{x\to0}g(x)={3}$, then evaluate the following limit:

\input{Limit-Concept-0001.HELP.tex}

\[\lim_{x\to0}f({x - 3})g(x)=\answer{15}\]
\end{problem}}%}

\latexProblemContent{
\ifVerboseLocation This is Derivative Concept Question 0001. \\ \fi
\begin{problem}

If you know that $\lim\limits_{x\to{1}}f(x)={1}$ and $\lim\limits_{x\to0}g(x)={-4}$, then evaluate the following limit:

\input{Limit-Concept-0001.HELP.tex}

\[\lim_{x\to0}f({x + 1})g(x)=\answer{-4}\]
\end{problem}}%}

\latexProblemContent{
\ifVerboseLocation This is Derivative Concept Question 0001. \\ \fi
\begin{problem}

If you know that $\lim\limits_{x\to{4}}f(x)={-5}$ and $\lim\limits_{x\to0}g(x)={3}$, then evaluate the following limit:

\input{Limit-Concept-0001.HELP.tex}

\[\lim_{x\to0}f({x + 4})g(x)=\answer{-15}\]
\end{problem}}%}

\latexProblemContent{
\ifVerboseLocation This is Derivative Concept Question 0001. \\ \fi
\begin{problem}

If you know that $\lim\limits_{x\to{-2}}f(x)={4}$ and $\lim\limits_{x\to0}g(x)={2}$, then evaluate the following limit:

\input{Limit-Concept-0001.HELP.tex}

\[\lim_{x\to0}f({x - 2})g(x)=\answer{8}\]
\end{problem}}%}

\latexProblemContent{
\ifVerboseLocation This is Derivative Concept Question 0001. \\ \fi
\begin{problem}

If you know that $\lim\limits_{x\to{-4}}f(x)={3}$ and $\lim\limits_{x\to0}g(x)={-2}$, then evaluate the following limit:

\input{Limit-Concept-0001.HELP.tex}

\[\lim_{x\to0}f({x - 4})g(x)=\answer{-6}\]
\end{problem}}%}

\latexProblemContent{
\ifVerboseLocation This is Derivative Concept Question 0001. \\ \fi
\begin{problem}

If you know that $\lim\limits_{x\to{-5}}f(x)={3}$ and $\lim\limits_{x\to0}g(x)={3}$, then evaluate the following limit:

\input{Limit-Concept-0001.HELP.tex}

\[\lim_{x\to0}f({x - 5})g(x)=\answer{9}\]
\end{problem}}%}

\latexProblemContent{
\ifVerboseLocation This is Derivative Concept Question 0001. \\ \fi
\begin{problem}

If you know that $\lim\limits_{x\to{1}}f(x)={3}$ and $\lim\limits_{x\to0}g(x)={-1}$, then evaluate the following limit:

\input{Limit-Concept-0001.HELP.tex}

\[\lim_{x\to0}f({x + 1})g(x)=\answer{-3}\]
\end{problem}}%}

\latexProblemContent{
\ifVerboseLocation This is Derivative Concept Question 0001. \\ \fi
\begin{problem}

If you know that $\lim\limits_{x\to{5}}f(x)={-3}$ and $\lim\limits_{x\to0}g(x)={3}$, then evaluate the following limit:

\input{Limit-Concept-0001.HELP.tex}

\[\lim_{x\to0}f({x + 5})g(x)=\answer{-9}\]
\end{problem}}%}

\latexProblemContent{
\ifVerboseLocation This is Derivative Concept Question 0001. \\ \fi
\begin{problem}

If you know that $\lim\limits_{x\to{-2}}f(x)={2}$ and $\lim\limits_{x\to0}g(x)={-2}$, then evaluate the following limit:

\input{Limit-Concept-0001.HELP.tex}

\[\lim_{x\to0}f({x - 2})g(x)=\answer{-4}\]
\end{problem}}%}

\latexProblemContent{
\ifVerboseLocation This is Derivative Concept Question 0001. \\ \fi
\begin{problem}

If you know that $\lim\limits_{x\to{3}}f(x)={4}$ and $\lim\limits_{x\to0}g(x)={1}$, then evaluate the following limit:

\input{Limit-Concept-0001.HELP.tex}

\[\lim_{x\to0}f({x + 3})g(x)=\answer{4}\]
\end{problem}}%}

\latexProblemContent{
\ifVerboseLocation This is Derivative Concept Question 0001. \\ \fi
\begin{problem}

If you know that $\lim\limits_{x\to{-5}}f(x)={-1}$ and $\lim\limits_{x\to0}g(x)={2}$, then evaluate the following limit:

\input{Limit-Concept-0001.HELP.tex}

\[\lim_{x\to0}f({x - 5})g(x)=\answer{-2}\]
\end{problem}}%}

\latexProblemContent{
\ifVerboseLocation This is Derivative Concept Question 0001. \\ \fi
\begin{problem}

If you know that $\lim\limits_{x\to{4}}f(x)={-4}$ and $\lim\limits_{x\to0}g(x)={-4}$, then evaluate the following limit:

\input{Limit-Concept-0001.HELP.tex}

\[\lim_{x\to0}f({x + 4})g(x)=\answer{16}\]
\end{problem}}%}

\latexProblemContent{
\ifVerboseLocation This is Derivative Concept Question 0001. \\ \fi
\begin{problem}

If you know that $\lim\limits_{x\to{-3}}f(x)={1}$ and $\lim\limits_{x\to0}g(x)={3}$, then evaluate the following limit:

\input{Limit-Concept-0001.HELP.tex}

\[\lim_{x\to0}f({x - 3})g(x)=\answer{3}\]
\end{problem}}%}

\latexProblemContent{
\ifVerboseLocation This is Derivative Concept Question 0001. \\ \fi
\begin{problem}

If you know that $\lim\limits_{x\to{-1}}f(x)={2}$ and $\lim\limits_{x\to0}g(x)={3}$, then evaluate the following limit:

\input{Limit-Concept-0001.HELP.tex}

\[\lim_{x\to0}f({x - 1})g(x)=\answer{6}\]
\end{problem}}%}

\latexProblemContent{
\ifVerboseLocation This is Derivative Concept Question 0001. \\ \fi
\begin{problem}

If you know that $\lim\limits_{x\to{3}}f(x)={-4}$ and $\lim\limits_{x\to0}g(x)={4}$, then evaluate the following limit:

\input{Limit-Concept-0001.HELP.tex}

\[\lim_{x\to0}f({x + 3})g(x)=\answer{-16}\]
\end{problem}}%}

\latexProblemContent{
\ifVerboseLocation This is Derivative Concept Question 0001. \\ \fi
\begin{problem}

If you know that $\lim\limits_{x\to{-4}}f(x)={-3}$ and $\lim\limits_{x\to0}g(x)={-2}$, then evaluate the following limit:

\input{Limit-Concept-0001.HELP.tex}

\[\lim_{x\to0}f({x - 4})g(x)=\answer{6}\]
\end{problem}}%}

\latexProblemContent{
\ifVerboseLocation This is Derivative Concept Question 0001. \\ \fi
\begin{problem}

If you know that $\lim\limits_{x\to{-4}}f(x)={2}$ and $\lim\limits_{x\to0}g(x)={-3}$, then evaluate the following limit:

\input{Limit-Concept-0001.HELP.tex}

\[\lim_{x\to0}f({x - 4})g(x)=\answer{-6}\]
\end{problem}}%}

\latexProblemContent{
\ifVerboseLocation This is Derivative Concept Question 0001. \\ \fi
\begin{problem}

If you know that $\lim\limits_{x\to{1}}f(x)={-4}$ and $\lim\limits_{x\to0}g(x)={-4}$, then evaluate the following limit:

\input{Limit-Concept-0001.HELP.tex}

\[\lim_{x\to0}f({x + 1})g(x)=\answer{16}\]
\end{problem}}%}

\latexProblemContent{
\ifVerboseLocation This is Derivative Concept Question 0001. \\ \fi
\begin{problem}

If you know that $\lim\limits_{x\to{4}}f(x)={3}$ and $\lim\limits_{x\to0}g(x)={2}$, then evaluate the following limit:

\input{Limit-Concept-0001.HELP.tex}

\[\lim_{x\to0}f({x + 4})g(x)=\answer{6}\]
\end{problem}}%}

\latexProblemContent{
\ifVerboseLocation This is Derivative Concept Question 0001. \\ \fi
\begin{problem}

If you know that $\lim\limits_{x\to{-3}}f(x)={3}$ and $\lim\limits_{x\to0}g(x)={-3}$, then evaluate the following limit:

\input{Limit-Concept-0001.HELP.tex}

\[\lim_{x\to0}f({x - 3})g(x)=\answer{-9}\]
\end{problem}}%}

\latexProblemContent{
\ifVerboseLocation This is Derivative Concept Question 0001. \\ \fi
\begin{problem}

If you know that $\lim\limits_{x\to{2}}f(x)={5}$ and $\lim\limits_{x\to0}g(x)={-3}$, then evaluate the following limit:

\input{Limit-Concept-0001.HELP.tex}

\[\lim_{x\to0}f({x + 2})g(x)=\answer{-15}\]
\end{problem}}%}

\latexProblemContent{
\ifVerboseLocation This is Derivative Concept Question 0001. \\ \fi
\begin{problem}

If you know that $\lim\limits_{x\to{3}}f(x)={1}$ and $\lim\limits_{x\to0}g(x)={-4}$, then evaluate the following limit:

\input{Limit-Concept-0001.HELP.tex}

\[\lim_{x\to0}f({x + 3})g(x)=\answer{-4}\]
\end{problem}}%}

\latexProblemContent{
\ifVerboseLocation This is Derivative Concept Question 0001. \\ \fi
\begin{problem}

If you know that $\lim\limits_{x\to{3}}f(x)={4}$ and $\lim\limits_{x\to0}g(x)={-5}$, then evaluate the following limit:

\input{Limit-Concept-0001.HELP.tex}

\[\lim_{x\to0}f({x + 3})g(x)=\answer{-20}\]
\end{problem}}%}

\latexProblemContent{
\ifVerboseLocation This is Derivative Concept Question 0001. \\ \fi
\begin{problem}

If you know that $\lim\limits_{x\to{5}}f(x)={4}$ and $\lim\limits_{x\to0}g(x)={1}$, then evaluate the following limit:

\input{Limit-Concept-0001.HELP.tex}

\[\lim_{x\to0}f({x + 5})g(x)=\answer{4}\]
\end{problem}}%}

\latexProblemContent{
\ifVerboseLocation This is Derivative Concept Question 0001. \\ \fi
\begin{problem}

If you know that $\lim\limits_{x\to{1}}f(x)={-2}$ and $\lim\limits_{x\to0}g(x)={-4}$, then evaluate the following limit:

\input{Limit-Concept-0001.HELP.tex}

\[\lim_{x\to0}f({x + 1})g(x)=\answer{8}\]
\end{problem}}%}

\latexProblemContent{
\ifVerboseLocation This is Derivative Concept Question 0001. \\ \fi
\begin{problem}

If you know that $\lim\limits_{x\to{-3}}f(x)={-5}$ and $\lim\limits_{x\to0}g(x)={-5}$, then evaluate the following limit:

\input{Limit-Concept-0001.HELP.tex}

\[\lim_{x\to0}f({x - 3})g(x)=\answer{25}\]
\end{problem}}%}

\latexProblemContent{
\ifVerboseLocation This is Derivative Concept Question 0001. \\ \fi
\begin{problem}

If you know that $\lim\limits_{x\to{-3}}f(x)={1}$ and $\lim\limits_{x\to0}g(x)={5}$, then evaluate the following limit:

\input{Limit-Concept-0001.HELP.tex}

\[\lim_{x\to0}f({x - 3})g(x)=\answer{5}\]
\end{problem}}%}

\latexProblemContent{
\ifVerboseLocation This is Derivative Concept Question 0001. \\ \fi
\begin{problem}

If you know that $\lim\limits_{x\to{2}}f(x)={2}$ and $\lim\limits_{x\to0}g(x)={5}$, then evaluate the following limit:

\input{Limit-Concept-0001.HELP.tex}

\[\lim_{x\to0}f({x + 2})g(x)=\answer{10}\]
\end{problem}}%}

\latexProblemContent{
\ifVerboseLocation This is Derivative Concept Question 0001. \\ \fi
\begin{problem}

If you know that $\lim\limits_{x\to{-2}}f(x)={-1}$ and $\lim\limits_{x\to0}g(x)={-4}$, then evaluate the following limit:

\input{Limit-Concept-0001.HELP.tex}

\[\lim_{x\to0}f({x - 2})g(x)=\answer{4}\]
\end{problem}}%}

\latexProblemContent{
\ifVerboseLocation This is Derivative Concept Question 0001. \\ \fi
\begin{problem}

If you know that $\lim\limits_{x\to{2}}f(x)={-2}$ and $\lim\limits_{x\to0}g(x)={3}$, then evaluate the following limit:

\input{Limit-Concept-0001.HELP.tex}

\[\lim_{x\to0}f({x + 2})g(x)=\answer{-6}\]
\end{problem}}%}

\latexProblemContent{
\ifVerboseLocation This is Derivative Concept Question 0001. \\ \fi
\begin{problem}

If you know that $\lim\limits_{x\to{5}}f(x)={-5}$ and $\lim\limits_{x\to0}g(x)={-3}$, then evaluate the following limit:

\input{Limit-Concept-0001.HELP.tex}

\[\lim_{x\to0}f({x + 5})g(x)=\answer{15}\]
\end{problem}}%}

\latexProblemContent{
\ifVerboseLocation This is Derivative Concept Question 0001. \\ \fi
\begin{problem}

If you know that $\lim\limits_{x\to{-4}}f(x)={-5}$ and $\lim\limits_{x\to0}g(x)={5}$, then evaluate the following limit:

\input{Limit-Concept-0001.HELP.tex}

\[\lim_{x\to0}f({x - 4})g(x)=\answer{-25}\]
\end{problem}}%}

\latexProblemContent{
\ifVerboseLocation This is Derivative Concept Question 0001. \\ \fi
\begin{problem}

If you know that $\lim\limits_{x\to{1}}f(x)={-2}$ and $\lim\limits_{x\to0}g(x)={-1}$, then evaluate the following limit:

\input{Limit-Concept-0001.HELP.tex}

\[\lim_{x\to0}f({x + 1})g(x)=\answer{2}\]
\end{problem}}%}

\latexProblemContent{
\ifVerboseLocation This is Derivative Concept Question 0001. \\ \fi
\begin{problem}

If you know that $\lim\limits_{x\to{3}}f(x)={3}$ and $\lim\limits_{x\to0}g(x)={5}$, then evaluate the following limit:

\input{Limit-Concept-0001.HELP.tex}

\[\lim_{x\to0}f({x + 3})g(x)=\answer{15}\]
\end{problem}}%}

\latexProblemContent{
\ifVerboseLocation This is Derivative Concept Question 0001. \\ \fi
\begin{problem}

If you know that $\lim\limits_{x\to{1}}f(x)={-3}$ and $\lim\limits_{x\to0}g(x)={2}$, then evaluate the following limit:

\input{Limit-Concept-0001.HELP.tex}

\[\lim_{x\to0}f({x + 1})g(x)=\answer{-6}\]
\end{problem}}%}

\latexProblemContent{
\ifVerboseLocation This is Derivative Concept Question 0001. \\ \fi
\begin{problem}

If you know that $\lim\limits_{x\to{3}}f(x)={-3}$ and $\lim\limits_{x\to0}g(x)={1}$, then evaluate the following limit:

\input{Limit-Concept-0001.HELP.tex}

\[\lim_{x\to0}f({x + 3})g(x)=\answer{-3}\]
\end{problem}}%}

\latexProblemContent{
\ifVerboseLocation This is Derivative Concept Question 0001. \\ \fi
\begin{problem}

If you know that $\lim\limits_{x\to{3}}f(x)={-3}$ and $\lim\limits_{x\to0}g(x)={3}$, then evaluate the following limit:

\input{Limit-Concept-0001.HELP.tex}

\[\lim_{x\to0}f({x + 3})g(x)=\answer{-9}\]
\end{problem}}%}

\latexProblemContent{
\ifVerboseLocation This is Derivative Concept Question 0001. \\ \fi
\begin{problem}

If you know that $\lim\limits_{x\to{-1}}f(x)={3}$ and $\lim\limits_{x\to0}g(x)={4}$, then evaluate the following limit:

\input{Limit-Concept-0001.HELP.tex}

\[\lim_{x\to0}f({x - 1})g(x)=\answer{12}\]
\end{problem}}%}

\latexProblemContent{
\ifVerboseLocation This is Derivative Concept Question 0001. \\ \fi
\begin{problem}

If you know that $\lim\limits_{x\to{-1}}f(x)={1}$ and $\lim\limits_{x\to0}g(x)={-5}$, then evaluate the following limit:

\input{Limit-Concept-0001.HELP.tex}

\[\lim_{x\to0}f({x - 1})g(x)=\answer{-5}\]
\end{problem}}%}

\latexProblemContent{
\ifVerboseLocation This is Derivative Concept Question 0001. \\ \fi
\begin{problem}

If you know that $\lim\limits_{x\to{-4}}f(x)={1}$ and $\lim\limits_{x\to0}g(x)={-3}$, then evaluate the following limit:

\input{Limit-Concept-0001.HELP.tex}

\[\lim_{x\to0}f({x - 4})g(x)=\answer{-3}\]
\end{problem}}%}

\latexProblemContent{
\ifVerboseLocation This is Derivative Concept Question 0001. \\ \fi
\begin{problem}

If you know that $\lim\limits_{x\to{4}}f(x)={2}$ and $\lim\limits_{x\to0}g(x)={-2}$, then evaluate the following limit:

\input{Limit-Concept-0001.HELP.tex}

\[\lim_{x\to0}f({x + 4})g(x)=\answer{-4}\]
\end{problem}}%}

\latexProblemContent{
\ifVerboseLocation This is Derivative Concept Question 0001. \\ \fi
\begin{problem}

If you know that $\lim\limits_{x\to{-2}}f(x)={-3}$ and $\lim\limits_{x\to0}g(x)={-1}$, then evaluate the following limit:

\input{Limit-Concept-0001.HELP.tex}

\[\lim_{x\to0}f({x - 2})g(x)=\answer{3}\]
\end{problem}}%}

\latexProblemContent{
\ifVerboseLocation This is Derivative Concept Question 0001. \\ \fi
\begin{problem}

If you know that $\lim\limits_{x\to{5}}f(x)={-4}$ and $\lim\limits_{x\to0}g(x)={2}$, then evaluate the following limit:

\input{Limit-Concept-0001.HELP.tex}

\[\lim_{x\to0}f({x + 5})g(x)=\answer{-8}\]
\end{problem}}%}

\latexProblemContent{
\ifVerboseLocation This is Derivative Concept Question 0001. \\ \fi
\begin{problem}

If you know that $\lim\limits_{x\to{-2}}f(x)={5}$ and $\lim\limits_{x\to0}g(x)={3}$, then evaluate the following limit:

\input{Limit-Concept-0001.HELP.tex}

\[\lim_{x\to0}f({x - 2})g(x)=\answer{15}\]
\end{problem}}%}

\latexProblemContent{
\ifVerboseLocation This is Derivative Concept Question 0001. \\ \fi
\begin{problem}

If you know that $\lim\limits_{x\to{1}}f(x)={2}$ and $\lim\limits_{x\to0}g(x)={3}$, then evaluate the following limit:

\input{Limit-Concept-0001.HELP.tex}

\[\lim_{x\to0}f({x + 1})g(x)=\answer{6}\]
\end{problem}}%}

\latexProblemContent{
\ifVerboseLocation This is Derivative Concept Question 0001. \\ \fi
\begin{problem}

If you know that $\lim\limits_{x\to{-3}}f(x)={-2}$ and $\lim\limits_{x\to0}g(x)={3}$, then evaluate the following limit:

\input{Limit-Concept-0001.HELP.tex}

\[\lim_{x\to0}f({x - 3})g(x)=\answer{-6}\]
\end{problem}}%}

\latexProblemContent{
\ifVerboseLocation This is Derivative Concept Question 0001. \\ \fi
\begin{problem}

If you know that $\lim\limits_{x\to{-5}}f(x)={5}$ and $\lim\limits_{x\to0}g(x)={-2}$, then evaluate the following limit:

\input{Limit-Concept-0001.HELP.tex}

\[\lim_{x\to0}f({x - 5})g(x)=\answer{-10}\]
\end{problem}}%}

\latexProblemContent{
\ifVerboseLocation This is Derivative Concept Question 0001. \\ \fi
\begin{problem}

If you know that $\lim\limits_{x\to{-2}}f(x)={3}$ and $\lim\limits_{x\to0}g(x)={2}$, then evaluate the following limit:

\input{Limit-Concept-0001.HELP.tex}

\[\lim_{x\to0}f({x - 2})g(x)=\answer{6}\]
\end{problem}}%}

\latexProblemContent{
\ifVerboseLocation This is Derivative Concept Question 0001. \\ \fi
\begin{problem}

If you know that $\lim\limits_{x\to{2}}f(x)={-2}$ and $\lim\limits_{x\to0}g(x)={-2}$, then evaluate the following limit:

\input{Limit-Concept-0001.HELP.tex}

\[\lim_{x\to0}f({x + 2})g(x)=\answer{4}\]
\end{problem}}%}

\latexProblemContent{
\ifVerboseLocation This is Derivative Concept Question 0001. \\ \fi
\begin{problem}

If you know that $\lim\limits_{x\to{-4}}f(x)={4}$ and $\lim\limits_{x\to0}g(x)={-5}$, then evaluate the following limit:

\input{Limit-Concept-0001.HELP.tex}

\[\lim_{x\to0}f({x - 4})g(x)=\answer{-20}\]
\end{problem}}%}

\latexProblemContent{
\ifVerboseLocation This is Derivative Concept Question 0001. \\ \fi
\begin{problem}

If you know that $\lim\limits_{x\to{-2}}f(x)={-4}$ and $\lim\limits_{x\to0}g(x)={3}$, then evaluate the following limit:

\input{Limit-Concept-0001.HELP.tex}

\[\lim_{x\to0}f({x - 2})g(x)=\answer{-12}\]
\end{problem}}%}

\latexProblemContent{
\ifVerboseLocation This is Derivative Concept Question 0001. \\ \fi
\begin{problem}

If you know that $\lim\limits_{x\to{2}}f(x)={4}$ and $\lim\limits_{x\to0}g(x)={5}$, then evaluate the following limit:

\input{Limit-Concept-0001.HELP.tex}

\[\lim_{x\to0}f({x + 2})g(x)=\answer{20}\]
\end{problem}}%}

\latexProblemContent{
\ifVerboseLocation This is Derivative Concept Question 0001. \\ \fi
\begin{problem}

If you know that $\lim\limits_{x\to{-3}}f(x)={-4}$ and $\lim\limits_{x\to0}g(x)={5}$, then evaluate the following limit:

\input{Limit-Concept-0001.HELP.tex}

\[\lim_{x\to0}f({x - 3})g(x)=\answer{-20}\]
\end{problem}}%}

\latexProblemContent{
\ifVerboseLocation This is Derivative Concept Question 0001. \\ \fi
\begin{problem}

If you know that $\lim\limits_{x\to{4}}f(x)={-2}$ and $\lim\limits_{x\to0}g(x)={3}$, then evaluate the following limit:

\input{Limit-Concept-0001.HELP.tex}

\[\lim_{x\to0}f({x + 4})g(x)=\answer{-6}\]
\end{problem}}%}

\latexProblemContent{
\ifVerboseLocation This is Derivative Concept Question 0001. \\ \fi
\begin{problem}

If you know that $\lim\limits_{x\to{1}}f(x)={4}$ and $\lim\limits_{x\to0}g(x)={3}$, then evaluate the following limit:

\input{Limit-Concept-0001.HELP.tex}

\[\lim_{x\to0}f({x + 1})g(x)=\answer{12}\]
\end{problem}}%}

\latexProblemContent{
\ifVerboseLocation This is Derivative Concept Question 0001. \\ \fi
\begin{problem}

If you know that $\lim\limits_{x\to{-5}}f(x)={5}$ and $\lim\limits_{x\to0}g(x)={4}$, then evaluate the following limit:

\input{Limit-Concept-0001.HELP.tex}

\[\lim_{x\to0}f({x - 5})g(x)=\answer{20}\]
\end{problem}}%}

\latexProblemContent{
\ifVerboseLocation This is Derivative Concept Question 0001. \\ \fi
\begin{problem}

If you know that $\lim\limits_{x\to{-4}}f(x)={1}$ and $\lim\limits_{x\to0}g(x)={-4}$, then evaluate the following limit:

\input{Limit-Concept-0001.HELP.tex}

\[\lim_{x\to0}f({x - 4})g(x)=\answer{-4}\]
\end{problem}}%}

\latexProblemContent{
\ifVerboseLocation This is Derivative Concept Question 0001. \\ \fi
\begin{problem}

If you know that $\lim\limits_{x\to{4}}f(x)={1}$ and $\lim\limits_{x\to0}g(x)={-2}$, then evaluate the following limit:

\input{Limit-Concept-0001.HELP.tex}

\[\lim_{x\to0}f({x + 4})g(x)=\answer{-2}\]
\end{problem}}%}

\latexProblemContent{
\ifVerboseLocation This is Derivative Concept Question 0001. \\ \fi
\begin{problem}

If you know that $\lim\limits_{x\to{-2}}f(x)={1}$ and $\lim\limits_{x\to0}g(x)={1}$, then evaluate the following limit:

\input{Limit-Concept-0001.HELP.tex}

\[\lim_{x\to0}f({x - 2})g(x)=\answer{1}\]
\end{problem}}%}

\latexProblemContent{
\ifVerboseLocation This is Derivative Concept Question 0001. \\ \fi
\begin{problem}

If you know that $\lim\limits_{x\to{-2}}f(x)={-1}$ and $\lim\limits_{x\to0}g(x)={2}$, then evaluate the following limit:

\input{Limit-Concept-0001.HELP.tex}

\[\lim_{x\to0}f({x - 2})g(x)=\answer{-2}\]
\end{problem}}%}

\latexProblemContent{
\ifVerboseLocation This is Derivative Concept Question 0001. \\ \fi
\begin{problem}

If you know that $\lim\limits_{x\to{-1}}f(x)={-4}$ and $\lim\limits_{x\to0}g(x)={-1}$, then evaluate the following limit:

\input{Limit-Concept-0001.HELP.tex}

\[\lim_{x\to0}f({x - 1})g(x)=\answer{4}\]
\end{problem}}%}

\latexProblemContent{
\ifVerboseLocation This is Derivative Concept Question 0001. \\ \fi
\begin{problem}

If you know that $\lim\limits_{x\to{5}}f(x)={1}$ and $\lim\limits_{x\to0}g(x)={1}$, then evaluate the following limit:

\input{Limit-Concept-0001.HELP.tex}

\[\lim_{x\to0}f({x + 5})g(x)=\answer{1}\]
\end{problem}}%}

\latexProblemContent{
\ifVerboseLocation This is Derivative Concept Question 0001. \\ \fi
\begin{problem}

If you know that $\lim\limits_{x\to{-3}}f(x)={2}$ and $\lim\limits_{x\to0}g(x)={2}$, then evaluate the following limit:

\input{Limit-Concept-0001.HELP.tex}

\[\lim_{x\to0}f({x - 3})g(x)=\answer{4}\]
\end{problem}}%}

\latexProblemContent{
\ifVerboseLocation This is Derivative Concept Question 0001. \\ \fi
\begin{problem}

If you know that $\lim\limits_{x\to{1}}f(x)={1}$ and $\lim\limits_{x\to0}g(x)={3}$, then evaluate the following limit:

\input{Limit-Concept-0001.HELP.tex}

\[\lim_{x\to0}f({x + 1})g(x)=\answer{3}\]
\end{problem}}%}

\latexProblemContent{
\ifVerboseLocation This is Derivative Concept Question 0001. \\ \fi
\begin{problem}

If you know that $\lim\limits_{x\to{4}}f(x)={3}$ and $\lim\limits_{x\to0}g(x)={4}$, then evaluate the following limit:

\input{Limit-Concept-0001.HELP.tex}

\[\lim_{x\to0}f({x + 4})g(x)=\answer{12}\]
\end{problem}}%}

\latexProblemContent{
\ifVerboseLocation This is Derivative Concept Question 0001. \\ \fi
\begin{problem}

If you know that $\lim\limits_{x\to{5}}f(x)={-2}$ and $\lim\limits_{x\to0}g(x)={5}$, then evaluate the following limit:

\input{Limit-Concept-0001.HELP.tex}

\[\lim_{x\to0}f({x + 5})g(x)=\answer{-10}\]
\end{problem}}%}

\latexProblemContent{
\ifVerboseLocation This is Derivative Concept Question 0001. \\ \fi
\begin{problem}

If you know that $\lim\limits_{x\to{-5}}f(x)={-1}$ and $\lim\limits_{x\to0}g(x)={3}$, then evaluate the following limit:

\input{Limit-Concept-0001.HELP.tex}

\[\lim_{x\to0}f({x - 5})g(x)=\answer{-3}\]
\end{problem}}%}

\latexProblemContent{
\ifVerboseLocation This is Derivative Concept Question 0001. \\ \fi
\begin{problem}

If you know that $\lim\limits_{x\to{3}}f(x)={2}$ and $\lim\limits_{x\to0}g(x)={-3}$, then evaluate the following limit:

\input{Limit-Concept-0001.HELP.tex}

\[\lim_{x\to0}f({x + 3})g(x)=\answer{-6}\]
\end{problem}}%}

\latexProblemContent{
\ifVerboseLocation This is Derivative Concept Question 0001. \\ \fi
\begin{problem}

If you know that $\lim\limits_{x\to{-1}}f(x)={-1}$ and $\lim\limits_{x\to0}g(x)={3}$, then evaluate the following limit:

\input{Limit-Concept-0001.HELP.tex}

\[\lim_{x\to0}f({x - 1})g(x)=\answer{-3}\]
\end{problem}}%}

\latexProblemContent{
\ifVerboseLocation This is Derivative Concept Question 0001. \\ \fi
\begin{problem}

If you know that $\lim\limits_{x\to{-3}}f(x)={1}$ and $\lim\limits_{x\to0}g(x)={-1}$, then evaluate the following limit:

\input{Limit-Concept-0001.HELP.tex}

\[\lim_{x\to0}f({x - 3})g(x)=\answer{-1}\]
\end{problem}}%}

\latexProblemContent{
\ifVerboseLocation This is Derivative Concept Question 0001. \\ \fi
\begin{problem}

If you know that $\lim\limits_{x\to{-5}}f(x)={-5}$ and $\lim\limits_{x\to0}g(x)={4}$, then evaluate the following limit:

\input{Limit-Concept-0001.HELP.tex}

\[\lim_{x\to0}f({x - 5})g(x)=\answer{-20}\]
\end{problem}}%}

\latexProblemContent{
\ifVerboseLocation This is Derivative Concept Question 0001. \\ \fi
\begin{problem}

If you know that $\lim\limits_{x\to{5}}f(x)={2}$ and $\lim\limits_{x\to0}g(x)={-5}$, then evaluate the following limit:

\input{Limit-Concept-0001.HELP.tex}

\[\lim_{x\to0}f({x + 5})g(x)=\answer{-10}\]
\end{problem}}%}

\latexProblemContent{
\ifVerboseLocation This is Derivative Concept Question 0001. \\ \fi
\begin{problem}

If you know that $\lim\limits_{x\to{4}}f(x)={-5}$ and $\lim\limits_{x\to0}g(x)={2}$, then evaluate the following limit:

\input{Limit-Concept-0001.HELP.tex}

\[\lim_{x\to0}f({x + 4})g(x)=\answer{-10}\]
\end{problem}}%}

\latexProblemContent{
\ifVerboseLocation This is Derivative Concept Question 0001. \\ \fi
\begin{problem}

If you know that $\lim\limits_{x\to{1}}f(x)={-1}$ and $\lim\limits_{x\to0}g(x)={4}$, then evaluate the following limit:

\input{Limit-Concept-0001.HELP.tex}

\[\lim_{x\to0}f({x + 1})g(x)=\answer{-4}\]
\end{problem}}%}

\latexProblemContent{
\ifVerboseLocation This is Derivative Concept Question 0001. \\ \fi
\begin{problem}

If you know that $\lim\limits_{x\to{2}}f(x)={4}$ and $\lim\limits_{x\to0}g(x)={-3}$, then evaluate the following limit:

\input{Limit-Concept-0001.HELP.tex}

\[\lim_{x\to0}f({x + 2})g(x)=\answer{-12}\]
\end{problem}}%}

\latexProblemContent{
\ifVerboseLocation This is Derivative Concept Question 0001. \\ \fi
\begin{problem}

If you know that $\lim\limits_{x\to{5}}f(x)={3}$ and $\lim\limits_{x\to0}g(x)={5}$, then evaluate the following limit:

\input{Limit-Concept-0001.HELP.tex}

\[\lim_{x\to0}f({x + 5})g(x)=\answer{15}\]
\end{problem}}%}

\latexProblemContent{
\ifVerboseLocation This is Derivative Concept Question 0001. \\ \fi
\begin{problem}

If you know that $\lim\limits_{x\to{5}}f(x)={-4}$ and $\lim\limits_{x\to0}g(x)={5}$, then evaluate the following limit:

\input{Limit-Concept-0001.HELP.tex}

\[\lim_{x\to0}f({x + 5})g(x)=\answer{-20}\]
\end{problem}}%}

\latexProblemContent{
\ifVerboseLocation This is Derivative Concept Question 0001. \\ \fi
\begin{problem}

If you know that $\lim\limits_{x\to{2}}f(x)={-1}$ and $\lim\limits_{x\to0}g(x)={-5}$, then evaluate the following limit:

\input{Limit-Concept-0001.HELP.tex}

\[\lim_{x\to0}f({x + 2})g(x)=\answer{5}\]
\end{problem}}%}

\latexProblemContent{
\ifVerboseLocation This is Derivative Concept Question 0001. \\ \fi
\begin{problem}

If you know that $\lim\limits_{x\to{4}}f(x)={-4}$ and $\lim\limits_{x\to0}g(x)={1}$, then evaluate the following limit:

\input{Limit-Concept-0001.HELP.tex}

\[\lim_{x\to0}f({x + 4})g(x)=\answer{-4}\]
\end{problem}}%}

\latexProblemContent{
\ifVerboseLocation This is Derivative Concept Question 0001. \\ \fi
\begin{problem}

If you know that $\lim\limits_{x\to{5}}f(x)={3}$ and $\lim\limits_{x\to0}g(x)={4}$, then evaluate the following limit:

\input{Limit-Concept-0001.HELP.tex}

\[\lim_{x\to0}f({x + 5})g(x)=\answer{12}\]
\end{problem}}%}

\latexProblemContent{
\ifVerboseLocation This is Derivative Concept Question 0001. \\ \fi
\begin{problem}

If you know that $\lim\limits_{x\to{-2}}f(x)={2}$ and $\lim\limits_{x\to0}g(x)={2}$, then evaluate the following limit:

\input{Limit-Concept-0001.HELP.tex}

\[\lim_{x\to0}f({x - 2})g(x)=\answer{4}\]
\end{problem}}%}

\latexProblemContent{
\ifVerboseLocation This is Derivative Concept Question 0001. \\ \fi
\begin{problem}

If you know that $\lim\limits_{x\to{-5}}f(x)={-5}$ and $\lim\limits_{x\to0}g(x)={2}$, then evaluate the following limit:

\input{Limit-Concept-0001.HELP.tex}

\[\lim_{x\to0}f({x - 5})g(x)=\answer{-10}\]
\end{problem}}%}

\latexProblemContent{
\ifVerboseLocation This is Derivative Concept Question 0001. \\ \fi
\begin{problem}

If you know that $\lim\limits_{x\to{5}}f(x)={1}$ and $\lim\limits_{x\to0}g(x)={4}$, then evaluate the following limit:

\input{Limit-Concept-0001.HELP.tex}

\[\lim_{x\to0}f({x + 5})g(x)=\answer{4}\]
\end{problem}}%}

\latexProblemContent{
\ifVerboseLocation This is Derivative Concept Question 0001. \\ \fi
\begin{problem}

If you know that $\lim\limits_{x\to{-1}}f(x)={-2}$ and $\lim\limits_{x\to0}g(x)={-5}$, then evaluate the following limit:

\input{Limit-Concept-0001.HELP.tex}

\[\lim_{x\to0}f({x - 1})g(x)=\answer{10}\]
\end{problem}}%}

\latexProblemContent{
\ifVerboseLocation This is Derivative Concept Question 0001. \\ \fi
\begin{problem}

If you know that $\lim\limits_{x\to{5}}f(x)={4}$ and $\lim\limits_{x\to0}g(x)={-2}$, then evaluate the following limit:

\input{Limit-Concept-0001.HELP.tex}

\[\lim_{x\to0}f({x + 5})g(x)=\answer{-8}\]
\end{problem}}%}

\latexProblemContent{
\ifVerboseLocation This is Derivative Concept Question 0001. \\ \fi
\begin{problem}

If you know that $\lim\limits_{x\to{3}}f(x)={-5}$ and $\lim\limits_{x\to0}g(x)={-1}$, then evaluate the following limit:

\input{Limit-Concept-0001.HELP.tex}

\[\lim_{x\to0}f({x + 3})g(x)=\answer{5}\]
\end{problem}}%}

\latexProblemContent{
\ifVerboseLocation This is Derivative Concept Question 0001. \\ \fi
\begin{problem}

If you know that $\lim\limits_{x\to{2}}f(x)={4}$ and $\lim\limits_{x\to0}g(x)={1}$, then evaluate the following limit:

\input{Limit-Concept-0001.HELP.tex}

\[\lim_{x\to0}f({x + 2})g(x)=\answer{4}\]
\end{problem}}%}

\latexProblemContent{
\ifVerboseLocation This is Derivative Concept Question 0001. \\ \fi
\begin{problem}

If you know that $\lim\limits_{x\to{1}}f(x)={1}$ and $\lim\limits_{x\to0}g(x)={5}$, then evaluate the following limit:

\input{Limit-Concept-0001.HELP.tex}

\[\lim_{x\to0}f({x + 1})g(x)=\answer{5}\]
\end{problem}}%}

\latexProblemContent{
\ifVerboseLocation This is Derivative Concept Question 0001. \\ \fi
\begin{problem}

If you know that $\lim\limits_{x\to{-4}}f(x)={4}$ and $\lim\limits_{x\to0}g(x)={1}$, then evaluate the following limit:

\input{Limit-Concept-0001.HELP.tex}

\[\lim_{x\to0}f({x - 4})g(x)=\answer{4}\]
\end{problem}}%}

\latexProblemContent{
\ifVerboseLocation This is Derivative Concept Question 0001. \\ \fi
\begin{problem}

If you know that $\lim\limits_{x\to{-3}}f(x)={-5}$ and $\lim\limits_{x\to0}g(x)={1}$, then evaluate the following limit:

\input{Limit-Concept-0001.HELP.tex}

\[\lim_{x\to0}f({x - 3})g(x)=\answer{-5}\]
\end{problem}}%}

\latexProblemContent{
\ifVerboseLocation This is Derivative Concept Question 0001. \\ \fi
\begin{problem}

If you know that $\lim\limits_{x\to{4}}f(x)={-3}$ and $\lim\limits_{x\to0}g(x)={-2}$, then evaluate the following limit:

\input{Limit-Concept-0001.HELP.tex}

\[\lim_{x\to0}f({x + 4})g(x)=\answer{6}\]
\end{problem}}%}

\latexProblemContent{
\ifVerboseLocation This is Derivative Concept Question 0001. \\ \fi
\begin{problem}

If you know that $\lim\limits_{x\to{-3}}f(x)={-1}$ and $\lim\limits_{x\to0}g(x)={-4}$, then evaluate the following limit:

\input{Limit-Concept-0001.HELP.tex}

\[\lim_{x\to0}f({x - 3})g(x)=\answer{4}\]
\end{problem}}%}

\latexProblemContent{
\ifVerboseLocation This is Derivative Concept Question 0001. \\ \fi
\begin{problem}

If you know that $\lim\limits_{x\to{-5}}f(x)={-4}$ and $\lim\limits_{x\to0}g(x)={1}$, then evaluate the following limit:

\input{Limit-Concept-0001.HELP.tex}

\[\lim_{x\to0}f({x - 5})g(x)=\answer{-4}\]
\end{problem}}%}

\latexProblemContent{
\ifVerboseLocation This is Derivative Concept Question 0001. \\ \fi
\begin{problem}

If you know that $\lim\limits_{x\to{1}}f(x)={-2}$ and $\lim\limits_{x\to0}g(x)={-2}$, then evaluate the following limit:

\input{Limit-Concept-0001.HELP.tex}

\[\lim_{x\to0}f({x + 1})g(x)=\answer{4}\]
\end{problem}}%}

\latexProblemContent{
\ifVerboseLocation This is Derivative Concept Question 0001. \\ \fi
\begin{problem}

If you know that $\lim\limits_{x\to{4}}f(x)={-1}$ and $\lim\limits_{x\to0}g(x)={-4}$, then evaluate the following limit:

\input{Limit-Concept-0001.HELP.tex}

\[\lim_{x\to0}f({x + 4})g(x)=\answer{4}\]
\end{problem}}%}

\latexProblemContent{
\ifVerboseLocation This is Derivative Concept Question 0001. \\ \fi
\begin{problem}

If you know that $\lim\limits_{x\to{4}}f(x)={-4}$ and $\lim\limits_{x\to0}g(x)={2}$, then evaluate the following limit:

\input{Limit-Concept-0001.HELP.tex}

\[\lim_{x\to0}f({x + 4})g(x)=\answer{-8}\]
\end{problem}}%}

\latexProblemContent{
\ifVerboseLocation This is Derivative Concept Question 0001. \\ \fi
\begin{problem}

If you know that $\lim\limits_{x\to{-5}}f(x)={-3}$ and $\lim\limits_{x\to0}g(x)={-3}$, then evaluate the following limit:

\input{Limit-Concept-0001.HELP.tex}

\[\lim_{x\to0}f({x - 5})g(x)=\answer{9}\]
\end{problem}}%}

\latexProblemContent{
\ifVerboseLocation This is Derivative Concept Question 0001. \\ \fi
\begin{problem}

If you know that $\lim\limits_{x\to{1}}f(x)={3}$ and $\lim\limits_{x\to0}g(x)={-2}$, then evaluate the following limit:

\input{Limit-Concept-0001.HELP.tex}

\[\lim_{x\to0}f({x + 1})g(x)=\answer{-6}\]
\end{problem}}%}

\latexProblemContent{
\ifVerboseLocation This is Derivative Concept Question 0001. \\ \fi
\begin{problem}

If you know that $\lim\limits_{x\to{3}}f(x)={-3}$ and $\lim\limits_{x\to0}g(x)={-5}$, then evaluate the following limit:

\input{Limit-Concept-0001.HELP.tex}

\[\lim_{x\to0}f({x + 3})g(x)=\answer{15}\]
\end{problem}}%}

\latexProblemContent{
\ifVerboseLocation This is Derivative Concept Question 0001. \\ \fi
\begin{problem}

If you know that $\lim\limits_{x\to{-1}}f(x)={-5}$ and $\lim\limits_{x\to0}g(x)={1}$, then evaluate the following limit:

\input{Limit-Concept-0001.HELP.tex}

\[\lim_{x\to0}f({x - 1})g(x)=\answer{-5}\]
\end{problem}}%}

\latexProblemContent{
\ifVerboseLocation This is Derivative Concept Question 0001. \\ \fi
\begin{problem}

If you know that $\lim\limits_{x\to{5}}f(x)={-3}$ and $\lim\limits_{x\to0}g(x)={-4}$, then evaluate the following limit:

\input{Limit-Concept-0001.HELP.tex}

\[\lim_{x\to0}f({x + 5})g(x)=\answer{12}\]
\end{problem}}%}

\latexProblemContent{
\ifVerboseLocation This is Derivative Concept Question 0001. \\ \fi
\begin{problem}

If you know that $\lim\limits_{x\to{5}}f(x)={4}$ and $\lim\limits_{x\to0}g(x)={-4}$, then evaluate the following limit:

\input{Limit-Concept-0001.HELP.tex}

\[\lim_{x\to0}f({x + 5})g(x)=\answer{-16}\]
\end{problem}}%}

\latexProblemContent{
\ifVerboseLocation This is Derivative Concept Question 0001. \\ \fi
\begin{problem}

If you know that $\lim\limits_{x\to{2}}f(x)={-5}$ and $\lim\limits_{x\to0}g(x)={3}$, then evaluate the following limit:

\input{Limit-Concept-0001.HELP.tex}

\[\lim_{x\to0}f({x + 2})g(x)=\answer{-15}\]
\end{problem}}%}

\latexProblemContent{
\ifVerboseLocation This is Derivative Concept Question 0001. \\ \fi
\begin{problem}

If you know that $\lim\limits_{x\to{5}}f(x)={-2}$ and $\lim\limits_{x\to0}g(x)={3}$, then evaluate the following limit:

\input{Limit-Concept-0001.HELP.tex}

\[\lim_{x\to0}f({x + 5})g(x)=\answer{-6}\]
\end{problem}}%}

\latexProblemContent{
\ifVerboseLocation This is Derivative Concept Question 0001. \\ \fi
\begin{problem}

If you know that $\lim\limits_{x\to{-2}}f(x)={1}$ and $\lim\limits_{x\to0}g(x)={2}$, then evaluate the following limit:

\input{Limit-Concept-0001.HELP.tex}

\[\lim_{x\to0}f({x - 2})g(x)=\answer{2}\]
\end{problem}}%}

\latexProblemContent{
\ifVerboseLocation This is Derivative Concept Question 0001. \\ \fi
\begin{problem}

If you know that $\lim\limits_{x\to{-1}}f(x)={-3}$ and $\lim\limits_{x\to0}g(x)={2}$, then evaluate the following limit:

\input{Limit-Concept-0001.HELP.tex}

\[\lim_{x\to0}f({x - 1})g(x)=\answer{-6}\]
\end{problem}}%}

\latexProblemContent{
\ifVerboseLocation This is Derivative Concept Question 0001. \\ \fi
\begin{problem}

If you know that $\lim\limits_{x\to{5}}f(x)={-5}$ and $\lim\limits_{x\to0}g(x)={-5}$, then evaluate the following limit:

\input{Limit-Concept-0001.HELP.tex}

\[\lim_{x\to0}f({x + 5})g(x)=\answer{25}\]
\end{problem}}%}

\latexProblemContent{
\ifVerboseLocation This is Derivative Concept Question 0001. \\ \fi
\begin{problem}

If you know that $\lim\limits_{x\to{-2}}f(x)={4}$ and $\lim\limits_{x\to0}g(x)={5}$, then evaluate the following limit:

\input{Limit-Concept-0001.HELP.tex}

\[\lim_{x\to0}f({x - 2})g(x)=\answer{20}\]
\end{problem}}%}

\latexProblemContent{
\ifVerboseLocation This is Derivative Concept Question 0001. \\ \fi
\begin{problem}

If you know that $\lim\limits_{x\to{-2}}f(x)={-1}$ and $\lim\limits_{x\to0}g(x)={-1}$, then evaluate the following limit:

\input{Limit-Concept-0001.HELP.tex}

\[\lim_{x\to0}f({x - 2})g(x)=\answer{1}\]
\end{problem}}%}

\latexProblemContent{
\ifVerboseLocation This is Derivative Concept Question 0001. \\ \fi
\begin{problem}

If you know that $\lim\limits_{x\to{1}}f(x)={-4}$ and $\lim\limits_{x\to0}g(x)={2}$, then evaluate the following limit:

\input{Limit-Concept-0001.HELP.tex}

\[\lim_{x\to0}f({x + 1})g(x)=\answer{-8}\]
\end{problem}}%}

\latexProblemContent{
\ifVerboseLocation This is Derivative Concept Question 0001. \\ \fi
\begin{problem}

If you know that $\lim\limits_{x\to{-5}}f(x)={5}$ and $\lim\limits_{x\to0}g(x)={-4}$, then evaluate the following limit:

\input{Limit-Concept-0001.HELP.tex}

\[\lim_{x\to0}f({x - 5})g(x)=\answer{-20}\]
\end{problem}}%}

\latexProblemContent{
\ifVerboseLocation This is Derivative Concept Question 0001. \\ \fi
\begin{problem}

If you know that $\lim\limits_{x\to{3}}f(x)={3}$ and $\lim\limits_{x\to0}g(x)={-3}$, then evaluate the following limit:

\input{Limit-Concept-0001.HELP.tex}

\[\lim_{x\to0}f({x + 3})g(x)=\answer{-9}\]
\end{problem}}%}

\latexProblemContent{
\ifVerboseLocation This is Derivative Concept Question 0001. \\ \fi
\begin{problem}

If you know that $\lim\limits_{x\to{-4}}f(x)={4}$ and $\lim\limits_{x\to0}g(x)={-4}$, then evaluate the following limit:

\input{Limit-Concept-0001.HELP.tex}

\[\lim_{x\to0}f({x - 4})g(x)=\answer{-16}\]
\end{problem}}%}

\latexProblemContent{
\ifVerboseLocation This is Derivative Concept Question 0001. \\ \fi
\begin{problem}

If you know that $\lim\limits_{x\to{-1}}f(x)={2}$ and $\lim\limits_{x\to0}g(x)={-3}$, then evaluate the following limit:

\input{Limit-Concept-0001.HELP.tex}

\[\lim_{x\to0}f({x - 1})g(x)=\answer{-6}\]
\end{problem}}%}

\latexProblemContent{
\ifVerboseLocation This is Derivative Concept Question 0001. \\ \fi
\begin{problem}

If you know that $\lim\limits_{x\to{2}}f(x)={2}$ and $\lim\limits_{x\to0}g(x)={1}$, then evaluate the following limit:

\input{Limit-Concept-0001.HELP.tex}

\[\lim_{x\to0}f({x + 2})g(x)=\answer{2}\]
\end{problem}}%}

\latexProblemContent{
\ifVerboseLocation This is Derivative Concept Question 0001. \\ \fi
\begin{problem}

If you know that $\lim\limits_{x\to{3}}f(x)={-2}$ and $\lim\limits_{x\to0}g(x)={4}$, then evaluate the following limit:

\input{Limit-Concept-0001.HELP.tex}

\[\lim_{x\to0}f({x + 3})g(x)=\answer{-8}\]
\end{problem}}%}

\latexProblemContent{
\ifVerboseLocation This is Derivative Concept Question 0001. \\ \fi
\begin{problem}

If you know that $\lim\limits_{x\to{2}}f(x)={-3}$ and $\lim\limits_{x\to0}g(x)={-2}$, then evaluate the following limit:

\input{Limit-Concept-0001.HELP.tex}

\[\lim_{x\to0}f({x + 2})g(x)=\answer{6}\]
\end{problem}}%}

\latexProblemContent{
\ifVerboseLocation This is Derivative Concept Question 0001. \\ \fi
\begin{problem}

If you know that $\lim\limits_{x\to{4}}f(x)={5}$ and $\lim\limits_{x\to0}g(x)={-1}$, then evaluate the following limit:

\input{Limit-Concept-0001.HELP.tex}

\[\lim_{x\to0}f({x + 4})g(x)=\answer{-5}\]
\end{problem}}%}

\latexProblemContent{
\ifVerboseLocation This is Derivative Concept Question 0001. \\ \fi
\begin{problem}

If you know that $\lim\limits_{x\to{2}}f(x)={5}$ and $\lim\limits_{x\to0}g(x)={4}$, then evaluate the following limit:

\input{Limit-Concept-0001.HELP.tex}

\[\lim_{x\to0}f({x + 2})g(x)=\answer{20}\]
\end{problem}}%}

\latexProblemContent{
\ifVerboseLocation This is Derivative Concept Question 0001. \\ \fi
\begin{problem}

If you know that $\lim\limits_{x\to{-3}}f(x)={-2}$ and $\lim\limits_{x\to0}g(x)={2}$, then evaluate the following limit:

\input{Limit-Concept-0001.HELP.tex}

\[\lim_{x\to0}f({x - 3})g(x)=\answer{-4}\]
\end{problem}}%}

\latexProblemContent{
\ifVerboseLocation This is Derivative Concept Question 0001. \\ \fi
\begin{problem}

If you know that $\lim\limits_{x\to{2}}f(x)={5}$ and $\lim\limits_{x\to0}g(x)={-4}$, then evaluate the following limit:

\input{Limit-Concept-0001.HELP.tex}

\[\lim_{x\to0}f({x + 2})g(x)=\answer{-20}\]
\end{problem}}%}

\latexProblemContent{
\ifVerboseLocation This is Derivative Concept Question 0001. \\ \fi
\begin{problem}

If you know that $\lim\limits_{x\to{-4}}f(x)={-2}$ and $\lim\limits_{x\to0}g(x)={4}$, then evaluate the following limit:

\input{Limit-Concept-0001.HELP.tex}

\[\lim_{x\to0}f({x - 4})g(x)=\answer{-8}\]
\end{problem}}%}

\latexProblemContent{
\ifVerboseLocation This is Derivative Concept Question 0001. \\ \fi
\begin{problem}

If you know that $\lim\limits_{x\to{-5}}f(x)={2}$ and $\lim\limits_{x\to0}g(x)={-4}$, then evaluate the following limit:

\input{Limit-Concept-0001.HELP.tex}

\[\lim_{x\to0}f({x - 5})g(x)=\answer{-8}\]
\end{problem}}%}

\latexProblemContent{
\ifVerboseLocation This is Derivative Concept Question 0001. \\ \fi
\begin{problem}

If you know that $\lim\limits_{x\to{-3}}f(x)={-4}$ and $\lim\limits_{x\to0}g(x)={-2}$, then evaluate the following limit:

\input{Limit-Concept-0001.HELP.tex}

\[\lim_{x\to0}f({x - 3})g(x)=\answer{8}\]
\end{problem}}%}

\latexProblemContent{
\ifVerboseLocation This is Derivative Concept Question 0001. \\ \fi
\begin{problem}

If you know that $\lim\limits_{x\to{3}}f(x)={-5}$ and $\lim\limits_{x\to0}g(x)={1}$, then evaluate the following limit:

\input{Limit-Concept-0001.HELP.tex}

\[\lim_{x\to0}f({x + 3})g(x)=\answer{-5}\]
\end{problem}}%}

\latexProblemContent{
\ifVerboseLocation This is Derivative Concept Question 0001. \\ \fi
\begin{problem}

If you know that $\lim\limits_{x\to{5}}f(x)={-1}$ and $\lim\limits_{x\to0}g(x)={-4}$, then evaluate the following limit:

\input{Limit-Concept-0001.HELP.tex}

\[\lim_{x\to0}f({x + 5})g(x)=\answer{4}\]
\end{problem}}%}

\latexProblemContent{
\ifVerboseLocation This is Derivative Concept Question 0001. \\ \fi
\begin{problem}

If you know that $\lim\limits_{x\to{-1}}f(x)={5}$ and $\lim\limits_{x\to0}g(x)={-3}$, then evaluate the following limit:

\input{Limit-Concept-0001.HELP.tex}

\[\lim_{x\to0}f({x - 1})g(x)=\answer{-15}\]
\end{problem}}%}

\latexProblemContent{
\ifVerboseLocation This is Derivative Concept Question 0001. \\ \fi
\begin{problem}

If you know that $\lim\limits_{x\to{-3}}f(x)={4}$ and $\lim\limits_{x\to0}g(x)={3}$, then evaluate the following limit:

\input{Limit-Concept-0001.HELP.tex}

\[\lim_{x\to0}f({x - 3})g(x)=\answer{12}\]
\end{problem}}%}

\latexProblemContent{
\ifVerboseLocation This is Derivative Concept Question 0001. \\ \fi
\begin{problem}

If you know that $\lim\limits_{x\to{-3}}f(x)={-2}$ and $\lim\limits_{x\to0}g(x)={-3}$, then evaluate the following limit:

\input{Limit-Concept-0001.HELP.tex}

\[\lim_{x\to0}f({x - 3})g(x)=\answer{6}\]
\end{problem}}%}

\latexProblemContent{
\ifVerboseLocation This is Derivative Concept Question 0001. \\ \fi
\begin{problem}

If you know that $\lim\limits_{x\to{-4}}f(x)={4}$ and $\lim\limits_{x\to0}g(x)={2}$, then evaluate the following limit:

\input{Limit-Concept-0001.HELP.tex}

\[\lim_{x\to0}f({x - 4})g(x)=\answer{8}\]
\end{problem}}%}

\latexProblemContent{
\ifVerboseLocation This is Derivative Concept Question 0001. \\ \fi
\begin{problem}

If you know that $\lim\limits_{x\to{-3}}f(x)={-4}$ and $\lim\limits_{x\to0}g(x)={-5}$, then evaluate the following limit:

\input{Limit-Concept-0001.HELP.tex}

\[\lim_{x\to0}f({x - 3})g(x)=\answer{20}\]
\end{problem}}%}

\latexProblemContent{
\ifVerboseLocation This is Derivative Concept Question 0001. \\ \fi
\begin{problem}

If you know that $\lim\limits_{x\to{-5}}f(x)={5}$ and $\lim\limits_{x\to0}g(x)={-5}$, then evaluate the following limit:

\input{Limit-Concept-0001.HELP.tex}

\[\lim_{x\to0}f({x - 5})g(x)=\answer{-25}\]
\end{problem}}%}

\latexProblemContent{
\ifVerboseLocation This is Derivative Concept Question 0001. \\ \fi
\begin{problem}

If you know that $\lim\limits_{x\to{-2}}f(x)={-5}$ and $\lim\limits_{x\to0}g(x)={-1}$, then evaluate the following limit:

\input{Limit-Concept-0001.HELP.tex}

\[\lim_{x\to0}f({x - 2})g(x)=\answer{5}\]
\end{problem}}%}

\latexProblemContent{
\ifVerboseLocation This is Derivative Concept Question 0001. \\ \fi
\begin{problem}

If you know that $\lim\limits_{x\to{4}}f(x)={1}$ and $\lim\limits_{x\to0}g(x)={-1}$, then evaluate the following limit:

\input{Limit-Concept-0001.HELP.tex}

\[\lim_{x\to0}f({x + 4})g(x)=\answer{-1}\]
\end{problem}}%}

\latexProblemContent{
\ifVerboseLocation This is Derivative Concept Question 0001. \\ \fi
\begin{problem}

If you know that $\lim\limits_{x\to{2}}f(x)={-2}$ and $\lim\limits_{x\to0}g(x)={4}$, then evaluate the following limit:

\input{Limit-Concept-0001.HELP.tex}

\[\lim_{x\to0}f({x + 2})g(x)=\answer{-8}\]
\end{problem}}%}

\latexProblemContent{
\ifVerboseLocation This is Derivative Concept Question 0001. \\ \fi
\begin{problem}

If you know that $\lim\limits_{x\to{-4}}f(x)={5}$ and $\lim\limits_{x\to0}g(x)={-4}$, then evaluate the following limit:

\input{Limit-Concept-0001.HELP.tex}

\[\lim_{x\to0}f({x - 4})g(x)=\answer{-20}\]
\end{problem}}%}

\latexProblemContent{
\ifVerboseLocation This is Derivative Concept Question 0001. \\ \fi
\begin{problem}

If you know that $\lim\limits_{x\to{5}}f(x)={2}$ and $\lim\limits_{x\to0}g(x)={-3}$, then evaluate the following limit:

\input{Limit-Concept-0001.HELP.tex}

\[\lim_{x\to0}f({x + 5})g(x)=\answer{-6}\]
\end{problem}}%}

\latexProblemContent{
\ifVerboseLocation This is Derivative Concept Question 0001. \\ \fi
\begin{problem}

If you know that $\lim\limits_{x\to{4}}f(x)={-3}$ and $\lim\limits_{x\to0}g(x)={-4}$, then evaluate the following limit:

\input{Limit-Concept-0001.HELP.tex}

\[\lim_{x\to0}f({x + 4})g(x)=\answer{12}\]
\end{problem}}%}

\latexProblemContent{
\ifVerboseLocation This is Derivative Concept Question 0001. \\ \fi
\begin{problem}

If you know that $\lim\limits_{x\to{-4}}f(x)={-2}$ and $\lim\limits_{x\to0}g(x)={-2}$, then evaluate the following limit:

\input{Limit-Concept-0001.HELP.tex}

\[\lim_{x\to0}f({x - 4})g(x)=\answer{4}\]
\end{problem}}%}

\latexProblemContent{
\ifVerboseLocation This is Derivative Concept Question 0001. \\ \fi
\begin{problem}

If you know that $\lim\limits_{x\to{5}}f(x)={4}$ and $\lim\limits_{x\to0}g(x)={3}$, then evaluate the following limit:

\input{Limit-Concept-0001.HELP.tex}

\[\lim_{x\to0}f({x + 5})g(x)=\answer{12}\]
\end{problem}}%}

\latexProblemContent{
\ifVerboseLocation This is Derivative Concept Question 0001. \\ \fi
\begin{problem}

If you know that $\lim\limits_{x\to{4}}f(x)={-1}$ and $\lim\limits_{x\to0}g(x)={1}$, then evaluate the following limit:

\input{Limit-Concept-0001.HELP.tex}

\[\lim_{x\to0}f({x + 4})g(x)=\answer{-1}\]
\end{problem}}%}

\latexProblemContent{
\ifVerboseLocation This is Derivative Concept Question 0001. \\ \fi
\begin{problem}

If you know that $\lim\limits_{x\to{-1}}f(x)={3}$ and $\lim\limits_{x\to0}g(x)={2}$, then evaluate the following limit:

\input{Limit-Concept-0001.HELP.tex}

\[\lim_{x\to0}f({x - 1})g(x)=\answer{6}\]
\end{problem}}%}

\latexProblemContent{
\ifVerboseLocation This is Derivative Concept Question 0001. \\ \fi
\begin{problem}

If you know that $\lim\limits_{x\to{1}}f(x)={4}$ and $\lim\limits_{x\to0}g(x)={5}$, then evaluate the following limit:

\input{Limit-Concept-0001.HELP.tex}

\[\lim_{x\to0}f({x + 1})g(x)=\answer{20}\]
\end{problem}}%}

\latexProblemContent{
\ifVerboseLocation This is Derivative Concept Question 0001. \\ \fi
\begin{problem}

If you know that $\lim\limits_{x\to{5}}f(x)={4}$ and $\lim\limits_{x\to0}g(x)={5}$, then evaluate the following limit:

\input{Limit-Concept-0001.HELP.tex}

\[\lim_{x\to0}f({x + 5})g(x)=\answer{20}\]
\end{problem}}%}

\latexProblemContent{
\ifVerboseLocation This is Derivative Concept Question 0001. \\ \fi
\begin{problem}

If you know that $\lim\limits_{x\to{-4}}f(x)={5}$ and $\lim\limits_{x\to0}g(x)={-1}$, then evaluate the following limit:

\input{Limit-Concept-0001.HELP.tex}

\[\lim_{x\to0}f({x - 4})g(x)=\answer{-5}\]
\end{problem}}%}

\latexProblemContent{
\ifVerboseLocation This is Derivative Concept Question 0001. \\ \fi
\begin{problem}

If you know that $\lim\limits_{x\to{-3}}f(x)={3}$ and $\lim\limits_{x\to0}g(x)={-2}$, then evaluate the following limit:

\input{Limit-Concept-0001.HELP.tex}

\[\lim_{x\to0}f({x - 3})g(x)=\answer{-6}\]
\end{problem}}%}

\latexProblemContent{
\ifVerboseLocation This is Derivative Concept Question 0001. \\ \fi
\begin{problem}

If you know that $\lim\limits_{x\to{-5}}f(x)={-1}$ and $\lim\limits_{x\to0}g(x)={1}$, then evaluate the following limit:

\input{Limit-Concept-0001.HELP.tex}

\[\lim_{x\to0}f({x - 5})g(x)=\answer{-1}\]
\end{problem}}%}

\latexProblemContent{
\ifVerboseLocation This is Derivative Concept Question 0001. \\ \fi
\begin{problem}

If you know that $\lim\limits_{x\to{4}}f(x)={-3}$ and $\lim\limits_{x\to0}g(x)={4}$, then evaluate the following limit:

\input{Limit-Concept-0001.HELP.tex}

\[\lim_{x\to0}f({x + 4})g(x)=\answer{-12}\]
\end{problem}}%}

\latexProblemContent{
\ifVerboseLocation This is Derivative Concept Question 0001. \\ \fi
\begin{problem}

If you know that $\lim\limits_{x\to{1}}f(x)={5}$ and $\lim\limits_{x\to0}g(x)={2}$, then evaluate the following limit:

\input{Limit-Concept-0001.HELP.tex}

\[\lim_{x\to0}f({x + 1})g(x)=\answer{10}\]
\end{problem}}%}

\latexProblemContent{
\ifVerboseLocation This is Derivative Concept Question 0001. \\ \fi
\begin{problem}

If you know that $\lim\limits_{x\to{-4}}f(x)={3}$ and $\lim\limits_{x\to0}g(x)={-5}$, then evaluate the following limit:

\input{Limit-Concept-0001.HELP.tex}

\[\lim_{x\to0}f({x - 4})g(x)=\answer{-15}\]
\end{problem}}%}

\latexProblemContent{
\ifVerboseLocation This is Derivative Concept Question 0001. \\ \fi
\begin{problem}

If you know that $\lim\limits_{x\to{2}}f(x)={-1}$ and $\lim\limits_{x\to0}g(x)={4}$, then evaluate the following limit:

\input{Limit-Concept-0001.HELP.tex}

\[\lim_{x\to0}f({x + 2})g(x)=\answer{-4}\]
\end{problem}}%}

\latexProblemContent{
\ifVerboseLocation This is Derivative Concept Question 0001. \\ \fi
\begin{problem}

If you know that $\lim\limits_{x\to{-5}}f(x)={4}$ and $\lim\limits_{x\to0}g(x)={1}$, then evaluate the following limit:

\input{Limit-Concept-0001.HELP.tex}

\[\lim_{x\to0}f({x - 5})g(x)=\answer{4}\]
\end{problem}}%}

\latexProblemContent{
\ifVerboseLocation This is Derivative Concept Question 0001. \\ \fi
\begin{problem}

If you know that $\lim\limits_{x\to{-1}}f(x)={-2}$ and $\lim\limits_{x\to0}g(x)={-2}$, then evaluate the following limit:

\input{Limit-Concept-0001.HELP.tex}

\[\lim_{x\to0}f({x - 1})g(x)=\answer{4}\]
\end{problem}}%}

\latexProblemContent{
\ifVerboseLocation This is Derivative Concept Question 0001. \\ \fi
\begin{problem}

If you know that $\lim\limits_{x\to{5}}f(x)={-1}$ and $\lim\limits_{x\to0}g(x)={2}$, then evaluate the following limit:

\input{Limit-Concept-0001.HELP.tex}

\[\lim_{x\to0}f({x + 5})g(x)=\answer{-2}\]
\end{problem}}%}

\latexProblemContent{
\ifVerboseLocation This is Derivative Concept Question 0001. \\ \fi
\begin{problem}

If you know that $\lim\limits_{x\to{2}}f(x)={1}$ and $\lim\limits_{x\to0}g(x)={5}$, then evaluate the following limit:

\input{Limit-Concept-0001.HELP.tex}

\[\lim_{x\to0}f({x + 2})g(x)=\answer{5}\]
\end{problem}}%}

\latexProblemContent{
\ifVerboseLocation This is Derivative Concept Question 0001. \\ \fi
\begin{problem}

If you know that $\lim\limits_{x\to{2}}f(x)={-2}$ and $\lim\limits_{x\to0}g(x)={1}$, then evaluate the following limit:

\input{Limit-Concept-0001.HELP.tex}

\[\lim_{x\to0}f({x + 2})g(x)=\answer{-2}\]
\end{problem}}%}

\latexProblemContent{
\ifVerboseLocation This is Derivative Concept Question 0001. \\ \fi
\begin{problem}

If you know that $\lim\limits_{x\to{-3}}f(x)={2}$ and $\lim\limits_{x\to0}g(x)={5}$, then evaluate the following limit:

\input{Limit-Concept-0001.HELP.tex}

\[\lim_{x\to0}f({x - 3})g(x)=\answer{10}\]
\end{problem}}%}

\latexProblemContent{
\ifVerboseLocation This is Derivative Concept Question 0001. \\ \fi
\begin{problem}

If you know that $\lim\limits_{x\to{-4}}f(x)={-2}$ and $\lim\limits_{x\to0}g(x)={-1}$, then evaluate the following limit:

\input{Limit-Concept-0001.HELP.tex}

\[\lim_{x\to0}f({x - 4})g(x)=\answer{2}\]
\end{problem}}%}

\latexProblemContent{
\ifVerboseLocation This is Derivative Concept Question 0001. \\ \fi
\begin{problem}

If you know that $\lim\limits_{x\to{5}}f(x)={-5}$ and $\lim\limits_{x\to0}g(x)={4}$, then evaluate the following limit:

\input{Limit-Concept-0001.HELP.tex}

\[\lim_{x\to0}f({x + 5})g(x)=\answer{-20}\]
\end{problem}}%}

\latexProblemContent{
\ifVerboseLocation This is Derivative Concept Question 0001. \\ \fi
\begin{problem}

If you know that $\lim\limits_{x\to{-3}}f(x)={-1}$ and $\lim\limits_{x\to0}g(x)={1}$, then evaluate the following limit:

\input{Limit-Concept-0001.HELP.tex}

\[\lim_{x\to0}f({x - 3})g(x)=\answer{-1}\]
\end{problem}}%}

\latexProblemContent{
\ifVerboseLocation This is Derivative Concept Question 0001. \\ \fi
\begin{problem}

If you know that $\lim\limits_{x\to{5}}f(x)={1}$ and $\lim\limits_{x\to0}g(x)={-3}$, then evaluate the following limit:

\input{Limit-Concept-0001.HELP.tex}

\[\lim_{x\to0}f({x + 5})g(x)=\answer{-3}\]
\end{problem}}%}

\latexProblemContent{
\ifVerboseLocation This is Derivative Concept Question 0001. \\ \fi
\begin{problem}

If you know that $\lim\limits_{x\to{-3}}f(x)={1}$ and $\lim\limits_{x\to0}g(x)={1}$, then evaluate the following limit:

\input{Limit-Concept-0001.HELP.tex}

\[\lim_{x\to0}f({x - 3})g(x)=\answer{1}\]
\end{problem}}%}

\latexProblemContent{
\ifVerboseLocation This is Derivative Concept Question 0001. \\ \fi
\begin{problem}

If you know that $\lim\limits_{x\to{-4}}f(x)={-4}$ and $\lim\limits_{x\to0}g(x)={1}$, then evaluate the following limit:

\input{Limit-Concept-0001.HELP.tex}

\[\lim_{x\to0}f({x - 4})g(x)=\answer{-4}\]
\end{problem}}%}

\latexProblemContent{
\ifVerboseLocation This is Derivative Concept Question 0001. \\ \fi
\begin{problem}

If you know that $\lim\limits_{x\to{-4}}f(x)={4}$ and $\lim\limits_{x\to0}g(x)={5}$, then evaluate the following limit:

\input{Limit-Concept-0001.HELP.tex}

\[\lim_{x\to0}f({x - 4})g(x)=\answer{20}\]
\end{problem}}%}

\latexProblemContent{
\ifVerboseLocation This is Derivative Concept Question 0001. \\ \fi
\begin{problem}

If you know that $\lim\limits_{x\to{1}}f(x)={1}$ and $\lim\limits_{x\to0}g(x)={1}$, then evaluate the following limit:

\input{Limit-Concept-0001.HELP.tex}

\[\lim_{x\to0}f({x + 1})g(x)=\answer{1}\]
\end{problem}}%}

\latexProblemContent{
\ifVerboseLocation This is Derivative Concept Question 0001. \\ \fi
\begin{problem}

If you know that $\lim\limits_{x\to{3}}f(x)={-2}$ and $\lim\limits_{x\to0}g(x)={-4}$, then evaluate the following limit:

\input{Limit-Concept-0001.HELP.tex}

\[\lim_{x\to0}f({x + 3})g(x)=\answer{8}\]
\end{problem}}%}

\latexProblemContent{
\ifVerboseLocation This is Derivative Concept Question 0001. \\ \fi
\begin{problem}

If you know that $\lim\limits_{x\to{-3}}f(x)={-3}$ and $\lim\limits_{x\to0}g(x)={2}$, then evaluate the following limit:

\input{Limit-Concept-0001.HELP.tex}

\[\lim_{x\to0}f({x - 3})g(x)=\answer{-6}\]
\end{problem}}%}

\latexProblemContent{
\ifVerboseLocation This is Derivative Concept Question 0001. \\ \fi
\begin{problem}

If you know that $\lim\limits_{x\to{-2}}f(x)={3}$ and $\lim\limits_{x\to0}g(x)={3}$, then evaluate the following limit:

\input{Limit-Concept-0001.HELP.tex}

\[\lim_{x\to0}f({x - 2})g(x)=\answer{9}\]
\end{problem}}%}

\latexProblemContent{
\ifVerboseLocation This is Derivative Concept Question 0001. \\ \fi
\begin{problem}

If you know that $\lim\limits_{x\to{1}}f(x)={5}$ and $\lim\limits_{x\to0}g(x)={-1}$, then evaluate the following limit:

\input{Limit-Concept-0001.HELP.tex}

\[\lim_{x\to0}f({x + 1})g(x)=\answer{-5}\]
\end{problem}}%}

\latexProblemContent{
\ifVerboseLocation This is Derivative Concept Question 0001. \\ \fi
\begin{problem}

If you know that $\lim\limits_{x\to{3}}f(x)={-4}$ and $\lim\limits_{x\to0}g(x)={5}$, then evaluate the following limit:

\input{Limit-Concept-0001.HELP.tex}

\[\lim_{x\to0}f({x + 3})g(x)=\answer{-20}\]
\end{problem}}%}

\latexProblemContent{
\ifVerboseLocation This is Derivative Concept Question 0001. \\ \fi
\begin{problem}

If you know that $\lim\limits_{x\to{-1}}f(x)={2}$ and $\lim\limits_{x\to0}g(x)={5}$, then evaluate the following limit:

\input{Limit-Concept-0001.HELP.tex}

\[\lim_{x\to0}f({x - 1})g(x)=\answer{10}\]
\end{problem}}%}

\latexProblemContent{
\ifVerboseLocation This is Derivative Concept Question 0001. \\ \fi
\begin{problem}

If you know that $\lim\limits_{x\to{-4}}f(x)={3}$ and $\lim\limits_{x\to0}g(x)={3}$, then evaluate the following limit:

\input{Limit-Concept-0001.HELP.tex}

\[\lim_{x\to0}f({x - 4})g(x)=\answer{9}\]
\end{problem}}%}

\latexProblemContent{
\ifVerboseLocation This is Derivative Concept Question 0001. \\ \fi
\begin{problem}

If you know that $\lim\limits_{x\to{2}}f(x)={5}$ and $\lim\limits_{x\to0}g(x)={-2}$, then evaluate the following limit:

\input{Limit-Concept-0001.HELP.tex}

\[\lim_{x\to0}f({x + 2})g(x)=\answer{-10}\]
\end{problem}}%}

\latexProblemContent{
\ifVerboseLocation This is Derivative Concept Question 0001. \\ \fi
\begin{problem}

If you know that $\lim\limits_{x\to{5}}f(x)={-2}$ and $\lim\limits_{x\to0}g(x)={1}$, then evaluate the following limit:

\input{Limit-Concept-0001.HELP.tex}

\[\lim_{x\to0}f({x + 5})g(x)=\answer{-2}\]
\end{problem}}%}

\latexProblemContent{
\ifVerboseLocation This is Derivative Concept Question 0001. \\ \fi
\begin{problem}

If you know that $\lim\limits_{x\to{3}}f(x)={-4}$ and $\lim\limits_{x\to0}g(x)={2}$, then evaluate the following limit:

\input{Limit-Concept-0001.HELP.tex}

\[\lim_{x\to0}f({x + 3})g(x)=\answer{-8}\]
\end{problem}}%}

\latexProblemContent{
\ifVerboseLocation This is Derivative Concept Question 0001. \\ \fi
\begin{problem}

If you know that $\lim\limits_{x\to{-2}}f(x)={5}$ and $\lim\limits_{x\to0}g(x)={-3}$, then evaluate the following limit:

\input{Limit-Concept-0001.HELP.tex}

\[\lim_{x\to0}f({x - 2})g(x)=\answer{-15}\]
\end{problem}}%}

\latexProblemContent{
\ifVerboseLocation This is Derivative Concept Question 0001. \\ \fi
\begin{problem}

If you know that $\lim\limits_{x\to{-4}}f(x)={5}$ and $\lim\limits_{x\to0}g(x)={-5}$, then evaluate the following limit:

\input{Limit-Concept-0001.HELP.tex}

\[\lim_{x\to0}f({x - 4})g(x)=\answer{-25}\]
\end{problem}}%}

\latexProblemContent{
\ifVerboseLocation This is Derivative Concept Question 0001. \\ \fi
\begin{problem}

If you know that $\lim\limits_{x\to{5}}f(x)={3}$ and $\lim\limits_{x\to0}g(x)={-3}$, then evaluate the following limit:

\input{Limit-Concept-0001.HELP.tex}

\[\lim_{x\to0}f({x + 5})g(x)=\answer{-9}\]
\end{problem}}%}

\latexProblemContent{
\ifVerboseLocation This is Derivative Concept Question 0001. \\ \fi
\begin{problem}

If you know that $\lim\limits_{x\to{3}}f(x)={5}$ and $\lim\limits_{x\to0}g(x)={-3}$, then evaluate the following limit:

\input{Limit-Concept-0001.HELP.tex}

\[\lim_{x\to0}f({x + 3})g(x)=\answer{-15}\]
\end{problem}}%}

\latexProblemContent{
\ifVerboseLocation This is Derivative Concept Question 0001. \\ \fi
\begin{problem}

If you know that $\lim\limits_{x\to{2}}f(x)={-4}$ and $\lim\limits_{x\to0}g(x)={2}$, then evaluate the following limit:

\input{Limit-Concept-0001.HELP.tex}

\[\lim_{x\to0}f({x + 2})g(x)=\answer{-8}\]
\end{problem}}%}

\latexProblemContent{
\ifVerboseLocation This is Derivative Concept Question 0001. \\ \fi
\begin{problem}

If you know that $\lim\limits_{x\to{3}}f(x)={-3}$ and $\lim\limits_{x\to0}g(x)={-3}$, then evaluate the following limit:

\input{Limit-Concept-0001.HELP.tex}

\[\lim_{x\to0}f({x + 3})g(x)=\answer{9}\]
\end{problem}}%}

\latexProblemContent{
\ifVerboseLocation This is Derivative Concept Question 0001. \\ \fi
\begin{problem}

If you know that $\lim\limits_{x\to{-2}}f(x)={2}$ and $\lim\limits_{x\to0}g(x)={-5}$, then evaluate the following limit:

\input{Limit-Concept-0001.HELP.tex}

\[\lim_{x\to0}f({x - 2})g(x)=\answer{-10}\]
\end{problem}}%}

\latexProblemContent{
\ifVerboseLocation This is Derivative Concept Question 0001. \\ \fi
\begin{problem}

If you know that $\lim\limits_{x\to{-4}}f(x)={4}$ and $\lim\limits_{x\to0}g(x)={-2}$, then evaluate the following limit:

\input{Limit-Concept-0001.HELP.tex}

\[\lim_{x\to0}f({x - 4})g(x)=\answer{-8}\]
\end{problem}}%}

\latexProblemContent{
\ifVerboseLocation This is Derivative Concept Question 0001. \\ \fi
\begin{problem}

If you know that $\lim\limits_{x\to{-2}}f(x)={-5}$ and $\lim\limits_{x\to0}g(x)={-2}$, then evaluate the following limit:

\input{Limit-Concept-0001.HELP.tex}

\[\lim_{x\to0}f({x - 2})g(x)=\answer{10}\]
\end{problem}}%}

\latexProblemContent{
\ifVerboseLocation This is Derivative Concept Question 0001. \\ \fi
\begin{problem}

If you know that $\lim\limits_{x\to{-1}}f(x)={-4}$ and $\lim\limits_{x\to0}g(x)={5}$, then evaluate the following limit:

\input{Limit-Concept-0001.HELP.tex}

\[\lim_{x\to0}f({x - 1})g(x)=\answer{-20}\]
\end{problem}}%}

\latexProblemContent{
\ifVerboseLocation This is Derivative Concept Question 0001. \\ \fi
\begin{problem}

If you know that $\lim\limits_{x\to{-5}}f(x)={-4}$ and $\lim\limits_{x\to0}g(x)={2}$, then evaluate the following limit:

\input{Limit-Concept-0001.HELP.tex}

\[\lim_{x\to0}f({x - 5})g(x)=\answer{-8}\]
\end{problem}}%}

\latexProblemContent{
\ifVerboseLocation This is Derivative Concept Question 0001. \\ \fi
\begin{problem}

If you know that $\lim\limits_{x\to{1}}f(x)={5}$ and $\lim\limits_{x\to0}g(x)={1}$, then evaluate the following limit:

\input{Limit-Concept-0001.HELP.tex}

\[\lim_{x\to0}f({x + 1})g(x)=\answer{5}\]
\end{problem}}%}

\latexProblemContent{
\ifVerboseLocation This is Derivative Concept Question 0001. \\ \fi
\begin{problem}

If you know that $\lim\limits_{x\to{5}}f(x)={1}$ and $\lim\limits_{x\to0}g(x)={5}$, then evaluate the following limit:

\input{Limit-Concept-0001.HELP.tex}

\[\lim_{x\to0}f({x + 5})g(x)=\answer{5}\]
\end{problem}}%}

\latexProblemContent{
\ifVerboseLocation This is Derivative Concept Question 0001. \\ \fi
\begin{problem}

If you know that $\lim\limits_{x\to{4}}f(x)={4}$ and $\lim\limits_{x\to0}g(x)={-2}$, then evaluate the following limit:

\input{Limit-Concept-0001.HELP.tex}

\[\lim_{x\to0}f({x + 4})g(x)=\answer{-8}\]
\end{problem}}%}

\latexProblemContent{
\ifVerboseLocation This is Derivative Concept Question 0001. \\ \fi
\begin{problem}

If you know that $\lim\limits_{x\to{-2}}f(x)={-4}$ and $\lim\limits_{x\to0}g(x)={4}$, then evaluate the following limit:

\input{Limit-Concept-0001.HELP.tex}

\[\lim_{x\to0}f({x - 2})g(x)=\answer{-16}\]
\end{problem}}%}

\latexProblemContent{
\ifVerboseLocation This is Derivative Concept Question 0001. \\ \fi
\begin{problem}

If you know that $\lim\limits_{x\to{3}}f(x)={5}$ and $\lim\limits_{x\to0}g(x)={5}$, then evaluate the following limit:

\input{Limit-Concept-0001.HELP.tex}

\[\lim_{x\to0}f({x + 3})g(x)=\answer{25}\]
\end{problem}}%}

\latexProblemContent{
\ifVerboseLocation This is Derivative Concept Question 0001. \\ \fi
\begin{problem}

If you know that $\lim\limits_{x\to{-5}}f(x)={-3}$ and $\lim\limits_{x\to0}g(x)={5}$, then evaluate the following limit:

\input{Limit-Concept-0001.HELP.tex}

\[\lim_{x\to0}f({x - 5})g(x)=\answer{-15}\]
\end{problem}}%}

\latexProblemContent{
\ifVerboseLocation This is Derivative Concept Question 0001. \\ \fi
\begin{problem}

If you know that $\lim\limits_{x\to{2}}f(x)={-4}$ and $\lim\limits_{x\to0}g(x)={-4}$, then evaluate the following limit:

\input{Limit-Concept-0001.HELP.tex}

\[\lim_{x\to0}f({x + 2})g(x)=\answer{16}\]
\end{problem}}%}

\latexProblemContent{
\ifVerboseLocation This is Derivative Concept Question 0001. \\ \fi
\begin{problem}

If you know that $\lim\limits_{x\to{5}}f(x)={1}$ and $\lim\limits_{x\to0}g(x)={-2}$, then evaluate the following limit:

\input{Limit-Concept-0001.HELP.tex}

\[\lim_{x\to0}f({x + 5})g(x)=\answer{-2}\]
\end{problem}}%}

\latexProblemContent{
\ifVerboseLocation This is Derivative Concept Question 0001. \\ \fi
\begin{problem}

If you know that $\lim\limits_{x\to{-1}}f(x)={-5}$ and $\lim\limits_{x\to0}g(x)={5}$, then evaluate the following limit:

\input{Limit-Concept-0001.HELP.tex}

\[\lim_{x\to0}f({x - 1})g(x)=\answer{-25}\]
\end{problem}}%}

\latexProblemContent{
\ifVerboseLocation This is Derivative Concept Question 0001. \\ \fi
\begin{problem}

If you know that $\lim\limits_{x\to{-2}}f(x)={-2}$ and $\lim\limits_{x\to0}g(x)={-5}$, then evaluate the following limit:

\input{Limit-Concept-0001.HELP.tex}

\[\lim_{x\to0}f({x - 2})g(x)=\answer{10}\]
\end{problem}}%}

\latexProblemContent{
\ifVerboseLocation This is Derivative Concept Question 0001. \\ \fi
\begin{problem}

If you know that $\lim\limits_{x\to{-1}}f(x)={-1}$ and $\lim\limits_{x\to0}g(x)={-4}$, then evaluate the following limit:

\input{Limit-Concept-0001.HELP.tex}

\[\lim_{x\to0}f({x - 1})g(x)=\answer{4}\]
\end{problem}}%}

\latexProblemContent{
\ifVerboseLocation This is Derivative Concept Question 0001. \\ \fi
\begin{problem}

If you know that $\lim\limits_{x\to{2}}f(x)={-5}$ and $\lim\limits_{x\to0}g(x)={-4}$, then evaluate the following limit:

\input{Limit-Concept-0001.HELP.tex}

\[\lim_{x\to0}f({x + 2})g(x)=\answer{20}\]
\end{problem}}%}

\latexProblemContent{
\ifVerboseLocation This is Derivative Concept Question 0001. \\ \fi
\begin{problem}

If you know that $\lim\limits_{x\to{-3}}f(x)={5}$ and $\lim\limits_{x\to0}g(x)={4}$, then evaluate the following limit:

\input{Limit-Concept-0001.HELP.tex}

\[\lim_{x\to0}f({x - 3})g(x)=\answer{20}\]
\end{problem}}%}

\latexProblemContent{
\ifVerboseLocation This is Derivative Concept Question 0001. \\ \fi
\begin{problem}

If you know that $\lim\limits_{x\to{-3}}f(x)={-1}$ and $\lim\limits_{x\to0}g(x)={3}$, then evaluate the following limit:

\input{Limit-Concept-0001.HELP.tex}

\[\lim_{x\to0}f({x - 3})g(x)=\answer{-3}\]
\end{problem}}%}

\latexProblemContent{
\ifVerboseLocation This is Derivative Concept Question 0001. \\ \fi
\begin{problem}

If you know that $\lim\limits_{x\to{-1}}f(x)={-2}$ and $\lim\limits_{x\to0}g(x)={2}$, then evaluate the following limit:

\input{Limit-Concept-0001.HELP.tex}

\[\lim_{x\to0}f({x - 1})g(x)=\answer{-4}\]
\end{problem}}%}

\latexProblemContent{
\ifVerboseLocation This is Derivative Concept Question 0001. \\ \fi
\begin{problem}

If you know that $\lim\limits_{x\to{1}}f(x)={-2}$ and $\lim\limits_{x\to0}g(x)={3}$, then evaluate the following limit:

\input{Limit-Concept-0001.HELP.tex}

\[\lim_{x\to0}f({x + 1})g(x)=\answer{-6}\]
\end{problem}}%}

\latexProblemContent{
\ifVerboseLocation This is Derivative Concept Question 0001. \\ \fi
\begin{problem}

If you know that $\lim\limits_{x\to{-5}}f(x)={-3}$ and $\lim\limits_{x\to0}g(x)={-4}$, then evaluate the following limit:

\input{Limit-Concept-0001.HELP.tex}

\[\lim_{x\to0}f({x - 5})g(x)=\answer{12}\]
\end{problem}}%}

\latexProblemContent{
\ifVerboseLocation This is Derivative Concept Question 0001. \\ \fi
\begin{problem}

If you know that $\lim\limits_{x\to{2}}f(x)={5}$ and $\lim\limits_{x\to0}g(x)={5}$, then evaluate the following limit:

\input{Limit-Concept-0001.HELP.tex}

\[\lim_{x\to0}f({x + 2})g(x)=\answer{25}\]
\end{problem}}%}

\latexProblemContent{
\ifVerboseLocation This is Derivative Concept Question 0001. \\ \fi
\begin{problem}

If you know that $\lim\limits_{x\to{-3}}f(x)={5}$ and $\lim\limits_{x\to0}g(x)={2}$, then evaluate the following limit:

\input{Limit-Concept-0001.HELP.tex}

\[\lim_{x\to0}f({x - 3})g(x)=\answer{10}\]
\end{problem}}%}

\latexProblemContent{
\ifVerboseLocation This is Derivative Concept Question 0001. \\ \fi
\begin{problem}

If you know that $\lim\limits_{x\to{1}}f(x)={2}$ and $\lim\limits_{x\to0}g(x)={5}$, then evaluate the following limit:

\input{Limit-Concept-0001.HELP.tex}

\[\lim_{x\to0}f({x + 1})g(x)=\answer{10}\]
\end{problem}}%}

\latexProblemContent{
\ifVerboseLocation This is Derivative Concept Question 0001. \\ \fi
\begin{problem}

If you know that $\lim\limits_{x\to{1}}f(x)={-2}$ and $\lim\limits_{x\to0}g(x)={1}$, then evaluate the following limit:

\input{Limit-Concept-0001.HELP.tex}

\[\lim_{x\to0}f({x + 1})g(x)=\answer{-2}\]
\end{problem}}%}

\latexProblemContent{
\ifVerboseLocation This is Derivative Concept Question 0001. \\ \fi
\begin{problem}

If you know that $\lim\limits_{x\to{1}}f(x)={4}$ and $\lim\limits_{x\to0}g(x)={-2}$, then evaluate the following limit:

\input{Limit-Concept-0001.HELP.tex}

\[\lim_{x\to0}f({x + 1})g(x)=\answer{-8}\]
\end{problem}}%}

\latexProblemContent{
\ifVerboseLocation This is Derivative Concept Question 0001. \\ \fi
\begin{problem}

If you know that $\lim\limits_{x\to{-4}}f(x)={-1}$ and $\lim\limits_{x\to0}g(x)={2}$, then evaluate the following limit:

\input{Limit-Concept-0001.HELP.tex}

\[\lim_{x\to0}f({x - 4})g(x)=\answer{-2}\]
\end{problem}}%}

\latexProblemContent{
\ifVerboseLocation This is Derivative Concept Question 0001. \\ \fi
\begin{problem}

If you know that $\lim\limits_{x\to{4}}f(x)={1}$ and $\lim\limits_{x\to0}g(x)={-4}$, then evaluate the following limit:

\input{Limit-Concept-0001.HELP.tex}

\[\lim_{x\to0}f({x + 4})g(x)=\answer{-4}\]
\end{problem}}%}

\latexProblemContent{
\ifVerboseLocation This is Derivative Concept Question 0001. \\ \fi
\begin{problem}

If you know that $\lim\limits_{x\to{3}}f(x)={-3}$ and $\lim\limits_{x\to0}g(x)={-4}$, then evaluate the following limit:

\input{Limit-Concept-0001.HELP.tex}

\[\lim_{x\to0}f({x + 3})g(x)=\answer{12}\]
\end{problem}}%}

\latexProblemContent{
\ifVerboseLocation This is Derivative Concept Question 0001. \\ \fi
\begin{problem}

If you know that $\lim\limits_{x\to{-5}}f(x)={3}$ and $\lim\limits_{x\to0}g(x)={-4}$, then evaluate the following limit:

\input{Limit-Concept-0001.HELP.tex}

\[\lim_{x\to0}f({x - 5})g(x)=\answer{-12}\]
\end{problem}}%}

\latexProblemContent{
\ifVerboseLocation This is Derivative Concept Question 0001. \\ \fi
\begin{problem}

If you know that $\lim\limits_{x\to{5}}f(x)={-4}$ and $\lim\limits_{x\to0}g(x)={-3}$, then evaluate the following limit:

\input{Limit-Concept-0001.HELP.tex}

\[\lim_{x\to0}f({x + 5})g(x)=\answer{12}\]
\end{problem}}%}

\latexProblemContent{
\ifVerboseLocation This is Derivative Concept Question 0001. \\ \fi
\begin{problem}

If you know that $\lim\limits_{x\to{-3}}f(x)={1}$ and $\lim\limits_{x\to0}g(x)={4}$, then evaluate the following limit:

\input{Limit-Concept-0001.HELP.tex}

\[\lim_{x\to0}f({x - 3})g(x)=\answer{4}\]
\end{problem}}%}

\latexProblemContent{
\ifVerboseLocation This is Derivative Concept Question 0001. \\ \fi
\begin{problem}

If you know that $\lim\limits_{x\to{3}}f(x)={3}$ and $\lim\limits_{x\to0}g(x)={-2}$, then evaluate the following limit:

\input{Limit-Concept-0001.HELP.tex}

\[\lim_{x\to0}f({x + 3})g(x)=\answer{-6}\]
\end{problem}}%}

\latexProblemContent{
\ifVerboseLocation This is Derivative Concept Question 0001. \\ \fi
\begin{problem}

If you know that $\lim\limits_{x\to{5}}f(x)={5}$ and $\lim\limits_{x\to0}g(x)={-2}$, then evaluate the following limit:

\input{Limit-Concept-0001.HELP.tex}

\[\lim_{x\to0}f({x + 5})g(x)=\answer{-10}\]
\end{problem}}%}

\latexProblemContent{
\ifVerboseLocation This is Derivative Concept Question 0001. \\ \fi
\begin{problem}

If you know that $\lim\limits_{x\to{-2}}f(x)={3}$ and $\lim\limits_{x\to0}g(x)={1}$, then evaluate the following limit:

\input{Limit-Concept-0001.HELP.tex}

\[\lim_{x\to0}f({x - 2})g(x)=\answer{3}\]
\end{problem}}%}

\latexProblemContent{
\ifVerboseLocation This is Derivative Concept Question 0001. \\ \fi
\begin{problem}

If you know that $\lim\limits_{x\to{-1}}f(x)={-5}$ and $\lim\limits_{x\to0}g(x)={-5}$, then evaluate the following limit:

\input{Limit-Concept-0001.HELP.tex}

\[\lim_{x\to0}f({x - 1})g(x)=\answer{25}\]
\end{problem}}%}

\latexProblemContent{
\ifVerboseLocation This is Derivative Concept Question 0001. \\ \fi
\begin{problem}

If you know that $\lim\limits_{x\to{1}}f(x)={5}$ and $\lim\limits_{x\to0}g(x)={4}$, then evaluate the following limit:

\input{Limit-Concept-0001.HELP.tex}

\[\lim_{x\to0}f({x + 1})g(x)=\answer{20}\]
\end{problem}}%}

\latexProblemContent{
\ifVerboseLocation This is Derivative Concept Question 0001. \\ \fi
\begin{problem}

If you know that $\lim\limits_{x\to{1}}f(x)={2}$ and $\lim\limits_{x\to0}g(x)={-3}$, then evaluate the following limit:

\input{Limit-Concept-0001.HELP.tex}

\[\lim_{x\to0}f({x + 1})g(x)=\answer{-6}\]
\end{problem}}%}

\latexProblemContent{
\ifVerboseLocation This is Derivative Concept Question 0001. \\ \fi
\begin{problem}

If you know that $\lim\limits_{x\to{-1}}f(x)={5}$ and $\lim\limits_{x\to0}g(x)={1}$, then evaluate the following limit:

\input{Limit-Concept-0001.HELP.tex}

\[\lim_{x\to0}f({x - 1})g(x)=\answer{5}\]
\end{problem}}%}

\latexProblemContent{
\ifVerboseLocation This is Derivative Concept Question 0001. \\ \fi
\begin{problem}

If you know that $\lim\limits_{x\to{-1}}f(x)={-2}$ and $\lim\limits_{x\to0}g(x)={5}$, then evaluate the following limit:

\input{Limit-Concept-0001.HELP.tex}

\[\lim_{x\to0}f({x - 1})g(x)=\answer{-10}\]
\end{problem}}%}

\latexProblemContent{
\ifVerboseLocation This is Derivative Concept Question 0001. \\ \fi
\begin{problem}

If you know that $\lim\limits_{x\to{-5}}f(x)={-3}$ and $\lim\limits_{x\to0}g(x)={-5}$, then evaluate the following limit:

\input{Limit-Concept-0001.HELP.tex}

\[\lim_{x\to0}f({x - 5})g(x)=\answer{15}\]
\end{problem}}%}

\latexProblemContent{
\ifVerboseLocation This is Derivative Concept Question 0001. \\ \fi
\begin{problem}

If you know that $\lim\limits_{x\to{-4}}f(x)={5}$ and $\lim\limits_{x\to0}g(x)={4}$, then evaluate the following limit:

\input{Limit-Concept-0001.HELP.tex}

\[\lim_{x\to0}f({x - 4})g(x)=\answer{20}\]
\end{problem}}%}

\latexProblemContent{
\ifVerboseLocation This is Derivative Concept Question 0001. \\ \fi
\begin{problem}

If you know that $\lim\limits_{x\to{1}}f(x)={-2}$ and $\lim\limits_{x\to0}g(x)={2}$, then evaluate the following limit:

\input{Limit-Concept-0001.HELP.tex}

\[\lim_{x\to0}f({x + 1})g(x)=\answer{-4}\]
\end{problem}}%}

\latexProblemContent{
\ifVerboseLocation This is Derivative Concept Question 0001. \\ \fi
\begin{problem}

If you know that $\lim\limits_{x\to{-4}}f(x)={-4}$ and $\lim\limits_{x\to0}g(x)={-2}$, then evaluate the following limit:

\input{Limit-Concept-0001.HELP.tex}

\[\lim_{x\to0}f({x - 4})g(x)=\answer{8}\]
\end{problem}}%}

\latexProblemContent{
\ifVerboseLocation This is Derivative Concept Question 0001. \\ \fi
\begin{problem}

If you know that $\lim\limits_{x\to{2}}f(x)={-5}$ and $\lim\limits_{x\to0}g(x)={1}$, then evaluate the following limit:

\input{Limit-Concept-0001.HELP.tex}

\[\lim_{x\to0}f({x + 2})g(x)=\answer{-5}\]
\end{problem}}%}

\latexProblemContent{
\ifVerboseLocation This is Derivative Concept Question 0001. \\ \fi
\begin{problem}

If you know that $\lim\limits_{x\to{-3}}f(x)={-3}$ and $\lim\limits_{x\to0}g(x)={-2}$, then evaluate the following limit:

\input{Limit-Concept-0001.HELP.tex}

\[\lim_{x\to0}f({x - 3})g(x)=\answer{6}\]
\end{problem}}%}

\latexProblemContent{
\ifVerboseLocation This is Derivative Concept Question 0001. \\ \fi
\begin{problem}

If you know that $\lim\limits_{x\to{1}}f(x)={-1}$ and $\lim\limits_{x\to0}g(x)={-2}$, then evaluate the following limit:

\input{Limit-Concept-0001.HELP.tex}

\[\lim_{x\to0}f({x + 1})g(x)=\answer{2}\]
\end{problem}}%}

\latexProblemContent{
\ifVerboseLocation This is Derivative Concept Question 0001. \\ \fi
\begin{problem}

If you know that $\lim\limits_{x\to{-5}}f(x)={4}$ and $\lim\limits_{x\to0}g(x)={-5}$, then evaluate the following limit:

\input{Limit-Concept-0001.HELP.tex}

\[\lim_{x\to0}f({x - 5})g(x)=\answer{-20}\]
\end{problem}}%}

\latexProblemContent{
\ifVerboseLocation This is Derivative Concept Question 0001. \\ \fi
\begin{problem}

If you know that $\lim\limits_{x\to{-1}}f(x)={3}$ and $\lim\limits_{x\to0}g(x)={-1}$, then evaluate the following limit:

\input{Limit-Concept-0001.HELP.tex}

\[\lim_{x\to0}f({x - 1})g(x)=\answer{-3}\]
\end{problem}}%}

\latexProblemContent{
\ifVerboseLocation This is Derivative Concept Question 0001. \\ \fi
\begin{problem}

If you know that $\lim\limits_{x\to{-3}}f(x)={-5}$ and $\lim\limits_{x\to0}g(x)={-1}$, then evaluate the following limit:

\input{Limit-Concept-0001.HELP.tex}

\[\lim_{x\to0}f({x - 3})g(x)=\answer{5}\]
\end{problem}}%}

\latexProblemContent{
\ifVerboseLocation This is Derivative Concept Question 0001. \\ \fi
\begin{problem}

If you know that $\lim\limits_{x\to{-1}}f(x)={3}$ and $\lim\limits_{x\to0}g(x)={-4}$, then evaluate the following limit:

\input{Limit-Concept-0001.HELP.tex}

\[\lim_{x\to0}f({x - 1})g(x)=\answer{-12}\]
\end{problem}}%}

\latexProblemContent{
\ifVerboseLocation This is Derivative Concept Question 0001. \\ \fi
\begin{problem}

If you know that $\lim\limits_{x\to{-1}}f(x)={-1}$ and $\lim\limits_{x\to0}g(x)={-2}$, then evaluate the following limit:

\input{Limit-Concept-0001.HELP.tex}

\[\lim_{x\to0}f({x - 1})g(x)=\answer{2}\]
\end{problem}}%}

\latexProblemContent{
\ifVerboseLocation This is Derivative Concept Question 0001. \\ \fi
\begin{problem}

If you know that $\lim\limits_{x\to{-3}}f(x)={4}$ and $\lim\limits_{x\to0}g(x)={5}$, then evaluate the following limit:

\input{Limit-Concept-0001.HELP.tex}

\[\lim_{x\to0}f({x - 3})g(x)=\answer{20}\]
\end{problem}}%}

\latexProblemContent{
\ifVerboseLocation This is Derivative Concept Question 0001. \\ \fi
\begin{problem}

If you know that $\lim\limits_{x\to{4}}f(x)={2}$ and $\lim\limits_{x\to0}g(x)={3}$, then evaluate the following limit:

\input{Limit-Concept-0001.HELP.tex}

\[\lim_{x\to0}f({x + 4})g(x)=\answer{6}\]
\end{problem}}%}

\latexProblemContent{
\ifVerboseLocation This is Derivative Concept Question 0001. \\ \fi
\begin{problem}

If you know that $\lim\limits_{x\to{-1}}f(x)={-3}$ and $\lim\limits_{x\to0}g(x)={-1}$, then evaluate the following limit:

\input{Limit-Concept-0001.HELP.tex}

\[\lim_{x\to0}f({x - 1})g(x)=\answer{3}\]
\end{problem}}%}

\latexProblemContent{
\ifVerboseLocation This is Derivative Concept Question 0001. \\ \fi
\begin{problem}

If you know that $\lim\limits_{x\to{-4}}f(x)={-3}$ and $\lim\limits_{x\to0}g(x)={-4}$, then evaluate the following limit:

\input{Limit-Concept-0001.HELP.tex}

\[\lim_{x\to0}f({x - 4})g(x)=\answer{12}\]
\end{problem}}%}

\latexProblemContent{
\ifVerboseLocation This is Derivative Concept Question 0001. \\ \fi
\begin{problem}

If you know that $\lim\limits_{x\to{3}}f(x)={-3}$ and $\lim\limits_{x\to0}g(x)={4}$, then evaluate the following limit:

\input{Limit-Concept-0001.HELP.tex}

\[\lim_{x\to0}f({x + 3})g(x)=\answer{-12}\]
\end{problem}}%}

\latexProblemContent{
\ifVerboseLocation This is Derivative Concept Question 0001. \\ \fi
\begin{problem}

If you know that $\lim\limits_{x\to{2}}f(x)={4}$ and $\lim\limits_{x\to0}g(x)={-2}$, then evaluate the following limit:

\input{Limit-Concept-0001.HELP.tex}

\[\lim_{x\to0}f({x + 2})g(x)=\answer{-8}\]
\end{problem}}%}

\latexProblemContent{
\ifVerboseLocation This is Derivative Concept Question 0001. \\ \fi
\begin{problem}

If you know that $\lim\limits_{x\to{-2}}f(x)={1}$ and $\lim\limits_{x\to0}g(x)={-3}$, then evaluate the following limit:

\input{Limit-Concept-0001.HELP.tex}

\[\lim_{x\to0}f({x - 2})g(x)=\answer{-3}\]
\end{problem}}%}

\latexProblemContent{
\ifVerboseLocation This is Derivative Concept Question 0001. \\ \fi
\begin{problem}

If you know that $\lim\limits_{x\to{-5}}f(x)={4}$ and $\lim\limits_{x\to0}g(x)={-2}$, then evaluate the following limit:

\input{Limit-Concept-0001.HELP.tex}

\[\lim_{x\to0}f({x - 5})g(x)=\answer{-8}\]
\end{problem}}%}

\latexProblemContent{
\ifVerboseLocation This is Derivative Concept Question 0001. \\ \fi
\begin{problem}

If you know that $\lim\limits_{x\to{2}}f(x)={-2}$ and $\lim\limits_{x\to0}g(x)={-4}$, then evaluate the following limit:

\input{Limit-Concept-0001.HELP.tex}

\[\lim_{x\to0}f({x + 2})g(x)=\answer{8}\]
\end{problem}}%}

\latexProblemContent{
\ifVerboseLocation This is Derivative Concept Question 0001. \\ \fi
\begin{problem}

If you know that $\lim\limits_{x\to{-4}}f(x)={3}$ and $\lim\limits_{x\to0}g(x)={4}$, then evaluate the following limit:

\input{Limit-Concept-0001.HELP.tex}

\[\lim_{x\to0}f({x - 4})g(x)=\answer{12}\]
\end{problem}}%}

\latexProblemContent{
\ifVerboseLocation This is Derivative Concept Question 0001. \\ \fi
\begin{problem}

If you know that $\lim\limits_{x\to{3}}f(x)={4}$ and $\lim\limits_{x\to0}g(x)={-2}$, then evaluate the following limit:

\input{Limit-Concept-0001.HELP.tex}

\[\lim_{x\to0}f({x + 3})g(x)=\answer{-8}\]
\end{problem}}%}

\latexProblemContent{
\ifVerboseLocation This is Derivative Concept Question 0001. \\ \fi
\begin{problem}

If you know that $\lim\limits_{x\to{4}}f(x)={2}$ and $\lim\limits_{x\to0}g(x)={4}$, then evaluate the following limit:

\input{Limit-Concept-0001.HELP.tex}

\[\lim_{x\to0}f({x + 4})g(x)=\answer{8}\]
\end{problem}}%}

\latexProblemContent{
\ifVerboseLocation This is Derivative Concept Question 0001. \\ \fi
\begin{problem}

If you know that $\lim\limits_{x\to{5}}f(x)={5}$ and $\lim\limits_{x\to0}g(x)={-1}$, then evaluate the following limit:

\input{Limit-Concept-0001.HELP.tex}

\[\lim_{x\to0}f({x + 5})g(x)=\answer{-5}\]
\end{problem}}%}

\latexProblemContent{
\ifVerboseLocation This is Derivative Concept Question 0001. \\ \fi
\begin{problem}

If you know that $\lim\limits_{x\to{4}}f(x)={-1}$ and $\lim\limits_{x\to0}g(x)={-3}$, then evaluate the following limit:

\input{Limit-Concept-0001.HELP.tex}

\[\lim_{x\to0}f({x + 4})g(x)=\answer{3}\]
\end{problem}}%}

\latexProblemContent{
\ifVerboseLocation This is Derivative Concept Question 0001. \\ \fi
\begin{problem}

If you know that $\lim\limits_{x\to{-2}}f(x)={1}$ and $\lim\limits_{x\to0}g(x)={-1}$, then evaluate the following limit:

\input{Limit-Concept-0001.HELP.tex}

\[\lim_{x\to0}f({x - 2})g(x)=\answer{-1}\]
\end{problem}}%}

\latexProblemContent{
\ifVerboseLocation This is Derivative Concept Question 0001. \\ \fi
\begin{problem}

If you know that $\lim\limits_{x\to{-3}}f(x)={-5}$ and $\lim\limits_{x\to0}g(x)={-4}$, then evaluate the following limit:

\input{Limit-Concept-0001.HELP.tex}

\[\lim_{x\to0}f({x - 3})g(x)=\answer{20}\]
\end{problem}}%}

\latexProblemContent{
\ifVerboseLocation This is Derivative Concept Question 0001. \\ \fi
\begin{problem}

If you know that $\lim\limits_{x\to{3}}f(x)={1}$ and $\lim\limits_{x\to0}g(x)={-1}$, then evaluate the following limit:

\input{Limit-Concept-0001.HELP.tex}

\[\lim_{x\to0}f({x + 3})g(x)=\answer{-1}\]
\end{problem}}%}

\latexProblemContent{
\ifVerboseLocation This is Derivative Concept Question 0001. \\ \fi
\begin{problem}

If you know that $\lim\limits_{x\to{1}}f(x)={-5}$ and $\lim\limits_{x\to0}g(x)={4}$, then evaluate the following limit:

\input{Limit-Concept-0001.HELP.tex}

\[\lim_{x\to0}f({x + 1})g(x)=\answer{-20}\]
\end{problem}}%}

\latexProblemContent{
\ifVerboseLocation This is Derivative Concept Question 0001. \\ \fi
\begin{problem}

If you know that $\lim\limits_{x\to{2}}f(x)={1}$ and $\lim\limits_{x\to0}g(x)={4}$, then evaluate the following limit:

\input{Limit-Concept-0001.HELP.tex}

\[\lim_{x\to0}f({x + 2})g(x)=\answer{4}\]
\end{problem}}%}

\latexProblemContent{
\ifVerboseLocation This is Derivative Concept Question 0001. \\ \fi
\begin{problem}

If you know that $\lim\limits_{x\to{-3}}f(x)={-2}$ and $\lim\limits_{x\to0}g(x)={5}$, then evaluate the following limit:

\input{Limit-Concept-0001.HELP.tex}

\[\lim_{x\to0}f({x - 3})g(x)=\answer{-10}\]
\end{problem}}%}

\latexProblemContent{
\ifVerboseLocation This is Derivative Concept Question 0001. \\ \fi
\begin{problem}

If you know that $\lim\limits_{x\to{1}}f(x)={-3}$ and $\lim\limits_{x\to0}g(x)={3}$, then evaluate the following limit:

\input{Limit-Concept-0001.HELP.tex}

\[\lim_{x\to0}f({x + 1})g(x)=\answer{-9}\]
\end{problem}}%}

\latexProblemContent{
\ifVerboseLocation This is Derivative Concept Question 0001. \\ \fi
\begin{problem}

If you know that $\lim\limits_{x\to{-4}}f(x)={-5}$ and $\lim\limits_{x\to0}g(x)={-1}$, then evaluate the following limit:

\input{Limit-Concept-0001.HELP.tex}

\[\lim_{x\to0}f({x - 4})g(x)=\answer{5}\]
\end{problem}}%}

\latexProblemContent{
\ifVerboseLocation This is Derivative Concept Question 0001. \\ \fi
\begin{problem}

If you know that $\lim\limits_{x\to{-1}}f(x)={3}$ and $\lim\limits_{x\to0}g(x)={-5}$, then evaluate the following limit:

\input{Limit-Concept-0001.HELP.tex}

\[\lim_{x\to0}f({x - 1})g(x)=\answer{-15}\]
\end{problem}}%}

\latexProblemContent{
\ifVerboseLocation This is Derivative Concept Question 0001. \\ \fi
\begin{problem}

If you know that $\lim\limits_{x\to{-4}}f(x)={-1}$ and $\lim\limits_{x\to0}g(x)={3}$, then evaluate the following limit:

\input{Limit-Concept-0001.HELP.tex}

\[\lim_{x\to0}f({x - 4})g(x)=\answer{-3}\]
\end{problem}}%}

\latexProblemContent{
\ifVerboseLocation This is Derivative Concept Question 0001. \\ \fi
\begin{problem}

If you know that $\lim\limits_{x\to{3}}f(x)={-4}$ and $\lim\limits_{x\to0}g(x)={1}$, then evaluate the following limit:

\input{Limit-Concept-0001.HELP.tex}

\[\lim_{x\to0}f({x + 3})g(x)=\answer{-4}\]
\end{problem}}%}

\latexProblemContent{
\ifVerboseLocation This is Derivative Concept Question 0001. \\ \fi
\begin{problem}

If you know that $\lim\limits_{x\to{4}}f(x)={5}$ and $\lim\limits_{x\to0}g(x)={-3}$, then evaluate the following limit:

\input{Limit-Concept-0001.HELP.tex}

\[\lim_{x\to0}f({x + 4})g(x)=\answer{-15}\]
\end{problem}}%}

\latexProblemContent{
\ifVerboseLocation This is Derivative Concept Question 0001. \\ \fi
\begin{problem}

If you know that $\lim\limits_{x\to{-2}}f(x)={-1}$ and $\lim\limits_{x\to0}g(x)={-5}$, then evaluate the following limit:

\input{Limit-Concept-0001.HELP.tex}

\[\lim_{x\to0}f({x - 2})g(x)=\answer{5}\]
\end{problem}}%}

\latexProblemContent{
\ifVerboseLocation This is Derivative Concept Question 0001. \\ \fi
\begin{problem}

If you know that $\lim\limits_{x\to{3}}f(x)={2}$ and $\lim\limits_{x\to0}g(x)={-1}$, then evaluate the following limit:

\input{Limit-Concept-0001.HELP.tex}

\[\lim_{x\to0}f({x + 3})g(x)=\answer{-2}\]
\end{problem}}%}

\latexProblemContent{
\ifVerboseLocation This is Derivative Concept Question 0001. \\ \fi
\begin{problem}

If you know that $\lim\limits_{x\to{-5}}f(x)={-4}$ and $\lim\limits_{x\to0}g(x)={5}$, then evaluate the following limit:

\input{Limit-Concept-0001.HELP.tex}

\[\lim_{x\to0}f({x - 5})g(x)=\answer{-20}\]
\end{problem}}%}

\latexProblemContent{
\ifVerboseLocation This is Derivative Concept Question 0001. \\ \fi
\begin{problem}

If you know that $\lim\limits_{x\to{-2}}f(x)={-1}$ and $\lim\limits_{x\to0}g(x)={3}$, then evaluate the following limit:

\input{Limit-Concept-0001.HELP.tex}

\[\lim_{x\to0}f({x - 2})g(x)=\answer{-3}\]
\end{problem}}%}

\latexProblemContent{
\ifVerboseLocation This is Derivative Concept Question 0001. \\ \fi
\begin{problem}

If you know that $\lim\limits_{x\to{1}}f(x)={4}$ and $\lim\limits_{x\to0}g(x)={-4}$, then evaluate the following limit:

\input{Limit-Concept-0001.HELP.tex}

\[\lim_{x\to0}f({x + 1})g(x)=\answer{-16}\]
\end{problem}}%}

\latexProblemContent{
\ifVerboseLocation This is Derivative Concept Question 0001. \\ \fi
\begin{problem}

If you know that $\lim\limits_{x\to{1}}f(x)={3}$ and $\lim\limits_{x\to0}g(x)={4}$, then evaluate the following limit:

\input{Limit-Concept-0001.HELP.tex}

\[\lim_{x\to0}f({x + 1})g(x)=\answer{12}\]
\end{problem}}%}

\latexProblemContent{
\ifVerboseLocation This is Derivative Concept Question 0001. \\ \fi
\begin{problem}

If you know that $\lim\limits_{x\to{-5}}f(x)={2}$ and $\lim\limits_{x\to0}g(x)={-3}$, then evaluate the following limit:

\input{Limit-Concept-0001.HELP.tex}

\[\lim_{x\to0}f({x - 5})g(x)=\answer{-6}\]
\end{problem}}%}

\latexProblemContent{
\ifVerboseLocation This is Derivative Concept Question 0001. \\ \fi
\begin{problem}

If you know that $\lim\limits_{x\to{-5}}f(x)={1}$ and $\lim\limits_{x\to0}g(x)={-3}$, then evaluate the following limit:

\input{Limit-Concept-0001.HELP.tex}

\[\lim_{x\to0}f({x - 5})g(x)=\answer{-3}\]
\end{problem}}%}

\latexProblemContent{
\ifVerboseLocation This is Derivative Concept Question 0001. \\ \fi
\begin{problem}

If you know that $\lim\limits_{x\to{4}}f(x)={3}$ and $\lim\limits_{x\to0}g(x)={1}$, then evaluate the following limit:

\input{Limit-Concept-0001.HELP.tex}

\[\lim_{x\to0}f({x + 4})g(x)=\answer{3}\]
\end{problem}}%}

\latexProblemContent{
\ifVerboseLocation This is Derivative Concept Question 0001. \\ \fi
\begin{problem}

If you know that $\lim\limits_{x\to{-1}}f(x)={1}$ and $\lim\limits_{x\to0}g(x)={1}$, then evaluate the following limit:

\input{Limit-Concept-0001.HELP.tex}

\[\lim_{x\to0}f({x - 1})g(x)=\answer{1}\]
\end{problem}}%}

\latexProblemContent{
\ifVerboseLocation This is Derivative Concept Question 0001. \\ \fi
\begin{problem}

If you know that $\lim\limits_{x\to{2}}f(x)={2}$ and $\lim\limits_{x\to0}g(x)={4}$, then evaluate the following limit:

\input{Limit-Concept-0001.HELP.tex}

\[\lim_{x\to0}f({x + 2})g(x)=\answer{8}\]
\end{problem}}%}

\latexProblemContent{
\ifVerboseLocation This is Derivative Concept Question 0001. \\ \fi
\begin{problem}

If you know that $\lim\limits_{x\to{-5}}f(x)={4}$ and $\lim\limits_{x\to0}g(x)={2}$, then evaluate the following limit:

\input{Limit-Concept-0001.HELP.tex}

\[\lim_{x\to0}f({x - 5})g(x)=\answer{8}\]
\end{problem}}%}

\latexProblemContent{
\ifVerboseLocation This is Derivative Concept Question 0001. \\ \fi
\begin{problem}

If you know that $\lim\limits_{x\to{-2}}f(x)={-3}$ and $\lim\limits_{x\to0}g(x)={-4}$, then evaluate the following limit:

\input{Limit-Concept-0001.HELP.tex}

\[\lim_{x\to0}f({x - 2})g(x)=\answer{12}\]
\end{problem}}%}

\latexProblemContent{
\ifVerboseLocation This is Derivative Concept Question 0001. \\ \fi
\begin{problem}

If you know that $\lim\limits_{x\to{-4}}f(x)={3}$ and $\lim\limits_{x\to0}g(x)={-3}$, then evaluate the following limit:

\input{Limit-Concept-0001.HELP.tex}

\[\lim_{x\to0}f({x - 4})g(x)=\answer{-9}\]
\end{problem}}%}

\latexProblemContent{
\ifVerboseLocation This is Derivative Concept Question 0001. \\ \fi
\begin{problem}

If you know that $\lim\limits_{x\to{-5}}f(x)={-1}$ and $\lim\limits_{x\to0}g(x)={5}$, then evaluate the following limit:

\input{Limit-Concept-0001.HELP.tex}

\[\lim_{x\to0}f({x - 5})g(x)=\answer{-5}\]
\end{problem}}%}

\latexProblemContent{
\ifVerboseLocation This is Derivative Concept Question 0001. \\ \fi
\begin{problem}

If you know that $\lim\limits_{x\to{5}}f(x)={5}$ and $\lim\limits_{x\to0}g(x)={5}$, then evaluate the following limit:

\input{Limit-Concept-0001.HELP.tex}

\[\lim_{x\to0}f({x + 5})g(x)=\answer{25}\]
\end{problem}}%}

\latexProblemContent{
\ifVerboseLocation This is Derivative Concept Question 0001. \\ \fi
\begin{problem}

If you know that $\lim\limits_{x\to{-4}}f(x)={2}$ and $\lim\limits_{x\to0}g(x)={3}$, then evaluate the following limit:

\input{Limit-Concept-0001.HELP.tex}

\[\lim_{x\to0}f({x - 4})g(x)=\answer{6}\]
\end{problem}}%}

\latexProblemContent{
\ifVerboseLocation This is Derivative Concept Question 0001. \\ \fi
\begin{problem}

If you know that $\lim\limits_{x\to{-1}}f(x)={-4}$ and $\lim\limits_{x\to0}g(x)={-4}$, then evaluate the following limit:

\input{Limit-Concept-0001.HELP.tex}

\[\lim_{x\to0}f({x - 1})g(x)=\answer{16}\]
\end{problem}}%}

\latexProblemContent{
\ifVerboseLocation This is Derivative Concept Question 0001. \\ \fi
\begin{problem}

If you know that $\lim\limits_{x\to{-3}}f(x)={5}$ and $\lim\limits_{x\to0}g(x)={-4}$, then evaluate the following limit:

\input{Limit-Concept-0001.HELP.tex}

\[\lim_{x\to0}f({x - 3})g(x)=\answer{-20}\]
\end{problem}}%}

\latexProblemContent{
\ifVerboseLocation This is Derivative Concept Question 0001. \\ \fi
\begin{problem}

If you know that $\lim\limits_{x\to{5}}f(x)={5}$ and $\lim\limits_{x\to0}g(x)={3}$, then evaluate the following limit:

\input{Limit-Concept-0001.HELP.tex}

\[\lim_{x\to0}f({x + 5})g(x)=\answer{15}\]
\end{problem}}%}

\latexProblemContent{
\ifVerboseLocation This is Derivative Concept Question 0001. \\ \fi
\begin{problem}

If you know that $\lim\limits_{x\to{1}}f(x)={-4}$ and $\lim\limits_{x\to0}g(x)={-3}$, then evaluate the following limit:

\input{Limit-Concept-0001.HELP.tex}

\[\lim_{x\to0}f({x + 1})g(x)=\answer{12}\]
\end{problem}}%}

\latexProblemContent{
\ifVerboseLocation This is Derivative Concept Question 0001. \\ \fi
\begin{problem}

If you know that $\lim\limits_{x\to{4}}f(x)={4}$ and $\lim\limits_{x\to0}g(x)={2}$, then evaluate the following limit:

\input{Limit-Concept-0001.HELP.tex}

\[\lim_{x\to0}f({x + 4})g(x)=\answer{8}\]
\end{problem}}%}

\latexProblemContent{
\ifVerboseLocation This is Derivative Concept Question 0001. \\ \fi
\begin{problem}

If you know that $\lim\limits_{x\to{4}}f(x)={-2}$ and $\lim\limits_{x\to0}g(x)={2}$, then evaluate the following limit:

\input{Limit-Concept-0001.HELP.tex}

\[\lim_{x\to0}f({x + 4})g(x)=\answer{-4}\]
\end{problem}}%}

\latexProblemContent{
\ifVerboseLocation This is Derivative Concept Question 0001. \\ \fi
\begin{problem}

If you know that $\lim\limits_{x\to{1}}f(x)={-5}$ and $\lim\limits_{x\to0}g(x)={5}$, then evaluate the following limit:

\input{Limit-Concept-0001.HELP.tex}

\[\lim_{x\to0}f({x + 1})g(x)=\answer{-25}\]
\end{problem}}%}

\latexProblemContent{
\ifVerboseLocation This is Derivative Concept Question 0001. \\ \fi
\begin{problem}

If you know that $\lim\limits_{x\to{-5}}f(x)={3}$ and $\lim\limits_{x\to0}g(x)={-2}$, then evaluate the following limit:

\input{Limit-Concept-0001.HELP.tex}

\[\lim_{x\to0}f({x - 5})g(x)=\answer{-6}\]
\end{problem}}%}

\latexProblemContent{
\ifVerboseLocation This is Derivative Concept Question 0001. \\ \fi
\begin{problem}

If you know that $\lim\limits_{x\to{-4}}f(x)={-5}$ and $\lim\limits_{x\to0}g(x)={-5}$, then evaluate the following limit:

\input{Limit-Concept-0001.HELP.tex}

\[\lim_{x\to0}f({x - 4})g(x)=\answer{25}\]
\end{problem}}%}

\latexProblemContent{
\ifVerboseLocation This is Derivative Concept Question 0001. \\ \fi
\begin{problem}

If you know that $\lim\limits_{x\to{-4}}f(x)={-2}$ and $\lim\limits_{x\to0}g(x)={-5}$, then evaluate the following limit:

\input{Limit-Concept-0001.HELP.tex}

\[\lim_{x\to0}f({x - 4})g(x)=\answer{10}\]
\end{problem}}%}

\latexProblemContent{
\ifVerboseLocation This is Derivative Concept Question 0001. \\ \fi
\begin{problem}

If you know that $\lim\limits_{x\to{3}}f(x)={2}$ and $\lim\limits_{x\to0}g(x)={3}$, then evaluate the following limit:

\input{Limit-Concept-0001.HELP.tex}

\[\lim_{x\to0}f({x + 3})g(x)=\answer{6}\]
\end{problem}}%}

\latexProblemContent{
\ifVerboseLocation This is Derivative Concept Question 0001. \\ \fi
\begin{problem}

If you know that $\lim\limits_{x\to{4}}f(x)={-4}$ and $\lim\limits_{x\to0}g(x)={3}$, then evaluate the following limit:

\input{Limit-Concept-0001.HELP.tex}

\[\lim_{x\to0}f({x + 4})g(x)=\answer{-12}\]
\end{problem}}%}

\latexProblemContent{
\ifVerboseLocation This is Derivative Concept Question 0001. \\ \fi
\begin{problem}

If you know that $\lim\limits_{x\to{-2}}f(x)={5}$ and $\lim\limits_{x\to0}g(x)={-5}$, then evaluate the following limit:

\input{Limit-Concept-0001.HELP.tex}

\[\lim_{x\to0}f({x - 2})g(x)=\answer{-25}\]
\end{problem}}%}

\latexProblemContent{
\ifVerboseLocation This is Derivative Concept Question 0001. \\ \fi
\begin{problem}

If you know that $\lim\limits_{x\to{3}}f(x)={2}$ and $\lim\limits_{x\to0}g(x)={-5}$, then evaluate the following limit:

\input{Limit-Concept-0001.HELP.tex}

\[\lim_{x\to0}f({x + 3})g(x)=\answer{-10}\]
\end{problem}}%}

\latexProblemContent{
\ifVerboseLocation This is Derivative Concept Question 0001. \\ \fi
\begin{problem}

If you know that $\lim\limits_{x\to{2}}f(x)={5}$ and $\lim\limits_{x\to0}g(x)={2}$, then evaluate the following limit:

\input{Limit-Concept-0001.HELP.tex}

\[\lim_{x\to0}f({x + 2})g(x)=\answer{10}\]
\end{problem}}%}

\latexProblemContent{
\ifVerboseLocation This is Derivative Concept Question 0001. \\ \fi
\begin{problem}

If you know that $\lim\limits_{x\to{5}}f(x)={-3}$ and $\lim\limits_{x\to0}g(x)={2}$, then evaluate the following limit:

\input{Limit-Concept-0001.HELP.tex}

\[\lim_{x\to0}f({x + 5})g(x)=\answer{-6}\]
\end{problem}}%}

