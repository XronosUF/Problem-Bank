\ProblemFileHeader{XTL_SV_QUESTIONCOUNT}% Process how many problems are in this file and how to detect if it has a desirable problem
\ifproblemToFind% If it has a desirable problem search the file.
%\tagged{Ans@ShortAns, Type@Compute, Topic@Derivative, Sub@Optimization, File@0046}{
\latexProblemContent{
\ifVerboseLocation This is Derivative Compute Question 0046. \\ \fi
\begin{problem}

A Norman window has the shape of a rectangle surmounted by a semicircle (i.e. the diameter of the semicircle is equal to the width of the rectangle).  If the perimeter of the window is ${4}$ ft, what area will allow in the greatest amount of light?

\input{Derivative-Compute-0046.HELP.tex}

\[\mbox{The largest possible area is\;} \answer{\frac{8}{\pi + 4}}\]
\end{problem}}%}

\latexProblemContent{
\ifVerboseLocation This is Derivative Compute Question 0046. \\ \fi
\begin{problem}

A Norman window has the shape of a rectangle surmounted by a semicircle (i.e. the diameter of the semicircle is equal to the width of the rectangle).  If the perimeter of the window is ${15}$ ft, what area will allow in the greatest amount of light?

\input{Derivative-Compute-0046.HELP.tex}

\[\mbox{The largest possible area is\;} \answer{\frac{225}{2 \, {\left(\pi + 4\right)}}}\]
\end{problem}}%}

\latexProblemContent{
\ifVerboseLocation This is Derivative Compute Question 0046. \\ \fi
\begin{problem}

A Norman window has the shape of a rectangle surmounted by a semicircle (i.e. the diameter of the semicircle is equal to the width of the rectangle).  If the perimeter of the window is ${6}$ ft, what area will allow in the greatest amount of light?

\input{Derivative-Compute-0046.HELP.tex}

\[\mbox{The largest possible area is\;} \answer{\frac{18}{\pi + 4}}\]
\end{problem}}%}

\latexProblemContent{
\ifVerboseLocation This is Derivative Compute Question 0046. \\ \fi
\begin{problem}

A Norman window has the shape of a rectangle surmounted by a semicircle (i.e. the diameter of the semicircle is equal to the width of the rectangle).  If the perimeter of the window is ${21}$ ft, what area will allow in the greatest amount of light?

\input{Derivative-Compute-0046.HELP.tex}

\[\mbox{The largest possible area is\;} \answer{\frac{441}{2 \, {\left(\pi + 4\right)}}}\]
\end{problem}}%}

\latexProblemContent{
\ifVerboseLocation This is Derivative Compute Question 0046. \\ \fi
\begin{problem}

A Norman window has the shape of a rectangle surmounted by a semicircle (i.e. the diameter of the semicircle is equal to the width of the rectangle).  If the perimeter of the window is ${1}$ ft, what area will allow in the greatest amount of light?

\input{Derivative-Compute-0046.HELP.tex}

\[\mbox{The largest possible area is\;} \answer{\frac{1}{2 \, {\left(\pi + 4\right)}}}\]
\end{problem}}%}

\latexProblemContent{
\ifVerboseLocation This is Derivative Compute Question 0046. \\ \fi
\begin{problem}

A Norman window has the shape of a rectangle surmounted by a semicircle (i.e. the diameter of the semicircle is equal to the width of the rectangle).  If the perimeter of the window is ${20}$ ft, what area will allow in the greatest amount of light?

\input{Derivative-Compute-0046.HELP.tex}

\[\mbox{The largest possible area is\;} \answer{\frac{200}{\pi + 4}}\]
\end{problem}}%}

\latexProblemContent{
\ifVerboseLocation This is Derivative Compute Question 0046. \\ \fi
\begin{problem}

A Norman window has the shape of a rectangle surmounted by a semicircle (i.e. the diameter of the semicircle is equal to the width of the rectangle).  If the perimeter of the window is ${30}$ ft, what area will allow in the greatest amount of light?

\input{Derivative-Compute-0046.HELP.tex}

\[\mbox{The largest possible area is\;} \answer{\frac{450}{\pi + 4}}\]
\end{problem}}%}

\latexProblemContent{
\ifVerboseLocation This is Derivative Compute Question 0046. \\ \fi
\begin{problem}

A Norman window has the shape of a rectangle surmounted by a semicircle (i.e. the diameter of the semicircle is equal to the width of the rectangle).  If the perimeter of the window is ${56}$ ft, what area will allow in the greatest amount of light?

\input{Derivative-Compute-0046.HELP.tex}

\[\mbox{The largest possible area is\;} \answer{\frac{1568}{\pi + 4}}\]
\end{problem}}%}

\latexProblemContent{
\ifVerboseLocation This is Derivative Compute Question 0046. \\ \fi
\begin{problem}

A Norman window has the shape of a rectangle surmounted by a semicircle (i.e. the diameter of the semicircle is equal to the width of the rectangle).  If the perimeter of the window is ${48}$ ft, what area will allow in the greatest amount of light?

\input{Derivative-Compute-0046.HELP.tex}

\[\mbox{The largest possible area is\;} \answer{\frac{1152}{\pi + 4}}\]
\end{problem}}%}

\latexProblemContent{
\ifVerboseLocation This is Derivative Compute Question 0046. \\ \fi
\begin{problem}

A Norman window has the shape of a rectangle surmounted by a semicircle (i.e. the diameter of the semicircle is equal to the width of the rectangle).  If the perimeter of the window is ${35}$ ft, what area will allow in the greatest amount of light?

\input{Derivative-Compute-0046.HELP.tex}

\[\mbox{The largest possible area is\;} \answer{\frac{1225}{2 \, {\left(\pi + 4\right)}}}\]
\end{problem}}%}

\latexProblemContent{
\ifVerboseLocation This is Derivative Compute Question 0046. \\ \fi
\begin{problem}

A Norman window has the shape of a rectangle surmounted by a semicircle (i.e. the diameter of the semicircle is equal to the width of the rectangle).  If the perimeter of the window is ${14}$ ft, what area will allow in the greatest amount of light?

\input{Derivative-Compute-0046.HELP.tex}

\[\mbox{The largest possible area is\;} \answer{\frac{98}{\pi + 4}}\]
\end{problem}}%}

\latexProblemContent{
\ifVerboseLocation This is Derivative Compute Question 0046. \\ \fi
\begin{problem}

A Norman window has the shape of a rectangle surmounted by a semicircle (i.e. the diameter of the semicircle is equal to the width of the rectangle).  If the perimeter of the window is ${2}$ ft, what area will allow in the greatest amount of light?

\input{Derivative-Compute-0046.HELP.tex}

\[\mbox{The largest possible area is\;} \answer{\frac{2}{\pi + 4}}\]
\end{problem}}%}

\latexProblemContent{
\ifVerboseLocation This is Derivative Compute Question 0046. \\ \fi
\begin{problem}

A Norman window has the shape of a rectangle surmounted by a semicircle (i.e. the diameter of the semicircle is equal to the width of the rectangle).  If the perimeter of the window is ${28}$ ft, what area will allow in the greatest amount of light?

\input{Derivative-Compute-0046.HELP.tex}

\[\mbox{The largest possible area is\;} \answer{\frac{392}{\pi + 4}}\]
\end{problem}}%}

\latexProblemContent{
\ifVerboseLocation This is Derivative Compute Question 0046. \\ \fi
\begin{problem}

A Norman window has the shape of a rectangle surmounted by a semicircle (i.e. the diameter of the semicircle is equal to the width of the rectangle).  If the perimeter of the window is ${24}$ ft, what area will allow in the greatest amount of light?

\input{Derivative-Compute-0046.HELP.tex}

\[\mbox{The largest possible area is\;} \answer{\frac{288}{\pi + 4}}\]
\end{problem}}%}

\latexProblemContent{
\ifVerboseLocation This is Derivative Compute Question 0046. \\ \fi
\begin{problem}

A Norman window has the shape of a rectangle surmounted by a semicircle (i.e. the diameter of the semicircle is equal to the width of the rectangle).  If the perimeter of the window is ${64}$ ft, what area will allow in the greatest amount of light?

\input{Derivative-Compute-0046.HELP.tex}

\[\mbox{The largest possible area is\;} \answer{\frac{2048}{\pi + 4}}\]
\end{problem}}%}

\latexProblemContent{
\ifVerboseLocation This is Derivative Compute Question 0046. \\ \fi
\begin{problem}

A Norman window has the shape of a rectangle surmounted by a semicircle (i.e. the diameter of the semicircle is equal to the width of the rectangle).  If the perimeter of the window is ${7}$ ft, what area will allow in the greatest amount of light?

\input{Derivative-Compute-0046.HELP.tex}

\[\mbox{The largest possible area is\;} \answer{\frac{49}{2 \, {\left(\pi + 4\right)}}}\]
\end{problem}}%}

\latexProblemContent{
\ifVerboseLocation This is Derivative Compute Question 0046. \\ \fi
\begin{problem}

A Norman window has the shape of a rectangle surmounted by a semicircle (i.e. the diameter of the semicircle is equal to the width of the rectangle).  If the perimeter of the window is ${9}$ ft, what area will allow in the greatest amount of light?

\input{Derivative-Compute-0046.HELP.tex}

\[\mbox{The largest possible area is\;} \answer{\frac{81}{2 \, {\left(\pi + 4\right)}}}\]
\end{problem}}%}

\latexProblemContent{
\ifVerboseLocation This is Derivative Compute Question 0046. \\ \fi
\begin{problem}

A Norman window has the shape of a rectangle surmounted by a semicircle (i.e. the diameter of the semicircle is equal to the width of the rectangle).  If the perimeter of the window is ${5}$ ft, what area will allow in the greatest amount of light?

\input{Derivative-Compute-0046.HELP.tex}

\[\mbox{The largest possible area is\;} \answer{\frac{25}{2 \, {\left(\pi + 4\right)}}}\]
\end{problem}}%}

\latexProblemContent{
\ifVerboseLocation This is Derivative Compute Question 0046. \\ \fi
\begin{problem}

A Norman window has the shape of a rectangle surmounted by a semicircle (i.e. the diameter of the semicircle is equal to the width of the rectangle).  If the perimeter of the window is ${12}$ ft, what area will allow in the greatest amount of light?

\input{Derivative-Compute-0046.HELP.tex}

\[\mbox{The largest possible area is\;} \answer{\frac{72}{\pi + 4}}\]
\end{problem}}%}

\latexProblemContent{
\ifVerboseLocation This is Derivative Compute Question 0046. \\ \fi
\begin{problem}

A Norman window has the shape of a rectangle surmounted by a semicircle (i.e. the diameter of the semicircle is equal to the width of the rectangle).  If the perimeter of the window is ${42}$ ft, what area will allow in the greatest amount of light?

\input{Derivative-Compute-0046.HELP.tex}

\[\mbox{The largest possible area is\;} \answer{\frac{882}{\pi + 4}}\]
\end{problem}}%}

\latexProblemContent{
\ifVerboseLocation This is Derivative Compute Question 0046. \\ \fi
\begin{problem}

A Norman window has the shape of a rectangle surmounted by a semicircle (i.e. the diameter of the semicircle is equal to the width of the rectangle).  If the perimeter of the window is ${16}$ ft, what area will allow in the greatest amount of light?

\input{Derivative-Compute-0046.HELP.tex}

\[\mbox{The largest possible area is\;} \answer{\frac{128}{\pi + 4}}\]
\end{problem}}%}

\latexProblemContent{
\ifVerboseLocation This is Derivative Compute Question 0046. \\ \fi
\begin{problem}

A Norman window has the shape of a rectangle surmounted by a semicircle (i.e. the diameter of the semicircle is equal to the width of the rectangle).  If the perimeter of the window is ${49}$ ft, what area will allow in the greatest amount of light?

\input{Derivative-Compute-0046.HELP.tex}

\[\mbox{The largest possible area is\;} \answer{\frac{2401}{2 \, {\left(\pi + 4\right)}}}\]
\end{problem}}%}

\latexProblemContent{
\ifVerboseLocation This is Derivative Compute Question 0046. \\ \fi
\begin{problem}

A Norman window has the shape of a rectangle surmounted by a semicircle (i.e. the diameter of the semicircle is equal to the width of the rectangle).  If the perimeter of the window is ${8}$ ft, what area will allow in the greatest amount of light?

\input{Derivative-Compute-0046.HELP.tex}

\[\mbox{The largest possible area is\;} \answer{\frac{32}{\pi + 4}}\]
\end{problem}}%}

\latexProblemContent{
\ifVerboseLocation This is Derivative Compute Question 0046. \\ \fi
\begin{problem}

A Norman window has the shape of a rectangle surmounted by a semicircle (i.e. the diameter of the semicircle is equal to the width of the rectangle).  If the perimeter of the window is ${18}$ ft, what area will allow in the greatest amount of light?

\input{Derivative-Compute-0046.HELP.tex}

\[\mbox{The largest possible area is\;} \answer{\frac{162}{\pi + 4}}\]
\end{problem}}%}

\latexProblemContent{
\ifVerboseLocation This is Derivative Compute Question 0046. \\ \fi
\begin{problem}

A Norman window has the shape of a rectangle surmounted by a semicircle (i.e. the diameter of the semicircle is equal to the width of the rectangle).  If the perimeter of the window is ${10}$ ft, what area will allow in the greatest amount of light?

\input{Derivative-Compute-0046.HELP.tex}

\[\mbox{The largest possible area is\;} \answer{\frac{50}{\pi + 4}}\]
\end{problem}}%}

\latexProblemContent{
\ifVerboseLocation This is Derivative Compute Question 0046. \\ \fi
\begin{problem}

A Norman window has the shape of a rectangle surmounted by a semicircle (i.e. the diameter of the semicircle is equal to the width of the rectangle).  If the perimeter of the window is ${3}$ ft, what area will allow in the greatest amount of light?

\input{Derivative-Compute-0046.HELP.tex}

\[\mbox{The largest possible area is\;} \answer{\frac{9}{2 \, {\left(\pi + 4\right)}}}\]
\end{problem}}%}

\latexProblemContent{
\ifVerboseLocation This is Derivative Compute Question 0046. \\ \fi
\begin{problem}

A Norman window has the shape of a rectangle surmounted by a semicircle (i.e. the diameter of the semicircle is equal to the width of the rectangle).  If the perimeter of the window is ${32}$ ft, what area will allow in the greatest amount of light?

\input{Derivative-Compute-0046.HELP.tex}

\[\mbox{The largest possible area is\;} \answer{\frac{512}{\pi + 4}}\]
\end{problem}}%}

\latexProblemContent{
\ifVerboseLocation This is Derivative Compute Question 0046. \\ \fi
\begin{problem}

A Norman window has the shape of a rectangle surmounted by a semicircle (i.e. the diameter of the semicircle is equal to the width of the rectangle).  If the perimeter of the window is ${40}$ ft, what area will allow in the greatest amount of light?

\input{Derivative-Compute-0046.HELP.tex}

\[\mbox{The largest possible area is\;} \answer{\frac{800}{\pi + 4}}\]
\end{problem}}%}

\latexProblemContent{
\ifVerboseLocation This is Derivative Compute Question 0046. \\ \fi
\begin{problem}

A Norman window has the shape of a rectangle surmounted by a semicircle (i.e. the diameter of the semicircle is equal to the width of the rectangle).  If the perimeter of the window is ${25}$ ft, what area will allow in the greatest amount of light?

\input{Derivative-Compute-0046.HELP.tex}

\[\mbox{The largest possible area is\;} \answer{\frac{625}{2 \, {\left(\pi + 4\right)}}}\]
\end{problem}}%}

\latexProblemContent{
\ifVerboseLocation This is Derivative Compute Question 0046. \\ \fi
\begin{problem}

A Norman window has the shape of a rectangle surmounted by a semicircle (i.e. the diameter of the semicircle is equal to the width of the rectangle).  If the perimeter of the window is ${36}$ ft, what area will allow in the greatest amount of light?

\input{Derivative-Compute-0046.HELP.tex}

\[\mbox{The largest possible area is\;} \answer{\frac{648}{\pi + 4}}\]
\end{problem}}%}

