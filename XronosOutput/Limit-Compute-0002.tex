\ProblemFileHeader{XTL_SV_QUESTIONCOUNT}% Process how many problems are in this file and how to detect if it has a desirable problem
\ifproblemToFind% If it has a desirable problem search the file.
%\tagged{Ans@ShortAns, Type@Compute, Topic@Limit, Sub@Rational, File@0002}{
\latexProblemContent{
\ifVerboseLocation This is Derivative Compute Question 0002. \\ \fi
\begin{problem}

Determine if the limit approaches a finite number, $\pm\infty$, or does not exist. (If the limit does not exist, write DNE)

\input{Limit-Compute-0002.HELP.tex}

\[\lim_{x\to{4}}\dfrac{2 \, x - 8}{x^{2} - 2 \, x - 8}=\answer{\frac{1}{3}}\]
\end{problem}}%}

\latexProblemContent{
\ifVerboseLocation This is Derivative Compute Question 0002. \\ \fi
\begin{problem}

Determine if the limit approaches a finite number, $\pm\infty$, or does not exist. (If the limit does not exist, write DNE)

\input{Limit-Compute-0002.HELP.tex}

\[\lim_{x\to{-4}}\dfrac{3 \, x + 12}{x^{2} + x - 12}=\answer{-\frac{3}{7}}\]
\end{problem}}%}

\latexProblemContent{
\ifVerboseLocation This is Derivative Compute Question 0002. \\ \fi
\begin{problem}

Determine if the limit approaches a finite number, $\pm\infty$, or does not exist. (If the limit does not exist, write DNE)

\input{Limit-Compute-0002.HELP.tex}

\[\lim_{x\to{-3}}\dfrac{x + 3}{x^{2} + 8 \, x + 15}=\answer{\frac{1}{2}}\]
\end{problem}}%}

\latexProblemContent{
\ifVerboseLocation This is Derivative Compute Question 0002. \\ \fi
\begin{problem}

Determine if the limit approaches a finite number, $\pm\infty$, or does not exist. (If the limit does not exist, write DNE)

\input{Limit-Compute-0002.HELP.tex}

\[\lim_{x\to{-2}}\dfrac{x + 2}{x^{2} - 5 \, x - 14}=\answer{-\frac{1}{9}}\]
\end{problem}}%}

\latexProblemContent{
\ifVerboseLocation This is Derivative Compute Question 0002. \\ \fi
\begin{problem}

Determine if the limit approaches a finite number, $\pm\infty$, or does not exist. (If the limit does not exist, write DNE)

\input{Limit-Compute-0002.HELP.tex}

\[\lim_{x\to{-7}}\dfrac{x + 7}{x^{2} + 2 \, x - 35}=\answer{-\frac{1}{12}}\]
\end{problem}}%}

\latexProblemContent{
\ifVerboseLocation This is Derivative Compute Question 0002. \\ \fi
\begin{problem}

Determine if the limit approaches a finite number, $\pm\infty$, or does not exist. (If the limit does not exist, write DNE)

\input{Limit-Compute-0002.HELP.tex}

\[\lim_{x\to{8}}\dfrac{3 \, x - 24}{x^{2} - 10 \, x + 16}=\answer{\frac{1}{2}}\]
\end{problem}}%}

\latexProblemContent{
\ifVerboseLocation This is Derivative Compute Question 0002. \\ \fi
\begin{problem}

Determine if the limit approaches a finite number, $\pm\infty$, or does not exist. (If the limit does not exist, write DNE)

\input{Limit-Compute-0002.HELP.tex}

\[\lim_{x\to{-6}}\dfrac{x + 6}{x^{2} + 15 \, x + 54}=\answer{\frac{1}{3}}\]
\end{problem}}%}

\latexProblemContent{
\ifVerboseLocation This is Derivative Compute Question 0002. \\ \fi
\begin{problem}

Determine if the limit approaches a finite number, $\pm\infty$, or does not exist. (If the limit does not exist, write DNE)

\input{Limit-Compute-0002.HELP.tex}

\[\lim_{x\to{-1}}\dfrac{3 \, x + 3}{x^{2} - 9 \, x - 10}=\answer{-\frac{3}{11}}\]
\end{problem}}%}

\latexProblemContent{
\ifVerboseLocation This is Derivative Compute Question 0002. \\ \fi
\begin{problem}

Determine if the limit approaches a finite number, $\pm\infty$, or does not exist. (If the limit does not exist, write DNE)

\input{Limit-Compute-0002.HELP.tex}

\[\lim_{x\to{6}}\dfrac{3 \, x - 18}{x^{2} - 8 \, x + 12}=\answer{\frac{3}{4}}\]
\end{problem}}%}

\latexProblemContent{
\ifVerboseLocation This is Derivative Compute Question 0002. \\ \fi
\begin{problem}

Determine if the limit approaches a finite number, $\pm\infty$, or does not exist. (If the limit does not exist, write DNE)

\input{Limit-Compute-0002.HELP.tex}

\[\lim_{x\to{-1}}\dfrac{3 \, x + 3}{x^{2} + 5 \, x + 4}=\answer{1}\]
\end{problem}}%}

\latexProblemContent{
\ifVerboseLocation This is Derivative Compute Question 0002. \\ \fi
\begin{problem}

Determine if the limit approaches a finite number, $\pm\infty$, or does not exist. (If the limit does not exist, write DNE)

\input{Limit-Compute-0002.HELP.tex}

\[\lim_{x\to{3}}\dfrac{3 \, x - 9}{x^{2} + 4 \, x - 21}=\answer{\frac{3}{10}}\]
\end{problem}}%}

\latexProblemContent{
\ifVerboseLocation This is Derivative Compute Question 0002. \\ \fi
\begin{problem}

Determine if the limit approaches a finite number, $\pm\infty$, or does not exist. (If the limit does not exist, write DNE)

\input{Limit-Compute-0002.HELP.tex}

\[\lim_{x\to{-1}}\dfrac{2 \, x + 2}{x^{2} - x - 2}=\answer{-\frac{2}{3}}\]
\end{problem}}%}

\latexProblemContent{
\ifVerboseLocation This is Derivative Compute Question 0002. \\ \fi
\begin{problem}

Determine if the limit approaches a finite number, $\pm\infty$, or does not exist. (If the limit does not exist, write DNE)

\input{Limit-Compute-0002.HELP.tex}

\[\lim_{x\to{-1}}\dfrac{x + 1}{x^{2} + 4 \, x + 3}=\answer{\frac{1}{2}}\]
\end{problem}}%}

\latexProblemContent{
\ifVerboseLocation This is Derivative Compute Question 0002. \\ \fi
\begin{problem}

Determine if the limit approaches a finite number, $\pm\infty$, or does not exist. (If the limit does not exist, write DNE)

\input{Limit-Compute-0002.HELP.tex}

\[\lim_{x\to{7}}\dfrac{2 \, x - 14}{x^{2} - 4 \, x - 21}=\answer{\frac{1}{5}}\]
\end{problem}}%}

\latexProblemContent{
\ifVerboseLocation This is Derivative Compute Question 0002. \\ \fi
\begin{problem}

Determine if the limit approaches a finite number, $\pm\infty$, or does not exist. (If the limit does not exist, write DNE)

\input{Limit-Compute-0002.HELP.tex}

\[\lim_{x\to{10}}\dfrac{2 \, x - 20}{x^{2} - 17 \, x + 70}=\answer{\frac{2}{3}}\]
\end{problem}}%}

\latexProblemContent{
\ifVerboseLocation This is Derivative Compute Question 0002. \\ \fi
\begin{problem}

Determine if the limit approaches a finite number, $\pm\infty$, or does not exist. (If the limit does not exist, write DNE)

\input{Limit-Compute-0002.HELP.tex}

\[\lim_{x\to{-10}}\dfrac{x + 10}{x^{2} + 5 \, x - 50}=\answer{-\frac{1}{15}}\]
\end{problem}}%}

\latexProblemContent{
\ifVerboseLocation This is Derivative Compute Question 0002. \\ \fi
\begin{problem}

Determine if the limit approaches a finite number, $\pm\infty$, or does not exist. (If the limit does not exist, write DNE)

\input{Limit-Compute-0002.HELP.tex}

\[\lim_{x\to{3}}\dfrac{2 \, x - 6}{x^{2} - 4 \, x + 3}=\answer{1}\]
\end{problem}}%}

\latexProblemContent{
\ifVerboseLocation This is Derivative Compute Question 0002. \\ \fi
\begin{problem}

Determine if the limit approaches a finite number, $\pm\infty$, or does not exist. (If the limit does not exist, write DNE)

\input{Limit-Compute-0002.HELP.tex}

\[\lim_{x\to{4}}\dfrac{x - 4}{x^{2} - 14 \, x + 40}=\answer{-\frac{1}{6}}\]
\end{problem}}%}

\latexProblemContent{
\ifVerboseLocation This is Derivative Compute Question 0002. \\ \fi
\begin{problem}

Determine if the limit approaches a finite number, $\pm\infty$, or does not exist. (If the limit does not exist, write DNE)

\input{Limit-Compute-0002.HELP.tex}

\[\lim_{x\to{-5}}\dfrac{x + 5}{x^{2} + 8 \, x + 15}=\answer{-\frac{1}{2}}\]
\end{problem}}%}

\latexProblemContent{
\ifVerboseLocation This is Derivative Compute Question 0002. \\ \fi
\begin{problem}

Determine if the limit approaches a finite number, $\pm\infty$, or does not exist. (If the limit does not exist, write DNE)

\input{Limit-Compute-0002.HELP.tex}

\[\lim_{x\to{5}}\dfrac{2 \, x - 10}{x^{2} - 4 \, x - 5}=\answer{\frac{1}{3}}\]
\end{problem}}%}

\latexProblemContent{
\ifVerboseLocation This is Derivative Compute Question 0002. \\ \fi
\begin{problem}

Determine if the limit approaches a finite number, $\pm\infty$, or does not exist. (If the limit does not exist, write DNE)

\input{Limit-Compute-0002.HELP.tex}

\[\lim_{x\to{3}}\dfrac{3 \, x - 9}{x^{2} - 12 \, x + 27}=\answer{-\frac{1}{2}}\]
\end{problem}}%}

\latexProblemContent{
\ifVerboseLocation This is Derivative Compute Question 0002. \\ \fi
\begin{problem}

Determine if the limit approaches a finite number, $\pm\infty$, or does not exist. (If the limit does not exist, write DNE)

\input{Limit-Compute-0002.HELP.tex}

\[\lim_{x\to{1}}\dfrac{x - 1}{x^{2} + x - 2}=\answer{\frac{1}{3}}\]
\end{problem}}%}

\latexProblemContent{
\ifVerboseLocation This is Derivative Compute Question 0002. \\ \fi
\begin{problem}

Determine if the limit approaches a finite number, $\pm\infty$, or does not exist. (If the limit does not exist, write DNE)

\input{Limit-Compute-0002.HELP.tex}

\[\lim_{x\to{-3}}\dfrac{2 \, x + 6}{x^{2} + 12 \, x + 27}=\answer{\frac{1}{3}}\]
\end{problem}}%}

\latexProblemContent{
\ifVerboseLocation This is Derivative Compute Question 0002. \\ \fi
\begin{problem}

Determine if the limit approaches a finite number, $\pm\infty$, or does not exist. (If the limit does not exist, write DNE)

\input{Limit-Compute-0002.HELP.tex}

\[\lim_{x\to{4}}\dfrac{x - 4}{x^{2} + x - 20}=\answer{\frac{1}{9}}\]
\end{problem}}%}

\latexProblemContent{
\ifVerboseLocation This is Derivative Compute Question 0002. \\ \fi
\begin{problem}

Determine if the limit approaches a finite number, $\pm\infty$, or does not exist. (If the limit does not exist, write DNE)

\input{Limit-Compute-0002.HELP.tex}

\[\lim_{x\to{-1}}\dfrac{3 \, x + 3}{x^{2} - 6 \, x - 7}=\answer{-\frac{3}{8}}\]
\end{problem}}%}

\latexProblemContent{
\ifVerboseLocation This is Derivative Compute Question 0002. \\ \fi
\begin{problem}

Determine if the limit approaches a finite number, $\pm\infty$, or does not exist. (If the limit does not exist, write DNE)

\input{Limit-Compute-0002.HELP.tex}

\[\lim_{x\to{-8}}\dfrac{2 \, x + 16}{x^{2} + 4 \, x - 32}=\answer{-\frac{1}{6}}\]
\end{problem}}%}

\latexProblemContent{
\ifVerboseLocation This is Derivative Compute Question 0002. \\ \fi
\begin{problem}

Determine if the limit approaches a finite number, $\pm\infty$, or does not exist. (If the limit does not exist, write DNE)

\input{Limit-Compute-0002.HELP.tex}

\[\lim_{x\to{10}}\dfrac{2 \, x - 20}{x^{2} - 8 \, x - 20}=\answer{\frac{1}{6}}\]
\end{problem}}%}

\latexProblemContent{
\ifVerboseLocation This is Derivative Compute Question 0002. \\ \fi
\begin{problem}

Determine if the limit approaches a finite number, $\pm\infty$, or does not exist. (If the limit does not exist, write DNE)

\input{Limit-Compute-0002.HELP.tex}

\[\lim_{x\to{-9}}\dfrac{2 \, x + 18}{x^{2} - 81}=\answer{-\frac{1}{9}}\]
\end{problem}}%}

\latexProblemContent{
\ifVerboseLocation This is Derivative Compute Question 0002. \\ \fi
\begin{problem}

Determine if the limit approaches a finite number, $\pm\infty$, or does not exist. (If the limit does not exist, write DNE)

\input{Limit-Compute-0002.HELP.tex}

\[\lim_{x\to{9}}\dfrac{x - 9}{x^{2} - 4 \, x - 45}=\answer{\frac{1}{14}}\]
\end{problem}}%}

\latexProblemContent{
\ifVerboseLocation This is Derivative Compute Question 0002. \\ \fi
\begin{problem}

Determine if the limit approaches a finite number, $\pm\infty$, or does not exist. (If the limit does not exist, write DNE)

\input{Limit-Compute-0002.HELP.tex}

\[\lim_{x\to{-7}}\dfrac{3 \, x + 21}{x^{2} + 12 \, x + 35}=\answer{-\frac{3}{2}}\]
\end{problem}}%}

\latexProblemContent{
\ifVerboseLocation This is Derivative Compute Question 0002. \\ \fi
\begin{problem}

Determine if the limit approaches a finite number, $\pm\infty$, or does not exist. (If the limit does not exist, write DNE)

\input{Limit-Compute-0002.HELP.tex}

\[\lim_{x\to{10}}\dfrac{x - 10}{x^{2} - 17 \, x + 70}=\answer{\frac{1}{3}}\]
\end{problem}}%}

\latexProblemContent{
\ifVerboseLocation This is Derivative Compute Question 0002. \\ \fi
\begin{problem}

Determine if the limit approaches a finite number, $\pm\infty$, or does not exist. (If the limit does not exist, write DNE)

\input{Limit-Compute-0002.HELP.tex}

\[\lim_{x\to{10}}\dfrac{2 \, x - 20}{x^{2} - 5 \, x - 50}=\answer{\frac{2}{15}}\]
\end{problem}}%}

\latexProblemContent{
\ifVerboseLocation This is Derivative Compute Question 0002. \\ \fi
\begin{problem}

Determine if the limit approaches a finite number, $\pm\infty$, or does not exist. (If the limit does not exist, write DNE)

\input{Limit-Compute-0002.HELP.tex}

\[\lim_{x\to{-4}}\dfrac{2 \, x + 8}{x^{2} + 3 \, x - 4}=\answer{-\frac{2}{5}}\]
\end{problem}}%}

\latexProblemContent{
\ifVerboseLocation This is Derivative Compute Question 0002. \\ \fi
\begin{problem}

Determine if the limit approaches a finite number, $\pm\infty$, or does not exist. (If the limit does not exist, write DNE)

\input{Limit-Compute-0002.HELP.tex}

\[\lim_{x\to{6}}\dfrac{x - 6}{x^{2} - 13 \, x + 42}=\answer{-1}\]
\end{problem}}%}

\latexProblemContent{
\ifVerboseLocation This is Derivative Compute Question 0002. \\ \fi
\begin{problem}

Determine if the limit approaches a finite number, $\pm\infty$, or does not exist. (If the limit does not exist, write DNE)

\input{Limit-Compute-0002.HELP.tex}

\[\lim_{x\to{3}}\dfrac{x - 3}{x^{2} - 4 \, x + 3}=\answer{\frac{1}{2}}\]
\end{problem}}%}

\latexProblemContent{
\ifVerboseLocation This is Derivative Compute Question 0002. \\ \fi
\begin{problem}

Determine if the limit approaches a finite number, $\pm\infty$, or does not exist. (If the limit does not exist, write DNE)

\input{Limit-Compute-0002.HELP.tex}

\[\lim_{x\to{-7}}\dfrac{x + 7}{x^{2} + 6 \, x - 7}=\answer{-\frac{1}{8}}\]
\end{problem}}%}

\latexProblemContent{
\ifVerboseLocation This is Derivative Compute Question 0002. \\ \fi
\begin{problem}

Determine if the limit approaches a finite number, $\pm\infty$, or does not exist. (If the limit does not exist, write DNE)

\input{Limit-Compute-0002.HELP.tex}

\[\lim_{x\to{-6}}\dfrac{x + 6}{x^{2} - 3 \, x - 54}=\answer{-\frac{1}{15}}\]
\end{problem}}%}

\latexProblemContent{
\ifVerboseLocation This is Derivative Compute Question 0002. \\ \fi
\begin{problem}

Determine if the limit approaches a finite number, $\pm\infty$, or does not exist. (If the limit does not exist, write DNE)

\input{Limit-Compute-0002.HELP.tex}

\[\lim_{x\to{-10}}\dfrac{x + 10}{x^{2} + 4 \, x - 60}=\answer{-\frac{1}{16}}\]
\end{problem}}%}

\latexProblemContent{
\ifVerboseLocation This is Derivative Compute Question 0002. \\ \fi
\begin{problem}

Determine if the limit approaches a finite number, $\pm\infty$, or does not exist. (If the limit does not exist, write DNE)

\input{Limit-Compute-0002.HELP.tex}

\[\lim_{x\to{5}}\dfrac{3 \, x - 15}{x^{2} - 8 \, x + 15}=\answer{\frac{3}{2}}\]
\end{problem}}%}

\latexProblemContent{
\ifVerboseLocation This is Derivative Compute Question 0002. \\ \fi
\begin{problem}

Determine if the limit approaches a finite number, $\pm\infty$, or does not exist. (If the limit does not exist, write DNE)

\input{Limit-Compute-0002.HELP.tex}

\[\lim_{x\to{8}}\dfrac{x - 8}{x^{2} - 13 \, x + 40}=\answer{\frac{1}{3}}\]
\end{problem}}%}

\latexProblemContent{
\ifVerboseLocation This is Derivative Compute Question 0002. \\ \fi
\begin{problem}

Determine if the limit approaches a finite number, $\pm\infty$, or does not exist. (If the limit does not exist, write DNE)

\input{Limit-Compute-0002.HELP.tex}

\[\lim_{x\to{4}}\dfrac{3 \, x - 12}{x^{2} - 16}=\answer{\frac{3}{8}}\]
\end{problem}}%}

\latexProblemContent{
\ifVerboseLocation This is Derivative Compute Question 0002. \\ \fi
\begin{problem}

Determine if the limit approaches a finite number, $\pm\infty$, or does not exist. (If the limit does not exist, write DNE)

\input{Limit-Compute-0002.HELP.tex}

\[\lim_{x\to{10}}\dfrac{x - 10}{x^{2} - 16 \, x + 60}=\answer{\frac{1}{4}}\]
\end{problem}}%}

\latexProblemContent{
\ifVerboseLocation This is Derivative Compute Question 0002. \\ \fi
\begin{problem}

Determine if the limit approaches a finite number, $\pm\infty$, or does not exist. (If the limit does not exist, write DNE)

\input{Limit-Compute-0002.HELP.tex}

\[\lim_{x\to{6}}\dfrac{2 \, x - 12}{x^{2} - 36}=\answer{\frac{1}{6}}\]
\end{problem}}%}

\latexProblemContent{
\ifVerboseLocation This is Derivative Compute Question 0002. \\ \fi
\begin{problem}

Determine if the limit approaches a finite number, $\pm\infty$, or does not exist. (If the limit does not exist, write DNE)

\input{Limit-Compute-0002.HELP.tex}

\[\lim_{x\to{-5}}\dfrac{2 \, x + 10}{x^{2} + 2 \, x - 15}=\answer{-\frac{1}{4}}\]
\end{problem}}%}

\latexProblemContent{
\ifVerboseLocation This is Derivative Compute Question 0002. \\ \fi
\begin{problem}

Determine if the limit approaches a finite number, $\pm\infty$, or does not exist. (If the limit does not exist, write DNE)

\input{Limit-Compute-0002.HELP.tex}

\[\lim_{x\to{-10}}\dfrac{2 \, x + 20}{x^{2} + 14 \, x + 40}=\answer{-\frac{1}{3}}\]
\end{problem}}%}

\latexProblemContent{
\ifVerboseLocation This is Derivative Compute Question 0002. \\ \fi
\begin{problem}

Determine if the limit approaches a finite number, $\pm\infty$, or does not exist. (If the limit does not exist, write DNE)

\input{Limit-Compute-0002.HELP.tex}

\[\lim_{x\to{-8}}\dfrac{3 \, x + 24}{x^{2} + 12 \, x + 32}=\answer{-\frac{3}{4}}\]
\end{problem}}%}

\latexProblemContent{
\ifVerboseLocation This is Derivative Compute Question 0002. \\ \fi
\begin{problem}

Determine if the limit approaches a finite number, $\pm\infty$, or does not exist. (If the limit does not exist, write DNE)

\input{Limit-Compute-0002.HELP.tex}

\[\lim_{x\to{4}}\dfrac{3 \, x - 12}{x^{2} + 2 \, x - 24}=\answer{\frac{3}{10}}\]
\end{problem}}%}

\latexProblemContent{
\ifVerboseLocation This is Derivative Compute Question 0002. \\ \fi
\begin{problem}

Determine if the limit approaches a finite number, $\pm\infty$, or does not exist. (If the limit does not exist, write DNE)

\input{Limit-Compute-0002.HELP.tex}

\[\lim_{x\to{6}}\dfrac{2 \, x - 12}{x^{2} - 4 \, x - 12}=\answer{\frac{1}{4}}\]
\end{problem}}%}

\latexProblemContent{
\ifVerboseLocation This is Derivative Compute Question 0002. \\ \fi
\begin{problem}

Determine if the limit approaches a finite number, $\pm\infty$, or does not exist. (If the limit does not exist, write DNE)

\input{Limit-Compute-0002.HELP.tex}

\[\lim_{x\to{8}}\dfrac{3 \, x - 24}{x^{2} - 7 \, x - 8}=\answer{\frac{1}{3}}\]
\end{problem}}%}

\latexProblemContent{
\ifVerboseLocation This is Derivative Compute Question 0002. \\ \fi
\begin{problem}

Determine if the limit approaches a finite number, $\pm\infty$, or does not exist. (If the limit does not exist, write DNE)

\input{Limit-Compute-0002.HELP.tex}

\[\lim_{x\to{2}}\dfrac{x - 2}{x^{2} + 5 \, x - 14}=\answer{\frac{1}{9}}\]
\end{problem}}%}

\latexProblemContent{
\ifVerboseLocation This is Derivative Compute Question 0002. \\ \fi
\begin{problem}

Determine if the limit approaches a finite number, $\pm\infty$, or does not exist. (If the limit does not exist, write DNE)

\input{Limit-Compute-0002.HELP.tex}

\[\lim_{x\to{-9}}\dfrac{2 \, x + 18}{x^{2} + 3 \, x - 54}=\answer{-\frac{2}{15}}\]
\end{problem}}%}

\latexProblemContent{
\ifVerboseLocation This is Derivative Compute Question 0002. \\ \fi
\begin{problem}

Determine if the limit approaches a finite number, $\pm\infty$, or does not exist. (If the limit does not exist, write DNE)

\input{Limit-Compute-0002.HELP.tex}

\[\lim_{x\to{-7}}\dfrac{2 \, x + 14}{x^{2} + 10 \, x + 21}=\answer{-\frac{1}{2}}\]
\end{problem}}%}

\latexProblemContent{
\ifVerboseLocation This is Derivative Compute Question 0002. \\ \fi
\begin{problem}

Determine if the limit approaches a finite number, $\pm\infty$, or does not exist. (If the limit does not exist, write DNE)

\input{Limit-Compute-0002.HELP.tex}

\[\lim_{x\to{-9}}\dfrac{3 \, x + 27}{x^{2} + 6 \, x - 27}=\answer{-\frac{1}{4}}\]
\end{problem}}%}

\latexProblemContent{
\ifVerboseLocation This is Derivative Compute Question 0002. \\ \fi
\begin{problem}

Determine if the limit approaches a finite number, $\pm\infty$, or does not exist. (If the limit does not exist, write DNE)

\input{Limit-Compute-0002.HELP.tex}

\[\lim_{x\to{5}}\dfrac{2 \, x - 10}{x^{2} - 3 \, x - 10}=\answer{\frac{2}{7}}\]
\end{problem}}%}

\latexProblemContent{
\ifVerboseLocation This is Derivative Compute Question 0002. \\ \fi
\begin{problem}

Determine if the limit approaches a finite number, $\pm\infty$, or does not exist. (If the limit does not exist, write DNE)

\input{Limit-Compute-0002.HELP.tex}

\[\lim_{x\to{9}}\dfrac{2 \, x - 18}{x^{2} - 19 \, x + 90}=\answer{-2}\]
\end{problem}}%}

\latexProblemContent{
\ifVerboseLocation This is Derivative Compute Question 0002. \\ \fi
\begin{problem}

Determine if the limit approaches a finite number, $\pm\infty$, or does not exist. (If the limit does not exist, write DNE)

\input{Limit-Compute-0002.HELP.tex}

\[\lim_{x\to{8}}\dfrac{3 \, x - 24}{x^{2} - 11 \, x + 24}=\answer{\frac{3}{5}}\]
\end{problem}}%}

\latexProblemContent{
\ifVerboseLocation This is Derivative Compute Question 0002. \\ \fi
\begin{problem}

Determine if the limit approaches a finite number, $\pm\infty$, or does not exist. (If the limit does not exist, write DNE)

\input{Limit-Compute-0002.HELP.tex}

\[\lim_{x\to{10}}\dfrac{2 \, x - 20}{x^{2} - 100}=\answer{\frac{1}{10}}\]
\end{problem}}%}

\latexProblemContent{
\ifVerboseLocation This is Derivative Compute Question 0002. \\ \fi
\begin{problem}

Determine if the limit approaches a finite number, $\pm\infty$, or does not exist. (If the limit does not exist, write DNE)

\input{Limit-Compute-0002.HELP.tex}

\[\lim_{x\to{-8}}\dfrac{2 \, x + 16}{x^{2} + 3 \, x - 40}=\answer{-\frac{2}{13}}\]
\end{problem}}%}

\latexProblemContent{
\ifVerboseLocation This is Derivative Compute Question 0002. \\ \fi
\begin{problem}

Determine if the limit approaches a finite number, $\pm\infty$, or does not exist. (If the limit does not exist, write DNE)

\input{Limit-Compute-0002.HELP.tex}

\[\lim_{x\to{1}}\dfrac{x - 1}{x^{2} - 9 \, x + 8}=\answer{-\frac{1}{7}}\]
\end{problem}}%}

\latexProblemContent{
\ifVerboseLocation This is Derivative Compute Question 0002. \\ \fi
\begin{problem}

Determine if the limit approaches a finite number, $\pm\infty$, or does not exist. (If the limit does not exist, write DNE)

\input{Limit-Compute-0002.HELP.tex}

\[\lim_{x\to{2}}\dfrac{x - 2}{x^{2} + 3 \, x - 10}=\answer{\frac{1}{7}}\]
\end{problem}}%}

\latexProblemContent{
\ifVerboseLocation This is Derivative Compute Question 0002. \\ \fi
\begin{problem}

Determine if the limit approaches a finite number, $\pm\infty$, or does not exist. (If the limit does not exist, write DNE)

\input{Limit-Compute-0002.HELP.tex}

\[\lim_{x\to{-7}}\dfrac{2 \, x + 14}{x^{2} - 3 \, x - 70}=\answer{-\frac{2}{17}}\]
\end{problem}}%}

\latexProblemContent{
\ifVerboseLocation This is Derivative Compute Question 0002. \\ \fi
\begin{problem}

Determine if the limit approaches a finite number, $\pm\infty$, or does not exist. (If the limit does not exist, write DNE)

\input{Limit-Compute-0002.HELP.tex}

\[\lim_{x\to{3}}\dfrac{2 \, x - 6}{x^{2} - 5 \, x + 6}=\answer{2}\]
\end{problem}}%}

\latexProblemContent{
\ifVerboseLocation This is Derivative Compute Question 0002. \\ \fi
\begin{problem}

Determine if the limit approaches a finite number, $\pm\infty$, or does not exist. (If the limit does not exist, write DNE)

\input{Limit-Compute-0002.HELP.tex}

\[\lim_{x\to{-6}}\dfrac{3 \, x + 18}{x^{2} + 2 \, x - 24}=\answer{-\frac{3}{10}}\]
\end{problem}}%}

\latexProblemContent{
\ifVerboseLocation This is Derivative Compute Question 0002. \\ \fi
\begin{problem}

Determine if the limit approaches a finite number, $\pm\infty$, or does not exist. (If the limit does not exist, write DNE)

\input{Limit-Compute-0002.HELP.tex}

\[\lim_{x\to{-10}}\dfrac{3 \, x + 30}{x^{2} + 8 \, x - 20}=\answer{-\frac{1}{4}}\]
\end{problem}}%}

\latexProblemContent{
\ifVerboseLocation This is Derivative Compute Question 0002. \\ \fi
\begin{problem}

Determine if the limit approaches a finite number, $\pm\infty$, or does not exist. (If the limit does not exist, write DNE)

\input{Limit-Compute-0002.HELP.tex}

\[\lim_{x\to{6}}\dfrac{3 \, x - 18}{x^{2} - x - 30}=\answer{\frac{3}{11}}\]
\end{problem}}%}

\latexProblemContent{
\ifVerboseLocation This is Derivative Compute Question 0002. \\ \fi
\begin{problem}

Determine if the limit approaches a finite number, $\pm\infty$, or does not exist. (If the limit does not exist, write DNE)

\input{Limit-Compute-0002.HELP.tex}

\[\lim_{x\to{10}}\dfrac{3 \, x - 30}{x^{2} - 14 \, x + 40}=\answer{\frac{1}{2}}\]
\end{problem}}%}

\latexProblemContent{
\ifVerboseLocation This is Derivative Compute Question 0002. \\ \fi
\begin{problem}

Determine if the limit approaches a finite number, $\pm\infty$, or does not exist. (If the limit does not exist, write DNE)

\input{Limit-Compute-0002.HELP.tex}

\[\lim_{x\to{1}}\dfrac{x - 1}{x^{2} - 8 \, x + 7}=\answer{-\frac{1}{6}}\]
\end{problem}}%}

\latexProblemContent{
\ifVerboseLocation This is Derivative Compute Question 0002. \\ \fi
\begin{problem}

Determine if the limit approaches a finite number, $\pm\infty$, or does not exist. (If the limit does not exist, write DNE)

\input{Limit-Compute-0002.HELP.tex}

\[\lim_{x\to{-9}}\dfrac{2 \, x + 18}{x^{2} + 8 \, x - 9}=\answer{-\frac{1}{5}}\]
\end{problem}}%}

\latexProblemContent{
\ifVerboseLocation This is Derivative Compute Question 0002. \\ \fi
\begin{problem}

Determine if the limit approaches a finite number, $\pm\infty$, or does not exist. (If the limit does not exist, write DNE)

\input{Limit-Compute-0002.HELP.tex}

\[\lim_{x\to{-3}}\dfrac{x + 3}{x^{2} - 4 \, x - 21}=\answer{-\frac{1}{10}}\]
\end{problem}}%}

\latexProblemContent{
\ifVerboseLocation This is Derivative Compute Question 0002. \\ \fi
\begin{problem}

Determine if the limit approaches a finite number, $\pm\infty$, or does not exist. (If the limit does not exist, write DNE)

\input{Limit-Compute-0002.HELP.tex}

\[\lim_{x\to{-5}}\dfrac{3 \, x + 15}{x^{2} + 4 \, x - 5}=\answer{-\frac{1}{2}}\]
\end{problem}}%}

\latexProblemContent{
\ifVerboseLocation This is Derivative Compute Question 0002. \\ \fi
\begin{problem}

Determine if the limit approaches a finite number, $\pm\infty$, or does not exist. (If the limit does not exist, write DNE)

\input{Limit-Compute-0002.HELP.tex}

\[\lim_{x\to{-3}}\dfrac{2 \, x + 6}{x^{2} - 5 \, x - 24}=\answer{-\frac{2}{11}}\]
\end{problem}}%}

\latexProblemContent{
\ifVerboseLocation This is Derivative Compute Question 0002. \\ \fi
\begin{problem}

Determine if the limit approaches a finite number, $\pm\infty$, or does not exist. (If the limit does not exist, write DNE)

\input{Limit-Compute-0002.HELP.tex}

\[\lim_{x\to{-2}}\dfrac{2 \, x + 4}{x^{2} - 6 \, x - 16}=\answer{-\frac{1}{5}}\]
\end{problem}}%}

\latexProblemContent{
\ifVerboseLocation This is Derivative Compute Question 0002. \\ \fi
\begin{problem}

Determine if the limit approaches a finite number, $\pm\infty$, or does not exist. (If the limit does not exist, write DNE)

\input{Limit-Compute-0002.HELP.tex}

\[\lim_{x\to{-3}}\dfrac{2 \, x + 6}{x^{2} + 8 \, x + 15}=\answer{1}\]
\end{problem}}%}

\latexProblemContent{
\ifVerboseLocation This is Derivative Compute Question 0002. \\ \fi
\begin{problem}

Determine if the limit approaches a finite number, $\pm\infty$, or does not exist. (If the limit does not exist, write DNE)

\input{Limit-Compute-0002.HELP.tex}

\[\lim_{x\to{-1}}\dfrac{3 \, x + 3}{x^{2} - 7 \, x - 8}=\answer{-\frac{1}{3}}\]
\end{problem}}%}

\latexProblemContent{
\ifVerboseLocation This is Derivative Compute Question 0002. \\ \fi
\begin{problem}

Determine if the limit approaches a finite number, $\pm\infty$, or does not exist. (If the limit does not exist, write DNE)

\input{Limit-Compute-0002.HELP.tex}

\[\lim_{x\to{6}}\dfrac{2 \, x - 12}{x^{2} - 7 \, x + 6}=\answer{\frac{2}{5}}\]
\end{problem}}%}

\latexProblemContent{
\ifVerboseLocation This is Derivative Compute Question 0002. \\ \fi
\begin{problem}

Determine if the limit approaches a finite number, $\pm\infty$, or does not exist. (If the limit does not exist, write DNE)

\input{Limit-Compute-0002.HELP.tex}

\[\lim_{x\to{-10}}\dfrac{2 \, x + 20}{x^{2} + 15 \, x + 50}=\answer{-\frac{2}{5}}\]
\end{problem}}%}

\latexProblemContent{
\ifVerboseLocation This is Derivative Compute Question 0002. \\ \fi
\begin{problem}

Determine if the limit approaches a finite number, $\pm\infty$, or does not exist. (If the limit does not exist, write DNE)

\input{Limit-Compute-0002.HELP.tex}

\[\lim_{x\to{-8}}\dfrac{2 \, x + 16}{x^{2} + 14 \, x + 48}=\answer{-1}\]
\end{problem}}%}

\latexProblemContent{
\ifVerboseLocation This is Derivative Compute Question 0002. \\ \fi
\begin{problem}

Determine if the limit approaches a finite number, $\pm\infty$, or does not exist. (If the limit does not exist, write DNE)

\input{Limit-Compute-0002.HELP.tex}

\[\lim_{x\to{7}}\dfrac{3 \, x - 21}{x^{2} - 11 \, x + 28}=\answer{1}\]
\end{problem}}%}

\latexProblemContent{
\ifVerboseLocation This is Derivative Compute Question 0002. \\ \fi
\begin{problem}

Determine if the limit approaches a finite number, $\pm\infty$, or does not exist. (If the limit does not exist, write DNE)

\input{Limit-Compute-0002.HELP.tex}

\[\lim_{x\to{-8}}\dfrac{x + 8}{x^{2} + 5 \, x - 24}=\answer{-\frac{1}{11}}\]
\end{problem}}%}

\latexProblemContent{
\ifVerboseLocation This is Derivative Compute Question 0002. \\ \fi
\begin{problem}

Determine if the limit approaches a finite number, $\pm\infty$, or does not exist. (If the limit does not exist, write DNE)

\input{Limit-Compute-0002.HELP.tex}

\[\lim_{x\to{7}}\dfrac{x - 7}{x^{2} - 13 \, x + 42}=\answer{1}\]
\end{problem}}%}

\latexProblemContent{
\ifVerboseLocation This is Derivative Compute Question 0002. \\ \fi
\begin{problem}

Determine if the limit approaches a finite number, $\pm\infty$, or does not exist. (If the limit does not exist, write DNE)

\input{Limit-Compute-0002.HELP.tex}

\[\lim_{x\to{-3}}\dfrac{2 \, x + 6}{x^{2} - 6 \, x - 27}=\answer{-\frac{1}{6}}\]
\end{problem}}%}

\latexProblemContent{
\ifVerboseLocation This is Derivative Compute Question 0002. \\ \fi
\begin{problem}

Determine if the limit approaches a finite number, $\pm\infty$, or does not exist. (If the limit does not exist, write DNE)

\input{Limit-Compute-0002.HELP.tex}

\[\lim_{x\to{-5}}\dfrac{x + 5}{x^{2} + 3 \, x - 10}=\answer{-\frac{1}{7}}\]
\end{problem}}%}

\latexProblemContent{
\ifVerboseLocation This is Derivative Compute Question 0002. \\ \fi
\begin{problem}

Determine if the limit approaches a finite number, $\pm\infty$, or does not exist. (If the limit does not exist, write DNE)

\input{Limit-Compute-0002.HELP.tex}

\[\lim_{x\to{5}}\dfrac{2 \, x - 10}{x^{2} - x - 20}=\answer{\frac{2}{9}}\]
\end{problem}}%}

\latexProblemContent{
\ifVerboseLocation This is Derivative Compute Question 0002. \\ \fi
\begin{problem}

Determine if the limit approaches a finite number, $\pm\infty$, or does not exist. (If the limit does not exist, write DNE)

\input{Limit-Compute-0002.HELP.tex}

\[\lim_{x\to{7}}\dfrac{3 \, x - 21}{x^{2} - 6 \, x - 7}=\answer{\frac{3}{8}}\]
\end{problem}}%}

\latexProblemContent{
\ifVerboseLocation This is Derivative Compute Question 0002. \\ \fi
\begin{problem}

Determine if the limit approaches a finite number, $\pm\infty$, or does not exist. (If the limit does not exist, write DNE)

\input{Limit-Compute-0002.HELP.tex}

\[\lim_{x\to{9}}\dfrac{2 \, x - 18}{x^{2} - 2 \, x - 63}=\answer{\frac{1}{8}}\]
\end{problem}}%}

\latexProblemContent{
\ifVerboseLocation This is Derivative Compute Question 0002. \\ \fi
\begin{problem}

Determine if the limit approaches a finite number, $\pm\infty$, or does not exist. (If the limit does not exist, write DNE)

\input{Limit-Compute-0002.HELP.tex}

\[\lim_{x\to{10}}\dfrac{3 \, x - 30}{x^{2} - 19 \, x + 90}=\answer{3}\]
\end{problem}}%}

\latexProblemContent{
\ifVerboseLocation This is Derivative Compute Question 0002. \\ \fi
\begin{problem}

Determine if the limit approaches a finite number, $\pm\infty$, or does not exist. (If the limit does not exist, write DNE)

\input{Limit-Compute-0002.HELP.tex}

\[\lim_{x\to{-1}}\dfrac{x + 1}{x^{2} + 7 \, x + 6}=\answer{\frac{1}{5}}\]
\end{problem}}%}

\latexProblemContent{
\ifVerboseLocation This is Derivative Compute Question 0002. \\ \fi
\begin{problem}

Determine if the limit approaches a finite number, $\pm\infty$, or does not exist. (If the limit does not exist, write DNE)

\input{Limit-Compute-0002.HELP.tex}

\[\lim_{x\to{9}}\dfrac{2 \, x - 18}{x^{2} - 17 \, x + 72}=\answer{2}\]
\end{problem}}%}

\latexProblemContent{
\ifVerboseLocation This is Derivative Compute Question 0002. \\ \fi
\begin{problem}

Determine if the limit approaches a finite number, $\pm\infty$, or does not exist. (If the limit does not exist, write DNE)

\input{Limit-Compute-0002.HELP.tex}

\[\lim_{x\to{4}}\dfrac{x - 4}{x^{2} - 7 \, x + 12}=\answer{1}\]
\end{problem}}%}

\latexProblemContent{
\ifVerboseLocation This is Derivative Compute Question 0002. \\ \fi
\begin{problem}

Determine if the limit approaches a finite number, $\pm\infty$, or does not exist. (If the limit does not exist, write DNE)

\input{Limit-Compute-0002.HELP.tex}

\[\lim_{x\to{-7}}\dfrac{x + 7}{x^{2} + x - 42}=\answer{-\frac{1}{13}}\]
\end{problem}}%}

\latexProblemContent{
\ifVerboseLocation This is Derivative Compute Question 0002. \\ \fi
\begin{problem}

Determine if the limit approaches a finite number, $\pm\infty$, or does not exist. (If the limit does not exist, write DNE)

\input{Limit-Compute-0002.HELP.tex}

\[\lim_{x\to{-6}}\dfrac{x + 6}{x^{2} + 9 \, x + 18}=\answer{-\frac{1}{3}}\]
\end{problem}}%}

\latexProblemContent{
\ifVerboseLocation This is Derivative Compute Question 0002. \\ \fi
\begin{problem}

Determine if the limit approaches a finite number, $\pm\infty$, or does not exist. (If the limit does not exist, write DNE)

\input{Limit-Compute-0002.HELP.tex}

\[\lim_{x\to{-9}}\dfrac{3 \, x + 27}{x^{2} + 15 \, x + 54}=\answer{-1}\]
\end{problem}}%}

\latexProblemContent{
\ifVerboseLocation This is Derivative Compute Question 0002. \\ \fi
\begin{problem}

Determine if the limit approaches a finite number, $\pm\infty$, or does not exist. (If the limit does not exist, write DNE)

\input{Limit-Compute-0002.HELP.tex}

\[\lim_{x\to{-1}}\dfrac{x + 1}{x^{2} + 9 \, x + 8}=\answer{\frac{1}{7}}\]
\end{problem}}%}

\latexProblemContent{
\ifVerboseLocation This is Derivative Compute Question 0002. \\ \fi
\begin{problem}

Determine if the limit approaches a finite number, $\pm\infty$, or does not exist. (If the limit does not exist, write DNE)

\input{Limit-Compute-0002.HELP.tex}

\[\lim_{x\to{-6}}\dfrac{2 \, x + 12}{x^{2} - 3 \, x - 54}=\answer{-\frac{2}{15}}\]
\end{problem}}%}

\latexProblemContent{
\ifVerboseLocation This is Derivative Compute Question 0002. \\ \fi
\begin{problem}

Determine if the limit approaches a finite number, $\pm\infty$, or does not exist. (If the limit does not exist, write DNE)

\input{Limit-Compute-0002.HELP.tex}

\[\lim_{x\to{9}}\dfrac{x - 9}{x^{2} - 8 \, x - 9}=\answer{\frac{1}{10}}\]
\end{problem}}%}

\latexProblemContent{
\ifVerboseLocation This is Derivative Compute Question 0002. \\ \fi
\begin{problem}

Determine if the limit approaches a finite number, $\pm\infty$, or does not exist. (If the limit does not exist, write DNE)

\input{Limit-Compute-0002.HELP.tex}

\[\lim_{x\to{-1}}\dfrac{x + 1}{x^{2} + 3 \, x + 2}=\answer{1}\]
\end{problem}}%}

\latexProblemContent{
\ifVerboseLocation This is Derivative Compute Question 0002. \\ \fi
\begin{problem}

Determine if the limit approaches a finite number, $\pm\infty$, or does not exist. (If the limit does not exist, write DNE)

\input{Limit-Compute-0002.HELP.tex}

\[\lim_{x\to{-4}}\dfrac{3 \, x + 12}{x^{2} + 12 \, x + 32}=\answer{\frac{3}{4}}\]
\end{problem}}%}

\latexProblemContent{
\ifVerboseLocation This is Derivative Compute Question 0002. \\ \fi
\begin{problem}

Determine if the limit approaches a finite number, $\pm\infty$, or does not exist. (If the limit does not exist, write DNE)

\input{Limit-Compute-0002.HELP.tex}

\[\lim_{x\to{-7}}\dfrac{3 \, x + 21}{x^{2} - 3 \, x - 70}=\answer{-\frac{3}{17}}\]
\end{problem}}%}

\latexProblemContent{
\ifVerboseLocation This is Derivative Compute Question 0002. \\ \fi
\begin{problem}

Determine if the limit approaches a finite number, $\pm\infty$, or does not exist. (If the limit does not exist, write DNE)

\input{Limit-Compute-0002.HELP.tex}

\[\lim_{x\to{-3}}\dfrac{x + 3}{x^{2} - 7 \, x - 30}=\answer{-\frac{1}{13}}\]
\end{problem}}%}

\latexProblemContent{
\ifVerboseLocation This is Derivative Compute Question 0002. \\ \fi
\begin{problem}

Determine if the limit approaches a finite number, $\pm\infty$, or does not exist. (If the limit does not exist, write DNE)

\input{Limit-Compute-0002.HELP.tex}

\[\lim_{x\to{-1}}\dfrac{2 \, x + 2}{x^{2} - 4 \, x - 5}=\answer{-\frac{1}{3}}\]
\end{problem}}%}

\latexProblemContent{
\ifVerboseLocation This is Derivative Compute Question 0002. \\ \fi
\begin{problem}

Determine if the limit approaches a finite number, $\pm\infty$, or does not exist. (If the limit does not exist, write DNE)

\input{Limit-Compute-0002.HELP.tex}

\[\lim_{x\to{3}}\dfrac{3 \, x - 9}{x^{2} - 8 \, x + 15}=\answer{-\frac{3}{2}}\]
\end{problem}}%}

\latexProblemContent{
\ifVerboseLocation This is Derivative Compute Question 0002. \\ \fi
\begin{problem}

Determine if the limit approaches a finite number, $\pm\infty$, or does not exist. (If the limit does not exist, write DNE)

\input{Limit-Compute-0002.HELP.tex}

\[\lim_{x\to{-4}}\dfrac{x + 4}{x^{2} + 9 \, x + 20}=\answer{1}\]
\end{problem}}%}

\latexProblemContent{
\ifVerboseLocation This is Derivative Compute Question 0002. \\ \fi
\begin{problem}

Determine if the limit approaches a finite number, $\pm\infty$, or does not exist. (If the limit does not exist, write DNE)

\input{Limit-Compute-0002.HELP.tex}

\[\lim_{x\to{3}}\dfrac{x - 3}{x^{2} - 12 \, x + 27}=\answer{-\frac{1}{6}}\]
\end{problem}}%}

\latexProblemContent{
\ifVerboseLocation This is Derivative Compute Question 0002. \\ \fi
\begin{problem}

Determine if the limit approaches a finite number, $\pm\infty$, or does not exist. (If the limit does not exist, write DNE)

\input{Limit-Compute-0002.HELP.tex}

\[\lim_{x\to{3}}\dfrac{x - 3}{x^{2} - 13 \, x + 30}=\answer{-\frac{1}{7}}\]
\end{problem}}%}

\latexProblemContent{
\ifVerboseLocation This is Derivative Compute Question 0002. \\ \fi
\begin{problem}

Determine if the limit approaches a finite number, $\pm\infty$, or does not exist. (If the limit does not exist, write DNE)

\input{Limit-Compute-0002.HELP.tex}

\[\lim_{x\to{-5}}\dfrac{x + 5}{x^{2} + 4 \, x - 5}=\answer{-\frac{1}{6}}\]
\end{problem}}%}

\latexProblemContent{
\ifVerboseLocation This is Derivative Compute Question 0002. \\ \fi
\begin{problem}

Determine if the limit approaches a finite number, $\pm\infty$, or does not exist. (If the limit does not exist, write DNE)

\input{Limit-Compute-0002.HELP.tex}

\[\lim_{x\to{-4}}\dfrac{x + 4}{x^{2} - 5 \, x - 36}=\answer{-\frac{1}{13}}\]
\end{problem}}%}

\latexProblemContent{
\ifVerboseLocation This is Derivative Compute Question 0002. \\ \fi
\begin{problem}

Determine if the limit approaches a finite number, $\pm\infty$, or does not exist. (If the limit does not exist, write DNE)

\input{Limit-Compute-0002.HELP.tex}

\[\lim_{x\to{6}}\dfrac{3 \, x - 18}{x^{2} - 11 \, x + 30}=\answer{3}\]
\end{problem}}%}

\latexProblemContent{
\ifVerboseLocation This is Derivative Compute Question 0002. \\ \fi
\begin{problem}

Determine if the limit approaches a finite number, $\pm\infty$, or does not exist. (If the limit does not exist, write DNE)

\input{Limit-Compute-0002.HELP.tex}

\[\lim_{x\to{-6}}\dfrac{3 \, x + 18}{x^{2} + 9 \, x + 18}=\answer{-1}\]
\end{problem}}%}

\latexProblemContent{
\ifVerboseLocation This is Derivative Compute Question 0002. \\ \fi
\begin{problem}

Determine if the limit approaches a finite number, $\pm\infty$, or does not exist. (If the limit does not exist, write DNE)

\input{Limit-Compute-0002.HELP.tex}

\[\lim_{x\to{3}}\dfrac{x - 3}{x^{2} - 10 \, x + 21}=\answer{-\frac{1}{4}}\]
\end{problem}}%}

\latexProblemContent{
\ifVerboseLocation This is Derivative Compute Question 0002. \\ \fi
\begin{problem}

Determine if the limit approaches a finite number, $\pm\infty$, or does not exist. (If the limit does not exist, write DNE)

\input{Limit-Compute-0002.HELP.tex}

\[\lim_{x\to{-5}}\dfrac{x + 5}{x^{2} + 13 \, x + 40}=\answer{\frac{1}{3}}\]
\end{problem}}%}

\latexProblemContent{
\ifVerboseLocation This is Derivative Compute Question 0002. \\ \fi
\begin{problem}

Determine if the limit approaches a finite number, $\pm\infty$, or does not exist. (If the limit does not exist, write DNE)

\input{Limit-Compute-0002.HELP.tex}

\[\lim_{x\to{5}}\dfrac{3 \, x - 15}{x^{2} + 2 \, x - 35}=\answer{\frac{1}{4}}\]
\end{problem}}%}

\latexProblemContent{
\ifVerboseLocation This is Derivative Compute Question 0002. \\ \fi
\begin{problem}

Determine if the limit approaches a finite number, $\pm\infty$, or does not exist. (If the limit does not exist, write DNE)

\input{Limit-Compute-0002.HELP.tex}

\[\lim_{x\to{6}}\dfrac{3 \, x - 18}{x^{2} - 13 \, x + 42}=\answer{-3}\]
\end{problem}}%}

\latexProblemContent{
\ifVerboseLocation This is Derivative Compute Question 0002. \\ \fi
\begin{problem}

Determine if the limit approaches a finite number, $\pm\infty$, or does not exist. (If the limit does not exist, write DNE)

\input{Limit-Compute-0002.HELP.tex}

\[\lim_{x\to{5}}\dfrac{x - 5}{x^{2} - 12 \, x + 35}=\answer{-\frac{1}{2}}\]
\end{problem}}%}

\latexProblemContent{
\ifVerboseLocation This is Derivative Compute Question 0002. \\ \fi
\begin{problem}

Determine if the limit approaches a finite number, $\pm\infty$, or does not exist. (If the limit does not exist, write DNE)

\input{Limit-Compute-0002.HELP.tex}

\[\lim_{x\to{-8}}\dfrac{x + 8}{x^{2} - x - 72}=\answer{-\frac{1}{17}}\]
\end{problem}}%}

\latexProblemContent{
\ifVerboseLocation This is Derivative Compute Question 0002. \\ \fi
\begin{problem}

Determine if the limit approaches a finite number, $\pm\infty$, or does not exist. (If the limit does not exist, write DNE)

\input{Limit-Compute-0002.HELP.tex}

\[\lim_{x\to{-9}}\dfrac{3 \, x + 27}{x^{2} + 5 \, x - 36}=\answer{-\frac{3}{13}}\]
\end{problem}}%}

\latexProblemContent{
\ifVerboseLocation This is Derivative Compute Question 0002. \\ \fi
\begin{problem}

Determine if the limit approaches a finite number, $\pm\infty$, or does not exist. (If the limit does not exist, write DNE)

\input{Limit-Compute-0002.HELP.tex}

\[\lim_{x\to{-4}}\dfrac{x + 4}{x^{2} + 12 \, x + 32}=\answer{\frac{1}{4}}\]
\end{problem}}%}

\latexProblemContent{
\ifVerboseLocation This is Derivative Compute Question 0002. \\ \fi
\begin{problem}

Determine if the limit approaches a finite number, $\pm\infty$, or does not exist. (If the limit does not exist, write DNE)

\input{Limit-Compute-0002.HELP.tex}

\[\lim_{x\to{-7}}\dfrac{x + 7}{x^{2} + 8 \, x + 7}=\answer{-\frac{1}{6}}\]
\end{problem}}%}

\latexProblemContent{
\ifVerboseLocation This is Derivative Compute Question 0002. \\ \fi
\begin{problem}

Determine if the limit approaches a finite number, $\pm\infty$, or does not exist. (If the limit does not exist, write DNE)

\input{Limit-Compute-0002.HELP.tex}

\[\lim_{x\to{8}}\dfrac{3 \, x - 24}{x^{2} - 4 \, x - 32}=\answer{\frac{1}{4}}\]
\end{problem}}%}

\latexProblemContent{
\ifVerboseLocation This is Derivative Compute Question 0002. \\ \fi
\begin{problem}

Determine if the limit approaches a finite number, $\pm\infty$, or does not exist. (If the limit does not exist, write DNE)

\input{Limit-Compute-0002.HELP.tex}

\[\lim_{x\to{-4}}\dfrac{x + 4}{x^{2} + 2 \, x - 8}=\answer{-\frac{1}{6}}\]
\end{problem}}%}

\latexProblemContent{
\ifVerboseLocation This is Derivative Compute Question 0002. \\ \fi
\begin{problem}

Determine if the limit approaches a finite number, $\pm\infty$, or does not exist. (If the limit does not exist, write DNE)

\input{Limit-Compute-0002.HELP.tex}

\[\lim_{x\to{-1}}\dfrac{2 \, x + 2}{x^{2} - 8 \, x - 9}=\answer{-\frac{1}{5}}\]
\end{problem}}%}

\latexProblemContent{
\ifVerboseLocation This is Derivative Compute Question 0002. \\ \fi
\begin{problem}

Determine if the limit approaches a finite number, $\pm\infty$, or does not exist. (If the limit does not exist, write DNE)

\input{Limit-Compute-0002.HELP.tex}

\[\lim_{x\to{-5}}\dfrac{2 \, x + 10}{x^{2} - 5 \, x - 50}=\answer{-\frac{2}{15}}\]
\end{problem}}%}

\latexProblemContent{
\ifVerboseLocation This is Derivative Compute Question 0002. \\ \fi
\begin{problem}

Determine if the limit approaches a finite number, $\pm\infty$, or does not exist. (If the limit does not exist, write DNE)

\input{Limit-Compute-0002.HELP.tex}

\[\lim_{x\to{-10}}\dfrac{2 \, x + 20}{x^{2} + 19 \, x + 90}=\answer{-2}\]
\end{problem}}%}

\latexProblemContent{
\ifVerboseLocation This is Derivative Compute Question 0002. \\ \fi
\begin{problem}

Determine if the limit approaches a finite number, $\pm\infty$, or does not exist. (If the limit does not exist, write DNE)

\input{Limit-Compute-0002.HELP.tex}

\[\lim_{x\to{-4}}\dfrac{3 \, x + 12}{x^{2} - 4 \, x - 32}=\answer{-\frac{1}{4}}\]
\end{problem}}%}

\latexProblemContent{
\ifVerboseLocation This is Derivative Compute Question 0002. \\ \fi
\begin{problem}

Determine if the limit approaches a finite number, $\pm\infty$, or does not exist. (If the limit does not exist, write DNE)

\input{Limit-Compute-0002.HELP.tex}

\[\lim_{x\to{4}}\dfrac{3 \, x - 12}{x^{2} + x - 20}=\answer{\frac{1}{3}}\]
\end{problem}}%}

\latexProblemContent{
\ifVerboseLocation This is Derivative Compute Question 0002. \\ \fi
\begin{problem}

Determine if the limit approaches a finite number, $\pm\infty$, or does not exist. (If the limit does not exist, write DNE)

\input{Limit-Compute-0002.HELP.tex}

\[\lim_{x\to{-8}}\dfrac{3 \, x + 24}{x^{2} + 11 \, x + 24}=\answer{-\frac{3}{5}}\]
\end{problem}}%}

\latexProblemContent{
\ifVerboseLocation This is Derivative Compute Question 0002. \\ \fi
\begin{problem}

Determine if the limit approaches a finite number, $\pm\infty$, or does not exist. (If the limit does not exist, write DNE)

\input{Limit-Compute-0002.HELP.tex}

\[\lim_{x\to{-7}}\dfrac{x + 7}{x^{2} + 13 \, x + 42}=\answer{-1}\]
\end{problem}}%}

\latexProblemContent{
\ifVerboseLocation This is Derivative Compute Question 0002. \\ \fi
\begin{problem}

Determine if the limit approaches a finite number, $\pm\infty$, or does not exist. (If the limit does not exist, write DNE)

\input{Limit-Compute-0002.HELP.tex}

\[\lim_{x\to{-7}}\dfrac{2 \, x + 14}{x^{2} + 2 \, x - 35}=\answer{-\frac{1}{6}}\]
\end{problem}}%}

\latexProblemContent{
\ifVerboseLocation This is Derivative Compute Question 0002. \\ \fi
\begin{problem}

Determine if the limit approaches a finite number, $\pm\infty$, or does not exist. (If the limit does not exist, write DNE)

\input{Limit-Compute-0002.HELP.tex}

\[\lim_{x\to{-9}}\dfrac{x + 9}{x^{2} + 11 \, x + 18}=\answer{-\frac{1}{7}}\]
\end{problem}}%}

\latexProblemContent{
\ifVerboseLocation This is Derivative Compute Question 0002. \\ \fi
\begin{problem}

Determine if the limit approaches a finite number, $\pm\infty$, or does not exist. (If the limit does not exist, write DNE)

\input{Limit-Compute-0002.HELP.tex}

\[\lim_{x\to{10}}\dfrac{x - 10}{x^{2} - 11 \, x + 10}=\answer{\frac{1}{9}}\]
\end{problem}}%}

\latexProblemContent{
\ifVerboseLocation This is Derivative Compute Question 0002. \\ \fi
\begin{problem}

Determine if the limit approaches a finite number, $\pm\infty$, or does not exist. (If the limit does not exist, write DNE)

\input{Limit-Compute-0002.HELP.tex}

\[\lim_{x\to{9}}\dfrac{x - 9}{x^{2} - 81}=\answer{\frac{1}{18}}\]
\end{problem}}%}

\latexProblemContent{
\ifVerboseLocation This is Derivative Compute Question 0002. \\ \fi
\begin{problem}

Determine if the limit approaches a finite number, $\pm\infty$, or does not exist. (If the limit does not exist, write DNE)

\input{Limit-Compute-0002.HELP.tex}

\[\lim_{x\to{2}}\dfrac{3 \, x - 6}{x^{2} + 7 \, x - 18}=\answer{\frac{3}{11}}\]
\end{problem}}%}

\latexProblemContent{
\ifVerboseLocation This is Derivative Compute Question 0002. \\ \fi
\begin{problem}

Determine if the limit approaches a finite number, $\pm\infty$, or does not exist. (If the limit does not exist, write DNE)

\input{Limit-Compute-0002.HELP.tex}

\[\lim_{x\to{8}}\dfrac{3 \, x - 24}{x^{2} - 5 \, x - 24}=\answer{\frac{3}{11}}\]
\end{problem}}%}

\latexProblemContent{
\ifVerboseLocation This is Derivative Compute Question 0002. \\ \fi
\begin{problem}

Determine if the limit approaches a finite number, $\pm\infty$, or does not exist. (If the limit does not exist, write DNE)

\input{Limit-Compute-0002.HELP.tex}

\[\lim_{x\to{10}}\dfrac{2 \, x - 20}{x^{2} - 3 \, x - 70}=\answer{\frac{2}{17}}\]
\end{problem}}%}

\latexProblemContent{
\ifVerboseLocation This is Derivative Compute Question 0002. \\ \fi
\begin{problem}

Determine if the limit approaches a finite number, $\pm\infty$, or does not exist. (If the limit does not exist, write DNE)

\input{Limit-Compute-0002.HELP.tex}

\[\lim_{x\to{2}}\dfrac{x - 2}{x^{2} - 5 \, x + 6}=\answer{-1}\]
\end{problem}}%}

\latexProblemContent{
\ifVerboseLocation This is Derivative Compute Question 0002. \\ \fi
\begin{problem}

Determine if the limit approaches a finite number, $\pm\infty$, or does not exist. (If the limit does not exist, write DNE)

\input{Limit-Compute-0002.HELP.tex}

\[\lim_{x\to{2}}\dfrac{3 \, x - 6}{x^{2} - 10 \, x + 16}=\answer{-\frac{1}{2}}\]
\end{problem}}%}

\latexProblemContent{
\ifVerboseLocation This is Derivative Compute Question 0002. \\ \fi
\begin{problem}

Determine if the limit approaches a finite number, $\pm\infty$, or does not exist. (If the limit does not exist, write DNE)

\input{Limit-Compute-0002.HELP.tex}

\[\lim_{x\to{-6}}\dfrac{x + 6}{x^{2} - 36}=\answer{-\frac{1}{12}}\]
\end{problem}}%}

\latexProblemContent{
\ifVerboseLocation This is Derivative Compute Question 0002. \\ \fi
\begin{problem}

Determine if the limit approaches a finite number, $\pm\infty$, or does not exist. (If the limit does not exist, write DNE)

\input{Limit-Compute-0002.HELP.tex}

\[\lim_{x\to{7}}\dfrac{3 \, x - 21}{x^{2} - 8 \, x + 7}=\answer{\frac{1}{2}}\]
\end{problem}}%}

\latexProblemContent{
\ifVerboseLocation This is Derivative Compute Question 0002. \\ \fi
\begin{problem}

Determine if the limit approaches a finite number, $\pm\infty$, or does not exist. (If the limit does not exist, write DNE)

\input{Limit-Compute-0002.HELP.tex}

\[\lim_{x\to{-6}}\dfrac{3 \, x + 18}{x^{2} + x - 30}=\answer{-\frac{3}{11}}\]
\end{problem}}%}

\latexProblemContent{
\ifVerboseLocation This is Derivative Compute Question 0002. \\ \fi
\begin{problem}

Determine if the limit approaches a finite number, $\pm\infty$, or does not exist. (If the limit does not exist, write DNE)

\input{Limit-Compute-0002.HELP.tex}

\[\lim_{x\to{-5}}\dfrac{2 \, x + 10}{x^{2} - 25}=\answer{-\frac{1}{5}}\]
\end{problem}}%}

\latexProblemContent{
\ifVerboseLocation This is Derivative Compute Question 0002. \\ \fi
\begin{problem}

Determine if the limit approaches a finite number, $\pm\infty$, or does not exist. (If the limit does not exist, write DNE)

\input{Limit-Compute-0002.HELP.tex}

\[\lim_{x\to{3}}\dfrac{x - 3}{x^{2} + 6 \, x - 27}=\answer{\frac{1}{12}}\]
\end{problem}}%}

\latexProblemContent{
\ifVerboseLocation This is Derivative Compute Question 0002. \\ \fi
\begin{problem}

Determine if the limit approaches a finite number, $\pm\infty$, or does not exist. (If the limit does not exist, write DNE)

\input{Limit-Compute-0002.HELP.tex}

\[\lim_{x\to{-6}}\dfrac{3 \, x + 18}{x^{2} + 7 \, x + 6}=\answer{-\frac{3}{5}}\]
\end{problem}}%}

\latexProblemContent{
\ifVerboseLocation This is Derivative Compute Question 0002. \\ \fi
\begin{problem}

Determine if the limit approaches a finite number, $\pm\infty$, or does not exist. (If the limit does not exist, write DNE)

\input{Limit-Compute-0002.HELP.tex}

\[\lim_{x\to{7}}\dfrac{3 \, x - 21}{x^{2} + 2 \, x - 63}=\answer{\frac{3}{16}}\]
\end{problem}}%}

\latexProblemContent{
\ifVerboseLocation This is Derivative Compute Question 0002. \\ \fi
\begin{problem}

Determine if the limit approaches a finite number, $\pm\infty$, or does not exist. (If the limit does not exist, write DNE)

\input{Limit-Compute-0002.HELP.tex}

\[\lim_{x\to{2}}\dfrac{2 \, x - 4}{x^{2} - 7 \, x + 10}=\answer{-\frac{2}{3}}\]
\end{problem}}%}

\latexProblemContent{
\ifVerboseLocation This is Derivative Compute Question 0002. \\ \fi
\begin{problem}

Determine if the limit approaches a finite number, $\pm\infty$, or does not exist. (If the limit does not exist, write DNE)

\input{Limit-Compute-0002.HELP.tex}

\[\lim_{x\to{-6}}\dfrac{x + 6}{x^{2} + 14 \, x + 48}=\answer{\frac{1}{2}}\]
\end{problem}}%}

\latexProblemContent{
\ifVerboseLocation This is Derivative Compute Question 0002. \\ \fi
\begin{problem}

Determine if the limit approaches a finite number, $\pm\infty$, or does not exist. (If the limit does not exist, write DNE)

\input{Limit-Compute-0002.HELP.tex}

\[\lim_{x\to{3}}\dfrac{3 \, x - 9}{x^{2} - 9 \, x + 18}=\answer{-1}\]
\end{problem}}%}

\latexProblemContent{
\ifVerboseLocation This is Derivative Compute Question 0002. \\ \fi
\begin{problem}

Determine if the limit approaches a finite number, $\pm\infty$, or does not exist. (If the limit does not exist, write DNE)

\input{Limit-Compute-0002.HELP.tex}

\[\lim_{x\to{-5}}\dfrac{x + 5}{x^{2} - 2 \, x - 35}=\answer{-\frac{1}{12}}\]
\end{problem}}%}

\latexProblemContent{
\ifVerboseLocation This is Derivative Compute Question 0002. \\ \fi
\begin{problem}

Determine if the limit approaches a finite number, $\pm\infty$, or does not exist. (If the limit does not exist, write DNE)

\input{Limit-Compute-0002.HELP.tex}

\[\lim_{x\to{7}}\dfrac{x - 7}{x^{2} - 16 \, x + 63}=\answer{-\frac{1}{2}}\]
\end{problem}}%}

\latexProblemContent{
\ifVerboseLocation This is Derivative Compute Question 0002. \\ \fi
\begin{problem}

Determine if the limit approaches a finite number, $\pm\infty$, or does not exist. (If the limit does not exist, write DNE)

\input{Limit-Compute-0002.HELP.tex}

\[\lim_{x\to{10}}\dfrac{x - 10}{x^{2} - 15 \, x + 50}=\answer{\frac{1}{5}}\]
\end{problem}}%}

\latexProblemContent{
\ifVerboseLocation This is Derivative Compute Question 0002. \\ \fi
\begin{problem}

Determine if the limit approaches a finite number, $\pm\infty$, or does not exist. (If the limit does not exist, write DNE)

\input{Limit-Compute-0002.HELP.tex}

\[\lim_{x\to{-9}}\dfrac{3 \, x + 27}{x^{2} + 2 \, x - 63}=\answer{-\frac{3}{16}}\]
\end{problem}}%}

\latexProblemContent{
\ifVerboseLocation This is Derivative Compute Question 0002. \\ \fi
\begin{problem}

Determine if the limit approaches a finite number, $\pm\infty$, or does not exist. (If the limit does not exist, write DNE)

\input{Limit-Compute-0002.HELP.tex}

\[\lim_{x\to{-6}}\dfrac{2 \, x + 12}{x^{2} + 8 \, x + 12}=\answer{-\frac{1}{2}}\]
\end{problem}}%}

\latexProblemContent{
\ifVerboseLocation This is Derivative Compute Question 0002. \\ \fi
\begin{problem}

Determine if the limit approaches a finite number, $\pm\infty$, or does not exist. (If the limit does not exist, write DNE)

\input{Limit-Compute-0002.HELP.tex}

\[\lim_{x\to{10}}\dfrac{x - 10}{x^{2} - 100}=\answer{\frac{1}{20}}\]
\end{problem}}%}

\latexProblemContent{
\ifVerboseLocation This is Derivative Compute Question 0002. \\ \fi
\begin{problem}

Determine if the limit approaches a finite number, $\pm\infty$, or does not exist. (If the limit does not exist, write DNE)

\input{Limit-Compute-0002.HELP.tex}

\[\lim_{x\to{-9}}\dfrac{3 \, x + 27}{x^{2} + 7 \, x - 18}=\answer{-\frac{3}{11}}\]
\end{problem}}%}

\latexProblemContent{
\ifVerboseLocation This is Derivative Compute Question 0002. \\ \fi
\begin{problem}

Determine if the limit approaches a finite number, $\pm\infty$, or does not exist. (If the limit does not exist, write DNE)

\input{Limit-Compute-0002.HELP.tex}

\[\lim_{x\to{-4}}\dfrac{2 \, x + 8}{x^{2} - x - 20}=\answer{-\frac{2}{9}}\]
\end{problem}}%}

\latexProblemContent{
\ifVerboseLocation This is Derivative Compute Question 0002. \\ \fi
\begin{problem}

Determine if the limit approaches a finite number, $\pm\infty$, or does not exist. (If the limit does not exist, write DNE)

\input{Limit-Compute-0002.HELP.tex}

\[\lim_{x\to{5}}\dfrac{3 \, x - 15}{x^{2} - 9 \, x + 20}=\answer{3}\]
\end{problem}}%}

\latexProblemContent{
\ifVerboseLocation This is Derivative Compute Question 0002. \\ \fi
\begin{problem}

Determine if the limit approaches a finite number, $\pm\infty$, or does not exist. (If the limit does not exist, write DNE)

\input{Limit-Compute-0002.HELP.tex}

\[\lim_{x\to{-4}}\dfrac{3 \, x + 12}{x^{2} - 2 \, x - 24}=\answer{-\frac{3}{10}}\]
\end{problem}}%}

\latexProblemContent{
\ifVerboseLocation This is Derivative Compute Question 0002. \\ \fi
\begin{problem}

Determine if the limit approaches a finite number, $\pm\infty$, or does not exist. (If the limit does not exist, write DNE)

\input{Limit-Compute-0002.HELP.tex}

\[\lim_{x\to{8}}\dfrac{2 \, x - 16}{x^{2} - 10 \, x + 16}=\answer{\frac{1}{3}}\]
\end{problem}}%}

\latexProblemContent{
\ifVerboseLocation This is Derivative Compute Question 0002. \\ \fi
\begin{problem}

Determine if the limit approaches a finite number, $\pm\infty$, or does not exist. (If the limit does not exist, write DNE)

\input{Limit-Compute-0002.HELP.tex}

\[\lim_{x\to{9}}\dfrac{x - 9}{x^{2} - 10 \, x + 9}=\answer{\frac{1}{8}}\]
\end{problem}}%}

\latexProblemContent{
\ifVerboseLocation This is Derivative Compute Question 0002. \\ \fi
\begin{problem}

Determine if the limit approaches a finite number, $\pm\infty$, or does not exist. (If the limit does not exist, write DNE)

\input{Limit-Compute-0002.HELP.tex}

\[\lim_{x\to{6}}\dfrac{3 \, x - 18}{x^{2} - 36}=\answer{\frac{1}{4}}\]
\end{problem}}%}

\latexProblemContent{
\ifVerboseLocation This is Derivative Compute Question 0002. \\ \fi
\begin{problem}

Determine if the limit approaches a finite number, $\pm\infty$, or does not exist. (If the limit does not exist, write DNE)

\input{Limit-Compute-0002.HELP.tex}

\[\lim_{x\to{7}}\dfrac{3 \, x - 21}{x^{2} - 9 \, x + 14}=\answer{\frac{3}{5}}\]
\end{problem}}%}

\latexProblemContent{
\ifVerboseLocation This is Derivative Compute Question 0002. \\ \fi
\begin{problem}

Determine if the limit approaches a finite number, $\pm\infty$, or does not exist. (If the limit does not exist, write DNE)

\input{Limit-Compute-0002.HELP.tex}

\[\lim_{x\to{-2}}\dfrac{2 \, x + 4}{x^{2} - 7 \, x - 18}=\answer{-\frac{2}{11}}\]
\end{problem}}%}

\latexProblemContent{
\ifVerboseLocation This is Derivative Compute Question 0002. \\ \fi
\begin{problem}

Determine if the limit approaches a finite number, $\pm\infty$, or does not exist. (If the limit does not exist, write DNE)

\input{Limit-Compute-0002.HELP.tex}

\[\lim_{x\to{1}}\dfrac{x - 1}{x^{2} + 7 \, x - 8}=\answer{\frac{1}{9}}\]
\end{problem}}%}

\latexProblemContent{
\ifVerboseLocation This is Derivative Compute Question 0002. \\ \fi
\begin{problem}

Determine if the limit approaches a finite number, $\pm\infty$, or does not exist. (If the limit does not exist, write DNE)

\input{Limit-Compute-0002.HELP.tex}

\[\lim_{x\to{-7}}\dfrac{x + 7}{x^{2} - 2 \, x - 63}=\answer{-\frac{1}{16}}\]
\end{problem}}%}

\latexProblemContent{
\ifVerboseLocation This is Derivative Compute Question 0002. \\ \fi
\begin{problem}

Determine if the limit approaches a finite number, $\pm\infty$, or does not exist. (If the limit does not exist, write DNE)

\input{Limit-Compute-0002.HELP.tex}

\[\lim_{x\to{-10}}\dfrac{x + 10}{x^{2} + 16 \, x + 60}=\answer{-\frac{1}{4}}\]
\end{problem}}%}

\latexProblemContent{
\ifVerboseLocation This is Derivative Compute Question 0002. \\ \fi
\begin{problem}

Determine if the limit approaches a finite number, $\pm\infty$, or does not exist. (If the limit does not exist, write DNE)

\input{Limit-Compute-0002.HELP.tex}

\[\lim_{x\to{-9}}\dfrac{x + 9}{x^{2} + 5 \, x - 36}=\answer{-\frac{1}{13}}\]
\end{problem}}%}

\latexProblemContent{
\ifVerboseLocation This is Derivative Compute Question 0002. \\ \fi
\begin{problem}

Determine if the limit approaches a finite number, $\pm\infty$, or does not exist. (If the limit does not exist, write DNE)

\input{Limit-Compute-0002.HELP.tex}

\[\lim_{x\to{-2}}\dfrac{x + 2}{x^{2} - 8 \, x - 20}=\answer{-\frac{1}{12}}\]
\end{problem}}%}

\latexProblemContent{
\ifVerboseLocation This is Derivative Compute Question 0002. \\ \fi
\begin{problem}

Determine if the limit approaches a finite number, $\pm\infty$, or does not exist. (If the limit does not exist, write DNE)

\input{Limit-Compute-0002.HELP.tex}

\[\lim_{x\to{-9}}\dfrac{3 \, x + 27}{x^{2} + 3 \, x - 54}=\answer{-\frac{1}{5}}\]
\end{problem}}%}

\latexProblemContent{
\ifVerboseLocation This is Derivative Compute Question 0002. \\ \fi
\begin{problem}

Determine if the limit approaches a finite number, $\pm\infty$, or does not exist. (If the limit does not exist, write DNE)

\input{Limit-Compute-0002.HELP.tex}

\[\lim_{x\to{-8}}\dfrac{x + 8}{x^{2} + 3 \, x - 40}=\answer{-\frac{1}{13}}\]
\end{problem}}%}

\latexProblemContent{
\ifVerboseLocation This is Derivative Compute Question 0002. \\ \fi
\begin{problem}

Determine if the limit approaches a finite number, $\pm\infty$, or does not exist. (If the limit does not exist, write DNE)

\input{Limit-Compute-0002.HELP.tex}

\[\lim_{x\to{6}}\dfrac{2 \, x - 12}{x^{2} - 13 \, x + 42}=\answer{-2}\]
\end{problem}}%}

\latexProblemContent{
\ifVerboseLocation This is Derivative Compute Question 0002. \\ \fi
\begin{problem}

Determine if the limit approaches a finite number, $\pm\infty$, or does not exist. (If the limit does not exist, write DNE)

\input{Limit-Compute-0002.HELP.tex}

\[\lim_{x\to{7}}\dfrac{3 \, x - 21}{x^{2} + 3 \, x - 70}=\answer{\frac{3}{17}}\]
\end{problem}}%}

\latexProblemContent{
\ifVerboseLocation This is Derivative Compute Question 0002. \\ \fi
\begin{problem}

Determine if the limit approaches a finite number, $\pm\infty$, or does not exist. (If the limit does not exist, write DNE)

\input{Limit-Compute-0002.HELP.tex}

\[\lim_{x\to{-1}}\dfrac{2 \, x + 2}{x^{2} - 2 \, x - 3}=\answer{-\frac{1}{2}}\]
\end{problem}}%}

\latexProblemContent{
\ifVerboseLocation This is Derivative Compute Question 0002. \\ \fi
\begin{problem}

Determine if the limit approaches a finite number, $\pm\infty$, or does not exist. (If the limit does not exist, write DNE)

\input{Limit-Compute-0002.HELP.tex}

\[\lim_{x\to{1}}\dfrac{3 \, x - 3}{x^{2} + 9 \, x - 10}=\answer{\frac{3}{11}}\]
\end{problem}}%}

\latexProblemContent{
\ifVerboseLocation This is Derivative Compute Question 0002. \\ \fi
\begin{problem}

Determine if the limit approaches a finite number, $\pm\infty$, or does not exist. (If the limit does not exist, write DNE)

\input{Limit-Compute-0002.HELP.tex}

\[\lim_{x\to{5}}\dfrac{2 \, x - 10}{x^{2} - 8 \, x + 15}=\answer{1}\]
\end{problem}}%}

\latexProblemContent{
\ifVerboseLocation This is Derivative Compute Question 0002. \\ \fi
\begin{problem}

Determine if the limit approaches a finite number, $\pm\infty$, or does not exist. (If the limit does not exist, write DNE)

\input{Limit-Compute-0002.HELP.tex}

\[\lim_{x\to{-8}}\dfrac{x + 8}{x^{2} - 64}=\answer{-\frac{1}{16}}\]
\end{problem}}%}

\latexProblemContent{
\ifVerboseLocation This is Derivative Compute Question 0002. \\ \fi
\begin{problem}

Determine if the limit approaches a finite number, $\pm\infty$, or does not exist. (If the limit does not exist, write DNE)

\input{Limit-Compute-0002.HELP.tex}

\[\lim_{x\to{-9}}\dfrac{3 \, x + 27}{x^{2} + 8 \, x - 9}=\answer{-\frac{3}{10}}\]
\end{problem}}%}

\latexProblemContent{
\ifVerboseLocation This is Derivative Compute Question 0002. \\ \fi
\begin{problem}

Determine if the limit approaches a finite number, $\pm\infty$, or does not exist. (If the limit does not exist, write DNE)

\input{Limit-Compute-0002.HELP.tex}

\[\lim_{x\to{9}}\dfrac{3 \, x - 27}{x^{2} - 6 \, x - 27}=\answer{\frac{1}{4}}\]
\end{problem}}%}

\latexProblemContent{
\ifVerboseLocation This is Derivative Compute Question 0002. \\ \fi
\begin{problem}

Determine if the limit approaches a finite number, $\pm\infty$, or does not exist. (If the limit does not exist, write DNE)

\input{Limit-Compute-0002.HELP.tex}

\[\lim_{x\to{-8}}\dfrac{2 \, x + 16}{x^{2} + 5 \, x - 24}=\answer{-\frac{2}{11}}\]
\end{problem}}%}

\latexProblemContent{
\ifVerboseLocation This is Derivative Compute Question 0002. \\ \fi
\begin{problem}

Determine if the limit approaches a finite number, $\pm\infty$, or does not exist. (If the limit does not exist, write DNE)

\input{Limit-Compute-0002.HELP.tex}

\[\lim_{x\to{7}}\dfrac{x - 7}{x^{2} - 15 \, x + 56}=\answer{-1}\]
\end{problem}}%}

\latexProblemContent{
\ifVerboseLocation This is Derivative Compute Question 0002. \\ \fi
\begin{problem}

Determine if the limit approaches a finite number, $\pm\infty$, or does not exist. (If the limit does not exist, write DNE)

\input{Limit-Compute-0002.HELP.tex}

\[\lim_{x\to{8}}\dfrac{3 \, x - 24}{x^{2} - 9 \, x + 8}=\answer{\frac{3}{7}}\]
\end{problem}}%}

\latexProblemContent{
\ifVerboseLocation This is Derivative Compute Question 0002. \\ \fi
\begin{problem}

Determine if the limit approaches a finite number, $\pm\infty$, or does not exist. (If the limit does not exist, write DNE)

\input{Limit-Compute-0002.HELP.tex}

\[\lim_{x\to{-3}}\dfrac{3 \, x + 9}{x^{2} - 7 \, x - 30}=\answer{-\frac{3}{13}}\]
\end{problem}}%}

\latexProblemContent{
\ifVerboseLocation This is Derivative Compute Question 0002. \\ \fi
\begin{problem}

Determine if the limit approaches a finite number, $\pm\infty$, or does not exist. (If the limit does not exist, write DNE)

\input{Limit-Compute-0002.HELP.tex}

\[\lim_{x\to{2}}\dfrac{2 \, x - 4}{x^{2} - 8 \, x + 12}=\answer{-\frac{1}{2}}\]
\end{problem}}%}

\latexProblemContent{
\ifVerboseLocation This is Derivative Compute Question 0002. \\ \fi
\begin{problem}

Determine if the limit approaches a finite number, $\pm\infty$, or does not exist. (If the limit does not exist, write DNE)

\input{Limit-Compute-0002.HELP.tex}

\[\lim_{x\to{9}}\dfrac{2 \, x - 18}{x^{2} - 10 \, x + 9}=\answer{\frac{1}{4}}\]
\end{problem}}%}

\latexProblemContent{
\ifVerboseLocation This is Derivative Compute Question 0002. \\ \fi
\begin{problem}

Determine if the limit approaches a finite number, $\pm\infty$, or does not exist. (If the limit does not exist, write DNE)

\input{Limit-Compute-0002.HELP.tex}

\[\lim_{x\to{3}}\dfrac{2 \, x - 6}{x^{2} + 2 \, x - 15}=\answer{\frac{1}{4}}\]
\end{problem}}%}

\latexProblemContent{
\ifVerboseLocation This is Derivative Compute Question 0002. \\ \fi
\begin{problem}

Determine if the limit approaches a finite number, $\pm\infty$, or does not exist. (If the limit does not exist, write DNE)

\input{Limit-Compute-0002.HELP.tex}

\[\lim_{x\to{1}}\dfrac{x - 1}{x^{2} + 4 \, x - 5}=\answer{\frac{1}{6}}\]
\end{problem}}%}

\latexProblemContent{
\ifVerboseLocation This is Derivative Compute Question 0002. \\ \fi
\begin{problem}

Determine if the limit approaches a finite number, $\pm\infty$, or does not exist. (If the limit does not exist, write DNE)

\input{Limit-Compute-0002.HELP.tex}

\[\lim_{x\to{-9}}\dfrac{2 \, x + 18}{x^{2} - x - 90}=\answer{-\frac{2}{19}}\]
\end{problem}}%}

\latexProblemContent{
\ifVerboseLocation This is Derivative Compute Question 0002. \\ \fi
\begin{problem}

Determine if the limit approaches a finite number, $\pm\infty$, or does not exist. (If the limit does not exist, write DNE)

\input{Limit-Compute-0002.HELP.tex}

\[\lim_{x\to{9}}\dfrac{x - 9}{x^{2} - 11 \, x + 18}=\answer{\frac{1}{7}}\]
\end{problem}}%}

\latexProblemContent{
\ifVerboseLocation This is Derivative Compute Question 0002. \\ \fi
\begin{problem}

Determine if the limit approaches a finite number, $\pm\infty$, or does not exist. (If the limit does not exist, write DNE)

\input{Limit-Compute-0002.HELP.tex}

\[\lim_{x\to{-5}}\dfrac{x + 5}{x^{2} + 7 \, x + 10}=\answer{-\frac{1}{3}}\]
\end{problem}}%}

\latexProblemContent{
\ifVerboseLocation This is Derivative Compute Question 0002. \\ \fi
\begin{problem}

Determine if the limit approaches a finite number, $\pm\infty$, or does not exist. (If the limit does not exist, write DNE)

\input{Limit-Compute-0002.HELP.tex}

\[\lim_{x\to{7}}\dfrac{3 \, x - 21}{x^{2} - 49}=\answer{\frac{3}{14}}\]
\end{problem}}%}

\latexProblemContent{
\ifVerboseLocation This is Derivative Compute Question 0002. \\ \fi
\begin{problem}

Determine if the limit approaches a finite number, $\pm\infty$, or does not exist. (If the limit does not exist, write DNE)

\input{Limit-Compute-0002.HELP.tex}

\[\lim_{x\to{7}}\dfrac{2 \, x - 14}{x^{2} - 10 \, x + 21}=\answer{\frac{1}{2}}\]
\end{problem}}%}

\latexProblemContent{
\ifVerboseLocation This is Derivative Compute Question 0002. \\ \fi
\begin{problem}

Determine if the limit approaches a finite number, $\pm\infty$, or does not exist. (If the limit does not exist, write DNE)

\input{Limit-Compute-0002.HELP.tex}

\[\lim_{x\to{1}}\dfrac{2 \, x - 2}{x^{2} + 8 \, x - 9}=\answer{\frac{1}{5}}\]
\end{problem}}%}

\latexProblemContent{
\ifVerboseLocation This is Derivative Compute Question 0002. \\ \fi
\begin{problem}

Determine if the limit approaches a finite number, $\pm\infty$, or does not exist. (If the limit does not exist, write DNE)

\input{Limit-Compute-0002.HELP.tex}

\[\lim_{x\to{-8}}\dfrac{x + 8}{x^{2} + 9 \, x + 8}=\answer{-\frac{1}{7}}\]
\end{problem}}%}

\latexProblemContent{
\ifVerboseLocation This is Derivative Compute Question 0002. \\ \fi
\begin{problem}

Determine if the limit approaches a finite number, $\pm\infty$, or does not exist. (If the limit does not exist, write DNE)

\input{Limit-Compute-0002.HELP.tex}

\[\lim_{x\to{-1}}\dfrac{2 \, x + 2}{x^{2} + 10 \, x + 9}=\answer{\frac{1}{4}}\]
\end{problem}}%}

\latexProblemContent{
\ifVerboseLocation This is Derivative Compute Question 0002. \\ \fi
\begin{problem}

Determine if the limit approaches a finite number, $\pm\infty$, or does not exist. (If the limit does not exist, write DNE)

\input{Limit-Compute-0002.HELP.tex}

\[\lim_{x\to{8}}\dfrac{2 \, x - 16}{x^{2} - 64}=\answer{\frac{1}{8}}\]
\end{problem}}%}

\latexProblemContent{
\ifVerboseLocation This is Derivative Compute Question 0002. \\ \fi
\begin{problem}

Determine if the limit approaches a finite number, $\pm\infty$, or does not exist. (If the limit does not exist, write DNE)

\input{Limit-Compute-0002.HELP.tex}

\[\lim_{x\to{8}}\dfrac{x - 8}{x^{2} + 2 \, x - 80}=\answer{\frac{1}{18}}\]
\end{problem}}%}

\latexProblemContent{
\ifVerboseLocation This is Derivative Compute Question 0002. \\ \fi
\begin{problem}

Determine if the limit approaches a finite number, $\pm\infty$, or does not exist. (If the limit does not exist, write DNE)

\input{Limit-Compute-0002.HELP.tex}

\[\lim_{x\to{7}}\dfrac{x - 7}{x^{2} - 6 \, x - 7}=\answer{\frac{1}{8}}\]
\end{problem}}%}

\latexProblemContent{
\ifVerboseLocation This is Derivative Compute Question 0002. \\ \fi
\begin{problem}

Determine if the limit approaches a finite number, $\pm\infty$, or does not exist. (If the limit does not exist, write DNE)

\input{Limit-Compute-0002.HELP.tex}

\[\lim_{x\to{4}}\dfrac{3 \, x - 12}{x^{2} - 13 \, x + 36}=\answer{-\frac{3}{5}}\]
\end{problem}}%}

\latexProblemContent{
\ifVerboseLocation This is Derivative Compute Question 0002. \\ \fi
\begin{problem}

Determine if the limit approaches a finite number, $\pm\infty$, or does not exist. (If the limit does not exist, write DNE)

\input{Limit-Compute-0002.HELP.tex}

\[\lim_{x\to{1}}\dfrac{3 \, x - 3}{x^{2} - 6 \, x + 5}=\answer{-\frac{3}{4}}\]
\end{problem}}%}

\latexProblemContent{
\ifVerboseLocation This is Derivative Compute Question 0002. \\ \fi
\begin{problem}

Determine if the limit approaches a finite number, $\pm\infty$, or does not exist. (If the limit does not exist, write DNE)

\input{Limit-Compute-0002.HELP.tex}

\[\lim_{x\to{9}}\dfrac{3 \, x - 27}{x^{2} - 7 \, x - 18}=\answer{\frac{3}{11}}\]
\end{problem}}%}

\latexProblemContent{
\ifVerboseLocation This is Derivative Compute Question 0002. \\ \fi
\begin{problem}

Determine if the limit approaches a finite number, $\pm\infty$, or does not exist. (If the limit does not exist, write DNE)

\input{Limit-Compute-0002.HELP.tex}

\[\lim_{x\to{10}}\dfrac{x - 10}{x^{2} - 8 \, x - 20}=\answer{\frac{1}{12}}\]
\end{problem}}%}

\latexProblemContent{
\ifVerboseLocation This is Derivative Compute Question 0002. \\ \fi
\begin{problem}

Determine if the limit approaches a finite number, $\pm\infty$, or does not exist. (If the limit does not exist, write DNE)

\input{Limit-Compute-0002.HELP.tex}

\[\lim_{x\to{10}}\dfrac{3 \, x - 30}{x^{2} - 13 \, x + 30}=\answer{\frac{3}{7}}\]
\end{problem}}%}

\latexProblemContent{
\ifVerboseLocation This is Derivative Compute Question 0002. \\ \fi
\begin{problem}

Determine if the limit approaches a finite number, $\pm\infty$, or does not exist. (If the limit does not exist, write DNE)

\input{Limit-Compute-0002.HELP.tex}

\[\lim_{x\to{4}}\dfrac{2 \, x - 8}{x^{2} - x - 12}=\answer{\frac{2}{7}}\]
\end{problem}}%}

\latexProblemContent{
\ifVerboseLocation This is Derivative Compute Question 0002. \\ \fi
\begin{problem}

Determine if the limit approaches a finite number, $\pm\infty$, or does not exist. (If the limit does not exist, write DNE)

\input{Limit-Compute-0002.HELP.tex}

\[\lim_{x\to{8}}\dfrac{3 \, x - 24}{x^{2} - 14 \, x + 48}=\answer{\frac{3}{2}}\]
\end{problem}}%}

\latexProblemContent{
\ifVerboseLocation This is Derivative Compute Question 0002. \\ \fi
\begin{problem}

Determine if the limit approaches a finite number, $\pm\infty$, or does not exist. (If the limit does not exist, write DNE)

\input{Limit-Compute-0002.HELP.tex}

\[\lim_{x\to{4}}\dfrac{2 \, x - 8}{x^{2} - 9 \, x + 20}=\answer{-2}\]
\end{problem}}%}

\latexProblemContent{
\ifVerboseLocation This is Derivative Compute Question 0002. \\ \fi
\begin{problem}

Determine if the limit approaches a finite number, $\pm\infty$, or does not exist. (If the limit does not exist, write DNE)

\input{Limit-Compute-0002.HELP.tex}

\[\lim_{x\to{7}}\dfrac{2 \, x - 14}{x^{2} - 3 \, x - 28}=\answer{\frac{2}{11}}\]
\end{problem}}%}

\latexProblemContent{
\ifVerboseLocation This is Derivative Compute Question 0002. \\ \fi
\begin{problem}

Determine if the limit approaches a finite number, $\pm\infty$, or does not exist. (If the limit does not exist, write DNE)

\input{Limit-Compute-0002.HELP.tex}

\[\lim_{x\to{3}}\dfrac{2 \, x - 6}{x^{2} + 7 \, x - 30}=\answer{\frac{2}{13}}\]
\end{problem}}%}

\latexProblemContent{
\ifVerboseLocation This is Derivative Compute Question 0002. \\ \fi
\begin{problem}

Determine if the limit approaches a finite number, $\pm\infty$, or does not exist. (If the limit does not exist, write DNE)

\input{Limit-Compute-0002.HELP.tex}

\[\lim_{x\to{8}}\dfrac{2 \, x - 16}{x^{2} - 2 \, x - 48}=\answer{\frac{1}{7}}\]
\end{problem}}%}

\latexProblemContent{
\ifVerboseLocation This is Derivative Compute Question 0002. \\ \fi
\begin{problem}

Determine if the limit approaches a finite number, $\pm\infty$, or does not exist. (If the limit does not exist, write DNE)

\input{Limit-Compute-0002.HELP.tex}

\[\lim_{x\to{3}}\dfrac{x - 3}{x^{2} - 9 \, x + 18}=\answer{-\frac{1}{3}}\]
\end{problem}}%}

\latexProblemContent{
\ifVerboseLocation This is Derivative Compute Question 0002. \\ \fi
\begin{problem}

Determine if the limit approaches a finite number, $\pm\infty$, or does not exist. (If the limit does not exist, write DNE)

\input{Limit-Compute-0002.HELP.tex}

\[\lim_{x\to{-2}}\dfrac{3 \, x + 6}{x^{2} + x - 2}=\answer{-1}\]
\end{problem}}%}

\latexProblemContent{
\ifVerboseLocation This is Derivative Compute Question 0002. \\ \fi
\begin{problem}

Determine if the limit approaches a finite number, $\pm\infty$, or does not exist. (If the limit does not exist, write DNE)

\input{Limit-Compute-0002.HELP.tex}

\[\lim_{x\to{8}}\dfrac{2 \, x - 16}{x^{2} + x - 72}=\answer{\frac{2}{17}}\]
\end{problem}}%}

\latexProblemContent{
\ifVerboseLocation This is Derivative Compute Question 0002. \\ \fi
\begin{problem}

Determine if the limit approaches a finite number, $\pm\infty$, or does not exist. (If the limit does not exist, write DNE)

\input{Limit-Compute-0002.HELP.tex}

\[\lim_{x\to{-7}}\dfrac{x + 7}{x^{2} + 15 \, x + 56}=\answer{1}\]
\end{problem}}%}

\latexProblemContent{
\ifVerboseLocation This is Derivative Compute Question 0002. \\ \fi
\begin{problem}

Determine if the limit approaches a finite number, $\pm\infty$, or does not exist. (If the limit does not exist, write DNE)

\input{Limit-Compute-0002.HELP.tex}

\[\lim_{x\to{-9}}\dfrac{2 \, x + 18}{x^{2} + 4 \, x - 45}=\answer{-\frac{1}{7}}\]
\end{problem}}%}

\latexProblemContent{
\ifVerboseLocation This is Derivative Compute Question 0002. \\ \fi
\begin{problem}

Determine if the limit approaches a finite number, $\pm\infty$, or does not exist. (If the limit does not exist, write DNE)

\input{Limit-Compute-0002.HELP.tex}

\[\lim_{x\to{2}}\dfrac{x - 2}{x^{2} + 6 \, x - 16}=\answer{\frac{1}{10}}\]
\end{problem}}%}

\latexProblemContent{
\ifVerboseLocation This is Derivative Compute Question 0002. \\ \fi
\begin{problem}

Determine if the limit approaches a finite number, $\pm\infty$, or does not exist. (If the limit does not exist, write DNE)

\input{Limit-Compute-0002.HELP.tex}

\[\lim_{x\to{10}}\dfrac{3 \, x - 30}{x^{2} - 5 \, x - 50}=\answer{\frac{1}{5}}\]
\end{problem}}%}

\latexProblemContent{
\ifVerboseLocation This is Derivative Compute Question 0002. \\ \fi
\begin{problem}

Determine if the limit approaches a finite number, $\pm\infty$, or does not exist. (If the limit does not exist, write DNE)

\input{Limit-Compute-0002.HELP.tex}

\[\lim_{x\to{1}}\dfrac{3 \, x - 3}{x^{2} - 1}=\answer{\frac{3}{2}}\]
\end{problem}}%}

\latexProblemContent{
\ifVerboseLocation This is Derivative Compute Question 0002. \\ \fi
\begin{problem}

Determine if the limit approaches a finite number, $\pm\infty$, or does not exist. (If the limit does not exist, write DNE)

\input{Limit-Compute-0002.HELP.tex}

\[\lim_{x\to{-9}}\dfrac{x + 9}{x^{2} + 10 \, x + 9}=\answer{-\frac{1}{8}}\]
\end{problem}}%}

\latexProblemContent{
\ifVerboseLocation This is Derivative Compute Question 0002. \\ \fi
\begin{problem}

Determine if the limit approaches a finite number, $\pm\infty$, or does not exist. (If the limit does not exist, write DNE)

\input{Limit-Compute-0002.HELP.tex}

\[\lim_{x\to{-9}}\dfrac{3 \, x + 27}{x^{2} - 81}=\answer{-\frac{1}{6}}\]
\end{problem}}%}

\latexProblemContent{
\ifVerboseLocation This is Derivative Compute Question 0002. \\ \fi
\begin{problem}

Determine if the limit approaches a finite number, $\pm\infty$, or does not exist. (If the limit does not exist, write DNE)

\input{Limit-Compute-0002.HELP.tex}

\[\lim_{x\to{7}}\dfrac{2 \, x - 14}{x^{2} - 15 \, x + 56}=\answer{-2}\]
\end{problem}}%}

\latexProblemContent{
\ifVerboseLocation This is Derivative Compute Question 0002. \\ \fi
\begin{problem}

Determine if the limit approaches a finite number, $\pm\infty$, or does not exist. (If the limit does not exist, write DNE)

\input{Limit-Compute-0002.HELP.tex}

\[\lim_{x\to{-6}}\dfrac{2 \, x + 12}{x^{2} + 11 \, x + 30}=\answer{-2}\]
\end{problem}}%}

\latexProblemContent{
\ifVerboseLocation This is Derivative Compute Question 0002. \\ \fi
\begin{problem}

Determine if the limit approaches a finite number, $\pm\infty$, or does not exist. (If the limit does not exist, write DNE)

\input{Limit-Compute-0002.HELP.tex}

\[\lim_{x\to{-8}}\dfrac{x + 8}{x^{2} + 15 \, x + 56}=\answer{-1}\]
\end{problem}}%}

\latexProblemContent{
\ifVerboseLocation This is Derivative Compute Question 0002. \\ \fi
\begin{problem}

Determine if the limit approaches a finite number, $\pm\infty$, or does not exist. (If the limit does not exist, write DNE)

\input{Limit-Compute-0002.HELP.tex}

\[\lim_{x\to{-2}}\dfrac{2 \, x + 4}{x^{2} + 7 \, x + 10}=\answer{\frac{2}{3}}\]
\end{problem}}%}

\latexProblemContent{
\ifVerboseLocation This is Derivative Compute Question 0002. \\ \fi
\begin{problem}

Determine if the limit approaches a finite number, $\pm\infty$, or does not exist. (If the limit does not exist, write DNE)

\input{Limit-Compute-0002.HELP.tex}

\[\lim_{x\to{3}}\dfrac{3 \, x - 9}{x^{2} - 4 \, x + 3}=\answer{\frac{3}{2}}\]
\end{problem}}%}

\latexProblemContent{
\ifVerboseLocation This is Derivative Compute Question 0002. \\ \fi
\begin{problem}

Determine if the limit approaches a finite number, $\pm\infty$, or does not exist. (If the limit does not exist, write DNE)

\input{Limit-Compute-0002.HELP.tex}

\[\lim_{x\to{9}}\dfrac{3 \, x - 27}{x^{2} - 13 \, x + 36}=\answer{\frac{3}{5}}\]
\end{problem}}%}

\latexProblemContent{
\ifVerboseLocation This is Derivative Compute Question 0002. \\ \fi
\begin{problem}

Determine if the limit approaches a finite number, $\pm\infty$, or does not exist. (If the limit does not exist, write DNE)

\input{Limit-Compute-0002.HELP.tex}

\[\lim_{x\to{7}}\dfrac{2 \, x - 14}{x^{2} + 2 \, x - 63}=\answer{\frac{1}{8}}\]
\end{problem}}%}

\latexProblemContent{
\ifVerboseLocation This is Derivative Compute Question 0002. \\ \fi
\begin{problem}

Determine if the limit approaches a finite number, $\pm\infty$, or does not exist. (If the limit does not exist, write DNE)

\input{Limit-Compute-0002.HELP.tex}

\[\lim_{x\to{-3}}\dfrac{x + 3}{x^{2} - 5 \, x - 24}=\answer{-\frac{1}{11}}\]
\end{problem}}%}

\latexProblemContent{
\ifVerboseLocation This is Derivative Compute Question 0002. \\ \fi
\begin{problem}

Determine if the limit approaches a finite number, $\pm\infty$, or does not exist. (If the limit does not exist, write DNE)

\input{Limit-Compute-0002.HELP.tex}

\[\lim_{x\to{-7}}\dfrac{x + 7}{x^{2} + 11 \, x + 28}=\answer{-\frac{1}{3}}\]
\end{problem}}%}

\latexProblemContent{
\ifVerboseLocation This is Derivative Compute Question 0002. \\ \fi
\begin{problem}

Determine if the limit approaches a finite number, $\pm\infty$, or does not exist. (If the limit does not exist, write DNE)

\input{Limit-Compute-0002.HELP.tex}

\[\lim_{x\to{5}}\dfrac{2 \, x - 10}{x^{2} - 7 \, x + 10}=\answer{\frac{2}{3}}\]
\end{problem}}%}

\latexProblemContent{
\ifVerboseLocation This is Derivative Compute Question 0002. \\ \fi
\begin{problem}

Determine if the limit approaches a finite number, $\pm\infty$, or does not exist. (If the limit does not exist, write DNE)

\input{Limit-Compute-0002.HELP.tex}

\[\lim_{x\to{-7}}\dfrac{2 \, x + 14}{x^{2} + 8 \, x + 7}=\answer{-\frac{1}{3}}\]
\end{problem}}%}

\latexProblemContent{
\ifVerboseLocation This is Derivative Compute Question 0002. \\ \fi
\begin{problem}

Determine if the limit approaches a finite number, $\pm\infty$, or does not exist. (If the limit does not exist, write DNE)

\input{Limit-Compute-0002.HELP.tex}

\[\lim_{x\to{-7}}\dfrac{x + 7}{x^{2} + 5 \, x - 14}=\answer{-\frac{1}{9}}\]
\end{problem}}%}

\latexProblemContent{
\ifVerboseLocation This is Derivative Compute Question 0002. \\ \fi
\begin{problem}

Determine if the limit approaches a finite number, $\pm\infty$, or does not exist. (If the limit does not exist, write DNE)

\input{Limit-Compute-0002.HELP.tex}

\[\lim_{x\to{-9}}\dfrac{x + 9}{x^{2} + 8 \, x - 9}=\answer{-\frac{1}{10}}\]
\end{problem}}%}

\latexProblemContent{
\ifVerboseLocation This is Derivative Compute Question 0002. \\ \fi
\begin{problem}

Determine if the limit approaches a finite number, $\pm\infty$, or does not exist. (If the limit does not exist, write DNE)

\input{Limit-Compute-0002.HELP.tex}

\[\lim_{x\to{-2}}\dfrac{3 \, x + 6}{x^{2} - 5 \, x - 14}=\answer{-\frac{1}{3}}\]
\end{problem}}%}

\latexProblemContent{
\ifVerboseLocation This is Derivative Compute Question 0002. \\ \fi
\begin{problem}

Determine if the limit approaches a finite number, $\pm\infty$, or does not exist. (If the limit does not exist, write DNE)

\input{Limit-Compute-0002.HELP.tex}

\[\lim_{x\to{3}}\dfrac{2 \, x - 6}{x^{2} + 5 \, x - 24}=\answer{\frac{2}{11}}\]
\end{problem}}%}

\latexProblemContent{
\ifVerboseLocation This is Derivative Compute Question 0002. \\ \fi
\begin{problem}

Determine if the limit approaches a finite number, $\pm\infty$, or does not exist. (If the limit does not exist, write DNE)

\input{Limit-Compute-0002.HELP.tex}

\[\lim_{x\to{-5}}\dfrac{3 \, x + 15}{x^{2} - 5 \, x - 50}=\answer{-\frac{1}{5}}\]
\end{problem}}%}

\latexProblemContent{
\ifVerboseLocation This is Derivative Compute Question 0002. \\ \fi
\begin{problem}

Determine if the limit approaches a finite number, $\pm\infty$, or does not exist. (If the limit does not exist, write DNE)

\input{Limit-Compute-0002.HELP.tex}

\[\lim_{x\to{8}}\dfrac{2 \, x - 16}{x^{2} - 12 \, x + 32}=\answer{\frac{1}{2}}\]
\end{problem}}%}

\latexProblemContent{
\ifVerboseLocation This is Derivative Compute Question 0002. \\ \fi
\begin{problem}

Determine if the limit approaches a finite number, $\pm\infty$, or does not exist. (If the limit does not exist, write DNE)

\input{Limit-Compute-0002.HELP.tex}

\[\lim_{x\to{3}}\dfrac{3 \, x - 9}{x^{2} + 5 \, x - 24}=\answer{\frac{3}{11}}\]
\end{problem}}%}

\latexProblemContent{
\ifVerboseLocation This is Derivative Compute Question 0002. \\ \fi
\begin{problem}

Determine if the limit approaches a finite number, $\pm\infty$, or does not exist. (If the limit does not exist, write DNE)

\input{Limit-Compute-0002.HELP.tex}

\[\lim_{x\to{-2}}\dfrac{3 \, x + 6}{x^{2} + 8 \, x + 12}=\answer{\frac{3}{4}}\]
\end{problem}}%}

\latexProblemContent{
\ifVerboseLocation This is Derivative Compute Question 0002. \\ \fi
\begin{problem}

Determine if the limit approaches a finite number, $\pm\infty$, or does not exist. (If the limit does not exist, write DNE)

\input{Limit-Compute-0002.HELP.tex}

\[\lim_{x\to{3}}\dfrac{3 \, x - 9}{x^{2} - 5 \, x + 6}=\answer{3}\]
\end{problem}}%}

\latexProblemContent{
\ifVerboseLocation This is Derivative Compute Question 0002. \\ \fi
\begin{problem}

Determine if the limit approaches a finite number, $\pm\infty$, or does not exist. (If the limit does not exist, write DNE)

\input{Limit-Compute-0002.HELP.tex}

\[\lim_{x\to{3}}\dfrac{2 \, x - 6}{x^{2} - x - 6}=\answer{\frac{2}{5}}\]
\end{problem}}%}

\latexProblemContent{
\ifVerboseLocation This is Derivative Compute Question 0002. \\ \fi
\begin{problem}

Determine if the limit approaches a finite number, $\pm\infty$, or does not exist. (If the limit does not exist, write DNE)

\input{Limit-Compute-0002.HELP.tex}

\[\lim_{x\to{-6}}\dfrac{x + 6}{x^{2} + 16 \, x + 60}=\answer{\frac{1}{4}}\]
\end{problem}}%}

\latexProblemContent{
\ifVerboseLocation This is Derivative Compute Question 0002. \\ \fi
\begin{problem}

Determine if the limit approaches a finite number, $\pm\infty$, or does not exist. (If the limit does not exist, write DNE)

\input{Limit-Compute-0002.HELP.tex}

\[\lim_{x\to{5}}\dfrac{x - 5}{x^{2} + 3 \, x - 40}=\answer{\frac{1}{13}}\]
\end{problem}}%}

\latexProblemContent{
\ifVerboseLocation This is Derivative Compute Question 0002. \\ \fi
\begin{problem}

Determine if the limit approaches a finite number, $\pm\infty$, or does not exist. (If the limit does not exist, write DNE)

\input{Limit-Compute-0002.HELP.tex}

\[\lim_{x\to{8}}\dfrac{x - 8}{x^{2} - 12 \, x + 32}=\answer{\frac{1}{4}}\]
\end{problem}}%}

\latexProblemContent{
\ifVerboseLocation This is Derivative Compute Question 0002. \\ \fi
\begin{problem}

Determine if the limit approaches a finite number, $\pm\infty$, or does not exist. (If the limit does not exist, write DNE)

\input{Limit-Compute-0002.HELP.tex}

\[\lim_{x\to{2}}\dfrac{x - 2}{x^{2} - 11 \, x + 18}=\answer{-\frac{1}{7}}\]
\end{problem}}%}

\latexProblemContent{
\ifVerboseLocation This is Derivative Compute Question 0002. \\ \fi
\begin{problem}

Determine if the limit approaches a finite number, $\pm\infty$, or does not exist. (If the limit does not exist, write DNE)

\input{Limit-Compute-0002.HELP.tex}

\[\lim_{x\to{1}}\dfrac{2 \, x - 2}{x^{2} + 2 \, x - 3}=\answer{\frac{1}{2}}\]
\end{problem}}%}

\latexProblemContent{
\ifVerboseLocation This is Derivative Compute Question 0002. \\ \fi
\begin{problem}

Determine if the limit approaches a finite number, $\pm\infty$, or does not exist. (If the limit does not exist, write DNE)

\input{Limit-Compute-0002.HELP.tex}

\[\lim_{x\to{9}}\dfrac{x - 9}{x^{2} - 3 \, x - 54}=\answer{\frac{1}{15}}\]
\end{problem}}%}

\latexProblemContent{
\ifVerboseLocation This is Derivative Compute Question 0002. \\ \fi
\begin{problem}

Determine if the limit approaches a finite number, $\pm\infty$, or does not exist. (If the limit does not exist, write DNE)

\input{Limit-Compute-0002.HELP.tex}

\[\lim_{x\to{4}}\dfrac{x - 4}{x^{2} - 10 \, x + 24}=\answer{-\frac{1}{2}}\]
\end{problem}}%}

\latexProblemContent{
\ifVerboseLocation This is Derivative Compute Question 0002. \\ \fi
\begin{problem}

Determine if the limit approaches a finite number, $\pm\infty$, or does not exist. (If the limit does not exist, write DNE)

\input{Limit-Compute-0002.HELP.tex}

\[\lim_{x\to{1}}\dfrac{3 \, x - 3}{x^{2} - 4 \, x + 3}=\answer{-\frac{3}{2}}\]
\end{problem}}%}

\latexProblemContent{
\ifVerboseLocation This is Derivative Compute Question 0002. \\ \fi
\begin{problem}

Determine if the limit approaches a finite number, $\pm\infty$, or does not exist. (If the limit does not exist, write DNE)

\input{Limit-Compute-0002.HELP.tex}

\[\lim_{x\to{9}}\dfrac{3 \, x - 27}{x^{2} - 5 \, x - 36}=\answer{\frac{3}{13}}\]
\end{problem}}%}

\latexProblemContent{
\ifVerboseLocation This is Derivative Compute Question 0002. \\ \fi
\begin{problem}

Determine if the limit approaches a finite number, $\pm\infty$, or does not exist. (If the limit does not exist, write DNE)

\input{Limit-Compute-0002.HELP.tex}

\[\lim_{x\to{-10}}\dfrac{3 \, x + 30}{x^{2} + 15 \, x + 50}=\answer{-\frac{3}{5}}\]
\end{problem}}%}

\latexProblemContent{
\ifVerboseLocation This is Derivative Compute Question 0002. \\ \fi
\begin{problem}

Determine if the limit approaches a finite number, $\pm\infty$, or does not exist. (If the limit does not exist, write DNE)

\input{Limit-Compute-0002.HELP.tex}

\[\lim_{x\to{1}}\dfrac{2 \, x - 2}{x^{2} + 5 \, x - 6}=\answer{\frac{2}{7}}\]
\end{problem}}%}

\latexProblemContent{
\ifVerboseLocation This is Derivative Compute Question 0002. \\ \fi
\begin{problem}

Determine if the limit approaches a finite number, $\pm\infty$, or does not exist. (If the limit does not exist, write DNE)

\input{Limit-Compute-0002.HELP.tex}

\[\lim_{x\to{-6}}\dfrac{2 \, x + 12}{x^{2} + 15 \, x + 54}=\answer{\frac{2}{3}}\]
\end{problem}}%}

\latexProblemContent{
\ifVerboseLocation This is Derivative Compute Question 0002. \\ \fi
\begin{problem}

Determine if the limit approaches a finite number, $\pm\infty$, or does not exist. (If the limit does not exist, write DNE)

\input{Limit-Compute-0002.HELP.tex}

\[\lim_{x\to{1}}\dfrac{2 \, x - 2}{x^{2} - 1}=\answer{1}\]
\end{problem}}%}

\latexProblemContent{
\ifVerboseLocation This is Derivative Compute Question 0002. \\ \fi
\begin{problem}

Determine if the limit approaches a finite number, $\pm\infty$, or does not exist. (If the limit does not exist, write DNE)

\input{Limit-Compute-0002.HELP.tex}

\[\lim_{x\to{-2}}\dfrac{x + 2}{x^{2} + 9 \, x + 14}=\answer{\frac{1}{5}}\]
\end{problem}}%}

\latexProblemContent{
\ifVerboseLocation This is Derivative Compute Question 0002. \\ \fi
\begin{problem}

Determine if the limit approaches a finite number, $\pm\infty$, or does not exist. (If the limit does not exist, write DNE)

\input{Limit-Compute-0002.HELP.tex}

\[\lim_{x\to{-8}}\dfrac{x + 8}{x^{2} + 14 \, x + 48}=\answer{-\frac{1}{2}}\]
\end{problem}}%}

\latexProblemContent{
\ifVerboseLocation This is Derivative Compute Question 0002. \\ \fi
\begin{problem}

Determine if the limit approaches a finite number, $\pm\infty$, or does not exist. (If the limit does not exist, write DNE)

\input{Limit-Compute-0002.HELP.tex}

\[\lim_{x\to{-2}}\dfrac{x + 2}{x^{2} + 7 \, x + 10}=\answer{\frac{1}{3}}\]
\end{problem}}%}

\latexProblemContent{
\ifVerboseLocation This is Derivative Compute Question 0002. \\ \fi
\begin{problem}

Determine if the limit approaches a finite number, $\pm\infty$, or does not exist. (If the limit does not exist, write DNE)

\input{Limit-Compute-0002.HELP.tex}

\[\lim_{x\to{-5}}\dfrac{2 \, x + 10}{x^{2} - 4 \, x - 45}=\answer{-\frac{1}{7}}\]
\end{problem}}%}

\latexProblemContent{
\ifVerboseLocation This is Derivative Compute Question 0002. \\ \fi
\begin{problem}

Determine if the limit approaches a finite number, $\pm\infty$, or does not exist. (If the limit does not exist, write DNE)

\input{Limit-Compute-0002.HELP.tex}

\[\lim_{x\to{7}}\dfrac{3 \, x - 21}{x^{2} + x - 56}=\answer{\frac{1}{5}}\]
\end{problem}}%}

\latexProblemContent{
\ifVerboseLocation This is Derivative Compute Question 0002. \\ \fi
\begin{problem}

Determine if the limit approaches a finite number, $\pm\infty$, or does not exist. (If the limit does not exist, write DNE)

\input{Limit-Compute-0002.HELP.tex}

\[\lim_{x\to{-8}}\dfrac{x + 8}{x^{2} + 7 \, x - 8}=\answer{-\frac{1}{9}}\]
\end{problem}}%}

\latexProblemContent{
\ifVerboseLocation This is Derivative Compute Question 0002. \\ \fi
\begin{problem}

Determine if the limit approaches a finite number, $\pm\infty$, or does not exist. (If the limit does not exist, write DNE)

\input{Limit-Compute-0002.HELP.tex}

\[\lim_{x\to{-9}}\dfrac{3 \, x + 27}{x^{2} + 16 \, x + 63}=\answer{-\frac{3}{2}}\]
\end{problem}}%}

\latexProblemContent{
\ifVerboseLocation This is Derivative Compute Question 0002. \\ \fi
\begin{problem}

Determine if the limit approaches a finite number, $\pm\infty$, or does not exist. (If the limit does not exist, write DNE)

\input{Limit-Compute-0002.HELP.tex}

\[\lim_{x\to{3}}\dfrac{2 \, x - 6}{x^{2} - 11 \, x + 24}=\answer{-\frac{2}{5}}\]
\end{problem}}%}

\latexProblemContent{
\ifVerboseLocation This is Derivative Compute Question 0002. \\ \fi
\begin{problem}

Determine if the limit approaches a finite number, $\pm\infty$, or does not exist. (If the limit does not exist, write DNE)

\input{Limit-Compute-0002.HELP.tex}

\[\lim_{x\to{-4}}\dfrac{3 \, x + 12}{x^{2} + 3 \, x - 4}=\answer{-\frac{3}{5}}\]
\end{problem}}%}

\latexProblemContent{
\ifVerboseLocation This is Derivative Compute Question 0002. \\ \fi
\begin{problem}

Determine if the limit approaches a finite number, $\pm\infty$, or does not exist. (If the limit does not exist, write DNE)

\input{Limit-Compute-0002.HELP.tex}

\[\lim_{x\to{-4}}\dfrac{3 \, x + 12}{x^{2} - 16}=\answer{-\frac{3}{8}}\]
\end{problem}}%}

\latexProblemContent{
\ifVerboseLocation This is Derivative Compute Question 0002. \\ \fi
\begin{problem}

Determine if the limit approaches a finite number, $\pm\infty$, or does not exist. (If the limit does not exist, write DNE)

\input{Limit-Compute-0002.HELP.tex}

\[\lim_{x\to{-4}}\dfrac{2 \, x + 8}{x^{2} + 5 \, x + 4}=\answer{-\frac{2}{3}}\]
\end{problem}}%}

\latexProblemContent{
\ifVerboseLocation This is Derivative Compute Question 0002. \\ \fi
\begin{problem}

Determine if the limit approaches a finite number, $\pm\infty$, or does not exist. (If the limit does not exist, write DNE)

\input{Limit-Compute-0002.HELP.tex}

\[\lim_{x\to{9}}\dfrac{3 \, x - 27}{x^{2} - 4 \, x - 45}=\answer{\frac{3}{14}}\]
\end{problem}}%}

\latexProblemContent{
\ifVerboseLocation This is Derivative Compute Question 0002. \\ \fi
\begin{problem}

Determine if the limit approaches a finite number, $\pm\infty$, or does not exist. (If the limit does not exist, write DNE)

\input{Limit-Compute-0002.HELP.tex}

\[\lim_{x\to{6}}\dfrac{2 \, x - 12}{x^{2} - 15 \, x + 54}=\answer{-\frac{2}{3}}\]
\end{problem}}%}

\latexProblemContent{
\ifVerboseLocation This is Derivative Compute Question 0002. \\ \fi
\begin{problem}

Determine if the limit approaches a finite number, $\pm\infty$, or does not exist. (If the limit does not exist, write DNE)

\input{Limit-Compute-0002.HELP.tex}

\[\lim_{x\to{-10}}\dfrac{2 \, x + 20}{x^{2} + 8 \, x - 20}=\answer{-\frac{1}{6}}\]
\end{problem}}%}

\latexProblemContent{
\ifVerboseLocation This is Derivative Compute Question 0002. \\ \fi
\begin{problem}

Determine if the limit approaches a finite number, $\pm\infty$, or does not exist. (If the limit does not exist, write DNE)

\input{Limit-Compute-0002.HELP.tex}

\[\lim_{x\to{10}}\dfrac{3 \, x - 30}{x^{2} - 15 \, x + 50}=\answer{\frac{3}{5}}\]
\end{problem}}%}

\latexProblemContent{
\ifVerboseLocation This is Derivative Compute Question 0002. \\ \fi
\begin{problem}

Determine if the limit approaches a finite number, $\pm\infty$, or does not exist. (If the limit does not exist, write DNE)

\input{Limit-Compute-0002.HELP.tex}

\[\lim_{x\to{-6}}\dfrac{2 \, x + 12}{x^{2} + 7 \, x + 6}=\answer{-\frac{2}{5}}\]
\end{problem}}%}

\latexProblemContent{
\ifVerboseLocation This is Derivative Compute Question 0002. \\ \fi
\begin{problem}

Determine if the limit approaches a finite number, $\pm\infty$, or does not exist. (If the limit does not exist, write DNE)

\input{Limit-Compute-0002.HELP.tex}

\[\lim_{x\to{4}}\dfrac{2 \, x - 8}{x^{2} - 7 \, x + 12}=\answer{2}\]
\end{problem}}%}

\latexProblemContent{
\ifVerboseLocation This is Derivative Compute Question 0002. \\ \fi
\begin{problem}

Determine if the limit approaches a finite number, $\pm\infty$, or does not exist. (If the limit does not exist, write DNE)

\input{Limit-Compute-0002.HELP.tex}

\[\lim_{x\to{-4}}\dfrac{3 \, x + 12}{x^{2} + 9 \, x + 20}=\answer{3}\]
\end{problem}}%}

\latexProblemContent{
\ifVerboseLocation This is Derivative Compute Question 0002. \\ \fi
\begin{problem}

Determine if the limit approaches a finite number, $\pm\infty$, or does not exist. (If the limit does not exist, write DNE)

\input{Limit-Compute-0002.HELP.tex}

\[\lim_{x\to{2}}\dfrac{3 \, x - 6}{x^{2} - 6 \, x + 8}=\answer{-\frac{3}{2}}\]
\end{problem}}%}

\latexProblemContent{
\ifVerboseLocation This is Derivative Compute Question 0002. \\ \fi
\begin{problem}

Determine if the limit approaches a finite number, $\pm\infty$, or does not exist. (If the limit does not exist, write DNE)

\input{Limit-Compute-0002.HELP.tex}

\[\lim_{x\to{6}}\dfrac{2 \, x - 12}{x^{2} - 8 \, x + 12}=\answer{\frac{1}{2}}\]
\end{problem}}%}

\latexProblemContent{
\ifVerboseLocation This is Derivative Compute Question 0002. \\ \fi
\begin{problem}

Determine if the limit approaches a finite number, $\pm\infty$, or does not exist. (If the limit does not exist, write DNE)

\input{Limit-Compute-0002.HELP.tex}

\[\lim_{x\to{-7}}\dfrac{x + 7}{x^{2} + 17 \, x + 70}=\answer{\frac{1}{3}}\]
\end{problem}}%}

\latexProblemContent{
\ifVerboseLocation This is Derivative Compute Question 0002. \\ \fi
\begin{problem}

Determine if the limit approaches a finite number, $\pm\infty$, or does not exist. (If the limit does not exist, write DNE)

\input{Limit-Compute-0002.HELP.tex}

\[\lim_{x\to{-4}}\dfrac{x + 4}{x^{2} + 7 \, x + 12}=\answer{-1}\]
\end{problem}}%}

\latexProblemContent{
\ifVerboseLocation This is Derivative Compute Question 0002. \\ \fi
\begin{problem}

Determine if the limit approaches a finite number, $\pm\infty$, or does not exist. (If the limit does not exist, write DNE)

\input{Limit-Compute-0002.HELP.tex}

\[\lim_{x\to{10}}\dfrac{2 \, x - 20}{x^{2} - 6 \, x - 40}=\answer{\frac{1}{7}}\]
\end{problem}}%}

\latexProblemContent{
\ifVerboseLocation This is Derivative Compute Question 0002. \\ \fi
\begin{problem}

Determine if the limit approaches a finite number, $\pm\infty$, or does not exist. (If the limit does not exist, write DNE)

\input{Limit-Compute-0002.HELP.tex}

\[\lim_{x\to{-1}}\dfrac{x + 1}{x^{2} - 8 \, x - 9}=\answer{-\frac{1}{10}}\]
\end{problem}}%}

\latexProblemContent{
\ifVerboseLocation This is Derivative Compute Question 0002. \\ \fi
\begin{problem}

Determine if the limit approaches a finite number, $\pm\infty$, or does not exist. (If the limit does not exist, write DNE)

\input{Limit-Compute-0002.HELP.tex}

\[\lim_{x\to{-10}}\dfrac{2 \, x + 20}{x^{2} + 11 \, x + 10}=\answer{-\frac{2}{9}}\]
\end{problem}}%}

\latexProblemContent{
\ifVerboseLocation This is Derivative Compute Question 0002. \\ \fi
\begin{problem}

Determine if the limit approaches a finite number, $\pm\infty$, or does not exist. (If the limit does not exist, write DNE)

\input{Limit-Compute-0002.HELP.tex}

\[\lim_{x\to{7}}\dfrac{x - 7}{x^{2} - 4 \, x - 21}=\answer{\frac{1}{10}}\]
\end{problem}}%}

\latexProblemContent{
\ifVerboseLocation This is Derivative Compute Question 0002. \\ \fi
\begin{problem}

Determine if the limit approaches a finite number, $\pm\infty$, or does not exist. (If the limit does not exist, write DNE)

\input{Limit-Compute-0002.HELP.tex}

\[\lim_{x\to{9}}\dfrac{2 \, x - 18}{x^{2} - 16 \, x + 63}=\answer{1}\]
\end{problem}}%}

\latexProblemContent{
\ifVerboseLocation This is Derivative Compute Question 0002. \\ \fi
\begin{problem}

Determine if the limit approaches a finite number, $\pm\infty$, or does not exist. (If the limit does not exist, write DNE)

\input{Limit-Compute-0002.HELP.tex}

\[\lim_{x\to{1}}\dfrac{x - 1}{x^{2} - 11 \, x + 10}=\answer{-\frac{1}{9}}\]
\end{problem}}%}

\latexProblemContent{
\ifVerboseLocation This is Derivative Compute Question 0002. \\ \fi
\begin{problem}

Determine if the limit approaches a finite number, $\pm\infty$, or does not exist. (If the limit does not exist, write DNE)

\input{Limit-Compute-0002.HELP.tex}

\[\lim_{x\to{-7}}\dfrac{3 \, x + 21}{x^{2} - 49}=\answer{-\frac{3}{14}}\]
\end{problem}}%}

\latexProblemContent{
\ifVerboseLocation This is Derivative Compute Question 0002. \\ \fi
\begin{problem}

Determine if the limit approaches a finite number, $\pm\infty$, or does not exist. (If the limit does not exist, write DNE)

\input{Limit-Compute-0002.HELP.tex}

\[\lim_{x\to{-10}}\dfrac{2 \, x + 20}{x^{2} + 9 \, x - 10}=\answer{-\frac{2}{11}}\]
\end{problem}}%}

\latexProblemContent{
\ifVerboseLocation This is Derivative Compute Question 0002. \\ \fi
\begin{problem}

Determine if the limit approaches a finite number, $\pm\infty$, or does not exist. (If the limit does not exist, write DNE)

\input{Limit-Compute-0002.HELP.tex}

\[\lim_{x\to{-6}}\dfrac{x + 6}{x^{2} + 10 \, x + 24}=\answer{-\frac{1}{2}}\]
\end{problem}}%}

\latexProblemContent{
\ifVerboseLocation This is Derivative Compute Question 0002. \\ \fi
\begin{problem}

Determine if the limit approaches a finite number, $\pm\infty$, or does not exist. (If the limit does not exist, write DNE)

\input{Limit-Compute-0002.HELP.tex}

\[\lim_{x\to{-3}}\dfrac{x + 3}{x^{2} - x - 12}=\answer{-\frac{1}{7}}\]
\end{problem}}%}

\latexProblemContent{
\ifVerboseLocation This is Derivative Compute Question 0002. \\ \fi
\begin{problem}

Determine if the limit approaches a finite number, $\pm\infty$, or does not exist. (If the limit does not exist, write DNE)

\input{Limit-Compute-0002.HELP.tex}

\[\lim_{x\to{10}}\dfrac{2 \, x - 20}{x^{2} - 18 \, x + 80}=\answer{1}\]
\end{problem}}%}

\latexProblemContent{
\ifVerboseLocation This is Derivative Compute Question 0002. \\ \fi
\begin{problem}

Determine if the limit approaches a finite number, $\pm\infty$, or does not exist. (If the limit does not exist, write DNE)

\input{Limit-Compute-0002.HELP.tex}

\[\lim_{x\to{-10}}\dfrac{2 \, x + 20}{x^{2} + x - 90}=\answer{-\frac{2}{19}}\]
\end{problem}}%}

\latexProblemContent{
\ifVerboseLocation This is Derivative Compute Question 0002. \\ \fi
\begin{problem}

Determine if the limit approaches a finite number, $\pm\infty$, or does not exist. (If the limit does not exist, write DNE)

\input{Limit-Compute-0002.HELP.tex}

\[\lim_{x\to{7}}\dfrac{x - 7}{x^{2} + 2 \, x - 63}=\answer{\frac{1}{16}}\]
\end{problem}}%}

\latexProblemContent{
\ifVerboseLocation This is Derivative Compute Question 0002. \\ \fi
\begin{problem}

Determine if the limit approaches a finite number, $\pm\infty$, or does not exist. (If the limit does not exist, write DNE)

\input{Limit-Compute-0002.HELP.tex}

\[\lim_{x\to{3}}\dfrac{3 \, x - 9}{x^{2} - 13 \, x + 30}=\answer{-\frac{3}{7}}\]
\end{problem}}%}

\latexProblemContent{
\ifVerboseLocation This is Derivative Compute Question 0002. \\ \fi
\begin{problem}

Determine if the limit approaches a finite number, $\pm\infty$, or does not exist. (If the limit does not exist, write DNE)

\input{Limit-Compute-0002.HELP.tex}

\[\lim_{x\to{6}}\dfrac{2 \, x - 12}{x^{2} - 9 \, x + 18}=\answer{\frac{2}{3}}\]
\end{problem}}%}

\latexProblemContent{
\ifVerboseLocation This is Derivative Compute Question 0002. \\ \fi
\begin{problem}

Determine if the limit approaches a finite number, $\pm\infty$, or does not exist. (If the limit does not exist, write DNE)

\input{Limit-Compute-0002.HELP.tex}

\[\lim_{x\to{-1}}\dfrac{2 \, x + 2}{x^{2} + 4 \, x + 3}=\answer{1}\]
\end{problem}}%}

\latexProblemContent{
\ifVerboseLocation This is Derivative Compute Question 0002. \\ \fi
\begin{problem}

Determine if the limit approaches a finite number, $\pm\infty$, or does not exist. (If the limit does not exist, write DNE)

\input{Limit-Compute-0002.HELP.tex}

\[\lim_{x\to{-3}}\dfrac{3 \, x + 9}{x^{2} - 5 \, x - 24}=\answer{-\frac{3}{11}}\]
\end{problem}}%}

\latexProblemContent{
\ifVerboseLocation This is Derivative Compute Question 0002. \\ \fi
\begin{problem}

Determine if the limit approaches a finite number, $\pm\infty$, or does not exist. (If the limit does not exist, write DNE)

\input{Limit-Compute-0002.HELP.tex}

\[\lim_{x\to{7}}\dfrac{2 \, x - 14}{x^{2} - 8 \, x + 7}=\answer{\frac{1}{3}}\]
\end{problem}}%}

\latexProblemContent{
\ifVerboseLocation This is Derivative Compute Question 0002. \\ \fi
\begin{problem}

Determine if the limit approaches a finite number, $\pm\infty$, or does not exist. (If the limit does not exist, write DNE)

\input{Limit-Compute-0002.HELP.tex}

\[\lim_{x\to{3}}\dfrac{3 \, x - 9}{x^{2} - x - 6}=\answer{\frac{3}{5}}\]
\end{problem}}%}

\latexProblemContent{
\ifVerboseLocation This is Derivative Compute Question 0002. \\ \fi
\begin{problem}

Determine if the limit approaches a finite number, $\pm\infty$, or does not exist. (If the limit does not exist, write DNE)

\input{Limit-Compute-0002.HELP.tex}

\[\lim_{x\to{-8}}\dfrac{2 \, x + 16}{x^{2} + 12 \, x + 32}=\answer{-\frac{1}{2}}\]
\end{problem}}%}

\latexProblemContent{
\ifVerboseLocation This is Derivative Compute Question 0002. \\ \fi
\begin{problem}

Determine if the limit approaches a finite number, $\pm\infty$, or does not exist. (If the limit does not exist, write DNE)

\input{Limit-Compute-0002.HELP.tex}

\[\lim_{x\to{-1}}\dfrac{x + 1}{x^{2} + 11 \, x + 10}=\answer{\frac{1}{9}}\]
\end{problem}}%}

\latexProblemContent{
\ifVerboseLocation This is Derivative Compute Question 0002. \\ \fi
\begin{problem}

Determine if the limit approaches a finite number, $\pm\infty$, or does not exist. (If the limit does not exist, write DNE)

\input{Limit-Compute-0002.HELP.tex}

\[\lim_{x\to{-4}}\dfrac{2 \, x + 8}{x^{2} + 7 \, x + 12}=\answer{-2}\]
\end{problem}}%}

\latexProblemContent{
\ifVerboseLocation This is Derivative Compute Question 0002. \\ \fi
\begin{problem}

Determine if the limit approaches a finite number, $\pm\infty$, or does not exist. (If the limit does not exist, write DNE)

\input{Limit-Compute-0002.HELP.tex}

\[\lim_{x\to{4}}\dfrac{3 \, x - 12}{x^{2} - 11 \, x + 28}=\answer{-1}\]
\end{problem}}%}

\latexProblemContent{
\ifVerboseLocation This is Derivative Compute Question 0002. \\ \fi
\begin{problem}

Determine if the limit approaches a finite number, $\pm\infty$, or does not exist. (If the limit does not exist, write DNE)

\input{Limit-Compute-0002.HELP.tex}

\[\lim_{x\to{-1}}\dfrac{x + 1}{x^{2} - 3 \, x - 4}=\answer{-\frac{1}{5}}\]
\end{problem}}%}

\latexProblemContent{
\ifVerboseLocation This is Derivative Compute Question 0002. \\ \fi
\begin{problem}

Determine if the limit approaches a finite number, $\pm\infty$, or does not exist. (If the limit does not exist, write DNE)

\input{Limit-Compute-0002.HELP.tex}

\[\lim_{x\to{7}}\dfrac{3 \, x - 21}{x^{2} - 3 \, x - 28}=\answer{\frac{3}{11}}\]
\end{problem}}%}

\latexProblemContent{
\ifVerboseLocation This is Derivative Compute Question 0002. \\ \fi
\begin{problem}

Determine if the limit approaches a finite number, $\pm\infty$, or does not exist. (If the limit does not exist, write DNE)

\input{Limit-Compute-0002.HELP.tex}

\[\lim_{x\to{-10}}\dfrac{x + 10}{x^{2} + 9 \, x - 10}=\answer{-\frac{1}{11}}\]
\end{problem}}%}

\latexProblemContent{
\ifVerboseLocation This is Derivative Compute Question 0002. \\ \fi
\begin{problem}

Determine if the limit approaches a finite number, $\pm\infty$, or does not exist. (If the limit does not exist, write DNE)

\input{Limit-Compute-0002.HELP.tex}

\[\lim_{x\to{-10}}\dfrac{x + 10}{x^{2} + 15 \, x + 50}=\answer{-\frac{1}{5}}\]
\end{problem}}%}

\latexProblemContent{
\ifVerboseLocation This is Derivative Compute Question 0002. \\ \fi
\begin{problem}

Determine if the limit approaches a finite number, $\pm\infty$, or does not exist. (If the limit does not exist, write DNE)

\input{Limit-Compute-0002.HELP.tex}

\[\lim_{x\to{5}}\dfrac{x - 5}{x^{2} - 6 \, x + 5}=\answer{\frac{1}{4}}\]
\end{problem}}%}

\latexProblemContent{
\ifVerboseLocation This is Derivative Compute Question 0002. \\ \fi
\begin{problem}

Determine if the limit approaches a finite number, $\pm\infty$, or does not exist. (If the limit does not exist, write DNE)

\input{Limit-Compute-0002.HELP.tex}

\[\lim_{x\to{5}}\dfrac{2 \, x - 10}{x^{2} - 2 \, x - 15}=\answer{\frac{1}{4}}\]
\end{problem}}%}

\latexProblemContent{
\ifVerboseLocation This is Derivative Compute Question 0002. \\ \fi
\begin{problem}

Determine if the limit approaches a finite number, $\pm\infty$, or does not exist. (If the limit does not exist, write DNE)

\input{Limit-Compute-0002.HELP.tex}

\[\lim_{x\to{2}}\dfrac{2 \, x - 4}{x^{2} + 5 \, x - 14}=\answer{\frac{2}{9}}\]
\end{problem}}%}

\latexProblemContent{
\ifVerboseLocation This is Derivative Compute Question 0002. \\ \fi
\begin{problem}

Determine if the limit approaches a finite number, $\pm\infty$, or does not exist. (If the limit does not exist, write DNE)

\input{Limit-Compute-0002.HELP.tex}

\[\lim_{x\to{-6}}\dfrac{2 \, x + 12}{x^{2} - x - 42}=\answer{-\frac{2}{13}}\]
\end{problem}}%}

\latexProblemContent{
\ifVerboseLocation This is Derivative Compute Question 0002. \\ \fi
\begin{problem}

Determine if the limit approaches a finite number, $\pm\infty$, or does not exist. (If the limit does not exist, write DNE)

\input{Limit-Compute-0002.HELP.tex}

\[\lim_{x\to{4}}\dfrac{3 \, x - 12}{x^{2} - 3 \, x - 4}=\answer{\frac{3}{5}}\]
\end{problem}}%}

\latexProblemContent{
\ifVerboseLocation This is Derivative Compute Question 0002. \\ \fi
\begin{problem}

Determine if the limit approaches a finite number, $\pm\infty$, or does not exist. (If the limit does not exist, write DNE)

\input{Limit-Compute-0002.HELP.tex}

\[\lim_{x\to{10}}\dfrac{3 \, x - 30}{x^{2} - 8 \, x - 20}=\answer{\frac{1}{4}}\]
\end{problem}}%}

\latexProblemContent{
\ifVerboseLocation This is Derivative Compute Question 0002. \\ \fi
\begin{problem}

Determine if the limit approaches a finite number, $\pm\infty$, or does not exist. (If the limit does not exist, write DNE)

\input{Limit-Compute-0002.HELP.tex}

\[\lim_{x\to{2}}\dfrac{x - 2}{x^{2} - 6 \, x + 8}=\answer{-\frac{1}{2}}\]
\end{problem}}%}

\latexProblemContent{
\ifVerboseLocation This is Derivative Compute Question 0002. \\ \fi
\begin{problem}

Determine if the limit approaches a finite number, $\pm\infty$, or does not exist. (If the limit does not exist, write DNE)

\input{Limit-Compute-0002.HELP.tex}

\[\lim_{x\to{-7}}\dfrac{3 \, x + 21}{x^{2} + 15 \, x + 56}=\answer{3}\]
\end{problem}}%}

\latexProblemContent{
\ifVerboseLocation This is Derivative Compute Question 0002. \\ \fi
\begin{problem}

Determine if the limit approaches a finite number, $\pm\infty$, or does not exist. (If the limit does not exist, write DNE)

\input{Limit-Compute-0002.HELP.tex}

\[\lim_{x\to{9}}\dfrac{3 \, x - 27}{x^{2} - 10 \, x + 9}=\answer{\frac{3}{8}}\]
\end{problem}}%}

\latexProblemContent{
\ifVerboseLocation This is Derivative Compute Question 0002. \\ \fi
\begin{problem}

Determine if the limit approaches a finite number, $\pm\infty$, or does not exist. (If the limit does not exist, write DNE)

\input{Limit-Compute-0002.HELP.tex}

\[\lim_{x\to{-3}}\dfrac{2 \, x + 6}{x^{2} + 5 \, x + 6}=\answer{-2}\]
\end{problem}}%}

\latexProblemContent{
\ifVerboseLocation This is Derivative Compute Question 0002. \\ \fi
\begin{problem}

Determine if the limit approaches a finite number, $\pm\infty$, or does not exist. (If the limit does not exist, write DNE)

\input{Limit-Compute-0002.HELP.tex}

\[\lim_{x\to{10}}\dfrac{2 \, x - 20}{x^{2} - 12 \, x + 20}=\answer{\frac{1}{4}}\]
\end{problem}}%}

\latexProblemContent{
\ifVerboseLocation This is Derivative Compute Question 0002. \\ \fi
\begin{problem}

Determine if the limit approaches a finite number, $\pm\infty$, or does not exist. (If the limit does not exist, write DNE)

\input{Limit-Compute-0002.HELP.tex}

\[\lim_{x\to{9}}\dfrac{x - 9}{x^{2} - 13 \, x + 36}=\answer{\frac{1}{5}}\]
\end{problem}}%}

\latexProblemContent{
\ifVerboseLocation This is Derivative Compute Question 0002. \\ \fi
\begin{problem}

Determine if the limit approaches a finite number, $\pm\infty$, or does not exist. (If the limit does not exist, write DNE)

\input{Limit-Compute-0002.HELP.tex}

\[\lim_{x\to{1}}\dfrac{2 \, x - 2}{x^{2} + x - 2}=\answer{\frac{2}{3}}\]
\end{problem}}%}

\latexProblemContent{
\ifVerboseLocation This is Derivative Compute Question 0002. \\ \fi
\begin{problem}

Determine if the limit approaches a finite number, $\pm\infty$, or does not exist. (If the limit does not exist, write DNE)

\input{Limit-Compute-0002.HELP.tex}

\[\lim_{x\to{6}}\dfrac{3 \, x - 18}{x^{2} - 2 \, x - 24}=\answer{\frac{3}{10}}\]
\end{problem}}%}

\latexProblemContent{
\ifVerboseLocation This is Derivative Compute Question 0002. \\ \fi
\begin{problem}

Determine if the limit approaches a finite number, $\pm\infty$, or does not exist. (If the limit does not exist, write DNE)

\input{Limit-Compute-0002.HELP.tex}

\[\lim_{x\to{3}}\dfrac{3 \, x - 9}{x^{2} + x - 12}=\answer{\frac{3}{7}}\]
\end{problem}}%}

\latexProblemContent{
\ifVerboseLocation This is Derivative Compute Question 0002. \\ \fi
\begin{problem}

Determine if the limit approaches a finite number, $\pm\infty$, or does not exist. (If the limit does not exist, write DNE)

\input{Limit-Compute-0002.HELP.tex}

\[\lim_{x\to{-1}}\dfrac{2 \, x + 2}{x^{2} - 9 \, x - 10}=\answer{-\frac{2}{11}}\]
\end{problem}}%}

\latexProblemContent{
\ifVerboseLocation This is Derivative Compute Question 0002. \\ \fi
\begin{problem}

Determine if the limit approaches a finite number, $\pm\infty$, or does not exist. (If the limit does not exist, write DNE)

\input{Limit-Compute-0002.HELP.tex}

\[\lim_{x\to{-7}}\dfrac{2 \, x + 14}{x^{2} + 6 \, x - 7}=\answer{-\frac{1}{4}}\]
\end{problem}}%}

\latexProblemContent{
\ifVerboseLocation This is Derivative Compute Question 0002. \\ \fi
\begin{problem}

Determine if the limit approaches a finite number, $\pm\infty$, or does not exist. (If the limit does not exist, write DNE)

\input{Limit-Compute-0002.HELP.tex}

\[\lim_{x\to{-1}}\dfrac{2 \, x + 2}{x^{2} - 5 \, x - 6}=\answer{-\frac{2}{7}}\]
\end{problem}}%}

\latexProblemContent{
\ifVerboseLocation This is Derivative Compute Question 0002. \\ \fi
\begin{problem}

Determine if the limit approaches a finite number, $\pm\infty$, or does not exist. (If the limit does not exist, write DNE)

\input{Limit-Compute-0002.HELP.tex}

\[\lim_{x\to{1}}\dfrac{3 \, x - 3}{x^{2} + 3 \, x - 4}=\answer{\frac{3}{5}}\]
\end{problem}}%}

\latexProblemContent{
\ifVerboseLocation This is Derivative Compute Question 0002. \\ \fi
\begin{problem}

Determine if the limit approaches a finite number, $\pm\infty$, or does not exist. (If the limit does not exist, write DNE)

\input{Limit-Compute-0002.HELP.tex}

\[\lim_{x\to{3}}\dfrac{2 \, x - 6}{x^{2} - 9 \, x + 18}=\answer{-\frac{2}{3}}\]
\end{problem}}%}

\latexProblemContent{
\ifVerboseLocation This is Derivative Compute Question 0002. \\ \fi
\begin{problem}

Determine if the limit approaches a finite number, $\pm\infty$, or does not exist. (If the limit does not exist, write DNE)

\input{Limit-Compute-0002.HELP.tex}

\[\lim_{x\to{-1}}\dfrac{x + 1}{x^{2} + 5 \, x + 4}=\answer{\frac{1}{3}}\]
\end{problem}}%}

\latexProblemContent{
\ifVerboseLocation This is Derivative Compute Question 0002. \\ \fi
\begin{problem}

Determine if the limit approaches a finite number, $\pm\infty$, or does not exist. (If the limit does not exist, write DNE)

\input{Limit-Compute-0002.HELP.tex}

\[\lim_{x\to{-9}}\dfrac{2 \, x + 18}{x^{2} + 16 \, x + 63}=\answer{-1}\]
\end{problem}}%}

\latexProblemContent{
\ifVerboseLocation This is Derivative Compute Question 0002. \\ \fi
\begin{problem}

Determine if the limit approaches a finite number, $\pm\infty$, or does not exist. (If the limit does not exist, write DNE)

\input{Limit-Compute-0002.HELP.tex}

\[\lim_{x\to{3}}\dfrac{3 \, x - 9}{x^{2} + 7 \, x - 30}=\answer{\frac{3}{13}}\]
\end{problem}}%}

\latexProblemContent{
\ifVerboseLocation This is Derivative Compute Question 0002. \\ \fi
\begin{problem}

Determine if the limit approaches a finite number, $\pm\infty$, or does not exist. (If the limit does not exist, write DNE)

\input{Limit-Compute-0002.HELP.tex}

\[\lim_{x\to{-5}}\dfrac{3 \, x + 15}{x^{2} - 2 \, x - 35}=\answer{-\frac{1}{4}}\]
\end{problem}}%}

\latexProblemContent{
\ifVerboseLocation This is Derivative Compute Question 0002. \\ \fi
\begin{problem}

Determine if the limit approaches a finite number, $\pm\infty$, or does not exist. (If the limit does not exist, write DNE)

\input{Limit-Compute-0002.HELP.tex}

\[\lim_{x\to{-6}}\dfrac{2 \, x + 12}{x^{2} + 14 \, x + 48}=\answer{1}\]
\end{problem}}%}

\latexProblemContent{
\ifVerboseLocation This is Derivative Compute Question 0002. \\ \fi
\begin{problem}

Determine if the limit approaches a finite number, $\pm\infty$, or does not exist. (If the limit does not exist, write DNE)

\input{Limit-Compute-0002.HELP.tex}

\[\lim_{x\to{6}}\dfrac{3 \, x - 18}{x^{2} - 3 \, x - 18}=\answer{\frac{1}{3}}\]
\end{problem}}%}

\latexProblemContent{
\ifVerboseLocation This is Derivative Compute Question 0002. \\ \fi
\begin{problem}

Determine if the limit approaches a finite number, $\pm\infty$, or does not exist. (If the limit does not exist, write DNE)

\input{Limit-Compute-0002.HELP.tex}

\[\lim_{x\to{-3}}\dfrac{2 \, x + 6}{x^{2} - x - 12}=\answer{-\frac{2}{7}}\]
\end{problem}}%}

\latexProblemContent{
\ifVerboseLocation This is Derivative Compute Question 0002. \\ \fi
\begin{problem}

Determine if the limit approaches a finite number, $\pm\infty$, or does not exist. (If the limit does not exist, write DNE)

\input{Limit-Compute-0002.HELP.tex}

\[\lim_{x\to{4}}\dfrac{x - 4}{x^{2} + 4 \, x - 32}=\answer{\frac{1}{12}}\]
\end{problem}}%}

\latexProblemContent{
\ifVerboseLocation This is Derivative Compute Question 0002. \\ \fi
\begin{problem}

Determine if the limit approaches a finite number, $\pm\infty$, or does not exist. (If the limit does not exist, write DNE)

\input{Limit-Compute-0002.HELP.tex}

\[\lim_{x\to{-7}}\dfrac{x + 7}{x^{2} - x - 56}=\answer{-\frac{1}{15}}\]
\end{problem}}%}

\latexProblemContent{
\ifVerboseLocation This is Derivative Compute Question 0002. \\ \fi
\begin{problem}

Determine if the limit approaches a finite number, $\pm\infty$, or does not exist. (If the limit does not exist, write DNE)

\input{Limit-Compute-0002.HELP.tex}

\[\lim_{x\to{8}}\dfrac{x - 8}{x^{2} - 6 \, x - 16}=\answer{\frac{1}{10}}\]
\end{problem}}%}

\latexProblemContent{
\ifVerboseLocation This is Derivative Compute Question 0002. \\ \fi
\begin{problem}

Determine if the limit approaches a finite number, $\pm\infty$, or does not exist. (If the limit does not exist, write DNE)

\input{Limit-Compute-0002.HELP.tex}

\[\lim_{x\to{-7}}\dfrac{x + 7}{x^{2} + 12 \, x + 35}=\answer{-\frac{1}{2}}\]
\end{problem}}%}

\latexProblemContent{
\ifVerboseLocation This is Derivative Compute Question 0002. \\ \fi
\begin{problem}

Determine if the limit approaches a finite number, $\pm\infty$, or does not exist. (If the limit does not exist, write DNE)

\input{Limit-Compute-0002.HELP.tex}

\[\lim_{x\to{5}}\dfrac{x - 5}{x^{2} - 7 \, x + 10}=\answer{\frac{1}{3}}\]
\end{problem}}%}

\latexProblemContent{
\ifVerboseLocation This is Derivative Compute Question 0002. \\ \fi
\begin{problem}

Determine if the limit approaches a finite number, $\pm\infty$, or does not exist. (If the limit does not exist, write DNE)

\input{Limit-Compute-0002.HELP.tex}

\[\lim_{x\to{-9}}\dfrac{2 \, x + 18}{x^{2} + 11 \, x + 18}=\answer{-\frac{2}{7}}\]
\end{problem}}%}

\latexProblemContent{
\ifVerboseLocation This is Derivative Compute Question 0002. \\ \fi
\begin{problem}

Determine if the limit approaches a finite number, $\pm\infty$, or does not exist. (If the limit does not exist, write DNE)

\input{Limit-Compute-0002.HELP.tex}

\[\lim_{x\to{10}}\dfrac{3 \, x - 30}{x^{2} - 2 \, x - 80}=\answer{\frac{1}{6}}\]
\end{problem}}%}

\latexProblemContent{
\ifVerboseLocation This is Derivative Compute Question 0002. \\ \fi
\begin{problem}

Determine if the limit approaches a finite number, $\pm\infty$, or does not exist. (If the limit does not exist, write DNE)

\input{Limit-Compute-0002.HELP.tex}

\[\lim_{x\to{-2}}\dfrac{3 \, x + 6}{x^{2} - 4 \, x - 12}=\answer{-\frac{3}{8}}\]
\end{problem}}%}

\latexProblemContent{
\ifVerboseLocation This is Derivative Compute Question 0002. \\ \fi
\begin{problem}

Determine if the limit approaches a finite number, $\pm\infty$, or does not exist. (If the limit does not exist, write DNE)

\input{Limit-Compute-0002.HELP.tex}

\[\lim_{x\to{-8}}\dfrac{3 \, x + 24}{x^{2} + 3 \, x - 40}=\answer{-\frac{3}{13}}\]
\end{problem}}%}

\latexProblemContent{
\ifVerboseLocation This is Derivative Compute Question 0002. \\ \fi
\begin{problem}

Determine if the limit approaches a finite number, $\pm\infty$, or does not exist. (If the limit does not exist, write DNE)

\input{Limit-Compute-0002.HELP.tex}

\[\lim_{x\to{1}}\dfrac{x - 1}{x^{2} + 8 \, x - 9}=\answer{\frac{1}{10}}\]
\end{problem}}%}

\latexProblemContent{
\ifVerboseLocation This is Derivative Compute Question 0002. \\ \fi
\begin{problem}

Determine if the limit approaches a finite number, $\pm\infty$, or does not exist. (If the limit does not exist, write DNE)

\input{Limit-Compute-0002.HELP.tex}

\[\lim_{x\to{-4}}\dfrac{x + 4}{x^{2} - 6 \, x - 40}=\answer{-\frac{1}{14}}\]
\end{problem}}%}

\latexProblemContent{
\ifVerboseLocation This is Derivative Compute Question 0002. \\ \fi
\begin{problem}

Determine if the limit approaches a finite number, $\pm\infty$, or does not exist. (If the limit does not exist, write DNE)

\input{Limit-Compute-0002.HELP.tex}

\[\lim_{x\to{-2}}\dfrac{3 \, x + 6}{x^{2} - 3 \, x - 10}=\answer{-\frac{3}{7}}\]
\end{problem}}%}

\latexProblemContent{
\ifVerboseLocation This is Derivative Compute Question 0002. \\ \fi
\begin{problem}

Determine if the limit approaches a finite number, $\pm\infty$, or does not exist. (If the limit does not exist, write DNE)

\input{Limit-Compute-0002.HELP.tex}

\[\lim_{x\to{-2}}\dfrac{x + 2}{x^{2} - x - 6}=\answer{-\frac{1}{5}}\]
\end{problem}}%}

\latexProblemContent{
\ifVerboseLocation This is Derivative Compute Question 0002. \\ \fi
\begin{problem}

Determine if the limit approaches a finite number, $\pm\infty$, or does not exist. (If the limit does not exist, write DNE)

\input{Limit-Compute-0002.HELP.tex}

\[\lim_{x\to{-2}}\dfrac{2 \, x + 4}{x^{2} + 8 \, x + 12}=\answer{\frac{1}{2}}\]
\end{problem}}%}

\latexProblemContent{
\ifVerboseLocation This is Derivative Compute Question 0002. \\ \fi
\begin{problem}

Determine if the limit approaches a finite number, $\pm\infty$, or does not exist. (If the limit does not exist, write DNE)

\input{Limit-Compute-0002.HELP.tex}

\[\lim_{x\to{4}}\dfrac{x - 4}{x^{2} - 3 \, x - 4}=\answer{\frac{1}{5}}\]
\end{problem}}%}

\latexProblemContent{
\ifVerboseLocation This is Derivative Compute Question 0002. \\ \fi
\begin{problem}

Determine if the limit approaches a finite number, $\pm\infty$, or does not exist. (If the limit does not exist, write DNE)

\input{Limit-Compute-0002.HELP.tex}

\[\lim_{x\to{-6}}\dfrac{3 \, x + 18}{x^{2} + 15 \, x + 54}=\answer{1}\]
\end{problem}}%}

\latexProblemContent{
\ifVerboseLocation This is Derivative Compute Question 0002. \\ \fi
\begin{problem}

Determine if the limit approaches a finite number, $\pm\infty$, or does not exist. (If the limit does not exist, write DNE)

\input{Limit-Compute-0002.HELP.tex}

\[\lim_{x\to{-3}}\dfrac{3 \, x + 9}{x^{2} + 7 \, x + 12}=\answer{3}\]
\end{problem}}%}

\latexProblemContent{
\ifVerboseLocation This is Derivative Compute Question 0002. \\ \fi
\begin{problem}

Determine if the limit approaches a finite number, $\pm\infty$, or does not exist. (If the limit does not exist, write DNE)

\input{Limit-Compute-0002.HELP.tex}

\[\lim_{x\to{-7}}\dfrac{3 \, x + 21}{x^{2} + 2 \, x - 35}=\answer{-\frac{1}{4}}\]
\end{problem}}%}

\latexProblemContent{
\ifVerboseLocation This is Derivative Compute Question 0002. \\ \fi
\begin{problem}

Determine if the limit approaches a finite number, $\pm\infty$, or does not exist. (If the limit does not exist, write DNE)

\input{Limit-Compute-0002.HELP.tex}

\[\lim_{x\to{-5}}\dfrac{3 \, x + 15}{x^{2} + 12 \, x + 35}=\answer{\frac{3}{2}}\]
\end{problem}}%}

\latexProblemContent{
\ifVerboseLocation This is Derivative Compute Question 0002. \\ \fi
\begin{problem}

Determine if the limit approaches a finite number, $\pm\infty$, or does not exist. (If the limit does not exist, write DNE)

\input{Limit-Compute-0002.HELP.tex}

\[\lim_{x\to{-3}}\dfrac{3 \, x + 9}{x^{2} - x - 12}=\answer{-\frac{3}{7}}\]
\end{problem}}%}

\latexProblemContent{
\ifVerboseLocation This is Derivative Compute Question 0002. \\ \fi
\begin{problem}

Determine if the limit approaches a finite number, $\pm\infty$, or does not exist. (If the limit does not exist, write DNE)

\input{Limit-Compute-0002.HELP.tex}

\[\lim_{x\to{6}}\dfrac{3 \, x - 18}{x^{2} - 5 \, x - 6}=\answer{\frac{3}{7}}\]
\end{problem}}%}

\latexProblemContent{
\ifVerboseLocation This is Derivative Compute Question 0002. \\ \fi
\begin{problem}

Determine if the limit approaches a finite number, $\pm\infty$, or does not exist. (If the limit does not exist, write DNE)

\input{Limit-Compute-0002.HELP.tex}

\[\lim_{x\to{-5}}\dfrac{2 \, x + 10}{x^{2} + 4 \, x - 5}=\answer{-\frac{1}{3}}\]
\end{problem}}%}

\latexProblemContent{
\ifVerboseLocation This is Derivative Compute Question 0002. \\ \fi
\begin{problem}

Determine if the limit approaches a finite number, $\pm\infty$, or does not exist. (If the limit does not exist, write DNE)

\input{Limit-Compute-0002.HELP.tex}

\[\lim_{x\to{3}}\dfrac{x - 3}{x^{2} + 4 \, x - 21}=\answer{\frac{1}{10}}\]
\end{problem}}%}

\latexProblemContent{
\ifVerboseLocation This is Derivative Compute Question 0002. \\ \fi
\begin{problem}

Determine if the limit approaches a finite number, $\pm\infty$, or does not exist. (If the limit does not exist, write DNE)

\input{Limit-Compute-0002.HELP.tex}

\[\lim_{x\to{-3}}\dfrac{3 \, x + 9}{x^{2} + 2 \, x - 3}=\answer{-\frac{3}{4}}\]
\end{problem}}%}

\latexProblemContent{
\ifVerboseLocation This is Derivative Compute Question 0002. \\ \fi
\begin{problem}

Determine if the limit approaches a finite number, $\pm\infty$, or does not exist. (If the limit does not exist, write DNE)

\input{Limit-Compute-0002.HELP.tex}

\[\lim_{x\to{4}}\dfrac{3 \, x - 12}{x^{2} + 4 \, x - 32}=\answer{\frac{1}{4}}\]
\end{problem}}%}

\latexProblemContent{
\ifVerboseLocation This is Derivative Compute Question 0002. \\ \fi
\begin{problem}

Determine if the limit approaches a finite number, $\pm\infty$, or does not exist. (If the limit does not exist, write DNE)

\input{Limit-Compute-0002.HELP.tex}

\[\lim_{x\to{4}}\dfrac{2 \, x - 8}{x^{2} - 16}=\answer{\frac{1}{4}}\]
\end{problem}}%}

\latexProblemContent{
\ifVerboseLocation This is Derivative Compute Question 0002. \\ \fi
\begin{problem}

Determine if the limit approaches a finite number, $\pm\infty$, or does not exist. (If the limit does not exist, write DNE)

\input{Limit-Compute-0002.HELP.tex}

\[\lim_{x\to{9}}\dfrac{3 \, x - 27}{x^{2} + x - 90}=\answer{\frac{3}{19}}\]
\end{problem}}%}

\latexProblemContent{
\ifVerboseLocation This is Derivative Compute Question 0002. \\ \fi
\begin{problem}

Determine if the limit approaches a finite number, $\pm\infty$, or does not exist. (If the limit does not exist, write DNE)

\input{Limit-Compute-0002.HELP.tex}

\[\lim_{x\to{6}}\dfrac{2 \, x - 12}{x^{2} - 10 \, x + 24}=\answer{1}\]
\end{problem}}%}

\latexProblemContent{
\ifVerboseLocation This is Derivative Compute Question 0002. \\ \fi
\begin{problem}

Determine if the limit approaches a finite number, $\pm\infty$, or does not exist. (If the limit does not exist, write DNE)

\input{Limit-Compute-0002.HELP.tex}

\[\lim_{x\to{-10}}\dfrac{2 \, x + 20}{x^{2} + 16 \, x + 60}=\answer{-\frac{1}{2}}\]
\end{problem}}%}

\latexProblemContent{
\ifVerboseLocation This is Derivative Compute Question 0002. \\ \fi
\begin{problem}

Determine if the limit approaches a finite number, $\pm\infty$, or does not exist. (If the limit does not exist, write DNE)

\input{Limit-Compute-0002.HELP.tex}

\[\lim_{x\to{-1}}\dfrac{3 \, x + 3}{x^{2} + 9 \, x + 8}=\answer{\frac{3}{7}}\]
\end{problem}}%}

\latexProblemContent{
\ifVerboseLocation This is Derivative Compute Question 0002. \\ \fi
\begin{problem}

Determine if the limit approaches a finite number, $\pm\infty$, or does not exist. (If the limit does not exist, write DNE)

\input{Limit-Compute-0002.HELP.tex}

\[\lim_{x\to{-3}}\dfrac{3 \, x + 9}{x^{2} - 9}=\answer{-\frac{1}{2}}\]
\end{problem}}%}

\latexProblemContent{
\ifVerboseLocation This is Derivative Compute Question 0002. \\ \fi
\begin{problem}

Determine if the limit approaches a finite number, $\pm\infty$, or does not exist. (If the limit does not exist, write DNE)

\input{Limit-Compute-0002.HELP.tex}

\[\lim_{x\to{-6}}\dfrac{2 \, x + 12}{x^{2} - 2 \, x - 48}=\answer{-\frac{1}{7}}\]
\end{problem}}%}

\latexProblemContent{
\ifVerboseLocation This is Derivative Compute Question 0002. \\ \fi
\begin{problem}

Determine if the limit approaches a finite number, $\pm\infty$, or does not exist. (If the limit does not exist, write DNE)

\input{Limit-Compute-0002.HELP.tex}

\[\lim_{x\to{-3}}\dfrac{2 \, x + 6}{x^{2} - 4 \, x - 21}=\answer{-\frac{1}{5}}\]
\end{problem}}%}

\latexProblemContent{
\ifVerboseLocation This is Derivative Compute Question 0002. \\ \fi
\begin{problem}

Determine if the limit approaches a finite number, $\pm\infty$, or does not exist. (If the limit does not exist, write DNE)

\input{Limit-Compute-0002.HELP.tex}

\[\lim_{x\to{6}}\dfrac{3 \, x - 18}{x^{2} - 15 \, x + 54}=\answer{-1}\]
\end{problem}}%}

\latexProblemContent{
\ifVerboseLocation This is Derivative Compute Question 0002. \\ \fi
\begin{problem}

Determine if the limit approaches a finite number, $\pm\infty$, or does not exist. (If the limit does not exist, write DNE)

\input{Limit-Compute-0002.HELP.tex}

\[\lim_{x\to{-8}}\dfrac{x + 8}{x^{2} + 11 \, x + 24}=\answer{-\frac{1}{5}}\]
\end{problem}}%}

\latexProblemContent{
\ifVerboseLocation This is Derivative Compute Question 0002. \\ \fi
\begin{problem}

Determine if the limit approaches a finite number, $\pm\infty$, or does not exist. (If the limit does not exist, write DNE)

\input{Limit-Compute-0002.HELP.tex}

\[\lim_{x\to{-6}}\dfrac{x + 6}{x^{2} + 5 \, x - 6}=\answer{-\frac{1}{7}}\]
\end{problem}}%}

\latexProblemContent{
\ifVerboseLocation This is Derivative Compute Question 0002. \\ \fi
\begin{problem}

Determine if the limit approaches a finite number, $\pm\infty$, or does not exist. (If the limit does not exist, write DNE)

\input{Limit-Compute-0002.HELP.tex}

\[\lim_{x\to{4}}\dfrac{x - 4}{x^{2} - 6 \, x + 8}=\answer{\frac{1}{2}}\]
\end{problem}}%}

\latexProblemContent{
\ifVerboseLocation This is Derivative Compute Question 0002. \\ \fi
\begin{problem}

Determine if the limit approaches a finite number, $\pm\infty$, or does not exist. (If the limit does not exist, write DNE)

\input{Limit-Compute-0002.HELP.tex}

\[\lim_{x\to{-2}}\dfrac{3 \, x + 6}{x^{2} + 7 \, x + 10}=\answer{1}\]
\end{problem}}%}

\latexProblemContent{
\ifVerboseLocation This is Derivative Compute Question 0002. \\ \fi
\begin{problem}

Determine if the limit approaches a finite number, $\pm\infty$, or does not exist. (If the limit does not exist, write DNE)

\input{Limit-Compute-0002.HELP.tex}

\[\lim_{x\to{3}}\dfrac{2 \, x - 6}{x^{2} - 12 \, x + 27}=\answer{-\frac{1}{3}}\]
\end{problem}}%}

\latexProblemContent{
\ifVerboseLocation This is Derivative Compute Question 0002. \\ \fi
\begin{problem}

Determine if the limit approaches a finite number, $\pm\infty$, or does not exist. (If the limit does not exist, write DNE)

\input{Limit-Compute-0002.HELP.tex}

\[\lim_{x\to{2}}\dfrac{x - 2}{x^{2} - 9 \, x + 14}=\answer{-\frac{1}{5}}\]
\end{problem}}%}

\latexProblemContent{
\ifVerboseLocation This is Derivative Compute Question 0002. \\ \fi
\begin{problem}

Determine if the limit approaches a finite number, $\pm\infty$, or does not exist. (If the limit does not exist, write DNE)

\input{Limit-Compute-0002.HELP.tex}

\[\lim_{x\to{10}}\dfrac{3 \, x - 30}{x^{2} - 7 \, x - 30}=\answer{\frac{3}{13}}\]
\end{problem}}%}

\latexProblemContent{
\ifVerboseLocation This is Derivative Compute Question 0002. \\ \fi
\begin{problem}

Determine if the limit approaches a finite number, $\pm\infty$, or does not exist. (If the limit does not exist, write DNE)

\input{Limit-Compute-0002.HELP.tex}

\[\lim_{x\to{-6}}\dfrac{2 \, x + 12}{x^{2} - 4 \, x - 60}=\answer{-\frac{1}{8}}\]
\end{problem}}%}

\latexProblemContent{
\ifVerboseLocation This is Derivative Compute Question 0002. \\ \fi
\begin{problem}

Determine if the limit approaches a finite number, $\pm\infty$, or does not exist. (If the limit does not exist, write DNE)

\input{Limit-Compute-0002.HELP.tex}

\[\lim_{x\to{-6}}\dfrac{2 \, x + 12}{x^{2} - 36}=\answer{-\frac{1}{6}}\]
\end{problem}}%}

\latexProblemContent{
\ifVerboseLocation This is Derivative Compute Question 0002. \\ \fi
\begin{problem}

Determine if the limit approaches a finite number, $\pm\infty$, or does not exist. (If the limit does not exist, write DNE)

\input{Limit-Compute-0002.HELP.tex}

\[\lim_{x\to{-6}}\dfrac{3 \, x + 18}{x^{2} + 5 \, x - 6}=\answer{-\frac{3}{7}}\]
\end{problem}}%}

\latexProblemContent{
\ifVerboseLocation This is Derivative Compute Question 0002. \\ \fi
\begin{problem}

Determine if the limit approaches a finite number, $\pm\infty$, or does not exist. (If the limit does not exist, write DNE)

\input{Limit-Compute-0002.HELP.tex}

\[\lim_{x\to{3}}\dfrac{3 \, x - 9}{x^{2} - 2 \, x - 3}=\answer{\frac{3}{4}}\]
\end{problem}}%}

\latexProblemContent{
\ifVerboseLocation This is Derivative Compute Question 0002. \\ \fi
\begin{problem}

Determine if the limit approaches a finite number, $\pm\infty$, or does not exist. (If the limit does not exist, write DNE)

\input{Limit-Compute-0002.HELP.tex}

\[\lim_{x\to{-9}}\dfrac{x + 9}{x^{2} + 14 \, x + 45}=\answer{-\frac{1}{4}}\]
\end{problem}}%}

\latexProblemContent{
\ifVerboseLocation This is Derivative Compute Question 0002. \\ \fi
\begin{problem}

Determine if the limit approaches a finite number, $\pm\infty$, or does not exist. (If the limit does not exist, write DNE)

\input{Limit-Compute-0002.HELP.tex}

\[\lim_{x\to{-8}}\dfrac{x + 8}{x^{2} + 6 \, x - 16}=\answer{-\frac{1}{10}}\]
\end{problem}}%}

\latexProblemContent{
\ifVerboseLocation This is Derivative Compute Question 0002. \\ \fi
\begin{problem}

Determine if the limit approaches a finite number, $\pm\infty$, or does not exist. (If the limit does not exist, write DNE)

\input{Limit-Compute-0002.HELP.tex}

\[\lim_{x\to{-6}}\dfrac{3 \, x + 18}{x^{2} + 3 \, x - 18}=\answer{-\frac{1}{3}}\]
\end{problem}}%}

\latexProblemContent{
\ifVerboseLocation This is Derivative Compute Question 0002. \\ \fi
\begin{problem}

Determine if the limit approaches a finite number, $\pm\infty$, or does not exist. (If the limit does not exist, write DNE)

\input{Limit-Compute-0002.HELP.tex}

\[\lim_{x\to{-3}}\dfrac{3 \, x + 9}{x^{2} - 4 \, x - 21}=\answer{-\frac{3}{10}}\]
\end{problem}}%}

\latexProblemContent{
\ifVerboseLocation This is Derivative Compute Question 0002. \\ \fi
\begin{problem}

Determine if the limit approaches a finite number, $\pm\infty$, or does not exist. (If the limit does not exist, write DNE)

\input{Limit-Compute-0002.HELP.tex}

\[\lim_{x\to{-10}}\dfrac{x + 10}{x^{2} + 7 \, x - 30}=\answer{-\frac{1}{13}}\]
\end{problem}}%}

\latexProblemContent{
\ifVerboseLocation This is Derivative Compute Question 0002. \\ \fi
\begin{problem}

Determine if the limit approaches a finite number, $\pm\infty$, or does not exist. (If the limit does not exist, write DNE)

\input{Limit-Compute-0002.HELP.tex}

\[\lim_{x\to{10}}\dfrac{3 \, x - 30}{x^{2} - 17 \, x + 70}=\answer{1}\]
\end{problem}}%}

\latexProblemContent{
\ifVerboseLocation This is Derivative Compute Question 0002. \\ \fi
\begin{problem}

Determine if the limit approaches a finite number, $\pm\infty$, or does not exist. (If the limit does not exist, write DNE)

\input{Limit-Compute-0002.HELP.tex}

\[\lim_{x\to{8}}\dfrac{2 \, x - 16}{x^{2} - 17 \, x + 72}=\answer{-2}\]
\end{problem}}%}

\latexProblemContent{
\ifVerboseLocation This is Derivative Compute Question 0002. \\ \fi
\begin{problem}

Determine if the limit approaches a finite number, $\pm\infty$, or does not exist. (If the limit does not exist, write DNE)

\input{Limit-Compute-0002.HELP.tex}

\[\lim_{x\to{-6}}\dfrac{x + 6}{x^{2} - x - 42}=\answer{-\frac{1}{13}}\]
\end{problem}}%}

\latexProblemContent{
\ifVerboseLocation This is Derivative Compute Question 0002. \\ \fi
\begin{problem}

Determine if the limit approaches a finite number, $\pm\infty$, or does not exist. (If the limit does not exist, write DNE)

\input{Limit-Compute-0002.HELP.tex}

\[\lim_{x\to{-10}}\dfrac{2 \, x + 20}{x^{2} + 2 \, x - 80}=\answer{-\frac{1}{9}}\]
\end{problem}}%}

\latexProblemContent{
\ifVerboseLocation This is Derivative Compute Question 0002. \\ \fi
\begin{problem}

Determine if the limit approaches a finite number, $\pm\infty$, or does not exist. (If the limit does not exist, write DNE)

\input{Limit-Compute-0002.HELP.tex}

\[\lim_{x\to{-10}}\dfrac{2 \, x + 20}{x^{2} + 12 \, x + 20}=\answer{-\frac{1}{4}}\]
\end{problem}}%}

\latexProblemContent{
\ifVerboseLocation This is Derivative Compute Question 0002. \\ \fi
\begin{problem}

Determine if the limit approaches a finite number, $\pm\infty$, or does not exist. (If the limit does not exist, write DNE)

\input{Limit-Compute-0002.HELP.tex}

\[\lim_{x\to{10}}\dfrac{x - 10}{x^{2} - 2 \, x - 80}=\answer{\frac{1}{18}}\]
\end{problem}}%}

\latexProblemContent{
\ifVerboseLocation This is Derivative Compute Question 0002. \\ \fi
\begin{problem}

Determine if the limit approaches a finite number, $\pm\infty$, or does not exist. (If the limit does not exist, write DNE)

\input{Limit-Compute-0002.HELP.tex}

\[\lim_{x\to{6}}\dfrac{2 \, x - 12}{x^{2} - 3 \, x - 18}=\answer{\frac{2}{9}}\]
\end{problem}}%}

\latexProblemContent{
\ifVerboseLocation This is Derivative Compute Question 0002. \\ \fi
\begin{problem}

Determine if the limit approaches a finite number, $\pm\infty$, or does not exist. (If the limit does not exist, write DNE)

\input{Limit-Compute-0002.HELP.tex}

\[\lim_{x\to{-9}}\dfrac{2 \, x + 18}{x^{2} + 5 \, x - 36}=\answer{-\frac{2}{13}}\]
\end{problem}}%}

\latexProblemContent{
\ifVerboseLocation This is Derivative Compute Question 0002. \\ \fi
\begin{problem}

Determine if the limit approaches a finite number, $\pm\infty$, or does not exist. (If the limit does not exist, write DNE)

\input{Limit-Compute-0002.HELP.tex}

\[\lim_{x\to{-5}}\dfrac{3 \, x + 15}{x^{2} + 15 \, x + 50}=\answer{\frac{3}{5}}\]
\end{problem}}%}

\latexProblemContent{
\ifVerboseLocation This is Derivative Compute Question 0002. \\ \fi
\begin{problem}

Determine if the limit approaches a finite number, $\pm\infty$, or does not exist. (If the limit does not exist, write DNE)

\input{Limit-Compute-0002.HELP.tex}

\[\lim_{x\to{-1}}\dfrac{3 \, x + 3}{x^{2} + 6 \, x + 5}=\answer{\frac{3}{4}}\]
\end{problem}}%}

\latexProblemContent{
\ifVerboseLocation This is Derivative Compute Question 0002. \\ \fi
\begin{problem}

Determine if the limit approaches a finite number, $\pm\infty$, or does not exist. (If the limit does not exist, write DNE)

\input{Limit-Compute-0002.HELP.tex}

\[\lim_{x\to{-9}}\dfrac{3 \, x + 27}{x^{2} + 13 \, x + 36}=\answer{-\frac{3}{5}}\]
\end{problem}}%}

\latexProblemContent{
\ifVerboseLocation This is Derivative Compute Question 0002. \\ \fi
\begin{problem}

Determine if the limit approaches a finite number, $\pm\infty$, or does not exist. (If the limit does not exist, write DNE)

\input{Limit-Compute-0002.HELP.tex}

\[\lim_{x\to{-5}}\dfrac{x + 5}{x^{2} - 5 \, x - 50}=\answer{-\frac{1}{15}}\]
\end{problem}}%}

\latexProblemContent{
\ifVerboseLocation This is Derivative Compute Question 0002. \\ \fi
\begin{problem}

Determine if the limit approaches a finite number, $\pm\infty$, or does not exist. (If the limit does not exist, write DNE)

\input{Limit-Compute-0002.HELP.tex}

\[\lim_{x\to{9}}\dfrac{2 \, x - 18}{x^{2} - 81}=\answer{\frac{1}{9}}\]
\end{problem}}%}

\latexProblemContent{
\ifVerboseLocation This is Derivative Compute Question 0002. \\ \fi
\begin{problem}

Determine if the limit approaches a finite number, $\pm\infty$, or does not exist. (If the limit does not exist, write DNE)

\input{Limit-Compute-0002.HELP.tex}

\[\lim_{x\to{-10}}\dfrac{3 \, x + 30}{x^{2} + 11 \, x + 10}=\answer{-\frac{1}{3}}\]
\end{problem}}%}

\latexProblemContent{
\ifVerboseLocation This is Derivative Compute Question 0002. \\ \fi
\begin{problem}

Determine if the limit approaches a finite number, $\pm\infty$, or does not exist. (If the limit does not exist, write DNE)

\input{Limit-Compute-0002.HELP.tex}

\[\lim_{x\to{-1}}\dfrac{2 \, x + 2}{x^{2} + 7 \, x + 6}=\answer{\frac{2}{5}}\]
\end{problem}}%}

\latexProblemContent{
\ifVerboseLocation This is Derivative Compute Question 0002. \\ \fi
\begin{problem}

Determine if the limit approaches a finite number, $\pm\infty$, or does not exist. (If the limit does not exist, write DNE)

\input{Limit-Compute-0002.HELP.tex}

\[\lim_{x\to{2}}\dfrac{x - 2}{x^{2} + 7 \, x - 18}=\answer{\frac{1}{11}}\]
\end{problem}}%}

\latexProblemContent{
\ifVerboseLocation This is Derivative Compute Question 0002. \\ \fi
\begin{problem}

Determine if the limit approaches a finite number, $\pm\infty$, or does not exist. (If the limit does not exist, write DNE)

\input{Limit-Compute-0002.HELP.tex}

\[\lim_{x\to{-5}}\dfrac{x + 5}{x^{2} + 12 \, x + 35}=\answer{\frac{1}{2}}\]
\end{problem}}%}

\latexProblemContent{
\ifVerboseLocation This is Derivative Compute Question 0002. \\ \fi
\begin{problem}

Determine if the limit approaches a finite number, $\pm\infty$, or does not exist. (If the limit does not exist, write DNE)

\input{Limit-Compute-0002.HELP.tex}

\[\lim_{x\to{-1}}\dfrac{x + 1}{x^{2} - 7 \, x - 8}=\answer{-\frac{1}{9}}\]
\end{problem}}%}

\latexProblemContent{
\ifVerboseLocation This is Derivative Compute Question 0002. \\ \fi
\begin{problem}

Determine if the limit approaches a finite number, $\pm\infty$, or does not exist. (If the limit does not exist, write DNE)

\input{Limit-Compute-0002.HELP.tex}

\[\lim_{x\to{-6}}\dfrac{x + 6}{x^{2} + 13 \, x + 42}=\answer{1}\]
\end{problem}}%}

\latexProblemContent{
\ifVerboseLocation This is Derivative Compute Question 0002. \\ \fi
\begin{problem}

Determine if the limit approaches a finite number, $\pm\infty$, or does not exist. (If the limit does not exist, write DNE)

\input{Limit-Compute-0002.HELP.tex}

\[\lim_{x\to{-5}}\dfrac{x + 5}{x^{2} + x - 20}=\answer{-\frac{1}{9}}\]
\end{problem}}%}

\latexProblemContent{
\ifVerboseLocation This is Derivative Compute Question 0002. \\ \fi
\begin{problem}

Determine if the limit approaches a finite number, $\pm\infty$, or does not exist. (If the limit does not exist, write DNE)

\input{Limit-Compute-0002.HELP.tex}

\[\lim_{x\to{5}}\dfrac{3 \, x - 15}{x^{2} - 15 \, x + 50}=\answer{-\frac{3}{5}}\]
\end{problem}}%}

\latexProblemContent{
\ifVerboseLocation This is Derivative Compute Question 0002. \\ \fi
\begin{problem}

Determine if the limit approaches a finite number, $\pm\infty$, or does not exist. (If the limit does not exist, write DNE)

\input{Limit-Compute-0002.HELP.tex}

\[\lim_{x\to{5}}\dfrac{x - 5}{x^{2} - 13 \, x + 40}=\answer{-\frac{1}{3}}\]
\end{problem}}%}

\latexProblemContent{
\ifVerboseLocation This is Derivative Compute Question 0002. \\ \fi
\begin{problem}

Determine if the limit approaches a finite number, $\pm\infty$, or does not exist. (If the limit does not exist, write DNE)

\input{Limit-Compute-0002.HELP.tex}

\[\lim_{x\to{8}}\dfrac{3 \, x - 24}{x^{2} - 15 \, x + 56}=\answer{3}\]
\end{problem}}%}

\latexProblemContent{
\ifVerboseLocation This is Derivative Compute Question 0002. \\ \fi
\begin{problem}

Determine if the limit approaches a finite number, $\pm\infty$, or does not exist. (If the limit does not exist, write DNE)

\input{Limit-Compute-0002.HELP.tex}

\[\lim_{x\to{1}}\dfrac{x - 1}{x^{2} + 9 \, x - 10}=\answer{\frac{1}{11}}\]
\end{problem}}%}

\latexProblemContent{
\ifVerboseLocation This is Derivative Compute Question 0002. \\ \fi
\begin{problem}

Determine if the limit approaches a finite number, $\pm\infty$, or does not exist. (If the limit does not exist, write DNE)

\input{Limit-Compute-0002.HELP.tex}

\[\lim_{x\to{-1}}\dfrac{2 \, x + 2}{x^{2} + 3 \, x + 2}=\answer{2}\]
\end{problem}}%}

\latexProblemContent{
\ifVerboseLocation This is Derivative Compute Question 0002. \\ \fi
\begin{problem}

Determine if the limit approaches a finite number, $\pm\infty$, or does not exist. (If the limit does not exist, write DNE)

\input{Limit-Compute-0002.HELP.tex}

\[\lim_{x\to{-2}}\dfrac{x + 2}{x^{2} - 4}=\answer{-\frac{1}{4}}\]
\end{problem}}%}

\latexProblemContent{
\ifVerboseLocation This is Derivative Compute Question 0002. \\ \fi
\begin{problem}

Determine if the limit approaches a finite number, $\pm\infty$, or does not exist. (If the limit does not exist, write DNE)

\input{Limit-Compute-0002.HELP.tex}

\[\lim_{x\to{2}}\dfrac{3 \, x - 6}{x^{2} + 2 \, x - 8}=\answer{\frac{1}{2}}\]
\end{problem}}%}

\latexProblemContent{
\ifVerboseLocation This is Derivative Compute Question 0002. \\ \fi
\begin{problem}

Determine if the limit approaches a finite number, $\pm\infty$, or does not exist. (If the limit does not exist, write DNE)

\input{Limit-Compute-0002.HELP.tex}

\[\lim_{x\to{2}}\dfrac{2 \, x - 4}{x^{2} - x - 2}=\answer{\frac{2}{3}}\]
\end{problem}}%}

\latexProblemContent{
\ifVerboseLocation This is Derivative Compute Question 0002. \\ \fi
\begin{problem}

Determine if the limit approaches a finite number, $\pm\infty$, or does not exist. (If the limit does not exist, write DNE)

\input{Limit-Compute-0002.HELP.tex}

\[\lim_{x\to{-10}}\dfrac{x + 10}{x^{2} + 14 \, x + 40}=\answer{-\frac{1}{6}}\]
\end{problem}}%}

\latexProblemContent{
\ifVerboseLocation This is Derivative Compute Question 0002. \\ \fi
\begin{problem}

Determine if the limit approaches a finite number, $\pm\infty$, or does not exist. (If the limit does not exist, write DNE)

\input{Limit-Compute-0002.HELP.tex}

\[\lim_{x\to{-7}}\dfrac{2 \, x + 14}{x^{2} + 12 \, x + 35}=\answer{-1}\]
\end{problem}}%}

\latexProblemContent{
\ifVerboseLocation This is Derivative Compute Question 0002. \\ \fi
\begin{problem}

Determine if the limit approaches a finite number, $\pm\infty$, or does not exist. (If the limit does not exist, write DNE)

\input{Limit-Compute-0002.HELP.tex}

\[\lim_{x\to{-1}}\dfrac{3 \, x + 3}{x^{2} - x - 2}=\answer{-1}\]
\end{problem}}%}

\latexProblemContent{
\ifVerboseLocation This is Derivative Compute Question 0002. \\ \fi
\begin{problem}

Determine if the limit approaches a finite number, $\pm\infty$, or does not exist. (If the limit does not exist, write DNE)

\input{Limit-Compute-0002.HELP.tex}

\[\lim_{x\to{10}}\dfrac{3 \, x - 30}{x^{2} - x - 90}=\answer{\frac{3}{19}}\]
\end{problem}}%}

\latexProblemContent{
\ifVerboseLocation This is Derivative Compute Question 0002. \\ \fi
\begin{problem}

Determine if the limit approaches a finite number, $\pm\infty$, or does not exist. (If the limit does not exist, write DNE)

\input{Limit-Compute-0002.HELP.tex}

\[\lim_{x\to{7}}\dfrac{2 \, x - 14}{x^{2} - 2 \, x - 35}=\answer{\frac{1}{6}}\]
\end{problem}}%}

\latexProblemContent{
\ifVerboseLocation This is Derivative Compute Question 0002. \\ \fi
\begin{problem}

Determine if the limit approaches a finite number, $\pm\infty$, or does not exist. (If the limit does not exist, write DNE)

\input{Limit-Compute-0002.HELP.tex}

\[\lim_{x\to{-2}}\dfrac{2 \, x + 4}{x^{2} + 9 \, x + 14}=\answer{\frac{2}{5}}\]
\end{problem}}%}

\latexProblemContent{
\ifVerboseLocation This is Derivative Compute Question 0002. \\ \fi
\begin{problem}

Determine if the limit approaches a finite number, $\pm\infty$, or does not exist. (If the limit does not exist, write DNE)

\input{Limit-Compute-0002.HELP.tex}

\[\lim_{x\to{3}}\dfrac{3 \, x - 9}{x^{2} - 11 \, x + 24}=\answer{-\frac{3}{5}}\]
\end{problem}}%}

\latexProblemContent{
\ifVerboseLocation This is Derivative Compute Question 0002. \\ \fi
\begin{problem}

Determine if the limit approaches a finite number, $\pm\infty$, or does not exist. (If the limit does not exist, write DNE)

\input{Limit-Compute-0002.HELP.tex}

\[\lim_{x\to{-4}}\dfrac{3 \, x + 12}{x^{2} + 10 \, x + 24}=\answer{\frac{3}{2}}\]
\end{problem}}%}

\latexProblemContent{
\ifVerboseLocation This is Derivative Compute Question 0002. \\ \fi
\begin{problem}

Determine if the limit approaches a finite number, $\pm\infty$, or does not exist. (If the limit does not exist, write DNE)

\input{Limit-Compute-0002.HELP.tex}

\[\lim_{x\to{-5}}\dfrac{2 \, x + 10}{x^{2} - 2 \, x - 35}=\answer{-\frac{1}{6}}\]
\end{problem}}%}

\latexProblemContent{
\ifVerboseLocation This is Derivative Compute Question 0002. \\ \fi
\begin{problem}

Determine if the limit approaches a finite number, $\pm\infty$, or does not exist. (If the limit does not exist, write DNE)

\input{Limit-Compute-0002.HELP.tex}

\[\lim_{x\to{9}}\dfrac{3 \, x - 27}{x^{2} - 11 \, x + 18}=\answer{\frac{3}{7}}\]
\end{problem}}%}

\latexProblemContent{
\ifVerboseLocation This is Derivative Compute Question 0002. \\ \fi
\begin{problem}

Determine if the limit approaches a finite number, $\pm\infty$, or does not exist. (If the limit does not exist, write DNE)

\input{Limit-Compute-0002.HELP.tex}

\[\lim_{x\to{2}}\dfrac{3 \, x - 6}{x^{2} + 8 \, x - 20}=\answer{\frac{1}{4}}\]
\end{problem}}%}

\latexProblemContent{
\ifVerboseLocation This is Derivative Compute Question 0002. \\ \fi
\begin{problem}

Determine if the limit approaches a finite number, $\pm\infty$, or does not exist. (If the limit does not exist, write DNE)

\input{Limit-Compute-0002.HELP.tex}

\[\lim_{x\to{5}}\dfrac{x - 5}{x^{2} - 8 \, x + 15}=\answer{\frac{1}{2}}\]
\end{problem}}%}

\latexProblemContent{
\ifVerboseLocation This is Derivative Compute Question 0002. \\ \fi
\begin{problem}

Determine if the limit approaches a finite number, $\pm\infty$, or does not exist. (If the limit does not exist, write DNE)

\input{Limit-Compute-0002.HELP.tex}

\[\lim_{x\to{7}}\dfrac{3 \, x - 21}{x^{2} - 17 \, x + 70}=\answer{-1}\]
\end{problem}}%}

\latexProblemContent{
\ifVerboseLocation This is Derivative Compute Question 0002. \\ \fi
\begin{problem}

Determine if the limit approaches a finite number, $\pm\infty$, or does not exist. (If the limit does not exist, write DNE)

\input{Limit-Compute-0002.HELP.tex}

\[\lim_{x\to{-4}}\dfrac{3 \, x + 12}{x^{2} + 11 \, x + 28}=\answer{1}\]
\end{problem}}%}

\latexProblemContent{
\ifVerboseLocation This is Derivative Compute Question 0002. \\ \fi
\begin{problem}

Determine if the limit approaches a finite number, $\pm\infty$, or does not exist. (If the limit does not exist, write DNE)

\input{Limit-Compute-0002.HELP.tex}

\[\lim_{x\to{-10}}\dfrac{x + 10}{x^{2} + 6 \, x - 40}=\answer{-\frac{1}{14}}\]
\end{problem}}%}

\latexProblemContent{
\ifVerboseLocation This is Derivative Compute Question 0002. \\ \fi
\begin{problem}

Determine if the limit approaches a finite number, $\pm\infty$, or does not exist. (If the limit does not exist, write DNE)

\input{Limit-Compute-0002.HELP.tex}

\[\lim_{x\to{4}}\dfrac{3 \, x - 12}{x^{2} - 14 \, x + 40}=\answer{-\frac{1}{2}}\]
\end{problem}}%}

\latexProblemContent{
\ifVerboseLocation This is Derivative Compute Question 0002. \\ \fi
\begin{problem}

Determine if the limit approaches a finite number, $\pm\infty$, or does not exist. (If the limit does not exist, write DNE)

\input{Limit-Compute-0002.HELP.tex}

\[\lim_{x\to{8}}\dfrac{3 \, x - 24}{x^{2} - 13 \, x + 40}=\answer{1}\]
\end{problem}}%}

\latexProblemContent{
\ifVerboseLocation This is Derivative Compute Question 0002. \\ \fi
\begin{problem}

Determine if the limit approaches a finite number, $\pm\infty$, or does not exist. (If the limit does not exist, write DNE)

\input{Limit-Compute-0002.HELP.tex}

\[\lim_{x\to{-8}}\dfrac{2 \, x + 16}{x^{2} + 2 \, x - 48}=\answer{-\frac{1}{7}}\]
\end{problem}}%}

\latexProblemContent{
\ifVerboseLocation This is Derivative Compute Question 0002. \\ \fi
\begin{problem}

Determine if the limit approaches a finite number, $\pm\infty$, or does not exist. (If the limit does not exist, write DNE)

\input{Limit-Compute-0002.HELP.tex}

\[\lim_{x\to{8}}\dfrac{3 \, x - 24}{x^{2} - 12 \, x + 32}=\answer{\frac{3}{4}}\]
\end{problem}}%}

\latexProblemContent{
\ifVerboseLocation This is Derivative Compute Question 0002. \\ \fi
\begin{problem}

Determine if the limit approaches a finite number, $\pm\infty$, or does not exist. (If the limit does not exist, write DNE)

\input{Limit-Compute-0002.HELP.tex}

\[\lim_{x\to{1}}\dfrac{3 \, x - 3}{x^{2} + 4 \, x - 5}=\answer{\frac{1}{2}}\]
\end{problem}}%}

\latexProblemContent{
\ifVerboseLocation This is Derivative Compute Question 0002. \\ \fi
\begin{problem}

Determine if the limit approaches a finite number, $\pm\infty$, or does not exist. (If the limit does not exist, write DNE)

\input{Limit-Compute-0002.HELP.tex}

\[\lim_{x\to{5}}\dfrac{3 \, x - 15}{x^{2} + x - 30}=\answer{\frac{3}{11}}\]
\end{problem}}%}

\latexProblemContent{
\ifVerboseLocation This is Derivative Compute Question 0002. \\ \fi
\begin{problem}

Determine if the limit approaches a finite number, $\pm\infty$, or does not exist. (If the limit does not exist, write DNE)

\input{Limit-Compute-0002.HELP.tex}

\[\lim_{x\to{-3}}\dfrac{2 \, x + 6}{x^{2} + 9 \, x + 18}=\answer{\frac{2}{3}}\]
\end{problem}}%}

\latexProblemContent{
\ifVerboseLocation This is Derivative Compute Question 0002. \\ \fi
\begin{problem}

Determine if the limit approaches a finite number, $\pm\infty$, or does not exist. (If the limit does not exist, write DNE)

\input{Limit-Compute-0002.HELP.tex}

\[\lim_{x\to{-4}}\dfrac{2 \, x + 8}{x^{2} + 12 \, x + 32}=\answer{\frac{1}{2}}\]
\end{problem}}%}

\latexProblemContent{
\ifVerboseLocation This is Derivative Compute Question 0002. \\ \fi
\begin{problem}

Determine if the limit approaches a finite number, $\pm\infty$, or does not exist. (If the limit does not exist, write DNE)

\input{Limit-Compute-0002.HELP.tex}

\[\lim_{x\to{-8}}\dfrac{3 \, x + 24}{x^{2} + 5 \, x - 24}=\answer{-\frac{3}{11}}\]
\end{problem}}%}

\latexProblemContent{
\ifVerboseLocation This is Derivative Compute Question 0002. \\ \fi
\begin{problem}

Determine if the limit approaches a finite number, $\pm\infty$, or does not exist. (If the limit does not exist, write DNE)

\input{Limit-Compute-0002.HELP.tex}

\[\lim_{x\to{-4}}\dfrac{3 \, x + 12}{x^{2} + 5 \, x + 4}=\answer{-1}\]
\end{problem}}%}

\latexProblemContent{
\ifVerboseLocation This is Derivative Compute Question 0002. \\ \fi
\begin{problem}

Determine if the limit approaches a finite number, $\pm\infty$, or does not exist. (If the limit does not exist, write DNE)

\input{Limit-Compute-0002.HELP.tex}

\[\lim_{x\to{4}}\dfrac{2 \, x - 8}{x^{2} - 3 \, x - 4}=\answer{\frac{2}{5}}\]
\end{problem}}%}

\latexProblemContent{
\ifVerboseLocation This is Derivative Compute Question 0002. \\ \fi
\begin{problem}

Determine if the limit approaches a finite number, $\pm\infty$, or does not exist. (If the limit does not exist, write DNE)

\input{Limit-Compute-0002.HELP.tex}

\[\lim_{x\to{6}}\dfrac{x - 6}{x^{2} - 36}=\answer{\frac{1}{12}}\]
\end{problem}}%}

\latexProblemContent{
\ifVerboseLocation This is Derivative Compute Question 0002. \\ \fi
\begin{problem}

Determine if the limit approaches a finite number, $\pm\infty$, or does not exist. (If the limit does not exist, write DNE)

\input{Limit-Compute-0002.HELP.tex}

\[\lim_{x\to{-9}}\dfrac{x + 9}{x^{2} + 2 \, x - 63}=\answer{-\frac{1}{16}}\]
\end{problem}}%}

\latexProblemContent{
\ifVerboseLocation This is Derivative Compute Question 0002. \\ \fi
\begin{problem}

Determine if the limit approaches a finite number, $\pm\infty$, or does not exist. (If the limit does not exist, write DNE)

\input{Limit-Compute-0002.HELP.tex}

\[\lim_{x\to{-4}}\dfrac{3 \, x + 12}{x^{2} + 7 \, x + 12}=\answer{-3}\]
\end{problem}}%}

\latexProblemContent{
\ifVerboseLocation This is Derivative Compute Question 0002. \\ \fi
\begin{problem}

Determine if the limit approaches a finite number, $\pm\infty$, or does not exist. (If the limit does not exist, write DNE)

\input{Limit-Compute-0002.HELP.tex}

\[\lim_{x\to{-7}}\dfrac{3 \, x + 21}{x^{2} + 5 \, x - 14}=\answer{-\frac{1}{3}}\]
\end{problem}}%}

\latexProblemContent{
\ifVerboseLocation This is Derivative Compute Question 0002. \\ \fi
\begin{problem}

Determine if the limit approaches a finite number, $\pm\infty$, or does not exist. (If the limit does not exist, write DNE)

\input{Limit-Compute-0002.HELP.tex}

\[\lim_{x\to{3}}\dfrac{x - 3}{x^{2} - x - 6}=\answer{\frac{1}{5}}\]
\end{problem}}%}

\latexProblemContent{
\ifVerboseLocation This is Derivative Compute Question 0002. \\ \fi
\begin{problem}

Determine if the limit approaches a finite number, $\pm\infty$, or does not exist. (If the limit does not exist, write DNE)

\input{Limit-Compute-0002.HELP.tex}

\[\lim_{x\to{-3}}\dfrac{2 \, x + 6}{x^{2} + 7 \, x + 12}=\answer{2}\]
\end{problem}}%}

\latexProblemContent{
\ifVerboseLocation This is Derivative Compute Question 0002. \\ \fi
\begin{problem}

Determine if the limit approaches a finite number, $\pm\infty$, or does not exist. (If the limit does not exist, write DNE)

\input{Limit-Compute-0002.HELP.tex}

\[\lim_{x\to{-4}}\dfrac{x + 4}{x^{2} + 6 \, x + 8}=\answer{-\frac{1}{2}}\]
\end{problem}}%}

\latexProblemContent{
\ifVerboseLocation This is Derivative Compute Question 0002. \\ \fi
\begin{problem}

Determine if the limit approaches a finite number, $\pm\infty$, or does not exist. (If the limit does not exist, write DNE)

\input{Limit-Compute-0002.HELP.tex}

\[\lim_{x\to{-3}}\dfrac{2 \, x + 6}{x^{2} + 2 \, x - 3}=\answer{-\frac{1}{2}}\]
\end{problem}}%}

\latexProblemContent{
\ifVerboseLocation This is Derivative Compute Question 0002. \\ \fi
\begin{problem}

Determine if the limit approaches a finite number, $\pm\infty$, or does not exist. (If the limit does not exist, write DNE)

\input{Limit-Compute-0002.HELP.tex}

\[\lim_{x\to{6}}\dfrac{x - 6}{x^{2} - 16 \, x + 60}=\answer{-\frac{1}{4}}\]
\end{problem}}%}

\latexProblemContent{
\ifVerboseLocation This is Derivative Compute Question 0002. \\ \fi
\begin{problem}

Determine if the limit approaches a finite number, $\pm\infty$, or does not exist. (If the limit does not exist, write DNE)

\input{Limit-Compute-0002.HELP.tex}

\[\lim_{x\to{1}}\dfrac{3 \, x - 3}{x^{2} + 7 \, x - 8}=\answer{\frac{1}{3}}\]
\end{problem}}%}

\latexProblemContent{
\ifVerboseLocation This is Derivative Compute Question 0002. \\ \fi
\begin{problem}

Determine if the limit approaches a finite number, $\pm\infty$, or does not exist. (If the limit does not exist, write DNE)

\input{Limit-Compute-0002.HELP.tex}

\[\lim_{x\to{2}}\dfrac{3 \, x - 6}{x^{2} + x - 6}=\answer{\frac{3}{5}}\]
\end{problem}}%}

\latexProblemContent{
\ifVerboseLocation This is Derivative Compute Question 0002. \\ \fi
\begin{problem}

Determine if the limit approaches a finite number, $\pm\infty$, or does not exist. (If the limit does not exist, write DNE)

\input{Limit-Compute-0002.HELP.tex}

\[\lim_{x\to{-7}}\dfrac{x + 7}{x^{2} - 3 \, x - 70}=\answer{-\frac{1}{17}}\]
\end{problem}}%}

\latexProblemContent{
\ifVerboseLocation This is Derivative Compute Question 0002. \\ \fi
\begin{problem}

Determine if the limit approaches a finite number, $\pm\infty$, or does not exist. (If the limit does not exist, write DNE)

\input{Limit-Compute-0002.HELP.tex}

\[\lim_{x\to{-5}}\dfrac{x + 5}{x^{2} - 25}=\answer{-\frac{1}{10}}\]
\end{problem}}%}

\latexProblemContent{
\ifVerboseLocation This is Derivative Compute Question 0002. \\ \fi
\begin{problem}

Determine if the limit approaches a finite number, $\pm\infty$, or does not exist. (If the limit does not exist, write DNE)

\input{Limit-Compute-0002.HELP.tex}

\[\lim_{x\to{4}}\dfrac{x - 4}{x^{2} + 5 \, x - 36}=\answer{\frac{1}{13}}\]
\end{problem}}%}

\latexProblemContent{
\ifVerboseLocation This is Derivative Compute Question 0002. \\ \fi
\begin{problem}

Determine if the limit approaches a finite number, $\pm\infty$, or does not exist. (If the limit does not exist, write DNE)

\input{Limit-Compute-0002.HELP.tex}

\[\lim_{x\to{4}}\dfrac{2 \, x - 8}{x^{2} - 11 \, x + 28}=\answer{-\frac{2}{3}}\]
\end{problem}}%}

\latexProblemContent{
\ifVerboseLocation This is Derivative Compute Question 0002. \\ \fi
\begin{problem}

Determine if the limit approaches a finite number, $\pm\infty$, or does not exist. (If the limit does not exist, write DNE)

\input{Limit-Compute-0002.HELP.tex}

\[\lim_{x\to{1}}\dfrac{2 \, x - 2}{x^{2} - 9 \, x + 8}=\answer{-\frac{2}{7}}\]
\end{problem}}%}

\latexProblemContent{
\ifVerboseLocation This is Derivative Compute Question 0002. \\ \fi
\begin{problem}

Determine if the limit approaches a finite number, $\pm\infty$, or does not exist. (If the limit does not exist, write DNE)

\input{Limit-Compute-0002.HELP.tex}

\[\lim_{x\to{-6}}\dfrac{2 \, x + 12}{x^{2} + 4 \, x - 12}=\answer{-\frac{1}{4}}\]
\end{problem}}%}

\latexProblemContent{
\ifVerboseLocation This is Derivative Compute Question 0002. \\ \fi
\begin{problem}

Determine if the limit approaches a finite number, $\pm\infty$, or does not exist. (If the limit does not exist, write DNE)

\input{Limit-Compute-0002.HELP.tex}

\[\lim_{x\to{9}}\dfrac{3 \, x - 27}{x^{2} - 12 \, x + 27}=\answer{\frac{1}{2}}\]
\end{problem}}%}

\latexProblemContent{
\ifVerboseLocation This is Derivative Compute Question 0002. \\ \fi
\begin{problem}

Determine if the limit approaches a finite number, $\pm\infty$, or does not exist. (If the limit does not exist, write DNE)

\input{Limit-Compute-0002.HELP.tex}

\[\lim_{x\to{9}}\dfrac{x - 9}{x^{2} - 6 \, x - 27}=\answer{\frac{1}{12}}\]
\end{problem}}%}

\latexProblemContent{
\ifVerboseLocation This is Derivative Compute Question 0002. \\ \fi
\begin{problem}

Determine if the limit approaches a finite number, $\pm\infty$, or does not exist. (If the limit does not exist, write DNE)

\input{Limit-Compute-0002.HELP.tex}

\[\lim_{x\to{10}}\dfrac{2 \, x - 20}{x^{2} - 14 \, x + 40}=\answer{\frac{1}{3}}\]
\end{problem}}%}

\latexProblemContent{
\ifVerboseLocation This is Derivative Compute Question 0002. \\ \fi
\begin{problem}

Determine if the limit approaches a finite number, $\pm\infty$, or does not exist. (If the limit does not exist, write DNE)

\input{Limit-Compute-0002.HELP.tex}

\[\lim_{x\to{4}}\dfrac{x - 4}{x^{2} - 5 \, x + 4}=\answer{\frac{1}{3}}\]
\end{problem}}%}

\latexProblemContent{
\ifVerboseLocation This is Derivative Compute Question 0002. \\ \fi
\begin{problem}

Determine if the limit approaches a finite number, $\pm\infty$, or does not exist. (If the limit does not exist, write DNE)

\input{Limit-Compute-0002.HELP.tex}

\[\lim_{x\to{2}}\dfrac{3 \, x - 6}{x^{2} + 5 \, x - 14}=\answer{\frac{1}{3}}\]
\end{problem}}%}

\latexProblemContent{
\ifVerboseLocation This is Derivative Compute Question 0002. \\ \fi
\begin{problem}

Determine if the limit approaches a finite number, $\pm\infty$, or does not exist. (If the limit does not exist, write DNE)

\input{Limit-Compute-0002.HELP.tex}

\[\lim_{x\to{10}}\dfrac{3 \, x - 30}{x^{2} - 4 \, x - 60}=\answer{\frac{3}{16}}\]
\end{problem}}%}

\latexProblemContent{
\ifVerboseLocation This is Derivative Compute Question 0002. \\ \fi
\begin{problem}

Determine if the limit approaches a finite number, $\pm\infty$, or does not exist. (If the limit does not exist, write DNE)

\input{Limit-Compute-0002.HELP.tex}

\[\lim_{x\to{-4}}\dfrac{x + 4}{x^{2} - 4 \, x - 32}=\answer{-\frac{1}{12}}\]
\end{problem}}%}

\latexProblemContent{
\ifVerboseLocation This is Derivative Compute Question 0002. \\ \fi
\begin{problem}

Determine if the limit approaches a finite number, $\pm\infty$, or does not exist. (If the limit does not exist, write DNE)

\input{Limit-Compute-0002.HELP.tex}

\[\lim_{x\to{-3}}\dfrac{2 \, x + 6}{x^{2} + 13 \, x + 30}=\answer{\frac{2}{7}}\]
\end{problem}}%}

\latexProblemContent{
\ifVerboseLocation This is Derivative Compute Question 0002. \\ \fi
\begin{problem}

Determine if the limit approaches a finite number, $\pm\infty$, or does not exist. (If the limit does not exist, write DNE)

\input{Limit-Compute-0002.HELP.tex}

\[\lim_{x\to{9}}\dfrac{2 \, x - 18}{x^{2} - 13 \, x + 36}=\answer{\frac{2}{5}}\]
\end{problem}}%}

\latexProblemContent{
\ifVerboseLocation This is Derivative Compute Question 0002. \\ \fi
\begin{problem}

Determine if the limit approaches a finite number, $\pm\infty$, or does not exist. (If the limit does not exist, write DNE)

\input{Limit-Compute-0002.HELP.tex}

\[\lim_{x\to{10}}\dfrac{x - 10}{x^{2} - x - 90}=\answer{\frac{1}{19}}\]
\end{problem}}%}

\latexProblemContent{
\ifVerboseLocation This is Derivative Compute Question 0002. \\ \fi
\begin{problem}

Determine if the limit approaches a finite number, $\pm\infty$, or does not exist. (If the limit does not exist, write DNE)

\input{Limit-Compute-0002.HELP.tex}

\[\lim_{x\to{2}}\dfrac{2 \, x - 4}{x^{2} - 3 \, x + 2}=\answer{2}\]
\end{problem}}%}

\latexProblemContent{
\ifVerboseLocation This is Derivative Compute Question 0002. \\ \fi
\begin{problem}

Determine if the limit approaches a finite number, $\pm\infty$, or does not exist. (If the limit does not exist, write DNE)

\input{Limit-Compute-0002.HELP.tex}

\[\lim_{x\to{6}}\dfrac{x - 6}{x^{2} - x - 30}=\answer{\frac{1}{11}}\]
\end{problem}}%}

\latexProblemContent{
\ifVerboseLocation This is Derivative Compute Question 0002. \\ \fi
\begin{problem}

Determine if the limit approaches a finite number, $\pm\infty$, or does not exist. (If the limit does not exist, write DNE)

\input{Limit-Compute-0002.HELP.tex}

\[\lim_{x\to{-1}}\dfrac{2 \, x + 2}{x^{2} - 7 \, x - 8}=\answer{-\frac{2}{9}}\]
\end{problem}}%}

\latexProblemContent{
\ifVerboseLocation This is Derivative Compute Question 0002. \\ \fi
\begin{problem}

Determine if the limit approaches a finite number, $\pm\infty$, or does not exist. (If the limit does not exist, write DNE)

\input{Limit-Compute-0002.HELP.tex}

\[\lim_{x\to{-2}}\dfrac{2 \, x + 4}{x^{2} + 12 \, x + 20}=\answer{\frac{1}{4}}\]
\end{problem}}%}

\latexProblemContent{
\ifVerboseLocation This is Derivative Compute Question 0002. \\ \fi
\begin{problem}

Determine if the limit approaches a finite number, $\pm\infty$, or does not exist. (If the limit does not exist, write DNE)

\input{Limit-Compute-0002.HELP.tex}

\[\lim_{x\to{-1}}\dfrac{2 \, x + 2}{x^{2} + 11 \, x + 10}=\answer{\frac{2}{9}}\]
\end{problem}}%}

\latexProblemContent{
\ifVerboseLocation This is Derivative Compute Question 0002. \\ \fi
\begin{problem}

Determine if the limit approaches a finite number, $\pm\infty$, or does not exist. (If the limit does not exist, write DNE)

\input{Limit-Compute-0002.HELP.tex}

\[\lim_{x\to{-10}}\dfrac{3 \, x + 30}{x^{2} + 18 \, x + 80}=\answer{-\frac{3}{2}}\]
\end{problem}}%}

\latexProblemContent{
\ifVerboseLocation This is Derivative Compute Question 0002. \\ \fi
\begin{problem}

Determine if the limit approaches a finite number, $\pm\infty$, or does not exist. (If the limit does not exist, write DNE)

\input{Limit-Compute-0002.HELP.tex}

\[\lim_{x\to{-2}}\dfrac{2 \, x + 4}{x^{2} + 11 \, x + 18}=\answer{\frac{2}{7}}\]
\end{problem}}%}

\latexProblemContent{
\ifVerboseLocation This is Derivative Compute Question 0002. \\ \fi
\begin{problem}

Determine if the limit approaches a finite number, $\pm\infty$, or does not exist. (If the limit does not exist, write DNE)

\input{Limit-Compute-0002.HELP.tex}

\[\lim_{x\to{4}}\dfrac{x - 4}{x^{2} + 6 \, x - 40}=\answer{\frac{1}{14}}\]
\end{problem}}%}

\latexProblemContent{
\ifVerboseLocation This is Derivative Compute Question 0002. \\ \fi
\begin{problem}

Determine if the limit approaches a finite number, $\pm\infty$, or does not exist. (If the limit does not exist, write DNE)

\input{Limit-Compute-0002.HELP.tex}

\[\lim_{x\to{-1}}\dfrac{x + 1}{x^{2} + 10 \, x + 9}=\answer{\frac{1}{8}}\]
\end{problem}}%}

\latexProblemContent{
\ifVerboseLocation This is Derivative Compute Question 0002. \\ \fi
\begin{problem}

Determine if the limit approaches a finite number, $\pm\infty$, or does not exist. (If the limit does not exist, write DNE)

\input{Limit-Compute-0002.HELP.tex}

\[\lim_{x\to{-10}}\dfrac{3 \, x + 30}{x^{2} + 9 \, x - 10}=\answer{-\frac{3}{11}}\]
\end{problem}}%}

\latexProblemContent{
\ifVerboseLocation This is Derivative Compute Question 0002. \\ \fi
\begin{problem}

Determine if the limit approaches a finite number, $\pm\infty$, or does not exist. (If the limit does not exist, write DNE)

\input{Limit-Compute-0002.HELP.tex}

\[\lim_{x\to{10}}\dfrac{x - 10}{x^{2} - 7 \, x - 30}=\answer{\frac{1}{13}}\]
\end{problem}}%}

\latexProblemContent{
\ifVerboseLocation This is Derivative Compute Question 0002. \\ \fi
\begin{problem}

Determine if the limit approaches a finite number, $\pm\infty$, or does not exist. (If the limit does not exist, write DNE)

\input{Limit-Compute-0002.HELP.tex}

\[\lim_{x\to{-9}}\dfrac{x + 9}{x^{2} + x - 72}=\answer{-\frac{1}{17}}\]
\end{problem}}%}

\latexProblemContent{
\ifVerboseLocation This is Derivative Compute Question 0002. \\ \fi
\begin{problem}

Determine if the limit approaches a finite number, $\pm\infty$, or does not exist. (If the limit does not exist, write DNE)

\input{Limit-Compute-0002.HELP.tex}

\[\lim_{x\to{-5}}\dfrac{x + 5}{x^{2} + 2 \, x - 15}=\answer{-\frac{1}{8}}\]
\end{problem}}%}

\latexProblemContent{
\ifVerboseLocation This is Derivative Compute Question 0002. \\ \fi
\begin{problem}

Determine if the limit approaches a finite number, $\pm\infty$, or does not exist. (If the limit does not exist, write DNE)

\input{Limit-Compute-0002.HELP.tex}

\[\lim_{x\to{-2}}\dfrac{3 \, x + 6}{x^{2} + 5 \, x + 6}=\answer{3}\]
\end{problem}}%}

\latexProblemContent{
\ifVerboseLocation This is Derivative Compute Question 0002. \\ \fi
\begin{problem}

Determine if the limit approaches a finite number, $\pm\infty$, or does not exist. (If the limit does not exist, write DNE)

\input{Limit-Compute-0002.HELP.tex}

\[\lim_{x\to{-10}}\dfrac{3 \, x + 30}{x^{2} + 14 \, x + 40}=\answer{-\frac{1}{2}}\]
\end{problem}}%}

\latexProblemContent{
\ifVerboseLocation This is Derivative Compute Question 0002. \\ \fi
\begin{problem}

Determine if the limit approaches a finite number, $\pm\infty$, or does not exist. (If the limit does not exist, write DNE)

\input{Limit-Compute-0002.HELP.tex}

\[\lim_{x\to{1}}\dfrac{3 \, x - 3}{x^{2} - 7 \, x + 6}=\answer{-\frac{3}{5}}\]
\end{problem}}%}

\latexProblemContent{
\ifVerboseLocation This is Derivative Compute Question 0002. \\ \fi
\begin{problem}

Determine if the limit approaches a finite number, $\pm\infty$, or does not exist. (If the limit does not exist, write DNE)

\input{Limit-Compute-0002.HELP.tex}

\[\lim_{x\to{4}}\dfrac{x - 4}{x^{2} - 16}=\answer{\frac{1}{8}}\]
\end{problem}}%}

\latexProblemContent{
\ifVerboseLocation This is Derivative Compute Question 0002. \\ \fi
\begin{problem}

Determine if the limit approaches a finite number, $\pm\infty$, or does not exist. (If the limit does not exist, write DNE)

\input{Limit-Compute-0002.HELP.tex}

\[\lim_{x\to{4}}\dfrac{3 \, x - 12}{x^{2} + 6 \, x - 40}=\answer{\frac{3}{14}}\]
\end{problem}}%}

\latexProblemContent{
\ifVerboseLocation This is Derivative Compute Question 0002. \\ \fi
\begin{problem}

Determine if the limit approaches a finite number, $\pm\infty$, or does not exist. (If the limit does not exist, write DNE)

\input{Limit-Compute-0002.HELP.tex}

\[\lim_{x\to{-5}}\dfrac{x + 5}{x^{2} + 15 \, x + 50}=\answer{\frac{1}{5}}\]
\end{problem}}%}

\latexProblemContent{
\ifVerboseLocation This is Derivative Compute Question 0002. \\ \fi
\begin{problem}

Determine if the limit approaches a finite number, $\pm\infty$, or does not exist. (If the limit does not exist, write DNE)

\input{Limit-Compute-0002.HELP.tex}

\[\lim_{x\to{-4}}\dfrac{x + 4}{x^{2} + 14 \, x + 40}=\answer{\frac{1}{6}}\]
\end{problem}}%}

\latexProblemContent{
\ifVerboseLocation This is Derivative Compute Question 0002. \\ \fi
\begin{problem}

Determine if the limit approaches a finite number, $\pm\infty$, or does not exist. (If the limit does not exist, write DNE)

\input{Limit-Compute-0002.HELP.tex}

\[\lim_{x\to{-4}}\dfrac{2 \, x + 8}{x^{2} - 4 \, x - 32}=\answer{-\frac{1}{6}}\]
\end{problem}}%}

\latexProblemContent{
\ifVerboseLocation This is Derivative Compute Question 0002. \\ \fi
\begin{problem}

Determine if the limit approaches a finite number, $\pm\infty$, or does not exist. (If the limit does not exist, write DNE)

\input{Limit-Compute-0002.HELP.tex}

\[\lim_{x\to{10}}\dfrac{2 \, x - 20}{x^{2} - 9 \, x - 10}=\answer{\frac{2}{11}}\]
\end{problem}}%}

\latexProblemContent{
\ifVerboseLocation This is Derivative Compute Question 0002. \\ \fi
\begin{problem}

Determine if the limit approaches a finite number, $\pm\infty$, or does not exist. (If the limit does not exist, write DNE)

\input{Limit-Compute-0002.HELP.tex}

\[\lim_{x\to{2}}\dfrac{x - 2}{x^{2} - 12 \, x + 20}=\answer{-\frac{1}{8}}\]
\end{problem}}%}

\latexProblemContent{
\ifVerboseLocation This is Derivative Compute Question 0002. \\ \fi
\begin{problem}

Determine if the limit approaches a finite number, $\pm\infty$, or does not exist. (If the limit does not exist, write DNE)

\input{Limit-Compute-0002.HELP.tex}

\[\lim_{x\to{-8}}\dfrac{3 \, x + 24}{x^{2} - 64}=\answer{-\frac{3}{16}}\]
\end{problem}}%}

\latexProblemContent{
\ifVerboseLocation This is Derivative Compute Question 0002. \\ \fi
\begin{problem}

Determine if the limit approaches a finite number, $\pm\infty$, or does not exist. (If the limit does not exist, write DNE)

\input{Limit-Compute-0002.HELP.tex}

\[\lim_{x\to{5}}\dfrac{3 \, x - 15}{x^{2} + 4 \, x - 45}=\answer{\frac{3}{14}}\]
\end{problem}}%}

\latexProblemContent{
\ifVerboseLocation This is Derivative Compute Question 0002. \\ \fi
\begin{problem}

Determine if the limit approaches a finite number, $\pm\infty$, or does not exist. (If the limit does not exist, write DNE)

\input{Limit-Compute-0002.HELP.tex}

\[\lim_{x\to{-7}}\dfrac{x + 7}{x^{2} + 9 \, x + 14}=\answer{-\frac{1}{5}}\]
\end{problem}}%}

\latexProblemContent{
\ifVerboseLocation This is Derivative Compute Question 0002. \\ \fi
\begin{problem}

Determine if the limit approaches a finite number, $\pm\infty$, or does not exist. (If the limit does not exist, write DNE)

\input{Limit-Compute-0002.HELP.tex}

\[\lim_{x\to{-5}}\dfrac{2 \, x + 10}{x^{2} + 14 \, x + 45}=\answer{\frac{1}{2}}\]
\end{problem}}%}

\latexProblemContent{
\ifVerboseLocation This is Derivative Compute Question 0002. \\ \fi
\begin{problem}

Determine if the limit approaches a finite number, $\pm\infty$, or does not exist. (If the limit does not exist, write DNE)

\input{Limit-Compute-0002.HELP.tex}

\[\lim_{x\to{4}}\dfrac{2 \, x - 8}{x^{2} + 3 \, x - 28}=\answer{\frac{2}{11}}\]
\end{problem}}%}

\latexProblemContent{
\ifVerboseLocation This is Derivative Compute Question 0002. \\ \fi
\begin{problem}

Determine if the limit approaches a finite number, $\pm\infty$, or does not exist. (If the limit does not exist, write DNE)

\input{Limit-Compute-0002.HELP.tex}

\[\lim_{x\to{5}}\dfrac{2 \, x - 10}{x^{2} - 14 \, x + 45}=\answer{-\frac{1}{2}}\]
\end{problem}}%}

\latexProblemContent{
\ifVerboseLocation This is Derivative Compute Question 0002. \\ \fi
\begin{problem}

Determine if the limit approaches a finite number, $\pm\infty$, or does not exist. (If the limit does not exist, write DNE)

\input{Limit-Compute-0002.HELP.tex}

\[\lim_{x\to{7}}\dfrac{2 \, x - 14}{x^{2} - 16 \, x + 63}=\answer{-1}\]
\end{problem}}%}

\latexProblemContent{
\ifVerboseLocation This is Derivative Compute Question 0002. \\ \fi
\begin{problem}

Determine if the limit approaches a finite number, $\pm\infty$, or does not exist. (If the limit does not exist, write DNE)

\input{Limit-Compute-0002.HELP.tex}

\[\lim_{x\to{1}}\dfrac{2 \, x - 2}{x^{2} - 7 \, x + 6}=\answer{-\frac{2}{5}}\]
\end{problem}}%}

\latexProblemContent{
\ifVerboseLocation This is Derivative Compute Question 0002. \\ \fi
\begin{problem}

Determine if the limit approaches a finite number, $\pm\infty$, or does not exist. (If the limit does not exist, write DNE)

\input{Limit-Compute-0002.HELP.tex}

\[\lim_{x\to{-3}}\dfrac{3 \, x + 9}{x^{2} + 13 \, x + 30}=\answer{\frac{3}{7}}\]
\end{problem}}%}

\latexProblemContent{
\ifVerboseLocation This is Derivative Compute Question 0002. \\ \fi
\begin{problem}

Determine if the limit approaches a finite number, $\pm\infty$, or does not exist. (If the limit does not exist, write DNE)

\input{Limit-Compute-0002.HELP.tex}

\[\lim_{x\to{7}}\dfrac{2 \, x - 14}{x^{2} - 12 \, x + 35}=\answer{1}\]
\end{problem}}%}

\latexProblemContent{
\ifVerboseLocation This is Derivative Compute Question 0002. \\ \fi
\begin{problem}

Determine if the limit approaches a finite number, $\pm\infty$, or does not exist. (If the limit does not exist, write DNE)

\input{Limit-Compute-0002.HELP.tex}

\[\lim_{x\to{-9}}\dfrac{3 \, x + 27}{x^{2} - x - 90}=\answer{-\frac{3}{19}}\]
\end{problem}}%}

\latexProblemContent{
\ifVerboseLocation This is Derivative Compute Question 0002. \\ \fi
\begin{problem}

Determine if the limit approaches a finite number, $\pm\infty$, or does not exist. (If the limit does not exist, write DNE)

\input{Limit-Compute-0002.HELP.tex}

\[\lim_{x\to{-9}}\dfrac{x + 9}{x^{2} + 15 \, x + 54}=\answer{-\frac{1}{3}}\]
\end{problem}}%}

\latexProblemContent{
\ifVerboseLocation This is Derivative Compute Question 0002. \\ \fi
\begin{problem}

Determine if the limit approaches a finite number, $\pm\infty$, or does not exist. (If the limit does not exist, write DNE)

\input{Limit-Compute-0002.HELP.tex}

\[\lim_{x\to{8}}\dfrac{2 \, x - 16}{x^{2} - 5 \, x - 24}=\answer{\frac{2}{11}}\]
\end{problem}}%}

\latexProblemContent{
\ifVerboseLocation This is Derivative Compute Question 0002. \\ \fi
\begin{problem}

Determine if the limit approaches a finite number, $\pm\infty$, or does not exist. (If the limit does not exist, write DNE)

\input{Limit-Compute-0002.HELP.tex}

\[\lim_{x\to{-9}}\dfrac{3 \, x + 27}{x^{2} + 11 \, x + 18}=\answer{-\frac{3}{7}}\]
\end{problem}}%}

\latexProblemContent{
\ifVerboseLocation This is Derivative Compute Question 0002. \\ \fi
\begin{problem}

Determine if the limit approaches a finite number, $\pm\infty$, or does not exist. (If the limit does not exist, write DNE)

\input{Limit-Compute-0002.HELP.tex}

\[\lim_{x\to{-5}}\dfrac{2 \, x + 10}{x^{2} + 15 \, x + 50}=\answer{\frac{2}{5}}\]
\end{problem}}%}

\latexProblemContent{
\ifVerboseLocation This is Derivative Compute Question 0002. \\ \fi
\begin{problem}

Determine if the limit approaches a finite number, $\pm\infty$, or does not exist. (If the limit does not exist, write DNE)

\input{Limit-Compute-0002.HELP.tex}

\[\lim_{x\to{-4}}\dfrac{2 \, x + 8}{x^{2} - 2 \, x - 24}=\answer{-\frac{1}{5}}\]
\end{problem}}%}

\latexProblemContent{
\ifVerboseLocation This is Derivative Compute Question 0002. \\ \fi
\begin{problem}

Determine if the limit approaches a finite number, $\pm\infty$, or does not exist. (If the limit does not exist, write DNE)

\input{Limit-Compute-0002.HELP.tex}

\[\lim_{x\to{-10}}\dfrac{x + 10}{x^{2} + 8 \, x - 20}=\answer{-\frac{1}{12}}\]
\end{problem}}%}

\latexProblemContent{
\ifVerboseLocation This is Derivative Compute Question 0002. \\ \fi
\begin{problem}

Determine if the limit approaches a finite number, $\pm\infty$, or does not exist. (If the limit does not exist, write DNE)

\input{Limit-Compute-0002.HELP.tex}

\[\lim_{x\to{2}}\dfrac{3 \, x - 6}{x^{2} - 12 \, x + 20}=\answer{-\frac{3}{8}}\]
\end{problem}}%}

\latexProblemContent{
\ifVerboseLocation This is Derivative Compute Question 0002. \\ \fi
\begin{problem}

Determine if the limit approaches a finite number, $\pm\infty$, or does not exist. (If the limit does not exist, write DNE)

\input{Limit-Compute-0002.HELP.tex}

\[\lim_{x\to{8}}\dfrac{2 \, x - 16}{x^{2} - 13 \, x + 40}=\answer{\frac{2}{3}}\]
\end{problem}}%}

\latexProblemContent{
\ifVerboseLocation This is Derivative Compute Question 0002. \\ \fi
\begin{problem}

Determine if the limit approaches a finite number, $\pm\infty$, or does not exist. (If the limit does not exist, write DNE)

\input{Limit-Compute-0002.HELP.tex}

\[\lim_{x\to{9}}\dfrac{x - 9}{x^{2} - 15 \, x + 54}=\answer{\frac{1}{3}}\]
\end{problem}}%}

\latexProblemContent{
\ifVerboseLocation This is Derivative Compute Question 0002. \\ \fi
\begin{problem}

Determine if the limit approaches a finite number, $\pm\infty$, or does not exist. (If the limit does not exist, write DNE)

\input{Limit-Compute-0002.HELP.tex}

\[\lim_{x\to{8}}\dfrac{2 \, x - 16}{x^{2} - 7 \, x - 8}=\answer{\frac{2}{9}}\]
\end{problem}}%}

