\ProblemFileHeader{XTL_SV_QUESTIONCOUNT}% Process how many problems are in this file and how to detect if it has a desirable problem
\ifproblemToFind% If it has a desirable problem search the file.
%\tagged{Ans@ShortAns, Type@Compute, Topic@Limit, Sub@Rational, File@0003}{
\latexProblemContent{
\ifVerboseLocation This is Derivative Compute Question 0003. \\ \fi
\begin{problem}

Determine if the limit approaches a finite number, $\pm\infty$, or does not exist. (If the limit does not exist, write DNE)

\input{Limit-Compute-0003.HELP.tex}

\[\lim_{x\to{-5}}\dfrac{x^{2} + 9 \, x + 20}{x + 4}=\answer{0}\]
\end{problem}}%}

\latexProblemContent{
\ifVerboseLocation This is Derivative Compute Question 0003. \\ \fi
\begin{problem}

Determine if the limit approaches a finite number, $\pm\infty$, or does not exist. (If the limit does not exist, write DNE)

\input{Limit-Compute-0003.HELP.tex}

\[\lim_{x\to{-4}}\dfrac{x^{2} + 7 \, x + 12}{x + 3}=\answer{0}\]
\end{problem}}%}

\latexProblemContent{
\ifVerboseLocation This is Derivative Compute Question 0003. \\ \fi
\begin{problem}

Determine if the limit approaches a finite number, $\pm\infty$, or does not exist. (If the limit does not exist, write DNE)

\input{Limit-Compute-0003.HELP.tex}

\[\lim_{x\to{-6}}\dfrac{3 \, x^{2} + 45 \, x + 162}{x + 9}=\answer{0}\]
\end{problem}}%}

\latexProblemContent{
\ifVerboseLocation This is Derivative Compute Question 0003. \\ \fi
\begin{problem}

Determine if the limit approaches a finite number, $\pm\infty$, or does not exist. (If the limit does not exist, write DNE)

\input{Limit-Compute-0003.HELP.tex}

\[\lim_{x\to{7}}\dfrac{x^{2} - x - 42}{x + 6}=\answer{0}\]
\end{problem}}%}

\latexProblemContent{
\ifVerboseLocation This is Derivative Compute Question 0003. \\ \fi
\begin{problem}

Determine if the limit approaches a finite number, $\pm\infty$, or does not exist. (If the limit does not exist, write DNE)

\input{Limit-Compute-0003.HELP.tex}

\[\lim_{x\to{-3}}\dfrac{3 \, x^{2} + 21 \, x + 36}{x + 4}=\answer{0}\]
\end{problem}}%}

\latexProblemContent{
\ifVerboseLocation This is Derivative Compute Question 0003. \\ \fi
\begin{problem}

Determine if the limit approaches a finite number, $\pm\infty$, or does not exist. (If the limit does not exist, write DNE)

\input{Limit-Compute-0003.HELP.tex}

\[\lim_{x\to{3}}\dfrac{x^{2} - 11 \, x + 24}{x - 8}=\answer{0}\]
\end{problem}}%}

\latexProblemContent{
\ifVerboseLocation This is Derivative Compute Question 0003. \\ \fi
\begin{problem}

Determine if the limit approaches a finite number, $\pm\infty$, or does not exist. (If the limit does not exist, write DNE)

\input{Limit-Compute-0003.HELP.tex}

\[\lim_{x\to{2}}\dfrac{2 \, x^{2} - 10 \, x + 12}{x - 3}=\answer{0}\]
\end{problem}}%}

\latexProblemContent{
\ifVerboseLocation This is Derivative Compute Question 0003. \\ \fi
\begin{problem}

Determine if the limit approaches a finite number, $\pm\infty$, or does not exist. (If the limit does not exist, write DNE)

\input{Limit-Compute-0003.HELP.tex}

\[\lim_{x\to{9}}\dfrac{3 \, x^{2} - 3 \, x - 216}{x + 8}=\answer{0}\]
\end{problem}}%}

\latexProblemContent{
\ifVerboseLocation This is Derivative Compute Question 0003. \\ \fi
\begin{problem}

Determine if the limit approaches a finite number, $\pm\infty$, or does not exist. (If the limit does not exist, write DNE)

\input{Limit-Compute-0003.HELP.tex}

\[\lim_{x\to{10}}\dfrac{3 \, x^{2} - 6 \, x - 240}{x + 8}=\answer{0}\]
\end{problem}}%}

\latexProblemContent{
\ifVerboseLocation This is Derivative Compute Question 0003. \\ \fi
\begin{problem}

Determine if the limit approaches a finite number, $\pm\infty$, or does not exist. (If the limit does not exist, write DNE)

\input{Limit-Compute-0003.HELP.tex}

\[\lim_{x\to{-1}}\dfrac{3 \, x^{2} + 12 \, x + 9}{x + 3}=\answer{0}\]
\end{problem}}%}

\latexProblemContent{
\ifVerboseLocation This is Derivative Compute Question 0003. \\ \fi
\begin{problem}

Determine if the limit approaches a finite number, $\pm\infty$, or does not exist. (If the limit does not exist, write DNE)

\input{Limit-Compute-0003.HELP.tex}

\[\lim_{x\to{9}}\dfrac{2 \, x^{2} - 20 \, x + 18}{x - 1}=\answer{0}\]
\end{problem}}%}

\latexProblemContent{
\ifVerboseLocation This is Derivative Compute Question 0003. \\ \fi
\begin{problem}

Determine if the limit approaches a finite number, $\pm\infty$, or does not exist. (If the limit does not exist, write DNE)

\input{Limit-Compute-0003.HELP.tex}

\[\lim_{x\to{4}}\dfrac{x^{2} - 9 \, x + 20}{x - 5}=\answer{0}\]
\end{problem}}%}

\latexProblemContent{
\ifVerboseLocation This is Derivative Compute Question 0003. \\ \fi
\begin{problem}

Determine if the limit approaches a finite number, $\pm\infty$, or does not exist. (If the limit does not exist, write DNE)

\input{Limit-Compute-0003.HELP.tex}

\[\lim_{x\to{8}}\dfrac{x^{2} + 2 \, x - 80}{x + 10}=\answer{0}\]
\end{problem}}%}

\latexProblemContent{
\ifVerboseLocation This is Derivative Compute Question 0003. \\ \fi
\begin{problem}

Determine if the limit approaches a finite number, $\pm\infty$, or does not exist. (If the limit does not exist, write DNE)

\input{Limit-Compute-0003.HELP.tex}

\[\lim_{x\to{-10}}\dfrac{x^{2} + 8 \, x - 20}{x - 2}=\answer{0}\]
\end{problem}}%}

\latexProblemContent{
\ifVerboseLocation This is Derivative Compute Question 0003. \\ \fi
\begin{problem}

Determine if the limit approaches a finite number, $\pm\infty$, or does not exist. (If the limit does not exist, write DNE)

\input{Limit-Compute-0003.HELP.tex}

\[\lim_{x\to{6}}\dfrac{2 \, x^{2} - 6 \, x - 36}{x + 3}=\answer{0}\]
\end{problem}}%}

\latexProblemContent{
\ifVerboseLocation This is Derivative Compute Question 0003. \\ \fi
\begin{problem}

Determine if the limit approaches a finite number, $\pm\infty$, or does not exist. (If the limit does not exist, write DNE)

\input{Limit-Compute-0003.HELP.tex}

\[\lim_{x\to{10}}\dfrac{3 \, x^{2} - 45 \, x + 150}{x - 5}=\answer{0}\]
\end{problem}}%}

\latexProblemContent{
\ifVerboseLocation This is Derivative Compute Question 0003. \\ \fi
\begin{problem}

Determine if the limit approaches a finite number, $\pm\infty$, or does not exist. (If the limit does not exist, write DNE)

\input{Limit-Compute-0003.HELP.tex}

\[\lim_{x\to{10}}\dfrac{x^{2} - 8 \, x - 20}{x + 2}=\answer{0}\]
\end{problem}}%}

\latexProblemContent{
\ifVerboseLocation This is Derivative Compute Question 0003. \\ \fi
\begin{problem}

Determine if the limit approaches a finite number, $\pm\infty$, or does not exist. (If the limit does not exist, write DNE)

\input{Limit-Compute-0003.HELP.tex}

\[\lim_{x\to{3}}\dfrac{3 \, x^{2} - 27 \, x + 54}{x - 6}=\answer{0}\]
\end{problem}}%}

\latexProblemContent{
\ifVerboseLocation This is Derivative Compute Question 0003. \\ \fi
\begin{problem}

Determine if the limit approaches a finite number, $\pm\infty$, or does not exist. (If the limit does not exist, write DNE)

\input{Limit-Compute-0003.HELP.tex}

\[\lim_{x\to{4}}\dfrac{x^{2} - 5 \, x + 4}{x - 1}=\answer{0}\]
\end{problem}}%}

\latexProblemContent{
\ifVerboseLocation This is Derivative Compute Question 0003. \\ \fi
\begin{problem}

Determine if the limit approaches a finite number, $\pm\infty$, or does not exist. (If the limit does not exist, write DNE)

\input{Limit-Compute-0003.HELP.tex}

\[\lim_{x\to{8}}\dfrac{x^{2} - 2 \, x - 48}{x + 6}=\answer{0}\]
\end{problem}}%}

\latexProblemContent{
\ifVerboseLocation This is Derivative Compute Question 0003. \\ \fi
\begin{problem}

Determine if the limit approaches a finite number, $\pm\infty$, or does not exist. (If the limit does not exist, write DNE)

\input{Limit-Compute-0003.HELP.tex}

\[\lim_{x\to{9}}\dfrac{2 \, x^{2} - 28 \, x + 90}{x - 5}=\answer{0}\]
\end{problem}}%}

\latexProblemContent{
\ifVerboseLocation This is Derivative Compute Question 0003. \\ \fi
\begin{problem}

Determine if the limit approaches a finite number, $\pm\infty$, or does not exist. (If the limit does not exist, write DNE)

\input{Limit-Compute-0003.HELP.tex}

\[\lim_{x\to{-10}}\dfrac{2 \, x^{2} + 14 \, x - 60}{x - 3}=\answer{0}\]
\end{problem}}%}

\latexProblemContent{
\ifVerboseLocation This is Derivative Compute Question 0003. \\ \fi
\begin{problem}

Determine if the limit approaches a finite number, $\pm\infty$, or does not exist. (If the limit does not exist, write DNE)

\input{Limit-Compute-0003.HELP.tex}

\[\lim_{x\to{-5}}\dfrac{2 \, x^{2} + 22 \, x + 60}{x + 6}=\answer{0}\]
\end{problem}}%}

\latexProblemContent{
\ifVerboseLocation This is Derivative Compute Question 0003. \\ \fi
\begin{problem}

Determine if the limit approaches a finite number, $\pm\infty$, or does not exist. (If the limit does not exist, write DNE)

\input{Limit-Compute-0003.HELP.tex}

\[\lim_{x\to{-8}}\dfrac{3 \, x^{2} + 45 \, x + 168}{x + 7}=\answer{0}\]
\end{problem}}%}

\latexProblemContent{
\ifVerboseLocation This is Derivative Compute Question 0003. \\ \fi
\begin{problem}

Determine if the limit approaches a finite number, $\pm\infty$, or does not exist. (If the limit does not exist, write DNE)

\input{Limit-Compute-0003.HELP.tex}

\[\lim_{x\to{1}}\dfrac{2 \, x^{2} + 16 \, x - 18}{x + 9}=\answer{0}\]
\end{problem}}%}

\latexProblemContent{
\ifVerboseLocation This is Derivative Compute Question 0003. \\ \fi
\begin{problem}

Determine if the limit approaches a finite number, $\pm\infty$, or does not exist. (If the limit does not exist, write DNE)

\input{Limit-Compute-0003.HELP.tex}

\[\lim_{x\to{5}}\dfrac{2 \, x^{2} - 4 \, x - 30}{x + 3}=\answer{0}\]
\end{problem}}%}

\latexProblemContent{
\ifVerboseLocation This is Derivative Compute Question 0003. \\ \fi
\begin{problem}

Determine if the limit approaches a finite number, $\pm\infty$, or does not exist. (If the limit does not exist, write DNE)

\input{Limit-Compute-0003.HELP.tex}

\[\lim_{x\to{5}}\dfrac{2 \, x^{2} + 4 \, x - 70}{x + 7}=\answer{0}\]
\end{problem}}%}

\latexProblemContent{
\ifVerboseLocation This is Derivative Compute Question 0003. \\ \fi
\begin{problem}

Determine if the limit approaches a finite number, $\pm\infty$, or does not exist. (If the limit does not exist, write DNE)

\input{Limit-Compute-0003.HELP.tex}

\[\lim_{x\to{-3}}\dfrac{2 \, x^{2} + 20 \, x + 42}{x + 7}=\answer{0}\]
\end{problem}}%}

\latexProblemContent{
\ifVerboseLocation This is Derivative Compute Question 0003. \\ \fi
\begin{problem}

Determine if the limit approaches a finite number, $\pm\infty$, or does not exist. (If the limit does not exist, write DNE)

\input{Limit-Compute-0003.HELP.tex}

\[\lim_{x\to{-5}}\dfrac{2 \, x^{2} + 16 \, x + 30}{x + 3}=\answer{0}\]
\end{problem}}%}

\latexProblemContent{
\ifVerboseLocation This is Derivative Compute Question 0003. \\ \fi
\begin{problem}

Determine if the limit approaches a finite number, $\pm\infty$, or does not exist. (If the limit does not exist, write DNE)

\input{Limit-Compute-0003.HELP.tex}

\[\lim_{x\to{3}}\dfrac{3 \, x^{2} + 15 \, x - 72}{x + 8}=\answer{0}\]
\end{problem}}%}

\latexProblemContent{
\ifVerboseLocation This is Derivative Compute Question 0003. \\ \fi
\begin{problem}

Determine if the limit approaches a finite number, $\pm\infty$, or does not exist. (If the limit does not exist, write DNE)

\input{Limit-Compute-0003.HELP.tex}

\[\lim_{x\to{-1}}\dfrac{3 \, x^{2} + 18 \, x + 15}{x + 5}=\answer{0}\]
\end{problem}}%}

\latexProblemContent{
\ifVerboseLocation This is Derivative Compute Question 0003. \\ \fi
\begin{problem}

Determine if the limit approaches a finite number, $\pm\infty$, or does not exist. (If the limit does not exist, write DNE)

\input{Limit-Compute-0003.HELP.tex}

\[\lim_{x\to{3}}\dfrac{3 \, x^{2} - 6 \, x - 9}{x + 1}=\answer{0}\]
\end{problem}}%}

\latexProblemContent{
\ifVerboseLocation This is Derivative Compute Question 0003. \\ \fi
\begin{problem}

Determine if the limit approaches a finite number, $\pm\infty$, or does not exist. (If the limit does not exist, write DNE)

\input{Limit-Compute-0003.HELP.tex}

\[\lim_{x\to{8}}\dfrac{2 \, x^{2} - 4 \, x - 96}{x + 6}=\answer{0}\]
\end{problem}}%}

\latexProblemContent{
\ifVerboseLocation This is Derivative Compute Question 0003. \\ \fi
\begin{problem}

Determine if the limit approaches a finite number, $\pm\infty$, or does not exist. (If the limit does not exist, write DNE)

\input{Limit-Compute-0003.HELP.tex}

\[\lim_{x\to{-1}}\dfrac{2 \, x^{2} + 8 \, x + 6}{x + 3}=\answer{0}\]
\end{problem}}%}

\latexProblemContent{
\ifVerboseLocation This is Derivative Compute Question 0003. \\ \fi
\begin{problem}

Determine if the limit approaches a finite number, $\pm\infty$, or does not exist. (If the limit does not exist, write DNE)

\input{Limit-Compute-0003.HELP.tex}

\[\lim_{x\to{9}}\dfrac{2 \, x^{2} - 38 \, x + 180}{x - 10}=\answer{0}\]
\end{problem}}%}

\latexProblemContent{
\ifVerboseLocation This is Derivative Compute Question 0003. \\ \fi
\begin{problem}

Determine if the limit approaches a finite number, $\pm\infty$, or does not exist. (If the limit does not exist, write DNE)

\input{Limit-Compute-0003.HELP.tex}

\[\lim_{x\to{7}}\dfrac{3 \, x^{2} - 9 \, x - 84}{x + 4}=\answer{0}\]
\end{problem}}%}

\latexProblemContent{
\ifVerboseLocation This is Derivative Compute Question 0003. \\ \fi
\begin{problem}

Determine if the limit approaches a finite number, $\pm\infty$, or does not exist. (If the limit does not exist, write DNE)

\input{Limit-Compute-0003.HELP.tex}

\[\lim_{x\to{-2}}\dfrac{x^{2} - 2 \, x - 8}{x - 4}=\answer{0}\]
\end{problem}}%}

\latexProblemContent{
\ifVerboseLocation This is Derivative Compute Question 0003. \\ \fi
\begin{problem}

Determine if the limit approaches a finite number, $\pm\infty$, or does not exist. (If the limit does not exist, write DNE)

\input{Limit-Compute-0003.HELP.tex}

\[\lim_{x\to{2}}\dfrac{x^{2} - 6 \, x + 8}{x - 4}=\answer{0}\]
\end{problem}}%}

\latexProblemContent{
\ifVerboseLocation This is Derivative Compute Question 0003. \\ \fi
\begin{problem}

Determine if the limit approaches a finite number, $\pm\infty$, or does not exist. (If the limit does not exist, write DNE)

\input{Limit-Compute-0003.HELP.tex}

\[\lim_{x\to{-3}}\dfrac{x^{2} - 4 \, x - 21}{x - 7}=\answer{0}\]
\end{problem}}%}

\latexProblemContent{
\ifVerboseLocation This is Derivative Compute Question 0003. \\ \fi
\begin{problem}

Determine if the limit approaches a finite number, $\pm\infty$, or does not exist. (If the limit does not exist, write DNE)

\input{Limit-Compute-0003.HELP.tex}

\[\lim_{x\to{9}}\dfrac{3 \, x^{2} + 3 \, x - 270}{x + 10}=\answer{0}\]
\end{problem}}%}

\latexProblemContent{
\ifVerboseLocation This is Derivative Compute Question 0003. \\ \fi
\begin{problem}

Determine if the limit approaches a finite number, $\pm\infty$, or does not exist. (If the limit does not exist, write DNE)

\input{Limit-Compute-0003.HELP.tex}

\[\lim_{x\to{2}}\dfrac{x^{2} + 4 \, x - 12}{x + 6}=\answer{0}\]
\end{problem}}%}

\latexProblemContent{
\ifVerboseLocation This is Derivative Compute Question 0003. \\ \fi
\begin{problem}

Determine if the limit approaches a finite number, $\pm\infty$, or does not exist. (If the limit does not exist, write DNE)

\input{Limit-Compute-0003.HELP.tex}

\[\lim_{x\to{-1}}\dfrac{3 \, x^{2} - 9 \, x - 12}{x - 4}=\answer{0}\]
\end{problem}}%}

\latexProblemContent{
\ifVerboseLocation This is Derivative Compute Question 0003. \\ \fi
\begin{problem}

Determine if the limit approaches a finite number, $\pm\infty$, or does not exist. (If the limit does not exist, write DNE)

\input{Limit-Compute-0003.HELP.tex}

\[\lim_{x\to{-8}}\dfrac{2 \, x^{2} + 18 \, x + 16}{x + 1}=\answer{0}\]
\end{problem}}%}

\latexProblemContent{
\ifVerboseLocation This is Derivative Compute Question 0003. \\ \fi
\begin{problem}

Determine if the limit approaches a finite number, $\pm\infty$, or does not exist. (If the limit does not exist, write DNE)

\input{Limit-Compute-0003.HELP.tex}

\[\lim_{x\to{10}}\dfrac{x^{2} - 5 \, x - 50}{x + 5}=\answer{0}\]
\end{problem}}%}

\latexProblemContent{
\ifVerboseLocation This is Derivative Compute Question 0003. \\ \fi
\begin{problem}

Determine if the limit approaches a finite number, $\pm\infty$, or does not exist. (If the limit does not exist, write DNE)

\input{Limit-Compute-0003.HELP.tex}

\[\lim_{x\to{1}}\dfrac{x^{2} - 9 \, x + 8}{x - 8}=\answer{0}\]
\end{problem}}%}

\latexProblemContent{
\ifVerboseLocation This is Derivative Compute Question 0003. \\ \fi
\begin{problem}

Determine if the limit approaches a finite number, $\pm\infty$, or does not exist. (If the limit does not exist, write DNE)

\input{Limit-Compute-0003.HELP.tex}

\[\lim_{x\to{2}}\dfrac{2 \, x^{2} + 4 \, x - 16}{x + 4}=\answer{0}\]
\end{problem}}%}

\latexProblemContent{
\ifVerboseLocation This is Derivative Compute Question 0003. \\ \fi
\begin{problem}

Determine if the limit approaches a finite number, $\pm\infty$, or does not exist. (If the limit does not exist, write DNE)

\input{Limit-Compute-0003.HELP.tex}

\[\lim_{x\to{-1}}\dfrac{3 \, x^{2} + 27 \, x + 24}{x + 8}=\answer{0}\]
\end{problem}}%}

\latexProblemContent{
\ifVerboseLocation This is Derivative Compute Question 0003. \\ \fi
\begin{problem}

Determine if the limit approaches a finite number, $\pm\infty$, or does not exist. (If the limit does not exist, write DNE)

\input{Limit-Compute-0003.HELP.tex}

\[\lim_{x\to{10}}\dfrac{2 \, x^{2} - 30 \, x + 100}{x - 5}=\answer{0}\]
\end{problem}}%}

\latexProblemContent{
\ifVerboseLocation This is Derivative Compute Question 0003. \\ \fi
\begin{problem}

Determine if the limit approaches a finite number, $\pm\infty$, or does not exist. (If the limit does not exist, write DNE)

\input{Limit-Compute-0003.HELP.tex}

\[\lim_{x\to{4}}\dfrac{3 \, x^{2} - 9 \, x - 12}{x + 1}=\answer{0}\]
\end{problem}}%}

\latexProblemContent{
\ifVerboseLocation This is Derivative Compute Question 0003. \\ \fi
\begin{problem}

Determine if the limit approaches a finite number, $\pm\infty$, or does not exist. (If the limit does not exist, write DNE)

\input{Limit-Compute-0003.HELP.tex}

\[\lim_{x\to{-3}}\dfrac{2 \, x^{2} - 18}{x - 3}=\answer{0}\]
\end{problem}}%}

\latexProblemContent{
\ifVerboseLocation This is Derivative Compute Question 0003. \\ \fi
\begin{problem}

Determine if the limit approaches a finite number, $\pm\infty$, or does not exist. (If the limit does not exist, write DNE)

\input{Limit-Compute-0003.HELP.tex}

\[\lim_{x\to{8}}\dfrac{2 \, x^{2} - 12 \, x - 32}{x + 2}=\answer{0}\]
\end{problem}}%}

\latexProblemContent{
\ifVerboseLocation This is Derivative Compute Question 0003. \\ \fi
\begin{problem}

Determine if the limit approaches a finite number, $\pm\infty$, or does not exist. (If the limit does not exist, write DNE)

\input{Limit-Compute-0003.HELP.tex}

\[\lim_{x\to{6}}\dfrac{3 \, x^{2} - 6 \, x - 72}{x + 4}=\answer{0}\]
\end{problem}}%}

\latexProblemContent{
\ifVerboseLocation This is Derivative Compute Question 0003. \\ \fi
\begin{problem}

Determine if the limit approaches a finite number, $\pm\infty$, or does not exist. (If the limit does not exist, write DNE)

\input{Limit-Compute-0003.HELP.tex}

\[\lim_{x\to{-1}}\dfrac{2 \, x^{2} - 4 \, x - 6}{x - 3}=\answer{0}\]
\end{problem}}%}

\latexProblemContent{
\ifVerboseLocation This is Derivative Compute Question 0003. \\ \fi
\begin{problem}

Determine if the limit approaches a finite number, $\pm\infty$, or does not exist. (If the limit does not exist, write DNE)

\input{Limit-Compute-0003.HELP.tex}

\[\lim_{x\to{-9}}\dfrac{3 \, x^{2} + 12 \, x - 135}{x - 5}=\answer{0}\]
\end{problem}}%}

\latexProblemContent{
\ifVerboseLocation This is Derivative Compute Question 0003. \\ \fi
\begin{problem}

Determine if the limit approaches a finite number, $\pm\infty$, or does not exist. (If the limit does not exist, write DNE)

\input{Limit-Compute-0003.HELP.tex}

\[\lim_{x\to{-2}}\dfrac{2 \, x^{2} + 24 \, x + 40}{x + 10}=\answer{0}\]
\end{problem}}%}

\latexProblemContent{
\ifVerboseLocation This is Derivative Compute Question 0003. \\ \fi
\begin{problem}

Determine if the limit approaches a finite number, $\pm\infty$, or does not exist. (If the limit does not exist, write DNE)

\input{Limit-Compute-0003.HELP.tex}

\[\lim_{x\to{8}}\dfrac{2 \, x^{2} - 34 \, x + 144}{x - 9}=\answer{0}\]
\end{problem}}%}

\latexProblemContent{
\ifVerboseLocation This is Derivative Compute Question 0003. \\ \fi
\begin{problem}

Determine if the limit approaches a finite number, $\pm\infty$, or does not exist. (If the limit does not exist, write DNE)

\input{Limit-Compute-0003.HELP.tex}

\[\lim_{x\to{-10}}\dfrac{3 \, x^{2} + 45 \, x + 150}{x + 5}=\answer{0}\]
\end{problem}}%}

\latexProblemContent{
\ifVerboseLocation This is Derivative Compute Question 0003. \\ \fi
\begin{problem}

Determine if the limit approaches a finite number, $\pm\infty$, or does not exist. (If the limit does not exist, write DNE)

\input{Limit-Compute-0003.HELP.tex}

\[\lim_{x\to{8}}\dfrac{3 \, x^{2} + 3 \, x - 216}{x + 9}=\answer{0}\]
\end{problem}}%}

\latexProblemContent{
\ifVerboseLocation This is Derivative Compute Question 0003. \\ \fi
\begin{problem}

Determine if the limit approaches a finite number, $\pm\infty$, or does not exist. (If the limit does not exist, write DNE)

\input{Limit-Compute-0003.HELP.tex}

\[\lim_{x\to{10}}\dfrac{x^{2} - x - 90}{x + 9}=\answer{0}\]
\end{problem}}%}

\latexProblemContent{
\ifVerboseLocation This is Derivative Compute Question 0003. \\ \fi
\begin{problem}

Determine if the limit approaches a finite number, $\pm\infty$, or does not exist. (If the limit does not exist, write DNE)

\input{Limit-Compute-0003.HELP.tex}

\[\lim_{x\to{4}}\dfrac{3 \, x^{2} + 9 \, x - 84}{x + 7}=\answer{0}\]
\end{problem}}%}

\latexProblemContent{
\ifVerboseLocation This is Derivative Compute Question 0003. \\ \fi
\begin{problem}

Determine if the limit approaches a finite number, $\pm\infty$, or does not exist. (If the limit does not exist, write DNE)

\input{Limit-Compute-0003.HELP.tex}

\[\lim_{x\to{6}}\dfrac{3 \, x^{2} - 45 \, x + 162}{x - 9}=\answer{0}\]
\end{problem}}%}

\latexProblemContent{
\ifVerboseLocation This is Derivative Compute Question 0003. \\ \fi
\begin{problem}

Determine if the limit approaches a finite number, $\pm\infty$, or does not exist. (If the limit does not exist, write DNE)

\input{Limit-Compute-0003.HELP.tex}

\[\lim_{x\to{5}}\dfrac{2 \, x^{2} + 10 \, x - 100}{x + 10}=\answer{0}\]
\end{problem}}%}

\latexProblemContent{
\ifVerboseLocation This is Derivative Compute Question 0003. \\ \fi
\begin{problem}

Determine if the limit approaches a finite number, $\pm\infty$, or does not exist. (If the limit does not exist, write DNE)

\input{Limit-Compute-0003.HELP.tex}

\[\lim_{x\to{3}}\dfrac{2 \, x^{2} + 2 \, x - 24}{x + 4}=\answer{0}\]
\end{problem}}%}

\latexProblemContent{
\ifVerboseLocation This is Derivative Compute Question 0003. \\ \fi
\begin{problem}

Determine if the limit approaches a finite number, $\pm\infty$, or does not exist. (If the limit does not exist, write DNE)

\input{Limit-Compute-0003.HELP.tex}

\[\lim_{x\to{-6}}\dfrac{x^{2} + 11 \, x + 30}{x + 5}=\answer{0}\]
\end{problem}}%}

\latexProblemContent{
\ifVerboseLocation This is Derivative Compute Question 0003. \\ \fi
\begin{problem}

Determine if the limit approaches a finite number, $\pm\infty$, or does not exist. (If the limit does not exist, write DNE)

\input{Limit-Compute-0003.HELP.tex}

\[\lim_{x\to{2}}\dfrac{2 \, x^{2} - 16 \, x + 24}{x - 6}=\answer{0}\]
\end{problem}}%}

\latexProblemContent{
\ifVerboseLocation This is Derivative Compute Question 0003. \\ \fi
\begin{problem}

Determine if the limit approaches a finite number, $\pm\infty$, or does not exist. (If the limit does not exist, write DNE)

\input{Limit-Compute-0003.HELP.tex}

\[\lim_{x\to{-6}}\dfrac{x^{2} + 9 \, x + 18}{x + 3}=\answer{0}\]
\end{problem}}%}

\latexProblemContent{
\ifVerboseLocation This is Derivative Compute Question 0003. \\ \fi
\begin{problem}

Determine if the limit approaches a finite number, $\pm\infty$, or does not exist. (If the limit does not exist, write DNE)

\input{Limit-Compute-0003.HELP.tex}

\[\lim_{x\to{-7}}\dfrac{x^{2} + 3 \, x - 28}{x - 4}=\answer{0}\]
\end{problem}}%}

\latexProblemContent{
\ifVerboseLocation This is Derivative Compute Question 0003. \\ \fi
\begin{problem}

Determine if the limit approaches a finite number, $\pm\infty$, or does not exist. (If the limit does not exist, write DNE)

\input{Limit-Compute-0003.HELP.tex}

\[\lim_{x\to{-4}}\dfrac{x^{2} - 6 \, x - 40}{x - 10}=\answer{0}\]
\end{problem}}%}

\latexProblemContent{
\ifVerboseLocation This is Derivative Compute Question 0003. \\ \fi
\begin{problem}

Determine if the limit approaches a finite number, $\pm\infty$, or does not exist. (If the limit does not exist, write DNE)

\input{Limit-Compute-0003.HELP.tex}

\[\lim_{x\to{7}}\dfrac{3 \, x^{2} - 15 \, x - 42}{x + 2}=\answer{0}\]
\end{problem}}%}

\latexProblemContent{
\ifVerboseLocation This is Derivative Compute Question 0003. \\ \fi
\begin{problem}

Determine if the limit approaches a finite number, $\pm\infty$, or does not exist. (If the limit does not exist, write DNE)

\input{Limit-Compute-0003.HELP.tex}

\[\lim_{x\to{6}}\dfrac{2 \, x^{2} - 8 \, x - 24}{x + 2}=\answer{0}\]
\end{problem}}%}

\latexProblemContent{
\ifVerboseLocation This is Derivative Compute Question 0003. \\ \fi
\begin{problem}

Determine if the limit approaches a finite number, $\pm\infty$, or does not exist. (If the limit does not exist, write DNE)

\input{Limit-Compute-0003.HELP.tex}

\[\lim_{x\to{6}}\dfrac{2 \, x^{2} - 22 \, x + 60}{x - 5}=\answer{0}\]
\end{problem}}%}

\latexProblemContent{
\ifVerboseLocation This is Derivative Compute Question 0003. \\ \fi
\begin{problem}

Determine if the limit approaches a finite number, $\pm\infty$, or does not exist. (If the limit does not exist, write DNE)

\input{Limit-Compute-0003.HELP.tex}

\[\lim_{x\to{3}}\dfrac{3 \, x^{2} - 21 \, x + 36}{x - 4}=\answer{0}\]
\end{problem}}%}

\latexProblemContent{
\ifVerboseLocation This is Derivative Compute Question 0003. \\ \fi
\begin{problem}

Determine if the limit approaches a finite number, $\pm\infty$, or does not exist. (If the limit does not exist, write DNE)

\input{Limit-Compute-0003.HELP.tex}

\[\lim_{x\to{8}}\dfrac{3 \, x^{2} - 39 \, x + 120}{x - 5}=\answer{0}\]
\end{problem}}%}

\latexProblemContent{
\ifVerboseLocation This is Derivative Compute Question 0003. \\ \fi
\begin{problem}

Determine if the limit approaches a finite number, $\pm\infty$, or does not exist. (If the limit does not exist, write DNE)

\input{Limit-Compute-0003.HELP.tex}

\[\lim_{x\to{10}}\dfrac{3 \, x^{2} - 57 \, x + 270}{x - 9}=\answer{0}\]
\end{problem}}%}

\latexProblemContent{
\ifVerboseLocation This is Derivative Compute Question 0003. \\ \fi
\begin{problem}

Determine if the limit approaches a finite number, $\pm\infty$, or does not exist. (If the limit does not exist, write DNE)

\input{Limit-Compute-0003.HELP.tex}

\[\lim_{x\to{5}}\dfrac{2 \, x^{2} + 8 \, x - 90}{x + 9}=\answer{0}\]
\end{problem}}%}

\latexProblemContent{
\ifVerboseLocation This is Derivative Compute Question 0003. \\ \fi
\begin{problem}

Determine if the limit approaches a finite number, $\pm\infty$, or does not exist. (If the limit does not exist, write DNE)

\input{Limit-Compute-0003.HELP.tex}

\[\lim_{x\to{-10}}\dfrac{3 \, x^{2} + 33 \, x + 30}{x + 1}=\answer{0}\]
\end{problem}}%}

\latexProblemContent{
\ifVerboseLocation This is Derivative Compute Question 0003. \\ \fi
\begin{problem}

Determine if the limit approaches a finite number, $\pm\infty$, or does not exist. (If the limit does not exist, write DNE)

\input{Limit-Compute-0003.HELP.tex}

\[\lim_{x\to{-8}}\dfrac{3 \, x^{2} + 30 \, x + 48}{x + 2}=\answer{0}\]
\end{problem}}%}

\latexProblemContent{
\ifVerboseLocation This is Derivative Compute Question 0003. \\ \fi
\begin{problem}

Determine if the limit approaches a finite number, $\pm\infty$, or does not exist. (If the limit does not exist, write DNE)

\input{Limit-Compute-0003.HELP.tex}

\[\lim_{x\to{-10}}\dfrac{3 \, x^{2} + 21 \, x - 90}{x - 3}=\answer{0}\]
\end{problem}}%}

\latexProblemContent{
\ifVerboseLocation This is Derivative Compute Question 0003. \\ \fi
\begin{problem}

Determine if the limit approaches a finite number, $\pm\infty$, or does not exist. (If the limit does not exist, write DNE)

\input{Limit-Compute-0003.HELP.tex}

\[\lim_{x\to{3}}\dfrac{3 \, x^{2} - 24 \, x + 45}{x - 5}=\answer{0}\]
\end{problem}}%}

\latexProblemContent{
\ifVerboseLocation This is Derivative Compute Question 0003. \\ \fi
\begin{problem}

Determine if the limit approaches a finite number, $\pm\infty$, or does not exist. (If the limit does not exist, write DNE)

\input{Limit-Compute-0003.HELP.tex}

\[\lim_{x\to{-10}}\dfrac{x^{2} + 7 \, x - 30}{x - 3}=\answer{0}\]
\end{problem}}%}

\latexProblemContent{
\ifVerboseLocation This is Derivative Compute Question 0003. \\ \fi
\begin{problem}

Determine if the limit approaches a finite number, $\pm\infty$, or does not exist. (If the limit does not exist, write DNE)

\input{Limit-Compute-0003.HELP.tex}

\[\lim_{x\to{10}}\dfrac{2 \, x^{2} - 4 \, x - 160}{x + 8}=\answer{0}\]
\end{problem}}%}

\latexProblemContent{
\ifVerboseLocation This is Derivative Compute Question 0003. \\ \fi
\begin{problem}

Determine if the limit approaches a finite number, $\pm\infty$, or does not exist. (If the limit does not exist, write DNE)

\input{Limit-Compute-0003.HELP.tex}

\[\lim_{x\to{10}}\dfrac{x^{2} - 100}{x + 10}=\answer{0}\]
\end{problem}}%}

\latexProblemContent{
\ifVerboseLocation This is Derivative Compute Question 0003. \\ \fi
\begin{problem}

Determine if the limit approaches a finite number, $\pm\infty$, or does not exist. (If the limit does not exist, write DNE)

\input{Limit-Compute-0003.HELP.tex}

\[\lim_{x\to{-7}}\dfrac{x^{2} + 4 \, x - 21}{x - 3}=\answer{0}\]
\end{problem}}%}

\latexProblemContent{
\ifVerboseLocation This is Derivative Compute Question 0003. \\ \fi
\begin{problem}

Determine if the limit approaches a finite number, $\pm\infty$, or does not exist. (If the limit does not exist, write DNE)

\input{Limit-Compute-0003.HELP.tex}

\[\lim_{x\to{1}}\dfrac{2 \, x^{2} + 8 \, x - 10}{x + 5}=\answer{0}\]
\end{problem}}%}

\latexProblemContent{
\ifVerboseLocation This is Derivative Compute Question 0003. \\ \fi
\begin{problem}

Determine if the limit approaches a finite number, $\pm\infty$, or does not exist. (If the limit does not exist, write DNE)

\input{Limit-Compute-0003.HELP.tex}

\[\lim_{x\to{-7}}\dfrac{2 \, x^{2} + 4 \, x - 70}{x - 5}=\answer{0}\]
\end{problem}}%}

\latexProblemContent{
\ifVerboseLocation This is Derivative Compute Question 0003. \\ \fi
\begin{problem}

Determine if the limit approaches a finite number, $\pm\infty$, or does not exist. (If the limit does not exist, write DNE)

\input{Limit-Compute-0003.HELP.tex}

\[\lim_{x\to{-3}}\dfrac{2 \, x^{2} + 22 \, x + 48}{x + 8}=\answer{0}\]
\end{problem}}%}

\latexProblemContent{
\ifVerboseLocation This is Derivative Compute Question 0003. \\ \fi
\begin{problem}

Determine if the limit approaches a finite number, $\pm\infty$, or does not exist. (If the limit does not exist, write DNE)

\input{Limit-Compute-0003.HELP.tex}

\[\lim_{x\to{-9}}\dfrac{3 \, x^{2} + 3 \, x - 216}{x - 8}=\answer{0}\]
\end{problem}}%}

\latexProblemContent{
\ifVerboseLocation This is Derivative Compute Question 0003. \\ \fi
\begin{problem}

Determine if the limit approaches a finite number, $\pm\infty$, or does not exist. (If the limit does not exist, write DNE)

\input{Limit-Compute-0003.HELP.tex}

\[\lim_{x\to{3}}\dfrac{x^{2} + 5 \, x - 24}{x + 8}=\answer{0}\]
\end{problem}}%}

\latexProblemContent{
\ifVerboseLocation This is Derivative Compute Question 0003. \\ \fi
\begin{problem}

Determine if the limit approaches a finite number, $\pm\infty$, or does not exist. (If the limit does not exist, write DNE)

\input{Limit-Compute-0003.HELP.tex}

\[\lim_{x\to{-6}}\dfrac{2 \, x^{2} - 8 \, x - 120}{x - 10}=\answer{0}\]
\end{problem}}%}

\latexProblemContent{
\ifVerboseLocation This is Derivative Compute Question 0003. \\ \fi
\begin{problem}

Determine if the limit approaches a finite number, $\pm\infty$, or does not exist. (If the limit does not exist, write DNE)

\input{Limit-Compute-0003.HELP.tex}

\[\lim_{x\to{8}}\dfrac{3 \, x^{2} - 12 \, x - 96}{x + 4}=\answer{0}\]
\end{problem}}%}

\latexProblemContent{
\ifVerboseLocation This is Derivative Compute Question 0003. \\ \fi
\begin{problem}

Determine if the limit approaches a finite number, $\pm\infty$, or does not exist. (If the limit does not exist, write DNE)

\input{Limit-Compute-0003.HELP.tex}

\[\lim_{x\to{-10}}\dfrac{2 \, x^{2} + 18 \, x - 20}{x - 1}=\answer{0}\]
\end{problem}}%}

\latexProblemContent{
\ifVerboseLocation This is Derivative Compute Question 0003. \\ \fi
\begin{problem}

Determine if the limit approaches a finite number, $\pm\infty$, or does not exist. (If the limit does not exist, write DNE)

\input{Limit-Compute-0003.HELP.tex}

\[\lim_{x\to{-1}}\dfrac{3 \, x^{2} - 27 \, x - 30}{x - 10}=\answer{0}\]
\end{problem}}%}

\latexProblemContent{
\ifVerboseLocation This is Derivative Compute Question 0003. \\ \fi
\begin{problem}

Determine if the limit approaches a finite number, $\pm\infty$, or does not exist. (If the limit does not exist, write DNE)

\input{Limit-Compute-0003.HELP.tex}

\[\lim_{x\to{-9}}\dfrac{2 \, x^{2} + 34 \, x + 144}{x + 8}=\answer{0}\]
\end{problem}}%}

\latexProblemContent{
\ifVerboseLocation This is Derivative Compute Question 0003. \\ \fi
\begin{problem}

Determine if the limit approaches a finite number, $\pm\infty$, or does not exist. (If the limit does not exist, write DNE)

\input{Limit-Compute-0003.HELP.tex}

\[\lim_{x\to{3}}\dfrac{2 \, x^{2} - 18 \, x + 36}{x - 6}=\answer{0}\]
\end{problem}}%}

\latexProblemContent{
\ifVerboseLocation This is Derivative Compute Question 0003. \\ \fi
\begin{problem}

Determine if the limit approaches a finite number, $\pm\infty$, or does not exist. (If the limit does not exist, write DNE)

\input{Limit-Compute-0003.HELP.tex}

\[\lim_{x\to{-8}}\dfrac{2 \, x^{2} + 20 \, x + 32}{x + 2}=\answer{0}\]
\end{problem}}%}

\latexProblemContent{
\ifVerboseLocation This is Derivative Compute Question 0003. \\ \fi
\begin{problem}

Determine if the limit approaches a finite number, $\pm\infty$, or does not exist. (If the limit does not exist, write DNE)

\input{Limit-Compute-0003.HELP.tex}

\[\lim_{x\to{-3}}\dfrac{3 \, x^{2} - 3 \, x - 36}{x - 4}=\answer{0}\]
\end{problem}}%}

\latexProblemContent{
\ifVerboseLocation This is Derivative Compute Question 0003. \\ \fi
\begin{problem}

Determine if the limit approaches a finite number, $\pm\infty$, or does not exist. (If the limit does not exist, write DNE)

\input{Limit-Compute-0003.HELP.tex}

\[\lim_{x\to{6}}\dfrac{x^{2} - 16 \, x + 60}{x - 10}=\answer{0}\]
\end{problem}}%}

\latexProblemContent{
\ifVerboseLocation This is Derivative Compute Question 0003. \\ \fi
\begin{problem}

Determine if the limit approaches a finite number, $\pm\infty$, or does not exist. (If the limit does not exist, write DNE)

\input{Limit-Compute-0003.HELP.tex}

\[\lim_{x\to{-9}}\dfrac{x^{2} + 12 \, x + 27}{x + 3}=\answer{0}\]
\end{problem}}%}

\latexProblemContent{
\ifVerboseLocation This is Derivative Compute Question 0003. \\ \fi
\begin{problem}

Determine if the limit approaches a finite number, $\pm\infty$, or does not exist. (If the limit does not exist, write DNE)

\input{Limit-Compute-0003.HELP.tex}

\[\lim_{x\to{-10}}\dfrac{2 \, x^{2} + 4 \, x - 160}{x - 8}=\answer{0}\]
\end{problem}}%}

\latexProblemContent{
\ifVerboseLocation This is Derivative Compute Question 0003. \\ \fi
\begin{problem}

Determine if the limit approaches a finite number, $\pm\infty$, or does not exist. (If the limit does not exist, write DNE)

\input{Limit-Compute-0003.HELP.tex}

\[\lim_{x\to{-5}}\dfrac{3 \, x^{2} + 45 \, x + 150}{x + 10}=\answer{0}\]
\end{problem}}%}

\latexProblemContent{
\ifVerboseLocation This is Derivative Compute Question 0003. \\ \fi
\begin{problem}

Determine if the limit approaches a finite number, $\pm\infty$, or does not exist. (If the limit does not exist, write DNE)

\input{Limit-Compute-0003.HELP.tex}

\[\lim_{x\to{-8}}\dfrac{2 \, x^{2} - 2 \, x - 144}{x - 9}=\answer{0}\]
\end{problem}}%}

\latexProblemContent{
\ifVerboseLocation This is Derivative Compute Question 0003. \\ \fi
\begin{problem}

Determine if the limit approaches a finite number, $\pm\infty$, or does not exist. (If the limit does not exist, write DNE)

\input{Limit-Compute-0003.HELP.tex}

\[\lim_{x\to{-9}}\dfrac{2 \, x^{2} + 22 \, x + 36}{x + 2}=\answer{0}\]
\end{problem}}%}

\latexProblemContent{
\ifVerboseLocation This is Derivative Compute Question 0003. \\ \fi
\begin{problem}

Determine if the limit approaches a finite number, $\pm\infty$, or does not exist. (If the limit does not exist, write DNE)

\input{Limit-Compute-0003.HELP.tex}

\[\lim_{x\to{-2}}\dfrac{3 \, x^{2} + 9 \, x + 6}{x + 1}=\answer{0}\]
\end{problem}}%}

\latexProblemContent{
\ifVerboseLocation This is Derivative Compute Question 0003. \\ \fi
\begin{problem}

Determine if the limit approaches a finite number, $\pm\infty$, or does not exist. (If the limit does not exist, write DNE)

\input{Limit-Compute-0003.HELP.tex}

\[\lim_{x\to{-2}}\dfrac{3 \, x^{2} - 9 \, x - 30}{x - 5}=\answer{0}\]
\end{problem}}%}

\latexProblemContent{
\ifVerboseLocation This is Derivative Compute Question 0003. \\ \fi
\begin{problem}

Determine if the limit approaches a finite number, $\pm\infty$, or does not exist. (If the limit does not exist, write DNE)

\input{Limit-Compute-0003.HELP.tex}

\[\lim_{x\to{-5}}\dfrac{3 \, x^{2} - 12 \, x - 135}{x - 9}=\answer{0}\]
\end{problem}}%}

\latexProblemContent{
\ifVerboseLocation This is Derivative Compute Question 0003. \\ \fi
\begin{problem}

Determine if the limit approaches a finite number, $\pm\infty$, or does not exist. (If the limit does not exist, write DNE)

\input{Limit-Compute-0003.HELP.tex}

\[\lim_{x\to{2}}\dfrac{x^{2} + 6 \, x - 16}{x + 8}=\answer{0}\]
\end{problem}}%}

\latexProblemContent{
\ifVerboseLocation This is Derivative Compute Question 0003. \\ \fi
\begin{problem}

Determine if the limit approaches a finite number, $\pm\infty$, or does not exist. (If the limit does not exist, write DNE)

\input{Limit-Compute-0003.HELP.tex}

\[\lim_{x\to{10}}\dfrac{x^{2} - 13 \, x + 30}{x - 3}=\answer{0}\]
\end{problem}}%}

\latexProblemContent{
\ifVerboseLocation This is Derivative Compute Question 0003. \\ \fi
\begin{problem}

Determine if the limit approaches a finite number, $\pm\infty$, or does not exist. (If the limit does not exist, write DNE)

\input{Limit-Compute-0003.HELP.tex}

\[\lim_{x\to{10}}\dfrac{x^{2} - 3 \, x - 70}{x + 7}=\answer{0}\]
\end{problem}}%}

\latexProblemContent{
\ifVerboseLocation This is Derivative Compute Question 0003. \\ \fi
\begin{problem}

Determine if the limit approaches a finite number, $\pm\infty$, or does not exist. (If the limit does not exist, write DNE)

\input{Limit-Compute-0003.HELP.tex}

\[\lim_{x\to{6}}\dfrac{3 \, x^{2} - 15 \, x - 18}{x + 1}=\answer{0}\]
\end{problem}}%}

\latexProblemContent{
\ifVerboseLocation This is Derivative Compute Question 0003. \\ \fi
\begin{problem}

Determine if the limit approaches a finite number, $\pm\infty$, or does not exist. (If the limit does not exist, write DNE)

\input{Limit-Compute-0003.HELP.tex}

\[\lim_{x\to{2}}\dfrac{x^{2} - 10 \, x + 16}{x - 8}=\answer{0}\]
\end{problem}}%}

\latexProblemContent{
\ifVerboseLocation This is Derivative Compute Question 0003. \\ \fi
\begin{problem}

Determine if the limit approaches a finite number, $\pm\infty$, or does not exist. (If the limit does not exist, write DNE)

\input{Limit-Compute-0003.HELP.tex}

\[\lim_{x\to{-7}}\dfrac{3 \, x^{2} + 3 \, x - 126}{x - 6}=\answer{0}\]
\end{problem}}%}

\latexProblemContent{
\ifVerboseLocation This is Derivative Compute Question 0003. \\ \fi
\begin{problem}

Determine if the limit approaches a finite number, $\pm\infty$, or does not exist. (If the limit does not exist, write DNE)

\input{Limit-Compute-0003.HELP.tex}

\[\lim_{x\to{10}}\dfrac{x^{2} - 4 \, x - 60}{x + 6}=\answer{0}\]
\end{problem}}%}

\latexProblemContent{
\ifVerboseLocation This is Derivative Compute Question 0003. \\ \fi
\begin{problem}

Determine if the limit approaches a finite number, $\pm\infty$, or does not exist. (If the limit does not exist, write DNE)

\input{Limit-Compute-0003.HELP.tex}

\[\lim_{x\to{-5}}\dfrac{3 \, x^{2} + 6 \, x - 45}{x - 3}=\answer{0}\]
\end{problem}}%}

\latexProblemContent{
\ifVerboseLocation This is Derivative Compute Question 0003. \\ \fi
\begin{problem}

Determine if the limit approaches a finite number, $\pm\infty$, or does not exist. (If the limit does not exist, write DNE)

\input{Limit-Compute-0003.HELP.tex}

\[\lim_{x\to{-7}}\dfrac{2 \, x^{2} - 6 \, x - 140}{x - 10}=\answer{0}\]
\end{problem}}%}

\latexProblemContent{
\ifVerboseLocation This is Derivative Compute Question 0003. \\ \fi
\begin{problem}

Determine if the limit approaches a finite number, $\pm\infty$, or does not exist. (If the limit does not exist, write DNE)

\input{Limit-Compute-0003.HELP.tex}

\[\lim_{x\to{-6}}\dfrac{x^{2} - x - 42}{x - 7}=\answer{0}\]
\end{problem}}%}

\latexProblemContent{
\ifVerboseLocation This is Derivative Compute Question 0003. \\ \fi
\begin{problem}

Determine if the limit approaches a finite number, $\pm\infty$, or does not exist. (If the limit does not exist, write DNE)

\input{Limit-Compute-0003.HELP.tex}

\[\lim_{x\to{1}}\dfrac{2 \, x^{2} + 14 \, x - 16}{x + 8}=\answer{0}\]
\end{problem}}%}

\latexProblemContent{
\ifVerboseLocation This is Derivative Compute Question 0003. \\ \fi
\begin{problem}

Determine if the limit approaches a finite number, $\pm\infty$, or does not exist. (If the limit does not exist, write DNE)

\input{Limit-Compute-0003.HELP.tex}

\[\lim_{x\to{6}}\dfrac{3 \, x^{2} - 108}{x + 6}=\answer{0}\]
\end{problem}}%}

\latexProblemContent{
\ifVerboseLocation This is Derivative Compute Question 0003. \\ \fi
\begin{problem}

Determine if the limit approaches a finite number, $\pm\infty$, or does not exist. (If the limit does not exist, write DNE)

\input{Limit-Compute-0003.HELP.tex}

\[\lim_{x\to{-4}}\dfrac{2 \, x^{2} - 6 \, x - 56}{x - 7}=\answer{0}\]
\end{problem}}%}

\latexProblemContent{
\ifVerboseLocation This is Derivative Compute Question 0003. \\ \fi
\begin{problem}

Determine if the limit approaches a finite number, $\pm\infty$, or does not exist. (If the limit does not exist, write DNE)

\input{Limit-Compute-0003.HELP.tex}

\[\lim_{x\to{8}}\dfrac{3 \, x^{2} - 54 \, x + 240}{x - 10}=\answer{0}\]
\end{problem}}%}

\latexProblemContent{
\ifVerboseLocation This is Derivative Compute Question 0003. \\ \fi
\begin{problem}

Determine if the limit approaches a finite number, $\pm\infty$, or does not exist. (If the limit does not exist, write DNE)

\input{Limit-Compute-0003.HELP.tex}

\[\lim_{x\to{-6}}\dfrac{3 \, x^{2} - 9 \, x - 162}{x - 9}=\answer{0}\]
\end{problem}}%}

\latexProblemContent{
\ifVerboseLocation This is Derivative Compute Question 0003. \\ \fi
\begin{problem}

Determine if the limit approaches a finite number, $\pm\infty$, or does not exist. (If the limit does not exist, write DNE)

\input{Limit-Compute-0003.HELP.tex}

\[\lim_{x\to{10}}\dfrac{3 \, x^{2} - 15 \, x - 150}{x + 5}=\answer{0}\]
\end{problem}}%}

\latexProblemContent{
\ifVerboseLocation This is Derivative Compute Question 0003. \\ \fi
\begin{problem}

Determine if the limit approaches a finite number, $\pm\infty$, or does not exist. (If the limit does not exist, write DNE)

\input{Limit-Compute-0003.HELP.tex}

\[\lim_{x\to{9}}\dfrac{3 \, x^{2} - 39 \, x + 108}{x - 4}=\answer{0}\]
\end{problem}}%}

\latexProblemContent{
\ifVerboseLocation This is Derivative Compute Question 0003. \\ \fi
\begin{problem}

Determine if the limit approaches a finite number, $\pm\infty$, or does not exist. (If the limit does not exist, write DNE)

\input{Limit-Compute-0003.HELP.tex}

\[\lim_{x\to{3}}\dfrac{3 \, x^{2} - 15 \, x + 18}{x - 2}=\answer{0}\]
\end{problem}}%}

\latexProblemContent{
\ifVerboseLocation This is Derivative Compute Question 0003. \\ \fi
\begin{problem}

Determine if the limit approaches a finite number, $\pm\infty$, or does not exist. (If the limit does not exist, write DNE)

\input{Limit-Compute-0003.HELP.tex}

\[\lim_{x\to{-7}}\dfrac{2 \, x^{2} - 2 \, x - 112}{x - 8}=\answer{0}\]
\end{problem}}%}

\latexProblemContent{
\ifVerboseLocation This is Derivative Compute Question 0003. \\ \fi
\begin{problem}

Determine if the limit approaches a finite number, $\pm\infty$, or does not exist. (If the limit does not exist, write DNE)

\input{Limit-Compute-0003.HELP.tex}

\[\lim_{x\to{7}}\dfrac{2 \, x^{2} - 2 \, x - 84}{x + 6}=\answer{0}\]
\end{problem}}%}

\latexProblemContent{
\ifVerboseLocation This is Derivative Compute Question 0003. \\ \fi
\begin{problem}

Determine if the limit approaches a finite number, $\pm\infty$, or does not exist. (If the limit does not exist, write DNE)

\input{Limit-Compute-0003.HELP.tex}

\[\lim_{x\to{-9}}\dfrac{2 \, x^{2} + 14 \, x - 36}{x - 2}=\answer{0}\]
\end{problem}}%}

\latexProblemContent{
\ifVerboseLocation This is Derivative Compute Question 0003. \\ \fi
\begin{problem}

Determine if the limit approaches a finite number, $\pm\infty$, or does not exist. (If the limit does not exist, write DNE)

\input{Limit-Compute-0003.HELP.tex}

\[\lim_{x\to{-10}}\dfrac{2 \, x^{2} - 200}{x - 10}=\answer{0}\]
\end{problem}}%}

\latexProblemContent{
\ifVerboseLocation This is Derivative Compute Question 0003. \\ \fi
\begin{problem}

Determine if the limit approaches a finite number, $\pm\infty$, or does not exist. (If the limit does not exist, write DNE)

\input{Limit-Compute-0003.HELP.tex}

\[\lim_{x\to{8}}\dfrac{2 \, x^{2} - 2 \, x - 112}{x + 7}=\answer{0}\]
\end{problem}}%}

\latexProblemContent{
\ifVerboseLocation This is Derivative Compute Question 0003. \\ \fi
\begin{problem}

Determine if the limit approaches a finite number, $\pm\infty$, or does not exist. (If the limit does not exist, write DNE)

\input{Limit-Compute-0003.HELP.tex}

\[\lim_{x\to{-6}}\dfrac{3 \, x^{2} + 27 \, x + 54}{x + 3}=\answer{0}\]
\end{problem}}%}

\latexProblemContent{
\ifVerboseLocation This is Derivative Compute Question 0003. \\ \fi
\begin{problem}

Determine if the limit approaches a finite number, $\pm\infty$, or does not exist. (If the limit does not exist, write DNE)

\input{Limit-Compute-0003.HELP.tex}

\[\lim_{x\to{-10}}\dfrac{3 \, x^{2} + 51 \, x + 210}{x + 7}=\answer{0}\]
\end{problem}}%}

\latexProblemContent{
\ifVerboseLocation This is Derivative Compute Question 0003. \\ \fi
\begin{problem}

Determine if the limit approaches a finite number, $\pm\infty$, or does not exist. (If the limit does not exist, write DNE)

\input{Limit-Compute-0003.HELP.tex}

\[\lim_{x\to{4}}\dfrac{3 \, x^{2} - 6 \, x - 24}{x + 2}=\answer{0}\]
\end{problem}}%}

\latexProblemContent{
\ifVerboseLocation This is Derivative Compute Question 0003. \\ \fi
\begin{problem}

Determine if the limit approaches a finite number, $\pm\infty$, or does not exist. (If the limit does not exist, write DNE)

\input{Limit-Compute-0003.HELP.tex}

\[\lim_{x\to{-7}}\dfrac{2 \, x^{2} + 16 \, x + 14}{x + 1}=\answer{0}\]
\end{problem}}%}

\latexProblemContent{
\ifVerboseLocation This is Derivative Compute Question 0003. \\ \fi
\begin{problem}

Determine if the limit approaches a finite number, $\pm\infty$, or does not exist. (If the limit does not exist, write DNE)

\input{Limit-Compute-0003.HELP.tex}

\[\lim_{x\to{9}}\dfrac{x^{2} - 14 \, x + 45}{x - 5}=\answer{0}\]
\end{problem}}%}

\latexProblemContent{
\ifVerboseLocation This is Derivative Compute Question 0003. \\ \fi
\begin{problem}

Determine if the limit approaches a finite number, $\pm\infty$, or does not exist. (If the limit does not exist, write DNE)

\input{Limit-Compute-0003.HELP.tex}

\[\lim_{x\to{-3}}\dfrac{3 \, x^{2} + 36 \, x + 81}{x + 9}=\answer{0}\]
\end{problem}}%}

\latexProblemContent{
\ifVerboseLocation This is Derivative Compute Question 0003. \\ \fi
\begin{problem}

Determine if the limit approaches a finite number, $\pm\infty$, or does not exist. (If the limit does not exist, write DNE)

\input{Limit-Compute-0003.HELP.tex}

\[\lim_{x\to{-6}}\dfrac{3 \, x^{2} - 6 \, x - 144}{x - 8}=\answer{0}\]
\end{problem}}%}

\latexProblemContent{
\ifVerboseLocation This is Derivative Compute Question 0003. \\ \fi
\begin{problem}

Determine if the limit approaches a finite number, $\pm\infty$, or does not exist. (If the limit does not exist, write DNE)

\input{Limit-Compute-0003.HELP.tex}

\[\lim_{x\to{-9}}\dfrac{x^{2} + 16 \, x + 63}{x + 7}=\answer{0}\]
\end{problem}}%}

\latexProblemContent{
\ifVerboseLocation This is Derivative Compute Question 0003. \\ \fi
\begin{problem}

Determine if the limit approaches a finite number, $\pm\infty$, or does not exist. (If the limit does not exist, write DNE)

\input{Limit-Compute-0003.HELP.tex}

\[\lim_{x\to{-7}}\dfrac{3 \, x^{2} - 147}{x - 7}=\answer{0}\]
\end{problem}}%}

\latexProblemContent{
\ifVerboseLocation This is Derivative Compute Question 0003. \\ \fi
\begin{problem}

Determine if the limit approaches a finite number, $\pm\infty$, or does not exist. (If the limit does not exist, write DNE)

\input{Limit-Compute-0003.HELP.tex}

\[\lim_{x\to{-3}}\dfrac{3 \, x^{2} + 12 \, x + 9}{x + 1}=\answer{0}\]
\end{problem}}%}

\latexProblemContent{
\ifVerboseLocation This is Derivative Compute Question 0003. \\ \fi
\begin{problem}

Determine if the limit approaches a finite number, $\pm\infty$, or does not exist. (If the limit does not exist, write DNE)

\input{Limit-Compute-0003.HELP.tex}

\[\lim_{x\to{6}}\dfrac{3 \, x^{2} - 3 \, x - 90}{x + 5}=\answer{0}\]
\end{problem}}%}

\latexProblemContent{
\ifVerboseLocation This is Derivative Compute Question 0003. \\ \fi
\begin{problem}

Determine if the limit approaches a finite number, $\pm\infty$, or does not exist. (If the limit does not exist, write DNE)

\input{Limit-Compute-0003.HELP.tex}

\[\lim_{x\to{6}}\dfrac{3 \, x^{2} - 12 \, x - 36}{x + 2}=\answer{0}\]
\end{problem}}%}

\latexProblemContent{
\ifVerboseLocation This is Derivative Compute Question 0003. \\ \fi
\begin{problem}

Determine if the limit approaches a finite number, $\pm\infty$, or does not exist. (If the limit does not exist, write DNE)

\input{Limit-Compute-0003.HELP.tex}

\[\lim_{x\to{-6}}\dfrac{x^{2} + 14 \, x + 48}{x + 8}=\answer{0}\]
\end{problem}}%}

\latexProblemContent{
\ifVerboseLocation This is Derivative Compute Question 0003. \\ \fi
\begin{problem}

Determine if the limit approaches a finite number, $\pm\infty$, or does not exist. (If the limit does not exist, write DNE)

\input{Limit-Compute-0003.HELP.tex}

\[\lim_{x\to{1}}\dfrac{3 \, x^{2} + 18 \, x - 21}{x + 7}=\answer{0}\]
\end{problem}}%}

\latexProblemContent{
\ifVerboseLocation This is Derivative Compute Question 0003. \\ \fi
\begin{problem}

Determine if the limit approaches a finite number, $\pm\infty$, or does not exist. (If the limit does not exist, write DNE)

\input{Limit-Compute-0003.HELP.tex}

\[\lim_{x\to{5}}\dfrac{2 \, x^{2} + 6 \, x - 80}{x + 8}=\answer{0}\]
\end{problem}}%}

\latexProblemContent{
\ifVerboseLocation This is Derivative Compute Question 0003. \\ \fi
\begin{problem}

Determine if the limit approaches a finite number, $\pm\infty$, or does not exist. (If the limit does not exist, write DNE)

\input{Limit-Compute-0003.HELP.tex}

\[\lim_{x\to{9}}\dfrac{x^{2} - 6 \, x - 27}{x + 3}=\answer{0}\]
\end{problem}}%}

\latexProblemContent{
\ifVerboseLocation This is Derivative Compute Question 0003. \\ \fi
\begin{problem}

Determine if the limit approaches a finite number, $\pm\infty$, or does not exist. (If the limit does not exist, write DNE)

\input{Limit-Compute-0003.HELP.tex}

\[\lim_{x\to{6}}\dfrac{2 \, x^{2} - 4 \, x - 48}{x + 4}=\answer{0}\]
\end{problem}}%}

\latexProblemContent{
\ifVerboseLocation This is Derivative Compute Question 0003. \\ \fi
\begin{problem}

Determine if the limit approaches a finite number, $\pm\infty$, or does not exist. (If the limit does not exist, write DNE)

\input{Limit-Compute-0003.HELP.tex}

\[\lim_{x\to{-8}}\dfrac{2 \, x^{2} + 24 \, x + 64}{x + 4}=\answer{0}\]
\end{problem}}%}

\latexProblemContent{
\ifVerboseLocation This is Derivative Compute Question 0003. \\ \fi
\begin{problem}

Determine if the limit approaches a finite number, $\pm\infty$, or does not exist. (If the limit does not exist, write DNE)

\input{Limit-Compute-0003.HELP.tex}

\[\lim_{x\to{-9}}\dfrac{3 \, x^{2} + 45 \, x + 162}{x + 6}=\answer{0}\]
\end{problem}}%}

\latexProblemContent{
\ifVerboseLocation This is Derivative Compute Question 0003. \\ \fi
\begin{problem}

Determine if the limit approaches a finite number, $\pm\infty$, or does not exist. (If the limit does not exist, write DNE)

\input{Limit-Compute-0003.HELP.tex}

\[\lim_{x\to{-3}}\dfrac{x^{2} + 5 \, x + 6}{x + 2}=\answer{0}\]
\end{problem}}%}

\latexProblemContent{
\ifVerboseLocation This is Derivative Compute Question 0003. \\ \fi
\begin{problem}

Determine if the limit approaches a finite number, $\pm\infty$, or does not exist. (If the limit does not exist, write DNE)

\input{Limit-Compute-0003.HELP.tex}

\[\lim_{x\to{-7}}\dfrac{2 \, x^{2} + 12 \, x - 14}{x - 1}=\answer{0}\]
\end{problem}}%}

\latexProblemContent{
\ifVerboseLocation This is Derivative Compute Question 0003. \\ \fi
\begin{problem}

Determine if the limit approaches a finite number, $\pm\infty$, or does not exist. (If the limit does not exist, write DNE)

\input{Limit-Compute-0003.HELP.tex}

\[\lim_{x\to{-10}}\dfrac{x^{2} + 17 \, x + 70}{x + 7}=\answer{0}\]
\end{problem}}%}

\latexProblemContent{
\ifVerboseLocation This is Derivative Compute Question 0003. \\ \fi
\begin{problem}

Determine if the limit approaches a finite number, $\pm\infty$, or does not exist. (If the limit does not exist, write DNE)

\input{Limit-Compute-0003.HELP.tex}

\[\lim_{x\to{6}}\dfrac{x^{2} - 2 \, x - 24}{x + 4}=\answer{0}\]
\end{problem}}%}

\latexProblemContent{
\ifVerboseLocation This is Derivative Compute Question 0003. \\ \fi
\begin{problem}

Determine if the limit approaches a finite number, $\pm\infty$, or does not exist. (If the limit does not exist, write DNE)

\input{Limit-Compute-0003.HELP.tex}

\[\lim_{x\to{10}}\dfrac{3 \, x^{2} - 9 \, x - 210}{x + 7}=\answer{0}\]
\end{problem}}%}

\latexProblemContent{
\ifVerboseLocation This is Derivative Compute Question 0003. \\ \fi
\begin{problem}

Determine if the limit approaches a finite number, $\pm\infty$, or does not exist. (If the limit does not exist, write DNE)

\input{Limit-Compute-0003.HELP.tex}

\[\lim_{x\to{-7}}\dfrac{x^{2} + 6 \, x - 7}{x - 1}=\answer{0}\]
\end{problem}}%}

\latexProblemContent{
\ifVerboseLocation This is Derivative Compute Question 0003. \\ \fi
\begin{problem}

Determine if the limit approaches a finite number, $\pm\infty$, or does not exist. (If the limit does not exist, write DNE)

\input{Limit-Compute-0003.HELP.tex}

\[\lim_{x\to{2}}\dfrac{x^{2} - x - 2}{x + 1}=\answer{0}\]
\end{problem}}%}

\latexProblemContent{
\ifVerboseLocation This is Derivative Compute Question 0003. \\ \fi
\begin{problem}

Determine if the limit approaches a finite number, $\pm\infty$, or does not exist. (If the limit does not exist, write DNE)

\input{Limit-Compute-0003.HELP.tex}

\[\lim_{x\to{7}}\dfrac{x^{2} - 12 \, x + 35}{x - 5}=\answer{0}\]
\end{problem}}%}

\latexProblemContent{
\ifVerboseLocation This is Derivative Compute Question 0003. \\ \fi
\begin{problem}

Determine if the limit approaches a finite number, $\pm\infty$, or does not exist. (If the limit does not exist, write DNE)

\input{Limit-Compute-0003.HELP.tex}

\[\lim_{x\to{10}}\dfrac{x^{2} - 18 \, x + 80}{x - 8}=\answer{0}\]
\end{problem}}%}

\latexProblemContent{
\ifVerboseLocation This is Derivative Compute Question 0003. \\ \fi
\begin{problem}

Determine if the limit approaches a finite number, $\pm\infty$, or does not exist. (If the limit does not exist, write DNE)

\input{Limit-Compute-0003.HELP.tex}

\[\lim_{x\to{4}}\dfrac{3 \, x^{2} - 30 \, x + 72}{x - 6}=\answer{0}\]
\end{problem}}%}

\latexProblemContent{
\ifVerboseLocation This is Derivative Compute Question 0003. \\ \fi
\begin{problem}

Determine if the limit approaches a finite number, $\pm\infty$, or does not exist. (If the limit does not exist, write DNE)

\input{Limit-Compute-0003.HELP.tex}

\[\lim_{x\to{3}}\dfrac{2 \, x^{2} - 22 \, x + 48}{x - 8}=\answer{0}\]
\end{problem}}%}

\latexProblemContent{
\ifVerboseLocation This is Derivative Compute Question 0003. \\ \fi
\begin{problem}

Determine if the limit approaches a finite number, $\pm\infty$, or does not exist. (If the limit does not exist, write DNE)

\input{Limit-Compute-0003.HELP.tex}

\[\lim_{x\to{-9}}\dfrac{x^{2} + 6 \, x - 27}{x - 3}=\answer{0}\]
\end{problem}}%}

\latexProblemContent{
\ifVerboseLocation This is Derivative Compute Question 0003. \\ \fi
\begin{problem}

Determine if the limit approaches a finite number, $\pm\infty$, or does not exist. (If the limit does not exist, write DNE)

\input{Limit-Compute-0003.HELP.tex}

\[\lim_{x\to{-4}}\dfrac{2 \, x^{2} + 14 \, x + 24}{x + 3}=\answer{0}\]
\end{problem}}%}

\latexProblemContent{
\ifVerboseLocation This is Derivative Compute Question 0003. \\ \fi
\begin{problem}

Determine if the limit approaches a finite number, $\pm\infty$, or does not exist. (If the limit does not exist, write DNE)

\input{Limit-Compute-0003.HELP.tex}

\[\lim_{x\to{3}}\dfrac{x^{2} - 9}{x + 3}=\answer{0}\]
\end{problem}}%}

\latexProblemContent{
\ifVerboseLocation This is Derivative Compute Question 0003. \\ \fi
\begin{problem}

Determine if the limit approaches a finite number, $\pm\infty$, or does not exist. (If the limit does not exist, write DNE)

\input{Limit-Compute-0003.HELP.tex}

\[\lim_{x\to{3}}\dfrac{2 \, x^{2} - 8 \, x + 6}{x - 1}=\answer{0}\]
\end{problem}}%}

\latexProblemContent{
\ifVerboseLocation This is Derivative Compute Question 0003. \\ \fi
\begin{problem}

Determine if the limit approaches a finite number, $\pm\infty$, or does not exist. (If the limit does not exist, write DNE)

\input{Limit-Compute-0003.HELP.tex}

\[\lim_{x\to{-3}}\dfrac{x^{2} - 2 \, x - 15}{x - 5}=\answer{0}\]
\end{problem}}%}

\latexProblemContent{
\ifVerboseLocation This is Derivative Compute Question 0003. \\ \fi
\begin{problem}

Determine if the limit approaches a finite number, $\pm\infty$, or does not exist. (If the limit does not exist, write DNE)

\input{Limit-Compute-0003.HELP.tex}

\[\lim_{x\to{-3}}\dfrac{x^{2} - 9}{x - 3}=\answer{0}\]
\end{problem}}%}

\latexProblemContent{
\ifVerboseLocation This is Derivative Compute Question 0003. \\ \fi
\begin{problem}

Determine if the limit approaches a finite number, $\pm\infty$, or does not exist. (If the limit does not exist, write DNE)

\input{Limit-Compute-0003.HELP.tex}

\[\lim_{x\to{-9}}\dfrac{3 \, x^{2} + 39 \, x + 108}{x + 4}=\answer{0}\]
\end{problem}}%}

\latexProblemContent{
\ifVerboseLocation This is Derivative Compute Question 0003. \\ \fi
\begin{problem}

Determine if the limit approaches a finite number, $\pm\infty$, or does not exist. (If the limit does not exist, write DNE)

\input{Limit-Compute-0003.HELP.tex}

\[\lim_{x\to{6}}\dfrac{3 \, x^{2} + 3 \, x - 126}{x + 7}=\answer{0}\]
\end{problem}}%}

\latexProblemContent{
\ifVerboseLocation This is Derivative Compute Question 0003. \\ \fi
\begin{problem}

Determine if the limit approaches a finite number, $\pm\infty$, or does not exist. (If the limit does not exist, write DNE)

\input{Limit-Compute-0003.HELP.tex}

\[\lim_{x\to{2}}\dfrac{3 \, x^{2} + 15 \, x - 42}{x + 7}=\answer{0}\]
\end{problem}}%}

\latexProblemContent{
\ifVerboseLocation This is Derivative Compute Question 0003. \\ \fi
\begin{problem}

Determine if the limit approaches a finite number, $\pm\infty$, or does not exist. (If the limit does not exist, write DNE)

\input{Limit-Compute-0003.HELP.tex}

\[\lim_{x\to{-4}}\dfrac{3 \, x^{2} + 27 \, x + 60}{x + 5}=\answer{0}\]
\end{problem}}%}

\latexProblemContent{
\ifVerboseLocation This is Derivative Compute Question 0003. \\ \fi
\begin{problem}

Determine if the limit approaches a finite number, $\pm\infty$, or does not exist. (If the limit does not exist, write DNE)

\input{Limit-Compute-0003.HELP.tex}

\[\lim_{x\to{1}}\dfrac{x^{2} + 6 \, x - 7}{x + 7}=\answer{0}\]
\end{problem}}%}

\latexProblemContent{
\ifVerboseLocation This is Derivative Compute Question 0003. \\ \fi
\begin{problem}

Determine if the limit approaches a finite number, $\pm\infty$, or does not exist. (If the limit does not exist, write DNE)

\input{Limit-Compute-0003.HELP.tex}

\[\lim_{x\to{-5}}\dfrac{2 \, x^{2} + 8 \, x - 10}{x - 1}=\answer{0}\]
\end{problem}}%}

\latexProblemContent{
\ifVerboseLocation This is Derivative Compute Question 0003. \\ \fi
\begin{problem}

Determine if the limit approaches a finite number, $\pm\infty$, or does not exist. (If the limit does not exist, write DNE)

\input{Limit-Compute-0003.HELP.tex}

\[\lim_{x\to{4}}\dfrac{x^{2} - 3 \, x - 4}{x + 1}=\answer{0}\]
\end{problem}}%}

\latexProblemContent{
\ifVerboseLocation This is Derivative Compute Question 0003. \\ \fi
\begin{problem}

Determine if the limit approaches a finite number, $\pm\infty$, or does not exist. (If the limit does not exist, write DNE)

\input{Limit-Compute-0003.HELP.tex}

\[\lim_{x\to{10}}\dfrac{x^{2} - 2 \, x - 80}{x + 8}=\answer{0}\]
\end{problem}}%}

\latexProblemContent{
\ifVerboseLocation This is Derivative Compute Question 0003. \\ \fi
\begin{problem}

Determine if the limit approaches a finite number, $\pm\infty$, or does not exist. (If the limit does not exist, write DNE)

\input{Limit-Compute-0003.HELP.tex}

\[\lim_{x\to{-5}}\dfrac{2 \, x^{2} + 18 \, x + 40}{x + 4}=\answer{0}\]
\end{problem}}%}

\latexProblemContent{
\ifVerboseLocation This is Derivative Compute Question 0003. \\ \fi
\begin{problem}

Determine if the limit approaches a finite number, $\pm\infty$, or does not exist. (If the limit does not exist, write DNE)

\input{Limit-Compute-0003.HELP.tex}

\[\lim_{x\to{-3}}\dfrac{2 \, x^{2} + 24 \, x + 54}{x + 9}=\answer{0}\]
\end{problem}}%}

\latexProblemContent{
\ifVerboseLocation This is Derivative Compute Question 0003. \\ \fi
\begin{problem}

Determine if the limit approaches a finite number, $\pm\infty$, or does not exist. (If the limit does not exist, write DNE)

\input{Limit-Compute-0003.HELP.tex}

\[\lim_{x\to{7}}\dfrac{x^{2} - 17 \, x + 70}{x - 10}=\answer{0}\]
\end{problem}}%}

\latexProblemContent{
\ifVerboseLocation This is Derivative Compute Question 0003. \\ \fi
\begin{problem}

Determine if the limit approaches a finite number, $\pm\infty$, or does not exist. (If the limit does not exist, write DNE)

\input{Limit-Compute-0003.HELP.tex}

\[\lim_{x\to{-4}}\dfrac{x^{2} + 3 \, x - 4}{x - 1}=\answer{0}\]
\end{problem}}%}

\latexProblemContent{
\ifVerboseLocation This is Derivative Compute Question 0003. \\ \fi
\begin{problem}

Determine if the limit approaches a finite number, $\pm\infty$, or does not exist. (If the limit does not exist, write DNE)

\input{Limit-Compute-0003.HELP.tex}

\[\lim_{x\to{8}}\dfrac{2 \, x^{2} - 36 \, x + 160}{x - 10}=\answer{0}\]
\end{problem}}%}

\latexProblemContent{
\ifVerboseLocation This is Derivative Compute Question 0003. \\ \fi
\begin{problem}

Determine if the limit approaches a finite number, $\pm\infty$, or does not exist. (If the limit does not exist, write DNE)

\input{Limit-Compute-0003.HELP.tex}

\[\lim_{x\to{-1}}\dfrac{x^{2} - x - 2}{x - 2}=\answer{0}\]
\end{problem}}%}

\latexProblemContent{
\ifVerboseLocation This is Derivative Compute Question 0003. \\ \fi
\begin{problem}

Determine if the limit approaches a finite number, $\pm\infty$, or does not exist. (If the limit does not exist, write DNE)

\input{Limit-Compute-0003.HELP.tex}

\[\lim_{x\to{-8}}\dfrac{3 \, x^{2} - 192}{x - 8}=\answer{0}\]
\end{problem}}%}

\latexProblemContent{
\ifVerboseLocation This is Derivative Compute Question 0003. \\ \fi
\begin{problem}

Determine if the limit approaches a finite number, $\pm\infty$, or does not exist. (If the limit does not exist, write DNE)

\input{Limit-Compute-0003.HELP.tex}

\[\lim_{x\to{1}}\dfrac{2 \, x^{2} - 6 \, x + 4}{x - 2}=\answer{0}\]
\end{problem}}%}

\latexProblemContent{
\ifVerboseLocation This is Derivative Compute Question 0003. \\ \fi
\begin{problem}

Determine if the limit approaches a finite number, $\pm\infty$, or does not exist. (If the limit does not exist, write DNE)

\input{Limit-Compute-0003.HELP.tex}

\[\lim_{x\to{-5}}\dfrac{x^{2} - 5 \, x - 50}{x - 10}=\answer{0}\]
\end{problem}}%}

\latexProblemContent{
\ifVerboseLocation This is Derivative Compute Question 0003. \\ \fi
\begin{problem}

Determine if the limit approaches a finite number, $\pm\infty$, or does not exist. (If the limit does not exist, write DNE)

\input{Limit-Compute-0003.HELP.tex}

\[\lim_{x\to{-7}}\dfrac{x^{2} + 5 \, x - 14}{x - 2}=\answer{0}\]
\end{problem}}%}

\latexProblemContent{
\ifVerboseLocation This is Derivative Compute Question 0003. \\ \fi
\begin{problem}

Determine if the limit approaches a finite number, $\pm\infty$, or does not exist. (If the limit does not exist, write DNE)

\input{Limit-Compute-0003.HELP.tex}

\[\lim_{x\to{3}}\dfrac{x^{2} - 2 \, x - 3}{x + 1}=\answer{0}\]
\end{problem}}%}

\latexProblemContent{
\ifVerboseLocation This is Derivative Compute Question 0003. \\ \fi
\begin{problem}

Determine if the limit approaches a finite number, $\pm\infty$, or does not exist. (If the limit does not exist, write DNE)

\input{Limit-Compute-0003.HELP.tex}

\[\lim_{x\to{3}}\dfrac{2 \, x^{2} + 4 \, x - 30}{x + 5}=\answer{0}\]
\end{problem}}%}

\latexProblemContent{
\ifVerboseLocation This is Derivative Compute Question 0003. \\ \fi
\begin{problem}

Determine if the limit approaches a finite number, $\pm\infty$, or does not exist. (If the limit does not exist, write DNE)

\input{Limit-Compute-0003.HELP.tex}

\[\lim_{x\to{-3}}\dfrac{2 \, x^{2} - 12 \, x - 54}{x - 9}=\answer{0}\]
\end{problem}}%}

\latexProblemContent{
\ifVerboseLocation This is Derivative Compute Question 0003. \\ \fi
\begin{problem}

Determine if the limit approaches a finite number, $\pm\infty$, or does not exist. (If the limit does not exist, write DNE)

\input{Limit-Compute-0003.HELP.tex}

\[\lim_{x\to{-2}}\dfrac{x^{2} - x - 6}{x - 3}=\answer{0}\]
\end{problem}}%}

\latexProblemContent{
\ifVerboseLocation This is Derivative Compute Question 0003. \\ \fi
\begin{problem}

Determine if the limit approaches a finite number, $\pm\infty$, or does not exist. (If the limit does not exist, write DNE)

\input{Limit-Compute-0003.HELP.tex}

\[\lim_{x\to{-2}}\dfrac{x^{2} + 3 \, x + 2}{x + 1}=\answer{0}\]
\end{problem}}%}

\latexProblemContent{
\ifVerboseLocation This is Derivative Compute Question 0003. \\ \fi
\begin{problem}

Determine if the limit approaches a finite number, $\pm\infty$, or does not exist. (If the limit does not exist, write DNE)

\input{Limit-Compute-0003.HELP.tex}

\[\lim_{x\to{9}}\dfrac{3 \, x^{2} - 6 \, x - 189}{x + 7}=\answer{0}\]
\end{problem}}%}

\latexProblemContent{
\ifVerboseLocation This is Derivative Compute Question 0003. \\ \fi
\begin{problem}

Determine if the limit approaches a finite number, $\pm\infty$, or does not exist. (If the limit does not exist, write DNE)

\input{Limit-Compute-0003.HELP.tex}

\[\lim_{x\to{2}}\dfrac{3 \, x^{2} - 12}{x + 2}=\answer{0}\]
\end{problem}}%}

\latexProblemContent{
\ifVerboseLocation This is Derivative Compute Question 0003. \\ \fi
\begin{problem}

Determine if the limit approaches a finite number, $\pm\infty$, or does not exist. (If the limit does not exist, write DNE)

\input{Limit-Compute-0003.HELP.tex}

\[\lim_{x\to{-6}}\dfrac{x^{2} + 2 \, x - 24}{x - 4}=\answer{0}\]
\end{problem}}%}

\latexProblemContent{
\ifVerboseLocation This is Derivative Compute Question 0003. \\ \fi
\begin{problem}

Determine if the limit approaches a finite number, $\pm\infty$, or does not exist. (If the limit does not exist, write DNE)

\input{Limit-Compute-0003.HELP.tex}

\[\lim_{x\to{-10}}\dfrac{3 \, x^{2} + 9 \, x - 210}{x - 7}=\answer{0}\]
\end{problem}}%}

\latexProblemContent{
\ifVerboseLocation This is Derivative Compute Question 0003. \\ \fi
\begin{problem}

Determine if the limit approaches a finite number, $\pm\infty$, or does not exist. (If the limit does not exist, write DNE)

\input{Limit-Compute-0003.HELP.tex}

\[\lim_{x\to{-4}}\dfrac{2 \, x^{2} - 12 \, x - 80}{x - 10}=\answer{0}\]
\end{problem}}%}

\latexProblemContent{
\ifVerboseLocation This is Derivative Compute Question 0003. \\ \fi
\begin{problem}

Determine if the limit approaches a finite number, $\pm\infty$, or does not exist. (If the limit does not exist, write DNE)

\input{Limit-Compute-0003.HELP.tex}

\[\lim_{x\to{1}}\dfrac{2 \, x^{2} - 2}{x + 1}=\answer{0}\]
\end{problem}}%}

\latexProblemContent{
\ifVerboseLocation This is Derivative Compute Question 0003. \\ \fi
\begin{problem}

Determine if the limit approaches a finite number, $\pm\infty$, or does not exist. (If the limit does not exist, write DNE)

\input{Limit-Compute-0003.HELP.tex}

\[\lim_{x\to{2}}\dfrac{3 \, x^{2} + 9 \, x - 30}{x + 5}=\answer{0}\]
\end{problem}}%}

\latexProblemContent{
\ifVerboseLocation This is Derivative Compute Question 0003. \\ \fi
\begin{problem}

Determine if the limit approaches a finite number, $\pm\infty$, or does not exist. (If the limit does not exist, write DNE)

\input{Limit-Compute-0003.HELP.tex}

\[\lim_{x\to{-1}}\dfrac{2 \, x^{2} - 2 \, x - 4}{x - 2}=\answer{0}\]
\end{problem}}%}

\latexProblemContent{
\ifVerboseLocation This is Derivative Compute Question 0003. \\ \fi
\begin{problem}

Determine if the limit approaches a finite number, $\pm\infty$, or does not exist. (If the limit does not exist, write DNE)

\input{Limit-Compute-0003.HELP.tex}

\[\lim_{x\to{-8}}\dfrac{3 \, x^{2} + 12 \, x - 96}{x - 4}=\answer{0}\]
\end{problem}}%}

\latexProblemContent{
\ifVerboseLocation This is Derivative Compute Question 0003. \\ \fi
\begin{problem}

Determine if the limit approaches a finite number, $\pm\infty$, or does not exist. (If the limit does not exist, write DNE)

\input{Limit-Compute-0003.HELP.tex}

\[\lim_{x\to{5}}\dfrac{x^{2} - 2 \, x - 15}{x + 3}=\answer{0}\]
\end{problem}}%}

\latexProblemContent{
\ifVerboseLocation This is Derivative Compute Question 0003. \\ \fi
\begin{problem}

Determine if the limit approaches a finite number, $\pm\infty$, or does not exist. (If the limit does not exist, write DNE)

\input{Limit-Compute-0003.HELP.tex}

\[\lim_{x\to{1}}\dfrac{2 \, x^{2} - 10 \, x + 8}{x - 4}=\answer{0}\]
\end{problem}}%}

\latexProblemContent{
\ifVerboseLocation This is Derivative Compute Question 0003. \\ \fi
\begin{problem}

Determine if the limit approaches a finite number, $\pm\infty$, or does not exist. (If the limit does not exist, write DNE)

\input{Limit-Compute-0003.HELP.tex}

\[\lim_{x\to{8}}\dfrac{x^{2} - 12 \, x + 32}{x - 4}=\answer{0}\]
\end{problem}}%}

\latexProblemContent{
\ifVerboseLocation This is Derivative Compute Question 0003. \\ \fi
\begin{problem}

Determine if the limit approaches a finite number, $\pm\infty$, or does not exist. (If the limit does not exist, write DNE)

\input{Limit-Compute-0003.HELP.tex}

\[\lim_{x\to{4}}\dfrac{2 \, x^{2} - 12 \, x + 16}{x - 2}=\answer{0}\]
\end{problem}}%}

\latexProblemContent{
\ifVerboseLocation This is Derivative Compute Question 0003. \\ \fi
\begin{problem}

Determine if the limit approaches a finite number, $\pm\infty$, or does not exist. (If the limit does not exist, write DNE)

\input{Limit-Compute-0003.HELP.tex}

\[\lim_{x\to{-2}}\dfrac{2 \, x^{2} + 22 \, x + 36}{x + 9}=\answer{0}\]
\end{problem}}%}

\latexProblemContent{
\ifVerboseLocation This is Derivative Compute Question 0003. \\ \fi
\begin{problem}

Determine if the limit approaches a finite number, $\pm\infty$, or does not exist. (If the limit does not exist, write DNE)

\input{Limit-Compute-0003.HELP.tex}

\[\lim_{x\to{-3}}\dfrac{2 \, x^{2} + 10 \, x + 12}{x + 2}=\answer{0}\]
\end{problem}}%}

\latexProblemContent{
\ifVerboseLocation This is Derivative Compute Question 0003. \\ \fi
\begin{problem}

Determine if the limit approaches a finite number, $\pm\infty$, or does not exist. (If the limit does not exist, write DNE)

\input{Limit-Compute-0003.HELP.tex}

\[\lim_{x\to{-9}}\dfrac{3 \, x^{2} + 24 \, x - 27}{x - 1}=\answer{0}\]
\end{problem}}%}

\latexProblemContent{
\ifVerboseLocation This is Derivative Compute Question 0003. \\ \fi
\begin{problem}

Determine if the limit approaches a finite number, $\pm\infty$, or does not exist. (If the limit does not exist, write DNE)

\input{Limit-Compute-0003.HELP.tex}

\[\lim_{x\to{3}}\dfrac{2 \, x^{2} - 10 \, x + 12}{x - 2}=\answer{0}\]
\end{problem}}%}

\latexProblemContent{
\ifVerboseLocation This is Derivative Compute Question 0003. \\ \fi
\begin{problem}

Determine if the limit approaches a finite number, $\pm\infty$, or does not exist. (If the limit does not exist, write DNE)

\input{Limit-Compute-0003.HELP.tex}

\[\lim_{x\to{4}}\dfrac{3 \, x^{2} - 3 \, x - 36}{x + 3}=\answer{0}\]
\end{problem}}%}

\latexProblemContent{
\ifVerboseLocation This is Derivative Compute Question 0003. \\ \fi
\begin{problem}

Determine if the limit approaches a finite number, $\pm\infty$, or does not exist. (If the limit does not exist, write DNE)

\input{Limit-Compute-0003.HELP.tex}

\[\lim_{x\to{-3}}\dfrac{x^{2} + 2 \, x - 3}{x - 1}=\answer{0}\]
\end{problem}}%}

\latexProblemContent{
\ifVerboseLocation This is Derivative Compute Question 0003. \\ \fi
\begin{problem}

Determine if the limit approaches a finite number, $\pm\infty$, or does not exist. (If the limit does not exist, write DNE)

\input{Limit-Compute-0003.HELP.tex}

\[\lim_{x\to{-6}}\dfrac{2 \, x^{2} - 4 \, x - 96}{x - 8}=\answer{0}\]
\end{problem}}%}

\latexProblemContent{
\ifVerboseLocation This is Derivative Compute Question 0003. \\ \fi
\begin{problem}

Determine if the limit approaches a finite number, $\pm\infty$, or does not exist. (If the limit does not exist, write DNE)

\input{Limit-Compute-0003.HELP.tex}

\[\lim_{x\to{-3}}\dfrac{3 \, x^{2} + 3 \, x - 18}{x - 2}=\answer{0}\]
\end{problem}}%}

\latexProblemContent{
\ifVerboseLocation This is Derivative Compute Question 0003. \\ \fi
\begin{problem}

Determine if the limit approaches a finite number, $\pm\infty$, or does not exist. (If the limit does not exist, write DNE)

\input{Limit-Compute-0003.HELP.tex}

\[\lim_{x\to{-6}}\dfrac{2 \, x^{2} + 4 \, x - 48}{x - 4}=\answer{0}\]
\end{problem}}%}

\latexProblemContent{
\ifVerboseLocation This is Derivative Compute Question 0003. \\ \fi
\begin{problem}

Determine if the limit approaches a finite number, $\pm\infty$, or does not exist. (If the limit does not exist, write DNE)

\input{Limit-Compute-0003.HELP.tex}

\[\lim_{x\to{-9}}\dfrac{2 \, x^{2} + 8 \, x - 90}{x - 5}=\answer{0}\]
\end{problem}}%}

\latexProblemContent{
\ifVerboseLocation This is Derivative Compute Question 0003. \\ \fi
\begin{problem}

Determine if the limit approaches a finite number, $\pm\infty$, or does not exist. (If the limit does not exist, write DNE)

\input{Limit-Compute-0003.HELP.tex}

\[\lim_{x\to{8}}\dfrac{x^{2} - 14 \, x + 48}{x - 6}=\answer{0}\]
\end{problem}}%}

\latexProblemContent{
\ifVerboseLocation This is Derivative Compute Question 0003. \\ \fi
\begin{problem}

Determine if the limit approaches a finite number, $\pm\infty$, or does not exist. (If the limit does not exist, write DNE)

\input{Limit-Compute-0003.HELP.tex}

\[\lim_{x\to{4}}\dfrac{3 \, x^{2} + 3 \, x - 60}{x + 5}=\answer{0}\]
\end{problem}}%}

\latexProblemContent{
\ifVerboseLocation This is Derivative Compute Question 0003. \\ \fi
\begin{problem}

Determine if the limit approaches a finite number, $\pm\infty$, or does not exist. (If the limit does not exist, write DNE)

\input{Limit-Compute-0003.HELP.tex}

\[\lim_{x\to{3}}\dfrac{2 \, x^{2} - 14 \, x + 24}{x - 4}=\answer{0}\]
\end{problem}}%}

\latexProblemContent{
\ifVerboseLocation This is Derivative Compute Question 0003. \\ \fi
\begin{problem}

Determine if the limit approaches a finite number, $\pm\infty$, or does not exist. (If the limit does not exist, write DNE)

\input{Limit-Compute-0003.HELP.tex}

\[\lim_{x\to{-6}}\dfrac{3 \, x^{2} + 3 \, x - 90}{x - 5}=\answer{0}\]
\end{problem}}%}

\latexProblemContent{
\ifVerboseLocation This is Derivative Compute Question 0003. \\ \fi
\begin{problem}

Determine if the limit approaches a finite number, $\pm\infty$, or does not exist. (If the limit does not exist, write DNE)

\input{Limit-Compute-0003.HELP.tex}

\[\lim_{x\to{9}}\dfrac{2 \, x^{2} - 26 \, x + 72}{x - 4}=\answer{0}\]
\end{problem}}%}

\latexProblemContent{
\ifVerboseLocation This is Derivative Compute Question 0003. \\ \fi
\begin{problem}

Determine if the limit approaches a finite number, $\pm\infty$, or does not exist. (If the limit does not exist, write DNE)

\input{Limit-Compute-0003.HELP.tex}

\[\lim_{x\to{-3}}\dfrac{2 \, x^{2} + 2 \, x - 12}{x - 2}=\answer{0}\]
\end{problem}}%}

\latexProblemContent{
\ifVerboseLocation This is Derivative Compute Question 0003. \\ \fi
\begin{problem}

Determine if the limit approaches a finite number, $\pm\infty$, or does not exist. (If the limit does not exist, write DNE)

\input{Limit-Compute-0003.HELP.tex}

\[\lim_{x\to{6}}\dfrac{x^{2} - 9 \, x + 18}{x - 3}=\answer{0}\]
\end{problem}}%}

\latexProblemContent{
\ifVerboseLocation This is Derivative Compute Question 0003. \\ \fi
\begin{problem}

Determine if the limit approaches a finite number, $\pm\infty$, or does not exist. (If the limit does not exist, write DNE)

\input{Limit-Compute-0003.HELP.tex}

\[\lim_{x\to{-9}}\dfrac{2 \, x^{2} + 28 \, x + 90}{x + 5}=\answer{0}\]
\end{problem}}%}

\latexProblemContent{
\ifVerboseLocation This is Derivative Compute Question 0003. \\ \fi
\begin{problem}

Determine if the limit approaches a finite number, $\pm\infty$, or does not exist. (If the limit does not exist, write DNE)

\input{Limit-Compute-0003.HELP.tex}

\[\lim_{x\to{-2}}\dfrac{3 \, x^{2} - 12 \, x - 36}{x - 6}=\answer{0}\]
\end{problem}}%}

\latexProblemContent{
\ifVerboseLocation This is Derivative Compute Question 0003. \\ \fi
\begin{problem}

Determine if the limit approaches a finite number, $\pm\infty$, or does not exist. (If the limit does not exist, write DNE)

\input{Limit-Compute-0003.HELP.tex}

\[\lim_{x\to{-10}}\dfrac{2 \, x^{2} + 32 \, x + 120}{x + 6}=\answer{0}\]
\end{problem}}%}

\latexProblemContent{
\ifVerboseLocation This is Derivative Compute Question 0003. \\ \fi
\begin{problem}

Determine if the limit approaches a finite number, $\pm\infty$, or does not exist. (If the limit does not exist, write DNE)

\input{Limit-Compute-0003.HELP.tex}

\[\lim_{x\to{8}}\dfrac{x^{2} - 17 \, x + 72}{x - 9}=\answer{0}\]
\end{problem}}%}

\latexProblemContent{
\ifVerboseLocation This is Derivative Compute Question 0003. \\ \fi
\begin{problem}

Determine if the limit approaches a finite number, $\pm\infty$, or does not exist. (If the limit does not exist, write DNE)

\input{Limit-Compute-0003.HELP.tex}

\[\lim_{x\to{8}}\dfrac{3 \, x^{2} + 6 \, x - 240}{x + 10}=\answer{0}\]
\end{problem}}%}

\latexProblemContent{
\ifVerboseLocation This is Derivative Compute Question 0003. \\ \fi
\begin{problem}

Determine if the limit approaches a finite number, $\pm\infty$, or does not exist. (If the limit does not exist, write DNE)

\input{Limit-Compute-0003.HELP.tex}

\[\lim_{x\to{7}}\dfrac{x^{2} - 6 \, x - 7}{x + 1}=\answer{0}\]
\end{problem}}%}

\latexProblemContent{
\ifVerboseLocation This is Derivative Compute Question 0003. \\ \fi
\begin{problem}

Determine if the limit approaches a finite number, $\pm\infty$, or does not exist. (If the limit does not exist, write DNE)

\input{Limit-Compute-0003.HELP.tex}

\[\lim_{x\to{7}}\dfrac{3 \, x^{2} - 51 \, x + 210}{x - 10}=\answer{0}\]
\end{problem}}%}

\latexProblemContent{
\ifVerboseLocation This is Derivative Compute Question 0003. \\ \fi
\begin{problem}

Determine if the limit approaches a finite number, $\pm\infty$, or does not exist. (If the limit does not exist, write DNE)

\input{Limit-Compute-0003.HELP.tex}

\[\lim_{x\to{-1}}\dfrac{2 \, x^{2} + 10 \, x + 8}{x + 4}=\answer{0}\]
\end{problem}}%}

\latexProblemContent{
\ifVerboseLocation This is Derivative Compute Question 0003. \\ \fi
\begin{problem}

Determine if the limit approaches a finite number, $\pm\infty$, or does not exist. (If the limit does not exist, write DNE)

\input{Limit-Compute-0003.HELP.tex}

\[\lim_{x\to{2}}\dfrac{2 \, x^{2} - 14 \, x + 20}{x - 5}=\answer{0}\]
\end{problem}}%}

\latexProblemContent{
\ifVerboseLocation This is Derivative Compute Question 0003. \\ \fi
\begin{problem}

Determine if the limit approaches a finite number, $\pm\infty$, or does not exist. (If the limit does not exist, write DNE)

\input{Limit-Compute-0003.HELP.tex}

\[\lim_{x\to{2}}\dfrac{x^{2} - 7 \, x + 10}{x - 5}=\answer{0}\]
\end{problem}}%}

\latexProblemContent{
\ifVerboseLocation This is Derivative Compute Question 0003. \\ \fi
\begin{problem}

Determine if the limit approaches a finite number, $\pm\infty$, or does not exist. (If the limit does not exist, write DNE)

\input{Limit-Compute-0003.HELP.tex}

\[\lim_{x\to{4}}\dfrac{x^{2} + 6 \, x - 40}{x + 10}=\answer{0}\]
\end{problem}}%}

\latexProblemContent{
\ifVerboseLocation This is Derivative Compute Question 0003. \\ \fi
\begin{problem}

Determine if the limit approaches a finite number, $\pm\infty$, or does not exist. (If the limit does not exist, write DNE)

\input{Limit-Compute-0003.HELP.tex}

\[\lim_{x\to{-8}}\dfrac{3 \, x^{2} + 36 \, x + 96}{x + 4}=\answer{0}\]
\end{problem}}%}

\latexProblemContent{
\ifVerboseLocation This is Derivative Compute Question 0003. \\ \fi
\begin{problem}

Determine if the limit approaches a finite number, $\pm\infty$, or does not exist. (If the limit does not exist, write DNE)

\input{Limit-Compute-0003.HELP.tex}

\[\lim_{x\to{4}}\dfrac{3 \, x^{2} + 18 \, x - 120}{x + 10}=\answer{0}\]
\end{problem}}%}

\latexProblemContent{
\ifVerboseLocation This is Derivative Compute Question 0003. \\ \fi
\begin{problem}

Determine if the limit approaches a finite number, $\pm\infty$, or does not exist. (If the limit does not exist, write DNE)

\input{Limit-Compute-0003.HELP.tex}

\[\lim_{x\to{1}}\dfrac{3 \, x^{2} - 15 \, x + 12}{x - 4}=\answer{0}\]
\end{problem}}%}

\latexProblemContent{
\ifVerboseLocation This is Derivative Compute Question 0003. \\ \fi
\begin{problem}

Determine if the limit approaches a finite number, $\pm\infty$, or does not exist. (If the limit does not exist, write DNE)

\input{Limit-Compute-0003.HELP.tex}

\[\lim_{x\to{-1}}\dfrac{3 \, x^{2} - 6 \, x - 9}{x - 3}=\answer{0}\]
\end{problem}}%}

\latexProblemContent{
\ifVerboseLocation This is Derivative Compute Question 0003. \\ \fi
\begin{problem}

Determine if the limit approaches a finite number, $\pm\infty$, or does not exist. (If the limit does not exist, write DNE)

\input{Limit-Compute-0003.HELP.tex}

\[\lim_{x\to{-1}}\dfrac{x^{2} + 3 \, x + 2}{x + 2}=\answer{0}\]
\end{problem}}%}

\latexProblemContent{
\ifVerboseLocation This is Derivative Compute Question 0003. \\ \fi
\begin{problem}

Determine if the limit approaches a finite number, $\pm\infty$, or does not exist. (If the limit does not exist, write DNE)

\input{Limit-Compute-0003.HELP.tex}

\[\lim_{x\to{-7}}\dfrac{2 \, x^{2} + 2 \, x - 84}{x - 6}=\answer{0}\]
\end{problem}}%}

\latexProblemContent{
\ifVerboseLocation This is Derivative Compute Question 0003. \\ \fi
\begin{problem}

Determine if the limit approaches a finite number, $\pm\infty$, or does not exist. (If the limit does not exist, write DNE)

\input{Limit-Compute-0003.HELP.tex}

\[\lim_{x\to{-1}}\dfrac{x^{2} - 2 \, x - 3}{x - 3}=\answer{0}\]
\end{problem}}%}

\latexProblemContent{
\ifVerboseLocation This is Derivative Compute Question 0003. \\ \fi
\begin{problem}

Determine if the limit approaches a finite number, $\pm\infty$, or does not exist. (If the limit does not exist, write DNE)

\input{Limit-Compute-0003.HELP.tex}

\[\lim_{x\to{4}}\dfrac{3 \, x^{2} - 21 \, x + 36}{x - 3}=\answer{0}\]
\end{problem}}%}

\latexProblemContent{
\ifVerboseLocation This is Derivative Compute Question 0003. \\ \fi
\begin{problem}

Determine if the limit approaches a finite number, $\pm\infty$, or does not exist. (If the limit does not exist, write DNE)

\input{Limit-Compute-0003.HELP.tex}

\[\lim_{x\to{1}}\dfrac{3 \, x^{2} - 21 \, x + 18}{x - 6}=\answer{0}\]
\end{problem}}%}

\latexProblemContent{
\ifVerboseLocation This is Derivative Compute Question 0003. \\ \fi
\begin{problem}

Determine if the limit approaches a finite number, $\pm\infty$, or does not exist. (If the limit does not exist, write DNE)

\input{Limit-Compute-0003.HELP.tex}

\[\lim_{x\to{10}}\dfrac{3 \, x^{2} - 42 \, x + 120}{x - 4}=\answer{0}\]
\end{problem}}%}

\latexProblemContent{
\ifVerboseLocation This is Derivative Compute Question 0003. \\ \fi
\begin{problem}

Determine if the limit approaches a finite number, $\pm\infty$, or does not exist. (If the limit does not exist, write DNE)

\input{Limit-Compute-0003.HELP.tex}

\[\lim_{x\to{4}}\dfrac{2 \, x^{2} - 24 \, x + 64}{x - 8}=\answer{0}\]
\end{problem}}%}

\latexProblemContent{
\ifVerboseLocation This is Derivative Compute Question 0003. \\ \fi
\begin{problem}

Determine if the limit approaches a finite number, $\pm\infty$, or does not exist. (If the limit does not exist, write DNE)

\input{Limit-Compute-0003.HELP.tex}

\[\lim_{x\to{-4}}\dfrac{3 \, x^{2} - 6 \, x - 72}{x - 6}=\answer{0}\]
\end{problem}}%}

\latexProblemContent{
\ifVerboseLocation This is Derivative Compute Question 0003. \\ \fi
\begin{problem}

Determine if the limit approaches a finite number, $\pm\infty$, or does not exist. (If the limit does not exist, write DNE)

\input{Limit-Compute-0003.HELP.tex}

\[\lim_{x\to{-9}}\dfrac{2 \, x^{2} + 10 \, x - 72}{x - 4}=\answer{0}\]
\end{problem}}%}

\latexProblemContent{
\ifVerboseLocation This is Derivative Compute Question 0003. \\ \fi
\begin{problem}

Determine if the limit approaches a finite number, $\pm\infty$, or does not exist. (If the limit does not exist, write DNE)

\input{Limit-Compute-0003.HELP.tex}

\[\lim_{x\to{-1}}\dfrac{2 \, x^{2} + 16 \, x + 14}{x + 7}=\answer{0}\]
\end{problem}}%}

\latexProblemContent{
\ifVerboseLocation This is Derivative Compute Question 0003. \\ \fi
\begin{problem}

Determine if the limit approaches a finite number, $\pm\infty$, or does not exist. (If the limit does not exist, write DNE)

\input{Limit-Compute-0003.HELP.tex}

\[\lim_{x\to{-2}}\dfrac{2 \, x^{2} + 10 \, x + 12}{x + 3}=\answer{0}\]
\end{problem}}%}

\latexProblemContent{
\ifVerboseLocation This is Derivative Compute Question 0003. \\ \fi
\begin{problem}

Determine if the limit approaches a finite number, $\pm\infty$, or does not exist. (If the limit does not exist, write DNE)

\input{Limit-Compute-0003.HELP.tex}

\[\lim_{x\to{9}}\dfrac{3 \, x^{2} - 243}{x + 9}=\answer{0}\]
\end{problem}}%}

\latexProblemContent{
\ifVerboseLocation This is Derivative Compute Question 0003. \\ \fi
\begin{problem}

Determine if the limit approaches a finite number, $\pm\infty$, or does not exist. (If the limit does not exist, write DNE)

\input{Limit-Compute-0003.HELP.tex}

\[\lim_{x\to{-2}}\dfrac{2 \, x^{2} - 16 \, x - 40}{x - 10}=\answer{0}\]
\end{problem}}%}

\latexProblemContent{
\ifVerboseLocation This is Derivative Compute Question 0003. \\ \fi
\begin{problem}

Determine if the limit approaches a finite number, $\pm\infty$, or does not exist. (If the limit does not exist, write DNE)

\input{Limit-Compute-0003.HELP.tex}

\[\lim_{x\to{-2}}\dfrac{3 \, x^{2} - 18 \, x - 48}{x - 8}=\answer{0}\]
\end{problem}}%}

\latexProblemContent{
\ifVerboseLocation This is Derivative Compute Question 0003. \\ \fi
\begin{problem}

Determine if the limit approaches a finite number, $\pm\infty$, or does not exist. (If the limit does not exist, write DNE)

\input{Limit-Compute-0003.HELP.tex}

\[\lim_{x\to{-5}}\dfrac{3 \, x^{2} + 33 \, x + 90}{x + 6}=\answer{0}\]
\end{problem}}%}

\latexProblemContent{
\ifVerboseLocation This is Derivative Compute Question 0003. \\ \fi
\begin{problem}

Determine if the limit approaches a finite number, $\pm\infty$, or does not exist. (If the limit does not exist, write DNE)

\input{Limit-Compute-0003.HELP.tex}

\[\lim_{x\to{7}}\dfrac{3 \, x^{2} - 45 \, x + 168}{x - 8}=\answer{0}\]
\end{problem}}%}

\latexProblemContent{
\ifVerboseLocation This is Derivative Compute Question 0003. \\ \fi
\begin{problem}

Determine if the limit approaches a finite number, $\pm\infty$, or does not exist. (If the limit does not exist, write DNE)

\input{Limit-Compute-0003.HELP.tex}

\[\lim_{x\to{1}}\dfrac{3 \, x^{2} + 21 \, x - 24}{x + 8}=\answer{0}\]
\end{problem}}%}

\latexProblemContent{
\ifVerboseLocation This is Derivative Compute Question 0003. \\ \fi
\begin{problem}

Determine if the limit approaches a finite number, $\pm\infty$, or does not exist. (If the limit does not exist, write DNE)

\input{Limit-Compute-0003.HELP.tex}

\[\lim_{x\to{-7}}\dfrac{x^{2} - 49}{x - 7}=\answer{0}\]
\end{problem}}%}

\latexProblemContent{
\ifVerboseLocation This is Derivative Compute Question 0003. \\ \fi
\begin{problem}

Determine if the limit approaches a finite number, $\pm\infty$, or does not exist. (If the limit does not exist, write DNE)

\input{Limit-Compute-0003.HELP.tex}

\[\lim_{x\to{5}}\dfrac{x^{2} - 11 \, x + 30}{x - 6}=\answer{0}\]
\end{problem}}%}

\latexProblemContent{
\ifVerboseLocation This is Derivative Compute Question 0003. \\ \fi
\begin{problem}

Determine if the limit approaches a finite number, $\pm\infty$, or does not exist. (If the limit does not exist, write DNE)

\input{Limit-Compute-0003.HELP.tex}

\[\lim_{x\to{-10}}\dfrac{2 \, x^{2} + 24 \, x + 40}{x + 2}=\answer{0}\]
\end{problem}}%}

\latexProblemContent{
\ifVerboseLocation This is Derivative Compute Question 0003. \\ \fi
\begin{problem}

Determine if the limit approaches a finite number, $\pm\infty$, or does not exist. (If the limit does not exist, write DNE)

\input{Limit-Compute-0003.HELP.tex}

\[\lim_{x\to{6}}\dfrac{2 \, x^{2} - 28 \, x + 96}{x - 8}=\answer{0}\]
\end{problem}}%}

\latexProblemContent{
\ifVerboseLocation This is Derivative Compute Question 0003. \\ \fi
\begin{problem}

Determine if the limit approaches a finite number, $\pm\infty$, or does not exist. (If the limit does not exist, write DNE)

\input{Limit-Compute-0003.HELP.tex}

\[\lim_{x\to{6}}\dfrac{3 \, x^{2} - 33 \, x + 90}{x - 5}=\answer{0}\]
\end{problem}}%}

\latexProblemContent{
\ifVerboseLocation This is Derivative Compute Question 0003. \\ \fi
\begin{problem}

Determine if the limit approaches a finite number, $\pm\infty$, or does not exist. (If the limit does not exist, write DNE)

\input{Limit-Compute-0003.HELP.tex}

\[\lim_{x\to{-2}}\dfrac{3 \, x^{2} + 3 \, x - 6}{x - 1}=\answer{0}\]
\end{problem}}%}

\latexProblemContent{
\ifVerboseLocation This is Derivative Compute Question 0003. \\ \fi
\begin{problem}

Determine if the limit approaches a finite number, $\pm\infty$, or does not exist. (If the limit does not exist, write DNE)

\input{Limit-Compute-0003.HELP.tex}

\[\lim_{x\to{4}}\dfrac{3 \, x^{2} - 36 \, x + 96}{x - 8}=\answer{0}\]
\end{problem}}%}

\latexProblemContent{
\ifVerboseLocation This is Derivative Compute Question 0003. \\ \fi
\begin{problem}

Determine if the limit approaches a finite number, $\pm\infty$, or does not exist. (If the limit does not exist, write DNE)

\input{Limit-Compute-0003.HELP.tex}

\[\lim_{x\to{-2}}\dfrac{2 \, x^{2} + 16 \, x + 24}{x + 6}=\answer{0}\]
\end{problem}}%}

\latexProblemContent{
\ifVerboseLocation This is Derivative Compute Question 0003. \\ \fi
\begin{problem}

Determine if the limit approaches a finite number, $\pm\infty$, or does not exist. (If the limit does not exist, write DNE)

\input{Limit-Compute-0003.HELP.tex}

\[\lim_{x\to{-8}}\dfrac{3 \, x^{2} + 3 \, x - 168}{x - 7}=\answer{0}\]
\end{problem}}%}

\latexProblemContent{
\ifVerboseLocation This is Derivative Compute Question 0003. \\ \fi
\begin{problem}

Determine if the limit approaches a finite number, $\pm\infty$, or does not exist. (If the limit does not exist, write DNE)

\input{Limit-Compute-0003.HELP.tex}

\[\lim_{x\to{-6}}\dfrac{x^{2} + 8 \, x + 12}{x + 2}=\answer{0}\]
\end{problem}}%}

\latexProblemContent{
\ifVerboseLocation This is Derivative Compute Question 0003. \\ \fi
\begin{problem}

Determine if the limit approaches a finite number, $\pm\infty$, or does not exist. (If the limit does not exist, write DNE)

\input{Limit-Compute-0003.HELP.tex}

\[\lim_{x\to{-3}}\dfrac{x^{2} - 5 \, x - 24}{x - 8}=\answer{0}\]
\end{problem}}%}

\latexProblemContent{
\ifVerboseLocation This is Derivative Compute Question 0003. \\ \fi
\begin{problem}

Determine if the limit approaches a finite number, $\pm\infty$, or does not exist. (If the limit does not exist, write DNE)

\input{Limit-Compute-0003.HELP.tex}

\[\lim_{x\to{6}}\dfrac{2 \, x^{2} - 20 \, x + 48}{x - 4}=\answer{0}\]
\end{problem}}%}

\latexProblemContent{
\ifVerboseLocation This is Derivative Compute Question 0003. \\ \fi
\begin{problem}

Determine if the limit approaches a finite number, $\pm\infty$, or does not exist. (If the limit does not exist, write DNE)

\input{Limit-Compute-0003.HELP.tex}

\[\lim_{x\to{8}}\dfrac{3 \, x^{2} - 9 \, x - 120}{x + 5}=\answer{0}\]
\end{problem}}%}

\latexProblemContent{
\ifVerboseLocation This is Derivative Compute Question 0003. \\ \fi
\begin{problem}

Determine if the limit approaches a finite number, $\pm\infty$, or does not exist. (If the limit does not exist, write DNE)

\input{Limit-Compute-0003.HELP.tex}

\[\lim_{x\to{-2}}\dfrac{2 \, x^{2} - 12 \, x - 32}{x - 8}=\answer{0}\]
\end{problem}}%}

\latexProblemContent{
\ifVerboseLocation This is Derivative Compute Question 0003. \\ \fi
\begin{problem}

Determine if the limit approaches a finite number, $\pm\infty$, or does not exist. (If the limit does not exist, write DNE)

\input{Limit-Compute-0003.HELP.tex}

\[\lim_{x\to{7}}\dfrac{2 \, x^{2} + 2 \, x - 112}{x + 8}=\answer{0}\]
\end{problem}}%}

\latexProblemContent{
\ifVerboseLocation This is Derivative Compute Question 0003. \\ \fi
\begin{problem}

Determine if the limit approaches a finite number, $\pm\infty$, or does not exist. (If the limit does not exist, write DNE)

\input{Limit-Compute-0003.HELP.tex}

\[\lim_{x\to{3}}\dfrac{2 \, x^{2} + 14 \, x - 60}{x + 10}=\answer{0}\]
\end{problem}}%}

\latexProblemContent{
\ifVerboseLocation This is Derivative Compute Question 0003. \\ \fi
\begin{problem}

Determine if the limit approaches a finite number, $\pm\infty$, or does not exist. (If the limit does not exist, write DNE)

\input{Limit-Compute-0003.HELP.tex}

\[\lim_{x\to{6}}\dfrac{2 \, x^{2} - 16 \, x + 24}{x - 2}=\answer{0}\]
\end{problem}}%}

\latexProblemContent{
\ifVerboseLocation This is Derivative Compute Question 0003. \\ \fi
\begin{problem}

Determine if the limit approaches a finite number, $\pm\infty$, or does not exist. (If the limit does not exist, write DNE)

\input{Limit-Compute-0003.HELP.tex}

\[\lim_{x\to{-10}}\dfrac{x^{2} + 4 \, x - 60}{x - 6}=\answer{0}\]
\end{problem}}%}

\latexProblemContent{
\ifVerboseLocation This is Derivative Compute Question 0003. \\ \fi
\begin{problem}

Determine if the limit approaches a finite number, $\pm\infty$, or does not exist. (If the limit does not exist, write DNE)

\input{Limit-Compute-0003.HELP.tex}

\[\lim_{x\to{-3}}\dfrac{x^{2} - 7 \, x - 30}{x - 10}=\answer{0}\]
\end{problem}}%}

\latexProblemContent{
\ifVerboseLocation This is Derivative Compute Question 0003. \\ \fi
\begin{problem}

Determine if the limit approaches a finite number, $\pm\infty$, or does not exist. (If the limit does not exist, write DNE)

\input{Limit-Compute-0003.HELP.tex}

\[\lim_{x\to{2}}\dfrac{3 \, x^{2} + 24 \, x - 60}{x + 10}=\answer{0}\]
\end{problem}}%}

\latexProblemContent{
\ifVerboseLocation This is Derivative Compute Question 0003. \\ \fi
\begin{problem}

Determine if the limit approaches a finite number, $\pm\infty$, or does not exist. (If the limit does not exist, write DNE)

\input{Limit-Compute-0003.HELP.tex}

\[\lim_{x\to{-10}}\dfrac{3 \, x^{2} + 3 \, x - 270}{x - 9}=\answer{0}\]
\end{problem}}%}

\latexProblemContent{
\ifVerboseLocation This is Derivative Compute Question 0003. \\ \fi
\begin{problem}

Determine if the limit approaches a finite number, $\pm\infty$, or does not exist. (If the limit does not exist, write DNE)

\input{Limit-Compute-0003.HELP.tex}

\[\lim_{x\to{1}}\dfrac{x^{2} - 4 \, x + 3}{x - 3}=\answer{0}\]
\end{problem}}%}

\latexProblemContent{
\ifVerboseLocation This is Derivative Compute Question 0003. \\ \fi
\begin{problem}

Determine if the limit approaches a finite number, $\pm\infty$, or does not exist. (If the limit does not exist, write DNE)

\input{Limit-Compute-0003.HELP.tex}

\[\lim_{x\to{4}}\dfrac{2 \, x^{2} + 8 \, x - 64}{x + 8}=\answer{0}\]
\end{problem}}%}

\latexProblemContent{
\ifVerboseLocation This is Derivative Compute Question 0003. \\ \fi
\begin{problem}

Determine if the limit approaches a finite number, $\pm\infty$, or does not exist. (If the limit does not exist, write DNE)

\input{Limit-Compute-0003.HELP.tex}

\[\lim_{x\to{-5}}\dfrac{2 \, x^{2} + 14 \, x + 20}{x + 2}=\answer{0}\]
\end{problem}}%}

\latexProblemContent{
\ifVerboseLocation This is Derivative Compute Question 0003. \\ \fi
\begin{problem}

Determine if the limit approaches a finite number, $\pm\infty$, or does not exist. (If the limit does not exist, write DNE)

\input{Limit-Compute-0003.HELP.tex}

\[\lim_{x\to{-2}}\dfrac{3 \, x^{2} + 21 \, x + 30}{x + 5}=\answer{0}\]
\end{problem}}%}

\latexProblemContent{
\ifVerboseLocation This is Derivative Compute Question 0003. \\ \fi
\begin{problem}

Determine if the limit approaches a finite number, $\pm\infty$, or does not exist. (If the limit does not exist, write DNE)

\input{Limit-Compute-0003.HELP.tex}

\[\lim_{x\to{10}}\dfrac{3 \, x^{2} - 54 \, x + 240}{x - 8}=\answer{0}\]
\end{problem}}%}

\latexProblemContent{
\ifVerboseLocation This is Derivative Compute Question 0003. \\ \fi
\begin{problem}

Determine if the limit approaches a finite number, $\pm\infty$, or does not exist. (If the limit does not exist, write DNE)

\input{Limit-Compute-0003.HELP.tex}

\[\lim_{x\to{-1}}\dfrac{3 \, x^{2} + 30 \, x + 27}{x + 9}=\answer{0}\]
\end{problem}}%}

\latexProblemContent{
\ifVerboseLocation This is Derivative Compute Question 0003. \\ \fi
\begin{problem}

Determine if the limit approaches a finite number, $\pm\infty$, or does not exist. (If the limit does not exist, write DNE)

\input{Limit-Compute-0003.HELP.tex}

\[\lim_{x\to{9}}\dfrac{x^{2} - 12 \, x + 27}{x - 3}=\answer{0}\]
\end{problem}}%}

\latexProblemContent{
\ifVerboseLocation This is Derivative Compute Question 0003. \\ \fi
\begin{problem}

Determine if the limit approaches a finite number, $\pm\infty$, or does not exist. (If the limit does not exist, write DNE)

\input{Limit-Compute-0003.HELP.tex}

\[\lim_{x\to{1}}\dfrac{2 \, x^{2} - 20 \, x + 18}{x - 9}=\answer{0}\]
\end{problem}}%}

\latexProblemContent{
\ifVerboseLocation This is Derivative Compute Question 0003. \\ \fi
\begin{problem}

Determine if the limit approaches a finite number, $\pm\infty$, or does not exist. (If the limit does not exist, write DNE)

\input{Limit-Compute-0003.HELP.tex}

\[\lim_{x\to{8}}\dfrac{x^{2} - 5 \, x - 24}{x + 3}=\answer{0}\]
\end{problem}}%}

\latexProblemContent{
\ifVerboseLocation This is Derivative Compute Question 0003. \\ \fi
\begin{problem}

Determine if the limit approaches a finite number, $\pm\infty$, or does not exist. (If the limit does not exist, write DNE)

\input{Limit-Compute-0003.HELP.tex}

\[\lim_{x\to{-9}}\dfrac{2 \, x^{2} - 162}{x - 9}=\answer{0}\]
\end{problem}}%}

\latexProblemContent{
\ifVerboseLocation This is Derivative Compute Question 0003. \\ \fi
\begin{problem}

Determine if the limit approaches a finite number, $\pm\infty$, or does not exist. (If the limit does not exist, write DNE)

\input{Limit-Compute-0003.HELP.tex}

\[\lim_{x\to{1}}\dfrac{2 \, x^{2} - 22 \, x + 20}{x - 10}=\answer{0}\]
\end{problem}}%}

\latexProblemContent{
\ifVerboseLocation This is Derivative Compute Question 0003. \\ \fi
\begin{problem}

Determine if the limit approaches a finite number, $\pm\infty$, or does not exist. (If the limit does not exist, write DNE)

\input{Limit-Compute-0003.HELP.tex}

\[\lim_{x\to{3}}\dfrac{3 \, x^{2} + 18 \, x - 81}{x + 9}=\answer{0}\]
\end{problem}}%}

\latexProblemContent{
\ifVerboseLocation This is Derivative Compute Question 0003. \\ \fi
\begin{problem}

Determine if the limit approaches a finite number, $\pm\infty$, or does not exist. (If the limit does not exist, write DNE)

\input{Limit-Compute-0003.HELP.tex}

\[\lim_{x\to{-3}}\dfrac{3 \, x^{2} + 30 \, x + 63}{x + 7}=\answer{0}\]
\end{problem}}%}

\latexProblemContent{
\ifVerboseLocation This is Derivative Compute Question 0003. \\ \fi
\begin{problem}

Determine if the limit approaches a finite number, $\pm\infty$, or does not exist. (If the limit does not exist, write DNE)

\input{Limit-Compute-0003.HELP.tex}

\[\lim_{x\to{-1}}\dfrac{x^{2} - 3 \, x - 4}{x - 4}=\answer{0}\]
\end{problem}}%}

\latexProblemContent{
\ifVerboseLocation This is Derivative Compute Question 0003. \\ \fi
\begin{problem}

Determine if the limit approaches a finite number, $\pm\infty$, or does not exist. (If the limit does not exist, write DNE)

\input{Limit-Compute-0003.HELP.tex}

\[\lim_{x\to{4}}\dfrac{x^{2} - 6 \, x + 8}{x - 2}=\answer{0}\]
\end{problem}}%}

\latexProblemContent{
\ifVerboseLocation This is Derivative Compute Question 0003. \\ \fi
\begin{problem}

Determine if the limit approaches a finite number, $\pm\infty$, or does not exist. (If the limit does not exist, write DNE)

\input{Limit-Compute-0003.HELP.tex}

\[\lim_{x\to{-10}}\dfrac{x^{2} + x - 90}{x - 9}=\answer{0}\]
\end{problem}}%}

\latexProblemContent{
\ifVerboseLocation This is Derivative Compute Question 0003. \\ \fi
\begin{problem}

Determine if the limit approaches a finite number, $\pm\infty$, or does not exist. (If the limit does not exist, write DNE)

\input{Limit-Compute-0003.HELP.tex}

\[\lim_{x\to{-4}}\dfrac{3 \, x^{2} + 9 \, x - 12}{x - 1}=\answer{0}\]
\end{problem}}%}

\latexProblemContent{
\ifVerboseLocation This is Derivative Compute Question 0003. \\ \fi
\begin{problem}

Determine if the limit approaches a finite number, $\pm\infty$, or does not exist. (If the limit does not exist, write DNE)

\input{Limit-Compute-0003.HELP.tex}

\[\lim_{x\to{5}}\dfrac{3 \, x^{2} + 12 \, x - 135}{x + 9}=\answer{0}\]
\end{problem}}%}

\latexProblemContent{
\ifVerboseLocation This is Derivative Compute Question 0003. \\ \fi
\begin{problem}

Determine if the limit approaches a finite number, $\pm\infty$, or does not exist. (If the limit does not exist, write DNE)

\input{Limit-Compute-0003.HELP.tex}

\[\lim_{x\to{2}}\dfrac{2 \, x^{2} - 2 \, x - 4}{x + 1}=\answer{0}\]
\end{problem}}%}

\latexProblemContent{
\ifVerboseLocation This is Derivative Compute Question 0003. \\ \fi
\begin{problem}

Determine if the limit approaches a finite number, $\pm\infty$, or does not exist. (If the limit does not exist, write DNE)

\input{Limit-Compute-0003.HELP.tex}

\[\lim_{x\to{-10}}\dfrac{x^{2} + 19 \, x + 90}{x + 9}=\answer{0}\]
\end{problem}}%}

\latexProblemContent{
\ifVerboseLocation This is Derivative Compute Question 0003. \\ \fi
\begin{problem}

Determine if the limit approaches a finite number, $\pm\infty$, or does not exist. (If the limit does not exist, write DNE)

\input{Limit-Compute-0003.HELP.tex}

\[\lim_{x\to{-9}}\dfrac{x^{2} + 2 \, x - 63}{x - 7}=\answer{0}\]
\end{problem}}%}

\latexProblemContent{
\ifVerboseLocation This is Derivative Compute Question 0003. \\ \fi
\begin{problem}

Determine if the limit approaches a finite number, $\pm\infty$, or does not exist. (If the limit does not exist, write DNE)

\input{Limit-Compute-0003.HELP.tex}

\[\lim_{x\to{-3}}\dfrac{x^{2} + 11 \, x + 24}{x + 8}=\answer{0}\]
\end{problem}}%}

\latexProblemContent{
\ifVerboseLocation This is Derivative Compute Question 0003. \\ \fi
\begin{problem}

Determine if the limit approaches a finite number, $\pm\infty$, or does not exist. (If the limit does not exist, write DNE)

\input{Limit-Compute-0003.HELP.tex}

\[\lim_{x\to{-2}}\dfrac{2 \, x^{2} + 6 \, x + 4}{x + 1}=\answer{0}\]
\end{problem}}%}

\latexProblemContent{
\ifVerboseLocation This is Derivative Compute Question 0003. \\ \fi
\begin{problem}

Determine if the limit approaches a finite number, $\pm\infty$, or does not exist. (If the limit does not exist, write DNE)

\input{Limit-Compute-0003.HELP.tex}

\[\lim_{x\to{-6}}\dfrac{3 \, x^{2} + 33 \, x + 90}{x + 5}=\answer{0}\]
\end{problem}}%}

\latexProblemContent{
\ifVerboseLocation This is Derivative Compute Question 0003. \\ \fi
\begin{problem}

Determine if the limit approaches a finite number, $\pm\infty$, or does not exist. (If the limit does not exist, write DNE)

\input{Limit-Compute-0003.HELP.tex}

\[\lim_{x\to{5}}\dfrac{2 \, x^{2} - 50}{x + 5}=\answer{0}\]
\end{problem}}%}

\latexProblemContent{
\ifVerboseLocation This is Derivative Compute Question 0003. \\ \fi
\begin{problem}

Determine if the limit approaches a finite number, $\pm\infty$, or does not exist. (If the limit does not exist, write DNE)

\input{Limit-Compute-0003.HELP.tex}

\[\lim_{x\to{-4}}\dfrac{x^{2} - 16}{x - 4}=\answer{0}\]
\end{problem}}%}

\latexProblemContent{
\ifVerboseLocation This is Derivative Compute Question 0003. \\ \fi
\begin{problem}

Determine if the limit approaches a finite number, $\pm\infty$, or does not exist. (If the limit does not exist, write DNE)

\input{Limit-Compute-0003.HELP.tex}

\[\lim_{x\to{-8}}\dfrac{x^{2} - 64}{x - 8}=\answer{0}\]
\end{problem}}%}

\latexProblemContent{
\ifVerboseLocation This is Derivative Compute Question 0003. \\ \fi
\begin{problem}

Determine if the limit approaches a finite number, $\pm\infty$, or does not exist. (If the limit does not exist, write DNE)

\input{Limit-Compute-0003.HELP.tex}

\[\lim_{x\to{9}}\dfrac{x^{2} - 5 \, x - 36}{x + 4}=\answer{0}\]
\end{problem}}%}

\latexProblemContent{
\ifVerboseLocation This is Derivative Compute Question 0003. \\ \fi
\begin{problem}

Determine if the limit approaches a finite number, $\pm\infty$, or does not exist. (If the limit does not exist, write DNE)

\input{Limit-Compute-0003.HELP.tex}

\[\lim_{x\to{-5}}\dfrac{x^{2} + 3 \, x - 10}{x - 2}=\answer{0}\]
\end{problem}}%}

\latexProblemContent{
\ifVerboseLocation This is Derivative Compute Question 0003. \\ \fi
\begin{problem}

Determine if the limit approaches a finite number, $\pm\infty$, or does not exist. (If the limit does not exist, write DNE)

\input{Limit-Compute-0003.HELP.tex}

\[\lim_{x\to{10}}\dfrac{2 \, x^{2} - 10 \, x - 100}{x + 5}=\answer{0}\]
\end{problem}}%}

\latexProblemContent{
\ifVerboseLocation This is Derivative Compute Question 0003. \\ \fi
\begin{problem}

Determine if the limit approaches a finite number, $\pm\infty$, or does not exist. (If the limit does not exist, write DNE)

\input{Limit-Compute-0003.HELP.tex}

\[\lim_{x\to{-9}}\dfrac{3 \, x^{2} + 30 \, x + 27}{x + 1}=\answer{0}\]
\end{problem}}%}

\latexProblemContent{
\ifVerboseLocation This is Derivative Compute Question 0003. \\ \fi
\begin{problem}

Determine if the limit approaches a finite number, $\pm\infty$, or does not exist. (If the limit does not exist, write DNE)

\input{Limit-Compute-0003.HELP.tex}

\[\lim_{x\to{7}}\dfrac{3 \, x^{2} - 24 \, x + 21}{x - 1}=\answer{0}\]
\end{problem}}%}

\latexProblemContent{
\ifVerboseLocation This is Derivative Compute Question 0003. \\ \fi
\begin{problem}

Determine if the limit approaches a finite number, $\pm\infty$, or does not exist. (If the limit does not exist, write DNE)

\input{Limit-Compute-0003.HELP.tex}

\[\lim_{x\to{10}}\dfrac{2 \, x^{2} - 38 \, x + 180}{x - 9}=\answer{0}\]
\end{problem}}%}

\latexProblemContent{
\ifVerboseLocation This is Derivative Compute Question 0003. \\ \fi
\begin{problem}

Determine if the limit approaches a finite number, $\pm\infty$, or does not exist. (If the limit does not exist, write DNE)

\input{Limit-Compute-0003.HELP.tex}

\[\lim_{x\to{7}}\dfrac{x^{2} - 8 \, x + 7}{x - 1}=\answer{0}\]
\end{problem}}%}

\latexProblemContent{
\ifVerboseLocation This is Derivative Compute Question 0003. \\ \fi
\begin{problem}

Determine if the limit approaches a finite number, $\pm\infty$, or does not exist. (If the limit does not exist, write DNE)

\input{Limit-Compute-0003.HELP.tex}

\[\lim_{x\to{-5}}\dfrac{x^{2} + 2 \, x - 15}{x - 3}=\answer{0}\]
\end{problem}}%}

\latexProblemContent{
\ifVerboseLocation This is Derivative Compute Question 0003. \\ \fi
\begin{problem}

Determine if the limit approaches a finite number, $\pm\infty$, or does not exist. (If the limit does not exist, write DNE)

\input{Limit-Compute-0003.HELP.tex}

\[\lim_{x\to{-8}}\dfrac{x^{2} + 10 \, x + 16}{x + 2}=\answer{0}\]
\end{problem}}%}

\latexProblemContent{
\ifVerboseLocation This is Derivative Compute Question 0003. \\ \fi
\begin{problem}

Determine if the limit approaches a finite number, $\pm\infty$, or does not exist. (If the limit does not exist, write DNE)

\input{Limit-Compute-0003.HELP.tex}

\[\lim_{x\to{-4}}\dfrac{3 \, x^{2} + 39 \, x + 108}{x + 9}=\answer{0}\]
\end{problem}}%}

\latexProblemContent{
\ifVerboseLocation This is Derivative Compute Question 0003. \\ \fi
\begin{problem}

Determine if the limit approaches a finite number, $\pm\infty$, or does not exist. (If the limit does not exist, write DNE)

\input{Limit-Compute-0003.HELP.tex}

\[\lim_{x\to{-3}}\dfrac{3 \, x^{2} + 39 \, x + 90}{x + 10}=\answer{0}\]
\end{problem}}%}

\latexProblemContent{
\ifVerboseLocation This is Derivative Compute Question 0003. \\ \fi
\begin{problem}

Determine if the limit approaches a finite number, $\pm\infty$, or does not exist. (If the limit does not exist, write DNE)

\input{Limit-Compute-0003.HELP.tex}

\[\lim_{x\to{-6}}\dfrac{2 \, x^{2} + 8 \, x - 24}{x - 2}=\answer{0}\]
\end{problem}}%}

\latexProblemContent{
\ifVerboseLocation This is Derivative Compute Question 0003. \\ \fi
\begin{problem}

Determine if the limit approaches a finite number, $\pm\infty$, or does not exist. (If the limit does not exist, write DNE)

\input{Limit-Compute-0003.HELP.tex}

\[\lim_{x\to{-4}}\dfrac{2 \, x^{2} + 6 \, x - 8}{x - 1}=\answer{0}\]
\end{problem}}%}

\latexProblemContent{
\ifVerboseLocation This is Derivative Compute Question 0003. \\ \fi
\begin{problem}

Determine if the limit approaches a finite number, $\pm\infty$, or does not exist. (If the limit does not exist, write DNE)

\input{Limit-Compute-0003.HELP.tex}

\[\lim_{x\to{3}}\dfrac{3 \, x^{2} - 39 \, x + 90}{x - 10}=\answer{0}\]
\end{problem}}%}

\latexProblemContent{
\ifVerboseLocation This is Derivative Compute Question 0003. \\ \fi
\begin{problem}

Determine if the limit approaches a finite number, $\pm\infty$, or does not exist. (If the limit does not exist, write DNE)

\input{Limit-Compute-0003.HELP.tex}

\[\lim_{x\to{4}}\dfrac{3 \, x^{2} - 27 \, x + 60}{x - 5}=\answer{0}\]
\end{problem}}%}

\latexProblemContent{
\ifVerboseLocation This is Derivative Compute Question 0003. \\ \fi
\begin{problem}

Determine if the limit approaches a finite number, $\pm\infty$, or does not exist. (If the limit does not exist, write DNE)

\input{Limit-Compute-0003.HELP.tex}

\[\lim_{x\to{-10}}\dfrac{2 \, x^{2} + 8 \, x - 120}{x - 6}=\answer{0}\]
\end{problem}}%}

\latexProblemContent{
\ifVerboseLocation This is Derivative Compute Question 0003. \\ \fi
\begin{problem}

Determine if the limit approaches a finite number, $\pm\infty$, or does not exist. (If the limit does not exist, write DNE)

\input{Limit-Compute-0003.HELP.tex}

\[\lim_{x\to{2}}\dfrac{x^{2} + 3 \, x - 10}{x + 5}=\answer{0}\]
\end{problem}}%}

\latexProblemContent{
\ifVerboseLocation This is Derivative Compute Question 0003. \\ \fi
\begin{problem}

Determine if the limit approaches a finite number, $\pm\infty$, or does not exist. (If the limit does not exist, write DNE)

\input{Limit-Compute-0003.HELP.tex}

\[\lim_{x\to{-7}}\dfrac{3 \, x^{2} + 24 \, x + 21}{x + 1}=\answer{0}\]
\end{problem}}%}

\latexProblemContent{
\ifVerboseLocation This is Derivative Compute Question 0003. \\ \fi
\begin{problem}

Determine if the limit approaches a finite number, $\pm\infty$, or does not exist. (If the limit does not exist, write DNE)

\input{Limit-Compute-0003.HELP.tex}

\[\lim_{x\to{-2}}\dfrac{x^{2} + 7 \, x + 10}{x + 5}=\answer{0}\]
\end{problem}}%}

\latexProblemContent{
\ifVerboseLocation This is Derivative Compute Question 0003. \\ \fi
\begin{problem}

Determine if the limit approaches a finite number, $\pm\infty$, or does not exist. (If the limit does not exist, write DNE)

\input{Limit-Compute-0003.HELP.tex}

\[\lim_{x\to{-4}}\dfrac{3 \, x^{2} + 33 \, x + 84}{x + 7}=\answer{0}\]
\end{problem}}%}

\latexProblemContent{
\ifVerboseLocation This is Derivative Compute Question 0003. \\ \fi
\begin{problem}

Determine if the limit approaches a finite number, $\pm\infty$, or does not exist. (If the limit does not exist, write DNE)

\input{Limit-Compute-0003.HELP.tex}

\[\lim_{x\to{-3}}\dfrac{3 \, x^{2} + 6 \, x - 9}{x - 1}=\answer{0}\]
\end{problem}}%}

\latexProblemContent{
\ifVerboseLocation This is Derivative Compute Question 0003. \\ \fi
\begin{problem}

Determine if the limit approaches a finite number, $\pm\infty$, or does not exist. (If the limit does not exist, write DNE)

\input{Limit-Compute-0003.HELP.tex}

\[\lim_{x\to{-2}}\dfrac{x^{2} - 6 \, x - 16}{x - 8}=\answer{0}\]
\end{problem}}%}

\latexProblemContent{
\ifVerboseLocation This is Derivative Compute Question 0003. \\ \fi
\begin{problem}

Determine if the limit approaches a finite number, $\pm\infty$, or does not exist. (If the limit does not exist, write DNE)

\input{Limit-Compute-0003.HELP.tex}

\[\lim_{x\to{8}}\dfrac{x^{2} - 64}{x + 8}=\answer{0}\]
\end{problem}}%}

\latexProblemContent{
\ifVerboseLocation This is Derivative Compute Question 0003. \\ \fi
\begin{problem}

Determine if the limit approaches a finite number, $\pm\infty$, or does not exist. (If the limit does not exist, write DNE)

\input{Limit-Compute-0003.HELP.tex}

\[\lim_{x\to{2}}\dfrac{3 \, x^{2} - 18 \, x + 24}{x - 4}=\answer{0}\]
\end{problem}}%}

\latexProblemContent{
\ifVerboseLocation This is Derivative Compute Question 0003. \\ \fi
\begin{problem}

Determine if the limit approaches a finite number, $\pm\infty$, or does not exist. (If the limit does not exist, write DNE)

\input{Limit-Compute-0003.HELP.tex}

\[\lim_{x\to{-1}}\dfrac{3 \, x^{2} - 12 \, x - 15}{x - 5}=\answer{0}\]
\end{problem}}%}

\latexProblemContent{
\ifVerboseLocation This is Derivative Compute Question 0003. \\ \fi
\begin{problem}

Determine if the limit approaches a finite number, $\pm\infty$, or does not exist. (If the limit does not exist, write DNE)

\input{Limit-Compute-0003.HELP.tex}

\[\lim_{x\to{-2}}\dfrac{3 \, x^{2} - 15 \, x - 42}{x - 7}=\answer{0}\]
\end{problem}}%}

\latexProblemContent{
\ifVerboseLocation This is Derivative Compute Question 0003. \\ \fi
\begin{problem}

Determine if the limit approaches a finite number, $\pm\infty$, or does not exist. (If the limit does not exist, write DNE)

\input{Limit-Compute-0003.HELP.tex}

\[\lim_{x\to{8}}\dfrac{x^{2} - x - 56}{x + 7}=\answer{0}\]
\end{problem}}%}

\latexProblemContent{
\ifVerboseLocation This is Derivative Compute Question 0003. \\ \fi
\begin{problem}

Determine if the limit approaches a finite number, $\pm\infty$, or does not exist. (If the limit does not exist, write DNE)

\input{Limit-Compute-0003.HELP.tex}

\[\lim_{x\to{3}}\dfrac{x^{2} + 7 \, x - 30}{x + 10}=\answer{0}\]
\end{problem}}%}

\latexProblemContent{
\ifVerboseLocation This is Derivative Compute Question 0003. \\ \fi
\begin{problem}

Determine if the limit approaches a finite number, $\pm\infty$, or does not exist. (If the limit does not exist, write DNE)

\input{Limit-Compute-0003.HELP.tex}

\[\lim_{x\to{-5}}\dfrac{3 \, x^{2} - 3 \, x - 90}{x - 6}=\answer{0}\]
\end{problem}}%}

\latexProblemContent{
\ifVerboseLocation This is Derivative Compute Question 0003. \\ \fi
\begin{problem}

Determine if the limit approaches a finite number, $\pm\infty$, or does not exist. (If the limit does not exist, write DNE)

\input{Limit-Compute-0003.HELP.tex}

\[\lim_{x\to{-1}}\dfrac{x^{2} + 6 \, x + 5}{x + 5}=\answer{0}\]
\end{problem}}%}

\latexProblemContent{
\ifVerboseLocation This is Derivative Compute Question 0003. \\ \fi
\begin{problem}

Determine if the limit approaches a finite number, $\pm\infty$, or does not exist. (If the limit does not exist, write DNE)

\input{Limit-Compute-0003.HELP.tex}

\[\lim_{x\to{8}}\dfrac{3 \, x^{2} - 30 \, x + 48}{x - 2}=\answer{0}\]
\end{problem}}%}

\latexProblemContent{
\ifVerboseLocation This is Derivative Compute Question 0003. \\ \fi
\begin{problem}

Determine if the limit approaches a finite number, $\pm\infty$, or does not exist. (If the limit does not exist, write DNE)

\input{Limit-Compute-0003.HELP.tex}

\[\lim_{x\to{10}}\dfrac{x^{2} - 15 \, x + 50}{x - 5}=\answer{0}\]
\end{problem}}%}

\latexProblemContent{
\ifVerboseLocation This is Derivative Compute Question 0003. \\ \fi
\begin{problem}

Determine if the limit approaches a finite number, $\pm\infty$, or does not exist. (If the limit does not exist, write DNE)

\input{Limit-Compute-0003.HELP.tex}

\[\lim_{x\to{-10}}\dfrac{3 \, x^{2} + 48 \, x + 180}{x + 6}=\answer{0}\]
\end{problem}}%}

\latexProblemContent{
\ifVerboseLocation This is Derivative Compute Question 0003. \\ \fi
\begin{problem}

Determine if the limit approaches a finite number, $\pm\infty$, or does not exist. (If the limit does not exist, write DNE)

\input{Limit-Compute-0003.HELP.tex}

\[\lim_{x\to{5}}\dfrac{x^{2} - 14 \, x + 45}{x - 9}=\answer{0}\]
\end{problem}}%}

\latexProblemContent{
\ifVerboseLocation This is Derivative Compute Question 0003. \\ \fi
\begin{problem}

Determine if the limit approaches a finite number, $\pm\infty$, or does not exist. (If the limit does not exist, write DNE)

\input{Limit-Compute-0003.HELP.tex}

\[\lim_{x\to{-9}}\dfrac{3 \, x^{2} + 57 \, x + 270}{x + 10}=\answer{0}\]
\end{problem}}%}

\latexProblemContent{
\ifVerboseLocation This is Derivative Compute Question 0003. \\ \fi
\begin{problem}

Determine if the limit approaches a finite number, $\pm\infty$, or does not exist. (If the limit does not exist, write DNE)

\input{Limit-Compute-0003.HELP.tex}

\[\lim_{x\to{-6}}\dfrac{x^{2} + 3 \, x - 18}{x - 3}=\answer{0}\]
\end{problem}}%}

\latexProblemContent{
\ifVerboseLocation This is Derivative Compute Question 0003. \\ \fi
\begin{problem}

Determine if the limit approaches a finite number, $\pm\infty$, or does not exist. (If the limit does not exist, write DNE)

\input{Limit-Compute-0003.HELP.tex}

\[\lim_{x\to{9}}\dfrac{3 \, x^{2} - 48 \, x + 189}{x - 7}=\answer{0}\]
\end{problem}}%}

\latexProblemContent{
\ifVerboseLocation This is Derivative Compute Question 0003. \\ \fi
\begin{problem}

Determine if the limit approaches a finite number, $\pm\infty$, or does not exist. (If the limit does not exist, write DNE)

\input{Limit-Compute-0003.HELP.tex}

\[\lim_{x\to{7}}\dfrac{2 \, x^{2} - 20 \, x + 42}{x - 3}=\answer{0}\]
\end{problem}}%}

\latexProblemContent{
\ifVerboseLocation This is Derivative Compute Question 0003. \\ \fi
\begin{problem}

Determine if the limit approaches a finite number, $\pm\infty$, or does not exist. (If the limit does not exist, write DNE)

\input{Limit-Compute-0003.HELP.tex}

\[\lim_{x\to{-10}}\dfrac{2 \, x^{2} + 34 \, x + 140}{x + 7}=\answer{0}\]
\end{problem}}%}

\latexProblemContent{
\ifVerboseLocation This is Derivative Compute Question 0003. \\ \fi
\begin{problem}

Determine if the limit approaches a finite number, $\pm\infty$, or does not exist. (If the limit does not exist, write DNE)

\input{Limit-Compute-0003.HELP.tex}

\[\lim_{x\to{-9}}\dfrac{2 \, x^{2} + 16 \, x - 18}{x - 1}=\answer{0}\]
\end{problem}}%}

\latexProblemContent{
\ifVerboseLocation This is Derivative Compute Question 0003. \\ \fi
\begin{problem}

Determine if the limit approaches a finite number, $\pm\infty$, or does not exist. (If the limit does not exist, write DNE)

\input{Limit-Compute-0003.HELP.tex}

\[\lim_{x\to{4}}\dfrac{3 \, x^{2} - 15 \, x + 12}{x - 1}=\answer{0}\]
\end{problem}}%}

\latexProblemContent{
\ifVerboseLocation This is Derivative Compute Question 0003. \\ \fi
\begin{problem}

Determine if the limit approaches a finite number, $\pm\infty$, or does not exist. (If the limit does not exist, write DNE)

\input{Limit-Compute-0003.HELP.tex}

\[\lim_{x\to{-8}}\dfrac{3 \, x^{2} + 39 \, x + 120}{x + 5}=\answer{0}\]
\end{problem}}%}

\latexProblemContent{
\ifVerboseLocation This is Derivative Compute Question 0003. \\ \fi
\begin{problem}

Determine if the limit approaches a finite number, $\pm\infty$, or does not exist. (If the limit does not exist, write DNE)

\input{Limit-Compute-0003.HELP.tex}

\[\lim_{x\to{-5}}\dfrac{3 \, x^{2} - 6 \, x - 105}{x - 7}=\answer{0}\]
\end{problem}}%}

\latexProblemContent{
\ifVerboseLocation This is Derivative Compute Question 0003. \\ \fi
\begin{problem}

Determine if the limit approaches a finite number, $\pm\infty$, or does not exist. (If the limit does not exist, write DNE)

\input{Limit-Compute-0003.HELP.tex}

\[\lim_{x\to{1}}\dfrac{3 \, x^{2} - 24 \, x + 21}{x - 7}=\answer{0}\]
\end{problem}}%}

\latexProblemContent{
\ifVerboseLocation This is Derivative Compute Question 0003. \\ \fi
\begin{problem}

Determine if the limit approaches a finite number, $\pm\infty$, or does not exist. (If the limit does not exist, write DNE)

\input{Limit-Compute-0003.HELP.tex}

\[\lim_{x\to{8}}\dfrac{x^{2} - 18 \, x + 80}{x - 10}=\answer{0}\]
\end{problem}}%}

\latexProblemContent{
\ifVerboseLocation This is Derivative Compute Question 0003. \\ \fi
\begin{problem}

Determine if the limit approaches a finite number, $\pm\infty$, or does not exist. (If the limit does not exist, write DNE)

\input{Limit-Compute-0003.HELP.tex}

\[\lim_{x\to{-7}}\dfrac{2 \, x^{2} - 4 \, x - 126}{x - 9}=\answer{0}\]
\end{problem}}%}

\latexProblemContent{
\ifVerboseLocation This is Derivative Compute Question 0003. \\ \fi
\begin{problem}

Determine if the limit approaches a finite number, $\pm\infty$, or does not exist. (If the limit does not exist, write DNE)

\input{Limit-Compute-0003.HELP.tex}

\[\lim_{x\to{5}}\dfrac{3 \, x^{2} + 9 \, x - 120}{x + 8}=\answer{0}\]
\end{problem}}%}

\latexProblemContent{
\ifVerboseLocation This is Derivative Compute Question 0003. \\ \fi
\begin{problem}

Determine if the limit approaches a finite number, $\pm\infty$, or does not exist. (If the limit does not exist, write DNE)

\input{Limit-Compute-0003.HELP.tex}

\[\lim_{x\to{2}}\dfrac{3 \, x^{2} - 36 \, x + 60}{x - 10}=\answer{0}\]
\end{problem}}%}

\latexProblemContent{
\ifVerboseLocation This is Derivative Compute Question 0003. \\ \fi
\begin{problem}

Determine if the limit approaches a finite number, $\pm\infty$, or does not exist. (If the limit does not exist, write DNE)

\input{Limit-Compute-0003.HELP.tex}

\[\lim_{x\to{-7}}\dfrac{x^{2} + 17 \, x + 70}{x + 10}=\answer{0}\]
\end{problem}}%}

\latexProblemContent{
\ifVerboseLocation This is Derivative Compute Question 0003. \\ \fi
\begin{problem}

Determine if the limit approaches a finite number, $\pm\infty$, or does not exist. (If the limit does not exist, write DNE)

\input{Limit-Compute-0003.HELP.tex}

\[\lim_{x\to{5}}\dfrac{x^{2} + 2 \, x - 35}{x + 7}=\answer{0}\]
\end{problem}}%}

\latexProblemContent{
\ifVerboseLocation This is Derivative Compute Question 0003. \\ \fi
\begin{problem}

Determine if the limit approaches a finite number, $\pm\infty$, or does not exist. (If the limit does not exist, write DNE)

\input{Limit-Compute-0003.HELP.tex}

\[\lim_{x\to{7}}\dfrac{2 \, x^{2} - 30 \, x + 112}{x - 8}=\answer{0}\]
\end{problem}}%}

\latexProblemContent{
\ifVerboseLocation This is Derivative Compute Question 0003. \\ \fi
\begin{problem}

Determine if the limit approaches a finite number, $\pm\infty$, or does not exist. (If the limit does not exist, write DNE)

\input{Limit-Compute-0003.HELP.tex}

\[\lim_{x\to{4}}\dfrac{x^{2} - 2 \, x - 8}{x + 2}=\answer{0}\]
\end{problem}}%}

\latexProblemContent{
\ifVerboseLocation This is Derivative Compute Question 0003. \\ \fi
\begin{problem}

Determine if the limit approaches a finite number, $\pm\infty$, or does not exist. (If the limit does not exist, write DNE)

\input{Limit-Compute-0003.HELP.tex}

\[\lim_{x\to{-5}}\dfrac{x^{2} - x - 30}{x - 6}=\answer{0}\]
\end{problem}}%}

\latexProblemContent{
\ifVerboseLocation This is Derivative Compute Question 0003. \\ \fi
\begin{problem}

Determine if the limit approaches a finite number, $\pm\infty$, or does not exist. (If the limit does not exist, write DNE)

\input{Limit-Compute-0003.HELP.tex}

\[\lim_{x\to{10}}\dfrac{x^{2} - 6 \, x - 40}{x + 4}=\answer{0}\]
\end{problem}}%}

\latexProblemContent{
\ifVerboseLocation This is Derivative Compute Question 0003. \\ \fi
\begin{problem}

Determine if the limit approaches a finite number, $\pm\infty$, or does not exist. (If the limit does not exist, write DNE)

\input{Limit-Compute-0003.HELP.tex}

\[\lim_{x\to{-2}}\dfrac{3 \, x^{2} + 33 \, x + 54}{x + 9}=\answer{0}\]
\end{problem}}%}

\latexProblemContent{
\ifVerboseLocation This is Derivative Compute Question 0003. \\ \fi
\begin{problem}

Determine if the limit approaches a finite number, $\pm\infty$, or does not exist. (If the limit does not exist, write DNE)

\input{Limit-Compute-0003.HELP.tex}

\[\lim_{x\to{9}}\dfrac{2 \, x^{2} - 16 \, x - 18}{x + 1}=\answer{0}\]
\end{problem}}%}

\latexProblemContent{
\ifVerboseLocation This is Derivative Compute Question 0003. \\ \fi
\begin{problem}

Determine if the limit approaches a finite number, $\pm\infty$, or does not exist. (If the limit does not exist, write DNE)

\input{Limit-Compute-0003.HELP.tex}

\[\lim_{x\to{-3}}\dfrac{x^{2} + 9 \, x + 18}{x + 6}=\answer{0}\]
\end{problem}}%}

\latexProblemContent{
\ifVerboseLocation This is Derivative Compute Question 0003. \\ \fi
\begin{problem}

Determine if the limit approaches a finite number, $\pm\infty$, or does not exist. (If the limit does not exist, write DNE)

\input{Limit-Compute-0003.HELP.tex}

\[\lim_{x\to{5}}\dfrac{2 \, x^{2} - 28 \, x + 90}{x - 9}=\answer{0}\]
\end{problem}}%}

\latexProblemContent{
\ifVerboseLocation This is Derivative Compute Question 0003. \\ \fi
\begin{problem}

Determine if the limit approaches a finite number, $\pm\infty$, or does not exist. (If the limit does not exist, write DNE)

\input{Limit-Compute-0003.HELP.tex}

\[\lim_{x\to{5}}\dfrac{3 \, x^{2} - 36 \, x + 105}{x - 7}=\answer{0}\]
\end{problem}}%}

\latexProblemContent{
\ifVerboseLocation This is Derivative Compute Question 0003. \\ \fi
\begin{problem}

Determine if the limit approaches a finite number, $\pm\infty$, or does not exist. (If the limit does not exist, write DNE)

\input{Limit-Compute-0003.HELP.tex}

\[\lim_{x\to{5}}\dfrac{x^{2} - 8 \, x + 15}{x - 3}=\answer{0}\]
\end{problem}}%}

\latexProblemContent{
\ifVerboseLocation This is Derivative Compute Question 0003. \\ \fi
\begin{problem}

Determine if the limit approaches a finite number, $\pm\infty$, or does not exist. (If the limit does not exist, write DNE)

\input{Limit-Compute-0003.HELP.tex}

\[\lim_{x\to{-8}}\dfrac{3 \, x^{2} + 18 \, x - 48}{x - 2}=\answer{0}\]
\end{problem}}%}

\latexProblemContent{
\ifVerboseLocation This is Derivative Compute Question 0003. \\ \fi
\begin{problem}

Determine if the limit approaches a finite number, $\pm\infty$, or does not exist. (If the limit does not exist, write DNE)

\input{Limit-Compute-0003.HELP.tex}

\[\lim_{x\to{2}}\dfrac{x^{2} + 7 \, x - 18}{x + 9}=\answer{0}\]
\end{problem}}%}

\latexProblemContent{
\ifVerboseLocation This is Derivative Compute Question 0003. \\ \fi
\begin{problem}

Determine if the limit approaches a finite number, $\pm\infty$, or does not exist. (If the limit does not exist, write DNE)

\input{Limit-Compute-0003.HELP.tex}

\[\lim_{x\to{2}}\dfrac{3 \, x^{2} - 9 \, x + 6}{x - 1}=\answer{0}\]
\end{problem}}%}

\latexProblemContent{
\ifVerboseLocation This is Derivative Compute Question 0003. \\ \fi
\begin{problem}

Determine if the limit approaches a finite number, $\pm\infty$, or does not exist. (If the limit does not exist, write DNE)

\input{Limit-Compute-0003.HELP.tex}

\[\lim_{x\to{-10}}\dfrac{2 \, x^{2} + 6 \, x - 140}{x - 7}=\answer{0}\]
\end{problem}}%}

\latexProblemContent{
\ifVerboseLocation This is Derivative Compute Question 0003. \\ \fi
\begin{problem}

Determine if the limit approaches a finite number, $\pm\infty$, or does not exist. (If the limit does not exist, write DNE)

\input{Limit-Compute-0003.HELP.tex}

\[\lim_{x\to{-7}}\dfrac{2 \, x^{2} + 26 \, x + 84}{x + 6}=\answer{0}\]
\end{problem}}%}

\latexProblemContent{
\ifVerboseLocation This is Derivative Compute Question 0003. \\ \fi
\begin{problem}

Determine if the limit approaches a finite number, $\pm\infty$, or does not exist. (If the limit does not exist, write DNE)

\input{Limit-Compute-0003.HELP.tex}

\[\lim_{x\to{6}}\dfrac{2 \, x^{2} - 72}{x + 6}=\answer{0}\]
\end{problem}}%}

\latexProblemContent{
\ifVerboseLocation This is Derivative Compute Question 0003. \\ \fi
\begin{problem}

Determine if the limit approaches a finite number, $\pm\infty$, or does not exist. (If the limit does not exist, write DNE)

\input{Limit-Compute-0003.HELP.tex}

\[\lim_{x\to{-6}}\dfrac{2 \, x^{2} + 22 \, x + 60}{x + 5}=\answer{0}\]
\end{problem}}%}

\latexProblemContent{
\ifVerboseLocation This is Derivative Compute Question 0003. \\ \fi
\begin{problem}

Determine if the limit approaches a finite number, $\pm\infty$, or does not exist. (If the limit does not exist, write DNE)

\input{Limit-Compute-0003.HELP.tex}

\[\lim_{x\to{-7}}\dfrac{2 \, x^{2} + 10 \, x - 28}{x - 2}=\answer{0}\]
\end{problem}}%}

\latexProblemContent{
\ifVerboseLocation This is Derivative Compute Question 0003. \\ \fi
\begin{problem}

Determine if the limit approaches a finite number, $\pm\infty$, or does not exist. (If the limit does not exist, write DNE)

\input{Limit-Compute-0003.HELP.tex}

\[\lim_{x\to{-5}}\dfrac{3 \, x^{2} - 9 \, x - 120}{x - 8}=\answer{0}\]
\end{problem}}%}

\latexProblemContent{
\ifVerboseLocation This is Derivative Compute Question 0003. \\ \fi
\begin{problem}

Determine if the limit approaches a finite number, $\pm\infty$, or does not exist. (If the limit does not exist, write DNE)

\input{Limit-Compute-0003.HELP.tex}

\[\lim_{x\to{3}}\dfrac{2 \, x^{2} - 4 \, x - 6}{x + 1}=\answer{0}\]
\end{problem}}%}

\latexProblemContent{
\ifVerboseLocation This is Derivative Compute Question 0003. \\ \fi
\begin{problem}

Determine if the limit approaches a finite number, $\pm\infty$, or does not exist. (If the limit does not exist, write DNE)

\input{Limit-Compute-0003.HELP.tex}

\[\lim_{x\to{5}}\dfrac{3 \, x^{2} - 42 \, x + 135}{x - 9}=\answer{0}\]
\end{problem}}%}

\latexProblemContent{
\ifVerboseLocation This is Derivative Compute Question 0003. \\ \fi
\begin{problem}

Determine if the limit approaches a finite number, $\pm\infty$, or does not exist. (If the limit does not exist, write DNE)

\input{Limit-Compute-0003.HELP.tex}

\[\lim_{x\to{3}}\dfrac{x^{2} - 10 \, x + 21}{x - 7}=\answer{0}\]
\end{problem}}%}

\latexProblemContent{
\ifVerboseLocation This is Derivative Compute Question 0003. \\ \fi
\begin{problem}

Determine if the limit approaches a finite number, $\pm\infty$, or does not exist. (If the limit does not exist, write DNE)

\input{Limit-Compute-0003.HELP.tex}

\[\lim_{x\to{10}}\dfrac{2 \, x^{2} - 32 \, x + 120}{x - 6}=\answer{0}\]
\end{problem}}%}

\latexProblemContent{
\ifVerboseLocation This is Derivative Compute Question 0003. \\ \fi
\begin{problem}

Determine if the limit approaches a finite number, $\pm\infty$, or does not exist. (If the limit does not exist, write DNE)

\input{Limit-Compute-0003.HELP.tex}

\[\lim_{x\to{-3}}\dfrac{x^{2} + 8 \, x + 15}{x + 5}=\answer{0}\]
\end{problem}}%}

\latexProblemContent{
\ifVerboseLocation This is Derivative Compute Question 0003. \\ \fi
\begin{problem}

Determine if the limit approaches a finite number, $\pm\infty$, or does not exist. (If the limit does not exist, write DNE)

\input{Limit-Compute-0003.HELP.tex}

\[\lim_{x\to{7}}\dfrac{2 \, x^{2} - 98}{x + 7}=\answer{0}\]
\end{problem}}%}

\latexProblemContent{
\ifVerboseLocation This is Derivative Compute Question 0003. \\ \fi
\begin{problem}

Determine if the limit approaches a finite number, $\pm\infty$, or does not exist. (If the limit does not exist, write DNE)

\input{Limit-Compute-0003.HELP.tex}

\[\lim_{x\to{-9}}\dfrac{x^{2} + 15 \, x + 54}{x + 6}=\answer{0}\]
\end{problem}}%}

\latexProblemContent{
\ifVerboseLocation This is Derivative Compute Question 0003. \\ \fi
\begin{problem}

Determine if the limit approaches a finite number, $\pm\infty$, or does not exist. (If the limit does not exist, write DNE)

\input{Limit-Compute-0003.HELP.tex}

\[\lim_{x\to{-5}}\dfrac{x^{2} + 7 \, x + 10}{x + 2}=\answer{0}\]
\end{problem}}%}

\latexProblemContent{
\ifVerboseLocation This is Derivative Compute Question 0003. \\ \fi
\begin{problem}

Determine if the limit approaches a finite number, $\pm\infty$, or does not exist. (If the limit does not exist, write DNE)

\input{Limit-Compute-0003.HELP.tex}

\[\lim_{x\to{-5}}\dfrac{2 \, x^{2} - 50}{x - 5}=\answer{0}\]
\end{problem}}%}

\latexProblemContent{
\ifVerboseLocation This is Derivative Compute Question 0003. \\ \fi
\begin{problem}

Determine if the limit approaches a finite number, $\pm\infty$, or does not exist. (If the limit does not exist, write DNE)

\input{Limit-Compute-0003.HELP.tex}

\[\lim_{x\to{-6}}\dfrac{x^{2} + 7 \, x + 6}{x + 1}=\answer{0}\]
\end{problem}}%}

\latexProblemContent{
\ifVerboseLocation This is Derivative Compute Question 0003. \\ \fi
\begin{problem}

Determine if the limit approaches a finite number, $\pm\infty$, or does not exist. (If the limit does not exist, write DNE)

\input{Limit-Compute-0003.HELP.tex}

\[\lim_{x\to{-8}}\dfrac{3 \, x^{2} + 42 \, x + 144}{x + 6}=\answer{0}\]
\end{problem}}%}

\latexProblemContent{
\ifVerboseLocation This is Derivative Compute Question 0003. \\ \fi
\begin{problem}

Determine if the limit approaches a finite number, $\pm\infty$, or does not exist. (If the limit does not exist, write DNE)

\input{Limit-Compute-0003.HELP.tex}

\[\lim_{x\to{-5}}\dfrac{2 \, x^{2} + 12 \, x + 10}{x + 1}=\answer{0}\]
\end{problem}}%}

\latexProblemContent{
\ifVerboseLocation This is Derivative Compute Question 0003. \\ \fi
\begin{problem}

Determine if the limit approaches a finite number, $\pm\infty$, or does not exist. (If the limit does not exist, write DNE)

\input{Limit-Compute-0003.HELP.tex}

\[\lim_{x\to{9}}\dfrac{3 \, x^{2} - 12 \, x - 135}{x + 5}=\answer{0}\]
\end{problem}}%}

\latexProblemContent{
\ifVerboseLocation This is Derivative Compute Question 0003. \\ \fi
\begin{problem}

Determine if the limit approaches a finite number, $\pm\infty$, or does not exist. (If the limit does not exist, write DNE)

\input{Limit-Compute-0003.HELP.tex}

\[\lim_{x\to{-5}}\dfrac{x^{2} + 8 \, x + 15}{x + 3}=\answer{0}\]
\end{problem}}%}

\latexProblemContent{
\ifVerboseLocation This is Derivative Compute Question 0003. \\ \fi
\begin{problem}

Determine if the limit approaches a finite number, $\pm\infty$, or does not exist. (If the limit does not exist, write DNE)

\input{Limit-Compute-0003.HELP.tex}

\[\lim_{x\to{-4}}\dfrac{2 \, x^{2} + 26 \, x + 72}{x + 9}=\answer{0}\]
\end{problem}}%}

\latexProblemContent{
\ifVerboseLocation This is Derivative Compute Question 0003. \\ \fi
\begin{problem}

Determine if the limit approaches a finite number, $\pm\infty$, or does not exist. (If the limit does not exist, write DNE)

\input{Limit-Compute-0003.HELP.tex}

\[\lim_{x\to{-7}}\dfrac{3 \, x^{2} - 6 \, x - 189}{x - 9}=\answer{0}\]
\end{problem}}%}

\latexProblemContent{
\ifVerboseLocation This is Derivative Compute Question 0003. \\ \fi
\begin{problem}

Determine if the limit approaches a finite number, $\pm\infty$, or does not exist. (If the limit does not exist, write DNE)

\input{Limit-Compute-0003.HELP.tex}

\[\lim_{x\to{-3}}\dfrac{2 \, x^{2} - 10 \, x - 48}{x - 8}=\answer{0}\]
\end{problem}}%}

\latexProblemContent{
\ifVerboseLocation This is Derivative Compute Question 0003. \\ \fi
\begin{problem}

Determine if the limit approaches a finite number, $\pm\infty$, or does not exist. (If the limit does not exist, write DNE)

\input{Limit-Compute-0003.HELP.tex}

\[\lim_{x\to{-10}}\dfrac{x^{2} + 15 \, x + 50}{x + 5}=\answer{0}\]
\end{problem}}%}

\latexProblemContent{
\ifVerboseLocation This is Derivative Compute Question 0003. \\ \fi
\begin{problem}

Determine if the limit approaches a finite number, $\pm\infty$, or does not exist. (If the limit does not exist, write DNE)

\input{Limit-Compute-0003.HELP.tex}

\[\lim_{x\to{6}}\dfrac{2 \, x^{2} - 26 \, x + 84}{x - 7}=\answer{0}\]
\end{problem}}%}

\latexProblemContent{
\ifVerboseLocation This is Derivative Compute Question 0003. \\ \fi
\begin{problem}

Determine if the limit approaches a finite number, $\pm\infty$, or does not exist. (If the limit does not exist, write DNE)

\input{Limit-Compute-0003.HELP.tex}

\[\lim_{x\to{-5}}\dfrac{x^{2} - 25}{x - 5}=\answer{0}\]
\end{problem}}%}

\latexProblemContent{
\ifVerboseLocation This is Derivative Compute Question 0003. \\ \fi
\begin{problem}

Determine if the limit approaches a finite number, $\pm\infty$, or does not exist. (If the limit does not exist, write DNE)

\input{Limit-Compute-0003.HELP.tex}

\[\lim_{x\to{-1}}\dfrac{x^{2} + 10 \, x + 9}{x + 9}=\answer{0}\]
\end{problem}}%}

\latexProblemContent{
\ifVerboseLocation This is Derivative Compute Question 0003. \\ \fi
\begin{problem}

Determine if the limit approaches a finite number, $\pm\infty$, or does not exist. (If the limit does not exist, write DNE)

\input{Limit-Compute-0003.HELP.tex}

\[\lim_{x\to{6}}\dfrac{3 \, x^{2} - 30 \, x + 72}{x - 4}=\answer{0}\]
\end{problem}}%}

\latexProblemContent{
\ifVerboseLocation This is Derivative Compute Question 0003. \\ \fi
\begin{problem}

Determine if the limit approaches a finite number, $\pm\infty$, or does not exist. (If the limit does not exist, write DNE)

\input{Limit-Compute-0003.HELP.tex}

\[\lim_{x\to{1}}\dfrac{x^{2} - 1}{x + 1}=\answer{0}\]
\end{problem}}%}

\latexProblemContent{
\ifVerboseLocation This is Derivative Compute Question 0003. \\ \fi
\begin{problem}

Determine if the limit approaches a finite number, $\pm\infty$, or does not exist. (If the limit does not exist, write DNE)

\input{Limit-Compute-0003.HELP.tex}

\[\lim_{x\to{4}}\dfrac{2 \, x^{2} - 26 \, x + 72}{x - 9}=\answer{0}\]
\end{problem}}%}

\latexProblemContent{
\ifVerboseLocation This is Derivative Compute Question 0003. \\ \fi
\begin{problem}

Determine if the limit approaches a finite number, $\pm\infty$, or does not exist. (If the limit does not exist, write DNE)

\input{Limit-Compute-0003.HELP.tex}

\[\lim_{x\to{3}}\dfrac{2 \, x^{2} + 8 \, x - 42}{x + 7}=\answer{0}\]
\end{problem}}%}

\latexProblemContent{
\ifVerboseLocation This is Derivative Compute Question 0003. \\ \fi
\begin{problem}

Determine if the limit approaches a finite number, $\pm\infty$, or does not exist. (If the limit does not exist, write DNE)

\input{Limit-Compute-0003.HELP.tex}

\[\lim_{x\to{-9}}\dfrac{x^{2} + 14 \, x + 45}{x + 5}=\answer{0}\]
\end{problem}}%}

\latexProblemContent{
\ifVerboseLocation This is Derivative Compute Question 0003. \\ \fi
\begin{problem}

Determine if the limit approaches a finite number, $\pm\infty$, or does not exist. (If the limit does not exist, write DNE)

\input{Limit-Compute-0003.HELP.tex}

\[\lim_{x\to{-6}}\dfrac{x^{2} - 4 \, x - 60}{x - 10}=\answer{0}\]
\end{problem}}%}

\latexProblemContent{
\ifVerboseLocation This is Derivative Compute Question 0003. \\ \fi
\begin{problem}

Determine if the limit approaches a finite number, $\pm\infty$, or does not exist. (If the limit does not exist, write DNE)

\input{Limit-Compute-0003.HELP.tex}

\[\lim_{x\to{-10}}\dfrac{x^{2} + 12 \, x + 20}{x + 2}=\answer{0}\]
\end{problem}}%}

\latexProblemContent{
\ifVerboseLocation This is Derivative Compute Question 0003. \\ \fi
\begin{problem}

Determine if the limit approaches a finite number, $\pm\infty$, or does not exist. (If the limit does not exist, write DNE)

\input{Limit-Compute-0003.HELP.tex}

\[\lim_{x\to{10}}\dfrac{3 \, x^{2} - 24 \, x - 60}{x + 2}=\answer{0}\]
\end{problem}}%}

\latexProblemContent{
\ifVerboseLocation This is Derivative Compute Question 0003. \\ \fi
\begin{problem}

Determine if the limit approaches a finite number, $\pm\infty$, or does not exist. (If the limit does not exist, write DNE)

\input{Limit-Compute-0003.HELP.tex}

\[\lim_{x\to{-10}}\dfrac{x^{2} + 18 \, x + 80}{x + 8}=\answer{0}\]
\end{problem}}%}

\latexProblemContent{
\ifVerboseLocation This is Derivative Compute Question 0003. \\ \fi
\begin{problem}

Determine if the limit approaches a finite number, $\pm\infty$, or does not exist. (If the limit does not exist, write DNE)

\input{Limit-Compute-0003.HELP.tex}

\[\lim_{x\to{-6}}\dfrac{2 \, x^{2} + 28 \, x + 96}{x + 8}=\answer{0}\]
\end{problem}}%}

\latexProblemContent{
\ifVerboseLocation This is Derivative Compute Question 0003. \\ \fi
\begin{problem}

Determine if the limit approaches a finite number, $\pm\infty$, or does not exist. (If the limit does not exist, write DNE)

\input{Limit-Compute-0003.HELP.tex}

\[\lim_{x\to{5}}\dfrac{x^{2} - 25}{x + 5}=\answer{0}\]
\end{problem}}%}

\latexProblemContent{
\ifVerboseLocation This is Derivative Compute Question 0003. \\ \fi
\begin{problem}

Determine if the limit approaches a finite number, $\pm\infty$, or does not exist. (If the limit does not exist, write DNE)

\input{Limit-Compute-0003.HELP.tex}

\[\lim_{x\to{-8}}\dfrac{2 \, x^{2} + 22 \, x + 48}{x + 3}=\answer{0}\]
\end{problem}}%}

\latexProblemContent{
\ifVerboseLocation This is Derivative Compute Question 0003. \\ \fi
\begin{problem}

Determine if the limit approaches a finite number, $\pm\infty$, or does not exist. (If the limit does not exist, write DNE)

\input{Limit-Compute-0003.HELP.tex}

\[\lim_{x\to{8}}\dfrac{x^{2} - 13 \, x + 40}{x - 5}=\answer{0}\]
\end{problem}}%}

\latexProblemContent{
\ifVerboseLocation This is Derivative Compute Question 0003. \\ \fi
\begin{problem}

Determine if the limit approaches a finite number, $\pm\infty$, or does not exist. (If the limit does not exist, write DNE)

\input{Limit-Compute-0003.HELP.tex}

\[\lim_{x\to{-4}}\dfrac{3 \, x^{2} - 48}{x - 4}=\answer{0}\]
\end{problem}}%}

\latexProblemContent{
\ifVerboseLocation This is Derivative Compute Question 0003. \\ \fi
\begin{problem}

Determine if the limit approaches a finite number, $\pm\infty$, or does not exist. (If the limit does not exist, write DNE)

\input{Limit-Compute-0003.HELP.tex}

\[\lim_{x\to{9}}\dfrac{2 \, x^{2} - 32 \, x + 126}{x - 7}=\answer{0}\]
\end{problem}}%}

\latexProblemContent{
\ifVerboseLocation This is Derivative Compute Question 0003. \\ \fi
\begin{problem}

Determine if the limit approaches a finite number, $\pm\infty$, or does not exist. (If the limit does not exist, write DNE)

\input{Limit-Compute-0003.HELP.tex}

\[\lim_{x\to{-7}}\dfrac{x^{2} + 12 \, x + 35}{x + 5}=\answer{0}\]
\end{problem}}%}

\latexProblemContent{
\ifVerboseLocation This is Derivative Compute Question 0003. \\ \fi
\begin{problem}

Determine if the limit approaches a finite number, $\pm\infty$, or does not exist. (If the limit does not exist, write DNE)

\input{Limit-Compute-0003.HELP.tex}

\[\lim_{x\to{6}}\dfrac{3 \, x^{2} + 6 \, x - 144}{x + 8}=\answer{0}\]
\end{problem}}%}

\latexProblemContent{
\ifVerboseLocation This is Derivative Compute Question 0003. \\ \fi
\begin{problem}

Determine if the limit approaches a finite number, $\pm\infty$, or does not exist. (If the limit does not exist, write DNE)

\input{Limit-Compute-0003.HELP.tex}

\[\lim_{x\to{-4}}\dfrac{2 \, x^{2} + 28 \, x + 80}{x + 10}=\answer{0}\]
\end{problem}}%}

\latexProblemContent{
\ifVerboseLocation This is Derivative Compute Question 0003. \\ \fi
\begin{problem}

Determine if the limit approaches a finite number, $\pm\infty$, or does not exist. (If the limit does not exist, write DNE)

\input{Limit-Compute-0003.HELP.tex}

\[\lim_{x\to{6}}\dfrac{x^{2} + x - 42}{x + 7}=\answer{0}\]
\end{problem}}%}

\latexProblemContent{
\ifVerboseLocation This is Derivative Compute Question 0003. \\ \fi
\begin{problem}

Determine if the limit approaches a finite number, $\pm\infty$, or does not exist. (If the limit does not exist, write DNE)

\input{Limit-Compute-0003.HELP.tex}

\[\lim_{x\to{2}}\dfrac{2 \, x^{2} - 24 \, x + 40}{x - 10}=\answer{0}\]
\end{problem}}%}

\latexProblemContent{
\ifVerboseLocation This is Derivative Compute Question 0003. \\ \fi
\begin{problem}

Determine if the limit approaches a finite number, $\pm\infty$, or does not exist. (If the limit does not exist, write DNE)

\input{Limit-Compute-0003.HELP.tex}

\[\lim_{x\to{-6}}\dfrac{3 \, x^{2} + 42 \, x + 144}{x + 8}=\answer{0}\]
\end{problem}}%}

\latexProblemContent{
\ifVerboseLocation This is Derivative Compute Question 0003. \\ \fi
\begin{problem}

Determine if the limit approaches a finite number, $\pm\infty$, or does not exist. (If the limit does not exist, write DNE)

\input{Limit-Compute-0003.HELP.tex}

\[\lim_{x\to{7}}\dfrac{2 \, x^{2} - 16 \, x + 14}{x - 1}=\answer{0}\]
\end{problem}}%}

\latexProblemContent{
\ifVerboseLocation This is Derivative Compute Question 0003. \\ \fi
\begin{problem}

Determine if the limit approaches a finite number, $\pm\infty$, or does not exist. (If the limit does not exist, write DNE)

\input{Limit-Compute-0003.HELP.tex}

\[\lim_{x\to{6}}\dfrac{x^{2} + 2 \, x - 48}{x + 8}=\answer{0}\]
\end{problem}}%}

\latexProblemContent{
\ifVerboseLocation This is Derivative Compute Question 0003. \\ \fi
\begin{problem}

Determine if the limit approaches a finite number, $\pm\infty$, or does not exist. (If the limit does not exist, write DNE)

\input{Limit-Compute-0003.HELP.tex}

\[\lim_{x\to{-8}}\dfrac{2 \, x^{2} + 2 \, x - 112}{x - 7}=\answer{0}\]
\end{problem}}%}

\latexProblemContent{
\ifVerboseLocation This is Derivative Compute Question 0003. \\ \fi
\begin{problem}

Determine if the limit approaches a finite number, $\pm\infty$, or does not exist. (If the limit does not exist, write DNE)

\input{Limit-Compute-0003.HELP.tex}

\[\lim_{x\to{6}}\dfrac{2 \, x^{2} + 6 \, x - 108}{x + 9}=\answer{0}\]
\end{problem}}%}

\latexProblemContent{
\ifVerboseLocation This is Derivative Compute Question 0003. \\ \fi
\begin{problem}

Determine if the limit approaches a finite number, $\pm\infty$, or does not exist. (If the limit does not exist, write DNE)

\input{Limit-Compute-0003.HELP.tex}

\[\lim_{x\to{-6}}\dfrac{3 \, x^{2} - 108}{x - 6}=\answer{0}\]
\end{problem}}%}

\latexProblemContent{
\ifVerboseLocation This is Derivative Compute Question 0003. \\ \fi
\begin{problem}

Determine if the limit approaches a finite number, $\pm\infty$, or does not exist. (If the limit does not exist, write DNE)

\input{Limit-Compute-0003.HELP.tex}

\[\lim_{x\to{-4}}\dfrac{x^{2} + x - 12}{x - 3}=\answer{0}\]
\end{problem}}%}

\latexProblemContent{
\ifVerboseLocation This is Derivative Compute Question 0003. \\ \fi
\begin{problem}

Determine if the limit approaches a finite number, $\pm\infty$, or does not exist. (If the limit does not exist, write DNE)

\input{Limit-Compute-0003.HELP.tex}

\[\lim_{x\to{1}}\dfrac{3 \, x^{2} + 24 \, x - 27}{x + 9}=\answer{0}\]
\end{problem}}%}

\latexProblemContent{
\ifVerboseLocation This is Derivative Compute Question 0003. \\ \fi
\begin{problem}

Determine if the limit approaches a finite number, $\pm\infty$, or does not exist. (If the limit does not exist, write DNE)

\input{Limit-Compute-0003.HELP.tex}

\[\lim_{x\to{-5}}\dfrac{x^{2} - 2 \, x - 35}{x - 7}=\answer{0}\]
\end{problem}}%}

\latexProblemContent{
\ifVerboseLocation This is Derivative Compute Question 0003. \\ \fi
\begin{problem}

Determine if the limit approaches a finite number, $\pm\infty$, or does not exist. (If the limit does not exist, write DNE)

\input{Limit-Compute-0003.HELP.tex}

\[\lim_{x\to{-8}}\dfrac{x^{2} + 4 \, x - 32}{x - 4}=\answer{0}\]
\end{problem}}%}

\latexProblemContent{
\ifVerboseLocation This is Derivative Compute Question 0003. \\ \fi
\begin{problem}

Determine if the limit approaches a finite number, $\pm\infty$, or does not exist. (If the limit does not exist, write DNE)

\input{Limit-Compute-0003.HELP.tex}

\[\lim_{x\to{-9}}\dfrac{x^{2} + 3 \, x - 54}{x - 6}=\answer{0}\]
\end{problem}}%}

\latexProblemContent{
\ifVerboseLocation This is Derivative Compute Question 0003. \\ \fi
\begin{problem}

Determine if the limit approaches a finite number, $\pm\infty$, or does not exist. (If the limit does not exist, write DNE)

\input{Limit-Compute-0003.HELP.tex}

\[\lim_{x\to{1}}\dfrac{x^{2} + 9 \, x - 10}{x + 10}=\answer{0}\]
\end{problem}}%}

\latexProblemContent{
\ifVerboseLocation This is Derivative Compute Question 0003. \\ \fi
\begin{problem}

Determine if the limit approaches a finite number, $\pm\infty$, or does not exist. (If the limit does not exist, write DNE)

\input{Limit-Compute-0003.HELP.tex}

\[\lim_{x\to{9}}\dfrac{2 \, x^{2} - 4 \, x - 126}{x + 7}=\answer{0}\]
\end{problem}}%}

\latexProblemContent{
\ifVerboseLocation This is Derivative Compute Question 0003. \\ \fi
\begin{problem}

Determine if the limit approaches a finite number, $\pm\infty$, or does not exist. (If the limit does not exist, write DNE)

\input{Limit-Compute-0003.HELP.tex}

\[\lim_{x\to{4}}\dfrac{3 \, x^{2} - 18 \, x + 24}{x - 2}=\answer{0}\]
\end{problem}}%}

\latexProblemContent{
\ifVerboseLocation This is Derivative Compute Question 0003. \\ \fi
\begin{problem}

Determine if the limit approaches a finite number, $\pm\infty$, or does not exist. (If the limit does not exist, write DNE)

\input{Limit-Compute-0003.HELP.tex}

\[\lim_{x\to{-1}}\dfrac{2 \, x^{2} + 6 \, x + 4}{x + 2}=\answer{0}\]
\end{problem}}%}

\latexProblemContent{
\ifVerboseLocation This is Derivative Compute Question 0003. \\ \fi
\begin{problem}

Determine if the limit approaches a finite number, $\pm\infty$, or does not exist. (If the limit does not exist, write DNE)

\input{Limit-Compute-0003.HELP.tex}

\[\lim_{x\to{9}}\dfrac{3 \, x^{2} - 42 \, x + 135}{x - 5}=\answer{0}\]
\end{problem}}%}

\latexProblemContent{
\ifVerboseLocation This is Derivative Compute Question 0003. \\ \fi
\begin{problem}

Determine if the limit approaches a finite number, $\pm\infty$, or does not exist. (If the limit does not exist, write DNE)

\input{Limit-Compute-0003.HELP.tex}

\[\lim_{x\to{-5}}\dfrac{2 \, x^{2} + 30 \, x + 100}{x + 10}=\answer{0}\]
\end{problem}}%}

\latexProblemContent{
\ifVerboseLocation This is Derivative Compute Question 0003. \\ \fi
\begin{problem}

Determine if the limit approaches a finite number, $\pm\infty$, or does not exist. (If the limit does not exist, write DNE)

\input{Limit-Compute-0003.HELP.tex}

\[\lim_{x\to{5}}\dfrac{3 \, x^{2} + 15 \, x - 150}{x + 10}=\answer{0}\]
\end{problem}}%}

\latexProblemContent{
\ifVerboseLocation This is Derivative Compute Question 0003. \\ \fi
\begin{problem}

Determine if the limit approaches a finite number, $\pm\infty$, or does not exist. (If the limit does not exist, write DNE)

\input{Limit-Compute-0003.HELP.tex}

\[\lim_{x\to{-1}}\dfrac{2 \, x^{2} - 14 \, x - 16}{x - 8}=\answer{0}\]
\end{problem}}%}

\latexProblemContent{
\ifVerboseLocation This is Derivative Compute Question 0003. \\ \fi
\begin{problem}

Determine if the limit approaches a finite number, $\pm\infty$, or does not exist. (If the limit does not exist, write DNE)

\input{Limit-Compute-0003.HELP.tex}

\[\lim_{x\to{2}}\dfrac{x^{2} + 2 \, x - 8}{x + 4}=\answer{0}\]
\end{problem}}%}

\latexProblemContent{
\ifVerboseLocation This is Derivative Compute Question 0003. \\ \fi
\begin{problem}

Determine if the limit approaches a finite number, $\pm\infty$, or does not exist. (If the limit does not exist, write DNE)

\input{Limit-Compute-0003.HELP.tex}

\[\lim_{x\to{3}}\dfrac{x^{2} - 12 \, x + 27}{x - 9}=\answer{0}\]
\end{problem}}%}

\latexProblemContent{
\ifVerboseLocation This is Derivative Compute Question 0003. \\ \fi
\begin{problem}

Determine if the limit approaches a finite number, $\pm\infty$, or does not exist. (If the limit does not exist, write DNE)

\input{Limit-Compute-0003.HELP.tex}

\[\lim_{x\to{-3}}\dfrac{2 \, x^{2} - 8 \, x - 42}{x - 7}=\answer{0}\]
\end{problem}}%}

\latexProblemContent{
\ifVerboseLocation This is Derivative Compute Question 0003. \\ \fi
\begin{problem}

Determine if the limit approaches a finite number, $\pm\infty$, or does not exist. (If the limit does not exist, write DNE)

\input{Limit-Compute-0003.HELP.tex}

\[\lim_{x\to{1}}\dfrac{3 \, x^{2} + 3 \, x - 6}{x + 2}=\answer{0}\]
\end{problem}}%}

\latexProblemContent{
\ifVerboseLocation This is Derivative Compute Question 0003. \\ \fi
\begin{problem}

Determine if the limit approaches a finite number, $\pm\infty$, or does not exist. (If the limit does not exist, write DNE)

\input{Limit-Compute-0003.HELP.tex}

\[\lim_{x\to{1}}\dfrac{3 \, x^{2} - 9 \, x + 6}{x - 2}=\answer{0}\]
\end{problem}}%}

\latexProblemContent{
\ifVerboseLocation This is Derivative Compute Question 0003. \\ \fi
\begin{problem}

Determine if the limit approaches a finite number, $\pm\infty$, or does not exist. (If the limit does not exist, write DNE)

\input{Limit-Compute-0003.HELP.tex}

\[\lim_{x\to{1}}\dfrac{2 \, x^{2} - 16 \, x + 14}{x - 7}=\answer{0}\]
\end{problem}}%}

\latexProblemContent{
\ifVerboseLocation This is Derivative Compute Question 0003. \\ \fi
\begin{problem}

Determine if the limit approaches a finite number, $\pm\infty$, or does not exist. (If the limit does not exist, write DNE)

\input{Limit-Compute-0003.HELP.tex}

\[\lim_{x\to{10}}\dfrac{3 \, x^{2} - 21 \, x - 90}{x + 3}=\answer{0}\]
\end{problem}}%}

\latexProblemContent{
\ifVerboseLocation This is Derivative Compute Question 0003. \\ \fi
\begin{problem}

Determine if the limit approaches a finite number, $\pm\infty$, or does not exist. (If the limit does not exist, write DNE)

\input{Limit-Compute-0003.HELP.tex}

\[\lim_{x\to{-1}}\dfrac{2 \, x^{2} - 6 \, x - 8}{x - 4}=\answer{0}\]
\end{problem}}%}

\latexProblemContent{
\ifVerboseLocation This is Derivative Compute Question 0003. \\ \fi
\begin{problem}

Determine if the limit approaches a finite number, $\pm\infty$, or does not exist. (If the limit does not exist, write DNE)

\input{Limit-Compute-0003.HELP.tex}

\[\lim_{x\to{-2}}\dfrac{3 \, x^{2} - 21 \, x - 54}{x - 9}=\answer{0}\]
\end{problem}}%}

\latexProblemContent{
\ifVerboseLocation This is Derivative Compute Question 0003. \\ \fi
\begin{problem}

Determine if the limit approaches a finite number, $\pm\infty$, or does not exist. (If the limit does not exist, write DNE)

\input{Limit-Compute-0003.HELP.tex}

\[\lim_{x\to{-9}}\dfrac{2 \, x^{2} + 30 \, x + 108}{x + 6}=\answer{0}\]
\end{problem}}%}

\latexProblemContent{
\ifVerboseLocation This is Derivative Compute Question 0003. \\ \fi
\begin{problem}

Determine if the limit approaches a finite number, $\pm\infty$, or does not exist. (If the limit does not exist, write DNE)

\input{Limit-Compute-0003.HELP.tex}

\[\lim_{x\to{-2}}\dfrac{3 \, x^{2} + 27 \, x + 42}{x + 7}=\answer{0}\]
\end{problem}}%}

\latexProblemContent{
\ifVerboseLocation This is Derivative Compute Question 0003. \\ \fi
\begin{problem}

Determine if the limit approaches a finite number, $\pm\infty$, or does not exist. (If the limit does not exist, write DNE)

\input{Limit-Compute-0003.HELP.tex}

\[\lim_{x\to{2}}\dfrac{x^{2} - 11 \, x + 18}{x - 9}=\answer{0}\]
\end{problem}}%}

\latexProblemContent{
\ifVerboseLocation This is Derivative Compute Question 0003. \\ \fi
\begin{problem}

Determine if the limit approaches a finite number, $\pm\infty$, or does not exist. (If the limit does not exist, write DNE)

\input{Limit-Compute-0003.HELP.tex}

\[\lim_{x\to{-5}}\dfrac{3 \, x^{2} + 42 \, x + 135}{x + 9}=\answer{0}\]
\end{problem}}%}

\latexProblemContent{
\ifVerboseLocation This is Derivative Compute Question 0003. \\ \fi
\begin{problem}

Determine if the limit approaches a finite number, $\pm\infty$, or does not exist. (If the limit does not exist, write DNE)

\input{Limit-Compute-0003.HELP.tex}

\[\lim_{x\to{-9}}\dfrac{x^{2} + 7 \, x - 18}{x - 2}=\answer{0}\]
\end{problem}}%}

\latexProblemContent{
\ifVerboseLocation This is Derivative Compute Question 0003. \\ \fi
\begin{problem}

Determine if the limit approaches a finite number, $\pm\infty$, or does not exist. (If the limit does not exist, write DNE)

\input{Limit-Compute-0003.HELP.tex}

\[\lim_{x\to{2}}\dfrac{3 \, x^{2} - 21 \, x + 30}{x - 5}=\answer{0}\]
\end{problem}}%}

\latexProblemContent{
\ifVerboseLocation This is Derivative Compute Question 0003. \\ \fi
\begin{problem}

Determine if the limit approaches a finite number, $\pm\infty$, or does not exist. (If the limit does not exist, write DNE)

\input{Limit-Compute-0003.HELP.tex}

\[\lim_{x\to{8}}\dfrac{2 \, x^{2} - 6 \, x - 80}{x + 5}=\answer{0}\]
\end{problem}}%}

\latexProblemContent{
\ifVerboseLocation This is Derivative Compute Question 0003. \\ \fi
\begin{problem}

Determine if the limit approaches a finite number, $\pm\infty$, or does not exist. (If the limit does not exist, write DNE)

\input{Limit-Compute-0003.HELP.tex}

\[\lim_{x\to{-7}}\dfrac{x^{2} - 3 \, x - 70}{x - 10}=\answer{0}\]
\end{problem}}%}

\latexProblemContent{
\ifVerboseLocation This is Derivative Compute Question 0003. \\ \fi
\begin{problem}

Determine if the limit approaches a finite number, $\pm\infty$, or does not exist. (If the limit does not exist, write DNE)

\input{Limit-Compute-0003.HELP.tex}

\[\lim_{x\to{4}}\dfrac{2 \, x^{2} + 12 \, x - 80}{x + 10}=\answer{0}\]
\end{problem}}%}

\latexProblemContent{
\ifVerboseLocation This is Derivative Compute Question 0003. \\ \fi
\begin{problem}

Determine if the limit approaches a finite number, $\pm\infty$, or does not exist. (If the limit does not exist, write DNE)

\input{Limit-Compute-0003.HELP.tex}

\[\lim_{x\to{3}}\dfrac{3 \, x^{2} - 36 \, x + 81}{x - 9}=\answer{0}\]
\end{problem}}%}

\latexProblemContent{
\ifVerboseLocation This is Derivative Compute Question 0003. \\ \fi
\begin{problem}

Determine if the limit approaches a finite number, $\pm\infty$, or does not exist. (If the limit does not exist, write DNE)

\input{Limit-Compute-0003.HELP.tex}

\[\lim_{x\to{7}}\dfrac{3 \, x^{2} - 3 \, x - 126}{x + 6}=\answer{0}\]
\end{problem}}%}

\latexProblemContent{
\ifVerboseLocation This is Derivative Compute Question 0003. \\ \fi
\begin{problem}

Determine if the limit approaches a finite number, $\pm\infty$, or does not exist. (If the limit does not exist, write DNE)

\input{Limit-Compute-0003.HELP.tex}

\[\lim_{x\to{7}}\dfrac{x^{2} - 49}{x + 7}=\answer{0}\]
\end{problem}}%}

\latexProblemContent{
\ifVerboseLocation This is Derivative Compute Question 0003. \\ \fi
\begin{problem}

Determine if the limit approaches a finite number, $\pm\infty$, or does not exist. (If the limit does not exist, write DNE)

\input{Limit-Compute-0003.HELP.tex}

\[\lim_{x\to{-7}}\dfrac{x^{2} + 10 \, x + 21}{x + 3}=\answer{0}\]
\end{problem}}%}

\latexProblemContent{
\ifVerboseLocation This is Derivative Compute Question 0003. \\ \fi
\begin{problem}

Determine if the limit approaches a finite number, $\pm\infty$, or does not exist. (If the limit does not exist, write DNE)

\input{Limit-Compute-0003.HELP.tex}

\[\lim_{x\to{-9}}\dfrac{3 \, x^{2} + 9 \, x - 162}{x - 6}=\answer{0}\]
\end{problem}}%}

\latexProblemContent{
\ifVerboseLocation This is Derivative Compute Question 0003. \\ \fi
\begin{problem}

Determine if the limit approaches a finite number, $\pm\infty$, or does not exist. (If the limit does not exist, write DNE)

\input{Limit-Compute-0003.HELP.tex}

\[\lim_{x\to{-10}}\dfrac{2 \, x^{2} + 28 \, x + 80}{x + 4}=\answer{0}\]
\end{problem}}%}

\latexProblemContent{
\ifVerboseLocation This is Derivative Compute Question 0003. \\ \fi
\begin{problem}

Determine if the limit approaches a finite number, $\pm\infty$, or does not exist. (If the limit does not exist, write DNE)

\input{Limit-Compute-0003.HELP.tex}

\[\lim_{x\to{4}}\dfrac{2 \, x^{2} - 6 \, x - 8}{x + 1}=\answer{0}\]
\end{problem}}%}

\latexProblemContent{
\ifVerboseLocation This is Derivative Compute Question 0003. \\ \fi
\begin{problem}

Determine if the limit approaches a finite number, $\pm\infty$, or does not exist. (If the limit does not exist, write DNE)

\input{Limit-Compute-0003.HELP.tex}

\[\lim_{x\to{8}}\dfrac{2 \, x^{2} - 128}{x + 8}=\answer{0}\]
\end{problem}}%}

\latexProblemContent{
\ifVerboseLocation This is Derivative Compute Question 0003. \\ \fi
\begin{problem}

Determine if the limit approaches a finite number, $\pm\infty$, or does not exist. (If the limit does not exist, write DNE)

\input{Limit-Compute-0003.HELP.tex}

\[\lim_{x\to{-5}}\dfrac{3 \, x^{2} + 36 \, x + 105}{x + 7}=\answer{0}\]
\end{problem}}%}

\latexProblemContent{
\ifVerboseLocation This is Derivative Compute Question 0003. \\ \fi
\begin{problem}

Determine if the limit approaches a finite number, $\pm\infty$, or does not exist. (If the limit does not exist, write DNE)

\input{Limit-Compute-0003.HELP.tex}

\[\lim_{x\to{8}}\dfrac{x^{2} - 9 \, x + 8}{x - 1}=\answer{0}\]
\end{problem}}%}

\latexProblemContent{
\ifVerboseLocation This is Derivative Compute Question 0003. \\ \fi
\begin{problem}

Determine if the limit approaches a finite number, $\pm\infty$, or does not exist. (If the limit does not exist, write DNE)

\input{Limit-Compute-0003.HELP.tex}

\[\lim_{x\to{-1}}\dfrac{3 \, x^{2} + 21 \, x + 18}{x + 6}=\answer{0}\]
\end{problem}}%}

\latexProblemContent{
\ifVerboseLocation This is Derivative Compute Question 0003. \\ \fi
\begin{problem}

Determine if the limit approaches a finite number, $\pm\infty$, or does not exist. (If the limit does not exist, write DNE)

\input{Limit-Compute-0003.HELP.tex}

\[\lim_{x\to{1}}\dfrac{x^{2} + 4 \, x - 5}{x + 5}=\answer{0}\]
\end{problem}}%}

\latexProblemContent{
\ifVerboseLocation This is Derivative Compute Question 0003. \\ \fi
\begin{problem}

Determine if the limit approaches a finite number, $\pm\infty$, or does not exist. (If the limit does not exist, write DNE)

\input{Limit-Compute-0003.HELP.tex}

\[\lim_{x\to{-9}}\dfrac{x^{2} + 13 \, x + 36}{x + 4}=\answer{0}\]
\end{problem}}%}

\latexProblemContent{
\ifVerboseLocation This is Derivative Compute Question 0003. \\ \fi
\begin{problem}

Determine if the limit approaches a finite number, $\pm\infty$, or does not exist. (If the limit does not exist, write DNE)

\input{Limit-Compute-0003.HELP.tex}

\[\lim_{x\to{-1}}\dfrac{2 \, x^{2} - 10 \, x - 12}{x - 6}=\answer{0}\]
\end{problem}}%}

\latexProblemContent{
\ifVerboseLocation This is Derivative Compute Question 0003. \\ \fi
\begin{problem}

Determine if the limit approaches a finite number, $\pm\infty$, or does not exist. (If the limit does not exist, write DNE)

\input{Limit-Compute-0003.HELP.tex}

\[\lim_{x\to{1}}\dfrac{x^{2} - 5 \, x + 4}{x - 4}=\answer{0}\]
\end{problem}}%}

\latexProblemContent{
\ifVerboseLocation This is Derivative Compute Question 0003. \\ \fi
\begin{problem}

Determine if the limit approaches a finite number, $\pm\infty$, or does not exist. (If the limit does not exist, write DNE)

\input{Limit-Compute-0003.HELP.tex}

\[\lim_{x\to{-2}}\dfrac{3 \, x^{2} + 30 \, x + 48}{x + 8}=\answer{0}\]
\end{problem}}%}

\latexProblemContent{
\ifVerboseLocation This is Derivative Compute Question 0003. \\ \fi
\begin{problem}

Determine if the limit approaches a finite number, $\pm\infty$, or does not exist. (If the limit does not exist, write DNE)

\input{Limit-Compute-0003.HELP.tex}

\[\lim_{x\to{-3}}\dfrac{3 \, x^{2} - 6 \, x - 45}{x - 5}=\answer{0}\]
\end{problem}}%}

\latexProblemContent{
\ifVerboseLocation This is Derivative Compute Question 0003. \\ \fi
\begin{problem}

Determine if the limit approaches a finite number, $\pm\infty$, or does not exist. (If the limit does not exist, write DNE)

\input{Limit-Compute-0003.HELP.tex}

\[\lim_{x\to{-1}}\dfrac{2 \, x^{2} - 2}{x - 1}=\answer{0}\]
\end{problem}}%}

\latexProblemContent{
\ifVerboseLocation This is Derivative Compute Question 0003. \\ \fi
\begin{problem}

Determine if the limit approaches a finite number, $\pm\infty$, or does not exist. (If the limit does not exist, write DNE)

\input{Limit-Compute-0003.HELP.tex}

\[\lim_{x\to{-8}}\dfrac{x^{2} + 18 \, x + 80}{x + 10}=\answer{0}\]
\end{problem}}%}

\latexProblemContent{
\ifVerboseLocation This is Derivative Compute Question 0003. \\ \fi
\begin{problem}

Determine if the limit approaches a finite number, $\pm\infty$, or does not exist. (If the limit does not exist, write DNE)

\input{Limit-Compute-0003.HELP.tex}

\[\lim_{x\to{-6}}\dfrac{x^{2} + 4 \, x - 12}{x - 2}=\answer{0}\]
\end{problem}}%}

\latexProblemContent{
\ifVerboseLocation This is Derivative Compute Question 0003. \\ \fi
\begin{problem}

Determine if the limit approaches a finite number, $\pm\infty$, or does not exist. (If the limit does not exist, write DNE)

\input{Limit-Compute-0003.HELP.tex}

\[\lim_{x\to{9}}\dfrac{x^{2} - 16 \, x + 63}{x - 7}=\answer{0}\]
\end{problem}}%}

\latexProblemContent{
\ifVerboseLocation This is Derivative Compute Question 0003. \\ \fi
\begin{problem}

Determine if the limit approaches a finite number, $\pm\infty$, or does not exist. (If the limit does not exist, write DNE)

\input{Limit-Compute-0003.HELP.tex}

\[\lim_{x\to{-1}}\dfrac{x^{2} - 1}{x - 1}=\answer{0}\]
\end{problem}}%}

\latexProblemContent{
\ifVerboseLocation This is Derivative Compute Question 0003. \\ \fi
\begin{problem}

Determine if the limit approaches a finite number, $\pm\infty$, or does not exist. (If the limit does not exist, write DNE)

\input{Limit-Compute-0003.HELP.tex}

\[\lim_{x\to{10}}\dfrac{2 \, x^{2} - 28 \, x + 80}{x - 4}=\answer{0}\]
\end{problem}}%}

\latexProblemContent{
\ifVerboseLocation This is Derivative Compute Question 0003. \\ \fi
\begin{problem}

Determine if the limit approaches a finite number, $\pm\infty$, or does not exist. (If the limit does not exist, write DNE)

\input{Limit-Compute-0003.HELP.tex}

\[\lim_{x\to{5}}\dfrac{2 \, x^{2} - 26 \, x + 80}{x - 8}=\answer{0}\]
\end{problem}}%}

\latexProblemContent{
\ifVerboseLocation This is Derivative Compute Question 0003. \\ \fi
\begin{problem}

Determine if the limit approaches a finite number, $\pm\infty$, or does not exist. (If the limit does not exist, write DNE)

\input{Limit-Compute-0003.HELP.tex}

\[\lim_{x\to{-2}}\dfrac{2 \, x^{2} - 14 \, x - 36}{x - 9}=\answer{0}\]
\end{problem}}%}

\latexProblemContent{
\ifVerboseLocation This is Derivative Compute Question 0003. \\ \fi
\begin{problem}

Determine if the limit approaches a finite number, $\pm\infty$, or does not exist. (If the limit does not exist, write DNE)

\input{Limit-Compute-0003.HELP.tex}

\[\lim_{x\to{7}}\dfrac{3 \, x^{2} - 27 \, x + 42}{x - 2}=\answer{0}\]
\end{problem}}%}

\latexProblemContent{
\ifVerboseLocation This is Derivative Compute Question 0003. \\ \fi
\begin{problem}

Determine if the limit approaches a finite number, $\pm\infty$, or does not exist. (If the limit does not exist, write DNE)

\input{Limit-Compute-0003.HELP.tex}

\[\lim_{x\to{7}}\dfrac{x^{2} - 5 \, x - 14}{x + 2}=\answer{0}\]
\end{problem}}%}

\latexProblemContent{
\ifVerboseLocation This is Derivative Compute Question 0003. \\ \fi
\begin{problem}

Determine if the limit approaches a finite number, $\pm\infty$, or does not exist. (If the limit does not exist, write DNE)

\input{Limit-Compute-0003.HELP.tex}

\[\lim_{x\to{9}}\dfrac{3 \, x^{2} - 18 \, x - 81}{x + 3}=\answer{0}\]
\end{problem}}%}

\latexProblemContent{
\ifVerboseLocation This is Derivative Compute Question 0003. \\ \fi
\begin{problem}

Determine if the limit approaches a finite number, $\pm\infty$, or does not exist. (If the limit does not exist, write DNE)

\input{Limit-Compute-0003.HELP.tex}

\[\lim_{x\to{-7}}\dfrac{x^{2} - 2 \, x - 63}{x - 9}=\answer{0}\]
\end{problem}}%}

\latexProblemContent{
\ifVerboseLocation This is Derivative Compute Question 0003. \\ \fi
\begin{problem}

Determine if the limit approaches a finite number, $\pm\infty$, or does not exist. (If the limit does not exist, write DNE)

\input{Limit-Compute-0003.HELP.tex}

\[\lim_{x\to{-9}}\dfrac{3 \, x^{2} + 15 \, x - 108}{x - 4}=\answer{0}\]
\end{problem}}%}

\latexProblemContent{
\ifVerboseLocation This is Derivative Compute Question 0003. \\ \fi
\begin{problem}

Determine if the limit approaches a finite number, $\pm\infty$, or does not exist. (If the limit does not exist, write DNE)

\input{Limit-Compute-0003.HELP.tex}

\[\lim_{x\to{-5}}\dfrac{x^{2} + 15 \, x + 50}{x + 10}=\answer{0}\]
\end{problem}}%}

\latexProblemContent{
\ifVerboseLocation This is Derivative Compute Question 0003. \\ \fi
\begin{problem}

Determine if the limit approaches a finite number, $\pm\infty$, or does not exist. (If the limit does not exist, write DNE)

\input{Limit-Compute-0003.HELP.tex}

\[\lim_{x\to{2}}\dfrac{x^{2} - 3 \, x + 2}{x - 1}=\answer{0}\]
\end{problem}}%}

\latexProblemContent{
\ifVerboseLocation This is Derivative Compute Question 0003. \\ \fi
\begin{problem}

Determine if the limit approaches a finite number, $\pm\infty$, or does not exist. (If the limit does not exist, write DNE)

\input{Limit-Compute-0003.HELP.tex}

\[\lim_{x\to{-4}}\dfrac{x^{2} + 9 \, x + 20}{x + 5}=\answer{0}\]
\end{problem}}%}

\latexProblemContent{
\ifVerboseLocation This is Derivative Compute Question 0003. \\ \fi
\begin{problem}

Determine if the limit approaches a finite number, $\pm\infty$, or does not exist. (If the limit does not exist, write DNE)

\input{Limit-Compute-0003.HELP.tex}

\[\lim_{x\to{5}}\dfrac{x^{2} - 12 \, x + 35}{x - 7}=\answer{0}\]
\end{problem}}%}

\latexProblemContent{
\ifVerboseLocation This is Derivative Compute Question 0003. \\ \fi
\begin{problem}

Determine if the limit approaches a finite number, $\pm\infty$, or does not exist. (If the limit does not exist, write DNE)

\input{Limit-Compute-0003.HELP.tex}

\[\lim_{x\to{-2}}\dfrac{3 \, x^{2} - 24 \, x - 60}{x - 10}=\answer{0}\]
\end{problem}}%}

\latexProblemContent{
\ifVerboseLocation This is Derivative Compute Question 0003. \\ \fi
\begin{problem}

Determine if the limit approaches a finite number, $\pm\infty$, or does not exist. (If the limit does not exist, write DNE)

\input{Limit-Compute-0003.HELP.tex}

\[\lim_{x\to{3}}\dfrac{3 \, x^{2} + 6 \, x - 45}{x + 5}=\answer{0}\]
\end{problem}}%}

\latexProblemContent{
\ifVerboseLocation This is Derivative Compute Question 0003. \\ \fi
\begin{problem}

Determine if the limit approaches a finite number, $\pm\infty$, or does not exist. (If the limit does not exist, write DNE)

\input{Limit-Compute-0003.HELP.tex}

\[\lim_{x\to{-4}}\dfrac{3 \, x^{2} + 21 \, x + 36}{x + 3}=\answer{0}\]
\end{problem}}%}

\latexProblemContent{
\ifVerboseLocation This is Derivative Compute Question 0003. \\ \fi
\begin{problem}

Determine if the limit approaches a finite number, $\pm\infty$, or does not exist. (If the limit does not exist, write DNE)

\input{Limit-Compute-0003.HELP.tex}

\[\lim_{x\to{10}}\dfrac{3 \, x^{2} - 36 \, x + 60}{x - 2}=\answer{0}\]
\end{problem}}%}

\latexProblemContent{
\ifVerboseLocation This is Derivative Compute Question 0003. \\ \fi
\begin{problem}

Determine if the limit approaches a finite number, $\pm\infty$, or does not exist. (If the limit does not exist, write DNE)

\input{Limit-Compute-0003.HELP.tex}

\[\lim_{x\to{-3}}\dfrac{2 \, x^{2} - 14 \, x - 60}{x - 10}=\answer{0}\]
\end{problem}}%}

\latexProblemContent{
\ifVerboseLocation This is Derivative Compute Question 0003. \\ \fi
\begin{problem}

Determine if the limit approaches a finite number, $\pm\infty$, or does not exist. (If the limit does not exist, write DNE)

\input{Limit-Compute-0003.HELP.tex}

\[\lim_{x\to{10}}\dfrac{2 \, x^{2} - 200}{x + 10}=\answer{0}\]
\end{problem}}%}

\latexProblemContent{
\ifVerboseLocation This is Derivative Compute Question 0003. \\ \fi
\begin{problem}

Determine if the limit approaches a finite number, $\pm\infty$, or does not exist. (If the limit does not exist, write DNE)

\input{Limit-Compute-0003.HELP.tex}

\[\lim_{x\to{-3}}\dfrac{3 \, x^{2} - 15 \, x - 72}{x - 8}=\answer{0}\]
\end{problem}}%}

\latexProblemContent{
\ifVerboseLocation This is Derivative Compute Question 0003. \\ \fi
\begin{problem}

Determine if the limit approaches a finite number, $\pm\infty$, or does not exist. (If the limit does not exist, write DNE)

\input{Limit-Compute-0003.HELP.tex}

\[\lim_{x\to{-8}}\dfrac{2 \, x^{2} + 4 \, x - 96}{x - 6}=\answer{0}\]
\end{problem}}%}

\latexProblemContent{
\ifVerboseLocation This is Derivative Compute Question 0003. \\ \fi
\begin{problem}

Determine if the limit approaches a finite number, $\pm\infty$, or does not exist. (If the limit does not exist, write DNE)

\input{Limit-Compute-0003.HELP.tex}

\[\lim_{x\to{10}}\dfrac{2 \, x^{2} - 14 \, x - 60}{x + 3}=\answer{0}\]
\end{problem}}%}

\latexProblemContent{
\ifVerboseLocation This is Derivative Compute Question 0003. \\ \fi
\begin{problem}

Determine if the limit approaches a finite number, $\pm\infty$, or does not exist. (If the limit does not exist, write DNE)

\input{Limit-Compute-0003.HELP.tex}

\[\lim_{x\to{-8}}\dfrac{x^{2} + 3 \, x - 40}{x - 5}=\answer{0}\]
\end{problem}}%}

\latexProblemContent{
\ifVerboseLocation This is Derivative Compute Question 0003. \\ \fi
\begin{problem}

Determine if the limit approaches a finite number, $\pm\infty$, or does not exist. (If the limit does not exist, write DNE)

\input{Limit-Compute-0003.HELP.tex}

\[\lim_{x\to{6}}\dfrac{3 \, x^{2} + 12 \, x - 180}{x + 10}=\answer{0}\]
\end{problem}}%}

\latexProblemContent{
\ifVerboseLocation This is Derivative Compute Question 0003. \\ \fi
\begin{problem}

Determine if the limit approaches a finite number, $\pm\infty$, or does not exist. (If the limit does not exist, write DNE)

\input{Limit-Compute-0003.HELP.tex}

\[\lim_{x\to{10}}\dfrac{2 \, x^{2} - 18 \, x - 20}{x + 1}=\answer{0}\]
\end{problem}}%}

\latexProblemContent{
\ifVerboseLocation This is Derivative Compute Question 0003. \\ \fi
\begin{problem}

Determine if the limit approaches a finite number, $\pm\infty$, or does not exist. (If the limit does not exist, write DNE)

\input{Limit-Compute-0003.HELP.tex}

\[\lim_{x\to{-6}}\dfrac{x^{2} + x - 30}{x - 5}=\answer{0}\]
\end{problem}}%}

\latexProblemContent{
\ifVerboseLocation This is Derivative Compute Question 0003. \\ \fi
\begin{problem}

Determine if the limit approaches a finite number, $\pm\infty$, or does not exist. (If the limit does not exist, write DNE)

\input{Limit-Compute-0003.HELP.tex}

\[\lim_{x\to{-4}}\dfrac{2 \, x^{2} + 18 \, x + 40}{x + 5}=\answer{0}\]
\end{problem}}%}

\latexProblemContent{
\ifVerboseLocation This is Derivative Compute Question 0003. \\ \fi
\begin{problem}

Determine if the limit approaches a finite number, $\pm\infty$, or does not exist. (If the limit does not exist, write DNE)

\input{Limit-Compute-0003.HELP.tex}

\[\lim_{x\to{-9}}\dfrac{2 \, x^{2} - 2 \, x - 180}{x - 10}=\answer{0}\]
\end{problem}}%}

\latexProblemContent{
\ifVerboseLocation This is Derivative Compute Question 0003. \\ \fi
\begin{problem}

Determine if the limit approaches a finite number, $\pm\infty$, or does not exist. (If the limit does not exist, write DNE)

\input{Limit-Compute-0003.HELP.tex}

\[\lim_{x\to{-7}}\dfrac{3 \, x^{2} - 9 \, x - 210}{x - 10}=\answer{0}\]
\end{problem}}%}

\latexProblemContent{
\ifVerboseLocation This is Derivative Compute Question 0003. \\ \fi
\begin{problem}

Determine if the limit approaches a finite number, $\pm\infty$, or does not exist. (If the limit does not exist, write DNE)

\input{Limit-Compute-0003.HELP.tex}

\[\lim_{x\to{-4}}\dfrac{3 \, x^{2} + 18 \, x + 24}{x + 2}=\answer{0}\]
\end{problem}}%}

\latexProblemContent{
\ifVerboseLocation This is Derivative Compute Question 0003. \\ \fi
\begin{problem}

Determine if the limit approaches a finite number, $\pm\infty$, or does not exist. (If the limit does not exist, write DNE)

\input{Limit-Compute-0003.HELP.tex}

\[\lim_{x\to{-9}}\dfrac{2 \, x^{2} + 38 \, x + 180}{x + 10}=\answer{0}\]
\end{problem}}%}

\latexProblemContent{
\ifVerboseLocation This is Derivative Compute Question 0003. \\ \fi
\begin{problem}

Determine if the limit approaches a finite number, $\pm\infty$, or does not exist. (If the limit does not exist, write DNE)

\input{Limit-Compute-0003.HELP.tex}

\[\lim_{x\to{6}}\dfrac{3 \, x^{2} - 48 \, x + 180}{x - 10}=\answer{0}\]
\end{problem}}%}

\latexProblemContent{
\ifVerboseLocation This is Derivative Compute Question 0003. \\ \fi
\begin{problem}

Determine if the limit approaches a finite number, $\pm\infty$, or does not exist. (If the limit does not exist, write DNE)

\input{Limit-Compute-0003.HELP.tex}

\[\lim_{x\to{-8}}\dfrac{x^{2} + x - 56}{x - 7}=\answer{0}\]
\end{problem}}%}

\latexProblemContent{
\ifVerboseLocation This is Derivative Compute Question 0003. \\ \fi
\begin{problem}

Determine if the limit approaches a finite number, $\pm\infty$, or does not exist. (If the limit does not exist, write DNE)

\input{Limit-Compute-0003.HELP.tex}

\[\lim_{x\to{6}}\dfrac{x^{2} - 7 \, x + 6}{x - 1}=\answer{0}\]
\end{problem}}%}

\latexProblemContent{
\ifVerboseLocation This is Derivative Compute Question 0003. \\ \fi
\begin{problem}

Determine if the limit approaches a finite number, $\pm\infty$, or does not exist. (If the limit does not exist, write DNE)

\input{Limit-Compute-0003.HELP.tex}

\[\lim_{x\to{5}}\dfrac{x^{2} - 3 \, x - 10}{x + 2}=\answer{0}\]
\end{problem}}%}

