\ProblemFileHeader{XTL_SV_QUESTIONCOUNT}% Process how many problems are in this file and how to detect if it has a desirable problem
\ifproblemToFind% If it has a desirable problem search the file.
%%\tagged{Ans@ShortAns, Type@Compute, Topic@Integral, Sub@TrigSub, Func@Trig, File@0037}{
\latexProblemContent{
\ifVerboseLocation This is Integration Compute Question 0037. \\ \fi
\begin{problem}

Compute the following integral:

\input{Integral-Compute-0037.HELP.tex}

\[
\int{{-\frac{4}{\sqrt{x^{2} + 9}}}\;dx} = \answer{{-4 \, {\rm arcsinh}\left(\frac{1}{3} \, x\right)}+C}
\]
\end{problem}}%}

\latexProblemContent{
\ifVerboseLocation This is Integration Compute Question 0037. \\ \fi
\begin{problem}

Compute the following integral:

\input{Integral-Compute-0037.HELP.tex}

\[
\int{{-\frac{4}{\sqrt{x^{2} - 64}}}\;dx} = \answer{{-4 \, \log\left(2 \, x + 2 \, \sqrt{x^{2} - 64}\right)}+C}
\]
\end{problem}}%}

\latexProblemContent{
\ifVerboseLocation This is Integration Compute Question 0037. \\ \fi
\begin{problem}

Compute the following integral:

\input{Integral-Compute-0037.HELP.tex}

\[
\int{{-\frac{9}{\sqrt{-x^{2} + 81}}}\;dx} = \answer{{-9 \, \arcsin\left(\frac{1}{9} \, x\right)}+C}
\]
\end{problem}}%}

\latexProblemContent{
\ifVerboseLocation This is Integration Compute Question 0037. \\ \fi
\begin{problem}

Compute the following integral:

\input{Integral-Compute-0037.HELP.tex}

\[
\int{{-\frac{1}{\sqrt{-x^{2} + 25}}}\;dx} = \answer{{-\arcsin\left(\frac{1}{5} \, x\right)}+C}
\]
\end{problem}}%}

\latexProblemContent{
\ifVerboseLocation This is Integration Compute Question 0037. \\ \fi
\begin{problem}

Compute the following integral:

\input{Integral-Compute-0037.HELP.tex}

\[
\int{{\frac{6}{\sqrt{-x^{2} + 9}}}\;dx} = \answer{{6 \, \arcsin\left(\frac{1}{3} \, x\right)}+C}
\]
\end{problem}}%}

\latexProblemContent{
\ifVerboseLocation This is Integration Compute Question 0037. \\ \fi
\begin{problem}

Compute the following integral:

\input{Integral-Compute-0037.HELP.tex}

\[
\int{{\frac{2}{\sqrt{-x^{2} + 64}}}\;dx} = \answer{{2 \, \arcsin\left(\frac{1}{8} \, x\right)}+C}
\]
\end{problem}}%}

\latexProblemContent{
\ifVerboseLocation This is Integration Compute Question 0037. \\ \fi
\begin{problem}

Compute the following integral:

\input{Integral-Compute-0037.HELP.tex}

\[
\int{{-\frac{6}{\sqrt{x^{2} + 16}}}\;dx} = \answer{{-6 \, {\rm arcsinh}\left(\frac{1}{4} \, x\right)}+C}
\]
\end{problem}}%}

\latexProblemContent{
\ifVerboseLocation This is Integration Compute Question 0037. \\ \fi
\begin{problem}

Compute the following integral:

\input{Integral-Compute-0037.HELP.tex}

\[
\int{{-\frac{1}{\sqrt{-x^{2} + 36}}}\;dx} = \answer{{-\arcsin\left(\frac{1}{6} \, x\right)}+C}
\]
\end{problem}}%}

\latexProblemContent{
\ifVerboseLocation This is Integration Compute Question 0037. \\ \fi
\begin{problem}

Compute the following integral:

\input{Integral-Compute-0037.HELP.tex}

\[
\int{{\frac{3}{\sqrt{x^{2} - 25}}}\;dx} = \answer{{3 \, \log\left(2 \, x + 2 \, \sqrt{x^{2} - 25}\right)}+C}
\]
\end{problem}}%}

\latexProblemContent{
\ifVerboseLocation This is Integration Compute Question 0037. \\ \fi
\begin{problem}

Compute the following integral:

\input{Integral-Compute-0037.HELP.tex}

\[
\int{{\frac{1}{\sqrt{-x^{2} + 36}}}\;dx} = \answer{{\arcsin\left(\frac{1}{6} \, x\right)}+C}
\]
\end{problem}}%}

\latexProblemContent{
\ifVerboseLocation This is Integration Compute Question 0037. \\ \fi
\begin{problem}

Compute the following integral:

\input{Integral-Compute-0037.HELP.tex}

\[
\int{{-\frac{2}{\sqrt{-x^{2} + 49}}}\;dx} = \answer{{-2 \, \arcsin\left(\frac{1}{7} \, x\right)}+C}
\]
\end{problem}}%}

\latexProblemContent{
\ifVerboseLocation This is Integration Compute Question 0037. \\ \fi
\begin{problem}

Compute the following integral:

\input{Integral-Compute-0037.HELP.tex}

\[
\int{{\frac{9}{\sqrt{x^{2} - 81}}}\;dx} = \answer{{9 \, \log\left(2 \, x + 2 \, \sqrt{x^{2} - 81}\right)}+C}
\]
\end{problem}}%}

\latexProblemContent{
\ifVerboseLocation This is Integration Compute Question 0037. \\ \fi
\begin{problem}

Compute the following integral:

\input{Integral-Compute-0037.HELP.tex}

\[
\int{{\frac{4}{\sqrt{x^{2} + 4}}}\;dx} = \answer{{4 \, {\rm arcsinh}\left(\frac{1}{2} \, x\right)}+C}
\]
\end{problem}}%}

\latexProblemContent{
\ifVerboseLocation This is Integration Compute Question 0037. \\ \fi
\begin{problem}

Compute the following integral:

\input{Integral-Compute-0037.HELP.tex}

\[
\int{{\frac{7}{\sqrt{-x^{2} + 1}}}\;dx} = \answer{{7 \, \arcsin\left(x\right)}+C}
\]
\end{problem}}%}

\latexProblemContent{
\ifVerboseLocation This is Integration Compute Question 0037. \\ \fi
\begin{problem}

Compute the following integral:

\input{Integral-Compute-0037.HELP.tex}

\[
\int{{-\frac{1}{\sqrt{x^{2} - 1}}}\;dx} = \answer{{-\log\left(2 \, x + 2 \, \sqrt{x^{2} - 1}\right)}+C}
\]
\end{problem}}%}

\latexProblemContent{
\ifVerboseLocation This is Integration Compute Question 0037. \\ \fi
\begin{problem}

Compute the following integral:

\input{Integral-Compute-0037.HELP.tex}

\[
\int{{\frac{6}{\sqrt{-x^{2} + 1}}}\;dx} = \answer{{6 \, \arcsin\left(x\right)}+C}
\]
\end{problem}}%}

\latexProblemContent{
\ifVerboseLocation This is Integration Compute Question 0037. \\ \fi
\begin{problem}

Compute the following integral:

\input{Integral-Compute-0037.HELP.tex}

\[
\int{{\frac{6}{\sqrt{x^{2} + 25}}}\;dx} = \answer{{6 \, {\rm arcsinh}\left(\frac{1}{5} \, x\right)}+C}
\]
\end{problem}}%}

\latexProblemContent{
\ifVerboseLocation This is Integration Compute Question 0037. \\ \fi
\begin{problem}

Compute the following integral:

\input{Integral-Compute-0037.HELP.tex}

\[
\int{{-\frac{8}{\sqrt{-x^{2} + 49}}}\;dx} = \answer{{-8 \, \arcsin\left(\frac{1}{7} \, x\right)}+C}
\]
\end{problem}}%}

\latexProblemContent{
\ifVerboseLocation This is Integration Compute Question 0037. \\ \fi
\begin{problem}

Compute the following integral:

\input{Integral-Compute-0037.HELP.tex}

\[
\int{{-\frac{7}{\sqrt{-x^{2} + 16}}}\;dx} = \answer{{-7 \, \arcsin\left(\frac{1}{4} \, x\right)}+C}
\]
\end{problem}}%}

\latexProblemContent{
\ifVerboseLocation This is Integration Compute Question 0037. \\ \fi
\begin{problem}

Compute the following integral:

\input{Integral-Compute-0037.HELP.tex}

\[
\int{{\frac{9}{\sqrt{x^{2} - 9}}}\;dx} = \answer{{9 \, \log\left(2 \, x + 2 \, \sqrt{x^{2} - 9}\right)}+C}
\]
\end{problem}}%}

\latexProblemContent{
\ifVerboseLocation This is Integration Compute Question 0037. \\ \fi
\begin{problem}

Compute the following integral:

\input{Integral-Compute-0037.HELP.tex}

\[
\int{{\frac{8}{\sqrt{-x^{2} + 16}}}\;dx} = \answer{{8 \, \arcsin\left(\frac{1}{4} \, x\right)}+C}
\]
\end{problem}}%}

\latexProblemContent{
\ifVerboseLocation This is Integration Compute Question 0037. \\ \fi
\begin{problem}

Compute the following integral:

\input{Integral-Compute-0037.HELP.tex}

\[
\int{{-\frac{3}{\sqrt{x^{2} + 9}}}\;dx} = \answer{{-3 \, {\rm arcsinh}\left(\frac{1}{3} \, x\right)}+C}
\]
\end{problem}}%}

\latexProblemContent{
\ifVerboseLocation This is Integration Compute Question 0037. \\ \fi
\begin{problem}

Compute the following integral:

\input{Integral-Compute-0037.HELP.tex}

\[
\int{{-\frac{7}{\sqrt{-x^{2} + 49}}}\;dx} = \answer{{-7 \, \arcsin\left(\frac{1}{7} \, x\right)}+C}
\]
\end{problem}}%}

\latexProblemContent{
\ifVerboseLocation This is Integration Compute Question 0037. \\ \fi
\begin{problem}

Compute the following integral:

\input{Integral-Compute-0037.HELP.tex}

\[
\int{{\frac{5}{\sqrt{-x^{2} + 1}}}\;dx} = \answer{{5 \, \arcsin\left(x\right)}+C}
\]
\end{problem}}%}

\latexProblemContent{
\ifVerboseLocation This is Integration Compute Question 0037. \\ \fi
\begin{problem}

Compute the following integral:

\input{Integral-Compute-0037.HELP.tex}

\[
\int{{-\frac{8}{\sqrt{-x^{2} + 1}}}\;dx} = \answer{{-8 \, \arcsin\left(x\right)}+C}
\]
\end{problem}}%}

\latexProblemContent{
\ifVerboseLocation This is Integration Compute Question 0037. \\ \fi
\begin{problem}

Compute the following integral:

\input{Integral-Compute-0037.HELP.tex}

\[
\int{{-\frac{4}{\sqrt{x^{2} - 36}}}\;dx} = \answer{{-4 \, \log\left(2 \, x + 2 \, \sqrt{x^{2} - 36}\right)}+C}
\]
\end{problem}}%}

\latexProblemContent{
\ifVerboseLocation This is Integration Compute Question 0037. \\ \fi
\begin{problem}

Compute the following integral:

\input{Integral-Compute-0037.HELP.tex}

\[
\int{{\frac{1}{\sqrt{-x^{2} + 1}}}\;dx} = \answer{{\arcsin\left(x\right)}+C}
\]
\end{problem}}%}

\latexProblemContent{
\ifVerboseLocation This is Integration Compute Question 0037. \\ \fi
\begin{problem}

Compute the following integral:

\input{Integral-Compute-0037.HELP.tex}

\[
\int{{\frac{7}{\sqrt{x^{2} - 9}}}\;dx} = \answer{{7 \, \log\left(2 \, x + 2 \, \sqrt{x^{2} - 9}\right)}+C}
\]
\end{problem}}%}

\latexProblemContent{
\ifVerboseLocation This is Integration Compute Question 0037. \\ \fi
\begin{problem}

Compute the following integral:

\input{Integral-Compute-0037.HELP.tex}

\[
\int{{-\frac{6}{\sqrt{x^{2} - 81}}}\;dx} = \answer{{-6 \, \log\left(2 \, x + 2 \, \sqrt{x^{2} - 81}\right)}+C}
\]
\end{problem}}%}

\latexProblemContent{
\ifVerboseLocation This is Integration Compute Question 0037. \\ \fi
\begin{problem}

Compute the following integral:

\input{Integral-Compute-0037.HELP.tex}

\[
\int{{-\frac{1}{\sqrt{x^{2} - 9}}}\;dx} = \answer{{-\log\left(2 \, x + 2 \, \sqrt{x^{2} - 9}\right)}+C}
\]
\end{problem}}%}

\latexProblemContent{
\ifVerboseLocation This is Integration Compute Question 0037. \\ \fi
\begin{problem}

Compute the following integral:

\input{Integral-Compute-0037.HELP.tex}

\[
\int{{\frac{7}{\sqrt{-x^{2} + 16}}}\;dx} = \answer{{7 \, \arcsin\left(\frac{1}{4} \, x\right)}+C}
\]
\end{problem}}%}

\latexProblemContent{
\ifVerboseLocation This is Integration Compute Question 0037. \\ \fi
\begin{problem}

Compute the following integral:

\input{Integral-Compute-0037.HELP.tex}

\[
\int{{\frac{3}{\sqrt{-x^{2} + 1}}}\;dx} = \answer{{3 \, \arcsin\left(x\right)}+C}
\]
\end{problem}}%}

\latexProblemContent{
\ifVerboseLocation This is Integration Compute Question 0037. \\ \fi
\begin{problem}

Compute the following integral:

\input{Integral-Compute-0037.HELP.tex}

\[
\int{{\frac{4}{\sqrt{-x^{2} + 4}}}\;dx} = \answer{{4 \, \arcsin\left(\frac{1}{2} \, x\right)}+C}
\]
\end{problem}}%}

\latexProblemContent{
\ifVerboseLocation This is Integration Compute Question 0037. \\ \fi
\begin{problem}

Compute the following integral:

\input{Integral-Compute-0037.HELP.tex}

\[
\int{{\frac{5}{\sqrt{-x^{2} + 81}}}\;dx} = \answer{{5 \, \arcsin\left(\frac{1}{9} \, x\right)}+C}
\]
\end{problem}}%}

\latexProblemContent{
\ifVerboseLocation This is Integration Compute Question 0037. \\ \fi
\begin{problem}

Compute the following integral:

\input{Integral-Compute-0037.HELP.tex}

\[
\int{{-\frac{3}{\sqrt{x^{2} - 64}}}\;dx} = \answer{{-3 \, \log\left(2 \, x + 2 \, \sqrt{x^{2} - 64}\right)}+C}
\]
\end{problem}}%}

\latexProblemContent{
\ifVerboseLocation This is Integration Compute Question 0037. \\ \fi
\begin{problem}

Compute the following integral:

\input{Integral-Compute-0037.HELP.tex}

\[
\int{{\frac{4}{\sqrt{x^{2} - 1}}}\;dx} = \answer{{4 \, \log\left(2 \, x + 2 \, \sqrt{x^{2} - 1}\right)}+C}
\]
\end{problem}}%}

\latexProblemContent{
\ifVerboseLocation This is Integration Compute Question 0037. \\ \fi
\begin{problem}

Compute the following integral:

\input{Integral-Compute-0037.HELP.tex}

\[
\int{{\frac{1}{\sqrt{x^{2} + 81}}}\;dx} = \answer{{{\rm arcsinh}\left(\frac{1}{9} \, x\right)}+C}
\]
\end{problem}}%}

\latexProblemContent{
\ifVerboseLocation This is Integration Compute Question 0037. \\ \fi
\begin{problem}

Compute the following integral:

\input{Integral-Compute-0037.HELP.tex}

\[
\int{{-\frac{2}{\sqrt{x^{2} + 4}}}\;dx} = \answer{{-2 \, {\rm arcsinh}\left(\frac{1}{2} \, x\right)}+C}
\]
\end{problem}}%}

\latexProblemContent{
\ifVerboseLocation This is Integration Compute Question 0037. \\ \fi
\begin{problem}

Compute the following integral:

\input{Integral-Compute-0037.HELP.tex}

\[
\int{{-\frac{4}{\sqrt{-x^{2} + 4}}}\;dx} = \answer{{-4 \, \arcsin\left(\frac{1}{2} \, x\right)}+C}
\]
\end{problem}}%}

\latexProblemContent{
\ifVerboseLocation This is Integration Compute Question 0037. \\ \fi
\begin{problem}

Compute the following integral:

\input{Integral-Compute-0037.HELP.tex}

\[
\int{{\frac{9}{\sqrt{x^{2} + 9}}}\;dx} = \answer{{9 \, {\rm arcsinh}\left(\frac{1}{3} \, x\right)}+C}
\]
\end{problem}}%}

\latexProblemContent{
\ifVerboseLocation This is Integration Compute Question 0037. \\ \fi
\begin{problem}

Compute the following integral:

\input{Integral-Compute-0037.HELP.tex}

\[
\int{{\frac{2}{\sqrt{x^{2} + 25}}}\;dx} = \answer{{2 \, {\rm arcsinh}\left(\frac{1}{5} \, x\right)}+C}
\]
\end{problem}}%}

\latexProblemContent{
\ifVerboseLocation This is Integration Compute Question 0037. \\ \fi
\begin{problem}

Compute the following integral:

\input{Integral-Compute-0037.HELP.tex}

\[
\int{{\frac{7}{\sqrt{x^{2} + 9}}}\;dx} = \answer{{7 \, {\rm arcsinh}\left(\frac{1}{3} \, x\right)}+C}
\]
\end{problem}}%}

\latexProblemContent{
\ifVerboseLocation This is Integration Compute Question 0037. \\ \fi
\begin{problem}

Compute the following integral:

\input{Integral-Compute-0037.HELP.tex}

\[
\int{{\frac{2}{\sqrt{-x^{2} + 4}}}\;dx} = \answer{{2 \, \arcsin\left(\frac{1}{2} \, x\right)}+C}
\]
\end{problem}}%}

\latexProblemContent{
\ifVerboseLocation This is Integration Compute Question 0037. \\ \fi
\begin{problem}

Compute the following integral:

\input{Integral-Compute-0037.HELP.tex}

\[
\int{{-\frac{5}{\sqrt{-x^{2} + 36}}}\;dx} = \answer{{-5 \, \arcsin\left(\frac{1}{6} \, x\right)}+C}
\]
\end{problem}}%}

\latexProblemContent{
\ifVerboseLocation This is Integration Compute Question 0037. \\ \fi
\begin{problem}

Compute the following integral:

\input{Integral-Compute-0037.HELP.tex}

\[
\int{{-\frac{1}{\sqrt{x^{2} + 64}}}\;dx} = \answer{{-{\rm arcsinh}\left(\frac{1}{8} \, x\right)}+C}
\]
\end{problem}}%}

\latexProblemContent{
\ifVerboseLocation This is Integration Compute Question 0037. \\ \fi
\begin{problem}

Compute the following integral:

\input{Integral-Compute-0037.HELP.tex}

\[
\int{{\frac{5}{\sqrt{x^{2} - 81}}}\;dx} = \answer{{5 \, \log\left(2 \, x + 2 \, \sqrt{x^{2} - 81}\right)}+C}
\]
\end{problem}}%}

\latexProblemContent{
\ifVerboseLocation This is Integration Compute Question 0037. \\ \fi
\begin{problem}

Compute the following integral:

\input{Integral-Compute-0037.HELP.tex}

\[
\int{{-\frac{8}{\sqrt{-x^{2} + 4}}}\;dx} = \answer{{-8 \, \arcsin\left(\frac{1}{2} \, x\right)}+C}
\]
\end{problem}}%}

\latexProblemContent{
\ifVerboseLocation This is Integration Compute Question 0037. \\ \fi
\begin{problem}

Compute the following integral:

\input{Integral-Compute-0037.HELP.tex}

\[
\int{{-\frac{8}{\sqrt{-x^{2} + 36}}}\;dx} = \answer{{-8 \, \arcsin\left(\frac{1}{6} \, x\right)}+C}
\]
\end{problem}}%}

\latexProblemContent{
\ifVerboseLocation This is Integration Compute Question 0037. \\ \fi
\begin{problem}

Compute the following integral:

\input{Integral-Compute-0037.HELP.tex}

\[
\int{{-\frac{2}{\sqrt{x^{2} + 64}}}\;dx} = \answer{{-2 \, {\rm arcsinh}\left(\frac{1}{8} \, x\right)}+C}
\]
\end{problem}}%}

\latexProblemContent{
\ifVerboseLocation This is Integration Compute Question 0037. \\ \fi
\begin{problem}

Compute the following integral:

\input{Integral-Compute-0037.HELP.tex}

\[
\int{{\frac{5}{\sqrt{x^{2} - 36}}}\;dx} = \answer{{5 \, \log\left(2 \, x + 2 \, \sqrt{x^{2} - 36}\right)}+C}
\]
\end{problem}}%}

\latexProblemContent{
\ifVerboseLocation This is Integration Compute Question 0037. \\ \fi
\begin{problem}

Compute the following integral:

\input{Integral-Compute-0037.HELP.tex}

\[
\int{{\frac{6}{\sqrt{x^{2} - 49}}}\;dx} = \answer{{6 \, \log\left(2 \, x + 2 \, \sqrt{x^{2} - 49}\right)}+C}
\]
\end{problem}}%}

\latexProblemContent{
\ifVerboseLocation This is Integration Compute Question 0037. \\ \fi
\begin{problem}

Compute the following integral:

\input{Integral-Compute-0037.HELP.tex}

\[
\int{{-\frac{2}{\sqrt{-x^{2} + 25}}}\;dx} = \answer{{-2 \, \arcsin\left(\frac{1}{5} \, x\right)}+C}
\]
\end{problem}}%}

\latexProblemContent{
\ifVerboseLocation This is Integration Compute Question 0037. \\ \fi
\begin{problem}

Compute the following integral:

\input{Integral-Compute-0037.HELP.tex}

\[
\int{{\frac{8}{\sqrt{x^{2} + 81}}}\;dx} = \answer{{8 \, {\rm arcsinh}\left(\frac{1}{9} \, x\right)}+C}
\]
\end{problem}}%}

\latexProblemContent{
\ifVerboseLocation This is Integration Compute Question 0037. \\ \fi
\begin{problem}

Compute the following integral:

\input{Integral-Compute-0037.HELP.tex}

\[
\int{{\frac{9}{\sqrt{-x^{2} + 36}}}\;dx} = \answer{{9 \, \arcsin\left(\frac{1}{6} \, x\right)}+C}
\]
\end{problem}}%}

\latexProblemContent{
\ifVerboseLocation This is Integration Compute Question 0037. \\ \fi
\begin{problem}

Compute the following integral:

\input{Integral-Compute-0037.HELP.tex}

\[
\int{{-\frac{1}{\sqrt{-x^{2} + 1}}}\;dx} = \answer{{-\arcsin\left(x\right)}+C}
\]
\end{problem}}%}

\latexProblemContent{
\ifVerboseLocation This is Integration Compute Question 0037. \\ \fi
\begin{problem}

Compute the following integral:

\input{Integral-Compute-0037.HELP.tex}

\[
\int{{-\frac{9}{\sqrt{x^{2} - 49}}}\;dx} = \answer{{-9 \, \log\left(2 \, x + 2 \, \sqrt{x^{2} - 49}\right)}+C}
\]
\end{problem}}%}

\latexProblemContent{
\ifVerboseLocation This is Integration Compute Question 0037. \\ \fi
\begin{problem}

Compute the following integral:

\input{Integral-Compute-0037.HELP.tex}

\[
\int{{-\frac{7}{\sqrt{x^{2} + 49}}}\;dx} = \answer{{-7 \, {\rm arcsinh}\left(\frac{1}{7} \, x\right)}+C}
\]
\end{problem}}%}

\latexProblemContent{
\ifVerboseLocation This is Integration Compute Question 0037. \\ \fi
\begin{problem}

Compute the following integral:

\input{Integral-Compute-0037.HELP.tex}

\[
\int{{-\frac{3}{\sqrt{x^{2} - 81}}}\;dx} = \answer{{-3 \, \log\left(2 \, x + 2 \, \sqrt{x^{2} - 81}\right)}+C}
\]
\end{problem}}%}

\latexProblemContent{
\ifVerboseLocation This is Integration Compute Question 0037. \\ \fi
\begin{problem}

Compute the following integral:

\input{Integral-Compute-0037.HELP.tex}

\[
\int{{-\frac{1}{\sqrt{x^{2} + 81}}}\;dx} = \answer{{-{\rm arcsinh}\left(\frac{1}{9} \, x\right)}+C}
\]
\end{problem}}%}

\latexProblemContent{
\ifVerboseLocation This is Integration Compute Question 0037. \\ \fi
\begin{problem}

Compute the following integral:

\input{Integral-Compute-0037.HELP.tex}

\[
\int{{-\frac{4}{\sqrt{x^{2} + 4}}}\;dx} = \answer{{-4 \, {\rm arcsinh}\left(\frac{1}{2} \, x\right)}+C}
\]
\end{problem}}%}

\latexProblemContent{
\ifVerboseLocation This is Integration Compute Question 0037. \\ \fi
\begin{problem}

Compute the following integral:

\input{Integral-Compute-0037.HELP.tex}

\[
\int{{\frac{9}{\sqrt{x^{2} - 1}}}\;dx} = \answer{{9 \, \log\left(2 \, x + 2 \, \sqrt{x^{2} - 1}\right)}+C}
\]
\end{problem}}%}

\latexProblemContent{
\ifVerboseLocation This is Integration Compute Question 0037. \\ \fi
\begin{problem}

Compute the following integral:

\input{Integral-Compute-0037.HELP.tex}

\[
\int{{-\frac{1}{\sqrt{-x^{2} + 64}}}\;dx} = \answer{{-\arcsin\left(\frac{1}{8} \, x\right)}+C}
\]
\end{problem}}%}

\latexProblemContent{
\ifVerboseLocation This is Integration Compute Question 0037. \\ \fi
\begin{problem}

Compute the following integral:

\input{Integral-Compute-0037.HELP.tex}

\[
\int{{\frac{9}{\sqrt{x^{2} + 64}}}\;dx} = \answer{{9 \, {\rm arcsinh}\left(\frac{1}{8} \, x\right)}+C}
\]
\end{problem}}%}

\latexProblemContent{
\ifVerboseLocation This is Integration Compute Question 0037. \\ \fi
\begin{problem}

Compute the following integral:

\input{Integral-Compute-0037.HELP.tex}

\[
\int{{-\frac{9}{\sqrt{-x^{2} + 36}}}\;dx} = \answer{{-9 \, \arcsin\left(\frac{1}{6} \, x\right)}+C}
\]
\end{problem}}%}

\latexProblemContent{
\ifVerboseLocation This is Integration Compute Question 0037. \\ \fi
\begin{problem}

Compute the following integral:

\input{Integral-Compute-0037.HELP.tex}

\[
\int{{\frac{5}{\sqrt{x^{2} + 64}}}\;dx} = \answer{{5 \, {\rm arcsinh}\left(\frac{1}{8} \, x\right)}+C}
\]
\end{problem}}%}

\latexProblemContent{
\ifVerboseLocation This is Integration Compute Question 0037. \\ \fi
\begin{problem}

Compute the following integral:

\input{Integral-Compute-0037.HELP.tex}

\[
\int{{\frac{7}{\sqrt{x^{2} - 81}}}\;dx} = \answer{{7 \, \log\left(2 \, x + 2 \, \sqrt{x^{2} - 81}\right)}+C}
\]
\end{problem}}%}

\latexProblemContent{
\ifVerboseLocation This is Integration Compute Question 0037. \\ \fi
\begin{problem}

Compute the following integral:

\input{Integral-Compute-0037.HELP.tex}

\[
\int{{\frac{3}{\sqrt{x^{2} + 81}}}\;dx} = \answer{{3 \, {\rm arcsinh}\left(\frac{1}{9} \, x\right)}+C}
\]
\end{problem}}%}

\latexProblemContent{
\ifVerboseLocation This is Integration Compute Question 0037. \\ \fi
\begin{problem}

Compute the following integral:

\input{Integral-Compute-0037.HELP.tex}

\[
\int{{-\frac{1}{\sqrt{x^{2} + 1}}}\;dx} = \answer{{-{\rm arcsinh}\left(x\right)}+C}
\]
\end{problem}}%}

\latexProblemContent{
\ifVerboseLocation This is Integration Compute Question 0037. \\ \fi
\begin{problem}

Compute the following integral:

\input{Integral-Compute-0037.HELP.tex}

\[
\int{{-\frac{2}{\sqrt{-x^{2} + 4}}}\;dx} = \answer{{-2 \, \arcsin\left(\frac{1}{2} \, x\right)}+C}
\]
\end{problem}}%}

\latexProblemContent{
\ifVerboseLocation This is Integration Compute Question 0037. \\ \fi
\begin{problem}

Compute the following integral:

\input{Integral-Compute-0037.HELP.tex}

\[
\int{{-\frac{7}{\sqrt{x^{2} - 49}}}\;dx} = \answer{{-7 \, \log\left(2 \, x + 2 \, \sqrt{x^{2} - 49}\right)}+C}
\]
\end{problem}}%}

\latexProblemContent{
\ifVerboseLocation This is Integration Compute Question 0037. \\ \fi
\begin{problem}

Compute the following integral:

\input{Integral-Compute-0037.HELP.tex}

\[
\int{{-\frac{8}{\sqrt{x^{2} + 49}}}\;dx} = \answer{{-8 \, {\rm arcsinh}\left(\frac{1}{7} \, x\right)}+C}
\]
\end{problem}}%}

\latexProblemContent{
\ifVerboseLocation This is Integration Compute Question 0037. \\ \fi
\begin{problem}

Compute the following integral:

\input{Integral-Compute-0037.HELP.tex}

\[
\int{{-\frac{9}{\sqrt{x^{2} - 36}}}\;dx} = \answer{{-9 \, \log\left(2 \, x + 2 \, \sqrt{x^{2} - 36}\right)}+C}
\]
\end{problem}}%}

\latexProblemContent{
\ifVerboseLocation This is Integration Compute Question 0037. \\ \fi
\begin{problem}

Compute the following integral:

\input{Integral-Compute-0037.HELP.tex}

\[
\int{{\frac{3}{\sqrt{x^{2} + 49}}}\;dx} = \answer{{3 \, {\rm arcsinh}\left(\frac{1}{7} \, x\right)}+C}
\]
\end{problem}}%}

\latexProblemContent{
\ifVerboseLocation This is Integration Compute Question 0037. \\ \fi
\begin{problem}

Compute the following integral:

\input{Integral-Compute-0037.HELP.tex}

\[
\int{{-\frac{6}{\sqrt{x^{2} - 1}}}\;dx} = \answer{{-6 \, \log\left(2 \, x + 2 \, \sqrt{x^{2} - 1}\right)}+C}
\]
\end{problem}}%}

\latexProblemContent{
\ifVerboseLocation This is Integration Compute Question 0037. \\ \fi
\begin{problem}

Compute the following integral:

\input{Integral-Compute-0037.HELP.tex}

\[
\int{{-\frac{5}{\sqrt{x^{2} - 16}}}\;dx} = \answer{{-5 \, \log\left(2 \, x + 2 \, \sqrt{x^{2} - 16}\right)}+C}
\]
\end{problem}}%}

\latexProblemContent{
\ifVerboseLocation This is Integration Compute Question 0037. \\ \fi
\begin{problem}

Compute the following integral:

\input{Integral-Compute-0037.HELP.tex}

\[
\int{{\frac{6}{\sqrt{x^{2} + 16}}}\;dx} = \answer{{6 \, {\rm arcsinh}\left(\frac{1}{4} \, x\right)}+C}
\]
\end{problem}}%}

\latexProblemContent{
\ifVerboseLocation This is Integration Compute Question 0037. \\ \fi
\begin{problem}

Compute the following integral:

\input{Integral-Compute-0037.HELP.tex}

\[
\int{{\frac{1}{\sqrt{x^{2} - 16}}}\;dx} = \answer{{\log\left(2 \, x + 2 \, \sqrt{x^{2} - 16}\right)}+C}
\]
\end{problem}}%}

\latexProblemContent{
\ifVerboseLocation This is Integration Compute Question 0037. \\ \fi
\begin{problem}

Compute the following integral:

\input{Integral-Compute-0037.HELP.tex}

\[
\int{{-\frac{7}{\sqrt{x^{2} + 25}}}\;dx} = \answer{{-7 \, {\rm arcsinh}\left(\frac{1}{5} \, x\right)}+C}
\]
\end{problem}}%}

\latexProblemContent{
\ifVerboseLocation This is Integration Compute Question 0037. \\ \fi
\begin{problem}

Compute the following integral:

\input{Integral-Compute-0037.HELP.tex}

\[
\int{{\frac{3}{\sqrt{x^{2} + 9}}}\;dx} = \answer{{3 \, {\rm arcsinh}\left(\frac{1}{3} \, x\right)}+C}
\]
\end{problem}}%}

\latexProblemContent{
\ifVerboseLocation This is Integration Compute Question 0037. \\ \fi
\begin{problem}

Compute the following integral:

\input{Integral-Compute-0037.HELP.tex}

\[
\int{{-\frac{3}{\sqrt{-x^{2} + 81}}}\;dx} = \answer{{-3 \, \arcsin\left(\frac{1}{9} \, x\right)}+C}
\]
\end{problem}}%}

\latexProblemContent{
\ifVerboseLocation This is Integration Compute Question 0037. \\ \fi
\begin{problem}

Compute the following integral:

\input{Integral-Compute-0037.HELP.tex}

\[
\int{{\frac{7}{\sqrt{x^{2} - 1}}}\;dx} = \answer{{7 \, \log\left(2 \, x + 2 \, \sqrt{x^{2} - 1}\right)}+C}
\]
\end{problem}}%}

\latexProblemContent{
\ifVerboseLocation This is Integration Compute Question 0037. \\ \fi
\begin{problem}

Compute the following integral:

\input{Integral-Compute-0037.HELP.tex}

\[
\int{{-\frac{7}{\sqrt{x^{2} - 25}}}\;dx} = \answer{{-7 \, \log\left(2 \, x + 2 \, \sqrt{x^{2} - 25}\right)}+C}
\]
\end{problem}}%}

\latexProblemContent{
\ifVerboseLocation This is Integration Compute Question 0037. \\ \fi
\begin{problem}

Compute the following integral:

\input{Integral-Compute-0037.HELP.tex}

\[
\int{{-\frac{8}{\sqrt{x^{2} - 81}}}\;dx} = \answer{{-8 \, \log\left(2 \, x + 2 \, \sqrt{x^{2} - 81}\right)}+C}
\]
\end{problem}}%}

\latexProblemContent{
\ifVerboseLocation This is Integration Compute Question 0037. \\ \fi
\begin{problem}

Compute the following integral:

\input{Integral-Compute-0037.HELP.tex}

\[
\int{{\frac{3}{\sqrt{x^{2} + 25}}}\;dx} = \answer{{3 \, {\rm arcsinh}\left(\frac{1}{5} \, x\right)}+C}
\]
\end{problem}}%}

\latexProblemContent{
\ifVerboseLocation This is Integration Compute Question 0037. \\ \fi
\begin{problem}

Compute the following integral:

\input{Integral-Compute-0037.HELP.tex}

\[
\int{{-\frac{5}{\sqrt{-x^{2} + 1}}}\;dx} = \answer{{-5 \, \arcsin\left(x\right)}+C}
\]
\end{problem}}%}

\latexProblemContent{
\ifVerboseLocation This is Integration Compute Question 0037. \\ \fi
\begin{problem}

Compute the following integral:

\input{Integral-Compute-0037.HELP.tex}

\[
\int{{-\frac{2}{\sqrt{x^{2} - 9}}}\;dx} = \answer{{-2 \, \log\left(2 \, x + 2 \, \sqrt{x^{2} - 9}\right)}+C}
\]
\end{problem}}%}

\latexProblemContent{
\ifVerboseLocation This is Integration Compute Question 0037. \\ \fi
\begin{problem}

Compute the following integral:

\input{Integral-Compute-0037.HELP.tex}

\[
\int{{\frac{1}{\sqrt{x^{2} + 64}}}\;dx} = \answer{{{\rm arcsinh}\left(\frac{1}{8} \, x\right)}+C}
\]
\end{problem}}%}

\latexProblemContent{
\ifVerboseLocation This is Integration Compute Question 0037. \\ \fi
\begin{problem}

Compute the following integral:

\input{Integral-Compute-0037.HELP.tex}

\[
\int{{\frac{8}{\sqrt{x^{2} + 4}}}\;dx} = \answer{{8 \, {\rm arcsinh}\left(\frac{1}{2} \, x\right)}+C}
\]
\end{problem}}%}

\latexProblemContent{
\ifVerboseLocation This is Integration Compute Question 0037. \\ \fi
\begin{problem}

Compute the following integral:

\input{Integral-Compute-0037.HELP.tex}

\[
\int{{\frac{8}{\sqrt{x^{2} + 16}}}\;dx} = \answer{{8 \, {\rm arcsinh}\left(\frac{1}{4} \, x\right)}+C}
\]
\end{problem}}%}

\latexProblemContent{
\ifVerboseLocation This is Integration Compute Question 0037. \\ \fi
\begin{problem}

Compute the following integral:

\input{Integral-Compute-0037.HELP.tex}

\[
\int{{\frac{3}{\sqrt{x^{2} + 64}}}\;dx} = \answer{{3 \, {\rm arcsinh}\left(\frac{1}{8} \, x\right)}+C}
\]
\end{problem}}%}

\latexProblemContent{
\ifVerboseLocation This is Integration Compute Question 0037. \\ \fi
\begin{problem}

Compute the following integral:

\input{Integral-Compute-0037.HELP.tex}

\[
\int{{-\frac{8}{\sqrt{x^{2} + 36}}}\;dx} = \answer{{-8 \, {\rm arcsinh}\left(\frac{1}{6} \, x\right)}+C}
\]
\end{problem}}%}

\latexProblemContent{
\ifVerboseLocation This is Integration Compute Question 0037. \\ \fi
\begin{problem}

Compute the following integral:

\input{Integral-Compute-0037.HELP.tex}

\[
\int{{\frac{1}{\sqrt{x^{2} - 1}}}\;dx} = \answer{{\log\left(2 \, x + 2 \, \sqrt{x^{2} - 1}\right)}+C}
\]
\end{problem}}%}

\latexProblemContent{
\ifVerboseLocation This is Integration Compute Question 0037. \\ \fi
\begin{problem}

Compute the following integral:

\input{Integral-Compute-0037.HELP.tex}

\[
\int{{\frac{3}{\sqrt{x^{2} + 4}}}\;dx} = \answer{{3 \, {\rm arcsinh}\left(\frac{1}{2} \, x\right)}+C}
\]
\end{problem}}%}

\latexProblemContent{
\ifVerboseLocation This is Integration Compute Question 0037. \\ \fi
\begin{problem}

Compute the following integral:

\input{Integral-Compute-0037.HELP.tex}

\[
\int{{-\frac{5}{\sqrt{x^{2} + 36}}}\;dx} = \answer{{-5 \, {\rm arcsinh}\left(\frac{1}{6} \, x\right)}+C}
\]
\end{problem}}%}

\latexProblemContent{
\ifVerboseLocation This is Integration Compute Question 0037. \\ \fi
\begin{problem}

Compute the following integral:

\input{Integral-Compute-0037.HELP.tex}

\[
\int{{-\frac{8}{\sqrt{x^{2} + 9}}}\;dx} = \answer{{-8 \, {\rm arcsinh}\left(\frac{1}{3} \, x\right)}+C}
\]
\end{problem}}%}

\latexProblemContent{
\ifVerboseLocation This is Integration Compute Question 0037. \\ \fi
\begin{problem}

Compute the following integral:

\input{Integral-Compute-0037.HELP.tex}

\[
\int{{-\frac{2}{\sqrt{-x^{2} + 81}}}\;dx} = \answer{{-2 \, \arcsin\left(\frac{1}{9} \, x\right)}+C}
\]
\end{problem}}%}

\latexProblemContent{
\ifVerboseLocation This is Integration Compute Question 0037. \\ \fi
\begin{problem}

Compute the following integral:

\input{Integral-Compute-0037.HELP.tex}

\[
\int{{-\frac{8}{\sqrt{-x^{2} + 81}}}\;dx} = \answer{{-8 \, \arcsin\left(\frac{1}{9} \, x\right)}+C}
\]
\end{problem}}%}

\latexProblemContent{
\ifVerboseLocation This is Integration Compute Question 0037. \\ \fi
\begin{problem}

Compute the following integral:

\input{Integral-Compute-0037.HELP.tex}

\[
\int{{\frac{9}{\sqrt{x^{2} + 4}}}\;dx} = \answer{{9 \, {\rm arcsinh}\left(\frac{1}{2} \, x\right)}+C}
\]
\end{problem}}%}

\latexProblemContent{
\ifVerboseLocation This is Integration Compute Question 0037. \\ \fi
\begin{problem}

Compute the following integral:

\input{Integral-Compute-0037.HELP.tex}

\[
\int{{\frac{9}{\sqrt{x^{2} - 4}}}\;dx} = \answer{{9 \, \log\left(2 \, x + 2 \, \sqrt{x^{2} - 4}\right)}+C}
\]
\end{problem}}%}

\latexProblemContent{
\ifVerboseLocation This is Integration Compute Question 0037. \\ \fi
\begin{problem}

Compute the following integral:

\input{Integral-Compute-0037.HELP.tex}

\[
\int{{-\frac{4}{\sqrt{-x^{2} + 9}}}\;dx} = \answer{{-4 \, \arcsin\left(\frac{1}{3} \, x\right)}+C}
\]
\end{problem}}%}

\latexProblemContent{
\ifVerboseLocation This is Integration Compute Question 0037. \\ \fi
\begin{problem}

Compute the following integral:

\input{Integral-Compute-0037.HELP.tex}

\[
\int{{-\frac{9}{\sqrt{x^{2} - 81}}}\;dx} = \answer{{-9 \, \log\left(2 \, x + 2 \, \sqrt{x^{2} - 81}\right)}+C}
\]
\end{problem}}%}

\latexProblemContent{
\ifVerboseLocation This is Integration Compute Question 0037. \\ \fi
\begin{problem}

Compute the following integral:

\input{Integral-Compute-0037.HELP.tex}

\[
\int{{-\frac{3}{\sqrt{-x^{2} + 25}}}\;dx} = \answer{{-3 \, \arcsin\left(\frac{1}{5} \, x\right)}+C}
\]
\end{problem}}%}

\latexProblemContent{
\ifVerboseLocation This is Integration Compute Question 0037. \\ \fi
\begin{problem}

Compute the following integral:

\input{Integral-Compute-0037.HELP.tex}

\[
\int{{-\frac{8}{\sqrt{x^{2} + 64}}}\;dx} = \answer{{-8 \, {\rm arcsinh}\left(\frac{1}{8} \, x\right)}+C}
\]
\end{problem}}%}

\latexProblemContent{
\ifVerboseLocation This is Integration Compute Question 0037. \\ \fi
\begin{problem}

Compute the following integral:

\input{Integral-Compute-0037.HELP.tex}

\[
\int{{\frac{5}{\sqrt{x^{2} - 25}}}\;dx} = \answer{{5 \, \log\left(2 \, x + 2 \, \sqrt{x^{2} - 25}\right)}+C}
\]
\end{problem}}%}

\latexProblemContent{
\ifVerboseLocation This is Integration Compute Question 0037. \\ \fi
\begin{problem}

Compute the following integral:

\input{Integral-Compute-0037.HELP.tex}

\[
\int{{\frac{3}{\sqrt{x^{2} - 4}}}\;dx} = \answer{{3 \, \log\left(2 \, x + 2 \, \sqrt{x^{2} - 4}\right)}+C}
\]
\end{problem}}%}

\latexProblemContent{
\ifVerboseLocation This is Integration Compute Question 0037. \\ \fi
\begin{problem}

Compute the following integral:

\input{Integral-Compute-0037.HELP.tex}

\[
\int{{-\frac{4}{\sqrt{-x^{2} + 1}}}\;dx} = \answer{{-4 \, \arcsin\left(x\right)}+C}
\]
\end{problem}}%}

\latexProblemContent{
\ifVerboseLocation This is Integration Compute Question 0037. \\ \fi
\begin{problem}

Compute the following integral:

\input{Integral-Compute-0037.HELP.tex}

\[
\int{{\frac{6}{\sqrt{x^{2} - 9}}}\;dx} = \answer{{6 \, \log\left(2 \, x + 2 \, \sqrt{x^{2} - 9}\right)}+C}
\]
\end{problem}}%}

\latexProblemContent{
\ifVerboseLocation This is Integration Compute Question 0037. \\ \fi
\begin{problem}

Compute the following integral:

\input{Integral-Compute-0037.HELP.tex}

\[
\int{{\frac{7}{\sqrt{x^{2} + 64}}}\;dx} = \answer{{7 \, {\rm arcsinh}\left(\frac{1}{8} \, x\right)}+C}
\]
\end{problem}}%}

\latexProblemContent{
\ifVerboseLocation This is Integration Compute Question 0037. \\ \fi
\begin{problem}

Compute the following integral:

\input{Integral-Compute-0037.HELP.tex}

\[
\int{{\frac{5}{\sqrt{-x^{2} + 25}}}\;dx} = \answer{{5 \, \arcsin\left(\frac{1}{5} \, x\right)}+C}
\]
\end{problem}}%}

\latexProblemContent{
\ifVerboseLocation This is Integration Compute Question 0037. \\ \fi
\begin{problem}

Compute the following integral:

\input{Integral-Compute-0037.HELP.tex}

\[
\int{{-\frac{2}{\sqrt{x^{2} + 25}}}\;dx} = \answer{{-2 \, {\rm arcsinh}\left(\frac{1}{5} \, x\right)}+C}
\]
\end{problem}}%}

\latexProblemContent{
\ifVerboseLocation This is Integration Compute Question 0037. \\ \fi
\begin{problem}

Compute the following integral:

\input{Integral-Compute-0037.HELP.tex}

\[
\int{{\frac{5}{\sqrt{-x^{2} + 4}}}\;dx} = \answer{{5 \, \arcsin\left(\frac{1}{2} \, x\right)}+C}
\]
\end{problem}}%}

\latexProblemContent{
\ifVerboseLocation This is Integration Compute Question 0037. \\ \fi
\begin{problem}

Compute the following integral:

\input{Integral-Compute-0037.HELP.tex}

\[
\int{{-\frac{9}{\sqrt{-x^{2} + 9}}}\;dx} = \answer{{-9 \, \arcsin\left(\frac{1}{3} \, x\right)}+C}
\]
\end{problem}}%}

\latexProblemContent{
\ifVerboseLocation This is Integration Compute Question 0037. \\ \fi
\begin{problem}

Compute the following integral:

\input{Integral-Compute-0037.HELP.tex}

\[
\int{{\frac{2}{\sqrt{-x^{2} + 9}}}\;dx} = \answer{{2 \, \arcsin\left(\frac{1}{3} \, x\right)}+C}
\]
\end{problem}}%}

\latexProblemContent{
\ifVerboseLocation This is Integration Compute Question 0037. \\ \fi
\begin{problem}

Compute the following integral:

\input{Integral-Compute-0037.HELP.tex}

\[
\int{{\frac{1}{\sqrt{x^{2} - 25}}}\;dx} = \answer{{\log\left(2 \, x + 2 \, \sqrt{x^{2} - 25}\right)}+C}
\]
\end{problem}}%}

\latexProblemContent{
\ifVerboseLocation This is Integration Compute Question 0037. \\ \fi
\begin{problem}

Compute the following integral:

\input{Integral-Compute-0037.HELP.tex}

\[
\int{{\frac{2}{\sqrt{-x^{2} + 16}}}\;dx} = \answer{{2 \, \arcsin\left(\frac{1}{4} \, x\right)}+C}
\]
\end{problem}}%}

\latexProblemContent{
\ifVerboseLocation This is Integration Compute Question 0037. \\ \fi
\begin{problem}

Compute the following integral:

\input{Integral-Compute-0037.HELP.tex}

\[
\int{{-\frac{2}{\sqrt{x^{2} - 4}}}\;dx} = \answer{{-2 \, \log\left(2 \, x + 2 \, \sqrt{x^{2} - 4}\right)}+C}
\]
\end{problem}}%}

\latexProblemContent{
\ifVerboseLocation This is Integration Compute Question 0037. \\ \fi
\begin{problem}

Compute the following integral:

\input{Integral-Compute-0037.HELP.tex}

\[
\int{{\frac{2}{\sqrt{x^{2} - 9}}}\;dx} = \answer{{2 \, \log\left(2 \, x + 2 \, \sqrt{x^{2} - 9}\right)}+C}
\]
\end{problem}}%}

\latexProblemContent{
\ifVerboseLocation This is Integration Compute Question 0037. \\ \fi
\begin{problem}

Compute the following integral:

\input{Integral-Compute-0037.HELP.tex}

\[
\int{{\frac{2}{\sqrt{x^{2} - 1}}}\;dx} = \answer{{2 \, \log\left(2 \, x + 2 \, \sqrt{x^{2} - 1}\right)}+C}
\]
\end{problem}}%}

\latexProblemContent{
\ifVerboseLocation This is Integration Compute Question 0037. \\ \fi
\begin{problem}

Compute the following integral:

\input{Integral-Compute-0037.HELP.tex}

\[
\int{{-\frac{9}{\sqrt{x^{2} + 36}}}\;dx} = \answer{{-9 \, {\rm arcsinh}\left(\frac{1}{6} \, x\right)}+C}
\]
\end{problem}}%}

\latexProblemContent{
\ifVerboseLocation This is Integration Compute Question 0037. \\ \fi
\begin{problem}

Compute the following integral:

\input{Integral-Compute-0037.HELP.tex}

\[
\int{{\frac{2}{\sqrt{x^{2} + 4}}}\;dx} = \answer{{2 \, {\rm arcsinh}\left(\frac{1}{2} \, x\right)}+C}
\]
\end{problem}}%}

\latexProblemContent{
\ifVerboseLocation This is Integration Compute Question 0037. \\ \fi
\begin{problem}

Compute the following integral:

\input{Integral-Compute-0037.HELP.tex}

\[
\int{{-\frac{8}{\sqrt{x^{2} + 4}}}\;dx} = \answer{{-8 \, {\rm arcsinh}\left(\frac{1}{2} \, x\right)}+C}
\]
\end{problem}}%}

\latexProblemContent{
\ifVerboseLocation This is Integration Compute Question 0037. \\ \fi
\begin{problem}

Compute the following integral:

\input{Integral-Compute-0037.HELP.tex}

\[
\int{{\frac{1}{\sqrt{-x^{2} + 81}}}\;dx} = \answer{{\arcsin\left(\frac{1}{9} \, x\right)}+C}
\]
\end{problem}}%}

\latexProblemContent{
\ifVerboseLocation This is Integration Compute Question 0037. \\ \fi
\begin{problem}

Compute the following integral:

\input{Integral-Compute-0037.HELP.tex}

\[
\int{{-\frac{7}{\sqrt{-x^{2} + 4}}}\;dx} = \answer{{-7 \, \arcsin\left(\frac{1}{2} \, x\right)}+C}
\]
\end{problem}}%}

\latexProblemContent{
\ifVerboseLocation This is Integration Compute Question 0037. \\ \fi
\begin{problem}

Compute the following integral:

\input{Integral-Compute-0037.HELP.tex}

\[
\int{{\frac{6}{\sqrt{-x^{2} + 81}}}\;dx} = \answer{{6 \, \arcsin\left(\frac{1}{9} \, x\right)}+C}
\]
\end{problem}}%}

\latexProblemContent{
\ifVerboseLocation This is Integration Compute Question 0037. \\ \fi
\begin{problem}

Compute the following integral:

\input{Integral-Compute-0037.HELP.tex}

\[
\int{{\frac{8}{\sqrt{x^{2} + 25}}}\;dx} = \answer{{8 \, {\rm arcsinh}\left(\frac{1}{5} \, x\right)}+C}
\]
\end{problem}}%}

\latexProblemContent{
\ifVerboseLocation This is Integration Compute Question 0037. \\ \fi
\begin{problem}

Compute the following integral:

\input{Integral-Compute-0037.HELP.tex}

\[
\int{{\frac{5}{\sqrt{x^{2} - 16}}}\;dx} = \answer{{5 \, \log\left(2 \, x + 2 \, \sqrt{x^{2} - 16}\right)}+C}
\]
\end{problem}}%}

\latexProblemContent{
\ifVerboseLocation This is Integration Compute Question 0037. \\ \fi
\begin{problem}

Compute the following integral:

\input{Integral-Compute-0037.HELP.tex}

\[
\int{{\frac{2}{\sqrt{x^{2} - 16}}}\;dx} = \answer{{2 \, \log\left(2 \, x + 2 \, \sqrt{x^{2} - 16}\right)}+C}
\]
\end{problem}}%}

\latexProblemContent{
\ifVerboseLocation This is Integration Compute Question 0037. \\ \fi
\begin{problem}

Compute the following integral:

\input{Integral-Compute-0037.HELP.tex}

\[
\int{{-\frac{2}{\sqrt{x^{2} - 49}}}\;dx} = \answer{{-2 \, \log\left(2 \, x + 2 \, \sqrt{x^{2} - 49}\right)}+C}
\]
\end{problem}}%}

\latexProblemContent{
\ifVerboseLocation This is Integration Compute Question 0037. \\ \fi
\begin{problem}

Compute the following integral:

\input{Integral-Compute-0037.HELP.tex}

\[
\int{{-\frac{1}{\sqrt{x^{2} - 4}}}\;dx} = \answer{{-\log\left(2 \, x + 2 \, \sqrt{x^{2} - 4}\right)}+C}
\]
\end{problem}}%}

\latexProblemContent{
\ifVerboseLocation This is Integration Compute Question 0037. \\ \fi
\begin{problem}

Compute the following integral:

\input{Integral-Compute-0037.HELP.tex}

\[
\int{{\frac{7}{\sqrt{-x^{2} + 49}}}\;dx} = \answer{{7 \, \arcsin\left(\frac{1}{7} \, x\right)}+C}
\]
\end{problem}}%}

\latexProblemContent{
\ifVerboseLocation This is Integration Compute Question 0037. \\ \fi
\begin{problem}

Compute the following integral:

\input{Integral-Compute-0037.HELP.tex}

\[
\int{{\frac{1}{\sqrt{x^{2} + 4}}}\;dx} = \answer{{{\rm arcsinh}\left(\frac{1}{2} \, x\right)}+C}
\]
\end{problem}}%}

\latexProblemContent{
\ifVerboseLocation This is Integration Compute Question 0037. \\ \fi
\begin{problem}

Compute the following integral:

\input{Integral-Compute-0037.HELP.tex}

\[
\int{{\frac{1}{\sqrt{-x^{2} + 16}}}\;dx} = \answer{{\arcsin\left(\frac{1}{4} \, x\right)}+C}
\]
\end{problem}}%}

\latexProblemContent{
\ifVerboseLocation This is Integration Compute Question 0037. \\ \fi
\begin{problem}

Compute the following integral:

\input{Integral-Compute-0037.HELP.tex}

\[
\int{{-\frac{6}{\sqrt{x^{2} - 64}}}\;dx} = \answer{{-6 \, \log\left(2 \, x + 2 \, \sqrt{x^{2} - 64}\right)}+C}
\]
\end{problem}}%}

\latexProblemContent{
\ifVerboseLocation This is Integration Compute Question 0037. \\ \fi
\begin{problem}

Compute the following integral:

\input{Integral-Compute-0037.HELP.tex}

\[
\int{{-\frac{8}{\sqrt{x^{2} - 64}}}\;dx} = \answer{{-8 \, \log\left(2 \, x + 2 \, \sqrt{x^{2} - 64}\right)}+C}
\]
\end{problem}}%}

\latexProblemContent{
\ifVerboseLocation This is Integration Compute Question 0037. \\ \fi
\begin{problem}

Compute the following integral:

\input{Integral-Compute-0037.HELP.tex}

\[
\int{{-\frac{9}{\sqrt{x^{2} - 25}}}\;dx} = \answer{{-9 \, \log\left(2 \, x + 2 \, \sqrt{x^{2} - 25}\right)}+C}
\]
\end{problem}}%}

\latexProblemContent{
\ifVerboseLocation This is Integration Compute Question 0037. \\ \fi
\begin{problem}

Compute the following integral:

\input{Integral-Compute-0037.HELP.tex}

\[
\int{{\frac{7}{\sqrt{x^{2} + 25}}}\;dx} = \answer{{7 \, {\rm arcsinh}\left(\frac{1}{5} \, x\right)}+C}
\]
\end{problem}}%}

\latexProblemContent{
\ifVerboseLocation This is Integration Compute Question 0037. \\ \fi
\begin{problem}

Compute the following integral:

\input{Integral-Compute-0037.HELP.tex}

\[
\int{{\frac{3}{\sqrt{-x^{2} + 16}}}\;dx} = \answer{{3 \, \arcsin\left(\frac{1}{4} \, x\right)}+C}
\]
\end{problem}}%}

\latexProblemContent{
\ifVerboseLocation This is Integration Compute Question 0037. \\ \fi
\begin{problem}

Compute the following integral:

\input{Integral-Compute-0037.HELP.tex}

\[
\int{{\frac{5}{\sqrt{x^{2} + 36}}}\;dx} = \answer{{5 \, {\rm arcsinh}\left(\frac{1}{6} \, x\right)}+C}
\]
\end{problem}}%}

\latexProblemContent{
\ifVerboseLocation This is Integration Compute Question 0037. \\ \fi
\begin{problem}

Compute the following integral:

\input{Integral-Compute-0037.HELP.tex}

\[
\int{{\frac{4}{\sqrt{x^{2} + 1}}}\;dx} = \answer{{4 \, {\rm arcsinh}\left(x\right)}+C}
\]
\end{problem}}%}

\latexProblemContent{
\ifVerboseLocation This is Integration Compute Question 0037. \\ \fi
\begin{problem}

Compute the following integral:

\input{Integral-Compute-0037.HELP.tex}

\[
\int{{\frac{8}{\sqrt{x^{2} + 1}}}\;dx} = \answer{{8 \, {\rm arcsinh}\left(x\right)}+C}
\]
\end{problem}}%}

\latexProblemContent{
\ifVerboseLocation This is Integration Compute Question 0037. \\ \fi
\begin{problem}

Compute the following integral:

\input{Integral-Compute-0037.HELP.tex}

\[
\int{{-\frac{6}{\sqrt{x^{2} + 64}}}\;dx} = \answer{{-6 \, {\rm arcsinh}\left(\frac{1}{8} \, x\right)}+C}
\]
\end{problem}}%}

\latexProblemContent{
\ifVerboseLocation This is Integration Compute Question 0037. \\ \fi
\begin{problem}

Compute the following integral:

\input{Integral-Compute-0037.HELP.tex}

\[
\int{{\frac{4}{\sqrt{-x^{2} + 64}}}\;dx} = \answer{{4 \, \arcsin\left(\frac{1}{8} \, x\right)}+C}
\]
\end{problem}}%}

\latexProblemContent{
\ifVerboseLocation This is Integration Compute Question 0037. \\ \fi
\begin{problem}

Compute the following integral:

\input{Integral-Compute-0037.HELP.tex}

\[
\int{{-\frac{6}{\sqrt{x^{2} + 25}}}\;dx} = \answer{{-6 \, {\rm arcsinh}\left(\frac{1}{5} \, x\right)}+C}
\]
\end{problem}}%}

\latexProblemContent{
\ifVerboseLocation This is Integration Compute Question 0037. \\ \fi
\begin{problem}

Compute the following integral:

\input{Integral-Compute-0037.HELP.tex}

\[
\int{{-\frac{7}{\sqrt{x^{2} + 1}}}\;dx} = \answer{{-7 \, {\rm arcsinh}\left(x\right)}+C}
\]
\end{problem}}%}

\latexProblemContent{
\ifVerboseLocation This is Integration Compute Question 0037. \\ \fi
\begin{problem}

Compute the following integral:

\input{Integral-Compute-0037.HELP.tex}

\[
\int{{-\frac{2}{\sqrt{x^{2} + 9}}}\;dx} = \answer{{-2 \, {\rm arcsinh}\left(\frac{1}{3} \, x\right)}+C}
\]
\end{problem}}%}

\latexProblemContent{
\ifVerboseLocation This is Integration Compute Question 0037. \\ \fi
\begin{problem}

Compute the following integral:

\input{Integral-Compute-0037.HELP.tex}

\[
\int{{\frac{8}{\sqrt{x^{2} - 81}}}\;dx} = \answer{{8 \, \log\left(2 \, x + 2 \, \sqrt{x^{2} - 81}\right)}+C}
\]
\end{problem}}%}

\latexProblemContent{
\ifVerboseLocation This is Integration Compute Question 0037. \\ \fi
\begin{problem}

Compute the following integral:

\input{Integral-Compute-0037.HELP.tex}

\[
\int{{\frac{9}{\sqrt{x^{2} - 16}}}\;dx} = \answer{{9 \, \log\left(2 \, x + 2 \, \sqrt{x^{2} - 16}\right)}+C}
\]
\end{problem}}%}

\latexProblemContent{
\ifVerboseLocation This is Integration Compute Question 0037. \\ \fi
\begin{problem}

Compute the following integral:

\input{Integral-Compute-0037.HELP.tex}

\[
\int{{-\frac{3}{\sqrt{x^{2} - 1}}}\;dx} = \answer{{-3 \, \log\left(2 \, x + 2 \, \sqrt{x^{2} - 1}\right)}+C}
\]
\end{problem}}%}

\latexProblemContent{
\ifVerboseLocation This is Integration Compute Question 0037. \\ \fi
\begin{problem}

Compute the following integral:

\input{Integral-Compute-0037.HELP.tex}

\[
\int{{\frac{1}{\sqrt{-x^{2} + 64}}}\;dx} = \answer{{\arcsin\left(\frac{1}{8} \, x\right)}+C}
\]
\end{problem}}%}

\latexProblemContent{
\ifVerboseLocation This is Integration Compute Question 0037. \\ \fi
\begin{problem}

Compute the following integral:

\input{Integral-Compute-0037.HELP.tex}

\[
\int{{-\frac{6}{\sqrt{-x^{2} + 16}}}\;dx} = \answer{{-6 \, \arcsin\left(\frac{1}{4} \, x\right)}+C}
\]
\end{problem}}%}

\latexProblemContent{
\ifVerboseLocation This is Integration Compute Question 0037. \\ \fi
\begin{problem}

Compute the following integral:

\input{Integral-Compute-0037.HELP.tex}

\[
\int{{-\frac{2}{\sqrt{-x^{2} + 36}}}\;dx} = \answer{{-2 \, \arcsin\left(\frac{1}{6} \, x\right)}+C}
\]
\end{problem}}%}

\latexProblemContent{
\ifVerboseLocation This is Integration Compute Question 0037. \\ \fi
\begin{problem}

Compute the following integral:

\input{Integral-Compute-0037.HELP.tex}

\[
\int{{\frac{6}{\sqrt{x^{2} - 1}}}\;dx} = \answer{{6 \, \log\left(2 \, x + 2 \, \sqrt{x^{2} - 1}\right)}+C}
\]
\end{problem}}%}

\latexProblemContent{
\ifVerboseLocation This is Integration Compute Question 0037. \\ \fi
\begin{problem}

Compute the following integral:

\input{Integral-Compute-0037.HELP.tex}

\[
\int{{-\frac{4}{\sqrt{x^{2} - 1}}}\;dx} = \answer{{-4 \, \log\left(2 \, x + 2 \, \sqrt{x^{2} - 1}\right)}+C}
\]
\end{problem}}%}

\latexProblemContent{
\ifVerboseLocation This is Integration Compute Question 0037. \\ \fi
\begin{problem}

Compute the following integral:

\input{Integral-Compute-0037.HELP.tex}

\[
\int{{\frac{7}{\sqrt{-x^{2} + 36}}}\;dx} = \answer{{7 \, \arcsin\left(\frac{1}{6} \, x\right)}+C}
\]
\end{problem}}%}

\latexProblemContent{
\ifVerboseLocation This is Integration Compute Question 0037. \\ \fi
\begin{problem}

Compute the following integral:

\input{Integral-Compute-0037.HELP.tex}

\[
\int{{-\frac{4}{\sqrt{-x^{2} + 49}}}\;dx} = \answer{{-4 \, \arcsin\left(\frac{1}{7} \, x\right)}+C}
\]
\end{problem}}%}

\latexProblemContent{
\ifVerboseLocation This is Integration Compute Question 0037. \\ \fi
\begin{problem}

Compute the following integral:

\input{Integral-Compute-0037.HELP.tex}

\[
\int{{-\frac{1}{\sqrt{x^{2} - 25}}}\;dx} = \answer{{-\log\left(2 \, x + 2 \, \sqrt{x^{2} - 25}\right)}+C}
\]
\end{problem}}%}

\latexProblemContent{
\ifVerboseLocation This is Integration Compute Question 0037. \\ \fi
\begin{problem}

Compute the following integral:

\input{Integral-Compute-0037.HELP.tex}

\[
\int{{\frac{7}{\sqrt{x^{2} - 49}}}\;dx} = \answer{{7 \, \log\left(2 \, x + 2 \, \sqrt{x^{2} - 49}\right)}+C}
\]
\end{problem}}%}

\latexProblemContent{
\ifVerboseLocation This is Integration Compute Question 0037. \\ \fi
\begin{problem}

Compute the following integral:

\input{Integral-Compute-0037.HELP.tex}

\[
\int{{\frac{4}{\sqrt{-x^{2} + 25}}}\;dx} = \answer{{4 \, \arcsin\left(\frac{1}{5} \, x\right)}+C}
\]
\end{problem}}%}

\latexProblemContent{
\ifVerboseLocation This is Integration Compute Question 0037. \\ \fi
\begin{problem}

Compute the following integral:

\input{Integral-Compute-0037.HELP.tex}

\[
\int{{\frac{7}{\sqrt{x^{2} - 4}}}\;dx} = \answer{{7 \, \log\left(2 \, x + 2 \, \sqrt{x^{2} - 4}\right)}+C}
\]
\end{problem}}%}

\latexProblemContent{
\ifVerboseLocation This is Integration Compute Question 0037. \\ \fi
\begin{problem}

Compute the following integral:

\input{Integral-Compute-0037.HELP.tex}

\[
\int{{\frac{1}{\sqrt{x^{2} - 36}}}\;dx} = \answer{{\log\left(2 \, x + 2 \, \sqrt{x^{2} - 36}\right)}+C}
\]
\end{problem}}%}

\latexProblemContent{
\ifVerboseLocation This is Integration Compute Question 0037. \\ \fi
\begin{problem}

Compute the following integral:

\input{Integral-Compute-0037.HELP.tex}

\[
\int{{-\frac{9}{\sqrt{x^{2} - 4}}}\;dx} = \answer{{-9 \, \log\left(2 \, x + 2 \, \sqrt{x^{2} - 4}\right)}+C}
\]
\end{problem}}%}

\latexProblemContent{
\ifVerboseLocation This is Integration Compute Question 0037. \\ \fi
\begin{problem}

Compute the following integral:

\input{Integral-Compute-0037.HELP.tex}

\[
\int{{\frac{1}{\sqrt{x^{2} + 36}}}\;dx} = \answer{{{\rm arcsinh}\left(\frac{1}{6} \, x\right)}+C}
\]
\end{problem}}%}

\latexProblemContent{
\ifVerboseLocation This is Integration Compute Question 0037. \\ \fi
\begin{problem}

Compute the following integral:

\input{Integral-Compute-0037.HELP.tex}

\[
\int{{-\frac{3}{\sqrt{x^{2} - 36}}}\;dx} = \answer{{-3 \, \log\left(2 \, x + 2 \, \sqrt{x^{2} - 36}\right)}+C}
\]
\end{problem}}%}

\latexProblemContent{
\ifVerboseLocation This is Integration Compute Question 0037. \\ \fi
\begin{problem}

Compute the following integral:

\input{Integral-Compute-0037.HELP.tex}

\[
\int{{\frac{4}{\sqrt{x^{2} + 9}}}\;dx} = \answer{{4 \, {\rm arcsinh}\left(\frac{1}{3} \, x\right)}+C}
\]
\end{problem}}%}

\latexProblemContent{
\ifVerboseLocation This is Integration Compute Question 0037. \\ \fi
\begin{problem}

Compute the following integral:

\input{Integral-Compute-0037.HELP.tex}

\[
\int{{-\frac{7}{\sqrt{x^{2} - 64}}}\;dx} = \answer{{-7 \, \log\left(2 \, x + 2 \, \sqrt{x^{2} - 64}\right)}+C}
\]
\end{problem}}%}

\latexProblemContent{
\ifVerboseLocation This is Integration Compute Question 0037. \\ \fi
\begin{problem}

Compute the following integral:

\input{Integral-Compute-0037.HELP.tex}

\[
\int{{\frac{5}{\sqrt{x^{2} + 9}}}\;dx} = \answer{{5 \, {\rm arcsinh}\left(\frac{1}{3} \, x\right)}+C}
\]
\end{problem}}%}

\latexProblemContent{
\ifVerboseLocation This is Integration Compute Question 0037. \\ \fi
\begin{problem}

Compute the following integral:

\input{Integral-Compute-0037.HELP.tex}

\[
\int{{\frac{6}{\sqrt{-x^{2} + 64}}}\;dx} = \answer{{6 \, \arcsin\left(\frac{1}{8} \, x\right)}+C}
\]
\end{problem}}%}

\latexProblemContent{
\ifVerboseLocation This is Integration Compute Question 0037. \\ \fi
\begin{problem}

Compute the following integral:

\input{Integral-Compute-0037.HELP.tex}

\[
\int{{-\frac{5}{\sqrt{-x^{2} + 9}}}\;dx} = \answer{{-5 \, \arcsin\left(\frac{1}{3} \, x\right)}+C}
\]
\end{problem}}%}

\latexProblemContent{
\ifVerboseLocation This is Integration Compute Question 0037. \\ \fi
\begin{problem}

Compute the following integral:

\input{Integral-Compute-0037.HELP.tex}

\[
\int{{\frac{2}{\sqrt{x^{2} + 16}}}\;dx} = \answer{{2 \, {\rm arcsinh}\left(\frac{1}{4} \, x\right)}+C}
\]
\end{problem}}%}

\latexProblemContent{
\ifVerboseLocation This is Integration Compute Question 0037. \\ \fi
\begin{problem}

Compute the following integral:

\input{Integral-Compute-0037.HELP.tex}

\[
\int{{-\frac{1}{\sqrt{x^{2} - 64}}}\;dx} = \answer{{-\log\left(2 \, x + 2 \, \sqrt{x^{2} - 64}\right)}+C}
\]
\end{problem}}%}

\latexProblemContent{
\ifVerboseLocation This is Integration Compute Question 0037. \\ \fi
\begin{problem}

Compute the following integral:

\input{Integral-Compute-0037.HELP.tex}

\[
\int{{-\frac{3}{\sqrt{-x^{2} + 4}}}\;dx} = \answer{{-3 \, \arcsin\left(\frac{1}{2} \, x\right)}+C}
\]
\end{problem}}%}

\latexProblemContent{
\ifVerboseLocation This is Integration Compute Question 0037. \\ \fi
\begin{problem}

Compute the following integral:

\input{Integral-Compute-0037.HELP.tex}

\[
\int{{\frac{8}{\sqrt{x^{2} + 64}}}\;dx} = \answer{{8 \, {\rm arcsinh}\left(\frac{1}{8} \, x\right)}+C}
\]
\end{problem}}%}

\latexProblemContent{
\ifVerboseLocation This is Integration Compute Question 0037. \\ \fi
\begin{problem}

Compute the following integral:

\input{Integral-Compute-0037.HELP.tex}

\[
\int{{-\frac{7}{\sqrt{x^{2} + 4}}}\;dx} = \answer{{-7 \, {\rm arcsinh}\left(\frac{1}{2} \, x\right)}+C}
\]
\end{problem}}%}

\latexProblemContent{
\ifVerboseLocation This is Integration Compute Question 0037. \\ \fi
\begin{problem}

Compute the following integral:

\input{Integral-Compute-0037.HELP.tex}

\[
\int{{\frac{4}{\sqrt{-x^{2} + 81}}}\;dx} = \answer{{4 \, \arcsin\left(\frac{1}{9} \, x\right)}+C}
\]
\end{problem}}%}

\latexProblemContent{
\ifVerboseLocation This is Integration Compute Question 0037. \\ \fi
\begin{problem}

Compute the following integral:

\input{Integral-Compute-0037.HELP.tex}

\[
\int{{\frac{6}{\sqrt{-x^{2} + 16}}}\;dx} = \answer{{6 \, \arcsin\left(\frac{1}{4} \, x\right)}+C}
\]
\end{problem}}%}

\latexProblemContent{
\ifVerboseLocation This is Integration Compute Question 0037. \\ \fi
\begin{problem}

Compute the following integral:

\input{Integral-Compute-0037.HELP.tex}

\[
\int{{-\frac{4}{\sqrt{-x^{2} + 64}}}\;dx} = \answer{{-4 \, \arcsin\left(\frac{1}{8} \, x\right)}+C}
\]
\end{problem}}%}

\latexProblemContent{
\ifVerboseLocation This is Integration Compute Question 0037. \\ \fi
\begin{problem}

Compute the following integral:

\input{Integral-Compute-0037.HELP.tex}

\[
\int{{-\frac{8}{\sqrt{x^{2} - 4}}}\;dx} = \answer{{-8 \, \log\left(2 \, x + 2 \, \sqrt{x^{2} - 4}\right)}+C}
\]
\end{problem}}%}

\latexProblemContent{
\ifVerboseLocation This is Integration Compute Question 0037. \\ \fi
\begin{problem}

Compute the following integral:

\input{Integral-Compute-0037.HELP.tex}

\[
\int{{-\frac{5}{\sqrt{-x^{2} + 25}}}\;dx} = \answer{{-5 \, \arcsin\left(\frac{1}{5} \, x\right)}+C}
\]
\end{problem}}%}

\latexProblemContent{
\ifVerboseLocation This is Integration Compute Question 0037. \\ \fi
\begin{problem}

Compute the following integral:

\input{Integral-Compute-0037.HELP.tex}

\[
\int{{\frac{3}{\sqrt{x^{2} + 1}}}\;dx} = \answer{{3 \, {\rm arcsinh}\left(x\right)}+C}
\]
\end{problem}}%}

\latexProblemContent{
\ifVerboseLocation This is Integration Compute Question 0037. \\ \fi
\begin{problem}

Compute the following integral:

\input{Integral-Compute-0037.HELP.tex}

\[
\int{{\frac{6}{\sqrt{x^{2} + 9}}}\;dx} = \answer{{6 \, {\rm arcsinh}\left(\frac{1}{3} \, x\right)}+C}
\]
\end{problem}}%}

\latexProblemContent{
\ifVerboseLocation This is Integration Compute Question 0037. \\ \fi
\begin{problem}

Compute the following integral:

\input{Integral-Compute-0037.HELP.tex}

\[
\int{{\frac{1}{\sqrt{x^{2} - 4}}}\;dx} = \answer{{\log\left(2 \, x + 2 \, \sqrt{x^{2} - 4}\right)}+C}
\]
\end{problem}}%}

\latexProblemContent{
\ifVerboseLocation This is Integration Compute Question 0037. \\ \fi
\begin{problem}

Compute the following integral:

\input{Integral-Compute-0037.HELP.tex}

\[
\int{{\frac{2}{\sqrt{x^{2} - 64}}}\;dx} = \answer{{2 \, \log\left(2 \, x + 2 \, \sqrt{x^{2} - 64}\right)}+C}
\]
\end{problem}}%}

\latexProblemContent{
\ifVerboseLocation This is Integration Compute Question 0037. \\ \fi
\begin{problem}

Compute the following integral:

\input{Integral-Compute-0037.HELP.tex}

\[
\int{{\frac{6}{\sqrt{x^{2} + 4}}}\;dx} = \answer{{6 \, {\rm arcsinh}\left(\frac{1}{2} \, x\right)}+C}
\]
\end{problem}}%}

\latexProblemContent{
\ifVerboseLocation This is Integration Compute Question 0037. \\ \fi
\begin{problem}

Compute the following integral:

\input{Integral-Compute-0037.HELP.tex}

\[
\int{{\frac{8}{\sqrt{-x^{2} + 49}}}\;dx} = \answer{{8 \, \arcsin\left(\frac{1}{7} \, x\right)}+C}
\]
\end{problem}}%}

\latexProblemContent{
\ifVerboseLocation This is Integration Compute Question 0037. \\ \fi
\begin{problem}

Compute the following integral:

\input{Integral-Compute-0037.HELP.tex}

\[
\int{{-\frac{4}{\sqrt{x^{2} - 9}}}\;dx} = \answer{{-4 \, \log\left(2 \, x + 2 \, \sqrt{x^{2} - 9}\right)}+C}
\]
\end{problem}}%}

\latexProblemContent{
\ifVerboseLocation This is Integration Compute Question 0037. \\ \fi
\begin{problem}

Compute the following integral:

\input{Integral-Compute-0037.HELP.tex}

\[
\int{{-\frac{5}{\sqrt{-x^{2} + 49}}}\;dx} = \answer{{-5 \, \arcsin\left(\frac{1}{7} \, x\right)}+C}
\]
\end{problem}}%}

\latexProblemContent{
\ifVerboseLocation This is Integration Compute Question 0037. \\ \fi
\begin{problem}

Compute the following integral:

\input{Integral-Compute-0037.HELP.tex}

\[
\int{{-\frac{8}{\sqrt{-x^{2} + 64}}}\;dx} = \answer{{-8 \, \arcsin\left(\frac{1}{8} \, x\right)}+C}
\]
\end{problem}}%}

\latexProblemContent{
\ifVerboseLocation This is Integration Compute Question 0037. \\ \fi
\begin{problem}

Compute the following integral:

\input{Integral-Compute-0037.HELP.tex}

\[
\int{{-\frac{6}{\sqrt{-x^{2} + 49}}}\;dx} = \answer{{-6 \, \arcsin\left(\frac{1}{7} \, x\right)}+C}
\]
\end{problem}}%}

\latexProblemContent{
\ifVerboseLocation This is Integration Compute Question 0037. \\ \fi
\begin{problem}

Compute the following integral:

\input{Integral-Compute-0037.HELP.tex}

\[
\int{{\frac{8}{\sqrt{x^{2} - 25}}}\;dx} = \answer{{8 \, \log\left(2 \, x + 2 \, \sqrt{x^{2} - 25}\right)}+C}
\]
\end{problem}}%}

\latexProblemContent{
\ifVerboseLocation This is Integration Compute Question 0037. \\ \fi
\begin{problem}

Compute the following integral:

\input{Integral-Compute-0037.HELP.tex}

\[
\int{{-\frac{7}{\sqrt{x^{2} - 16}}}\;dx} = \answer{{-7 \, \log\left(2 \, x + 2 \, \sqrt{x^{2} - 16}\right)}+C}
\]
\end{problem}}%}

\latexProblemContent{
\ifVerboseLocation This is Integration Compute Question 0037. \\ \fi
\begin{problem}

Compute the following integral:

\input{Integral-Compute-0037.HELP.tex}

\[
\int{{\frac{3}{\sqrt{-x^{2} + 25}}}\;dx} = \answer{{3 \, \arcsin\left(\frac{1}{5} \, x\right)}+C}
\]
\end{problem}}%}

\latexProblemContent{
\ifVerboseLocation This is Integration Compute Question 0037. \\ \fi
\begin{problem}

Compute the following integral:

\input{Integral-Compute-0037.HELP.tex}

\[
\int{{\frac{3}{\sqrt{-x^{2} + 4}}}\;dx} = \answer{{3 \, \arcsin\left(\frac{1}{2} \, x\right)}+C}
\]
\end{problem}}%}

\latexProblemContent{
\ifVerboseLocation This is Integration Compute Question 0037. \\ \fi
\begin{problem}

Compute the following integral:

\input{Integral-Compute-0037.HELP.tex}

\[
\int{{\frac{1}{\sqrt{x^{2} - 9}}}\;dx} = \answer{{\log\left(2 \, x + 2 \, \sqrt{x^{2} - 9}\right)}+C}
\]
\end{problem}}%}

\latexProblemContent{
\ifVerboseLocation This is Integration Compute Question 0037. \\ \fi
\begin{problem}

Compute the following integral:

\input{Integral-Compute-0037.HELP.tex}

\[
\int{{-\frac{3}{\sqrt{x^{2} + 64}}}\;dx} = \answer{{-3 \, {\rm arcsinh}\left(\frac{1}{8} \, x\right)}+C}
\]
\end{problem}}%}

\latexProblemContent{
\ifVerboseLocation This is Integration Compute Question 0037. \\ \fi
\begin{problem}

Compute the following integral:

\input{Integral-Compute-0037.HELP.tex}

\[
\int{{-\frac{2}{\sqrt{x^{2} + 36}}}\;dx} = \answer{{-2 \, {\rm arcsinh}\left(\frac{1}{6} \, x\right)}+C}
\]
\end{problem}}%}

\latexProblemContent{
\ifVerboseLocation This is Integration Compute Question 0037. \\ \fi
\begin{problem}

Compute the following integral:

\input{Integral-Compute-0037.HELP.tex}

\[
\int{{\frac{1}{\sqrt{-x^{2} + 25}}}\;dx} = \answer{{\arcsin\left(\frac{1}{5} \, x\right)}+C}
\]
\end{problem}}%}

\latexProblemContent{
\ifVerboseLocation This is Integration Compute Question 0037. \\ \fi
\begin{problem}

Compute the following integral:

\input{Integral-Compute-0037.HELP.tex}

\[
\int{{-\frac{9}{\sqrt{x^{2} + 9}}}\;dx} = \answer{{-9 \, {\rm arcsinh}\left(\frac{1}{3} \, x\right)}+C}
\]
\end{problem}}%}

\latexProblemContent{
\ifVerboseLocation This is Integration Compute Question 0037. \\ \fi
\begin{problem}

Compute the following integral:

\input{Integral-Compute-0037.HELP.tex}

\[
\int{{-\frac{6}{\sqrt{-x^{2} + 81}}}\;dx} = \answer{{-6 \, \arcsin\left(\frac{1}{9} \, x\right)}+C}
\]
\end{problem}}%}

\latexProblemContent{
\ifVerboseLocation This is Integration Compute Question 0037. \\ \fi
\begin{problem}

Compute the following integral:

\input{Integral-Compute-0037.HELP.tex}

\[
\int{{\frac{6}{\sqrt{-x^{2} + 4}}}\;dx} = \answer{{6 \, \arcsin\left(\frac{1}{2} \, x\right)}+C}
\]
\end{problem}}%}

\latexProblemContent{
\ifVerboseLocation This is Integration Compute Question 0037. \\ \fi
\begin{problem}

Compute the following integral:

\input{Integral-Compute-0037.HELP.tex}

\[
\int{{\frac{6}{\sqrt{x^{2} + 64}}}\;dx} = \answer{{6 \, {\rm arcsinh}\left(\frac{1}{8} \, x\right)}+C}
\]
\end{problem}}%}

\latexProblemContent{
\ifVerboseLocation This is Integration Compute Question 0037. \\ \fi
\begin{problem}

Compute the following integral:

\input{Integral-Compute-0037.HELP.tex}

\[
\int{{\frac{2}{\sqrt{x^{2} + 36}}}\;dx} = \answer{{2 \, {\rm arcsinh}\left(\frac{1}{6} \, x\right)}+C}
\]
\end{problem}}%}

\latexProblemContent{
\ifVerboseLocation This is Integration Compute Question 0037. \\ \fi
\begin{problem}

Compute the following integral:

\input{Integral-Compute-0037.HELP.tex}

\[
\int{{-\frac{6}{\sqrt{x^{2} + 4}}}\;dx} = \answer{{-6 \, {\rm arcsinh}\left(\frac{1}{2} \, x\right)}+C}
\]
\end{problem}}%}

\latexProblemContent{
\ifVerboseLocation This is Integration Compute Question 0037. \\ \fi
\begin{problem}

Compute the following integral:

\input{Integral-Compute-0037.HELP.tex}

\[
\int{{\frac{2}{\sqrt{-x^{2} + 1}}}\;dx} = \answer{{2 \, \arcsin\left(x\right)}+C}
\]
\end{problem}}%}

\latexProblemContent{
\ifVerboseLocation This is Integration Compute Question 0037. \\ \fi
\begin{problem}

Compute the following integral:

\input{Integral-Compute-0037.HELP.tex}

\[
\int{{-\frac{1}{\sqrt{-x^{2} + 4}}}\;dx} = \answer{{-\arcsin\left(\frac{1}{2} \, x\right)}+C}
\]
\end{problem}}%}

\latexProblemContent{
\ifVerboseLocation This is Integration Compute Question 0037. \\ \fi
\begin{problem}

Compute the following integral:

\input{Integral-Compute-0037.HELP.tex}

\[
\int{{\frac{9}{\sqrt{x^{2} + 49}}}\;dx} = \answer{{9 \, {\rm arcsinh}\left(\frac{1}{7} \, x\right)}+C}
\]
\end{problem}}%}

\latexProblemContent{
\ifVerboseLocation This is Integration Compute Question 0037. \\ \fi
\begin{problem}

Compute the following integral:

\input{Integral-Compute-0037.HELP.tex}

\[
\int{{-\frac{4}{\sqrt{x^{2} - 81}}}\;dx} = \answer{{-4 \, \log\left(2 \, x + 2 \, \sqrt{x^{2} - 81}\right)}+C}
\]
\end{problem}}%}

\latexProblemContent{
\ifVerboseLocation This is Integration Compute Question 0037. \\ \fi
\begin{problem}

Compute the following integral:

\input{Integral-Compute-0037.HELP.tex}

\[
\int{{-\frac{1}{\sqrt{-x^{2} + 81}}}\;dx} = \answer{{-\arcsin\left(\frac{1}{9} \, x\right)}+C}
\]
\end{problem}}%}

\latexProblemContent{
\ifVerboseLocation This is Integration Compute Question 0037. \\ \fi
\begin{problem}

Compute the following integral:

\input{Integral-Compute-0037.HELP.tex}

\[
\int{{\frac{1}{\sqrt{x^{2} + 49}}}\;dx} = \answer{{{\rm arcsinh}\left(\frac{1}{7} \, x\right)}+C}
\]
\end{problem}}%}

\latexProblemContent{
\ifVerboseLocation This is Integration Compute Question 0037. \\ \fi
\begin{problem}

Compute the following integral:

\input{Integral-Compute-0037.HELP.tex}

\[
\int{{-\frac{9}{\sqrt{x^{2} + 16}}}\;dx} = \answer{{-9 \, {\rm arcsinh}\left(\frac{1}{4} \, x\right)}+C}
\]
\end{problem}}%}

\latexProblemContent{
\ifVerboseLocation This is Integration Compute Question 0037. \\ \fi
\begin{problem}

Compute the following integral:

\input{Integral-Compute-0037.HELP.tex}

\[
\int{{-\frac{3}{\sqrt{x^{2} - 4}}}\;dx} = \answer{{-3 \, \log\left(2 \, x + 2 \, \sqrt{x^{2} - 4}\right)}+C}
\]
\end{problem}}%}

\latexProblemContent{
\ifVerboseLocation This is Integration Compute Question 0037. \\ \fi
\begin{problem}

Compute the following integral:

\input{Integral-Compute-0037.HELP.tex}

\[
\int{{-\frac{3}{\sqrt{x^{2} - 9}}}\;dx} = \answer{{-3 \, \log\left(2 \, x + 2 \, \sqrt{x^{2} - 9}\right)}+C}
\]
\end{problem}}%}

\latexProblemContent{
\ifVerboseLocation This is Integration Compute Question 0037. \\ \fi
\begin{problem}

Compute the following integral:

\input{Integral-Compute-0037.HELP.tex}

\[
\int{{-\frac{5}{\sqrt{x^{2} - 9}}}\;dx} = \answer{{-5 \, \log\left(2 \, x + 2 \, \sqrt{x^{2} - 9}\right)}+C}
\]
\end{problem}}%}

\latexProblemContent{
\ifVerboseLocation This is Integration Compute Question 0037. \\ \fi
\begin{problem}

Compute the following integral:

\input{Integral-Compute-0037.HELP.tex}

\[
\int{{\frac{6}{\sqrt{-x^{2} + 36}}}\;dx} = \answer{{6 \, \arcsin\left(\frac{1}{6} \, x\right)}+C}
\]
\end{problem}}%}

\latexProblemContent{
\ifVerboseLocation This is Integration Compute Question 0037. \\ \fi
\begin{problem}

Compute the following integral:

\input{Integral-Compute-0037.HELP.tex}

\[
\int{{-\frac{2}{\sqrt{x^{2} + 1}}}\;dx} = \answer{{-2 \, {\rm arcsinh}\left(x\right)}+C}
\]
\end{problem}}%}

\latexProblemContent{
\ifVerboseLocation This is Integration Compute Question 0037. \\ \fi
\begin{problem}

Compute the following integral:

\input{Integral-Compute-0037.HELP.tex}

\[
\int{{-\frac{3}{\sqrt{-x^{2} + 36}}}\;dx} = \answer{{-3 \, \arcsin\left(\frac{1}{6} \, x\right)}+C}
\]
\end{problem}}%}

\latexProblemContent{
\ifVerboseLocation This is Integration Compute Question 0037. \\ \fi
\begin{problem}

Compute the following integral:

\input{Integral-Compute-0037.HELP.tex}

\[
\int{{-\frac{3}{\sqrt{x^{2} - 25}}}\;dx} = \answer{{-3 \, \log\left(2 \, x + 2 \, \sqrt{x^{2} - 25}\right)}+C}
\]
\end{problem}}%}

\latexProblemContent{
\ifVerboseLocation This is Integration Compute Question 0037. \\ \fi
\begin{problem}

Compute the following integral:

\input{Integral-Compute-0037.HELP.tex}

\[
\int{{-\frac{2}{\sqrt{x^{2} - 64}}}\;dx} = \answer{{-2 \, \log\left(2 \, x + 2 \, \sqrt{x^{2} - 64}\right)}+C}
\]
\end{problem}}%}

\latexProblemContent{
\ifVerboseLocation This is Integration Compute Question 0037. \\ \fi
\begin{problem}

Compute the following integral:

\input{Integral-Compute-0037.HELP.tex}

\[
\int{{-\frac{4}{\sqrt{-x^{2} + 36}}}\;dx} = \answer{{-4 \, \arcsin\left(\frac{1}{6} \, x\right)}+C}
\]
\end{problem}}%}

\latexProblemContent{
\ifVerboseLocation This is Integration Compute Question 0037. \\ \fi
\begin{problem}

Compute the following integral:

\input{Integral-Compute-0037.HELP.tex}

\[
\int{{\frac{6}{\sqrt{-x^{2} + 49}}}\;dx} = \answer{{6 \, \arcsin\left(\frac{1}{7} \, x\right)}+C}
\]
\end{problem}}%}

\latexProblemContent{
\ifVerboseLocation This is Integration Compute Question 0037. \\ \fi
\begin{problem}

Compute the following integral:

\input{Integral-Compute-0037.HELP.tex}

\[
\int{{\frac{6}{\sqrt{x^{2} + 36}}}\;dx} = \answer{{6 \, {\rm arcsinh}\left(\frac{1}{6} \, x\right)}+C}
\]
\end{problem}}%}

\latexProblemContent{
\ifVerboseLocation This is Integration Compute Question 0037. \\ \fi
\begin{problem}

Compute the following integral:

\input{Integral-Compute-0037.HELP.tex}

\[
\int{{-\frac{3}{\sqrt{x^{2} + 1}}}\;dx} = \answer{{-3 \, {\rm arcsinh}\left(x\right)}+C}
\]
\end{problem}}%}

\latexProblemContent{
\ifVerboseLocation This is Integration Compute Question 0037. \\ \fi
\begin{problem}

Compute the following integral:

\input{Integral-Compute-0037.HELP.tex}

\[
\int{{\frac{2}{\sqrt{x^{2} + 49}}}\;dx} = \answer{{2 \, {\rm arcsinh}\left(\frac{1}{7} \, x\right)}+C}
\]
\end{problem}}%}

\latexProblemContent{
\ifVerboseLocation This is Integration Compute Question 0037. \\ \fi
\begin{problem}

Compute the following integral:

\input{Integral-Compute-0037.HELP.tex}

\[
\int{{\frac{6}{\sqrt{x^{2} + 1}}}\;dx} = \answer{{6 \, {\rm arcsinh}\left(x\right)}+C}
\]
\end{problem}}%}

\latexProblemContent{
\ifVerboseLocation This is Integration Compute Question 0037. \\ \fi
\begin{problem}

Compute the following integral:

\input{Integral-Compute-0037.HELP.tex}

\[
\int{{-\frac{6}{\sqrt{x^{2} + 81}}}\;dx} = \answer{{-6 \, {\rm arcsinh}\left(\frac{1}{9} \, x\right)}+C}
\]
\end{problem}}%}

\latexProblemContent{
\ifVerboseLocation This is Integration Compute Question 0037. \\ \fi
\begin{problem}

Compute the following integral:

\input{Integral-Compute-0037.HELP.tex}

\[
\int{{\frac{9}{\sqrt{x^{2} + 1}}}\;dx} = \answer{{9 \, {\rm arcsinh}\left(x\right)}+C}
\]
\end{problem}}%}

\latexProblemContent{
\ifVerboseLocation This is Integration Compute Question 0037. \\ \fi
\begin{problem}

Compute the following integral:

\input{Integral-Compute-0037.HELP.tex}

\[
\int{{-\frac{2}{\sqrt{x^{2} - 1}}}\;dx} = \answer{{-2 \, \log\left(2 \, x + 2 \, \sqrt{x^{2} - 1}\right)}+C}
\]
\end{problem}}%}

\latexProblemContent{
\ifVerboseLocation This is Integration Compute Question 0037. \\ \fi
\begin{problem}

Compute the following integral:

\input{Integral-Compute-0037.HELP.tex}

\[
\int{{-\frac{9}{\sqrt{-x^{2} + 4}}}\;dx} = \answer{{-9 \, \arcsin\left(\frac{1}{2} \, x\right)}+C}
\]
\end{problem}}%}

\latexProblemContent{
\ifVerboseLocation This is Integration Compute Question 0037. \\ \fi
\begin{problem}

Compute the following integral:

\input{Integral-Compute-0037.HELP.tex}

\[
\int{{-\frac{9}{\sqrt{-x^{2} + 49}}}\;dx} = \answer{{-9 \, \arcsin\left(\frac{1}{7} \, x\right)}+C}
\]
\end{problem}}%}

\latexProblemContent{
\ifVerboseLocation This is Integration Compute Question 0037. \\ \fi
\begin{problem}

Compute the following integral:

\input{Integral-Compute-0037.HELP.tex}

\[
\int{{-\frac{3}{\sqrt{x^{2} + 81}}}\;dx} = \answer{{-3 \, {\rm arcsinh}\left(\frac{1}{9} \, x\right)}+C}
\]
\end{problem}}%}

\latexProblemContent{
\ifVerboseLocation This is Integration Compute Question 0037. \\ \fi
\begin{problem}

Compute the following integral:

\input{Integral-Compute-0037.HELP.tex}

\[
\int{{\frac{9}{\sqrt{x^{2} + 25}}}\;dx} = \answer{{9 \, {\rm arcsinh}\left(\frac{1}{5} \, x\right)}+C}
\]
\end{problem}}%}

\latexProblemContent{
\ifVerboseLocation This is Integration Compute Question 0037. \\ \fi
\begin{problem}

Compute the following integral:

\input{Integral-Compute-0037.HELP.tex}

\[
\int{{\frac{5}{\sqrt{x^{2} + 16}}}\;dx} = \answer{{5 \, {\rm arcsinh}\left(\frac{1}{4} \, x\right)}+C}
\]
\end{problem}}%}

\latexProblemContent{
\ifVerboseLocation This is Integration Compute Question 0037. \\ \fi
\begin{problem}

Compute the following integral:

\input{Integral-Compute-0037.HELP.tex}

\[
\int{{-\frac{8}{\sqrt{x^{2} - 36}}}\;dx} = \answer{{-8 \, \log\left(2 \, x + 2 \, \sqrt{x^{2} - 36}\right)}+C}
\]
\end{problem}}%}

\latexProblemContent{
\ifVerboseLocation This is Integration Compute Question 0037. \\ \fi
\begin{problem}

Compute the following integral:

\input{Integral-Compute-0037.HELP.tex}

\[
\int{{\frac{8}{\sqrt{x^{2} - 9}}}\;dx} = \answer{{8 \, \log\left(2 \, x + 2 \, \sqrt{x^{2} - 9}\right)}+C}
\]
\end{problem}}%}

\latexProblemContent{
\ifVerboseLocation This is Integration Compute Question 0037. \\ \fi
\begin{problem}

Compute the following integral:

\input{Integral-Compute-0037.HELP.tex}

\[
\int{{-\frac{4}{\sqrt{x^{2} - 49}}}\;dx} = \answer{{-4 \, \log\left(2 \, x + 2 \, \sqrt{x^{2} - 49}\right)}+C}
\]
\end{problem}}%}

\latexProblemContent{
\ifVerboseLocation This is Integration Compute Question 0037. \\ \fi
\begin{problem}

Compute the following integral:

\input{Integral-Compute-0037.HELP.tex}

\[
\int{{\frac{6}{\sqrt{x^{2} + 49}}}\;dx} = \answer{{6 \, {\rm arcsinh}\left(\frac{1}{7} \, x\right)}+C}
\]
\end{problem}}%}

\latexProblemContent{
\ifVerboseLocation This is Integration Compute Question 0037. \\ \fi
\begin{problem}

Compute the following integral:

\input{Integral-Compute-0037.HELP.tex}

\[
\int{{-\frac{9}{\sqrt{-x^{2} + 25}}}\;dx} = \answer{{-9 \, \arcsin\left(\frac{1}{5} \, x\right)}+C}
\]
\end{problem}}%}

\latexProblemContent{
\ifVerboseLocation This is Integration Compute Question 0037. \\ \fi
\begin{problem}

Compute the following integral:

\input{Integral-Compute-0037.HELP.tex}

\[
\int{{\frac{4}{\sqrt{x^{2} - 81}}}\;dx} = \answer{{4 \, \log\left(2 \, x + 2 \, \sqrt{x^{2} - 81}\right)}+C}
\]
\end{problem}}%}

\latexProblemContent{
\ifVerboseLocation This is Integration Compute Question 0037. \\ \fi
\begin{problem}

Compute the following integral:

\input{Integral-Compute-0037.HELP.tex}

\[
\int{{-\frac{4}{\sqrt{x^{2} + 16}}}\;dx} = \answer{{-4 \, {\rm arcsinh}\left(\frac{1}{4} \, x\right)}+C}
\]
\end{problem}}%}

\latexProblemContent{
\ifVerboseLocation This is Integration Compute Question 0037. \\ \fi
\begin{problem}

Compute the following integral:

\input{Integral-Compute-0037.HELP.tex}

\[
\int{{\frac{8}{\sqrt{x^{2} - 1}}}\;dx} = \answer{{8 \, \log\left(2 \, x + 2 \, \sqrt{x^{2} - 1}\right)}+C}
\]
\end{problem}}%}

\latexProblemContent{
\ifVerboseLocation This is Integration Compute Question 0037. \\ \fi
\begin{problem}

Compute the following integral:

\input{Integral-Compute-0037.HELP.tex}

\[
\int{{-\frac{7}{\sqrt{-x^{2} + 64}}}\;dx} = \answer{{-7 \, \arcsin\left(\frac{1}{8} \, x\right)}+C}
\]
\end{problem}}%}

\latexProblemContent{
\ifVerboseLocation This is Integration Compute Question 0037. \\ \fi
\begin{problem}

Compute the following integral:

\input{Integral-Compute-0037.HELP.tex}

\[
\int{{\frac{4}{\sqrt{-x^{2} + 9}}}\;dx} = \answer{{4 \, \arcsin\left(\frac{1}{3} \, x\right)}+C}
\]
\end{problem}}%}

\latexProblemContent{
\ifVerboseLocation This is Integration Compute Question 0037. \\ \fi
\begin{problem}

Compute the following integral:

\input{Integral-Compute-0037.HELP.tex}

\[
\int{{\frac{9}{\sqrt{x^{2} - 64}}}\;dx} = \answer{{9 \, \log\left(2 \, x + 2 \, \sqrt{x^{2} - 64}\right)}+C}
\]
\end{problem}}%}

\latexProblemContent{
\ifVerboseLocation This is Integration Compute Question 0037. \\ \fi
\begin{problem}

Compute the following integral:

\input{Integral-Compute-0037.HELP.tex}

\[
\int{{\frac{3}{\sqrt{x^{2} - 16}}}\;dx} = \answer{{3 \, \log\left(2 \, x + 2 \, \sqrt{x^{2} - 16}\right)}+C}
\]
\end{problem}}%}

\latexProblemContent{
\ifVerboseLocation This is Integration Compute Question 0037. \\ \fi
\begin{problem}

Compute the following integral:

\input{Integral-Compute-0037.HELP.tex}

\[
\int{{-\frac{6}{\sqrt{x^{2} - 9}}}\;dx} = \answer{{-6 \, \log\left(2 \, x + 2 \, \sqrt{x^{2} - 9}\right)}+C}
\]
\end{problem}}%}

\latexProblemContent{
\ifVerboseLocation This is Integration Compute Question 0037. \\ \fi
\begin{problem}

Compute the following integral:

\input{Integral-Compute-0037.HELP.tex}

\[
\int{{-\frac{8}{\sqrt{x^{2} - 25}}}\;dx} = \answer{{-8 \, \log\left(2 \, x + 2 \, \sqrt{x^{2} - 25}\right)}+C}
\]
\end{problem}}%}

\latexProblemContent{
\ifVerboseLocation This is Integration Compute Question 0037. \\ \fi
\begin{problem}

Compute the following integral:

\input{Integral-Compute-0037.HELP.tex}

\[
\int{{\frac{2}{\sqrt{x^{2} - 36}}}\;dx} = \answer{{2 \, \log\left(2 \, x + 2 \, \sqrt{x^{2} - 36}\right)}+C}
\]
\end{problem}}%}

\latexProblemContent{
\ifVerboseLocation This is Integration Compute Question 0037. \\ \fi
\begin{problem}

Compute the following integral:

\input{Integral-Compute-0037.HELP.tex}

\[
\int{{\frac{8}{\sqrt{x^{2} + 36}}}\;dx} = \answer{{8 \, {\rm arcsinh}\left(\frac{1}{6} \, x\right)}+C}
\]
\end{problem}}%}

\latexProblemContent{
\ifVerboseLocation This is Integration Compute Question 0037. \\ \fi
\begin{problem}

Compute the following integral:

\input{Integral-Compute-0037.HELP.tex}

\[
\int{{-\frac{6}{\sqrt{-x^{2} + 36}}}\;dx} = \answer{{-6 \, \arcsin\left(\frac{1}{6} \, x\right)}+C}
\]
\end{problem}}%}

\latexProblemContent{
\ifVerboseLocation This is Integration Compute Question 0037. \\ \fi
\begin{problem}

Compute the following integral:

\input{Integral-Compute-0037.HELP.tex}

\[
\int{{-\frac{6}{\sqrt{x^{2} - 16}}}\;dx} = \answer{{-6 \, \log\left(2 \, x + 2 \, \sqrt{x^{2} - 16}\right)}+C}
\]
\end{problem}}%}

\latexProblemContent{
\ifVerboseLocation This is Integration Compute Question 0037. \\ \fi
\begin{problem}

Compute the following integral:

\input{Integral-Compute-0037.HELP.tex}

\[
\int{{-\frac{7}{\sqrt{x^{2} + 81}}}\;dx} = \answer{{-7 \, {\rm arcsinh}\left(\frac{1}{9} \, x\right)}+C}
\]
\end{problem}}%}

\latexProblemContent{
\ifVerboseLocation This is Integration Compute Question 0037. \\ \fi
\begin{problem}

Compute the following integral:

\input{Integral-Compute-0037.HELP.tex}

\[
\int{{\frac{4}{\sqrt{x^{2} + 25}}}\;dx} = \answer{{4 \, {\rm arcsinh}\left(\frac{1}{5} \, x\right)}+C}
\]
\end{problem}}%}

\latexProblemContent{
\ifVerboseLocation This is Integration Compute Question 0037. \\ \fi
\begin{problem}

Compute the following integral:

\input{Integral-Compute-0037.HELP.tex}

\[
\int{{\frac{7}{\sqrt{x^{2} + 4}}}\;dx} = \answer{{7 \, {\rm arcsinh}\left(\frac{1}{2} \, x\right)}+C}
\]
\end{problem}}%}

\latexProblemContent{
\ifVerboseLocation This is Integration Compute Question 0037. \\ \fi
\begin{problem}

Compute the following integral:

\input{Integral-Compute-0037.HELP.tex}

\[
\int{{-\frac{9}{\sqrt{-x^{2} + 64}}}\;dx} = \answer{{-9 \, \arcsin\left(\frac{1}{8} \, x\right)}+C}
\]
\end{problem}}%}

\latexProblemContent{
\ifVerboseLocation This is Integration Compute Question 0037. \\ \fi
\begin{problem}

Compute the following integral:

\input{Integral-Compute-0037.HELP.tex}

\[
\int{{-\frac{4}{\sqrt{x^{2} - 4}}}\;dx} = \answer{{-4 \, \log\left(2 \, x + 2 \, \sqrt{x^{2} - 4}\right)}+C}
\]
\end{problem}}%}

\latexProblemContent{
\ifVerboseLocation This is Integration Compute Question 0037. \\ \fi
\begin{problem}

Compute the following integral:

\input{Integral-Compute-0037.HELP.tex}

\[
\int{{\frac{8}{\sqrt{x^{2} - 49}}}\;dx} = \answer{{8 \, \log\left(2 \, x + 2 \, \sqrt{x^{2} - 49}\right)}+C}
\]
\end{problem}}%}

\latexProblemContent{
\ifVerboseLocation This is Integration Compute Question 0037. \\ \fi
\begin{problem}

Compute the following integral:

\input{Integral-Compute-0037.HELP.tex}

\[
\int{{-\frac{5}{\sqrt{x^{2} + 64}}}\;dx} = \answer{{-5 \, {\rm arcsinh}\left(\frac{1}{8} \, x\right)}+C}
\]
\end{problem}}%}

\latexProblemContent{
\ifVerboseLocation This is Integration Compute Question 0037. \\ \fi
\begin{problem}

Compute the following integral:

\input{Integral-Compute-0037.HELP.tex}

\[
\int{{\frac{5}{\sqrt{x^{2} + 25}}}\;dx} = \answer{{5 \, {\rm arcsinh}\left(\frac{1}{5} \, x\right)}+C}
\]
\end{problem}}%}

\latexProblemContent{
\ifVerboseLocation This is Integration Compute Question 0037. \\ \fi
\begin{problem}

Compute the following integral:

\input{Integral-Compute-0037.HELP.tex}

\[
\int{{\frac{7}{\sqrt{-x^{2} + 4}}}\;dx} = \answer{{7 \, \arcsin\left(\frac{1}{2} \, x\right)}+C}
\]
\end{problem}}%}

\latexProblemContent{
\ifVerboseLocation This is Integration Compute Question 0037. \\ \fi
\begin{problem}

Compute the following integral:

\input{Integral-Compute-0037.HELP.tex}

\[
\int{{-\frac{9}{\sqrt{x^{2} - 1}}}\;dx} = \answer{{-9 \, \log\left(2 \, x + 2 \, \sqrt{x^{2} - 1}\right)}+C}
\]
\end{problem}}%}

\latexProblemContent{
\ifVerboseLocation This is Integration Compute Question 0037. \\ \fi
\begin{problem}

Compute the following integral:

\input{Integral-Compute-0037.HELP.tex}

\[
\int{{\frac{5}{\sqrt{x^{2} - 4}}}\;dx} = \answer{{5 \, \log\left(2 \, x + 2 \, \sqrt{x^{2} - 4}\right)}+C}
\]
\end{problem}}%}

\latexProblemContent{
\ifVerboseLocation This is Integration Compute Question 0037. \\ \fi
\begin{problem}

Compute the following integral:

\input{Integral-Compute-0037.HELP.tex}

\[
\int{{\frac{8}{\sqrt{-x^{2} + 36}}}\;dx} = \answer{{8 \, \arcsin\left(\frac{1}{6} \, x\right)}+C}
\]
\end{problem}}%}

\latexProblemContent{
\ifVerboseLocation This is Integration Compute Question 0037. \\ \fi
\begin{problem}

Compute the following integral:

\input{Integral-Compute-0037.HELP.tex}

\[
\int{{-\frac{6}{\sqrt{x^{2} - 4}}}\;dx} = \answer{{-6 \, \log\left(2 \, x + 2 \, \sqrt{x^{2} - 4}\right)}+C}
\]
\end{problem}}%}

\latexProblemContent{
\ifVerboseLocation This is Integration Compute Question 0037. \\ \fi
\begin{problem}

Compute the following integral:

\input{Integral-Compute-0037.HELP.tex}

\[
\int{{-\frac{7}{\sqrt{-x^{2} + 1}}}\;dx} = \answer{{-7 \, \arcsin\left(x\right)}+C}
\]
\end{problem}}%}

\latexProblemContent{
\ifVerboseLocation This is Integration Compute Question 0037. \\ \fi
\begin{problem}

Compute the following integral:

\input{Integral-Compute-0037.HELP.tex}

\[
\int{{\frac{2}{\sqrt{-x^{2} + 36}}}\;dx} = \answer{{2 \, \arcsin\left(\frac{1}{6} \, x\right)}+C}
\]
\end{problem}}%}

\latexProblemContent{
\ifVerboseLocation This is Integration Compute Question 0037. \\ \fi
\begin{problem}

Compute the following integral:

\input{Integral-Compute-0037.HELP.tex}

\[
\int{{-\frac{3}{\sqrt{x^{2} + 49}}}\;dx} = \answer{{-3 \, {\rm arcsinh}\left(\frac{1}{7} \, x\right)}+C}
\]
\end{problem}}%}

\latexProblemContent{
\ifVerboseLocation This is Integration Compute Question 0037. \\ \fi
\begin{problem}

Compute the following integral:

\input{Integral-Compute-0037.HELP.tex}

\[
\int{{\frac{9}{\sqrt{-x^{2} + 16}}}\;dx} = \answer{{9 \, \arcsin\left(\frac{1}{4} \, x\right)}+C}
\]
\end{problem}}%}

\latexProblemContent{
\ifVerboseLocation This is Integration Compute Question 0037. \\ \fi
\begin{problem}

Compute the following integral:

\input{Integral-Compute-0037.HELP.tex}

\[
\int{{-\frac{5}{\sqrt{x^{2} - 81}}}\;dx} = \answer{{-5 \, \log\left(2 \, x + 2 \, \sqrt{x^{2} - 81}\right)}+C}
\]
\end{problem}}%}

\latexProblemContent{
\ifVerboseLocation This is Integration Compute Question 0037. \\ \fi
\begin{problem}

Compute the following integral:

\input{Integral-Compute-0037.HELP.tex}

\[
\int{{-\frac{6}{\sqrt{-x^{2} + 1}}}\;dx} = \answer{{-6 \, \arcsin\left(x\right)}+C}
\]
\end{problem}}%}

\latexProblemContent{
\ifVerboseLocation This is Integration Compute Question 0037. \\ \fi
\begin{problem}

Compute the following integral:

\input{Integral-Compute-0037.HELP.tex}

\[
\int{{-\frac{8}{\sqrt{x^{2} - 9}}}\;dx} = \answer{{-8 \, \log\left(2 \, x + 2 \, \sqrt{x^{2} - 9}\right)}+C}
\]
\end{problem}}%}

\latexProblemContent{
\ifVerboseLocation This is Integration Compute Question 0037. \\ \fi
\begin{problem}

Compute the following integral:

\input{Integral-Compute-0037.HELP.tex}

\[
\int{{\frac{3}{\sqrt{-x^{2} + 36}}}\;dx} = \answer{{3 \, \arcsin\left(\frac{1}{6} \, x\right)}+C}
\]
\end{problem}}%}

\latexProblemContent{
\ifVerboseLocation This is Integration Compute Question 0037. \\ \fi
\begin{problem}

Compute the following integral:

\input{Integral-Compute-0037.HELP.tex}

\[
\int{{-\frac{6}{\sqrt{x^{2} - 25}}}\;dx} = \answer{{-6 \, \log\left(2 \, x + 2 \, \sqrt{x^{2} - 25}\right)}+C}
\]
\end{problem}}%}

\latexProblemContent{
\ifVerboseLocation This is Integration Compute Question 0037. \\ \fi
\begin{problem}

Compute the following integral:

\input{Integral-Compute-0037.HELP.tex}

\[
\int{{\frac{8}{\sqrt{-x^{2} + 25}}}\;dx} = \answer{{8 \, \arcsin\left(\frac{1}{5} \, x\right)}+C}
\]
\end{problem}}%}

\latexProblemContent{
\ifVerboseLocation This is Integration Compute Question 0037. \\ \fi
\begin{problem}

Compute the following integral:

\input{Integral-Compute-0037.HELP.tex}

\[
\int{{-\frac{3}{\sqrt{-x^{2} + 9}}}\;dx} = \answer{{-3 \, \arcsin\left(\frac{1}{3} \, x\right)}+C}
\]
\end{problem}}%}

\latexProblemContent{
\ifVerboseLocation This is Integration Compute Question 0037. \\ \fi
\begin{problem}

Compute the following integral:

\input{Integral-Compute-0037.HELP.tex}

\[
\int{{\frac{7}{\sqrt{-x^{2} + 81}}}\;dx} = \answer{{7 \, \arcsin\left(\frac{1}{9} \, x\right)}+C}
\]
\end{problem}}%}

\latexProblemContent{
\ifVerboseLocation This is Integration Compute Question 0037. \\ \fi
\begin{problem}

Compute the following integral:

\input{Integral-Compute-0037.HELP.tex}

\[
\int{{\frac{5}{\sqrt{x^{2} + 1}}}\;dx} = \answer{{5 \, {\rm arcsinh}\left(x\right)}+C}
\]
\end{problem}}%}

\latexProblemContent{
\ifVerboseLocation This is Integration Compute Question 0037. \\ \fi
\begin{problem}

Compute the following integral:

\input{Integral-Compute-0037.HELP.tex}

\[
\int{{-\frac{9}{\sqrt{-x^{2} + 1}}}\;dx} = \answer{{-9 \, \arcsin\left(x\right)}+C}
\]
\end{problem}}%}

\latexProblemContent{
\ifVerboseLocation This is Integration Compute Question 0037. \\ \fi
\begin{problem}

Compute the following integral:

\input{Integral-Compute-0037.HELP.tex}

\[
\int{{-\frac{6}{\sqrt{x^{2} + 1}}}\;dx} = \answer{{-6 \, {\rm arcsinh}\left(x\right)}+C}
\]
\end{problem}}%}

\latexProblemContent{
\ifVerboseLocation This is Integration Compute Question 0037. \\ \fi
\begin{problem}

Compute the following integral:

\input{Integral-Compute-0037.HELP.tex}

\[
\int{{-\frac{2}{\sqrt{-x^{2} + 16}}}\;dx} = \answer{{-2 \, \arcsin\left(\frac{1}{4} \, x\right)}+C}
\]
\end{problem}}%}

\latexProblemContent{
\ifVerboseLocation This is Integration Compute Question 0037. \\ \fi
\begin{problem}

Compute the following integral:

\input{Integral-Compute-0037.HELP.tex}

\[
\int{{-\frac{1}{\sqrt{x^{2} - 49}}}\;dx} = \answer{{-\log\left(2 \, x + 2 \, \sqrt{x^{2} - 49}\right)}+C}
\]
\end{problem}}%}

\latexProblemContent{
\ifVerboseLocation This is Integration Compute Question 0037. \\ \fi
\begin{problem}

Compute the following integral:

\input{Integral-Compute-0037.HELP.tex}

\[
\int{{-\frac{1}{\sqrt{-x^{2} + 16}}}\;dx} = \answer{{-\arcsin\left(\frac{1}{4} \, x\right)}+C}
\]
\end{problem}}%}

\latexProblemContent{
\ifVerboseLocation This is Integration Compute Question 0037. \\ \fi
\begin{problem}

Compute the following integral:

\input{Integral-Compute-0037.HELP.tex}

\[
\int{{\frac{3}{\sqrt{-x^{2} + 9}}}\;dx} = \answer{{3 \, \arcsin\left(\frac{1}{3} \, x\right)}+C}
\]
\end{problem}}%}

\latexProblemContent{
\ifVerboseLocation This is Integration Compute Question 0037. \\ \fi
\begin{problem}

Compute the following integral:

\input{Integral-Compute-0037.HELP.tex}

\[
\int{{\frac{5}{\sqrt{x^{2} - 64}}}\;dx} = \answer{{5 \, \log\left(2 \, x + 2 \, \sqrt{x^{2} - 64}\right)}+C}
\]
\end{problem}}%}

\latexProblemContent{
\ifVerboseLocation This is Integration Compute Question 0037. \\ \fi
\begin{problem}

Compute the following integral:

\input{Integral-Compute-0037.HELP.tex}

\[
\int{{-\frac{3}{\sqrt{-x^{2} + 49}}}\;dx} = \answer{{-3 \, \arcsin\left(\frac{1}{7} \, x\right)}+C}
\]
\end{problem}}%}

\latexProblemContent{
\ifVerboseLocation This is Integration Compute Question 0037. \\ \fi
\begin{problem}

Compute the following integral:

\input{Integral-Compute-0037.HELP.tex}

\[
\int{{-\frac{1}{\sqrt{x^{2} + 36}}}\;dx} = \answer{{-{\rm arcsinh}\left(\frac{1}{6} \, x\right)}+C}
\]
\end{problem}}%}

\latexProblemContent{
\ifVerboseLocation This is Integration Compute Question 0037. \\ \fi
\begin{problem}

Compute the following integral:

\input{Integral-Compute-0037.HELP.tex}

\[
\int{{\frac{4}{\sqrt{x^{2} - 64}}}\;dx} = \answer{{4 \, \log\left(2 \, x + 2 \, \sqrt{x^{2} - 64}\right)}+C}
\]
\end{problem}}%}

\latexProblemContent{
\ifVerboseLocation This is Integration Compute Question 0037. \\ \fi
\begin{problem}

Compute the following integral:

\input{Integral-Compute-0037.HELP.tex}

\[
\int{{\frac{9}{\sqrt{x^{2} - 49}}}\;dx} = \answer{{9 \, \log\left(2 \, x + 2 \, \sqrt{x^{2} - 49}\right)}+C}
\]
\end{problem}}%}

\latexProblemContent{
\ifVerboseLocation This is Integration Compute Question 0037. \\ \fi
\begin{problem}

Compute the following integral:

\input{Integral-Compute-0037.HELP.tex}

\[
\int{{-\frac{4}{\sqrt{x^{2} + 1}}}\;dx} = \answer{{-4 \, {\rm arcsinh}\left(x\right)}+C}
\]
\end{problem}}%}

\latexProblemContent{
\ifVerboseLocation This is Integration Compute Question 0037. \\ \fi
\begin{problem}

Compute the following integral:

\input{Integral-Compute-0037.HELP.tex}

\[
\int{{\frac{5}{\sqrt{x^{2} + 4}}}\;dx} = \answer{{5 \, {\rm arcsinh}\left(\frac{1}{2} \, x\right)}+C}
\]
\end{problem}}%}

\latexProblemContent{
\ifVerboseLocation This is Integration Compute Question 0037. \\ \fi
\begin{problem}

Compute the following integral:

\input{Integral-Compute-0037.HELP.tex}

\[
\int{{-\frac{7}{\sqrt{-x^{2} + 25}}}\;dx} = \answer{{-7 \, \arcsin\left(\frac{1}{5} \, x\right)}+C}
\]
\end{problem}}%}

\latexProblemContent{
\ifVerboseLocation This is Integration Compute Question 0037. \\ \fi
\begin{problem}

Compute the following integral:

\input{Integral-Compute-0037.HELP.tex}

\[
\int{{-\frac{9}{\sqrt{x^{2} + 49}}}\;dx} = \answer{{-9 \, {\rm arcsinh}\left(\frac{1}{7} \, x\right)}+C}
\]
\end{problem}}%}

\latexProblemContent{
\ifVerboseLocation This is Integration Compute Question 0037. \\ \fi
\begin{problem}

Compute the following integral:

\input{Integral-Compute-0037.HELP.tex}

\[
\int{{-\frac{5}{\sqrt{x^{2} - 64}}}\;dx} = \answer{{-5 \, \log\left(2 \, x + 2 \, \sqrt{x^{2} - 64}\right)}+C}
\]
\end{problem}}%}

\latexProblemContent{
\ifVerboseLocation This is Integration Compute Question 0037. \\ \fi
\begin{problem}

Compute the following integral:

\input{Integral-Compute-0037.HELP.tex}

\[
\int{{\frac{7}{\sqrt{-x^{2} + 9}}}\;dx} = \answer{{7 \, \arcsin\left(\frac{1}{3} \, x\right)}+C}
\]
\end{problem}}%}

\latexProblemContent{
\ifVerboseLocation This is Integration Compute Question 0037. \\ \fi
\begin{problem}

Compute the following integral:

\input{Integral-Compute-0037.HELP.tex}

\[
\int{{\frac{4}{\sqrt{-x^{2} + 16}}}\;dx} = \answer{{4 \, \arcsin\left(\frac{1}{4} \, x\right)}+C}
\]
\end{problem}}%}

\latexProblemContent{
\ifVerboseLocation This is Integration Compute Question 0037. \\ \fi
\begin{problem}

Compute the following integral:

\input{Integral-Compute-0037.HELP.tex}

\[
\int{{-\frac{4}{\sqrt{x^{2} + 64}}}\;dx} = \answer{{-4 \, {\rm arcsinh}\left(\frac{1}{8} \, x\right)}+C}
\]
\end{problem}}%}

\latexProblemContent{
\ifVerboseLocation This is Integration Compute Question 0037. \\ \fi
\begin{problem}

Compute the following integral:

\input{Integral-Compute-0037.HELP.tex}

\[
\int{{-\frac{2}{\sqrt{x^{2} + 49}}}\;dx} = \answer{{-2 \, {\rm arcsinh}\left(\frac{1}{7} \, x\right)}+C}
\]
\end{problem}}%}

\latexProblemContent{
\ifVerboseLocation This is Integration Compute Question 0037. \\ \fi
\begin{problem}

Compute the following integral:

\input{Integral-Compute-0037.HELP.tex}

\[
\int{{-\frac{1}{\sqrt{x^{2} + 16}}}\;dx} = \answer{{-{\rm arcsinh}\left(\frac{1}{4} \, x\right)}+C}
\]
\end{problem}}%}

\latexProblemContent{
\ifVerboseLocation This is Integration Compute Question 0037. \\ \fi
\begin{problem}

Compute the following integral:

\input{Integral-Compute-0037.HELP.tex}

\[
\int{{-\frac{7}{\sqrt{x^{2} - 81}}}\;dx} = \answer{{-7 \, \log\left(2 \, x + 2 \, \sqrt{x^{2} - 81}\right)}+C}
\]
\end{problem}}%}

\latexProblemContent{
\ifVerboseLocation This is Integration Compute Question 0037. \\ \fi
\begin{problem}

Compute the following integral:

\input{Integral-Compute-0037.HELP.tex}

\[
\int{{\frac{3}{\sqrt{x^{2} + 16}}}\;dx} = \answer{{3 \, {\rm arcsinh}\left(\frac{1}{4} \, x\right)}+C}
\]
\end{problem}}%}

\latexProblemContent{
\ifVerboseLocation This is Integration Compute Question 0037. \\ \fi
\begin{problem}

Compute the following integral:

\input{Integral-Compute-0037.HELP.tex}

\[
\int{{-\frac{7}{\sqrt{x^{2} - 9}}}\;dx} = \answer{{-7 \, \log\left(2 \, x + 2 \, \sqrt{x^{2} - 9}\right)}+C}
\]
\end{problem}}%}

\latexProblemContent{
\ifVerboseLocation This is Integration Compute Question 0037. \\ \fi
\begin{problem}

Compute the following integral:

\input{Integral-Compute-0037.HELP.tex}

\[
\int{{\frac{4}{\sqrt{x^{2} - 16}}}\;dx} = \answer{{4 \, \log\left(2 \, x + 2 \, \sqrt{x^{2} - 16}\right)}+C}
\]
\end{problem}}%}

\latexProblemContent{
\ifVerboseLocation This is Integration Compute Question 0037. \\ \fi
\begin{problem}

Compute the following integral:

\input{Integral-Compute-0037.HELP.tex}

\[
\int{{-\frac{6}{\sqrt{-x^{2} + 9}}}\;dx} = \answer{{-6 \, \arcsin\left(\frac{1}{3} \, x\right)}+C}
\]
\end{problem}}%}

\latexProblemContent{
\ifVerboseLocation This is Integration Compute Question 0037. \\ \fi
\begin{problem}

Compute the following integral:

\input{Integral-Compute-0037.HELP.tex}

\[
\int{{\frac{1}{\sqrt{-x^{2} + 49}}}\;dx} = \answer{{\arcsin\left(\frac{1}{7} \, x\right)}+C}
\]
\end{problem}}%}

\latexProblemContent{
\ifVerboseLocation This is Integration Compute Question 0037. \\ \fi
\begin{problem}

Compute the following integral:

\input{Integral-Compute-0037.HELP.tex}

\[
\int{{-\frac{6}{\sqrt{x^{2} + 9}}}\;dx} = \answer{{-6 \, {\rm arcsinh}\left(\frac{1}{3} \, x\right)}+C}
\]
\end{problem}}%}

\latexProblemContent{
\ifVerboseLocation This is Integration Compute Question 0037. \\ \fi
\begin{problem}

Compute the following integral:

\input{Integral-Compute-0037.HELP.tex}

\[
\int{{\frac{3}{\sqrt{x^{2} - 49}}}\;dx} = \answer{{3 \, \log\left(2 \, x + 2 \, \sqrt{x^{2} - 49}\right)}+C}
\]
\end{problem}}%}

\latexProblemContent{
\ifVerboseLocation This is Integration Compute Question 0037. \\ \fi
\begin{problem}

Compute the following integral:

\input{Integral-Compute-0037.HELP.tex}

\[
\int{{-\frac{4}{\sqrt{x^{2} + 81}}}\;dx} = \answer{{-4 \, {\rm arcsinh}\left(\frac{1}{9} \, x\right)}+C}
\]
\end{problem}}%}

\latexProblemContent{
\ifVerboseLocation This is Integration Compute Question 0037. \\ \fi
\begin{problem}

Compute the following integral:

\input{Integral-Compute-0037.HELP.tex}

\[
\int{{-\frac{6}{\sqrt{x^{2} - 49}}}\;dx} = \answer{{-6 \, \log\left(2 \, x + 2 \, \sqrt{x^{2} - 49}\right)}+C}
\]
\end{problem}}%}

\latexProblemContent{
\ifVerboseLocation This is Integration Compute Question 0037. \\ \fi
\begin{problem}

Compute the following integral:

\input{Integral-Compute-0037.HELP.tex}

\[
\int{{-\frac{3}{\sqrt{x^{2} + 4}}}\;dx} = \answer{{-3 \, {\rm arcsinh}\left(\frac{1}{2} \, x\right)}+C}
\]
\end{problem}}%}

\latexProblemContent{
\ifVerboseLocation This is Integration Compute Question 0037. \\ \fi
\begin{problem}

Compute the following integral:

\input{Integral-Compute-0037.HELP.tex}

\[
\int{{-\frac{3}{\sqrt{x^{2} - 16}}}\;dx} = \answer{{-3 \, \log\left(2 \, x + 2 \, \sqrt{x^{2} - 16}\right)}+C}
\]
\end{problem}}%}

\latexProblemContent{
\ifVerboseLocation This is Integration Compute Question 0037. \\ \fi
\begin{problem}

Compute the following integral:

\input{Integral-Compute-0037.HELP.tex}

\[
\int{{-\frac{7}{\sqrt{x^{2} - 4}}}\;dx} = \answer{{-7 \, \log\left(2 \, x + 2 \, \sqrt{x^{2} - 4}\right)}+C}
\]
\end{problem}}%}

\latexProblemContent{
\ifVerboseLocation This is Integration Compute Question 0037. \\ \fi
\begin{problem}

Compute the following integral:

\input{Integral-Compute-0037.HELP.tex}

\[
\int{{-\frac{6}{\sqrt{-x^{2} + 25}}}\;dx} = \answer{{-6 \, \arcsin\left(\frac{1}{5} \, x\right)}+C}
\]
\end{problem}}%}

\latexProblemContent{
\ifVerboseLocation This is Integration Compute Question 0037. \\ \fi
\begin{problem}

Compute the following integral:

\input{Integral-Compute-0037.HELP.tex}

\[
\int{{\frac{2}{\sqrt{x^{2} + 1}}}\;dx} = \answer{{2 \, {\rm arcsinh}\left(x\right)}+C}
\]
\end{problem}}%}

\latexProblemContent{
\ifVerboseLocation This is Integration Compute Question 0037. \\ \fi
\begin{problem}

Compute the following integral:

\input{Integral-Compute-0037.HELP.tex}

\[
\int{{\frac{3}{\sqrt{-x^{2} + 49}}}\;dx} = \answer{{3 \, \arcsin\left(\frac{1}{7} \, x\right)}+C}
\]
\end{problem}}%}

\latexProblemContent{
\ifVerboseLocation This is Integration Compute Question 0037. \\ \fi
\begin{problem}

Compute the following integral:

\input{Integral-Compute-0037.HELP.tex}

\[
\int{{\frac{5}{\sqrt{x^{2} - 49}}}\;dx} = \answer{{5 \, \log\left(2 \, x + 2 \, \sqrt{x^{2} - 49}\right)}+C}
\]
\end{problem}}%}

\latexProblemContent{
\ifVerboseLocation This is Integration Compute Question 0037. \\ \fi
\begin{problem}

Compute the following integral:

\input{Integral-Compute-0037.HELP.tex}

\[
\int{{\frac{9}{\sqrt{-x^{2} + 1}}}\;dx} = \answer{{9 \, \arcsin\left(x\right)}+C}
\]
\end{problem}}%}

\latexProblemContent{
\ifVerboseLocation This is Integration Compute Question 0037. \\ \fi
\begin{problem}

Compute the following integral:

\input{Integral-Compute-0037.HELP.tex}

\[
\int{{-\frac{7}{\sqrt{x^{2} - 1}}}\;dx} = \answer{{-7 \, \log\left(2 \, x + 2 \, \sqrt{x^{2} - 1}\right)}+C}
\]
\end{problem}}%}

\latexProblemContent{
\ifVerboseLocation This is Integration Compute Question 0037. \\ \fi
\begin{problem}

Compute the following integral:

\input{Integral-Compute-0037.HELP.tex}

\[
\int{{-\frac{9}{\sqrt{x^{2} - 9}}}\;dx} = \answer{{-9 \, \log\left(2 \, x + 2 \, \sqrt{x^{2} - 9}\right)}+C}
\]
\end{problem}}%}

\latexProblemContent{
\ifVerboseLocation This is Integration Compute Question 0037. \\ \fi
\begin{problem}

Compute the following integral:

\input{Integral-Compute-0037.HELP.tex}

\[
\int{{\frac{5}{\sqrt{x^{2} + 81}}}\;dx} = \answer{{5 \, {\rm arcsinh}\left(\frac{1}{9} \, x\right)}+C}
\]
\end{problem}}%}

\latexProblemContent{
\ifVerboseLocation This is Integration Compute Question 0037. \\ \fi
\begin{problem}

Compute the following integral:

\input{Integral-Compute-0037.HELP.tex}

\[
\int{{-\frac{6}{\sqrt{-x^{2} + 4}}}\;dx} = \answer{{-6 \, \arcsin\left(\frac{1}{2} \, x\right)}+C}
\]
\end{problem}}%}

\latexProblemContent{
\ifVerboseLocation This is Integration Compute Question 0037. \\ \fi
\begin{problem}

Compute the following integral:

\input{Integral-Compute-0037.HELP.tex}

\[
\int{{\frac{6}{\sqrt{x^{2} - 25}}}\;dx} = \answer{{6 \, \log\left(2 \, x + 2 \, \sqrt{x^{2} - 25}\right)}+C}
\]
\end{problem}}%}

\latexProblemContent{
\ifVerboseLocation This is Integration Compute Question 0037. \\ \fi
\begin{problem}

Compute the following integral:

\input{Integral-Compute-0037.HELP.tex}

\[
\int{{-\frac{3}{\sqrt{-x^{2} + 16}}}\;dx} = \answer{{-3 \, \arcsin\left(\frac{1}{4} \, x\right)}+C}
\]
\end{problem}}%}

\latexProblemContent{
\ifVerboseLocation This is Integration Compute Question 0037. \\ \fi
\begin{problem}

Compute the following integral:

\input{Integral-Compute-0037.HELP.tex}

\[
\int{{-\frac{4}{\sqrt{x^{2} - 25}}}\;dx} = \answer{{-4 \, \log\left(2 \, x + 2 \, \sqrt{x^{2} - 25}\right)}+C}
\]
\end{problem}}%}

\latexProblemContent{
\ifVerboseLocation This is Integration Compute Question 0037. \\ \fi
\begin{problem}

Compute the following integral:

\input{Integral-Compute-0037.HELP.tex}

\[
\int{{-\frac{5}{\sqrt{x^{2} + 1}}}\;dx} = \answer{{-5 \, {\rm arcsinh}\left(x\right)}+C}
\]
\end{problem}}%}

\latexProblemContent{
\ifVerboseLocation This is Integration Compute Question 0037. \\ \fi
\begin{problem}

Compute the following integral:

\input{Integral-Compute-0037.HELP.tex}

\[
\int{{-\frac{4}{\sqrt{x^{2} + 36}}}\;dx} = \answer{{-4 \, {\rm arcsinh}\left(\frac{1}{6} \, x\right)}+C}
\]
\end{problem}}%}

\latexProblemContent{
\ifVerboseLocation This is Integration Compute Question 0037. \\ \fi
\begin{problem}

Compute the following integral:

\input{Integral-Compute-0037.HELP.tex}

\[
\int{{\frac{1}{\sqrt{x^{2} + 1}}}\;dx} = \answer{{{\rm arcsinh}\left(x\right)}+C}
\]
\end{problem}}%}

\latexProblemContent{
\ifVerboseLocation This is Integration Compute Question 0037. \\ \fi
\begin{problem}

Compute the following integral:

\input{Integral-Compute-0037.HELP.tex}

\[
\int{{\frac{4}{\sqrt{x^{2} + 16}}}\;dx} = \answer{{4 \, {\rm arcsinh}\left(\frac{1}{4} \, x\right)}+C}
\]
\end{problem}}%}

\latexProblemContent{
\ifVerboseLocation This is Integration Compute Question 0037. \\ \fi
\begin{problem}

Compute the following integral:

\input{Integral-Compute-0037.HELP.tex}

\[
\int{{\frac{8}{\sqrt{x^{2} - 36}}}\;dx} = \answer{{8 \, \log\left(2 \, x + 2 \, \sqrt{x^{2} - 36}\right)}+C}
\]
\end{problem}}%}

\latexProblemContent{
\ifVerboseLocation This is Integration Compute Question 0037. \\ \fi
\begin{problem}

Compute the following integral:

\input{Integral-Compute-0037.HELP.tex}

\[
\int{{\frac{3}{\sqrt{x^{2} + 36}}}\;dx} = \answer{{3 \, {\rm arcsinh}\left(\frac{1}{6} \, x\right)}+C}
\]
\end{problem}}%}

\latexProblemContent{
\ifVerboseLocation This is Integration Compute Question 0037. \\ \fi
\begin{problem}

Compute the following integral:

\input{Integral-Compute-0037.HELP.tex}

\[
\int{{-\frac{2}{\sqrt{-x^{2} + 64}}}\;dx} = \answer{{-2 \, \arcsin\left(\frac{1}{8} \, x\right)}+C}
\]
\end{problem}}%}

\latexProblemContent{
\ifVerboseLocation This is Integration Compute Question 0037. \\ \fi
\begin{problem}

Compute the following integral:

\input{Integral-Compute-0037.HELP.tex}

\[
\int{{\frac{7}{\sqrt{x^{2} + 49}}}\;dx} = \answer{{7 \, {\rm arcsinh}\left(\frac{1}{7} \, x\right)}+C}
\]
\end{problem}}%}

\latexProblemContent{
\ifVerboseLocation This is Integration Compute Question 0037. \\ \fi
\begin{problem}

Compute the following integral:

\input{Integral-Compute-0037.HELP.tex}

\[
\int{{\frac{1}{\sqrt{x^{2} - 81}}}\;dx} = \answer{{\log\left(2 \, x + 2 \, \sqrt{x^{2} - 81}\right)}+C}
\]
\end{problem}}%}

\latexProblemContent{
\ifVerboseLocation This is Integration Compute Question 0037. \\ \fi
\begin{problem}

Compute the following integral:

\input{Integral-Compute-0037.HELP.tex}

\[
\int{{-\frac{7}{\sqrt{-x^{2} + 36}}}\;dx} = \answer{{-7 \, \arcsin\left(\frac{1}{6} \, x\right)}+C}
\]
\end{problem}}%}

\latexProblemContent{
\ifVerboseLocation This is Integration Compute Question 0037. \\ \fi
\begin{problem}

Compute the following integral:

\input{Integral-Compute-0037.HELP.tex}

\[
\int{{\frac{6}{\sqrt{x^{2} - 64}}}\;dx} = \answer{{6 \, \log\left(2 \, x + 2 \, \sqrt{x^{2} - 64}\right)}+C}
\]
\end{problem}}%}

\latexProblemContent{
\ifVerboseLocation This is Integration Compute Question 0037. \\ \fi
\begin{problem}

Compute the following integral:

\input{Integral-Compute-0037.HELP.tex}

\[
\int{{-\frac{5}{\sqrt{x^{2} - 49}}}\;dx} = \answer{{-5 \, \log\left(2 \, x + 2 \, \sqrt{x^{2} - 49}\right)}+C}
\]
\end{problem}}%}

\latexProblemContent{
\ifVerboseLocation This is Integration Compute Question 0037. \\ \fi
\begin{problem}

Compute the following integral:

\input{Integral-Compute-0037.HELP.tex}

\[
\int{{\frac{9}{\sqrt{x^{2} + 81}}}\;dx} = \answer{{9 \, {\rm arcsinh}\left(\frac{1}{9} \, x\right)}+C}
\]
\end{problem}}%}

\latexProblemContent{
\ifVerboseLocation This is Integration Compute Question 0037. \\ \fi
\begin{problem}

Compute the following integral:

\input{Integral-Compute-0037.HELP.tex}

\[
\int{{\frac{9}{\sqrt{-x^{2} + 4}}}\;dx} = \answer{{9 \, \arcsin\left(\frac{1}{2} \, x\right)}+C}
\]
\end{problem}}%}

\latexProblemContent{
\ifVerboseLocation This is Integration Compute Question 0037. \\ \fi
\begin{problem}

Compute the following integral:

\input{Integral-Compute-0037.HELP.tex}

\[
\int{{-\frac{8}{\sqrt{-x^{2} + 9}}}\;dx} = \answer{{-8 \, \arcsin\left(\frac{1}{3} \, x\right)}+C}
\]
\end{problem}}%}

\latexProblemContent{
\ifVerboseLocation This is Integration Compute Question 0037. \\ \fi
\begin{problem}

Compute the following integral:

\input{Integral-Compute-0037.HELP.tex}

\[
\int{{\frac{4}{\sqrt{x^{2} - 9}}}\;dx} = \answer{{4 \, \log\left(2 \, x + 2 \, \sqrt{x^{2} - 9}\right)}+C}
\]
\end{problem}}%}

\latexProblemContent{
\ifVerboseLocation This is Integration Compute Question 0037. \\ \fi
\begin{problem}

Compute the following integral:

\input{Integral-Compute-0037.HELP.tex}

\[
\int{{\frac{7}{\sqrt{x^{2} + 16}}}\;dx} = \answer{{7 \, {\rm arcsinh}\left(\frac{1}{4} \, x\right)}+C}
\]
\end{problem}}%}

\latexProblemContent{
\ifVerboseLocation This is Integration Compute Question 0037. \\ \fi
\begin{problem}

Compute the following integral:

\input{Integral-Compute-0037.HELP.tex}

\[
\int{{\frac{1}{\sqrt{x^{2} + 16}}}\;dx} = \answer{{{\rm arcsinh}\left(\frac{1}{4} \, x\right)}+C}
\]
\end{problem}}%}

\latexProblemContent{
\ifVerboseLocation This is Integration Compute Question 0037. \\ \fi
\begin{problem}

Compute the following integral:

\input{Integral-Compute-0037.HELP.tex}

\[
\int{{\frac{5}{\sqrt{-x^{2} + 9}}}\;dx} = \answer{{5 \, \arcsin\left(\frac{1}{3} \, x\right)}+C}
\]
\end{problem}}%}

\latexProblemContent{
\ifVerboseLocation This is Integration Compute Question 0037. \\ \fi
\begin{problem}

Compute the following integral:

\input{Integral-Compute-0037.HELP.tex}

\[
\int{{\frac{1}{\sqrt{x^{2} - 64}}}\;dx} = \answer{{\log\left(2 \, x + 2 \, \sqrt{x^{2} - 64}\right)}+C}
\]
\end{problem}}%}

\latexProblemContent{
\ifVerboseLocation This is Integration Compute Question 0037. \\ \fi
\begin{problem}

Compute the following integral:

\input{Integral-Compute-0037.HELP.tex}

\[
\int{{-\frac{6}{\sqrt{-x^{2} + 64}}}\;dx} = \answer{{-6 \, \arcsin\left(\frac{1}{8} \, x\right)}+C}
\]
\end{problem}}%}

\latexProblemContent{
\ifVerboseLocation This is Integration Compute Question 0037. \\ \fi
\begin{problem}

Compute the following integral:

\input{Integral-Compute-0037.HELP.tex}

\[
\int{{\frac{9}{\sqrt{x^{2} + 36}}}\;dx} = \answer{{9 \, {\rm arcsinh}\left(\frac{1}{6} \, x\right)}+C}
\]
\end{problem}}%}

\latexProblemContent{
\ifVerboseLocation This is Integration Compute Question 0037. \\ \fi
\begin{problem}

Compute the following integral:

\input{Integral-Compute-0037.HELP.tex}

\[
\int{{-\frac{9}{\sqrt{x^{2} + 25}}}\;dx} = \answer{{-9 \, {\rm arcsinh}\left(\frac{1}{5} \, x\right)}+C}
\]
\end{problem}}%}

\latexProblemContent{
\ifVerboseLocation This is Integration Compute Question 0037. \\ \fi
\begin{problem}

Compute the following integral:

\input{Integral-Compute-0037.HELP.tex}

\[
\int{{\frac{3}{\sqrt{-x^{2} + 81}}}\;dx} = \answer{{3 \, \arcsin\left(\frac{1}{9} \, x\right)}+C}
\]
\end{problem}}%}

\latexProblemContent{
\ifVerboseLocation This is Integration Compute Question 0037. \\ \fi
\begin{problem}

Compute the following integral:

\input{Integral-Compute-0037.HELP.tex}

\[
\int{{\frac{8}{\sqrt{x^{2} - 16}}}\;dx} = \answer{{8 \, \log\left(2 \, x + 2 \, \sqrt{x^{2} - 16}\right)}+C}
\]
\end{problem}}%}

\latexProblemContent{
\ifVerboseLocation This is Integration Compute Question 0037. \\ \fi
\begin{problem}

Compute the following integral:

\input{Integral-Compute-0037.HELP.tex}

\[
\int{{-\frac{1}{\sqrt{-x^{2} + 49}}}\;dx} = \answer{{-\arcsin\left(\frac{1}{7} \, x\right)}+C}
\]
\end{problem}}%}

\latexProblemContent{
\ifVerboseLocation This is Integration Compute Question 0037. \\ \fi
\begin{problem}

Compute the following integral:

\input{Integral-Compute-0037.HELP.tex}

\[
\int{{\frac{3}{\sqrt{-x^{2} + 64}}}\;dx} = \answer{{3 \, \arcsin\left(\frac{1}{8} \, x\right)}+C}
\]
\end{problem}}%}

\latexProblemContent{
\ifVerboseLocation This is Integration Compute Question 0037. \\ \fi
\begin{problem}

Compute the following integral:

\input{Integral-Compute-0037.HELP.tex}

\[
\int{{\frac{8}{\sqrt{x^{2} - 64}}}\;dx} = \answer{{8 \, \log\left(2 \, x + 2 \, \sqrt{x^{2} - 64}\right)}+C}
\]
\end{problem}}%}

\latexProblemContent{
\ifVerboseLocation This is Integration Compute Question 0037. \\ \fi
\begin{problem}

Compute the following integral:

\input{Integral-Compute-0037.HELP.tex}

\[
\int{{-\frac{5}{\sqrt{-x^{2} + 16}}}\;dx} = \answer{{-5 \, \arcsin\left(\frac{1}{4} \, x\right)}+C}
\]
\end{problem}}%}

\latexProblemContent{
\ifVerboseLocation This is Integration Compute Question 0037. \\ \fi
\begin{problem}

Compute the following integral:

\input{Integral-Compute-0037.HELP.tex}

\[
\int{{-\frac{3}{\sqrt{x^{2} + 36}}}\;dx} = \answer{{-3 \, {\rm arcsinh}\left(\frac{1}{6} \, x\right)}+C}
\]
\end{problem}}%}

\latexProblemContent{
\ifVerboseLocation This is Integration Compute Question 0037. \\ \fi
\begin{problem}

Compute the following integral:

\input{Integral-Compute-0037.HELP.tex}

\[
\int{{\frac{9}{\sqrt{x^{2} + 16}}}\;dx} = \answer{{9 \, {\rm arcsinh}\left(\frac{1}{4} \, x\right)}+C}
\]
\end{problem}}%}

\latexProblemContent{
\ifVerboseLocation This is Integration Compute Question 0037. \\ \fi
\begin{problem}

Compute the following integral:

\input{Integral-Compute-0037.HELP.tex}

\[
\int{{\frac{2}{\sqrt{x^{2} + 81}}}\;dx} = \answer{{2 \, {\rm arcsinh}\left(\frac{1}{9} \, x\right)}+C}
\]
\end{problem}}%}

\latexProblemContent{
\ifVerboseLocation This is Integration Compute Question 0037. \\ \fi
\begin{problem}

Compute the following integral:

\input{Integral-Compute-0037.HELP.tex}

\[
\int{{\frac{4}{\sqrt{-x^{2} + 49}}}\;dx} = \answer{{4 \, \arcsin\left(\frac{1}{7} \, x\right)}+C}
\]
\end{problem}}%}

\latexProblemContent{
\ifVerboseLocation This is Integration Compute Question 0037. \\ \fi
\begin{problem}

Compute the following integral:

\input{Integral-Compute-0037.HELP.tex}

\[
\int{{\frac{1}{\sqrt{x^{2} - 49}}}\;dx} = \answer{{\log\left(2 \, x + 2 \, \sqrt{x^{2} - 49}\right)}+C}
\]
\end{problem}}%}

\latexProblemContent{
\ifVerboseLocation This is Integration Compute Question 0037. \\ \fi
\begin{problem}

Compute the following integral:

\input{Integral-Compute-0037.HELP.tex}

\[
\int{{-\frac{5}{\sqrt{x^{2} + 49}}}\;dx} = \answer{{-5 \, {\rm arcsinh}\left(\frac{1}{7} \, x\right)}+C}
\]
\end{problem}}%}

\latexProblemContent{
\ifVerboseLocation This is Integration Compute Question 0037. \\ \fi
\begin{problem}

Compute the following integral:

\input{Integral-Compute-0037.HELP.tex}

\[
\int{{\frac{2}{\sqrt{x^{2} + 64}}}\;dx} = \answer{{2 \, {\rm arcsinh}\left(\frac{1}{8} \, x\right)}+C}
\]
\end{problem}}%}

\latexProblemContent{
\ifVerboseLocation This is Integration Compute Question 0037. \\ \fi
\begin{problem}

Compute the following integral:

\input{Integral-Compute-0037.HELP.tex}

\[
\int{{-\frac{5}{\sqrt{-x^{2} + 81}}}\;dx} = \answer{{-5 \, \arcsin\left(\frac{1}{9} \, x\right)}+C}
\]
\end{problem}}%}

\latexProblemContent{
\ifVerboseLocation This is Integration Compute Question 0037. \\ \fi
\begin{problem}

Compute the following integral:

\input{Integral-Compute-0037.HELP.tex}

\[
\int{{-\frac{2}{\sqrt{x^{2} + 81}}}\;dx} = \answer{{-2 \, {\rm arcsinh}\left(\frac{1}{9} \, x\right)}+C}
\]
\end{problem}}%}

\latexProblemContent{
\ifVerboseLocation This is Integration Compute Question 0037. \\ \fi
\begin{problem}

Compute the following integral:

\input{Integral-Compute-0037.HELP.tex}

\[
\int{{-\frac{1}{\sqrt{x^{2} - 36}}}\;dx} = \answer{{-\log\left(2 \, x + 2 \, \sqrt{x^{2} - 36}\right)}+C}
\]
\end{problem}}%}

\latexProblemContent{
\ifVerboseLocation This is Integration Compute Question 0037. \\ \fi
\begin{problem}

Compute the following integral:

\input{Integral-Compute-0037.HELP.tex}

\[
\int{{\frac{8}{\sqrt{-x^{2} + 64}}}\;dx} = \answer{{8 \, \arcsin\left(\frac{1}{8} \, x\right)}+C}
\]
\end{problem}}%}

\latexProblemContent{
\ifVerboseLocation This is Integration Compute Question 0037. \\ \fi
\begin{problem}

Compute the following integral:

\input{Integral-Compute-0037.HELP.tex}

\[
\int{{-\frac{2}{\sqrt{x^{2} - 16}}}\;dx} = \answer{{-2 \, \log\left(2 \, x + 2 \, \sqrt{x^{2} - 16}\right)}+C}
\]
\end{problem}}%}

\latexProblemContent{
\ifVerboseLocation This is Integration Compute Question 0037. \\ \fi
\begin{problem}

Compute the following integral:

\input{Integral-Compute-0037.HELP.tex}

\[
\int{{\frac{4}{\sqrt{x^{2} + 36}}}\;dx} = \answer{{4 \, {\rm arcsinh}\left(\frac{1}{6} \, x\right)}+C}
\]
\end{problem}}%}

\latexProblemContent{
\ifVerboseLocation This is Integration Compute Question 0037. \\ \fi
\begin{problem}

Compute the following integral:

\input{Integral-Compute-0037.HELP.tex}

\[
\int{{-\frac{5}{\sqrt{x^{2} - 36}}}\;dx} = \answer{{-5 \, \log\left(2 \, x + 2 \, \sqrt{x^{2} - 36}\right)}+C}
\]
\end{problem}}%}

\latexProblemContent{
\ifVerboseLocation This is Integration Compute Question 0037. \\ \fi
\begin{problem}

Compute the following integral:

\input{Integral-Compute-0037.HELP.tex}

\[
\int{{\frac{9}{\sqrt{-x^{2} + 9}}}\;dx} = \answer{{9 \, \arcsin\left(\frac{1}{3} \, x\right)}+C}
\]
\end{problem}}%}

\latexProblemContent{
\ifVerboseLocation This is Integration Compute Question 0037. \\ \fi
\begin{problem}

Compute the following integral:

\input{Integral-Compute-0037.HELP.tex}

\[
\int{{-\frac{9}{\sqrt{x^{2} + 1}}}\;dx} = \answer{{-9 \, {\rm arcsinh}\left(x\right)}+C}
\]
\end{problem}}%}

\latexProblemContent{
\ifVerboseLocation This is Integration Compute Question 0037. \\ \fi
\begin{problem}

Compute the following integral:

\input{Integral-Compute-0037.HELP.tex}

\[
\int{{-\frac{9}{\sqrt{x^{2} + 4}}}\;dx} = \answer{{-9 \, {\rm arcsinh}\left(\frac{1}{2} \, x\right)}+C}
\]
\end{problem}}%}

\latexProblemContent{
\ifVerboseLocation This is Integration Compute Question 0037. \\ \fi
\begin{problem}

Compute the following integral:

\input{Integral-Compute-0037.HELP.tex}

\[
\int{{-\frac{2}{\sqrt{-x^{2} + 9}}}\;dx} = \answer{{-2 \, \arcsin\left(\frac{1}{3} \, x\right)}+C}
\]
\end{problem}}%}

\latexProblemContent{
\ifVerboseLocation This is Integration Compute Question 0037. \\ \fi
\begin{problem}

Compute the following integral:

\input{Integral-Compute-0037.HELP.tex}

\[
\int{{-\frac{8}{\sqrt{x^{2} + 1}}}\;dx} = \answer{{-8 \, {\rm arcsinh}\left(x\right)}+C}
\]
\end{problem}}%}

\latexProblemContent{
\ifVerboseLocation This is Integration Compute Question 0037. \\ \fi
\begin{problem}

Compute the following integral:

\input{Integral-Compute-0037.HELP.tex}

\[
\int{{\frac{7}{\sqrt{x^{2} + 1}}}\;dx} = \answer{{7 \, {\rm arcsinh}\left(x\right)}+C}
\]
\end{problem}}%}

\latexProblemContent{
\ifVerboseLocation This is Integration Compute Question 0037. \\ \fi
\begin{problem}

Compute the following integral:

\input{Integral-Compute-0037.HELP.tex}

\[
\int{{\frac{1}{\sqrt{x^{2} + 9}}}\;dx} = \answer{{{\rm arcsinh}\left(\frac{1}{3} \, x\right)}+C}
\]
\end{problem}}%}

\latexProblemContent{
\ifVerboseLocation This is Integration Compute Question 0037. \\ \fi
\begin{problem}

Compute the following integral:

\input{Integral-Compute-0037.HELP.tex}

\[
\int{{-\frac{3}{\sqrt{-x^{2} + 1}}}\;dx} = \answer{{-3 \, \arcsin\left(x\right)}+C}
\]
\end{problem}}%}

\latexProblemContent{
\ifVerboseLocation This is Integration Compute Question 0037. \\ \fi
\begin{problem}

Compute the following integral:

\input{Integral-Compute-0037.HELP.tex}

\[
\int{{-\frac{7}{\sqrt{x^{2} + 16}}}\;dx} = \answer{{-7 \, {\rm arcsinh}\left(\frac{1}{4} \, x\right)}+C}
\]
\end{problem}}%}

\latexProblemContent{
\ifVerboseLocation This is Integration Compute Question 0037. \\ \fi
\begin{problem}

Compute the following integral:

\input{Integral-Compute-0037.HELP.tex}

\[
\int{{-\frac{8}{\sqrt{x^{2} - 16}}}\;dx} = \answer{{-8 \, \log\left(2 \, x + 2 \, \sqrt{x^{2} - 16}\right)}+C}
\]
\end{problem}}%}

\latexProblemContent{
\ifVerboseLocation This is Integration Compute Question 0037. \\ \fi
\begin{problem}

Compute the following integral:

\input{Integral-Compute-0037.HELP.tex}

\[
\int{{-\frac{6}{\sqrt{x^{2} + 49}}}\;dx} = \answer{{-6 \, {\rm arcsinh}\left(\frac{1}{7} \, x\right)}+C}
\]
\end{problem}}%}

\latexProblemContent{
\ifVerboseLocation This is Integration Compute Question 0037. \\ \fi
\begin{problem}

Compute the following integral:

\input{Integral-Compute-0037.HELP.tex}

\[
\int{{\frac{4}{\sqrt{x^{2} + 64}}}\;dx} = \answer{{4 \, {\rm arcsinh}\left(\frac{1}{8} \, x\right)}+C}
\]
\end{problem}}%}

\latexProblemContent{
\ifVerboseLocation This is Integration Compute Question 0037. \\ \fi
\begin{problem}

Compute the following integral:

\input{Integral-Compute-0037.HELP.tex}

\[
\int{{\frac{9}{\sqrt{-x^{2} + 64}}}\;dx} = \answer{{9 \, \arcsin\left(\frac{1}{8} \, x\right)}+C}
\]
\end{problem}}%}

\latexProblemContent{
\ifVerboseLocation This is Integration Compute Question 0037. \\ \fi
\begin{problem}

Compute the following integral:

\input{Integral-Compute-0037.HELP.tex}

\[
\int{{-\frac{8}{\sqrt{-x^{2} + 16}}}\;dx} = \answer{{-8 \, \arcsin\left(\frac{1}{4} \, x\right)}+C}
\]
\end{problem}}%}

\latexProblemContent{
\ifVerboseLocation This is Integration Compute Question 0037. \\ \fi
\begin{problem}

Compute the following integral:

\input{Integral-Compute-0037.HELP.tex}

\[
\int{{-\frac{8}{\sqrt{x^{2} - 1}}}\;dx} = \answer{{-8 \, \log\left(2 \, x + 2 \, \sqrt{x^{2} - 1}\right)}+C}
\]
\end{problem}}%}

\latexProblemContent{
\ifVerboseLocation This is Integration Compute Question 0037. \\ \fi
\begin{problem}

Compute the following integral:

\input{Integral-Compute-0037.HELP.tex}

\[
\int{{-\frac{3}{\sqrt{x^{2} + 16}}}\;dx} = \answer{{-3 \, {\rm arcsinh}\left(\frac{1}{4} \, x\right)}+C}
\]
\end{problem}}%}

\latexProblemContent{
\ifVerboseLocation This is Integration Compute Question 0037. \\ \fi
\begin{problem}

Compute the following integral:

\input{Integral-Compute-0037.HELP.tex}

\[
\int{{\frac{8}{\sqrt{x^{2} + 9}}}\;dx} = \answer{{8 \, {\rm arcsinh}\left(\frac{1}{3} \, x\right)}+C}
\]
\end{problem}}%}

\latexProblemContent{
\ifVerboseLocation This is Integration Compute Question 0037. \\ \fi
\begin{problem}

Compute the following integral:

\input{Integral-Compute-0037.HELP.tex}

\[
\int{{\frac{7}{\sqrt{-x^{2} + 64}}}\;dx} = \answer{{7 \, \arcsin\left(\frac{1}{8} \, x\right)}+C}
\]
\end{problem}}%}

\latexProblemContent{
\ifVerboseLocation This is Integration Compute Question 0037. \\ \fi
\begin{problem}

Compute the following integral:

\input{Integral-Compute-0037.HELP.tex}

\[
\int{{\frac{7}{\sqrt{x^{2} - 64}}}\;dx} = \answer{{7 \, \log\left(2 \, x + 2 \, \sqrt{x^{2} - 64}\right)}+C}
\]
\end{problem}}%}

\latexProblemContent{
\ifVerboseLocation This is Integration Compute Question 0037. \\ \fi
\begin{problem}

Compute the following integral:

\input{Integral-Compute-0037.HELP.tex}

\[
\int{{-\frac{6}{\sqrt{x^{2} - 36}}}\;dx} = \answer{{-6 \, \log\left(2 \, x + 2 \, \sqrt{x^{2} - 36}\right)}+C}
\]
\end{problem}}%}

\latexProblemContent{
\ifVerboseLocation This is Integration Compute Question 0037. \\ \fi
\begin{problem}

Compute the following integral:

\input{Integral-Compute-0037.HELP.tex}

\[
\int{{-\frac{5}{\sqrt{x^{2} + 81}}}\;dx} = \answer{{-5 \, {\rm arcsinh}\left(\frac{1}{9} \, x\right)}+C}
\]
\end{problem}}%}

\latexProblemContent{
\ifVerboseLocation This is Integration Compute Question 0037. \\ \fi
\begin{problem}

Compute the following integral:

\input{Integral-Compute-0037.HELP.tex}

\[
\int{{\frac{9}{\sqrt{x^{2} - 25}}}\;dx} = \answer{{9 \, \log\left(2 \, x + 2 \, \sqrt{x^{2} - 25}\right)}+C}
\]
\end{problem}}%}

\latexProblemContent{
\ifVerboseLocation This is Integration Compute Question 0037. \\ \fi
\begin{problem}

Compute the following integral:

\input{Integral-Compute-0037.HELP.tex}

\[
\int{{-\frac{5}{\sqrt{x^{2} + 25}}}\;dx} = \answer{{-5 \, {\rm arcsinh}\left(\frac{1}{5} \, x\right)}+C}
\]
\end{problem}}%}

\latexProblemContent{
\ifVerboseLocation This is Integration Compute Question 0037. \\ \fi
\begin{problem}

Compute the following integral:

\input{Integral-Compute-0037.HELP.tex}

\[
\int{{-\frac{6}{\sqrt{x^{2} + 36}}}\;dx} = \answer{{-6 \, {\rm arcsinh}\left(\frac{1}{6} \, x\right)}+C}
\]
\end{problem}}%}

\latexProblemContent{
\ifVerboseLocation This is Integration Compute Question 0037. \\ \fi
\begin{problem}

Compute the following integral:

\input{Integral-Compute-0037.HELP.tex}

\[
\int{{-\frac{1}{\sqrt{x^{2} + 9}}}\;dx} = \answer{{-{\rm arcsinh}\left(\frac{1}{3} \, x\right)}+C}
\]
\end{problem}}%}

\latexProblemContent{
\ifVerboseLocation This is Integration Compute Question 0037. \\ \fi
\begin{problem}

Compute the following integral:

\input{Integral-Compute-0037.HELP.tex}

\[
\int{{\frac{8}{\sqrt{-x^{2} + 1}}}\;dx} = \answer{{8 \, \arcsin\left(x\right)}+C}
\]
\end{problem}}%}

\latexProblemContent{
\ifVerboseLocation This is Integration Compute Question 0037. \\ \fi
\begin{problem}

Compute the following integral:

\input{Integral-Compute-0037.HELP.tex}

\[
\int{{\frac{5}{\sqrt{-x^{2} + 16}}}\;dx} = \answer{{5 \, \arcsin\left(\frac{1}{4} \, x\right)}+C}
\]
\end{problem}}%}

\latexProblemContent{
\ifVerboseLocation This is Integration Compute Question 0037. \\ \fi
\begin{problem}

Compute the following integral:

\input{Integral-Compute-0037.HELP.tex}

\[
\int{{\frac{1}{\sqrt{-x^{2} + 9}}}\;dx} = \answer{{\arcsin\left(\frac{1}{3} \, x\right)}+C}
\]
\end{problem}}%}

\latexProblemContent{
\ifVerboseLocation This is Integration Compute Question 0037. \\ \fi
\begin{problem}

Compute the following integral:

\input{Integral-Compute-0037.HELP.tex}

\[
\int{{-\frac{7}{\sqrt{x^{2} + 64}}}\;dx} = \answer{{-7 \, {\rm arcsinh}\left(\frac{1}{8} \, x\right)}+C}
\]
\end{problem}}%}

\latexProblemContent{
\ifVerboseLocation This is Integration Compute Question 0037. \\ \fi
\begin{problem}

Compute the following integral:

\input{Integral-Compute-0037.HELP.tex}

\[
\int{{-\frac{5}{\sqrt{-x^{2} + 4}}}\;dx} = \answer{{-5 \, \arcsin\left(\frac{1}{2} \, x\right)}+C}
\]
\end{problem}}%}

\latexProblemContent{
\ifVerboseLocation This is Integration Compute Question 0037. \\ \fi
\begin{problem}

Compute the following integral:

\input{Integral-Compute-0037.HELP.tex}

\[
\int{{-\frac{1}{\sqrt{-x^{2} + 9}}}\;dx} = \answer{{-\arcsin\left(\frac{1}{3} \, x\right)}+C}
\]
\end{problem}}%}

