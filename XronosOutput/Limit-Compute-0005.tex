\ProblemFileHeader{XTL_SV_QUESTIONCOUNT}% Process how many problems are in this file and how to detect if it has a desirable problem
\ifproblemToFind% If it has a desirable problem search the file.
%\tagged{Ans@ShortAns, Type@Compute, Topic@Limit, Sub@Rational, File@0005}{
\latexProblemContent{
\ifVerboseLocation This is Derivative Compute Question 0005. \\ \fi
\begin{problem}

Determine if the limit approaches a finite number, $\pm\infty$, or does not exist. (If the limit does not exist, write DNE)

\input{Limit-Compute-0005.HELP.tex}

\[\lim_{x\to{-4}}\dfrac{x^{2} + 2 \, x - 8}{x^{2} - 2 \, x - 24}=\answer{\frac{3}{5}}\]
\end{problem}}%}

\latexProblemContent{
\ifVerboseLocation This is Derivative Compute Question 0005. \\ \fi
\begin{problem}

Determine if the limit approaches a finite number, $\pm\infty$, or does not exist. (If the limit does not exist, write DNE)

\input{Limit-Compute-0005.HELP.tex}

\[\lim_{x\to{1}}\dfrac{x^{2} - 1}{x^{2} - 1}=\answer{1}\]
\end{problem}}%}

\latexProblemContent{
\ifVerboseLocation This is Derivative Compute Question 0005. \\ \fi
\begin{problem}

Determine if the limit approaches a finite number, $\pm\infty$, or does not exist. (If the limit does not exist, write DNE)

\input{Limit-Compute-0005.HELP.tex}

\[\lim_{x\to{1}}\dfrac{x^{2} - 3 \, x + 2}{x^{2} - 1}=\answer{-\frac{1}{2}}\]
\end{problem}}%}

\latexProblemContent{
\ifVerboseLocation This is Derivative Compute Question 0005. \\ \fi
\begin{problem}

Determine if the limit approaches a finite number, $\pm\infty$, or does not exist. (If the limit does not exist, write DNE)

\input{Limit-Compute-0005.HELP.tex}

\[\lim_{x\to{-2}}\dfrac{x^{2} - 4}{x^{2} - 2 \, x - 8}=\answer{\frac{2}{3}}\]
\end{problem}}%}

\latexProblemContent{
\ifVerboseLocation This is Derivative Compute Question 0005. \\ \fi
\begin{problem}

Determine if the limit approaches a finite number, $\pm\infty$, or does not exist. (If the limit does not exist, write DNE)

\input{Limit-Compute-0005.HELP.tex}

\[\lim_{x\to{4}}\dfrac{x^{2} - 5 \, x + 4}{x^{2} + x - 20}=\answer{\frac{1}{3}}\]
\end{problem}}%}

\latexProblemContent{
\ifVerboseLocation This is Derivative Compute Question 0005. \\ \fi
\begin{problem}

Determine if the limit approaches a finite number, $\pm\infty$, or does not exist. (If the limit does not exist, write DNE)

\input{Limit-Compute-0005.HELP.tex}

\[\lim_{x\to{-3}}\dfrac{x^{2} - 9}{x^{2} + 7 \, x + 12}=\answer{-6}\]
\end{problem}}%}

\latexProblemContent{
\ifVerboseLocation This is Derivative Compute Question 0005. \\ \fi
\begin{problem}

Determine if the limit approaches a finite number, $\pm\infty$, or does not exist. (If the limit does not exist, write DNE)

\input{Limit-Compute-0005.HELP.tex}

\[\lim_{x\to{2}}\dfrac{x^{2} - 6 \, x + 8}{x^{2} - 6 \, x + 8}=\answer{1}\]
\end{problem}}%}

\latexProblemContent{
\ifVerboseLocation This is Derivative Compute Question 0005. \\ \fi
\begin{problem}

Determine if the limit approaches a finite number, $\pm\infty$, or does not exist. (If the limit does not exist, write DNE)

\input{Limit-Compute-0005.HELP.tex}

\[\lim_{x\to{-3}}\dfrac{x^{2} - x - 12}{x^{2} + x - 6}=\answer{\frac{7}{5}}\]
\end{problem}}%}

\latexProblemContent{
\ifVerboseLocation This is Derivative Compute Question 0005. \\ \fi
\begin{problem}

Determine if the limit approaches a finite number, $\pm\infty$, or does not exist. (If the limit does not exist, write DNE)

\input{Limit-Compute-0005.HELP.tex}

\[\lim_{x\to{5}}\dfrac{x^{2} - 25}{x^{2} - x - 20}=\answer{\frac{10}{9}}\]
\end{problem}}%}

\latexProblemContent{
\ifVerboseLocation This is Derivative Compute Question 0005. \\ \fi
\begin{problem}

Determine if the limit approaches a finite number, $\pm\infty$, or does not exist. (If the limit does not exist, write DNE)

\input{Limit-Compute-0005.HELP.tex}

\[\lim_{x\to{-1}}\dfrac{x^{2} + 4 \, x + 3}{x^{2} + 7 \, x + 6}=\answer{\frac{2}{5}}\]
\end{problem}}%}

\latexProblemContent{
\ifVerboseLocation This is Derivative Compute Question 0005. \\ \fi
\begin{problem}

Determine if the limit approaches a finite number, $\pm\infty$, or does not exist. (If the limit does not exist, write DNE)

\input{Limit-Compute-0005.HELP.tex}

\[\lim_{x\to{-6}}\dfrac{x^{2} + 4 \, x - 12}{x^{2} + 4 \, x - 12}=\answer{1}\]
\end{problem}}%}

\latexProblemContent{
\ifVerboseLocation This is Derivative Compute Question 0005. \\ \fi
\begin{problem}

Determine if the limit approaches a finite number, $\pm\infty$, or does not exist. (If the limit does not exist, write DNE)

\input{Limit-Compute-0005.HELP.tex}

\[\lim_{x\to{3}}\dfrac{x^{2} - 8 \, x + 15}{x^{2} - 7 \, x + 12}=\answer{2}\]
\end{problem}}%}

\latexProblemContent{
\ifVerboseLocation This is Derivative Compute Question 0005. \\ \fi
\begin{problem}

Determine if the limit approaches a finite number, $\pm\infty$, or does not exist. (If the limit does not exist, write DNE)

\input{Limit-Compute-0005.HELP.tex}

\[\lim_{x\to{-5}}\dfrac{x^{2} + 2 \, x - 15}{x^{2} + 9 \, x + 20}=\answer{8}\]
\end{problem}}%}

\latexProblemContent{
\ifVerboseLocation This is Derivative Compute Question 0005. \\ \fi
\begin{problem}

Determine if the limit approaches a finite number, $\pm\infty$, or does not exist. (If the limit does not exist, write DNE)

\input{Limit-Compute-0005.HELP.tex}

\[\lim_{x\to{4}}\dfrac{x^{2} - 16}{x^{2} + 2 \, x - 24}=\answer{\frac{4}{5}}\]
\end{problem}}%}

\latexProblemContent{
\ifVerboseLocation This is Derivative Compute Question 0005. \\ \fi
\begin{problem}

Determine if the limit approaches a finite number, $\pm\infty$, or does not exist. (If the limit does not exist, write DNE)

\input{Limit-Compute-0005.HELP.tex}

\[\lim_{x\to{-5}}\dfrac{x^{2} + 6 \, x + 5}{x^{2} + 8 \, x + 15}=\answer{2}\]
\end{problem}}%}

\latexProblemContent{
\ifVerboseLocation This is Derivative Compute Question 0005. \\ \fi
\begin{problem}

Determine if the limit approaches a finite number, $\pm\infty$, or does not exist. (If the limit does not exist, write DNE)

\input{Limit-Compute-0005.HELP.tex}

\[\lim_{x\to{-2}}\dfrac{x^{2} + 6 \, x + 8}{x^{2} - 2 \, x - 8}=\answer{-\frac{1}{3}}\]
\end{problem}}%}

\latexProblemContent{
\ifVerboseLocation This is Derivative Compute Question 0005. \\ \fi
\begin{problem}

Determine if the limit approaches a finite number, $\pm\infty$, or does not exist. (If the limit does not exist, write DNE)

\input{Limit-Compute-0005.HELP.tex}

\[\lim_{x\to{6}}\dfrac{x^{2} - 5 \, x - 6}{x^{2} - 3 \, x - 18}=\answer{\frac{7}{9}}\]
\end{problem}}%}

\latexProblemContent{
\ifVerboseLocation This is Derivative Compute Question 0005. \\ \fi
\begin{problem}

Determine if the limit approaches a finite number, $\pm\infty$, or does not exist. (If the limit does not exist, write DNE)

\input{Limit-Compute-0005.HELP.tex}

\[\lim_{x\to{5}}\dfrac{x^{2} - 9 \, x + 20}{x^{2} - 8 \, x + 15}=\answer{\frac{1}{2}}\]
\end{problem}}%}

\latexProblemContent{
\ifVerboseLocation This is Derivative Compute Question 0005. \\ \fi
\begin{problem}

Determine if the limit approaches a finite number, $\pm\infty$, or does not exist. (If the limit does not exist, write DNE)

\input{Limit-Compute-0005.HELP.tex}

\[\lim_{x\to{2}}\dfrac{x^{2} + x - 6}{x^{2} - 7 \, x + 10}=\answer{-\frac{5}{3}}\]
\end{problem}}%}

\latexProblemContent{
\ifVerboseLocation This is Derivative Compute Question 0005. \\ \fi
\begin{problem}

Determine if the limit approaches a finite number, $\pm\infty$, or does not exist. (If the limit does not exist, write DNE)

\input{Limit-Compute-0005.HELP.tex}

\[\lim_{x\to{5}}\dfrac{x^{2} - 10 \, x + 25}{x^{2} - 8 \, x + 15}=\answer{0}\]
\end{problem}}%}

\latexProblemContent{
\ifVerboseLocation This is Derivative Compute Question 0005. \\ \fi
\begin{problem}

Determine if the limit approaches a finite number, $\pm\infty$, or does not exist. (If the limit does not exist, write DNE)

\input{Limit-Compute-0005.HELP.tex}

\[\lim_{x\to{-3}}\dfrac{x^{2} + 2 \, x - 3}{x^{2} - 2 \, x - 15}=\answer{\frac{1}{2}}\]
\end{problem}}%}

\latexProblemContent{
\ifVerboseLocation This is Derivative Compute Question 0005. \\ \fi
\begin{problem}

Determine if the limit approaches a finite number, $\pm\infty$, or does not exist. (If the limit does not exist, write DNE)

\input{Limit-Compute-0005.HELP.tex}

\[\lim_{x\to{-1}}\dfrac{x^{2} + 3 \, x + 2}{x^{2} - x - 2}=\answer{-\frac{1}{3}}\]
\end{problem}}%}

\latexProblemContent{
\ifVerboseLocation This is Derivative Compute Question 0005. \\ \fi
\begin{problem}

Determine if the limit approaches a finite number, $\pm\infty$, or does not exist. (If the limit does not exist, write DNE)

\input{Limit-Compute-0005.HELP.tex}

\[\lim_{x\to{5}}\dfrac{x^{2} - 7 \, x + 10}{x^{2} - 9 \, x + 20}=\answer{3}\]
\end{problem}}%}

\latexProblemContent{
\ifVerboseLocation This is Derivative Compute Question 0005. \\ \fi
\begin{problem}

Determine if the limit approaches a finite number, $\pm\infty$, or does not exist. (If the limit does not exist, write DNE)

\input{Limit-Compute-0005.HELP.tex}

\[\lim_{x\to{4}}\dfrac{x^{2} - 6 \, x + 8}{x^{2} - 16}=\answer{\frac{1}{4}}\]
\end{problem}}%}

\latexProblemContent{
\ifVerboseLocation This is Derivative Compute Question 0005. \\ \fi
\begin{problem}

Determine if the limit approaches a finite number, $\pm\infty$, or does not exist. (If the limit does not exist, write DNE)

\input{Limit-Compute-0005.HELP.tex}

\[\lim_{x\to{-1}}\dfrac{x^{2} + 5 \, x + 4}{x^{2} + 4 \, x + 3}=\answer{\frac{3}{2}}\]
\end{problem}}%}

\latexProblemContent{
\ifVerboseLocation This is Derivative Compute Question 0005. \\ \fi
\begin{problem}

Determine if the limit approaches a finite number, $\pm\infty$, or does not exist. (If the limit does not exist, write DNE)

\input{Limit-Compute-0005.HELP.tex}

\[\lim_{x\to{-5}}\dfrac{x^{2} + 2 \, x - 15}{x^{2} + 3 \, x - 10}=\answer{\frac{8}{7}}\]
\end{problem}}%}

\latexProblemContent{
\ifVerboseLocation This is Derivative Compute Question 0005. \\ \fi
\begin{problem}

Determine if the limit approaches a finite number, $\pm\infty$, or does not exist. (If the limit does not exist, write DNE)

\input{Limit-Compute-0005.HELP.tex}

\[\lim_{x\to{4}}\dfrac{x^{2} - 2 \, x - 8}{x^{2} - x - 12}=\answer{\frac{6}{7}}\]
\end{problem}}%}

\latexProblemContent{
\ifVerboseLocation This is Derivative Compute Question 0005. \\ \fi
\begin{problem}

Determine if the limit approaches a finite number, $\pm\infty$, or does not exist. (If the limit does not exist, write DNE)

\input{Limit-Compute-0005.HELP.tex}

\[\lim_{x\to{-6}}\dfrac{x^{2} + 9 \, x + 18}{x^{2} + 4 \, x - 12}=\answer{\frac{3}{8}}\]
\end{problem}}%}

\latexProblemContent{
\ifVerboseLocation This is Derivative Compute Question 0005. \\ \fi
\begin{problem}

Determine if the limit approaches a finite number, $\pm\infty$, or does not exist. (If the limit does not exist, write DNE)

\input{Limit-Compute-0005.HELP.tex}

\[\lim_{x\to{1}}\dfrac{x^{2} - 6 \, x + 5}{x^{2} + x - 2}=\answer{-\frac{4}{3}}\]
\end{problem}}%}

\latexProblemContent{
\ifVerboseLocation This is Derivative Compute Question 0005. \\ \fi
\begin{problem}

Determine if the limit approaches a finite number, $\pm\infty$, or does not exist. (If the limit does not exist, write DNE)

\input{Limit-Compute-0005.HELP.tex}

\[\lim_{x\to{-4}}\dfrac{x^{2} + 5 \, x + 4}{x^{2} + 2 \, x - 8}=\answer{\frac{1}{2}}\]
\end{problem}}%}

\latexProblemContent{
\ifVerboseLocation This is Derivative Compute Question 0005. \\ \fi
\begin{problem}

Determine if the limit approaches a finite number, $\pm\infty$, or does not exist. (If the limit does not exist, write DNE)

\input{Limit-Compute-0005.HELP.tex}

\[\lim_{x\to{-1}}\dfrac{x^{2} - 2 \, x - 3}{x^{2} - 4 \, x - 5}=\answer{\frac{2}{3}}\]
\end{problem}}%}

\latexProblemContent{
\ifVerboseLocation This is Derivative Compute Question 0005. \\ \fi
\begin{problem}

Determine if the limit approaches a finite number, $\pm\infty$, or does not exist. (If the limit does not exist, write DNE)

\input{Limit-Compute-0005.HELP.tex}

\[\lim_{x\to{-3}}\dfrac{x^{2} - 2 \, x - 15}{x^{2} + 5 \, x + 6}=\answer{8}\]
\end{problem}}%}

\latexProblemContent{
\ifVerboseLocation This is Derivative Compute Question 0005. \\ \fi
\begin{problem}

Determine if the limit approaches a finite number, $\pm\infty$, or does not exist. (If the limit does not exist, write DNE)

\input{Limit-Compute-0005.HELP.tex}

\[\lim_{x\to{3}}\dfrac{x^{2} - 6 \, x + 9}{x^{2} - 4 \, x + 3}=\answer{0}\]
\end{problem}}%}

\latexProblemContent{
\ifVerboseLocation This is Derivative Compute Question 0005. \\ \fi
\begin{problem}

Determine if the limit approaches a finite number, $\pm\infty$, or does not exist. (If the limit does not exist, write DNE)

\input{Limit-Compute-0005.HELP.tex}

\[\lim_{x\to{-1}}\dfrac{x^{2} + 6 \, x + 5}{x^{2} + 7 \, x + 6}=\answer{\frac{4}{5}}\]
\end{problem}}%}

\latexProblemContent{
\ifVerboseLocation This is Derivative Compute Question 0005. \\ \fi
\begin{problem}

Determine if the limit approaches a finite number, $\pm\infty$, or does not exist. (If the limit does not exist, write DNE)

\input{Limit-Compute-0005.HELP.tex}

\[\lim_{x\to{2}}\dfrac{x^{2} - 5 \, x + 6}{x^{2} + 2 \, x - 8}=\answer{-\frac{1}{6}}\]
\end{problem}}%}

\latexProblemContent{
\ifVerboseLocation This is Derivative Compute Question 0005. \\ \fi
\begin{problem}

Determine if the limit approaches a finite number, $\pm\infty$, or does not exist. (If the limit does not exist, write DNE)

\input{Limit-Compute-0005.HELP.tex}

\[\lim_{x\to{4}}\dfrac{x^{2} + x - 20}{x^{2} - 2 \, x - 8}=\answer{\frac{3}{2}}\]
\end{problem}}%}

\latexProblemContent{
\ifVerboseLocation This is Derivative Compute Question 0005. \\ \fi
\begin{problem}

Determine if the limit approaches a finite number, $\pm\infty$, or does not exist. (If the limit does not exist, write DNE)

\input{Limit-Compute-0005.HELP.tex}

\[\lim_{x\to{-4}}\dfrac{x^{2} + 7 \, x + 12}{x^{2} - x - 20}=\answer{\frac{1}{9}}\]
\end{problem}}%}

\latexProblemContent{
\ifVerboseLocation This is Derivative Compute Question 0005. \\ \fi
\begin{problem}

Determine if the limit approaches a finite number, $\pm\infty$, or does not exist. (If the limit does not exist, write DNE)

\input{Limit-Compute-0005.HELP.tex}

\[\lim_{x\to{-1}}\dfrac{x^{2} - x - 2}{x^{2} - 3 \, x - 4}=\answer{\frac{3}{5}}\]
\end{problem}}%}

\latexProblemContent{
\ifVerboseLocation This is Derivative Compute Question 0005. \\ \fi
\begin{problem}

Determine if the limit approaches a finite number, $\pm\infty$, or does not exist. (If the limit does not exist, write DNE)

\input{Limit-Compute-0005.HELP.tex}

\[\lim_{x\to{-3}}\dfrac{x^{2} + 2 \, x - 3}{x^{2} - 9}=\answer{\frac{2}{3}}\]
\end{problem}}%}

\latexProblemContent{
\ifVerboseLocation This is Derivative Compute Question 0005. \\ \fi
\begin{problem}

Determine if the limit approaches a finite number, $\pm\infty$, or does not exist. (If the limit does not exist, write DNE)

\input{Limit-Compute-0005.HELP.tex}

\[\lim_{x\to{-6}}\dfrac{x^{2} + 9 \, x + 18}{x^{2} + 9 \, x + 18}=\answer{1}\]
\end{problem}}%}

\latexProblemContent{
\ifVerboseLocation This is Derivative Compute Question 0005. \\ \fi
\begin{problem}

Determine if the limit approaches a finite number, $\pm\infty$, or does not exist. (If the limit does not exist, write DNE)

\input{Limit-Compute-0005.HELP.tex}

\[\lim_{x\to{-5}}\dfrac{x^{2} + 8 \, x + 15}{x^{2} + 6 \, x + 5}=\answer{\frac{1}{2}}\]
\end{problem}}%}

\latexProblemContent{
\ifVerboseLocation This is Derivative Compute Question 0005. \\ \fi
\begin{problem}

Determine if the limit approaches a finite number, $\pm\infty$, or does not exist. (If the limit does not exist, write DNE)

\input{Limit-Compute-0005.HELP.tex}

\[\lim_{x\to{6}}\dfrac{x^{2} - 3 \, x - 18}{x^{2} - 10 \, x + 24}=\answer{\frac{9}{2}}\]
\end{problem}}%}

\latexProblemContent{
\ifVerboseLocation This is Derivative Compute Question 0005. \\ \fi
\begin{problem}

Determine if the limit approaches a finite number, $\pm\infty$, or does not exist. (If the limit does not exist, write DNE)

\input{Limit-Compute-0005.HELP.tex}

\[\lim_{x\to{1}}\dfrac{x^{2} - 3 \, x + 2}{x^{2} - 7 \, x + 6}=\answer{\frac{1}{5}}\]
\end{problem}}%}

\latexProblemContent{
\ifVerboseLocation This is Derivative Compute Question 0005. \\ \fi
\begin{problem}

Determine if the limit approaches a finite number, $\pm\infty$, or does not exist. (If the limit does not exist, write DNE)

\input{Limit-Compute-0005.HELP.tex}

\[\lim_{x\to{1}}\dfrac{x^{2} - 2 \, x + 1}{x^{2} - 1}=\answer{0}\]
\end{problem}}%}

\latexProblemContent{
\ifVerboseLocation This is Derivative Compute Question 0005. \\ \fi
\begin{problem}

Determine if the limit approaches a finite number, $\pm\infty$, or does not exist. (If the limit does not exist, write DNE)

\input{Limit-Compute-0005.HELP.tex}

\[\lim_{x\to{-4}}\dfrac{x^{2} + 5 \, x + 4}{x^{2} + 7 \, x + 12}=\answer{3}\]
\end{problem}}%}

\latexProblemContent{
\ifVerboseLocation This is Derivative Compute Question 0005. \\ \fi
\begin{problem}

Determine if the limit approaches a finite number, $\pm\infty$, or does not exist. (If the limit does not exist, write DNE)

\input{Limit-Compute-0005.HELP.tex}

\[\lim_{x\to{1}}\dfrac{x^{2} + 4 \, x - 5}{x^{2} + 4 \, x - 5}=\answer{1}\]
\end{problem}}%}

\latexProblemContent{
\ifVerboseLocation This is Derivative Compute Question 0005. \\ \fi
\begin{problem}

Determine if the limit approaches a finite number, $\pm\infty$, or does not exist. (If the limit does not exist, write DNE)

\input{Limit-Compute-0005.HELP.tex}

\[\lim_{x\to{-6}}\dfrac{x^{2} + 9 \, x + 18}{x^{2} + 8 \, x + 12}=\answer{\frac{3}{4}}\]
\end{problem}}%}

\latexProblemContent{
\ifVerboseLocation This is Derivative Compute Question 0005. \\ \fi
\begin{problem}

Determine if the limit approaches a finite number, $\pm\infty$, or does not exist. (If the limit does not exist, write DNE)

\input{Limit-Compute-0005.HELP.tex}

\[\lim_{x\to{-2}}\dfrac{x^{2} - 4}{x^{2} + 7 \, x + 10}=\answer{-\frac{4}{3}}\]
\end{problem}}%}

\latexProblemContent{
\ifVerboseLocation This is Derivative Compute Question 0005. \\ \fi
\begin{problem}

Determine if the limit approaches a finite number, $\pm\infty$, or does not exist. (If the limit does not exist, write DNE)

\input{Limit-Compute-0005.HELP.tex}

\[\lim_{x\to{3}}\dfrac{x^{2} - 6 \, x + 9}{x^{2} - 9}=\answer{0}\]
\end{problem}}%}

\latexProblemContent{
\ifVerboseLocation This is Derivative Compute Question 0005. \\ \fi
\begin{problem}

Determine if the limit approaches a finite number, $\pm\infty$, or does not exist. (If the limit does not exist, write DNE)

\input{Limit-Compute-0005.HELP.tex}

\[\lim_{x\to{2}}\dfrac{x^{2} - 7 \, x + 10}{x^{2} + x - 6}=\answer{-\frac{3}{5}}\]
\end{problem}}%}

\latexProblemContent{
\ifVerboseLocation This is Derivative Compute Question 0005. \\ \fi
\begin{problem}

Determine if the limit approaches a finite number, $\pm\infty$, or does not exist. (If the limit does not exist, write DNE)

\input{Limit-Compute-0005.HELP.tex}

\[\lim_{x\to{-6}}\dfrac{x^{2} + 4 \, x - 12}{x^{2} + 2 \, x - 24}=\answer{\frac{4}{5}}\]
\end{problem}}%}

\latexProblemContent{
\ifVerboseLocation This is Derivative Compute Question 0005. \\ \fi
\begin{problem}

Determine if the limit approaches a finite number, $\pm\infty$, or does not exist. (If the limit does not exist, write DNE)

\input{Limit-Compute-0005.HELP.tex}

\[\lim_{x\to{-2}}\dfrac{x^{2} + 7 \, x + 10}{x^{2} + 8 \, x + 12}=\answer{\frac{3}{4}}\]
\end{problem}}%}

\latexProblemContent{
\ifVerboseLocation This is Derivative Compute Question 0005. \\ \fi
\begin{problem}

Determine if the limit approaches a finite number, $\pm\infty$, or does not exist. (If the limit does not exist, write DNE)

\input{Limit-Compute-0005.HELP.tex}

\[\lim_{x\to{6}}\dfrac{x^{2} - 2 \, x - 24}{x^{2} - x - 30}=\answer{\frac{10}{11}}\]
\end{problem}}%}

\latexProblemContent{
\ifVerboseLocation This is Derivative Compute Question 0005. \\ \fi
\begin{problem}

Determine if the limit approaches a finite number, $\pm\infty$, or does not exist. (If the limit does not exist, write DNE)

\input{Limit-Compute-0005.HELP.tex}

\[\lim_{x\to{2}}\dfrac{x^{2} - x - 2}{x^{2} - 5 \, x + 6}=\answer{-3}\]
\end{problem}}%}

\latexProblemContent{
\ifVerboseLocation This is Derivative Compute Question 0005. \\ \fi
\begin{problem}

Determine if the limit approaches a finite number, $\pm\infty$, or does not exist. (If the limit does not exist, write DNE)

\input{Limit-Compute-0005.HELP.tex}

\[\lim_{x\to{5}}\dfrac{x^{2} - 2 \, x - 15}{x^{2} - 11 \, x + 30}=\answer{-8}\]
\end{problem}}%}

\latexProblemContent{
\ifVerboseLocation This is Derivative Compute Question 0005. \\ \fi
\begin{problem}

Determine if the limit approaches a finite number, $\pm\infty$, or does not exist. (If the limit does not exist, write DNE)

\input{Limit-Compute-0005.HELP.tex}

\[\lim_{x\to{-4}}\dfrac{x^{2} + 9 \, x + 20}{x^{2} + 7 \, x + 12}=\answer{-1}\]
\end{problem}}%}

\latexProblemContent{
\ifVerboseLocation This is Derivative Compute Question 0005. \\ \fi
\begin{problem}

Determine if the limit approaches a finite number, $\pm\infty$, or does not exist. (If the limit does not exist, write DNE)

\input{Limit-Compute-0005.HELP.tex}

\[\lim_{x\to{-6}}\dfrac{x^{2} + 3 \, x - 18}{x^{2} + 8 \, x + 12}=\answer{\frac{9}{4}}\]
\end{problem}}%}

\latexProblemContent{
\ifVerboseLocation This is Derivative Compute Question 0005. \\ \fi
\begin{problem}

Determine if the limit approaches a finite number, $\pm\infty$, or does not exist. (If the limit does not exist, write DNE)

\input{Limit-Compute-0005.HELP.tex}

\[\lim_{x\to{-6}}\dfrac{x^{2} + 2 \, x - 24}{x^{2} + 5 \, x - 6}=\answer{\frac{10}{7}}\]
\end{problem}}%}

\latexProblemContent{
\ifVerboseLocation This is Derivative Compute Question 0005. \\ \fi
\begin{problem}

Determine if the limit approaches a finite number, $\pm\infty$, or does not exist. (If the limit does not exist, write DNE)

\input{Limit-Compute-0005.HELP.tex}

\[\lim_{x\to{1}}\dfrac{x^{2} - 6 \, x + 5}{x^{2} - 1}=\answer{-2}\]
\end{problem}}%}

\latexProblemContent{
\ifVerboseLocation This is Derivative Compute Question 0005. \\ \fi
\begin{problem}

Determine if the limit approaches a finite number, $\pm\infty$, or does not exist. (If the limit does not exist, write DNE)

\input{Limit-Compute-0005.HELP.tex}

\[\lim_{x\to{-4}}\dfrac{x^{2} + 5 \, x + 4}{x^{2} - 2 \, x - 24}=\answer{\frac{3}{10}}\]
\end{problem}}%}

\latexProblemContent{
\ifVerboseLocation This is Derivative Compute Question 0005. \\ \fi
\begin{problem}

Determine if the limit approaches a finite number, $\pm\infty$, or does not exist. (If the limit does not exist, write DNE)

\input{Limit-Compute-0005.HELP.tex}

\[\lim_{x\to{6}}\dfrac{x^{2} - x - 30}{x^{2} - 2 \, x - 24}=\answer{\frac{11}{10}}\]
\end{problem}}%}

\latexProblemContent{
\ifVerboseLocation This is Derivative Compute Question 0005. \\ \fi
\begin{problem}

Determine if the limit approaches a finite number, $\pm\infty$, or does not exist. (If the limit does not exist, write DNE)

\input{Limit-Compute-0005.HELP.tex}

\[\lim_{x\to{-6}}\dfrac{x^{2} + 2 \, x - 24}{x^{2} + 2 \, x - 24}=\answer{1}\]
\end{problem}}%}

\latexProblemContent{
\ifVerboseLocation This is Derivative Compute Question 0005. \\ \fi
\begin{problem}

Determine if the limit approaches a finite number, $\pm\infty$, or does not exist. (If the limit does not exist, write DNE)

\input{Limit-Compute-0005.HELP.tex}

\[\lim_{x\to{-1}}\dfrac{x^{2} - 3 \, x - 4}{x^{2} + 3 \, x + 2}=\answer{-5}\]
\end{problem}}%}

\latexProblemContent{
\ifVerboseLocation This is Derivative Compute Question 0005. \\ \fi
\begin{problem}

Determine if the limit approaches a finite number, $\pm\infty$, or does not exist. (If the limit does not exist, write DNE)

\input{Limit-Compute-0005.HELP.tex}

\[\lim_{x\to{5}}\dfrac{x^{2} - 3 \, x - 10}{x^{2} - 7 \, x + 10}=\answer{\frac{7}{3}}\]
\end{problem}}%}

\latexProblemContent{
\ifVerboseLocation This is Derivative Compute Question 0005. \\ \fi
\begin{problem}

Determine if the limit approaches a finite number, $\pm\infty$, or does not exist. (If the limit does not exist, write DNE)

\input{Limit-Compute-0005.HELP.tex}

\[\lim_{x\to{-2}}\dfrac{x^{2} + 4 \, x + 4}{x^{2} - 4 \, x - 12}=\answer{0}\]
\end{problem}}%}

\latexProblemContent{
\ifVerboseLocation This is Derivative Compute Question 0005. \\ \fi
\begin{problem}

Determine if the limit approaches a finite number, $\pm\infty$, or does not exist. (If the limit does not exist, write DNE)

\input{Limit-Compute-0005.HELP.tex}

\[\lim_{x\to{-4}}\dfrac{x^{2} + 6 \, x + 8}{x^{2} + 10 \, x + 24}=\answer{-1}\]
\end{problem}}%}

\latexProblemContent{
\ifVerboseLocation This is Derivative Compute Question 0005. \\ \fi
\begin{problem}

Determine if the limit approaches a finite number, $\pm\infty$, or does not exist. (If the limit does not exist, write DNE)

\input{Limit-Compute-0005.HELP.tex}

\[\lim_{x\to{-1}}\dfrac{x^{2} + 2 \, x + 1}{x^{2} + 5 \, x + 4}=\answer{0}\]
\end{problem}}%}

\latexProblemContent{
\ifVerboseLocation This is Derivative Compute Question 0005. \\ \fi
\begin{problem}

Determine if the limit approaches a finite number, $\pm\infty$, or does not exist. (If the limit does not exist, write DNE)

\input{Limit-Compute-0005.HELP.tex}

\[\lim_{x\to{2}}\dfrac{x^{2} - 5 \, x + 6}{x^{2} - 8 \, x + 12}=\answer{\frac{1}{4}}\]
\end{problem}}%}

\latexProblemContent{
\ifVerboseLocation This is Derivative Compute Question 0005. \\ \fi
\begin{problem}

Determine if the limit approaches a finite number, $\pm\infty$, or does not exist. (If the limit does not exist, write DNE)

\input{Limit-Compute-0005.HELP.tex}

\[\lim_{x\to{4}}\dfrac{x^{2} - 16}{x^{2} - 5 \, x + 4}=\answer{\frac{8}{3}}\]
\end{problem}}%}

\latexProblemContent{
\ifVerboseLocation This is Derivative Compute Question 0005. \\ \fi
\begin{problem}

Determine if the limit approaches a finite number, $\pm\infty$, or does not exist. (If the limit does not exist, write DNE)

\input{Limit-Compute-0005.HELP.tex}

\[\lim_{x\to{-1}}\dfrac{x^{2} + 5 \, x + 4}{x^{2} - 5 \, x - 6}=\answer{-\frac{3}{7}}\]
\end{problem}}%}

\latexProblemContent{
\ifVerboseLocation This is Derivative Compute Question 0005. \\ \fi
\begin{problem}

Determine if the limit approaches a finite number, $\pm\infty$, or does not exist. (If the limit does not exist, write DNE)

\input{Limit-Compute-0005.HELP.tex}

\[\lim_{x\to{4}}\dfrac{x^{2} + x - 20}{x^{2} - x - 12}=\answer{\frac{9}{7}}\]
\end{problem}}%}

\latexProblemContent{
\ifVerboseLocation This is Derivative Compute Question 0005. \\ \fi
\begin{problem}

Determine if the limit approaches a finite number, $\pm\infty$, or does not exist. (If the limit does not exist, write DNE)

\input{Limit-Compute-0005.HELP.tex}

\[\lim_{x\to{-6}}\dfrac{x^{2} + x - 30}{x^{2} + 11 \, x + 30}=\answer{11}\]
\end{problem}}%}

\latexProblemContent{
\ifVerboseLocation This is Derivative Compute Question 0005. \\ \fi
\begin{problem}

Determine if the limit approaches a finite number, $\pm\infty$, or does not exist. (If the limit does not exist, write DNE)

\input{Limit-Compute-0005.HELP.tex}

\[\lim_{x\to{1}}\dfrac{x^{2} - 3 \, x + 2}{x^{2} + 4 \, x - 5}=\answer{-\frac{1}{6}}\]
\end{problem}}%}

\latexProblemContent{
\ifVerboseLocation This is Derivative Compute Question 0005. \\ \fi
\begin{problem}

Determine if the limit approaches a finite number, $\pm\infty$, or does not exist. (If the limit does not exist, write DNE)

\input{Limit-Compute-0005.HELP.tex}

\[\lim_{x\to{1}}\dfrac{x^{2} - 2 \, x + 1}{x^{2} - 4 \, x + 3}=\answer{0}\]
\end{problem}}%}

\latexProblemContent{
\ifVerboseLocation This is Derivative Compute Question 0005. \\ \fi
\begin{problem}

Determine if the limit approaches a finite number, $\pm\infty$, or does not exist. (If the limit does not exist, write DNE)

\input{Limit-Compute-0005.HELP.tex}

\[\lim_{x\to{-1}}\dfrac{x^{2} + 2 \, x + 1}{x^{2} - 3 \, x - 4}=\answer{0}\]
\end{problem}}%}

\latexProblemContent{
\ifVerboseLocation This is Derivative Compute Question 0005. \\ \fi
\begin{problem}

Determine if the limit approaches a finite number, $\pm\infty$, or does not exist. (If the limit does not exist, write DNE)

\input{Limit-Compute-0005.HELP.tex}

\[\lim_{x\to{2}}\dfrac{x^{2} - 6 \, x + 8}{x^{2} + 2 \, x - 8}=\answer{-\frac{1}{3}}\]
\end{problem}}%}

\latexProblemContent{
\ifVerboseLocation This is Derivative Compute Question 0005. \\ \fi
\begin{problem}

Determine if the limit approaches a finite number, $\pm\infty$, or does not exist. (If the limit does not exist, write DNE)

\input{Limit-Compute-0005.HELP.tex}

\[\lim_{x\to{1}}\dfrac{x^{2} + 3 \, x - 4}{x^{2} + 3 \, x - 4}=\answer{1}\]
\end{problem}}%}

\latexProblemContent{
\ifVerboseLocation This is Derivative Compute Question 0005. \\ \fi
\begin{problem}

Determine if the limit approaches a finite number, $\pm\infty$, or does not exist. (If the limit does not exist, write DNE)

\input{Limit-Compute-0005.HELP.tex}

\[\lim_{x\to{-2}}\dfrac{x^{2} + 3 \, x + 2}{x^{2} - 4 \, x - 12}=\answer{\frac{1}{8}}\]
\end{problem}}%}

\latexProblemContent{
\ifVerboseLocation This is Derivative Compute Question 0005. \\ \fi
\begin{problem}

Determine if the limit approaches a finite number, $\pm\infty$, or does not exist. (If the limit does not exist, write DNE)

\input{Limit-Compute-0005.HELP.tex}

\[\lim_{x\to{2}}\dfrac{x^{2} - 3 \, x + 2}{x^{2} - 4}=\answer{\frac{1}{4}}\]
\end{problem}}%}

\latexProblemContent{
\ifVerboseLocation This is Derivative Compute Question 0005. \\ \fi
\begin{problem}

Determine if the limit approaches a finite number, $\pm\infty$, or does not exist. (If the limit does not exist, write DNE)

\input{Limit-Compute-0005.HELP.tex}

\[\lim_{x\to{6}}\dfrac{x^{2} - 7 \, x + 6}{x^{2} - 11 \, x + 30}=\answer{5}\]
\end{problem}}%}

\latexProblemContent{
\ifVerboseLocation This is Derivative Compute Question 0005. \\ \fi
\begin{problem}

Determine if the limit approaches a finite number, $\pm\infty$, or does not exist. (If the limit does not exist, write DNE)

\input{Limit-Compute-0005.HELP.tex}

\[\lim_{x\to{4}}\dfrac{x^{2} - 9 \, x + 20}{x^{2} - 6 \, x + 8}=\answer{-\frac{1}{2}}\]
\end{problem}}%}

\latexProblemContent{
\ifVerboseLocation This is Derivative Compute Question 0005. \\ \fi
\begin{problem}

Determine if the limit approaches a finite number, $\pm\infty$, or does not exist. (If the limit does not exist, write DNE)

\input{Limit-Compute-0005.HELP.tex}

\[\lim_{x\to{3}}\dfrac{x^{2} + x - 12}{x^{2} + 3 \, x - 18}=\answer{\frac{7}{9}}\]
\end{problem}}%}

\latexProblemContent{
\ifVerboseLocation This is Derivative Compute Question 0005. \\ \fi
\begin{problem}

Determine if the limit approaches a finite number, $\pm\infty$, or does not exist. (If the limit does not exist, write DNE)

\input{Limit-Compute-0005.HELP.tex}

\[\lim_{x\to{4}}\dfrac{x^{2} - 9 \, x + 20}{x^{2} + 2 \, x - 24}=\answer{-\frac{1}{10}}\]
\end{problem}}%}

\latexProblemContent{
\ifVerboseLocation This is Derivative Compute Question 0005. \\ \fi
\begin{problem}

Determine if the limit approaches a finite number, $\pm\infty$, or does not exist. (If the limit does not exist, write DNE)

\input{Limit-Compute-0005.HELP.tex}

\[\lim_{x\to{1}}\dfrac{x^{2} + x - 2}{x^{2} - 7 \, x + 6}=\answer{-\frac{3}{5}}\]
\end{problem}}%}

\latexProblemContent{
\ifVerboseLocation This is Derivative Compute Question 0005. \\ \fi
\begin{problem}

Determine if the limit approaches a finite number, $\pm\infty$, or does not exist. (If the limit does not exist, write DNE)

\input{Limit-Compute-0005.HELP.tex}

\[\lim_{x\to{3}}\dfrac{x^{2} - 5 \, x + 6}{x^{2} - 5 \, x + 6}=\answer{1}\]
\end{problem}}%}

\latexProblemContent{
\ifVerboseLocation This is Derivative Compute Question 0005. \\ \fi
\begin{problem}

Determine if the limit approaches a finite number, $\pm\infty$, or does not exist. (If the limit does not exist, write DNE)

\input{Limit-Compute-0005.HELP.tex}

\[\lim_{x\to{3}}\dfrac{x^{2} - 6 \, x + 9}{x^{2} - 2 \, x - 3}=\answer{0}\]
\end{problem}}%}

\latexProblemContent{
\ifVerboseLocation This is Derivative Compute Question 0005. \\ \fi
\begin{problem}

Determine if the limit approaches a finite number, $\pm\infty$, or does not exist. (If the limit does not exist, write DNE)

\input{Limit-Compute-0005.HELP.tex}

\[\lim_{x\to{3}}\dfrac{x^{2} - 8 \, x + 15}{x^{2} - 5 \, x + 6}=\answer{-2}\]
\end{problem}}%}

\latexProblemContent{
\ifVerboseLocation This is Derivative Compute Question 0005. \\ \fi
\begin{problem}

Determine if the limit approaches a finite number, $\pm\infty$, or does not exist. (If the limit does not exist, write DNE)

\input{Limit-Compute-0005.HELP.tex}

\[\lim_{x\to{4}}\dfrac{x^{2} - 7 \, x + 12}{x^{2} + x - 20}=\answer{\frac{1}{9}}\]
\end{problem}}%}

\latexProblemContent{
\ifVerboseLocation This is Derivative Compute Question 0005. \\ \fi
\begin{problem}

Determine if the limit approaches a finite number, $\pm\infty$, or does not exist. (If the limit does not exist, write DNE)

\input{Limit-Compute-0005.HELP.tex}

\[\lim_{x\to{-3}}\dfrac{x^{2} + 6 \, x + 9}{x^{2} + 9 \, x + 18}=\answer{0}\]
\end{problem}}%}

\latexProblemContent{
\ifVerboseLocation This is Derivative Compute Question 0005. \\ \fi
\begin{problem}

Determine if the limit approaches a finite number, $\pm\infty$, or does not exist. (If the limit does not exist, write DNE)

\input{Limit-Compute-0005.HELP.tex}

\[\lim_{x\to{5}}\dfrac{x^{2} - 25}{x^{2} - 11 \, x + 30}=\answer{-10}\]
\end{problem}}%}

\latexProblemContent{
\ifVerboseLocation This is Derivative Compute Question 0005. \\ \fi
\begin{problem}

Determine if the limit approaches a finite number, $\pm\infty$, or does not exist. (If the limit does not exist, write DNE)

\input{Limit-Compute-0005.HELP.tex}

\[\lim_{x\to{1}}\dfrac{x^{2} + x - 2}{x^{2} + 2 \, x - 3}=\answer{\frac{3}{4}}\]
\end{problem}}%}

\latexProblemContent{
\ifVerboseLocation This is Derivative Compute Question 0005. \\ \fi
\begin{problem}

Determine if the limit approaches a finite number, $\pm\infty$, or does not exist. (If the limit does not exist, write DNE)

\input{Limit-Compute-0005.HELP.tex}

\[\lim_{x\to{-2}}\dfrac{x^{2} + 7 \, x + 10}{x^{2} - 4 \, x - 12}=\answer{-\frac{3}{8}}\]
\end{problem}}%}

\latexProblemContent{
\ifVerboseLocation This is Derivative Compute Question 0005. \\ \fi
\begin{problem}

Determine if the limit approaches a finite number, $\pm\infty$, or does not exist. (If the limit does not exist, write DNE)

\input{Limit-Compute-0005.HELP.tex}

\[\lim_{x\to{6}}\dfrac{x^{2} - 3 \, x - 18}{x^{2} - 8 \, x + 12}=\answer{\frac{9}{4}}\]
\end{problem}}%}

\latexProblemContent{
\ifVerboseLocation This is Derivative Compute Question 0005. \\ \fi
\begin{problem}

Determine if the limit approaches a finite number, $\pm\infty$, or does not exist. (If the limit does not exist, write DNE)

\input{Limit-Compute-0005.HELP.tex}

\[\lim_{x\to{6}}\dfrac{x^{2} - 4 \, x - 12}{x^{2} - 8 \, x + 12}=\answer{2}\]
\end{problem}}%}

\latexProblemContent{
\ifVerboseLocation This is Derivative Compute Question 0005. \\ \fi
\begin{problem}

Determine if the limit approaches a finite number, $\pm\infty$, or does not exist. (If the limit does not exist, write DNE)

\input{Limit-Compute-0005.HELP.tex}

\[\lim_{x\to{-3}}\dfrac{x^{2} + 7 \, x + 12}{x^{2} + 7 \, x + 12}=\answer{1}\]
\end{problem}}%}

\latexProblemContent{
\ifVerboseLocation This is Derivative Compute Question 0005. \\ \fi
\begin{problem}

Determine if the limit approaches a finite number, $\pm\infty$, or does not exist. (If the limit does not exist, write DNE)

\input{Limit-Compute-0005.HELP.tex}

\[\lim_{x\to{6}}\dfrac{x^{2} - 8 \, x + 12}{x^{2} - 10 \, x + 24}=\answer{2}\]
\end{problem}}%}

\latexProblemContent{
\ifVerboseLocation This is Derivative Compute Question 0005. \\ \fi
\begin{problem}

Determine if the limit approaches a finite number, $\pm\infty$, or does not exist. (If the limit does not exist, write DNE)

\input{Limit-Compute-0005.HELP.tex}

\[\lim_{x\to{-4}}\dfrac{x^{2} + 9 \, x + 20}{x^{2} + 5 \, x + 4}=\answer{-\frac{1}{3}}\]
\end{problem}}%}

\latexProblemContent{
\ifVerboseLocation This is Derivative Compute Question 0005. \\ \fi
\begin{problem}

Determine if the limit approaches a finite number, $\pm\infty$, or does not exist. (If the limit does not exist, write DNE)

\input{Limit-Compute-0005.HELP.tex}

\[\lim_{x\to{-2}}\dfrac{x^{2} + 6 \, x + 8}{x^{2} - 3 \, x - 10}=\answer{-\frac{2}{7}}\]
\end{problem}}%}

\latexProblemContent{
\ifVerboseLocation This is Derivative Compute Question 0005. \\ \fi
\begin{problem}

Determine if the limit approaches a finite number, $\pm\infty$, or does not exist. (If the limit does not exist, write DNE)

\input{Limit-Compute-0005.HELP.tex}

\[\lim_{x\to{-2}}\dfrac{x^{2} - 4}{x^{2} - x - 6}=\answer{\frac{4}{5}}\]
\end{problem}}%}

\latexProblemContent{
\ifVerboseLocation This is Derivative Compute Question 0005. \\ \fi
\begin{problem}

Determine if the limit approaches a finite number, $\pm\infty$, or does not exist. (If the limit does not exist, write DNE)

\input{Limit-Compute-0005.HELP.tex}

\[\lim_{x\to{3}}\dfrac{x^{2} + 2 \, x - 15}{x^{2} - 2 \, x - 3}=\answer{2}\]
\end{problem}}%}

\latexProblemContent{
\ifVerboseLocation This is Derivative Compute Question 0005. \\ \fi
\begin{problem}

Determine if the limit approaches a finite number, $\pm\infty$, or does not exist. (If the limit does not exist, write DNE)

\input{Limit-Compute-0005.HELP.tex}

\[\lim_{x\to{2}}\dfrac{x^{2} + 2 \, x - 8}{x^{2} + 2 \, x - 8}=\answer{1}\]
\end{problem}}%}

\latexProblemContent{
\ifVerboseLocation This is Derivative Compute Question 0005. \\ \fi
\begin{problem}

Determine if the limit approaches a finite number, $\pm\infty$, or does not exist. (If the limit does not exist, write DNE)

\input{Limit-Compute-0005.HELP.tex}

\[\lim_{x\to{-4}}\dfrac{x^{2} + 3 \, x - 4}{x^{2} - x - 20}=\answer{\frac{5}{9}}\]
\end{problem}}%}

\latexProblemContent{
\ifVerboseLocation This is Derivative Compute Question 0005. \\ \fi
\begin{problem}

Determine if the limit approaches a finite number, $\pm\infty$, or does not exist. (If the limit does not exist, write DNE)

\input{Limit-Compute-0005.HELP.tex}

\[\lim_{x\to{-1}}\dfrac{x^{2} - 1}{x^{2} + 5 \, x + 4}=\answer{-\frac{2}{3}}\]
\end{problem}}%}

\latexProblemContent{
\ifVerboseLocation This is Derivative Compute Question 0005. \\ \fi
\begin{problem}

Determine if the limit approaches a finite number, $\pm\infty$, or does not exist. (If the limit does not exist, write DNE)

\input{Limit-Compute-0005.HELP.tex}

\[\lim_{x\to{-3}}\dfrac{x^{2} + x - 6}{x^{2} + 4 \, x + 3}=\answer{\frac{5}{2}}\]
\end{problem}}%}

\latexProblemContent{
\ifVerboseLocation This is Derivative Compute Question 0005. \\ \fi
\begin{problem}

Determine if the limit approaches a finite number, $\pm\infty$, or does not exist. (If the limit does not exist, write DNE)

\input{Limit-Compute-0005.HELP.tex}

\[\lim_{x\to{-4}}\dfrac{x^{2} + 3 \, x - 4}{x^{2} + x - 12}=\answer{\frac{5}{7}}\]
\end{problem}}%}

\latexProblemContent{
\ifVerboseLocation This is Derivative Compute Question 0005. \\ \fi
\begin{problem}

Determine if the limit approaches a finite number, $\pm\infty$, or does not exist. (If the limit does not exist, write DNE)

\input{Limit-Compute-0005.HELP.tex}

\[\lim_{x\to{-6}}\dfrac{x^{2} + 5 \, x - 6}{x^{2} + 5 \, x - 6}=\answer{1}\]
\end{problem}}%}

\latexProblemContent{
\ifVerboseLocation This is Derivative Compute Question 0005. \\ \fi
\begin{problem}

Determine if the limit approaches a finite number, $\pm\infty$, or does not exist. (If the limit does not exist, write DNE)

\input{Limit-Compute-0005.HELP.tex}

\[\lim_{x\to{2}}\dfrac{x^{2} - x - 2}{x^{2} - 8 \, x + 12}=\answer{-\frac{3}{4}}\]
\end{problem}}%}

\latexProblemContent{
\ifVerboseLocation This is Derivative Compute Question 0005. \\ \fi
\begin{problem}

Determine if the limit approaches a finite number, $\pm\infty$, or does not exist. (If the limit does not exist, write DNE)

\input{Limit-Compute-0005.HELP.tex}

\[\lim_{x\to{-2}}\dfrac{x^{2} - 3 \, x - 10}{x^{2} + 3 \, x + 2}=\answer{7}\]
\end{problem}}%}

\latexProblemContent{
\ifVerboseLocation This is Derivative Compute Question 0005. \\ \fi
\begin{problem}

Determine if the limit approaches a finite number, $\pm\infty$, or does not exist. (If the limit does not exist, write DNE)

\input{Limit-Compute-0005.HELP.tex}

\[\lim_{x\to{-1}}\dfrac{x^{2} + 6 \, x + 5}{x^{2} - 3 \, x - 4}=\answer{-\frac{4}{5}}\]
\end{problem}}%}

\latexProblemContent{
\ifVerboseLocation This is Derivative Compute Question 0005. \\ \fi
\begin{problem}

Determine if the limit approaches a finite number, $\pm\infty$, or does not exist. (If the limit does not exist, write DNE)

\input{Limit-Compute-0005.HELP.tex}

\[\lim_{x\to{-5}}\dfrac{x^{2} - 25}{x^{2} + 8 \, x + 15}=\answer{5}\]
\end{problem}}%}

\latexProblemContent{
\ifVerboseLocation This is Derivative Compute Question 0005. \\ \fi
\begin{problem}

Determine if the limit approaches a finite number, $\pm\infty$, or does not exist. (If the limit does not exist, write DNE)

\input{Limit-Compute-0005.HELP.tex}

\[\lim_{x\to{2}}\dfrac{x^{2} - 6 \, x + 8}{x^{2} - 7 \, x + 10}=\answer{\frac{2}{3}}\]
\end{problem}}%}

\latexProblemContent{
\ifVerboseLocation This is Derivative Compute Question 0005. \\ \fi
\begin{problem}

Determine if the limit approaches a finite number, $\pm\infty$, or does not exist. (If the limit does not exist, write DNE)

\input{Limit-Compute-0005.HELP.tex}

\[\lim_{x\to{-5}}\dfrac{x^{2} + 2 \, x - 15}{x^{2} + 7 \, x + 10}=\answer{\frac{8}{3}}\]
\end{problem}}%}

\latexProblemContent{
\ifVerboseLocation This is Derivative Compute Question 0005. \\ \fi
\begin{problem}

Determine if the limit approaches a finite number, $\pm\infty$, or does not exist. (If the limit does not exist, write DNE)

\input{Limit-Compute-0005.HELP.tex}

\[\lim_{x\to{-5}}\dfrac{x^{2} + 7 \, x + 10}{x^{2} + 6 \, x + 5}=\answer{\frac{3}{4}}\]
\end{problem}}%}

\latexProblemContent{
\ifVerboseLocation This is Derivative Compute Question 0005. \\ \fi
\begin{problem}

Determine if the limit approaches a finite number, $\pm\infty$, or does not exist. (If the limit does not exist, write DNE)

\input{Limit-Compute-0005.HELP.tex}

\[\lim_{x\to{-3}}\dfrac{x^{2} - 2 \, x - 15}{x^{2} - 9}=\answer{\frac{4}{3}}\]
\end{problem}}%}

\latexProblemContent{
\ifVerboseLocation This is Derivative Compute Question 0005. \\ \fi
\begin{problem}

Determine if the limit approaches a finite number, $\pm\infty$, or does not exist. (If the limit does not exist, write DNE)

\input{Limit-Compute-0005.HELP.tex}

\[\lim_{x\to{5}}\dfrac{x^{2} - 25}{x^{2} - 6 \, x + 5}=\answer{\frac{5}{2}}\]
\end{problem}}%}

\latexProblemContent{
\ifVerboseLocation This is Derivative Compute Question 0005. \\ \fi
\begin{problem}

Determine if the limit approaches a finite number, $\pm\infty$, or does not exist. (If the limit does not exist, write DNE)

\input{Limit-Compute-0005.HELP.tex}

\[\lim_{x\to{2}}\dfrac{x^{2} - 3 \, x + 2}{x^{2} - 3 \, x + 2}=\answer{1}\]
\end{problem}}%}

\latexProblemContent{
\ifVerboseLocation This is Derivative Compute Question 0005. \\ \fi
\begin{problem}

Determine if the limit approaches a finite number, $\pm\infty$, or does not exist. (If the limit does not exist, write DNE)

\input{Limit-Compute-0005.HELP.tex}

\[\lim_{x\to{-3}}\dfrac{x^{2} - 2 \, x - 15}{x^{2} + 9 \, x + 18}=\answer{-\frac{8}{3}}\]
\end{problem}}%}

\latexProblemContent{
\ifVerboseLocation This is Derivative Compute Question 0005. \\ \fi
\begin{problem}

Determine if the limit approaches a finite number, $\pm\infty$, or does not exist. (If the limit does not exist, write DNE)

\input{Limit-Compute-0005.HELP.tex}

\[\lim_{x\to{1}}\dfrac{x^{2} + 2 \, x - 3}{x^{2} + 2 \, x - 3}=\answer{1}\]
\end{problem}}%}

\latexProblemContent{
\ifVerboseLocation This is Derivative Compute Question 0005. \\ \fi
\begin{problem}

Determine if the limit approaches a finite number, $\pm\infty$, or does not exist. (If the limit does not exist, write DNE)

\input{Limit-Compute-0005.HELP.tex}

\[\lim_{x\to{3}}\dfrac{x^{2} - x - 6}{x^{2} + 3 \, x - 18}=\answer{\frac{5}{9}}\]
\end{problem}}%}

\latexProblemContent{
\ifVerboseLocation This is Derivative Compute Question 0005. \\ \fi
\begin{problem}

Determine if the limit approaches a finite number, $\pm\infty$, or does not exist. (If the limit does not exist, write DNE)

\input{Limit-Compute-0005.HELP.tex}

\[\lim_{x\to{3}}\dfrac{x^{2} - 4 \, x + 3}{x^{2} - 2 \, x - 3}=\answer{\frac{1}{2}}\]
\end{problem}}%}

\latexProblemContent{
\ifVerboseLocation This is Derivative Compute Question 0005. \\ \fi
\begin{problem}

Determine if the limit approaches a finite number, $\pm\infty$, or does not exist. (If the limit does not exist, write DNE)

\input{Limit-Compute-0005.HELP.tex}

\[\lim_{x\to{-5}}\dfrac{x^{2} + 10 \, x + 25}{x^{2} + 6 \, x + 5}=\answer{0}\]
\end{problem}}%}

\latexProblemContent{
\ifVerboseLocation This is Derivative Compute Question 0005. \\ \fi
\begin{problem}

Determine if the limit approaches a finite number, $\pm\infty$, or does not exist. (If the limit does not exist, write DNE)

\input{Limit-Compute-0005.HELP.tex}

\[\lim_{x\to{6}}\dfrac{x^{2} - x - 30}{x^{2} - 3 \, x - 18}=\answer{\frac{11}{9}}\]
\end{problem}}%}

\latexProblemContent{
\ifVerboseLocation This is Derivative Compute Question 0005. \\ \fi
\begin{problem}

Determine if the limit approaches a finite number, $\pm\infty$, or does not exist. (If the limit does not exist, write DNE)

\input{Limit-Compute-0005.HELP.tex}

\[\lim_{x\to{-2}}\dfrac{x^{2} - x - 6}{x^{2} + 3 \, x + 2}=\answer{5}\]
\end{problem}}%}

\latexProblemContent{
\ifVerboseLocation This is Derivative Compute Question 0005. \\ \fi
\begin{problem}

Determine if the limit approaches a finite number, $\pm\infty$, or does not exist. (If the limit does not exist, write DNE)

\input{Limit-Compute-0005.HELP.tex}

\[\lim_{x\to{5}}\dfrac{x^{2} - 10 \, x + 25}{x^{2} - x - 20}=\answer{0}\]
\end{problem}}%}

\latexProblemContent{
\ifVerboseLocation This is Derivative Compute Question 0005. \\ \fi
\begin{problem}

Determine if the limit approaches a finite number, $\pm\infty$, or does not exist. (If the limit does not exist, write DNE)

\input{Limit-Compute-0005.HELP.tex}

\[\lim_{x\to{4}}\dfrac{x^{2} - 2 \, x - 8}{x^{2} - 16}=\answer{\frac{3}{4}}\]
\end{problem}}%}

\latexProblemContent{
\ifVerboseLocation This is Derivative Compute Question 0005. \\ \fi
\begin{problem}

Determine if the limit approaches a finite number, $\pm\infty$, or does not exist. (If the limit does not exist, write DNE)

\input{Limit-Compute-0005.HELP.tex}

\[\lim_{x\to{-5}}\dfrac{x^{2} + 4 \, x - 5}{x^{2} + 7 \, x + 10}=\answer{2}\]
\end{problem}}%}

\latexProblemContent{
\ifVerboseLocation This is Derivative Compute Question 0005. \\ \fi
\begin{problem}

Determine if the limit approaches a finite number, $\pm\infty$, or does not exist. (If the limit does not exist, write DNE)

\input{Limit-Compute-0005.HELP.tex}

\[\lim_{x\to{-6}}\dfrac{x^{2} + 7 \, x + 6}{x^{2} + 9 \, x + 18}=\answer{\frac{5}{3}}\]
\end{problem}}%}

\latexProblemContent{
\ifVerboseLocation This is Derivative Compute Question 0005. \\ \fi
\begin{problem}

Determine if the limit approaches a finite number, $\pm\infty$, or does not exist. (If the limit does not exist, write DNE)

\input{Limit-Compute-0005.HELP.tex}

\[\lim_{x\to{6}}\dfrac{x^{2} - 8 \, x + 12}{x^{2} - 9 \, x + 18}=\answer{\frac{4}{3}}\]
\end{problem}}%}

\latexProblemContent{
\ifVerboseLocation This is Derivative Compute Question 0005. \\ \fi
\begin{problem}

Determine if the limit approaches a finite number, $\pm\infty$, or does not exist. (If the limit does not exist, write DNE)

\input{Limit-Compute-0005.HELP.tex}

\[\lim_{x\to{-4}}\dfrac{x^{2} - x - 20}{x^{2} + 3 \, x - 4}=\answer{\frac{9}{5}}\]
\end{problem}}%}

\latexProblemContent{
\ifVerboseLocation This is Derivative Compute Question 0005. \\ \fi
\begin{problem}

Determine if the limit approaches a finite number, $\pm\infty$, or does not exist. (If the limit does not exist, write DNE)

\input{Limit-Compute-0005.HELP.tex}

\[\lim_{x\to{-1}}\dfrac{x^{2} + 3 \, x + 2}{x^{2} - 5 \, x - 6}=\answer{-\frac{1}{7}}\]
\end{problem}}%}

\latexProblemContent{
\ifVerboseLocation This is Derivative Compute Question 0005. \\ \fi
\begin{problem}

Determine if the limit approaches a finite number, $\pm\infty$, or does not exist. (If the limit does not exist, write DNE)

\input{Limit-Compute-0005.HELP.tex}

\[\lim_{x\to{-2}}\dfrac{x^{2} - 3 \, x - 10}{x^{2} + 8 \, x + 12}=\answer{-\frac{7}{4}}\]
\end{problem}}%}

\latexProblemContent{
\ifVerboseLocation This is Derivative Compute Question 0005. \\ \fi
\begin{problem}

Determine if the limit approaches a finite number, $\pm\infty$, or does not exist. (If the limit does not exist, write DNE)

\input{Limit-Compute-0005.HELP.tex}

\[\lim_{x\to{2}}\dfrac{x^{2} - 4 \, x + 4}{x^{2} - 6 \, x + 8}=\answer{0}\]
\end{problem}}%}

\latexProblemContent{
\ifVerboseLocation This is Derivative Compute Question 0005. \\ \fi
\begin{problem}

Determine if the limit approaches a finite number, $\pm\infty$, or does not exist. (If the limit does not exist, write DNE)

\input{Limit-Compute-0005.HELP.tex}

\[\lim_{x\to{4}}\dfrac{x^{2} - 5 \, x + 4}{x^{2} - 16}=\answer{\frac{3}{8}}\]
\end{problem}}%}

\latexProblemContent{
\ifVerboseLocation This is Derivative Compute Question 0005. \\ \fi
\begin{problem}

Determine if the limit approaches a finite number, $\pm\infty$, or does not exist. (If the limit does not exist, write DNE)

\input{Limit-Compute-0005.HELP.tex}

\[\lim_{x\to{-4}}\dfrac{x^{2} - 16}{x^{2} - 2 \, x - 24}=\answer{\frac{4}{5}}\]
\end{problem}}%}

\latexProblemContent{
\ifVerboseLocation This is Derivative Compute Question 0005. \\ \fi
\begin{problem}

Determine if the limit approaches a finite number, $\pm\infty$, or does not exist. (If the limit does not exist, write DNE)

\input{Limit-Compute-0005.HELP.tex}

\[\lim_{x\to{6}}\dfrac{x^{2} - 3 \, x - 18}{x^{2} - 9 \, x + 18}=\answer{3}\]
\end{problem}}%}

\latexProblemContent{
\ifVerboseLocation This is Derivative Compute Question 0005. \\ \fi
\begin{problem}

Determine if the limit approaches a finite number, $\pm\infty$, or does not exist. (If the limit does not exist, write DNE)

\input{Limit-Compute-0005.HELP.tex}

\[\lim_{x\to{-3}}\dfrac{x^{2} + 6 \, x + 9}{x^{2} + 8 \, x + 15}=\answer{0}\]
\end{problem}}%}

\latexProblemContent{
\ifVerboseLocation This is Derivative Compute Question 0005. \\ \fi
\begin{problem}

Determine if the limit approaches a finite number, $\pm\infty$, or does not exist. (If the limit does not exist, write DNE)

\input{Limit-Compute-0005.HELP.tex}

\[\lim_{x\to{6}}\dfrac{x^{2} - 5 \, x - 6}{x^{2} - 36}=\answer{\frac{7}{12}}\]
\end{problem}}%}

\latexProblemContent{
\ifVerboseLocation This is Derivative Compute Question 0005. \\ \fi
\begin{problem}

Determine if the limit approaches a finite number, $\pm\infty$, or does not exist. (If the limit does not exist, write DNE)

\input{Limit-Compute-0005.HELP.tex}

\[\lim_{x\to{-6}}\dfrac{x^{2} + 8 \, x + 12}{x^{2} + 10 \, x + 24}=\answer{2}\]
\end{problem}}%}

\latexProblemContent{
\ifVerboseLocation This is Derivative Compute Question 0005. \\ \fi
\begin{problem}

Determine if the limit approaches a finite number, $\pm\infty$, or does not exist. (If the limit does not exist, write DNE)

\input{Limit-Compute-0005.HELP.tex}

\[\lim_{x\to{1}}\dfrac{x^{2} + x - 2}{x^{2} - 3 \, x + 2}=\answer{-3}\]
\end{problem}}%}

\latexProblemContent{
\ifVerboseLocation This is Derivative Compute Question 0005. \\ \fi
\begin{problem}

Determine if the limit approaches a finite number, $\pm\infty$, or does not exist. (If the limit does not exist, write DNE)

\input{Limit-Compute-0005.HELP.tex}

\[\lim_{x\to{5}}\dfrac{x^{2} - 2 \, x - 15}{x^{2} - 6 \, x + 5}=\answer{2}\]
\end{problem}}%}

\latexProblemContent{
\ifVerboseLocation This is Derivative Compute Question 0005. \\ \fi
\begin{problem}

Determine if the limit approaches a finite number, $\pm\infty$, or does not exist. (If the limit does not exist, write DNE)

\input{Limit-Compute-0005.HELP.tex}

\[\lim_{x\to{-5}}\dfrac{x^{2} + 8 \, x + 15}{x^{2} + 3 \, x - 10}=\answer{\frac{2}{7}}\]
\end{problem}}%}

\latexProblemContent{
\ifVerboseLocation This is Derivative Compute Question 0005. \\ \fi
\begin{problem}

Determine if the limit approaches a finite number, $\pm\infty$, or does not exist. (If the limit does not exist, write DNE)

\input{Limit-Compute-0005.HELP.tex}

\[\lim_{x\to{5}}\dfrac{x^{2} - 7 \, x + 10}{x^{2} - 6 \, x + 5}=\answer{\frac{3}{4}}\]
\end{problem}}%}

\latexProblemContent{
\ifVerboseLocation This is Derivative Compute Question 0005. \\ \fi
\begin{problem}

Determine if the limit approaches a finite number, $\pm\infty$, or does not exist. (If the limit does not exist, write DNE)

\input{Limit-Compute-0005.HELP.tex}

\[\lim_{x\to{4}}\dfrac{x^{2} - 6 \, x + 8}{x^{2} + x - 20}=\answer{\frac{2}{9}}\]
\end{problem}}%}

\latexProblemContent{
\ifVerboseLocation This is Derivative Compute Question 0005. \\ \fi
\begin{problem}

Determine if the limit approaches a finite number, $\pm\infty$, or does not exist. (If the limit does not exist, write DNE)

\input{Limit-Compute-0005.HELP.tex}

\[\lim_{x\to{5}}\dfrac{x^{2} - 4 \, x - 5}{x^{2} - 4 \, x - 5}=\answer{1}\]
\end{problem}}%}

\latexProblemContent{
\ifVerboseLocation This is Derivative Compute Question 0005. \\ \fi
\begin{problem}

Determine if the limit approaches a finite number, $\pm\infty$, or does not exist. (If the limit does not exist, write DNE)

\input{Limit-Compute-0005.HELP.tex}

\[\lim_{x\to{-5}}\dfrac{x^{2} - 25}{x^{2} - 25}=\answer{1}\]
\end{problem}}%}

\latexProblemContent{
\ifVerboseLocation This is Derivative Compute Question 0005. \\ \fi
\begin{problem}

Determine if the limit approaches a finite number, $\pm\infty$, or does not exist. (If the limit does not exist, write DNE)

\input{Limit-Compute-0005.HELP.tex}

\[\lim_{x\to{4}}\dfrac{x^{2} - 2 \, x - 8}{x^{2} - 2 \, x - 8}=\answer{1}\]
\end{problem}}%}

\latexProblemContent{
\ifVerboseLocation This is Derivative Compute Question 0005. \\ \fi
\begin{problem}

Determine if the limit approaches a finite number, $\pm\infty$, or does not exist. (If the limit does not exist, write DNE)

\input{Limit-Compute-0005.HELP.tex}

\[\lim_{x\to{-6}}\dfrac{x^{2} + 8 \, x + 12}{x^{2} + 9 \, x + 18}=\answer{\frac{4}{3}}\]
\end{problem}}%}

\latexProblemContent{
\ifVerboseLocation This is Derivative Compute Question 0005. \\ \fi
\begin{problem}

Determine if the limit approaches a finite number, $\pm\infty$, or does not exist. (If the limit does not exist, write DNE)

\input{Limit-Compute-0005.HELP.tex}

\[\lim_{x\to{-5}}\dfrac{x^{2} + 3 \, x - 10}{x^{2} + 3 \, x - 10}=\answer{1}\]
\end{problem}}%}

\latexProblemContent{
\ifVerboseLocation This is Derivative Compute Question 0005. \\ \fi
\begin{problem}

Determine if the limit approaches a finite number, $\pm\infty$, or does not exist. (If the limit does not exist, write DNE)

\input{Limit-Compute-0005.HELP.tex}

\[\lim_{x\to{5}}\dfrac{x^{2} - 2 \, x - 15}{x^{2} - 2 \, x - 15}=\answer{1}\]
\end{problem}}%}

\latexProblemContent{
\ifVerboseLocation This is Derivative Compute Question 0005. \\ \fi
\begin{problem}

Determine if the limit approaches a finite number, $\pm\infty$, or does not exist. (If the limit does not exist, write DNE)

\input{Limit-Compute-0005.HELP.tex}

\[\lim_{x\to{-2}}\dfrac{x^{2} + 5 \, x + 6}{x^{2} + 7 \, x + 10}=\answer{\frac{1}{3}}\]
\end{problem}}%}

\latexProblemContent{
\ifVerboseLocation This is Derivative Compute Question 0005. \\ \fi
\begin{problem}

Determine if the limit approaches a finite number, $\pm\infty$, or does not exist. (If the limit does not exist, write DNE)

\input{Limit-Compute-0005.HELP.tex}

\[\lim_{x\to{-1}}\dfrac{x^{2} + 6 \, x + 5}{x^{2} - 5 \, x - 6}=\answer{-\frac{4}{7}}\]
\end{problem}}%}

\latexProblemContent{
\ifVerboseLocation This is Derivative Compute Question 0005. \\ \fi
\begin{problem}

Determine if the limit approaches a finite number, $\pm\infty$, or does not exist. (If the limit does not exist, write DNE)

\input{Limit-Compute-0005.HELP.tex}

\[\lim_{x\to{-1}}\dfrac{x^{2} + 5 \, x + 4}{x^{2} + 7 \, x + 6}=\answer{\frac{3}{5}}\]
\end{problem}}%}

\latexProblemContent{
\ifVerboseLocation This is Derivative Compute Question 0005. \\ \fi
\begin{problem}

Determine if the limit approaches a finite number, $\pm\infty$, or does not exist. (If the limit does not exist, write DNE)

\input{Limit-Compute-0005.HELP.tex}

\[\lim_{x\to{-1}}\dfrac{x^{2} + 3 \, x + 2}{x^{2} - 3 \, x - 4}=\answer{-\frac{1}{5}}\]
\end{problem}}%}

\latexProblemContent{
\ifVerboseLocation This is Derivative Compute Question 0005. \\ \fi
\begin{problem}

Determine if the limit approaches a finite number, $\pm\infty$, or does not exist. (If the limit does not exist, write DNE)

\input{Limit-Compute-0005.HELP.tex}

\[\lim_{x\to{-3}}\dfrac{x^{2} - x - 12}{x^{2} + 4 \, x + 3}=\answer{\frac{7}{2}}\]
\end{problem}}%}

\latexProblemContent{
\ifVerboseLocation This is Derivative Compute Question 0005. \\ \fi
\begin{problem}

Determine if the limit approaches a finite number, $\pm\infty$, or does not exist. (If the limit does not exist, write DNE)

\input{Limit-Compute-0005.HELP.tex}

\[\lim_{x\to{3}}\dfrac{x^{2} - 5 \, x + 6}{x^{2} - 8 \, x + 15}=\answer{-\frac{1}{2}}\]
\end{problem}}%}

\latexProblemContent{
\ifVerboseLocation This is Derivative Compute Question 0005. \\ \fi
\begin{problem}

Determine if the limit approaches a finite number, $\pm\infty$, or does not exist. (If the limit does not exist, write DNE)

\input{Limit-Compute-0005.HELP.tex}

\[\lim_{x\to{2}}\dfrac{x^{2} - 5 \, x + 6}{x^{2} + 4 \, x - 12}=\answer{-\frac{1}{8}}\]
\end{problem}}%}

\latexProblemContent{
\ifVerboseLocation This is Derivative Compute Question 0005. \\ \fi
\begin{problem}

Determine if the limit approaches a finite number, $\pm\infty$, or does not exist. (If the limit does not exist, write DNE)

\input{Limit-Compute-0005.HELP.tex}

\[\lim_{x\to{5}}\dfrac{x^{2} - 9 \, x + 20}{x^{2} - 3 \, x - 10}=\answer{\frac{1}{7}}\]
\end{problem}}%}

\latexProblemContent{
\ifVerboseLocation This is Derivative Compute Question 0005. \\ \fi
\begin{problem}

Determine if the limit approaches a finite number, $\pm\infty$, or does not exist. (If the limit does not exist, write DNE)

\input{Limit-Compute-0005.HELP.tex}

\[\lim_{x\to{4}}\dfrac{x^{2} - 5 \, x + 4}{x^{2} - 9 \, x + 20}=\answer{-3}\]
\end{problem}}%}

\latexProblemContent{
\ifVerboseLocation This is Derivative Compute Question 0005. \\ \fi
\begin{problem}

Determine if the limit approaches a finite number, $\pm\infty$, or does not exist. (If the limit does not exist, write DNE)

\input{Limit-Compute-0005.HELP.tex}

\[\lim_{x\to{-3}}\dfrac{x^{2} + 5 \, x + 6}{x^{2} + 7 \, x + 12}=\answer{-1}\]
\end{problem}}%}

\latexProblemContent{
\ifVerboseLocation This is Derivative Compute Question 0005. \\ \fi
\begin{problem}

Determine if the limit approaches a finite number, $\pm\infty$, or does not exist. (If the limit does not exist, write DNE)

\input{Limit-Compute-0005.HELP.tex}

\[\lim_{x\to{5}}\dfrac{x^{2} - 7 \, x + 10}{x^{2} - 4 \, x - 5}=\answer{\frac{1}{2}}\]
\end{problem}}%}

\latexProblemContent{
\ifVerboseLocation This is Derivative Compute Question 0005. \\ \fi
\begin{problem}

Determine if the limit approaches a finite number, $\pm\infty$, or does not exist. (If the limit does not exist, write DNE)

\input{Limit-Compute-0005.HELP.tex}

\[\lim_{x\to{-1}}\dfrac{x^{2} + 6 \, x + 5}{x^{2} + 4 \, x + 3}=\answer{2}\]
\end{problem}}%}

\latexProblemContent{
\ifVerboseLocation This is Derivative Compute Question 0005. \\ \fi
\begin{problem}

Determine if the limit approaches a finite number, $\pm\infty$, or does not exist. (If the limit does not exist, write DNE)

\input{Limit-Compute-0005.HELP.tex}

\[\lim_{x\to{-5}}\dfrac{x^{2} + 4 \, x - 5}{x^{2} + 8 \, x + 15}=\answer{3}\]
\end{problem}}%}

\latexProblemContent{
\ifVerboseLocation This is Derivative Compute Question 0005. \\ \fi
\begin{problem}

Determine if the limit approaches a finite number, $\pm\infty$, or does not exist. (If the limit does not exist, write DNE)

\input{Limit-Compute-0005.HELP.tex}

\[\lim_{x\to{5}}\dfrac{x^{2} - 8 \, x + 15}{x^{2} + x - 30}=\answer{\frac{2}{11}}\]
\end{problem}}%}

\latexProblemContent{
\ifVerboseLocation This is Derivative Compute Question 0005. \\ \fi
\begin{problem}

Determine if the limit approaches a finite number, $\pm\infty$, or does not exist. (If the limit does not exist, write DNE)

\input{Limit-Compute-0005.HELP.tex}

\[\lim_{x\to{3}}\dfrac{x^{2} - 2 \, x - 3}{x^{2} - 2 \, x - 3}=\answer{1}\]
\end{problem}}%}

\latexProblemContent{
\ifVerboseLocation This is Derivative Compute Question 0005. \\ \fi
\begin{problem}

Determine if the limit approaches a finite number, $\pm\infty$, or does not exist. (If the limit does not exist, write DNE)

\input{Limit-Compute-0005.HELP.tex}

\[\lim_{x\to{-6}}\dfrac{x^{2} + x - 30}{x^{2} + 9 \, x + 18}=\answer{\frac{11}{3}}\]
\end{problem}}%}

\latexProblemContent{
\ifVerboseLocation This is Derivative Compute Question 0005. \\ \fi
\begin{problem}

Determine if the limit approaches a finite number, $\pm\infty$, or does not exist. (If the limit does not exist, write DNE)

\input{Limit-Compute-0005.HELP.tex}

\[\lim_{x\to{-2}}\dfrac{x^{2} + x - 2}{x^{2} - 4}=\answer{\frac{3}{4}}\]
\end{problem}}%}

\latexProblemContent{
\ifVerboseLocation This is Derivative Compute Question 0005. \\ \fi
\begin{problem}

Determine if the limit approaches a finite number, $\pm\infty$, or does not exist. (If the limit does not exist, write DNE)

\input{Limit-Compute-0005.HELP.tex}

\[\lim_{x\to{5}}\dfrac{x^{2} - 10 \, x + 25}{x^{2} + x - 30}=\answer{0}\]
\end{problem}}%}

\latexProblemContent{
\ifVerboseLocation This is Derivative Compute Question 0005. \\ \fi
\begin{problem}

Determine if the limit approaches a finite number, $\pm\infty$, or does not exist. (If the limit does not exist, write DNE)

\input{Limit-Compute-0005.HELP.tex}

\[\lim_{x\to{-6}}\dfrac{x^{2} + 3 \, x - 18}{x^{2} + 4 \, x - 12}=\answer{\frac{9}{8}}\]
\end{problem}}%}

\latexProblemContent{
\ifVerboseLocation This is Derivative Compute Question 0005. \\ \fi
\begin{problem}

Determine if the limit approaches a finite number, $\pm\infty$, or does not exist. (If the limit does not exist, write DNE)

\input{Limit-Compute-0005.HELP.tex}

\[\lim_{x\to{3}}\dfrac{x^{2} - x - 6}{x^{2} - 2 \, x - 3}=\answer{\frac{5}{4}}\]
\end{problem}}%}

\latexProblemContent{
\ifVerboseLocation This is Derivative Compute Question 0005. \\ \fi
\begin{problem}

Determine if the limit approaches a finite number, $\pm\infty$, or does not exist. (If the limit does not exist, write DNE)

\input{Limit-Compute-0005.HELP.tex}

\[\lim_{x\to{1}}\dfrac{x^{2} + 3 \, x - 4}{x^{2} - 5 \, x + 4}=\answer{-\frac{5}{3}}\]
\end{problem}}%}

\latexProblemContent{
\ifVerboseLocation This is Derivative Compute Question 0005. \\ \fi
\begin{problem}

Determine if the limit approaches a finite number, $\pm\infty$, or does not exist. (If the limit does not exist, write DNE)

\input{Limit-Compute-0005.HELP.tex}

\[\lim_{x\to{-4}}\dfrac{x^{2} + x - 12}{x^{2} - 16}=\answer{\frac{7}{8}}\]
\end{problem}}%}

\latexProblemContent{
\ifVerboseLocation This is Derivative Compute Question 0005. \\ \fi
\begin{problem}

Determine if the limit approaches a finite number, $\pm\infty$, or does not exist. (If the limit does not exist, write DNE)

\input{Limit-Compute-0005.HELP.tex}

\[\lim_{x\to{3}}\dfrac{x^{2} - 5 \, x + 6}{x^{2} - 9 \, x + 18}=\answer{-\frac{1}{3}}\]
\end{problem}}%}

\latexProblemContent{
\ifVerboseLocation This is Derivative Compute Question 0005. \\ \fi
\begin{problem}

Determine if the limit approaches a finite number, $\pm\infty$, or does not exist. (If the limit does not exist, write DNE)

\input{Limit-Compute-0005.HELP.tex}

\[\lim_{x\to{2}}\dfrac{x^{2} - 4 \, x + 4}{x^{2} - 4}=\answer{0}\]
\end{problem}}%}

\latexProblemContent{
\ifVerboseLocation This is Derivative Compute Question 0005. \\ \fi
\begin{problem}

Determine if the limit approaches a finite number, $\pm\infty$, or does not exist. (If the limit does not exist, write DNE)

\input{Limit-Compute-0005.HELP.tex}

\[\lim_{x\to{3}}\dfrac{x^{2} - 6 \, x + 9}{x^{2} - 7 \, x + 12}=\answer{0}\]
\end{problem}}%}

\latexProblemContent{
\ifVerboseLocation This is Derivative Compute Question 0005. \\ \fi
\begin{problem}

Determine if the limit approaches a finite number, $\pm\infty$, or does not exist. (If the limit does not exist, write DNE)

\input{Limit-Compute-0005.HELP.tex}

\[\lim_{x\to{6}}\dfrac{x^{2} - 2 \, x - 24}{x^{2} - 8 \, x + 12}=\answer{\frac{5}{2}}\]
\end{problem}}%}

\latexProblemContent{
\ifVerboseLocation This is Derivative Compute Question 0005. \\ \fi
\begin{problem}

Determine if the limit approaches a finite number, $\pm\infty$, or does not exist. (If the limit does not exist, write DNE)

\input{Limit-Compute-0005.HELP.tex}

\[\lim_{x\to{2}}\dfrac{x^{2} - 4 \, x + 4}{x^{2} + 3 \, x - 10}=\answer{0}\]
\end{problem}}%}

\latexProblemContent{
\ifVerboseLocation This is Derivative Compute Question 0005. \\ \fi
\begin{problem}

Determine if the limit approaches a finite number, $\pm\infty$, or does not exist. (If the limit does not exist, write DNE)

\input{Limit-Compute-0005.HELP.tex}

\[\lim_{x\to{4}}\dfrac{x^{2} - x - 12}{x^{2} - 6 \, x + 8}=\answer{\frac{7}{2}}\]
\end{problem}}%}

\latexProblemContent{
\ifVerboseLocation This is Derivative Compute Question 0005. \\ \fi
\begin{problem}

Determine if the limit approaches a finite number, $\pm\infty$, or does not exist. (If the limit does not exist, write DNE)

\input{Limit-Compute-0005.HELP.tex}

\[\lim_{x\to{-3}}\dfrac{x^{2} + 7 \, x + 12}{x^{2} - 3 \, x - 18}=\answer{-\frac{1}{9}}\]
\end{problem}}%}

\latexProblemContent{
\ifVerboseLocation This is Derivative Compute Question 0005. \\ \fi
\begin{problem}

Determine if the limit approaches a finite number, $\pm\infty$, or does not exist. (If the limit does not exist, write DNE)

\input{Limit-Compute-0005.HELP.tex}

\[\lim_{x\to{-2}}\dfrac{x^{2} - 3 \, x - 10}{x^{2} - 2 \, x - 8}=\answer{\frac{7}{6}}\]
\end{problem}}%}

\latexProblemContent{
\ifVerboseLocation This is Derivative Compute Question 0005. \\ \fi
\begin{problem}

Determine if the limit approaches a finite number, $\pm\infty$, or does not exist. (If the limit does not exist, write DNE)

\input{Limit-Compute-0005.HELP.tex}

\[\lim_{x\to{-4}}\dfrac{x^{2} + 3 \, x - 4}{x^{2} + 9 \, x + 20}=\answer{-5}\]
\end{problem}}%}

\latexProblemContent{
\ifVerboseLocation This is Derivative Compute Question 0005. \\ \fi
\begin{problem}

Determine if the limit approaches a finite number, $\pm\infty$, or does not exist. (If the limit does not exist, write DNE)

\input{Limit-Compute-0005.HELP.tex}

\[\lim_{x\to{-2}}\dfrac{x^{2} - 3 \, x - 10}{x^{2} - 3 \, x - 10}=\answer{1}\]
\end{problem}}%}

\latexProblemContent{
\ifVerboseLocation This is Derivative Compute Question 0005. \\ \fi
\begin{problem}

Determine if the limit approaches a finite number, $\pm\infty$, or does not exist. (If the limit does not exist, write DNE)

\input{Limit-Compute-0005.HELP.tex}

\[\lim_{x\to{-2}}\dfrac{x^{2} - 2 \, x - 8}{x^{2} - 4 \, x - 12}=\answer{\frac{3}{4}}\]
\end{problem}}%}

\latexProblemContent{
\ifVerboseLocation This is Derivative Compute Question 0005. \\ \fi
\begin{problem}

Determine if the limit approaches a finite number, $\pm\infty$, or does not exist. (If the limit does not exist, write DNE)

\input{Limit-Compute-0005.HELP.tex}

\[\lim_{x\to{2}}\dfrac{x^{2} - x - 2}{x^{2} + 3 \, x - 10}=\answer{\frac{3}{7}}\]
\end{problem}}%}

\latexProblemContent{
\ifVerboseLocation This is Derivative Compute Question 0005. \\ \fi
\begin{problem}

Determine if the limit approaches a finite number, $\pm\infty$, or does not exist. (If the limit does not exist, write DNE)

\input{Limit-Compute-0005.HELP.tex}

\[\lim_{x\to{-4}}\dfrac{x^{2} + 8 \, x + 16}{x^{2} + 9 \, x + 20}=\answer{0}\]
\end{problem}}%}

\latexProblemContent{
\ifVerboseLocation This is Derivative Compute Question 0005. \\ \fi
\begin{problem}

Determine if the limit approaches a finite number, $\pm\infty$, or does not exist. (If the limit does not exist, write DNE)

\input{Limit-Compute-0005.HELP.tex}

\[\lim_{x\to{-2}}\dfrac{x^{2} - x - 6}{x^{2} + 7 \, x + 10}=\answer{-\frac{5}{3}}\]
\end{problem}}%}

\latexProblemContent{
\ifVerboseLocation This is Derivative Compute Question 0005. \\ \fi
\begin{problem}

Determine if the limit approaches a finite number, $\pm\infty$, or does not exist. (If the limit does not exist, write DNE)

\input{Limit-Compute-0005.HELP.tex}

\[\lim_{x\to{5}}\dfrac{x^{2} - x - 20}{x^{2} - 2 \, x - 15}=\answer{\frac{9}{8}}\]
\end{problem}}%}

\latexProblemContent{
\ifVerboseLocation This is Derivative Compute Question 0005. \\ \fi
\begin{problem}

Determine if the limit approaches a finite number, $\pm\infty$, or does not exist. (If the limit does not exist, write DNE)

\input{Limit-Compute-0005.HELP.tex}

\[\lim_{x\to{-6}}\dfrac{x^{2} + 5 \, x - 6}{x^{2} + 10 \, x + 24}=\answer{\frac{7}{2}}\]
\end{problem}}%}

\latexProblemContent{
\ifVerboseLocation This is Derivative Compute Question 0005. \\ \fi
\begin{problem}

Determine if the limit approaches a finite number, $\pm\infty$, or does not exist. (If the limit does not exist, write DNE)

\input{Limit-Compute-0005.HELP.tex}

\[\lim_{x\to{1}}\dfrac{x^{2} + 4 \, x - 5}{x^{2} + 2 \, x - 3}=\answer{\frac{3}{2}}\]
\end{problem}}%}

\latexProblemContent{
\ifVerboseLocation This is Derivative Compute Question 0005. \\ \fi
\begin{problem}

Determine if the limit approaches a finite number, $\pm\infty$, or does not exist. (If the limit does not exist, write DNE)

\input{Limit-Compute-0005.HELP.tex}

\[\lim_{x\to{-6}}\dfrac{x^{2} + 3 \, x - 18}{x^{2} + 11 \, x + 30}=\answer{9}\]
\end{problem}}%}

\latexProblemContent{
\ifVerboseLocation This is Derivative Compute Question 0005. \\ \fi
\begin{problem}

Determine if the limit approaches a finite number, $\pm\infty$, or does not exist. (If the limit does not exist, write DNE)

\input{Limit-Compute-0005.HELP.tex}

\[\lim_{x\to{3}}\dfrac{x^{2} - x - 6}{x^{2} - 7 \, x + 12}=\answer{-5}\]
\end{problem}}%}

\latexProblemContent{
\ifVerboseLocation This is Derivative Compute Question 0005. \\ \fi
\begin{problem}

Determine if the limit approaches a finite number, $\pm\infty$, or does not exist. (If the limit does not exist, write DNE)

\input{Limit-Compute-0005.HELP.tex}

\[\lim_{x\to{4}}\dfrac{x^{2} - x - 12}{x^{2} - 7 \, x + 12}=\answer{7}\]
\end{problem}}%}

\latexProblemContent{
\ifVerboseLocation This is Derivative Compute Question 0005. \\ \fi
\begin{problem}

Determine if the limit approaches a finite number, $\pm\infty$, or does not exist. (If the limit does not exist, write DNE)

\input{Limit-Compute-0005.HELP.tex}

\[\lim_{x\to{6}}\dfrac{x^{2} - 2 \, x - 24}{x^{2} - 11 \, x + 30}=\answer{10}\]
\end{problem}}%}

\latexProblemContent{
\ifVerboseLocation This is Derivative Compute Question 0005. \\ \fi
\begin{problem}

Determine if the limit approaches a finite number, $\pm\infty$, or does not exist. (If the limit does not exist, write DNE)

\input{Limit-Compute-0005.HELP.tex}

\[\lim_{x\to{3}}\dfrac{x^{2} + 2 \, x - 15}{x^{2} + 3 \, x - 18}=\answer{\frac{8}{9}}\]
\end{problem}}%}

\latexProblemContent{
\ifVerboseLocation This is Derivative Compute Question 0005. \\ \fi
\begin{problem}

Determine if the limit approaches a finite number, $\pm\infty$, or does not exist. (If the limit does not exist, write DNE)

\input{Limit-Compute-0005.HELP.tex}

\[\lim_{x\to{2}}\dfrac{x^{2} + 2 \, x - 8}{x^{2} + 3 \, x - 10}=\answer{\frac{6}{7}}\]
\end{problem}}%}

\latexProblemContent{
\ifVerboseLocation This is Derivative Compute Question 0005. \\ \fi
\begin{problem}

Determine if the limit approaches a finite number, $\pm\infty$, or does not exist. (If the limit does not exist, write DNE)

\input{Limit-Compute-0005.HELP.tex}

\[\lim_{x\to{-2}}\dfrac{x^{2} + x - 2}{x^{2} + 8 \, x + 12}=\answer{-\frac{3}{4}}\]
\end{problem}}%}

\latexProblemContent{
\ifVerboseLocation This is Derivative Compute Question 0005. \\ \fi
\begin{problem}

Determine if the limit approaches a finite number, $\pm\infty$, or does not exist. (If the limit does not exist, write DNE)

\input{Limit-Compute-0005.HELP.tex}

\[\lim_{x\to{3}}\dfrac{x^{2} + 2 \, x - 15}{x^{2} - 5 \, x + 6}=\answer{8}\]
\end{problem}}%}

\latexProblemContent{
\ifVerboseLocation This is Derivative Compute Question 0005. \\ \fi
\begin{problem}

Determine if the limit approaches a finite number, $\pm\infty$, or does not exist. (If the limit does not exist, write DNE)

\input{Limit-Compute-0005.HELP.tex}

\[\lim_{x\to{2}}\dfrac{x^{2} + 2 \, x - 8}{x^{2} - 6 \, x + 8}=\answer{-3}\]
\end{problem}}%}

\latexProblemContent{
\ifVerboseLocation This is Derivative Compute Question 0005. \\ \fi
\begin{problem}

Determine if the limit approaches a finite number, $\pm\infty$, or does not exist. (If the limit does not exist, write DNE)

\input{Limit-Compute-0005.HELP.tex}

\[\lim_{x\to{6}}\dfrac{x^{2} - 2 \, x - 24}{x^{2} - 10 \, x + 24}=\answer{5}\]
\end{problem}}%}

\latexProblemContent{
\ifVerboseLocation This is Derivative Compute Question 0005. \\ \fi
\begin{problem}

Determine if the limit approaches a finite number, $\pm\infty$, or does not exist. (If the limit does not exist, write DNE)

\input{Limit-Compute-0005.HELP.tex}

\[\lim_{x\to{-4}}\dfrac{x^{2} + 7 \, x + 12}{x^{2} + 9 \, x + 20}=\answer{-1}\]
\end{problem}}%}

\latexProblemContent{
\ifVerboseLocation This is Derivative Compute Question 0005. \\ \fi
\begin{problem}

Determine if the limit approaches a finite number, $\pm\infty$, or does not exist. (If the limit does not exist, write DNE)

\input{Limit-Compute-0005.HELP.tex}

\[\lim_{x\to{1}}\dfrac{x^{2} - 1}{x^{2} + x - 2}=\answer{\frac{2}{3}}\]
\end{problem}}%}

\latexProblemContent{
\ifVerboseLocation This is Derivative Compute Question 0005. \\ \fi
\begin{problem}

Determine if the limit approaches a finite number, $\pm\infty$, or does not exist. (If the limit does not exist, write DNE)

\input{Limit-Compute-0005.HELP.tex}

\[\lim_{x\to{-3}}\dfrac{x^{2} - x - 12}{x^{2} - 2 \, x - 15}=\answer{\frac{7}{8}}\]
\end{problem}}%}

\latexProblemContent{
\ifVerboseLocation This is Derivative Compute Question 0005. \\ \fi
\begin{problem}

Determine if the limit approaches a finite number, $\pm\infty$, or does not exist. (If the limit does not exist, write DNE)

\input{Limit-Compute-0005.HELP.tex}

\[\lim_{x\to{-5}}\dfrac{x^{2} + 3 \, x - 10}{x^{2} + x - 20}=\answer{\frac{7}{9}}\]
\end{problem}}%}

\latexProblemContent{
\ifVerboseLocation This is Derivative Compute Question 0005. \\ \fi
\begin{problem}

Determine if the limit approaches a finite number, $\pm\infty$, or does not exist. (If the limit does not exist, write DNE)

\input{Limit-Compute-0005.HELP.tex}

\[\lim_{x\to{-1}}\dfrac{x^{2} - 3 \, x - 4}{x^{2} + 7 \, x + 6}=\answer{-1}\]
\end{problem}}%}

\latexProblemContent{
\ifVerboseLocation This is Derivative Compute Question 0005. \\ \fi
\begin{problem}

Determine if the limit approaches a finite number, $\pm\infty$, or does not exist. (If the limit does not exist, write DNE)

\input{Limit-Compute-0005.HELP.tex}

\[\lim_{x\to{-3}}\dfrac{x^{2} + x - 6}{x^{2} + 7 \, x + 12}=\answer{-5}\]
\end{problem}}%}

\latexProblemContent{
\ifVerboseLocation This is Derivative Compute Question 0005. \\ \fi
\begin{problem}

Determine if the limit approaches a finite number, $\pm\infty$, or does not exist. (If the limit does not exist, write DNE)

\input{Limit-Compute-0005.HELP.tex}

\[\lim_{x\to{-3}}\dfrac{x^{2} + 6 \, x + 9}{x^{2} + 2 \, x - 3}=\answer{0}\]
\end{problem}}%}

\latexProblemContent{
\ifVerboseLocation This is Derivative Compute Question 0005. \\ \fi
\begin{problem}

Determine if the limit approaches a finite number, $\pm\infty$, or does not exist. (If the limit does not exist, write DNE)

\input{Limit-Compute-0005.HELP.tex}

\[\lim_{x\to{-4}}\dfrac{x^{2} + 6 \, x + 8}{x^{2} - 2 \, x - 24}=\answer{\frac{1}{5}}\]
\end{problem}}%}

\latexProblemContent{
\ifVerboseLocation This is Derivative Compute Question 0005. \\ \fi
\begin{problem}

Determine if the limit approaches a finite number, $\pm\infty$, or does not exist. (If the limit does not exist, write DNE)

\input{Limit-Compute-0005.HELP.tex}

\[\lim_{x\to{3}}\dfrac{x^{2} + x - 12}{x^{2} - 4 \, x + 3}=\answer{\frac{7}{2}}\]
\end{problem}}%}

\latexProblemContent{
\ifVerboseLocation This is Derivative Compute Question 0005. \\ \fi
\begin{problem}

Determine if the limit approaches a finite number, $\pm\infty$, or does not exist. (If the limit does not exist, write DNE)

\input{Limit-Compute-0005.HELP.tex}

\[\lim_{x\to{2}}\dfrac{x^{2} - 6 \, x + 8}{x^{2} - x - 2}=\answer{-\frac{2}{3}}\]
\end{problem}}%}

\latexProblemContent{
\ifVerboseLocation This is Derivative Compute Question 0005. \\ \fi
\begin{problem}

Determine if the limit approaches a finite number, $\pm\infty$, or does not exist. (If the limit does not exist, write DNE)

\input{Limit-Compute-0005.HELP.tex}

\[\lim_{x\to{4}}\dfrac{x^{2} + x - 20}{x^{2} - 7 \, x + 12}=\answer{9}\]
\end{problem}}%}

\latexProblemContent{
\ifVerboseLocation This is Derivative Compute Question 0005. \\ \fi
\begin{problem}

Determine if the limit approaches a finite number, $\pm\infty$, or does not exist. (If the limit does not exist, write DNE)

\input{Limit-Compute-0005.HELP.tex}

\[\lim_{x\to{4}}\dfrac{x^{2} - 3 \, x - 4}{x^{2} - 10 \, x + 24}=\answer{-\frac{5}{2}}\]
\end{problem}}%}

\latexProblemContent{
\ifVerboseLocation This is Derivative Compute Question 0005. \\ \fi
\begin{problem}

Determine if the limit approaches a finite number, $\pm\infty$, or does not exist. (If the limit does not exist, write DNE)

\input{Limit-Compute-0005.HELP.tex}

\[\lim_{x\to{-2}}\dfrac{x^{2} + 7 \, x + 10}{x^{2} + 7 \, x + 10}=\answer{1}\]
\end{problem}}%}

\latexProblemContent{
\ifVerboseLocation This is Derivative Compute Question 0005. \\ \fi
\begin{problem}

Determine if the limit approaches a finite number, $\pm\infty$, or does not exist. (If the limit does not exist, write DNE)

\input{Limit-Compute-0005.HELP.tex}

\[\lim_{x\to{-6}}\dfrac{x^{2} + 7 \, x + 6}{x^{2} + 5 \, x - 6}=\answer{\frac{5}{7}}\]
\end{problem}}%}

\latexProblemContent{
\ifVerboseLocation This is Derivative Compute Question 0005. \\ \fi
\begin{problem}

Determine if the limit approaches a finite number, $\pm\infty$, or does not exist. (If the limit does not exist, write DNE)

\input{Limit-Compute-0005.HELP.tex}

\[\lim_{x\to{2}}\dfrac{x^{2} - 7 \, x + 10}{x^{2} - 4}=\answer{-\frac{3}{4}}\]
\end{problem}}%}

\latexProblemContent{
\ifVerboseLocation This is Derivative Compute Question 0005. \\ \fi
\begin{problem}

Determine if the limit approaches a finite number, $\pm\infty$, or does not exist. (If the limit does not exist, write DNE)

\input{Limit-Compute-0005.HELP.tex}

\[\lim_{x\to{4}}\dfrac{x^{2} - 7 \, x + 12}{x^{2} - 6 \, x + 8}=\answer{\frac{1}{2}}\]
\end{problem}}%}

\latexProblemContent{
\ifVerboseLocation This is Derivative Compute Question 0005. \\ \fi
\begin{problem}

Determine if the limit approaches a finite number, $\pm\infty$, or does not exist. (If the limit does not exist, write DNE)

\input{Limit-Compute-0005.HELP.tex}

\[\lim_{x\to{-1}}\dfrac{x^{2} + 3 \, x + 2}{x^{2} - 1}=\answer{-\frac{1}{2}}\]
\end{problem}}%}

\latexProblemContent{
\ifVerboseLocation This is Derivative Compute Question 0005. \\ \fi
\begin{problem}

Determine if the limit approaches a finite number, $\pm\infty$, or does not exist. (If the limit does not exist, write DNE)

\input{Limit-Compute-0005.HELP.tex}

\[\lim_{x\to{6}}\dfrac{x^{2} - 3 \, x - 18}{x^{2} - 2 \, x - 24}=\answer{\frac{9}{10}}\]
\end{problem}}%}

\latexProblemContent{
\ifVerboseLocation This is Derivative Compute Question 0005. \\ \fi
\begin{problem}

Determine if the limit approaches a finite number, $\pm\infty$, or does not exist. (If the limit does not exist, write DNE)

\input{Limit-Compute-0005.HELP.tex}

\[\lim_{x\to{2}}\dfrac{x^{2} - 3 \, x + 2}{x^{2} + 3 \, x - 10}=\answer{\frac{1}{7}}\]
\end{problem}}%}

\latexProblemContent{
\ifVerboseLocation This is Derivative Compute Question 0005. \\ \fi
\begin{problem}

Determine if the limit approaches a finite number, $\pm\infty$, or does not exist. (If the limit does not exist, write DNE)

\input{Limit-Compute-0005.HELP.tex}

\[\lim_{x\to{3}}\dfrac{x^{2} - 5 \, x + 6}{x^{2} + 3 \, x - 18}=\answer{\frac{1}{9}}\]
\end{problem}}%}

\latexProblemContent{
\ifVerboseLocation This is Derivative Compute Question 0005. \\ \fi
\begin{problem}

Determine if the limit approaches a finite number, $\pm\infty$, or does not exist. (If the limit does not exist, write DNE)

\input{Limit-Compute-0005.HELP.tex}

\[\lim_{x\to{-5}}\dfrac{x^{2} + 7 \, x + 10}{x^{2} + 2 \, x - 15}=\answer{\frac{3}{8}}\]
\end{problem}}%}

\latexProblemContent{
\ifVerboseLocation This is Derivative Compute Question 0005. \\ \fi
\begin{problem}

Determine if the limit approaches a finite number, $\pm\infty$, or does not exist. (If the limit does not exist, write DNE)

\input{Limit-Compute-0005.HELP.tex}

\[\lim_{x\to{-5}}\dfrac{x^{2} + 6 \, x + 5}{x^{2} - x - 30}=\answer{\frac{4}{11}}\]
\end{problem}}%}

\latexProblemContent{
\ifVerboseLocation This is Derivative Compute Question 0005. \\ \fi
\begin{problem}

Determine if the limit approaches a finite number, $\pm\infty$, or does not exist. (If the limit does not exist, write DNE)

\input{Limit-Compute-0005.HELP.tex}

\[\lim_{x\to{-2}}\dfrac{x^{2} + 4 \, x + 4}{x^{2} + 8 \, x + 12}=\answer{0}\]
\end{problem}}%}

\latexProblemContent{
\ifVerboseLocation This is Derivative Compute Question 0005. \\ \fi
\begin{problem}

Determine if the limit approaches a finite number, $\pm\infty$, or does not exist. (If the limit does not exist, write DNE)

\input{Limit-Compute-0005.HELP.tex}

\[\lim_{x\to{1}}\dfrac{x^{2} + 3 \, x - 4}{x^{2} - 6 \, x + 5}=\answer{-\frac{5}{4}}\]
\end{problem}}%}

\latexProblemContent{
\ifVerboseLocation This is Derivative Compute Question 0005. \\ \fi
\begin{problem}

Determine if the limit approaches a finite number, $\pm\infty$, or does not exist. (If the limit does not exist, write DNE)

\input{Limit-Compute-0005.HELP.tex}

\[\lim_{x\to{4}}\dfrac{x^{2} - x - 12}{x^{2} - x - 12}=\answer{1}\]
\end{problem}}%}

\latexProblemContent{
\ifVerboseLocation This is Derivative Compute Question 0005. \\ \fi
\begin{problem}

Determine if the limit approaches a finite number, $\pm\infty$, or does not exist. (If the limit does not exist, write DNE)

\input{Limit-Compute-0005.HELP.tex}

\[\lim_{x\to{4}}\dfrac{x^{2} - 2 \, x - 8}{x^{2} - 9 \, x + 20}=\answer{-6}\]
\end{problem}}%}

\latexProblemContent{
\ifVerboseLocation This is Derivative Compute Question 0005. \\ \fi
\begin{problem}

Determine if the limit approaches a finite number, $\pm\infty$, or does not exist. (If the limit does not exist, write DNE)

\input{Limit-Compute-0005.HELP.tex}

\[\lim_{x\to{-1}}\dfrac{x^{2} + 5 \, x + 4}{x^{2} + 5 \, x + 4}=\answer{1}\]
\end{problem}}%}

\latexProblemContent{
\ifVerboseLocation This is Derivative Compute Question 0005. \\ \fi
\begin{problem}

Determine if the limit approaches a finite number, $\pm\infty$, or does not exist. (If the limit does not exist, write DNE)

\input{Limit-Compute-0005.HELP.tex}

\[\lim_{x\to{-3}}\dfrac{x^{2} + 5 \, x + 6}{x^{2} + x - 6}=\answer{\frac{1}{5}}\]
\end{problem}}%}

\latexProblemContent{
\ifVerboseLocation This is Derivative Compute Question 0005. \\ \fi
\begin{problem}

Determine if the limit approaches a finite number, $\pm\infty$, or does not exist. (If the limit does not exist, write DNE)

\input{Limit-Compute-0005.HELP.tex}

\[\lim_{x\to{-1}}\dfrac{x^{2} + 2 \, x + 1}{x^{2} - x - 2}=\answer{0}\]
\end{problem}}%}

\latexProblemContent{
\ifVerboseLocation This is Derivative Compute Question 0005. \\ \fi
\begin{problem}

Determine if the limit approaches a finite number, $\pm\infty$, or does not exist. (If the limit does not exist, write DNE)

\input{Limit-Compute-0005.HELP.tex}

\[\lim_{x\to{5}}\dfrac{x^{2} - 7 \, x + 10}{x^{2} + x - 30}=\answer{\frac{3}{11}}\]
\end{problem}}%}

\latexProblemContent{
\ifVerboseLocation This is Derivative Compute Question 0005. \\ \fi
\begin{problem}

Determine if the limit approaches a finite number, $\pm\infty$, or does not exist. (If the limit does not exist, write DNE)

\input{Limit-Compute-0005.HELP.tex}

\[\lim_{x\to{-1}}\dfrac{x^{2} - 1}{x^{2} - 5 \, x - 6}=\answer{\frac{2}{7}}\]
\end{problem}}%}

\latexProblemContent{
\ifVerboseLocation This is Derivative Compute Question 0005. \\ \fi
\begin{problem}

Determine if the limit approaches a finite number, $\pm\infty$, or does not exist. (If the limit does not exist, write DNE)

\input{Limit-Compute-0005.HELP.tex}

\[\lim_{x\to{6}}\dfrac{x^{2} - 8 \, x + 12}{x^{2} - x - 30}=\answer{\frac{4}{11}}\]
\end{problem}}%}

\latexProblemContent{
\ifVerboseLocation This is Derivative Compute Question 0005. \\ \fi
\begin{problem}

Determine if the limit approaches a finite number, $\pm\infty$, or does not exist. (If the limit does not exist, write DNE)

\input{Limit-Compute-0005.HELP.tex}

\[\lim_{x\to{-6}}\dfrac{x^{2} + 10 \, x + 24}{x^{2} + x - 30}=\answer{\frac{2}{11}}\]
\end{problem}}%}

\latexProblemContent{
\ifVerboseLocation This is Derivative Compute Question 0005. \\ \fi
\begin{problem}

Determine if the limit approaches a finite number, $\pm\infty$, or does not exist. (If the limit does not exist, write DNE)

\input{Limit-Compute-0005.HELP.tex}

\[\lim_{x\to{4}}\dfrac{x^{2} - 6 \, x + 8}{x^{2} - 5 \, x + 4}=\answer{\frac{2}{3}}\]
\end{problem}}%}

\latexProblemContent{
\ifVerboseLocation This is Derivative Compute Question 0005. \\ \fi
\begin{problem}

Determine if the limit approaches a finite number, $\pm\infty$, or does not exist. (If the limit does not exist, write DNE)

\input{Limit-Compute-0005.HELP.tex}

\[\lim_{x\to{6}}\dfrac{x^{2} - 9 \, x + 18}{x^{2} - 8 \, x + 12}=\answer{\frac{3}{4}}\]
\end{problem}}%}

\latexProblemContent{
\ifVerboseLocation This is Derivative Compute Question 0005. \\ \fi
\begin{problem}

Determine if the limit approaches a finite number, $\pm\infty$, or does not exist. (If the limit does not exist, write DNE)

\input{Limit-Compute-0005.HELP.tex}

\[\lim_{x\to{5}}\dfrac{x^{2} - x - 20}{x^{2} - 7 \, x + 10}=\answer{3}\]
\end{problem}}%}

\latexProblemContent{
\ifVerboseLocation This is Derivative Compute Question 0005. \\ \fi
\begin{problem}

Determine if the limit approaches a finite number, $\pm\infty$, or does not exist. (If the limit does not exist, write DNE)

\input{Limit-Compute-0005.HELP.tex}

\[\lim_{x\to{-3}}\dfrac{x^{2} - 2 \, x - 15}{x^{2} + 7 \, x + 12}=\answer{-8}\]
\end{problem}}%}

\latexProblemContent{
\ifVerboseLocation This is Derivative Compute Question 0005. \\ \fi
\begin{problem}

Determine if the limit approaches a finite number, $\pm\infty$, or does not exist. (If the limit does not exist, write DNE)

\input{Limit-Compute-0005.HELP.tex}

\[\lim_{x\to{-4}}\dfrac{x^{2} + 7 \, x + 12}{x^{2} + 2 \, x - 8}=\answer{\frac{1}{6}}\]
\end{problem}}%}

\latexProblemContent{
\ifVerboseLocation This is Derivative Compute Question 0005. \\ \fi
\begin{problem}

Determine if the limit approaches a finite number, $\pm\infty$, or does not exist. (If the limit does not exist, write DNE)

\input{Limit-Compute-0005.HELP.tex}

\[\lim_{x\to{3}}\dfrac{x^{2} - 5 \, x + 6}{x^{2} - 7 \, x + 12}=\answer{-1}\]
\end{problem}}%}

\latexProblemContent{
\ifVerboseLocation This is Derivative Compute Question 0005. \\ \fi
\begin{problem}

Determine if the limit approaches a finite number, $\pm\infty$, or does not exist. (If the limit does not exist, write DNE)

\input{Limit-Compute-0005.HELP.tex}

\[\lim_{x\to{1}}\dfrac{x^{2} - 6 \, x + 5}{x^{2} - 3 \, x + 2}=\answer{4}\]
\end{problem}}%}

\latexProblemContent{
\ifVerboseLocation This is Derivative Compute Question 0005. \\ \fi
\begin{problem}

Determine if the limit approaches a finite number, $\pm\infty$, or does not exist. (If the limit does not exist, write DNE)

\input{Limit-Compute-0005.HELP.tex}

\[\lim_{x\to{-4}}\dfrac{x^{2} + x - 12}{x^{2} + 6 \, x + 8}=\answer{\frac{7}{2}}\]
\end{problem}}%}

\latexProblemContent{
\ifVerboseLocation This is Derivative Compute Question 0005. \\ \fi
\begin{problem}

Determine if the limit approaches a finite number, $\pm\infty$, or does not exist. (If the limit does not exist, write DNE)

\input{Limit-Compute-0005.HELP.tex}

\[\lim_{x\to{-2}}\dfrac{x^{2} - 2 \, x - 8}{x^{2} - 4}=\answer{\frac{3}{2}}\]
\end{problem}}%}

\latexProblemContent{
\ifVerboseLocation This is Derivative Compute Question 0005. \\ \fi
\begin{problem}

Determine if the limit approaches a finite number, $\pm\infty$, or does not exist. (If the limit does not exist, write DNE)

\input{Limit-Compute-0005.HELP.tex}

\[\lim_{x\to{5}}\dfrac{x^{2} - 2 \, x - 15}{x^{2} - 8 \, x + 15}=\answer{4}\]
\end{problem}}%}

\latexProblemContent{
\ifVerboseLocation This is Derivative Compute Question 0005. \\ \fi
\begin{problem}

Determine if the limit approaches a finite number, $\pm\infty$, or does not exist. (If the limit does not exist, write DNE)

\input{Limit-Compute-0005.HELP.tex}

\[\lim_{x\to{3}}\dfrac{x^{2} - 9}{x^{2} + 2 \, x - 15}=\answer{\frac{3}{4}}\]
\end{problem}}%}

\latexProblemContent{
\ifVerboseLocation This is Derivative Compute Question 0005. \\ \fi
\begin{problem}

Determine if the limit approaches a finite number, $\pm\infty$, or does not exist. (If the limit does not exist, write DNE)

\input{Limit-Compute-0005.HELP.tex}

\[\lim_{x\to{3}}\dfrac{x^{2} - 2 \, x - 3}{x^{2} - 5 \, x + 6}=\answer{4}\]
\end{problem}}%}

\latexProblemContent{
\ifVerboseLocation This is Derivative Compute Question 0005. \\ \fi
\begin{problem}

Determine if the limit approaches a finite number, $\pm\infty$, or does not exist. (If the limit does not exist, write DNE)

\input{Limit-Compute-0005.HELP.tex}

\[\lim_{x\to{-4}}\dfrac{x^{2} + 2 \, x - 8}{x^{2} + 5 \, x + 4}=\answer{2}\]
\end{problem}}%}

\latexProblemContent{
\ifVerboseLocation This is Derivative Compute Question 0005. \\ \fi
\begin{problem}

Determine if the limit approaches a finite number, $\pm\infty$, or does not exist. (If the limit does not exist, write DNE)

\input{Limit-Compute-0005.HELP.tex}

\[\lim_{x\to{-5}}\dfrac{x^{2} + 10 \, x + 25}{x^{2} + 4 \, x - 5}=\answer{0}\]
\end{problem}}%}

\latexProblemContent{
\ifVerboseLocation This is Derivative Compute Question 0005. \\ \fi
\begin{problem}

Determine if the limit approaches a finite number, $\pm\infty$, or does not exist. (If the limit does not exist, write DNE)

\input{Limit-Compute-0005.HELP.tex}

\[\lim_{x\to{2}}\dfrac{x^{2} - 3 \, x + 2}{x^{2} - 7 \, x + 10}=\answer{-\frac{1}{3}}\]
\end{problem}}%}

\latexProblemContent{
\ifVerboseLocation This is Derivative Compute Question 0005. \\ \fi
\begin{problem}

Determine if the limit approaches a finite number, $\pm\infty$, or does not exist. (If the limit does not exist, write DNE)

\input{Limit-Compute-0005.HELP.tex}

\[\lim_{x\to{4}}\dfrac{x^{2} - 8 \, x + 16}{x^{2} - 16}=\answer{0}\]
\end{problem}}%}

\latexProblemContent{
\ifVerboseLocation This is Derivative Compute Question 0005. \\ \fi
\begin{problem}

Determine if the limit approaches a finite number, $\pm\infty$, or does not exist. (If the limit does not exist, write DNE)

\input{Limit-Compute-0005.HELP.tex}

\[\lim_{x\to{-6}}\dfrac{x^{2} + 2 \, x - 24}{x^{2} + 9 \, x + 18}=\answer{\frac{10}{3}}\]
\end{problem}}%}

\latexProblemContent{
\ifVerboseLocation This is Derivative Compute Question 0005. \\ \fi
\begin{problem}

Determine if the limit approaches a finite number, $\pm\infty$, or does not exist. (If the limit does not exist, write DNE)

\input{Limit-Compute-0005.HELP.tex}

\[\lim_{x\to{-3}}\dfrac{x^{2} - 9}{x^{2} + 2 \, x - 3}=\answer{\frac{3}{2}}\]
\end{problem}}%}

\latexProblemContent{
\ifVerboseLocation This is Derivative Compute Question 0005. \\ \fi
\begin{problem}

Determine if the limit approaches a finite number, $\pm\infty$, or does not exist. (If the limit does not exist, write DNE)

\input{Limit-Compute-0005.HELP.tex}

\[\lim_{x\to{5}}\dfrac{x^{2} - 3 \, x - 10}{x^{2} - 3 \, x - 10}=\answer{1}\]
\end{problem}}%}

\latexProblemContent{
\ifVerboseLocation This is Derivative Compute Question 0005. \\ \fi
\begin{problem}

Determine if the limit approaches a finite number, $\pm\infty$, or does not exist. (If the limit does not exist, write DNE)

\input{Limit-Compute-0005.HELP.tex}

\[\lim_{x\to{-4}}\dfrac{x^{2} + 9 \, x + 20}{x^{2} + 10 \, x + 24}=\answer{\frac{1}{2}}\]
\end{problem}}%}

\latexProblemContent{
\ifVerboseLocation This is Derivative Compute Question 0005. \\ \fi
\begin{problem}

Determine if the limit approaches a finite number, $\pm\infty$, or does not exist. (If the limit does not exist, write DNE)

\input{Limit-Compute-0005.HELP.tex}

\[\lim_{x\to{6}}\dfrac{x^{2} - 10 \, x + 24}{x^{2} - 9 \, x + 18}=\answer{\frac{2}{3}}\]
\end{problem}}%}

\latexProblemContent{
\ifVerboseLocation This is Derivative Compute Question 0005. \\ \fi
\begin{problem}

Determine if the limit approaches a finite number, $\pm\infty$, or does not exist. (If the limit does not exist, write DNE)

\input{Limit-Compute-0005.HELP.tex}

\[\lim_{x\to{-2}}\dfrac{x^{2} + 3 \, x + 2}{x^{2} + 7 \, x + 10}=\answer{-\frac{1}{3}}\]
\end{problem}}%}

\latexProblemContent{
\ifVerboseLocation This is Derivative Compute Question 0005. \\ \fi
\begin{problem}

Determine if the limit approaches a finite number, $\pm\infty$, or does not exist. (If the limit does not exist, write DNE)

\input{Limit-Compute-0005.HELP.tex}

\[\lim_{x\to{-5}}\dfrac{x^{2} + 10 \, x + 25}{x^{2} + 7 \, x + 10}=\answer{0}\]
\end{problem}}%}

\latexProblemContent{
\ifVerboseLocation This is Derivative Compute Question 0005. \\ \fi
\begin{problem}

Determine if the limit approaches a finite number, $\pm\infty$, or does not exist. (If the limit does not exist, write DNE)

\input{Limit-Compute-0005.HELP.tex}

\[\lim_{x\to{3}}\dfrac{x^{2} - x - 6}{x^{2} - 9}=\answer{\frac{5}{6}}\]
\end{problem}}%}

\latexProblemContent{
\ifVerboseLocation This is Derivative Compute Question 0005. \\ \fi
\begin{problem}

Determine if the limit approaches a finite number, $\pm\infty$, or does not exist. (If the limit does not exist, write DNE)

\input{Limit-Compute-0005.HELP.tex}

\[\lim_{x\to{2}}\dfrac{x^{2} - x - 2}{x^{2} - 7 \, x + 10}=\answer{-1}\]
\end{problem}}%}

\latexProblemContent{
\ifVerboseLocation This is Derivative Compute Question 0005. \\ \fi
\begin{problem}

Determine if the limit approaches a finite number, $\pm\infty$, or does not exist. (If the limit does not exist, write DNE)

\input{Limit-Compute-0005.HELP.tex}

\[\lim_{x\to{-2}}\dfrac{x^{2} + x - 2}{x^{2} - x - 6}=\answer{\frac{3}{5}}\]
\end{problem}}%}

\latexProblemContent{
\ifVerboseLocation This is Derivative Compute Question 0005. \\ \fi
\begin{problem}

Determine if the limit approaches a finite number, $\pm\infty$, or does not exist. (If the limit does not exist, write DNE)

\input{Limit-Compute-0005.HELP.tex}

\[\lim_{x\to{6}}\dfrac{x^{2} - 11 \, x + 30}{x^{2} - x - 30}=\answer{\frac{1}{11}}\]
\end{problem}}%}

\latexProblemContent{
\ifVerboseLocation This is Derivative Compute Question 0005. \\ \fi
\begin{problem}

Determine if the limit approaches a finite number, $\pm\infty$, or does not exist. (If the limit does not exist, write DNE)

\input{Limit-Compute-0005.HELP.tex}

\[\lim_{x\to{2}}\dfrac{x^{2} + x - 6}{x^{2} - 4}=\answer{\frac{5}{4}}\]
\end{problem}}%}

\latexProblemContent{
\ifVerboseLocation This is Derivative Compute Question 0005. \\ \fi
\begin{problem}

Determine if the limit approaches a finite number, $\pm\infty$, or does not exist. (If the limit does not exist, write DNE)

\input{Limit-Compute-0005.HELP.tex}

\[\lim_{x\to{5}}\dfrac{x^{2} - 2 \, x - 15}{x^{2} - 25}=\answer{\frac{4}{5}}\]
\end{problem}}%}

\latexProblemContent{
\ifVerboseLocation This is Derivative Compute Question 0005. \\ \fi
\begin{problem}

Determine if the limit approaches a finite number, $\pm\infty$, or does not exist. (If the limit does not exist, write DNE)

\input{Limit-Compute-0005.HELP.tex}

\[\lim_{x\to{-2}}\dfrac{x^{2} - x - 6}{x^{2} - 4}=\answer{\frac{5}{4}}\]
\end{problem}}%}

\latexProblemContent{
\ifVerboseLocation This is Derivative Compute Question 0005. \\ \fi
\begin{problem}

Determine if the limit approaches a finite number, $\pm\infty$, or does not exist. (If the limit does not exist, write DNE)

\input{Limit-Compute-0005.HELP.tex}

\[\lim_{x\to{-4}}\dfrac{x^{2} + 6 \, x + 8}{x^{2} + 7 \, x + 12}=\answer{2}\]
\end{problem}}%}

\latexProblemContent{
\ifVerboseLocation This is Derivative Compute Question 0005. \\ \fi
\begin{problem}

Determine if the limit approaches a finite number, $\pm\infty$, or does not exist. (If the limit does not exist, write DNE)

\input{Limit-Compute-0005.HELP.tex}

\[\lim_{x\to{-4}}\dfrac{x^{2} + x - 12}{x^{2} + 9 \, x + 20}=\answer{-7}\]
\end{problem}}%}

\latexProblemContent{
\ifVerboseLocation This is Derivative Compute Question 0005. \\ \fi
\begin{problem}

Determine if the limit approaches a finite number, $\pm\infty$, or does not exist. (If the limit does not exist, write DNE)

\input{Limit-Compute-0005.HELP.tex}

\[\lim_{x\to{3}}\dfrac{x^{2} - 2 \, x - 3}{x^{2} - 7 \, x + 12}=\answer{-4}\]
\end{problem}}%}

\latexProblemContent{
\ifVerboseLocation This is Derivative Compute Question 0005. \\ \fi
\begin{problem}

Determine if the limit approaches a finite number, $\pm\infty$, or does not exist. (If the limit does not exist, write DNE)

\input{Limit-Compute-0005.HELP.tex}

\[\lim_{x\to{-6}}\dfrac{x^{2} + x - 30}{x^{2} + 7 \, x + 6}=\answer{\frac{11}{5}}\]
\end{problem}}%}

\latexProblemContent{
\ifVerboseLocation This is Derivative Compute Question 0005. \\ \fi
\begin{problem}

Determine if the limit approaches a finite number, $\pm\infty$, or does not exist. (If the limit does not exist, write DNE)

\input{Limit-Compute-0005.HELP.tex}

\[\lim_{x\to{4}}\dfrac{x^{2} - 3 \, x - 4}{x^{2} - 9 \, x + 20}=\answer{-5}\]
\end{problem}}%}

\latexProblemContent{
\ifVerboseLocation This is Derivative Compute Question 0005. \\ \fi
\begin{problem}

Determine if the limit approaches a finite number, $\pm\infty$, or does not exist. (If the limit does not exist, write DNE)

\input{Limit-Compute-0005.HELP.tex}

\[\lim_{x\to{2}}\dfrac{x^{2} + 3 \, x - 10}{x^{2} - 8 \, x + 12}=\answer{-\frac{7}{4}}\]
\end{problem}}%}

\latexProblemContent{
\ifVerboseLocation This is Derivative Compute Question 0005. \\ \fi
\begin{problem}

Determine if the limit approaches a finite number, $\pm\infty$, or does not exist. (If the limit does not exist, write DNE)

\input{Limit-Compute-0005.HELP.tex}

\[\lim_{x\to{-6}}\dfrac{x^{2} + 5 \, x - 6}{x^{2} + 2 \, x - 24}=\answer{\frac{7}{10}}\]
\end{problem}}%}

\latexProblemContent{
\ifVerboseLocation This is Derivative Compute Question 0005. \\ \fi
\begin{problem}

Determine if the limit approaches a finite number, $\pm\infty$, or does not exist. (If the limit does not exist, write DNE)

\input{Limit-Compute-0005.HELP.tex}

\[\lim_{x\to{5}}\dfrac{x^{2} - 2 \, x - 15}{x^{2} - 9 \, x + 20}=\answer{8}\]
\end{problem}}%}

\latexProblemContent{
\ifVerboseLocation This is Derivative Compute Question 0005. \\ \fi
\begin{problem}

Determine if the limit approaches a finite number, $\pm\infty$, or does not exist. (If the limit does not exist, write DNE)

\input{Limit-Compute-0005.HELP.tex}

\[\lim_{x\to{-2}}\dfrac{x^{2} + 3 \, x + 2}{x^{2} + x - 2}=\answer{\frac{1}{3}}\]
\end{problem}}%}

\latexProblemContent{
\ifVerboseLocation This is Derivative Compute Question 0005. \\ \fi
\begin{problem}

Determine if the limit approaches a finite number, $\pm\infty$, or does not exist. (If the limit does not exist, write DNE)

\input{Limit-Compute-0005.HELP.tex}

\[\lim_{x\to{3}}\dfrac{x^{2} - 5 \, x + 6}{x^{2} - 2 \, x - 3}=\answer{\frac{1}{4}}\]
\end{problem}}%}

\latexProblemContent{
\ifVerboseLocation This is Derivative Compute Question 0005. \\ \fi
\begin{problem}

Determine if the limit approaches a finite number, $\pm\infty$, or does not exist. (If the limit does not exist, write DNE)

\input{Limit-Compute-0005.HELP.tex}

\[\lim_{x\to{-3}}\dfrac{x^{2} + 4 \, x + 3}{x^{2} - 2 \, x - 15}=\answer{\frac{1}{4}}\]
\end{problem}}%}

\latexProblemContent{
\ifVerboseLocation This is Derivative Compute Question 0005. \\ \fi
\begin{problem}

Determine if the limit approaches a finite number, $\pm\infty$, or does not exist. (If the limit does not exist, write DNE)

\input{Limit-Compute-0005.HELP.tex}

\[\lim_{x\to{-1}}\dfrac{x^{2} - 3 \, x - 4}{x^{2} - 4 \, x - 5}=\answer{\frac{5}{6}}\]
\end{problem}}%}

\latexProblemContent{
\ifVerboseLocation This is Derivative Compute Question 0005. \\ \fi
\begin{problem}

Determine if the limit approaches a finite number, $\pm\infty$, or does not exist. (If the limit does not exist, write DNE)

\input{Limit-Compute-0005.HELP.tex}

\[\lim_{x\to{4}}\dfrac{x^{2} - 9 \, x + 20}{x^{2} + x - 20}=\answer{-\frac{1}{9}}\]
\end{problem}}%}

\latexProblemContent{
\ifVerboseLocation This is Derivative Compute Question 0005. \\ \fi
\begin{problem}

Determine if the limit approaches a finite number, $\pm\infty$, or does not exist. (If the limit does not exist, write DNE)

\input{Limit-Compute-0005.HELP.tex}

\[\lim_{x\to{-2}}\dfrac{x^{2} + 6 \, x + 8}{x^{2} + 7 \, x + 10}=\answer{\frac{2}{3}}\]
\end{problem}}%}

\latexProblemContent{
\ifVerboseLocation This is Derivative Compute Question 0005. \\ \fi
\begin{problem}

Determine if the limit approaches a finite number, $\pm\infty$, or does not exist. (If the limit does not exist, write DNE)

\input{Limit-Compute-0005.HELP.tex}

\[\lim_{x\to{6}}\dfrac{x^{2} - 7 \, x + 6}{x^{2} - 4 \, x - 12}=\answer{\frac{5}{8}}\]
\end{problem}}%}

\latexProblemContent{
\ifVerboseLocation This is Derivative Compute Question 0005. \\ \fi
\begin{problem}

Determine if the limit approaches a finite number, $\pm\infty$, or does not exist. (If the limit does not exist, write DNE)

\input{Limit-Compute-0005.HELP.tex}

\[\lim_{x\to{-6}}\dfrac{x^{2} + 10 \, x + 24}{x^{2} + 3 \, x - 18}=\answer{\frac{2}{9}}\]
\end{problem}}%}

\latexProblemContent{
\ifVerboseLocation This is Derivative Compute Question 0005. \\ \fi
\begin{problem}

Determine if the limit approaches a finite number, $\pm\infty$, or does not exist. (If the limit does not exist, write DNE)

\input{Limit-Compute-0005.HELP.tex}

\[\lim_{x\to{-5}}\dfrac{x^{2} + 2 \, x - 15}{x^{2} + 6 \, x + 5}=\answer{2}\]
\end{problem}}%}

\latexProblemContent{
\ifVerboseLocation This is Derivative Compute Question 0005. \\ \fi
\begin{problem}

Determine if the limit approaches a finite number, $\pm\infty$, or does not exist. (If the limit does not exist, write DNE)

\input{Limit-Compute-0005.HELP.tex}

\[\lim_{x\to{5}}\dfrac{x^{2} - 3 \, x - 10}{x^{2} + x - 30}=\answer{\frac{7}{11}}\]
\end{problem}}%}

\latexProblemContent{
\ifVerboseLocation This is Derivative Compute Question 0005. \\ \fi
\begin{problem}

Determine if the limit approaches a finite number, $\pm\infty$, or does not exist. (If the limit does not exist, write DNE)

\input{Limit-Compute-0005.HELP.tex}

\[\lim_{x\to{2}}\dfrac{x^{2} - 3 \, x + 2}{x^{2} + 4 \, x - 12}=\answer{\frac{1}{8}}\]
\end{problem}}%}

\latexProblemContent{
\ifVerboseLocation This is Derivative Compute Question 0005. \\ \fi
\begin{problem}

Determine if the limit approaches a finite number, $\pm\infty$, or does not exist. (If the limit does not exist, write DNE)

\input{Limit-Compute-0005.HELP.tex}

\[\lim_{x\to{-4}}\dfrac{x^{2} + 9 \, x + 20}{x^{2} - 16}=\answer{-\frac{1}{8}}\]
\end{problem}}%}

\latexProblemContent{
\ifVerboseLocation This is Derivative Compute Question 0005. \\ \fi
\begin{problem}

Determine if the limit approaches a finite number, $\pm\infty$, or does not exist. (If the limit does not exist, write DNE)

\input{Limit-Compute-0005.HELP.tex}

\[\lim_{x\to{5}}\dfrac{x^{2} - 9 \, x + 20}{x^{2} - 6 \, x + 5}=\answer{\frac{1}{4}}\]
\end{problem}}%}

\latexProblemContent{
\ifVerboseLocation This is Derivative Compute Question 0005. \\ \fi
\begin{problem}

Determine if the limit approaches a finite number, $\pm\infty$, or does not exist. (If the limit does not exist, write DNE)

\input{Limit-Compute-0005.HELP.tex}

\[\lim_{x\to{5}}\dfrac{x^{2} - 7 \, x + 10}{x^{2} - 11 \, x + 30}=\answer{-3}\]
\end{problem}}%}

\latexProblemContent{
\ifVerboseLocation This is Derivative Compute Question 0005. \\ \fi
\begin{problem}

Determine if the limit approaches a finite number, $\pm\infty$, or does not exist. (If the limit does not exist, write DNE)

\input{Limit-Compute-0005.HELP.tex}

\[\lim_{x\to{-6}}\dfrac{x^{2} + 2 \, x - 24}{x^{2} + 11 \, x + 30}=\answer{10}\]
\end{problem}}%}

\latexProblemContent{
\ifVerboseLocation This is Derivative Compute Question 0005. \\ \fi
\begin{problem}

Determine if the limit approaches a finite number, $\pm\infty$, or does not exist. (If the limit does not exist, write DNE)

\input{Limit-Compute-0005.HELP.tex}

\[\lim_{x\to{5}}\dfrac{x^{2} - 25}{x^{2} - 8 \, x + 15}=\answer{5}\]
\end{problem}}%}

\latexProblemContent{
\ifVerboseLocation This is Derivative Compute Question 0005. \\ \fi
\begin{problem}

Determine if the limit approaches a finite number, $\pm\infty$, or does not exist. (If the limit does not exist, write DNE)

\input{Limit-Compute-0005.HELP.tex}

\[\lim_{x\to{-1}}\dfrac{x^{2} - 2 \, x - 3}{x^{2} - 5 \, x - 6}=\answer{\frac{4}{7}}\]
\end{problem}}%}

\latexProblemContent{
\ifVerboseLocation This is Derivative Compute Question 0005. \\ \fi
\begin{problem}

Determine if the limit approaches a finite number, $\pm\infty$, or does not exist. (If the limit does not exist, write DNE)

\input{Limit-Compute-0005.HELP.tex}

\[\lim_{x\to{3}}\dfrac{x^{2} - 9}{x^{2} + 3 \, x - 18}=\answer{\frac{2}{3}}\]
\end{problem}}%}

\latexProblemContent{
\ifVerboseLocation This is Derivative Compute Question 0005. \\ \fi
\begin{problem}

Determine if the limit approaches a finite number, $\pm\infty$, or does not exist. (If the limit does not exist, write DNE)

\input{Limit-Compute-0005.HELP.tex}

\[\lim_{x\to{-1}}\dfrac{x^{2} - 4 \, x - 5}{x^{2} - 5 \, x - 6}=\answer{\frac{6}{7}}\]
\end{problem}}%}

\latexProblemContent{
\ifVerboseLocation This is Derivative Compute Question 0005. \\ \fi
\begin{problem}

Determine if the limit approaches a finite number, $\pm\infty$, or does not exist. (If the limit does not exist, write DNE)

\input{Limit-Compute-0005.HELP.tex}

\[\lim_{x\to{1}}\dfrac{x^{2} - 5 \, x + 4}{x^{2} + 5 \, x - 6}=\answer{-\frac{3}{7}}\]
\end{problem}}%}

\latexProblemContent{
\ifVerboseLocation This is Derivative Compute Question 0005. \\ \fi
\begin{problem}

Determine if the limit approaches a finite number, $\pm\infty$, or does not exist. (If the limit does not exist, write DNE)

\input{Limit-Compute-0005.HELP.tex}

\[\lim_{x\to{1}}\dfrac{x^{2} + 4 \, x - 5}{x^{2} + x - 2}=\answer{2}\]
\end{problem}}%}

\latexProblemContent{
\ifVerboseLocation This is Derivative Compute Question 0005. \\ \fi
\begin{problem}

Determine if the limit approaches a finite number, $\pm\infty$, or does not exist. (If the limit does not exist, write DNE)

\input{Limit-Compute-0005.HELP.tex}

\[\lim_{x\to{3}}\dfrac{x^{2} - 9}{x^{2} + x - 12}=\answer{\frac{6}{7}}\]
\end{problem}}%}

\latexProblemContent{
\ifVerboseLocation This is Derivative Compute Question 0005. \\ \fi
\begin{problem}

Determine if the limit approaches a finite number, $\pm\infty$, or does not exist. (If the limit does not exist, write DNE)

\input{Limit-Compute-0005.HELP.tex}

\[\lim_{x\to{-5}}\dfrac{x^{2} + 4 \, x - 5}{x^{2} + 11 \, x + 30}=\answer{-6}\]
\end{problem}}%}

\latexProblemContent{
\ifVerboseLocation This is Derivative Compute Question 0005. \\ \fi
\begin{problem}

Determine if the limit approaches a finite number, $\pm\infty$, or does not exist. (If the limit does not exist, write DNE)

\input{Limit-Compute-0005.HELP.tex}

\[\lim_{x\to{-6}}\dfrac{x^{2} + 9 \, x + 18}{x^{2} - 36}=\answer{\frac{1}{4}}\]
\end{problem}}%}

\latexProblemContent{
\ifVerboseLocation This is Derivative Compute Question 0005. \\ \fi
\begin{problem}

Determine if the limit approaches a finite number, $\pm\infty$, or does not exist. (If the limit does not exist, write DNE)

\input{Limit-Compute-0005.HELP.tex}

\[\lim_{x\to{3}}\dfrac{x^{2} - 7 \, x + 12}{x^{2} - 9 \, x + 18}=\answer{\frac{1}{3}}\]
\end{problem}}%}

\latexProblemContent{
\ifVerboseLocation This is Derivative Compute Question 0005. \\ \fi
\begin{problem}

Determine if the limit approaches a finite number, $\pm\infty$, or does not exist. (If the limit does not exist, write DNE)

\input{Limit-Compute-0005.HELP.tex}

\[\lim_{x\to{6}}\dfrac{x^{2} - 3 \, x - 18}{x^{2} - 7 \, x + 6}=\answer{\frac{9}{5}}\]
\end{problem}}%}

\latexProblemContent{
\ifVerboseLocation This is Derivative Compute Question 0005. \\ \fi
\begin{problem}

Determine if the limit approaches a finite number, $\pm\infty$, or does not exist. (If the limit does not exist, write DNE)

\input{Limit-Compute-0005.HELP.tex}

\[\lim_{x\to{-1}}\dfrac{x^{2} + 3 \, x + 2}{x^{2} + 5 \, x + 4}=\answer{\frac{1}{3}}\]
\end{problem}}%}

\latexProblemContent{
\ifVerboseLocation This is Derivative Compute Question 0005. \\ \fi
\begin{problem}

Determine if the limit approaches a finite number, $\pm\infty$, or does not exist. (If the limit does not exist, write DNE)

\input{Limit-Compute-0005.HELP.tex}

\[\lim_{x\to{2}}\dfrac{x^{2} - 7 \, x + 10}{x^{2} - 3 \, x + 2}=\answer{-3}\]
\end{problem}}%}

\latexProblemContent{
\ifVerboseLocation This is Derivative Compute Question 0005. \\ \fi
\begin{problem}

Determine if the limit approaches a finite number, $\pm\infty$, or does not exist. (If the limit does not exist, write DNE)

\input{Limit-Compute-0005.HELP.tex}

\[\lim_{x\to{-1}}\dfrac{x^{2} + 5 \, x + 4}{x^{2} - x - 2}=\answer{-1}\]
\end{problem}}%}

\latexProblemContent{
\ifVerboseLocation This is Derivative Compute Question 0005. \\ \fi
\begin{problem}

Determine if the limit approaches a finite number, $\pm\infty$, or does not exist. (If the limit does not exist, write DNE)

\input{Limit-Compute-0005.HELP.tex}

\[\lim_{x\to{6}}\dfrac{x^{2} - 4 \, x - 12}{x^{2} - 7 \, x + 6}=\answer{\frac{8}{5}}\]
\end{problem}}%}

\latexProblemContent{
\ifVerboseLocation This is Derivative Compute Question 0005. \\ \fi
\begin{problem}

Determine if the limit approaches a finite number, $\pm\infty$, or does not exist. (If the limit does not exist, write DNE)

\input{Limit-Compute-0005.HELP.tex}

\[\lim_{x\to{-1}}\dfrac{x^{2} - 2 \, x - 3}{x^{2} - 3 \, x - 4}=\answer{\frac{4}{5}}\]
\end{problem}}%}

\latexProblemContent{
\ifVerboseLocation This is Derivative Compute Question 0005. \\ \fi
\begin{problem}

Determine if the limit approaches a finite number, $\pm\infty$, or does not exist. (If the limit does not exist, write DNE)

\input{Limit-Compute-0005.HELP.tex}

\[\lim_{x\to{5}}\dfrac{x^{2} - 9 \, x + 20}{x^{2} - 7 \, x + 10}=\answer{\frac{1}{3}}\]
\end{problem}}%}

\latexProblemContent{
\ifVerboseLocation This is Derivative Compute Question 0005. \\ \fi
\begin{problem}

Determine if the limit approaches a finite number, $\pm\infty$, or does not exist. (If the limit does not exist, write DNE)

\input{Limit-Compute-0005.HELP.tex}

\[\lim_{x\to{-4}}\dfrac{x^{2} + 8 \, x + 16}{x^{2} + 7 \, x + 12}=\answer{0}\]
\end{problem}}%}

\latexProblemContent{
\ifVerboseLocation This is Derivative Compute Question 0005. \\ \fi
\begin{problem}

Determine if the limit approaches a finite number, $\pm\infty$, or does not exist. (If the limit does not exist, write DNE)

\input{Limit-Compute-0005.HELP.tex}

\[\lim_{x\to{2}}\dfrac{x^{2} + 3 \, x - 10}{x^{2} - 5 \, x + 6}=\answer{-7}\]
\end{problem}}%}

\latexProblemContent{
\ifVerboseLocation This is Derivative Compute Question 0005. \\ \fi
\begin{problem}

Determine if the limit approaches a finite number, $\pm\infty$, or does not exist. (If the limit does not exist, write DNE)

\input{Limit-Compute-0005.HELP.tex}

\[\lim_{x\to{3}}\dfrac{x^{2} + x - 12}{x^{2} - 9 \, x + 18}=\answer{-\frac{7}{3}}\]
\end{problem}}%}

\latexProblemContent{
\ifVerboseLocation This is Derivative Compute Question 0005. \\ \fi
\begin{problem}

Determine if the limit approaches a finite number, $\pm\infty$, or does not exist. (If the limit does not exist, write DNE)

\input{Limit-Compute-0005.HELP.tex}

\[\lim_{x\to{5}}\dfrac{x^{2} - 10 \, x + 25}{x^{2} - 3 \, x - 10}=\answer{0}\]
\end{problem}}%}

\latexProblemContent{
\ifVerboseLocation This is Derivative Compute Question 0005. \\ \fi
\begin{problem}

Determine if the limit approaches a finite number, $\pm\infty$, or does not exist. (If the limit does not exist, write DNE)

\input{Limit-Compute-0005.HELP.tex}

\[\lim_{x\to{4}}\dfrac{x^{2} - 16}{x^{2} - 7 \, x + 12}=\answer{8}\]
\end{problem}}%}

\latexProblemContent{
\ifVerboseLocation This is Derivative Compute Question 0005. \\ \fi
\begin{problem}

Determine if the limit approaches a finite number, $\pm\infty$, or does not exist. (If the limit does not exist, write DNE)

\input{Limit-Compute-0005.HELP.tex}

\[\lim_{x\to{-2}}\dfrac{x^{2} + 3 \, x + 2}{x^{2} + 8 \, x + 12}=\answer{-\frac{1}{4}}\]
\end{problem}}%}

\latexProblemContent{
\ifVerboseLocation This is Derivative Compute Question 0005. \\ \fi
\begin{problem}

Determine if the limit approaches a finite number, $\pm\infty$, or does not exist. (If the limit does not exist, write DNE)

\input{Limit-Compute-0005.HELP.tex}

\[\lim_{x\to{2}}\dfrac{x^{2} - x - 2}{x^{2} - 3 \, x + 2}=\answer{3}\]
\end{problem}}%}

\latexProblemContent{
\ifVerboseLocation This is Derivative Compute Question 0005. \\ \fi
\begin{problem}

Determine if the limit approaches a finite number, $\pm\infty$, or does not exist. (If the limit does not exist, write DNE)

\input{Limit-Compute-0005.HELP.tex}

\[\lim_{x\to{5}}\dfrac{x^{2} - 3 \, x - 10}{x^{2} - 2 \, x - 15}=\answer{\frac{7}{8}}\]
\end{problem}}%}

\latexProblemContent{
\ifVerboseLocation This is Derivative Compute Question 0005. \\ \fi
\begin{problem}

Determine if the limit approaches a finite number, $\pm\infty$, or does not exist. (If the limit does not exist, write DNE)

\input{Limit-Compute-0005.HELP.tex}

\[\lim_{x\to{-3}}\dfrac{x^{2} - 9}{x^{2} + 5 \, x + 6}=\answer{6}\]
\end{problem}}%}

\latexProblemContent{
\ifVerboseLocation This is Derivative Compute Question 0005. \\ \fi
\begin{problem}

Determine if the limit approaches a finite number, $\pm\infty$, or does not exist. (If the limit does not exist, write DNE)

\input{Limit-Compute-0005.HELP.tex}

\[\lim_{x\to{-6}}\dfrac{x^{2} + 8 \, x + 12}{x^{2} + 2 \, x - 24}=\answer{\frac{2}{5}}\]
\end{problem}}%}

\latexProblemContent{
\ifVerboseLocation This is Derivative Compute Question 0005. \\ \fi
\begin{problem}

Determine if the limit approaches a finite number, $\pm\infty$, or does not exist. (If the limit does not exist, write DNE)

\input{Limit-Compute-0005.HELP.tex}

\[\lim_{x\to{1}}\dfrac{x^{2} + 2 \, x - 3}{x^{2} - 4 \, x + 3}=\answer{-2}\]
\end{problem}}%}

\latexProblemContent{
\ifVerboseLocation This is Derivative Compute Question 0005. \\ \fi
\begin{problem}

Determine if the limit approaches a finite number, $\pm\infty$, or does not exist. (If the limit does not exist, write DNE)

\input{Limit-Compute-0005.HELP.tex}

\[\lim_{x\to{-2}}\dfrac{x^{2} + 7 \, x + 10}{x^{2} - 2 \, x - 8}=\answer{-\frac{1}{2}}\]
\end{problem}}%}

\latexProblemContent{
\ifVerboseLocation This is Derivative Compute Question 0005. \\ \fi
\begin{problem}

Determine if the limit approaches a finite number, $\pm\infty$, or does not exist. (If the limit does not exist, write DNE)

\input{Limit-Compute-0005.HELP.tex}

\[\lim_{x\to{5}}\dfrac{x^{2} - x - 20}{x^{2} - 4 \, x - 5}=\answer{\frac{3}{2}}\]
\end{problem}}%}

\latexProblemContent{
\ifVerboseLocation This is Derivative Compute Question 0005. \\ \fi
\begin{problem}

Determine if the limit approaches a finite number, $\pm\infty$, or does not exist. (If the limit does not exist, write DNE)

\input{Limit-Compute-0005.HELP.tex}

\[\lim_{x\to{-1}}\dfrac{x^{2} - 3 \, x - 4}{x^{2} + 4 \, x + 3}=\answer{-\frac{5}{2}}\]
\end{problem}}%}

\latexProblemContent{
\ifVerboseLocation This is Derivative Compute Question 0005. \\ \fi
\begin{problem}

Determine if the limit approaches a finite number, $\pm\infty$, or does not exist. (If the limit does not exist, write DNE)

\input{Limit-Compute-0005.HELP.tex}

\[\lim_{x\to{-3}}\dfrac{x^{2} + 8 \, x + 15}{x^{2} + 9 \, x + 18}=\answer{\frac{2}{3}}\]
\end{problem}}%}

\latexProblemContent{
\ifVerboseLocation This is Derivative Compute Question 0005. \\ \fi
\begin{problem}

Determine if the limit approaches a finite number, $\pm\infty$, or does not exist. (If the limit does not exist, write DNE)

\input{Limit-Compute-0005.HELP.tex}

\[\lim_{x\to{-1}}\dfrac{x^{2} - 2 \, x - 3}{x^{2} + 4 \, x + 3}=\answer{-2}\]
\end{problem}}%}

\latexProblemContent{
\ifVerboseLocation This is Derivative Compute Question 0005. \\ \fi
\begin{problem}

Determine if the limit approaches a finite number, $\pm\infty$, or does not exist. (If the limit does not exist, write DNE)

\input{Limit-Compute-0005.HELP.tex}

\[\lim_{x\to{2}}\dfrac{x^{2} - 4 \, x + 4}{x^{2} - 3 \, x + 2}=\answer{0}\]
\end{problem}}%}

\latexProblemContent{
\ifVerboseLocation This is Derivative Compute Question 0005. \\ \fi
\begin{problem}

Determine if the limit approaches a finite number, $\pm\infty$, or does not exist. (If the limit does not exist, write DNE)

\input{Limit-Compute-0005.HELP.tex}

\[\lim_{x\to{-1}}\dfrac{x^{2} - 1}{x^{2} + 3 \, x + 2}=\answer{-2}\]
\end{problem}}%}

\latexProblemContent{
\ifVerboseLocation This is Derivative Compute Question 0005. \\ \fi
\begin{problem}

Determine if the limit approaches a finite number, $\pm\infty$, or does not exist. (If the limit does not exist, write DNE)

\input{Limit-Compute-0005.HELP.tex}

\[\lim_{x\to{6}}\dfrac{x^{2} - x - 30}{x^{2} - 8 \, x + 12}=\answer{\frac{11}{4}}\]
\end{problem}}%}

\latexProblemContent{
\ifVerboseLocation This is Derivative Compute Question 0005. \\ \fi
\begin{problem}

Determine if the limit approaches a finite number, $\pm\infty$, or does not exist. (If the limit does not exist, write DNE)

\input{Limit-Compute-0005.HELP.tex}

\[\lim_{x\to{-4}}\dfrac{x^{2} + 8 \, x + 16}{x^{2} + 2 \, x - 8}=\answer{0}\]
\end{problem}}%}

\latexProblemContent{
\ifVerboseLocation This is Derivative Compute Question 0005. \\ \fi
\begin{problem}

Determine if the limit approaches a finite number, $\pm\infty$, or does not exist. (If the limit does not exist, write DNE)

\input{Limit-Compute-0005.HELP.tex}

\[\lim_{x\to{-1}}\dfrac{x^{2} + 6 \, x + 5}{x^{2} - x - 2}=\answer{-\frac{4}{3}}\]
\end{problem}}%}

\latexProblemContent{
\ifVerboseLocation This is Derivative Compute Question 0005. \\ \fi
\begin{problem}

Determine if the limit approaches a finite number, $\pm\infty$, or does not exist. (If the limit does not exist, write DNE)

\input{Limit-Compute-0005.HELP.tex}

\[\lim_{x\to{-2}}\dfrac{x^{2} - 3 \, x - 10}{x^{2} - x - 6}=\answer{\frac{7}{5}}\]
\end{problem}}%}

\latexProblemContent{
\ifVerboseLocation This is Derivative Compute Question 0005. \\ \fi
\begin{problem}

Determine if the limit approaches a finite number, $\pm\infty$, or does not exist. (If the limit does not exist, write DNE)

\input{Limit-Compute-0005.HELP.tex}

\[\lim_{x\to{-6}}\dfrac{x^{2} + 11 \, x + 30}{x^{2} + 4 \, x - 12}=\answer{\frac{1}{8}}\]
\end{problem}}%}

\latexProblemContent{
\ifVerboseLocation This is Derivative Compute Question 0005. \\ \fi
\begin{problem}

Determine if the limit approaches a finite number, $\pm\infty$, or does not exist. (If the limit does not exist, write DNE)

\input{Limit-Compute-0005.HELP.tex}

\[\lim_{x\to{4}}\dfrac{x^{2} - 2 \, x - 8}{x^{2} - 6 \, x + 8}=\answer{3}\]
\end{problem}}%}

\latexProblemContent{
\ifVerboseLocation This is Derivative Compute Question 0005. \\ \fi
\begin{problem}

Determine if the limit approaches a finite number, $\pm\infty$, or does not exist. (If the limit does not exist, write DNE)

\input{Limit-Compute-0005.HELP.tex}

\[\lim_{x\to{-6}}\dfrac{x^{2} + 11 \, x + 30}{x^{2} + 11 \, x + 30}=\answer{1}\]
\end{problem}}%}

\latexProblemContent{
\ifVerboseLocation This is Derivative Compute Question 0005. \\ \fi
\begin{problem}

Determine if the limit approaches a finite number, $\pm\infty$, or does not exist. (If the limit does not exist, write DNE)

\input{Limit-Compute-0005.HELP.tex}

\[\lim_{x\to{6}}\dfrac{x^{2} - 2 \, x - 24}{x^{2} - 9 \, x + 18}=\answer{\frac{10}{3}}\]
\end{problem}}%}

\latexProblemContent{
\ifVerboseLocation This is Derivative Compute Question 0005. \\ \fi
\begin{problem}

Determine if the limit approaches a finite number, $\pm\infty$, or does not exist. (If the limit does not exist, write DNE)

\input{Limit-Compute-0005.HELP.tex}

\[\lim_{x\to{2}}\dfrac{x^{2} + 3 \, x - 10}{x^{2} + x - 6}=\answer{\frac{7}{5}}\]
\end{problem}}%}

\latexProblemContent{
\ifVerboseLocation This is Derivative Compute Question 0005. \\ \fi
\begin{problem}

Determine if the limit approaches a finite number, $\pm\infty$, or does not exist. (If the limit does not exist, write DNE)

\input{Limit-Compute-0005.HELP.tex}

\[\lim_{x\to{-5}}\dfrac{x^{2} + 9 \, x + 20}{x^{2} + 6 \, x + 5}=\answer{\frac{1}{4}}\]
\end{problem}}%}

\latexProblemContent{
\ifVerboseLocation This is Derivative Compute Question 0005. \\ \fi
\begin{problem}

Determine if the limit approaches a finite number, $\pm\infty$, or does not exist. (If the limit does not exist, write DNE)

\input{Limit-Compute-0005.HELP.tex}

\[\lim_{x\to{4}}\dfrac{x^{2} - 7 \, x + 12}{x^{2} - 5 \, x + 4}=\answer{\frac{1}{3}}\]
\end{problem}}%}

\latexProblemContent{
\ifVerboseLocation This is Derivative Compute Question 0005. \\ \fi
\begin{problem}

Determine if the limit approaches a finite number, $\pm\infty$, or does not exist. (If the limit does not exist, write DNE)

\input{Limit-Compute-0005.HELP.tex}

\[\lim_{x\to{-2}}\dfrac{x^{2} - 4}{x^{2} - 4 \, x - 12}=\answer{\frac{1}{2}}\]
\end{problem}}%}

\latexProblemContent{
\ifVerboseLocation This is Derivative Compute Question 0005. \\ \fi
\begin{problem}

Determine if the limit approaches a finite number, $\pm\infty$, or does not exist. (If the limit does not exist, write DNE)

\input{Limit-Compute-0005.HELP.tex}

\[\lim_{x\to{6}}\dfrac{x^{2} - 5 \, x - 6}{x^{2} - 8 \, x + 12}=\answer{\frac{7}{4}}\]
\end{problem}}%}

\latexProblemContent{
\ifVerboseLocation This is Derivative Compute Question 0005. \\ \fi
\begin{problem}

Determine if the limit approaches a finite number, $\pm\infty$, or does not exist. (If the limit does not exist, write DNE)

\input{Limit-Compute-0005.HELP.tex}

\[\lim_{x\to{-5}}\dfrac{x^{2} + x - 20}{x^{2} + 6 \, x + 5}=\answer{\frac{9}{4}}\]
\end{problem}}%}

\latexProblemContent{
\ifVerboseLocation This is Derivative Compute Question 0005. \\ \fi
\begin{problem}

Determine if the limit approaches a finite number, $\pm\infty$, or does not exist. (If the limit does not exist, write DNE)

\input{Limit-Compute-0005.HELP.tex}

\[\lim_{x\to{3}}\dfrac{x^{2} + 2 \, x - 15}{x^{2} - 8 \, x + 15}=\answer{-4}\]
\end{problem}}%}

\latexProblemContent{
\ifVerboseLocation This is Derivative Compute Question 0005. \\ \fi
\begin{problem}

Determine if the limit approaches a finite number, $\pm\infty$, or does not exist. (If the limit does not exist, write DNE)

\input{Limit-Compute-0005.HELP.tex}

\[\lim_{x\to{4}}\dfrac{x^{2} - 2 \, x - 8}{x^{2} - 3 \, x - 4}=\answer{\frac{6}{5}}\]
\end{problem}}%}

\latexProblemContent{
\ifVerboseLocation This is Derivative Compute Question 0005. \\ \fi
\begin{problem}

Determine if the limit approaches a finite number, $\pm\infty$, or does not exist. (If the limit does not exist, write DNE)

\input{Limit-Compute-0005.HELP.tex}

\[\lim_{x\to{-6}}\dfrac{x^{2} + 3 \, x - 18}{x^{2} + 5 \, x - 6}=\answer{\frac{9}{7}}\]
\end{problem}}%}

\latexProblemContent{
\ifVerboseLocation This is Derivative Compute Question 0005. \\ \fi
\begin{problem}

Determine if the limit approaches a finite number, $\pm\infty$, or does not exist. (If the limit does not exist, write DNE)

\input{Limit-Compute-0005.HELP.tex}

\[\lim_{x\to{5}}\dfrac{x^{2} - 3 \, x - 10}{x^{2} - 11 \, x + 30}=\answer{-7}\]
\end{problem}}%}

\latexProblemContent{
\ifVerboseLocation This is Derivative Compute Question 0005. \\ \fi
\begin{problem}

Determine if the limit approaches a finite number, $\pm\infty$, or does not exist. (If the limit does not exist, write DNE)

\input{Limit-Compute-0005.HELP.tex}

\[\lim_{x\to{1}}\dfrac{x^{2} + 4 \, x - 5}{x^{2} - 5 \, x + 4}=\answer{-2}\]
\end{problem}}%}

\latexProblemContent{
\ifVerboseLocation This is Derivative Compute Question 0005. \\ \fi
\begin{problem}

Determine if the limit approaches a finite number, $\pm\infty$, or does not exist. (If the limit does not exist, write DNE)

\input{Limit-Compute-0005.HELP.tex}

\[\lim_{x\to{-5}}\dfrac{x^{2} - 25}{x^{2} + 6 \, x + 5}=\answer{\frac{5}{2}}\]
\end{problem}}%}

\latexProblemContent{
\ifVerboseLocation This is Derivative Compute Question 0005. \\ \fi
\begin{problem}

Determine if the limit approaches a finite number, $\pm\infty$, or does not exist. (If the limit does not exist, write DNE)

\input{Limit-Compute-0005.HELP.tex}

\[\lim_{x\to{6}}\dfrac{x^{2} - 11 \, x + 30}{x^{2} - 3 \, x - 18}=\answer{\frac{1}{9}}\]
\end{problem}}%}

\latexProblemContent{
\ifVerboseLocation This is Derivative Compute Question 0005. \\ \fi
\begin{problem}

Determine if the limit approaches a finite number, $\pm\infty$, or does not exist. (If the limit does not exist, write DNE)

\input{Limit-Compute-0005.HELP.tex}

\[\lim_{x\to{-3}}\dfrac{x^{2} + x - 6}{x^{2} + 8 \, x + 15}=\answer{-\frac{5}{2}}\]
\end{problem}}%}

\latexProblemContent{
\ifVerboseLocation This is Derivative Compute Question 0005. \\ \fi
\begin{problem}

Determine if the limit approaches a finite number, $\pm\infty$, or does not exist. (If the limit does not exist, write DNE)

\input{Limit-Compute-0005.HELP.tex}

\[\lim_{x\to{-6}}\dfrac{x^{2} + 10 \, x + 24}{x^{2} + 2 \, x - 24}=\answer{\frac{1}{5}}\]
\end{problem}}%}

\latexProblemContent{
\ifVerboseLocation This is Derivative Compute Question 0005. \\ \fi
\begin{problem}

Determine if the limit approaches a finite number, $\pm\infty$, or does not exist. (If the limit does not exist, write DNE)

\input{Limit-Compute-0005.HELP.tex}

\[\lim_{x\to{4}}\dfrac{x^{2} - 16}{x^{2} - 9 \, x + 20}=\answer{-8}\]
\end{problem}}%}

\latexProblemContent{
\ifVerboseLocation This is Derivative Compute Question 0005. \\ \fi
\begin{problem}

Determine if the limit approaches a finite number, $\pm\infty$, or does not exist. (If the limit does not exist, write DNE)

\input{Limit-Compute-0005.HELP.tex}

\[\lim_{x\to{-4}}\dfrac{x^{2} + 8 \, x + 16}{x^{2} + 6 \, x + 8}=\answer{0}\]
\end{problem}}%}

\latexProblemContent{
\ifVerboseLocation This is Derivative Compute Question 0005. \\ \fi
\begin{problem}

Determine if the limit approaches a finite number, $\pm\infty$, or does not exist. (If the limit does not exist, write DNE)

\input{Limit-Compute-0005.HELP.tex}

\[\lim_{x\to{-3}}\dfrac{x^{2} + 8 \, x + 15}{x^{2} + 8 \, x + 15}=\answer{1}\]
\end{problem}}%}

\latexProblemContent{
\ifVerboseLocation This is Derivative Compute Question 0005. \\ \fi
\begin{problem}

Determine if the limit approaches a finite number, $\pm\infty$, or does not exist. (If the limit does not exist, write DNE)

\input{Limit-Compute-0005.HELP.tex}

\[\lim_{x\to{4}}\dfrac{x^{2} - 8 \, x + 16}{x^{2} - 6 \, x + 8}=\answer{0}\]
\end{problem}}%}

\latexProblemContent{
\ifVerboseLocation This is Derivative Compute Question 0005. \\ \fi
\begin{problem}

Determine if the limit approaches a finite number, $\pm\infty$, or does not exist. (If the limit does not exist, write DNE)

\input{Limit-Compute-0005.HELP.tex}

\[\lim_{x\to{-2}}\dfrac{x^{2} + 7 \, x + 10}{x^{2} + 5 \, x + 6}=\answer{3}\]
\end{problem}}%}

\latexProblemContent{
\ifVerboseLocation This is Derivative Compute Question 0005. \\ \fi
\begin{problem}

Determine if the limit approaches a finite number, $\pm\infty$, or does not exist. (If the limit does not exist, write DNE)

\input{Limit-Compute-0005.HELP.tex}

\[\lim_{x\to{-5}}\dfrac{x^{2} + 3 \, x - 10}{x^{2} + 7 \, x + 10}=\answer{\frac{7}{3}}\]
\end{problem}}%}

\latexProblemContent{
\ifVerboseLocation This is Derivative Compute Question 0005. \\ \fi
\begin{problem}

Determine if the limit approaches a finite number, $\pm\infty$, or does not exist. (If the limit does not exist, write DNE)

\input{Limit-Compute-0005.HELP.tex}

\[\lim_{x\to{3}}\dfrac{x^{2} + 2 \, x - 15}{x^{2} - 4 \, x + 3}=\answer{4}\]
\end{problem}}%}

\latexProblemContent{
\ifVerboseLocation This is Derivative Compute Question 0005. \\ \fi
\begin{problem}

Determine if the limit approaches a finite number, $\pm\infty$, or does not exist. (If the limit does not exist, write DNE)

\input{Limit-Compute-0005.HELP.tex}

\[\lim_{x\to{2}}\dfrac{x^{2} - 4}{x^{2} - 5 \, x + 6}=\answer{-4}\]
\end{problem}}%}

\latexProblemContent{
\ifVerboseLocation This is Derivative Compute Question 0005. \\ \fi
\begin{problem}

Determine if the limit approaches a finite number, $\pm\infty$, or does not exist. (If the limit does not exist, write DNE)

\input{Limit-Compute-0005.HELP.tex}

\[\lim_{x\to{3}}\dfrac{x^{2} - 2 \, x - 3}{x^{2} - x - 6}=\answer{\frac{4}{5}}\]
\end{problem}}%}

\latexProblemContent{
\ifVerboseLocation This is Derivative Compute Question 0005. \\ \fi
\begin{problem}

Determine if the limit approaches a finite number, $\pm\infty$, or does not exist. (If the limit does not exist, write DNE)

\input{Limit-Compute-0005.HELP.tex}

\[\lim_{x\to{-3}}\dfrac{x^{2} + 6 \, x + 9}{x^{2} - 3 \, x - 18}=\answer{0}\]
\end{problem}}%}

\latexProblemContent{
\ifVerboseLocation This is Derivative Compute Question 0005. \\ \fi
\begin{problem}

Determine if the limit approaches a finite number, $\pm\infty$, or does not exist. (If the limit does not exist, write DNE)

\input{Limit-Compute-0005.HELP.tex}

\[\lim_{x\to{-3}}\dfrac{x^{2} - 9}{x^{2} + 8 \, x + 15}=\answer{-3}\]
\end{problem}}%}

\latexProblemContent{
\ifVerboseLocation This is Derivative Compute Question 0005. \\ \fi
\begin{problem}

Determine if the limit approaches a finite number, $\pm\infty$, or does not exist. (If the limit does not exist, write DNE)

\input{Limit-Compute-0005.HELP.tex}

\[\lim_{x\to{-2}}\dfrac{x^{2} + 4 \, x + 4}{x^{2} + 6 \, x + 8}=\answer{0}\]
\end{problem}}%}

\latexProblemContent{
\ifVerboseLocation This is Derivative Compute Question 0005. \\ \fi
\begin{problem}

Determine if the limit approaches a finite number, $\pm\infty$, or does not exist. (If the limit does not exist, write DNE)

\input{Limit-Compute-0005.HELP.tex}

\[\lim_{x\to{3}}\dfrac{x^{2} - 2 \, x - 3}{x^{2} + 3 \, x - 18}=\answer{\frac{4}{9}}\]
\end{problem}}%}

\latexProblemContent{
\ifVerboseLocation This is Derivative Compute Question 0005. \\ \fi
\begin{problem}

Determine if the limit approaches a finite number, $\pm\infty$, or does not exist. (If the limit does not exist, write DNE)

\input{Limit-Compute-0005.HELP.tex}

\[\lim_{x\to{4}}\dfrac{x^{2} - 7 \, x + 12}{x^{2} - 9 \, x + 20}=\answer{-1}\]
\end{problem}}%}

\latexProblemContent{
\ifVerboseLocation This is Derivative Compute Question 0005. \\ \fi
\begin{problem}

Determine if the limit approaches a finite number, $\pm\infty$, or does not exist. (If the limit does not exist, write DNE)

\input{Limit-Compute-0005.HELP.tex}

\[\lim_{x\to{-3}}\dfrac{x^{2} + 5 \, x + 6}{x^{2} - 2 \, x - 15}=\answer{\frac{1}{8}}\]
\end{problem}}%}

\latexProblemContent{
\ifVerboseLocation This is Derivative Compute Question 0005. \\ \fi
\begin{problem}

Determine if the limit approaches a finite number, $\pm\infty$, or does not exist. (If the limit does not exist, write DNE)

\input{Limit-Compute-0005.HELP.tex}

\[\lim_{x\to{-5}}\dfrac{x^{2} + 10 \, x + 25}{x^{2} - x - 30}=\answer{0}\]
\end{problem}}%}

\latexProblemContent{
\ifVerboseLocation This is Derivative Compute Question 0005. \\ \fi
\begin{problem}

Determine if the limit approaches a finite number, $\pm\infty$, or does not exist. (If the limit does not exist, write DNE)

\input{Limit-Compute-0005.HELP.tex}

\[\lim_{x\to{3}}\dfrac{x^{2} - 9}{x^{2} - 9 \, x + 18}=\answer{-2}\]
\end{problem}}%}

\latexProblemContent{
\ifVerboseLocation This is Derivative Compute Question 0005. \\ \fi
\begin{problem}

Determine if the limit approaches a finite number, $\pm\infty$, or does not exist. (If the limit does not exist, write DNE)

\input{Limit-Compute-0005.HELP.tex}

\[\lim_{x\to{1}}\dfrac{x^{2} + 4 \, x - 5}{x^{2} - 6 \, x + 5}=\answer{-\frac{3}{2}}\]
\end{problem}}%}

\latexProblemContent{
\ifVerboseLocation This is Derivative Compute Question 0005. \\ \fi
\begin{problem}

Determine if the limit approaches a finite number, $\pm\infty$, or does not exist. (If the limit does not exist, write DNE)

\input{Limit-Compute-0005.HELP.tex}

\[\lim_{x\to{-2}}\dfrac{x^{2} + 7 \, x + 10}{x^{2} + x - 2}=\answer{-1}\]
\end{problem}}%}

\latexProblemContent{
\ifVerboseLocation This is Derivative Compute Question 0005. \\ \fi
\begin{problem}

Determine if the limit approaches a finite number, $\pm\infty$, or does not exist. (If the limit does not exist, write DNE)

\input{Limit-Compute-0005.HELP.tex}

\[\lim_{x\to{-3}}\dfrac{x^{2} + 6 \, x + 9}{x^{2} - x - 12}=\answer{0}\]
\end{problem}}%}

\latexProblemContent{
\ifVerboseLocation This is Derivative Compute Question 0005. \\ \fi
\begin{problem}

Determine if the limit approaches a finite number, $\pm\infty$, or does not exist. (If the limit does not exist, write DNE)

\input{Limit-Compute-0005.HELP.tex}

\[\lim_{x\to{6}}\dfrac{x^{2} - 10 \, x + 24}{x^{2} - x - 30}=\answer{\frac{2}{11}}\]
\end{problem}}%}

\latexProblemContent{
\ifVerboseLocation This is Derivative Compute Question 0005. \\ \fi
\begin{problem}

Determine if the limit approaches a finite number, $\pm\infty$, or does not exist. (If the limit does not exist, write DNE)

\input{Limit-Compute-0005.HELP.tex}

\[\lim_{x\to{6}}\dfrac{x^{2} - 8 \, x + 12}{x^{2} - 8 \, x + 12}=\answer{1}\]
\end{problem}}%}

\latexProblemContent{
\ifVerboseLocation This is Derivative Compute Question 0005. \\ \fi
\begin{problem}

Determine if the limit approaches a finite number, $\pm\infty$, or does not exist. (If the limit does not exist, write DNE)

\input{Limit-Compute-0005.HELP.tex}

\[\lim_{x\to{-3}}\dfrac{x^{2} + 8 \, x + 15}{x^{2} + x - 6}=\answer{-\frac{2}{5}}\]
\end{problem}}%}

\latexProblemContent{
\ifVerboseLocation This is Derivative Compute Question 0005. \\ \fi
\begin{problem}

Determine if the limit approaches a finite number, $\pm\infty$, or does not exist. (If the limit does not exist, write DNE)

\input{Limit-Compute-0005.HELP.tex}

\[\lim_{x\to{-3}}\dfrac{x^{2} + 2 \, x - 3}{x^{2} + 5 \, x + 6}=\answer{4}\]
\end{problem}}%}

\latexProblemContent{
\ifVerboseLocation This is Derivative Compute Question 0005. \\ \fi
\begin{problem}

Determine if the limit approaches a finite number, $\pm\infty$, or does not exist. (If the limit does not exist, write DNE)

\input{Limit-Compute-0005.HELP.tex}

\[\lim_{x\to{1}}\dfrac{x^{2} - 5 \, x + 4}{x^{2} - 4 \, x + 3}=\answer{\frac{3}{2}}\]
\end{problem}}%}

\latexProblemContent{
\ifVerboseLocation This is Derivative Compute Question 0005. \\ \fi
\begin{problem}

Determine if the limit approaches a finite number, $\pm\infty$, or does not exist. (If the limit does not exist, write DNE)

\input{Limit-Compute-0005.HELP.tex}

\[\lim_{x\to{-5}}\dfrac{x^{2} + 9 \, x + 20}{x^{2} + 4 \, x - 5}=\answer{\frac{1}{6}}\]
\end{problem}}%}

\latexProblemContent{
\ifVerboseLocation This is Derivative Compute Question 0005. \\ \fi
\begin{problem}

Determine if the limit approaches a finite number, $\pm\infty$, or does not exist. (If the limit does not exist, write DNE)

\input{Limit-Compute-0005.HELP.tex}

\[\lim_{x\to{6}}\dfrac{x^{2} - 2 \, x - 24}{x^{2} - 4 \, x - 12}=\answer{\frac{5}{4}}\]
\end{problem}}%}

\latexProblemContent{
\ifVerboseLocation This is Derivative Compute Question 0005. \\ \fi
\begin{problem}

Determine if the limit approaches a finite number, $\pm\infty$, or does not exist. (If the limit does not exist, write DNE)

\input{Limit-Compute-0005.HELP.tex}

\[\lim_{x\to{1}}\dfrac{x^{2} - 2 \, x + 1}{x^{2} + 3 \, x - 4}=\answer{0}\]
\end{problem}}%}

\latexProblemContent{
\ifVerboseLocation This is Derivative Compute Question 0005. \\ \fi
\begin{problem}

Determine if the limit approaches a finite number, $\pm\infty$, or does not exist. (If the limit does not exist, write DNE)

\input{Limit-Compute-0005.HELP.tex}

\[\lim_{x\to{-1}}\dfrac{x^{2} - x - 2}{x^{2} + 4 \, x + 3}=\answer{-\frac{3}{2}}\]
\end{problem}}%}

\latexProblemContent{
\ifVerboseLocation This is Derivative Compute Question 0005. \\ \fi
\begin{problem}

Determine if the limit approaches a finite number, $\pm\infty$, or does not exist. (If the limit does not exist, write DNE)

\input{Limit-Compute-0005.HELP.tex}

\[\lim_{x\to{-1}}\dfrac{x^{2} + 2 \, x + 1}{x^{2} - 4 \, x - 5}=\answer{0}\]
\end{problem}}%}

\latexProblemContent{
\ifVerboseLocation This is Derivative Compute Question 0005. \\ \fi
\begin{problem}

Determine if the limit approaches a finite number, $\pm\infty$, or does not exist. (If the limit does not exist, write DNE)

\input{Limit-Compute-0005.HELP.tex}

\[\lim_{x\to{-5}}\dfrac{x^{2} + 2 \, x - 15}{x^{2} + 8 \, x + 15}=\answer{4}\]
\end{problem}}%}

\latexProblemContent{
\ifVerboseLocation This is Derivative Compute Question 0005. \\ \fi
\begin{problem}

Determine if the limit approaches a finite number, $\pm\infty$, or does not exist. (If the limit does not exist, write DNE)

\input{Limit-Compute-0005.HELP.tex}

\[\lim_{x\to{-1}}\dfrac{x^{2} + 6 \, x + 5}{x^{2} + 6 \, x + 5}=\answer{1}\]
\end{problem}}%}

\latexProblemContent{
\ifVerboseLocation This is Derivative Compute Question 0005. \\ \fi
\begin{problem}

Determine if the limit approaches a finite number, $\pm\infty$, or does not exist. (If the limit does not exist, write DNE)

\input{Limit-Compute-0005.HELP.tex}

\[\lim_{x\to{-3}}\dfrac{x^{2} + x - 6}{x^{2} - x - 12}=\answer{\frac{5}{7}}\]
\end{problem}}%}

\latexProblemContent{
\ifVerboseLocation This is Derivative Compute Question 0005. \\ \fi
\begin{problem}

Determine if the limit approaches a finite number, $\pm\infty$, or does not exist. (If the limit does not exist, write DNE)

\input{Limit-Compute-0005.HELP.tex}

\[\lim_{x\to{6}}\dfrac{x^{2} - 5 \, x - 6}{x^{2} - 5 \, x - 6}=\answer{1}\]
\end{problem}}%}

\latexProblemContent{
\ifVerboseLocation This is Derivative Compute Question 0005. \\ \fi
\begin{problem}

Determine if the limit approaches a finite number, $\pm\infty$, or does not exist. (If the limit does not exist, write DNE)

\input{Limit-Compute-0005.HELP.tex}

\[\lim_{x\to{-4}}\dfrac{x^{2} - x - 20}{x^{2} - x - 20}=\answer{1}\]
\end{problem}}%}

\latexProblemContent{
\ifVerboseLocation This is Derivative Compute Question 0005. \\ \fi
\begin{problem}

Determine if the limit approaches a finite number, $\pm\infty$, or does not exist. (If the limit does not exist, write DNE)

\input{Limit-Compute-0005.HELP.tex}

\[\lim_{x\to{1}}\dfrac{x^{2} + 3 \, x - 4}{x^{2} + 4 \, x - 5}=\answer{\frac{5}{6}}\]
\end{problem}}%}

\latexProblemContent{
\ifVerboseLocation This is Derivative Compute Question 0005. \\ \fi
\begin{problem}

Determine if the limit approaches a finite number, $\pm\infty$, or does not exist. (If the limit does not exist, write DNE)

\input{Limit-Compute-0005.HELP.tex}

\[\lim_{x\to{-6}}\dfrac{x^{2} + 4 \, x - 12}{x^{2} + x - 30}=\answer{\frac{8}{11}}\]
\end{problem}}%}

\latexProblemContent{
\ifVerboseLocation This is Derivative Compute Question 0005. \\ \fi
\begin{problem}

Determine if the limit approaches a finite number, $\pm\infty$, or does not exist. (If the limit does not exist, write DNE)

\input{Limit-Compute-0005.HELP.tex}

\[\lim_{x\to{4}}\dfrac{x^{2} - x - 12}{x^{2} - 9 \, x + 20}=\answer{-7}\]
\end{problem}}%}

\latexProblemContent{
\ifVerboseLocation This is Derivative Compute Question 0005. \\ \fi
\begin{problem}

Determine if the limit approaches a finite number, $\pm\infty$, or does not exist. (If the limit does not exist, write DNE)

\input{Limit-Compute-0005.HELP.tex}

\[\lim_{x\to{1}}\dfrac{x^{2} - 3 \, x + 2}{x^{2} - 4 \, x + 3}=\answer{\frac{1}{2}}\]
\end{problem}}%}

\latexProblemContent{
\ifVerboseLocation This is Derivative Compute Question 0005. \\ \fi
\begin{problem}

Determine if the limit approaches a finite number, $\pm\infty$, or does not exist. (If the limit does not exist, write DNE)

\input{Limit-Compute-0005.HELP.tex}

\[\lim_{x\to{2}}\dfrac{x^{2} - 3 \, x + 2}{x^{2} - x - 2}=\answer{\frac{1}{3}}\]
\end{problem}}%}

\latexProblemContent{
\ifVerboseLocation This is Derivative Compute Question 0005. \\ \fi
\begin{problem}

Determine if the limit approaches a finite number, $\pm\infty$, or does not exist. (If the limit does not exist, write DNE)

\input{Limit-Compute-0005.HELP.tex}

\[\lim_{x\to{-5}}\dfrac{x^{2} + 2 \, x - 15}{x^{2} + x - 20}=\answer{\frac{8}{9}}\]
\end{problem}}%}

\latexProblemContent{
\ifVerboseLocation This is Derivative Compute Question 0005. \\ \fi
\begin{problem}

Determine if the limit approaches a finite number, $\pm\infty$, or does not exist. (If the limit does not exist, write DNE)

\input{Limit-Compute-0005.HELP.tex}

\[\lim_{x\to{-4}}\dfrac{x^{2} + 8 \, x + 16}{x^{2} + 3 \, x - 4}=\answer{0}\]
\end{problem}}%}

\latexProblemContent{
\ifVerboseLocation This is Derivative Compute Question 0005. \\ \fi
\begin{problem}

Determine if the limit approaches a finite number, $\pm\infty$, or does not exist. (If the limit does not exist, write DNE)

\input{Limit-Compute-0005.HELP.tex}

\[\lim_{x\to{3}}\dfrac{x^{2} - x - 6}{x^{2} - 4 \, x + 3}=\answer{\frac{5}{2}}\]
\end{problem}}%}

\latexProblemContent{
\ifVerboseLocation This is Derivative Compute Question 0005. \\ \fi
\begin{problem}

Determine if the limit approaches a finite number, $\pm\infty$, or does not exist. (If the limit does not exist, write DNE)

\input{Limit-Compute-0005.HELP.tex}

\[\lim_{x\to{4}}\dfrac{x^{2} - 7 \, x + 12}{x^{2} - 7 \, x + 12}=\answer{1}\]
\end{problem}}%}

\latexProblemContent{
\ifVerboseLocation This is Derivative Compute Question 0005. \\ \fi
\begin{problem}

Determine if the limit approaches a finite number, $\pm\infty$, or does not exist. (If the limit does not exist, write DNE)

\input{Limit-Compute-0005.HELP.tex}

\[\lim_{x\to{-3}}\dfrac{x^{2} - x - 12}{x^{2} - x - 12}=\answer{1}\]
\end{problem}}%}

\latexProblemContent{
\ifVerboseLocation This is Derivative Compute Question 0005. \\ \fi
\begin{problem}

Determine if the limit approaches a finite number, $\pm\infty$, or does not exist. (If the limit does not exist, write DNE)

\input{Limit-Compute-0005.HELP.tex}

\[\lim_{x\to{6}}\dfrac{x^{2} - 7 \, x + 6}{x^{2} - 5 \, x - 6}=\answer{\frac{5}{7}}\]
\end{problem}}%}

\latexProblemContent{
\ifVerboseLocation This is Derivative Compute Question 0005. \\ \fi
\begin{problem}

Determine if the limit approaches a finite number, $\pm\infty$, or does not exist. (If the limit does not exist, write DNE)

\input{Limit-Compute-0005.HELP.tex}

\[\lim_{x\to{-6}}\dfrac{x^{2} + 7 \, x + 6}{x^{2} + 8 \, x + 12}=\answer{\frac{5}{4}}\]
\end{problem}}%}

\latexProblemContent{
\ifVerboseLocation This is Derivative Compute Question 0005. \\ \fi
\begin{problem}

Determine if the limit approaches a finite number, $\pm\infty$, or does not exist. (If the limit does not exist, write DNE)

\input{Limit-Compute-0005.HELP.tex}

\[\lim_{x\to{-4}}\dfrac{x^{2} - x - 20}{x^{2} + 5 \, x + 4}=\answer{3}\]
\end{problem}}%}

\latexProblemContent{
\ifVerboseLocation This is Derivative Compute Question 0005. \\ \fi
\begin{problem}

Determine if the limit approaches a finite number, $\pm\infty$, or does not exist. (If the limit does not exist, write DNE)

\input{Limit-Compute-0005.HELP.tex}

\[\lim_{x\to{-3}}\dfrac{x^{2} - x - 12}{x^{2} + 9 \, x + 18}=\answer{-\frac{7}{3}}\]
\end{problem}}%}

\latexProblemContent{
\ifVerboseLocation This is Derivative Compute Question 0005. \\ \fi
\begin{problem}

Determine if the limit approaches a finite number, $\pm\infty$, or does not exist. (If the limit does not exist, write DNE)

\input{Limit-Compute-0005.HELP.tex}

\[\lim_{x\to{-1}}\dfrac{x^{2} + 2 \, x + 1}{x^{2} - 2 \, x - 3}=\answer{0}\]
\end{problem}}%}

\latexProblemContent{
\ifVerboseLocation This is Derivative Compute Question 0005. \\ \fi
\begin{problem}

Determine if the limit approaches a finite number, $\pm\infty$, or does not exist. (If the limit does not exist, write DNE)

\input{Limit-Compute-0005.HELP.tex}

\[\lim_{x\to{-6}}\dfrac{x^{2} + 7 \, x + 6}{x^{2} + x - 30}=\answer{\frac{5}{11}}\]
\end{problem}}%}

\latexProblemContent{
\ifVerboseLocation This is Derivative Compute Question 0005. \\ \fi
\begin{problem}

Determine if the limit approaches a finite number, $\pm\infty$, or does not exist. (If the limit does not exist, write DNE)

\input{Limit-Compute-0005.HELP.tex}

\[\lim_{x\to{-6}}\dfrac{x^{2} + 10 \, x + 24}{x^{2} + 11 \, x + 30}=\answer{2}\]
\end{problem}}%}

\latexProblemContent{
\ifVerboseLocation This is Derivative Compute Question 0005. \\ \fi
\begin{problem}

Determine if the limit approaches a finite number, $\pm\infty$, or does not exist. (If the limit does not exist, write DNE)

\input{Limit-Compute-0005.HELP.tex}

\[\lim_{x\to{5}}\dfrac{x^{2} - x - 20}{x^{2} - 6 \, x + 5}=\answer{\frac{9}{4}}\]
\end{problem}}%}

\latexProblemContent{
\ifVerboseLocation This is Derivative Compute Question 0005. \\ \fi
\begin{problem}

Determine if the limit approaches a finite number, $\pm\infty$, or does not exist. (If the limit does not exist, write DNE)

\input{Limit-Compute-0005.HELP.tex}

\[\lim_{x\to{6}}\dfrac{x^{2} - x - 30}{x^{2} - 4 \, x - 12}=\answer{\frac{11}{8}}\]
\end{problem}}%}

\latexProblemContent{
\ifVerboseLocation This is Derivative Compute Question 0005. \\ \fi
\begin{problem}

Determine if the limit approaches a finite number, $\pm\infty$, or does not exist. (If the limit does not exist, write DNE)

\input{Limit-Compute-0005.HELP.tex}

\[\lim_{x\to{-1}}\dfrac{x^{2} - 4 \, x - 5}{x^{2} + 4 \, x + 3}=\answer{-3}\]
\end{problem}}%}

\latexProblemContent{
\ifVerboseLocation This is Derivative Compute Question 0005. \\ \fi
\begin{problem}

Determine if the limit approaches a finite number, $\pm\infty$, or does not exist. (If the limit does not exist, write DNE)

\input{Limit-Compute-0005.HELP.tex}

\[\lim_{x\to{5}}\dfrac{x^{2} - x - 20}{x^{2} - 8 \, x + 15}=\answer{\frac{9}{2}}\]
\end{problem}}%}

\latexProblemContent{
\ifVerboseLocation This is Derivative Compute Question 0005. \\ \fi
\begin{problem}

Determine if the limit approaches a finite number, $\pm\infty$, or does not exist. (If the limit does not exist, write DNE)

\input{Limit-Compute-0005.HELP.tex}

\[\lim_{x\to{5}}\dfrac{x^{2} - 7 \, x + 10}{x^{2} - 8 \, x + 15}=\answer{\frac{3}{2}}\]
\end{problem}}%}

\latexProblemContent{
\ifVerboseLocation This is Derivative Compute Question 0005. \\ \fi
\begin{problem}

Determine if the limit approaches a finite number, $\pm\infty$, or does not exist. (If the limit does not exist, write DNE)

\input{Limit-Compute-0005.HELP.tex}

\[\lim_{x\to{2}}\dfrac{x^{2} - 7 \, x + 10}{x^{2} + 2 \, x - 8}=\answer{-\frac{1}{2}}\]
\end{problem}}%}

\latexProblemContent{
\ifVerboseLocation This is Derivative Compute Question 0005. \\ \fi
\begin{problem}

Determine if the limit approaches a finite number, $\pm\infty$, or does not exist. (If the limit does not exist, write DNE)

\input{Limit-Compute-0005.HELP.tex}

\[\lim_{x\to{6}}\dfrac{x^{2} - 4 \, x - 12}{x^{2} - 5 \, x - 6}=\answer{\frac{8}{7}}\]
\end{problem}}%}

\latexProblemContent{
\ifVerboseLocation This is Derivative Compute Question 0005. \\ \fi
\begin{problem}

Determine if the limit approaches a finite number, $\pm\infty$, or does not exist. (If the limit does not exist, write DNE)

\input{Limit-Compute-0005.HELP.tex}

\[\lim_{x\to{3}}\dfrac{x^{2} - 4 \, x + 3}{x^{2} + x - 12}=\answer{\frac{2}{7}}\]
\end{problem}}%}

\latexProblemContent{
\ifVerboseLocation This is Derivative Compute Question 0005. \\ \fi
\begin{problem}

Determine if the limit approaches a finite number, $\pm\infty$, or does not exist. (If the limit does not exist, write DNE)

\input{Limit-Compute-0005.HELP.tex}

\[\lim_{x\to{3}}\dfrac{x^{2} - 7 \, x + 12}{x^{2} - x - 6}=\answer{-\frac{1}{5}}\]
\end{problem}}%}

\latexProblemContent{
\ifVerboseLocation This is Derivative Compute Question 0005. \\ \fi
\begin{problem}

Determine if the limit approaches a finite number, $\pm\infty$, or does not exist. (If the limit does not exist, write DNE)

\input{Limit-Compute-0005.HELP.tex}

\[\lim_{x\to{-5}}\dfrac{x^{2} + 4 \, x - 5}{x^{2} + x - 20}=\answer{\frac{2}{3}}\]
\end{problem}}%}

\latexProblemContent{
\ifVerboseLocation This is Derivative Compute Question 0005. \\ \fi
\begin{problem}

Determine if the limit approaches a finite number, $\pm\infty$, or does not exist. (If the limit does not exist, write DNE)

\input{Limit-Compute-0005.HELP.tex}

\[\lim_{x\to{-3}}\dfrac{x^{2} + 5 \, x + 6}{x^{2} + 4 \, x + 3}=\answer{\frac{1}{2}}\]
\end{problem}}%}

\latexProblemContent{
\ifVerboseLocation This is Derivative Compute Question 0005. \\ \fi
\begin{problem}

Determine if the limit approaches a finite number, $\pm\infty$, or does not exist. (If the limit does not exist, write DNE)

\input{Limit-Compute-0005.HELP.tex}

\[\lim_{x\to{6}}\dfrac{x^{2} - x - 30}{x^{2} - 10 \, x + 24}=\answer{\frac{11}{2}}\]
\end{problem}}%}

\latexProblemContent{
\ifVerboseLocation This is Derivative Compute Question 0005. \\ \fi
\begin{problem}

Determine if the limit approaches a finite number, $\pm\infty$, or does not exist. (If the limit does not exist, write DNE)

\input{Limit-Compute-0005.HELP.tex}

\[\lim_{x\to{-2}}\dfrac{x^{2} - 2 \, x - 8}{x^{2} + 6 \, x + 8}=\answer{-3}\]
\end{problem}}%}

\latexProblemContent{
\ifVerboseLocation This is Derivative Compute Question 0005. \\ \fi
\begin{problem}

Determine if the limit approaches a finite number, $\pm\infty$, or does not exist. (If the limit does not exist, write DNE)

\input{Limit-Compute-0005.HELP.tex}

\[\lim_{x\to{-6}}\dfrac{x^{2} + 5 \, x - 6}{x^{2} + 8 \, x + 12}=\answer{\frac{7}{4}}\]
\end{problem}}%}

\latexProblemContent{
\ifVerboseLocation This is Derivative Compute Question 0005. \\ \fi
\begin{problem}

Determine if the limit approaches a finite number, $\pm\infty$, or does not exist. (If the limit does not exist, write DNE)

\input{Limit-Compute-0005.HELP.tex}

\[\lim_{x\to{-2}}\dfrac{x^{2} + 5 \, x + 6}{x^{2} + 8 \, x + 12}=\answer{\frac{1}{4}}\]
\end{problem}}%}

\latexProblemContent{
\ifVerboseLocation This is Derivative Compute Question 0005. \\ \fi
\begin{problem}

Determine if the limit approaches a finite number, $\pm\infty$, or does not exist. (If the limit does not exist, write DNE)

\input{Limit-Compute-0005.HELP.tex}

\[\lim_{x\to{-3}}\dfrac{x^{2} + 8 \, x + 15}{x^{2} + 7 \, x + 12}=\answer{2}\]
\end{problem}}%}

\latexProblemContent{
\ifVerboseLocation This is Derivative Compute Question 0005. \\ \fi
\begin{problem}

Determine if the limit approaches a finite number, $\pm\infty$, or does not exist. (If the limit does not exist, write DNE)

\input{Limit-Compute-0005.HELP.tex}

\[\lim_{x\to{5}}\dfrac{x^{2} - 8 \, x + 15}{x^{2} - x - 20}=\answer{\frac{2}{9}}\]
\end{problem}}%}

\latexProblemContent{
\ifVerboseLocation This is Derivative Compute Question 0005. \\ \fi
\begin{problem}

Determine if the limit approaches a finite number, $\pm\infty$, or does not exist. (If the limit does not exist, write DNE)

\input{Limit-Compute-0005.HELP.tex}

\[\lim_{x\to{-4}}\dfrac{x^{2} + 5 \, x + 4}{x^{2} + 3 \, x - 4}=\answer{\frac{3}{5}}\]
\end{problem}}%}

\latexProblemContent{
\ifVerboseLocation This is Derivative Compute Question 0005. \\ \fi
\begin{problem}

Determine if the limit approaches a finite number, $\pm\infty$, or does not exist. (If the limit does not exist, write DNE)

\input{Limit-Compute-0005.HELP.tex}

\[\lim_{x\to{5}}\dfrac{x^{2} - 8 \, x + 15}{x^{2} - 3 \, x - 10}=\answer{\frac{2}{7}}\]
\end{problem}}%}

\latexProblemContent{
\ifVerboseLocation This is Derivative Compute Question 0005. \\ \fi
\begin{problem}

Determine if the limit approaches a finite number, $\pm\infty$, or does not exist. (If the limit does not exist, write DNE)

\input{Limit-Compute-0005.HELP.tex}

\[\lim_{x\to{-1}}\dfrac{x^{2} - x - 2}{x^{2} + 6 \, x + 5}=\answer{-\frac{3}{4}}\]
\end{problem}}%}

\latexProblemContent{
\ifVerboseLocation This is Derivative Compute Question 0005. \\ \fi
\begin{problem}

Determine if the limit approaches a finite number, $\pm\infty$, or does not exist. (If the limit does not exist, write DNE)

\input{Limit-Compute-0005.HELP.tex}

\[\lim_{x\to{-4}}\dfrac{x^{2} + 7 \, x + 12}{x^{2} - 16}=\answer{\frac{1}{8}}\]
\end{problem}}%}

\latexProblemContent{
\ifVerboseLocation This is Derivative Compute Question 0005. \\ \fi
\begin{problem}

Determine if the limit approaches a finite number, $\pm\infty$, or does not exist. (If the limit does not exist, write DNE)

\input{Limit-Compute-0005.HELP.tex}

\[\lim_{x\to{-3}}\dfrac{x^{2} - x - 12}{x^{2} - 3 \, x - 18}=\answer{\frac{7}{9}}\]
\end{problem}}%}

\latexProblemContent{
\ifVerboseLocation This is Derivative Compute Question 0005. \\ \fi
\begin{problem}

Determine if the limit approaches a finite number, $\pm\infty$, or does not exist. (If the limit does not exist, write DNE)

\input{Limit-Compute-0005.HELP.tex}

\[\lim_{x\to{-3}}\dfrac{x^{2} + 4 \, x + 3}{x^{2} + 2 \, x - 3}=\answer{\frac{1}{2}}\]
\end{problem}}%}

\latexProblemContent{
\ifVerboseLocation This is Derivative Compute Question 0005. \\ \fi
\begin{problem}

Determine if the limit approaches a finite number, $\pm\infty$, or does not exist. (If the limit does not exist, write DNE)

\input{Limit-Compute-0005.HELP.tex}

\[\lim_{x\to{4}}\dfrac{x^{2} - 6 \, x + 8}{x^{2} - 6 \, x + 8}=\answer{1}\]
\end{problem}}%}

\latexProblemContent{
\ifVerboseLocation This is Derivative Compute Question 0005. \\ \fi
\begin{problem}

Determine if the limit approaches a finite number, $\pm\infty$, or does not exist. (If the limit does not exist, write DNE)

\input{Limit-Compute-0005.HELP.tex}

\[\lim_{x\to{-6}}\dfrac{x^{2} + 11 \, x + 30}{x^{2} + 8 \, x + 12}=\answer{\frac{1}{4}}\]
\end{problem}}%}

\latexProblemContent{
\ifVerboseLocation This is Derivative Compute Question 0005. \\ \fi
\begin{problem}

Determine if the limit approaches a finite number, $\pm\infty$, or does not exist. (If the limit does not exist, write DNE)

\input{Limit-Compute-0005.HELP.tex}

\[\lim_{x\to{-2}}\dfrac{x^{2} - 2 \, x - 8}{x^{2} + 5 \, x + 6}=\answer{-6}\]
\end{problem}}%}

\latexProblemContent{
\ifVerboseLocation This is Derivative Compute Question 0005. \\ \fi
\begin{problem}

Determine if the limit approaches a finite number, $\pm\infty$, or does not exist. (If the limit does not exist, write DNE)

\input{Limit-Compute-0005.HELP.tex}

\[\lim_{x\to{-4}}\dfrac{x^{2} + 7 \, x + 12}{x^{2} + 6 \, x + 8}=\answer{\frac{1}{2}}\]
\end{problem}}%}

\latexProblemContent{
\ifVerboseLocation This is Derivative Compute Question 0005. \\ \fi
\begin{problem}

Determine if the limit approaches a finite number, $\pm\infty$, or does not exist. (If the limit does not exist, write DNE)

\input{Limit-Compute-0005.HELP.tex}

\[\lim_{x\to{1}}\dfrac{x^{2} + 4 \, x - 5}{x^{2} + 5 \, x - 6}=\answer{\frac{6}{7}}\]
\end{problem}}%}

\latexProblemContent{
\ifVerboseLocation This is Derivative Compute Question 0005. \\ \fi
\begin{problem}

Determine if the limit approaches a finite number, $\pm\infty$, or does not exist. (If the limit does not exist, write DNE)

\input{Limit-Compute-0005.HELP.tex}

\[\lim_{x\to{-2}}\dfrac{x^{2} + 3 \, x + 2}{x^{2} - 2 \, x - 8}=\answer{\frac{1}{6}}\]
\end{problem}}%}

\latexProblemContent{
\ifVerboseLocation This is Derivative Compute Question 0005. \\ \fi
\begin{problem}

Determine if the limit approaches a finite number, $\pm\infty$, or does not exist. (If the limit does not exist, write DNE)

\input{Limit-Compute-0005.HELP.tex}

\[\lim_{x\to{-3}}\dfrac{x^{2} - 2 \, x - 15}{x^{2} - x - 12}=\answer{\frac{8}{7}}\]
\end{problem}}%}

\latexProblemContent{
\ifVerboseLocation This is Derivative Compute Question 0005. \\ \fi
\begin{problem}

Determine if the limit approaches a finite number, $\pm\infty$, or does not exist. (If the limit does not exist, write DNE)

\input{Limit-Compute-0005.HELP.tex}

\[\lim_{x\to{-1}}\dfrac{x^{2} + 5 \, x + 4}{x^{2} + 3 \, x + 2}=\answer{3}\]
\end{problem}}%}

\latexProblemContent{
\ifVerboseLocation This is Derivative Compute Question 0005. \\ \fi
\begin{problem}

Determine if the limit approaches a finite number, $\pm\infty$, or does not exist. (If the limit does not exist, write DNE)

\input{Limit-Compute-0005.HELP.tex}

\[\lim_{x\to{5}}\dfrac{x^{2} - 7 \, x + 10}{x^{2} - 7 \, x + 10}=\answer{1}\]
\end{problem}}%}

\latexProblemContent{
\ifVerboseLocation This is Derivative Compute Question 0005. \\ \fi
\begin{problem}

Determine if the limit approaches a finite number, $\pm\infty$, or does not exist. (If the limit does not exist, write DNE)

\input{Limit-Compute-0005.HELP.tex}

\[\lim_{x\to{-4}}\dfrac{x^{2} + 3 \, x - 4}{x^{2} + 7 \, x + 12}=\answer{5}\]
\end{problem}}%}

\latexProblemContent{
\ifVerboseLocation This is Derivative Compute Question 0005. \\ \fi
\begin{problem}

Determine if the limit approaches a finite number, $\pm\infty$, or does not exist. (If the limit does not exist, write DNE)

\input{Limit-Compute-0005.HELP.tex}

\[\lim_{x\to{-6}}\dfrac{x^{2} + 2 \, x - 24}{x^{2} - 36}=\answer{\frac{5}{6}}\]
\end{problem}}%}

\latexProblemContent{
\ifVerboseLocation This is Derivative Compute Question 0005. \\ \fi
\begin{problem}

Determine if the limit approaches a finite number, $\pm\infty$, or does not exist. (If the limit does not exist, write DNE)

\input{Limit-Compute-0005.HELP.tex}

\[\lim_{x\to{6}}\dfrac{x^{2} - 7 \, x + 6}{x^{2} - 3 \, x - 18}=\answer{\frac{5}{9}}\]
\end{problem}}%}

\latexProblemContent{
\ifVerboseLocation This is Derivative Compute Question 0005. \\ \fi
\begin{problem}

Determine if the limit approaches a finite number, $\pm\infty$, or does not exist. (If the limit does not exist, write DNE)

\input{Limit-Compute-0005.HELP.tex}

\[\lim_{x\to{-2}}\dfrac{x^{2} + 6 \, x + 8}{x^{2} + 8 \, x + 12}=\answer{\frac{1}{2}}\]
\end{problem}}%}

\latexProblemContent{
\ifVerboseLocation This is Derivative Compute Question 0005. \\ \fi
\begin{problem}

Determine if the limit approaches a finite number, $\pm\infty$, or does not exist. (If the limit does not exist, write DNE)

\input{Limit-Compute-0005.HELP.tex}

\[\lim_{x\to{4}}\dfrac{x^{2} + x - 20}{x^{2} - 16}=\answer{\frac{9}{8}}\]
\end{problem}}%}

\latexProblemContent{
\ifVerboseLocation This is Derivative Compute Question 0005. \\ \fi
\begin{problem}

Determine if the limit approaches a finite number, $\pm\infty$, or does not exist. (If the limit does not exist, write DNE)

\input{Limit-Compute-0005.HELP.tex}

\[\lim_{x\to{5}}\dfrac{x^{2} - 7 \, x + 10}{x^{2} - 25}=\answer{\frac{3}{10}}\]
\end{problem}}%}

\latexProblemContent{
\ifVerboseLocation This is Derivative Compute Question 0005. \\ \fi
\begin{problem}

Determine if the limit approaches a finite number, $\pm\infty$, or does not exist. (If the limit does not exist, write DNE)

\input{Limit-Compute-0005.HELP.tex}

\[\lim_{x\to{-5}}\dfrac{x^{2} + 6 \, x + 5}{x^{2} + x - 20}=\answer{\frac{4}{9}}\]
\end{problem}}%}

\latexProblemContent{
\ifVerboseLocation This is Derivative Compute Question 0005. \\ \fi
\begin{problem}

Determine if the limit approaches a finite number, $\pm\infty$, or does not exist. (If the limit does not exist, write DNE)

\input{Limit-Compute-0005.HELP.tex}

\[\lim_{x\to{1}}\dfrac{x^{2} - 2 \, x + 1}{x^{2} + 5 \, x - 6}=\answer{0}\]
\end{problem}}%}

\latexProblemContent{
\ifVerboseLocation This is Derivative Compute Question 0005. \\ \fi
\begin{problem}

Determine if the limit approaches a finite number, $\pm\infty$, or does not exist. (If the limit does not exist, write DNE)

\input{Limit-Compute-0005.HELP.tex}

\[\lim_{x\to{-1}}\dfrac{x^{2} + 5 \, x + 4}{x^{2} - 1}=\answer{-\frac{3}{2}}\]
\end{problem}}%}

\latexProblemContent{
\ifVerboseLocation This is Derivative Compute Question 0005. \\ \fi
\begin{problem}

Determine if the limit approaches a finite number, $\pm\infty$, or does not exist. (If the limit does not exist, write DNE)

\input{Limit-Compute-0005.HELP.tex}

\[\lim_{x\to{3}}\dfrac{x^{2} - 2 \, x - 3}{x^{2} + x - 12}=\answer{\frac{4}{7}}\]
\end{problem}}%}

\latexProblemContent{
\ifVerboseLocation This is Derivative Compute Question 0005. \\ \fi
\begin{problem}

Determine if the limit approaches a finite number, $\pm\infty$, or does not exist. (If the limit does not exist, write DNE)

\input{Limit-Compute-0005.HELP.tex}

\[\lim_{x\to{1}}\dfrac{x^{2} + 3 \, x - 4}{x^{2} - 3 \, x + 2}=\answer{-5}\]
\end{problem}}%}

\latexProblemContent{
\ifVerboseLocation This is Derivative Compute Question 0005. \\ \fi
\begin{problem}

Determine if the limit approaches a finite number, $\pm\infty$, or does not exist. (If the limit does not exist, write DNE)

\input{Limit-Compute-0005.HELP.tex}

\[\lim_{x\to{-2}}\dfrac{x^{2} + 3 \, x + 2}{x^{2} + 6 \, x + 8}=\answer{-\frac{1}{2}}\]
\end{problem}}%}

\latexProblemContent{
\ifVerboseLocation This is Derivative Compute Question 0005. \\ \fi
\begin{problem}

Determine if the limit approaches a finite number, $\pm\infty$, or does not exist. (If the limit does not exist, write DNE)

\input{Limit-Compute-0005.HELP.tex}

\[\lim_{x\to{-2}}\dfrac{x^{2} - 4}{x^{2} + 8 \, x + 12}=\answer{-1}\]
\end{problem}}%}

\latexProblemContent{
\ifVerboseLocation This is Derivative Compute Question 0005. \\ \fi
\begin{problem}

Determine if the limit approaches a finite number, $\pm\infty$, or does not exist. (If the limit does not exist, write DNE)

\input{Limit-Compute-0005.HELP.tex}

\[\lim_{x\to{5}}\dfrac{x^{2} - 4 \, x - 5}{x^{2} - 8 \, x + 15}=\answer{3}\]
\end{problem}}%}

\latexProblemContent{
\ifVerboseLocation This is Derivative Compute Question 0005. \\ \fi
\begin{problem}

Determine if the limit approaches a finite number, $\pm\infty$, or does not exist. (If the limit does not exist, write DNE)

\input{Limit-Compute-0005.HELP.tex}

\[\lim_{x\to{6}}\dfrac{x^{2} - 7 \, x + 6}{x^{2} - 7 \, x + 6}=\answer{1}\]
\end{problem}}%}

\latexProblemContent{
\ifVerboseLocation This is Derivative Compute Question 0005. \\ \fi
\begin{problem}

Determine if the limit approaches a finite number, $\pm\infty$, or does not exist. (If the limit does not exist, write DNE)

\input{Limit-Compute-0005.HELP.tex}

\[\lim_{x\to{2}}\dfrac{x^{2} - 4 \, x + 4}{x^{2} + 2 \, x - 8}=\answer{0}\]
\end{problem}}%}

\latexProblemContent{
\ifVerboseLocation This is Derivative Compute Question 0005. \\ \fi
\begin{problem}

Determine if the limit approaches a finite number, $\pm\infty$, or does not exist. (If the limit does not exist, write DNE)

\input{Limit-Compute-0005.HELP.tex}

\[\lim_{x\to{-6}}\dfrac{x^{2} + 4 \, x - 12}{x^{2} + 5 \, x - 6}=\answer{\frac{8}{7}}\]
\end{problem}}%}

\latexProblemContent{
\ifVerboseLocation This is Derivative Compute Question 0005. \\ \fi
\begin{problem}

Determine if the limit approaches a finite number, $\pm\infty$, or does not exist. (If the limit does not exist, write DNE)

\input{Limit-Compute-0005.HELP.tex}

\[\lim_{x\to{6}}\dfrac{x^{2} - 11 \, x + 30}{x^{2} - 9 \, x + 18}=\answer{\frac{1}{3}}\]
\end{problem}}%}

\latexProblemContent{
\ifVerboseLocation This is Derivative Compute Question 0005. \\ \fi
\begin{problem}

Determine if the limit approaches a finite number, $\pm\infty$, or does not exist. (If the limit does not exist, write DNE)

\input{Limit-Compute-0005.HELP.tex}

\[\lim_{x\to{3}}\dfrac{x^{2} - 6 \, x + 9}{x^{2} - 9 \, x + 18}=\answer{0}\]
\end{problem}}%}

\latexProblemContent{
\ifVerboseLocation This is Derivative Compute Question 0005. \\ \fi
\begin{problem}

Determine if the limit approaches a finite number, $\pm\infty$, or does not exist. (If the limit does not exist, write DNE)

\input{Limit-Compute-0005.HELP.tex}

\[\lim_{x\to{4}}\dfrac{x^{2} - 2 \, x - 8}{x^{2} - 5 \, x + 4}=\answer{2}\]
\end{problem}}%}

\latexProblemContent{
\ifVerboseLocation This is Derivative Compute Question 0005. \\ \fi
\begin{problem}

Determine if the limit approaches a finite number, $\pm\infty$, or does not exist. (If the limit does not exist, write DNE)

\input{Limit-Compute-0005.HELP.tex}

\[\lim_{x\to{1}}\dfrac{x^{2} + 3 \, x - 4}{x^{2} + x - 2}=\answer{\frac{5}{3}}\]
\end{problem}}%}

\latexProblemContent{
\ifVerboseLocation This is Derivative Compute Question 0005. \\ \fi
\begin{problem}

Determine if the limit approaches a finite number, $\pm\infty$, or does not exist. (If the limit does not exist, write DNE)

\input{Limit-Compute-0005.HELP.tex}

\[\lim_{x\to{4}}\dfrac{x^{2} - 16}{x^{2} + x - 20}=\answer{\frac{8}{9}}\]
\end{problem}}%}

\latexProblemContent{
\ifVerboseLocation This is Derivative Compute Question 0005. \\ \fi
\begin{problem}

Determine if the limit approaches a finite number, $\pm\infty$, or does not exist. (If the limit does not exist, write DNE)

\input{Limit-Compute-0005.HELP.tex}

\[\lim_{x\to{4}}\dfrac{x^{2} - 5 \, x + 4}{x^{2} - 2 \, x - 8}=\answer{\frac{1}{2}}\]
\end{problem}}%}

\latexProblemContent{
\ifVerboseLocation This is Derivative Compute Question 0005. \\ \fi
\begin{problem}

Determine if the limit approaches a finite number, $\pm\infty$, or does not exist. (If the limit does not exist, write DNE)

\input{Limit-Compute-0005.HELP.tex}

\[\lim_{x\to{6}}\dfrac{x^{2} - 5 \, x - 6}{x^{2} - 2 \, x - 24}=\answer{\frac{7}{10}}\]
\end{problem}}%}

\latexProblemContent{
\ifVerboseLocation This is Derivative Compute Question 0005. \\ \fi
\begin{problem}

Determine if the limit approaches a finite number, $\pm\infty$, or does not exist. (If the limit does not exist, write DNE)

\input{Limit-Compute-0005.HELP.tex}

\[\lim_{x\to{6}}\dfrac{x^{2} - 11 \, x + 30}{x^{2} - 8 \, x + 12}=\answer{\frac{1}{4}}\]
\end{problem}}%}

\latexProblemContent{
\ifVerboseLocation This is Derivative Compute Question 0005. \\ \fi
\begin{problem}

Determine if the limit approaches a finite number, $\pm\infty$, or does not exist. (If the limit does not exist, write DNE)

\input{Limit-Compute-0005.HELP.tex}

\[\lim_{x\to{1}}\dfrac{x^{2} + 3 \, x - 4}{x^{2} + 2 \, x - 3}=\answer{\frac{5}{4}}\]
\end{problem}}%}

\latexProblemContent{
\ifVerboseLocation This is Derivative Compute Question 0005. \\ \fi
\begin{problem}

Determine if the limit approaches a finite number, $\pm\infty$, or does not exist. (If the limit does not exist, write DNE)

\input{Limit-Compute-0005.HELP.tex}

\[\lim_{x\to{-4}}\dfrac{x^{2} + x - 12}{x^{2} + x - 12}=\answer{1}\]
\end{problem}}%}

\latexProblemContent{
\ifVerboseLocation This is Derivative Compute Question 0005. \\ \fi
\begin{problem}

Determine if the limit approaches a finite number, $\pm\infty$, or does not exist. (If the limit does not exist, write DNE)

\input{Limit-Compute-0005.HELP.tex}

\[\lim_{x\to{4}}\dfrac{x^{2} - 16}{x^{2} - 10 \, x + 24}=\answer{-4}\]
\end{problem}}%}

\latexProblemContent{
\ifVerboseLocation This is Derivative Compute Question 0005. \\ \fi
\begin{problem}

Determine if the limit approaches a finite number, $\pm\infty$, or does not exist. (If the limit does not exist, write DNE)

\input{Limit-Compute-0005.HELP.tex}

\[\lim_{x\to{-6}}\dfrac{x^{2} + 4 \, x - 12}{x^{2} - 36}=\answer{\frac{2}{3}}\]
\end{problem}}%}

\latexProblemContent{
\ifVerboseLocation This is Derivative Compute Question 0005. \\ \fi
\begin{problem}

Determine if the limit approaches a finite number, $\pm\infty$, or does not exist. (If the limit does not exist, write DNE)

\input{Limit-Compute-0005.HELP.tex}

\[\lim_{x\to{6}}\dfrac{x^{2} - 8 \, x + 12}{x^{2} - 36}=\answer{\frac{1}{3}}\]
\end{problem}}%}

\latexProblemContent{
\ifVerboseLocation This is Derivative Compute Question 0005. \\ \fi
\begin{problem}

Determine if the limit approaches a finite number, $\pm\infty$, or does not exist. (If the limit does not exist, write DNE)

\input{Limit-Compute-0005.HELP.tex}

\[\lim_{x\to{-5}}\dfrac{x^{2} + 3 \, x - 10}{x^{2} + 4 \, x - 5}=\answer{\frac{7}{6}}\]
\end{problem}}%}

\latexProblemContent{
\ifVerboseLocation This is Derivative Compute Question 0005. \\ \fi
\begin{problem}

Determine if the limit approaches a finite number, $\pm\infty$, or does not exist. (If the limit does not exist, write DNE)

\input{Limit-Compute-0005.HELP.tex}

\[\lim_{x\to{-3}}\dfrac{x^{2} + 6 \, x + 9}{x^{2} + 7 \, x + 12}=\answer{0}\]
\end{problem}}%}

\latexProblemContent{
\ifVerboseLocation This is Derivative Compute Question 0005. \\ \fi
\begin{problem}

Determine if the limit approaches a finite number, $\pm\infty$, or does not exist. (If the limit does not exist, write DNE)

\input{Limit-Compute-0005.HELP.tex}

\[\lim_{x\to{-6}}\dfrac{x^{2} + 2 \, x - 24}{x^{2} + 8 \, x + 12}=\answer{\frac{5}{2}}\]
\end{problem}}%}

\latexProblemContent{
\ifVerboseLocation This is Derivative Compute Question 0005. \\ \fi
\begin{problem}

Determine if the limit approaches a finite number, $\pm\infty$, or does not exist. (If the limit does not exist, write DNE)

\input{Limit-Compute-0005.HELP.tex}

\[\lim_{x\to{-5}}\dfrac{x^{2} + 3 \, x - 10}{x^{2} - x - 30}=\answer{\frac{7}{11}}\]
\end{problem}}%}

\latexProblemContent{
\ifVerboseLocation This is Derivative Compute Question 0005. \\ \fi
\begin{problem}

Determine if the limit approaches a finite number, $\pm\infty$, or does not exist. (If the limit does not exist, write DNE)

\input{Limit-Compute-0005.HELP.tex}

\[\lim_{x\to{6}}\dfrac{x^{2} - 11 \, x + 30}{x^{2} - 2 \, x - 24}=\answer{\frac{1}{10}}\]
\end{problem}}%}

\latexProblemContent{
\ifVerboseLocation This is Derivative Compute Question 0005. \\ \fi
\begin{problem}

Determine if the limit approaches a finite number, $\pm\infty$, or does not exist. (If the limit does not exist, write DNE)

\input{Limit-Compute-0005.HELP.tex}

\[\lim_{x\to{3}}\dfrac{x^{2} - 7 \, x + 12}{x^{2} - 7 \, x + 12}=\answer{1}\]
\end{problem}}%}

\latexProblemContent{
\ifVerboseLocation This is Derivative Compute Question 0005. \\ \fi
\begin{problem}

Determine if the limit approaches a finite number, $\pm\infty$, or does not exist. (If the limit does not exist, write DNE)

\input{Limit-Compute-0005.HELP.tex}

\[\lim_{x\to{-2}}\dfrac{x^{2} + 6 \, x + 8}{x^{2} - 4 \, x - 12}=\answer{-\frac{1}{4}}\]
\end{problem}}%}

\latexProblemContent{
\ifVerboseLocation This is Derivative Compute Question 0005. \\ \fi
\begin{problem}

Determine if the limit approaches a finite number, $\pm\infty$, or does not exist. (If the limit does not exist, write DNE)

\input{Limit-Compute-0005.HELP.tex}

\[\lim_{x\to{-3}}\dfrac{x^{2} + 4 \, x + 3}{x^{2} - 9}=\answer{\frac{1}{3}}\]
\end{problem}}%}

\latexProblemContent{
\ifVerboseLocation This is Derivative Compute Question 0005. \\ \fi
\begin{problem}

Determine if the limit approaches a finite number, $\pm\infty$, or does not exist. (If the limit does not exist, write DNE)

\input{Limit-Compute-0005.HELP.tex}

\[\lim_{x\to{5}}\dfrac{x^{2} - 10 \, x + 25}{x^{2} - 2 \, x - 15}=\answer{0}\]
\end{problem}}%}

\latexProblemContent{
\ifVerboseLocation This is Derivative Compute Question 0005. \\ \fi
\begin{problem}

Determine if the limit approaches a finite number, $\pm\infty$, or does not exist. (If the limit does not exist, write DNE)

\input{Limit-Compute-0005.HELP.tex}

\[\lim_{x\to{-2}}\dfrac{x^{2} + x - 2}{x^{2} + 7 \, x + 10}=\answer{-1}\]
\end{problem}}%}

\latexProblemContent{
\ifVerboseLocation This is Derivative Compute Question 0005. \\ \fi
\begin{problem}

Determine if the limit approaches a finite number, $\pm\infty$, or does not exist. (If the limit does not exist, write DNE)

\input{Limit-Compute-0005.HELP.tex}

\[\lim_{x\to{2}}\dfrac{x^{2} - 6 \, x + 8}{x^{2} + 3 \, x - 10}=\answer{-\frac{2}{7}}\]
\end{problem}}%}

\latexProblemContent{
\ifVerboseLocation This is Derivative Compute Question 0005. \\ \fi
\begin{problem}

Determine if the limit approaches a finite number, $\pm\infty$, or does not exist. (If the limit does not exist, write DNE)

\input{Limit-Compute-0005.HELP.tex}

\[\lim_{x\to{2}}\dfrac{x^{2} - 3 \, x + 2}{x^{2} - 6 \, x + 8}=\answer{-\frac{1}{2}}\]
\end{problem}}%}

\latexProblemContent{
\ifVerboseLocation This is Derivative Compute Question 0005. \\ \fi
\begin{problem}

Determine if the limit approaches a finite number, $\pm\infty$, or does not exist. (If the limit does not exist, write DNE)

\input{Limit-Compute-0005.HELP.tex}

\[\lim_{x\to{3}}\dfrac{x^{2} - 7 \, x + 12}{x^{2} - 4 \, x + 3}=\answer{-\frac{1}{2}}\]
\end{problem}}%}

\latexProblemContent{
\ifVerboseLocation This is Derivative Compute Question 0005. \\ \fi
\begin{problem}

Determine if the limit approaches a finite number, $\pm\infty$, or does not exist. (If the limit does not exist, write DNE)

\input{Limit-Compute-0005.HELP.tex}

\[\lim_{x\to{-5}}\dfrac{x^{2} + 6 \, x + 5}{x^{2} + 9 \, x + 20}=\answer{4}\]
\end{problem}}%}

\latexProblemContent{
\ifVerboseLocation This is Derivative Compute Question 0005. \\ \fi
\begin{problem}

Determine if the limit approaches a finite number, $\pm\infty$, or does not exist. (If the limit does not exist, write DNE)

\input{Limit-Compute-0005.HELP.tex}

\[\lim_{x\to{-5}}\dfrac{x^{2} + 9 \, x + 20}{x^{2} + 3 \, x - 10}=\answer{\frac{1}{7}}\]
\end{problem}}%}

\latexProblemContent{
\ifVerboseLocation This is Derivative Compute Question 0005. \\ \fi
\begin{problem}

Determine if the limit approaches a finite number, $\pm\infty$, or does not exist. (If the limit does not exist, write DNE)

\input{Limit-Compute-0005.HELP.tex}

\[\lim_{x\to{-3}}\dfrac{x^{2} + 7 \, x + 12}{x^{2} + 2 \, x - 3}=\answer{-\frac{1}{4}}\]
\end{problem}}%}

\latexProblemContent{
\ifVerboseLocation This is Derivative Compute Question 0005. \\ \fi
\begin{problem}

Determine if the limit approaches a finite number, $\pm\infty$, or does not exist. (If the limit does not exist, write DNE)

\input{Limit-Compute-0005.HELP.tex}

\[\lim_{x\to{2}}\dfrac{x^{2} - x - 2}{x^{2} - 6 \, x + 8}=\answer{-\frac{3}{2}}\]
\end{problem}}%}

\latexProblemContent{
\ifVerboseLocation This is Derivative Compute Question 0005. \\ \fi
\begin{problem}

Determine if the limit approaches a finite number, $\pm\infty$, or does not exist. (If the limit does not exist, write DNE)

\input{Limit-Compute-0005.HELP.tex}

\[\lim_{x\to{5}}\dfrac{x^{2} - 6 \, x + 5}{x^{2} - 7 \, x + 10}=\answer{\frac{4}{3}}\]
\end{problem}}%}

\latexProblemContent{
\ifVerboseLocation This is Derivative Compute Question 0005. \\ \fi
\begin{problem}

Determine if the limit approaches a finite number, $\pm\infty$, or does not exist. (If the limit does not exist, write DNE)

\input{Limit-Compute-0005.HELP.tex}

\[\lim_{x\to{5}}\dfrac{x^{2} - 4 \, x - 5}{x^{2} - 9 \, x + 20}=\answer{6}\]
\end{problem}}%}

\latexProblemContent{
\ifVerboseLocation This is Derivative Compute Question 0005. \\ \fi
\begin{problem}

Determine if the limit approaches a finite number, $\pm\infty$, or does not exist. (If the limit does not exist, write DNE)

\input{Limit-Compute-0005.HELP.tex}

\[\lim_{x\to{2}}\dfrac{x^{2} - 6 \, x + 8}{x^{2} - 5 \, x + 6}=\answer{2}\]
\end{problem}}%}

\latexProblemContent{
\ifVerboseLocation This is Derivative Compute Question 0005. \\ \fi
\begin{problem}

Determine if the limit approaches a finite number, $\pm\infty$, or does not exist. (If the limit does not exist, write DNE)

\input{Limit-Compute-0005.HELP.tex}

\[\lim_{x\to{-6}}\dfrac{x^{2} + 8 \, x + 12}{x^{2} - 36}=\answer{\frac{1}{3}}\]
\end{problem}}%}

\latexProblemContent{
\ifVerboseLocation This is Derivative Compute Question 0005. \\ \fi
\begin{problem}

Determine if the limit approaches a finite number, $\pm\infty$, or does not exist. (If the limit does not exist, write DNE)

\input{Limit-Compute-0005.HELP.tex}

\[\lim_{x\to{-5}}\dfrac{x^{2} + 9 \, x + 20}{x^{2} + 8 \, x + 15}=\answer{\frac{1}{2}}\]
\end{problem}}%}

\latexProblemContent{
\ifVerboseLocation This is Derivative Compute Question 0005. \\ \fi
\begin{problem}

Determine if the limit approaches a finite number, $\pm\infty$, or does not exist. (If the limit does not exist, write DNE)

\input{Limit-Compute-0005.HELP.tex}

\[\lim_{x\to{-1}}\dfrac{x^{2} - 3 \, x - 4}{x^{2} - 2 \, x - 3}=\answer{\frac{5}{4}}\]
\end{problem}}%}

\latexProblemContent{
\ifVerboseLocation This is Derivative Compute Question 0005. \\ \fi
\begin{problem}

Determine if the limit approaches a finite number, $\pm\infty$, or does not exist. (If the limit does not exist, write DNE)

\input{Limit-Compute-0005.HELP.tex}

\[\lim_{x\to{-1}}\dfrac{x^{2} + 2 \, x + 1}{x^{2} + 4 \, x + 3}=\answer{0}\]
\end{problem}}%}

\latexProblemContent{
\ifVerboseLocation This is Derivative Compute Question 0005. \\ \fi
\begin{problem}

Determine if the limit approaches a finite number, $\pm\infty$, or does not exist. (If the limit does not exist, write DNE)

\input{Limit-Compute-0005.HELP.tex}

\[\lim_{x\to{-1}}\dfrac{x^{2} - x - 2}{x^{2} - 4 \, x - 5}=\answer{\frac{1}{2}}\]
\end{problem}}%}

\latexProblemContent{
\ifVerboseLocation This is Derivative Compute Question 0005. \\ \fi
\begin{problem}

Determine if the limit approaches a finite number, $\pm\infty$, or does not exist. (If the limit does not exist, write DNE)

\input{Limit-Compute-0005.HELP.tex}

\[\lim_{x\to{6}}\dfrac{x^{2} - 9 \, x + 18}{x^{2} - 9 \, x + 18}=\answer{1}\]
\end{problem}}%}

\latexProblemContent{
\ifVerboseLocation This is Derivative Compute Question 0005. \\ \fi
\begin{problem}

Determine if the limit approaches a finite number, $\pm\infty$, or does not exist. (If the limit does not exist, write DNE)

\input{Limit-Compute-0005.HELP.tex}

\[\lim_{x\to{5}}\dfrac{x^{2} - 25}{x^{2} - 9 \, x + 20}=\answer{10}\]
\end{problem}}%}

\latexProblemContent{
\ifVerboseLocation This is Derivative Compute Question 0005. \\ \fi
\begin{problem}

Determine if the limit approaches a finite number, $\pm\infty$, or does not exist. (If the limit does not exist, write DNE)

\input{Limit-Compute-0005.HELP.tex}

\[\lim_{x\to{6}}\dfrac{x^{2} - 4 \, x - 12}{x^{2} - 36}=\answer{\frac{2}{3}}\]
\end{problem}}%}

\latexProblemContent{
\ifVerboseLocation This is Derivative Compute Question 0005. \\ \fi
\begin{problem}

Determine if the limit approaches a finite number, $\pm\infty$, or does not exist. (If the limit does not exist, write DNE)

\input{Limit-Compute-0005.HELP.tex}

\[\lim_{x\to{-5}}\dfrac{x^{2} + 8 \, x + 15}{x^{2} - x - 30}=\answer{\frac{2}{11}}\]
\end{problem}}%}

\latexProblemContent{
\ifVerboseLocation This is Derivative Compute Question 0005. \\ \fi
\begin{problem}

Determine if the limit approaches a finite number, $\pm\infty$, or does not exist. (If the limit does not exist, write DNE)

\input{Limit-Compute-0005.HELP.tex}

\[\lim_{x\to{4}}\dfrac{x^{2} - 2 \, x - 8}{x^{2} - 7 \, x + 12}=\answer{6}\]
\end{problem}}%}

\latexProblemContent{
\ifVerboseLocation This is Derivative Compute Question 0005. \\ \fi
\begin{problem}

Determine if the limit approaches a finite number, $\pm\infty$, or does not exist. (If the limit does not exist, write DNE)

\input{Limit-Compute-0005.HELP.tex}

\[\lim_{x\to{-5}}\dfrac{x^{2} + 3 \, x - 10}{x^{2} + 6 \, x + 5}=\answer{\frac{7}{4}}\]
\end{problem}}%}

\latexProblemContent{
\ifVerboseLocation This is Derivative Compute Question 0005. \\ \fi
\begin{problem}

Determine if the limit approaches a finite number, $\pm\infty$, or does not exist. (If the limit does not exist, write DNE)

\input{Limit-Compute-0005.HELP.tex}

\[\lim_{x\to{4}}\dfrac{x^{2} - 9 \, x + 20}{x^{2} - 9 \, x + 20}=\answer{1}\]
\end{problem}}%}

\latexProblemContent{
\ifVerboseLocation This is Derivative Compute Question 0005. \\ \fi
\begin{problem}

Determine if the limit approaches a finite number, $\pm\infty$, or does not exist. (If the limit does not exist, write DNE)

\input{Limit-Compute-0005.HELP.tex}

\[\lim_{x\to{-4}}\dfrac{x^{2} + x - 12}{x^{2} + 7 \, x + 12}=\answer{7}\]
\end{problem}}%}

\latexProblemContent{
\ifVerboseLocation This is Derivative Compute Question 0005. \\ \fi
\begin{problem}

Determine if the limit approaches a finite number, $\pm\infty$, or does not exist. (If the limit does not exist, write DNE)

\input{Limit-Compute-0005.HELP.tex}

\[\lim_{x\to{2}}\dfrac{x^{2} + x - 6}{x^{2} + 3 \, x - 10}=\answer{\frac{5}{7}}\]
\end{problem}}%}

\latexProblemContent{
\ifVerboseLocation This is Derivative Compute Question 0005. \\ \fi
\begin{problem}

Determine if the limit approaches a finite number, $\pm\infty$, or does not exist. (If the limit does not exist, write DNE)

\input{Limit-Compute-0005.HELP.tex}

\[\lim_{x\to{4}}\dfrac{x^{2} - x - 12}{x^{2} - 3 \, x - 4}=\answer{\frac{7}{5}}\]
\end{problem}}%}

\latexProblemContent{
\ifVerboseLocation This is Derivative Compute Question 0005. \\ \fi
\begin{problem}

Determine if the limit approaches a finite number, $\pm\infty$, or does not exist. (If the limit does not exist, write DNE)

\input{Limit-Compute-0005.HELP.tex}

\[\lim_{x\to{1}}\dfrac{x^{2} - 4 \, x + 3}{x^{2} - 4 \, x + 3}=\answer{1}\]
\end{problem}}%}

\latexProblemContent{
\ifVerboseLocation This is Derivative Compute Question 0005. \\ \fi
\begin{problem}

Determine if the limit approaches a finite number, $\pm\infty$, or does not exist. (If the limit does not exist, write DNE)

\input{Limit-Compute-0005.HELP.tex}

\[\lim_{x\to{5}}\dfrac{x^{2} - 6 \, x + 5}{x^{2} - 11 \, x + 30}=\answer{-4}\]
\end{problem}}%}

\latexProblemContent{
\ifVerboseLocation This is Derivative Compute Question 0005. \\ \fi
\begin{problem}

Determine if the limit approaches a finite number, $\pm\infty$, or does not exist. (If the limit does not exist, write DNE)

\input{Limit-Compute-0005.HELP.tex}

\[\lim_{x\to{-1}}\dfrac{x^{2} + 4 \, x + 3}{x^{2} + 6 \, x + 5}=\answer{\frac{1}{2}}\]
\end{problem}}%}

\latexProblemContent{
\ifVerboseLocation This is Derivative Compute Question 0005. \\ \fi
\begin{problem}

Determine if the limit approaches a finite number, $\pm\infty$, or does not exist. (If the limit does not exist, write DNE)

\input{Limit-Compute-0005.HELP.tex}

\[\lim_{x\to{4}}\dfrac{x^{2} - 8 \, x + 16}{x^{2} - 5 \, x + 4}=\answer{0}\]
\end{problem}}%}

\latexProblemContent{
\ifVerboseLocation This is Derivative Compute Question 0005. \\ \fi
\begin{problem}

Determine if the limit approaches a finite number, $\pm\infty$, or does not exist. (If the limit does not exist, write DNE)

\input{Limit-Compute-0005.HELP.tex}

\[\lim_{x\to{3}}\dfrac{x^{2} - 7 \, x + 12}{x^{2} - 5 \, x + 6}=\answer{-1}\]
\end{problem}}%}

\latexProblemContent{
\ifVerboseLocation This is Derivative Compute Question 0005. \\ \fi
\begin{problem}

Determine if the limit approaches a finite number, $\pm\infty$, or does not exist. (If the limit does not exist, write DNE)

\input{Limit-Compute-0005.HELP.tex}

\[\lim_{x\to{3}}\dfrac{x^{2} - 8 \, x + 15}{x^{2} - 9 \, x + 18}=\answer{\frac{2}{3}}\]
\end{problem}}%}

\latexProblemContent{
\ifVerboseLocation This is Derivative Compute Question 0005. \\ \fi
\begin{problem}

Determine if the limit approaches a finite number, $\pm\infty$, or does not exist. (If the limit does not exist, write DNE)

\input{Limit-Compute-0005.HELP.tex}

\[\lim_{x\to{6}}\dfrac{x^{2} - 3 \, x - 18}{x^{2} - 36}=\answer{\frac{3}{4}}\]
\end{problem}}%}

\latexProblemContent{
\ifVerboseLocation This is Derivative Compute Question 0005. \\ \fi
\begin{problem}

Determine if the limit approaches a finite number, $\pm\infty$, or does not exist. (If the limit does not exist, write DNE)

\input{Limit-Compute-0005.HELP.tex}

\[\lim_{x\to{1}}\dfrac{x^{2} - 6 \, x + 5}{x^{2} + 3 \, x - 4}=\answer{-\frac{4}{5}}\]
\end{problem}}%}

\latexProblemContent{
\ifVerboseLocation This is Derivative Compute Question 0005. \\ \fi
\begin{problem}

Determine if the limit approaches a finite number, $\pm\infty$, or does not exist. (If the limit does not exist, write DNE)

\input{Limit-Compute-0005.HELP.tex}

\[\lim_{x\to{-2}}\dfrac{x^{2} + 6 \, x + 8}{x^{2} - 4}=\answer{-\frac{1}{2}}\]
\end{problem}}%}

