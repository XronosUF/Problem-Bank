\ProblemFileHeader{XTL_SV_QUESTIONCOUNT}% Process how many problems are in this file and how to detect if it has a desirable problem
\ifproblemToFind% If it has a desirable problem search the file.
%%\tagged{Ans@ShortAns, Type@Compute, Topic@Integral, Sub@Definite, Sub@Theorems-FTC, Func@Poly, File@0011}{
\latexProblemContent{
\ifVerboseLocation This is Integration Compute Question 0011. \\ \fi
\begin{problem}

Use the Fundamental Theorem of Calculus to evaluate the integral.

\input{Integral-Compute-0011.HELP.tex}

\[
\int_{8}^{19} {\frac{x - 3}{8 \, \sqrt{x}}}\;dx = \answer{\frac{1}{24} \, \left(4 \, \sqrt{2}\right) + \frac{5}{6} \, \sqrt{19}}
\]
\end{problem}}%}

\latexProblemContent{
\ifVerboseLocation This is Integration Compute Question 0011. \\ \fi
\begin{problem}

Use the Fundamental Theorem of Calculus to evaluate the integral.

\input{Integral-Compute-0011.HELP.tex}

\[
\int_{16}^{21} {\frac{x + 1}{5 \, \sqrt{x}}}\;dx = \answer{\frac{16}{5} \, \sqrt{21} - \frac{152}{15}}
\]
\end{problem}}%}

\latexProblemContent{
\ifVerboseLocation This is Integration Compute Question 0011. \\ \fi
\begin{problem}

Use the Fundamental Theorem of Calculus to evaluate the integral.

\input{Integral-Compute-0011.HELP.tex}

\[
\int_{16}^{17} {-\frac{x + 3}{6 \, \sqrt{x}}}\;dx = \answer{-\frac{26}{9} \, \sqrt{17} + \frac{100}{9}}
\]
\end{problem}}%}

\latexProblemContent{
\ifVerboseLocation This is Integration Compute Question 0011. \\ \fi
\begin{problem}

Use the Fundamental Theorem of Calculus to evaluate the integral.

\input{Integral-Compute-0011.HELP.tex}

\[
\int_{4}^{4} {-\frac{x - 3}{3 \, \sqrt{x}}}\;dx = \answer{0}
\]
\end{problem}}%}

\latexProblemContent{
\ifVerboseLocation This is Integration Compute Question 0011. \\ \fi
\begin{problem}

Use the Fundamental Theorem of Calculus to evaluate the integral.

\input{Integral-Compute-0011.HELP.tex}

\[
\int_{6}^{14} {\frac{x + 2}{4 \, \sqrt{x}}}\;dx = \answer{\frac{10}{3} \, \sqrt{14} - 2 \, \sqrt{6}}
\]
\end{problem}}%}

\latexProblemContent{
\ifVerboseLocation This is Integration Compute Question 0011. \\ \fi
\begin{problem}

Use the Fundamental Theorem of Calculus to evaluate the integral.

\input{Integral-Compute-0011.HELP.tex}

\[
\int_{7}^{11} {\frac{x - 2}{3 \, \sqrt{x}}}\;dx = \answer{\frac{10}{9} \, \sqrt{11} - \frac{2}{9} \, \sqrt{7}}
\]
\end{problem}}%}

\latexProblemContent{
\ifVerboseLocation This is Integration Compute Question 0011. \\ \fi
\begin{problem}

Use the Fundamental Theorem of Calculus to evaluate the integral.

\input{Integral-Compute-0011.HELP.tex}

\[
\int_{15}^{17} {\frac{x - 3}{4 \, \sqrt{x}}}\;dx = \answer{\frac{4}{3} \, \sqrt{17} - \sqrt{15}}
\]
\end{problem}}%}

\latexProblemContent{
\ifVerboseLocation This is Integration Compute Question 0011. \\ \fi
\begin{problem}

Use the Fundamental Theorem of Calculus to evaluate the integral.

\input{Integral-Compute-0011.HELP.tex}

\[
\int_{7}^{14} {\frac{x - 1}{7 \, \sqrt{x}}}\;dx = \answer{\frac{22}{21} \, \sqrt{14} - \frac{8}{21} \, \sqrt{7}}
\]
\end{problem}}%}

\latexProblemContent{
\ifVerboseLocation This is Integration Compute Question 0011. \\ \fi
\begin{problem}

Use the Fundamental Theorem of Calculus to evaluate the integral.

\input{Integral-Compute-0011.HELP.tex}

\[
\int_{18}^{18} {-\frac{x - 1}{4 \, \sqrt{x}}}\;dx = \answer{0}
\]
\end{problem}}%}

\latexProblemContent{
\ifVerboseLocation This is Integration Compute Question 0011. \\ \fi
\begin{problem}

Use the Fundamental Theorem of Calculus to evaluate the integral.

\input{Integral-Compute-0011.HELP.tex}

\[
\int_{6}^{20} {-\frac{x - 4}{6 \, \sqrt{x}}}\;dx = \answer{-\frac{2}{3} \, \sqrt{6} - \frac{16}{9} \, \sqrt{5}}
\]
\end{problem}}%}

\latexProblemContent{
\ifVerboseLocation This is Integration Compute Question 0011. \\ \fi
\begin{problem}

Use the Fundamental Theorem of Calculus to evaluate the integral.

\input{Integral-Compute-0011.HELP.tex}

\[
\int_{9}^{16} {-\frac{x + 1}{\sqrt{x}}}\;dx = \answer{-\frac{80}{3}}
\]
\end{problem}}%}

\latexProblemContent{
\ifVerboseLocation This is Integration Compute Question 0011. \\ \fi
\begin{problem}

Use the Fundamental Theorem of Calculus to evaluate the integral.

\input{Integral-Compute-0011.HELP.tex}

\[
\int_{20}^{21} {\frac{x - 4}{4 \, \sqrt{x}}}\;dx = \answer{\frac{3}{2} \, \sqrt{21} - \frac{8}{3} \, \sqrt{5}}
\]
\end{problem}}%}

\latexProblemContent{
\ifVerboseLocation This is Integration Compute Question 0011. \\ \fi
\begin{problem}

Use the Fundamental Theorem of Calculus to evaluate the integral.

\input{Integral-Compute-0011.HELP.tex}

\[
\int_{15}^{19} {\frac{x - 1}{5 \, \sqrt{x}}}\;dx = \answer{\frac{32}{15} \, \sqrt{19} - \frac{8}{5} \, \sqrt{15}}
\]
\end{problem}}%}

\latexProblemContent{
\ifVerboseLocation This is Integration Compute Question 0011. \\ \fi
\begin{problem}

Use the Fundamental Theorem of Calculus to evaluate the integral.

\input{Integral-Compute-0011.HELP.tex}

\[
\int_{6}^{12} {\frac{x - 1}{\sqrt{x}}}\;dx = \answer{-2 \, \sqrt{6} + 12 \, \sqrt{3}}
\]
\end{problem}}%}

\latexProblemContent{
\ifVerboseLocation This is Integration Compute Question 0011. \\ \fi
\begin{problem}

Use the Fundamental Theorem of Calculus to evaluate the integral.

\input{Integral-Compute-0011.HELP.tex}

\[
\int_{5}^{12} {\frac{x + 3}{3 \, \sqrt{x}}}\;dx = \answer{-\frac{28}{9} \, \sqrt{5} + \frac{28}{3} \, \sqrt{3}}
\]
\end{problem}}%}

\latexProblemContent{
\ifVerboseLocation This is Integration Compute Question 0011. \\ \fi
\begin{problem}

Use the Fundamental Theorem of Calculus to evaluate the integral.

\input{Integral-Compute-0011.HELP.tex}

\[
\int_{9}^{17} {\frac{x - 5}{7 \, \sqrt{x}}}\;dx = \answer{\frac{4}{21} \, \sqrt{17} + \frac{12}{7}}
\]
\end{problem}}%}

\latexProblemContent{
\ifVerboseLocation This is Integration Compute Question 0011. \\ \fi
\begin{problem}

Use the Fundamental Theorem of Calculus to evaluate the integral.

\input{Integral-Compute-0011.HELP.tex}

\[
\int_{18}^{20} {-\frac{x - 1}{6 \, \sqrt{x}}}\;dx = \answer{-\frac{34}{9} \, \sqrt{5} + 5 \, \sqrt{2}}
\]
\end{problem}}%}

\latexProblemContent{
\ifVerboseLocation This is Integration Compute Question 0011. \\ \fi
\begin{problem}

Use the Fundamental Theorem of Calculus to evaluate the integral.

\input{Integral-Compute-0011.HELP.tex}

\[
\int_{4}^{19} {\frac{x + 4}{8 \, \sqrt{x}}}\;dx = \answer{\frac{31}{12} \, \sqrt{19} - \frac{8}{3}}
\]
\end{problem}}%}

\latexProblemContent{
\ifVerboseLocation This is Integration Compute Question 0011. \\ \fi
\begin{problem}

Use the Fundamental Theorem of Calculus to evaluate the integral.

\input{Integral-Compute-0011.HELP.tex}

\[
\int_{2}^{2} {\frac{x + 5}{3 \, \sqrt{x}}}\;dx = \answer{0}
\]
\end{problem}}%}

\latexProblemContent{
\ifVerboseLocation This is Integration Compute Question 0011. \\ \fi
\begin{problem}

Use the Fundamental Theorem of Calculus to evaluate the integral.

\input{Integral-Compute-0011.HELP.tex}

\[
\int_{9}^{20} {-\frac{x - 3}{5 \, \sqrt{x}}}\;dx = \answer{-\frac{44}{15} \, \sqrt{5}}
\]
\end{problem}}%}

\latexProblemContent{
\ifVerboseLocation This is Integration Compute Question 0011. \\ \fi
\begin{problem}

Use the Fundamental Theorem of Calculus to evaluate the integral.

\input{Integral-Compute-0011.HELP.tex}

\[
\int_{7}^{17} {\frac{x - 4}{4 \, \sqrt{x}}}\;dx = \answer{\frac{5}{6} \, \sqrt{17} + \frac{5}{6} \, \sqrt{7}}
\]
\end{problem}}%}

\latexProblemContent{
\ifVerboseLocation This is Integration Compute Question 0011. \\ \fi
\begin{problem}

Use the Fundamental Theorem of Calculus to evaluate the integral.

\input{Integral-Compute-0011.HELP.tex}

\[
\int_{4}^{11} {\frac{x + 5}{2 \, \sqrt{x}}}\;dx = \answer{\frac{26}{3} \, \sqrt{11} - \frac{38}{3}}
\]
\end{problem}}%}

\latexProblemContent{
\ifVerboseLocation This is Integration Compute Question 0011. \\ \fi
\begin{problem}

Use the Fundamental Theorem of Calculus to evaluate the integral.

\input{Integral-Compute-0011.HELP.tex}

\[
\int_{12}^{12} {-\frac{x + 2}{7 \, \sqrt{x}}}\;dx = \answer{0}
\]
\end{problem}}%}

\latexProblemContent{
\ifVerboseLocation This is Integration Compute Question 0011. \\ \fi
\begin{problem}

Use the Fundamental Theorem of Calculus to evaluate the integral.

\input{Integral-Compute-0011.HELP.tex}

\[
\int_{18}^{21} {-\frac{x + 4}{\sqrt{x}}}\;dx = \answer{-22 \, \sqrt{21} + 60 \, \sqrt{2}}
\]
\end{problem}}%}

\latexProblemContent{
\ifVerboseLocation This is Integration Compute Question 0011. \\ \fi
\begin{problem}

Use the Fundamental Theorem of Calculus to evaluate the integral.

\input{Integral-Compute-0011.HELP.tex}

\[
\int_{7}^{20} {-\frac{x - 5}{8 \, \sqrt{x}}}\;dx = \answer{-\frac{2}{3} \, \sqrt{7} - \frac{5}{6} \, \sqrt{5}}
\]
\end{problem}}%}

\latexProblemContent{
\ifVerboseLocation This is Integration Compute Question 0011. \\ \fi
\begin{problem}

Use the Fundamental Theorem of Calculus to evaluate the integral.

\input{Integral-Compute-0011.HELP.tex}

\[
\int_{16}^{20} {-\frac{x - 5}{7 \, \sqrt{x}}}\;dx = \answer{-\frac{20}{21} \, \sqrt{5} + \frac{8}{21}}
\]
\end{problem}}%}

\latexProblemContent{
\ifVerboseLocation This is Integration Compute Question 0011. \\ \fi
\begin{problem}

Use the Fundamental Theorem of Calculus to evaluate the integral.

\input{Integral-Compute-0011.HELP.tex}

\[
\int_{17}^{17} {\frac{x - 1}{8 \, \sqrt{x}}}\;dx = \answer{0}
\]
\end{problem}}%}

\latexProblemContent{
\ifVerboseLocation This is Integration Compute Question 0011. \\ \fi
\begin{problem}

Use the Fundamental Theorem of Calculus to evaluate the integral.

\input{Integral-Compute-0011.HELP.tex}

\[
\int_{13}^{20} {\frac{x - 4}{4 \, \sqrt{x}}}\;dx = \answer{-\frac{1}{6} \, \sqrt{13} + \frac{8}{3} \, \sqrt{5}}
\]
\end{problem}}%}

\latexProblemContent{
\ifVerboseLocation This is Integration Compute Question 0011. \\ \fi
\begin{problem}

Use the Fundamental Theorem of Calculus to evaluate the integral.

\input{Integral-Compute-0011.HELP.tex}

\[
\int_{15}^{18} {-\frac{x - 4}{5 \, \sqrt{x}}}\;dx = \answer{\frac{2}{5} \, \sqrt{15} - \frac{12}{5} \, \sqrt{2}}
\]
\end{problem}}%}

\latexProblemContent{
\ifVerboseLocation This is Integration Compute Question 0011. \\ \fi
\begin{problem}

Use the Fundamental Theorem of Calculus to evaluate the integral.

\input{Integral-Compute-0011.HELP.tex}

\[
\int_{10}^{18} {-\frac{x - 4}{6 \, \sqrt{x}}}\;dx = \answer{-\frac{2}{9} \, \sqrt{10} - 2 \, \sqrt{2}}
\]
\end{problem}}%}

\latexProblemContent{
\ifVerboseLocation This is Integration Compute Question 0011. \\ \fi
\begin{problem}

Use the Fundamental Theorem of Calculus to evaluate the integral.

\input{Integral-Compute-0011.HELP.tex}

\[
\int_{3}^{7} {\frac{x - 2}{7 \, \sqrt{x}}}\;dx = \answer{\frac{2}{21} \, \sqrt{7} + \frac{2}{7} \, \sqrt{3}}
\]
\end{problem}}%}

\latexProblemContent{
\ifVerboseLocation This is Integration Compute Question 0011. \\ \fi
\begin{problem}

Use the Fundamental Theorem of Calculus to evaluate the integral.

\input{Integral-Compute-0011.HELP.tex}

\[
\int_{10}^{18} {-\frac{x + 3}{8 \, \sqrt{x}}}\;dx = \answer{\frac{19}{12} \, \sqrt{10} - \frac{27}{4} \, \sqrt{2}}
\]
\end{problem}}%}

\latexProblemContent{
\ifVerboseLocation This is Integration Compute Question 0011. \\ \fi
\begin{problem}

Use the Fundamental Theorem of Calculus to evaluate the integral.

\input{Integral-Compute-0011.HELP.tex}

\[
\int_{10}^{13} {-\frac{x - 1}{2 \, \sqrt{x}}}\;dx = \answer{-\frac{10}{3} \, \sqrt{13} + \frac{7}{3} \, \sqrt{10}}
\]
\end{problem}}%}

\latexProblemContent{
\ifVerboseLocation This is Integration Compute Question 0011. \\ \fi
\begin{problem}

Use the Fundamental Theorem of Calculus to evaluate the integral.

\input{Integral-Compute-0011.HELP.tex}

\[
\int_{1}^{11} {\frac{x - 5}{3 \, \sqrt{x}}}\;dx = \answer{-\frac{8}{9} \, \sqrt{11} + \frac{28}{9}}
\]
\end{problem}}%}

\latexProblemContent{
\ifVerboseLocation This is Integration Compute Question 0011. \\ \fi
\begin{problem}

Use the Fundamental Theorem of Calculus to evaluate the integral.

\input{Integral-Compute-0011.HELP.tex}

\[
\int_{11}^{15} {-\frac{x + 1}{2 \, \sqrt{x}}}\;dx = \answer{-6 \, \sqrt{15} + \frac{14}{3} \, \sqrt{11}}
\]
\end{problem}}%}

\latexProblemContent{
\ifVerboseLocation This is Integration Compute Question 0011. \\ \fi
\begin{problem}

Use the Fundamental Theorem of Calculus to evaluate the integral.

\input{Integral-Compute-0011.HELP.tex}

\[
\int_{20}^{21} {-\frac{x + 2}{5 \, \sqrt{x}}}\;dx = \answer{-\frac{18}{5} \, \sqrt{21} + \frac{104}{15} \, \sqrt{5}}
\]
\end{problem}}%}

\latexProblemContent{
\ifVerboseLocation This is Integration Compute Question 0011. \\ \fi
\begin{problem}

Use the Fundamental Theorem of Calculus to evaluate the integral.

\input{Integral-Compute-0011.HELP.tex}

\[
\int_{10}^{15} {-\frac{x + 4}{4 \, \sqrt{x}}}\;dx = \answer{-\frac{9}{2} \, \sqrt{15} + \frac{11}{3} \, \sqrt{10}}
\]
\end{problem}}%}

\latexProblemContent{
\ifVerboseLocation This is Integration Compute Question 0011. \\ \fi
\begin{problem}

Use the Fundamental Theorem of Calculus to evaluate the integral.

\input{Integral-Compute-0011.HELP.tex}

\[
\int_{4}^{17} {\frac{x - 5}{3 \, \sqrt{x}}}\;dx = \answer{\frac{4}{9} \, \sqrt{17} + \frac{44}{9}}
\]
\end{problem}}%}

\latexProblemContent{
\ifVerboseLocation This is Integration Compute Question 0011. \\ \fi
\begin{problem}

Use the Fundamental Theorem of Calculus to evaluate the integral.

\input{Integral-Compute-0011.HELP.tex}

\[
\int_{4}^{12} {-\frac{x - 2}{8 \, \sqrt{x}}}\;dx = \answer{-\sqrt{3} - \frac{1}{3}}
\]
\end{problem}}%}

\latexProblemContent{
\ifVerboseLocation This is Integration Compute Question 0011. \\ \fi
\begin{problem}

Use the Fundamental Theorem of Calculus to evaluate the integral.

\input{Integral-Compute-0011.HELP.tex}

\[
\int_{15}^{21} {-\frac{x + 1}{4 \, \sqrt{x}}}\;dx = \answer{-4 \, \sqrt{21} + 3 \, \sqrt{15}}
\]
\end{problem}}%}

\latexProblemContent{
\ifVerboseLocation This is Integration Compute Question 0011. \\ \fi
\begin{problem}

Use the Fundamental Theorem of Calculus to evaluate the integral.

\input{Integral-Compute-0011.HELP.tex}

\[
\int_{17}^{20} {\frac{x - 2}{7 \, \sqrt{x}}}\;dx = \answer{-\frac{22}{21} \, \sqrt{17} + \frac{8}{3} \, \sqrt{5}}
\]
\end{problem}}%}

\latexProblemContent{
\ifVerboseLocation This is Integration Compute Question 0011. \\ \fi
\begin{problem}

Use the Fundamental Theorem of Calculus to evaluate the integral.

\input{Integral-Compute-0011.HELP.tex}

\[
\int_{19}^{20} {-\frac{x + 5}{4 \, \sqrt{x}}}\;dx = \answer{\frac{17}{3} \, \sqrt{19} - \frac{35}{3} \, \sqrt{5}}
\]
\end{problem}}%}

\latexProblemContent{
\ifVerboseLocation This is Integration Compute Question 0011. \\ \fi
\begin{problem}

Use the Fundamental Theorem of Calculus to evaluate the integral.

\input{Integral-Compute-0011.HELP.tex}

\[
\int_{1}^{6} {-\frac{x + 5}{7 \, \sqrt{x}}}\;dx = \answer{-2 \, \sqrt{6} + \frac{32}{21}}
\]
\end{problem}}%}

\latexProblemContent{
\ifVerboseLocation This is Integration Compute Question 0011. \\ \fi
\begin{problem}

Use the Fundamental Theorem of Calculus to evaluate the integral.

\input{Integral-Compute-0011.HELP.tex}

\[
\int_{9}^{17} {-\frac{x + 1}{7 \, \sqrt{x}}}\;dx = \answer{-\frac{40}{21} \, \sqrt{17} + \frac{24}{7}}
\]
\end{problem}}%}

\latexProblemContent{
\ifVerboseLocation This is Integration Compute Question 0011. \\ \fi
\begin{problem}

Use the Fundamental Theorem of Calculus to evaluate the integral.

\input{Integral-Compute-0011.HELP.tex}

\[
\int_{11}^{16} {\frac{x + 1}{2 \, \sqrt{x}}}\;dx = \answer{-\frac{14}{3} \, \sqrt{11} + \frac{76}{3}}
\]
\end{problem}}%}

\latexProblemContent{
\ifVerboseLocation This is Integration Compute Question 0011. \\ \fi
\begin{problem}

Use the Fundamental Theorem of Calculus to evaluate the integral.

\input{Integral-Compute-0011.HELP.tex}

\[
\int_{19}^{20} {-\frac{x - 5}{7 \, \sqrt{x}}}\;dx = \answer{\frac{8}{21} \, \sqrt{19} - \frac{20}{21} \, \sqrt{5}}
\]
\end{problem}}%}

\latexProblemContent{
\ifVerboseLocation This is Integration Compute Question 0011. \\ \fi
\begin{problem}

Use the Fundamental Theorem of Calculus to evaluate the integral.

\input{Integral-Compute-0011.HELP.tex}

\[
\int_{10}^{11} {\frac{x - 2}{3 \, \sqrt{x}}}\;dx = \answer{\frac{10}{9} \, \sqrt{11} - \frac{8}{9} \, \sqrt{10}}
\]
\end{problem}}%}

\latexProblemContent{
\ifVerboseLocation This is Integration Compute Question 0011. \\ \fi
\begin{problem}

Use the Fundamental Theorem of Calculus to evaluate the integral.

\input{Integral-Compute-0011.HELP.tex}

\[
\int_{11}^{12} {\frac{x - 5}{\sqrt{x}}}\;dx = \answer{\frac{8}{3} \, \sqrt{11} - 4 \, \sqrt{3}}
\]
\end{problem}}%}

\latexProblemContent{
\ifVerboseLocation This is Integration Compute Question 0011. \\ \fi
\begin{problem}

Use the Fundamental Theorem of Calculus to evaluate the integral.

\input{Integral-Compute-0011.HELP.tex}

\[
\int_{16}^{19} {-\frac{x - 1}{8 \, \sqrt{x}}}\;dx = \answer{-\frac{4}{3} \, \sqrt{19} + \frac{13}{3}}
\]
\end{problem}}%}

\latexProblemContent{
\ifVerboseLocation This is Integration Compute Question 0011. \\ \fi
\begin{problem}

Use the Fundamental Theorem of Calculus to evaluate the integral.

\input{Integral-Compute-0011.HELP.tex}

\[
\int_{5}^{20} {\frac{x + 4}{5 \, \sqrt{x}}}\;dx = \answer{\frac{94}{15} \, \sqrt{5}}
\]
\end{problem}}%}

\latexProblemContent{
\ifVerboseLocation This is Integration Compute Question 0011. \\ \fi
\begin{problem}

Use the Fundamental Theorem of Calculus to evaluate the integral.

\input{Integral-Compute-0011.HELP.tex}

\[
\int_{20}^{20} {-\frac{x + 4}{\sqrt{x}}}\;dx = \answer{0}
\]
\end{problem}}%}

\latexProblemContent{
\ifVerboseLocation This is Integration Compute Question 0011. \\ \fi
\begin{problem}

Use the Fundamental Theorem of Calculus to evaluate the integral.

\input{Integral-Compute-0011.HELP.tex}

\[
\int_{4}^{21} {-\frac{x - 4}{7 \, \sqrt{x}}}\;dx = \answer{-\frac{6}{7} \, \sqrt{21} - \frac{32}{21}}
\]
\end{problem}}%}

\latexProblemContent{
\ifVerboseLocation This is Integration Compute Question 0011. \\ \fi
\begin{problem}

Use the Fundamental Theorem of Calculus to evaluate the integral.

\input{Integral-Compute-0011.HELP.tex}

\[
\int_{14}^{19} {\frac{x + 3}{8 \, \sqrt{x}}}\;dx = \answer{\frac{7}{3} \, \sqrt{19} - \frac{23}{12} \, \sqrt{14}}
\]
\end{problem}}%}

\latexProblemContent{
\ifVerboseLocation This is Integration Compute Question 0011. \\ \fi
\begin{problem}

Use the Fundamental Theorem of Calculus to evaluate the integral.

\input{Integral-Compute-0011.HELP.tex}

\[
\int_{5}^{6} {\frac{x + 3}{7 \, \sqrt{x}}}\;dx = \answer{\frac{10}{7} \, \sqrt{6} - \frac{4}{3} \, \sqrt{5}}
\]
\end{problem}}%}

\latexProblemContent{
\ifVerboseLocation This is Integration Compute Question 0011. \\ \fi
\begin{problem}

Use the Fundamental Theorem of Calculus to evaluate the integral.

\input{Integral-Compute-0011.HELP.tex}

\[
\int_{14}^{14} {-\frac{x + 4}{6 \, \sqrt{x}}}\;dx = \answer{0}
\]
\end{problem}}%}

\latexProblemContent{
\ifVerboseLocation This is Integration Compute Question 0011. \\ \fi
\begin{problem}

Use the Fundamental Theorem of Calculus to evaluate the integral.

\input{Integral-Compute-0011.HELP.tex}

\[
\int_{13}^{16} {\frac{x + 3}{4 \, \sqrt{x}}}\;dx = \answer{-\frac{11}{3} \, \sqrt{13} + \frac{50}{3}}
\]
\end{problem}}%}

\latexProblemContent{
\ifVerboseLocation This is Integration Compute Question 0011. \\ \fi
\begin{problem}

Use the Fundamental Theorem of Calculus to evaluate the integral.

\input{Integral-Compute-0011.HELP.tex}

\[
\int_{2}^{4} {-\frac{x + 3}{6 \, \sqrt{x}}}\;dx = \answer{\frac{11}{9} \, \sqrt{2} - \frac{26}{9}}
\]
\end{problem}}%}

\latexProblemContent{
\ifVerboseLocation This is Integration Compute Question 0011. \\ \fi
\begin{problem}

Use the Fundamental Theorem of Calculus to evaluate the integral.

\input{Integral-Compute-0011.HELP.tex}

\[
\int_{13}^{17} {\frac{x + 5}{4 \, \sqrt{x}}}\;dx = \answer{\frac{16}{3} \, \sqrt{17} - \frac{14}{3} \, \sqrt{13}}
\]
\end{problem}}%}

\latexProblemContent{
\ifVerboseLocation This is Integration Compute Question 0011. \\ \fi
\begin{problem}

Use the Fundamental Theorem of Calculus to evaluate the integral.

\input{Integral-Compute-0011.HELP.tex}

\[
\int_{1}^{5} {\frac{x + 2}{7 \, \sqrt{x}}}\;dx = \answer{\frac{22}{21} \, \sqrt{5} - \frac{2}{3}}
\]
\end{problem}}%}

\latexProblemContent{
\ifVerboseLocation This is Integration Compute Question 0011. \\ \fi
\begin{problem}

Use the Fundamental Theorem of Calculus to evaluate the integral.

\input{Integral-Compute-0011.HELP.tex}

\[
\int_{6}^{17} {-\frac{x - 1}{4 \, \sqrt{x}}}\;dx = \answer{-\frac{7}{3} \, \sqrt{17} + \frac{1}{2} \, \sqrt{6}}
\]
\end{problem}}%}

\latexProblemContent{
\ifVerboseLocation This is Integration Compute Question 0011. \\ \fi
\begin{problem}

Use the Fundamental Theorem of Calculus to evaluate the integral.

\input{Integral-Compute-0011.HELP.tex}

\[
\int_{12}^{12} {-\frac{x + 1}{2 \, \sqrt{x}}}\;dx = \answer{0}
\]
\end{problem}}%}

\latexProblemContent{
\ifVerboseLocation This is Integration Compute Question 0011. \\ \fi
\begin{problem}

Use the Fundamental Theorem of Calculus to evaluate the integral.

\input{Integral-Compute-0011.HELP.tex}

\[
\int_{16}^{20} {\frac{x - 5}{8 \, \sqrt{x}}}\;dx = \answer{\frac{5}{6} \, \sqrt{5} - \frac{1}{3}}
\]
\end{problem}}%}

\latexProblemContent{
\ifVerboseLocation This is Integration Compute Question 0011. \\ \fi
\begin{problem}

Use the Fundamental Theorem of Calculus to evaluate the integral.

\input{Integral-Compute-0011.HELP.tex}

\[
\int_{3}^{16} {\frac{x + 3}{\sqrt{x}}}\;dx = \answer{-8 \, \sqrt{3} + \frac{200}{3}}
\]
\end{problem}}%}

\latexProblemContent{
\ifVerboseLocation This is Integration Compute Question 0011. \\ \fi
\begin{problem}

Use the Fundamental Theorem of Calculus to evaluate the integral.

\input{Integral-Compute-0011.HELP.tex}

\[
\int_{19}^{21} {\frac{x + 1}{6 \, \sqrt{x}}}\;dx = \answer{\frac{8}{3} \, \sqrt{21} - \frac{22}{9} \, \sqrt{19}}
\]
\end{problem}}%}

\latexProblemContent{
\ifVerboseLocation This is Integration Compute Question 0011. \\ \fi
\begin{problem}

Use the Fundamental Theorem of Calculus to evaluate the integral.

\input{Integral-Compute-0011.HELP.tex}

\[
\int_{19}^{20} {-\frac{x + 4}{4 \, \sqrt{x}}}\;dx = \answer{\frac{31}{6} \, \sqrt{19} - \frac{32}{3} \, \sqrt{5}}
\]
\end{problem}}%}

\latexProblemContent{
\ifVerboseLocation This is Integration Compute Question 0011. \\ \fi
\begin{problem}

Use the Fundamental Theorem of Calculus to evaluate the integral.

\input{Integral-Compute-0011.HELP.tex}

\[
\int_{20}^{20} {-\frac{x + 4}{2 \, \sqrt{x}}}\;dx = \answer{0}
\]
\end{problem}}%}

\latexProblemContent{
\ifVerboseLocation This is Integration Compute Question 0011. \\ \fi
\begin{problem}

Use the Fundamental Theorem of Calculus to evaluate the integral.

\input{Integral-Compute-0011.HELP.tex}

\[
\int_{8}^{16} {\frac{x + 1}{7 \, \sqrt{x}}}\;dx = \answer{-\frac{44}{21} \, \sqrt{2} + \frac{152}{21}}
\]
\end{problem}}%}

\latexProblemContent{
\ifVerboseLocation This is Integration Compute Question 0011. \\ \fi
\begin{problem}

Use the Fundamental Theorem of Calculus to evaluate the integral.

\input{Integral-Compute-0011.HELP.tex}

\[
\int_{8}^{16} {\frac{x + 1}{2 \, \sqrt{x}}}\;dx = \answer{-\frac{22}{3} \, \sqrt{2} + \frac{76}{3}}
\]
\end{problem}}%}

\latexProblemContent{
\ifVerboseLocation This is Integration Compute Question 0011. \\ \fi
\begin{problem}

Use the Fundamental Theorem of Calculus to evaluate the integral.

\input{Integral-Compute-0011.HELP.tex}

\[
\int_{13}^{19} {\frac{x + 4}{7 \, \sqrt{x}}}\;dx = \answer{\frac{62}{21} \, \sqrt{19} - \frac{50}{21} \, \sqrt{13}}
\]
\end{problem}}%}

\latexProblemContent{
\ifVerboseLocation This is Integration Compute Question 0011. \\ \fi
\begin{problem}

Use the Fundamental Theorem of Calculus to evaluate the integral.

\input{Integral-Compute-0011.HELP.tex}

\[
\int_{5}^{8} {\frac{x - 4}{6 \, \sqrt{x}}}\;dx = \answer{\frac{7}{9} \, \sqrt{5} - \frac{8}{9} \, \sqrt{2}}
\]
\end{problem}}%}

\latexProblemContent{
\ifVerboseLocation This is Integration Compute Question 0011. \\ \fi
\begin{problem}

Use the Fundamental Theorem of Calculus to evaluate the integral.

\input{Integral-Compute-0011.HELP.tex}

\[
\int_{11}^{15} {-\frac{x + 1}{4 \, \sqrt{x}}}\;dx = \answer{-3 \, \sqrt{15} + \frac{7}{3} \, \sqrt{11}}
\]
\end{problem}}%}

\latexProblemContent{
\ifVerboseLocation This is Integration Compute Question 0011. \\ \fi
\begin{problem}

Use the Fundamental Theorem of Calculus to evaluate the integral.

\input{Integral-Compute-0011.HELP.tex}

\[
\int_{8}^{12} {-\frac{x + 1}{6 \, \sqrt{x}}}\;dx = \answer{-\frac{10}{3} \, \sqrt{3} + \frac{22}{9} \, \sqrt{2}}
\]
\end{problem}}%}

\latexProblemContent{
\ifVerboseLocation This is Integration Compute Question 0011. \\ \fi
\begin{problem}

Use the Fundamental Theorem of Calculus to evaluate the integral.

\input{Integral-Compute-0011.HELP.tex}

\[
\int_{4}^{17} {\frac{x - 1}{3 \, \sqrt{x}}}\;dx = \answer{\frac{28}{9} \, \sqrt{17} - \frac{4}{9}}
\]
\end{problem}}%}

\latexProblemContent{
\ifVerboseLocation This is Integration Compute Question 0011. \\ \fi
\begin{problem}

Use the Fundamental Theorem of Calculus to evaluate the integral.

\input{Integral-Compute-0011.HELP.tex}

\[
\int_{8}^{12} {-\frac{x + 3}{7 \, \sqrt{x}}}\;dx = \answer{-4 \, \sqrt{3} + \frac{68}{21} \, \sqrt{2}}
\]
\end{problem}}%}

\latexProblemContent{
\ifVerboseLocation This is Integration Compute Question 0011. \\ \fi
\begin{problem}

Use the Fundamental Theorem of Calculus to evaluate the integral.

\input{Integral-Compute-0011.HELP.tex}

\[
\int_{13}^{17} {\frac{x - 3}{\sqrt{x}}}\;dx = \answer{\frac{16}{3} \, \sqrt{17} - \frac{8}{3} \, \sqrt{13}}
\]
\end{problem}}%}

\latexProblemContent{
\ifVerboseLocation This is Integration Compute Question 0011. \\ \fi
\begin{problem}

Use the Fundamental Theorem of Calculus to evaluate the integral.

\input{Integral-Compute-0011.HELP.tex}

\[
\int_{1}^{11} {-\frac{x + 4}{4 \, \sqrt{x}}}\;dx = \answer{-\frac{23}{6} \, \sqrt{11} + \frac{13}{6}}
\]
\end{problem}}%}

\latexProblemContent{
\ifVerboseLocation This is Integration Compute Question 0011. \\ \fi
\begin{problem}

Use the Fundamental Theorem of Calculus to evaluate the integral.

\input{Integral-Compute-0011.HELP.tex}

\[
\int_{14}^{15} {\frac{x - 4}{8 \, \sqrt{x}}}\;dx = \answer{\frac{1}{4} \, \sqrt{15} - \frac{1}{6} \, \sqrt{14}}
\]
\end{problem}}%}

\latexProblemContent{
\ifVerboseLocation This is Integration Compute Question 0011. \\ \fi
\begin{problem}

Use the Fundamental Theorem of Calculus to evaluate the integral.

\input{Integral-Compute-0011.HELP.tex}

\[
\int_{6}^{6} {\frac{x - 5}{\sqrt{x}}}\;dx = \answer{0}
\]
\end{problem}}%}

\latexProblemContent{
\ifVerboseLocation This is Integration Compute Question 0011. \\ \fi
\begin{problem}

Use the Fundamental Theorem of Calculus to evaluate the integral.

\input{Integral-Compute-0011.HELP.tex}

\[
\int_{4}^{15} {-\frac{x - 1}{6 \, \sqrt{x}}}\;dx = \answer{-\frac{4}{3} \, \sqrt{15} + \frac{2}{9}}
\]
\end{problem}}%}

\latexProblemContent{
\ifVerboseLocation This is Integration Compute Question 0011. \\ \fi
\begin{problem}

Use the Fundamental Theorem of Calculus to evaluate the integral.

\input{Integral-Compute-0011.HELP.tex}

\[
\int_{2}^{9} {-\frac{x + 5}{8 \, \sqrt{x}}}\;dx = \answer{\frac{17}{12} \, \sqrt{2} - 6}
\]
\end{problem}}%}

\latexProblemContent{
\ifVerboseLocation This is Integration Compute Question 0011. \\ \fi
\begin{problem}

Use the Fundamental Theorem of Calculus to evaluate the integral.

\input{Integral-Compute-0011.HELP.tex}

\[
\int_{4}^{8} {\frac{x - 1}{7 \, \sqrt{x}}}\;dx = \answer{\frac{20}{21} \, \sqrt{2} - \frac{4}{21}}
\]
\end{problem}}%}

\latexProblemContent{
\ifVerboseLocation This is Integration Compute Question 0011. \\ \fi
\begin{problem}

Use the Fundamental Theorem of Calculus to evaluate the integral.

\input{Integral-Compute-0011.HELP.tex}

\[
\int_{8}^{11} {\frac{x + 4}{6 \, \sqrt{x}}}\;dx = \answer{\frac{23}{9} \, \sqrt{11} - \frac{40}{9} \, \sqrt{2}}
\]
\end{problem}}%}

\latexProblemContent{
\ifVerboseLocation This is Integration Compute Question 0011. \\ \fi
\begin{problem}

Use the Fundamental Theorem of Calculus to evaluate the integral.

\input{Integral-Compute-0011.HELP.tex}

\[
\int_{18}^{21} {-\frac{x + 3}{5 \, \sqrt{x}}}\;dx = \answer{-4 \, \sqrt{21} + \frac{54}{5} \, \sqrt{2}}
\]
\end{problem}}%}

\latexProblemContent{
\ifVerboseLocation This is Integration Compute Question 0011. \\ \fi
\begin{problem}

Use the Fundamental Theorem of Calculus to evaluate the integral.

\input{Integral-Compute-0011.HELP.tex}

\[
\int_{18}^{20} {\frac{x - 3}{5 \, \sqrt{x}}}\;dx = \answer{\frac{44}{15} \, \sqrt{5} - \frac{18}{5} \, \sqrt{2}}
\]
\end{problem}}%}

\latexProblemContent{
\ifVerboseLocation This is Integration Compute Question 0011. \\ \fi
\begin{problem}

Use the Fundamental Theorem of Calculus to evaluate the integral.

\input{Integral-Compute-0011.HELP.tex}

\[
\int_{13}^{13} {\frac{x + 5}{8 \, \sqrt{x}}}\;dx = \answer{0}
\]
\end{problem}}%}

\latexProblemContent{
\ifVerboseLocation This is Integration Compute Question 0011. \\ \fi
\begin{problem}

Use the Fundamental Theorem of Calculus to evaluate the integral.

\input{Integral-Compute-0011.HELP.tex}

\[
\int_{3}^{15} {-\frac{x - 5}{5 \, \sqrt{x}}}\;dx = \answer{-\frac{8}{5} \, \sqrt{3}}
\]
\end{problem}}%}

\latexProblemContent{
\ifVerboseLocation This is Integration Compute Question 0011. \\ \fi
\begin{problem}

Use the Fundamental Theorem of Calculus to evaluate the integral.

\input{Integral-Compute-0011.HELP.tex}

\[
\int_{20}^{20} {-\frac{x - 1}{8 \, \sqrt{x}}}\;dx = \answer{0}
\]
\end{problem}}%}

\latexProblemContent{
\ifVerboseLocation This is Integration Compute Question 0011. \\ \fi
\begin{problem}

Use the Fundamental Theorem of Calculus to evaluate the integral.

\input{Integral-Compute-0011.HELP.tex}

\[
\int_{19}^{21} {-\frac{x - 2}{3 \, \sqrt{x}}}\;dx = \answer{-\frac{10}{3} \, \sqrt{21} + \frac{26}{9} \, \sqrt{19}}
\]
\end{problem}}%}

\latexProblemContent{
\ifVerboseLocation This is Integration Compute Question 0011. \\ \fi
\begin{problem}

Use the Fundamental Theorem of Calculus to evaluate the integral.

\input{Integral-Compute-0011.HELP.tex}

\[
\int_{14}^{19} {-\frac{x + 2}{7 \, \sqrt{x}}}\;dx = \answer{-\frac{50}{21} \, \sqrt{19} + \frac{40}{21} \, \sqrt{14}}
\]
\end{problem}}%}

\latexProblemContent{
\ifVerboseLocation This is Integration Compute Question 0011. \\ \fi
\begin{problem}

Use the Fundamental Theorem of Calculus to evaluate the integral.

\input{Integral-Compute-0011.HELP.tex}

\[
\int_{3}^{19} {\frac{x + 1}{\sqrt{x}}}\;dx = \answer{\frac{44}{3} \, \sqrt{19} - 4 \, \sqrt{3}}
\]
\end{problem}}%}

\latexProblemContent{
\ifVerboseLocation This is Integration Compute Question 0011. \\ \fi
\begin{problem}

Use the Fundamental Theorem of Calculus to evaluate the integral.

\input{Integral-Compute-0011.HELP.tex}

\[
\int_{11}^{15} {\frac{x + 5}{6 \, \sqrt{x}}}\;dx = \answer{\frac{10}{3} \, \sqrt{15} - \frac{26}{9} \, \sqrt{11}}
\]
\end{problem}}%}

\latexProblemContent{
\ifVerboseLocation This is Integration Compute Question 0011. \\ \fi
\begin{problem}

Use the Fundamental Theorem of Calculus to evaluate the integral.

\input{Integral-Compute-0011.HELP.tex}

\[
\int_{18}^{18} {-\frac{x - 3}{3 \, \sqrt{x}}}\;dx = \answer{0}
\]
\end{problem}}%}

\latexProblemContent{
\ifVerboseLocation This is Integration Compute Question 0011. \\ \fi
\begin{problem}

Use the Fundamental Theorem of Calculus to evaluate the integral.

\input{Integral-Compute-0011.HELP.tex}

\[
\int_{3}^{16} {\frac{x + 2}{5 \, \sqrt{x}}}\;dx = \answer{-\frac{6}{5} \, \sqrt{3} + \frac{176}{15}}
\]
\end{problem}}%}

\latexProblemContent{
\ifVerboseLocation This is Integration Compute Question 0011. \\ \fi
\begin{problem}

Use the Fundamental Theorem of Calculus to evaluate the integral.

\input{Integral-Compute-0011.HELP.tex}

\[
\int_{12}^{21} {\frac{x - 1}{7 \, \sqrt{x}}}\;dx = \answer{\frac{12}{7} \, \sqrt{21} - \frac{12}{7} \, \sqrt{3}}
\]
\end{problem}}%}

\latexProblemContent{
\ifVerboseLocation This is Integration Compute Question 0011. \\ \fi
\begin{problem}

Use the Fundamental Theorem of Calculus to evaluate the integral.

\input{Integral-Compute-0011.HELP.tex}

\[
\int_{3}^{19} {\frac{x - 2}{4 \, \sqrt{x}}}\;dx = \answer{\frac{13}{6} \, \sqrt{19} + \frac{1}{2} \, \sqrt{3}}
\]
\end{problem}}%}

\latexProblemContent{
\ifVerboseLocation This is Integration Compute Question 0011. \\ \fi
\begin{problem}

Use the Fundamental Theorem of Calculus to evaluate the integral.

\input{Integral-Compute-0011.HELP.tex}

\[
\int_{2}^{5} {\frac{x + 4}{5 \, \sqrt{x}}}\;dx = \answer{\frac{34}{15} \, \sqrt{5} - \frac{28}{15} \, \sqrt{2}}
\]
\end{problem}}%}

\latexProblemContent{
\ifVerboseLocation This is Integration Compute Question 0011. \\ \fi
\begin{problem}

Use the Fundamental Theorem of Calculus to evaluate the integral.

\input{Integral-Compute-0011.HELP.tex}

\[
\int_{4}^{17} {-\frac{x - 5}{4 \, \sqrt{x}}}\;dx = \answer{-\frac{1}{3} \, \sqrt{17} - \frac{11}{3}}
\]
\end{problem}}%}

\latexProblemContent{
\ifVerboseLocation This is Integration Compute Question 0011. \\ \fi
\begin{problem}

Use the Fundamental Theorem of Calculus to evaluate the integral.

\input{Integral-Compute-0011.HELP.tex}

\[
\int_{17}^{21} {-\frac{x + 3}{4 \, \sqrt{x}}}\;dx = \answer{-5 \, \sqrt{21} + \frac{13}{3} \, \sqrt{17}}
\]
\end{problem}}%}

\latexProblemContent{
\ifVerboseLocation This is Integration Compute Question 0011. \\ \fi
\begin{problem}

Use the Fundamental Theorem of Calculus to evaluate the integral.

\input{Integral-Compute-0011.HELP.tex}

\[
\int_{5}^{15} {-\frac{x - 5}{3 \, \sqrt{x}}}\;dx = \answer{-\frac{20}{9} \, \sqrt{5}}
\]
\end{problem}}%}

\latexProblemContent{
\ifVerboseLocation This is Integration Compute Question 0011. \\ \fi
\begin{problem}

Use the Fundamental Theorem of Calculus to evaluate the integral.

\input{Integral-Compute-0011.HELP.tex}

\[
\int_{8}^{17} {-\frac{x + 1}{6 \, \sqrt{x}}}\;dx = \answer{-\frac{20}{9} \, \sqrt{17} + \frac{22}{9} \, \sqrt{2}}
\]
\end{problem}}%}

\latexProblemContent{
\ifVerboseLocation This is Integration Compute Question 0011. \\ \fi
\begin{problem}

Use the Fundamental Theorem of Calculus to evaluate the integral.

\input{Integral-Compute-0011.HELP.tex}

\[
\int_{9}^{18} {\frac{x - 5}{6 \, \sqrt{x}}}\;dx = \answer{\sqrt{2} + 2}
\]
\end{problem}}%}

\latexProblemContent{
\ifVerboseLocation This is Integration Compute Question 0011. \\ \fi
\begin{problem}

Use the Fundamental Theorem of Calculus to evaluate the integral.

\input{Integral-Compute-0011.HELP.tex}

\[
\int_{18}^{21} {\frac{x + 2}{\sqrt{x}}}\;dx = \answer{18 \, \sqrt{21} - 48 \, \sqrt{2}}
\]
\end{problem}}%}

\latexProblemContent{
\ifVerboseLocation This is Integration Compute Question 0011. \\ \fi
\begin{problem}

Use the Fundamental Theorem of Calculus to evaluate the integral.

\input{Integral-Compute-0011.HELP.tex}

\[
\int_{3}^{18} {\frac{x + 5}{4 \, \sqrt{x}}}\;dx = \answer{-3 \, \sqrt{3} + \frac{33}{2} \, \sqrt{2}}
\]
\end{problem}}%}

\latexProblemContent{
\ifVerboseLocation This is Integration Compute Question 0011. \\ \fi
\begin{problem}

Use the Fundamental Theorem of Calculus to evaluate the integral.

\input{Integral-Compute-0011.HELP.tex}

\[
\int_{19}^{19} {-\frac{x - 3}{3 \, \sqrt{x}}}\;dx = \answer{0}
\]
\end{problem}}%}

\latexProblemContent{
\ifVerboseLocation This is Integration Compute Question 0011. \\ \fi
\begin{problem}

Use the Fundamental Theorem of Calculus to evaluate the integral.

\input{Integral-Compute-0011.HELP.tex}

\[
\int_{9}^{21} {-\frac{x - 2}{3 \, \sqrt{x}}}\;dx = \answer{-\frac{10}{3} \, \sqrt{21} + 2}
\]
\end{problem}}%}

\latexProblemContent{
\ifVerboseLocation This is Integration Compute Question 0011. \\ \fi
\begin{problem}

Use the Fundamental Theorem of Calculus to evaluate the integral.

\input{Integral-Compute-0011.HELP.tex}

\[
\int_{1}^{7} {-\frac{x + 2}{8 \, \sqrt{x}}}\;dx = \answer{-\frac{13}{12} \, \sqrt{7} + \frac{7}{12}}
\]
\end{problem}}%}

\latexProblemContent{
\ifVerboseLocation This is Integration Compute Question 0011. \\ \fi
\begin{problem}

Use the Fundamental Theorem of Calculus to evaluate the integral.

\input{Integral-Compute-0011.HELP.tex}

\[
\int_{15}^{18} {\frac{x + 3}{8 \, \sqrt{x}}}\;dx = \answer{-2 \, \sqrt{15} + \frac{27}{4} \, \sqrt{2}}
\]
\end{problem}}%}

\latexProblemContent{
\ifVerboseLocation This is Integration Compute Question 0011. \\ \fi
\begin{problem}

Use the Fundamental Theorem of Calculus to evaluate the integral.

\input{Integral-Compute-0011.HELP.tex}

\[
\int_{6}^{20} {\frac{x - 3}{3 \, \sqrt{x}}}\;dx = \answer{\frac{1}{9} \, \left(6 \, \sqrt{6}\right) + \frac{44}{9} \, \sqrt{5}}
\]
\end{problem}}%}

\latexProblemContent{
\ifVerboseLocation This is Integration Compute Question 0011. \\ \fi
\begin{problem}

Use the Fundamental Theorem of Calculus to evaluate the integral.

\input{Integral-Compute-0011.HELP.tex}

\[
\int_{17}^{17} {\frac{x - 2}{3 \, \sqrt{x}}}\;dx = \answer{0}
\]
\end{problem}}%}

\latexProblemContent{
\ifVerboseLocation This is Integration Compute Question 0011. \\ \fi
\begin{problem}

Use the Fundamental Theorem of Calculus to evaluate the integral.

\input{Integral-Compute-0011.HELP.tex}

\[
\int_{14}^{20} {\frac{x - 2}{2 \, \sqrt{x}}}\;dx = \answer{-\frac{8}{3} \, \sqrt{14} + \frac{28}{3} \, \sqrt{5}}
\]
\end{problem}}%}

\latexProblemContent{
\ifVerboseLocation This is Integration Compute Question 0011. \\ \fi
\begin{problem}

Use the Fundamental Theorem of Calculus to evaluate the integral.

\input{Integral-Compute-0011.HELP.tex}

\[
\int_{18}^{20} {-\frac{x - 2}{7 \, \sqrt{x}}}\;dx = \answer{-\frac{8}{3} \, \sqrt{5} + \frac{24}{7} \, \sqrt{2}}
\]
\end{problem}}%}

\latexProblemContent{
\ifVerboseLocation This is Integration Compute Question 0011. \\ \fi
\begin{problem}

Use the Fundamental Theorem of Calculus to evaluate the integral.

\input{Integral-Compute-0011.HELP.tex}

\[
\int_{1}^{11} {\frac{x - 2}{2 \, \sqrt{x}}}\;dx = \answer{\frac{5}{3} \, \sqrt{11} + \frac{5}{3}}
\]
\end{problem}}%}

\latexProblemContent{
\ifVerboseLocation This is Integration Compute Question 0011. \\ \fi
\begin{problem}

Use the Fundamental Theorem of Calculus to evaluate the integral.

\input{Integral-Compute-0011.HELP.tex}

\[
\int_{1}^{9} {\frac{x - 1}{5 \, \sqrt{x}}}\;dx = \answer{\frac{8}{3}}
\]
\end{problem}}%}

\latexProblemContent{
\ifVerboseLocation This is Integration Compute Question 0011. \\ \fi
\begin{problem}

Use the Fundamental Theorem of Calculus to evaluate the integral.

\input{Integral-Compute-0011.HELP.tex}

\[
\int_{6}^{21} {-\frac{x + 4}{\sqrt{x}}}\;dx = \answer{-22 \, \sqrt{21} + 12 \, \sqrt{6}}
\]
\end{problem}}%}

\latexProblemContent{
\ifVerboseLocation This is Integration Compute Question 0011. \\ \fi
\begin{problem}

Use the Fundamental Theorem of Calculus to evaluate the integral.

\input{Integral-Compute-0011.HELP.tex}

\[
\int_{1}^{14} {\frac{x + 2}{5 \, \sqrt{x}}}\;dx = \answer{\frac{8}{3} \, \sqrt{14} - \frac{14}{15}}
\]
\end{problem}}%}

\latexProblemContent{
\ifVerboseLocation This is Integration Compute Question 0011. \\ \fi
\begin{problem}

Use the Fundamental Theorem of Calculus to evaluate the integral.

\input{Integral-Compute-0011.HELP.tex}

\[
\int_{11}^{11} {\frac{x - 3}{8 \, \sqrt{x}}}\;dx = \answer{0}
\]
\end{problem}}%}

\latexProblemContent{
\ifVerboseLocation This is Integration Compute Question 0011. \\ \fi
\begin{problem}

Use the Fundamental Theorem of Calculus to evaluate the integral.

\input{Integral-Compute-0011.HELP.tex}

\[
\int_{5}^{5} {\frac{x - 2}{7 \, \sqrt{x}}}\;dx = \answer{0}
\]
\end{problem}}%}

\latexProblemContent{
\ifVerboseLocation This is Integration Compute Question 0011. \\ \fi
\begin{problem}

Use the Fundamental Theorem of Calculus to evaluate the integral.

\input{Integral-Compute-0011.HELP.tex}

\[
\int_{7}^{12} {\frac{x + 3}{3 \, \sqrt{x}}}\;dx = \answer{-\frac{32}{9} \, \sqrt{7} + \frac{28}{3} \, \sqrt{3}}
\]
\end{problem}}%}

\latexProblemContent{
\ifVerboseLocation This is Integration Compute Question 0011. \\ \fi
\begin{problem}

Use the Fundamental Theorem of Calculus to evaluate the integral.

\input{Integral-Compute-0011.HELP.tex}

\[
\int_{6}^{7} {\frac{x - 4}{6 \, \sqrt{x}}}\;dx = \answer{-\frac{5}{9} \, \sqrt{7} + \frac{2}{3} \, \sqrt{6}}
\]
\end{problem}}%}

\latexProblemContent{
\ifVerboseLocation This is Integration Compute Question 0011. \\ \fi
\begin{problem}

Use the Fundamental Theorem of Calculus to evaluate the integral.

\input{Integral-Compute-0011.HELP.tex}

\[
\int_{18}^{21} {-\frac{x + 3}{8 \, \sqrt{x}}}\;dx = \answer{-\frac{5}{2} \, \sqrt{21} + \frac{27}{4} \, \sqrt{2}}
\]
\end{problem}}%}

\latexProblemContent{
\ifVerboseLocation This is Integration Compute Question 0011. \\ \fi
\begin{problem}

Use the Fundamental Theorem of Calculus to evaluate the integral.

\input{Integral-Compute-0011.HELP.tex}

\[
\int_{18}^{20} {\frac{x - 1}{2 \, \sqrt{x}}}\;dx = \answer{\frac{34}{3} \, \sqrt{5} - 15 \, \sqrt{2}}
\]
\end{problem}}%}

\latexProblemContent{
\ifVerboseLocation This is Integration Compute Question 0011. \\ \fi
\begin{problem}

Use the Fundamental Theorem of Calculus to evaluate the integral.

\input{Integral-Compute-0011.HELP.tex}

\[
\int_{15}^{15} {-\frac{x - 5}{6 \, \sqrt{x}}}\;dx = \answer{0}
\]
\end{problem}}%}

\latexProblemContent{
\ifVerboseLocation This is Integration Compute Question 0011. \\ \fi
\begin{problem}

Use the Fundamental Theorem of Calculus to evaluate the integral.

\input{Integral-Compute-0011.HELP.tex}

\[
\int_{13}^{19} {\frac{x - 4}{7 \, \sqrt{x}}}\;dx = \answer{\frac{2}{3} \, \sqrt{19} - \frac{2}{21} \, \sqrt{13}}
\]
\end{problem}}%}

\latexProblemContent{
\ifVerboseLocation This is Integration Compute Question 0011. \\ \fi
\begin{problem}

Use the Fundamental Theorem of Calculus to evaluate the integral.

\input{Integral-Compute-0011.HELP.tex}

\[
\int_{4}^{8} {\frac{x + 4}{\sqrt{x}}}\;dx = \answer{\frac{80}{3} \, \sqrt{2} - \frac{64}{3}}
\]
\end{problem}}%}

\latexProblemContent{
\ifVerboseLocation This is Integration Compute Question 0011. \\ \fi
\begin{problem}

Use the Fundamental Theorem of Calculus to evaluate the integral.

\input{Integral-Compute-0011.HELP.tex}

\[
\int_{7}^{8} {-\frac{x + 5}{2 \, \sqrt{x}}}\;dx = \answer{\frac{22}{3} \, \sqrt{7} - \frac{46}{3} \, \sqrt{2}}
\]
\end{problem}}%}

\latexProblemContent{
\ifVerboseLocation This is Integration Compute Question 0011. \\ \fi
\begin{problem}

Use the Fundamental Theorem of Calculus to evaluate the integral.

\input{Integral-Compute-0011.HELP.tex}

\[
\int_{20}^{20} {-\frac{x - 4}{6 \, \sqrt{x}}}\;dx = \answer{0}
\]
\end{problem}}%}

\latexProblemContent{
\ifVerboseLocation This is Integration Compute Question 0011. \\ \fi
\begin{problem}

Use the Fundamental Theorem of Calculus to evaluate the integral.

\input{Integral-Compute-0011.HELP.tex}

\[
\int_{6}^{16} {\frac{x + 3}{6 \, \sqrt{x}}}\;dx = \answer{-\frac{5}{3} \, \sqrt{6} + \frac{100}{9}}
\]
\end{problem}}%}

\latexProblemContent{
\ifVerboseLocation This is Integration Compute Question 0011. \\ \fi
\begin{problem}

Use the Fundamental Theorem of Calculus to evaluate the integral.

\input{Integral-Compute-0011.HELP.tex}

\[
\int_{4}^{21} {\frac{x + 4}{3 \, \sqrt{x}}}\;dx = \answer{\frac{22}{3} \, \sqrt{21} - \frac{64}{9}}
\]
\end{problem}}%}

\latexProblemContent{
\ifVerboseLocation This is Integration Compute Question 0011. \\ \fi
\begin{problem}

Use the Fundamental Theorem of Calculus to evaluate the integral.

\input{Integral-Compute-0011.HELP.tex}

\[
\int_{12}^{18} {-\frac{x - 5}{\sqrt{x}}}\;dx = \answer{-4 \, \sqrt{3} - 6 \, \sqrt{2}}
\]
\end{problem}}%}

\latexProblemContent{
\ifVerboseLocation This is Integration Compute Question 0011. \\ \fi
\begin{problem}

Use the Fundamental Theorem of Calculus to evaluate the integral.

\input{Integral-Compute-0011.HELP.tex}

\[
\int_{18}^{19} {-\frac{x - 5}{7 \, \sqrt{x}}}\;dx = \answer{-\frac{8}{21} \, \sqrt{19} + \frac{6}{7} \, \sqrt{2}}
\]
\end{problem}}%}

\latexProblemContent{
\ifVerboseLocation This is Integration Compute Question 0011. \\ \fi
\begin{problem}

Use the Fundamental Theorem of Calculus to evaluate the integral.

\input{Integral-Compute-0011.HELP.tex}

\[
\int_{2}^{7} {\frac{x + 2}{2 \, \sqrt{x}}}\;dx = \answer{\frac{13}{3} \, \sqrt{7} - \frac{8}{3} \, \sqrt{2}}
\]
\end{problem}}%}

\latexProblemContent{
\ifVerboseLocation This is Integration Compute Question 0011. \\ \fi
\begin{problem}

Use the Fundamental Theorem of Calculus to evaluate the integral.

\input{Integral-Compute-0011.HELP.tex}

\[
\int_{4}^{5} {\frac{x - 1}{8 \, \sqrt{x}}}\;dx = \answer{\frac{1}{6} \, \sqrt{5} - \frac{1}{6}}
\]
\end{problem}}%}

\latexProblemContent{
\ifVerboseLocation This is Integration Compute Question 0011. \\ \fi
\begin{problem}

Use the Fundamental Theorem of Calculus to evaluate the integral.

\input{Integral-Compute-0011.HELP.tex}

\[
\int_{2}^{15} {\frac{x + 2}{5 \, \sqrt{x}}}\;dx = \answer{\frac{14}{5} \, \sqrt{15} - \frac{16}{15} \, \sqrt{2}}
\]
\end{problem}}%}

\latexProblemContent{
\ifVerboseLocation This is Integration Compute Question 0011. \\ \fi
\begin{problem}

Use the Fundamental Theorem of Calculus to evaluate the integral.

\input{Integral-Compute-0011.HELP.tex}

\[
\int_{19}^{19} {\frac{x + 1}{2 \, \sqrt{x}}}\;dx = \answer{0}
\]
\end{problem}}%}

\latexProblemContent{
\ifVerboseLocation This is Integration Compute Question 0011. \\ \fi
\begin{problem}

Use the Fundamental Theorem of Calculus to evaluate the integral.

\input{Integral-Compute-0011.HELP.tex}

\[
\int_{15}^{17} {\frac{x + 5}{7 \, \sqrt{x}}}\;dx = \answer{\frac{64}{21} \, \sqrt{17} - \frac{20}{7} \, \sqrt{15}}
\]
\end{problem}}%}

\latexProblemContent{
\ifVerboseLocation This is Integration Compute Question 0011. \\ \fi
\begin{problem}

Use the Fundamental Theorem of Calculus to evaluate the integral.

\input{Integral-Compute-0011.HELP.tex}

\[
\int_{10}^{13} {-\frac{x + 4}{8 \, \sqrt{x}}}\;dx = \answer{-\frac{25}{12} \, \sqrt{13} + \frac{11}{6} \, \sqrt{10}}
\]
\end{problem}}%}

\latexProblemContent{
\ifVerboseLocation This is Integration Compute Question 0011. \\ \fi
\begin{problem}

Use the Fundamental Theorem of Calculus to evaluate the integral.

\input{Integral-Compute-0011.HELP.tex}

\[
\int_{20}^{20} {\frac{x + 5}{6 \, \sqrt{x}}}\;dx = \answer{0}
\]
\end{problem}}%}

\latexProblemContent{
\ifVerboseLocation This is Integration Compute Question 0011. \\ \fi
\begin{problem}

Use the Fundamental Theorem of Calculus to evaluate the integral.

\input{Integral-Compute-0011.HELP.tex}

\[
\int_{13}^{19} {\frac{x + 5}{6 \, \sqrt{x}}}\;dx = \answer{\frac{34}{9} \, \sqrt{19} - \frac{28}{9} \, \sqrt{13}}
\]
\end{problem}}%}

\latexProblemContent{
\ifVerboseLocation This is Integration Compute Question 0011. \\ \fi
\begin{problem}

Use the Fundamental Theorem of Calculus to evaluate the integral.

\input{Integral-Compute-0011.HELP.tex}

\[
\int_{15}^{16} {-\frac{x - 1}{8 \, \sqrt{x}}}\;dx = \answer{\sqrt{15} - \frac{13}{3}}
\]
\end{problem}}%}

\latexProblemContent{
\ifVerboseLocation This is Integration Compute Question 0011. \\ \fi
\begin{problem}

Use the Fundamental Theorem of Calculus to evaluate the integral.

\input{Integral-Compute-0011.HELP.tex}

\[
\int_{13}^{13} {-\frac{x + 5}{6 \, \sqrt{x}}}\;dx = \answer{0}
\]
\end{problem}}%}

\latexProblemContent{
\ifVerboseLocation This is Integration Compute Question 0011. \\ \fi
\begin{problem}

Use the Fundamental Theorem of Calculus to evaluate the integral.

\input{Integral-Compute-0011.HELP.tex}

\[
\int_{17}^{20} {-\frac{x + 3}{8 \, \sqrt{x}}}\;dx = \answer{\frac{13}{6} \, \sqrt{17} - \frac{29}{6} \, \sqrt{5}}
\]
\end{problem}}%}

\latexProblemContent{
\ifVerboseLocation This is Integration Compute Question 0011. \\ \fi
\begin{problem}

Use the Fundamental Theorem of Calculus to evaluate the integral.

\input{Integral-Compute-0011.HELP.tex}

\[
\int_{19}^{19} {-\frac{x + 1}{6 \, \sqrt{x}}}\;dx = \answer{0}
\]
\end{problem}}%}

\latexProblemContent{
\ifVerboseLocation This is Integration Compute Question 0011. \\ \fi
\begin{problem}

Use the Fundamental Theorem of Calculus to evaluate the integral.

\input{Integral-Compute-0011.HELP.tex}

\[
\int_{1}^{16} {\frac{x - 3}{2 \, \sqrt{x}}}\;dx = \answer{12}
\]
\end{problem}}%}

\latexProblemContent{
\ifVerboseLocation This is Integration Compute Question 0011. \\ \fi
\begin{problem}

Use the Fundamental Theorem of Calculus to evaluate the integral.

\input{Integral-Compute-0011.HELP.tex}

\[
\int_{1}^{21} {\frac{x + 3}{4 \, \sqrt{x}}}\;dx = \answer{5 \, \sqrt{21} - \frac{5}{3}}
\]
\end{problem}}%}

\latexProblemContent{
\ifVerboseLocation This is Integration Compute Question 0011. \\ \fi
\begin{problem}

Use the Fundamental Theorem of Calculus to evaluate the integral.

\input{Integral-Compute-0011.HELP.tex}

\[
\int_{10}^{12} {\frac{x + 2}{6 \, \sqrt{x}}}\;dx = \answer{-\frac{16}{9} \, \sqrt{10} + 4 \, \sqrt{3}}
\]
\end{problem}}%}

\latexProblemContent{
\ifVerboseLocation This is Integration Compute Question 0011. \\ \fi
\begin{problem}

Use the Fundamental Theorem of Calculus to evaluate the integral.

\input{Integral-Compute-0011.HELP.tex}

\[
\int_{6}^{7} {\frac{x + 2}{6 \, \sqrt{x}}}\;dx = \answer{\frac{13}{9} \, \sqrt{7} - \frac{4}{3} \, \sqrt{6}}
\]
\end{problem}}%}

\latexProblemContent{
\ifVerboseLocation This is Integration Compute Question 0011. \\ \fi
\begin{problem}

Use the Fundamental Theorem of Calculus to evaluate the integral.

\input{Integral-Compute-0011.HELP.tex}

\[
\int_{9}^{15} {\frac{x + 1}{6 \, \sqrt{x}}}\;dx = \answer{2 \, \sqrt{15} - 4}
\]
\end{problem}}%}

\latexProblemContent{
\ifVerboseLocation This is Integration Compute Question 0011. \\ \fi
\begin{problem}

Use the Fundamental Theorem of Calculus to evaluate the integral.

\input{Integral-Compute-0011.HELP.tex}

\[
\int_{18}^{20} {-\frac{x + 4}{2 \, \sqrt{x}}}\;dx = \answer{-\frac{64}{3} \, \sqrt{5} + 30 \, \sqrt{2}}
\]
\end{problem}}%}

\latexProblemContent{
\ifVerboseLocation This is Integration Compute Question 0011. \\ \fi
\begin{problem}

Use the Fundamental Theorem of Calculus to evaluate the integral.

\input{Integral-Compute-0011.HELP.tex}

\[
\int_{19}^{21} {-\frac{x - 2}{6 \, \sqrt{x}}}\;dx = \answer{-\frac{5}{3} \, \sqrt{21} + \frac{13}{9} \, \sqrt{19}}
\]
\end{problem}}%}

\latexProblemContent{
\ifVerboseLocation This is Integration Compute Question 0011. \\ \fi
\begin{problem}

Use the Fundamental Theorem of Calculus to evaluate the integral.

\input{Integral-Compute-0011.HELP.tex}

\[
\int_{20}^{20} {-\frac{x - 1}{5 \, \sqrt{x}}}\;dx = \answer{0}
\]
\end{problem}}%}

\latexProblemContent{
\ifVerboseLocation This is Integration Compute Question 0011. \\ \fi
\begin{problem}

Use the Fundamental Theorem of Calculus to evaluate the integral.

\input{Integral-Compute-0011.HELP.tex}

\[
\int_{1}^{3} {-\frac{x + 3}{2 \, \sqrt{x}}}\;dx = \answer{-4 \, \sqrt{3} + \frac{10}{3}}
\]
\end{problem}}%}

\latexProblemContent{
\ifVerboseLocation This is Integration Compute Question 0011. \\ \fi
\begin{problem}

Use the Fundamental Theorem of Calculus to evaluate the integral.

\input{Integral-Compute-0011.HELP.tex}

\[
\int_{9}^{12} {\frac{x - 4}{2 \, \sqrt{x}}}\;dx = \answer{3}
\]
\end{problem}}%}

\latexProblemContent{
\ifVerboseLocation This is Integration Compute Question 0011. \\ \fi
\begin{problem}

Use the Fundamental Theorem of Calculus to evaluate the integral.

\input{Integral-Compute-0011.HELP.tex}

\[
\int_{10}^{20} {\frac{x - 3}{7 \, \sqrt{x}}}\;dx = \answer{-\frac{2}{21} \, \sqrt{10} + \frac{44}{21} \, \sqrt{5}}
\]
\end{problem}}%}

\latexProblemContent{
\ifVerboseLocation This is Integration Compute Question 0011. \\ \fi
\begin{problem}

Use the Fundamental Theorem of Calculus to evaluate the integral.

\input{Integral-Compute-0011.HELP.tex}

\[
\int_{18}^{21} {-\frac{x + 1}{8 \, \sqrt{x}}}\;dx = \answer{-2 \, \sqrt{21} + \frac{21}{4} \, \sqrt{2}}
\]
\end{problem}}%}

\latexProblemContent{
\ifVerboseLocation This is Integration Compute Question 0011. \\ \fi
\begin{problem}

Use the Fundamental Theorem of Calculus to evaluate the integral.

\input{Integral-Compute-0011.HELP.tex}

\[
\int_{8}^{12} {-\frac{x + 5}{2 \, \sqrt{x}}}\;dx = \answer{-18 \, \sqrt{3} + \frac{46}{3} \, \sqrt{2}}
\]
\end{problem}}%}

\latexProblemContent{
\ifVerboseLocation This is Integration Compute Question 0011. \\ \fi
\begin{problem}

Use the Fundamental Theorem of Calculus to evaluate the integral.

\input{Integral-Compute-0011.HELP.tex}

\[
\int_{20}^{21} {-\frac{x + 1}{3 \, \sqrt{x}}}\;dx = \answer{-\frac{16}{3} \, \sqrt{21} + \frac{92}{9} \, \sqrt{5}}
\]
\end{problem}}%}

\latexProblemContent{
\ifVerboseLocation This is Integration Compute Question 0011. \\ \fi
\begin{problem}

Use the Fundamental Theorem of Calculus to evaluate the integral.

\input{Integral-Compute-0011.HELP.tex}

\[
\int_{19}^{20} {\frac{x + 4}{6 \, \sqrt{x}}}\;dx = \answer{-\frac{31}{9} \, \sqrt{19} + \frac{64}{9} \, \sqrt{5}}
\]
\end{problem}}%}

\latexProblemContent{
\ifVerboseLocation This is Integration Compute Question 0011. \\ \fi
\begin{problem}

Use the Fundamental Theorem of Calculus to evaluate the integral.

\input{Integral-Compute-0011.HELP.tex}

\[
\int_{17}^{17} {-\frac{x - 4}{4 \, \sqrt{x}}}\;dx = \answer{0}
\]
\end{problem}}%}

\latexProblemContent{
\ifVerboseLocation This is Integration Compute Question 0011. \\ \fi
\begin{problem}

Use the Fundamental Theorem of Calculus to evaluate the integral.

\input{Integral-Compute-0011.HELP.tex}

\[
\int_{8}^{19} {\frac{x + 2}{\sqrt{x}}}\;dx = \answer{\frac{50}{3} \, \sqrt{19} - \frac{56}{3} \, \sqrt{2}}
\]
\end{problem}}%}

\latexProblemContent{
\ifVerboseLocation This is Integration Compute Question 0011. \\ \fi
\begin{problem}

Use the Fundamental Theorem of Calculus to evaluate the integral.

\input{Integral-Compute-0011.HELP.tex}

\[
\int_{15}^{20} {-\frac{x - 5}{2 \, \sqrt{x}}}\;dx = \answer{-\frac{10}{3} \, \sqrt{5}}
\]
\end{problem}}%}

\latexProblemContent{
\ifVerboseLocation This is Integration Compute Question 0011. \\ \fi
\begin{problem}

Use the Fundamental Theorem of Calculus to evaluate the integral.

\input{Integral-Compute-0011.HELP.tex}

\[
\int_{15}^{15} {\frac{x + 1}{8 \, \sqrt{x}}}\;dx = \answer{0}
\]
\end{problem}}%}

\latexProblemContent{
\ifVerboseLocation This is Integration Compute Question 0011. \\ \fi
\begin{problem}

Use the Fundamental Theorem of Calculus to evaluate the integral.

\input{Integral-Compute-0011.HELP.tex}

\[
\int_{11}^{20} {-\frac{x + 3}{5 \, \sqrt{x}}}\;dx = \answer{\frac{8}{3} \, \sqrt{11} - \frac{116}{15} \, \sqrt{5}}
\]
\end{problem}}%}

\latexProblemContent{
\ifVerboseLocation This is Integration Compute Question 0011. \\ \fi
\begin{problem}

Use the Fundamental Theorem of Calculus to evaluate the integral.

\input{Integral-Compute-0011.HELP.tex}

\[
\int_{1}^{4} {-\frac{x + 1}{3 \, \sqrt{x}}}\;dx = \answer{-\frac{20}{9}}
\]
\end{problem}}%}

\latexProblemContent{
\ifVerboseLocation This is Integration Compute Question 0011. \\ \fi
\begin{problem}

Use the Fundamental Theorem of Calculus to evaluate the integral.

\input{Integral-Compute-0011.HELP.tex}

\[
\int_{12}^{17} {\frac{x + 5}{3 \, \sqrt{x}}}\;dx = \answer{\frac{64}{9} \, \sqrt{17} - 12 \, \sqrt{3}}
\]
\end{problem}}%}

\latexProblemContent{
\ifVerboseLocation This is Integration Compute Question 0011. \\ \fi
\begin{problem}

Use the Fundamental Theorem of Calculus to evaluate the integral.

\input{Integral-Compute-0011.HELP.tex}

\[
\int_{13}^{21} {-\frac{x + 3}{8 \, \sqrt{x}}}\;dx = \answer{-\frac{5}{2} \, \sqrt{21} + \frac{11}{6} \, \sqrt{13}}
\]
\end{problem}}%}

\latexProblemContent{
\ifVerboseLocation This is Integration Compute Question 0011. \\ \fi
\begin{problem}

Use the Fundamental Theorem of Calculus to evaluate the integral.

\input{Integral-Compute-0011.HELP.tex}

\[
\int_{13}^{17} {-\frac{x - 2}{\sqrt{x}}}\;dx = \answer{-\frac{22}{3} \, \sqrt{17} + \frac{14}{3} \, \sqrt{13}}
\]
\end{problem}}%}

\latexProblemContent{
\ifVerboseLocation This is Integration Compute Question 0011. \\ \fi
\begin{problem}

Use the Fundamental Theorem of Calculus to evaluate the integral.

\input{Integral-Compute-0011.HELP.tex}

\[
\int_{3}^{5} {-\frac{x + 2}{\sqrt{x}}}\;dx = \answer{-\frac{22}{3} \, \sqrt{5} + 6 \, \sqrt{3}}
\]
\end{problem}}%}

\latexProblemContent{
\ifVerboseLocation This is Integration Compute Question 0011. \\ \fi
\begin{problem}

Use the Fundamental Theorem of Calculus to evaluate the integral.

\input{Integral-Compute-0011.HELP.tex}

\[
\int_{3}^{18} {\frac{x + 2}{2 \, \sqrt{x}}}\;dx = \answer{-3 \, \sqrt{3} + 24 \, \sqrt{2}}
\]
\end{problem}}%}

\latexProblemContent{
\ifVerboseLocation This is Integration Compute Question 0011. \\ \fi
\begin{problem}

Use the Fundamental Theorem of Calculus to evaluate the integral.

\input{Integral-Compute-0011.HELP.tex}

\[
\int_{6}^{8} {\frac{x + 1}{3 \, \sqrt{x}}}\;dx = \answer{-2 \, \sqrt{6} + \frac{44}{9} \, \sqrt{2}}
\]
\end{problem}}%}

\latexProblemContent{
\ifVerboseLocation This is Integration Compute Question 0011. \\ \fi
\begin{problem}

Use the Fundamental Theorem of Calculus to evaluate the integral.

\input{Integral-Compute-0011.HELP.tex}

\[
\int_{11}^{16} {-\frac{x + 5}{5 \, \sqrt{x}}}\;dx = \answer{\frac{52}{15} \, \sqrt{11} - \frac{248}{15}}
\]
\end{problem}}%}

\latexProblemContent{
\ifVerboseLocation This is Integration Compute Question 0011. \\ \fi
\begin{problem}

Use the Fundamental Theorem of Calculus to evaluate the integral.

\input{Integral-Compute-0011.HELP.tex}

\[
\int_{7}^{11} {\frac{x - 4}{6 \, \sqrt{x}}}\;dx = \answer{-\frac{1}{9} \, \sqrt{11} + \frac{5}{9} \, \sqrt{7}}
\]
\end{problem}}%}

\latexProblemContent{
\ifVerboseLocation This is Integration Compute Question 0011. \\ \fi
\begin{problem}

Use the Fundamental Theorem of Calculus to evaluate the integral.

\input{Integral-Compute-0011.HELP.tex}

\[
\int_{5}^{13} {-\frac{x - 4}{\sqrt{x}}}\;dx = \answer{-\frac{2}{3} \, \sqrt{13} - \frac{14}{3} \, \sqrt{5}}
\]
\end{problem}}%}

\latexProblemContent{
\ifVerboseLocation This is Integration Compute Question 0011. \\ \fi
\begin{problem}

Use the Fundamental Theorem of Calculus to evaluate the integral.

\input{Integral-Compute-0011.HELP.tex}

\[
\int_{4}^{15} {-\frac{x + 5}{7 \, \sqrt{x}}}\;dx = \answer{-\frac{20}{7} \, \sqrt{15} + \frac{76}{21}}
\]
\end{problem}}%}

\latexProblemContent{
\ifVerboseLocation This is Integration Compute Question 0011. \\ \fi
\begin{problem}

Use the Fundamental Theorem of Calculus to evaluate the integral.

\input{Integral-Compute-0011.HELP.tex}

\[
\int_{12}^{20} {-\frac{x - 2}{2 \, \sqrt{x}}}\;dx = \answer{-\frac{28}{3} \, \sqrt{5} + 4 \, \sqrt{3}}
\]
\end{problem}}%}

\latexProblemContent{
\ifVerboseLocation This is Integration Compute Question 0011. \\ \fi
\begin{problem}

Use the Fundamental Theorem of Calculus to evaluate the integral.

\input{Integral-Compute-0011.HELP.tex}

\[
\int_{2}^{9} {-\frac{x - 3}{\sqrt{x}}}\;dx = \answer{-\frac{14}{3} \, \sqrt{2}}
\]
\end{problem}}%}

\latexProblemContent{
\ifVerboseLocation This is Integration Compute Question 0011. \\ \fi
\begin{problem}

Use the Fundamental Theorem of Calculus to evaluate the integral.

\input{Integral-Compute-0011.HELP.tex}

\[
\int_{2}^{13} {-\frac{x + 3}{3 \, \sqrt{x}}}\;dx = \answer{-\frac{44}{9} \, \sqrt{13} + \frac{22}{9} \, \sqrt{2}}
\]
\end{problem}}%}

\latexProblemContent{
\ifVerboseLocation This is Integration Compute Question 0011. \\ \fi
\begin{problem}

Use the Fundamental Theorem of Calculus to evaluate the integral.

\input{Integral-Compute-0011.HELP.tex}

\[
\int_{12}^{12} {-\frac{x - 3}{6 \, \sqrt{x}}}\;dx = \answer{0}
\]
\end{problem}}%}

\latexProblemContent{
\ifVerboseLocation This is Integration Compute Question 0011. \\ \fi
\begin{problem}

Use the Fundamental Theorem of Calculus to evaluate the integral.

\input{Integral-Compute-0011.HELP.tex}

\[
\int_{19}^{19} {-\frac{x + 4}{8 \, \sqrt{x}}}\;dx = \answer{0}
\]
\end{problem}}%}

\latexProblemContent{
\ifVerboseLocation This is Integration Compute Question 0011. \\ \fi
\begin{problem}

Use the Fundamental Theorem of Calculus to evaluate the integral.

\input{Integral-Compute-0011.HELP.tex}

\[
\int_{5}^{20} {\frac{x - 4}{3 \, \sqrt{x}}}\;dx = \answer{\frac{46}{9} \, \sqrt{5}}
\]
\end{problem}}%}

\latexProblemContent{
\ifVerboseLocation This is Integration Compute Question 0011. \\ \fi
\begin{problem}

Use the Fundamental Theorem of Calculus to evaluate the integral.

\input{Integral-Compute-0011.HELP.tex}

\[
\int_{20}^{20} {-\frac{x + 4}{4 \, \sqrt{x}}}\;dx = \answer{0}
\]
\end{problem}}%}

\latexProblemContent{
\ifVerboseLocation This is Integration Compute Question 0011. \\ \fi
\begin{problem}

Use the Fundamental Theorem of Calculus to evaluate the integral.

\input{Integral-Compute-0011.HELP.tex}

\[
\int_{8}^{10} {-\frac{x - 5}{3 \, \sqrt{x}}}\;dx = \answer{\frac{1}{9} \, \left(10 \, \sqrt{10}\right) - \frac{28}{9} \, \sqrt{2}}
\]
\end{problem}}%}

\latexProblemContent{
\ifVerboseLocation This is Integration Compute Question 0011. \\ \fi
\begin{problem}

Use the Fundamental Theorem of Calculus to evaluate the integral.

\input{Integral-Compute-0011.HELP.tex}

\[
\int_{6}^{12} {\frac{x - 4}{6 \, \sqrt{x}}}\;dx = \answer{\frac{2}{3} \, \sqrt{6}}
\]
\end{problem}}%}

\latexProblemContent{
\ifVerboseLocation This is Integration Compute Question 0011. \\ \fi
\begin{problem}

Use the Fundamental Theorem of Calculus to evaluate the integral.

\input{Integral-Compute-0011.HELP.tex}

\[
\int_{10}^{19} {-\frac{x - 2}{\sqrt{x}}}\;dx = \answer{-\frac{26}{3} \, \sqrt{19} + \frac{8}{3} \, \sqrt{10}}
\]
\end{problem}}%}

\latexProblemContent{
\ifVerboseLocation This is Integration Compute Question 0011. \\ \fi
\begin{problem}

Use the Fundamental Theorem of Calculus to evaluate the integral.

\input{Integral-Compute-0011.HELP.tex}

\[
\int_{7}^{14} {-\frac{x + 3}{3 \, \sqrt{x}}}\;dx = \answer{-\frac{46}{9} \, \sqrt{14} + \frac{32}{9} \, \sqrt{7}}
\]
\end{problem}}%}

\latexProblemContent{
\ifVerboseLocation This is Integration Compute Question 0011. \\ \fi
\begin{problem}

Use the Fundamental Theorem of Calculus to evaluate the integral.

\input{Integral-Compute-0011.HELP.tex}

\[
\int_{6}^{17} {\frac{x + 1}{8 \, \sqrt{x}}}\;dx = \answer{\frac{5}{3} \, \sqrt{17} - \frac{3}{4} \, \sqrt{6}}
\]
\end{problem}}%}

\latexProblemContent{
\ifVerboseLocation This is Integration Compute Question 0011. \\ \fi
\begin{problem}

Use the Fundamental Theorem of Calculus to evaluate the integral.

\input{Integral-Compute-0011.HELP.tex}

\[
\int_{8}^{14} {-\frac{x + 2}{4 \, \sqrt{x}}}\;dx = \answer{-\frac{10}{3} \, \sqrt{14} + \frac{14}{3} \, \sqrt{2}}
\]
\end{problem}}%}

\latexProblemContent{
\ifVerboseLocation This is Integration Compute Question 0011. \\ \fi
\begin{problem}

Use the Fundamental Theorem of Calculus to evaluate the integral.

\input{Integral-Compute-0011.HELP.tex}

\[
\int_{7}^{20} {-\frac{x + 2}{8 \, \sqrt{x}}}\;dx = \answer{\frac{13}{12} \, \sqrt{7} - \frac{13}{3} \, \sqrt{5}}
\]
\end{problem}}%}

\latexProblemContent{
\ifVerboseLocation This is Integration Compute Question 0011. \\ \fi
\begin{problem}

Use the Fundamental Theorem of Calculus to evaluate the integral.

\input{Integral-Compute-0011.HELP.tex}

\[
\int_{15}^{18} {\frac{x + 2}{4 \, \sqrt{x}}}\;dx = \answer{-\frac{7}{2} \, \sqrt{15} + 12 \, \sqrt{2}}
\]
\end{problem}}%}

\latexProblemContent{
\ifVerboseLocation This is Integration Compute Question 0011. \\ \fi
\begin{problem}

Use the Fundamental Theorem of Calculus to evaluate the integral.

\input{Integral-Compute-0011.HELP.tex}

\[
\int_{10}^{15} {-\frac{x - 1}{5 \, \sqrt{x}}}\;dx = \answer{-\frac{8}{5} \, \sqrt{15} + \frac{14}{15} \, \sqrt{10}}
\]
\end{problem}}%}

\latexProblemContent{
\ifVerboseLocation This is Integration Compute Question 0011. \\ \fi
\begin{problem}

Use the Fundamental Theorem of Calculus to evaluate the integral.

\input{Integral-Compute-0011.HELP.tex}

\[
\int_{9}^{11} {\frac{x + 5}{2 \, \sqrt{x}}}\;dx = \answer{\frac{26}{3} \, \sqrt{11} - 24}
\]
\end{problem}}%}

\latexProblemContent{
\ifVerboseLocation This is Integration Compute Question 0011. \\ \fi
\begin{problem}

Use the Fundamental Theorem of Calculus to evaluate the integral.

\input{Integral-Compute-0011.HELP.tex}

\[
\int_{8}^{10} {\frac{x - 4}{\sqrt{x}}}\;dx = \answer{-\frac{4}{3} \, \sqrt{10} + \frac{16}{3} \, \sqrt{2}}
\]
\end{problem}}%}

\latexProblemContent{
\ifVerboseLocation This is Integration Compute Question 0011. \\ \fi
\begin{problem}

Use the Fundamental Theorem of Calculus to evaluate the integral.

\input{Integral-Compute-0011.HELP.tex}

\[
\int_{13}^{20} {-\frac{x - 3}{2 \, \sqrt{x}}}\;dx = \answer{\frac{4}{3} \, \sqrt{13} - \frac{22}{3} \, \sqrt{5}}
\]
\end{problem}}%}

\latexProblemContent{
\ifVerboseLocation This is Integration Compute Question 0011. \\ \fi
\begin{problem}

Use the Fundamental Theorem of Calculus to evaluate the integral.

\input{Integral-Compute-0011.HELP.tex}

\[
\int_{13}^{18} {\frac{x + 5}{\sqrt{x}}}\;dx = \answer{-\frac{56}{3} \, \sqrt{13} + 66 \, \sqrt{2}}
\]
\end{problem}}%}

\latexProblemContent{
\ifVerboseLocation This is Integration Compute Question 0011. \\ \fi
\begin{problem}

Use the Fundamental Theorem of Calculus to evaluate the integral.

\input{Integral-Compute-0011.HELP.tex}

\[
\int_{17}^{17} {-\frac{x - 3}{4 \, \sqrt{x}}}\;dx = \answer{0}
\]
\end{problem}}%}

\latexProblemContent{
\ifVerboseLocation This is Integration Compute Question 0011. \\ \fi
\begin{problem}

Use the Fundamental Theorem of Calculus to evaluate the integral.

\input{Integral-Compute-0011.HELP.tex}

\[
\int_{14}^{15} {-\frac{x + 2}{6 \, \sqrt{x}}}\;dx = \answer{-\frac{7}{3} \, \sqrt{15} + \frac{20}{9} \, \sqrt{14}}
\]
\end{problem}}%}

\latexProblemContent{
\ifVerboseLocation This is Integration Compute Question 0011. \\ \fi
\begin{problem}

Use the Fundamental Theorem of Calculus to evaluate the integral.

\input{Integral-Compute-0011.HELP.tex}

\[
\int_{15}^{17} {\frac{x + 3}{7 \, \sqrt{x}}}\;dx = \answer{\frac{52}{21} \, \sqrt{17} - \frac{16}{7} \, \sqrt{15}}
\]
\end{problem}}%}

\latexProblemContent{
\ifVerboseLocation This is Integration Compute Question 0011. \\ \fi
\begin{problem}

Use the Fundamental Theorem of Calculus to evaluate the integral.

\input{Integral-Compute-0011.HELP.tex}

\[
\int_{14}^{20} {\frac{x + 5}{4 \, \sqrt{x}}}\;dx = \answer{-\frac{29}{6} \, \sqrt{14} + \frac{35}{3} \, \sqrt{5}}
\]
\end{problem}}%}

\latexProblemContent{
\ifVerboseLocation This is Integration Compute Question 0011. \\ \fi
\begin{problem}

Use the Fundamental Theorem of Calculus to evaluate the integral.

\input{Integral-Compute-0011.HELP.tex}

\[
\int_{2}^{13} {\frac{x - 5}{8 \, \sqrt{x}}}\;dx = \answer{-\frac{1}{6} \, \sqrt{13} + \frac{13}{12} \, \sqrt{2}}
\]
\end{problem}}%}

\latexProblemContent{
\ifVerboseLocation This is Integration Compute Question 0011. \\ \fi
\begin{problem}

Use the Fundamental Theorem of Calculus to evaluate the integral.

\input{Integral-Compute-0011.HELP.tex}

\[
\int_{16}^{20} {\frac{x + 1}{7 \, \sqrt{x}}}\;dx = \answer{\frac{92}{21} \, \sqrt{5} - \frac{152}{21}}
\]
\end{problem}}%}

\latexProblemContent{
\ifVerboseLocation This is Integration Compute Question 0011. \\ \fi
\begin{problem}

Use the Fundamental Theorem of Calculus to evaluate the integral.

\input{Integral-Compute-0011.HELP.tex}

\[
\int_{14}^{16} {\frac{x + 2}{8 \, \sqrt{x}}}\;dx = \answer{-\frac{5}{3} \, \sqrt{14} + \frac{22}{3}}
\]
\end{problem}}%}

\latexProblemContent{
\ifVerboseLocation This is Integration Compute Question 0011. \\ \fi
\begin{problem}

Use the Fundamental Theorem of Calculus to evaluate the integral.

\input{Integral-Compute-0011.HELP.tex}

\[
\int_{9}^{18} {\frac{x + 1}{7 \, \sqrt{x}}}\;dx = \answer{6 \, \sqrt{2} - \frac{24}{7}}
\]
\end{problem}}%}

\latexProblemContent{
\ifVerboseLocation This is Integration Compute Question 0011. \\ \fi
\begin{problem}

Use the Fundamental Theorem of Calculus to evaluate the integral.

\input{Integral-Compute-0011.HELP.tex}

\[
\int_{7}^{9} {-\frac{x - 2}{8 \, \sqrt{x}}}\;dx = \answer{\frac{1}{12} \, \sqrt{7} - \frac{3}{4}}
\]
\end{problem}}%}

\latexProblemContent{
\ifVerboseLocation This is Integration Compute Question 0011. \\ \fi
\begin{problem}

Use the Fundamental Theorem of Calculus to evaluate the integral.

\input{Integral-Compute-0011.HELP.tex}

\[
\int_{16}^{20} {\frac{x - 1}{2 \, \sqrt{x}}}\;dx = \answer{\frac{34}{3} \, \sqrt{5} - \frac{52}{3}}
\]
\end{problem}}%}

\latexProblemContent{
\ifVerboseLocation This is Integration Compute Question 0011. \\ \fi
\begin{problem}

Use the Fundamental Theorem of Calculus to evaluate the integral.

\input{Integral-Compute-0011.HELP.tex}

\[
\int_{19}^{20} {\frac{x + 3}{5 \, \sqrt{x}}}\;dx = \answer{-\frac{56}{15} \, \sqrt{19} + \frac{116}{15} \, \sqrt{5}}
\]
\end{problem}}%}

\latexProblemContent{
\ifVerboseLocation This is Integration Compute Question 0011. \\ \fi
\begin{problem}

Use the Fundamental Theorem of Calculus to evaluate the integral.

\input{Integral-Compute-0011.HELP.tex}

\[
\int_{20}^{20} {\frac{x + 1}{7 \, \sqrt{x}}}\;dx = \answer{0}
\]
\end{problem}}%}

\latexProblemContent{
\ifVerboseLocation This is Integration Compute Question 0011. \\ \fi
\begin{problem}

Use the Fundamental Theorem of Calculus to evaluate the integral.

\input{Integral-Compute-0011.HELP.tex}

\[
\int_{17}^{21} {-\frac{x + 1}{2 \, \sqrt{x}}}\;dx = \answer{-8 \, \sqrt{21} + \frac{20}{3} \, \sqrt{17}}
\]
\end{problem}}%}

\latexProblemContent{
\ifVerboseLocation This is Integration Compute Question 0011. \\ \fi
\begin{problem}

Use the Fundamental Theorem of Calculus to evaluate the integral.

\input{Integral-Compute-0011.HELP.tex}

\[
\int_{19}^{19} {-\frac{x - 5}{8 \, \sqrt{x}}}\;dx = \answer{0}
\]
\end{problem}}%}

\latexProblemContent{
\ifVerboseLocation This is Integration Compute Question 0011. \\ \fi
\begin{problem}

Use the Fundamental Theorem of Calculus to evaluate the integral.

\input{Integral-Compute-0011.HELP.tex}

\[
\int_{15}^{16} {\frac{x + 3}{3 \, \sqrt{x}}}\;dx = \answer{-\frac{16}{3} \, \sqrt{15} + \frac{200}{9}}
\]
\end{problem}}%}

\latexProblemContent{
\ifVerboseLocation This is Integration Compute Question 0011. \\ \fi
\begin{problem}

Use the Fundamental Theorem of Calculus to evaluate the integral.

\input{Integral-Compute-0011.HELP.tex}

\[
\int_{11}^{21} {\frac{x + 2}{5 \, \sqrt{x}}}\;dx = \answer{\frac{18}{5} \, \sqrt{21} - \frac{34}{15} \, \sqrt{11}}
\]
\end{problem}}%}

\latexProblemContent{
\ifVerboseLocation This is Integration Compute Question 0011. \\ \fi
\begin{problem}

Use the Fundamental Theorem of Calculus to evaluate the integral.

\input{Integral-Compute-0011.HELP.tex}

\[
\int_{20}^{20} {-\frac{x + 2}{\sqrt{x}}}\;dx = \answer{0}
\]
\end{problem}}%}

\latexProblemContent{
\ifVerboseLocation This is Integration Compute Question 0011. \\ \fi
\begin{problem}

Use the Fundamental Theorem of Calculus to evaluate the integral.

\input{Integral-Compute-0011.HELP.tex}

\[
\int_{8}^{16} {\frac{x + 3}{6 \, \sqrt{x}}}\;dx = \answer{-\frac{34}{9} \, \sqrt{2} + \frac{100}{9}}
\]
\end{problem}}%}

\latexProblemContent{
\ifVerboseLocation This is Integration Compute Question 0011. \\ \fi
\begin{problem}

Use the Fundamental Theorem of Calculus to evaluate the integral.

\input{Integral-Compute-0011.HELP.tex}

\[
\int_{15}^{19} {\frac{x + 1}{5 \, \sqrt{x}}}\;dx = \answer{\frac{44}{15} \, \sqrt{19} - \frac{12}{5} \, \sqrt{15}}
\]
\end{problem}}%}

\latexProblemContent{
\ifVerboseLocation This is Integration Compute Question 0011. \\ \fi
\begin{problem}

Use the Fundamental Theorem of Calculus to evaluate the integral.

\input{Integral-Compute-0011.HELP.tex}

\[
\int_{19}^{20} {-\frac{x - 4}{3 \, \sqrt{x}}}\;dx = \answer{\frac{14}{9} \, \sqrt{19} - \frac{32}{9} \, \sqrt{5}}
\]
\end{problem}}%}

\latexProblemContent{
\ifVerboseLocation This is Integration Compute Question 0011. \\ \fi
\begin{problem}

Use the Fundamental Theorem of Calculus to evaluate the integral.

\input{Integral-Compute-0011.HELP.tex}

\[
\int_{7}^{18} {\frac{x - 4}{2 \, \sqrt{x}}}\;dx = \answer{\frac{5}{3} \, \sqrt{7} + 6 \, \sqrt{2}}
\]
\end{problem}}%}

\latexProblemContent{
\ifVerboseLocation This is Integration Compute Question 0011. \\ \fi
\begin{problem}

Use the Fundamental Theorem of Calculus to evaluate the integral.

\input{Integral-Compute-0011.HELP.tex}

\[
\int_{20}^{21} {-\frac{x + 4}{4 \, \sqrt{x}}}\;dx = \answer{-\frac{11}{2} \, \sqrt{21} + \frac{32}{3} \, \sqrt{5}}
\]
\end{problem}}%}

\latexProblemContent{
\ifVerboseLocation This is Integration Compute Question 0011. \\ \fi
\begin{problem}

Use the Fundamental Theorem of Calculus to evaluate the integral.

\input{Integral-Compute-0011.HELP.tex}

\[
\int_{2}^{14} {\frac{x - 3}{3 \, \sqrt{x}}}\;dx = \answer{\frac{10}{9} \, \sqrt{14} + \frac{14}{9} \, \sqrt{2}}
\]
\end{problem}}%}

\latexProblemContent{
\ifVerboseLocation This is Integration Compute Question 0011. \\ \fi
\begin{problem}

Use the Fundamental Theorem of Calculus to evaluate the integral.

\input{Integral-Compute-0011.HELP.tex}

\[
\int_{5}^{11} {-\frac{x + 4}{2 \, \sqrt{x}}}\;dx = \answer{-\frac{23}{3} \, \sqrt{11} + \frac{17}{3} \, \sqrt{5}}
\]
\end{problem}}%}

\latexProblemContent{
\ifVerboseLocation This is Integration Compute Question 0011. \\ \fi
\begin{problem}

Use the Fundamental Theorem of Calculus to evaluate the integral.

\input{Integral-Compute-0011.HELP.tex}

\[
\int_{11}^{18} {-\frac{x + 4}{5 \, \sqrt{x}}}\;dx = \answer{\frac{46}{15} \, \sqrt{11} - 12 \, \sqrt{2}}
\]
\end{problem}}%}

\latexProblemContent{
\ifVerboseLocation This is Integration Compute Question 0011. \\ \fi
\begin{problem}

Use the Fundamental Theorem of Calculus to evaluate the integral.

\input{Integral-Compute-0011.HELP.tex}

\[
\int_{7}^{8} {\frac{x - 1}{3 \, \sqrt{x}}}\;dx = \answer{-\frac{8}{9} \, \sqrt{7} + \frac{20}{9} \, \sqrt{2}}
\]
\end{problem}}%}

\latexProblemContent{
\ifVerboseLocation This is Integration Compute Question 0011. \\ \fi
\begin{problem}

Use the Fundamental Theorem of Calculus to evaluate the integral.

\input{Integral-Compute-0011.HELP.tex}

\[
\int_{14}^{19} {-\frac{x - 2}{\sqrt{x}}}\;dx = \answer{-\frac{26}{3} \, \sqrt{19} + \frac{16}{3} \, \sqrt{14}}
\]
\end{problem}}%}

\latexProblemContent{
\ifVerboseLocation This is Integration Compute Question 0011. \\ \fi
\begin{problem}

Use the Fundamental Theorem of Calculus to evaluate the integral.

\input{Integral-Compute-0011.HELP.tex}

\[
\int_{18}^{21} {\frac{x + 3}{3 \, \sqrt{x}}}\;dx = \answer{\frac{20}{3} \, \sqrt{21} - 18 \, \sqrt{2}}
\]
\end{problem}}%}

\latexProblemContent{
\ifVerboseLocation This is Integration Compute Question 0011. \\ \fi
\begin{problem}

Use the Fundamental Theorem of Calculus to evaluate the integral.

\input{Integral-Compute-0011.HELP.tex}

\[
\int_{10}^{10} {\frac{x - 5}{5 \, \sqrt{x}}}\;dx = \answer{0}
\]
\end{problem}}%}

\latexProblemContent{
\ifVerboseLocation This is Integration Compute Question 0011. \\ \fi
\begin{problem}

Use the Fundamental Theorem of Calculus to evaluate the integral.

\input{Integral-Compute-0011.HELP.tex}

\[
\int_{4}^{17} {\frac{x + 5}{3 \, \sqrt{x}}}\;dx = \answer{\frac{64}{9} \, \sqrt{17} - \frac{76}{9}}
\]
\end{problem}}%}

\latexProblemContent{
\ifVerboseLocation This is Integration Compute Question 0011. \\ \fi
\begin{problem}

Use the Fundamental Theorem of Calculus to evaluate the integral.

\input{Integral-Compute-0011.HELP.tex}

\[
\int_{19}^{19} {-\frac{x - 1}{6 \, \sqrt{x}}}\;dx = \answer{0}
\]
\end{problem}}%}

\latexProblemContent{
\ifVerboseLocation This is Integration Compute Question 0011. \\ \fi
\begin{problem}

Use the Fundamental Theorem of Calculus to evaluate the integral.

\input{Integral-Compute-0011.HELP.tex}

\[
\int_{17}^{17} {-\frac{x - 5}{\sqrt{x}}}\;dx = \answer{0}
\]
\end{problem}}%}

\latexProblemContent{
\ifVerboseLocation This is Integration Compute Question 0011. \\ \fi
\begin{problem}

Use the Fundamental Theorem of Calculus to evaluate the integral.

\input{Integral-Compute-0011.HELP.tex}

\[
\int_{18}^{21} {\frac{x - 2}{8 \, \sqrt{x}}}\;dx = \answer{\frac{5}{4} \, \sqrt{21} - 3 \, \sqrt{2}}
\]
\end{problem}}%}

\latexProblemContent{
\ifVerboseLocation This is Integration Compute Question 0011. \\ \fi
\begin{problem}

Use the Fundamental Theorem of Calculus to evaluate the integral.

\input{Integral-Compute-0011.HELP.tex}

\[
\int_{1}^{6} {-\frac{x + 5}{3 \, \sqrt{x}}}\;dx = \answer{-\frac{14}{3} \, \sqrt{6} + \frac{32}{9}}
\]
\end{problem}}%}

\latexProblemContent{
\ifVerboseLocation This is Integration Compute Question 0011. \\ \fi
\begin{problem}

Use the Fundamental Theorem of Calculus to evaluate the integral.

\input{Integral-Compute-0011.HELP.tex}

\[
\int_{4}^{11} {-\frac{x - 5}{3 \, \sqrt{x}}}\;dx = \answer{\frac{8}{9} \, \sqrt{11} - \frac{44}{9}}
\]
\end{problem}}%}

\latexProblemContent{
\ifVerboseLocation This is Integration Compute Question 0011. \\ \fi
\begin{problem}

Use the Fundamental Theorem of Calculus to evaluate the integral.

\input{Integral-Compute-0011.HELP.tex}

\[
\int_{2}^{17} {\frac{x + 4}{2 \, \sqrt{x}}}\;dx = \answer{\frac{29}{3} \, \sqrt{17} - \frac{14}{3} \, \sqrt{2}}
\]
\end{problem}}%}

\latexProblemContent{
\ifVerboseLocation This is Integration Compute Question 0011. \\ \fi
\begin{problem}

Use the Fundamental Theorem of Calculus to evaluate the integral.

\input{Integral-Compute-0011.HELP.tex}

\[
\int_{1}^{16} {-\frac{x + 5}{6 \, \sqrt{x}}}\;dx = \answer{-12}
\]
\end{problem}}%}

\latexProblemContent{
\ifVerboseLocation This is Integration Compute Question 0011. \\ \fi
\begin{problem}

Use the Fundamental Theorem of Calculus to evaluate the integral.

\input{Integral-Compute-0011.HELP.tex}

\[
\int_{4}^{17} {-\frac{x + 5}{8 \, \sqrt{x}}}\;dx = \answer{-\frac{8}{3} \, \sqrt{17} + \frac{19}{6}}
\]
\end{problem}}%}

\latexProblemContent{
\ifVerboseLocation This is Integration Compute Question 0011. \\ \fi
\begin{problem}

Use the Fundamental Theorem of Calculus to evaluate the integral.

\input{Integral-Compute-0011.HELP.tex}

\[
\int_{4}^{21} {\frac{x + 3}{2 \, \sqrt{x}}}\;dx = \answer{10 \, \sqrt{21} - \frac{26}{3}}
\]
\end{problem}}%}

\latexProblemContent{
\ifVerboseLocation This is Integration Compute Question 0011. \\ \fi
\begin{problem}

Use the Fundamental Theorem of Calculus to evaluate the integral.

\input{Integral-Compute-0011.HELP.tex}

\[
\int_{6}^{11} {-\frac{x - 4}{7 \, \sqrt{x}}}\;dx = \answer{\frac{2}{21} \, \sqrt{11} - \frac{4}{7} \, \sqrt{6}}
\]
\end{problem}}%}

\latexProblemContent{
\ifVerboseLocation This is Integration Compute Question 0011. \\ \fi
\begin{problem}

Use the Fundamental Theorem of Calculus to evaluate the integral.

\input{Integral-Compute-0011.HELP.tex}

\[
\int_{8}^{10} {\frac{x + 1}{3 \, \sqrt{x}}}\;dx = \answer{\frac{26}{9} \, \sqrt{10} - \frac{44}{9} \, \sqrt{2}}
\]
\end{problem}}%}

\latexProblemContent{
\ifVerboseLocation This is Integration Compute Question 0011. \\ \fi
\begin{problem}

Use the Fundamental Theorem of Calculus to evaluate the integral.

\input{Integral-Compute-0011.HELP.tex}

\[
\int_{11}^{20} {\frac{x - 4}{7 \, \sqrt{x}}}\;dx = \answer{\frac{2}{21} \, \sqrt{11} + \frac{32}{21} \, \sqrt{5}}
\]
\end{problem}}%}

\latexProblemContent{
\ifVerboseLocation This is Integration Compute Question 0011. \\ \fi
\begin{problem}

Use the Fundamental Theorem of Calculus to evaluate the integral.

\input{Integral-Compute-0011.HELP.tex}

\[
\int_{11}^{18} {-\frac{x + 3}{3 \, \sqrt{x}}}\;dx = \answer{\frac{40}{9} \, \sqrt{11} - 18 \, \sqrt{2}}
\]
\end{problem}}%}

\latexProblemContent{
\ifVerboseLocation This is Integration Compute Question 0011. \\ \fi
\begin{problem}

Use the Fundamental Theorem of Calculus to evaluate the integral.

\input{Integral-Compute-0011.HELP.tex}

\[
\int_{16}^{20} {\frac{x - 5}{2 \, \sqrt{x}}}\;dx = \answer{\frac{10}{3} \, \sqrt{5} - \frac{4}{3}}
\]
\end{problem}}%}

\latexProblemContent{
\ifVerboseLocation This is Integration Compute Question 0011. \\ \fi
\begin{problem}

Use the Fundamental Theorem of Calculus to evaluate the integral.

\input{Integral-Compute-0011.HELP.tex}

\[
\int_{1}^{8} {\frac{x - 4}{8 \, \sqrt{x}}}\;dx = \answer{-\frac{1}{24} \, \left(16 \, \sqrt{2}\right) + \frac{11}{12}}
\]
\end{problem}}%}

\latexProblemContent{
\ifVerboseLocation This is Integration Compute Question 0011. \\ \fi
\begin{problem}

Use the Fundamental Theorem of Calculus to evaluate the integral.

\input{Integral-Compute-0011.HELP.tex}

\[
\int_{8}^{13} {-\frac{x - 1}{6 \, \sqrt{x}}}\;dx = \answer{-\frac{10}{9} \, \sqrt{13} + \frac{10}{9} \, \sqrt{2}}
\]
\end{problem}}%}

\latexProblemContent{
\ifVerboseLocation This is Integration Compute Question 0011. \\ \fi
\begin{problem}

Use the Fundamental Theorem of Calculus to evaluate the integral.

\input{Integral-Compute-0011.HELP.tex}

\[
\int_{2}^{11} {-\frac{x + 2}{\sqrt{x}}}\;dx = \answer{-\frac{34}{3} \, \sqrt{11} + \frac{16}{3} \, \sqrt{2}}
\]
\end{problem}}%}

\latexProblemContent{
\ifVerboseLocation This is Integration Compute Question 0011. \\ \fi
\begin{problem}

Use the Fundamental Theorem of Calculus to evaluate the integral.

\input{Integral-Compute-0011.HELP.tex}

\[
\int_{9}^{13} {\frac{x - 4}{6 \, \sqrt{x}}}\;dx = \answer{\frac{1}{9} \, \sqrt{13} + 1}
\]
\end{problem}}%}

\latexProblemContent{
\ifVerboseLocation This is Integration Compute Question 0011. \\ \fi
\begin{problem}

Use the Fundamental Theorem of Calculus to evaluate the integral.

\input{Integral-Compute-0011.HELP.tex}

\[
\int_{13}^{20} {\frac{x - 4}{7 \, \sqrt{x}}}\;dx = \answer{-\frac{2}{21} \, \sqrt{13} + \frac{32}{21} \, \sqrt{5}}
\]
\end{problem}}%}

\latexProblemContent{
\ifVerboseLocation This is Integration Compute Question 0011. \\ \fi
\begin{problem}

Use the Fundamental Theorem of Calculus to evaluate the integral.

\input{Integral-Compute-0011.HELP.tex}

\[
\int_{19}^{19} {\frac{x - 3}{6 \, \sqrt{x}}}\;dx = \answer{0}
\]
\end{problem}}%}

\latexProblemContent{
\ifVerboseLocation This is Integration Compute Question 0011. \\ \fi
\begin{problem}

Use the Fundamental Theorem of Calculus to evaluate the integral.

\input{Integral-Compute-0011.HELP.tex}

\[
\int_{20}^{20} {\frac{x - 1}{2 \, \sqrt{x}}}\;dx = \answer{0}
\]
\end{problem}}%}

\latexProblemContent{
\ifVerboseLocation This is Integration Compute Question 0011. \\ \fi
\begin{problem}

Use the Fundamental Theorem of Calculus to evaluate the integral.

\input{Integral-Compute-0011.HELP.tex}

\[
\int_{18}^{19} {-\frac{x - 1}{2 \, \sqrt{x}}}\;dx = \answer{-\frac{16}{3} \, \sqrt{19} + 15 \, \sqrt{2}}
\]
\end{problem}}%}

\latexProblemContent{
\ifVerboseLocation This is Integration Compute Question 0011. \\ \fi
\begin{problem}

Use the Fundamental Theorem of Calculus to evaluate the integral.

\input{Integral-Compute-0011.HELP.tex}

\[
\int_{4}^{8} {-\frac{x + 3}{3 \, \sqrt{x}}}\;dx = \answer{-\frac{68}{9} \, \sqrt{2} + \frac{52}{9}}
\]
\end{problem}}%}

\latexProblemContent{
\ifVerboseLocation This is Integration Compute Question 0011. \\ \fi
\begin{problem}

Use the Fundamental Theorem of Calculus to evaluate the integral.

\input{Integral-Compute-0011.HELP.tex}

\[
\int_{20}^{20} {\frac{x - 5}{2 \, \sqrt{x}}}\;dx = \answer{0}
\]
\end{problem}}%}

\latexProblemContent{
\ifVerboseLocation This is Integration Compute Question 0011. \\ \fi
\begin{problem}

Use the Fundamental Theorem of Calculus to evaluate the integral.

\input{Integral-Compute-0011.HELP.tex}

\[
\int_{18}^{19} {\frac{x - 4}{7 \, \sqrt{x}}}\;dx = \answer{\frac{2}{3} \, \sqrt{19} - \frac{12}{7} \, \sqrt{2}}
\]
\end{problem}}%}

\latexProblemContent{
\ifVerboseLocation This is Integration Compute Question 0011. \\ \fi
\begin{problem}

Use the Fundamental Theorem of Calculus to evaluate the integral.

\input{Integral-Compute-0011.HELP.tex}

\[
\int_{20}^{21} {\frac{x - 3}{3 \, \sqrt{x}}}\;dx = \answer{\frac{8}{3} \, \sqrt{21} - \frac{44}{9} \, \sqrt{5}}
\]
\end{problem}}%}

\latexProblemContent{
\ifVerboseLocation This is Integration Compute Question 0011. \\ \fi
\begin{problem}

Use the Fundamental Theorem of Calculus to evaluate the integral.

\input{Integral-Compute-0011.HELP.tex}

\[
\int_{20}^{20} {-\frac{x - 3}{5 \, \sqrt{x}}}\;dx = \answer{0}
\]
\end{problem}}%}

\latexProblemContent{
\ifVerboseLocation This is Integration Compute Question 0011. \\ \fi
\begin{problem}

Use the Fundamental Theorem of Calculus to evaluate the integral.

\input{Integral-Compute-0011.HELP.tex}

\[
\int_{19}^{19} {-\frac{x - 1}{2 \, \sqrt{x}}}\;dx = \answer{0}
\]
\end{problem}}%}

\latexProblemContent{
\ifVerboseLocation This is Integration Compute Question 0011. \\ \fi
\begin{problem}

Use the Fundamental Theorem of Calculus to evaluate the integral.

\input{Integral-Compute-0011.HELP.tex}

\[
\int_{19}^{19} {\frac{x + 3}{5 \, \sqrt{x}}}\;dx = \answer{0}
\]
\end{problem}}%}

\latexProblemContent{
\ifVerboseLocation This is Integration Compute Question 0011. \\ \fi
\begin{problem}

Use the Fundamental Theorem of Calculus to evaluate the integral.

\input{Integral-Compute-0011.HELP.tex}

\[
\int_{19}^{21} {-\frac{x + 1}{2 \, \sqrt{x}}}\;dx = \answer{-8 \, \sqrt{21} + \frac{22}{3} \, \sqrt{19}}
\]
\end{problem}}%}

\latexProblemContent{
\ifVerboseLocation This is Integration Compute Question 0011. \\ \fi
\begin{problem}

Use the Fundamental Theorem of Calculus to evaluate the integral.

\input{Integral-Compute-0011.HELP.tex}

\[
\int_{10}^{13} {\frac{x - 3}{5 \, \sqrt{x}}}\;dx = \answer{\frac{8}{15} \, \sqrt{13} - \frac{2}{15} \, \sqrt{10}}
\]
\end{problem}}%}

\latexProblemContent{
\ifVerboseLocation This is Integration Compute Question 0011. \\ \fi
\begin{problem}

Use the Fundamental Theorem of Calculus to evaluate the integral.

\input{Integral-Compute-0011.HELP.tex}

\[
\int_{11}^{18} {\frac{x - 1}{4 \, \sqrt{x}}}\;dx = \answer{-\frac{4}{3} \, \sqrt{11} + \frac{15}{2} \, \sqrt{2}}
\]
\end{problem}}%}

\latexProblemContent{
\ifVerboseLocation This is Integration Compute Question 0011. \\ \fi
\begin{problem}

Use the Fundamental Theorem of Calculus to evaluate the integral.

\input{Integral-Compute-0011.HELP.tex}

\[
\int_{3}^{5} {\frac{x + 4}{3 \, \sqrt{x}}}\;dx = \answer{\frac{34}{9} \, \sqrt{5} - \frac{10}{3} \, \sqrt{3}}
\]
\end{problem}}%}

\latexProblemContent{
\ifVerboseLocation This is Integration Compute Question 0011. \\ \fi
\begin{problem}

Use the Fundamental Theorem of Calculus to evaluate the integral.

\input{Integral-Compute-0011.HELP.tex}

\[
\int_{20}^{21} {\frac{x + 2}{3 \, \sqrt{x}}}\;dx = \answer{6 \, \sqrt{21} - \frac{104}{9} \, \sqrt{5}}
\]
\end{problem}}%}

\latexProblemContent{
\ifVerboseLocation This is Integration Compute Question 0011. \\ \fi
\begin{problem}

Use the Fundamental Theorem of Calculus to evaluate the integral.

\input{Integral-Compute-0011.HELP.tex}

\[
\int_{2}^{4} {\frac{x + 1}{\sqrt{x}}}\;dx = \answer{-\frac{10}{3} \, \sqrt{2} + \frac{28}{3}}
\]
\end{problem}}%}

\latexProblemContent{
\ifVerboseLocation This is Integration Compute Question 0011. \\ \fi
\begin{problem}

Use the Fundamental Theorem of Calculus to evaluate the integral.

\input{Integral-Compute-0011.HELP.tex}

\[
\int_{4}^{4} {-\frac{x - 5}{7 \, \sqrt{x}}}\;dx = \answer{0}
\]
\end{problem}}%}

\latexProblemContent{
\ifVerboseLocation This is Integration Compute Question 0011. \\ \fi
\begin{problem}

Use the Fundamental Theorem of Calculus to evaluate the integral.

\input{Integral-Compute-0011.HELP.tex}

\[
\int_{2}^{11} {-\frac{x + 5}{7 \, \sqrt{x}}}\;dx = \answer{-\frac{52}{21} \, \sqrt{11} + \frac{34}{21} \, \sqrt{2}}
\]
\end{problem}}%}

\latexProblemContent{
\ifVerboseLocation This is Integration Compute Question 0011. \\ \fi
\begin{problem}

Use the Fundamental Theorem of Calculus to evaluate the integral.

\input{Integral-Compute-0011.HELP.tex}

\[
\int_{7}^{16} {-\frac{x + 3}{6 \, \sqrt{x}}}\;dx = \answer{\frac{16}{9} \, \sqrt{7} - \frac{100}{9}}
\]
\end{problem}}%}

\latexProblemContent{
\ifVerboseLocation This is Integration Compute Question 0011. \\ \fi
\begin{problem}

Use the Fundamental Theorem of Calculus to evaluate the integral.

\input{Integral-Compute-0011.HELP.tex}

\[
\int_{5}^{16} {\frac{x + 5}{5 \, \sqrt{x}}}\;dx = \answer{-\frac{8}{3} \, \sqrt{5} + \frac{248}{15}}
\]
\end{problem}}%}

\latexProblemContent{
\ifVerboseLocation This is Integration Compute Question 0011. \\ \fi
\begin{problem}

Use the Fundamental Theorem of Calculus to evaluate the integral.

\input{Integral-Compute-0011.HELP.tex}

\[
\int_{18}^{21} {\frac{x - 5}{8 \, \sqrt{x}}}\;dx = \answer{\frac{1}{2} \, \sqrt{21} - \frac{3}{4} \, \sqrt{2}}
\]
\end{problem}}%}

\latexProblemContent{
\ifVerboseLocation This is Integration Compute Question 0011. \\ \fi
\begin{problem}

Use the Fundamental Theorem of Calculus to evaluate the integral.

\input{Integral-Compute-0011.HELP.tex}

\[
\int_{9}^{13} {\frac{x - 2}{\sqrt{x}}}\;dx = \answer{\frac{14}{3} \, \sqrt{13} - 6}
\]
\end{problem}}%}

\latexProblemContent{
\ifVerboseLocation This is Integration Compute Question 0011. \\ \fi
\begin{problem}

Use the Fundamental Theorem of Calculus to evaluate the integral.

\input{Integral-Compute-0011.HELP.tex}

\[
\int_{6}^{21} {-\frac{x - 2}{\sqrt{x}}}\;dx = \answer{-10 \, \sqrt{21}}
\]
\end{problem}}%}

\latexProblemContent{
\ifVerboseLocation This is Integration Compute Question 0011. \\ \fi
\begin{problem}

Use the Fundamental Theorem of Calculus to evaluate the integral.

\input{Integral-Compute-0011.HELP.tex}

\[
\int_{15}^{15} {-\frac{x + 2}{\sqrt{x}}}\;dx = \answer{0}
\]
\end{problem}}%}

\latexProblemContent{
\ifVerboseLocation This is Integration Compute Question 0011. \\ \fi
\begin{problem}

Use the Fundamental Theorem of Calculus to evaluate the integral.

\input{Integral-Compute-0011.HELP.tex}

\[
\int_{20}^{21} {\frac{x + 3}{\sqrt{x}}}\;dx = \answer{20 \, \sqrt{21} - \frac{116}{3} \, \sqrt{5}}
\]
\end{problem}}%}

\latexProblemContent{
\ifVerboseLocation This is Integration Compute Question 0011. \\ \fi
\begin{problem}

Use the Fundamental Theorem of Calculus to evaluate the integral.

\input{Integral-Compute-0011.HELP.tex}

\[
\int_{12}^{15} {\frac{x - 5}{4 \, \sqrt{x}}}\;dx = \answer{\sqrt{3}}
\]
\end{problem}}%}

\latexProblemContent{
\ifVerboseLocation This is Integration Compute Question 0011. \\ \fi
\begin{problem}

Use the Fundamental Theorem of Calculus to evaluate the integral.

\input{Integral-Compute-0011.HELP.tex}

\[
\int_{11}^{19} {\frac{x + 2}{5 \, \sqrt{x}}}\;dx = \answer{\frac{10}{3} \, \sqrt{19} - \frac{34}{15} \, \sqrt{11}}
\]
\end{problem}}%}

\latexProblemContent{
\ifVerboseLocation This is Integration Compute Question 0011. \\ \fi
\begin{problem}

Use the Fundamental Theorem of Calculus to evaluate the integral.

\input{Integral-Compute-0011.HELP.tex}

\[
\int_{3}^{11} {\frac{x - 1}{5 \, \sqrt{x}}}\;dx = \answer{\frac{16}{15} \, \sqrt{11}}
\]
\end{problem}}%}

\latexProblemContent{
\ifVerboseLocation This is Integration Compute Question 0011. \\ \fi
\begin{problem}

Use the Fundamental Theorem of Calculus to evaluate the integral.

\input{Integral-Compute-0011.HELP.tex}

\[
\int_{7}^{8} {\frac{x - 2}{\sqrt{x}}}\;dx = \answer{-\frac{2}{3} \, \sqrt{7} + \frac{8}{3} \, \sqrt{2}}
\]
\end{problem}}%}

\latexProblemContent{
\ifVerboseLocation This is Integration Compute Question 0011. \\ \fi
\begin{problem}

Use the Fundamental Theorem of Calculus to evaluate the integral.

\input{Integral-Compute-0011.HELP.tex}

\[
\int_{5}^{21} {\frac{x + 2}{\sqrt{x}}}\;dx = \answer{18 \, \sqrt{21} - \frac{22}{3} \, \sqrt{5}}
\]
\end{problem}}%}

\latexProblemContent{
\ifVerboseLocation This is Integration Compute Question 0011. \\ \fi
\begin{problem}

Use the Fundamental Theorem of Calculus to evaluate the integral.

\input{Integral-Compute-0011.HELP.tex}

\[
\int_{7}^{10} {-\frac{x + 5}{7 \, \sqrt{x}}}\;dx = \answer{-\frac{50}{21} \, \sqrt{10} + \frac{44}{21} \, \sqrt{7}}
\]
\end{problem}}%}

\latexProblemContent{
\ifVerboseLocation This is Integration Compute Question 0011. \\ \fi
\begin{problem}

Use the Fundamental Theorem of Calculus to evaluate the integral.

\input{Integral-Compute-0011.HELP.tex}

\[
\int_{7}^{11} {-\frac{x + 4}{3 \, \sqrt{x}}}\;dx = \answer{-\frac{46}{9} \, \sqrt{11} + \frac{38}{9} \, \sqrt{7}}
\]
\end{problem}}%}

\latexProblemContent{
\ifVerboseLocation This is Integration Compute Question 0011. \\ \fi
\begin{problem}

Use the Fundamental Theorem of Calculus to evaluate the integral.

\input{Integral-Compute-0011.HELP.tex}

\[
\int_{2}^{5} {\frac{x + 1}{4 \, \sqrt{x}}}\;dx = \answer{\frac{4}{3} \, \sqrt{5} - \frac{5}{6} \, \sqrt{2}}
\]
\end{problem}}%}

\latexProblemContent{
\ifVerboseLocation This is Integration Compute Question 0011. \\ \fi
\begin{problem}

Use the Fundamental Theorem of Calculus to evaluate the integral.

\input{Integral-Compute-0011.HELP.tex}

\[
\int_{9}^{9} {-\frac{x - 2}{4 \, \sqrt{x}}}\;dx = \answer{0}
\]
\end{problem}}%}

\latexProblemContent{
\ifVerboseLocation This is Integration Compute Question 0011. \\ \fi
\begin{problem}

Use the Fundamental Theorem of Calculus to evaluate the integral.

\input{Integral-Compute-0011.HELP.tex}

\[
\int_{7}^{18} {\frac{x + 3}{6 \, \sqrt{x}}}\;dx = \answer{-\frac{16}{9} \, \sqrt{7} + 9 \, \sqrt{2}}
\]
\end{problem}}%}

\latexProblemContent{
\ifVerboseLocation This is Integration Compute Question 0011. \\ \fi
\begin{problem}

Use the Fundamental Theorem of Calculus to evaluate the integral.

\input{Integral-Compute-0011.HELP.tex}

\[
\int_{2}^{11} {\frac{x - 5}{5 \, \sqrt{x}}}\;dx = \answer{-\frac{8}{15} \, \sqrt{11} + \frac{26}{15} \, \sqrt{2}}
\]
\end{problem}}%}

\latexProblemContent{
\ifVerboseLocation This is Integration Compute Question 0011. \\ \fi
\begin{problem}

Use the Fundamental Theorem of Calculus to evaluate the integral.

\input{Integral-Compute-0011.HELP.tex}

\[
\int_{5}^{19} {\frac{x - 3}{2 \, \sqrt{x}}}\;dx = \answer{\frac{10}{3} \, \sqrt{19} + \frac{4}{3} \, \sqrt{5}}
\]
\end{problem}}%}

\latexProblemContent{
\ifVerboseLocation This is Integration Compute Question 0011. \\ \fi
\begin{problem}

Use the Fundamental Theorem of Calculus to evaluate the integral.

\input{Integral-Compute-0011.HELP.tex}

\[
\int_{8}^{10} {-\frac{x + 2}{8 \, \sqrt{x}}}\;dx = \answer{-\frac{4}{3} \, \sqrt{10} + \frac{7}{3} \, \sqrt{2}}
\]
\end{problem}}%}

\latexProblemContent{
\ifVerboseLocation This is Integration Compute Question 0011. \\ \fi
\begin{problem}

Use the Fundamental Theorem of Calculus to evaluate the integral.

\input{Integral-Compute-0011.HELP.tex}

\[
\int_{9}^{12} {-\frac{x - 3}{4 \, \sqrt{x}}}\;dx = \answer{-\sqrt{3}}
\]
\end{problem}}%}

\latexProblemContent{
\ifVerboseLocation This is Integration Compute Question 0011. \\ \fi
\begin{problem}

Use the Fundamental Theorem of Calculus to evaluate the integral.

\input{Integral-Compute-0011.HELP.tex}

\[
\int_{4}^{14} {\frac{x - 4}{6 \, \sqrt{x}}}\;dx = \answer{\frac{2}{9} \, \sqrt{14} + \frac{16}{9}}
\]
\end{problem}}%}

\latexProblemContent{
\ifVerboseLocation This is Integration Compute Question 0011. \\ \fi
\begin{problem}

Use the Fundamental Theorem of Calculus to evaluate the integral.

\input{Integral-Compute-0011.HELP.tex}

\[
\int_{9}^{19} {\frac{x + 1}{3 \, \sqrt{x}}}\;dx = \answer{\frac{44}{9} \, \sqrt{19} - 8}
\]
\end{problem}}%}

\latexProblemContent{
\ifVerboseLocation This is Integration Compute Question 0011. \\ \fi
\begin{problem}

Use the Fundamental Theorem of Calculus to evaluate the integral.

\input{Integral-Compute-0011.HELP.tex}

\[
\int_{3}^{3} {-\frac{x - 5}{\sqrt{x}}}\;dx = \answer{0}
\]
\end{problem}}%}

\latexProblemContent{
\ifVerboseLocation This is Integration Compute Question 0011. \\ \fi
\begin{problem}

Use the Fundamental Theorem of Calculus to evaluate the integral.

\input{Integral-Compute-0011.HELP.tex}

\[
\int_{18}^{19} {\frac{x - 3}{3 \, \sqrt{x}}}\;dx = \answer{\frac{20}{9} \, \sqrt{19} - 6 \, \sqrt{2}}
\]
\end{problem}}%}

\latexProblemContent{
\ifVerboseLocation This is Integration Compute Question 0011. \\ \fi
\begin{problem}

Use the Fundamental Theorem of Calculus to evaluate the integral.

\input{Integral-Compute-0011.HELP.tex}

\[
\int_{18}^{18} {-\frac{x + 4}{8 \, \sqrt{x}}}\;dx = \answer{0}
\]
\end{problem}}%}

\latexProblemContent{
\ifVerboseLocation This is Integration Compute Question 0011. \\ \fi
\begin{problem}

Use the Fundamental Theorem of Calculus to evaluate the integral.

\input{Integral-Compute-0011.HELP.tex}

\[
\int_{18}^{19} {\frac{x + 4}{5 \, \sqrt{x}}}\;dx = \answer{\frac{62}{15} \, \sqrt{19} - 12 \, \sqrt{2}}
\]
\end{problem}}%}

\latexProblemContent{
\ifVerboseLocation This is Integration Compute Question 0011. \\ \fi
\begin{problem}

Use the Fundamental Theorem of Calculus to evaluate the integral.

\input{Integral-Compute-0011.HELP.tex}

\[
\int_{7}^{20} {-\frac{x + 5}{2 \, \sqrt{x}}}\;dx = \answer{\frac{22}{3} \, \sqrt{7} - \frac{70}{3} \, \sqrt{5}}
\]
\end{problem}}%}

\latexProblemContent{
\ifVerboseLocation This is Integration Compute Question 0011. \\ \fi
\begin{problem}

Use the Fundamental Theorem of Calculus to evaluate the integral.

\input{Integral-Compute-0011.HELP.tex}

\[
\int_{7}^{11} {\frac{x - 3}{\sqrt{x}}}\;dx = \answer{\frac{4}{3} \, \sqrt{11} + \frac{4}{3} \, \sqrt{7}}
\]
\end{problem}}%}

\latexProblemContent{
\ifVerboseLocation This is Integration Compute Question 0011. \\ \fi
\begin{problem}

Use the Fundamental Theorem of Calculus to evaluate the integral.

\input{Integral-Compute-0011.HELP.tex}

\[
\int_{20}^{20} {\frac{x + 5}{8 \, \sqrt{x}}}\;dx = \answer{0}
\]
\end{problem}}%}

\latexProblemContent{
\ifVerboseLocation This is Integration Compute Question 0011. \\ \fi
\begin{problem}

Use the Fundamental Theorem of Calculus to evaluate the integral.

\input{Integral-Compute-0011.HELP.tex}

\[
\int_{18}^{20} {\frac{x - 2}{2 \, \sqrt{x}}}\;dx = \answer{\frac{28}{3} \, \sqrt{5} - 12 \, \sqrt{2}}
\]
\end{problem}}%}

\latexProblemContent{
\ifVerboseLocation This is Integration Compute Question 0011. \\ \fi
\begin{problem}

Use the Fundamental Theorem of Calculus to evaluate the integral.

\input{Integral-Compute-0011.HELP.tex}

\[
\int_{6}^{6} {\frac{x - 3}{2 \, \sqrt{x}}}\;dx = \answer{0}
\]
\end{problem}}%}

\latexProblemContent{
\ifVerboseLocation This is Integration Compute Question 0011. \\ \fi
\begin{problem}

Use the Fundamental Theorem of Calculus to evaluate the integral.

\input{Integral-Compute-0011.HELP.tex}

\[
\int_{1}^{4} {-\frac{x - 1}{2 \, \sqrt{x}}}\;dx = \answer{-\frac{4}{3}}
\]
\end{problem}}%}

\latexProblemContent{
\ifVerboseLocation This is Integration Compute Question 0011. \\ \fi
\begin{problem}

Use the Fundamental Theorem of Calculus to evaluate the integral.

\input{Integral-Compute-0011.HELP.tex}

\[
\int_{10}^{14} {\frac{x + 5}{8 \, \sqrt{x}}}\;dx = \answer{\frac{29}{12} \, \sqrt{14} - \frac{25}{12} \, \sqrt{10}}
\]
\end{problem}}%}

\latexProblemContent{
\ifVerboseLocation This is Integration Compute Question 0011. \\ \fi
\begin{problem}

Use the Fundamental Theorem of Calculus to evaluate the integral.

\input{Integral-Compute-0011.HELP.tex}

\[
\int_{1}^{9} {-\frac{x + 4}{2 \, \sqrt{x}}}\;dx = \answer{-\frac{50}{3}}
\]
\end{problem}}%}

\latexProblemContent{
\ifVerboseLocation This is Integration Compute Question 0011. \\ \fi
\begin{problem}

Use the Fundamental Theorem of Calculus to evaluate the integral.

\input{Integral-Compute-0011.HELP.tex}

\[
\int_{19}^{21} {-\frac{x - 4}{8 \, \sqrt{x}}}\;dx = \answer{-\frac{3}{4} \, \sqrt{21} + \frac{7}{12} \, \sqrt{19}}
\]
\end{problem}}%}

\latexProblemContent{
\ifVerboseLocation This is Integration Compute Question 0011. \\ \fi
\begin{problem}

Use the Fundamental Theorem of Calculus to evaluate the integral.

\input{Integral-Compute-0011.HELP.tex}

\[
\int_{13}^{19} {\frac{x - 4}{4 \, \sqrt{x}}}\;dx = \answer{\frac{7}{6} \, \sqrt{19} - \frac{1}{6} \, \sqrt{13}}
\]
\end{problem}}%}

\latexProblemContent{
\ifVerboseLocation This is Integration Compute Question 0011. \\ \fi
\begin{problem}

Use the Fundamental Theorem of Calculus to evaluate the integral.

\input{Integral-Compute-0011.HELP.tex}

\[
\int_{3}^{10} {-\frac{x + 3}{2 \, \sqrt{x}}}\;dx = \answer{-\frac{19}{3} \, \sqrt{10} + 4 \, \sqrt{3}}
\]
\end{problem}}%}

\latexProblemContent{
\ifVerboseLocation This is Integration Compute Question 0011. \\ \fi
\begin{problem}

Use the Fundamental Theorem of Calculus to evaluate the integral.

\input{Integral-Compute-0011.HELP.tex}

\[
\int_{1}^{11} {\frac{x + 3}{6 \, \sqrt{x}}}\;dx = \answer{\frac{20}{9} \, \sqrt{11} - \frac{10}{9}}
\]
\end{problem}}%}

\latexProblemContent{
\ifVerboseLocation This is Integration Compute Question 0011. \\ \fi
\begin{problem}

Use the Fundamental Theorem of Calculus to evaluate the integral.

\input{Integral-Compute-0011.HELP.tex}

\[
\int_{16}^{18} {\frac{x + 4}{8 \, \sqrt{x}}}\;dx = \answer{\frac{15}{2} \, \sqrt{2} - \frac{28}{3}}
\]
\end{problem}}%}

\latexProblemContent{
\ifVerboseLocation This is Integration Compute Question 0011. \\ \fi
\begin{problem}

Use the Fundamental Theorem of Calculus to evaluate the integral.

\input{Integral-Compute-0011.HELP.tex}

\[
\int_{11}^{13} {\frac{x - 4}{3 \, \sqrt{x}}}\;dx = \answer{\frac{2}{9} \, \sqrt{13} + \frac{2}{9} \, \sqrt{11}}
\]
\end{problem}}%}

\latexProblemContent{
\ifVerboseLocation This is Integration Compute Question 0011. \\ \fi
\begin{problem}

Use the Fundamental Theorem of Calculus to evaluate the integral.

\input{Integral-Compute-0011.HELP.tex}

\[
\int_{8}^{18} {-\frac{x - 2}{6 \, \sqrt{x}}}\;dx = \answer{-\frac{32}{9} \, \sqrt{2}}
\]
\end{problem}}%}

\latexProblemContent{
\ifVerboseLocation This is Integration Compute Question 0011. \\ \fi
\begin{problem}

Use the Fundamental Theorem of Calculus to evaluate the integral.

\input{Integral-Compute-0011.HELP.tex}

\[
\int_{7}^{13} {\frac{x + 3}{7 \, \sqrt{x}}}\;dx = \answer{\frac{44}{21} \, \sqrt{13} - \frac{32}{21} \, \sqrt{7}}
\]
\end{problem}}%}

\latexProblemContent{
\ifVerboseLocation This is Integration Compute Question 0011. \\ \fi
\begin{problem}

Use the Fundamental Theorem of Calculus to evaluate the integral.

\input{Integral-Compute-0011.HELP.tex}

\[
\int_{20}^{20} {\frac{x - 5}{8 \, \sqrt{x}}}\;dx = \answer{0}
\]
\end{problem}}%}

\latexProblemContent{
\ifVerboseLocation This is Integration Compute Question 0011. \\ \fi
\begin{problem}

Use the Fundamental Theorem of Calculus to evaluate the integral.

\input{Integral-Compute-0011.HELP.tex}

\[
\int_{1}^{15} {\frac{x + 5}{6 \, \sqrt{x}}}\;dx = \answer{\frac{10}{3} \, \sqrt{15} - \frac{16}{9}}
\]
\end{problem}}%}

\latexProblemContent{
\ifVerboseLocation This is Integration Compute Question 0011. \\ \fi
\begin{problem}

Use the Fundamental Theorem of Calculus to evaluate the integral.

\input{Integral-Compute-0011.HELP.tex}

\[
\int_{5}^{16} {-\frac{x + 3}{3 \, \sqrt{x}}}\;dx = \answer{\frac{28}{9} \, \sqrt{5} - \frac{200}{9}}
\]
\end{problem}}%}

\latexProblemContent{
\ifVerboseLocation This is Integration Compute Question 0011. \\ \fi
\begin{problem}

Use the Fundamental Theorem of Calculus to evaluate the integral.

\input{Integral-Compute-0011.HELP.tex}

\[
\int_{20}^{20} {-\frac{x - 4}{7 \, \sqrt{x}}}\;dx = \answer{0}
\]
\end{problem}}%}

\latexProblemContent{
\ifVerboseLocation This is Integration Compute Question 0011. \\ \fi
\begin{problem}

Use the Fundamental Theorem of Calculus to evaluate the integral.

\input{Integral-Compute-0011.HELP.tex}

\[
\int_{5}^{17} {\frac{x + 5}{\sqrt{x}}}\;dx = \answer{\frac{64}{3} \, \sqrt{17} - \frac{40}{3} \, \sqrt{5}}
\]
\end{problem}}%}

\latexProblemContent{
\ifVerboseLocation This is Integration Compute Question 0011. \\ \fi
\begin{problem}

Use the Fundamental Theorem of Calculus to evaluate the integral.

\input{Integral-Compute-0011.HELP.tex}

\[
\int_{10}^{12} {\frac{x - 5}{\sqrt{x}}}\;dx = \answer{\frac{1}{3} \, \left(10 \, \sqrt{10}\right) - 4 \, \sqrt{3}}
\]
\end{problem}}%}

\latexProblemContent{
\ifVerboseLocation This is Integration Compute Question 0011. \\ \fi
\begin{problem}

Use the Fundamental Theorem of Calculus to evaluate the integral.

\input{Integral-Compute-0011.HELP.tex}

\[
\int_{3}^{20} {-\frac{x - 2}{4 \, \sqrt{x}}}\;dx = \answer{-\frac{14}{3} \, \sqrt{5} - \frac{1}{2} \, \sqrt{3}}
\]
\end{problem}}%}

\latexProblemContent{
\ifVerboseLocation This is Integration Compute Question 0011. \\ \fi
\begin{problem}

Use the Fundamental Theorem of Calculus to evaluate the integral.

\input{Integral-Compute-0011.HELP.tex}

\[
\int_{10}^{17} {-\frac{x + 5}{8 \, \sqrt{x}}}\;dx = \answer{-\frac{8}{3} \, \sqrt{17} + \frac{25}{12} \, \sqrt{10}}
\]
\end{problem}}%}

\latexProblemContent{
\ifVerboseLocation This is Integration Compute Question 0011. \\ \fi
\begin{problem}

Use the Fundamental Theorem of Calculus to evaluate the integral.

\input{Integral-Compute-0011.HELP.tex}

\[
\int_{17}^{19} {-\frac{x + 3}{7 \, \sqrt{x}}}\;dx = \answer{-\frac{8}{3} \, \sqrt{19} + \frac{52}{21} \, \sqrt{17}}
\]
\end{problem}}%}

\latexProblemContent{
\ifVerboseLocation This is Integration Compute Question 0011. \\ \fi
\begin{problem}

Use the Fundamental Theorem of Calculus to evaluate the integral.

\input{Integral-Compute-0011.HELP.tex}

\[
\int_{1}^{10} {-\frac{x + 3}{8 \, \sqrt{x}}}\;dx = \answer{-\frac{19}{12} \, \sqrt{10} + \frac{5}{6}}
\]
\end{problem}}%}

\latexProblemContent{
\ifVerboseLocation This is Integration Compute Question 0011. \\ \fi
\begin{problem}

Use the Fundamental Theorem of Calculus to evaluate the integral.

\input{Integral-Compute-0011.HELP.tex}

\[
\int_{19}^{19} {-\frac{x - 4}{4 \, \sqrt{x}}}\;dx = \answer{0}
\]
\end{problem}}%}

\latexProblemContent{
\ifVerboseLocation This is Integration Compute Question 0011. \\ \fi
\begin{problem}

Use the Fundamental Theorem of Calculus to evaluate the integral.

\input{Integral-Compute-0011.HELP.tex}

\[
\int_{8}^{15} {\frac{x - 1}{2 \, \sqrt{x}}}\;dx = \answer{4 \, \sqrt{15} - \frac{10}{3} \, \sqrt{2}}
\]
\end{problem}}%}

\latexProblemContent{
\ifVerboseLocation This is Integration Compute Question 0011. \\ \fi
\begin{problem}

Use the Fundamental Theorem of Calculus to evaluate the integral.

\input{Integral-Compute-0011.HELP.tex}

\[
\int_{6}^{16} {\frac{x + 2}{2 \, \sqrt{x}}}\;dx = \answer{-4 \, \sqrt{6} + \frac{88}{3}}
\]
\end{problem}}%}

\latexProblemContent{
\ifVerboseLocation This is Integration Compute Question 0011. \\ \fi
\begin{problem}

Use the Fundamental Theorem of Calculus to evaluate the integral.

\input{Integral-Compute-0011.HELP.tex}

\[
\int_{15}^{15} {\frac{x + 5}{5 \, \sqrt{x}}}\;dx = \answer{0}
\]
\end{problem}}%}

\latexProblemContent{
\ifVerboseLocation This is Integration Compute Question 0011. \\ \fi
\begin{problem}

Use the Fundamental Theorem of Calculus to evaluate the integral.

\input{Integral-Compute-0011.HELP.tex}

\[
\int_{3}^{12} {-\frac{x - 4}{2 \, \sqrt{x}}}\;dx = \answer{-3 \, \sqrt{3}}
\]
\end{problem}}%}

\latexProblemContent{
\ifVerboseLocation This is Integration Compute Question 0011. \\ \fi
\begin{problem}

Use the Fundamental Theorem of Calculus to evaluate the integral.

\input{Integral-Compute-0011.HELP.tex}

\[
\int_{6}^{16} {\frac{x + 3}{7 \, \sqrt{x}}}\;dx = \answer{-\frac{10}{7} \, \sqrt{6} + \frac{200}{21}}
\]
\end{problem}}%}

\latexProblemContent{
\ifVerboseLocation This is Integration Compute Question 0011. \\ \fi
\begin{problem}

Use the Fundamental Theorem of Calculus to evaluate the integral.

\input{Integral-Compute-0011.HELP.tex}

\[
\int_{2}^{11} {-\frac{x + 5}{3 \, \sqrt{x}}}\;dx = \answer{-\frac{52}{9} \, \sqrt{11} + \frac{34}{9} \, \sqrt{2}}
\]
\end{problem}}%}

\latexProblemContent{
\ifVerboseLocation This is Integration Compute Question 0011. \\ \fi
\begin{problem}

Use the Fundamental Theorem of Calculus to evaluate the integral.

\input{Integral-Compute-0011.HELP.tex}

\[
\int_{14}^{16} {-\frac{x - 3}{5 \, \sqrt{x}}}\;dx = \answer{\frac{2}{3} \, \sqrt{14} - \frac{56}{15}}
\]
\end{problem}}%}

\latexProblemContent{
\ifVerboseLocation This is Integration Compute Question 0011. \\ \fi
\begin{problem}

Use the Fundamental Theorem of Calculus to evaluate the integral.

\input{Integral-Compute-0011.HELP.tex}

\[
\int_{15}^{21} {\frac{x - 1}{7 \, \sqrt{x}}}\;dx = \answer{\frac{12}{7} \, \sqrt{21} - \frac{8}{7} \, \sqrt{15}}
\]
\end{problem}}%}

\latexProblemContent{
\ifVerboseLocation This is Integration Compute Question 0011. \\ \fi
\begin{problem}

Use the Fundamental Theorem of Calculus to evaluate the integral.

\input{Integral-Compute-0011.HELP.tex}

\[
\int_{16}^{21} {-\frac{x - 5}{\sqrt{x}}}\;dx = \answer{-4 \, \sqrt{21} + \frac{8}{3}}
\]
\end{problem}}%}

\latexProblemContent{
\ifVerboseLocation This is Integration Compute Question 0011. \\ \fi
\begin{problem}

Use the Fundamental Theorem of Calculus to evaluate the integral.

\input{Integral-Compute-0011.HELP.tex}

\[
\int_{11}^{19} {\frac{x + 2}{6 \, \sqrt{x}}}\;dx = \answer{\frac{25}{9} \, \sqrt{19} - \frac{17}{9} \, \sqrt{11}}
\]
\end{problem}}%}

\latexProblemContent{
\ifVerboseLocation This is Integration Compute Question 0011. \\ \fi
\begin{problem}

Use the Fundamental Theorem of Calculus to evaluate the integral.

\input{Integral-Compute-0011.HELP.tex}

\[
\int_{1}^{18} {\frac{x - 1}{2 \, \sqrt{x}}}\;dx = \answer{15 \, \sqrt{2} + \frac{2}{3}}
\]
\end{problem}}%}

\latexProblemContent{
\ifVerboseLocation This is Integration Compute Question 0011. \\ \fi
\begin{problem}

Use the Fundamental Theorem of Calculus to evaluate the integral.

\input{Integral-Compute-0011.HELP.tex}

\[
\int_{19}^{21} {-\frac{x + 4}{\sqrt{x}}}\;dx = \answer{-22 \, \sqrt{21} + \frac{62}{3} \, \sqrt{19}}
\]
\end{problem}}%}

\latexProblemContent{
\ifVerboseLocation This is Integration Compute Question 0011. \\ \fi
\begin{problem}

Use the Fundamental Theorem of Calculus to evaluate the integral.

\input{Integral-Compute-0011.HELP.tex}

\[
\int_{11}^{13} {-\frac{x + 2}{8 \, \sqrt{x}}}\;dx = \answer{-\frac{19}{12} \, \sqrt{13} + \frac{17}{12} \, \sqrt{11}}
\]
\end{problem}}%}

\latexProblemContent{
\ifVerboseLocation This is Integration Compute Question 0011. \\ \fi
\begin{problem}

Use the Fundamental Theorem of Calculus to evaluate the integral.

\input{Integral-Compute-0011.HELP.tex}

\[
\int_{12}^{12} {\frac{x + 1}{5 \, \sqrt{x}}}\;dx = \answer{0}
\]
\end{problem}}%}

\latexProblemContent{
\ifVerboseLocation This is Integration Compute Question 0011. \\ \fi
\begin{problem}

Use the Fundamental Theorem of Calculus to evaluate the integral.

\input{Integral-Compute-0011.HELP.tex}

\[
\int_{5}^{15} {-\frac{x - 3}{5 \, \sqrt{x}}}\;dx = \answer{-\frac{4}{5} \, \sqrt{15} - \frac{8}{15} \, \sqrt{5}}
\]
\end{problem}}%}

\latexProblemContent{
\ifVerboseLocation This is Integration Compute Question 0011. \\ \fi
\begin{problem}

Use the Fundamental Theorem of Calculus to evaluate the integral.

\input{Integral-Compute-0011.HELP.tex}

\[
\int_{1}^{10} {-\frac{x + 4}{\sqrt{x}}}\;dx = \answer{-\frac{44}{3} \, \sqrt{10} + \frac{26}{3}}
\]
\end{problem}}%}

\latexProblemContent{
\ifVerboseLocation This is Integration Compute Question 0011. \\ \fi
\begin{problem}

Use the Fundamental Theorem of Calculus to evaluate the integral.

\input{Integral-Compute-0011.HELP.tex}

\[
\int_{13}^{14} {-\frac{x - 1}{8 \, \sqrt{x}}}\;dx = \answer{-\frac{11}{12} \, \sqrt{14} + \frac{5}{6} \, \sqrt{13}}
\]
\end{problem}}%}

\latexProblemContent{
\ifVerboseLocation This is Integration Compute Question 0011. \\ \fi
\begin{problem}

Use the Fundamental Theorem of Calculus to evaluate the integral.

\input{Integral-Compute-0011.HELP.tex}

\[
\int_{12}^{15} {-\frac{x + 3}{\sqrt{x}}}\;dx = \answer{-16 \, \sqrt{15} + 28 \, \sqrt{3}}
\]
\end{problem}}%}

\latexProblemContent{
\ifVerboseLocation This is Integration Compute Question 0011. \\ \fi
\begin{problem}

Use the Fundamental Theorem of Calculus to evaluate the integral.

\input{Integral-Compute-0011.HELP.tex}

\[
\int_{13}^{16} {-\frac{x - 5}{4 \, \sqrt{x}}}\;dx = \answer{-\frac{1}{3} \, \sqrt{13} - \frac{2}{3}}
\]
\end{problem}}%}

\latexProblemContent{
\ifVerboseLocation This is Integration Compute Question 0011. \\ \fi
\begin{problem}

Use the Fundamental Theorem of Calculus to evaluate the integral.

\input{Integral-Compute-0011.HELP.tex}

\[
\int_{7}^{17} {\frac{x + 5}{3 \, \sqrt{x}}}\;dx = \answer{\frac{64}{9} \, \sqrt{17} - \frac{44}{9} \, \sqrt{7}}
\]
\end{problem}}%}

\latexProblemContent{
\ifVerboseLocation This is Integration Compute Question 0011. \\ \fi
\begin{problem}

Use the Fundamental Theorem of Calculus to evaluate the integral.

\input{Integral-Compute-0011.HELP.tex}

\[
\int_{20}^{21} {\frac{x - 1}{6 \, \sqrt{x}}}\;dx = \answer{2 \, \sqrt{21} - \frac{34}{9} \, \sqrt{5}}
\]
\end{problem}}%}

\latexProblemContent{
\ifVerboseLocation This is Integration Compute Question 0011. \\ \fi
\begin{problem}

Use the Fundamental Theorem of Calculus to evaluate the integral.

\input{Integral-Compute-0011.HELP.tex}

\[
\int_{10}^{11} {-\frac{x + 4}{\sqrt{x}}}\;dx = \answer{-\frac{46}{3} \, \sqrt{11} + \frac{44}{3} \, \sqrt{10}}
\]
\end{problem}}%}

\latexProblemContent{
\ifVerboseLocation This is Integration Compute Question 0011. \\ \fi
\begin{problem}

Use the Fundamental Theorem of Calculus to evaluate the integral.

\input{Integral-Compute-0011.HELP.tex}

\[
\int_{15}^{15} {\frac{x - 5}{4 \, \sqrt{x}}}\;dx = \answer{0}
\]
\end{problem}}%}

\latexProblemContent{
\ifVerboseLocation This is Integration Compute Question 0011. \\ \fi
\begin{problem}

Use the Fundamental Theorem of Calculus to evaluate the integral.

\input{Integral-Compute-0011.HELP.tex}

\[
\int_{19}^{20} {\frac{x - 2}{6 \, \sqrt{x}}}\;dx = \answer{-\frac{13}{9} \, \sqrt{19} + \frac{28}{9} \, \sqrt{5}}
\]
\end{problem}}%}

\latexProblemContent{
\ifVerboseLocation This is Integration Compute Question 0011. \\ \fi
\begin{problem}

Use the Fundamental Theorem of Calculus to evaluate the integral.

\input{Integral-Compute-0011.HELP.tex}

\[
\int_{5}^{10} {\frac{x + 1}{4 \, \sqrt{x}}}\;dx = \answer{\frac{13}{6} \, \sqrt{10} - \frac{4}{3} \, \sqrt{5}}
\]
\end{problem}}%}

\latexProblemContent{
\ifVerboseLocation This is Integration Compute Question 0011. \\ \fi
\begin{problem}

Use the Fundamental Theorem of Calculus to evaluate the integral.

\input{Integral-Compute-0011.HELP.tex}

\[
\int_{11}^{11} {\frac{x + 4}{\sqrt{x}}}\;dx = \answer{0}
\]
\end{problem}}%}

\latexProblemContent{
\ifVerboseLocation This is Integration Compute Question 0011. \\ \fi
\begin{problem}

Use the Fundamental Theorem of Calculus to evaluate the integral.

\input{Integral-Compute-0011.HELP.tex}

\[
\int_{15}^{20} {\frac{x - 1}{4 \, \sqrt{x}}}\;dx = \answer{-2 \, \sqrt{15} + \frac{17}{3} \, \sqrt{5}}
\]
\end{problem}}%}

\latexProblemContent{
\ifVerboseLocation This is Integration Compute Question 0011. \\ \fi
\begin{problem}

Use the Fundamental Theorem of Calculus to evaluate the integral.

\input{Integral-Compute-0011.HELP.tex}

\[
\int_{18}^{18} {\frac{x - 5}{7 \, \sqrt{x}}}\;dx = \answer{0}
\]
\end{problem}}%}

\latexProblemContent{
\ifVerboseLocation This is Integration Compute Question 0011. \\ \fi
\begin{problem}

Use the Fundamental Theorem of Calculus to evaluate the integral.

\input{Integral-Compute-0011.HELP.tex}

\[
\int_{1}^{8} {\frac{x + 3}{\sqrt{x}}}\;dx = \answer{\frac{68}{3} \, \sqrt{2} - \frac{20}{3}}
\]
\end{problem}}%}

\latexProblemContent{
\ifVerboseLocation This is Integration Compute Question 0011. \\ \fi
\begin{problem}

Use the Fundamental Theorem of Calculus to evaluate the integral.

\input{Integral-Compute-0011.HELP.tex}

\[
\int_{10}^{13} {-\frac{x + 5}{7 \, \sqrt{x}}}\;dx = \answer{-\frac{8}{3} \, \sqrt{13} + \frac{50}{21} \, \sqrt{10}}
\]
\end{problem}}%}

\latexProblemContent{
\ifVerboseLocation This is Integration Compute Question 0011. \\ \fi
\begin{problem}

Use the Fundamental Theorem of Calculus to evaluate the integral.

\input{Integral-Compute-0011.HELP.tex}

\[
\int_{6}^{19} {\frac{x - 3}{8 \, \sqrt{x}}}\;dx = \answer{\frac{1}{24} \, \left(6 \, \sqrt{6}\right) + \frac{5}{6} \, \sqrt{19}}
\]
\end{problem}}%}

\latexProblemContent{
\ifVerboseLocation This is Integration Compute Question 0011. \\ \fi
\begin{problem}

Use the Fundamental Theorem of Calculus to evaluate the integral.

\input{Integral-Compute-0011.HELP.tex}

\[
\int_{5}^{17} {-\frac{x + 3}{5 \, \sqrt{x}}}\;dx = \answer{-\frac{52}{15} \, \sqrt{17} + \frac{28}{15} \, \sqrt{5}}
\]
\end{problem}}%}

\latexProblemContent{
\ifVerboseLocation This is Integration Compute Question 0011. \\ \fi
\begin{problem}

Use the Fundamental Theorem of Calculus to evaluate the integral.

\input{Integral-Compute-0011.HELP.tex}

\[
\int_{2}^{7} {\frac{x + 2}{\sqrt{x}}}\;dx = \answer{\frac{26}{3} \, \sqrt{7} - \frac{16}{3} \, \sqrt{2}}
\]
\end{problem}}%}

\latexProblemContent{
\ifVerboseLocation This is Integration Compute Question 0011. \\ \fi
\begin{problem}

Use the Fundamental Theorem of Calculus to evaluate the integral.

\input{Integral-Compute-0011.HELP.tex}

\[
\int_{18}^{19} {\frac{x - 4}{6 \, \sqrt{x}}}\;dx = \answer{\frac{7}{9} \, \sqrt{19} - 2 \, \sqrt{2}}
\]
\end{problem}}%}

\latexProblemContent{
\ifVerboseLocation This is Integration Compute Question 0011. \\ \fi
\begin{problem}

Use the Fundamental Theorem of Calculus to evaluate the integral.

\input{Integral-Compute-0011.HELP.tex}

\[
\int_{6}^{13} {\frac{x - 5}{7 \, \sqrt{x}}}\;dx = \answer{-\frac{4}{21} \, \sqrt{13} + \frac{6}{7} \, \sqrt{6}}
\]
\end{problem}}%}

\latexProblemContent{
\ifVerboseLocation This is Integration Compute Question 0011. \\ \fi
\begin{problem}

Use the Fundamental Theorem of Calculus to evaluate the integral.

\input{Integral-Compute-0011.HELP.tex}

\[
\int_{13}^{14} {\frac{x - 3}{\sqrt{x}}}\;dx = \answer{\frac{10}{3} \, \sqrt{14} - \frac{8}{3} \, \sqrt{13}}
\]
\end{problem}}%}

\latexProblemContent{
\ifVerboseLocation This is Integration Compute Question 0011. \\ \fi
\begin{problem}

Use the Fundamental Theorem of Calculus to evaluate the integral.

\input{Integral-Compute-0011.HELP.tex}

\[
\int_{4}^{17} {-\frac{x - 2}{6 \, \sqrt{x}}}\;dx = \answer{-\frac{11}{9} \, \sqrt{17} - \frac{4}{9}}
\]
\end{problem}}%}

\latexProblemContent{
\ifVerboseLocation This is Integration Compute Question 0011. \\ \fi
\begin{problem}

Use the Fundamental Theorem of Calculus to evaluate the integral.

\input{Integral-Compute-0011.HELP.tex}

\[
\int_{8}^{20} {\frac{x - 2}{8 \, \sqrt{x}}}\;dx = \answer{\frac{7}{3} \, \sqrt{5} - \frac{1}{3} \, \sqrt{2}}
\]
\end{problem}}%}

\latexProblemContent{
\ifVerboseLocation This is Integration Compute Question 0011. \\ \fi
\begin{problem}

Use the Fundamental Theorem of Calculus to evaluate the integral.

\input{Integral-Compute-0011.HELP.tex}

\[
\int_{14}^{19} {\frac{x - 5}{7 \, \sqrt{x}}}\;dx = \answer{\frac{8}{21} \, \sqrt{19} + \frac{2}{21} \, \sqrt{14}}
\]
\end{problem}}%}

\latexProblemContent{
\ifVerboseLocation This is Integration Compute Question 0011. \\ \fi
\begin{problem}

Use the Fundamental Theorem of Calculus to evaluate the integral.

\input{Integral-Compute-0011.HELP.tex}

\[
\int_{9}^{18} {\frac{x - 3}{3 \, \sqrt{x}}}\;dx = \answer{6 \, \sqrt{2}}
\]
\end{problem}}%}

\latexProblemContent{
\ifVerboseLocation This is Integration Compute Question 0011. \\ \fi
\begin{problem}

Use the Fundamental Theorem of Calculus to evaluate the integral.

\input{Integral-Compute-0011.HELP.tex}

\[
\int_{3}^{12} {\frac{x - 3}{6 \, \sqrt{x}}}\;dx = \answer{\frac{4}{3} \, \sqrt{3}}
\]
\end{problem}}%}

\latexProblemContent{
\ifVerboseLocation This is Integration Compute Question 0011. \\ \fi
\begin{problem}

Use the Fundamental Theorem of Calculus to evaluate the integral.

\input{Integral-Compute-0011.HELP.tex}

\[
\int_{9}^{17} {\frac{x + 3}{7 \, \sqrt{x}}}\;dx = \answer{\frac{52}{21} \, \sqrt{17} - \frac{36}{7}}
\]
\end{problem}}%}

\latexProblemContent{
\ifVerboseLocation This is Integration Compute Question 0011. \\ \fi
\begin{problem}

Use the Fundamental Theorem of Calculus to evaluate the integral.

\input{Integral-Compute-0011.HELP.tex}

\[
\int_{13}^{13} {-\frac{x + 4}{4 \, \sqrt{x}}}\;dx = \answer{0}
\]
\end{problem}}%}

\latexProblemContent{
\ifVerboseLocation This is Integration Compute Question 0011. \\ \fi
\begin{problem}

Use the Fundamental Theorem of Calculus to evaluate the integral.

\input{Integral-Compute-0011.HELP.tex}

\[
\int_{17}^{18} {-\frac{x + 2}{7 \, \sqrt{x}}}\;dx = \answer{\frac{46}{21} \, \sqrt{17} - \frac{48}{7} \, \sqrt{2}}
\]
\end{problem}}%}

\latexProblemContent{
\ifVerboseLocation This is Integration Compute Question 0011. \\ \fi
\begin{problem}

Use the Fundamental Theorem of Calculus to evaluate the integral.

\input{Integral-Compute-0011.HELP.tex}

\[
\int_{13}^{16} {-\frac{x + 4}{6 \, \sqrt{x}}}\;dx = \answer{\frac{25}{9} \, \sqrt{13} - \frac{112}{9}}
\]
\end{problem}}%}

\latexProblemContent{
\ifVerboseLocation This is Integration Compute Question 0011. \\ \fi
\begin{problem}

Use the Fundamental Theorem of Calculus to evaluate the integral.

\input{Integral-Compute-0011.HELP.tex}

\[
\int_{17}^{18} {\frac{x - 2}{5 \, \sqrt{x}}}\;dx = \answer{-\frac{22}{15} \, \sqrt{17} + \frac{24}{5} \, \sqrt{2}}
\]
\end{problem}}%}

\latexProblemContent{
\ifVerboseLocation This is Integration Compute Question 0011. \\ \fi
\begin{problem}

Use the Fundamental Theorem of Calculus to evaluate the integral.

\input{Integral-Compute-0011.HELP.tex}

\[
\int_{2}^{4} {\frac{x + 4}{7 \, \sqrt{x}}}\;dx = \answer{-\frac{4}{3} \, \sqrt{2} + \frac{64}{21}}
\]
\end{problem}}%}

\latexProblemContent{
\ifVerboseLocation This is Integration Compute Question 0011. \\ \fi
\begin{problem}

Use the Fundamental Theorem of Calculus to evaluate the integral.

\input{Integral-Compute-0011.HELP.tex}

\[
\int_{1}^{15} {\frac{x - 3}{\sqrt{x}}}\;dx = \answer{4 \, \sqrt{15} + \frac{16}{3}}
\]
\end{problem}}%}

\latexProblemContent{
\ifVerboseLocation This is Integration Compute Question 0011. \\ \fi
\begin{problem}

Use the Fundamental Theorem of Calculus to evaluate the integral.

\input{Integral-Compute-0011.HELP.tex}

\[
\int_{4}^{20} {-\frac{x + 3}{3 \, \sqrt{x}}}\;dx = \answer{-\frac{116}{9} \, \sqrt{5} + \frac{52}{9}}
\]
\end{problem}}%}

\latexProblemContent{
\ifVerboseLocation This is Integration Compute Question 0011. \\ \fi
\begin{problem}

Use the Fundamental Theorem of Calculus to evaluate the integral.

\input{Integral-Compute-0011.HELP.tex}

\[
\int_{15}^{20} {\frac{x + 1}{6 \, \sqrt{x}}}\;dx = \answer{-2 \, \sqrt{15} + \frac{46}{9} \, \sqrt{5}}
\]
\end{problem}}%}

\latexProblemContent{
\ifVerboseLocation This is Integration Compute Question 0011. \\ \fi
\begin{problem}

Use the Fundamental Theorem of Calculus to evaluate the integral.

\input{Integral-Compute-0011.HELP.tex}

\[
\int_{13}^{14} {-\frac{x + 5}{3 \, \sqrt{x}}}\;dx = \answer{-\frac{58}{9} \, \sqrt{14} + \frac{56}{9} \, \sqrt{13}}
\]
\end{problem}}%}

\latexProblemContent{
\ifVerboseLocation This is Integration Compute Question 0011. \\ \fi
\begin{problem}

Use the Fundamental Theorem of Calculus to evaluate the integral.

\input{Integral-Compute-0011.HELP.tex}

\[
\int_{19}^{21} {\frac{x - 2}{3 \, \sqrt{x}}}\;dx = \answer{\frac{10}{3} \, \sqrt{21} - \frac{26}{9} \, \sqrt{19}}
\]
\end{problem}}%}

\latexProblemContent{
\ifVerboseLocation This is Integration Compute Question 0011. \\ \fi
\begin{problem}

Use the Fundamental Theorem of Calculus to evaluate the integral.

\input{Integral-Compute-0011.HELP.tex}

\[
\int_{10}^{12} {-\frac{x + 2}{2 \, \sqrt{x}}}\;dx = \answer{\frac{16}{3} \, \sqrt{10} - 12 \, \sqrt{3}}
\]
\end{problem}}%}

\latexProblemContent{
\ifVerboseLocation This is Integration Compute Question 0011. \\ \fi
\begin{problem}

Use the Fundamental Theorem of Calculus to evaluate the integral.

\input{Integral-Compute-0011.HELP.tex}

\[
\int_{5}^{19} {\frac{x + 4}{4 \, \sqrt{x}}}\;dx = \answer{\frac{31}{6} \, \sqrt{19} - \frac{17}{6} \, \sqrt{5}}
\]
\end{problem}}%}

\latexProblemContent{
\ifVerboseLocation This is Integration Compute Question 0011. \\ \fi
\begin{problem}

Use the Fundamental Theorem of Calculus to evaluate the integral.

\input{Integral-Compute-0011.HELP.tex}

\[
\int_{16}^{21} {-\frac{x + 3}{5 \, \sqrt{x}}}\;dx = \answer{-4 \, \sqrt{21} + \frac{40}{3}}
\]
\end{problem}}%}

\latexProblemContent{
\ifVerboseLocation This is Integration Compute Question 0011. \\ \fi
\begin{problem}

Use the Fundamental Theorem of Calculus to evaluate the integral.

\input{Integral-Compute-0011.HELP.tex}

\[
\int_{9}^{17} {-\frac{x - 5}{\sqrt{x}}}\;dx = \answer{-\frac{4}{3} \, \sqrt{17} - 12}
\]
\end{problem}}%}

\latexProblemContent{
\ifVerboseLocation This is Integration Compute Question 0011. \\ \fi
\begin{problem}

Use the Fundamental Theorem of Calculus to evaluate the integral.

\input{Integral-Compute-0011.HELP.tex}

\[
\int_{15}^{15} {\frac{x - 5}{5 \, \sqrt{x}}}\;dx = \answer{0}
\]
\end{problem}}%}

\latexProblemContent{
\ifVerboseLocation This is Integration Compute Question 0011. \\ \fi
\begin{problem}

Use the Fundamental Theorem of Calculus to evaluate the integral.

\input{Integral-Compute-0011.HELP.tex}

\[
\int_{9}^{20} {-\frac{x - 5}{\sqrt{x}}}\;dx = \answer{-\frac{20}{3} \, \sqrt{5} - 12}
\]
\end{problem}}%}

\latexProblemContent{
\ifVerboseLocation This is Integration Compute Question 0011. \\ \fi
\begin{problem}

Use the Fundamental Theorem of Calculus to evaluate the integral.

\input{Integral-Compute-0011.HELP.tex}

\[
\int_{13}^{16} {\frac{x + 1}{7 \, \sqrt{x}}}\;dx = \answer{-\frac{32}{21} \, \sqrt{13} + \frac{152}{21}}
\]
\end{problem}}%}

\latexProblemContent{
\ifVerboseLocation This is Integration Compute Question 0011. \\ \fi
\begin{problem}

Use the Fundamental Theorem of Calculus to evaluate the integral.

\input{Integral-Compute-0011.HELP.tex}

\[
\int_{17}^{17} {\frac{x - 5}{7 \, \sqrt{x}}}\;dx = \answer{0}
\]
\end{problem}}%}

\latexProblemContent{
\ifVerboseLocation This is Integration Compute Question 0011. \\ \fi
\begin{problem}

Use the Fundamental Theorem of Calculus to evaluate the integral.

\input{Integral-Compute-0011.HELP.tex}

\[
\int_{13}^{19} {-\frac{x + 3}{\sqrt{x}}}\;dx = \answer{-\frac{56}{3} \, \sqrt{19} + \frac{44}{3} \, \sqrt{13}}
\]
\end{problem}}%}

\latexProblemContent{
\ifVerboseLocation This is Integration Compute Question 0011. \\ \fi
\begin{problem}

Use the Fundamental Theorem of Calculus to evaluate the integral.

\input{Integral-Compute-0011.HELP.tex}

\[
\int_{19}^{19} {-\frac{x - 1}{8 \, \sqrt{x}}}\;dx = \answer{0}
\]
\end{problem}}%}

\latexProblemContent{
\ifVerboseLocation This is Integration Compute Question 0011. \\ \fi
\begin{problem}

Use the Fundamental Theorem of Calculus to evaluate the integral.

\input{Integral-Compute-0011.HELP.tex}

\[
\int_{8}^{9} {\frac{x - 3}{7 \, \sqrt{x}}}\;dx = \answer{\frac{4}{21} \, \sqrt{2}}
\]
\end{problem}}%}

\latexProblemContent{
\ifVerboseLocation This is Integration Compute Question 0011. \\ \fi
\begin{problem}

Use the Fundamental Theorem of Calculus to evaluate the integral.

\input{Integral-Compute-0011.HELP.tex}

\[
\int_{7}^{20} {-\frac{x + 4}{\sqrt{x}}}\;dx = \answer{\frac{38}{3} \, \sqrt{7} - \frac{128}{3} \, \sqrt{5}}
\]
\end{problem}}%}

\latexProblemContent{
\ifVerboseLocation This is Integration Compute Question 0011. \\ \fi
\begin{problem}

Use the Fundamental Theorem of Calculus to evaluate the integral.

\input{Integral-Compute-0011.HELP.tex}

\[
\int_{15}^{19} {\frac{x + 5}{4 \, \sqrt{x}}}\;dx = \answer{\frac{17}{3} \, \sqrt{19} - 5 \, \sqrt{15}}
\]
\end{problem}}%}

\latexProblemContent{
\ifVerboseLocation This is Integration Compute Question 0011. \\ \fi
\begin{problem}

Use the Fundamental Theorem of Calculus to evaluate the integral.

\input{Integral-Compute-0011.HELP.tex}

\[
\int_{3}^{17} {\frac{x + 3}{2 \, \sqrt{x}}}\;dx = \answer{\frac{26}{3} \, \sqrt{17} - 4 \, \sqrt{3}}
\]
\end{problem}}%}

\latexProblemContent{
\ifVerboseLocation This is Integration Compute Question 0011. \\ \fi
\begin{problem}

Use the Fundamental Theorem of Calculus to evaluate the integral.

\input{Integral-Compute-0011.HELP.tex}

\[
\int_{18}^{19} {-\frac{x + 1}{5 \, \sqrt{x}}}\;dx = \answer{-\frac{44}{15} \, \sqrt{19} + \frac{42}{5} \, \sqrt{2}}
\]
\end{problem}}%}

\latexProblemContent{
\ifVerboseLocation This is Integration Compute Question 0011. \\ \fi
\begin{problem}

Use the Fundamental Theorem of Calculus to evaluate the integral.

\input{Integral-Compute-0011.HELP.tex}

\[
\int_{13}^{14} {-\frac{x + 1}{8 \, \sqrt{x}}}\;dx = \answer{-\frac{17}{12} \, \sqrt{14} + \frac{4}{3} \, \sqrt{13}}
\]
\end{problem}}%}

\latexProblemContent{
\ifVerboseLocation This is Integration Compute Question 0011. \\ \fi
\begin{problem}

Use the Fundamental Theorem of Calculus to evaluate the integral.

\input{Integral-Compute-0011.HELP.tex}

\[
\int_{2}^{14} {\frac{x + 3}{3 \, \sqrt{x}}}\;dx = \answer{\frac{46}{9} \, \sqrt{14} - \frac{22}{9} \, \sqrt{2}}
\]
\end{problem}}%}

\latexProblemContent{
\ifVerboseLocation This is Integration Compute Question 0011. \\ \fi
\begin{problem}

Use the Fundamental Theorem of Calculus to evaluate the integral.

\input{Integral-Compute-0011.HELP.tex}

\[
\int_{1}^{11} {-\frac{x + 3}{\sqrt{x}}}\;dx = \answer{-\frac{40}{3} \, \sqrt{11} + \frac{20}{3}}
\]
\end{problem}}%}

\latexProblemContent{
\ifVerboseLocation This is Integration Compute Question 0011. \\ \fi
\begin{problem}

Use the Fundamental Theorem of Calculus to evaluate the integral.

\input{Integral-Compute-0011.HELP.tex}

\[
\int_{8}^{8} {-\frac{x - 1}{6 \, \sqrt{x}}}\;dx = \answer{0}
\]
\end{problem}}%}

\latexProblemContent{
\ifVerboseLocation This is Integration Compute Question 0011. \\ \fi
\begin{problem}

Use the Fundamental Theorem of Calculus to evaluate the integral.

\input{Integral-Compute-0011.HELP.tex}

\[
\int_{16}^{17} {\frac{x - 4}{2 \, \sqrt{x}}}\;dx = \answer{\frac{5}{3} \, \sqrt{17} - \frac{16}{3}}
\]
\end{problem}}%}

\latexProblemContent{
\ifVerboseLocation This is Integration Compute Question 0011. \\ \fi
\begin{problem}

Use the Fundamental Theorem of Calculus to evaluate the integral.

\input{Integral-Compute-0011.HELP.tex}

\[
\int_{8}^{15} {-\frac{x + 4}{3 \, \sqrt{x}}}\;dx = \answer{-6 \, \sqrt{15} + \frac{80}{9} \, \sqrt{2}}
\]
\end{problem}}%}

\latexProblemContent{
\ifVerboseLocation This is Integration Compute Question 0011. \\ \fi
\begin{problem}

Use the Fundamental Theorem of Calculus to evaluate the integral.

\input{Integral-Compute-0011.HELP.tex}

\[
\int_{19}^{21} {-\frac{x - 4}{\sqrt{x}}}\;dx = \answer{-6 \, \sqrt{21} + \frac{14}{3} \, \sqrt{19}}
\]
\end{problem}}%}

\latexProblemContent{
\ifVerboseLocation This is Integration Compute Question 0011. \\ \fi
\begin{problem}

Use the Fundamental Theorem of Calculus to evaluate the integral.

\input{Integral-Compute-0011.HELP.tex}

\[
\int_{19}^{20} {-\frac{x + 1}{\sqrt{x}}}\;dx = \answer{\frac{44}{3} \, \sqrt{19} - \frac{92}{3} \, \sqrt{5}}
\]
\end{problem}}%}

\latexProblemContent{
\ifVerboseLocation This is Integration Compute Question 0011. \\ \fi
\begin{problem}

Use the Fundamental Theorem of Calculus to evaluate the integral.

\input{Integral-Compute-0011.HELP.tex}

\[
\int_{3}^{18} {-\frac{x - 3}{4 \, \sqrt{x}}}\;dx = \answer{-\sqrt{3} - \frac{9}{2} \, \sqrt{2}}
\]
\end{problem}}%}

\latexProblemContent{
\ifVerboseLocation This is Integration Compute Question 0011. \\ \fi
\begin{problem}

Use the Fundamental Theorem of Calculus to evaluate the integral.

\input{Integral-Compute-0011.HELP.tex}

\[
\int_{11}^{15} {\frac{x + 1}{4 \, \sqrt{x}}}\;dx = \answer{3 \, \sqrt{15} - \frac{7}{3} \, \sqrt{11}}
\]
\end{problem}}%}

\latexProblemContent{
\ifVerboseLocation This is Integration Compute Question 0011. \\ \fi
\begin{problem}

Use the Fundamental Theorem of Calculus to evaluate the integral.

\input{Integral-Compute-0011.HELP.tex}

\[
\int_{1}^{19} {\frac{x + 1}{8 \, \sqrt{x}}}\;dx = \answer{\frac{11}{6} \, \sqrt{19} - \frac{1}{3}}
\]
\end{problem}}%}

\latexProblemContent{
\ifVerboseLocation This is Integration Compute Question 0011. \\ \fi
\begin{problem}

Use the Fundamental Theorem of Calculus to evaluate the integral.

\input{Integral-Compute-0011.HELP.tex}

\[
\int_{10}^{10} {\frac{x + 3}{4 \, \sqrt{x}}}\;dx = \answer{0}
\]
\end{problem}}%}

\latexProblemContent{
\ifVerboseLocation This is Integration Compute Question 0011. \\ \fi
\begin{problem}

Use the Fundamental Theorem of Calculus to evaluate the integral.

\input{Integral-Compute-0011.HELP.tex}

\[
\int_{9}^{16} {-\frac{x - 1}{\sqrt{x}}}\;dx = \answer{-\frac{68}{3}}
\]
\end{problem}}%}

\latexProblemContent{
\ifVerboseLocation This is Integration Compute Question 0011. \\ \fi
\begin{problem}

Use the Fundamental Theorem of Calculus to evaluate the integral.

\input{Integral-Compute-0011.HELP.tex}

\[
\int_{3}^{17} {\frac{x - 4}{2 \, \sqrt{x}}}\;dx = \answer{\frac{5}{3} \, \sqrt{17} + 3 \, \sqrt{3}}
\]
\end{problem}}%}

\latexProblemContent{
\ifVerboseLocation This is Integration Compute Question 0011. \\ \fi
\begin{problem}

Use the Fundamental Theorem of Calculus to evaluate the integral.

\input{Integral-Compute-0011.HELP.tex}

\[
\int_{7}^{13} {\frac{x + 2}{7 \, \sqrt{x}}}\;dx = \answer{\frac{38}{21} \, \sqrt{13} - \frac{26}{21} \, \sqrt{7}}
\]
\end{problem}}%}

\latexProblemContent{
\ifVerboseLocation This is Integration Compute Question 0011. \\ \fi
\begin{problem}

Use the Fundamental Theorem of Calculus to evaluate the integral.

\input{Integral-Compute-0011.HELP.tex}

\[
\int_{17}^{21} {-\frac{x - 3}{2 \, \sqrt{x}}}\;dx = \answer{-4 \, \sqrt{21} + \frac{8}{3} \, \sqrt{17}}
\]
\end{problem}}%}

\latexProblemContent{
\ifVerboseLocation This is Integration Compute Question 0011. \\ \fi
\begin{problem}

Use the Fundamental Theorem of Calculus to evaluate the integral.

\input{Integral-Compute-0011.HELP.tex}

\[
\int_{4}^{11} {-\frac{x + 1}{\sqrt{x}}}\;dx = \answer{-\frac{28}{3} \, \sqrt{11} + \frac{28}{3}}
\]
\end{problem}}%}

\latexProblemContent{
\ifVerboseLocation This is Integration Compute Question 0011. \\ \fi
\begin{problem}

Use the Fundamental Theorem of Calculus to evaluate the integral.

\input{Integral-Compute-0011.HELP.tex}

\[
\int_{7}^{9} {\frac{x - 2}{3 \, \sqrt{x}}}\;dx = \answer{-\frac{2}{9} \, \sqrt{7} + 2}
\]
\end{problem}}%}

\latexProblemContent{
\ifVerboseLocation This is Integration Compute Question 0011. \\ \fi
\begin{problem}

Use the Fundamental Theorem of Calculus to evaluate the integral.

\input{Integral-Compute-0011.HELP.tex}

\[
\int_{9}^{14} {\frac{x - 2}{4 \, \sqrt{x}}}\;dx = \answer{\frac{4}{3} \, \sqrt{14} - \frac{3}{2}}
\]
\end{problem}}%}

\latexProblemContent{
\ifVerboseLocation This is Integration Compute Question 0011. \\ \fi
\begin{problem}

Use the Fundamental Theorem of Calculus to evaluate the integral.

\input{Integral-Compute-0011.HELP.tex}

\[
\int_{14}^{19} {-\frac{x - 1}{\sqrt{x}}}\;dx = \answer{-\frac{32}{3} \, \sqrt{19} + \frac{22}{3} \, \sqrt{14}}
\]
\end{problem}}%}

\latexProblemContent{
\ifVerboseLocation This is Integration Compute Question 0011. \\ \fi
\begin{problem}

Use the Fundamental Theorem of Calculus to evaluate the integral.

\input{Integral-Compute-0011.HELP.tex}

\[
\int_{18}^{18} {-\frac{x + 2}{6 \, \sqrt{x}}}\;dx = \answer{0}
\]
\end{problem}}%}

\latexProblemContent{
\ifVerboseLocation This is Integration Compute Question 0011. \\ \fi
\begin{problem}

Use the Fundamental Theorem of Calculus to evaluate the integral.

\input{Integral-Compute-0011.HELP.tex}

\[
\int_{11}^{19} {-\frac{x - 5}{3 \, \sqrt{x}}}\;dx = \answer{-\frac{8}{9} \, \sqrt{19} - \frac{8}{9} \, \sqrt{11}}
\]
\end{problem}}%}

\latexProblemContent{
\ifVerboseLocation This is Integration Compute Question 0011. \\ \fi
\begin{problem}

Use the Fundamental Theorem of Calculus to evaluate the integral.

\input{Integral-Compute-0011.HELP.tex}

\[
\int_{3}^{4} {\frac{x - 4}{\sqrt{x}}}\;dx = \answer{6 \, \sqrt{3} - \frac{32}{3}}
\]
\end{problem}}%}

\latexProblemContent{
\ifVerboseLocation This is Integration Compute Question 0011. \\ \fi
\begin{problem}

Use the Fundamental Theorem of Calculus to evaluate the integral.

\input{Integral-Compute-0011.HELP.tex}

\[
\int_{5}^{9} {-\frac{x - 1}{6 \, \sqrt{x}}}\;dx = \answer{\frac{2}{9} \, \sqrt{5} - 2}
\]
\end{problem}}%}

\latexProblemContent{
\ifVerboseLocation This is Integration Compute Question 0011. \\ \fi
\begin{problem}

Use the Fundamental Theorem of Calculus to evaluate the integral.

\input{Integral-Compute-0011.HELP.tex}

\[
\int_{11}^{14} {\frac{x + 2}{3 \, \sqrt{x}}}\;dx = \answer{\frac{40}{9} \, \sqrt{14} - \frac{34}{9} \, \sqrt{11}}
\]
\end{problem}}%}

\latexProblemContent{
\ifVerboseLocation This is Integration Compute Question 0011. \\ \fi
\begin{problem}

Use the Fundamental Theorem of Calculus to evaluate the integral.

\input{Integral-Compute-0011.HELP.tex}

\[
\int_{17}^{19} {-\frac{x + 5}{7 \, \sqrt{x}}}\;dx = \answer{-\frac{68}{21} \, \sqrt{19} + \frac{64}{21} \, \sqrt{17}}
\]
\end{problem}}%}

\latexProblemContent{
\ifVerboseLocation This is Integration Compute Question 0011. \\ \fi
\begin{problem}

Use the Fundamental Theorem of Calculus to evaluate the integral.

\input{Integral-Compute-0011.HELP.tex}

\[
\int_{2}^{3} {\frac{x - 4}{8 \, \sqrt{x}}}\;dx = \answer{-\frac{3}{4} \, \sqrt{3} + \frac{5}{6} \, \sqrt{2}}
\]
\end{problem}}%}

\latexProblemContent{
\ifVerboseLocation This is Integration Compute Question 0011. \\ \fi
\begin{problem}

Use the Fundamental Theorem of Calculus to evaluate the integral.

\input{Integral-Compute-0011.HELP.tex}

\[
\int_{2}^{15} {-\frac{x + 5}{\sqrt{x}}}\;dx = \answer{-20 \, \sqrt{15} + \frac{34}{3} \, \sqrt{2}}
\]
\end{problem}}%}

\latexProblemContent{
\ifVerboseLocation This is Integration Compute Question 0011. \\ \fi
\begin{problem}

Use the Fundamental Theorem of Calculus to evaluate the integral.

\input{Integral-Compute-0011.HELP.tex}

\[
\int_{13}^{21} {\frac{x - 1}{5 \, \sqrt{x}}}\;dx = \answer{\frac{12}{5} \, \sqrt{21} - \frac{4}{3} \, \sqrt{13}}
\]
\end{problem}}%}

\latexProblemContent{
\ifVerboseLocation This is Integration Compute Question 0011. \\ \fi
\begin{problem}

Use the Fundamental Theorem of Calculus to evaluate the integral.

\input{Integral-Compute-0011.HELP.tex}

\[
\int_{3}^{5} {-\frac{x + 4}{\sqrt{x}}}\;dx = \answer{-\frac{34}{3} \, \sqrt{5} + 10 \, \sqrt{3}}
\]
\end{problem}}%}

\latexProblemContent{
\ifVerboseLocation This is Integration Compute Question 0011. \\ \fi
\begin{problem}

Use the Fundamental Theorem of Calculus to evaluate the integral.

\input{Integral-Compute-0011.HELP.tex}

\[
\int_{3}^{13} {\frac{x - 4}{8 \, \sqrt{x}}}\;dx = \answer{\frac{1}{12} \, \sqrt{13} + \frac{3}{4} \, \sqrt{3}}
\]
\end{problem}}%}

\latexProblemContent{
\ifVerboseLocation This is Integration Compute Question 0011. \\ \fi
\begin{problem}

Use the Fundamental Theorem of Calculus to evaluate the integral.

\input{Integral-Compute-0011.HELP.tex}

\[
\int_{8}^{16} {-\frac{x - 1}{4 \, \sqrt{x}}}\;dx = \answer{\frac{5}{3} \, \sqrt{2} - \frac{26}{3}}
\]
\end{problem}}%}

\latexProblemContent{
\ifVerboseLocation This is Integration Compute Question 0011. \\ \fi
\begin{problem}

Use the Fundamental Theorem of Calculus to evaluate the integral.

\input{Integral-Compute-0011.HELP.tex}

\[
\int_{2}^{20} {\frac{x + 4}{8 \, \sqrt{x}}}\;dx = \answer{\frac{16}{3} \, \sqrt{5} - \frac{7}{6} \, \sqrt{2}}
\]
\end{problem}}%}

\latexProblemContent{
\ifVerboseLocation This is Integration Compute Question 0011. \\ \fi
\begin{problem}

Use the Fundamental Theorem of Calculus to evaluate the integral.

\input{Integral-Compute-0011.HELP.tex}

\[
\int_{17}^{20} {-\frac{x - 1}{\sqrt{x}}}\;dx = \answer{\frac{28}{3} \, \sqrt{17} - \frac{68}{3} \, \sqrt{5}}
\]
\end{problem}}%}

\latexProblemContent{
\ifVerboseLocation This is Integration Compute Question 0011. \\ \fi
\begin{problem}

Use the Fundamental Theorem of Calculus to evaluate the integral.

\input{Integral-Compute-0011.HELP.tex}

\[
\int_{19}^{19} {-\frac{x - 2}{7 \, \sqrt{x}}}\;dx = \answer{0}
\]
\end{problem}}%}

\latexProblemContent{
\ifVerboseLocation This is Integration Compute Question 0011. \\ \fi
\begin{problem}

Use the Fundamental Theorem of Calculus to evaluate the integral.

\input{Integral-Compute-0011.HELP.tex}

\[
\int_{10}^{12} {\frac{x - 4}{3 \, \sqrt{x}}}\;dx = \answer{\frac{4}{9} \, \sqrt{10}}
\]
\end{problem}}%}

\latexProblemContent{
\ifVerboseLocation This is Integration Compute Question 0011. \\ \fi
\begin{problem}

Use the Fundamental Theorem of Calculus to evaluate the integral.

\input{Integral-Compute-0011.HELP.tex}

\[
\int_{5}^{21} {-\frac{x + 4}{\sqrt{x}}}\;dx = \answer{-22 \, \sqrt{21} + \frac{34}{3} \, \sqrt{5}}
\]
\end{problem}}%}

\latexProblemContent{
\ifVerboseLocation This is Integration Compute Question 0011. \\ \fi
\begin{problem}

Use the Fundamental Theorem of Calculus to evaluate the integral.

\input{Integral-Compute-0011.HELP.tex}

\[
\int_{19}^{19} {\frac{x + 3}{3 \, \sqrt{x}}}\;dx = \answer{0}
\]
\end{problem}}%}

\latexProblemContent{
\ifVerboseLocation This is Integration Compute Question 0011. \\ \fi
\begin{problem}

Use the Fundamental Theorem of Calculus to evaluate the integral.

\input{Integral-Compute-0011.HELP.tex}

\[
\int_{2}^{19} {\frac{x - 2}{3 \, \sqrt{x}}}\;dx = \answer{\frac{1}{9} \, \left(8 \, \sqrt{2}\right) + \frac{26}{9} \, \sqrt{19}}
\]
\end{problem}}%}

\latexProblemContent{
\ifVerboseLocation This is Integration Compute Question 0011. \\ \fi
\begin{problem}

Use the Fundamental Theorem of Calculus to evaluate the integral.

\input{Integral-Compute-0011.HELP.tex}

\[
\int_{13}^{20} {\frac{x - 3}{8 \, \sqrt{x}}}\;dx = \answer{-\frac{1}{3} \, \sqrt{13} + \frac{11}{6} \, \sqrt{5}}
\]
\end{problem}}%}

\latexProblemContent{
\ifVerboseLocation This is Integration Compute Question 0011. \\ \fi
\begin{problem}

Use the Fundamental Theorem of Calculus to evaluate the integral.

\input{Integral-Compute-0011.HELP.tex}

\[
\int_{4}^{9} {-\frac{x + 4}{8 \, \sqrt{x}}}\;dx = \answer{-\frac{31}{12}}
\]
\end{problem}}%}

\latexProblemContent{
\ifVerboseLocation This is Integration Compute Question 0011. \\ \fi
\begin{problem}

Use the Fundamental Theorem of Calculus to evaluate the integral.

\input{Integral-Compute-0011.HELP.tex}

\[
\int_{5}^{21} {-\frac{x - 3}{\sqrt{x}}}\;dx = \answer{-8 \, \sqrt{21} - \frac{8}{3} \, \sqrt{5}}
\]
\end{problem}}%}

\latexProblemContent{
\ifVerboseLocation This is Integration Compute Question 0011. \\ \fi
\begin{problem}

Use the Fundamental Theorem of Calculus to evaluate the integral.

\input{Integral-Compute-0011.HELP.tex}

\[
\int_{13}^{17} {-\frac{x + 2}{\sqrt{x}}}\;dx = \answer{-\frac{46}{3} \, \sqrt{17} + \frac{38}{3} \, \sqrt{13}}
\]
\end{problem}}%}

\latexProblemContent{
\ifVerboseLocation This is Integration Compute Question 0011. \\ \fi
\begin{problem}

Use the Fundamental Theorem of Calculus to evaluate the integral.

\input{Integral-Compute-0011.HELP.tex}

\[
\int_{18}^{20} {\frac{x + 2}{7 \, \sqrt{x}}}\;dx = \answer{\frac{104}{21} \, \sqrt{5} - \frac{48}{7} \, \sqrt{2}}
\]
\end{problem}}%}

\latexProblemContent{
\ifVerboseLocation This is Integration Compute Question 0011. \\ \fi
\begin{problem}

Use the Fundamental Theorem of Calculus to evaluate the integral.

\input{Integral-Compute-0011.HELP.tex}

\[
\int_{7}^{8} {\frac{x - 4}{3 \, \sqrt{x}}}\;dx = \answer{\frac{10}{9} \, \sqrt{7} - \frac{16}{9} \, \sqrt{2}}
\]
\end{problem}}%}

\latexProblemContent{
\ifVerboseLocation This is Integration Compute Question 0011. \\ \fi
\begin{problem}

Use the Fundamental Theorem of Calculus to evaluate the integral.

\input{Integral-Compute-0011.HELP.tex}

\[
\int_{17}^{17} {\frac{x + 3}{2 \, \sqrt{x}}}\;dx = \answer{0}
\]
\end{problem}}%}

\latexProblemContent{
\ifVerboseLocation This is Integration Compute Question 0011. \\ \fi
\begin{problem}

Use the Fundamental Theorem of Calculus to evaluate the integral.

\input{Integral-Compute-0011.HELP.tex}

\[
\int_{5}^{10} {\frac{x - 1}{3 \, \sqrt{x}}}\;dx = \answer{\frac{14}{9} \, \sqrt{10} - \frac{4}{9} \, \sqrt{5}}
\]
\end{problem}}%}

\latexProblemContent{
\ifVerboseLocation This is Integration Compute Question 0011. \\ \fi
\begin{problem}

Use the Fundamental Theorem of Calculus to evaluate the integral.

\input{Integral-Compute-0011.HELP.tex}

\[
\int_{13}^{17} {\frac{x - 3}{3 \, \sqrt{x}}}\;dx = \answer{\frac{16}{9} \, \sqrt{17} - \frac{8}{9} \, \sqrt{13}}
\]
\end{problem}}%}

\latexProblemContent{
\ifVerboseLocation This is Integration Compute Question 0011. \\ \fi
\begin{problem}

Use the Fundamental Theorem of Calculus to evaluate the integral.

\input{Integral-Compute-0011.HELP.tex}

\[
\int_{3}^{8} {-\frac{x - 2}{3 \, \sqrt{x}}}\;dx = \answer{-\frac{1}{9} \, \left(8 \, \sqrt{2}\right) - \frac{2}{3} \, \sqrt{3}}
\]
\end{problem}}%}

\latexProblemContent{
\ifVerboseLocation This is Integration Compute Question 0011. \\ \fi
\begin{problem}

Use the Fundamental Theorem of Calculus to evaluate the integral.

\input{Integral-Compute-0011.HELP.tex}

\[
\int_{10}^{16} {-\frac{x - 3}{5 \, \sqrt{x}}}\;dx = \answer{\frac{2}{15} \, \sqrt{10} - \frac{56}{15}}
\]
\end{problem}}%}

\latexProblemContent{
\ifVerboseLocation This is Integration Compute Question 0011. \\ \fi
\begin{problem}

Use the Fundamental Theorem of Calculus to evaluate the integral.

\input{Integral-Compute-0011.HELP.tex}

\[
\int_{20}^{20} {\frac{x + 1}{8 \, \sqrt{x}}}\;dx = \answer{0}
\]
\end{problem}}%}

\latexProblemContent{
\ifVerboseLocation This is Integration Compute Question 0011. \\ \fi
\begin{problem}

Use the Fundamental Theorem of Calculus to evaluate the integral.

\input{Integral-Compute-0011.HELP.tex}

\[
\int_{5}^{13} {-\frac{x - 1}{3 \, \sqrt{x}}}\;dx = \answer{-\frac{20}{9} \, \sqrt{13} + \frac{4}{9} \, \sqrt{5}}
\]
\end{problem}}%}

\latexProblemContent{
\ifVerboseLocation This is Integration Compute Question 0011. \\ \fi
\begin{problem}

Use the Fundamental Theorem of Calculus to evaluate the integral.

\input{Integral-Compute-0011.HELP.tex}

\[
\int_{20}^{20} {\frac{x - 3}{2 \, \sqrt{x}}}\;dx = \answer{0}
\]
\end{problem}}%}

\latexProblemContent{
\ifVerboseLocation This is Integration Compute Question 0011. \\ \fi
\begin{problem}

Use the Fundamental Theorem of Calculus to evaluate the integral.

\input{Integral-Compute-0011.HELP.tex}

\[
\int_{19}^{19} {\frac{x + 5}{\sqrt{x}}}\;dx = \answer{0}
\]
\end{problem}}%}

\latexProblemContent{
\ifVerboseLocation This is Integration Compute Question 0011. \\ \fi
\begin{problem}

Use the Fundamental Theorem of Calculus to evaluate the integral.

\input{Integral-Compute-0011.HELP.tex}

\[
\int_{20}^{21} {-\frac{x - 3}{8 \, \sqrt{x}}}\;dx = \answer{-\sqrt{21} + \frac{11}{6} \, \sqrt{5}}
\]
\end{problem}}%}

\latexProblemContent{
\ifVerboseLocation This is Integration Compute Question 0011. \\ \fi
\begin{problem}

Use the Fundamental Theorem of Calculus to evaluate the integral.

\input{Integral-Compute-0011.HELP.tex}

\[
\int_{9}^{17} {\frac{x + 4}{5 \, \sqrt{x}}}\;dx = \answer{\frac{58}{15} \, \sqrt{17} - \frac{42}{5}}
\]
\end{problem}}%}

\latexProblemContent{
\ifVerboseLocation This is Integration Compute Question 0011. \\ \fi
\begin{problem}

Use the Fundamental Theorem of Calculus to evaluate the integral.

\input{Integral-Compute-0011.HELP.tex}

\[
\int_{15}^{17} {-\frac{x + 2}{8 \, \sqrt{x}}}\;dx = \answer{-\frac{23}{12} \, \sqrt{17} + \frac{7}{4} \, \sqrt{15}}
\]
\end{problem}}%}

\latexProblemContent{
\ifVerboseLocation This is Integration Compute Question 0011. \\ \fi
\begin{problem}

Use the Fundamental Theorem of Calculus to evaluate the integral.

\input{Integral-Compute-0011.HELP.tex}

\[
\int_{19}^{20} {\frac{x - 2}{\sqrt{x}}}\;dx = \answer{-\frac{26}{3} \, \sqrt{19} + \frac{56}{3} \, \sqrt{5}}
\]
\end{problem}}%}

\latexProblemContent{
\ifVerboseLocation This is Integration Compute Question 0011. \\ \fi
\begin{problem}

Use the Fundamental Theorem of Calculus to evaluate the integral.

\input{Integral-Compute-0011.HELP.tex}

\[
\int_{14}^{21} {-\frac{x + 1}{7 \, \sqrt{x}}}\;dx = \answer{-\frac{16}{7} \, \sqrt{21} + \frac{34}{21} \, \sqrt{14}}
\]
\end{problem}}%}

\latexProblemContent{
\ifVerboseLocation This is Integration Compute Question 0011. \\ \fi
\begin{problem}

Use the Fundamental Theorem of Calculus to evaluate the integral.

\input{Integral-Compute-0011.HELP.tex}

\[
\int_{13}^{16} {-\frac{x + 2}{\sqrt{x}}}\;dx = \answer{\frac{38}{3} \, \sqrt{13} - \frac{176}{3}}
\]
\end{problem}}%}

\latexProblemContent{
\ifVerboseLocation This is Integration Compute Question 0011. \\ \fi
\begin{problem}

Use the Fundamental Theorem of Calculus to evaluate the integral.

\input{Integral-Compute-0011.HELP.tex}

\[
\int_{1}^{12} {-\frac{x - 3}{7 \, \sqrt{x}}}\;dx = \answer{-\frac{4}{7} \, \sqrt{3} - \frac{16}{21}}
\]
\end{problem}}%}

\latexProblemContent{
\ifVerboseLocation This is Integration Compute Question 0011. \\ \fi
\begin{problem}

Use the Fundamental Theorem of Calculus to evaluate the integral.

\input{Integral-Compute-0011.HELP.tex}

\[
\int_{18}^{18} {\frac{x + 3}{5 \, \sqrt{x}}}\;dx = \answer{0}
\]
\end{problem}}%}

\latexProblemContent{
\ifVerboseLocation This is Integration Compute Question 0011. \\ \fi
\begin{problem}

Use the Fundamental Theorem of Calculus to evaluate the integral.

\input{Integral-Compute-0011.HELP.tex}

\[
\int_{3}^{3} {\frac{x + 3}{8 \, \sqrt{x}}}\;dx = \answer{0}
\]
\end{problem}}%}

\latexProblemContent{
\ifVerboseLocation This is Integration Compute Question 0011. \\ \fi
\begin{problem}

Use the Fundamental Theorem of Calculus to evaluate the integral.

\input{Integral-Compute-0011.HELP.tex}

\[
\int_{7}^{7} {-\frac{x + 3}{6 \, \sqrt{x}}}\;dx = \answer{0}
\]
\end{problem}}%}

\latexProblemContent{
\ifVerboseLocation This is Integration Compute Question 0011. \\ \fi
\begin{problem}

Use the Fundamental Theorem of Calculus to evaluate the integral.

\input{Integral-Compute-0011.HELP.tex}

\[
\int_{13}^{20} {\frac{x + 3}{2 \, \sqrt{x}}}\;dx = \answer{-\frac{22}{3} \, \sqrt{13} + \frac{58}{3} \, \sqrt{5}}
\]
\end{problem}}%}

\latexProblemContent{
\ifVerboseLocation This is Integration Compute Question 0011. \\ \fi
\begin{problem}

Use the Fundamental Theorem of Calculus to evaluate the integral.

\input{Integral-Compute-0011.HELP.tex}

\[
\int_{2}^{19} {\frac{x + 4}{4 \, \sqrt{x}}}\;dx = \answer{\frac{31}{6} \, \sqrt{19} - \frac{7}{3} \, \sqrt{2}}
\]
\end{problem}}%}

\latexProblemContent{
\ifVerboseLocation This is Integration Compute Question 0011. \\ \fi
\begin{problem}

Use the Fundamental Theorem of Calculus to evaluate the integral.

\input{Integral-Compute-0011.HELP.tex}

\[
\int_{3}^{19} {\frac{x - 1}{8 \, \sqrt{x}}}\;dx = \answer{\frac{4}{3} \, \sqrt{19}}
\]
\end{problem}}%}

\latexProblemContent{
\ifVerboseLocation This is Integration Compute Question 0011. \\ \fi
\begin{problem}

Use the Fundamental Theorem of Calculus to evaluate the integral.

\input{Integral-Compute-0011.HELP.tex}

\[
\int_{14}^{14} {\frac{x + 5}{2 \, \sqrt{x}}}\;dx = \answer{0}
\]
\end{problem}}%}

\latexProblemContent{
\ifVerboseLocation This is Integration Compute Question 0011. \\ \fi
\begin{problem}

Use the Fundamental Theorem of Calculus to evaluate the integral.

\input{Integral-Compute-0011.HELP.tex}

\[
\int_{2}^{5} {-\frac{x + 1}{4 \, \sqrt{x}}}\;dx = \answer{-\frac{4}{3} \, \sqrt{5} + \frac{5}{6} \, \sqrt{2}}
\]
\end{problem}}%}

\latexProblemContent{
\ifVerboseLocation This is Integration Compute Question 0011. \\ \fi
\begin{problem}

Use the Fundamental Theorem of Calculus to evaluate the integral.

\input{Integral-Compute-0011.HELP.tex}

\[
\int_{14}^{20} {\frac{x + 1}{\sqrt{x}}}\;dx = \answer{-\frac{34}{3} \, \sqrt{14} + \frac{92}{3} \, \sqrt{5}}
\]
\end{problem}}%}

\latexProblemContent{
\ifVerboseLocation This is Integration Compute Question 0011. \\ \fi
\begin{problem}

Use the Fundamental Theorem of Calculus to evaluate the integral.

\input{Integral-Compute-0011.HELP.tex}

\[
\int_{1}^{6} {\frac{x - 1}{\sqrt{x}}}\;dx = \answer{\frac{1}{3} \, \left(6 \, \sqrt{6}\right) + \frac{4}{3}}
\]
\end{problem}}%}

\latexProblemContent{
\ifVerboseLocation This is Integration Compute Question 0011. \\ \fi
\begin{problem}

Use the Fundamental Theorem of Calculus to evaluate the integral.

\input{Integral-Compute-0011.HELP.tex}

\[
\int_{17}^{19} {\frac{x - 4}{5 \, \sqrt{x}}}\;dx = \answer{\frac{14}{15} \, \sqrt{19} - \frac{2}{3} \, \sqrt{17}}
\]
\end{problem}}%}

\latexProblemContent{
\ifVerboseLocation This is Integration Compute Question 0011. \\ \fi
\begin{problem}

Use the Fundamental Theorem of Calculus to evaluate the integral.

\input{Integral-Compute-0011.HELP.tex}

\[
\int_{3}^{4} {\frac{x + 2}{6 \, \sqrt{x}}}\;dx = \answer{-\sqrt{3} + \frac{20}{9}}
\]
\end{problem}}%}

\latexProblemContent{
\ifVerboseLocation This is Integration Compute Question 0011. \\ \fi
\begin{problem}

Use the Fundamental Theorem of Calculus to evaluate the integral.

\input{Integral-Compute-0011.HELP.tex}

\[
\int_{19}^{19} {\frac{x + 5}{7 \, \sqrt{x}}}\;dx = \answer{0}
\]
\end{problem}}%}

\latexProblemContent{
\ifVerboseLocation This is Integration Compute Question 0011. \\ \fi
\begin{problem}

Use the Fundamental Theorem of Calculus to evaluate the integral.

\input{Integral-Compute-0011.HELP.tex}

\[
\int_{12}^{17} {\frac{x - 4}{4 \, \sqrt{x}}}\;dx = \answer{\frac{5}{6} \, \sqrt{17}}
\]
\end{problem}}%}

\latexProblemContent{
\ifVerboseLocation This is Integration Compute Question 0011. \\ \fi
\begin{problem}

Use the Fundamental Theorem of Calculus to evaluate the integral.

\input{Integral-Compute-0011.HELP.tex}

\[
\int_{17}^{21} {\frac{x - 3}{2 \, \sqrt{x}}}\;dx = \answer{4 \, \sqrt{21} - \frac{8}{3} \, \sqrt{17}}
\]
\end{problem}}%}

\latexProblemContent{
\ifVerboseLocation This is Integration Compute Question 0011. \\ \fi
\begin{problem}

Use the Fundamental Theorem of Calculus to evaluate the integral.

\input{Integral-Compute-0011.HELP.tex}

\[
\int_{12}^{12} {\frac{x - 2}{6 \, \sqrt{x}}}\;dx = \answer{0}
\]
\end{problem}}%}

\latexProblemContent{
\ifVerboseLocation This is Integration Compute Question 0011. \\ \fi
\begin{problem}

Use the Fundamental Theorem of Calculus to evaluate the integral.

\input{Integral-Compute-0011.HELP.tex}

\[
\int_{10}^{19} {\frac{x - 2}{8 \, \sqrt{x}}}\;dx = \answer{\frac{13}{12} \, \sqrt{19} - \frac{1}{3} \, \sqrt{10}}
\]
\end{problem}}%}

\latexProblemContent{
\ifVerboseLocation This is Integration Compute Question 0011. \\ \fi
\begin{problem}

Use the Fundamental Theorem of Calculus to evaluate the integral.

\input{Integral-Compute-0011.HELP.tex}

\[
\int_{4}^{19} {-\frac{x - 2}{3 \, \sqrt{x}}}\;dx = \answer{-\frac{26}{9} \, \sqrt{19} - \frac{8}{9}}
\]
\end{problem}}%}

\latexProblemContent{
\ifVerboseLocation This is Integration Compute Question 0011. \\ \fi
\begin{problem}

Use the Fundamental Theorem of Calculus to evaluate the integral.

\input{Integral-Compute-0011.HELP.tex}

\[
\int_{14}^{20} {-\frac{x + 4}{7 \, \sqrt{x}}}\;dx = \answer{\frac{52}{21} \, \sqrt{14} - \frac{128}{21} \, \sqrt{5}}
\]
\end{problem}}%}

\latexProblemContent{
\ifVerboseLocation This is Integration Compute Question 0011. \\ \fi
\begin{problem}

Use the Fundamental Theorem of Calculus to evaluate the integral.

\input{Integral-Compute-0011.HELP.tex}

\[
\int_{6}^{15} {-\frac{x + 3}{3 \, \sqrt{x}}}\;dx = \answer{-\frac{16}{3} \, \sqrt{15} + \frac{10}{3} \, \sqrt{6}}
\]
\end{problem}}%}

\latexProblemContent{
\ifVerboseLocation This is Integration Compute Question 0011. \\ \fi
\begin{problem}

Use the Fundamental Theorem of Calculus to evaluate the integral.

\input{Integral-Compute-0011.HELP.tex}

\[
\int_{8}^{14} {\frac{x - 3}{4 \, \sqrt{x}}}\;dx = \answer{\frac{1}{12} \, \left(4 \, \sqrt{2}\right) + \frac{5}{6} \, \sqrt{14}}
\]
\end{problem}}%}

\latexProblemContent{
\ifVerboseLocation This is Integration Compute Question 0011. \\ \fi
\begin{problem}

Use the Fundamental Theorem of Calculus to evaluate the integral.

\input{Integral-Compute-0011.HELP.tex}

\[
\int_{13}^{13} {-\frac{x + 4}{2 \, \sqrt{x}}}\;dx = \answer{0}
\]
\end{problem}}%}

\latexProblemContent{
\ifVerboseLocation This is Integration Compute Question 0011. \\ \fi
\begin{problem}

Use the Fundamental Theorem of Calculus to evaluate the integral.

\input{Integral-Compute-0011.HELP.tex}

\[
\int_{4}^{19} {\frac{x + 5}{8 \, \sqrt{x}}}\;dx = \answer{\frac{17}{6} \, \sqrt{19} - \frac{19}{6}}
\]
\end{problem}}%}

\latexProblemContent{
\ifVerboseLocation This is Integration Compute Question 0011. \\ \fi
\begin{problem}

Use the Fundamental Theorem of Calculus to evaluate the integral.

\input{Integral-Compute-0011.HELP.tex}

\[
\int_{10}^{12} {-\frac{x + 4}{5 \, \sqrt{x}}}\;dx = \answer{\frac{44}{15} \, \sqrt{10} - \frac{32}{5} \, \sqrt{3}}
\]
\end{problem}}%}

\latexProblemContent{
\ifVerboseLocation This is Integration Compute Question 0011. \\ \fi
\begin{problem}

Use the Fundamental Theorem of Calculus to evaluate the integral.

\input{Integral-Compute-0011.HELP.tex}

\[
\int_{20}^{20} {\frac{x - 2}{4 \, \sqrt{x}}}\;dx = \answer{0}
\]
\end{problem}}%}

\latexProblemContent{
\ifVerboseLocation This is Integration Compute Question 0011. \\ \fi
\begin{problem}

Use the Fundamental Theorem of Calculus to evaluate the integral.

\input{Integral-Compute-0011.HELP.tex}

\[
\int_{2}^{8} {-\frac{x + 3}{7 \, \sqrt{x}}}\;dx = \answer{-\frac{46}{21} \, \sqrt{2}}
\]
\end{problem}}%}

\latexProblemContent{
\ifVerboseLocation This is Integration Compute Question 0011. \\ \fi
\begin{problem}

Use the Fundamental Theorem of Calculus to evaluate the integral.

\input{Integral-Compute-0011.HELP.tex}

\[
\int_{15}^{18} {-\frac{x + 1}{4 \, \sqrt{x}}}\;dx = \answer{3 \, \sqrt{15} - \frac{21}{2} \, \sqrt{2}}
\]
\end{problem}}%}

\latexProblemContent{
\ifVerboseLocation This is Integration Compute Question 0011. \\ \fi
\begin{problem}

Use the Fundamental Theorem of Calculus to evaluate the integral.

\input{Integral-Compute-0011.HELP.tex}

\[
\int_{9}^{15} {\frac{x - 3}{7 \, \sqrt{x}}}\;dx = \answer{\frac{4}{7} \, \sqrt{15}}
\]
\end{problem}}%}

\latexProblemContent{
\ifVerboseLocation This is Integration Compute Question 0011. \\ \fi
\begin{problem}

Use the Fundamental Theorem of Calculus to evaluate the integral.

\input{Integral-Compute-0011.HELP.tex}

\[
\int_{14}^{14} {-\frac{x + 1}{4 \, \sqrt{x}}}\;dx = \answer{0}
\]
\end{problem}}%}

\latexProblemContent{
\ifVerboseLocation This is Integration Compute Question 0011. \\ \fi
\begin{problem}

Use the Fundamental Theorem of Calculus to evaluate the integral.

\input{Integral-Compute-0011.HELP.tex}

\[
\int_{17}^{18} {\frac{x - 4}{\sqrt{x}}}\;dx = \answer{-\frac{10}{3} \, \sqrt{17} + 12 \, \sqrt{2}}
\]
\end{problem}}%}

\latexProblemContent{
\ifVerboseLocation This is Integration Compute Question 0011. \\ \fi
\begin{problem}

Use the Fundamental Theorem of Calculus to evaluate the integral.

\input{Integral-Compute-0011.HELP.tex}

\[
\int_{8}^{10} {\frac{x + 1}{6 \, \sqrt{x}}}\;dx = \answer{\frac{13}{9} \, \sqrt{10} - \frac{22}{9} \, \sqrt{2}}
\]
\end{problem}}%}

\latexProblemContent{
\ifVerboseLocation This is Integration Compute Question 0011. \\ \fi
\begin{problem}

Use the Fundamental Theorem of Calculus to evaluate the integral.

\input{Integral-Compute-0011.HELP.tex}

\[
\int_{16}^{20} {\frac{x + 4}{2 \, \sqrt{x}}}\;dx = \answer{\frac{64}{3} \, \sqrt{5} - \frac{112}{3}}
\]
\end{problem}}%}

\latexProblemContent{
\ifVerboseLocation This is Integration Compute Question 0011. \\ \fi
\begin{problem}

Use the Fundamental Theorem of Calculus to evaluate the integral.

\input{Integral-Compute-0011.HELP.tex}

\[
\int_{1}^{10} {-\frac{x + 1}{3 \, \sqrt{x}}}\;dx = \answer{-\frac{26}{9} \, \sqrt{10} + \frac{8}{9}}
\]
\end{problem}}%}

\latexProblemContent{
\ifVerboseLocation This is Integration Compute Question 0011. \\ \fi
\begin{problem}

Use the Fundamental Theorem of Calculus to evaluate the integral.

\input{Integral-Compute-0011.HELP.tex}

\[
\int_{4}^{6} {\frac{x + 3}{6 \, \sqrt{x}}}\;dx = \answer{\frac{5}{3} \, \sqrt{6} - \frac{26}{9}}
\]
\end{problem}}%}

\latexProblemContent{
\ifVerboseLocation This is Integration Compute Question 0011. \\ \fi
\begin{problem}

Use the Fundamental Theorem of Calculus to evaluate the integral.

\input{Integral-Compute-0011.HELP.tex}

\[
\int_{20}^{20} {-\frac{x + 5}{2 \, \sqrt{x}}}\;dx = \answer{0}
\]
\end{problem}}%}

\latexProblemContent{
\ifVerboseLocation This is Integration Compute Question 0011. \\ \fi
\begin{problem}

Use the Fundamental Theorem of Calculus to evaluate the integral.

\input{Integral-Compute-0011.HELP.tex}

\[
\int_{2}^{8} {\frac{x - 5}{3 \, \sqrt{x}}}\;dx = \answer{-\frac{2}{9} \, \sqrt{2}}
\]
\end{problem}}%}

\latexProblemContent{
\ifVerboseLocation This is Integration Compute Question 0011. \\ \fi
\begin{problem}

Use the Fundamental Theorem of Calculus to evaluate the integral.

\input{Integral-Compute-0011.HELP.tex}

\[
\int_{14}^{21} {\frac{x + 4}{7 \, \sqrt{x}}}\;dx = \answer{\frac{22}{7} \, \sqrt{21} - \frac{52}{21} \, \sqrt{14}}
\]
\end{problem}}%}

\latexProblemContent{
\ifVerboseLocation This is Integration Compute Question 0011. \\ \fi
\begin{problem}

Use the Fundamental Theorem of Calculus to evaluate the integral.

\input{Integral-Compute-0011.HELP.tex}

\[
\int_{18}^{18} {-\frac{x - 5}{\sqrt{x}}}\;dx = \answer{0}
\]
\end{problem}}%}

\latexProblemContent{
\ifVerboseLocation This is Integration Compute Question 0011. \\ \fi
\begin{problem}

Use the Fundamental Theorem of Calculus to evaluate the integral.

\input{Integral-Compute-0011.HELP.tex}

\[
\int_{17}^{21} {\frac{x + 1}{5 \, \sqrt{x}}}\;dx = \answer{\frac{16}{5} \, \sqrt{21} - \frac{8}{3} \, \sqrt{17}}
\]
\end{problem}}%}

\latexProblemContent{
\ifVerboseLocation This is Integration Compute Question 0011. \\ \fi
\begin{problem}

Use the Fundamental Theorem of Calculus to evaluate the integral.

\input{Integral-Compute-0011.HELP.tex}

\[
\int_{14}^{15} {\frac{x + 1}{2 \, \sqrt{x}}}\;dx = \answer{6 \, \sqrt{15} - \frac{17}{3} \, \sqrt{14}}
\]
\end{problem}}%}

\latexProblemContent{
\ifVerboseLocation This is Integration Compute Question 0011. \\ \fi
\begin{problem}

Use the Fundamental Theorem of Calculus to evaluate the integral.

\input{Integral-Compute-0011.HELP.tex}

\[
\int_{7}^{8} {-\frac{x - 2}{5 \, \sqrt{x}}}\;dx = \answer{\frac{2}{15} \, \sqrt{7} - \frac{8}{15} \, \sqrt{2}}
\]
\end{problem}}%}

\latexProblemContent{
\ifVerboseLocation This is Integration Compute Question 0011. \\ \fi
\begin{problem}

Use the Fundamental Theorem of Calculus to evaluate the integral.

\input{Integral-Compute-0011.HELP.tex}

\[
\int_{20}^{20} {\frac{x + 2}{6 \, \sqrt{x}}}\;dx = \answer{0}
\]
\end{problem}}%}

\latexProblemContent{
\ifVerboseLocation This is Integration Compute Question 0011. \\ \fi
\begin{problem}

Use the Fundamental Theorem of Calculus to evaluate the integral.

\input{Integral-Compute-0011.HELP.tex}

\[
\int_{14}^{21} {\frac{x + 2}{\sqrt{x}}}\;dx = \answer{18 \, \sqrt{21} - \frac{40}{3} \, \sqrt{14}}
\]
\end{problem}}%}

\latexProblemContent{
\ifVerboseLocation This is Integration Compute Question 0011. \\ \fi
\begin{problem}

Use the Fundamental Theorem of Calculus to evaluate the integral.

\input{Integral-Compute-0011.HELP.tex}

\[
\int_{13}^{17} {\frac{x + 2}{8 \, \sqrt{x}}}\;dx = \answer{\frac{23}{12} \, \sqrt{17} - \frac{19}{12} \, \sqrt{13}}
\]
\end{problem}}%}

\latexProblemContent{
\ifVerboseLocation This is Integration Compute Question 0011. \\ \fi
\begin{problem}

Use the Fundamental Theorem of Calculus to evaluate the integral.

\input{Integral-Compute-0011.HELP.tex}

\[
\int_{2}^{17} {\frac{x - 1}{7 \, \sqrt{x}}}\;dx = \answer{\frac{1}{21} \, \left(2 \, \sqrt{2}\right) + \frac{4}{3} \, \sqrt{17}}
\]
\end{problem}}%}

\latexProblemContent{
\ifVerboseLocation This is Integration Compute Question 0011. \\ \fi
\begin{problem}

Use the Fundamental Theorem of Calculus to evaluate the integral.

\input{Integral-Compute-0011.HELP.tex}

\[
\int_{19}^{20} {\frac{x - 4}{3 \, \sqrt{x}}}\;dx = \answer{-\frac{14}{9} \, \sqrt{19} + \frac{32}{9} \, \sqrt{5}}
\]
\end{problem}}%}

\latexProblemContent{
\ifVerboseLocation This is Integration Compute Question 0011. \\ \fi
\begin{problem}

Use the Fundamental Theorem of Calculus to evaluate the integral.

\input{Integral-Compute-0011.HELP.tex}

\[
\int_{4}^{15} {\frac{x - 3}{\sqrt{x}}}\;dx = \answer{4 \, \sqrt{15} + \frac{20}{3}}
\]
\end{problem}}%}

\latexProblemContent{
\ifVerboseLocation This is Integration Compute Question 0011. \\ \fi
\begin{problem}

Use the Fundamental Theorem of Calculus to evaluate the integral.

\input{Integral-Compute-0011.HELP.tex}

\[
\int_{1}^{5} {-\frac{x + 4}{2 \, \sqrt{x}}}\;dx = \answer{-\frac{17}{3} \, \sqrt{5} + \frac{13}{3}}
\]
\end{problem}}%}

\latexProblemContent{
\ifVerboseLocation This is Integration Compute Question 0011. \\ \fi
\begin{problem}

Use the Fundamental Theorem of Calculus to evaluate the integral.

\input{Integral-Compute-0011.HELP.tex}

\[
\int_{8}^{9} {\frac{x - 1}{5 \, \sqrt{x}}}\;dx = \answer{-\frac{4}{3} \, \sqrt{2} + \frac{12}{5}}
\]
\end{problem}}%}

\latexProblemContent{
\ifVerboseLocation This is Integration Compute Question 0011. \\ \fi
\begin{problem}

Use the Fundamental Theorem of Calculus to evaluate the integral.

\input{Integral-Compute-0011.HELP.tex}

\[
\int_{15}^{19} {-\frac{x + 3}{6 \, \sqrt{x}}}\;dx = \answer{-\frac{28}{9} \, \sqrt{19} + \frac{8}{3} \, \sqrt{15}}
\]
\end{problem}}%}

\latexProblemContent{
\ifVerboseLocation This is Integration Compute Question 0011. \\ \fi
\begin{problem}

Use the Fundamental Theorem of Calculus to evaluate the integral.

\input{Integral-Compute-0011.HELP.tex}

\[
\int_{16}^{18} {-\frac{x - 2}{\sqrt{x}}}\;dx = \answer{-24 \, \sqrt{2} + \frac{80}{3}}
\]
\end{problem}}%}

\latexProblemContent{
\ifVerboseLocation This is Integration Compute Question 0011. \\ \fi
\begin{problem}

Use the Fundamental Theorem of Calculus to evaluate the integral.

\input{Integral-Compute-0011.HELP.tex}

\[
\int_{6}^{7} {\frac{x + 4}{3 \, \sqrt{x}}}\;dx = \answer{\frac{38}{9} \, \sqrt{7} - 4 \, \sqrt{6}}
\]
\end{problem}}%}

\latexProblemContent{
\ifVerboseLocation This is Integration Compute Question 0011. \\ \fi
\begin{problem}

Use the Fundamental Theorem of Calculus to evaluate the integral.

\input{Integral-Compute-0011.HELP.tex}

\[
\int_{9}^{17} {\frac{x - 2}{5 \, \sqrt{x}}}\;dx = \answer{\frac{22}{15} \, \sqrt{17} - \frac{6}{5}}
\]
\end{problem}}%}

\latexProblemContent{
\ifVerboseLocation This is Integration Compute Question 0011. \\ \fi
\begin{problem}

Use the Fundamental Theorem of Calculus to evaluate the integral.

\input{Integral-Compute-0011.HELP.tex}

\[
\int_{14}^{20} {\frac{x - 4}{4 \, \sqrt{x}}}\;dx = \answer{-\frac{1}{3} \, \sqrt{14} + \frac{8}{3} \, \sqrt{5}}
\]
\end{problem}}%}

\latexProblemContent{
\ifVerboseLocation This is Integration Compute Question 0011. \\ \fi
\begin{problem}

Use the Fundamental Theorem of Calculus to evaluate the integral.

\input{Integral-Compute-0011.HELP.tex}

\[
\int_{8}^{10} {\frac{x - 4}{4 \, \sqrt{x}}}\;dx = \answer{-\frac{1}{3} \, \sqrt{10} + \frac{4}{3} \, \sqrt{2}}
\]
\end{problem}}%}

\latexProblemContent{
\ifVerboseLocation This is Integration Compute Question 0011. \\ \fi
\begin{problem}

Use the Fundamental Theorem of Calculus to evaluate the integral.

\input{Integral-Compute-0011.HELP.tex}

\[
\int_{10}^{16} {\frac{x + 5}{7 \, \sqrt{x}}}\;dx = \answer{-\frac{50}{21} \, \sqrt{10} + \frac{248}{21}}
\]
\end{problem}}%}

\latexProblemContent{
\ifVerboseLocation This is Integration Compute Question 0011. \\ \fi
\begin{problem}

Use the Fundamental Theorem of Calculus to evaluate the integral.

\input{Integral-Compute-0011.HELP.tex}

\[
\int_{13}^{18} {-\frac{x - 4}{6 \, \sqrt{x}}}\;dx = \answer{\frac{1}{9} \, \sqrt{13} - 2 \, \sqrt{2}}
\]
\end{problem}}%}

\latexProblemContent{
\ifVerboseLocation This is Integration Compute Question 0011. \\ \fi
\begin{problem}

Use the Fundamental Theorem of Calculus to evaluate the integral.

\input{Integral-Compute-0011.HELP.tex}

\[
\int_{20}^{20} {\frac{x + 5}{4 \, \sqrt{x}}}\;dx = \answer{0}
\]
\end{problem}}%}

