\ProblemFileHeader{XTL_SV_QUESTIONCOUNT}% Process how many problems are in this file and how to detect if it has a desirable problem
\ifproblemToFind% If it has a desirable problem search the file.
%\tagged{Ans@ShortAns, Type@Compute, Topic@Derivative, Sub@Trig, Sub@LHopital, File@0050}{
\latexProblemContent{
\ifVerboseLocation This is Derivative Compute Question 0050. \\ \fi
\begin{problem}

Find the limit.  Use L'H$\hat{o}$pital's rule where appropriate.

\input{Derivative-Compute-0050.HELP.tex}

\[\lim\limits_{x\to\infty} {{\left(-\frac{6}{x} + 1\right)}^{4 \, x} + 7}=\answer{e^{\left(-24\right)} + 7}\]
\end{problem}}%}

\latexProblemContent{
\ifVerboseLocation This is Derivative Compute Question 0050. \\ \fi
\begin{problem}

Find the limit.  Use L'H$\hat{o}$pital's rule where appropriate.

\input{Derivative-Compute-0050.HELP.tex}

\[\lim\limits_{x\to\infty} {{\left(-\frac{2}{x} + 1\right)}^{8 \, x} + 5}=\answer{e^{\left(-16\right)} + 5}\]
\end{problem}}%}

\latexProblemContent{
\ifVerboseLocation This is Derivative Compute Question 0050. \\ \fi
\begin{problem}

Find the limit.  Use L'H$\hat{o}$pital's rule where appropriate.

\input{Derivative-Compute-0050.HELP.tex}

\[\lim\limits_{x\to\infty} {{\left(-\frac{8}{x} + 1\right)}^{-9 \, x} + 19}=\answer{e^{72} + 19}\]
\end{problem}}%}

\latexProblemContent{
\ifVerboseLocation This is Derivative Compute Question 0050. \\ \fi
\begin{problem}

Find the limit.  Use L'H$\hat{o}$pital's rule where appropriate.

\input{Derivative-Compute-0050.HELP.tex}

\[\lim\limits_{x\to\infty} {{\left(-\frac{3}{x} + 1\right)}^{7 \, x} - 13}=\answer{e^{\left(-21\right)} - 13}\]
\end{problem}}%}

\latexProblemContent{
\ifVerboseLocation This is Derivative Compute Question 0050. \\ \fi
\begin{problem}

Find the limit.  Use L'H$\hat{o}$pital's rule where appropriate.

\input{Derivative-Compute-0050.HELP.tex}

\[\lim\limits_{x\to\infty} {{\left(-\frac{1}{x} + 1\right)}^{6 \, x} - 4}=\answer{e^{\left(-6\right)} - 4}\]
\end{problem}}%}

\latexProblemContent{
\ifVerboseLocation This is Derivative Compute Question 0050. \\ \fi
\begin{problem}

Find the limit.  Use L'H$\hat{o}$pital's rule where appropriate.

\input{Derivative-Compute-0050.HELP.tex}

\[\lim\limits_{x\to\infty} {{\left(-\frac{1}{x} + 1\right)}^{8 \, x} + 16}=\answer{e^{\left(-8\right)} + 16}\]
\end{problem}}%}

\latexProblemContent{
\ifVerboseLocation This is Derivative Compute Question 0050. \\ \fi
\begin{problem}

Find the limit.  Use L'H$\hat{o}$pital's rule where appropriate.

\input{Derivative-Compute-0050.HELP.tex}

\[\lim\limits_{x\to\infty} {{\left(\frac{6}{x} + 1\right)}^{10 \, x} - 3}=\answer{e^{60} - 3}\]
\end{problem}}%}

\latexProblemContent{
\ifVerboseLocation This is Derivative Compute Question 0050. \\ \fi
\begin{problem}

Find the limit.  Use L'H$\hat{o}$pital's rule where appropriate.

\input{Derivative-Compute-0050.HELP.tex}

\[\lim\limits_{x\to\infty} {{\left(-\frac{4}{x} + 1\right)}^{-6 \, x} - 12}=\answer{e^{24} - 12}\]
\end{problem}}%}

\latexProblemContent{
\ifVerboseLocation This is Derivative Compute Question 0050. \\ \fi
\begin{problem}

Find the limit.  Use L'H$\hat{o}$pital's rule where appropriate.

\input{Derivative-Compute-0050.HELP.tex}

\[\lim\limits_{x\to\infty} {{\left(\frac{4}{x} + 1\right)}^{5 \, x} + 2}=\answer{e^{20} + 2}\]
\end{problem}}%}

\latexProblemContent{
\ifVerboseLocation This is Derivative Compute Question 0050. \\ \fi
\begin{problem}

Find the limit.  Use L'H$\hat{o}$pital's rule where appropriate.

\input{Derivative-Compute-0050.HELP.tex}

\[\lim\limits_{x\to\infty} {{\left(-\frac{8}{x} + 1\right)}^{-3 \, x} - 16}=\answer{e^{24} - 16}\]
\end{problem}}%}

\latexProblemContent{
\ifVerboseLocation This is Derivative Compute Question 0050. \\ \fi
\begin{problem}

Find the limit.  Use L'H$\hat{o}$pital's rule where appropriate.

\input{Derivative-Compute-0050.HELP.tex}

\[\lim\limits_{x\to\infty} {{\left(-\frac{10}{x} + 1\right)}^{-7 \, x} - 9}=\answer{e^{70} - 9}\]
\end{problem}}%}

\latexProblemContent{
\ifVerboseLocation This is Derivative Compute Question 0050. \\ \fi
\begin{problem}

Find the limit.  Use L'H$\hat{o}$pital's rule where appropriate.

\input{Derivative-Compute-0050.HELP.tex}

\[\lim\limits_{x\to\infty} {{\left(-\frac{5}{x} + 1\right)}^{-3 \, x} - 18}=\answer{e^{15} - 18}\]
\end{problem}}%}

\latexProblemContent{
\ifVerboseLocation This is Derivative Compute Question 0050. \\ \fi
\begin{problem}

Find the limit.  Use L'H$\hat{o}$pital's rule where appropriate.

\input{Derivative-Compute-0050.HELP.tex}

\[\lim\limits_{x\to\infty} {{\left(\frac{3}{x} + 1\right)}^{-3 \, x} - 17}=\answer{e^{\left(-9\right)} - 17}\]
\end{problem}}%}

\latexProblemContent{
\ifVerboseLocation This is Derivative Compute Question 0050. \\ \fi
\begin{problem}

Find the limit.  Use L'H$\hat{o}$pital's rule where appropriate.

\input{Derivative-Compute-0050.HELP.tex}

\[\lim\limits_{x\to\infty} {{\left(\frac{5}{x} + 1\right)}^{-7 \, x} + 14}=\answer{e^{\left(-35\right)} + 14}\]
\end{problem}}%}

\latexProblemContent{
\ifVerboseLocation This is Derivative Compute Question 0050. \\ \fi
\begin{problem}

Find the limit.  Use L'H$\hat{o}$pital's rule where appropriate.

\input{Derivative-Compute-0050.HELP.tex}

\[\lim\limits_{x\to\infty} {{\left(-\frac{9}{x} + 1\right)}^{4 \, x} - 12}=\answer{e^{\left(-36\right)} - 12}\]
\end{problem}}%}

\latexProblemContent{
\ifVerboseLocation This is Derivative Compute Question 0050. \\ \fi
\begin{problem}

Find the limit.  Use L'H$\hat{o}$pital's rule where appropriate.

\input{Derivative-Compute-0050.HELP.tex}

\[\lim\limits_{x\to\infty} {{\left(-\frac{8}{x} + 1\right)}^{3 \, x} - 4}=\answer{e^{\left(-24\right)} - 4}\]
\end{problem}}%}

\latexProblemContent{
\ifVerboseLocation This is Derivative Compute Question 0050. \\ \fi
\begin{problem}

Find the limit.  Use L'H$\hat{o}$pital's rule where appropriate.

\input{Derivative-Compute-0050.HELP.tex}

\[\lim\limits_{x\to\infty} {{\left(-\frac{9}{x} + 1\right)}^{-6 \, x} + 19}=\answer{e^{54} + 19}\]
\end{problem}}%}

\latexProblemContent{
\ifVerboseLocation This is Derivative Compute Question 0050. \\ \fi
\begin{problem}

Find the limit.  Use L'H$\hat{o}$pital's rule where appropriate.

\input{Derivative-Compute-0050.HELP.tex}

\[\lim\limits_{x\to\infty} {{\left(-\frac{8}{x} + 1\right)}^{-7 \, x} - 15}=\answer{e^{56} - 15}\]
\end{problem}}%}

\latexProblemContent{
\ifVerboseLocation This is Derivative Compute Question 0050. \\ \fi
\begin{problem}

Find the limit.  Use L'H$\hat{o}$pital's rule where appropriate.

\input{Derivative-Compute-0050.HELP.tex}

\[\lim\limits_{x\to\infty} {{\left(\frac{10}{x} + 1\right)}^{-9 \, x} + 2}=\answer{e^{\left(-90\right)} + 2}\]
\end{problem}}%}

\latexProblemContent{
\ifVerboseLocation This is Derivative Compute Question 0050. \\ \fi
\begin{problem}

Find the limit.  Use L'H$\hat{o}$pital's rule where appropriate.

\input{Derivative-Compute-0050.HELP.tex}

\[\lim\limits_{x\to\infty} {{\left(\frac{4}{x} + 1\right)}^{-x} + 12}=\answer{e^{\left(-4\right)} + 12}\]
\end{problem}}%}

\latexProblemContent{
\ifVerboseLocation This is Derivative Compute Question 0050. \\ \fi
\begin{problem}

Find the limit.  Use L'H$\hat{o}$pital's rule where appropriate.

\input{Derivative-Compute-0050.HELP.tex}

\[\lim\limits_{x\to\infty} {{\left(\frac{3}{x} + 1\right)}^{3 \, x} - 2}=\answer{e^{9} - 2}\]
\end{problem}}%}

\latexProblemContent{
\ifVerboseLocation This is Derivative Compute Question 0050. \\ \fi
\begin{problem}

Find the limit.  Use L'H$\hat{o}$pital's rule where appropriate.

\input{Derivative-Compute-0050.HELP.tex}

\[\lim\limits_{x\to\infty} {{\left(-\frac{4}{x} + 1\right)}^{4 \, x} + 14}=\answer{e^{\left(-16\right)} + 14}\]
\end{problem}}%}

\latexProblemContent{
\ifVerboseLocation This is Derivative Compute Question 0050. \\ \fi
\begin{problem}

Find the limit.  Use L'H$\hat{o}$pital's rule where appropriate.

\input{Derivative-Compute-0050.HELP.tex}

\[\lim\limits_{x\to\infty} {{\left(\frac{8}{x} + 1\right)}^{2 \, x} - 6}=\answer{e^{16} - 6}\]
\end{problem}}%}

\latexProblemContent{
\ifVerboseLocation This is Derivative Compute Question 0050. \\ \fi
\begin{problem}

Find the limit.  Use L'H$\hat{o}$pital's rule where appropriate.

\input{Derivative-Compute-0050.HELP.tex}

\[\lim\limits_{x\to\infty} {{\left(\frac{7}{x} + 1\right)}^{4 \, x} - 10}=\answer{e^{28} - 10}\]
\end{problem}}%}

\latexProblemContent{
\ifVerboseLocation This is Derivative Compute Question 0050. \\ \fi
\begin{problem}

Find the limit.  Use L'H$\hat{o}$pital's rule where appropriate.

\input{Derivative-Compute-0050.HELP.tex}

\[\lim\limits_{x\to\infty} {{\left(\frac{2}{x} + 1\right)}^{3 \, x} + 6}=\answer{e^{6} + 6}\]
\end{problem}}%}

\latexProblemContent{
\ifVerboseLocation This is Derivative Compute Question 0050. \\ \fi
\begin{problem}

Find the limit.  Use L'H$\hat{o}$pital's rule where appropriate.

\input{Derivative-Compute-0050.HELP.tex}

\[\lim\limits_{x\to\infty} {{\left(-\frac{7}{x} + 1\right)}^{5 \, x} - 13}=\answer{e^{\left(-35\right)} - 13}\]
\end{problem}}%}

\latexProblemContent{
\ifVerboseLocation This is Derivative Compute Question 0050. \\ \fi
\begin{problem}

Find the limit.  Use L'H$\hat{o}$pital's rule where appropriate.

\input{Derivative-Compute-0050.HELP.tex}

\[\lim\limits_{x\to\infty} {{\left(-\frac{6}{x} + 1\right)}^{-4 \, x} - 10}=\answer{e^{24} - 10}\]
\end{problem}}%}

\latexProblemContent{
\ifVerboseLocation This is Derivative Compute Question 0050. \\ \fi
\begin{problem}

Find the limit.  Use L'H$\hat{o}$pital's rule where appropriate.

\input{Derivative-Compute-0050.HELP.tex}

\[\lim\limits_{x\to\infty} {{\left(-\frac{2}{x} + 1\right)}^{2 \, x} - 7}=\answer{e^{\left(-4\right)} - 7}\]
\end{problem}}%}

\latexProblemContent{
\ifVerboseLocation This is Derivative Compute Question 0050. \\ \fi
\begin{problem}

Find the limit.  Use L'H$\hat{o}$pital's rule where appropriate.

\input{Derivative-Compute-0050.HELP.tex}

\[\lim\limits_{x\to\infty} {{\left(\frac{4}{x} + 1\right)}^{8 \, x} - 20}=\answer{e^{32} - 20}\]
\end{problem}}%}

\latexProblemContent{
\ifVerboseLocation This is Derivative Compute Question 0050. \\ \fi
\begin{problem}

Find the limit.  Use L'H$\hat{o}$pital's rule where appropriate.

\input{Derivative-Compute-0050.HELP.tex}

\[\lim\limits_{x\to\infty} {{\left(-\frac{3}{x} + 1\right)}^{8 \, x} + 7}=\answer{e^{\left(-24\right)} + 7}\]
\end{problem}}%}

\latexProblemContent{
\ifVerboseLocation This is Derivative Compute Question 0050. \\ \fi
\begin{problem}

Find the limit.  Use L'H$\hat{o}$pital's rule where appropriate.

\input{Derivative-Compute-0050.HELP.tex}

\[\lim\limits_{x\to\infty} {{\left(\frac{6}{x} + 1\right)}^{-2 \, x} + 5}=\answer{e^{\left(-12\right)} + 5}\]
\end{problem}}%}

\latexProblemContent{
\ifVerboseLocation This is Derivative Compute Question 0050. \\ \fi
\begin{problem}

Find the limit.  Use L'H$\hat{o}$pital's rule where appropriate.

\input{Derivative-Compute-0050.HELP.tex}

\[\lim\limits_{x\to\infty} {{\left(\frac{9}{x} + 1\right)}^{10 \, x} + 16}=\answer{e^{90} + 16}\]
\end{problem}}%}

\latexProblemContent{
\ifVerboseLocation This is Derivative Compute Question 0050. \\ \fi
\begin{problem}

Find the limit.  Use L'H$\hat{o}$pital's rule where appropriate.

\input{Derivative-Compute-0050.HELP.tex}

\[\lim\limits_{x\to\infty} {{\left(-\frac{5}{x} + 1\right)}^{-10 \, x} - 12}=\answer{e^{50} - 12}\]
\end{problem}}%}

\latexProblemContent{
\ifVerboseLocation This is Derivative Compute Question 0050. \\ \fi
\begin{problem}

Find the limit.  Use L'H$\hat{o}$pital's rule where appropriate.

\input{Derivative-Compute-0050.HELP.tex}

\[\lim\limits_{x\to\infty} {{\left(-\frac{2}{x} + 1\right)}^{3 \, x} - 17}=\answer{e^{\left(-6\right)} - 17}\]
\end{problem}}%}

\latexProblemContent{
\ifVerboseLocation This is Derivative Compute Question 0050. \\ \fi
\begin{problem}

Find the limit.  Use L'H$\hat{o}$pital's rule where appropriate.

\input{Derivative-Compute-0050.HELP.tex}

\[\lim\limits_{x\to\infty} {{\left(\frac{10}{x} + 1\right)}^{7 \, x} - 20}=\answer{e^{70} - 20}\]
\end{problem}}%}

\latexProblemContent{
\ifVerboseLocation This is Derivative Compute Question 0050. \\ \fi
\begin{problem}

Find the limit.  Use L'H$\hat{o}$pital's rule where appropriate.

\input{Derivative-Compute-0050.HELP.tex}

\[\lim\limits_{x\to\infty} {{\left(\frac{8}{x} + 1\right)}^{-8 \, x} + 15}=\answer{e^{\left(-64\right)} + 15}\]
\end{problem}}%}

\latexProblemContent{
\ifVerboseLocation This is Derivative Compute Question 0050. \\ \fi
\begin{problem}

Find the limit.  Use L'H$\hat{o}$pital's rule where appropriate.

\input{Derivative-Compute-0050.HELP.tex}

\[\lim\limits_{x\to\infty} {{\left(\frac{10}{x} + 1\right)}^{-9 \, x} - 4}=\answer{e^{\left(-90\right)} - 4}\]
\end{problem}}%}

\latexProblemContent{
\ifVerboseLocation This is Derivative Compute Question 0050. \\ \fi
\begin{problem}

Find the limit.  Use L'H$\hat{o}$pital's rule where appropriate.

\input{Derivative-Compute-0050.HELP.tex}

\[\lim\limits_{x\to\infty} {{\left(\frac{7}{x} + 1\right)}^{5 \, x} - 5}=\answer{e^{35} - 5}\]
\end{problem}}%}

\latexProblemContent{
\ifVerboseLocation This is Derivative Compute Question 0050. \\ \fi
\begin{problem}

Find the limit.  Use L'H$\hat{o}$pital's rule where appropriate.

\input{Derivative-Compute-0050.HELP.tex}

\[\lim\limits_{x\to\infty} {{\left(-\frac{7}{x} + 1\right)}^{x} - 2}=\answer{e^{\left(-7\right)} - 2}\]
\end{problem}}%}

\latexProblemContent{
\ifVerboseLocation This is Derivative Compute Question 0050. \\ \fi
\begin{problem}

Find the limit.  Use L'H$\hat{o}$pital's rule where appropriate.

\input{Derivative-Compute-0050.HELP.tex}

\[\lim\limits_{x\to\infty} {{\left(-\frac{8}{x} + 1\right)}^{-6 \, x} + 20}=\answer{e^{48} + 20}\]
\end{problem}}%}

\latexProblemContent{
\ifVerboseLocation This is Derivative Compute Question 0050. \\ \fi
\begin{problem}

Find the limit.  Use L'H$\hat{o}$pital's rule where appropriate.

\input{Derivative-Compute-0050.HELP.tex}

\[\lim\limits_{x\to\infty} {{\left(\frac{3}{x} + 1\right)}^{8 \, x}}=\answer{e^{24}}\]
\end{problem}}%}

\latexProblemContent{
\ifVerboseLocation This is Derivative Compute Question 0050. \\ \fi
\begin{problem}

Find the limit.  Use L'H$\hat{o}$pital's rule where appropriate.

\input{Derivative-Compute-0050.HELP.tex}

\[\lim\limits_{x\to\infty} {{\left(-\frac{1}{x} + 1\right)}^{4 \, x} + 7}=\answer{e^{\left(-4\right)} + 7}\]
\end{problem}}%}

\latexProblemContent{
\ifVerboseLocation This is Derivative Compute Question 0050. \\ \fi
\begin{problem}

Find the limit.  Use L'H$\hat{o}$pital's rule where appropriate.

\input{Derivative-Compute-0050.HELP.tex}

\[\lim\limits_{x\to\infty} {{\left(\frac{8}{x} + 1\right)}^{-6 \, x} - 8}=\answer{e^{\left(-48\right)} - 8}\]
\end{problem}}%}

\latexProblemContent{
\ifVerboseLocation This is Derivative Compute Question 0050. \\ \fi
\begin{problem}

Find the limit.  Use L'H$\hat{o}$pital's rule where appropriate.

\input{Derivative-Compute-0050.HELP.tex}

\[\lim\limits_{x\to\infty} {{\left(\frac{10}{x} + 1\right)}^{-6 \, x} + 9}=\answer{e^{\left(-60\right)} + 9}\]
\end{problem}}%}

\latexProblemContent{
\ifVerboseLocation This is Derivative Compute Question 0050. \\ \fi
\begin{problem}

Find the limit.  Use L'H$\hat{o}$pital's rule where appropriate.

\input{Derivative-Compute-0050.HELP.tex}

\[\lim\limits_{x\to\infty} {{\left(-\frac{8}{x} + 1\right)}^{-x} - 2}=\answer{e^{8} - 2}\]
\end{problem}}%}

\latexProblemContent{
\ifVerboseLocation This is Derivative Compute Question 0050. \\ \fi
\begin{problem}

Find the limit.  Use L'H$\hat{o}$pital's rule where appropriate.

\input{Derivative-Compute-0050.HELP.tex}

\[\lim\limits_{x\to\infty} {{\left(-\frac{10}{x} + 1\right)}^{3 \, x} - 17}=\answer{e^{\left(-30\right)} - 17}\]
\end{problem}}%}

\latexProblemContent{
\ifVerboseLocation This is Derivative Compute Question 0050. \\ \fi
\begin{problem}

Find the limit.  Use L'H$\hat{o}$pital's rule where appropriate.

\input{Derivative-Compute-0050.HELP.tex}

\[\lim\limits_{x\to\infty} {{\left(\frac{7}{x} + 1\right)}^{-10 \, x} - 8}=\answer{e^{\left(-70\right)} - 8}\]
\end{problem}}%}

\latexProblemContent{
\ifVerboseLocation This is Derivative Compute Question 0050. \\ \fi
\begin{problem}

Find the limit.  Use L'H$\hat{o}$pital's rule where appropriate.

\input{Derivative-Compute-0050.HELP.tex}

\[\lim\limits_{x\to\infty} {{\left(\frac{1}{x} + 1\right)}^{-10 \, x} + 11}=\answer{e^{\left(-10\right)} + 11}\]
\end{problem}}%}

\latexProblemContent{
\ifVerboseLocation This is Derivative Compute Question 0050. \\ \fi
\begin{problem}

Find the limit.  Use L'H$\hat{o}$pital's rule where appropriate.

\input{Derivative-Compute-0050.HELP.tex}

\[\lim\limits_{x\to\infty} {{\left(\frac{9}{x} + 1\right)}^{-2 \, x} + 3}=\answer{e^{\left(-18\right)} + 3}\]
\end{problem}}%}

\latexProblemContent{
\ifVerboseLocation This is Derivative Compute Question 0050. \\ \fi
\begin{problem}

Find the limit.  Use L'H$\hat{o}$pital's rule where appropriate.

\input{Derivative-Compute-0050.HELP.tex}

\[\lim\limits_{x\to\infty} {{\left(-\frac{4}{x} + 1\right)}^{-5 \, x} + 17}=\answer{e^{20} + 17}\]
\end{problem}}%}

\latexProblemContent{
\ifVerboseLocation This is Derivative Compute Question 0050. \\ \fi
\begin{problem}

Find the limit.  Use L'H$\hat{o}$pital's rule where appropriate.

\input{Derivative-Compute-0050.HELP.tex}

\[\lim\limits_{x\to\infty} {{\left(\frac{5}{x} + 1\right)}^{10 \, x} - 5}=\answer{e^{50} - 5}\]
\end{problem}}%}

\latexProblemContent{
\ifVerboseLocation This is Derivative Compute Question 0050. \\ \fi
\begin{problem}

Find the limit.  Use L'H$\hat{o}$pital's rule where appropriate.

\input{Derivative-Compute-0050.HELP.tex}

\[\lim\limits_{x\to\infty} {{\left(-\frac{5}{x} + 1\right)}^{9 \, x} + 9}=\answer{e^{\left(-45\right)} + 9}\]
\end{problem}}%}

\latexProblemContent{
\ifVerboseLocation This is Derivative Compute Question 0050. \\ \fi
\begin{problem}

Find the limit.  Use L'H$\hat{o}$pital's rule where appropriate.

\input{Derivative-Compute-0050.HELP.tex}

\[\lim\limits_{x\to\infty} {{\left(-\frac{2}{x} + 1\right)}^{4 \, x} + 16}=\answer{e^{\left(-8\right)} + 16}\]
\end{problem}}%}

\latexProblemContent{
\ifVerboseLocation This is Derivative Compute Question 0050. \\ \fi
\begin{problem}

Find the limit.  Use L'H$\hat{o}$pital's rule where appropriate.

\input{Derivative-Compute-0050.HELP.tex}

\[\lim\limits_{x\to\infty} {{\left(\frac{10}{x} + 1\right)}^{2 \, x} + 4}=\answer{e^{20} + 4}\]
\end{problem}}%}

\latexProblemContent{
\ifVerboseLocation This is Derivative Compute Question 0050. \\ \fi
\begin{problem}

Find the limit.  Use L'H$\hat{o}$pital's rule where appropriate.

\input{Derivative-Compute-0050.HELP.tex}

\[\lim\limits_{x\to\infty} {{\left(-\frac{7}{x} + 1\right)}^{-5 \, x} + 8}=\answer{e^{35} + 8}\]
\end{problem}}%}

\latexProblemContent{
\ifVerboseLocation This is Derivative Compute Question 0050. \\ \fi
\begin{problem}

Find the limit.  Use L'H$\hat{o}$pital's rule where appropriate.

\input{Derivative-Compute-0050.HELP.tex}

\[\lim\limits_{x\to\infty} {{\left(\frac{5}{x} + 1\right)}^{-x} - 7}=\answer{e^{\left(-5\right)} - 7}\]
\end{problem}}%}

\latexProblemContent{
\ifVerboseLocation This is Derivative Compute Question 0050. \\ \fi
\begin{problem}

Find the limit.  Use L'H$\hat{o}$pital's rule where appropriate.

\input{Derivative-Compute-0050.HELP.tex}

\[\lim\limits_{x\to\infty} {{\left(\frac{5}{x} + 1\right)}^{-2 \, x} + 16}=\answer{e^{\left(-10\right)} + 16}\]
\end{problem}}%}

\latexProblemContent{
\ifVerboseLocation This is Derivative Compute Question 0050. \\ \fi
\begin{problem}

Find the limit.  Use L'H$\hat{o}$pital's rule where appropriate.

\input{Derivative-Compute-0050.HELP.tex}

\[\lim\limits_{x\to\infty} {{\left(\frac{3}{x} + 1\right)}^{-8 \, x} - 14}=\answer{e^{\left(-24\right)} - 14}\]
\end{problem}}%}

\latexProblemContent{
\ifVerboseLocation This is Derivative Compute Question 0050. \\ \fi
\begin{problem}

Find the limit.  Use L'H$\hat{o}$pital's rule where appropriate.

\input{Derivative-Compute-0050.HELP.tex}

\[\lim\limits_{x\to\infty} {{\left(-\frac{8}{x} + 1\right)}^{-7 \, x} + 13}=\answer{e^{56} + 13}\]
\end{problem}}%}

\latexProblemContent{
\ifVerboseLocation This is Derivative Compute Question 0050. \\ \fi
\begin{problem}

Find the limit.  Use L'H$\hat{o}$pital's rule where appropriate.

\input{Derivative-Compute-0050.HELP.tex}

\[\lim\limits_{x\to\infty} {{\left(\frac{8}{x} + 1\right)}^{10 \, x} + 2}=\answer{e^{80} + 2}\]
\end{problem}}%}

\latexProblemContent{
\ifVerboseLocation This is Derivative Compute Question 0050. \\ \fi
\begin{problem}

Find the limit.  Use L'H$\hat{o}$pital's rule where appropriate.

\input{Derivative-Compute-0050.HELP.tex}

\[\lim\limits_{x\to\infty} {{\left(\frac{3}{x} + 1\right)}^{10 \, x} - 7}=\answer{e^{30} - 7}\]
\end{problem}}%}

\latexProblemContent{
\ifVerboseLocation This is Derivative Compute Question 0050. \\ \fi
\begin{problem}

Find the limit.  Use L'H$\hat{o}$pital's rule where appropriate.

\input{Derivative-Compute-0050.HELP.tex}

\[\lim\limits_{x\to\infty} {{\left(\frac{8}{x} + 1\right)}^{-5 \, x} + 2}=\answer{e^{\left(-40\right)} + 2}\]
\end{problem}}%}

\latexProblemContent{
\ifVerboseLocation This is Derivative Compute Question 0050. \\ \fi
\begin{problem}

Find the limit.  Use L'H$\hat{o}$pital's rule where appropriate.

\input{Derivative-Compute-0050.HELP.tex}

\[\lim\limits_{x\to\infty} {{\left(\frac{2}{x} + 1\right)}^{-8 \, x} - 14}=\answer{e^{\left(-16\right)} - 14}\]
\end{problem}}%}

\latexProblemContent{
\ifVerboseLocation This is Derivative Compute Question 0050. \\ \fi
\begin{problem}

Find the limit.  Use L'H$\hat{o}$pital's rule where appropriate.

\input{Derivative-Compute-0050.HELP.tex}

\[\lim\limits_{x\to\infty} {{\left(\frac{7}{x} + 1\right)}^{-2 \, x} + 15}=\answer{e^{\left(-14\right)} + 15}\]
\end{problem}}%}

\latexProblemContent{
\ifVerboseLocation This is Derivative Compute Question 0050. \\ \fi
\begin{problem}

Find the limit.  Use L'H$\hat{o}$pital's rule where appropriate.

\input{Derivative-Compute-0050.HELP.tex}

\[\lim\limits_{x\to\infty} {{\left(-\frac{1}{x} + 1\right)}^{-10 \, x} + 1}=\answer{e^{10} + 1}\]
\end{problem}}%}

\latexProblemContent{
\ifVerboseLocation This is Derivative Compute Question 0050. \\ \fi
\begin{problem}

Find the limit.  Use L'H$\hat{o}$pital's rule where appropriate.

\input{Derivative-Compute-0050.HELP.tex}

\[\lim\limits_{x\to\infty} {{\left(-\frac{10}{x} + 1\right)}^{x}}=\answer{e^{\left(-10\right)}}\]
\end{problem}}%}

\latexProblemContent{
\ifVerboseLocation This is Derivative Compute Question 0050. \\ \fi
\begin{problem}

Find the limit.  Use L'H$\hat{o}$pital's rule where appropriate.

\input{Derivative-Compute-0050.HELP.tex}

\[\lim\limits_{x\to\infty} {{\left(\frac{8}{x} + 1\right)}^{9 \, x} + 18}=\answer{e^{72} + 18}\]
\end{problem}}%}

\latexProblemContent{
\ifVerboseLocation This is Derivative Compute Question 0050. \\ \fi
\begin{problem}

Find the limit.  Use L'H$\hat{o}$pital's rule where appropriate.

\input{Derivative-Compute-0050.HELP.tex}

\[\lim\limits_{x\to\infty} {{\left(-\frac{4}{x} + 1\right)}^{5 \, x} - 10}=\answer{e^{\left(-20\right)} - 10}\]
\end{problem}}%}

\latexProblemContent{
\ifVerboseLocation This is Derivative Compute Question 0050. \\ \fi
\begin{problem}

Find the limit.  Use L'H$\hat{o}$pital's rule where appropriate.

\input{Derivative-Compute-0050.HELP.tex}

\[\lim\limits_{x\to\infty} {{\left(-\frac{2}{x} + 1\right)}^{9 \, x} + 6}=\answer{e^{\left(-18\right)} + 6}\]
\end{problem}}%}

\latexProblemContent{
\ifVerboseLocation This is Derivative Compute Question 0050. \\ \fi
\begin{problem}

Find the limit.  Use L'H$\hat{o}$pital's rule where appropriate.

\input{Derivative-Compute-0050.HELP.tex}

\[\lim\limits_{x\to\infty} {{\left(-\frac{9}{x} + 1\right)}^{4 \, x} - 7}=\answer{e^{\left(-36\right)} - 7}\]
\end{problem}}%}

\latexProblemContent{
\ifVerboseLocation This is Derivative Compute Question 0050. \\ \fi
\begin{problem}

Find the limit.  Use L'H$\hat{o}$pital's rule where appropriate.

\input{Derivative-Compute-0050.HELP.tex}

\[\lim\limits_{x\to\infty} {{\left(-\frac{2}{x} + 1\right)}^{2 \, x} - 13}=\answer{e^{\left(-4\right)} - 13}\]
\end{problem}}%}

\latexProblemContent{
\ifVerboseLocation This is Derivative Compute Question 0050. \\ \fi
\begin{problem}

Find the limit.  Use L'H$\hat{o}$pital's rule where appropriate.

\input{Derivative-Compute-0050.HELP.tex}

\[\lim\limits_{x\to\infty} {{\left(\frac{1}{x} + 1\right)}^{10 \, x} - 5}=\answer{e^{10} - 5}\]
\end{problem}}%}

\latexProblemContent{
\ifVerboseLocation This is Derivative Compute Question 0050. \\ \fi
\begin{problem}

Find the limit.  Use L'H$\hat{o}$pital's rule where appropriate.

\input{Derivative-Compute-0050.HELP.tex}

\[\lim\limits_{x\to\infty} {{\left(\frac{2}{x} + 1\right)}^{-6 \, x} + 1}=\answer{e^{\left(-12\right)} + 1}\]
\end{problem}}%}

\latexProblemContent{
\ifVerboseLocation This is Derivative Compute Question 0050. \\ \fi
\begin{problem}

Find the limit.  Use L'H$\hat{o}$pital's rule where appropriate.

\input{Derivative-Compute-0050.HELP.tex}

\[\lim\limits_{x\to\infty} {{\left(\frac{1}{x} + 1\right)}^{-x} - 16}=\answer{e^{\left(-1\right)} - 16}\]
\end{problem}}%}

\latexProblemContent{
\ifVerboseLocation This is Derivative Compute Question 0050. \\ \fi
\begin{problem}

Find the limit.  Use L'H$\hat{o}$pital's rule where appropriate.

\input{Derivative-Compute-0050.HELP.tex}

\[\lim\limits_{x\to\infty} {{\left(\frac{1}{x} + 1\right)}^{8 \, x} + 8}=\answer{e^{8} + 8}\]
\end{problem}}%}

\latexProblemContent{
\ifVerboseLocation This is Derivative Compute Question 0050. \\ \fi
\begin{problem}

Find the limit.  Use L'H$\hat{o}$pital's rule where appropriate.

\input{Derivative-Compute-0050.HELP.tex}

\[\lim\limits_{x\to\infty} {{\left(-\frac{8}{x} + 1\right)}^{6 \, x} + 9}=\answer{e^{\left(-48\right)} + 9}\]
\end{problem}}%}

\latexProblemContent{
\ifVerboseLocation This is Derivative Compute Question 0050. \\ \fi
\begin{problem}

Find the limit.  Use L'H$\hat{o}$pital's rule where appropriate.

\input{Derivative-Compute-0050.HELP.tex}

\[\lim\limits_{x\to\infty} {{\left(-\frac{9}{x} + 1\right)}^{-7 \, x} + 5}=\answer{e^{63} + 5}\]
\end{problem}}%}

\latexProblemContent{
\ifVerboseLocation This is Derivative Compute Question 0050. \\ \fi
\begin{problem}

Find the limit.  Use L'H$\hat{o}$pital's rule where appropriate.

\input{Derivative-Compute-0050.HELP.tex}

\[\lim\limits_{x\to\infty} {{\left(-\frac{5}{x} + 1\right)}^{10 \, x} + 20}=\answer{e^{\left(-50\right)} + 20}\]
\end{problem}}%}

\latexProblemContent{
\ifVerboseLocation This is Derivative Compute Question 0050. \\ \fi
\begin{problem}

Find the limit.  Use L'H$\hat{o}$pital's rule where appropriate.

\input{Derivative-Compute-0050.HELP.tex}

\[\lim\limits_{x\to\infty} {{\left(\frac{8}{x} + 1\right)}^{8 \, x} + 12}=\answer{e^{64} + 12}\]
\end{problem}}%}

\latexProblemContent{
\ifVerboseLocation This is Derivative Compute Question 0050. \\ \fi
\begin{problem}

Find the limit.  Use L'H$\hat{o}$pital's rule where appropriate.

\input{Derivative-Compute-0050.HELP.tex}

\[\lim\limits_{x\to\infty} {{\left(\frac{5}{x} + 1\right)}^{-x} + 11}=\answer{e^{\left(-5\right)} + 11}\]
\end{problem}}%}

\latexProblemContent{
\ifVerboseLocation This is Derivative Compute Question 0050. \\ \fi
\begin{problem}

Find the limit.  Use L'H$\hat{o}$pital's rule where appropriate.

\input{Derivative-Compute-0050.HELP.tex}

\[\lim\limits_{x\to\infty} {{\left(\frac{3}{x} + 1\right)}^{9 \, x} + 9}=\answer{e^{27} + 9}\]
\end{problem}}%}

\latexProblemContent{
\ifVerboseLocation This is Derivative Compute Question 0050. \\ \fi
\begin{problem}

Find the limit.  Use L'H$\hat{o}$pital's rule where appropriate.

\input{Derivative-Compute-0050.HELP.tex}

\[\lim\limits_{x\to\infty} {{\left(\frac{2}{x} + 1\right)}^{-8 \, x} - 11}=\answer{e^{\left(-16\right)} - 11}\]
\end{problem}}%}

\latexProblemContent{
\ifVerboseLocation This is Derivative Compute Question 0050. \\ \fi
\begin{problem}

Find the limit.  Use L'H$\hat{o}$pital's rule where appropriate.

\input{Derivative-Compute-0050.HELP.tex}

\[\lim\limits_{x\to\infty} {{\left(\frac{5}{x} + 1\right)}^{-10 \, x} + 19}=\answer{e^{\left(-50\right)} + 19}\]
\end{problem}}%}

\latexProblemContent{
\ifVerboseLocation This is Derivative Compute Question 0050. \\ \fi
\begin{problem}

Find the limit.  Use L'H$\hat{o}$pital's rule where appropriate.

\input{Derivative-Compute-0050.HELP.tex}

\[\lim\limits_{x\to\infty} {{\left(-\frac{3}{x} + 1\right)}^{7 \, x} - 4}=\answer{e^{\left(-21\right)} - 4}\]
\end{problem}}%}

\latexProblemContent{
\ifVerboseLocation This is Derivative Compute Question 0050. \\ \fi
\begin{problem}

Find the limit.  Use L'H$\hat{o}$pital's rule where appropriate.

\input{Derivative-Compute-0050.HELP.tex}

\[\lim\limits_{x\to\infty} {{\left(-\frac{9}{x} + 1\right)}^{-9 \, x} - 2}=\answer{e^{81} - 2}\]
\end{problem}}%}

\latexProblemContent{
\ifVerboseLocation This is Derivative Compute Question 0050. \\ \fi
\begin{problem}

Find the limit.  Use L'H$\hat{o}$pital's rule where appropriate.

\input{Derivative-Compute-0050.HELP.tex}

\[\lim\limits_{x\to\infty} {{\left(\frac{10}{x} + 1\right)}^{2 \, x} - 6}=\answer{e^{20} - 6}\]
\end{problem}}%}

\latexProblemContent{
\ifVerboseLocation This is Derivative Compute Question 0050. \\ \fi
\begin{problem}

Find the limit.  Use L'H$\hat{o}$pital's rule where appropriate.

\input{Derivative-Compute-0050.HELP.tex}

\[\lim\limits_{x\to\infty} {{\left(\frac{10}{x} + 1\right)}^{2 \, x} - 12}=\answer{e^{20} - 12}\]
\end{problem}}%}

\latexProblemContent{
\ifVerboseLocation This is Derivative Compute Question 0050. \\ \fi
\begin{problem}

Find the limit.  Use L'H$\hat{o}$pital's rule where appropriate.

\input{Derivative-Compute-0050.HELP.tex}

\[\lim\limits_{x\to\infty} {{\left(\frac{7}{x} + 1\right)}^{3 \, x} - 17}=\answer{e^{21} - 17}\]
\end{problem}}%}

\latexProblemContent{
\ifVerboseLocation This is Derivative Compute Question 0050. \\ \fi
\begin{problem}

Find the limit.  Use L'H$\hat{o}$pital's rule where appropriate.

\input{Derivative-Compute-0050.HELP.tex}

\[\lim\limits_{x\to\infty} {{\left(\frac{9}{x} + 1\right)}^{-4 \, x} - 11}=\answer{e^{\left(-36\right)} - 11}\]
\end{problem}}%}

\latexProblemContent{
\ifVerboseLocation This is Derivative Compute Question 0050. \\ \fi
\begin{problem}

Find the limit.  Use L'H$\hat{o}$pital's rule where appropriate.

\input{Derivative-Compute-0050.HELP.tex}

\[\lim\limits_{x\to\infty} {{\left(\frac{5}{x} + 1\right)}^{9 \, x} + 13}=\answer{e^{45} + 13}\]
\end{problem}}%}

\latexProblemContent{
\ifVerboseLocation This is Derivative Compute Question 0050. \\ \fi
\begin{problem}

Find the limit.  Use L'H$\hat{o}$pital's rule where appropriate.

\input{Derivative-Compute-0050.HELP.tex}

\[\lim\limits_{x\to\infty} {{\left(\frac{2}{x} + 1\right)}^{9 \, x} - 15}=\answer{e^{18} - 15}\]
\end{problem}}%}

\latexProblemContent{
\ifVerboseLocation This is Derivative Compute Question 0050. \\ \fi
\begin{problem}

Find the limit.  Use L'H$\hat{o}$pital's rule where appropriate.

\input{Derivative-Compute-0050.HELP.tex}

\[\lim\limits_{x\to\infty} {{\left(\frac{7}{x} + 1\right)}^{x} + 1}=\answer{e^{7} + 1}\]
\end{problem}}%}

\latexProblemContent{
\ifVerboseLocation This is Derivative Compute Question 0050. \\ \fi
\begin{problem}

Find the limit.  Use L'H$\hat{o}$pital's rule where appropriate.

\input{Derivative-Compute-0050.HELP.tex}

\[\lim\limits_{x\to\infty} {{\left(-\frac{8}{x} + 1\right)}^{10 \, x} - 11}=\answer{e^{\left(-80\right)} - 11}\]
\end{problem}}%}

\latexProblemContent{
\ifVerboseLocation This is Derivative Compute Question 0050. \\ \fi
\begin{problem}

Find the limit.  Use L'H$\hat{o}$pital's rule where appropriate.

\input{Derivative-Compute-0050.HELP.tex}

\[\lim\limits_{x\to\infty} {{\left(\frac{3}{x} + 1\right)}^{-2 \, x} - 2}=\answer{e^{\left(-6\right)} - 2}\]
\end{problem}}%}

\latexProblemContent{
\ifVerboseLocation This is Derivative Compute Question 0050. \\ \fi
\begin{problem}

Find the limit.  Use L'H$\hat{o}$pital's rule where appropriate.

\input{Derivative-Compute-0050.HELP.tex}

\[\lim\limits_{x\to\infty} {{\left(-\frac{8}{x} + 1\right)}^{2 \, x} - 17}=\answer{e^{\left(-16\right)} - 17}\]
\end{problem}}%}

\latexProblemContent{
\ifVerboseLocation This is Derivative Compute Question 0050. \\ \fi
\begin{problem}

Find the limit.  Use L'H$\hat{o}$pital's rule where appropriate.

\input{Derivative-Compute-0050.HELP.tex}

\[\lim\limits_{x\to\infty} {{\left(-\frac{2}{x} + 1\right)}^{4 \, x} - 13}=\answer{e^{\left(-8\right)} - 13}\]
\end{problem}}%}

\latexProblemContent{
\ifVerboseLocation This is Derivative Compute Question 0050. \\ \fi
\begin{problem}

Find the limit.  Use L'H$\hat{o}$pital's rule where appropriate.

\input{Derivative-Compute-0050.HELP.tex}

\[\lim\limits_{x\to\infty} {{\left(-\frac{4}{x} + 1\right)}^{-6 \, x} - 17}=\answer{e^{24} - 17}\]
\end{problem}}%}

\latexProblemContent{
\ifVerboseLocation This is Derivative Compute Question 0050. \\ \fi
\begin{problem}

Find the limit.  Use L'H$\hat{o}$pital's rule where appropriate.

\input{Derivative-Compute-0050.HELP.tex}

\[\lim\limits_{x\to\infty} {{\left(\frac{4}{x} + 1\right)}^{-7 \, x} + 9}=\answer{e^{\left(-28\right)} + 9}\]
\end{problem}}%}

\latexProblemContent{
\ifVerboseLocation This is Derivative Compute Question 0050. \\ \fi
\begin{problem}

Find the limit.  Use L'H$\hat{o}$pital's rule where appropriate.

\input{Derivative-Compute-0050.HELP.tex}

\[\lim\limits_{x\to\infty} {{\left(\frac{4}{x} + 1\right)}^{-7 \, x} - 7}=\answer{e^{\left(-28\right)} - 7}\]
\end{problem}}%}

\latexProblemContent{
\ifVerboseLocation This is Derivative Compute Question 0050. \\ \fi
\begin{problem}

Find the limit.  Use L'H$\hat{o}$pital's rule where appropriate.

\input{Derivative-Compute-0050.HELP.tex}

\[\lim\limits_{x\to\infty} {{\left(\frac{1}{x} + 1\right)}^{-3 \, x} + 5}=\answer{e^{\left(-3\right)} + 5}\]
\end{problem}}%}

\latexProblemContent{
\ifVerboseLocation This is Derivative Compute Question 0050. \\ \fi
\begin{problem}

Find the limit.  Use L'H$\hat{o}$pital's rule where appropriate.

\input{Derivative-Compute-0050.HELP.tex}

\[\lim\limits_{x\to\infty} {{\left(\frac{5}{x} + 1\right)}^{4 \, x} + 11}=\answer{e^{20} + 11}\]
\end{problem}}%}

\latexProblemContent{
\ifVerboseLocation This is Derivative Compute Question 0050. \\ \fi
\begin{problem}

Find the limit.  Use L'H$\hat{o}$pital's rule where appropriate.

\input{Derivative-Compute-0050.HELP.tex}

\[\lim\limits_{x\to\infty} {{\left(\frac{1}{x} + 1\right)}^{-6 \, x} + 8}=\answer{e^{\left(-6\right)} + 8}\]
\end{problem}}%}

\latexProblemContent{
\ifVerboseLocation This is Derivative Compute Question 0050. \\ \fi
\begin{problem}

Find the limit.  Use L'H$\hat{o}$pital's rule where appropriate.

\input{Derivative-Compute-0050.HELP.tex}

\[\lim\limits_{x\to\infty} {{\left(-\frac{2}{x} + 1\right)}^{2 \, x} + 16}=\answer{e^{\left(-4\right)} + 16}\]
\end{problem}}%}

\latexProblemContent{
\ifVerboseLocation This is Derivative Compute Question 0050. \\ \fi
\begin{problem}

Find the limit.  Use L'H$\hat{o}$pital's rule where appropriate.

\input{Derivative-Compute-0050.HELP.tex}

\[\lim\limits_{x\to\infty} {{\left(-\frac{9}{x} + 1\right)}^{-2 \, x} - 10}=\answer{e^{18} - 10}\]
\end{problem}}%}

\latexProblemContent{
\ifVerboseLocation This is Derivative Compute Question 0050. \\ \fi
\begin{problem}

Find the limit.  Use L'H$\hat{o}$pital's rule where appropriate.

\input{Derivative-Compute-0050.HELP.tex}

\[\lim\limits_{x\to\infty} {{\left(-\frac{1}{x} + 1\right)}^{-8 \, x} - 12}=\answer{e^{8} - 12}\]
\end{problem}}%}

\latexProblemContent{
\ifVerboseLocation This is Derivative Compute Question 0050. \\ \fi
\begin{problem}

Find the limit.  Use L'H$\hat{o}$pital's rule where appropriate.

\input{Derivative-Compute-0050.HELP.tex}

\[\lim\limits_{x\to\infty} {{\left(-\frac{7}{x} + 1\right)}^{-10 \, x} - 5}=\answer{e^{70} - 5}\]
\end{problem}}%}

\latexProblemContent{
\ifVerboseLocation This is Derivative Compute Question 0050. \\ \fi
\begin{problem}

Find the limit.  Use L'H$\hat{o}$pital's rule where appropriate.

\input{Derivative-Compute-0050.HELP.tex}

\[\lim\limits_{x\to\infty} {{\left(-\frac{3}{x} + 1\right)}^{6 \, x} + 5}=\answer{e^{\left(-18\right)} + 5}\]
\end{problem}}%}

\latexProblemContent{
\ifVerboseLocation This is Derivative Compute Question 0050. \\ \fi
\begin{problem}

Find the limit.  Use L'H$\hat{o}$pital's rule where appropriate.

\input{Derivative-Compute-0050.HELP.tex}

\[\lim\limits_{x\to\infty} {{\left(-\frac{4}{x} + 1\right)}^{7 \, x} - 11}=\answer{e^{\left(-28\right)} - 11}\]
\end{problem}}%}

\latexProblemContent{
\ifVerboseLocation This is Derivative Compute Question 0050. \\ \fi
\begin{problem}

Find the limit.  Use L'H$\hat{o}$pital's rule where appropriate.

\input{Derivative-Compute-0050.HELP.tex}

\[\lim\limits_{x\to\infty} {{\left(-\frac{1}{x} + 1\right)}^{10 \, x} + 3}=\answer{e^{\left(-10\right)} + 3}\]
\end{problem}}%}

\latexProblemContent{
\ifVerboseLocation This is Derivative Compute Question 0050. \\ \fi
\begin{problem}

Find the limit.  Use L'H$\hat{o}$pital's rule where appropriate.

\input{Derivative-Compute-0050.HELP.tex}

\[\lim\limits_{x\to\infty} {{\left(-\frac{2}{x} + 1\right)}^{4 \, x} + 20}=\answer{e^{\left(-8\right)} + 20}\]
\end{problem}}%}

\latexProblemContent{
\ifVerboseLocation This is Derivative Compute Question 0050. \\ \fi
\begin{problem}

Find the limit.  Use L'H$\hat{o}$pital's rule where appropriate.

\input{Derivative-Compute-0050.HELP.tex}

\[\lim\limits_{x\to\infty} {{\left(-\frac{10}{x} + 1\right)}^{-8 \, x} + 7}=\answer{e^{80} + 7}\]
\end{problem}}%}

\latexProblemContent{
\ifVerboseLocation This is Derivative Compute Question 0050. \\ \fi
\begin{problem}

Find the limit.  Use L'H$\hat{o}$pital's rule where appropriate.

\input{Derivative-Compute-0050.HELP.tex}

\[\lim\limits_{x\to\infty} {{\left(-\frac{2}{x} + 1\right)}^{-3 \, x} + 11}=\answer{e^{6} + 11}\]
\end{problem}}%}

\latexProblemContent{
\ifVerboseLocation This is Derivative Compute Question 0050. \\ \fi
\begin{problem}

Find the limit.  Use L'H$\hat{o}$pital's rule where appropriate.

\input{Derivative-Compute-0050.HELP.tex}

\[\lim\limits_{x\to\infty} {{\left(-\frac{1}{x} + 1\right)}^{5 \, x} + 20}=\answer{e^{\left(-5\right)} + 20}\]
\end{problem}}%}

\latexProblemContent{
\ifVerboseLocation This is Derivative Compute Question 0050. \\ \fi
\begin{problem}

Find the limit.  Use L'H$\hat{o}$pital's rule where appropriate.

\input{Derivative-Compute-0050.HELP.tex}

\[\lim\limits_{x\to\infty} {{\left(\frac{4}{x} + 1\right)}^{6 \, x} - 8}=\answer{e^{24} - 8}\]
\end{problem}}%}

\latexProblemContent{
\ifVerboseLocation This is Derivative Compute Question 0050. \\ \fi
\begin{problem}

Find the limit.  Use L'H$\hat{o}$pital's rule where appropriate.

\input{Derivative-Compute-0050.HELP.tex}

\[\lim\limits_{x\to\infty} {{\left(\frac{10}{x} + 1\right)}^{5 \, x} + 13}=\answer{e^{50} + 13}\]
\end{problem}}%}

\latexProblemContent{
\ifVerboseLocation This is Derivative Compute Question 0050. \\ \fi
\begin{problem}

Find the limit.  Use L'H$\hat{o}$pital's rule where appropriate.

\input{Derivative-Compute-0050.HELP.tex}

\[\lim\limits_{x\to\infty} {{\left(-\frac{2}{x} + 1\right)}^{5 \, x} - 5}=\answer{e^{\left(-10\right)} - 5}\]
\end{problem}}%}

\latexProblemContent{
\ifVerboseLocation This is Derivative Compute Question 0050. \\ \fi
\begin{problem}

Find the limit.  Use L'H$\hat{o}$pital's rule where appropriate.

\input{Derivative-Compute-0050.HELP.tex}

\[\lim\limits_{x\to\infty} {{\left(\frac{2}{x} + 1\right)}^{7 \, x} - 19}=\answer{e^{14} - 19}\]
\end{problem}}%}

\latexProblemContent{
\ifVerboseLocation This is Derivative Compute Question 0050. \\ \fi
\begin{problem}

Find the limit.  Use L'H$\hat{o}$pital's rule where appropriate.

\input{Derivative-Compute-0050.HELP.tex}

\[\lim\limits_{x\to\infty} {{\left(-\frac{10}{x} + 1\right)}^{-10 \, x} + 4}=\answer{e^{100} + 4}\]
\end{problem}}%}

\latexProblemContent{
\ifVerboseLocation This is Derivative Compute Question 0050. \\ \fi
\begin{problem}

Find the limit.  Use L'H$\hat{o}$pital's rule where appropriate.

\input{Derivative-Compute-0050.HELP.tex}

\[\lim\limits_{x\to\infty} {{\left(\frac{5}{x} + 1\right)}^{-5 \, x} - 7}=\answer{e^{\left(-25\right)} - 7}\]
\end{problem}}%}

\latexProblemContent{
\ifVerboseLocation This is Derivative Compute Question 0050. \\ \fi
\begin{problem}

Find the limit.  Use L'H$\hat{o}$pital's rule where appropriate.

\input{Derivative-Compute-0050.HELP.tex}

\[\lim\limits_{x\to\infty} {{\left(\frac{1}{x} + 1\right)}^{-10 \, x} + 8}=\answer{e^{\left(-10\right)} + 8}\]
\end{problem}}%}

\latexProblemContent{
\ifVerboseLocation This is Derivative Compute Question 0050. \\ \fi
\begin{problem}

Find the limit.  Use L'H$\hat{o}$pital's rule where appropriate.

\input{Derivative-Compute-0050.HELP.tex}

\[\lim\limits_{x\to\infty} {{\left(-\frac{9}{x} + 1\right)}^{4 \, x}}=\answer{e^{\left(-36\right)}}\]
\end{problem}}%}

\latexProblemContent{
\ifVerboseLocation This is Derivative Compute Question 0050. \\ \fi
\begin{problem}

Find the limit.  Use L'H$\hat{o}$pital's rule where appropriate.

\input{Derivative-Compute-0050.HELP.tex}

\[\lim\limits_{x\to\infty} {{\left(-\frac{3}{x} + 1\right)}^{10 \, x} + 3}=\answer{e^{\left(-30\right)} + 3}\]
\end{problem}}%}

\latexProblemContent{
\ifVerboseLocation This is Derivative Compute Question 0050. \\ \fi
\begin{problem}

Find the limit.  Use L'H$\hat{o}$pital's rule where appropriate.

\input{Derivative-Compute-0050.HELP.tex}

\[\lim\limits_{x\to\infty} {{\left(-\frac{9}{x} + 1\right)}^{x} - 8}=\answer{e^{\left(-9\right)} - 8}\]
\end{problem}}%}

\latexProblemContent{
\ifVerboseLocation This is Derivative Compute Question 0050. \\ \fi
\begin{problem}

Find the limit.  Use L'H$\hat{o}$pital's rule where appropriate.

\input{Derivative-Compute-0050.HELP.tex}

\[\lim\limits_{x\to\infty} {{\left(-\frac{1}{x} + 1\right)}^{-3 \, x} + 14}=\answer{e^{3} + 14}\]
\end{problem}}%}

\latexProblemContent{
\ifVerboseLocation This is Derivative Compute Question 0050. \\ \fi
\begin{problem}

Find the limit.  Use L'H$\hat{o}$pital's rule where appropriate.

\input{Derivative-Compute-0050.HELP.tex}

\[\lim\limits_{x\to\infty} {{\left(-\frac{9}{x} + 1\right)}^{4 \, x} - 2}=\answer{e^{\left(-36\right)} - 2}\]
\end{problem}}%}

\latexProblemContent{
\ifVerboseLocation This is Derivative Compute Question 0050. \\ \fi
\begin{problem}

Find the limit.  Use L'H$\hat{o}$pital's rule where appropriate.

\input{Derivative-Compute-0050.HELP.tex}

\[\lim\limits_{x\to\infty} {{\left(\frac{2}{x} + 1\right)}^{2 \, x} + 3}=\answer{e^{4} + 3}\]
\end{problem}}%}

\latexProblemContent{
\ifVerboseLocation This is Derivative Compute Question 0050. \\ \fi
\begin{problem}

Find the limit.  Use L'H$\hat{o}$pital's rule where appropriate.

\input{Derivative-Compute-0050.HELP.tex}

\[\lim\limits_{x\to\infty} {{\left(-\frac{8}{x} + 1\right)}^{x} + 1}=\answer{e^{\left(-8\right)} + 1}\]
\end{problem}}%}

\latexProblemContent{
\ifVerboseLocation This is Derivative Compute Question 0050. \\ \fi
\begin{problem}

Find the limit.  Use L'H$\hat{o}$pital's rule where appropriate.

\input{Derivative-Compute-0050.HELP.tex}

\[\lim\limits_{x\to\infty} {{\left(\frac{1}{x} + 1\right)}^{-10 \, x} - 8}=\answer{e^{\left(-10\right)} - 8}\]
\end{problem}}%}

\latexProblemContent{
\ifVerboseLocation This is Derivative Compute Question 0050. \\ \fi
\begin{problem}

Find the limit.  Use L'H$\hat{o}$pital's rule where appropriate.

\input{Derivative-Compute-0050.HELP.tex}

\[\lim\limits_{x\to\infty} {{\left(-\frac{6}{x} + 1\right)}^{6 \, x} - 14}=\answer{e^{\left(-36\right)} - 14}\]
\end{problem}}%}

\latexProblemContent{
\ifVerboseLocation This is Derivative Compute Question 0050. \\ \fi
\begin{problem}

Find the limit.  Use L'H$\hat{o}$pital's rule where appropriate.

\input{Derivative-Compute-0050.HELP.tex}

\[\lim\limits_{x\to\infty} {{\left(-\frac{3}{x} + 1\right)}^{-2 \, x} + 2}=\answer{e^{6} + 2}\]
\end{problem}}%}

\latexProblemContent{
\ifVerboseLocation This is Derivative Compute Question 0050. \\ \fi
\begin{problem}

Find the limit.  Use L'H$\hat{o}$pital's rule where appropriate.

\input{Derivative-Compute-0050.HELP.tex}

\[\lim\limits_{x\to\infty} {{\left(-\frac{1}{x} + 1\right)}^{4 \, x} + 12}=\answer{e^{\left(-4\right)} + 12}\]
\end{problem}}%}

\latexProblemContent{
\ifVerboseLocation This is Derivative Compute Question 0050. \\ \fi
\begin{problem}

Find the limit.  Use L'H$\hat{o}$pital's rule where appropriate.

\input{Derivative-Compute-0050.HELP.tex}

\[\lim\limits_{x\to\infty} {{\left(-\frac{6}{x} + 1\right)}^{7 \, x} - 14}=\answer{e^{\left(-42\right)} - 14}\]
\end{problem}}%}

\latexProblemContent{
\ifVerboseLocation This is Derivative Compute Question 0050. \\ \fi
\begin{problem}

Find the limit.  Use L'H$\hat{o}$pital's rule where appropriate.

\input{Derivative-Compute-0050.HELP.tex}

\[\lim\limits_{x\to\infty} {{\left(-\frac{7}{x} + 1\right)}^{-5 \, x} + 12}=\answer{e^{35} + 12}\]
\end{problem}}%}

\latexProblemContent{
\ifVerboseLocation This is Derivative Compute Question 0050. \\ \fi
\begin{problem}

Find the limit.  Use L'H$\hat{o}$pital's rule where appropriate.

\input{Derivative-Compute-0050.HELP.tex}

\[\lim\limits_{x\to\infty} {{\left(\frac{4}{x} + 1\right)}^{-6 \, x} + 20}=\answer{e^{\left(-24\right)} + 20}\]
\end{problem}}%}

\latexProblemContent{
\ifVerboseLocation This is Derivative Compute Question 0050. \\ \fi
\begin{problem}

Find the limit.  Use L'H$\hat{o}$pital's rule where appropriate.

\input{Derivative-Compute-0050.HELP.tex}

\[\lim\limits_{x\to\infty} {{\left(-\frac{7}{x} + 1\right)}^{-5 \, x} - 13}=\answer{e^{35} - 13}\]
\end{problem}}%}

\latexProblemContent{
\ifVerboseLocation This is Derivative Compute Question 0050. \\ \fi
\begin{problem}

Find the limit.  Use L'H$\hat{o}$pital's rule where appropriate.

\input{Derivative-Compute-0050.HELP.tex}

\[\lim\limits_{x\to\infty} {{\left(-\frac{6}{x} + 1\right)}^{3 \, x}}=\answer{e^{\left(-18\right)}}\]
\end{problem}}%}

\latexProblemContent{
\ifVerboseLocation This is Derivative Compute Question 0050. \\ \fi
\begin{problem}

Find the limit.  Use L'H$\hat{o}$pital's rule where appropriate.

\input{Derivative-Compute-0050.HELP.tex}

\[\lim\limits_{x\to\infty} {{\left(-\frac{1}{x} + 1\right)}^{7 \, x} - 4}=\answer{e^{\left(-7\right)} - 4}\]
\end{problem}}%}

\latexProblemContent{
\ifVerboseLocation This is Derivative Compute Question 0050. \\ \fi
\begin{problem}

Find the limit.  Use L'H$\hat{o}$pital's rule where appropriate.

\input{Derivative-Compute-0050.HELP.tex}

\[\lim\limits_{x\to\infty} {{\left(-\frac{10}{x} + 1\right)}^{10 \, x} - 14}=\answer{e^{\left(-100\right)} - 14}\]
\end{problem}}%}

\latexProblemContent{
\ifVerboseLocation This is Derivative Compute Question 0050. \\ \fi
\begin{problem}

Find the limit.  Use L'H$\hat{o}$pital's rule where appropriate.

\input{Derivative-Compute-0050.HELP.tex}

\[\lim\limits_{x\to\infty} {{\left(\frac{6}{x} + 1\right)}^{8 \, x} - 7}=\answer{e^{48} - 7}\]
\end{problem}}%}

\latexProblemContent{
\ifVerboseLocation This is Derivative Compute Question 0050. \\ \fi
\begin{problem}

Find the limit.  Use L'H$\hat{o}$pital's rule where appropriate.

\input{Derivative-Compute-0050.HELP.tex}

\[\lim\limits_{x\to\infty} {{\left(-\frac{5}{x} + 1\right)}^{-6 \, x} + 6}=\answer{e^{30} + 6}\]
\end{problem}}%}

\latexProblemContent{
\ifVerboseLocation This is Derivative Compute Question 0050. \\ \fi
\begin{problem}

Find the limit.  Use L'H$\hat{o}$pital's rule where appropriate.

\input{Derivative-Compute-0050.HELP.tex}

\[\lim\limits_{x\to\infty} {{\left(-\frac{10}{x} + 1\right)}^{-2 \, x} - 5}=\answer{e^{20} - 5}\]
\end{problem}}%}

\latexProblemContent{
\ifVerboseLocation This is Derivative Compute Question 0050. \\ \fi
\begin{problem}

Find the limit.  Use L'H$\hat{o}$pital's rule where appropriate.

\input{Derivative-Compute-0050.HELP.tex}

\[\lim\limits_{x\to\infty} {{\left(-\frac{8}{x} + 1\right)}^{-10 \, x} - 15}=\answer{e^{80} - 15}\]
\end{problem}}%}

\latexProblemContent{
\ifVerboseLocation This is Derivative Compute Question 0050. \\ \fi
\begin{problem}

Find the limit.  Use L'H$\hat{o}$pital's rule where appropriate.

\input{Derivative-Compute-0050.HELP.tex}

\[\lim\limits_{x\to\infty} {{\left(-\frac{10}{x} + 1\right)}^{3 \, x} - 5}=\answer{e^{\left(-30\right)} - 5}\]
\end{problem}}%}

\latexProblemContent{
\ifVerboseLocation This is Derivative Compute Question 0050. \\ \fi
\begin{problem}

Find the limit.  Use L'H$\hat{o}$pital's rule where appropriate.

\input{Derivative-Compute-0050.HELP.tex}

\[\lim\limits_{x\to\infty} {{\left(-\frac{1}{x} + 1\right)}^{10 \, x} + 16}=\answer{e^{\left(-10\right)} + 16}\]
\end{problem}}%}

\latexProblemContent{
\ifVerboseLocation This is Derivative Compute Question 0050. \\ \fi
\begin{problem}

Find the limit.  Use L'H$\hat{o}$pital's rule where appropriate.

\input{Derivative-Compute-0050.HELP.tex}

\[\lim\limits_{x\to\infty} {{\left(\frac{1}{x} + 1\right)}^{2 \, x} + 2}=\answer{e^{2} + 2}\]
\end{problem}}%}

\latexProblemContent{
\ifVerboseLocation This is Derivative Compute Question 0050. \\ \fi
\begin{problem}

Find the limit.  Use L'H$\hat{o}$pital's rule where appropriate.

\input{Derivative-Compute-0050.HELP.tex}

\[\lim\limits_{x\to\infty} {{\left(\frac{4}{x} + 1\right)}^{4 \, x} + 10}=\answer{e^{16} + 10}\]
\end{problem}}%}

\latexProblemContent{
\ifVerboseLocation This is Derivative Compute Question 0050. \\ \fi
\begin{problem}

Find the limit.  Use L'H$\hat{o}$pital's rule where appropriate.

\input{Derivative-Compute-0050.HELP.tex}

\[\lim\limits_{x\to\infty} {{\left(-\frac{6}{x} + 1\right)}^{-2 \, x} + 11}=\answer{e^{12} + 11}\]
\end{problem}}%}

\latexProblemContent{
\ifVerboseLocation This is Derivative Compute Question 0050. \\ \fi
\begin{problem}

Find the limit.  Use L'H$\hat{o}$pital's rule where appropriate.

\input{Derivative-Compute-0050.HELP.tex}

\[\lim\limits_{x\to\infty} {{\left(-\frac{3}{x} + 1\right)}^{10 \, x} - 16}=\answer{e^{\left(-30\right)} - 16}\]
\end{problem}}%}

\latexProblemContent{
\ifVerboseLocation This is Derivative Compute Question 0050. \\ \fi
\begin{problem}

Find the limit.  Use L'H$\hat{o}$pital's rule where appropriate.

\input{Derivative-Compute-0050.HELP.tex}

\[\lim\limits_{x\to\infty} {{\left(-\frac{6}{x} + 1\right)}^{5 \, x} - 6}=\answer{e^{\left(-30\right)} - 6}\]
\end{problem}}%}

\latexProblemContent{
\ifVerboseLocation This is Derivative Compute Question 0050. \\ \fi
\begin{problem}

Find the limit.  Use L'H$\hat{o}$pital's rule where appropriate.

\input{Derivative-Compute-0050.HELP.tex}

\[\lim\limits_{x\to\infty} {{\left(-\frac{1}{x} + 1\right)}^{-9 \, x} - 11}=\answer{e^{9} - 11}\]
\end{problem}}%}

\latexProblemContent{
\ifVerboseLocation This is Derivative Compute Question 0050. \\ \fi
\begin{problem}

Find the limit.  Use L'H$\hat{o}$pital's rule where appropriate.

\input{Derivative-Compute-0050.HELP.tex}

\[\lim\limits_{x\to\infty} {{\left(-\frac{1}{x} + 1\right)}^{-x} - 15}=\answer{e - 15}\]
\end{problem}}%}

\latexProblemContent{
\ifVerboseLocation This is Derivative Compute Question 0050. \\ \fi
\begin{problem}

Find the limit.  Use L'H$\hat{o}$pital's rule where appropriate.

\input{Derivative-Compute-0050.HELP.tex}

\[\lim\limits_{x\to\infty} {{\left(-\frac{3}{x} + 1\right)}^{-10 \, x} - 8}=\answer{e^{30} - 8}\]
\end{problem}}%}

\latexProblemContent{
\ifVerboseLocation This is Derivative Compute Question 0050. \\ \fi
\begin{problem}

Find the limit.  Use L'H$\hat{o}$pital's rule where appropriate.

\input{Derivative-Compute-0050.HELP.tex}

\[\lim\limits_{x\to\infty} {{\left(\frac{9}{x} + 1\right)}^{-8 \, x} - 3}=\answer{e^{\left(-72\right)} - 3}\]
\end{problem}}%}

\latexProblemContent{
\ifVerboseLocation This is Derivative Compute Question 0050. \\ \fi
\begin{problem}

Find the limit.  Use L'H$\hat{o}$pital's rule where appropriate.

\input{Derivative-Compute-0050.HELP.tex}

\[\lim\limits_{x\to\infty} {{\left(\frac{10}{x} + 1\right)}^{8 \, x} + 7}=\answer{e^{80} + 7}\]
\end{problem}}%}

\latexProblemContent{
\ifVerboseLocation This is Derivative Compute Question 0050. \\ \fi
\begin{problem}

Find the limit.  Use L'H$\hat{o}$pital's rule where appropriate.

\input{Derivative-Compute-0050.HELP.tex}

\[\lim\limits_{x\to\infty} {{\left(\frac{6}{x} + 1\right)}^{-x} - 19}=\answer{e^{\left(-6\right)} - 19}\]
\end{problem}}%}

\latexProblemContent{
\ifVerboseLocation This is Derivative Compute Question 0050. \\ \fi
\begin{problem}

Find the limit.  Use L'H$\hat{o}$pital's rule where appropriate.

\input{Derivative-Compute-0050.HELP.tex}

\[\lim\limits_{x\to\infty} {{\left(-\frac{1}{x} + 1\right)}^{-7 \, x} + 15}=\answer{e^{7} + 15}\]
\end{problem}}%}

\latexProblemContent{
\ifVerboseLocation This is Derivative Compute Question 0050. \\ \fi
\begin{problem}

Find the limit.  Use L'H$\hat{o}$pital's rule where appropriate.

\input{Derivative-Compute-0050.HELP.tex}

\[\lim\limits_{x\to\infty} {{\left(-\frac{6}{x} + 1\right)}^{7 \, x} + 5}=\answer{e^{\left(-42\right)} + 5}\]
\end{problem}}%}

\latexProblemContent{
\ifVerboseLocation This is Derivative Compute Question 0050. \\ \fi
\begin{problem}

Find the limit.  Use L'H$\hat{o}$pital's rule where appropriate.

\input{Derivative-Compute-0050.HELP.tex}

\[\lim\limits_{x\to\infty} {{\left(-\frac{4}{x} + 1\right)}^{5 \, x} + 20}=\answer{e^{\left(-20\right)} + 20}\]
\end{problem}}%}

\latexProblemContent{
\ifVerboseLocation This is Derivative Compute Question 0050. \\ \fi
\begin{problem}

Find the limit.  Use L'H$\hat{o}$pital's rule where appropriate.

\input{Derivative-Compute-0050.HELP.tex}

\[\lim\limits_{x\to\infty} {{\left(-\frac{7}{x} + 1\right)}^{6 \, x} + 14}=\answer{e^{\left(-42\right)} + 14}\]
\end{problem}}%}

\latexProblemContent{
\ifVerboseLocation This is Derivative Compute Question 0050. \\ \fi
\begin{problem}

Find the limit.  Use L'H$\hat{o}$pital's rule where appropriate.

\input{Derivative-Compute-0050.HELP.tex}

\[\lim\limits_{x\to\infty} {{\left(\frac{10}{x} + 1\right)}^{2 \, x} + 19}=\answer{e^{20} + 19}\]
\end{problem}}%}

\latexProblemContent{
\ifVerboseLocation This is Derivative Compute Question 0050. \\ \fi
\begin{problem}

Find the limit.  Use L'H$\hat{o}$pital's rule where appropriate.

\input{Derivative-Compute-0050.HELP.tex}

\[\lim\limits_{x\to\infty} {{\left(\frac{7}{x} + 1\right)}^{-4 \, x} + 8}=\answer{e^{\left(-28\right)} + 8}\]
\end{problem}}%}

\latexProblemContent{
\ifVerboseLocation This is Derivative Compute Question 0050. \\ \fi
\begin{problem}

Find the limit.  Use L'H$\hat{o}$pital's rule where appropriate.

\input{Derivative-Compute-0050.HELP.tex}

\[\lim\limits_{x\to\infty} {{\left(\frac{7}{x} + 1\right)}^{-4 \, x} + 1}=\answer{e^{\left(-28\right)} + 1}\]
\end{problem}}%}

\latexProblemContent{
\ifVerboseLocation This is Derivative Compute Question 0050. \\ \fi
\begin{problem}

Find the limit.  Use L'H$\hat{o}$pital's rule where appropriate.

\input{Derivative-Compute-0050.HELP.tex}

\[\lim\limits_{x\to\infty} {{\left(\frac{10}{x} + 1\right)}^{9 \, x} - 5}=\answer{e^{90} - 5}\]
\end{problem}}%}

\latexProblemContent{
\ifVerboseLocation This is Derivative Compute Question 0050. \\ \fi
\begin{problem}

Find the limit.  Use L'H$\hat{o}$pital's rule where appropriate.

\input{Derivative-Compute-0050.HELP.tex}

\[\lim\limits_{x\to\infty} {{\left(\frac{1}{x} + 1\right)}^{10 \, x}}=\answer{e^{10}}\]
\end{problem}}%}

\latexProblemContent{
\ifVerboseLocation This is Derivative Compute Question 0050. \\ \fi
\begin{problem}

Find the limit.  Use L'H$\hat{o}$pital's rule where appropriate.

\input{Derivative-Compute-0050.HELP.tex}

\[\lim\limits_{x\to\infty} {{\left(-\frac{8}{x} + 1\right)}^{10 \, x} + 6}=\answer{e^{\left(-80\right)} + 6}\]
\end{problem}}%}

\latexProblemContent{
\ifVerboseLocation This is Derivative Compute Question 0050. \\ \fi
\begin{problem}

Find the limit.  Use L'H$\hat{o}$pital's rule where appropriate.

\input{Derivative-Compute-0050.HELP.tex}

\[\lim\limits_{x\to\infty} {{\left(-\frac{2}{x} + 1\right)}^{-7 \, x} + 6}=\answer{e^{14} + 6}\]
\end{problem}}%}

\latexProblemContent{
\ifVerboseLocation This is Derivative Compute Question 0050. \\ \fi
\begin{problem}

Find the limit.  Use L'H$\hat{o}$pital's rule where appropriate.

\input{Derivative-Compute-0050.HELP.tex}

\[\lim\limits_{x\to\infty} {{\left(\frac{1}{x} + 1\right)}^{4 \, x} + 20}=\answer{e^{4} + 20}\]
\end{problem}}%}

\latexProblemContent{
\ifVerboseLocation This is Derivative Compute Question 0050. \\ \fi
\begin{problem}

Find the limit.  Use L'H$\hat{o}$pital's rule where appropriate.

\input{Derivative-Compute-0050.HELP.tex}

\[\lim\limits_{x\to\infty} {{\left(\frac{2}{x} + 1\right)}^{10 \, x} + 20}=\answer{e^{20} + 20}\]
\end{problem}}%}

\latexProblemContent{
\ifVerboseLocation This is Derivative Compute Question 0050. \\ \fi
\begin{problem}

Find the limit.  Use L'H$\hat{o}$pital's rule where appropriate.

\input{Derivative-Compute-0050.HELP.tex}

\[\lim\limits_{x\to\infty} {{\left(-\frac{1}{x} + 1\right)}^{9 \, x} - 11}=\answer{e^{\left(-9\right)} - 11}\]
\end{problem}}%}

\latexProblemContent{
\ifVerboseLocation This is Derivative Compute Question 0050. \\ \fi
\begin{problem}

Find the limit.  Use L'H$\hat{o}$pital's rule where appropriate.

\input{Derivative-Compute-0050.HELP.tex}

\[\lim\limits_{x\to\infty} {{\left(\frac{8}{x} + 1\right)}^{8 \, x} + 16}=\answer{e^{64} + 16}\]
\end{problem}}%}

\latexProblemContent{
\ifVerboseLocation This is Derivative Compute Question 0050. \\ \fi
\begin{problem}

Find the limit.  Use L'H$\hat{o}$pital's rule where appropriate.

\input{Derivative-Compute-0050.HELP.tex}

\[\lim\limits_{x\to\infty} {{\left(\frac{5}{x} + 1\right)}^{7 \, x} + 3}=\answer{e^{35} + 3}\]
\end{problem}}%}

\latexProblemContent{
\ifVerboseLocation This is Derivative Compute Question 0050. \\ \fi
\begin{problem}

Find the limit.  Use L'H$\hat{o}$pital's rule where appropriate.

\input{Derivative-Compute-0050.HELP.tex}

\[\lim\limits_{x\to\infty} {{\left(-\frac{9}{x} + 1\right)}^{-3 \, x} + 20}=\answer{e^{27} + 20}\]
\end{problem}}%}

\latexProblemContent{
\ifVerboseLocation This is Derivative Compute Question 0050. \\ \fi
\begin{problem}

Find the limit.  Use L'H$\hat{o}$pital's rule where appropriate.

\input{Derivative-Compute-0050.HELP.tex}

\[\lim\limits_{x\to\infty} {{\left(\frac{3}{x} + 1\right)}^{8 \, x} - 17}=\answer{e^{24} - 17}\]
\end{problem}}%}

\latexProblemContent{
\ifVerboseLocation This is Derivative Compute Question 0050. \\ \fi
\begin{problem}

Find the limit.  Use L'H$\hat{o}$pital's rule where appropriate.

\input{Derivative-Compute-0050.HELP.tex}

\[\lim\limits_{x\to\infty} {{\left(-\frac{10}{x} + 1\right)}^{10 \, x} - 13}=\answer{e^{\left(-100\right)} - 13}\]
\end{problem}}%}

\latexProblemContent{
\ifVerboseLocation This is Derivative Compute Question 0050. \\ \fi
\begin{problem}

Find the limit.  Use L'H$\hat{o}$pital's rule where appropriate.

\input{Derivative-Compute-0050.HELP.tex}

\[\lim\limits_{x\to\infty} {{\left(-\frac{6}{x} + 1\right)}^{-4 \, x} - 9}=\answer{e^{24} - 9}\]
\end{problem}}%}

\latexProblemContent{
\ifVerboseLocation This is Derivative Compute Question 0050. \\ \fi
\begin{problem}

Find the limit.  Use L'H$\hat{o}$pital's rule where appropriate.

\input{Derivative-Compute-0050.HELP.tex}

\[\lim\limits_{x\to\infty} {{\left(-\frac{2}{x} + 1\right)}^{5 \, x} + 8}=\answer{e^{\left(-10\right)} + 8}\]
\end{problem}}%}

\latexProblemContent{
\ifVerboseLocation This is Derivative Compute Question 0050. \\ \fi
\begin{problem}

Find the limit.  Use L'H$\hat{o}$pital's rule where appropriate.

\input{Derivative-Compute-0050.HELP.tex}

\[\lim\limits_{x\to\infty} {{\left(\frac{1}{x} + 1\right)}^{-10 \, x} - 18}=\answer{e^{\left(-10\right)} - 18}\]
\end{problem}}%}

\latexProblemContent{
\ifVerboseLocation This is Derivative Compute Question 0050. \\ \fi
\begin{problem}

Find the limit.  Use L'H$\hat{o}$pital's rule where appropriate.

\input{Derivative-Compute-0050.HELP.tex}

\[\lim\limits_{x\to\infty} {{\left(-\frac{9}{x} + 1\right)}^{-6 \, x} + 9}=\answer{e^{54} + 9}\]
\end{problem}}%}

\latexProblemContent{
\ifVerboseLocation This is Derivative Compute Question 0050. \\ \fi
\begin{problem}

Find the limit.  Use L'H$\hat{o}$pital's rule where appropriate.

\input{Derivative-Compute-0050.HELP.tex}

\[\lim\limits_{x\to\infty} {{\left(-\frac{5}{x} + 1\right)}^{-2 \, x} - 15}=\answer{e^{10} - 15}\]
\end{problem}}%}

\latexProblemContent{
\ifVerboseLocation This is Derivative Compute Question 0050. \\ \fi
\begin{problem}

Find the limit.  Use L'H$\hat{o}$pital's rule where appropriate.

\input{Derivative-Compute-0050.HELP.tex}

\[\lim\limits_{x\to\infty} {{\left(-\frac{1}{x} + 1\right)}^{-x} + 20}=\answer{e + 20}\]
\end{problem}}%}

\latexProblemContent{
\ifVerboseLocation This is Derivative Compute Question 0050. \\ \fi
\begin{problem}

Find the limit.  Use L'H$\hat{o}$pital's rule where appropriate.

\input{Derivative-Compute-0050.HELP.tex}

\[\lim\limits_{x\to\infty} {{\left(-\frac{3}{x} + 1\right)}^{10 \, x} + 5}=\answer{e^{\left(-30\right)} + 5}\]
\end{problem}}%}

\latexProblemContent{
\ifVerboseLocation This is Derivative Compute Question 0050. \\ \fi
\begin{problem}

Find the limit.  Use L'H$\hat{o}$pital's rule where appropriate.

\input{Derivative-Compute-0050.HELP.tex}

\[\lim\limits_{x\to\infty} {{\left(-\frac{3}{x} + 1\right)}^{-2 \, x} + 15}=\answer{e^{6} + 15}\]
\end{problem}}%}

\latexProblemContent{
\ifVerboseLocation This is Derivative Compute Question 0050. \\ \fi
\begin{problem}

Find the limit.  Use L'H$\hat{o}$pital's rule where appropriate.

\input{Derivative-Compute-0050.HELP.tex}

\[\lim\limits_{x\to\infty} {{\left(-\frac{5}{x} + 1\right)}^{6 \, x} - 1}=\answer{e^{\left(-30\right)} - 1}\]
\end{problem}}%}

\latexProblemContent{
\ifVerboseLocation This is Derivative Compute Question 0050. \\ \fi
\begin{problem}

Find the limit.  Use L'H$\hat{o}$pital's rule where appropriate.

\input{Derivative-Compute-0050.HELP.tex}

\[\lim\limits_{x\to\infty} {{\left(-\frac{1}{x} + 1\right)}^{-4 \, x} - 19}=\answer{e^{4} - 19}\]
\end{problem}}%}

\latexProblemContent{
\ifVerboseLocation This is Derivative Compute Question 0050. \\ \fi
\begin{problem}

Find the limit.  Use L'H$\hat{o}$pital's rule where appropriate.

\input{Derivative-Compute-0050.HELP.tex}

\[\lim\limits_{x\to\infty} {{\left(-\frac{2}{x} + 1\right)}^{-4 \, x} + 16}=\answer{e^{8} + 16}\]
\end{problem}}%}

\latexProblemContent{
\ifVerboseLocation This is Derivative Compute Question 0050. \\ \fi
\begin{problem}

Find the limit.  Use L'H$\hat{o}$pital's rule where appropriate.

\input{Derivative-Compute-0050.HELP.tex}

\[\lim\limits_{x\to\infty} {{\left(-\frac{8}{x} + 1\right)}^{10 \, x} - 15}=\answer{e^{\left(-80\right)} - 15}\]
\end{problem}}%}

\latexProblemContent{
\ifVerboseLocation This is Derivative Compute Question 0050. \\ \fi
\begin{problem}

Find the limit.  Use L'H$\hat{o}$pital's rule where appropriate.

\input{Derivative-Compute-0050.HELP.tex}

\[\lim\limits_{x\to\infty} {{\left(-\frac{7}{x} + 1\right)}^{-10 \, x} - 10}=\answer{e^{70} - 10}\]
\end{problem}}%}

\latexProblemContent{
\ifVerboseLocation This is Derivative Compute Question 0050. \\ \fi
\begin{problem}

Find the limit.  Use L'H$\hat{o}$pital's rule where appropriate.

\input{Derivative-Compute-0050.HELP.tex}

\[\lim\limits_{x\to\infty} {{\left(\frac{3}{x} + 1\right)}^{-7 \, x} - 20}=\answer{e^{\left(-21\right)} - 20}\]
\end{problem}}%}

\latexProblemContent{
\ifVerboseLocation This is Derivative Compute Question 0050. \\ \fi
\begin{problem}

Find the limit.  Use L'H$\hat{o}$pital's rule where appropriate.

\input{Derivative-Compute-0050.HELP.tex}

\[\lim\limits_{x\to\infty} {{\left(\frac{9}{x} + 1\right)}^{-9 \, x} - 9}=\answer{e^{\left(-81\right)} - 9}\]
\end{problem}}%}

\latexProblemContent{
\ifVerboseLocation This is Derivative Compute Question 0050. \\ \fi
\begin{problem}

Find the limit.  Use L'H$\hat{o}$pital's rule where appropriate.

\input{Derivative-Compute-0050.HELP.tex}

\[\lim\limits_{x\to\infty} {{\left(-\frac{7}{x} + 1\right)}^{10 \, x} - 14}=\answer{e^{\left(-70\right)} - 14}\]
\end{problem}}%}

\latexProblemContent{
\ifVerboseLocation This is Derivative Compute Question 0050. \\ \fi
\begin{problem}

Find the limit.  Use L'H$\hat{o}$pital's rule where appropriate.

\input{Derivative-Compute-0050.HELP.tex}

\[\lim\limits_{x\to\infty} {{\left(-\frac{7}{x} + 1\right)}^{2 \, x} + 7}=\answer{e^{\left(-14\right)} + 7}\]
\end{problem}}%}

\latexProblemContent{
\ifVerboseLocation This is Derivative Compute Question 0050. \\ \fi
\begin{problem}

Find the limit.  Use L'H$\hat{o}$pital's rule where appropriate.

\input{Derivative-Compute-0050.HELP.tex}

\[\lim\limits_{x\to\infty} {{\left(-\frac{10}{x} + 1\right)}^{-x} + 13}=\answer{e^{10} + 13}\]
\end{problem}}%}

\latexProblemContent{
\ifVerboseLocation This is Derivative Compute Question 0050. \\ \fi
\begin{problem}

Find the limit.  Use L'H$\hat{o}$pital's rule where appropriate.

\input{Derivative-Compute-0050.HELP.tex}

\[\lim\limits_{x\to\infty} {{\left(-\frac{1}{x} + 1\right)}^{10 \, x} - 15}=\answer{e^{\left(-10\right)} - 15}\]
\end{problem}}%}

\latexProblemContent{
\ifVerboseLocation This is Derivative Compute Question 0050. \\ \fi
\begin{problem}

Find the limit.  Use L'H$\hat{o}$pital's rule where appropriate.

\input{Derivative-Compute-0050.HELP.tex}

\[\lim\limits_{x\to\infty} {{\left(\frac{6}{x} + 1\right)}^{4 \, x} - 1}=\answer{e^{24} - 1}\]
\end{problem}}%}

\latexProblemContent{
\ifVerboseLocation This is Derivative Compute Question 0050. \\ \fi
\begin{problem}

Find the limit.  Use L'H$\hat{o}$pital's rule where appropriate.

\input{Derivative-Compute-0050.HELP.tex}

\[\lim\limits_{x\to\infty} {{\left(-\frac{5}{x} + 1\right)}^{9 \, x} + 17}=\answer{e^{\left(-45\right)} + 17}\]
\end{problem}}%}

\latexProblemContent{
\ifVerboseLocation This is Derivative Compute Question 0050. \\ \fi
\begin{problem}

Find the limit.  Use L'H$\hat{o}$pital's rule where appropriate.

\input{Derivative-Compute-0050.HELP.tex}

\[\lim\limits_{x\to\infty} {{\left(-\frac{4}{x} + 1\right)}^{-9 \, x} - 8}=\answer{e^{36} - 8}\]
\end{problem}}%}

\latexProblemContent{
\ifVerboseLocation This is Derivative Compute Question 0050. \\ \fi
\begin{problem}

Find the limit.  Use L'H$\hat{o}$pital's rule where appropriate.

\input{Derivative-Compute-0050.HELP.tex}

\[\lim\limits_{x\to\infty} {{\left(\frac{5}{x} + 1\right)}^{-4 \, x} - 3}=\answer{e^{\left(-20\right)} - 3}\]
\end{problem}}%}

\latexProblemContent{
\ifVerboseLocation This is Derivative Compute Question 0050. \\ \fi
\begin{problem}

Find the limit.  Use L'H$\hat{o}$pital's rule where appropriate.

\input{Derivative-Compute-0050.HELP.tex}

\[\lim\limits_{x\to\infty} {{\left(-\frac{4}{x} + 1\right)}^{4 \, x} - 17}=\answer{e^{\left(-16\right)} - 17}\]
\end{problem}}%}

\latexProblemContent{
\ifVerboseLocation This is Derivative Compute Question 0050. \\ \fi
\begin{problem}

Find the limit.  Use L'H$\hat{o}$pital's rule where appropriate.

\input{Derivative-Compute-0050.HELP.tex}

\[\lim\limits_{x\to\infty} {{\left(\frac{1}{x} + 1\right)}^{-x} - 20}=\answer{e^{\left(-1\right)} - 20}\]
\end{problem}}%}

\latexProblemContent{
\ifVerboseLocation This is Derivative Compute Question 0050. \\ \fi
\begin{problem}

Find the limit.  Use L'H$\hat{o}$pital's rule where appropriate.

\input{Derivative-Compute-0050.HELP.tex}

\[\lim\limits_{x\to\infty} {{\left(-\frac{1}{x} + 1\right)}^{4 \, x} + 11}=\answer{e^{\left(-4\right)} + 11}\]
\end{problem}}%}

\latexProblemContent{
\ifVerboseLocation This is Derivative Compute Question 0050. \\ \fi
\begin{problem}

Find the limit.  Use L'H$\hat{o}$pital's rule where appropriate.

\input{Derivative-Compute-0050.HELP.tex}

\[\lim\limits_{x\to\infty} {{\left(-\frac{1}{x} + 1\right)}^{9 \, x} - 18}=\answer{e^{\left(-9\right)} - 18}\]
\end{problem}}%}

\latexProblemContent{
\ifVerboseLocation This is Derivative Compute Question 0050. \\ \fi
\begin{problem}

Find the limit.  Use L'H$\hat{o}$pital's rule where appropriate.

\input{Derivative-Compute-0050.HELP.tex}

\[\lim\limits_{x\to\infty} {{\left(\frac{7}{x} + 1\right)}^{-4 \, x} - 1}=\answer{e^{\left(-28\right)} - 1}\]
\end{problem}}%}

\latexProblemContent{
\ifVerboseLocation This is Derivative Compute Question 0050. \\ \fi
\begin{problem}

Find the limit.  Use L'H$\hat{o}$pital's rule where appropriate.

\input{Derivative-Compute-0050.HELP.tex}

\[\lim\limits_{x\to\infty} {{\left(\frac{6}{x} + 1\right)}^{4 \, x} - 9}=\answer{e^{24} - 9}\]
\end{problem}}%}

\latexProblemContent{
\ifVerboseLocation This is Derivative Compute Question 0050. \\ \fi
\begin{problem}

Find the limit.  Use L'H$\hat{o}$pital's rule where appropriate.

\input{Derivative-Compute-0050.HELP.tex}

\[\lim\limits_{x\to\infty} {{\left(\frac{2}{x} + 1\right)}^{-3 \, x} + 5}=\answer{e^{\left(-6\right)} + 5}\]
\end{problem}}%}

\latexProblemContent{
\ifVerboseLocation This is Derivative Compute Question 0050. \\ \fi
\begin{problem}

Find the limit.  Use L'H$\hat{o}$pital's rule where appropriate.

\input{Derivative-Compute-0050.HELP.tex}

\[\lim\limits_{x\to\infty} {{\left(-\frac{8}{x} + 1\right)}^{-6 \, x} + 14}=\answer{e^{48} + 14}\]
\end{problem}}%}

\latexProblemContent{
\ifVerboseLocation This is Derivative Compute Question 0050. \\ \fi
\begin{problem}

Find the limit.  Use L'H$\hat{o}$pital's rule where appropriate.

\input{Derivative-Compute-0050.HELP.tex}

\[\lim\limits_{x\to\infty} {{\left(-\frac{6}{x} + 1\right)}^{-7 \, x} + 6}=\answer{e^{42} + 6}\]
\end{problem}}%}

\latexProblemContent{
\ifVerboseLocation This is Derivative Compute Question 0050. \\ \fi
\begin{problem}

Find the limit.  Use L'H$\hat{o}$pital's rule where appropriate.

\input{Derivative-Compute-0050.HELP.tex}

\[\lim\limits_{x\to\infty} {{\left(-\frac{2}{x} + 1\right)}^{3 \, x} - 10}=\answer{e^{\left(-6\right)} - 10}\]
\end{problem}}%}

\latexProblemContent{
\ifVerboseLocation This is Derivative Compute Question 0050. \\ \fi
\begin{problem}

Find the limit.  Use L'H$\hat{o}$pital's rule where appropriate.

\input{Derivative-Compute-0050.HELP.tex}

\[\lim\limits_{x\to\infty} {{\left(-\frac{6}{x} + 1\right)}^{4 \, x} - 14}=\answer{e^{\left(-24\right)} - 14}\]
\end{problem}}%}

\latexProblemContent{
\ifVerboseLocation This is Derivative Compute Question 0050. \\ \fi
\begin{problem}

Find the limit.  Use L'H$\hat{o}$pital's rule where appropriate.

\input{Derivative-Compute-0050.HELP.tex}

\[\lim\limits_{x\to\infty} {{\left(\frac{6}{x} + 1\right)}^{-x} - 7}=\answer{e^{\left(-6\right)} - 7}\]
\end{problem}}%}

\latexProblemContent{
\ifVerboseLocation This is Derivative Compute Question 0050. \\ \fi
\begin{problem}

Find the limit.  Use L'H$\hat{o}$pital's rule where appropriate.

\input{Derivative-Compute-0050.HELP.tex}

\[\lim\limits_{x\to\infty} {{\left(-\frac{3}{x} + 1\right)}^{8 \, x} - 19}=\answer{e^{\left(-24\right)} - 19}\]
\end{problem}}%}

\latexProblemContent{
\ifVerboseLocation This is Derivative Compute Question 0050. \\ \fi
\begin{problem}

Find the limit.  Use L'H$\hat{o}$pital's rule where appropriate.

\input{Derivative-Compute-0050.HELP.tex}

\[\lim\limits_{x\to\infty} {{\left(-\frac{3}{x} + 1\right)}^{-x} + 9}=\answer{e^{3} + 9}\]
\end{problem}}%}

\latexProblemContent{
\ifVerboseLocation This is Derivative Compute Question 0050. \\ \fi
\begin{problem}

Find the limit.  Use L'H$\hat{o}$pital's rule where appropriate.

\input{Derivative-Compute-0050.HELP.tex}

\[\lim\limits_{x\to\infty} {{\left(-\frac{3}{x} + 1\right)}^{2 \, x} - 7}=\answer{e^{\left(-6\right)} - 7}\]
\end{problem}}%}

\latexProblemContent{
\ifVerboseLocation This is Derivative Compute Question 0050. \\ \fi
\begin{problem}

Find the limit.  Use L'H$\hat{o}$pital's rule where appropriate.

\input{Derivative-Compute-0050.HELP.tex}

\[\lim\limits_{x\to\infty} {{\left(\frac{3}{x} + 1\right)}^{-x} - 19}=\answer{e^{\left(-3\right)} - 19}\]
\end{problem}}%}

\latexProblemContent{
\ifVerboseLocation This is Derivative Compute Question 0050. \\ \fi
\begin{problem}

Find the limit.  Use L'H$\hat{o}$pital's rule where appropriate.

\input{Derivative-Compute-0050.HELP.tex}

\[\lim\limits_{x\to\infty} {{\left(-\frac{10}{x} + 1\right)}^{-5 \, x} + 11}=\answer{e^{50} + 11}\]
\end{problem}}%}

\latexProblemContent{
\ifVerboseLocation This is Derivative Compute Question 0050. \\ \fi
\begin{problem}

Find the limit.  Use L'H$\hat{o}$pital's rule where appropriate.

\input{Derivative-Compute-0050.HELP.tex}

\[\lim\limits_{x\to\infty} {{\left(-\frac{1}{x} + 1\right)}^{-7 \, x} - 1}=\answer{e^{7} - 1}\]
\end{problem}}%}

\latexProblemContent{
\ifVerboseLocation This is Derivative Compute Question 0050. \\ \fi
\begin{problem}

Find the limit.  Use L'H$\hat{o}$pital's rule where appropriate.

\input{Derivative-Compute-0050.HELP.tex}

\[\lim\limits_{x\to\infty} {{\left(\frac{2}{x} + 1\right)}^{-10 \, x} - 3}=\answer{e^{\left(-20\right)} - 3}\]
\end{problem}}%}

\latexProblemContent{
\ifVerboseLocation This is Derivative Compute Question 0050. \\ \fi
\begin{problem}

Find the limit.  Use L'H$\hat{o}$pital's rule where appropriate.

\input{Derivative-Compute-0050.HELP.tex}

\[\lim\limits_{x\to\infty} {{\left(\frac{3}{x} + 1\right)}^{-x} + 20}=\answer{e^{\left(-3\right)} + 20}\]
\end{problem}}%}

\latexProblemContent{
\ifVerboseLocation This is Derivative Compute Question 0050. \\ \fi
\begin{problem}

Find the limit.  Use L'H$\hat{o}$pital's rule where appropriate.

\input{Derivative-Compute-0050.HELP.tex}

\[\lim\limits_{x\to\infty} {{\left(-\frac{10}{x} + 1\right)}^{-8 \, x} - 17}=\answer{e^{80} - 17}\]
\end{problem}}%}

\latexProblemContent{
\ifVerboseLocation This is Derivative Compute Question 0050. \\ \fi
\begin{problem}

Find the limit.  Use L'H$\hat{o}$pital's rule where appropriate.

\input{Derivative-Compute-0050.HELP.tex}

\[\lim\limits_{x\to\infty} {{\left(-\frac{1}{x} + 1\right)}^{-5 \, x} + 18}=\answer{e^{5} + 18}\]
\end{problem}}%}

\latexProblemContent{
\ifVerboseLocation This is Derivative Compute Question 0050. \\ \fi
\begin{problem}

Find the limit.  Use L'H$\hat{o}$pital's rule where appropriate.

\input{Derivative-Compute-0050.HELP.tex}

\[\lim\limits_{x\to\infty} {{\left(-\frac{1}{x} + 1\right)}^{-9 \, x} - 19}=\answer{e^{9} - 19}\]
\end{problem}}%}

\latexProblemContent{
\ifVerboseLocation This is Derivative Compute Question 0050. \\ \fi
\begin{problem}

Find the limit.  Use L'H$\hat{o}$pital's rule where appropriate.

\input{Derivative-Compute-0050.HELP.tex}

\[\lim\limits_{x\to\infty} {{\left(\frac{9}{x} + 1\right)}^{7 \, x} - 4}=\answer{e^{63} - 4}\]
\end{problem}}%}

\latexProblemContent{
\ifVerboseLocation This is Derivative Compute Question 0050. \\ \fi
\begin{problem}

Find the limit.  Use L'H$\hat{o}$pital's rule where appropriate.

\input{Derivative-Compute-0050.HELP.tex}

\[\lim\limits_{x\to\infty} {{\left(\frac{3}{x} + 1\right)}^{10 \, x} - 17}=\answer{e^{30} - 17}\]
\end{problem}}%}

\latexProblemContent{
\ifVerboseLocation This is Derivative Compute Question 0050. \\ \fi
\begin{problem}

Find the limit.  Use L'H$\hat{o}$pital's rule where appropriate.

\input{Derivative-Compute-0050.HELP.tex}

\[\lim\limits_{x\to\infty} {{\left(\frac{2}{x} + 1\right)}^{x} + 11}=\answer{e^{2} + 11}\]
\end{problem}}%}

\latexProblemContent{
\ifVerboseLocation This is Derivative Compute Question 0050. \\ \fi
\begin{problem}

Find the limit.  Use L'H$\hat{o}$pital's rule where appropriate.

\input{Derivative-Compute-0050.HELP.tex}

\[\lim\limits_{x\to\infty} {{\left(-\frac{10}{x} + 1\right)}^{5 \, x} - 18}=\answer{e^{\left(-50\right)} - 18}\]
\end{problem}}%}

\latexProblemContent{
\ifVerboseLocation This is Derivative Compute Question 0050. \\ \fi
\begin{problem}

Find the limit.  Use L'H$\hat{o}$pital's rule where appropriate.

\input{Derivative-Compute-0050.HELP.tex}

\[\lim\limits_{x\to\infty} {{\left(\frac{8}{x} + 1\right)}^{-3 \, x} - 1}=\answer{e^{\left(-24\right)} - 1}\]
\end{problem}}%}

\latexProblemContent{
\ifVerboseLocation This is Derivative Compute Question 0050. \\ \fi
\begin{problem}

Find the limit.  Use L'H$\hat{o}$pital's rule where appropriate.

\input{Derivative-Compute-0050.HELP.tex}

\[\lim\limits_{x\to\infty} {{\left(-\frac{2}{x} + 1\right)}^{-7 \, x} - 8}=\answer{e^{14} - 8}\]
\end{problem}}%}

\latexProblemContent{
\ifVerboseLocation This is Derivative Compute Question 0050. \\ \fi
\begin{problem}

Find the limit.  Use L'H$\hat{o}$pital's rule where appropriate.

\input{Derivative-Compute-0050.HELP.tex}

\[\lim\limits_{x\to\infty} {{\left(\frac{4}{x} + 1\right)}^{3 \, x} + 15}=\answer{e^{12} + 15}\]
\end{problem}}%}

\latexProblemContent{
\ifVerboseLocation This is Derivative Compute Question 0050. \\ \fi
\begin{problem}

Find the limit.  Use L'H$\hat{o}$pital's rule where appropriate.

\input{Derivative-Compute-0050.HELP.tex}

\[\lim\limits_{x\to\infty} {{\left(\frac{1}{x} + 1\right)}^{10 \, x} - 10}=\answer{e^{10} - 10}\]
\end{problem}}%}

\latexProblemContent{
\ifVerboseLocation This is Derivative Compute Question 0050. \\ \fi
\begin{problem}

Find the limit.  Use L'H$\hat{o}$pital's rule where appropriate.

\input{Derivative-Compute-0050.HELP.tex}

\[\lim\limits_{x\to\infty} {{\left(-\frac{5}{x} + 1\right)}^{-8 \, x} - 14}=\answer{e^{40} - 14}\]
\end{problem}}%}

\latexProblemContent{
\ifVerboseLocation This is Derivative Compute Question 0050. \\ \fi
\begin{problem}

Find the limit.  Use L'H$\hat{o}$pital's rule where appropriate.

\input{Derivative-Compute-0050.HELP.tex}

\[\lim\limits_{x\to\infty} {{\left(-\frac{10}{x} + 1\right)}^{4 \, x} - 7}=\answer{e^{\left(-40\right)} - 7}\]
\end{problem}}%}

\latexProblemContent{
\ifVerboseLocation This is Derivative Compute Question 0050. \\ \fi
\begin{problem}

Find the limit.  Use L'H$\hat{o}$pital's rule where appropriate.

\input{Derivative-Compute-0050.HELP.tex}

\[\lim\limits_{x\to\infty} {{\left(-\frac{5}{x} + 1\right)}^{-x} + 15}=\answer{e^{5} + 15}\]
\end{problem}}%}

\latexProblemContent{
\ifVerboseLocation This is Derivative Compute Question 0050. \\ \fi
\begin{problem}

Find the limit.  Use L'H$\hat{o}$pital's rule where appropriate.

\input{Derivative-Compute-0050.HELP.tex}

\[\lim\limits_{x\to\infty} {{\left(\frac{4}{x} + 1\right)}^{9 \, x} - 13}=\answer{e^{36} - 13}\]
\end{problem}}%}

\latexProblemContent{
\ifVerboseLocation This is Derivative Compute Question 0050. \\ \fi
\begin{problem}

Find the limit.  Use L'H$\hat{o}$pital's rule where appropriate.

\input{Derivative-Compute-0050.HELP.tex}

\[\lim\limits_{x\to\infty} {{\left(\frac{5}{x} + 1\right)}^{-8 \, x} + 13}=\answer{e^{\left(-40\right)} + 13}\]
\end{problem}}%}

\latexProblemContent{
\ifVerboseLocation This is Derivative Compute Question 0050. \\ \fi
\begin{problem}

Find the limit.  Use L'H$\hat{o}$pital's rule where appropriate.

\input{Derivative-Compute-0050.HELP.tex}

\[\lim\limits_{x\to\infty} {{\left(-\frac{9}{x} + 1\right)}^{-x} - 18}=\answer{e^{9} - 18}\]
\end{problem}}%}

\latexProblemContent{
\ifVerboseLocation This is Derivative Compute Question 0050. \\ \fi
\begin{problem}

Find the limit.  Use L'H$\hat{o}$pital's rule where appropriate.

\input{Derivative-Compute-0050.HELP.tex}

\[\lim\limits_{x\to\infty} {{\left(-\frac{3}{x} + 1\right)}^{2 \, x} - 4}=\answer{e^{\left(-6\right)} - 4}\]
\end{problem}}%}

\latexProblemContent{
\ifVerboseLocation This is Derivative Compute Question 0050. \\ \fi
\begin{problem}

Find the limit.  Use L'H$\hat{o}$pital's rule where appropriate.

\input{Derivative-Compute-0050.HELP.tex}

\[\lim\limits_{x\to\infty} {{\left(-\frac{1}{x} + 1\right)}^{-2 \, x} - 10}=\answer{e^{2} - 10}\]
\end{problem}}%}

\latexProblemContent{
\ifVerboseLocation This is Derivative Compute Question 0050. \\ \fi
\begin{problem}

Find the limit.  Use L'H$\hat{o}$pital's rule where appropriate.

\input{Derivative-Compute-0050.HELP.tex}

\[\lim\limits_{x\to\infty} {{\left(\frac{10}{x} + 1\right)}^{3 \, x} + 9}=\answer{e^{30} + 9}\]
\end{problem}}%}

\latexProblemContent{
\ifVerboseLocation This is Derivative Compute Question 0050. \\ \fi
\begin{problem}

Find the limit.  Use L'H$\hat{o}$pital's rule where appropriate.

\input{Derivative-Compute-0050.HELP.tex}

\[\lim\limits_{x\to\infty} {{\left(-\frac{10}{x} + 1\right)}^{-2 \, x} + 1}=\answer{e^{20} + 1}\]
\end{problem}}%}

\latexProblemContent{
\ifVerboseLocation This is Derivative Compute Question 0050. \\ \fi
\begin{problem}

Find the limit.  Use L'H$\hat{o}$pital's rule where appropriate.

\input{Derivative-Compute-0050.HELP.tex}

\[\lim\limits_{x\to\infty} {{\left(\frac{5}{x} + 1\right)}^{-7 \, x} + 12}=\answer{e^{\left(-35\right)} + 12}\]
\end{problem}}%}

\latexProblemContent{
\ifVerboseLocation This is Derivative Compute Question 0050. \\ \fi
\begin{problem}

Find the limit.  Use L'H$\hat{o}$pital's rule where appropriate.

\input{Derivative-Compute-0050.HELP.tex}

\[\lim\limits_{x\to\infty} {{\left(\frac{4}{x} + 1\right)}^{x} - 20}=\answer{e^{4} - 20}\]
\end{problem}}%}

\latexProblemContent{
\ifVerboseLocation This is Derivative Compute Question 0050. \\ \fi
\begin{problem}

Find the limit.  Use L'H$\hat{o}$pital's rule where appropriate.

\input{Derivative-Compute-0050.HELP.tex}

\[\lim\limits_{x\to\infty} {{\left(-\frac{1}{x} + 1\right)}^{10 \, x} + 2}=\answer{e^{\left(-10\right)} + 2}\]
\end{problem}}%}

\latexProblemContent{
\ifVerboseLocation This is Derivative Compute Question 0050. \\ \fi
\begin{problem}

Find the limit.  Use L'H$\hat{o}$pital's rule where appropriate.

\input{Derivative-Compute-0050.HELP.tex}

\[\lim\limits_{x\to\infty} {{\left(\frac{2}{x} + 1\right)}^{-5 \, x} - 18}=\answer{e^{\left(-10\right)} - 18}\]
\end{problem}}%}

\latexProblemContent{
\ifVerboseLocation This is Derivative Compute Question 0050. \\ \fi
\begin{problem}

Find the limit.  Use L'H$\hat{o}$pital's rule where appropriate.

\input{Derivative-Compute-0050.HELP.tex}

\[\lim\limits_{x\to\infty} {{\left(\frac{2}{x} + 1\right)}^{9 \, x} - 16}=\answer{e^{18} - 16}\]
\end{problem}}%}

\latexProblemContent{
\ifVerboseLocation This is Derivative Compute Question 0050. \\ \fi
\begin{problem}

Find the limit.  Use L'H$\hat{o}$pital's rule where appropriate.

\input{Derivative-Compute-0050.HELP.tex}

\[\lim\limits_{x\to\infty} {{\left(\frac{10}{x} + 1\right)}^{-4 \, x} - 13}=\answer{e^{\left(-40\right)} - 13}\]
\end{problem}}%}

\latexProblemContent{
\ifVerboseLocation This is Derivative Compute Question 0050. \\ \fi
\begin{problem}

Find the limit.  Use L'H$\hat{o}$pital's rule where appropriate.

\input{Derivative-Compute-0050.HELP.tex}

\[\lim\limits_{x\to\infty} {{\left(\frac{8}{x} + 1\right)}^{-4 \, x} + 8}=\answer{e^{\left(-32\right)} + 8}\]
\end{problem}}%}

\latexProblemContent{
\ifVerboseLocation This is Derivative Compute Question 0050. \\ \fi
\begin{problem}

Find the limit.  Use L'H$\hat{o}$pital's rule where appropriate.

\input{Derivative-Compute-0050.HELP.tex}

\[\lim\limits_{x\to\infty} {{\left(-\frac{5}{x} + 1\right)}^{-5 \, x} - 15}=\answer{e^{25} - 15}\]
\end{problem}}%}

\latexProblemContent{
\ifVerboseLocation This is Derivative Compute Question 0050. \\ \fi
\begin{problem}

Find the limit.  Use L'H$\hat{o}$pital's rule where appropriate.

\input{Derivative-Compute-0050.HELP.tex}

\[\lim\limits_{x\to\infty} {{\left(\frac{2}{x} + 1\right)}^{-x} - 10}=\answer{e^{\left(-2\right)} - 10}\]
\end{problem}}%}

\latexProblemContent{
\ifVerboseLocation This is Derivative Compute Question 0050. \\ \fi
\begin{problem}

Find the limit.  Use L'H$\hat{o}$pital's rule where appropriate.

\input{Derivative-Compute-0050.HELP.tex}

\[\lim\limits_{x\to\infty} {{\left(\frac{9}{x} + 1\right)}^{-8 \, x} + 5}=\answer{e^{\left(-72\right)} + 5}\]
\end{problem}}%}

\latexProblemContent{
\ifVerboseLocation This is Derivative Compute Question 0050. \\ \fi
\begin{problem}

Find the limit.  Use L'H$\hat{o}$pital's rule where appropriate.

\input{Derivative-Compute-0050.HELP.tex}

\[\lim\limits_{x\to\infty} {{\left(\frac{8}{x} + 1\right)}^{-7 \, x} - 6}=\answer{e^{\left(-56\right)} - 6}\]
\end{problem}}%}

\latexProblemContent{
\ifVerboseLocation This is Derivative Compute Question 0050. \\ \fi
\begin{problem}

Find the limit.  Use L'H$\hat{o}$pital's rule where appropriate.

\input{Derivative-Compute-0050.HELP.tex}

\[\lim\limits_{x\to\infty} {{\left(\frac{9}{x} + 1\right)}^{-5 \, x} - 14}=\answer{e^{\left(-45\right)} - 14}\]
\end{problem}}%}

\latexProblemContent{
\ifVerboseLocation This is Derivative Compute Question 0050. \\ \fi
\begin{problem}

Find the limit.  Use L'H$\hat{o}$pital's rule where appropriate.

\input{Derivative-Compute-0050.HELP.tex}

\[\lim\limits_{x\to\infty} {{\left(-\frac{4}{x} + 1\right)}^{5 \, x} - 17}=\answer{e^{\left(-20\right)} - 17}\]
\end{problem}}%}

\latexProblemContent{
\ifVerboseLocation This is Derivative Compute Question 0050. \\ \fi
\begin{problem}

Find the limit.  Use L'H$\hat{o}$pital's rule where appropriate.

\input{Derivative-Compute-0050.HELP.tex}

\[\lim\limits_{x\to\infty} {{\left(\frac{1}{x} + 1\right)}^{-4 \, x} - 5}=\answer{e^{\left(-4\right)} - 5}\]
\end{problem}}%}

\latexProblemContent{
\ifVerboseLocation This is Derivative Compute Question 0050. \\ \fi
\begin{problem}

Find the limit.  Use L'H$\hat{o}$pital's rule where appropriate.

\input{Derivative-Compute-0050.HELP.tex}

\[\lim\limits_{x\to\infty} {{\left(-\frac{2}{x} + 1\right)}^{3 \, x} + 11}=\answer{e^{\left(-6\right)} + 11}\]
\end{problem}}%}

\latexProblemContent{
\ifVerboseLocation This is Derivative Compute Question 0050. \\ \fi
\begin{problem}

Find the limit.  Use L'H$\hat{o}$pital's rule where appropriate.

\input{Derivative-Compute-0050.HELP.tex}

\[\lim\limits_{x\to\infty} {{\left(-\frac{9}{x} + 1\right)}^{4 \, x} - 20}=\answer{e^{\left(-36\right)} - 20}\]
\end{problem}}%}

\latexProblemContent{
\ifVerboseLocation This is Derivative Compute Question 0050. \\ \fi
\begin{problem}

Find the limit.  Use L'H$\hat{o}$pital's rule where appropriate.

\input{Derivative-Compute-0050.HELP.tex}

\[\lim\limits_{x\to\infty} {{\left(\frac{2}{x} + 1\right)}^{-10 \, x} - 8}=\answer{e^{\left(-20\right)} - 8}\]
\end{problem}}%}

\latexProblemContent{
\ifVerboseLocation This is Derivative Compute Question 0050. \\ \fi
\begin{problem}

Find the limit.  Use L'H$\hat{o}$pital's rule where appropriate.

\input{Derivative-Compute-0050.HELP.tex}

\[\lim\limits_{x\to\infty} {{\left(\frac{6}{x} + 1\right)}^{2 \, x} + 2}=\answer{e^{12} + 2}\]
\end{problem}}%}

\latexProblemContent{
\ifVerboseLocation This is Derivative Compute Question 0050. \\ \fi
\begin{problem}

Find the limit.  Use L'H$\hat{o}$pital's rule where appropriate.

\input{Derivative-Compute-0050.HELP.tex}

\[\lim\limits_{x\to\infty} {{\left(-\frac{10}{x} + 1\right)}^{5 \, x} + 4}=\answer{e^{\left(-50\right)} + 4}\]
\end{problem}}%}

\latexProblemContent{
\ifVerboseLocation This is Derivative Compute Question 0050. \\ \fi
\begin{problem}

Find the limit.  Use L'H$\hat{o}$pital's rule where appropriate.

\input{Derivative-Compute-0050.HELP.tex}

\[\lim\limits_{x\to\infty} {{\left(-\frac{9}{x} + 1\right)}^{7 \, x} - 13}=\answer{e^{\left(-63\right)} - 13}\]
\end{problem}}%}

\latexProblemContent{
\ifVerboseLocation This is Derivative Compute Question 0050. \\ \fi
\begin{problem}

Find the limit.  Use L'H$\hat{o}$pital's rule where appropriate.

\input{Derivative-Compute-0050.HELP.tex}

\[\lim\limits_{x\to\infty} {{\left(-\frac{3}{x} + 1\right)}^{-5 \, x} + 14}=\answer{e^{15} + 14}\]
\end{problem}}%}

\latexProblemContent{
\ifVerboseLocation This is Derivative Compute Question 0050. \\ \fi
\begin{problem}

Find the limit.  Use L'H$\hat{o}$pital's rule where appropriate.

\input{Derivative-Compute-0050.HELP.tex}

\[\lim\limits_{x\to\infty} {{\left(\frac{2}{x} + 1\right)}^{x} - 6}=\answer{e^{2} - 6}\]
\end{problem}}%}

\latexProblemContent{
\ifVerboseLocation This is Derivative Compute Question 0050. \\ \fi
\begin{problem}

Find the limit.  Use L'H$\hat{o}$pital's rule where appropriate.

\input{Derivative-Compute-0050.HELP.tex}

\[\lim\limits_{x\to\infty} {{\left(-\frac{2}{x} + 1\right)}^{-3 \, x} + 2}=\answer{e^{6} + 2}\]
\end{problem}}%}

\latexProblemContent{
\ifVerboseLocation This is Derivative Compute Question 0050. \\ \fi
\begin{problem}

Find the limit.  Use L'H$\hat{o}$pital's rule where appropriate.

\input{Derivative-Compute-0050.HELP.tex}

\[\lim\limits_{x\to\infty} {{\left(\frac{3}{x} + 1\right)}^{3 \, x} + 9}=\answer{e^{9} + 9}\]
\end{problem}}%}

\latexProblemContent{
\ifVerboseLocation This is Derivative Compute Question 0050. \\ \fi
\begin{problem}

Find the limit.  Use L'H$\hat{o}$pital's rule where appropriate.

\input{Derivative-Compute-0050.HELP.tex}

\[\lim\limits_{x\to\infty} {{\left(-\frac{2}{x} + 1\right)}^{-x} + 15}=\answer{e^{2} + 15}\]
\end{problem}}%}

\latexProblemContent{
\ifVerboseLocation This is Derivative Compute Question 0050. \\ \fi
\begin{problem}

Find the limit.  Use L'H$\hat{o}$pital's rule where appropriate.

\input{Derivative-Compute-0050.HELP.tex}

\[\lim\limits_{x\to\infty} {{\left(\frac{10}{x} + 1\right)}^{-6 \, x} - 2}=\answer{e^{\left(-60\right)} - 2}\]
\end{problem}}%}

\latexProblemContent{
\ifVerboseLocation This is Derivative Compute Question 0050. \\ \fi
\begin{problem}

Find the limit.  Use L'H$\hat{o}$pital's rule where appropriate.

\input{Derivative-Compute-0050.HELP.tex}

\[\lim\limits_{x\to\infty} {{\left(\frac{10}{x} + 1\right)}^{2 \, x} - 3}=\answer{e^{20} - 3}\]
\end{problem}}%}

\latexProblemContent{
\ifVerboseLocation This is Derivative Compute Question 0050. \\ \fi
\begin{problem}

Find the limit.  Use L'H$\hat{o}$pital's rule where appropriate.

\input{Derivative-Compute-0050.HELP.tex}

\[\lim\limits_{x\to\infty} {{\left(\frac{10}{x} + 1\right)}^{8 \, x} - 18}=\answer{e^{80} - 18}\]
\end{problem}}%}

\latexProblemContent{
\ifVerboseLocation This is Derivative Compute Question 0050. \\ \fi
\begin{problem}

Find the limit.  Use L'H$\hat{o}$pital's rule where appropriate.

\input{Derivative-Compute-0050.HELP.tex}

\[\lim\limits_{x\to\infty} {{\left(\frac{10}{x} + 1\right)}^{-4 \, x} - 17}=\answer{e^{\left(-40\right)} - 17}\]
\end{problem}}%}

\latexProblemContent{
\ifVerboseLocation This is Derivative Compute Question 0050. \\ \fi
\begin{problem}

Find the limit.  Use L'H$\hat{o}$pital's rule where appropriate.

\input{Derivative-Compute-0050.HELP.tex}

\[\lim\limits_{x\to\infty} {{\left(\frac{5}{x} + 1\right)}^{-x} + 20}=\answer{e^{\left(-5\right)} + 20}\]
\end{problem}}%}

\latexProblemContent{
\ifVerboseLocation This is Derivative Compute Question 0050. \\ \fi
\begin{problem}

Find the limit.  Use L'H$\hat{o}$pital's rule where appropriate.

\input{Derivative-Compute-0050.HELP.tex}

\[\lim\limits_{x\to\infty} {{\left(\frac{4}{x} + 1\right)}^{-2 \, x} + 20}=\answer{e^{\left(-8\right)} + 20}\]
\end{problem}}%}

\latexProblemContent{
\ifVerboseLocation This is Derivative Compute Question 0050. \\ \fi
\begin{problem}

Find the limit.  Use L'H$\hat{o}$pital's rule where appropriate.

\input{Derivative-Compute-0050.HELP.tex}

\[\lim\limits_{x\to\infty} {{\left(-\frac{1}{x} + 1\right)}^{-9 \, x} + 12}=\answer{e^{9} + 12}\]
\end{problem}}%}

\latexProblemContent{
\ifVerboseLocation This is Derivative Compute Question 0050. \\ \fi
\begin{problem}

Find the limit.  Use L'H$\hat{o}$pital's rule where appropriate.

\input{Derivative-Compute-0050.HELP.tex}

\[\lim\limits_{x\to\infty} {{\left(\frac{6}{x} + 1\right)}^{-6 \, x} - 8}=\answer{e^{\left(-36\right)} - 8}\]
\end{problem}}%}

\latexProblemContent{
\ifVerboseLocation This is Derivative Compute Question 0050. \\ \fi
\begin{problem}

Find the limit.  Use L'H$\hat{o}$pital's rule where appropriate.

\input{Derivative-Compute-0050.HELP.tex}

\[\lim\limits_{x\to\infty} {{\left(\frac{3}{x} + 1\right)}^{x} + 17}=\answer{e^{3} + 17}\]
\end{problem}}%}

\latexProblemContent{
\ifVerboseLocation This is Derivative Compute Question 0050. \\ \fi
\begin{problem}

Find the limit.  Use L'H$\hat{o}$pital's rule where appropriate.

\input{Derivative-Compute-0050.HELP.tex}

\[\lim\limits_{x\to\infty} {{\left(\frac{2}{x} + 1\right)}^{7 \, x} - 2}=\answer{e^{14} - 2}\]
\end{problem}}%}

\latexProblemContent{
\ifVerboseLocation This is Derivative Compute Question 0050. \\ \fi
\begin{problem}

Find the limit.  Use L'H$\hat{o}$pital's rule where appropriate.

\input{Derivative-Compute-0050.HELP.tex}

\[\lim\limits_{x\to\infty} {{\left(-\frac{3}{x} + 1\right)}^{-9 \, x} - 14}=\answer{e^{27} - 14}\]
\end{problem}}%}

\latexProblemContent{
\ifVerboseLocation This is Derivative Compute Question 0050. \\ \fi
\begin{problem}

Find the limit.  Use L'H$\hat{o}$pital's rule where appropriate.

\input{Derivative-Compute-0050.HELP.tex}

\[\lim\limits_{x\to\infty} {{\left(-\frac{4}{x} + 1\right)}^{-8 \, x} + 12}=\answer{e^{32} + 12}\]
\end{problem}}%}

\latexProblemContent{
\ifVerboseLocation This is Derivative Compute Question 0050. \\ \fi
\begin{problem}

Find the limit.  Use L'H$\hat{o}$pital's rule where appropriate.

\input{Derivative-Compute-0050.HELP.tex}

\[\lim\limits_{x\to\infty} {{\left(\frac{8}{x} + 1\right)}^{5 \, x} + 20}=\answer{e^{40} + 20}\]
\end{problem}}%}

\latexProblemContent{
\ifVerboseLocation This is Derivative Compute Question 0050. \\ \fi
\begin{problem}

Find the limit.  Use L'H$\hat{o}$pital's rule where appropriate.

\input{Derivative-Compute-0050.HELP.tex}

\[\lim\limits_{x\to\infty} {{\left(\frac{3}{x} + 1\right)}^{2 \, x} + 17}=\answer{e^{6} + 17}\]
\end{problem}}%}

\latexProblemContent{
\ifVerboseLocation This is Derivative Compute Question 0050. \\ \fi
\begin{problem}

Find the limit.  Use L'H$\hat{o}$pital's rule where appropriate.

\input{Derivative-Compute-0050.HELP.tex}

\[\lim\limits_{x\to\infty} {{\left(\frac{9}{x} + 1\right)}^{4 \, x} - 9}=\answer{e^{36} - 9}\]
\end{problem}}%}

\latexProblemContent{
\ifVerboseLocation This is Derivative Compute Question 0050. \\ \fi
\begin{problem}

Find the limit.  Use L'H$\hat{o}$pital's rule where appropriate.

\input{Derivative-Compute-0050.HELP.tex}

\[\lim\limits_{x\to\infty} {{\left(\frac{7}{x} + 1\right)}^{-3 \, x} - 19}=\answer{e^{\left(-21\right)} - 19}\]
\end{problem}}%}

\latexProblemContent{
\ifVerboseLocation This is Derivative Compute Question 0050. \\ \fi
\begin{problem}

Find the limit.  Use L'H$\hat{o}$pital's rule where appropriate.

\input{Derivative-Compute-0050.HELP.tex}

\[\lim\limits_{x\to\infty} {{\left(-\frac{10}{x} + 1\right)}^{-5 \, x} + 17}=\answer{e^{50} + 17}\]
\end{problem}}%}

\latexProblemContent{
\ifVerboseLocation This is Derivative Compute Question 0050. \\ \fi
\begin{problem}

Find the limit.  Use L'H$\hat{o}$pital's rule where appropriate.

\input{Derivative-Compute-0050.HELP.tex}

\[\lim\limits_{x\to\infty} {{\left(\frac{2}{x} + 1\right)}^{7 \, x} + 17}=\answer{e^{14} + 17}\]
\end{problem}}%}

\latexProblemContent{
\ifVerboseLocation This is Derivative Compute Question 0050. \\ \fi
\begin{problem}

Find the limit.  Use L'H$\hat{o}$pital's rule where appropriate.

\input{Derivative-Compute-0050.HELP.tex}

\[\lim\limits_{x\to\infty} {{\left(-\frac{7}{x} + 1\right)}^{-5 \, x} + 1}=\answer{e^{35} + 1}\]
\end{problem}}%}

\latexProblemContent{
\ifVerboseLocation This is Derivative Compute Question 0050. \\ \fi
\begin{problem}

Find the limit.  Use L'H$\hat{o}$pital's rule where appropriate.

\input{Derivative-Compute-0050.HELP.tex}

\[\lim\limits_{x\to\infty} {{\left(-\frac{8}{x} + 1\right)}^{-6 \, x} - 15}=\answer{e^{48} - 15}\]
\end{problem}}%}

\latexProblemContent{
\ifVerboseLocation This is Derivative Compute Question 0050. \\ \fi
\begin{problem}

Find the limit.  Use L'H$\hat{o}$pital's rule where appropriate.

\input{Derivative-Compute-0050.HELP.tex}

\[\lim\limits_{x\to\infty} {{\left(-\frac{6}{x} + 1\right)}^{8 \, x} + 10}=\answer{e^{\left(-48\right)} + 10}\]
\end{problem}}%}

\latexProblemContent{
\ifVerboseLocation This is Derivative Compute Question 0050. \\ \fi
\begin{problem}

Find the limit.  Use L'H$\hat{o}$pital's rule where appropriate.

\input{Derivative-Compute-0050.HELP.tex}

\[\lim\limits_{x\to\infty} {{\left(\frac{1}{x} + 1\right)}^{5 \, x} - 4}=\answer{e^{5} - 4}\]
\end{problem}}%}

\latexProblemContent{
\ifVerboseLocation This is Derivative Compute Question 0050. \\ \fi
\begin{problem}

Find the limit.  Use L'H$\hat{o}$pital's rule where appropriate.

\input{Derivative-Compute-0050.HELP.tex}

\[\lim\limits_{x\to\infty} {{\left(\frac{1}{x} + 1\right)}^{5 \, x} + 19}=\answer{e^{5} + 19}\]
\end{problem}}%}

\latexProblemContent{
\ifVerboseLocation This is Derivative Compute Question 0050. \\ \fi
\begin{problem}

Find the limit.  Use L'H$\hat{o}$pital's rule where appropriate.

\input{Derivative-Compute-0050.HELP.tex}

\[\lim\limits_{x\to\infty} {{\left(-\frac{6}{x} + 1\right)}^{-x} + 16}=\answer{e^{6} + 16}\]
\end{problem}}%}

\latexProblemContent{
\ifVerboseLocation This is Derivative Compute Question 0050. \\ \fi
\begin{problem}

Find the limit.  Use L'H$\hat{o}$pital's rule where appropriate.

\input{Derivative-Compute-0050.HELP.tex}

\[\lim\limits_{x\to\infty} {{\left(-\frac{2}{x} + 1\right)}^{7 \, x} - 10}=\answer{e^{\left(-14\right)} - 10}\]
\end{problem}}%}

\latexProblemContent{
\ifVerboseLocation This is Derivative Compute Question 0050. \\ \fi
\begin{problem}

Find the limit.  Use L'H$\hat{o}$pital's rule where appropriate.

\input{Derivative-Compute-0050.HELP.tex}

\[\lim\limits_{x\to\infty} {{\left(\frac{10}{x} + 1\right)}^{5 \, x} - 3}=\answer{e^{50} - 3}\]
\end{problem}}%}

\latexProblemContent{
\ifVerboseLocation This is Derivative Compute Question 0050. \\ \fi
\begin{problem}

Find the limit.  Use L'H$\hat{o}$pital's rule where appropriate.

\input{Derivative-Compute-0050.HELP.tex}

\[\lim\limits_{x\to\infty} {{\left(\frac{9}{x} + 1\right)}^{4 \, x} + 15}=\answer{e^{36} + 15}\]
\end{problem}}%}

\latexProblemContent{
\ifVerboseLocation This is Derivative Compute Question 0050. \\ \fi
\begin{problem}

Find the limit.  Use L'H$\hat{o}$pital's rule where appropriate.

\input{Derivative-Compute-0050.HELP.tex}

\[\lim\limits_{x\to\infty} {{\left(-\frac{1}{x} + 1\right)}^{-8 \, x} + 8}=\answer{e^{8} + 8}\]
\end{problem}}%}

\latexProblemContent{
\ifVerboseLocation This is Derivative Compute Question 0050. \\ \fi
\begin{problem}

Find the limit.  Use L'H$\hat{o}$pital's rule where appropriate.

\input{Derivative-Compute-0050.HELP.tex}

\[\lim\limits_{x\to\infty} {{\left(\frac{4}{x} + 1\right)}^{4 \, x} - 4}=\answer{e^{16} - 4}\]
\end{problem}}%}

\latexProblemContent{
\ifVerboseLocation This is Derivative Compute Question 0050. \\ \fi
\begin{problem}

Find the limit.  Use L'H$\hat{o}$pital's rule where appropriate.

\input{Derivative-Compute-0050.HELP.tex}

\[\lim\limits_{x\to\infty} {{\left(\frac{4}{x} + 1\right)}^{6 \, x} - 18}=\answer{e^{24} - 18}\]
\end{problem}}%}

\latexProblemContent{
\ifVerboseLocation This is Derivative Compute Question 0050. \\ \fi
\begin{problem}

Find the limit.  Use L'H$\hat{o}$pital's rule where appropriate.

\input{Derivative-Compute-0050.HELP.tex}

\[\lim\limits_{x\to\infty} {{\left(-\frac{5}{x} + 1\right)}^{-2 \, x} - 19}=\answer{e^{10} - 19}\]
\end{problem}}%}

\latexProblemContent{
\ifVerboseLocation This is Derivative Compute Question 0050. \\ \fi
\begin{problem}

Find the limit.  Use L'H$\hat{o}$pital's rule where appropriate.

\input{Derivative-Compute-0050.HELP.tex}

\[\lim\limits_{x\to\infty} {{\left(\frac{4}{x} + 1\right)}^{7 \, x} + 6}=\answer{e^{28} + 6}\]
\end{problem}}%}

\latexProblemContent{
\ifVerboseLocation This is Derivative Compute Question 0050. \\ \fi
\begin{problem}

Find the limit.  Use L'H$\hat{o}$pital's rule where appropriate.

\input{Derivative-Compute-0050.HELP.tex}

\[\lim\limits_{x\to\infty} {{\left(-\frac{2}{x} + 1\right)}^{-10 \, x} + 8}=\answer{e^{20} + 8}\]
\end{problem}}%}

\latexProblemContent{
\ifVerboseLocation This is Derivative Compute Question 0050. \\ \fi
\begin{problem}

Find the limit.  Use L'H$\hat{o}$pital's rule where appropriate.

\input{Derivative-Compute-0050.HELP.tex}

\[\lim\limits_{x\to\infty} {{\left(-\frac{7}{x} + 1\right)}^{2 \, x} - 15}=\answer{e^{\left(-14\right)} - 15}\]
\end{problem}}%}

\latexProblemContent{
\ifVerboseLocation This is Derivative Compute Question 0050. \\ \fi
\begin{problem}

Find the limit.  Use L'H$\hat{o}$pital's rule where appropriate.

\input{Derivative-Compute-0050.HELP.tex}

\[\lim\limits_{x\to\infty} {{\left(-\frac{7}{x} + 1\right)}^{-5 \, x} - 5}=\answer{e^{35} - 5}\]
\end{problem}}%}

\latexProblemContent{
\ifVerboseLocation This is Derivative Compute Question 0050. \\ \fi
\begin{problem}

Find the limit.  Use L'H$\hat{o}$pital's rule where appropriate.

\input{Derivative-Compute-0050.HELP.tex}

\[\lim\limits_{x\to\infty} {{\left(\frac{7}{x} + 1\right)}^{2 \, x} + 17}=\answer{e^{14} + 17}\]
\end{problem}}%}

\latexProblemContent{
\ifVerboseLocation This is Derivative Compute Question 0050. \\ \fi
\begin{problem}

Find the limit.  Use L'H$\hat{o}$pital's rule where appropriate.

\input{Derivative-Compute-0050.HELP.tex}

\[\lim\limits_{x\to\infty} {{\left(-\frac{10}{x} + 1\right)}^{-3 \, x} - 11}=\answer{e^{30} - 11}\]
\end{problem}}%}

\latexProblemContent{
\ifVerboseLocation This is Derivative Compute Question 0050. \\ \fi
\begin{problem}

Find the limit.  Use L'H$\hat{o}$pital's rule where appropriate.

\input{Derivative-Compute-0050.HELP.tex}

\[\lim\limits_{x\to\infty} {{\left(-\frac{8}{x} + 1\right)}^{-3 \, x} + 20}=\answer{e^{24} + 20}\]
\end{problem}}%}

\latexProblemContent{
\ifVerboseLocation This is Derivative Compute Question 0050. \\ \fi
\begin{problem}

Find the limit.  Use L'H$\hat{o}$pital's rule where appropriate.

\input{Derivative-Compute-0050.HELP.tex}

\[\lim\limits_{x\to\infty} {{\left(-\frac{1}{x} + 1\right)}^{-6 \, x} - 11}=\answer{e^{6} - 11}\]
\end{problem}}%}

\latexProblemContent{
\ifVerboseLocation This is Derivative Compute Question 0050. \\ \fi
\begin{problem}

Find the limit.  Use L'H$\hat{o}$pital's rule where appropriate.

\input{Derivative-Compute-0050.HELP.tex}

\[\lim\limits_{x\to\infty} {{\left(-\frac{8}{x} + 1\right)}^{3 \, x} - 20}=\answer{e^{\left(-24\right)} - 20}\]
\end{problem}}%}

\latexProblemContent{
\ifVerboseLocation This is Derivative Compute Question 0050. \\ \fi
\begin{problem}

Find the limit.  Use L'H$\hat{o}$pital's rule where appropriate.

\input{Derivative-Compute-0050.HELP.tex}

\[\lim\limits_{x\to\infty} {{\left(\frac{2}{x} + 1\right)}^{7 \, x} + 5}=\answer{e^{14} + 5}\]
\end{problem}}%}

\latexProblemContent{
\ifVerboseLocation This is Derivative Compute Question 0050. \\ \fi
\begin{problem}

Find the limit.  Use L'H$\hat{o}$pital's rule where appropriate.

\input{Derivative-Compute-0050.HELP.tex}

\[\lim\limits_{x\to\infty} {{\left(\frac{6}{x} + 1\right)}^{5 \, x} - 17}=\answer{e^{30} - 17}\]
\end{problem}}%}

\latexProblemContent{
\ifVerboseLocation This is Derivative Compute Question 0050. \\ \fi
\begin{problem}

Find the limit.  Use L'H$\hat{o}$pital's rule where appropriate.

\input{Derivative-Compute-0050.HELP.tex}

\[\lim\limits_{x\to\infty} {{\left(\frac{4}{x} + 1\right)}^{8 \, x} - 15}=\answer{e^{32} - 15}\]
\end{problem}}%}

\latexProblemContent{
\ifVerboseLocation This is Derivative Compute Question 0050. \\ \fi
\begin{problem}

Find the limit.  Use L'H$\hat{o}$pital's rule where appropriate.

\input{Derivative-Compute-0050.HELP.tex}

\[\lim\limits_{x\to\infty} {{\left(-\frac{2}{x} + 1\right)}^{7 \, x} - 16}=\answer{e^{\left(-14\right)} - 16}\]
\end{problem}}%}

\latexProblemContent{
\ifVerboseLocation This is Derivative Compute Question 0050. \\ \fi
\begin{problem}

Find the limit.  Use L'H$\hat{o}$pital's rule where appropriate.

\input{Derivative-Compute-0050.HELP.tex}

\[\lim\limits_{x\to\infty} {{\left(-\frac{5}{x} + 1\right)}^{-3 \, x} + 8}=\answer{e^{15} + 8}\]
\end{problem}}%}

\latexProblemContent{
\ifVerboseLocation This is Derivative Compute Question 0050. \\ \fi
\begin{problem}

Find the limit.  Use L'H$\hat{o}$pital's rule where appropriate.

\input{Derivative-Compute-0050.HELP.tex}

\[\lim\limits_{x\to\infty} {{\left(\frac{1}{x} + 1\right)}^{2 \, x} - 10}=\answer{e^{2} - 10}\]
\end{problem}}%}

\latexProblemContent{
\ifVerboseLocation This is Derivative Compute Question 0050. \\ \fi
\begin{problem}

Find the limit.  Use L'H$\hat{o}$pital's rule where appropriate.

\input{Derivative-Compute-0050.HELP.tex}

\[\lim\limits_{x\to\infty} {{\left(\frac{8}{x} + 1\right)}^{-x} - 20}=\answer{e^{\left(-8\right)} - 20}\]
\end{problem}}%}

\latexProblemContent{
\ifVerboseLocation This is Derivative Compute Question 0050. \\ \fi
\begin{problem}

Find the limit.  Use L'H$\hat{o}$pital's rule where appropriate.

\input{Derivative-Compute-0050.HELP.tex}

\[\lim\limits_{x\to\infty} {{\left(-\frac{1}{x} + 1\right)}^{5 \, x} + 14}=\answer{e^{\left(-5\right)} + 14}\]
\end{problem}}%}

\latexProblemContent{
\ifVerboseLocation This is Derivative Compute Question 0050. \\ \fi
\begin{problem}

Find the limit.  Use L'H$\hat{o}$pital's rule where appropriate.

\input{Derivative-Compute-0050.HELP.tex}

\[\lim\limits_{x\to\infty} {{\left(\frac{3}{x} + 1\right)}^{6 \, x} + 19}=\answer{e^{18} + 19}\]
\end{problem}}%}

\latexProblemContent{
\ifVerboseLocation This is Derivative Compute Question 0050. \\ \fi
\begin{problem}

Find the limit.  Use L'H$\hat{o}$pital's rule where appropriate.

\input{Derivative-Compute-0050.HELP.tex}

\[\lim\limits_{x\to\infty} {{\left(\frac{3}{x} + 1\right)}^{5 \, x}}=\answer{e^{15}}\]
\end{problem}}%}

\latexProblemContent{
\ifVerboseLocation This is Derivative Compute Question 0050. \\ \fi
\begin{problem}

Find the limit.  Use L'H$\hat{o}$pital's rule where appropriate.

\input{Derivative-Compute-0050.HELP.tex}

\[\lim\limits_{x\to\infty} {{\left(-\frac{3}{x} + 1\right)}^{-4 \, x}}=\answer{e^{12}}\]
\end{problem}}%}

\latexProblemContent{
\ifVerboseLocation This is Derivative Compute Question 0050. \\ \fi
\begin{problem}

Find the limit.  Use L'H$\hat{o}$pital's rule where appropriate.

\input{Derivative-Compute-0050.HELP.tex}

\[\lim\limits_{x\to\infty} {{\left(-\frac{5}{x} + 1\right)}^{-5 \, x} - 14}=\answer{e^{25} - 14}\]
\end{problem}}%}

\latexProblemContent{
\ifVerboseLocation This is Derivative Compute Question 0050. \\ \fi
\begin{problem}

Find the limit.  Use L'H$\hat{o}$pital's rule where appropriate.

\input{Derivative-Compute-0050.HELP.tex}

\[\lim\limits_{x\to\infty} {{\left(\frac{4}{x} + 1\right)}^{-10 \, x} + 1}=\answer{e^{\left(-40\right)} + 1}\]
\end{problem}}%}

\latexProblemContent{
\ifVerboseLocation This is Derivative Compute Question 0050. \\ \fi
\begin{problem}

Find the limit.  Use L'H$\hat{o}$pital's rule where appropriate.

\input{Derivative-Compute-0050.HELP.tex}

\[\lim\limits_{x\to\infty} {{\left(-\frac{10}{x} + 1\right)}^{9 \, x} - 17}=\answer{e^{\left(-90\right)} - 17}\]
\end{problem}}%}

\latexProblemContent{
\ifVerboseLocation This is Derivative Compute Question 0050. \\ \fi
\begin{problem}

Find the limit.  Use L'H$\hat{o}$pital's rule where appropriate.

\input{Derivative-Compute-0050.HELP.tex}

\[\lim\limits_{x\to\infty} {{\left(\frac{10}{x} + 1\right)}^{8 \, x} + 3}=\answer{e^{80} + 3}\]
\end{problem}}%}

\latexProblemContent{
\ifVerboseLocation This is Derivative Compute Question 0050. \\ \fi
\begin{problem}

Find the limit.  Use L'H$\hat{o}$pital's rule where appropriate.

\input{Derivative-Compute-0050.HELP.tex}

\[\lim\limits_{x\to\infty} {{\left(\frac{1}{x} + 1\right)}^{10 \, x} + 5}=\answer{e^{10} + 5}\]
\end{problem}}%}

\latexProblemContent{
\ifVerboseLocation This is Derivative Compute Question 0050. \\ \fi
\begin{problem}

Find the limit.  Use L'H$\hat{o}$pital's rule where appropriate.

\input{Derivative-Compute-0050.HELP.tex}

\[\lim\limits_{x\to\infty} {{\left(\frac{1}{x} + 1\right)}^{-5 \, x} + 7}=\answer{e^{\left(-5\right)} + 7}\]
\end{problem}}%}

\latexProblemContent{
\ifVerboseLocation This is Derivative Compute Question 0050. \\ \fi
\begin{problem}

Find the limit.  Use L'H$\hat{o}$pital's rule where appropriate.

\input{Derivative-Compute-0050.HELP.tex}

\[\lim\limits_{x\to\infty} {{\left(\frac{1}{x} + 1\right)}^{-x} + 3}=\answer{e^{\left(-1\right)} + 3}\]
\end{problem}}%}

\latexProblemContent{
\ifVerboseLocation This is Derivative Compute Question 0050. \\ \fi
\begin{problem}

Find the limit.  Use L'H$\hat{o}$pital's rule where appropriate.

\input{Derivative-Compute-0050.HELP.tex}

\[\lim\limits_{x\to\infty} {{\left(\frac{2}{x} + 1\right)}^{8 \, x} + 5}=\answer{e^{16} + 5}\]
\end{problem}}%}

\latexProblemContent{
\ifVerboseLocation This is Derivative Compute Question 0050. \\ \fi
\begin{problem}

Find the limit.  Use L'H$\hat{o}$pital's rule where appropriate.

\input{Derivative-Compute-0050.HELP.tex}

\[\lim\limits_{x\to\infty} {{\left(-\frac{4}{x} + 1\right)}^{-7 \, x} - 3}=\answer{e^{28} - 3}\]
\end{problem}}%}

\latexProblemContent{
\ifVerboseLocation This is Derivative Compute Question 0050. \\ \fi
\begin{problem}

Find the limit.  Use L'H$\hat{o}$pital's rule where appropriate.

\input{Derivative-Compute-0050.HELP.tex}

\[\lim\limits_{x\to\infty} {{\left(\frac{2}{x} + 1\right)}^{-6 \, x} + 10}=\answer{e^{\left(-12\right)} + 10}\]
\end{problem}}%}

\latexProblemContent{
\ifVerboseLocation This is Derivative Compute Question 0050. \\ \fi
\begin{problem}

Find the limit.  Use L'H$\hat{o}$pital's rule where appropriate.

\input{Derivative-Compute-0050.HELP.tex}

\[\lim\limits_{x\to\infty} {{\left(-\frac{6}{x} + 1\right)}^{-4 \, x} - 18}=\answer{e^{24} - 18}\]
\end{problem}}%}

\latexProblemContent{
\ifVerboseLocation This is Derivative Compute Question 0050. \\ \fi
\begin{problem}

Find the limit.  Use L'H$\hat{o}$pital's rule where appropriate.

\input{Derivative-Compute-0050.HELP.tex}

\[\lim\limits_{x\to\infty} {{\left(-\frac{6}{x} + 1\right)}^{2 \, x} - 7}=\answer{e^{\left(-12\right)} - 7}\]
\end{problem}}%}

\latexProblemContent{
\ifVerboseLocation This is Derivative Compute Question 0050. \\ \fi
\begin{problem}

Find the limit.  Use L'H$\hat{o}$pital's rule where appropriate.

\input{Derivative-Compute-0050.HELP.tex}

\[\lim\limits_{x\to\infty} {{\left(-\frac{8}{x} + 1\right)}^{-3 \, x} + 11}=\answer{e^{24} + 11}\]
\end{problem}}%}

\latexProblemContent{
\ifVerboseLocation This is Derivative Compute Question 0050. \\ \fi
\begin{problem}

Find the limit.  Use L'H$\hat{o}$pital's rule where appropriate.

\input{Derivative-Compute-0050.HELP.tex}

\[\lim\limits_{x\to\infty} {{\left(\frac{8}{x} + 1\right)}^{2 \, x}}=\answer{e^{16}}\]
\end{problem}}%}

\latexProblemContent{
\ifVerboseLocation This is Derivative Compute Question 0050. \\ \fi
\begin{problem}

Find the limit.  Use L'H$\hat{o}$pital's rule where appropriate.

\input{Derivative-Compute-0050.HELP.tex}

\[\lim\limits_{x\to\infty} {{\left(-\frac{2}{x} + 1\right)}^{-5 \, x} + 13}=\answer{e^{10} + 13}\]
\end{problem}}%}

\latexProblemContent{
\ifVerboseLocation This is Derivative Compute Question 0050. \\ \fi
\begin{problem}

Find the limit.  Use L'H$\hat{o}$pital's rule where appropriate.

\input{Derivative-Compute-0050.HELP.tex}

\[\lim\limits_{x\to\infty} {{\left(-\frac{7}{x} + 1\right)}^{8 \, x} + 18}=\answer{e^{\left(-56\right)} + 18}\]
\end{problem}}%}

\latexProblemContent{
\ifVerboseLocation This is Derivative Compute Question 0050. \\ \fi
\begin{problem}

Find the limit.  Use L'H$\hat{o}$pital's rule where appropriate.

\input{Derivative-Compute-0050.HELP.tex}

\[\lim\limits_{x\to\infty} {{\left(\frac{1}{x} + 1\right)}^{-2 \, x} + 20}=\answer{e^{\left(-2\right)} + 20}\]
\end{problem}}%}

\latexProblemContent{
\ifVerboseLocation This is Derivative Compute Question 0050. \\ \fi
\begin{problem}

Find the limit.  Use L'H$\hat{o}$pital's rule where appropriate.

\input{Derivative-Compute-0050.HELP.tex}

\[\lim\limits_{x\to\infty} {{\left(-\frac{8}{x} + 1\right)}^{9 \, x} - 7}=\answer{e^{\left(-72\right)} - 7}\]
\end{problem}}%}

\latexProblemContent{
\ifVerboseLocation This is Derivative Compute Question 0050. \\ \fi
\begin{problem}

Find the limit.  Use L'H$\hat{o}$pital's rule where appropriate.

\input{Derivative-Compute-0050.HELP.tex}

\[\lim\limits_{x\to\infty} {{\left(\frac{8}{x} + 1\right)}^{4 \, x} + 13}=\answer{e^{32} + 13}\]
\end{problem}}%}

\latexProblemContent{
\ifVerboseLocation This is Derivative Compute Question 0050. \\ \fi
\begin{problem}

Find the limit.  Use L'H$\hat{o}$pital's rule where appropriate.

\input{Derivative-Compute-0050.HELP.tex}

\[\lim\limits_{x\to\infty} {{\left(\frac{6}{x} + 1\right)}^{6 \, x} - 18}=\answer{e^{36} - 18}\]
\end{problem}}%}

\latexProblemContent{
\ifVerboseLocation This is Derivative Compute Question 0050. \\ \fi
\begin{problem}

Find the limit.  Use L'H$\hat{o}$pital's rule where appropriate.

\input{Derivative-Compute-0050.HELP.tex}

\[\lim\limits_{x\to\infty} {{\left(\frac{2}{x} + 1\right)}^{10 \, x} + 15}=\answer{e^{20} + 15}\]
\end{problem}}%}

\latexProblemContent{
\ifVerboseLocation This is Derivative Compute Question 0050. \\ \fi
\begin{problem}

Find the limit.  Use L'H$\hat{o}$pital's rule where appropriate.

\input{Derivative-Compute-0050.HELP.tex}

\[\lim\limits_{x\to\infty} {{\left(-\frac{1}{x} + 1\right)}^{-6 \, x} + 9}=\answer{e^{6} + 9}\]
\end{problem}}%}

\latexProblemContent{
\ifVerboseLocation This is Derivative Compute Question 0050. \\ \fi
\begin{problem}

Find the limit.  Use L'H$\hat{o}$pital's rule where appropriate.

\input{Derivative-Compute-0050.HELP.tex}

\[\lim\limits_{x\to\infty} {{\left(-\frac{1}{x} + 1\right)}^{-7 \, x} + 11}=\answer{e^{7} + 11}\]
\end{problem}}%}

\latexProblemContent{
\ifVerboseLocation This is Derivative Compute Question 0050. \\ \fi
\begin{problem}

Find the limit.  Use L'H$\hat{o}$pital's rule where appropriate.

\input{Derivative-Compute-0050.HELP.tex}

\[\lim\limits_{x\to\infty} {{\left(-\frac{8}{x} + 1\right)}^{9 \, x} - 8}=\answer{e^{\left(-72\right)} - 8}\]
\end{problem}}%}

\latexProblemContent{
\ifVerboseLocation This is Derivative Compute Question 0050. \\ \fi
\begin{problem}

Find the limit.  Use L'H$\hat{o}$pital's rule where appropriate.

\input{Derivative-Compute-0050.HELP.tex}

\[\lim\limits_{x\to\infty} {{\left(-\frac{6}{x} + 1\right)}^{-4 \, x} + 12}=\answer{e^{24} + 12}\]
\end{problem}}%}

\latexProblemContent{
\ifVerboseLocation This is Derivative Compute Question 0050. \\ \fi
\begin{problem}

Find the limit.  Use L'H$\hat{o}$pital's rule where appropriate.

\input{Derivative-Compute-0050.HELP.tex}

\[\lim\limits_{x\to\infty} {{\left(-\frac{5}{x} + 1\right)}^{6 \, x} - 14}=\answer{e^{\left(-30\right)} - 14}\]
\end{problem}}%}

\latexProblemContent{
\ifVerboseLocation This is Derivative Compute Question 0050. \\ \fi
\begin{problem}

Find the limit.  Use L'H$\hat{o}$pital's rule where appropriate.

\input{Derivative-Compute-0050.HELP.tex}

\[\lim\limits_{x\to\infty} {{\left(\frac{7}{x} + 1\right)}^{8 \, x} - 11}=\answer{e^{56} - 11}\]
\end{problem}}%}

\latexProblemContent{
\ifVerboseLocation This is Derivative Compute Question 0050. \\ \fi
\begin{problem}

Find the limit.  Use L'H$\hat{o}$pital's rule where appropriate.

\input{Derivative-Compute-0050.HELP.tex}

\[\lim\limits_{x\to\infty} {{\left(\frac{9}{x} + 1\right)}^{-5 \, x} - 7}=\answer{e^{\left(-45\right)} - 7}\]
\end{problem}}%}

\latexProblemContent{
\ifVerboseLocation This is Derivative Compute Question 0050. \\ \fi
\begin{problem}

Find the limit.  Use L'H$\hat{o}$pital's rule where appropriate.

\input{Derivative-Compute-0050.HELP.tex}

\[\lim\limits_{x\to\infty} {{\left(\frac{3}{x} + 1\right)}^{5 \, x} - 15}=\answer{e^{15} - 15}\]
\end{problem}}%}

\latexProblemContent{
\ifVerboseLocation This is Derivative Compute Question 0050. \\ \fi
\begin{problem}

Find the limit.  Use L'H$\hat{o}$pital's rule where appropriate.

\input{Derivative-Compute-0050.HELP.tex}

\[\lim\limits_{x\to\infty} {{\left(-\frac{6}{x} + 1\right)}^{9 \, x} + 9}=\answer{e^{\left(-54\right)} + 9}\]
\end{problem}}%}

\latexProblemContent{
\ifVerboseLocation This is Derivative Compute Question 0050. \\ \fi
\begin{problem}

Find the limit.  Use L'H$\hat{o}$pital's rule where appropriate.

\input{Derivative-Compute-0050.HELP.tex}

\[\lim\limits_{x\to\infty} {{\left(\frac{8}{x} + 1\right)}^{3 \, x} + 4}=\answer{e^{24} + 4}\]
\end{problem}}%}

\latexProblemContent{
\ifVerboseLocation This is Derivative Compute Question 0050. \\ \fi
\begin{problem}

Find the limit.  Use L'H$\hat{o}$pital's rule where appropriate.

\input{Derivative-Compute-0050.HELP.tex}

\[\lim\limits_{x\to\infty} {{\left(\frac{4}{x} + 1\right)}^{-2 \, x} + 2}=\answer{e^{\left(-8\right)} + 2}\]
\end{problem}}%}

\latexProblemContent{
\ifVerboseLocation This is Derivative Compute Question 0050. \\ \fi
\begin{problem}

Find the limit.  Use L'H$\hat{o}$pital's rule where appropriate.

\input{Derivative-Compute-0050.HELP.tex}

\[\lim\limits_{x\to\infty} {{\left(\frac{8}{x} + 1\right)}^{-7 \, x} - 12}=\answer{e^{\left(-56\right)} - 12}\]
\end{problem}}%}

\latexProblemContent{
\ifVerboseLocation This is Derivative Compute Question 0050. \\ \fi
\begin{problem}

Find the limit.  Use L'H$\hat{o}$pital's rule where appropriate.

\input{Derivative-Compute-0050.HELP.tex}

\[\lim\limits_{x\to\infty} {{\left(-\frac{1}{x} + 1\right)}^{2 \, x} + 13}=\answer{e^{\left(-2\right)} + 13}\]
\end{problem}}%}

\latexProblemContent{
\ifVerboseLocation This is Derivative Compute Question 0050. \\ \fi
\begin{problem}

Find the limit.  Use L'H$\hat{o}$pital's rule where appropriate.

\input{Derivative-Compute-0050.HELP.tex}

\[\lim\limits_{x\to\infty} {{\left(\frac{8}{x} + 1\right)}^{8 \, x} + 10}=\answer{e^{64} + 10}\]
\end{problem}}%}

\latexProblemContent{
\ifVerboseLocation This is Derivative Compute Question 0050. \\ \fi
\begin{problem}

Find the limit.  Use L'H$\hat{o}$pital's rule where appropriate.

\input{Derivative-Compute-0050.HELP.tex}

\[\lim\limits_{x\to\infty} {{\left(-\frac{5}{x} + 1\right)}^{2 \, x} - 17}=\answer{e^{\left(-10\right)} - 17}\]
\end{problem}}%}

\latexProblemContent{
\ifVerboseLocation This is Derivative Compute Question 0050. \\ \fi
\begin{problem}

Find the limit.  Use L'H$\hat{o}$pital's rule where appropriate.

\input{Derivative-Compute-0050.HELP.tex}

\[\lim\limits_{x\to\infty} {{\left(\frac{2}{x} + 1\right)}^{-10 \, x} + 14}=\answer{e^{\left(-20\right)} + 14}\]
\end{problem}}%}

\latexProblemContent{
\ifVerboseLocation This is Derivative Compute Question 0050. \\ \fi
\begin{problem}

Find the limit.  Use L'H$\hat{o}$pital's rule where appropriate.

\input{Derivative-Compute-0050.HELP.tex}

\[\lim\limits_{x\to\infty} {{\left(\frac{9}{x} + 1\right)}^{-8 \, x} - 17}=\answer{e^{\left(-72\right)} - 17}\]
\end{problem}}%}

\latexProblemContent{
\ifVerboseLocation This is Derivative Compute Question 0050. \\ \fi
\begin{problem}

Find the limit.  Use L'H$\hat{o}$pital's rule where appropriate.

\input{Derivative-Compute-0050.HELP.tex}

\[\lim\limits_{x\to\infty} {{\left(\frac{4}{x} + 1\right)}^{5 \, x} + 10}=\answer{e^{20} + 10}\]
\end{problem}}%}

\latexProblemContent{
\ifVerboseLocation This is Derivative Compute Question 0050. \\ \fi
\begin{problem}

Find the limit.  Use L'H$\hat{o}$pital's rule where appropriate.

\input{Derivative-Compute-0050.HELP.tex}

\[\lim\limits_{x\to\infty} {{\left(-\frac{10}{x} + 1\right)}^{-3 \, x} - 5}=\answer{e^{30} - 5}\]
\end{problem}}%}

\latexProblemContent{
\ifVerboseLocation This is Derivative Compute Question 0050. \\ \fi
\begin{problem}

Find the limit.  Use L'H$\hat{o}$pital's rule where appropriate.

\input{Derivative-Compute-0050.HELP.tex}

\[\lim\limits_{x\to\infty} {{\left(-\frac{4}{x} + 1\right)}^{x} + 14}=\answer{e^{\left(-4\right)} + 14}\]
\end{problem}}%}

\latexProblemContent{
\ifVerboseLocation This is Derivative Compute Question 0050. \\ \fi
\begin{problem}

Find the limit.  Use L'H$\hat{o}$pital's rule where appropriate.

\input{Derivative-Compute-0050.HELP.tex}

\[\lim\limits_{x\to\infty} {{\left(-\frac{10}{x} + 1\right)}^{-7 \, x} - 20}=\answer{e^{70} - 20}\]
\end{problem}}%}

\latexProblemContent{
\ifVerboseLocation This is Derivative Compute Question 0050. \\ \fi
\begin{problem}

Find the limit.  Use L'H$\hat{o}$pital's rule where appropriate.

\input{Derivative-Compute-0050.HELP.tex}

\[\lim\limits_{x\to\infty} {{\left(-\frac{7}{x} + 1\right)}^{-5 \, x} - 12}=\answer{e^{35} - 12}\]
\end{problem}}%}

\latexProblemContent{
\ifVerboseLocation This is Derivative Compute Question 0050. \\ \fi
\begin{problem}

Find the limit.  Use L'H$\hat{o}$pital's rule where appropriate.

\input{Derivative-Compute-0050.HELP.tex}

\[\lim\limits_{x\to\infty} {{\left(\frac{6}{x} + 1\right)}^{-9 \, x} + 1}=\answer{e^{\left(-54\right)} + 1}\]
\end{problem}}%}

\latexProblemContent{
\ifVerboseLocation This is Derivative Compute Question 0050. \\ \fi
\begin{problem}

Find the limit.  Use L'H$\hat{o}$pital's rule where appropriate.

\input{Derivative-Compute-0050.HELP.tex}

\[\lim\limits_{x\to\infty} {{\left(-\frac{2}{x} + 1\right)}^{x} - 18}=\answer{e^{\left(-2\right)} - 18}\]
\end{problem}}%}

\latexProblemContent{
\ifVerboseLocation This is Derivative Compute Question 0050. \\ \fi
\begin{problem}

Find the limit.  Use L'H$\hat{o}$pital's rule where appropriate.

\input{Derivative-Compute-0050.HELP.tex}

\[\lim\limits_{x\to\infty} {{\left(\frac{4}{x} + 1\right)}^{-x} + 10}=\answer{e^{\left(-4\right)} + 10}\]
\end{problem}}%}

\latexProblemContent{
\ifVerboseLocation This is Derivative Compute Question 0050. \\ \fi
\begin{problem}

Find the limit.  Use L'H$\hat{o}$pital's rule where appropriate.

\input{Derivative-Compute-0050.HELP.tex}

\[\lim\limits_{x\to\infty} {{\left(\frac{8}{x} + 1\right)}^{2 \, x} - 8}=\answer{e^{16} - 8}\]
\end{problem}}%}

\latexProblemContent{
\ifVerboseLocation This is Derivative Compute Question 0050. \\ \fi
\begin{problem}

Find the limit.  Use L'H$\hat{o}$pital's rule where appropriate.

\input{Derivative-Compute-0050.HELP.tex}

\[\lim\limits_{x\to\infty} {{\left(-\frac{3}{x} + 1\right)}^{-2 \, x} - 20}=\answer{e^{6} - 20}\]
\end{problem}}%}

\latexProblemContent{
\ifVerboseLocation This is Derivative Compute Question 0050. \\ \fi
\begin{problem}

Find the limit.  Use L'H$\hat{o}$pital's rule where appropriate.

\input{Derivative-Compute-0050.HELP.tex}

\[\lim\limits_{x\to\infty} {{\left(\frac{10}{x} + 1\right)}^{8 \, x} + 18}=\answer{e^{80} + 18}\]
\end{problem}}%}

\latexProblemContent{
\ifVerboseLocation This is Derivative Compute Question 0050. \\ \fi
\begin{problem}

Find the limit.  Use L'H$\hat{o}$pital's rule where appropriate.

\input{Derivative-Compute-0050.HELP.tex}

\[\lim\limits_{x\to\infty} {{\left(-\frac{7}{x} + 1\right)}^{-x}}=\answer{e^{7}}\]
\end{problem}}%}

\latexProblemContent{
\ifVerboseLocation This is Derivative Compute Question 0050. \\ \fi
\begin{problem}

Find the limit.  Use L'H$\hat{o}$pital's rule where appropriate.

\input{Derivative-Compute-0050.HELP.tex}

\[\lim\limits_{x\to\infty} {{\left(\frac{9}{x} + 1\right)}^{8 \, x} + 18}=\answer{e^{72} + 18}\]
\end{problem}}%}

\latexProblemContent{
\ifVerboseLocation This is Derivative Compute Question 0050. \\ \fi
\begin{problem}

Find the limit.  Use L'H$\hat{o}$pital's rule where appropriate.

\input{Derivative-Compute-0050.HELP.tex}

\[\lim\limits_{x\to\infty} {{\left(\frac{1}{x} + 1\right)}^{-5 \, x}}=\answer{e^{\left(-5\right)}}\]
\end{problem}}%}

\latexProblemContent{
\ifVerboseLocation This is Derivative Compute Question 0050. \\ \fi
\begin{problem}

Find the limit.  Use L'H$\hat{o}$pital's rule where appropriate.

\input{Derivative-Compute-0050.HELP.tex}

\[\lim\limits_{x\to\infty} {{\left(-\frac{6}{x} + 1\right)}^{4 \, x} - 12}=\answer{e^{\left(-24\right)} - 12}\]
\end{problem}}%}

\latexProblemContent{
\ifVerboseLocation This is Derivative Compute Question 0050. \\ \fi
\begin{problem}

Find the limit.  Use L'H$\hat{o}$pital's rule where appropriate.

\input{Derivative-Compute-0050.HELP.tex}

\[\lim\limits_{x\to\infty} {{\left(-\frac{2}{x} + 1\right)}^{9 \, x} + 4}=\answer{e^{\left(-18\right)} + 4}\]
\end{problem}}%}

\latexProblemContent{
\ifVerboseLocation This is Derivative Compute Question 0050. \\ \fi
\begin{problem}

Find the limit.  Use L'H$\hat{o}$pital's rule where appropriate.

\input{Derivative-Compute-0050.HELP.tex}

\[\lim\limits_{x\to\infty} {{\left(\frac{9}{x} + 1\right)}^{9 \, x} + 18}=\answer{e^{81} + 18}\]
\end{problem}}%}

\latexProblemContent{
\ifVerboseLocation This is Derivative Compute Question 0050. \\ \fi
\begin{problem}

Find the limit.  Use L'H$\hat{o}$pital's rule where appropriate.

\input{Derivative-Compute-0050.HELP.tex}

\[\lim\limits_{x\to\infty} {{\left(-\frac{10}{x} + 1\right)}^{8 \, x} + 11}=\answer{e^{\left(-80\right)} + 11}\]
\end{problem}}%}

\latexProblemContent{
\ifVerboseLocation This is Derivative Compute Question 0050. \\ \fi
\begin{problem}

Find the limit.  Use L'H$\hat{o}$pital's rule where appropriate.

\input{Derivative-Compute-0050.HELP.tex}

\[\lim\limits_{x\to\infty} {{\left(-\frac{9}{x} + 1\right)}^{-5 \, x} - 3}=\answer{e^{45} - 3}\]
\end{problem}}%}

\latexProblemContent{
\ifVerboseLocation This is Derivative Compute Question 0050. \\ \fi
\begin{problem}

Find the limit.  Use L'H$\hat{o}$pital's rule where appropriate.

\input{Derivative-Compute-0050.HELP.tex}

\[\lim\limits_{x\to\infty} {{\left(\frac{1}{x} + 1\right)}^{8 \, x} - 18}=\answer{e^{8} - 18}\]
\end{problem}}%}

\latexProblemContent{
\ifVerboseLocation This is Derivative Compute Question 0050. \\ \fi
\begin{problem}

Find the limit.  Use L'H$\hat{o}$pital's rule where appropriate.

\input{Derivative-Compute-0050.HELP.tex}

\[\lim\limits_{x\to\infty} {{\left(\frac{6}{x} + 1\right)}^{x} - 14}=\answer{e^{6} - 14}\]
\end{problem}}%}

\latexProblemContent{
\ifVerboseLocation This is Derivative Compute Question 0050. \\ \fi
\begin{problem}

Find the limit.  Use L'H$\hat{o}$pital's rule where appropriate.

\input{Derivative-Compute-0050.HELP.tex}

\[\lim\limits_{x\to\infty} {{\left(\frac{3}{x} + 1\right)}^{x} - 5}=\answer{e^{3} - 5}\]
\end{problem}}%}

\latexProblemContent{
\ifVerboseLocation This is Derivative Compute Question 0050. \\ \fi
\begin{problem}

Find the limit.  Use L'H$\hat{o}$pital's rule where appropriate.

\input{Derivative-Compute-0050.HELP.tex}

\[\lim\limits_{x\to\infty} {{\left(\frac{4}{x} + 1\right)}^{-7 \, x} + 5}=\answer{e^{\left(-28\right)} + 5}\]
\end{problem}}%}

\latexProblemContent{
\ifVerboseLocation This is Derivative Compute Question 0050. \\ \fi
\begin{problem}

Find the limit.  Use L'H$\hat{o}$pital's rule where appropriate.

\input{Derivative-Compute-0050.HELP.tex}

\[\lim\limits_{x\to\infty} {{\left(-\frac{2}{x} + 1\right)}^{-x} + 5}=\answer{e^{2} + 5}\]
\end{problem}}%}

\latexProblemContent{
\ifVerboseLocation This is Derivative Compute Question 0050. \\ \fi
\begin{problem}

Find the limit.  Use L'H$\hat{o}$pital's rule where appropriate.

\input{Derivative-Compute-0050.HELP.tex}

\[\lim\limits_{x\to\infty} {{\left(-\frac{9}{x} + 1\right)}^{-7 \, x} - 2}=\answer{e^{63} - 2}\]
\end{problem}}%}

\latexProblemContent{
\ifVerboseLocation This is Derivative Compute Question 0050. \\ \fi
\begin{problem}

Find the limit.  Use L'H$\hat{o}$pital's rule where appropriate.

\input{Derivative-Compute-0050.HELP.tex}

\[\lim\limits_{x\to\infty} {{\left(\frac{8}{x} + 1\right)}^{6 \, x} + 10}=\answer{e^{48} + 10}\]
\end{problem}}%}

\latexProblemContent{
\ifVerboseLocation This is Derivative Compute Question 0050. \\ \fi
\begin{problem}

Find the limit.  Use L'H$\hat{o}$pital's rule where appropriate.

\input{Derivative-Compute-0050.HELP.tex}

\[\lim\limits_{x\to\infty} {{\left(-\frac{5}{x} + 1\right)}^{-7 \, x} + 11}=\answer{e^{35} + 11}\]
\end{problem}}%}

\latexProblemContent{
\ifVerboseLocation This is Derivative Compute Question 0050. \\ \fi
\begin{problem}

Find the limit.  Use L'H$\hat{o}$pital's rule where appropriate.

\input{Derivative-Compute-0050.HELP.tex}

\[\lim\limits_{x\to\infty} {{\left(-\frac{6}{x} + 1\right)}^{-7 \, x} - 1}=\answer{e^{42} - 1}\]
\end{problem}}%}

\latexProblemContent{
\ifVerboseLocation This is Derivative Compute Question 0050. \\ \fi
\begin{problem}

Find the limit.  Use L'H$\hat{o}$pital's rule where appropriate.

\input{Derivative-Compute-0050.HELP.tex}

\[\lim\limits_{x\to\infty} {{\left(-\frac{5}{x} + 1\right)}^{10 \, x} - 5}=\answer{e^{\left(-50\right)} - 5}\]
\end{problem}}%}

\latexProblemContent{
\ifVerboseLocation This is Derivative Compute Question 0050. \\ \fi
\begin{problem}

Find the limit.  Use L'H$\hat{o}$pital's rule where appropriate.

\input{Derivative-Compute-0050.HELP.tex}

\[\lim\limits_{x\to\infty} {{\left(-\frac{7}{x} + 1\right)}^{4 \, x} - 12}=\answer{e^{\left(-28\right)} - 12}\]
\end{problem}}%}

\latexProblemContent{
\ifVerboseLocation This is Derivative Compute Question 0050. \\ \fi
\begin{problem}

Find the limit.  Use L'H$\hat{o}$pital's rule where appropriate.

\input{Derivative-Compute-0050.HELP.tex}

\[\lim\limits_{x\to\infty} {{\left(\frac{4}{x} + 1\right)}^{7 \, x} - 12}=\answer{e^{28} - 12}\]
\end{problem}}%}

\latexProblemContent{
\ifVerboseLocation This is Derivative Compute Question 0050. \\ \fi
\begin{problem}

Find the limit.  Use L'H$\hat{o}$pital's rule where appropriate.

\input{Derivative-Compute-0050.HELP.tex}

\[\lim\limits_{x\to\infty} {{\left(\frac{3}{x} + 1\right)}^{10 \, x} - 8}=\answer{e^{30} - 8}\]
\end{problem}}%}

\latexProblemContent{
\ifVerboseLocation This is Derivative Compute Question 0050. \\ \fi
\begin{problem}

Find the limit.  Use L'H$\hat{o}$pital's rule where appropriate.

\input{Derivative-Compute-0050.HELP.tex}

\[\lim\limits_{x\to\infty} {{\left(-\frac{9}{x} + 1\right)}^{-4 \, x} + 14}=\answer{e^{36} + 14}\]
\end{problem}}%}

\latexProblemContent{
\ifVerboseLocation This is Derivative Compute Question 0050. \\ \fi
\begin{problem}

Find the limit.  Use L'H$\hat{o}$pital's rule where appropriate.

\input{Derivative-Compute-0050.HELP.tex}

\[\lim\limits_{x\to\infty} {{\left(\frac{8}{x} + 1\right)}^{5 \, x} - 12}=\answer{e^{40} - 12}\]
\end{problem}}%}

\latexProblemContent{
\ifVerboseLocation This is Derivative Compute Question 0050. \\ \fi
\begin{problem}

Find the limit.  Use L'H$\hat{o}$pital's rule where appropriate.

\input{Derivative-Compute-0050.HELP.tex}

\[\lim\limits_{x\to\infty} {{\left(\frac{1}{x} + 1\right)}^{-2 \, x} + 18}=\answer{e^{\left(-2\right)} + 18}\]
\end{problem}}%}

\latexProblemContent{
\ifVerboseLocation This is Derivative Compute Question 0050. \\ \fi
\begin{problem}

Find the limit.  Use L'H$\hat{o}$pital's rule where appropriate.

\input{Derivative-Compute-0050.HELP.tex}

\[\lim\limits_{x\to\infty} {{\left(-\frac{3}{x} + 1\right)}^{7 \, x} + 19}=\answer{e^{\left(-21\right)} + 19}\]
\end{problem}}%}

\latexProblemContent{
\ifVerboseLocation This is Derivative Compute Question 0050. \\ \fi
\begin{problem}

Find the limit.  Use L'H$\hat{o}$pital's rule where appropriate.

\input{Derivative-Compute-0050.HELP.tex}

\[\lim\limits_{x\to\infty} {{\left(\frac{6}{x} + 1\right)}^{-7 \, x} - 12}=\answer{e^{\left(-42\right)} - 12}\]
\end{problem}}%}

\latexProblemContent{
\ifVerboseLocation This is Derivative Compute Question 0050. \\ \fi
\begin{problem}

Find the limit.  Use L'H$\hat{o}$pital's rule where appropriate.

\input{Derivative-Compute-0050.HELP.tex}

\[\lim\limits_{x\to\infty} {{\left(-\frac{8}{x} + 1\right)}^{-8 \, x} + 12}=\answer{e^{64} + 12}\]
\end{problem}}%}

\latexProblemContent{
\ifVerboseLocation This is Derivative Compute Question 0050. \\ \fi
\begin{problem}

Find the limit.  Use L'H$\hat{o}$pital's rule where appropriate.

\input{Derivative-Compute-0050.HELP.tex}

\[\lim\limits_{x\to\infty} {{\left(-\frac{8}{x} + 1\right)}^{7 \, x} - 10}=\answer{e^{\left(-56\right)} - 10}\]
\end{problem}}%}

\latexProblemContent{
\ifVerboseLocation This is Derivative Compute Question 0050. \\ \fi
\begin{problem}

Find the limit.  Use L'H$\hat{o}$pital's rule where appropriate.

\input{Derivative-Compute-0050.HELP.tex}

\[\lim\limits_{x\to\infty} {{\left(\frac{4}{x} + 1\right)}^{6 \, x} + 12}=\answer{e^{24} + 12}\]
\end{problem}}%}

\latexProblemContent{
\ifVerboseLocation This is Derivative Compute Question 0050. \\ \fi
\begin{problem}

Find the limit.  Use L'H$\hat{o}$pital's rule where appropriate.

\input{Derivative-Compute-0050.HELP.tex}

\[\lim\limits_{x\to\infty} {{\left(-\frac{1}{x} + 1\right)}^{x} + 3}=\answer{e^{\left(-1\right)} + 3}\]
\end{problem}}%}

\latexProblemContent{
\ifVerboseLocation This is Derivative Compute Question 0050. \\ \fi
\begin{problem}

Find the limit.  Use L'H$\hat{o}$pital's rule where appropriate.

\input{Derivative-Compute-0050.HELP.tex}

\[\lim\limits_{x\to\infty} {{\left(-\frac{8}{x} + 1\right)}^{-4 \, x} + 19}=\answer{e^{32} + 19}\]
\end{problem}}%}

\latexProblemContent{
\ifVerboseLocation This is Derivative Compute Question 0050. \\ \fi
\begin{problem}

Find the limit.  Use L'H$\hat{o}$pital's rule where appropriate.

\input{Derivative-Compute-0050.HELP.tex}

\[\lim\limits_{x\to\infty} {{\left(-\frac{9}{x} + 1\right)}^{-6 \, x} + 17}=\answer{e^{54} + 17}\]
\end{problem}}%}

\latexProblemContent{
\ifVerboseLocation This is Derivative Compute Question 0050. \\ \fi
\begin{problem}

Find the limit.  Use L'H$\hat{o}$pital's rule where appropriate.

\input{Derivative-Compute-0050.HELP.tex}

\[\lim\limits_{x\to\infty} {{\left(\frac{10}{x} + 1\right)}^{7 \, x} - 8}=\answer{e^{70} - 8}\]
\end{problem}}%}

\latexProblemContent{
\ifVerboseLocation This is Derivative Compute Question 0050. \\ \fi
\begin{problem}

Find the limit.  Use L'H$\hat{o}$pital's rule where appropriate.

\input{Derivative-Compute-0050.HELP.tex}

\[\lim\limits_{x\to\infty} {{\left(\frac{8}{x} + 1\right)}^{-7 \, x} + 15}=\answer{e^{\left(-56\right)} + 15}\]
\end{problem}}%}

\latexProblemContent{
\ifVerboseLocation This is Derivative Compute Question 0050. \\ \fi
\begin{problem}

Find the limit.  Use L'H$\hat{o}$pital's rule where appropriate.

\input{Derivative-Compute-0050.HELP.tex}

\[\lim\limits_{x\to\infty} {{\left(-\frac{4}{x} + 1\right)}^{-2 \, x} - 7}=\answer{e^{8} - 7}\]
\end{problem}}%}

\latexProblemContent{
\ifVerboseLocation This is Derivative Compute Question 0050. \\ \fi
\begin{problem}

Find the limit.  Use L'H$\hat{o}$pital's rule where appropriate.

\input{Derivative-Compute-0050.HELP.tex}

\[\lim\limits_{x\to\infty} {{\left(-\frac{7}{x} + 1\right)}^{2 \, x} + 1}=\answer{e^{\left(-14\right)} + 1}\]
\end{problem}}%}

\latexProblemContent{
\ifVerboseLocation This is Derivative Compute Question 0050. \\ \fi
\begin{problem}

Find the limit.  Use L'H$\hat{o}$pital's rule where appropriate.

\input{Derivative-Compute-0050.HELP.tex}

\[\lim\limits_{x\to\infty} {{\left(-\frac{1}{x} + 1\right)}^{-10 \, x} + 7}=\answer{e^{10} + 7}\]
\end{problem}}%}

\latexProblemContent{
\ifVerboseLocation This is Derivative Compute Question 0050. \\ \fi
\begin{problem}

Find the limit.  Use L'H$\hat{o}$pital's rule where appropriate.

\input{Derivative-Compute-0050.HELP.tex}

\[\lim\limits_{x\to\infty} {{\left(\frac{4}{x} + 1\right)}^{x} + 8}=\answer{e^{4} + 8}\]
\end{problem}}%}

\latexProblemContent{
\ifVerboseLocation This is Derivative Compute Question 0050. \\ \fi
\begin{problem}

Find the limit.  Use L'H$\hat{o}$pital's rule where appropriate.

\input{Derivative-Compute-0050.HELP.tex}

\[\lim\limits_{x\to\infty} {{\left(-\frac{2}{x} + 1\right)}^{-10 \, x} - 9}=\answer{e^{20} - 9}\]
\end{problem}}%}

\latexProblemContent{
\ifVerboseLocation This is Derivative Compute Question 0050. \\ \fi
\begin{problem}

Find the limit.  Use L'H$\hat{o}$pital's rule where appropriate.

\input{Derivative-Compute-0050.HELP.tex}

\[\lim\limits_{x\to\infty} {{\left(-\frac{5}{x} + 1\right)}^{-10 \, x} + 5}=\answer{e^{50} + 5}\]
\end{problem}}%}

\latexProblemContent{
\ifVerboseLocation This is Derivative Compute Question 0050. \\ \fi
\begin{problem}

Find the limit.  Use L'H$\hat{o}$pital's rule where appropriate.

\input{Derivative-Compute-0050.HELP.tex}

\[\lim\limits_{x\to\infty} {{\left(-\frac{6}{x} + 1\right)}^{-6 \, x} + 11}=\answer{e^{36} + 11}\]
\end{problem}}%}

\latexProblemContent{
\ifVerboseLocation This is Derivative Compute Question 0050. \\ \fi
\begin{problem}

Find the limit.  Use L'H$\hat{o}$pital's rule where appropriate.

\input{Derivative-Compute-0050.HELP.tex}

\[\lim\limits_{x\to\infty} {{\left(\frac{8}{x} + 1\right)}^{-9 \, x} - 12}=\answer{e^{\left(-72\right)} - 12}\]
\end{problem}}%}

\latexProblemContent{
\ifVerboseLocation This is Derivative Compute Question 0050. \\ \fi
\begin{problem}

Find the limit.  Use L'H$\hat{o}$pital's rule where appropriate.

\input{Derivative-Compute-0050.HELP.tex}

\[\lim\limits_{x\to\infty} {{\left(-\frac{3}{x} + 1\right)}^{-5 \, x} + 13}=\answer{e^{15} + 13}\]
\end{problem}}%}

\latexProblemContent{
\ifVerboseLocation This is Derivative Compute Question 0050. \\ \fi
\begin{problem}

Find the limit.  Use L'H$\hat{o}$pital's rule where appropriate.

\input{Derivative-Compute-0050.HELP.tex}

\[\lim\limits_{x\to\infty} {{\left(-\frac{10}{x} + 1\right)}^{-x} + 19}=\answer{e^{10} + 19}\]
\end{problem}}%}

\latexProblemContent{
\ifVerboseLocation This is Derivative Compute Question 0050. \\ \fi
\begin{problem}

Find the limit.  Use L'H$\hat{o}$pital's rule where appropriate.

\input{Derivative-Compute-0050.HELP.tex}

\[\lim\limits_{x\to\infty} {{\left(-\frac{4}{x} + 1\right)}^{-7 \, x} + 3}=\answer{e^{28} + 3}\]
\end{problem}}%}

\latexProblemContent{
\ifVerboseLocation This is Derivative Compute Question 0050. \\ \fi
\begin{problem}

Find the limit.  Use L'H$\hat{o}$pital's rule where appropriate.

\input{Derivative-Compute-0050.HELP.tex}

\[\lim\limits_{x\to\infty} {{\left(-\frac{1}{x} + 1\right)}^{-6 \, x} - 3}=\answer{e^{6} - 3}\]
\end{problem}}%}

\latexProblemContent{
\ifVerboseLocation This is Derivative Compute Question 0050. \\ \fi
\begin{problem}

Find the limit.  Use L'H$\hat{o}$pital's rule where appropriate.

\input{Derivative-Compute-0050.HELP.tex}

\[\lim\limits_{x\to\infty} {{\left(-\frac{10}{x} + 1\right)}^{2 \, x} + 6}=\answer{e^{\left(-20\right)} + 6}\]
\end{problem}}%}

\latexProblemContent{
\ifVerboseLocation This is Derivative Compute Question 0050. \\ \fi
\begin{problem}

Find the limit.  Use L'H$\hat{o}$pital's rule where appropriate.

\input{Derivative-Compute-0050.HELP.tex}

\[\lim\limits_{x\to\infty} {{\left(-\frac{3}{x} + 1\right)}^{-4 \, x} - 16}=\answer{e^{12} - 16}\]
\end{problem}}%}

\latexProblemContent{
\ifVerboseLocation This is Derivative Compute Question 0050. \\ \fi
\begin{problem}

Find the limit.  Use L'H$\hat{o}$pital's rule where appropriate.

\input{Derivative-Compute-0050.HELP.tex}

\[\lim\limits_{x\to\infty} {{\left(-\frac{1}{x} + 1\right)}^{-9 \, x} - 9}=\answer{e^{9} - 9}\]
\end{problem}}%}

\latexProblemContent{
\ifVerboseLocation This is Derivative Compute Question 0050. \\ \fi
\begin{problem}

Find the limit.  Use L'H$\hat{o}$pital's rule where appropriate.

\input{Derivative-Compute-0050.HELP.tex}

\[\lim\limits_{x\to\infty} {{\left(-\frac{7}{x} + 1\right)}^{x} - 1}=\answer{e^{\left(-7\right)} - 1}\]
\end{problem}}%}

\latexProblemContent{
\ifVerboseLocation This is Derivative Compute Question 0050. \\ \fi
\begin{problem}

Find the limit.  Use L'H$\hat{o}$pital's rule where appropriate.

\input{Derivative-Compute-0050.HELP.tex}

\[\lim\limits_{x\to\infty} {{\left(\frac{4}{x} + 1\right)}^{-9 \, x} + 9}=\answer{e^{\left(-36\right)} + 9}\]
\end{problem}}%}

\latexProblemContent{
\ifVerboseLocation This is Derivative Compute Question 0050. \\ \fi
\begin{problem}

Find the limit.  Use L'H$\hat{o}$pital's rule where appropriate.

\input{Derivative-Compute-0050.HELP.tex}

\[\lim\limits_{x\to\infty} {{\left(-\frac{10}{x} + 1\right)}^{-10 \, x} - 3}=\answer{e^{100} - 3}\]
\end{problem}}%}

\latexProblemContent{
\ifVerboseLocation This is Derivative Compute Question 0050. \\ \fi
\begin{problem}

Find the limit.  Use L'H$\hat{o}$pital's rule where appropriate.

\input{Derivative-Compute-0050.HELP.tex}

\[\lim\limits_{x\to\infty} {{\left(\frac{3}{x} + 1\right)}^{-6 \, x} + 19}=\answer{e^{\left(-18\right)} + 19}\]
\end{problem}}%}

\latexProblemContent{
\ifVerboseLocation This is Derivative Compute Question 0050. \\ \fi
\begin{problem}

Find the limit.  Use L'H$\hat{o}$pital's rule where appropriate.

\input{Derivative-Compute-0050.HELP.tex}

\[\lim\limits_{x\to\infty} {{\left(-\frac{5}{x} + 1\right)}^{-8 \, x} + 8}=\answer{e^{40} + 8}\]
\end{problem}}%}

\latexProblemContent{
\ifVerboseLocation This is Derivative Compute Question 0050. \\ \fi
\begin{problem}

Find the limit.  Use L'H$\hat{o}$pital's rule where appropriate.

\input{Derivative-Compute-0050.HELP.tex}

\[\lim\limits_{x\to\infty} {{\left(-\frac{10}{x} + 1\right)}^{10 \, x} - 17}=\answer{e^{\left(-100\right)} - 17}\]
\end{problem}}%}

\latexProblemContent{
\ifVerboseLocation This is Derivative Compute Question 0050. \\ \fi
\begin{problem}

Find the limit.  Use L'H$\hat{o}$pital's rule where appropriate.

\input{Derivative-Compute-0050.HELP.tex}

\[\lim\limits_{x\to\infty} {{\left(\frac{9}{x} + 1\right)}^{-6 \, x} + 4}=\answer{e^{\left(-54\right)} + 4}\]
\end{problem}}%}

\latexProblemContent{
\ifVerboseLocation This is Derivative Compute Question 0050. \\ \fi
\begin{problem}

Find the limit.  Use L'H$\hat{o}$pital's rule where appropriate.

\input{Derivative-Compute-0050.HELP.tex}

\[\lim\limits_{x\to\infty} {{\left(\frac{10}{x} + 1\right)}^{6 \, x} + 2}=\answer{e^{60} + 2}\]
\end{problem}}%}

\latexProblemContent{
\ifVerboseLocation This is Derivative Compute Question 0050. \\ \fi
\begin{problem}

Find the limit.  Use L'H$\hat{o}$pital's rule where appropriate.

\input{Derivative-Compute-0050.HELP.tex}

\[\lim\limits_{x\to\infty} {{\left(\frac{1}{x} + 1\right)}^{-10 \, x} + 13}=\answer{e^{\left(-10\right)} + 13}\]
\end{problem}}%}

\latexProblemContent{
\ifVerboseLocation This is Derivative Compute Question 0050. \\ \fi
\begin{problem}

Find the limit.  Use L'H$\hat{o}$pital's rule where appropriate.

\input{Derivative-Compute-0050.HELP.tex}

\[\lim\limits_{x\to\infty} {{\left(\frac{10}{x} + 1\right)}^{8 \, x} + 14}=\answer{e^{80} + 14}\]
\end{problem}}%}

\latexProblemContent{
\ifVerboseLocation This is Derivative Compute Question 0050. \\ \fi
\begin{problem}

Find the limit.  Use L'H$\hat{o}$pital's rule where appropriate.

\input{Derivative-Compute-0050.HELP.tex}

\[\lim\limits_{x\to\infty} {{\left(\frac{2}{x} + 1\right)}^{-6 \, x} - 3}=\answer{e^{\left(-12\right)} - 3}\]
\end{problem}}%}

\latexProblemContent{
\ifVerboseLocation This is Derivative Compute Question 0050. \\ \fi
\begin{problem}

Find the limit.  Use L'H$\hat{o}$pital's rule where appropriate.

\input{Derivative-Compute-0050.HELP.tex}

\[\lim\limits_{x\to\infty} {{\left(-\frac{6}{x} + 1\right)}^{-10 \, x} - 16}=\answer{e^{60} - 16}\]
\end{problem}}%}

\latexProblemContent{
\ifVerboseLocation This is Derivative Compute Question 0050. \\ \fi
\begin{problem}

Find the limit.  Use L'H$\hat{o}$pital's rule where appropriate.

\input{Derivative-Compute-0050.HELP.tex}

\[\lim\limits_{x\to\infty} {{\left(-\frac{1}{x} + 1\right)}^{6 \, x} - 16}=\answer{e^{\left(-6\right)} - 16}\]
\end{problem}}%}

\latexProblemContent{
\ifVerboseLocation This is Derivative Compute Question 0050. \\ \fi
\begin{problem}

Find the limit.  Use L'H$\hat{o}$pital's rule where appropriate.

\input{Derivative-Compute-0050.HELP.tex}

\[\lim\limits_{x\to\infty} {{\left(\frac{8}{x} + 1\right)}^{-3 \, x} - 15}=\answer{e^{\left(-24\right)} - 15}\]
\end{problem}}%}

\latexProblemContent{
\ifVerboseLocation This is Derivative Compute Question 0050. \\ \fi
\begin{problem}

Find the limit.  Use L'H$\hat{o}$pital's rule where appropriate.

\input{Derivative-Compute-0050.HELP.tex}

\[\lim\limits_{x\to\infty} {{\left(-\frac{5}{x} + 1\right)}^{5 \, x} + 15}=\answer{e^{\left(-25\right)} + 15}\]
\end{problem}}%}

\latexProblemContent{
\ifVerboseLocation This is Derivative Compute Question 0050. \\ \fi
\begin{problem}

Find the limit.  Use L'H$\hat{o}$pital's rule where appropriate.

\input{Derivative-Compute-0050.HELP.tex}

\[\lim\limits_{x\to\infty} {{\left(\frac{4}{x} + 1\right)}^{-7 \, x} - 9}=\answer{e^{\left(-28\right)} - 9}\]
\end{problem}}%}

\latexProblemContent{
\ifVerboseLocation This is Derivative Compute Question 0050. \\ \fi
\begin{problem}

Find the limit.  Use L'H$\hat{o}$pital's rule where appropriate.

\input{Derivative-Compute-0050.HELP.tex}

\[\lim\limits_{x\to\infty} {{\left(-\frac{3}{x} + 1\right)}^{-6 \, x} - 20}=\answer{e^{18} - 20}\]
\end{problem}}%}

\latexProblemContent{
\ifVerboseLocation This is Derivative Compute Question 0050. \\ \fi
\begin{problem}

Find the limit.  Use L'H$\hat{o}$pital's rule where appropriate.

\input{Derivative-Compute-0050.HELP.tex}

\[\lim\limits_{x\to\infty} {{\left(\frac{3}{x} + 1\right)}^{6 \, x} + 20}=\answer{e^{18} + 20}\]
\end{problem}}%}

\latexProblemContent{
\ifVerboseLocation This is Derivative Compute Question 0050. \\ \fi
\begin{problem}

Find the limit.  Use L'H$\hat{o}$pital's rule where appropriate.

\input{Derivative-Compute-0050.HELP.tex}

\[\lim\limits_{x\to\infty} {{\left(-\frac{5}{x} + 1\right)}^{-6 \, x} - 2}=\answer{e^{30} - 2}\]
\end{problem}}%}

\latexProblemContent{
\ifVerboseLocation This is Derivative Compute Question 0050. \\ \fi
\begin{problem}

Find the limit.  Use L'H$\hat{o}$pital's rule where appropriate.

\input{Derivative-Compute-0050.HELP.tex}

\[\lim\limits_{x\to\infty} {{\left(\frac{2}{x} + 1\right)}^{-6 \, x} + 3}=\answer{e^{\left(-12\right)} + 3}\]
\end{problem}}%}

\latexProblemContent{
\ifVerboseLocation This is Derivative Compute Question 0050. \\ \fi
\begin{problem}

Find the limit.  Use L'H$\hat{o}$pital's rule where appropriate.

\input{Derivative-Compute-0050.HELP.tex}

\[\lim\limits_{x\to\infty} {{\left(-\frac{6}{x} + 1\right)}^{-3 \, x} - 20}=\answer{e^{18} - 20}\]
\end{problem}}%}

\latexProblemContent{
\ifVerboseLocation This is Derivative Compute Question 0050. \\ \fi
\begin{problem}

Find the limit.  Use L'H$\hat{o}$pital's rule where appropriate.

\input{Derivative-Compute-0050.HELP.tex}

\[\lim\limits_{x\to\infty} {{\left(\frac{1}{x} + 1\right)}^{-5 \, x} + 12}=\answer{e^{\left(-5\right)} + 12}\]
\end{problem}}%}

\latexProblemContent{
\ifVerboseLocation This is Derivative Compute Question 0050. \\ \fi
\begin{problem}

Find the limit.  Use L'H$\hat{o}$pital's rule where appropriate.

\input{Derivative-Compute-0050.HELP.tex}

\[\lim\limits_{x\to\infty} {{\left(-\frac{2}{x} + 1\right)}^{5 \, x} - 20}=\answer{e^{\left(-10\right)} - 20}\]
\end{problem}}%}

\latexProblemContent{
\ifVerboseLocation This is Derivative Compute Question 0050. \\ \fi
\begin{problem}

Find the limit.  Use L'H$\hat{o}$pital's rule where appropriate.

\input{Derivative-Compute-0050.HELP.tex}

\[\lim\limits_{x\to\infty} {{\left(-\frac{7}{x} + 1\right)}^{-4 \, x} + 18}=\answer{e^{28} + 18}\]
\end{problem}}%}

\latexProblemContent{
\ifVerboseLocation This is Derivative Compute Question 0050. \\ \fi
\begin{problem}

Find the limit.  Use L'H$\hat{o}$pital's rule where appropriate.

\input{Derivative-Compute-0050.HELP.tex}

\[\lim\limits_{x\to\infty} {{\left(\frac{6}{x} + 1\right)}^{3 \, x} - 13}=\answer{e^{18} - 13}\]
\end{problem}}%}

\latexProblemContent{
\ifVerboseLocation This is Derivative Compute Question 0050. \\ \fi
\begin{problem}

Find the limit.  Use L'H$\hat{o}$pital's rule where appropriate.

\input{Derivative-Compute-0050.HELP.tex}

\[\lim\limits_{x\to\infty} {{\left(-\frac{1}{x} + 1\right)}^{2 \, x} + 11}=\answer{e^{\left(-2\right)} + 11}\]
\end{problem}}%}

\latexProblemContent{
\ifVerboseLocation This is Derivative Compute Question 0050. \\ \fi
\begin{problem}

Find the limit.  Use L'H$\hat{o}$pital's rule where appropriate.

\input{Derivative-Compute-0050.HELP.tex}

\[\lim\limits_{x\to\infty} {{\left(-\frac{2}{x} + 1\right)}^{-4 \, x} + 3}=\answer{e^{8} + 3}\]
\end{problem}}%}

\latexProblemContent{
\ifVerboseLocation This is Derivative Compute Question 0050. \\ \fi
\begin{problem}

Find the limit.  Use L'H$\hat{o}$pital's rule where appropriate.

\input{Derivative-Compute-0050.HELP.tex}

\[\lim\limits_{x\to\infty} {{\left(-\frac{9}{x} + 1\right)}^{-8 \, x} - 10}=\answer{e^{72} - 10}\]
\end{problem}}%}

\latexProblemContent{
\ifVerboseLocation This is Derivative Compute Question 0050. \\ \fi
\begin{problem}

Find the limit.  Use L'H$\hat{o}$pital's rule where appropriate.

\input{Derivative-Compute-0050.HELP.tex}

\[\lim\limits_{x\to\infty} {{\left(\frac{6}{x} + 1\right)}^{-10 \, x} - 14}=\answer{e^{\left(-60\right)} - 14}\]
\end{problem}}%}

\latexProblemContent{
\ifVerboseLocation This is Derivative Compute Question 0050. \\ \fi
\begin{problem}

Find the limit.  Use L'H$\hat{o}$pital's rule where appropriate.

\input{Derivative-Compute-0050.HELP.tex}

\[\lim\limits_{x\to\infty} {{\left(\frac{1}{x} + 1\right)}^{-7 \, x} - 14}=\answer{e^{\left(-7\right)} - 14}\]
\end{problem}}%}

\latexProblemContent{
\ifVerboseLocation This is Derivative Compute Question 0050. \\ \fi
\begin{problem}

Find the limit.  Use L'H$\hat{o}$pital's rule where appropriate.

\input{Derivative-Compute-0050.HELP.tex}

\[\lim\limits_{x\to\infty} {{\left(-\frac{10}{x} + 1\right)}^{6 \, x} - 5}=\answer{e^{\left(-60\right)} - 5}\]
\end{problem}}%}

\latexProblemContent{
\ifVerboseLocation This is Derivative Compute Question 0050. \\ \fi
\begin{problem}

Find the limit.  Use L'H$\hat{o}$pital's rule where appropriate.

\input{Derivative-Compute-0050.HELP.tex}

\[\lim\limits_{x\to\infty} {{\left(\frac{1}{x} + 1\right)}^{-2 \, x} + 10}=\answer{e^{\left(-2\right)} + 10}\]
\end{problem}}%}

\latexProblemContent{
\ifVerboseLocation This is Derivative Compute Question 0050. \\ \fi
\begin{problem}

Find the limit.  Use L'H$\hat{o}$pital's rule where appropriate.

\input{Derivative-Compute-0050.HELP.tex}

\[\lim\limits_{x\to\infty} {{\left(-\frac{8}{x} + 1\right)}^{8 \, x} - 5}=\answer{e^{\left(-64\right)} - 5}\]
\end{problem}}%}

\latexProblemContent{
\ifVerboseLocation This is Derivative Compute Question 0050. \\ \fi
\begin{problem}

Find the limit.  Use L'H$\hat{o}$pital's rule where appropriate.

\input{Derivative-Compute-0050.HELP.tex}

\[\lim\limits_{x\to\infty} {{\left(\frac{4}{x} + 1\right)}^{9 \, x} + 12}=\answer{e^{36} + 12}\]
\end{problem}}%}

\latexProblemContent{
\ifVerboseLocation This is Derivative Compute Question 0050. \\ \fi
\begin{problem}

Find the limit.  Use L'H$\hat{o}$pital's rule where appropriate.

\input{Derivative-Compute-0050.HELP.tex}

\[\lim\limits_{x\to\infty} {{\left(-\frac{9}{x} + 1\right)}^{9 \, x} - 9}=\answer{e^{\left(-81\right)} - 9}\]
\end{problem}}%}

\latexProblemContent{
\ifVerboseLocation This is Derivative Compute Question 0050. \\ \fi
\begin{problem}

Find the limit.  Use L'H$\hat{o}$pital's rule where appropriate.

\input{Derivative-Compute-0050.HELP.tex}

\[\lim\limits_{x\to\infty} {{\left(\frac{6}{x} + 1\right)}^{2 \, x} + 19}=\answer{e^{12} + 19}\]
\end{problem}}%}

\latexProblemContent{
\ifVerboseLocation This is Derivative Compute Question 0050. \\ \fi
\begin{problem}

Find the limit.  Use L'H$\hat{o}$pital's rule where appropriate.

\input{Derivative-Compute-0050.HELP.tex}

\[\lim\limits_{x\to\infty} {{\left(-\frac{1}{x} + 1\right)}^{x} + 16}=\answer{e^{\left(-1\right)} + 16}\]
\end{problem}}%}

\latexProblemContent{
\ifVerboseLocation This is Derivative Compute Question 0050. \\ \fi
\begin{problem}

Find the limit.  Use L'H$\hat{o}$pital's rule where appropriate.

\input{Derivative-Compute-0050.HELP.tex}

\[\lim\limits_{x\to\infty} {{\left(-\frac{8}{x} + 1\right)}^{7 \, x} - 17}=\answer{e^{\left(-56\right)} - 17}\]
\end{problem}}%}

\latexProblemContent{
\ifVerboseLocation This is Derivative Compute Question 0050. \\ \fi
\begin{problem}

Find the limit.  Use L'H$\hat{o}$pital's rule where appropriate.

\input{Derivative-Compute-0050.HELP.tex}

\[\lim\limits_{x\to\infty} {{\left(-\frac{2}{x} + 1\right)}^{6 \, x} - 15}=\answer{e^{\left(-12\right)} - 15}\]
\end{problem}}%}

\latexProblemContent{
\ifVerboseLocation This is Derivative Compute Question 0050. \\ \fi
\begin{problem}

Find the limit.  Use L'H$\hat{o}$pital's rule where appropriate.

\input{Derivative-Compute-0050.HELP.tex}

\[\lim\limits_{x\to\infty} {{\left(\frac{3}{x} + 1\right)}^{-x} - 1}=\answer{e^{\left(-3\right)} - 1}\]
\end{problem}}%}

\latexProblemContent{
\ifVerboseLocation This is Derivative Compute Question 0050. \\ \fi
\begin{problem}

Find the limit.  Use L'H$\hat{o}$pital's rule where appropriate.

\input{Derivative-Compute-0050.HELP.tex}

\[\lim\limits_{x\to\infty} {{\left(-\frac{5}{x} + 1\right)}^{-8 \, x} - 9}=\answer{e^{40} - 9}\]
\end{problem}}%}

\latexProblemContent{
\ifVerboseLocation This is Derivative Compute Question 0050. \\ \fi
\begin{problem}

Find the limit.  Use L'H$\hat{o}$pital's rule where appropriate.

\input{Derivative-Compute-0050.HELP.tex}

\[\lim\limits_{x\to\infty} {{\left(-\frac{2}{x} + 1\right)}^{6 \, x} + 19}=\answer{e^{\left(-12\right)} + 19}\]
\end{problem}}%}

\latexProblemContent{
\ifVerboseLocation This is Derivative Compute Question 0050. \\ \fi
\begin{problem}

Find the limit.  Use L'H$\hat{o}$pital's rule where appropriate.

\input{Derivative-Compute-0050.HELP.tex}

\[\lim\limits_{x\to\infty} {{\left(-\frac{4}{x} + 1\right)}^{5 \, x} + 2}=\answer{e^{\left(-20\right)} + 2}\]
\end{problem}}%}

\latexProblemContent{
\ifVerboseLocation This is Derivative Compute Question 0050. \\ \fi
\begin{problem}

Find the limit.  Use L'H$\hat{o}$pital's rule where appropriate.

\input{Derivative-Compute-0050.HELP.tex}

\[\lim\limits_{x\to\infty} {{\left(-\frac{1}{x} + 1\right)}^{6 \, x} + 6}=\answer{e^{\left(-6\right)} + 6}\]
\end{problem}}%}

\latexProblemContent{
\ifVerboseLocation This is Derivative Compute Question 0050. \\ \fi
\begin{problem}

Find the limit.  Use L'H$\hat{o}$pital's rule where appropriate.

\input{Derivative-Compute-0050.HELP.tex}

\[\lim\limits_{x\to\infty} {{\left(\frac{10}{x} + 1\right)}^{x} - 10}=\answer{e^{10} - 10}\]
\end{problem}}%}

\latexProblemContent{
\ifVerboseLocation This is Derivative Compute Question 0050. \\ \fi
\begin{problem}

Find the limit.  Use L'H$\hat{o}$pital's rule where appropriate.

\input{Derivative-Compute-0050.HELP.tex}

\[\lim\limits_{x\to\infty} {{\left(-\frac{1}{x} + 1\right)}^{3 \, x} + 17}=\answer{e^{\left(-3\right)} + 17}\]
\end{problem}}%}

\latexProblemContent{
\ifVerboseLocation This is Derivative Compute Question 0050. \\ \fi
\begin{problem}

Find the limit.  Use L'H$\hat{o}$pital's rule where appropriate.

\input{Derivative-Compute-0050.HELP.tex}

\[\lim\limits_{x\to\infty} {{\left(-\frac{4}{x} + 1\right)}^{-7 \, x} - 2}=\answer{e^{28} - 2}\]
\end{problem}}%}

\latexProblemContent{
\ifVerboseLocation This is Derivative Compute Question 0050. \\ \fi
\begin{problem}

Find the limit.  Use L'H$\hat{o}$pital's rule where appropriate.

\input{Derivative-Compute-0050.HELP.tex}

\[\lim\limits_{x\to\infty} {{\left(-\frac{7}{x} + 1\right)}^{-x} - 15}=\answer{e^{7} - 15}\]
\end{problem}}%}

\latexProblemContent{
\ifVerboseLocation This is Derivative Compute Question 0050. \\ \fi
\begin{problem}

Find the limit.  Use L'H$\hat{o}$pital's rule where appropriate.

\input{Derivative-Compute-0050.HELP.tex}

\[\lim\limits_{x\to\infty} {{\left(-\frac{8}{x} + 1\right)}^{7 \, x} - 2}=\answer{e^{\left(-56\right)} - 2}\]
\end{problem}}%}

\latexProblemContent{
\ifVerboseLocation This is Derivative Compute Question 0050. \\ \fi
\begin{problem}

Find the limit.  Use L'H$\hat{o}$pital's rule where appropriate.

\input{Derivative-Compute-0050.HELP.tex}

\[\lim\limits_{x\to\infty} {{\left(-\frac{2}{x} + 1\right)}^{-x} + 16}=\answer{e^{2} + 16}\]
\end{problem}}%}

\latexProblemContent{
\ifVerboseLocation This is Derivative Compute Question 0050. \\ \fi
\begin{problem}

Find the limit.  Use L'H$\hat{o}$pital's rule where appropriate.

\input{Derivative-Compute-0050.HELP.tex}

\[\lim\limits_{x\to\infty} {{\left(-\frac{5}{x} + 1\right)}^{10 \, x} + 6}=\answer{e^{\left(-50\right)} + 6}\]
\end{problem}}%}

\latexProblemContent{
\ifVerboseLocation This is Derivative Compute Question 0050. \\ \fi
\begin{problem}

Find the limit.  Use L'H$\hat{o}$pital's rule where appropriate.

\input{Derivative-Compute-0050.HELP.tex}

\[\lim\limits_{x\to\infty} {{\left(-\frac{3}{x} + 1\right)}^{3 \, x} + 9}=\answer{e^{\left(-9\right)} + 9}\]
\end{problem}}%}

\latexProblemContent{
\ifVerboseLocation This is Derivative Compute Question 0050. \\ \fi
\begin{problem}

Find the limit.  Use L'H$\hat{o}$pital's rule where appropriate.

\input{Derivative-Compute-0050.HELP.tex}

\[\lim\limits_{x\to\infty} {{\left(\frac{2}{x} + 1\right)}^{-7 \, x} + 12}=\answer{e^{\left(-14\right)} + 12}\]
\end{problem}}%}

\latexProblemContent{
\ifVerboseLocation This is Derivative Compute Question 0050. \\ \fi
\begin{problem}

Find the limit.  Use L'H$\hat{o}$pital's rule where appropriate.

\input{Derivative-Compute-0050.HELP.tex}

\[\lim\limits_{x\to\infty} {{\left(\frac{10}{x} + 1\right)}^{-9 \, x} - 7}=\answer{e^{\left(-90\right)} - 7}\]
\end{problem}}%}

\latexProblemContent{
\ifVerboseLocation This is Derivative Compute Question 0050. \\ \fi
\begin{problem}

Find the limit.  Use L'H$\hat{o}$pital's rule where appropriate.

\input{Derivative-Compute-0050.HELP.tex}

\[\lim\limits_{x\to\infty} {{\left(-\frac{5}{x} + 1\right)}^{2 \, x} + 7}=\answer{e^{\left(-10\right)} + 7}\]
\end{problem}}%}

\latexProblemContent{
\ifVerboseLocation This is Derivative Compute Question 0050. \\ \fi
\begin{problem}

Find the limit.  Use L'H$\hat{o}$pital's rule where appropriate.

\input{Derivative-Compute-0050.HELP.tex}

\[\lim\limits_{x\to\infty} {{\left(-\frac{7}{x} + 1\right)}^{5 \, x} - 1}=\answer{e^{\left(-35\right)} - 1}\]
\end{problem}}%}

\latexProblemContent{
\ifVerboseLocation This is Derivative Compute Question 0050. \\ \fi
\begin{problem}

Find the limit.  Use L'H$\hat{o}$pital's rule where appropriate.

\input{Derivative-Compute-0050.HELP.tex}

\[\lim\limits_{x\to\infty} {{\left(-\frac{4}{x} + 1\right)}^{-x} - 11}=\answer{e^{4} - 11}\]
\end{problem}}%}

\latexProblemContent{
\ifVerboseLocation This is Derivative Compute Question 0050. \\ \fi
\begin{problem}

Find the limit.  Use L'H$\hat{o}$pital's rule where appropriate.

\input{Derivative-Compute-0050.HELP.tex}

\[\lim\limits_{x\to\infty} {{\left(\frac{1}{x} + 1\right)}^{6 \, x} - 1}=\answer{e^{6} - 1}\]
\end{problem}}%}

\latexProblemContent{
\ifVerboseLocation This is Derivative Compute Question 0050. \\ \fi
\begin{problem}

Find the limit.  Use L'H$\hat{o}$pital's rule where appropriate.

\input{Derivative-Compute-0050.HELP.tex}

\[\lim\limits_{x\to\infty} {{\left(-\frac{9}{x} + 1\right)}^{3 \, x}}=\answer{e^{\left(-27\right)}}\]
\end{problem}}%}

\latexProblemContent{
\ifVerboseLocation This is Derivative Compute Question 0050. \\ \fi
\begin{problem}

Find the limit.  Use L'H$\hat{o}$pital's rule where appropriate.

\input{Derivative-Compute-0050.HELP.tex}

\[\lim\limits_{x\to\infty} {{\left(-\frac{8}{x} + 1\right)}^{-7 \, x} + 12}=\answer{e^{56} + 12}\]
\end{problem}}%}

\latexProblemContent{
\ifVerboseLocation This is Derivative Compute Question 0050. \\ \fi
\begin{problem}

Find the limit.  Use L'H$\hat{o}$pital's rule where appropriate.

\input{Derivative-Compute-0050.HELP.tex}

\[\lim\limits_{x\to\infty} {{\left(-\frac{5}{x} + 1\right)}^{-4 \, x} + 16}=\answer{e^{20} + 16}\]
\end{problem}}%}

\latexProblemContent{
\ifVerboseLocation This is Derivative Compute Question 0050. \\ \fi
\begin{problem}

Find the limit.  Use L'H$\hat{o}$pital's rule where appropriate.

\input{Derivative-Compute-0050.HELP.tex}

\[\lim\limits_{x\to\infty} {{\left(-\frac{8}{x} + 1\right)}^{-6 \, x} - 6}=\answer{e^{48} - 6}\]
\end{problem}}%}

\latexProblemContent{
\ifVerboseLocation This is Derivative Compute Question 0050. \\ \fi
\begin{problem}

Find the limit.  Use L'H$\hat{o}$pital's rule where appropriate.

\input{Derivative-Compute-0050.HELP.tex}

\[\lim\limits_{x\to\infty} {{\left(-\frac{10}{x} + 1\right)}^{x} + 5}=\answer{e^{\left(-10\right)} + 5}\]
\end{problem}}%}

\latexProblemContent{
\ifVerboseLocation This is Derivative Compute Question 0050. \\ \fi
\begin{problem}

Find the limit.  Use L'H$\hat{o}$pital's rule where appropriate.

\input{Derivative-Compute-0050.HELP.tex}

\[\lim\limits_{x\to\infty} {{\left(\frac{7}{x} + 1\right)}^{-5 \, x} + 15}=\answer{e^{\left(-35\right)} + 15}\]
\end{problem}}%}

\latexProblemContent{
\ifVerboseLocation This is Derivative Compute Question 0050. \\ \fi
\begin{problem}

Find the limit.  Use L'H$\hat{o}$pital's rule where appropriate.

\input{Derivative-Compute-0050.HELP.tex}

\[\lim\limits_{x\to\infty} {{\left(\frac{10}{x} + 1\right)}^{-3 \, x} - 19}=\answer{e^{\left(-30\right)} - 19}\]
\end{problem}}%}

\latexProblemContent{
\ifVerboseLocation This is Derivative Compute Question 0050. \\ \fi
\begin{problem}

Find the limit.  Use L'H$\hat{o}$pital's rule where appropriate.

\input{Derivative-Compute-0050.HELP.tex}

\[\lim\limits_{x\to\infty} {{\left(-\frac{1}{x} + 1\right)}^{10 \, x} - 20}=\answer{e^{\left(-10\right)} - 20}\]
\end{problem}}%}

\latexProblemContent{
\ifVerboseLocation This is Derivative Compute Question 0050. \\ \fi
\begin{problem}

Find the limit.  Use L'H$\hat{o}$pital's rule where appropriate.

\input{Derivative-Compute-0050.HELP.tex}

\[\lim\limits_{x\to\infty} {{\left(\frac{4}{x} + 1\right)}^{2 \, x} - 20}=\answer{e^{8} - 20}\]
\end{problem}}%}

\latexProblemContent{
\ifVerboseLocation This is Derivative Compute Question 0050. \\ \fi
\begin{problem}

Find the limit.  Use L'H$\hat{o}$pital's rule where appropriate.

\input{Derivative-Compute-0050.HELP.tex}

\[\lim\limits_{x\to\infty} {{\left(\frac{9}{x} + 1\right)}^{4 \, x} - 18}=\answer{e^{36} - 18}\]
\end{problem}}%}

\latexProblemContent{
\ifVerboseLocation This is Derivative Compute Question 0050. \\ \fi
\begin{problem}

Find the limit.  Use L'H$\hat{o}$pital's rule where appropriate.

\input{Derivative-Compute-0050.HELP.tex}

\[\lim\limits_{x\to\infty} {{\left(-\frac{3}{x} + 1\right)}^{5 \, x} - 5}=\answer{e^{\left(-15\right)} - 5}\]
\end{problem}}%}

\latexProblemContent{
\ifVerboseLocation This is Derivative Compute Question 0050. \\ \fi
\begin{problem}

Find the limit.  Use L'H$\hat{o}$pital's rule where appropriate.

\input{Derivative-Compute-0050.HELP.tex}

\[\lim\limits_{x\to\infty} {{\left(-\frac{3}{x} + 1\right)}^{-6 \, x}}=\answer{e^{18}}\]
\end{problem}}%}

\latexProblemContent{
\ifVerboseLocation This is Derivative Compute Question 0050. \\ \fi
\begin{problem}

Find the limit.  Use L'H$\hat{o}$pital's rule where appropriate.

\input{Derivative-Compute-0050.HELP.tex}

\[\lim\limits_{x\to\infty} {{\left(\frac{3}{x} + 1\right)}^{-x} + 8}=\answer{e^{\left(-3\right)} + 8}\]
\end{problem}}%}

\latexProblemContent{
\ifVerboseLocation This is Derivative Compute Question 0050. \\ \fi
\begin{problem}

Find the limit.  Use L'H$\hat{o}$pital's rule where appropriate.

\input{Derivative-Compute-0050.HELP.tex}

\[\lim\limits_{x\to\infty} {{\left(\frac{5}{x} + 1\right)}^{7 \, x} + 20}=\answer{e^{35} + 20}\]
\end{problem}}%}

\latexProblemContent{
\ifVerboseLocation This is Derivative Compute Question 0050. \\ \fi
\begin{problem}

Find the limit.  Use L'H$\hat{o}$pital's rule where appropriate.

\input{Derivative-Compute-0050.HELP.tex}

\[\lim\limits_{x\to\infty} {{\left(\frac{8}{x} + 1\right)}^{8 \, x} + 15}=\answer{e^{64} + 15}\]
\end{problem}}%}

\latexProblemContent{
\ifVerboseLocation This is Derivative Compute Question 0050. \\ \fi
\begin{problem}

Find the limit.  Use L'H$\hat{o}$pital's rule where appropriate.

\input{Derivative-Compute-0050.HELP.tex}

\[\lim\limits_{x\to\infty} {{\left(-\frac{9}{x} + 1\right)}^{7 \, x} - 2}=\answer{e^{\left(-63\right)} - 2}\]
\end{problem}}%}

\latexProblemContent{
\ifVerboseLocation This is Derivative Compute Question 0050. \\ \fi
\begin{problem}

Find the limit.  Use L'H$\hat{o}$pital's rule where appropriate.

\input{Derivative-Compute-0050.HELP.tex}

\[\lim\limits_{x\to\infty} {{\left(\frac{4}{x} + 1\right)}^{-7 \, x} + 17}=\answer{e^{\left(-28\right)} + 17}\]
\end{problem}}%}

\latexProblemContent{
\ifVerboseLocation This is Derivative Compute Question 0050. \\ \fi
\begin{problem}

Find the limit.  Use L'H$\hat{o}$pital's rule where appropriate.

\input{Derivative-Compute-0050.HELP.tex}

\[\lim\limits_{x\to\infty} {{\left(-\frac{6}{x} + 1\right)}^{10 \, x} - 4}=\answer{e^{\left(-60\right)} - 4}\]
\end{problem}}%}

\latexProblemContent{
\ifVerboseLocation This is Derivative Compute Question 0050. \\ \fi
\begin{problem}

Find the limit.  Use L'H$\hat{o}$pital's rule where appropriate.

\input{Derivative-Compute-0050.HELP.tex}

\[\lim\limits_{x\to\infty} {{\left(\frac{6}{x} + 1\right)}^{-8 \, x} + 1}=\answer{e^{\left(-48\right)} + 1}\]
\end{problem}}%}

\latexProblemContent{
\ifVerboseLocation This is Derivative Compute Question 0050. \\ \fi
\begin{problem}

Find the limit.  Use L'H$\hat{o}$pital's rule where appropriate.

\input{Derivative-Compute-0050.HELP.tex}

\[\lim\limits_{x\to\infty} {{\left(-\frac{1}{x} + 1\right)}^{-5 \, x} + 19}=\answer{e^{5} + 19}\]
\end{problem}}%}

\latexProblemContent{
\ifVerboseLocation This is Derivative Compute Question 0050. \\ \fi
\begin{problem}

Find the limit.  Use L'H$\hat{o}$pital's rule where appropriate.

\input{Derivative-Compute-0050.HELP.tex}

\[\lim\limits_{x\to\infty} {{\left(\frac{7}{x} + 1\right)}^{-7 \, x} - 1}=\answer{e^{\left(-49\right)} - 1}\]
\end{problem}}%}

\latexProblemContent{
\ifVerboseLocation This is Derivative Compute Question 0050. \\ \fi
\begin{problem}

Find the limit.  Use L'H$\hat{o}$pital's rule where appropriate.

\input{Derivative-Compute-0050.HELP.tex}

\[\lim\limits_{x\to\infty} {{\left(\frac{10}{x} + 1\right)}^{-6 \, x} - 10}=\answer{e^{\left(-60\right)} - 10}\]
\end{problem}}%}

\latexProblemContent{
\ifVerboseLocation This is Derivative Compute Question 0050. \\ \fi
\begin{problem}

Find the limit.  Use L'H$\hat{o}$pital's rule where appropriate.

\input{Derivative-Compute-0050.HELP.tex}

\[\lim\limits_{x\to\infty} {{\left(\frac{6}{x} + 1\right)}^{-10 \, x} - 1}=\answer{e^{\left(-60\right)} - 1}\]
\end{problem}}%}

\latexProblemContent{
\ifVerboseLocation This is Derivative Compute Question 0050. \\ \fi
\begin{problem}

Find the limit.  Use L'H$\hat{o}$pital's rule where appropriate.

\input{Derivative-Compute-0050.HELP.tex}

\[\lim\limits_{x\to\infty} {{\left(\frac{5}{x} + 1\right)}^{2 \, x} + 10}=\answer{e^{10} + 10}\]
\end{problem}}%}

\latexProblemContent{
\ifVerboseLocation This is Derivative Compute Question 0050. \\ \fi
\begin{problem}

Find the limit.  Use L'H$\hat{o}$pital's rule where appropriate.

\input{Derivative-Compute-0050.HELP.tex}

\[\lim\limits_{x\to\infty} {{\left(-\frac{10}{x} + 1\right)}^{-5 \, x} - 7}=\answer{e^{50} - 7}\]
\end{problem}}%}

\latexProblemContent{
\ifVerboseLocation This is Derivative Compute Question 0050. \\ \fi
\begin{problem}

Find the limit.  Use L'H$\hat{o}$pital's rule where appropriate.

\input{Derivative-Compute-0050.HELP.tex}

\[\lim\limits_{x\to\infty} {{\left(-\frac{4}{x} + 1\right)}^{-6 \, x} + 19}=\answer{e^{24} + 19}\]
\end{problem}}%}

\latexProblemContent{
\ifVerboseLocation This is Derivative Compute Question 0050. \\ \fi
\begin{problem}

Find the limit.  Use L'H$\hat{o}$pital's rule where appropriate.

\input{Derivative-Compute-0050.HELP.tex}

\[\lim\limits_{x\to\infty} {{\left(-\frac{3}{x} + 1\right)}^{-10 \, x} - 19}=\answer{e^{30} - 19}\]
\end{problem}}%}

\latexProblemContent{
\ifVerboseLocation This is Derivative Compute Question 0050. \\ \fi
\begin{problem}

Find the limit.  Use L'H$\hat{o}$pital's rule where appropriate.

\input{Derivative-Compute-0050.HELP.tex}

\[\lim\limits_{x\to\infty} {{\left(\frac{4}{x} + 1\right)}^{-2 \, x} + 16}=\answer{e^{\left(-8\right)} + 16}\]
\end{problem}}%}

\latexProblemContent{
\ifVerboseLocation This is Derivative Compute Question 0050. \\ \fi
\begin{problem}

Find the limit.  Use L'H$\hat{o}$pital's rule where appropriate.

\input{Derivative-Compute-0050.HELP.tex}

\[\lim\limits_{x\to\infty} {{\left(\frac{8}{x} + 1\right)}^{8 \, x} - 8}=\answer{e^{64} - 8}\]
\end{problem}}%}

\latexProblemContent{
\ifVerboseLocation This is Derivative Compute Question 0050. \\ \fi
\begin{problem}

Find the limit.  Use L'H$\hat{o}$pital's rule where appropriate.

\input{Derivative-Compute-0050.HELP.tex}

\[\lim\limits_{x\to\infty} {{\left(-\frac{6}{x} + 1\right)}^{6 \, x} + 2}=\answer{e^{\left(-36\right)} + 2}\]
\end{problem}}%}

