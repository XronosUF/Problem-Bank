\ProblemFileHeader{XTL_SV_QUESTIONCOUNT}% Process how many problems are in this file and how to detect if it has a desirable problem
\ifproblemToFind% If it has a desirable problem search the file.
%%\tagged{Ans@MC, Type@Concept, Topic@Integral, Sub@Definite, Sub@Theorems-FTC, Func@Trig, File@0003}{
\latexProblemContent{
\ifVerboseLocation This is Integration Concept Question 0003. \\ \fi
\begin{problem}

What is wrong with the following equation:

\[
\int_{\frac{5}{6} \, \pi}^{\frac{3}{2} \, \pi} {-2 \, \cot\left(x\right) \csc\left(x\right)}\;dx = {\frac{2}{\sin\left(x\right)}}\Bigg\vert_{\frac{5}{6} \, \pi}^{\frac{3}{2} \, \pi} = {-6}
\]

\input{Integral-Concept-0003.HELP.tex}

\begin{multipleChoice}
\choice{The antiderivative is incorrect.}
\choice[correct]{The integrand is not defined over the entire interval.}
\choice{The bounds are evaluated in the wrong order.}
\choice{Nothing is wrong.  The equation is correct, as is.}
\end{multipleChoice}

\end{problem}}%}

\latexProblemContent{
\ifVerboseLocation This is Integration Concept Question 0003. \\ \fi
\begin{problem}

What is wrong with the following equation:

\[
\int_{\frac{5}{6} \, \pi}^{\frac{5}{4} \, \pi} {2 \, \cot\left(x\right) \csc\left(x\right)}\;dx = {-\frac{2}{\sin\left(x\right)}}\Bigg\vert_{\frac{5}{6} \, \pi}^{\frac{5}{4} \, \pi} = {2 \, \sqrt{2} + 4}
\]

\input{Integral-Concept-0003.HELP.tex}

\begin{multipleChoice}
\choice{The antiderivative is incorrect.}
\choice[correct]{The integrand is not defined over the entire interval.}
\choice{The bounds are evaluated in the wrong order.}
\choice{Nothing is wrong.  The equation is correct, as is.}
\end{multipleChoice}

\end{problem}}%}

\latexProblemContent{
\ifVerboseLocation This is Integration Concept Question 0003. \\ \fi
\begin{problem}

What is wrong with the following equation:

\[
\int_{\frac{5}{6} \, \pi}^{\frac{4}{3} \, \pi} {10 \, \cot\left(x\right) \csc\left(x\right)}\;dx = {-\frac{10}{\sin\left(x\right)}}\Bigg\vert_{\frac{5}{6} \, \pi}^{\frac{4}{3} \, \pi} = {\frac{20}{3} \, \sqrt{3} + 20}
\]

\input{Integral-Concept-0003.HELP.tex}

\begin{multipleChoice}
\choice{The antiderivative is incorrect.}
\choice[correct]{The integrand is not defined over the entire interval.}
\choice{The bounds are evaluated in the wrong order.}
\choice{Nothing is wrong.  The equation is correct, as is.}
\end{multipleChoice}

\end{problem}}%}

\latexProblemContent{
\ifVerboseLocation This is Integration Concept Question 0003. \\ \fi
\begin{problem}

What is wrong with the following equation:

\[
\int_{\frac{1}{6} \, \pi}^{\frac{3}{2} \, \pi} {-7 \, \csc\left(x\right)^{2}}\;dx = {\frac{7}{\tan\left(x\right)}}\Bigg\vert_{\frac{1}{6} \, \pi}^{\frac{3}{2} \, \pi} = {-7 \, \sqrt{3}}
\]

\input{Integral-Concept-0003.HELP.tex}

\begin{multipleChoice}
\choice{The antiderivative is incorrect.}
\choice[correct]{The integrand is not defined over the entire interval.}
\choice{The bounds are evaluated in the wrong order.}
\choice{Nothing is wrong.  The equation is correct, as is.}
\end{multipleChoice}

\end{problem}}%}

\latexProblemContent{
\ifVerboseLocation This is Integration Concept Question 0003. \\ \fi
\begin{problem}

What is wrong with the following equation:

\[
\int_{\frac{1}{2} \, \pi}^{\frac{11}{6} \, \pi} {4 \, \csc\left(x\right)^{2}}\;dx = {-\frac{4}{\tan\left(x\right)}}\Bigg\vert_{\frac{1}{2} \, \pi}^{\frac{11}{6} \, \pi} = {4 \, \sqrt{3}}
\]

\input{Integral-Concept-0003.HELP.tex}

\begin{multipleChoice}
\choice{The antiderivative is incorrect.}
\choice[correct]{The integrand is not defined over the entire interval.}
\choice{The bounds are evaluated in the wrong order.}
\choice{Nothing is wrong.  The equation is correct, as is.}
\end{multipleChoice}

\end{problem}}%}

\latexProblemContent{
\ifVerboseLocation This is Integration Concept Question 0003. \\ \fi
\begin{problem}

What is wrong with the following equation:

\[
\int_{\frac{5}{6} \, \pi}^{\frac{7}{6} \, \pi} {-7 \, \csc\left(x\right)^{2}}\;dx = {\frac{7}{\tan\left(x\right)}}\Bigg\vert_{\frac{5}{6} \, \pi}^{\frac{7}{6} \, \pi} = {14 \, \sqrt{3}}
\]

\input{Integral-Concept-0003.HELP.tex}

\begin{multipleChoice}
\choice{The antiderivative is incorrect.}
\choice[correct]{The integrand is not defined over the entire interval.}
\choice{The bounds are evaluated in the wrong order.}
\choice{Nothing is wrong.  The equation is correct, as is.}
\end{multipleChoice}

\end{problem}}%}

\latexProblemContent{
\ifVerboseLocation This is Integration Concept Question 0003. \\ \fi
\begin{problem}

What is wrong with the following equation:

\[
\int_{\frac{1}{3} \, \pi}^{\frac{11}{6} \, \pi} {3 \, \cot\left(x\right) \csc\left(x\right)}\;dx = {-\frac{3}{\sin\left(x\right)}}\Bigg\vert_{\frac{1}{3} \, \pi}^{\frac{11}{6} \, \pi} = {2 \, \sqrt{3} + 6}
\]

\input{Integral-Concept-0003.HELP.tex}

\begin{multipleChoice}
\choice{The antiderivative is incorrect.}
\choice[correct]{The integrand is not defined over the entire interval.}
\choice{The bounds are evaluated in the wrong order.}
\choice{Nothing is wrong.  The equation is correct, as is.}
\end{multipleChoice}

\end{problem}}%}

\latexProblemContent{
\ifVerboseLocation This is Integration Concept Question 0003. \\ \fi
\begin{problem}

What is wrong with the following equation:

\[
\int_{\frac{3}{4} \, \pi}^{\frac{4}{3} \, \pi} {-\csc\left(x\right)^{2}}\;dx = {\frac{1}{\tan\left(x\right)}}\Bigg\vert_{\frac{3}{4} \, \pi}^{\frac{4}{3} \, \pi} = {\frac{1}{3} \, \sqrt{3} + 1}
\]

\input{Integral-Concept-0003.HELP.tex}

\begin{multipleChoice}
\choice{The antiderivative is incorrect.}
\choice[correct]{The integrand is not defined over the entire interval.}
\choice{The bounds are evaluated in the wrong order.}
\choice{Nothing is wrong.  The equation is correct, as is.}
\end{multipleChoice}

\end{problem}}%}

\latexProblemContent{
\ifVerboseLocation This is Integration Concept Question 0003. \\ \fi
\begin{problem}

What is wrong with the following equation:

\[
\int_{\frac{1}{2} \, \pi}^{\frac{3}{2} \, \pi} {-9 \, \csc\left(x\right)^{2}}\;dx = {\frac{9}{\tan\left(x\right)}}\Bigg\vert_{\frac{1}{2} \, \pi}^{\frac{3}{2} \, \pi} = {0}
\]

\input{Integral-Concept-0003.HELP.tex}

\begin{multipleChoice}
\choice{The antiderivative is incorrect.}
\choice[correct]{The integrand is not defined over the entire interval.}
\choice{The bounds are evaluated in the wrong order.}
\choice{Nothing is wrong.  The equation is correct, as is.}
\end{multipleChoice}

\end{problem}}%}

\latexProblemContent{
\ifVerboseLocation This is Integration Concept Question 0003. \\ \fi
\begin{problem}

What is wrong with the following equation:

\[
\int_{\frac{1}{4} \, \pi}^{\frac{5}{3} \, \pi} {5 \, \csc\left(x\right)^{2}}\;dx = {-\frac{5}{\tan\left(x\right)}}\Bigg\vert_{\frac{1}{4} \, \pi}^{\frac{5}{3} \, \pi} = {\frac{5}{3} \, \sqrt{3} + 5}
\]

\input{Integral-Concept-0003.HELP.tex}

\begin{multipleChoice}
\choice{The antiderivative is incorrect.}
\choice[correct]{The integrand is not defined over the entire interval.}
\choice{The bounds are evaluated in the wrong order.}
\choice{Nothing is wrong.  The equation is correct, as is.}
\end{multipleChoice}

\end{problem}}%}

\latexProblemContent{
\ifVerboseLocation This is Integration Concept Question 0003. \\ \fi
\begin{problem}

What is wrong with the following equation:

\[
\int_{\frac{1}{3} \, \pi}^{\frac{7}{4} \, \pi} {5 \, \csc\left(x\right)^{2}}\;dx = {-\frac{5}{\tan\left(x\right)}}\Bigg\vert_{\frac{1}{3} \, \pi}^{\frac{7}{4} \, \pi} = {\frac{5}{3} \, \sqrt{3} + 5}
\]

\input{Integral-Concept-0003.HELP.tex}

\begin{multipleChoice}
\choice{The antiderivative is incorrect.}
\choice[correct]{The integrand is not defined over the entire interval.}
\choice{The bounds are evaluated in the wrong order.}
\choice{Nothing is wrong.  The equation is correct, as is.}
\end{multipleChoice}

\end{problem}}%}

\latexProblemContent{
\ifVerboseLocation This is Integration Concept Question 0003. \\ \fi
\begin{problem}

What is wrong with the following equation:

\[
\int_{\frac{1}{2} \, \pi}^{\frac{5}{3} \, \pi} {-10 \, \cot\left(x\right) \csc\left(x\right)}\;dx = {\frac{10}{\sin\left(x\right)}}\Bigg\vert_{\frac{1}{2} \, \pi}^{\frac{5}{3} \, \pi} = {-\frac{20}{3} \, \sqrt{3} - 10}
\]

\input{Integral-Concept-0003.HELP.tex}

\begin{multipleChoice}
\choice{The antiderivative is incorrect.}
\choice[correct]{The integrand is not defined over the entire interval.}
\choice{The bounds are evaluated in the wrong order.}
\choice{Nothing is wrong.  The equation is correct, as is.}
\end{multipleChoice}

\end{problem}}%}

\latexProblemContent{
\ifVerboseLocation This is Integration Concept Question 0003. \\ \fi
\begin{problem}

What is wrong with the following equation:

\[
\int_{\frac{1}{2} \, \pi}^{\frac{11}{6} \, \pi} {-8 \, \csc\left(x\right)^{2}}\;dx = {\frac{8}{\tan\left(x\right)}}\Bigg\vert_{\frac{1}{2} \, \pi}^{\frac{11}{6} \, \pi} = {-8 \, \sqrt{3}}
\]

\input{Integral-Concept-0003.HELP.tex}

\begin{multipleChoice}
\choice{The antiderivative is incorrect.}
\choice[correct]{The integrand is not defined over the entire interval.}
\choice{The bounds are evaluated in the wrong order.}
\choice{Nothing is wrong.  The equation is correct, as is.}
\end{multipleChoice}

\end{problem}}%}

\latexProblemContent{
\ifVerboseLocation This is Integration Concept Question 0003. \\ \fi
\begin{problem}

What is wrong with the following equation:

\[
\int_{\frac{3}{4} \, \pi}^{\frac{4}{3} \, \pi} {-7 \, \cot\left(x\right) \csc\left(x\right)}\;dx = {\frac{7}{\sin\left(x\right)}}\Bigg\vert_{\frac{3}{4} \, \pi}^{\frac{4}{3} \, \pi} = {-\frac{14}{3} \, \sqrt{3} - 7 \, \sqrt{2}}
\]

\input{Integral-Concept-0003.HELP.tex}

\begin{multipleChoice}
\choice{The antiderivative is incorrect.}
\choice[correct]{The integrand is not defined over the entire interval.}
\choice{The bounds are evaluated in the wrong order.}
\choice{Nothing is wrong.  The equation is correct, as is.}
\end{multipleChoice}

\end{problem}}%}

\latexProblemContent{
\ifVerboseLocation This is Integration Concept Question 0003. \\ \fi
\begin{problem}

What is wrong with the following equation:

\[
\int_{\frac{1}{3} \, \pi}^{\frac{5}{4} \, \pi} {2 \, \csc\left(x\right)^{2}}\;dx = {-\frac{2}{\tan\left(x\right)}}\Bigg\vert_{\frac{1}{3} \, \pi}^{\frac{5}{4} \, \pi} = {\frac{2}{3} \, \sqrt{3} - 2}
\]

\input{Integral-Concept-0003.HELP.tex}

\begin{multipleChoice}
\choice{The antiderivative is incorrect.}
\choice[correct]{The integrand is not defined over the entire interval.}
\choice{The bounds are evaluated in the wrong order.}
\choice{Nothing is wrong.  The equation is correct, as is.}
\end{multipleChoice}

\end{problem}}%}

\latexProblemContent{
\ifVerboseLocation This is Integration Concept Question 0003. \\ \fi
\begin{problem}

What is wrong with the following equation:

\[
\int_{\frac{1}{6} \, \pi}^{\frac{7}{6} \, \pi} {10 \, \cot\left(x\right) \csc\left(x\right)}\;dx = {-\frac{10}{\sin\left(x\right)}}\Bigg\vert_{\frac{1}{6} \, \pi}^{\frac{7}{6} \, \pi} = {40}
\]

\input{Integral-Concept-0003.HELP.tex}

\begin{multipleChoice}
\choice{The antiderivative is incorrect.}
\choice[correct]{The integrand is not defined over the entire interval.}
\choice{The bounds are evaluated in the wrong order.}
\choice{Nothing is wrong.  The equation is correct, as is.}
\end{multipleChoice}

\end{problem}}%}

\latexProblemContent{
\ifVerboseLocation This is Integration Concept Question 0003. \\ \fi
\begin{problem}

What is wrong with the following equation:

\[
\int_{\frac{5}{6} \, \pi}^{\frac{5}{4} \, \pi} {-10 \, \csc\left(x\right)^{2}}\;dx = {\frac{10}{\tan\left(x\right)}}\Bigg\vert_{\frac{5}{6} \, \pi}^{\frac{5}{4} \, \pi} = {10 \, \sqrt{3} + 10}
\]

\input{Integral-Concept-0003.HELP.tex}

\begin{multipleChoice}
\choice{The antiderivative is incorrect.}
\choice[correct]{The integrand is not defined over the entire interval.}
\choice{The bounds are evaluated in the wrong order.}
\choice{Nothing is wrong.  The equation is correct, as is.}
\end{multipleChoice}

\end{problem}}%}

\latexProblemContent{
\ifVerboseLocation This is Integration Concept Question 0003. \\ \fi
\begin{problem}

What is wrong with the following equation:

\[
\int_{\frac{3}{4} \, \pi}^{\frac{4}{3} \, \pi} {-6 \, \csc\left(x\right)^{2}}\;dx = {\frac{6}{\tan\left(x\right)}}\Bigg\vert_{\frac{3}{4} \, \pi}^{\frac{4}{3} \, \pi} = {2 \, \sqrt{3} + 6}
\]

\input{Integral-Concept-0003.HELP.tex}

\begin{multipleChoice}
\choice{The antiderivative is incorrect.}
\choice[correct]{The integrand is not defined over the entire interval.}
\choice{The bounds are evaluated in the wrong order.}
\choice{Nothing is wrong.  The equation is correct, as is.}
\end{multipleChoice}

\end{problem}}%}

\latexProblemContent{
\ifVerboseLocation This is Integration Concept Question 0003. \\ \fi
\begin{problem}

What is wrong with the following equation:

\[
\int_{\frac{1}{4} \, \pi}^{\frac{4}{3} \, \pi} {10 \, \csc\left(x\right)^{2}}\;dx = {-\frac{10}{\tan\left(x\right)}}\Bigg\vert_{\frac{1}{4} \, \pi}^{\frac{4}{3} \, \pi} = {-\frac{10}{3} \, \sqrt{3} + 10}
\]

\input{Integral-Concept-0003.HELP.tex}

\begin{multipleChoice}
\choice{The antiderivative is incorrect.}
\choice[correct]{The integrand is not defined over the entire interval.}
\choice{The bounds are evaluated in the wrong order.}
\choice{Nothing is wrong.  The equation is correct, as is.}
\end{multipleChoice}

\end{problem}}%}

\latexProblemContent{
\ifVerboseLocation This is Integration Concept Question 0003. \\ \fi
\begin{problem}

What is wrong with the following equation:

\[
\int_{\frac{5}{6} \, \pi}^{\frac{5}{4} \, \pi} {-3 \, \cot\left(x\right) \csc\left(x\right)}\;dx = {\frac{3}{\sin\left(x\right)}}\Bigg\vert_{\frac{5}{6} \, \pi}^{\frac{5}{4} \, \pi} = {-3 \, \sqrt{2} - 6}
\]

\input{Integral-Concept-0003.HELP.tex}

\begin{multipleChoice}
\choice{The antiderivative is incorrect.}
\choice[correct]{The integrand is not defined over the entire interval.}
\choice{The bounds are evaluated in the wrong order.}
\choice{Nothing is wrong.  The equation is correct, as is.}
\end{multipleChoice}

\end{problem}}%}

\latexProblemContent{
\ifVerboseLocation This is Integration Concept Question 0003. \\ \fi
\begin{problem}

What is wrong with the following equation:

\[
\int_{\frac{1}{6} \, \pi}^{\frac{5}{3} \, \pi} {6 \, \cot\left(x\right) \csc\left(x\right)}\;dx = {-\frac{6}{\sin\left(x\right)}}\Bigg\vert_{\frac{1}{6} \, \pi}^{\frac{5}{3} \, \pi} = {4 \, \sqrt{3} + 12}
\]

\input{Integral-Concept-0003.HELP.tex}

\begin{multipleChoice}
\choice{The antiderivative is incorrect.}
\choice[correct]{The integrand is not defined over the entire interval.}
\choice{The bounds are evaluated in the wrong order.}
\choice{Nothing is wrong.  The equation is correct, as is.}
\end{multipleChoice}

\end{problem}}%}

\latexProblemContent{
\ifVerboseLocation This is Integration Concept Question 0003. \\ \fi
\begin{problem}

What is wrong with the following equation:

\[
\int_{\frac{5}{6} \, \pi}^{\frac{5}{3} \, \pi} {-6 \, \csc\left(x\right)^{2}}\;dx = {\frac{6}{\tan\left(x\right)}}\Bigg\vert_{\frac{5}{6} \, \pi}^{\frac{5}{3} \, \pi} = {4 \, \sqrt{3}}
\]

\input{Integral-Concept-0003.HELP.tex}

\begin{multipleChoice}
\choice{The antiderivative is incorrect.}
\choice[correct]{The integrand is not defined over the entire interval.}
\choice{The bounds are evaluated in the wrong order.}
\choice{Nothing is wrong.  The equation is correct, as is.}
\end{multipleChoice}

\end{problem}}%}

\latexProblemContent{
\ifVerboseLocation This is Integration Concept Question 0003. \\ \fi
\begin{problem}

What is wrong with the following equation:

\[
\int_{\frac{3}{4} \, \pi}^{\frac{3}{2} \, \pi} {-10 \, \cot\left(x\right) \csc\left(x\right)}\;dx = {\frac{10}{\sin\left(x\right)}}\Bigg\vert_{\frac{3}{4} \, \pi}^{\frac{3}{2} \, \pi} = {-10 \, \sqrt{2} - 10}
\]

\input{Integral-Concept-0003.HELP.tex}

\begin{multipleChoice}
\choice{The antiderivative is incorrect.}
\choice[correct]{The integrand is not defined over the entire interval.}
\choice{The bounds are evaluated in the wrong order.}
\choice{Nothing is wrong.  The equation is correct, as is.}
\end{multipleChoice}

\end{problem}}%}

\latexProblemContent{
\ifVerboseLocation This is Integration Concept Question 0003. \\ \fi
\begin{problem}

What is wrong with the following equation:

\[
\int_{\frac{1}{3} \, \pi}^{\frac{7}{4} \, \pi} {7 \, \csc\left(x\right)^{2}}\;dx = {-\frac{7}{\tan\left(x\right)}}\Bigg\vert_{\frac{1}{3} \, \pi}^{\frac{7}{4} \, \pi} = {\frac{7}{3} \, \sqrt{3} + 7}
\]

\input{Integral-Concept-0003.HELP.tex}

\begin{multipleChoice}
\choice{The antiderivative is incorrect.}
\choice[correct]{The integrand is not defined over the entire interval.}
\choice{The bounds are evaluated in the wrong order.}
\choice{Nothing is wrong.  The equation is correct, as is.}
\end{multipleChoice}

\end{problem}}%}

\latexProblemContent{
\ifVerboseLocation This is Integration Concept Question 0003. \\ \fi
\begin{problem}

What is wrong with the following equation:

\[
\int_{\frac{3}{4} \, \pi}^{\frac{4}{3} \, \pi} {-6 \, \cot\left(x\right) \csc\left(x\right)}\;dx = {\frac{6}{\sin\left(x\right)}}\Bigg\vert_{\frac{3}{4} \, \pi}^{\frac{4}{3} \, \pi} = {-4 \, \sqrt{3} - 6 \, \sqrt{2}}
\]

\input{Integral-Concept-0003.HELP.tex}

\begin{multipleChoice}
\choice{The antiderivative is incorrect.}
\choice[correct]{The integrand is not defined over the entire interval.}
\choice{The bounds are evaluated in the wrong order.}
\choice{Nothing is wrong.  The equation is correct, as is.}
\end{multipleChoice}

\end{problem}}%}

\latexProblemContent{
\ifVerboseLocation This is Integration Concept Question 0003. \\ \fi
\begin{problem}

What is wrong with the following equation:

\[
\int_{\frac{1}{4} \, \pi}^{\frac{5}{4} \, \pi} {3 \, \csc\left(x\right)^{2}}\;dx = {-\frac{3}{\tan\left(x\right)}}\Bigg\vert_{\frac{1}{4} \, \pi}^{\frac{5}{4} \, \pi} = {0}
\]

\input{Integral-Concept-0003.HELP.tex}

\begin{multipleChoice}
\choice{The antiderivative is incorrect.}
\choice[correct]{The integrand is not defined over the entire interval.}
\choice{The bounds are evaluated in the wrong order.}
\choice{Nothing is wrong.  The equation is correct, as is.}
\end{multipleChoice}

\end{problem}}%}

\latexProblemContent{
\ifVerboseLocation This is Integration Concept Question 0003. \\ \fi
\begin{problem}

What is wrong with the following equation:

\[
\int_{\frac{1}{3} \, \pi}^{\frac{11}{6} \, \pi} {10 \, \csc\left(x\right)^{2}}\;dx = {-\frac{10}{\tan\left(x\right)}}\Bigg\vert_{\frac{1}{3} \, \pi}^{\frac{11}{6} \, \pi} = {\frac{40}{3} \, \sqrt{3}}
\]

\input{Integral-Concept-0003.HELP.tex}

\begin{multipleChoice}
\choice{The antiderivative is incorrect.}
\choice[correct]{The integrand is not defined over the entire interval.}
\choice{The bounds are evaluated in the wrong order.}
\choice{Nothing is wrong.  The equation is correct, as is.}
\end{multipleChoice}

\end{problem}}%}

\latexProblemContent{
\ifVerboseLocation This is Integration Concept Question 0003. \\ \fi
\begin{problem}

What is wrong with the following equation:

\[
\int_{\frac{5}{6} \, \pi}^{\frac{4}{3} \, \pi} {5 \, \csc\left(x\right)^{2}}\;dx = {-\frac{5}{\tan\left(x\right)}}\Bigg\vert_{\frac{5}{6} \, \pi}^{\frac{4}{3} \, \pi} = {-\frac{20}{3} \, \sqrt{3}}
\]

\input{Integral-Concept-0003.HELP.tex}

\begin{multipleChoice}
\choice{The antiderivative is incorrect.}
\choice[correct]{The integrand is not defined over the entire interval.}
\choice{The bounds are evaluated in the wrong order.}
\choice{Nothing is wrong.  The equation is correct, as is.}
\end{multipleChoice}

\end{problem}}%}

\latexProblemContent{
\ifVerboseLocation This is Integration Concept Question 0003. \\ \fi
\begin{problem}

What is wrong with the following equation:

\[
\int_{\frac{5}{6} \, \pi}^{\frac{7}{4} \, \pi} {-3 \, \cot\left(x\right) \csc\left(x\right)}\;dx = {\frac{3}{\sin\left(x\right)}}\Bigg\vert_{\frac{5}{6} \, \pi}^{\frac{7}{4} \, \pi} = {-3 \, \sqrt{2} - 6}
\]

\input{Integral-Concept-0003.HELP.tex}

\begin{multipleChoice}
\choice{The antiderivative is incorrect.}
\choice[correct]{The integrand is not defined over the entire interval.}
\choice{The bounds are evaluated in the wrong order.}
\choice{Nothing is wrong.  The equation is correct, as is.}
\end{multipleChoice}

\end{problem}}%}

\latexProblemContent{
\ifVerboseLocation This is Integration Concept Question 0003. \\ \fi
\begin{problem}

What is wrong with the following equation:

\[
\int_{\frac{1}{4} \, \pi}^{\frac{11}{6} \, \pi} {-3 \, \cot\left(x\right) \csc\left(x\right)}\;dx = {\frac{3}{\sin\left(x\right)}}\Bigg\vert_{\frac{1}{4} \, \pi}^{\frac{11}{6} \, \pi} = {-3 \, \sqrt{2} - 6}
\]

\input{Integral-Concept-0003.HELP.tex}

\begin{multipleChoice}
\choice{The antiderivative is incorrect.}
\choice[correct]{The integrand is not defined over the entire interval.}
\choice{The bounds are evaluated in the wrong order.}
\choice{Nothing is wrong.  The equation is correct, as is.}
\end{multipleChoice}

\end{problem}}%}

\latexProblemContent{
\ifVerboseLocation This is Integration Concept Question 0003. \\ \fi
\begin{problem}

What is wrong with the following equation:

\[
\int_{\frac{1}{3} \, \pi}^{\frac{7}{4} \, \pi} {-8 \, \cot\left(x\right) \csc\left(x\right)}\;dx = {\frac{8}{\sin\left(x\right)}}\Bigg\vert_{\frac{1}{3} \, \pi}^{\frac{7}{4} \, \pi} = {-\frac{16}{3} \, \sqrt{3} - 8 \, \sqrt{2}}
\]

\input{Integral-Concept-0003.HELP.tex}

\begin{multipleChoice}
\choice{The antiderivative is incorrect.}
\choice[correct]{The integrand is not defined over the entire interval.}
\choice{The bounds are evaluated in the wrong order.}
\choice{Nothing is wrong.  The equation is correct, as is.}
\end{multipleChoice}

\end{problem}}%}

\latexProblemContent{
\ifVerboseLocation This is Integration Concept Question 0003. \\ \fi
\begin{problem}

What is wrong with the following equation:

\[
\int_{\frac{3}{4} \, \pi}^{\frac{7}{4} \, \pi} {10 \, \cot\left(x\right) \csc\left(x\right)}\;dx = {-\frac{10}{\sin\left(x\right)}}\Bigg\vert_{\frac{3}{4} \, \pi}^{\frac{7}{4} \, \pi} = {20 \, \sqrt{2}}
\]

\input{Integral-Concept-0003.HELP.tex}

\begin{multipleChoice}
\choice{The antiderivative is incorrect.}
\choice[correct]{The integrand is not defined over the entire interval.}
\choice{The bounds are evaluated in the wrong order.}
\choice{Nothing is wrong.  The equation is correct, as is.}
\end{multipleChoice}

\end{problem}}%}

\latexProblemContent{
\ifVerboseLocation This is Integration Concept Question 0003. \\ \fi
\begin{problem}

What is wrong with the following equation:

\[
\int_{\frac{3}{4} \, \pi}^{\frac{5}{4} \, \pi} {-9 \, \cot\left(x\right) \csc\left(x\right)}\;dx = {\frac{9}{\sin\left(x\right)}}\Bigg\vert_{\frac{3}{4} \, \pi}^{\frac{5}{4} \, \pi} = {-18 \, \sqrt{2}}
\]

\input{Integral-Concept-0003.HELP.tex}

\begin{multipleChoice}
\choice{The antiderivative is incorrect.}
\choice[correct]{The integrand is not defined over the entire interval.}
\choice{The bounds are evaluated in the wrong order.}
\choice{Nothing is wrong.  The equation is correct, as is.}
\end{multipleChoice}

\end{problem}}%}

\latexProblemContent{
\ifVerboseLocation This is Integration Concept Question 0003. \\ \fi
\begin{problem}

What is wrong with the following equation:

\[
\int_{\frac{1}{6} \, \pi}^{\frac{11}{6} \, \pi} {-3 \, \cot\left(x\right) \csc\left(x\right)}\;dx = {\frac{3}{\sin\left(x\right)}}\Bigg\vert_{\frac{1}{6} \, \pi}^{\frac{11}{6} \, \pi} = {-12}
\]

\input{Integral-Concept-0003.HELP.tex}

\begin{multipleChoice}
\choice{The antiderivative is incorrect.}
\choice[correct]{The integrand is not defined over the entire interval.}
\choice{The bounds are evaluated in the wrong order.}
\choice{Nothing is wrong.  The equation is correct, as is.}
\end{multipleChoice}

\end{problem}}%}

\latexProblemContent{
\ifVerboseLocation This is Integration Concept Question 0003. \\ \fi
\begin{problem}

What is wrong with the following equation:

\[
\int_{\frac{5}{6} \, \pi}^{\frac{3}{2} \, \pi} {9 \, \csc\left(x\right)^{2}}\;dx = {-\frac{9}{\tan\left(x\right)}}\Bigg\vert_{\frac{5}{6} \, \pi}^{\frac{3}{2} \, \pi} = {-9 \, \sqrt{3}}
\]

\input{Integral-Concept-0003.HELP.tex}

\begin{multipleChoice}
\choice{The antiderivative is incorrect.}
\choice[correct]{The integrand is not defined over the entire interval.}
\choice{The bounds are evaluated in the wrong order.}
\choice{Nothing is wrong.  The equation is correct, as is.}
\end{multipleChoice}

\end{problem}}%}

\latexProblemContent{
\ifVerboseLocation This is Integration Concept Question 0003. \\ \fi
\begin{problem}

What is wrong with the following equation:

\[
\int_{\frac{1}{3} \, \pi}^{\frac{7}{4} \, \pi} {3 \, \cot\left(x\right) \csc\left(x\right)}\;dx = {-\frac{3}{\sin\left(x\right)}}\Bigg\vert_{\frac{1}{3} \, \pi}^{\frac{7}{4} \, \pi} = {2 \, \sqrt{3} + 3 \, \sqrt{2}}
\]

\input{Integral-Concept-0003.HELP.tex}

\begin{multipleChoice}
\choice{The antiderivative is incorrect.}
\choice[correct]{The integrand is not defined over the entire interval.}
\choice{The bounds are evaluated in the wrong order.}
\choice{Nothing is wrong.  The equation is correct, as is.}
\end{multipleChoice}

\end{problem}}%}

\latexProblemContent{
\ifVerboseLocation This is Integration Concept Question 0003. \\ \fi
\begin{problem}

What is wrong with the following equation:

\[
\int_{\frac{1}{4} \, \pi}^{\frac{11}{6} \, \pi} {-2 \, \cot\left(x\right) \csc\left(x\right)}\;dx = {\frac{2}{\sin\left(x\right)}}\Bigg\vert_{\frac{1}{4} \, \pi}^{\frac{11}{6} \, \pi} = {-2 \, \sqrt{2} - 4}
\]

\input{Integral-Concept-0003.HELP.tex}

\begin{multipleChoice}
\choice{The antiderivative is incorrect.}
\choice[correct]{The integrand is not defined over the entire interval.}
\choice{The bounds are evaluated in the wrong order.}
\choice{Nothing is wrong.  The equation is correct, as is.}
\end{multipleChoice}

\end{problem}}%}

\latexProblemContent{
\ifVerboseLocation This is Integration Concept Question 0003. \\ \fi
\begin{problem}

What is wrong with the following equation:

\[
\int_{\frac{3}{4} \, \pi}^{\frac{3}{2} \, \pi} {8 \, \cot\left(x\right) \csc\left(x\right)}\;dx = {-\frac{8}{\sin\left(x\right)}}\Bigg\vert_{\frac{3}{4} \, \pi}^{\frac{3}{2} \, \pi} = {8 \, \sqrt{2} + 8}
\]

\input{Integral-Concept-0003.HELP.tex}

\begin{multipleChoice}
\choice{The antiderivative is incorrect.}
\choice[correct]{The integrand is not defined over the entire interval.}
\choice{The bounds are evaluated in the wrong order.}
\choice{Nothing is wrong.  The equation is correct, as is.}
\end{multipleChoice}

\end{problem}}%}

\latexProblemContent{
\ifVerboseLocation This is Integration Concept Question 0003. \\ \fi
\begin{problem}

What is wrong with the following equation:

\[
\int_{\frac{1}{2} \, \pi}^{\frac{7}{6} \, \pi} {10 \, \csc\left(x\right)^{2}}\;dx = {-\frac{10}{\tan\left(x\right)}}\Bigg\vert_{\frac{1}{2} \, \pi}^{\frac{7}{6} \, \pi} = {-10 \, \sqrt{3}}
\]

\input{Integral-Concept-0003.HELP.tex}

\begin{multipleChoice}
\choice{The antiderivative is incorrect.}
\choice[correct]{The integrand is not defined over the entire interval.}
\choice{The bounds are evaluated in the wrong order.}
\choice{Nothing is wrong.  The equation is correct, as is.}
\end{multipleChoice}

\end{problem}}%}

\latexProblemContent{
\ifVerboseLocation This is Integration Concept Question 0003. \\ \fi
\begin{problem}

What is wrong with the following equation:

\[
\int_{\frac{1}{6} \, \pi}^{\frac{5}{3} \, \pi} {10 \, \csc\left(x\right)^{2}}\;dx = {-\frac{10}{\tan\left(x\right)}}\Bigg\vert_{\frac{1}{6} \, \pi}^{\frac{5}{3} \, \pi} = {\frac{40}{3} \, \sqrt{3}}
\]

\input{Integral-Concept-0003.HELP.tex}

\begin{multipleChoice}
\choice{The antiderivative is incorrect.}
\choice[correct]{The integrand is not defined over the entire interval.}
\choice{The bounds are evaluated in the wrong order.}
\choice{Nothing is wrong.  The equation is correct, as is.}
\end{multipleChoice}

\end{problem}}%}

\latexProblemContent{
\ifVerboseLocation This is Integration Concept Question 0003. \\ \fi
\begin{problem}

What is wrong with the following equation:

\[
\int_{\frac{2}{3} \, \pi}^{\frac{4}{3} \, \pi} {\csc\left(x\right)^{2}}\;dx = {-\frac{1}{\tan\left(x\right)}}\Bigg\vert_{\frac{2}{3} \, \pi}^{\frac{4}{3} \, \pi} = {-\frac{2}{3} \, \sqrt{3}}
\]

\input{Integral-Concept-0003.HELP.tex}

\begin{multipleChoice}
\choice{The antiderivative is incorrect.}
\choice[correct]{The integrand is not defined over the entire interval.}
\choice{The bounds are evaluated in the wrong order.}
\choice{Nothing is wrong.  The equation is correct, as is.}
\end{multipleChoice}

\end{problem}}%}

\latexProblemContent{
\ifVerboseLocation This is Integration Concept Question 0003. \\ \fi
\begin{problem}

What is wrong with the following equation:

\[
\int_{\frac{1}{2} \, \pi}^{\frac{7}{6} \, \pi} {-6 \, \cot\left(x\right) \csc\left(x\right)}\;dx = {\frac{6}{\sin\left(x\right)}}\Bigg\vert_{\frac{1}{2} \, \pi}^{\frac{7}{6} \, \pi} = {-18}
\]

\input{Integral-Concept-0003.HELP.tex}

\begin{multipleChoice}
\choice{The antiderivative is incorrect.}
\choice[correct]{The integrand is not defined over the entire interval.}
\choice{The bounds are evaluated in the wrong order.}
\choice{Nothing is wrong.  The equation is correct, as is.}
\end{multipleChoice}

\end{problem}}%}

\latexProblemContent{
\ifVerboseLocation This is Integration Concept Question 0003. \\ \fi
\begin{problem}

What is wrong with the following equation:

\[
\int_{\frac{2}{3} \, \pi}^{\frac{4}{3} \, \pi} {3 \, \csc\left(x\right)^{2}}\;dx = {-\frac{3}{\tan\left(x\right)}}\Bigg\vert_{\frac{2}{3} \, \pi}^{\frac{4}{3} \, \pi} = {-2 \, \sqrt{3}}
\]

\input{Integral-Concept-0003.HELP.tex}

\begin{multipleChoice}
\choice{The antiderivative is incorrect.}
\choice[correct]{The integrand is not defined over the entire interval.}
\choice{The bounds are evaluated in the wrong order.}
\choice{Nothing is wrong.  The equation is correct, as is.}
\end{multipleChoice}

\end{problem}}%}

\latexProblemContent{
\ifVerboseLocation This is Integration Concept Question 0003. \\ \fi
\begin{problem}

What is wrong with the following equation:

\[
\int_{\frac{5}{6} \, \pi}^{\frac{4}{3} \, \pi} {6 \, \cot\left(x\right) \csc\left(x\right)}\;dx = {-\frac{6}{\sin\left(x\right)}}\Bigg\vert_{\frac{5}{6} \, \pi}^{\frac{4}{3} \, \pi} = {4 \, \sqrt{3} + 12}
\]

\input{Integral-Concept-0003.HELP.tex}

\begin{multipleChoice}
\choice{The antiderivative is incorrect.}
\choice[correct]{The integrand is not defined over the entire interval.}
\choice{The bounds are evaluated in the wrong order.}
\choice{Nothing is wrong.  The equation is correct, as is.}
\end{multipleChoice}

\end{problem}}%}

\latexProblemContent{
\ifVerboseLocation This is Integration Concept Question 0003. \\ \fi
\begin{problem}

What is wrong with the following equation:

\[
\int_{\frac{3}{4} \, \pi}^{\frac{3}{2} \, \pi} {-5 \, \csc\left(x\right)^{2}}\;dx = {\frac{5}{\tan\left(x\right)}}\Bigg\vert_{\frac{3}{4} \, \pi}^{\frac{3}{2} \, \pi} = {5}
\]

\input{Integral-Concept-0003.HELP.tex}

\begin{multipleChoice}
\choice{The antiderivative is incorrect.}
\choice[correct]{The integrand is not defined over the entire interval.}
\choice{The bounds are evaluated in the wrong order.}
\choice{Nothing is wrong.  The equation is correct, as is.}
\end{multipleChoice}

\end{problem}}%}

\latexProblemContent{
\ifVerboseLocation This is Integration Concept Question 0003. \\ \fi
\begin{problem}

What is wrong with the following equation:

\[
\int_{\frac{5}{6} \, \pi}^{\frac{5}{3} \, \pi} {-8 \, \csc\left(x\right)^{2}}\;dx = {\frac{8}{\tan\left(x\right)}}\Bigg\vert_{\frac{5}{6} \, \pi}^{\frac{5}{3} \, \pi} = {\frac{16}{3} \, \sqrt{3}}
\]

\input{Integral-Concept-0003.HELP.tex}

\begin{multipleChoice}
\choice{The antiderivative is incorrect.}
\choice[correct]{The integrand is not defined over the entire interval.}
\choice{The bounds are evaluated in the wrong order.}
\choice{Nothing is wrong.  The equation is correct, as is.}
\end{multipleChoice}

\end{problem}}%}

\latexProblemContent{
\ifVerboseLocation This is Integration Concept Question 0003. \\ \fi
\begin{problem}

What is wrong with the following equation:

\[
\int_{\frac{1}{3} \, \pi}^{\frac{5}{4} \, \pi} {3 \, \cot\left(x\right) \csc\left(x\right)}\;dx = {-\frac{3}{\sin\left(x\right)}}\Bigg\vert_{\frac{1}{3} \, \pi}^{\frac{5}{4} \, \pi} = {2 \, \sqrt{3} + 3 \, \sqrt{2}}
\]

\input{Integral-Concept-0003.HELP.tex}

\begin{multipleChoice}
\choice{The antiderivative is incorrect.}
\choice[correct]{The integrand is not defined over the entire interval.}
\choice{The bounds are evaluated in the wrong order.}
\choice{Nothing is wrong.  The equation is correct, as is.}
\end{multipleChoice}

\end{problem}}%}

\latexProblemContent{
\ifVerboseLocation This is Integration Concept Question 0003. \\ \fi
\begin{problem}

What is wrong with the following equation:

\[
\int_{\frac{1}{3} \, \pi}^{\frac{7}{6} \, \pi} {2 \, \cot\left(x\right) \csc\left(x\right)}\;dx = {-\frac{2}{\sin\left(x\right)}}\Bigg\vert_{\frac{1}{3} \, \pi}^{\frac{7}{6} \, \pi} = {\frac{4}{3} \, \sqrt{3} + 4}
\]

\input{Integral-Concept-0003.HELP.tex}

\begin{multipleChoice}
\choice{The antiderivative is incorrect.}
\choice[correct]{The integrand is not defined over the entire interval.}
\choice{The bounds are evaluated in the wrong order.}
\choice{Nothing is wrong.  The equation is correct, as is.}
\end{multipleChoice}

\end{problem}}%}

\latexProblemContent{
\ifVerboseLocation This is Integration Concept Question 0003. \\ \fi
\begin{problem}

What is wrong with the following equation:

\[
\int_{\frac{2}{3} \, \pi}^{\frac{5}{4} \, \pi} {-7 \, \csc\left(x\right)^{2}}\;dx = {\frac{7}{\tan\left(x\right)}}\Bigg\vert_{\frac{2}{3} \, \pi}^{\frac{5}{4} \, \pi} = {\frac{7}{3} \, \sqrt{3} + 7}
\]

\input{Integral-Concept-0003.HELP.tex}

\begin{multipleChoice}
\choice{The antiderivative is incorrect.}
\choice[correct]{The integrand is not defined over the entire interval.}
\choice{The bounds are evaluated in the wrong order.}
\choice{Nothing is wrong.  The equation is correct, as is.}
\end{multipleChoice}

\end{problem}}%}

\latexProblemContent{
\ifVerboseLocation This is Integration Concept Question 0003. \\ \fi
\begin{problem}

What is wrong with the following equation:

\[
\int_{\frac{3}{4} \, \pi}^{\frac{5}{4} \, \pi} {2 \, \cot\left(x\right) \csc\left(x\right)}\;dx = {-\frac{2}{\sin\left(x\right)}}\Bigg\vert_{\frac{3}{4} \, \pi}^{\frac{5}{4} \, \pi} = {4 \, \sqrt{2}}
\]

\input{Integral-Concept-0003.HELP.tex}

\begin{multipleChoice}
\choice{The antiderivative is incorrect.}
\choice[correct]{The integrand is not defined over the entire interval.}
\choice{The bounds are evaluated in the wrong order.}
\choice{Nothing is wrong.  The equation is correct, as is.}
\end{multipleChoice}

\end{problem}}%}

\latexProblemContent{
\ifVerboseLocation This is Integration Concept Question 0003. \\ \fi
\begin{problem}

What is wrong with the following equation:

\[
\int_{\frac{1}{3} \, \pi}^{\frac{7}{4} \, \pi} {-10 \, \csc\left(x\right)^{2}}\;dx = {\frac{10}{\tan\left(x\right)}}\Bigg\vert_{\frac{1}{3} \, \pi}^{\frac{7}{4} \, \pi} = {-\frac{10}{3} \, \sqrt{3} - 10}
\]

\input{Integral-Concept-0003.HELP.tex}

\begin{multipleChoice}
\choice{The antiderivative is incorrect.}
\choice[correct]{The integrand is not defined over the entire interval.}
\choice{The bounds are evaluated in the wrong order.}
\choice{Nothing is wrong.  The equation is correct, as is.}
\end{multipleChoice}

\end{problem}}%}

\latexProblemContent{
\ifVerboseLocation This is Integration Concept Question 0003. \\ \fi
\begin{problem}

What is wrong with the following equation:

\[
\int_{\frac{2}{3} \, \pi}^{\frac{5}{4} \, \pi} {3 \, \csc\left(x\right)^{2}}\;dx = {-\frac{3}{\tan\left(x\right)}}\Bigg\vert_{\frac{2}{3} \, \pi}^{\frac{5}{4} \, \pi} = {-\sqrt{3} - 3}
\]

\input{Integral-Concept-0003.HELP.tex}

\begin{multipleChoice}
\choice{The antiderivative is incorrect.}
\choice[correct]{The integrand is not defined over the entire interval.}
\choice{The bounds are evaluated in the wrong order.}
\choice{Nothing is wrong.  The equation is correct, as is.}
\end{multipleChoice}

\end{problem}}%}

\latexProblemContent{
\ifVerboseLocation This is Integration Concept Question 0003. \\ \fi
\begin{problem}

What is wrong with the following equation:

\[
\int_{\frac{1}{4} \, \pi}^{\frac{4}{3} \, \pi} {-2 \, \csc\left(x\right)^{2}}\;dx = {\frac{2}{\tan\left(x\right)}}\Bigg\vert_{\frac{1}{4} \, \pi}^{\frac{4}{3} \, \pi} = {\frac{2}{3} \, \sqrt{3} - 2}
\]

\input{Integral-Concept-0003.HELP.tex}

\begin{multipleChoice}
\choice{The antiderivative is incorrect.}
\choice[correct]{The integrand is not defined over the entire interval.}
\choice{The bounds are evaluated in the wrong order.}
\choice{Nothing is wrong.  The equation is correct, as is.}
\end{multipleChoice}

\end{problem}}%}

\latexProblemContent{
\ifVerboseLocation This is Integration Concept Question 0003. \\ \fi
\begin{problem}

What is wrong with the following equation:

\[
\int_{\frac{1}{3} \, \pi}^{\frac{7}{6} \, \pi} {-7 \, \csc\left(x\right)^{2}}\;dx = {\frac{7}{\tan\left(x\right)}}\Bigg\vert_{\frac{1}{3} \, \pi}^{\frac{7}{6} \, \pi} = {\frac{14}{3} \, \sqrt{3}}
\]

\input{Integral-Concept-0003.HELP.tex}

\begin{multipleChoice}
\choice{The antiderivative is incorrect.}
\choice[correct]{The integrand is not defined over the entire interval.}
\choice{The bounds are evaluated in the wrong order.}
\choice{Nothing is wrong.  The equation is correct, as is.}
\end{multipleChoice}

\end{problem}}%}

\latexProblemContent{
\ifVerboseLocation This is Integration Concept Question 0003. \\ \fi
\begin{problem}

What is wrong with the following equation:

\[
\int_{\frac{1}{2} \, \pi}^{\frac{4}{3} \, \pi} {4 \, \csc\left(x\right)^{2}}\;dx = {-\frac{4}{\tan\left(x\right)}}\Bigg\vert_{\frac{1}{2} \, \pi}^{\frac{4}{3} \, \pi} = {-\frac{4}{3} \, \sqrt{3}}
\]

\input{Integral-Concept-0003.HELP.tex}

\begin{multipleChoice}
\choice{The antiderivative is incorrect.}
\choice[correct]{The integrand is not defined over the entire interval.}
\choice{The bounds are evaluated in the wrong order.}
\choice{Nothing is wrong.  The equation is correct, as is.}
\end{multipleChoice}

\end{problem}}%}

\latexProblemContent{
\ifVerboseLocation This is Integration Concept Question 0003. \\ \fi
\begin{problem}

What is wrong with the following equation:

\[
\int_{\frac{1}{4} \, \pi}^{\frac{4}{3} \, \pi} {10 \, \cot\left(x\right) \csc\left(x\right)}\;dx = {-\frac{10}{\sin\left(x\right)}}\Bigg\vert_{\frac{1}{4} \, \pi}^{\frac{4}{3} \, \pi} = {\frac{20}{3} \, \sqrt{3} + 10 \, \sqrt{2}}
\]

\input{Integral-Concept-0003.HELP.tex}

\begin{multipleChoice}
\choice{The antiderivative is incorrect.}
\choice[correct]{The integrand is not defined over the entire interval.}
\choice{The bounds are evaluated in the wrong order.}
\choice{Nothing is wrong.  The equation is correct, as is.}
\end{multipleChoice}

\end{problem}}%}

\latexProblemContent{
\ifVerboseLocation This is Integration Concept Question 0003. \\ \fi
\begin{problem}

What is wrong with the following equation:

\[
\int_{\frac{5}{6} \, \pi}^{\frac{3}{2} \, \pi} {3 \, \cot\left(x\right) \csc\left(x\right)}\;dx = {-\frac{3}{\sin\left(x\right)}}\Bigg\vert_{\frac{5}{6} \, \pi}^{\frac{3}{2} \, \pi} = {9}
\]

\input{Integral-Concept-0003.HELP.tex}

\begin{multipleChoice}
\choice{The antiderivative is incorrect.}
\choice[correct]{The integrand is not defined over the entire interval.}
\choice{The bounds are evaluated in the wrong order.}
\choice{Nothing is wrong.  The equation is correct, as is.}
\end{multipleChoice}

\end{problem}}%}

\latexProblemContent{
\ifVerboseLocation This is Integration Concept Question 0003. \\ \fi
\begin{problem}

What is wrong with the following equation:

\[
\int_{\frac{5}{6} \, \pi}^{\frac{7}{6} \, \pi} {2 \, \csc\left(x\right)^{2}}\;dx = {-\frac{2}{\tan\left(x\right)}}\Bigg\vert_{\frac{5}{6} \, \pi}^{\frac{7}{6} \, \pi} = {-4 \, \sqrt{3}}
\]

\input{Integral-Concept-0003.HELP.tex}

\begin{multipleChoice}
\choice{The antiderivative is incorrect.}
\choice[correct]{The integrand is not defined over the entire interval.}
\choice{The bounds are evaluated in the wrong order.}
\choice{Nothing is wrong.  The equation is correct, as is.}
\end{multipleChoice}

\end{problem}}%}

\latexProblemContent{
\ifVerboseLocation This is Integration Concept Question 0003. \\ \fi
\begin{problem}

What is wrong with the following equation:

\[
\int_{\frac{5}{6} \, \pi}^{\frac{3}{2} \, \pi} {-3 \, \csc\left(x\right)^{2}}\;dx = {\frac{3}{\tan\left(x\right)}}\Bigg\vert_{\frac{5}{6} \, \pi}^{\frac{3}{2} \, \pi} = {3 \, \sqrt{3}}
\]

\input{Integral-Concept-0003.HELP.tex}

\begin{multipleChoice}
\choice{The antiderivative is incorrect.}
\choice[correct]{The integrand is not defined over the entire interval.}
\choice{The bounds are evaluated in the wrong order.}
\choice{Nothing is wrong.  The equation is correct, as is.}
\end{multipleChoice}

\end{problem}}%}

\latexProblemContent{
\ifVerboseLocation This is Integration Concept Question 0003. \\ \fi
\begin{problem}

What is wrong with the following equation:

\[
\int_{\frac{1}{3} \, \pi}^{\frac{4}{3} \, \pi} {-\csc\left(x\right)^{2}}\;dx = {\frac{1}{\tan\left(x\right)}}\Bigg\vert_{\frac{1}{3} \, \pi}^{\frac{4}{3} \, \pi} = {0}
\]

\input{Integral-Concept-0003.HELP.tex}

\begin{multipleChoice}
\choice{The antiderivative is incorrect.}
\choice[correct]{The integrand is not defined over the entire interval.}
\choice{The bounds are evaluated in the wrong order.}
\choice{Nothing is wrong.  The equation is correct, as is.}
\end{multipleChoice}

\end{problem}}%}

\latexProblemContent{
\ifVerboseLocation This is Integration Concept Question 0003. \\ \fi
\begin{problem}

What is wrong with the following equation:

\[
\int_{\frac{3}{4} \, \pi}^{\frac{5}{3} \, \pi} {2 \, \cot\left(x\right) \csc\left(x\right)}\;dx = {-\frac{2}{\sin\left(x\right)}}\Bigg\vert_{\frac{3}{4} \, \pi}^{\frac{5}{3} \, \pi} = {\frac{4}{3} \, \sqrt{3} + 2 \, \sqrt{2}}
\]

\input{Integral-Concept-0003.HELP.tex}

\begin{multipleChoice}
\choice{The antiderivative is incorrect.}
\choice[correct]{The integrand is not defined over the entire interval.}
\choice{The bounds are evaluated in the wrong order.}
\choice{Nothing is wrong.  The equation is correct, as is.}
\end{multipleChoice}

\end{problem}}%}

\latexProblemContent{
\ifVerboseLocation This is Integration Concept Question 0003. \\ \fi
\begin{problem}

What is wrong with the following equation:

\[
\int_{\frac{3}{4} \, \pi}^{\frac{7}{4} \, \pi} {2 \, \csc\left(x\right)^{2}}\;dx = {-\frac{2}{\tan\left(x\right)}}\Bigg\vert_{\frac{3}{4} \, \pi}^{\frac{7}{4} \, \pi} = {0}
\]

\input{Integral-Concept-0003.HELP.tex}

\begin{multipleChoice}
\choice{The antiderivative is incorrect.}
\choice[correct]{The integrand is not defined over the entire interval.}
\choice{The bounds are evaluated in the wrong order.}
\choice{Nothing is wrong.  The equation is correct, as is.}
\end{multipleChoice}

\end{problem}}%}

\latexProblemContent{
\ifVerboseLocation This is Integration Concept Question 0003. \\ \fi
\begin{problem}

What is wrong with the following equation:

\[
\int_{\frac{5}{6} \, \pi}^{\frac{7}{6} \, \pi} {5 \, \cot\left(x\right) \csc\left(x\right)}\;dx = {-\frac{5}{\sin\left(x\right)}}\Bigg\vert_{\frac{5}{6} \, \pi}^{\frac{7}{6} \, \pi} = {20}
\]

\input{Integral-Concept-0003.HELP.tex}

\begin{multipleChoice}
\choice{The antiderivative is incorrect.}
\choice[correct]{The integrand is not defined over the entire interval.}
\choice{The bounds are evaluated in the wrong order.}
\choice{Nothing is wrong.  The equation is correct, as is.}
\end{multipleChoice}

\end{problem}}%}

\latexProblemContent{
\ifVerboseLocation This is Integration Concept Question 0003. \\ \fi
\begin{problem}

What is wrong with the following equation:

\[
\int_{\frac{1}{2} \, \pi}^{\frac{7}{4} \, \pi} {3 \, \cot\left(x\right) \csc\left(x\right)}\;dx = {-\frac{3}{\sin\left(x\right)}}\Bigg\vert_{\frac{1}{2} \, \pi}^{\frac{7}{4} \, \pi} = {3 \, \sqrt{2} + 3}
\]

\input{Integral-Concept-0003.HELP.tex}

\begin{multipleChoice}
\choice{The antiderivative is incorrect.}
\choice[correct]{The integrand is not defined over the entire interval.}
\choice{The bounds are evaluated in the wrong order.}
\choice{Nothing is wrong.  The equation is correct, as is.}
\end{multipleChoice}

\end{problem}}%}

\latexProblemContent{
\ifVerboseLocation This is Integration Concept Question 0003. \\ \fi
\begin{problem}

What is wrong with the following equation:

\[
\int_{\frac{1}{3} \, \pi}^{\frac{7}{6} \, \pi} {6 \, \cot\left(x\right) \csc\left(x\right)}\;dx = {-\frac{6}{\sin\left(x\right)}}\Bigg\vert_{\frac{1}{3} \, \pi}^{\frac{7}{6} \, \pi} = {4 \, \sqrt{3} + 12}
\]

\input{Integral-Concept-0003.HELP.tex}

\begin{multipleChoice}
\choice{The antiderivative is incorrect.}
\choice[correct]{The integrand is not defined over the entire interval.}
\choice{The bounds are evaluated in the wrong order.}
\choice{Nothing is wrong.  The equation is correct, as is.}
\end{multipleChoice}

\end{problem}}%}

\latexProblemContent{
\ifVerboseLocation This is Integration Concept Question 0003. \\ \fi
\begin{problem}

What is wrong with the following equation:

\[
\int_{\frac{1}{4} \, \pi}^{\frac{11}{6} \, \pi} {-10 \, \csc\left(x\right)^{2}}\;dx = {\frac{10}{\tan\left(x\right)}}\Bigg\vert_{\frac{1}{4} \, \pi}^{\frac{11}{6} \, \pi} = {-10 \, \sqrt{3} - 10}
\]

\input{Integral-Concept-0003.HELP.tex}

\begin{multipleChoice}
\choice{The antiderivative is incorrect.}
\choice[correct]{The integrand is not defined over the entire interval.}
\choice{The bounds are evaluated in the wrong order.}
\choice{Nothing is wrong.  The equation is correct, as is.}
\end{multipleChoice}

\end{problem}}%}

\latexProblemContent{
\ifVerboseLocation This is Integration Concept Question 0003. \\ \fi
\begin{problem}

What is wrong with the following equation:

\[
\int_{\frac{1}{4} \, \pi}^{\frac{3}{2} \, \pi} {-7 \, \cot\left(x\right) \csc\left(x\right)}\;dx = {\frac{7}{\sin\left(x\right)}}\Bigg\vert_{\frac{1}{4} \, \pi}^{\frac{3}{2} \, \pi} = {-7 \, \sqrt{2} - 7}
\]

\input{Integral-Concept-0003.HELP.tex}

\begin{multipleChoice}
\choice{The antiderivative is incorrect.}
\choice[correct]{The integrand is not defined over the entire interval.}
\choice{The bounds are evaluated in the wrong order.}
\choice{Nothing is wrong.  The equation is correct, as is.}
\end{multipleChoice}

\end{problem}}%}

\latexProblemContent{
\ifVerboseLocation This is Integration Concept Question 0003. \\ \fi
\begin{problem}

What is wrong with the following equation:

\[
\int_{\frac{5}{6} \, \pi}^{\frac{11}{6} \, \pi} {9 \, \csc\left(x\right)^{2}}\;dx = {-\frac{9}{\tan\left(x\right)}}\Bigg\vert_{\frac{5}{6} \, \pi}^{\frac{11}{6} \, \pi} = {0}
\]

\input{Integral-Concept-0003.HELP.tex}

\begin{multipleChoice}
\choice{The antiderivative is incorrect.}
\choice[correct]{The integrand is not defined over the entire interval.}
\choice{The bounds are evaluated in the wrong order.}
\choice{Nothing is wrong.  The equation is correct, as is.}
\end{multipleChoice}

\end{problem}}%}

\latexProblemContent{
\ifVerboseLocation This is Integration Concept Question 0003. \\ \fi
\begin{problem}

What is wrong with the following equation:

\[
\int_{\frac{1}{3} \, \pi}^{\frac{7}{6} \, \pi} {\csc\left(x\right)^{2}}\;dx = {-\frac{1}{\tan\left(x\right)}}\Bigg\vert_{\frac{1}{3} \, \pi}^{\frac{7}{6} \, \pi} = {-\frac{2}{3} \, \sqrt{3}}
\]

\input{Integral-Concept-0003.HELP.tex}

\begin{multipleChoice}
\choice{The antiderivative is incorrect.}
\choice[correct]{The integrand is not defined over the entire interval.}
\choice{The bounds are evaluated in the wrong order.}
\choice{Nothing is wrong.  The equation is correct, as is.}
\end{multipleChoice}

\end{problem}}%}

\latexProblemContent{
\ifVerboseLocation This is Integration Concept Question 0003. \\ \fi
\begin{problem}

What is wrong with the following equation:

\[
\int_{\frac{2}{3} \, \pi}^{\frac{5}{3} \, \pi} {4 \, \cot\left(x\right) \csc\left(x\right)}\;dx = {-\frac{4}{\sin\left(x\right)}}\Bigg\vert_{\frac{2}{3} \, \pi}^{\frac{5}{3} \, \pi} = {\frac{16}{3} \, \sqrt{3}}
\]

\input{Integral-Concept-0003.HELP.tex}

\begin{multipleChoice}
\choice{The antiderivative is incorrect.}
\choice[correct]{The integrand is not defined over the entire interval.}
\choice{The bounds are evaluated in the wrong order.}
\choice{Nothing is wrong.  The equation is correct, as is.}
\end{multipleChoice}

\end{problem}}%}

\latexProblemContent{
\ifVerboseLocation This is Integration Concept Question 0003. \\ \fi
\begin{problem}

What is wrong with the following equation:

\[
\int_{\frac{1}{4} \, \pi}^{\frac{7}{4} \, \pi} {-5 \, \csc\left(x\right)^{2}}\;dx = {\frac{5}{\tan\left(x\right)}}\Bigg\vert_{\frac{1}{4} \, \pi}^{\frac{7}{4} \, \pi} = {-10}
\]

\input{Integral-Concept-0003.HELP.tex}

\begin{multipleChoice}
\choice{The antiderivative is incorrect.}
\choice[correct]{The integrand is not defined over the entire interval.}
\choice{The bounds are evaluated in the wrong order.}
\choice{Nothing is wrong.  The equation is correct, as is.}
\end{multipleChoice}

\end{problem}}%}

\latexProblemContent{
\ifVerboseLocation This is Integration Concept Question 0003. \\ \fi
\begin{problem}

What is wrong with the following equation:

\[
\int_{\frac{5}{6} \, \pi}^{\frac{5}{3} \, \pi} {-\cot\left(x\right) \csc\left(x\right)}\;dx = {\frac{1}{\sin\left(x\right)}}\Bigg\vert_{\frac{5}{6} \, \pi}^{\frac{5}{3} \, \pi} = {-\frac{2}{3} \, \sqrt{3} - 2}
\]

\input{Integral-Concept-0003.HELP.tex}

\begin{multipleChoice}
\choice{The antiderivative is incorrect.}
\choice[correct]{The integrand is not defined over the entire interval.}
\choice{The bounds are evaluated in the wrong order.}
\choice{Nothing is wrong.  The equation is correct, as is.}
\end{multipleChoice}

\end{problem}}%}

\latexProblemContent{
\ifVerboseLocation This is Integration Concept Question 0003. \\ \fi
\begin{problem}

What is wrong with the following equation:

\[
\int_{\frac{1}{4} \, \pi}^{\frac{5}{3} \, \pi} {-8 \, \csc\left(x\right)^{2}}\;dx = {\frac{8}{\tan\left(x\right)}}\Bigg\vert_{\frac{1}{4} \, \pi}^{\frac{5}{3} \, \pi} = {-\frac{8}{3} \, \sqrt{3} - 8}
\]

\input{Integral-Concept-0003.HELP.tex}

\begin{multipleChoice}
\choice{The antiderivative is incorrect.}
\choice[correct]{The integrand is not defined over the entire interval.}
\choice{The bounds are evaluated in the wrong order.}
\choice{Nothing is wrong.  The equation is correct, as is.}
\end{multipleChoice}

\end{problem}}%}

\latexProblemContent{
\ifVerboseLocation This is Integration Concept Question 0003. \\ \fi
\begin{problem}

What is wrong with the following equation:

\[
\int_{\frac{3}{4} \, \pi}^{\frac{7}{6} \, \pi} {-3 \, \cot\left(x\right) \csc\left(x\right)}\;dx = {\frac{3}{\sin\left(x\right)}}\Bigg\vert_{\frac{3}{4} \, \pi}^{\frac{7}{6} \, \pi} = {-3 \, \sqrt{2} - 6}
\]

\input{Integral-Concept-0003.HELP.tex}

\begin{multipleChoice}
\choice{The antiderivative is incorrect.}
\choice[correct]{The integrand is not defined over the entire interval.}
\choice{The bounds are evaluated in the wrong order.}
\choice{Nothing is wrong.  The equation is correct, as is.}
\end{multipleChoice}

\end{problem}}%}

\latexProblemContent{
\ifVerboseLocation This is Integration Concept Question 0003. \\ \fi
\begin{problem}

What is wrong with the following equation:

\[
\int_{\frac{1}{4} \, \pi}^{\frac{11}{6} \, \pi} {-10 \, \cot\left(x\right) \csc\left(x\right)}\;dx = {\frac{10}{\sin\left(x\right)}}\Bigg\vert_{\frac{1}{4} \, \pi}^{\frac{11}{6} \, \pi} = {-10 \, \sqrt{2} - 20}
\]

\input{Integral-Concept-0003.HELP.tex}

\begin{multipleChoice}
\choice{The antiderivative is incorrect.}
\choice[correct]{The integrand is not defined over the entire interval.}
\choice{The bounds are evaluated in the wrong order.}
\choice{Nothing is wrong.  The equation is correct, as is.}
\end{multipleChoice}

\end{problem}}%}

\latexProblemContent{
\ifVerboseLocation This is Integration Concept Question 0003. \\ \fi
\begin{problem}

What is wrong with the following equation:

\[
\int_{\frac{1}{6} \, \pi}^{\frac{11}{6} \, \pi} {-7 \, \cot\left(x\right) \csc\left(x\right)}\;dx = {\frac{7}{\sin\left(x\right)}}\Bigg\vert_{\frac{1}{6} \, \pi}^{\frac{11}{6} \, \pi} = {-28}
\]

\input{Integral-Concept-0003.HELP.tex}

\begin{multipleChoice}
\choice{The antiderivative is incorrect.}
\choice[correct]{The integrand is not defined over the entire interval.}
\choice{The bounds are evaluated in the wrong order.}
\choice{Nothing is wrong.  The equation is correct, as is.}
\end{multipleChoice}

\end{problem}}%}

\latexProblemContent{
\ifVerboseLocation This is Integration Concept Question 0003. \\ \fi
\begin{problem}

What is wrong with the following equation:

\[
\int_{\frac{2}{3} \, \pi}^{\frac{5}{4} \, \pi} {-2 \, \csc\left(x\right)^{2}}\;dx = {\frac{2}{\tan\left(x\right)}}\Bigg\vert_{\frac{2}{3} \, \pi}^{\frac{5}{4} \, \pi} = {\frac{2}{3} \, \sqrt{3} + 2}
\]

\input{Integral-Concept-0003.HELP.tex}

\begin{multipleChoice}
\choice{The antiderivative is incorrect.}
\choice[correct]{The integrand is not defined over the entire interval.}
\choice{The bounds are evaluated in the wrong order.}
\choice{Nothing is wrong.  The equation is correct, as is.}
\end{multipleChoice}

\end{problem}}%}

\latexProblemContent{
\ifVerboseLocation This is Integration Concept Question 0003. \\ \fi
\begin{problem}

What is wrong with the following equation:

\[
\int_{\frac{1}{3} \, \pi}^{\frac{7}{6} \, \pi} {\cot\left(x\right) \csc\left(x\right)}\;dx = {-\frac{1}{\sin\left(x\right)}}\Bigg\vert_{\frac{1}{3} \, \pi}^{\frac{7}{6} \, \pi} = {\frac{2}{3} \, \sqrt{3} + 2}
\]

\input{Integral-Concept-0003.HELP.tex}

\begin{multipleChoice}
\choice{The antiderivative is incorrect.}
\choice[correct]{The integrand is not defined over the entire interval.}
\choice{The bounds are evaluated in the wrong order.}
\choice{Nothing is wrong.  The equation is correct, as is.}
\end{multipleChoice}

\end{problem}}%}

\latexProblemContent{
\ifVerboseLocation This is Integration Concept Question 0003. \\ \fi
\begin{problem}

What is wrong with the following equation:

\[
\int_{\frac{1}{6} \, \pi}^{\frac{7}{6} \, \pi} {-9 \, \csc\left(x\right)^{2}}\;dx = {\frac{9}{\tan\left(x\right)}}\Bigg\vert_{\frac{1}{6} \, \pi}^{\frac{7}{6} \, \pi} = {0}
\]

\input{Integral-Concept-0003.HELP.tex}

\begin{multipleChoice}
\choice{The antiderivative is incorrect.}
\choice[correct]{The integrand is not defined over the entire interval.}
\choice{The bounds are evaluated in the wrong order.}
\choice{Nothing is wrong.  The equation is correct, as is.}
\end{multipleChoice}

\end{problem}}%}

\latexProblemContent{
\ifVerboseLocation This is Integration Concept Question 0003. \\ \fi
\begin{problem}

What is wrong with the following equation:

\[
\int_{\frac{2}{3} \, \pi}^{\frac{7}{4} \, \pi} {-\csc\left(x\right)^{2}}\;dx = {\frac{1}{\tan\left(x\right)}}\Bigg\vert_{\frac{2}{3} \, \pi}^{\frac{7}{4} \, \pi} = {\frac{1}{3} \, \sqrt{3} - 1}
\]

\input{Integral-Concept-0003.HELP.tex}

\begin{multipleChoice}
\choice{The antiderivative is incorrect.}
\choice[correct]{The integrand is not defined over the entire interval.}
\choice{The bounds are evaluated in the wrong order.}
\choice{Nothing is wrong.  The equation is correct, as is.}
\end{multipleChoice}

\end{problem}}%}

\latexProblemContent{
\ifVerboseLocation This is Integration Concept Question 0003. \\ \fi
\begin{problem}

What is wrong with the following equation:

\[
\int_{\frac{1}{4} \, \pi}^{\frac{7}{6} \, \pi} {-2 \, \cot\left(x\right) \csc\left(x\right)}\;dx = {\frac{2}{\sin\left(x\right)}}\Bigg\vert_{\frac{1}{4} \, \pi}^{\frac{7}{6} \, \pi} = {-2 \, \sqrt{2} - 4}
\]

\input{Integral-Concept-0003.HELP.tex}

\begin{multipleChoice}
\choice{The antiderivative is incorrect.}
\choice[correct]{The integrand is not defined over the entire interval.}
\choice{The bounds are evaluated in the wrong order.}
\choice{Nothing is wrong.  The equation is correct, as is.}
\end{multipleChoice}

\end{problem}}%}

\latexProblemContent{
\ifVerboseLocation This is Integration Concept Question 0003. \\ \fi
\begin{problem}

What is wrong with the following equation:

\[
\int_{\frac{1}{4} \, \pi}^{\frac{4}{3} \, \pi} {2 \, \cot\left(x\right) \csc\left(x\right)}\;dx = {-\frac{2}{\sin\left(x\right)}}\Bigg\vert_{\frac{1}{4} \, \pi}^{\frac{4}{3} \, \pi} = {\frac{4}{3} \, \sqrt{3} + 2 \, \sqrt{2}}
\]

\input{Integral-Concept-0003.HELP.tex}

\begin{multipleChoice}
\choice{The antiderivative is incorrect.}
\choice[correct]{The integrand is not defined over the entire interval.}
\choice{The bounds are evaluated in the wrong order.}
\choice{Nothing is wrong.  The equation is correct, as is.}
\end{multipleChoice}

\end{problem}}%}

\latexProblemContent{
\ifVerboseLocation This is Integration Concept Question 0003. \\ \fi
\begin{problem}

What is wrong with the following equation:

\[
\int_{\frac{5}{6} \, \pi}^{\frac{3}{2} \, \pi} {2 \, \cot\left(x\right) \csc\left(x\right)}\;dx = {-\frac{2}{\sin\left(x\right)}}\Bigg\vert_{\frac{5}{6} \, \pi}^{\frac{3}{2} \, \pi} = {6}
\]

\input{Integral-Concept-0003.HELP.tex}

\begin{multipleChoice}
\choice{The antiderivative is incorrect.}
\choice[correct]{The integrand is not defined over the entire interval.}
\choice{The bounds are evaluated in the wrong order.}
\choice{Nothing is wrong.  The equation is correct, as is.}
\end{multipleChoice}

\end{problem}}%}

\latexProblemContent{
\ifVerboseLocation This is Integration Concept Question 0003. \\ \fi
\begin{problem}

What is wrong with the following equation:

\[
\int_{\frac{1}{3} \, \pi}^{\frac{5}{3} \, \pi} {-3 \, \csc\left(x\right)^{2}}\;dx = {\frac{3}{\tan\left(x\right)}}\Bigg\vert_{\frac{1}{3} \, \pi}^{\frac{5}{3} \, \pi} = {-2 \, \sqrt{3}}
\]

\input{Integral-Concept-0003.HELP.tex}

\begin{multipleChoice}
\choice{The antiderivative is incorrect.}
\choice[correct]{The integrand is not defined over the entire interval.}
\choice{The bounds are evaluated in the wrong order.}
\choice{Nothing is wrong.  The equation is correct, as is.}
\end{multipleChoice}

\end{problem}}%}

\latexProblemContent{
\ifVerboseLocation This is Integration Concept Question 0003. \\ \fi
\begin{problem}

What is wrong with the following equation:

\[
\int_{\frac{1}{3} \, \pi}^{\frac{5}{4} \, \pi} {-8 \, \cot\left(x\right) \csc\left(x\right)}\;dx = {\frac{8}{\sin\left(x\right)}}\Bigg\vert_{\frac{1}{3} \, \pi}^{\frac{5}{4} \, \pi} = {-\frac{16}{3} \, \sqrt{3} - 8 \, \sqrt{2}}
\]

\input{Integral-Concept-0003.HELP.tex}

\begin{multipleChoice}
\choice{The antiderivative is incorrect.}
\choice[correct]{The integrand is not defined over the entire interval.}
\choice{The bounds are evaluated in the wrong order.}
\choice{Nothing is wrong.  The equation is correct, as is.}
\end{multipleChoice}

\end{problem}}%}

\latexProblemContent{
\ifVerboseLocation This is Integration Concept Question 0003. \\ \fi
\begin{problem}

What is wrong with the following equation:

\[
\int_{\frac{2}{3} \, \pi}^{\frac{4}{3} \, \pi} {-5 \, \cot\left(x\right) \csc\left(x\right)}\;dx = {\frac{5}{\sin\left(x\right)}}\Bigg\vert_{\frac{2}{3} \, \pi}^{\frac{4}{3} \, \pi} = {-\frac{20}{3} \, \sqrt{3}}
\]

\input{Integral-Concept-0003.HELP.tex}

\begin{multipleChoice}
\choice{The antiderivative is incorrect.}
\choice[correct]{The integrand is not defined over the entire interval.}
\choice{The bounds are evaluated in the wrong order.}
\choice{Nothing is wrong.  The equation is correct, as is.}
\end{multipleChoice}

\end{problem}}%}

\latexProblemContent{
\ifVerboseLocation This is Integration Concept Question 0003. \\ \fi
\begin{problem}

What is wrong with the following equation:

\[
\int_{\frac{2}{3} \, \pi}^{\frac{3}{2} \, \pi} {\csc\left(x\right)^{2}}\;dx = {-\frac{1}{\tan\left(x\right)}}\Bigg\vert_{\frac{2}{3} \, \pi}^{\frac{3}{2} \, \pi} = {-\frac{1}{3} \, \sqrt{3}}
\]

\input{Integral-Concept-0003.HELP.tex}

\begin{multipleChoice}
\choice{The antiderivative is incorrect.}
\choice[correct]{The integrand is not defined over the entire interval.}
\choice{The bounds are evaluated in the wrong order.}
\choice{Nothing is wrong.  The equation is correct, as is.}
\end{multipleChoice}

\end{problem}}%}

\latexProblemContent{
\ifVerboseLocation This is Integration Concept Question 0003. \\ \fi
\begin{problem}

What is wrong with the following equation:

\[
\int_{\frac{1}{3} \, \pi}^{\frac{5}{4} \, \pi} {-3 \, \csc\left(x\right)^{2}}\;dx = {\frac{3}{\tan\left(x\right)}}\Bigg\vert_{\frac{1}{3} \, \pi}^{\frac{5}{4} \, \pi} = {-\sqrt{3} + 3}
\]

\input{Integral-Concept-0003.HELP.tex}

\begin{multipleChoice}
\choice{The antiderivative is incorrect.}
\choice[correct]{The integrand is not defined over the entire interval.}
\choice{The bounds are evaluated in the wrong order.}
\choice{Nothing is wrong.  The equation is correct, as is.}
\end{multipleChoice}

\end{problem}}%}

\latexProblemContent{
\ifVerboseLocation This is Integration Concept Question 0003. \\ \fi
\begin{problem}

What is wrong with the following equation:

\[
\int_{\frac{1}{2} \, \pi}^{\frac{4}{3} \, \pi} {-4 \, \csc\left(x\right)^{2}}\;dx = {\frac{4}{\tan\left(x\right)}}\Bigg\vert_{\frac{1}{2} \, \pi}^{\frac{4}{3} \, \pi} = {\frac{4}{3} \, \sqrt{3}}
\]

\input{Integral-Concept-0003.HELP.tex}

\begin{multipleChoice}
\choice{The antiderivative is incorrect.}
\choice[correct]{The integrand is not defined over the entire interval.}
\choice{The bounds are evaluated in the wrong order.}
\choice{Nothing is wrong.  The equation is correct, as is.}
\end{multipleChoice}

\end{problem}}%}

\latexProblemContent{
\ifVerboseLocation This is Integration Concept Question 0003. \\ \fi
\begin{problem}

What is wrong with the following equation:

\[
\int_{\frac{3}{4} \, \pi}^{\frac{7}{4} \, \pi} {3 \, \csc\left(x\right)^{2}}\;dx = {-\frac{3}{\tan\left(x\right)}}\Bigg\vert_{\frac{3}{4} \, \pi}^{\frac{7}{4} \, \pi} = {0}
\]

\input{Integral-Concept-0003.HELP.tex}

\begin{multipleChoice}
\choice{The antiderivative is incorrect.}
\choice[correct]{The integrand is not defined over the entire interval.}
\choice{The bounds are evaluated in the wrong order.}
\choice{Nothing is wrong.  The equation is correct, as is.}
\end{multipleChoice}

\end{problem}}%}

\latexProblemContent{
\ifVerboseLocation This is Integration Concept Question 0003. \\ \fi
\begin{problem}

What is wrong with the following equation:

\[
\int_{\frac{2}{3} \, \pi}^{\frac{11}{6} \, \pi} {-7 \, \csc\left(x\right)^{2}}\;dx = {\frac{7}{\tan\left(x\right)}}\Bigg\vert_{\frac{2}{3} \, \pi}^{\frac{11}{6} \, \pi} = {-\frac{14}{3} \, \sqrt{3}}
\]

\input{Integral-Concept-0003.HELP.tex}

\begin{multipleChoice}
\choice{The antiderivative is incorrect.}
\choice[correct]{The integrand is not defined over the entire interval.}
\choice{The bounds are evaluated in the wrong order.}
\choice{Nothing is wrong.  The equation is correct, as is.}
\end{multipleChoice}

\end{problem}}%}

\latexProblemContent{
\ifVerboseLocation This is Integration Concept Question 0003. \\ \fi
\begin{problem}

What is wrong with the following equation:

\[
\int_{\frac{1}{4} \, \pi}^{\frac{7}{6} \, \pi} {-4 \, \cot\left(x\right) \csc\left(x\right)}\;dx = {\frac{4}{\sin\left(x\right)}}\Bigg\vert_{\frac{1}{4} \, \pi}^{\frac{7}{6} \, \pi} = {-4 \, \sqrt{2} - 8}
\]

\input{Integral-Concept-0003.HELP.tex}

\begin{multipleChoice}
\choice{The antiderivative is incorrect.}
\choice[correct]{The integrand is not defined over the entire interval.}
\choice{The bounds are evaluated in the wrong order.}
\choice{Nothing is wrong.  The equation is correct, as is.}
\end{multipleChoice}

\end{problem}}%}

\latexProblemContent{
\ifVerboseLocation This is Integration Concept Question 0003. \\ \fi
\begin{problem}

What is wrong with the following equation:

\[
\int_{\frac{5}{6} \, \pi}^{\frac{7}{6} \, \pi} {6 \, \csc\left(x\right)^{2}}\;dx = {-\frac{6}{\tan\left(x\right)}}\Bigg\vert_{\frac{5}{6} \, \pi}^{\frac{7}{6} \, \pi} = {-12 \, \sqrt{3}}
\]

\input{Integral-Concept-0003.HELP.tex}

\begin{multipleChoice}
\choice{The antiderivative is incorrect.}
\choice[correct]{The integrand is not defined over the entire interval.}
\choice{The bounds are evaluated in the wrong order.}
\choice{Nothing is wrong.  The equation is correct, as is.}
\end{multipleChoice}

\end{problem}}%}

\latexProblemContent{
\ifVerboseLocation This is Integration Concept Question 0003. \\ \fi
\begin{problem}

What is wrong with the following equation:

\[
\int_{\frac{1}{3} \, \pi}^{\frac{7}{4} \, \pi} {7 \, \cot\left(x\right) \csc\left(x\right)}\;dx = {-\frac{7}{\sin\left(x\right)}}\Bigg\vert_{\frac{1}{3} \, \pi}^{\frac{7}{4} \, \pi} = {\frac{14}{3} \, \sqrt{3} + 7 \, \sqrt{2}}
\]

\input{Integral-Concept-0003.HELP.tex}

\begin{multipleChoice}
\choice{The antiderivative is incorrect.}
\choice[correct]{The integrand is not defined over the entire interval.}
\choice{The bounds are evaluated in the wrong order.}
\choice{Nothing is wrong.  The equation is correct, as is.}
\end{multipleChoice}

\end{problem}}%}

\latexProblemContent{
\ifVerboseLocation This is Integration Concept Question 0003. \\ \fi
\begin{problem}

What is wrong with the following equation:

\[
\int_{\frac{2}{3} \, \pi}^{\frac{4}{3} \, \pi} {\cot\left(x\right) \csc\left(x\right)}\;dx = {-\frac{1}{\sin\left(x\right)}}\Bigg\vert_{\frac{2}{3} \, \pi}^{\frac{4}{3} \, \pi} = {\frac{4}{3} \, \sqrt{3}}
\]

\input{Integral-Concept-0003.HELP.tex}

\begin{multipleChoice}
\choice{The antiderivative is incorrect.}
\choice[correct]{The integrand is not defined over the entire interval.}
\choice{The bounds are evaluated in the wrong order.}
\choice{Nothing is wrong.  The equation is correct, as is.}
\end{multipleChoice}

\end{problem}}%}

\latexProblemContent{
\ifVerboseLocation This is Integration Concept Question 0003. \\ \fi
\begin{problem}

What is wrong with the following equation:

\[
\int_{\frac{3}{4} \, \pi}^{\frac{5}{3} \, \pi} {-6 \, \csc\left(x\right)^{2}}\;dx = {\frac{6}{\tan\left(x\right)}}\Bigg\vert_{\frac{3}{4} \, \pi}^{\frac{5}{3} \, \pi} = {-2 \, \sqrt{3} + 6}
\]

\input{Integral-Concept-0003.HELP.tex}

\begin{multipleChoice}
\choice{The antiderivative is incorrect.}
\choice[correct]{The integrand is not defined over the entire interval.}
\choice{The bounds are evaluated in the wrong order.}
\choice{Nothing is wrong.  The equation is correct, as is.}
\end{multipleChoice}

\end{problem}}%}

\latexProblemContent{
\ifVerboseLocation This is Integration Concept Question 0003. \\ \fi
\begin{problem}

What is wrong with the following equation:

\[
\int_{\frac{2}{3} \, \pi}^{\frac{3}{2} \, \pi} {4 \, \csc\left(x\right)^{2}}\;dx = {-\frac{4}{\tan\left(x\right)}}\Bigg\vert_{\frac{2}{3} \, \pi}^{\frac{3}{2} \, \pi} = {-\frac{4}{3} \, \sqrt{3}}
\]

\input{Integral-Concept-0003.HELP.tex}

\begin{multipleChoice}
\choice{The antiderivative is incorrect.}
\choice[correct]{The integrand is not defined over the entire interval.}
\choice{The bounds are evaluated in the wrong order.}
\choice{Nothing is wrong.  The equation is correct, as is.}
\end{multipleChoice}

\end{problem}}%}

\latexProblemContent{
\ifVerboseLocation This is Integration Concept Question 0003. \\ \fi
\begin{problem}

What is wrong with the following equation:

\[
\int_{\frac{1}{6} \, \pi}^{\frac{5}{3} \, \pi} {8 \, \cot\left(x\right) \csc\left(x\right)}\;dx = {-\frac{8}{\sin\left(x\right)}}\Bigg\vert_{\frac{1}{6} \, \pi}^{\frac{5}{3} \, \pi} = {\frac{16}{3} \, \sqrt{3} + 16}
\]

\input{Integral-Concept-0003.HELP.tex}

\begin{multipleChoice}
\choice{The antiderivative is incorrect.}
\choice[correct]{The integrand is not defined over the entire interval.}
\choice{The bounds are evaluated in the wrong order.}
\choice{Nothing is wrong.  The equation is correct, as is.}
\end{multipleChoice}

\end{problem}}%}

\latexProblemContent{
\ifVerboseLocation This is Integration Concept Question 0003. \\ \fi
\begin{problem}

What is wrong with the following equation:

\[
\int_{\frac{5}{6} \, \pi}^{\frac{5}{4} \, \pi} {-7 \, \cot\left(x\right) \csc\left(x\right)}\;dx = {\frac{7}{\sin\left(x\right)}}\Bigg\vert_{\frac{5}{6} \, \pi}^{\frac{5}{4} \, \pi} = {-7 \, \sqrt{2} - 14}
\]

\input{Integral-Concept-0003.HELP.tex}

\begin{multipleChoice}
\choice{The antiderivative is incorrect.}
\choice[correct]{The integrand is not defined over the entire interval.}
\choice{The bounds are evaluated in the wrong order.}
\choice{Nothing is wrong.  The equation is correct, as is.}
\end{multipleChoice}

\end{problem}}%}

\latexProblemContent{
\ifVerboseLocation This is Integration Concept Question 0003. \\ \fi
\begin{problem}

What is wrong with the following equation:

\[
\int_{\frac{2}{3} \, \pi}^{\frac{5}{3} \, \pi} {4 \, \csc\left(x\right)^{2}}\;dx = {-\frac{4}{\tan\left(x\right)}}\Bigg\vert_{\frac{2}{3} \, \pi}^{\frac{5}{3} \, \pi} = {0}
\]

\input{Integral-Concept-0003.HELP.tex}

\begin{multipleChoice}
\choice{The antiderivative is incorrect.}
\choice[correct]{The integrand is not defined over the entire interval.}
\choice{The bounds are evaluated in the wrong order.}
\choice{Nothing is wrong.  The equation is correct, as is.}
\end{multipleChoice}

\end{problem}}%}

\latexProblemContent{
\ifVerboseLocation This is Integration Concept Question 0003. \\ \fi
\begin{problem}

What is wrong with the following equation:

\[
\int_{\frac{3}{4} \, \pi}^{\frac{7}{4} \, \pi} {-8 \, \csc\left(x\right)^{2}}\;dx = {\frac{8}{\tan\left(x\right)}}\Bigg\vert_{\frac{3}{4} \, \pi}^{\frac{7}{4} \, \pi} = {0}
\]

\input{Integral-Concept-0003.HELP.tex}

\begin{multipleChoice}
\choice{The antiderivative is incorrect.}
\choice[correct]{The integrand is not defined over the entire interval.}
\choice{The bounds are evaluated in the wrong order.}
\choice{Nothing is wrong.  The equation is correct, as is.}
\end{multipleChoice}

\end{problem}}%}

\latexProblemContent{
\ifVerboseLocation This is Integration Concept Question 0003. \\ \fi
\begin{problem}

What is wrong with the following equation:

\[
\int_{\frac{5}{6} \, \pi}^{\frac{3}{2} \, \pi} {7 \, \csc\left(x\right)^{2}}\;dx = {-\frac{7}{\tan\left(x\right)}}\Bigg\vert_{\frac{5}{6} \, \pi}^{\frac{3}{2} \, \pi} = {-7 \, \sqrt{3}}
\]

\input{Integral-Concept-0003.HELP.tex}

\begin{multipleChoice}
\choice{The antiderivative is incorrect.}
\choice[correct]{The integrand is not defined over the entire interval.}
\choice{The bounds are evaluated in the wrong order.}
\choice{Nothing is wrong.  The equation is correct, as is.}
\end{multipleChoice}

\end{problem}}%}

\latexProblemContent{
\ifVerboseLocation This is Integration Concept Question 0003. \\ \fi
\begin{problem}

What is wrong with the following equation:

\[
\int_{\frac{1}{2} \, \pi}^{\frac{3}{2} \, \pi} {-7 \, \csc\left(x\right)^{2}}\;dx = {\frac{7}{\tan\left(x\right)}}\Bigg\vert_{\frac{1}{2} \, \pi}^{\frac{3}{2} \, \pi} = {0}
\]

\input{Integral-Concept-0003.HELP.tex}

\begin{multipleChoice}
\choice{The antiderivative is incorrect.}
\choice[correct]{The integrand is not defined over the entire interval.}
\choice{The bounds are evaluated in the wrong order.}
\choice{Nothing is wrong.  The equation is correct, as is.}
\end{multipleChoice}

\end{problem}}%}

\latexProblemContent{
\ifVerboseLocation This is Integration Concept Question 0003. \\ \fi
\begin{problem}

What is wrong with the following equation:

\[
\int_{\frac{1}{4} \, \pi}^{\frac{11}{6} \, \pi} {4 \, \csc\left(x\right)^{2}}\;dx = {-\frac{4}{\tan\left(x\right)}}\Bigg\vert_{\frac{1}{4} \, \pi}^{\frac{11}{6} \, \pi} = {4 \, \sqrt{3} + 4}
\]

\input{Integral-Concept-0003.HELP.tex}

\begin{multipleChoice}
\choice{The antiderivative is incorrect.}
\choice[correct]{The integrand is not defined over the entire interval.}
\choice{The bounds are evaluated in the wrong order.}
\choice{Nothing is wrong.  The equation is correct, as is.}
\end{multipleChoice}

\end{problem}}%}

\latexProblemContent{
\ifVerboseLocation This is Integration Concept Question 0003. \\ \fi
\begin{problem}

What is wrong with the following equation:

\[
\int_{\frac{1}{2} \, \pi}^{\frac{4}{3} \, \pi} {8 \, \csc\left(x\right)^{2}}\;dx = {-\frac{8}{\tan\left(x\right)}}\Bigg\vert_{\frac{1}{2} \, \pi}^{\frac{4}{3} \, \pi} = {-\frac{8}{3} \, \sqrt{3}}
\]

\input{Integral-Concept-0003.HELP.tex}

\begin{multipleChoice}
\choice{The antiderivative is incorrect.}
\choice[correct]{The integrand is not defined over the entire interval.}
\choice{The bounds are evaluated in the wrong order.}
\choice{Nothing is wrong.  The equation is correct, as is.}
\end{multipleChoice}

\end{problem}}%}

\latexProblemContent{
\ifVerboseLocation This is Integration Concept Question 0003. \\ \fi
\begin{problem}

What is wrong with the following equation:

\[
\int_{\frac{5}{6} \, \pi}^{\frac{5}{3} \, \pi} {9 \, \cot\left(x\right) \csc\left(x\right)}\;dx = {-\frac{9}{\sin\left(x\right)}}\Bigg\vert_{\frac{5}{6} \, \pi}^{\frac{5}{3} \, \pi} = {6 \, \sqrt{3} + 18}
\]

\input{Integral-Concept-0003.HELP.tex}

\begin{multipleChoice}
\choice{The antiderivative is incorrect.}
\choice[correct]{The integrand is not defined over the entire interval.}
\choice{The bounds are evaluated in the wrong order.}
\choice{Nothing is wrong.  The equation is correct, as is.}
\end{multipleChoice}

\end{problem}}%}

\latexProblemContent{
\ifVerboseLocation This is Integration Concept Question 0003. \\ \fi
\begin{problem}

What is wrong with the following equation:

\[
\int_{\frac{2}{3} \, \pi}^{\frac{7}{6} \, \pi} {-6 \, \cot\left(x\right) \csc\left(x\right)}\;dx = {\frac{6}{\sin\left(x\right)}}\Bigg\vert_{\frac{2}{3} \, \pi}^{\frac{7}{6} \, \pi} = {-4 \, \sqrt{3} - 12}
\]

\input{Integral-Concept-0003.HELP.tex}

\begin{multipleChoice}
\choice{The antiderivative is incorrect.}
\choice[correct]{The integrand is not defined over the entire interval.}
\choice{The bounds are evaluated in the wrong order.}
\choice{Nothing is wrong.  The equation is correct, as is.}
\end{multipleChoice}

\end{problem}}%}

\latexProblemContent{
\ifVerboseLocation This is Integration Concept Question 0003. \\ \fi
\begin{problem}

What is wrong with the following equation:

\[
\int_{\frac{1}{6} \, \pi}^{\frac{4}{3} \, \pi} {-2 \, \csc\left(x\right)^{2}}\;dx = {\frac{2}{\tan\left(x\right)}}\Bigg\vert_{\frac{1}{6} \, \pi}^{\frac{4}{3} \, \pi} = {-\frac{4}{3} \, \sqrt{3}}
\]

\input{Integral-Concept-0003.HELP.tex}

\begin{multipleChoice}
\choice{The antiderivative is incorrect.}
\choice[correct]{The integrand is not defined over the entire interval.}
\choice{The bounds are evaluated in the wrong order.}
\choice{Nothing is wrong.  The equation is correct, as is.}
\end{multipleChoice}

\end{problem}}%}

\latexProblemContent{
\ifVerboseLocation This is Integration Concept Question 0003. \\ \fi
\begin{problem}

What is wrong with the following equation:

\[
\int_{\frac{1}{4} \, \pi}^{\frac{11}{6} \, \pi} {10 \, \cot\left(x\right) \csc\left(x\right)}\;dx = {-\frac{10}{\sin\left(x\right)}}\Bigg\vert_{\frac{1}{4} \, \pi}^{\frac{11}{6} \, \pi} = {10 \, \sqrt{2} + 20}
\]

\input{Integral-Concept-0003.HELP.tex}

\begin{multipleChoice}
\choice{The antiderivative is incorrect.}
\choice[correct]{The integrand is not defined over the entire interval.}
\choice{The bounds are evaluated in the wrong order.}
\choice{Nothing is wrong.  The equation is correct, as is.}
\end{multipleChoice}

\end{problem}}%}

\latexProblemContent{
\ifVerboseLocation This is Integration Concept Question 0003. \\ \fi
\begin{problem}

What is wrong with the following equation:

\[
\int_{\frac{5}{6} \, \pi}^{\frac{5}{3} \, \pi} {-6 \, \cot\left(x\right) \csc\left(x\right)}\;dx = {\frac{6}{\sin\left(x\right)}}\Bigg\vert_{\frac{5}{6} \, \pi}^{\frac{5}{3} \, \pi} = {-4 \, \sqrt{3} - 12}
\]

\input{Integral-Concept-0003.HELP.tex}

\begin{multipleChoice}
\choice{The antiderivative is incorrect.}
\choice[correct]{The integrand is not defined over the entire interval.}
\choice{The bounds are evaluated in the wrong order.}
\choice{Nothing is wrong.  The equation is correct, as is.}
\end{multipleChoice}

\end{problem}}%}

\latexProblemContent{
\ifVerboseLocation This is Integration Concept Question 0003. \\ \fi
\begin{problem}

What is wrong with the following equation:

\[
\int_{\frac{2}{3} \, \pi}^{\frac{4}{3} \, \pi} {9 \, \cot\left(x\right) \csc\left(x\right)}\;dx = {-\frac{9}{\sin\left(x\right)}}\Bigg\vert_{\frac{2}{3} \, \pi}^{\frac{4}{3} \, \pi} = {12 \, \sqrt{3}}
\]

\input{Integral-Concept-0003.HELP.tex}

\begin{multipleChoice}
\choice{The antiderivative is incorrect.}
\choice[correct]{The integrand is not defined over the entire interval.}
\choice{The bounds are evaluated in the wrong order.}
\choice{Nothing is wrong.  The equation is correct, as is.}
\end{multipleChoice}

\end{problem}}%}

\latexProblemContent{
\ifVerboseLocation This is Integration Concept Question 0003. \\ \fi
\begin{problem}

What is wrong with the following equation:

\[
\int_{\frac{1}{6} \, \pi}^{\frac{7}{4} \, \pi} {8 \, \csc\left(x\right)^{2}}\;dx = {-\frac{8}{\tan\left(x\right)}}\Bigg\vert_{\frac{1}{6} \, \pi}^{\frac{7}{4} \, \pi} = {8 \, \sqrt{3} + 8}
\]

\input{Integral-Concept-0003.HELP.tex}

\begin{multipleChoice}
\choice{The antiderivative is incorrect.}
\choice[correct]{The integrand is not defined over the entire interval.}
\choice{The bounds are evaluated in the wrong order.}
\choice{Nothing is wrong.  The equation is correct, as is.}
\end{multipleChoice}

\end{problem}}%}

\latexProblemContent{
\ifVerboseLocation This is Integration Concept Question 0003. \\ \fi
\begin{problem}

What is wrong with the following equation:

\[
\int_{\frac{1}{4} \, \pi}^{\frac{5}{3} \, \pi} {-6 \, \cot\left(x\right) \csc\left(x\right)}\;dx = {\frac{6}{\sin\left(x\right)}}\Bigg\vert_{\frac{1}{4} \, \pi}^{\frac{5}{3} \, \pi} = {-4 \, \sqrt{3} - 6 \, \sqrt{2}}
\]

\input{Integral-Concept-0003.HELP.tex}

\begin{multipleChoice}
\choice{The antiderivative is incorrect.}
\choice[correct]{The integrand is not defined over the entire interval.}
\choice{The bounds are evaluated in the wrong order.}
\choice{Nothing is wrong.  The equation is correct, as is.}
\end{multipleChoice}

\end{problem}}%}

\latexProblemContent{
\ifVerboseLocation This is Integration Concept Question 0003. \\ \fi
\begin{problem}

What is wrong with the following equation:

\[
\int_{\frac{1}{2} \, \pi}^{\frac{11}{6} \, \pi} {10 \, \cot\left(x\right) \csc\left(x\right)}\;dx = {-\frac{10}{\sin\left(x\right)}}\Bigg\vert_{\frac{1}{2} \, \pi}^{\frac{11}{6} \, \pi} = {30}
\]

\input{Integral-Concept-0003.HELP.tex}

\begin{multipleChoice}
\choice{The antiderivative is incorrect.}
\choice[correct]{The integrand is not defined over the entire interval.}
\choice{The bounds are evaluated in the wrong order.}
\choice{Nothing is wrong.  The equation is correct, as is.}
\end{multipleChoice}

\end{problem}}%}

\latexProblemContent{
\ifVerboseLocation This is Integration Concept Question 0003. \\ \fi
\begin{problem}

What is wrong with the following equation:

\[
\int_{\frac{2}{3} \, \pi}^{\frac{5}{4} \, \pi} {4 \, \csc\left(x\right)^{2}}\;dx = {-\frac{4}{\tan\left(x\right)}}\Bigg\vert_{\frac{2}{3} \, \pi}^{\frac{5}{4} \, \pi} = {-\frac{4}{3} \, \sqrt{3} - 4}
\]

\input{Integral-Concept-0003.HELP.tex}

\begin{multipleChoice}
\choice{The antiderivative is incorrect.}
\choice[correct]{The integrand is not defined over the entire interval.}
\choice{The bounds are evaluated in the wrong order.}
\choice{Nothing is wrong.  The equation is correct, as is.}
\end{multipleChoice}

\end{problem}}%}

\latexProblemContent{
\ifVerboseLocation This is Integration Concept Question 0003. \\ \fi
\begin{problem}

What is wrong with the following equation:

\[
\int_{\frac{3}{4} \, \pi}^{\frac{7}{6} \, \pi} {-5 \, \csc\left(x\right)^{2}}\;dx = {\frac{5}{\tan\left(x\right)}}\Bigg\vert_{\frac{3}{4} \, \pi}^{\frac{7}{6} \, \pi} = {5 \, \sqrt{3} + 5}
\]

\input{Integral-Concept-0003.HELP.tex}

\begin{multipleChoice}
\choice{The antiderivative is incorrect.}
\choice[correct]{The integrand is not defined over the entire interval.}
\choice{The bounds are evaluated in the wrong order.}
\choice{Nothing is wrong.  The equation is correct, as is.}
\end{multipleChoice}

\end{problem}}%}

\latexProblemContent{
\ifVerboseLocation This is Integration Concept Question 0003. \\ \fi
\begin{problem}

What is wrong with the following equation:

\[
\int_{\frac{3}{4} \, \pi}^{\frac{11}{6} \, \pi} {6 \, \csc\left(x\right)^{2}}\;dx = {-\frac{6}{\tan\left(x\right)}}\Bigg\vert_{\frac{3}{4} \, \pi}^{\frac{11}{6} \, \pi} = {6 \, \sqrt{3} - 6}
\]

\input{Integral-Concept-0003.HELP.tex}

\begin{multipleChoice}
\choice{The antiderivative is incorrect.}
\choice[correct]{The integrand is not defined over the entire interval.}
\choice{The bounds are evaluated in the wrong order.}
\choice{Nothing is wrong.  The equation is correct, as is.}
\end{multipleChoice}

\end{problem}}%}

\latexProblemContent{
\ifVerboseLocation This is Integration Concept Question 0003. \\ \fi
\begin{problem}

What is wrong with the following equation:

\[
\int_{\frac{1}{4} \, \pi}^{\frac{5}{4} \, \pi} {6 \, \cot\left(x\right) \csc\left(x\right)}\;dx = {-\frac{6}{\sin\left(x\right)}}\Bigg\vert_{\frac{1}{4} \, \pi}^{\frac{5}{4} \, \pi} = {12 \, \sqrt{2}}
\]

\input{Integral-Concept-0003.HELP.tex}

\begin{multipleChoice}
\choice{The antiderivative is incorrect.}
\choice[correct]{The integrand is not defined over the entire interval.}
\choice{The bounds are evaluated in the wrong order.}
\choice{Nothing is wrong.  The equation is correct, as is.}
\end{multipleChoice}

\end{problem}}%}

\latexProblemContent{
\ifVerboseLocation This is Integration Concept Question 0003. \\ \fi
\begin{problem}

What is wrong with the following equation:

\[
\int_{\frac{1}{2} \, \pi}^{\frac{5}{3} \, \pi} {6 \, \csc\left(x\right)^{2}}\;dx = {-\frac{6}{\tan\left(x\right)}}\Bigg\vert_{\frac{1}{2} \, \pi}^{\frac{5}{3} \, \pi} = {2 \, \sqrt{3}}
\]

\input{Integral-Concept-0003.HELP.tex}

\begin{multipleChoice}
\choice{The antiderivative is incorrect.}
\choice[correct]{The integrand is not defined over the entire interval.}
\choice{The bounds are evaluated in the wrong order.}
\choice{Nothing is wrong.  The equation is correct, as is.}
\end{multipleChoice}

\end{problem}}%}

\latexProblemContent{
\ifVerboseLocation This is Integration Concept Question 0003. \\ \fi
\begin{problem}

What is wrong with the following equation:

\[
\int_{\frac{1}{2} \, \pi}^{\frac{7}{6} \, \pi} {-7 \, \csc\left(x\right)^{2}}\;dx = {\frac{7}{\tan\left(x\right)}}\Bigg\vert_{\frac{1}{2} \, \pi}^{\frac{7}{6} \, \pi} = {7 \, \sqrt{3}}
\]

\input{Integral-Concept-0003.HELP.tex}

\begin{multipleChoice}
\choice{The antiderivative is incorrect.}
\choice[correct]{The integrand is not defined over the entire interval.}
\choice{The bounds are evaluated in the wrong order.}
\choice{Nothing is wrong.  The equation is correct, as is.}
\end{multipleChoice}

\end{problem}}%}

\latexProblemContent{
\ifVerboseLocation This is Integration Concept Question 0003. \\ \fi
\begin{problem}

What is wrong with the following equation:

\[
\int_{\frac{1}{3} \, \pi}^{\frac{3}{2} \, \pi} {-9 \, \csc\left(x\right)^{2}}\;dx = {\frac{9}{\tan\left(x\right)}}\Bigg\vert_{\frac{1}{3} \, \pi}^{\frac{3}{2} \, \pi} = {-3 \, \sqrt{3}}
\]

\input{Integral-Concept-0003.HELP.tex}

\begin{multipleChoice}
\choice{The antiderivative is incorrect.}
\choice[correct]{The integrand is not defined over the entire interval.}
\choice{The bounds are evaluated in the wrong order.}
\choice{Nothing is wrong.  The equation is correct, as is.}
\end{multipleChoice}

\end{problem}}%}

\latexProblemContent{
\ifVerboseLocation This is Integration Concept Question 0003. \\ \fi
\begin{problem}

What is wrong with the following equation:

\[
\int_{\frac{1}{6} \, \pi}^{\frac{7}{6} \, \pi} {2 \, \cot\left(x\right) \csc\left(x\right)}\;dx = {-\frac{2}{\sin\left(x\right)}}\Bigg\vert_{\frac{1}{6} \, \pi}^{\frac{7}{6} \, \pi} = {8}
\]

\input{Integral-Concept-0003.HELP.tex}

\begin{multipleChoice}
\choice{The antiderivative is incorrect.}
\choice[correct]{The integrand is not defined over the entire interval.}
\choice{The bounds are evaluated in the wrong order.}
\choice{Nothing is wrong.  The equation is correct, as is.}
\end{multipleChoice}

\end{problem}}%}

\latexProblemContent{
\ifVerboseLocation This is Integration Concept Question 0003. \\ \fi
\begin{problem}

What is wrong with the following equation:

\[
\int_{\frac{1}{3} \, \pi}^{\frac{5}{3} \, \pi} {-9 \, \csc\left(x\right)^{2}}\;dx = {\frac{9}{\tan\left(x\right)}}\Bigg\vert_{\frac{1}{3} \, \pi}^{\frac{5}{3} \, \pi} = {-6 \, \sqrt{3}}
\]

\input{Integral-Concept-0003.HELP.tex}

\begin{multipleChoice}
\choice{The antiderivative is incorrect.}
\choice[correct]{The integrand is not defined over the entire interval.}
\choice{The bounds are evaluated in the wrong order.}
\choice{Nothing is wrong.  The equation is correct, as is.}
\end{multipleChoice}

\end{problem}}%}

\latexProblemContent{
\ifVerboseLocation This is Integration Concept Question 0003. \\ \fi
\begin{problem}

What is wrong with the following equation:

\[
\int_{\frac{5}{6} \, \pi}^{\frac{5}{3} \, \pi} {-2 \, \csc\left(x\right)^{2}}\;dx = {\frac{2}{\tan\left(x\right)}}\Bigg\vert_{\frac{5}{6} \, \pi}^{\frac{5}{3} \, \pi} = {\frac{4}{3} \, \sqrt{3}}
\]

\input{Integral-Concept-0003.HELP.tex}

\begin{multipleChoice}
\choice{The antiderivative is incorrect.}
\choice[correct]{The integrand is not defined over the entire interval.}
\choice{The bounds are evaluated in the wrong order.}
\choice{Nothing is wrong.  The equation is correct, as is.}
\end{multipleChoice}

\end{problem}}%}

\latexProblemContent{
\ifVerboseLocation This is Integration Concept Question 0003. \\ \fi
\begin{problem}

What is wrong with the following equation:

\[
\int_{\frac{1}{2} \, \pi}^{\frac{4}{3} \, \pi} {-\csc\left(x\right)^{2}}\;dx = {\frac{1}{\tan\left(x\right)}}\Bigg\vert_{\frac{1}{2} \, \pi}^{\frac{4}{3} \, \pi} = {\frac{1}{3} \, \sqrt{3}}
\]

\input{Integral-Concept-0003.HELP.tex}

\begin{multipleChoice}
\choice{The antiderivative is incorrect.}
\choice[correct]{The integrand is not defined over the entire interval.}
\choice{The bounds are evaluated in the wrong order.}
\choice{Nothing is wrong.  The equation is correct, as is.}
\end{multipleChoice}

\end{problem}}%}

\latexProblemContent{
\ifVerboseLocation This is Integration Concept Question 0003. \\ \fi
\begin{problem}

What is wrong with the following equation:

\[
\int_{\frac{5}{6} \, \pi}^{\frac{5}{4} \, \pi} {-2 \, \csc\left(x\right)^{2}}\;dx = {\frac{2}{\tan\left(x\right)}}\Bigg\vert_{\frac{5}{6} \, \pi}^{\frac{5}{4} \, \pi} = {2 \, \sqrt{3} + 2}
\]

\input{Integral-Concept-0003.HELP.tex}

\begin{multipleChoice}
\choice{The antiderivative is incorrect.}
\choice[correct]{The integrand is not defined over the entire interval.}
\choice{The bounds are evaluated in the wrong order.}
\choice{Nothing is wrong.  The equation is correct, as is.}
\end{multipleChoice}

\end{problem}}%}

\latexProblemContent{
\ifVerboseLocation This is Integration Concept Question 0003. \\ \fi
\begin{problem}

What is wrong with the following equation:

\[
\int_{\frac{1}{6} \, \pi}^{\frac{4}{3} \, \pi} {-2 \, \cot\left(x\right) \csc\left(x\right)}\;dx = {\frac{2}{\sin\left(x\right)}}\Bigg\vert_{\frac{1}{6} \, \pi}^{\frac{4}{3} \, \pi} = {-\frac{4}{3} \, \sqrt{3} - 4}
\]

\input{Integral-Concept-0003.HELP.tex}

\begin{multipleChoice}
\choice{The antiderivative is incorrect.}
\choice[correct]{The integrand is not defined over the entire interval.}
\choice{The bounds are evaluated in the wrong order.}
\choice{Nothing is wrong.  The equation is correct, as is.}
\end{multipleChoice}

\end{problem}}%}

\latexProblemContent{
\ifVerboseLocation This is Integration Concept Question 0003. \\ \fi
\begin{problem}

What is wrong with the following equation:

\[
\int_{\frac{1}{2} \, \pi}^{\frac{7}{6} \, \pi} {3 \, \cot\left(x\right) \csc\left(x\right)}\;dx = {-\frac{3}{\sin\left(x\right)}}\Bigg\vert_{\frac{1}{2} \, \pi}^{\frac{7}{6} \, \pi} = {9}
\]

\input{Integral-Concept-0003.HELP.tex}

\begin{multipleChoice}
\choice{The antiderivative is incorrect.}
\choice[correct]{The integrand is not defined over the entire interval.}
\choice{The bounds are evaluated in the wrong order.}
\choice{Nothing is wrong.  The equation is correct, as is.}
\end{multipleChoice}

\end{problem}}%}

\latexProblemContent{
\ifVerboseLocation This is Integration Concept Question 0003. \\ \fi
\begin{problem}

What is wrong with the following equation:

\[
\int_{\frac{1}{6} \, \pi}^{\frac{11}{6} \, \pi} {9 \, \csc\left(x\right)^{2}}\;dx = {-\frac{9}{\tan\left(x\right)}}\Bigg\vert_{\frac{1}{6} \, \pi}^{\frac{11}{6} \, \pi} = {18 \, \sqrt{3}}
\]

\input{Integral-Concept-0003.HELP.tex}

\begin{multipleChoice}
\choice{The antiderivative is incorrect.}
\choice[correct]{The integrand is not defined over the entire interval.}
\choice{The bounds are evaluated in the wrong order.}
\choice{Nothing is wrong.  The equation is correct, as is.}
\end{multipleChoice}

\end{problem}}%}

\latexProblemContent{
\ifVerboseLocation This is Integration Concept Question 0003. \\ \fi
\begin{problem}

What is wrong with the following equation:

\[
\int_{\frac{1}{4} \, \pi}^{\frac{3}{2} \, \pi} {-6 \, \cot\left(x\right) \csc\left(x\right)}\;dx = {\frac{6}{\sin\left(x\right)}}\Bigg\vert_{\frac{1}{4} \, \pi}^{\frac{3}{2} \, \pi} = {-6 \, \sqrt{2} - 6}
\]

\input{Integral-Concept-0003.HELP.tex}

\begin{multipleChoice}
\choice{The antiderivative is incorrect.}
\choice[correct]{The integrand is not defined over the entire interval.}
\choice{The bounds are evaluated in the wrong order.}
\choice{Nothing is wrong.  The equation is correct, as is.}
\end{multipleChoice}

\end{problem}}%}

\latexProblemContent{
\ifVerboseLocation This is Integration Concept Question 0003. \\ \fi
\begin{problem}

What is wrong with the following equation:

\[
\int_{\frac{1}{2} \, \pi}^{\frac{3}{2} \, \pi} {9 \, \cot\left(x\right) \csc\left(x\right)}\;dx = {-\frac{9}{\sin\left(x\right)}}\Bigg\vert_{\frac{1}{2} \, \pi}^{\frac{3}{2} \, \pi} = {18}
\]

\input{Integral-Concept-0003.HELP.tex}

\begin{multipleChoice}
\choice{The antiderivative is incorrect.}
\choice[correct]{The integrand is not defined over the entire interval.}
\choice{The bounds are evaluated in the wrong order.}
\choice{Nothing is wrong.  The equation is correct, as is.}
\end{multipleChoice}

\end{problem}}%}

\latexProblemContent{
\ifVerboseLocation This is Integration Concept Question 0003. \\ \fi
\begin{problem}

What is wrong with the following equation:

\[
\int_{\frac{1}{3} \, \pi}^{\frac{11}{6} \, \pi} {-7 \, \cot\left(x\right) \csc\left(x\right)}\;dx = {\frac{7}{\sin\left(x\right)}}\Bigg\vert_{\frac{1}{3} \, \pi}^{\frac{11}{6} \, \pi} = {-\frac{14}{3} \, \sqrt{3} - 14}
\]

\input{Integral-Concept-0003.HELP.tex}

\begin{multipleChoice}
\choice{The antiderivative is incorrect.}
\choice[correct]{The integrand is not defined over the entire interval.}
\choice{The bounds are evaluated in the wrong order.}
\choice{Nothing is wrong.  The equation is correct, as is.}
\end{multipleChoice}

\end{problem}}%}

\latexProblemContent{
\ifVerboseLocation This is Integration Concept Question 0003. \\ \fi
\begin{problem}

What is wrong with the following equation:

\[
\int_{\frac{5}{6} \, \pi}^{\frac{5}{3} \, \pi} {6 \, \csc\left(x\right)^{2}}\;dx = {-\frac{6}{\tan\left(x\right)}}\Bigg\vert_{\frac{5}{6} \, \pi}^{\frac{5}{3} \, \pi} = {-4 \, \sqrt{3}}
\]

\input{Integral-Concept-0003.HELP.tex}

\begin{multipleChoice}
\choice{The antiderivative is incorrect.}
\choice[correct]{The integrand is not defined over the entire interval.}
\choice{The bounds are evaluated in the wrong order.}
\choice{Nothing is wrong.  The equation is correct, as is.}
\end{multipleChoice}

\end{problem}}%}

\latexProblemContent{
\ifVerboseLocation This is Integration Concept Question 0003. \\ \fi
\begin{problem}

What is wrong with the following equation:

\[
\int_{\frac{5}{6} \, \pi}^{\frac{7}{4} \, \pi} {9 \, \cot\left(x\right) \csc\left(x\right)}\;dx = {-\frac{9}{\sin\left(x\right)}}\Bigg\vert_{\frac{5}{6} \, \pi}^{\frac{7}{4} \, \pi} = {9 \, \sqrt{2} + 18}
\]

\input{Integral-Concept-0003.HELP.tex}

\begin{multipleChoice}
\choice{The antiderivative is incorrect.}
\choice[correct]{The integrand is not defined over the entire interval.}
\choice{The bounds are evaluated in the wrong order.}
\choice{Nothing is wrong.  The equation is correct, as is.}
\end{multipleChoice}

\end{problem}}%}

\latexProblemContent{
\ifVerboseLocation This is Integration Concept Question 0003. \\ \fi
\begin{problem}

What is wrong with the following equation:

\[
\int_{\frac{5}{6} \, \pi}^{\frac{3}{2} \, \pi} {-7 \, \csc\left(x\right)^{2}}\;dx = {\frac{7}{\tan\left(x\right)}}\Bigg\vert_{\frac{5}{6} \, \pi}^{\frac{3}{2} \, \pi} = {7 \, \sqrt{3}}
\]

\input{Integral-Concept-0003.HELP.tex}

\begin{multipleChoice}
\choice{The antiderivative is incorrect.}
\choice[correct]{The integrand is not defined over the entire interval.}
\choice{The bounds are evaluated in the wrong order.}
\choice{Nothing is wrong.  The equation is correct, as is.}
\end{multipleChoice}

\end{problem}}%}

\latexProblemContent{
\ifVerboseLocation This is Integration Concept Question 0003. \\ \fi
\begin{problem}

What is wrong with the following equation:

\[
\int_{\frac{1}{2} \, \pi}^{\frac{7}{4} \, \pi} {4 \, \cot\left(x\right) \csc\left(x\right)}\;dx = {-\frac{4}{\sin\left(x\right)}}\Bigg\vert_{\frac{1}{2} \, \pi}^{\frac{7}{4} \, \pi} = {4 \, \sqrt{2} + 4}
\]

\input{Integral-Concept-0003.HELP.tex}

\begin{multipleChoice}
\choice{The antiderivative is incorrect.}
\choice[correct]{The integrand is not defined over the entire interval.}
\choice{The bounds are evaluated in the wrong order.}
\choice{Nothing is wrong.  The equation is correct, as is.}
\end{multipleChoice}

\end{problem}}%}

\latexProblemContent{
\ifVerboseLocation This is Integration Concept Question 0003. \\ \fi
\begin{problem}

What is wrong with the following equation:

\[
\int_{\frac{1}{4} \, \pi}^{\frac{7}{6} \, \pi} {6 \, \cot\left(x\right) \csc\left(x\right)}\;dx = {-\frac{6}{\sin\left(x\right)}}\Bigg\vert_{\frac{1}{4} \, \pi}^{\frac{7}{6} \, \pi} = {6 \, \sqrt{2} + 12}
\]

\input{Integral-Concept-0003.HELP.tex}

\begin{multipleChoice}
\choice{The antiderivative is incorrect.}
\choice[correct]{The integrand is not defined over the entire interval.}
\choice{The bounds are evaluated in the wrong order.}
\choice{Nothing is wrong.  The equation is correct, as is.}
\end{multipleChoice}

\end{problem}}%}

\latexProblemContent{
\ifVerboseLocation This is Integration Concept Question 0003. \\ \fi
\begin{problem}

What is wrong with the following equation:

\[
\int_{\frac{1}{4} \, \pi}^{\frac{7}{4} \, \pi} {\cot\left(x\right) \csc\left(x\right)}\;dx = {-\frac{1}{\sin\left(x\right)}}\Bigg\vert_{\frac{1}{4} \, \pi}^{\frac{7}{4} \, \pi} = {2 \, \sqrt{2}}
\]

\input{Integral-Concept-0003.HELP.tex}

\begin{multipleChoice}
\choice{The antiderivative is incorrect.}
\choice[correct]{The integrand is not defined over the entire interval.}
\choice{The bounds are evaluated in the wrong order.}
\choice{Nothing is wrong.  The equation is correct, as is.}
\end{multipleChoice}

\end{problem}}%}

\latexProblemContent{
\ifVerboseLocation This is Integration Concept Question 0003. \\ \fi
\begin{problem}

What is wrong with the following equation:

\[
\int_{\frac{1}{4} \, \pi}^{\frac{5}{4} \, \pi} {8 \, \cot\left(x\right) \csc\left(x\right)}\;dx = {-\frac{8}{\sin\left(x\right)}}\Bigg\vert_{\frac{1}{4} \, \pi}^{\frac{5}{4} \, \pi} = {16 \, \sqrt{2}}
\]

\input{Integral-Concept-0003.HELP.tex}

\begin{multipleChoice}
\choice{The antiderivative is incorrect.}
\choice[correct]{The integrand is not defined over the entire interval.}
\choice{The bounds are evaluated in the wrong order.}
\choice{Nothing is wrong.  The equation is correct, as is.}
\end{multipleChoice}

\end{problem}}%}

\latexProblemContent{
\ifVerboseLocation This is Integration Concept Question 0003. \\ \fi
\begin{problem}

What is wrong with the following equation:

\[
\int_{\frac{1}{3} \, \pi}^{\frac{7}{6} \, \pi} {5 \, \cot\left(x\right) \csc\left(x\right)}\;dx = {-\frac{5}{\sin\left(x\right)}}\Bigg\vert_{\frac{1}{3} \, \pi}^{\frac{7}{6} \, \pi} = {\frac{10}{3} \, \sqrt{3} + 10}
\]

\input{Integral-Concept-0003.HELP.tex}

\begin{multipleChoice}
\choice{The antiderivative is incorrect.}
\choice[correct]{The integrand is not defined over the entire interval.}
\choice{The bounds are evaluated in the wrong order.}
\choice{Nothing is wrong.  The equation is correct, as is.}
\end{multipleChoice}

\end{problem}}%}

\latexProblemContent{
\ifVerboseLocation This is Integration Concept Question 0003. \\ \fi
\begin{problem}

What is wrong with the following equation:

\[
\int_{\frac{5}{6} \, \pi}^{\frac{4}{3} \, \pi} {-4 \, \cot\left(x\right) \csc\left(x\right)}\;dx = {\frac{4}{\sin\left(x\right)}}\Bigg\vert_{\frac{5}{6} \, \pi}^{\frac{4}{3} \, \pi} = {-\frac{8}{3} \, \sqrt{3} - 8}
\]

\input{Integral-Concept-0003.HELP.tex}

\begin{multipleChoice}
\choice{The antiderivative is incorrect.}
\choice[correct]{The integrand is not defined over the entire interval.}
\choice{The bounds are evaluated in the wrong order.}
\choice{Nothing is wrong.  The equation is correct, as is.}
\end{multipleChoice}

\end{problem}}%}

\latexProblemContent{
\ifVerboseLocation This is Integration Concept Question 0003. \\ \fi
\begin{problem}

What is wrong with the following equation:

\[
\int_{\frac{1}{2} \, \pi}^{\frac{4}{3} \, \pi} {2 \, \cot\left(x\right) \csc\left(x\right)}\;dx = {-\frac{2}{\sin\left(x\right)}}\Bigg\vert_{\frac{1}{2} \, \pi}^{\frac{4}{3} \, \pi} = {\frac{4}{3} \, \sqrt{3} + 2}
\]

\input{Integral-Concept-0003.HELP.tex}

\begin{multipleChoice}
\choice{The antiderivative is incorrect.}
\choice[correct]{The integrand is not defined over the entire interval.}
\choice{The bounds are evaluated in the wrong order.}
\choice{Nothing is wrong.  The equation is correct, as is.}
\end{multipleChoice}

\end{problem}}%}

\latexProblemContent{
\ifVerboseLocation This is Integration Concept Question 0003. \\ \fi
\begin{problem}

What is wrong with the following equation:

\[
\int_{\frac{1}{4} \, \pi}^{\frac{4}{3} \, \pi} {-3 \, \csc\left(x\right)^{2}}\;dx = {\frac{3}{\tan\left(x\right)}}\Bigg\vert_{\frac{1}{4} \, \pi}^{\frac{4}{3} \, \pi} = {\sqrt{3} - 3}
\]

\input{Integral-Concept-0003.HELP.tex}

\begin{multipleChoice}
\choice{The antiderivative is incorrect.}
\choice[correct]{The integrand is not defined over the entire interval.}
\choice{The bounds are evaluated in the wrong order.}
\choice{Nothing is wrong.  The equation is correct, as is.}
\end{multipleChoice}

\end{problem}}%}

\latexProblemContent{
\ifVerboseLocation This is Integration Concept Question 0003. \\ \fi
\begin{problem}

What is wrong with the following equation:

\[
\int_{\frac{1}{6} \, \pi}^{\frac{5}{3} \, \pi} {-2 \, \cot\left(x\right) \csc\left(x\right)}\;dx = {\frac{2}{\sin\left(x\right)}}\Bigg\vert_{\frac{1}{6} \, \pi}^{\frac{5}{3} \, \pi} = {-\frac{4}{3} \, \sqrt{3} - 4}
\]

\input{Integral-Concept-0003.HELP.tex}

\begin{multipleChoice}
\choice{The antiderivative is incorrect.}
\choice[correct]{The integrand is not defined over the entire interval.}
\choice{The bounds are evaluated in the wrong order.}
\choice{Nothing is wrong.  The equation is correct, as is.}
\end{multipleChoice}

\end{problem}}%}

\latexProblemContent{
\ifVerboseLocation This is Integration Concept Question 0003. \\ \fi
\begin{problem}

What is wrong with the following equation:

\[
\int_{\frac{1}{6} \, \pi}^{\frac{7}{6} \, \pi} {-6 \, \csc\left(x\right)^{2}}\;dx = {\frac{6}{\tan\left(x\right)}}\Bigg\vert_{\frac{1}{6} \, \pi}^{\frac{7}{6} \, \pi} = {0}
\]

\input{Integral-Concept-0003.HELP.tex}

\begin{multipleChoice}
\choice{The antiderivative is incorrect.}
\choice[correct]{The integrand is not defined over the entire interval.}
\choice{The bounds are evaluated in the wrong order.}
\choice{Nothing is wrong.  The equation is correct, as is.}
\end{multipleChoice}

\end{problem}}%}

\latexProblemContent{
\ifVerboseLocation This is Integration Concept Question 0003. \\ \fi
\begin{problem}

What is wrong with the following equation:

\[
\int_{\frac{3}{4} \, \pi}^{\frac{5}{3} \, \pi} {5 \, \csc\left(x\right)^{2}}\;dx = {-\frac{5}{\tan\left(x\right)}}\Bigg\vert_{\frac{3}{4} \, \pi}^{\frac{5}{3} \, \pi} = {\frac{5}{3} \, \sqrt{3} - 5}
\]

\input{Integral-Concept-0003.HELP.tex}

\begin{multipleChoice}
\choice{The antiderivative is incorrect.}
\choice[correct]{The integrand is not defined over the entire interval.}
\choice{The bounds are evaluated in the wrong order.}
\choice{Nothing is wrong.  The equation is correct, as is.}
\end{multipleChoice}

\end{problem}}%}

\latexProblemContent{
\ifVerboseLocation This is Integration Concept Question 0003. \\ \fi
\begin{problem}

What is wrong with the following equation:

\[
\int_{\frac{1}{2} \, \pi}^{\frac{5}{3} \, \pi} {9 \, \cot\left(x\right) \csc\left(x\right)}\;dx = {-\frac{9}{\sin\left(x\right)}}\Bigg\vert_{\frac{1}{2} \, \pi}^{\frac{5}{3} \, \pi} = {6 \, \sqrt{3} + 9}
\]

\input{Integral-Concept-0003.HELP.tex}

\begin{multipleChoice}
\choice{The antiderivative is incorrect.}
\choice[correct]{The integrand is not defined over the entire interval.}
\choice{The bounds are evaluated in the wrong order.}
\choice{Nothing is wrong.  The equation is correct, as is.}
\end{multipleChoice}

\end{problem}}%}

\latexProblemContent{
\ifVerboseLocation This is Integration Concept Question 0003. \\ \fi
\begin{problem}

What is wrong with the following equation:

\[
\int_{\frac{1}{3} \, \pi}^{\frac{5}{3} \, \pi} {8 \, \cot\left(x\right) \csc\left(x\right)}\;dx = {-\frac{8}{\sin\left(x\right)}}\Bigg\vert_{\frac{1}{3} \, \pi}^{\frac{5}{3} \, \pi} = {\frac{32}{3} \, \sqrt{3}}
\]

\input{Integral-Concept-0003.HELP.tex}

\begin{multipleChoice}
\choice{The antiderivative is incorrect.}
\choice[correct]{The integrand is not defined over the entire interval.}
\choice{The bounds are evaluated in the wrong order.}
\choice{Nothing is wrong.  The equation is correct, as is.}
\end{multipleChoice}

\end{problem}}%}

\latexProblemContent{
\ifVerboseLocation This is Integration Concept Question 0003. \\ \fi
\begin{problem}

What is wrong with the following equation:

\[
\int_{\frac{1}{4} \, \pi}^{\frac{3}{2} \, \pi} {-9 \, \cot\left(x\right) \csc\left(x\right)}\;dx = {\frac{9}{\sin\left(x\right)}}\Bigg\vert_{\frac{1}{4} \, \pi}^{\frac{3}{2} \, \pi} = {-9 \, \sqrt{2} - 9}
\]

\input{Integral-Concept-0003.HELP.tex}

\begin{multipleChoice}
\choice{The antiderivative is incorrect.}
\choice[correct]{The integrand is not defined over the entire interval.}
\choice{The bounds are evaluated in the wrong order.}
\choice{Nothing is wrong.  The equation is correct, as is.}
\end{multipleChoice}

\end{problem}}%}

\latexProblemContent{
\ifVerboseLocation This is Integration Concept Question 0003. \\ \fi
\begin{problem}

What is wrong with the following equation:

\[
\int_{\frac{1}{3} \, \pi}^{\frac{3}{2} \, \pi} {10 \, \cot\left(x\right) \csc\left(x\right)}\;dx = {-\frac{10}{\sin\left(x\right)}}\Bigg\vert_{\frac{1}{3} \, \pi}^{\frac{3}{2} \, \pi} = {\frac{20}{3} \, \sqrt{3} + 10}
\]

\input{Integral-Concept-0003.HELP.tex}

\begin{multipleChoice}
\choice{The antiderivative is incorrect.}
\choice[correct]{The integrand is not defined over the entire interval.}
\choice{The bounds are evaluated in the wrong order.}
\choice{Nothing is wrong.  The equation is correct, as is.}
\end{multipleChoice}

\end{problem}}%}

\latexProblemContent{
\ifVerboseLocation This is Integration Concept Question 0003. \\ \fi
\begin{problem}

What is wrong with the following equation:

\[
\int_{\frac{1}{6} \, \pi}^{\frac{5}{4} \, \pi} {-8 \, \cot\left(x\right) \csc\left(x\right)}\;dx = {\frac{8}{\sin\left(x\right)}}\Bigg\vert_{\frac{1}{6} \, \pi}^{\frac{5}{4} \, \pi} = {-8 \, \sqrt{2} - 16}
\]

\input{Integral-Concept-0003.HELP.tex}

\begin{multipleChoice}
\choice{The antiderivative is incorrect.}
\choice[correct]{The integrand is not defined over the entire interval.}
\choice{The bounds are evaluated in the wrong order.}
\choice{Nothing is wrong.  The equation is correct, as is.}
\end{multipleChoice}

\end{problem}}%}

\latexProblemContent{
\ifVerboseLocation This is Integration Concept Question 0003. \\ \fi
\begin{problem}

What is wrong with the following equation:

\[
\int_{\frac{1}{3} \, \pi}^{\frac{3}{2} \, \pi} {2 \, \cot\left(x\right) \csc\left(x\right)}\;dx = {-\frac{2}{\sin\left(x\right)}}\Bigg\vert_{\frac{1}{3} \, \pi}^{\frac{3}{2} \, \pi} = {\frac{4}{3} \, \sqrt{3} + 2}
\]

\input{Integral-Concept-0003.HELP.tex}

\begin{multipleChoice}
\choice{The antiderivative is incorrect.}
\choice[correct]{The integrand is not defined over the entire interval.}
\choice{The bounds are evaluated in the wrong order.}
\choice{Nothing is wrong.  The equation is correct, as is.}
\end{multipleChoice}

\end{problem}}%}

\latexProblemContent{
\ifVerboseLocation This is Integration Concept Question 0003. \\ \fi
\begin{problem}

What is wrong with the following equation:

\[
\int_{\frac{1}{3} \, \pi}^{\frac{3}{2} \, \pi} {3 \, \cot\left(x\right) \csc\left(x\right)}\;dx = {-\frac{3}{\sin\left(x\right)}}\Bigg\vert_{\frac{1}{3} \, \pi}^{\frac{3}{2} \, \pi} = {2 \, \sqrt{3} + 3}
\]

\input{Integral-Concept-0003.HELP.tex}

\begin{multipleChoice}
\choice{The antiderivative is incorrect.}
\choice[correct]{The integrand is not defined over the entire interval.}
\choice{The bounds are evaluated in the wrong order.}
\choice{Nothing is wrong.  The equation is correct, as is.}
\end{multipleChoice}

\end{problem}}%}

\latexProblemContent{
\ifVerboseLocation This is Integration Concept Question 0003. \\ \fi
\begin{problem}

What is wrong with the following equation:

\[
\int_{\frac{5}{6} \, \pi}^{\frac{7}{4} \, \pi} {-\csc\left(x\right)^{2}}\;dx = {\frac{1}{\tan\left(x\right)}}\Bigg\vert_{\frac{5}{6} \, \pi}^{\frac{7}{4} \, \pi} = {\sqrt{3} - 1}
\]

\input{Integral-Concept-0003.HELP.tex}

\begin{multipleChoice}
\choice{The antiderivative is incorrect.}
\choice[correct]{The integrand is not defined over the entire interval.}
\choice{The bounds are evaluated in the wrong order.}
\choice{Nothing is wrong.  The equation is correct, as is.}
\end{multipleChoice}

\end{problem}}%}

\latexProblemContent{
\ifVerboseLocation This is Integration Concept Question 0003. \\ \fi
\begin{problem}

What is wrong with the following equation:

\[
\int_{\frac{5}{6} \, \pi}^{\frac{11}{6} \, \pi} {-6 \, \cot\left(x\right) \csc\left(x\right)}\;dx = {\frac{6}{\sin\left(x\right)}}\Bigg\vert_{\frac{5}{6} \, \pi}^{\frac{11}{6} \, \pi} = {-24}
\]

\input{Integral-Concept-0003.HELP.tex}

\begin{multipleChoice}
\choice{The antiderivative is incorrect.}
\choice[correct]{The integrand is not defined over the entire interval.}
\choice{The bounds are evaluated in the wrong order.}
\choice{Nothing is wrong.  The equation is correct, as is.}
\end{multipleChoice}

\end{problem}}%}

\latexProblemContent{
\ifVerboseLocation This is Integration Concept Question 0003. \\ \fi
\begin{problem}

What is wrong with the following equation:

\[
\int_{\frac{2}{3} \, \pi}^{\frac{5}{3} \, \pi} {-10 \, \cot\left(x\right) \csc\left(x\right)}\;dx = {\frac{10}{\sin\left(x\right)}}\Bigg\vert_{\frac{2}{3} \, \pi}^{\frac{5}{3} \, \pi} = {-\frac{40}{3} \, \sqrt{3}}
\]

\input{Integral-Concept-0003.HELP.tex}

\begin{multipleChoice}
\choice{The antiderivative is incorrect.}
\choice[correct]{The integrand is not defined over the entire interval.}
\choice{The bounds are evaluated in the wrong order.}
\choice{Nothing is wrong.  The equation is correct, as is.}
\end{multipleChoice}

\end{problem}}%}

\latexProblemContent{
\ifVerboseLocation This is Integration Concept Question 0003. \\ \fi
\begin{problem}

What is wrong with the following equation:

\[
\int_{\frac{3}{4} \, \pi}^{\frac{3}{2} \, \pi} {8 \, \csc\left(x\right)^{2}}\;dx = {-\frac{8}{\tan\left(x\right)}}\Bigg\vert_{\frac{3}{4} \, \pi}^{\frac{3}{2} \, \pi} = {-8}
\]

\input{Integral-Concept-0003.HELP.tex}

\begin{multipleChoice}
\choice{The antiderivative is incorrect.}
\choice[correct]{The integrand is not defined over the entire interval.}
\choice{The bounds are evaluated in the wrong order.}
\choice{Nothing is wrong.  The equation is correct, as is.}
\end{multipleChoice}

\end{problem}}%}

\latexProblemContent{
\ifVerboseLocation This is Integration Concept Question 0003. \\ \fi
\begin{problem}

What is wrong with the following equation:

\[
\int_{\frac{1}{3} \, \pi}^{\frac{11}{6} \, \pi} {8 \, \csc\left(x\right)^{2}}\;dx = {-\frac{8}{\tan\left(x\right)}}\Bigg\vert_{\frac{1}{3} \, \pi}^{\frac{11}{6} \, \pi} = {\frac{32}{3} \, \sqrt{3}}
\]

\input{Integral-Concept-0003.HELP.tex}

\begin{multipleChoice}
\choice{The antiderivative is incorrect.}
\choice[correct]{The integrand is not defined over the entire interval.}
\choice{The bounds are evaluated in the wrong order.}
\choice{Nothing is wrong.  The equation is correct, as is.}
\end{multipleChoice}

\end{problem}}%}

\latexProblemContent{
\ifVerboseLocation This is Integration Concept Question 0003. \\ \fi
\begin{problem}

What is wrong with the following equation:

\[
\int_{\frac{1}{6} \, \pi}^{\frac{11}{6} \, \pi} {-4 \, \csc\left(x\right)^{2}}\;dx = {\frac{4}{\tan\left(x\right)}}\Bigg\vert_{\frac{1}{6} \, \pi}^{\frac{11}{6} \, \pi} = {-8 \, \sqrt{3}}
\]

\input{Integral-Concept-0003.HELP.tex}

\begin{multipleChoice}
\choice{The antiderivative is incorrect.}
\choice[correct]{The integrand is not defined over the entire interval.}
\choice{The bounds are evaluated in the wrong order.}
\choice{Nothing is wrong.  The equation is correct, as is.}
\end{multipleChoice}

\end{problem}}%}

\latexProblemContent{
\ifVerboseLocation This is Integration Concept Question 0003. \\ \fi
\begin{problem}

What is wrong with the following equation:

\[
\int_{\frac{2}{3} \, \pi}^{\frac{7}{6} \, \pi} {2 \, \cot\left(x\right) \csc\left(x\right)}\;dx = {-\frac{2}{\sin\left(x\right)}}\Bigg\vert_{\frac{2}{3} \, \pi}^{\frac{7}{6} \, \pi} = {\frac{4}{3} \, \sqrt{3} + 4}
\]

\input{Integral-Concept-0003.HELP.tex}

\begin{multipleChoice}
\choice{The antiderivative is incorrect.}
\choice[correct]{The integrand is not defined over the entire interval.}
\choice{The bounds are evaluated in the wrong order.}
\choice{Nothing is wrong.  The equation is correct, as is.}
\end{multipleChoice}

\end{problem}}%}

\latexProblemContent{
\ifVerboseLocation This is Integration Concept Question 0003. \\ \fi
\begin{problem}

What is wrong with the following equation:

\[
\int_{\frac{3}{4} \, \pi}^{\frac{3}{2} \, \pi} {9 \, \csc\left(x\right)^{2}}\;dx = {-\frac{9}{\tan\left(x\right)}}\Bigg\vert_{\frac{3}{4} \, \pi}^{\frac{3}{2} \, \pi} = {-9}
\]

\input{Integral-Concept-0003.HELP.tex}

\begin{multipleChoice}
\choice{The antiderivative is incorrect.}
\choice[correct]{The integrand is not defined over the entire interval.}
\choice{The bounds are evaluated in the wrong order.}
\choice{Nothing is wrong.  The equation is correct, as is.}
\end{multipleChoice}

\end{problem}}%}

\latexProblemContent{
\ifVerboseLocation This is Integration Concept Question 0003. \\ \fi
\begin{problem}

What is wrong with the following equation:

\[
\int_{\frac{1}{2} \, \pi}^{\frac{3}{2} \, \pi} {-8 \, \cot\left(x\right) \csc\left(x\right)}\;dx = {\frac{8}{\sin\left(x\right)}}\Bigg\vert_{\frac{1}{2} \, \pi}^{\frac{3}{2} \, \pi} = {-16}
\]

\input{Integral-Concept-0003.HELP.tex}

\begin{multipleChoice}
\choice{The antiderivative is incorrect.}
\choice[correct]{The integrand is not defined over the entire interval.}
\choice{The bounds are evaluated in the wrong order.}
\choice{Nothing is wrong.  The equation is correct, as is.}
\end{multipleChoice}

\end{problem}}%}

\latexProblemContent{
\ifVerboseLocation This is Integration Concept Question 0003. \\ \fi
\begin{problem}

What is wrong with the following equation:

\[
\int_{\frac{1}{4} \, \pi}^{\frac{11}{6} \, \pi} {-7 \, \cot\left(x\right) \csc\left(x\right)}\;dx = {\frac{7}{\sin\left(x\right)}}\Bigg\vert_{\frac{1}{4} \, \pi}^{\frac{11}{6} \, \pi} = {-7 \, \sqrt{2} - 14}
\]

\input{Integral-Concept-0003.HELP.tex}

\begin{multipleChoice}
\choice{The antiderivative is incorrect.}
\choice[correct]{The integrand is not defined over the entire interval.}
\choice{The bounds are evaluated in the wrong order.}
\choice{Nothing is wrong.  The equation is correct, as is.}
\end{multipleChoice}

\end{problem}}%}

\latexProblemContent{
\ifVerboseLocation This is Integration Concept Question 0003. \\ \fi
\begin{problem}

What is wrong with the following equation:

\[
\int_{\frac{1}{2} \, \pi}^{\frac{5}{4} \, \pi} {8 \, \cot\left(x\right) \csc\left(x\right)}\;dx = {-\frac{8}{\sin\left(x\right)}}\Bigg\vert_{\frac{1}{2} \, \pi}^{\frac{5}{4} \, \pi} = {8 \, \sqrt{2} + 8}
\]

\input{Integral-Concept-0003.HELP.tex}

\begin{multipleChoice}
\choice{The antiderivative is incorrect.}
\choice[correct]{The integrand is not defined over the entire interval.}
\choice{The bounds are evaluated in the wrong order.}
\choice{Nothing is wrong.  The equation is correct, as is.}
\end{multipleChoice}

\end{problem}}%}

\latexProblemContent{
\ifVerboseLocation This is Integration Concept Question 0003. \\ \fi
\begin{problem}

What is wrong with the following equation:

\[
\int_{\frac{5}{6} \, \pi}^{\frac{7}{4} \, \pi} {-5 \, \cot\left(x\right) \csc\left(x\right)}\;dx = {\frac{5}{\sin\left(x\right)}}\Bigg\vert_{\frac{5}{6} \, \pi}^{\frac{7}{4} \, \pi} = {-5 \, \sqrt{2} - 10}
\]

\input{Integral-Concept-0003.HELP.tex}

\begin{multipleChoice}
\choice{The antiderivative is incorrect.}
\choice[correct]{The integrand is not defined over the entire interval.}
\choice{The bounds are evaluated in the wrong order.}
\choice{Nothing is wrong.  The equation is correct, as is.}
\end{multipleChoice}

\end{problem}}%}

\latexProblemContent{
\ifVerboseLocation This is Integration Concept Question 0003. \\ \fi
\begin{problem}

What is wrong with the following equation:

\[
\int_{\frac{2}{3} \, \pi}^{\frac{11}{6} \, \pi} {-4 \, \csc\left(x\right)^{2}}\;dx = {\frac{4}{\tan\left(x\right)}}\Bigg\vert_{\frac{2}{3} \, \pi}^{\frac{11}{6} \, \pi} = {-\frac{8}{3} \, \sqrt{3}}
\]

\input{Integral-Concept-0003.HELP.tex}

\begin{multipleChoice}
\choice{The antiderivative is incorrect.}
\choice[correct]{The integrand is not defined over the entire interval.}
\choice{The bounds are evaluated in the wrong order.}
\choice{Nothing is wrong.  The equation is correct, as is.}
\end{multipleChoice}

\end{problem}}%}

\latexProblemContent{
\ifVerboseLocation This is Integration Concept Question 0003. \\ \fi
\begin{problem}

What is wrong with the following equation:

\[
\int_{\frac{5}{6} \, \pi}^{\frac{7}{6} \, \pi} {\csc\left(x\right)^{2}}\;dx = {-\frac{1}{\tan\left(x\right)}}\Bigg\vert_{\frac{5}{6} \, \pi}^{\frac{7}{6} \, \pi} = {-2 \, \sqrt{3}}
\]

\input{Integral-Concept-0003.HELP.tex}

\begin{multipleChoice}
\choice{The antiderivative is incorrect.}
\choice[correct]{The integrand is not defined over the entire interval.}
\choice{The bounds are evaluated in the wrong order.}
\choice{Nothing is wrong.  The equation is correct, as is.}
\end{multipleChoice}

\end{problem}}%}

\latexProblemContent{
\ifVerboseLocation This is Integration Concept Question 0003. \\ \fi
\begin{problem}

What is wrong with the following equation:

\[
\int_{\frac{2}{3} \, \pi}^{\frac{7}{6} \, \pi} {5 \, \csc\left(x\right)^{2}}\;dx = {-\frac{5}{\tan\left(x\right)}}\Bigg\vert_{\frac{2}{3} \, \pi}^{\frac{7}{6} \, \pi} = {-\frac{20}{3} \, \sqrt{3}}
\]

\input{Integral-Concept-0003.HELP.tex}

\begin{multipleChoice}
\choice{The antiderivative is incorrect.}
\choice[correct]{The integrand is not defined over the entire interval.}
\choice{The bounds are evaluated in the wrong order.}
\choice{Nothing is wrong.  The equation is correct, as is.}
\end{multipleChoice}

\end{problem}}%}

\latexProblemContent{
\ifVerboseLocation This is Integration Concept Question 0003. \\ \fi
\begin{problem}

What is wrong with the following equation:

\[
\int_{\frac{1}{2} \, \pi}^{\frac{4}{3} \, \pi} {-9 \, \cot\left(x\right) \csc\left(x\right)}\;dx = {\frac{9}{\sin\left(x\right)}}\Bigg\vert_{\frac{1}{2} \, \pi}^{\frac{4}{3} \, \pi} = {-6 \, \sqrt{3} - 9}
\]

\input{Integral-Concept-0003.HELP.tex}

\begin{multipleChoice}
\choice{The antiderivative is incorrect.}
\choice[correct]{The integrand is not defined over the entire interval.}
\choice{The bounds are evaluated in the wrong order.}
\choice{Nothing is wrong.  The equation is correct, as is.}
\end{multipleChoice}

\end{problem}}%}

\latexProblemContent{
\ifVerboseLocation This is Integration Concept Question 0003. \\ \fi
\begin{problem}

What is wrong with the following equation:

\[
\int_{\frac{3}{4} \, \pi}^{\frac{11}{6} \, \pi} {3 \, \csc\left(x\right)^{2}}\;dx = {-\frac{3}{\tan\left(x\right)}}\Bigg\vert_{\frac{3}{4} \, \pi}^{\frac{11}{6} \, \pi} = {3 \, \sqrt{3} - 3}
\]

\input{Integral-Concept-0003.HELP.tex}

\begin{multipleChoice}
\choice{The antiderivative is incorrect.}
\choice[correct]{The integrand is not defined over the entire interval.}
\choice{The bounds are evaluated in the wrong order.}
\choice{Nothing is wrong.  The equation is correct, as is.}
\end{multipleChoice}

\end{problem}}%}

\latexProblemContent{
\ifVerboseLocation This is Integration Concept Question 0003. \\ \fi
\begin{problem}

What is wrong with the following equation:

\[
\int_{\frac{3}{4} \, \pi}^{\frac{7}{6} \, \pi} {-9 \, \cot\left(x\right) \csc\left(x\right)}\;dx = {\frac{9}{\sin\left(x\right)}}\Bigg\vert_{\frac{3}{4} \, \pi}^{\frac{7}{6} \, \pi} = {-9 \, \sqrt{2} - 18}
\]

\input{Integral-Concept-0003.HELP.tex}

\begin{multipleChoice}
\choice{The antiderivative is incorrect.}
\choice[correct]{The integrand is not defined over the entire interval.}
\choice{The bounds are evaluated in the wrong order.}
\choice{Nothing is wrong.  The equation is correct, as is.}
\end{multipleChoice}

\end{problem}}%}

\latexProblemContent{
\ifVerboseLocation This is Integration Concept Question 0003. \\ \fi
\begin{problem}

What is wrong with the following equation:

\[
\int_{\frac{1}{6} \, \pi}^{\frac{11}{6} \, \pi} {-\csc\left(x\right)^{2}}\;dx = {\frac{1}{\tan\left(x\right)}}\Bigg\vert_{\frac{1}{6} \, \pi}^{\frac{11}{6} \, \pi} = {-2 \, \sqrt{3}}
\]

\input{Integral-Concept-0003.HELP.tex}

\begin{multipleChoice}
\choice{The antiderivative is incorrect.}
\choice[correct]{The integrand is not defined over the entire interval.}
\choice{The bounds are evaluated in the wrong order.}
\choice{Nothing is wrong.  The equation is correct, as is.}
\end{multipleChoice}

\end{problem}}%}

\latexProblemContent{
\ifVerboseLocation This is Integration Concept Question 0003. \\ \fi
\begin{problem}

What is wrong with the following equation:

\[
\int_{\frac{1}{6} \, \pi}^{\frac{5}{4} \, \pi} {-4 \, \csc\left(x\right)^{2}}\;dx = {\frac{4}{\tan\left(x\right)}}\Bigg\vert_{\frac{1}{6} \, \pi}^{\frac{5}{4} \, \pi} = {-4 \, \sqrt{3} + 4}
\]

\input{Integral-Concept-0003.HELP.tex}

\begin{multipleChoice}
\choice{The antiderivative is incorrect.}
\choice[correct]{The integrand is not defined over the entire interval.}
\choice{The bounds are evaluated in the wrong order.}
\choice{Nothing is wrong.  The equation is correct, as is.}
\end{multipleChoice}

\end{problem}}%}

\latexProblemContent{
\ifVerboseLocation This is Integration Concept Question 0003. \\ \fi
\begin{problem}

What is wrong with the following equation:

\[
\int_{\frac{5}{6} \, \pi}^{\frac{5}{4} \, \pi} {-6 \, \cot\left(x\right) \csc\left(x\right)}\;dx = {\frac{6}{\sin\left(x\right)}}\Bigg\vert_{\frac{5}{6} \, \pi}^{\frac{5}{4} \, \pi} = {-6 \, \sqrt{2} - 12}
\]

\input{Integral-Concept-0003.HELP.tex}

\begin{multipleChoice}
\choice{The antiderivative is incorrect.}
\choice[correct]{The integrand is not defined over the entire interval.}
\choice{The bounds are evaluated in the wrong order.}
\choice{Nothing is wrong.  The equation is correct, as is.}
\end{multipleChoice}

\end{problem}}%}

\latexProblemContent{
\ifVerboseLocation This is Integration Concept Question 0003. \\ \fi
\begin{problem}

What is wrong with the following equation:

\[
\int_{\frac{1}{3} \, \pi}^{\frac{5}{4} \, \pi} {-6 \, \csc\left(x\right)^{2}}\;dx = {\frac{6}{\tan\left(x\right)}}\Bigg\vert_{\frac{1}{3} \, \pi}^{\frac{5}{4} \, \pi} = {-2 \, \sqrt{3} + 6}
\]

\input{Integral-Concept-0003.HELP.tex}

\begin{multipleChoice}
\choice{The antiderivative is incorrect.}
\choice[correct]{The integrand is not defined over the entire interval.}
\choice{The bounds are evaluated in the wrong order.}
\choice{Nothing is wrong.  The equation is correct, as is.}
\end{multipleChoice}

\end{problem}}%}

\latexProblemContent{
\ifVerboseLocation This is Integration Concept Question 0003. \\ \fi
\begin{problem}

What is wrong with the following equation:

\[
\int_{\frac{5}{6} \, \pi}^{\frac{3}{2} \, \pi} {-8 \, \cot\left(x\right) \csc\left(x\right)}\;dx = {\frac{8}{\sin\left(x\right)}}\Bigg\vert_{\frac{5}{6} \, \pi}^{\frac{3}{2} \, \pi} = {-24}
\]

\input{Integral-Concept-0003.HELP.tex}

\begin{multipleChoice}
\choice{The antiderivative is incorrect.}
\choice[correct]{The integrand is not defined over the entire interval.}
\choice{The bounds are evaluated in the wrong order.}
\choice{Nothing is wrong.  The equation is correct, as is.}
\end{multipleChoice}

\end{problem}}%}

\latexProblemContent{
\ifVerboseLocation This is Integration Concept Question 0003. \\ \fi
\begin{problem}

What is wrong with the following equation:

\[
\int_{\frac{3}{4} \, \pi}^{\frac{7}{6} \, \pi} {8 \, \cot\left(x\right) \csc\left(x\right)}\;dx = {-\frac{8}{\sin\left(x\right)}}\Bigg\vert_{\frac{3}{4} \, \pi}^{\frac{7}{6} \, \pi} = {8 \, \sqrt{2} + 16}
\]

\input{Integral-Concept-0003.HELP.tex}

\begin{multipleChoice}
\choice{The antiderivative is incorrect.}
\choice[correct]{The integrand is not defined over the entire interval.}
\choice{The bounds are evaluated in the wrong order.}
\choice{Nothing is wrong.  The equation is correct, as is.}
\end{multipleChoice}

\end{problem}}%}

\latexProblemContent{
\ifVerboseLocation This is Integration Concept Question 0003. \\ \fi
\begin{problem}

What is wrong with the following equation:

\[
\int_{\frac{2}{3} \, \pi}^{\frac{5}{3} \, \pi} {-8 \, \cot\left(x\right) \csc\left(x\right)}\;dx = {\frac{8}{\sin\left(x\right)}}\Bigg\vert_{\frac{2}{3} \, \pi}^{\frac{5}{3} \, \pi} = {-\frac{32}{3} \, \sqrt{3}}
\]

\input{Integral-Concept-0003.HELP.tex}

\begin{multipleChoice}
\choice{The antiderivative is incorrect.}
\choice[correct]{The integrand is not defined over the entire interval.}
\choice{The bounds are evaluated in the wrong order.}
\choice{Nothing is wrong.  The equation is correct, as is.}
\end{multipleChoice}

\end{problem}}%}

\latexProblemContent{
\ifVerboseLocation This is Integration Concept Question 0003. \\ \fi
\begin{problem}

What is wrong with the following equation:

\[
\int_{\frac{1}{4} \, \pi}^{\frac{4}{3} \, \pi} {9 \, \cot\left(x\right) \csc\left(x\right)}\;dx = {-\frac{9}{\sin\left(x\right)}}\Bigg\vert_{\frac{1}{4} \, \pi}^{\frac{4}{3} \, \pi} = {6 \, \sqrt{3} + 9 \, \sqrt{2}}
\]

\input{Integral-Concept-0003.HELP.tex}

\begin{multipleChoice}
\choice{The antiderivative is incorrect.}
\choice[correct]{The integrand is not defined over the entire interval.}
\choice{The bounds are evaluated in the wrong order.}
\choice{Nothing is wrong.  The equation is correct, as is.}
\end{multipleChoice}

\end{problem}}%}

\latexProblemContent{
\ifVerboseLocation This is Integration Concept Question 0003. \\ \fi
\begin{problem}

What is wrong with the following equation:

\[
\int_{\frac{1}{4} \, \pi}^{\frac{11}{6} \, \pi} {\cot\left(x\right) \csc\left(x\right)}\;dx = {-\frac{1}{\sin\left(x\right)}}\Bigg\vert_{\frac{1}{4} \, \pi}^{\frac{11}{6} \, \pi} = {\sqrt{2} + 2}
\]

\input{Integral-Concept-0003.HELP.tex}

\begin{multipleChoice}
\choice{The antiderivative is incorrect.}
\choice[correct]{The integrand is not defined over the entire interval.}
\choice{The bounds are evaluated in the wrong order.}
\choice{Nothing is wrong.  The equation is correct, as is.}
\end{multipleChoice}

\end{problem}}%}

\latexProblemContent{
\ifVerboseLocation This is Integration Concept Question 0003. \\ \fi
\begin{problem}

What is wrong with the following equation:

\[
\int_{\frac{1}{6} \, \pi}^{\frac{11}{6} \, \pi} {4 \, \csc\left(x\right)^{2}}\;dx = {-\frac{4}{\tan\left(x\right)}}\Bigg\vert_{\frac{1}{6} \, \pi}^{\frac{11}{6} \, \pi} = {8 \, \sqrt{3}}
\]

\input{Integral-Concept-0003.HELP.tex}

\begin{multipleChoice}
\choice{The antiderivative is incorrect.}
\choice[correct]{The integrand is not defined over the entire interval.}
\choice{The bounds are evaluated in the wrong order.}
\choice{Nothing is wrong.  The equation is correct, as is.}
\end{multipleChoice}

\end{problem}}%}

\latexProblemContent{
\ifVerboseLocation This is Integration Concept Question 0003. \\ \fi
\begin{problem}

What is wrong with the following equation:

\[
\int_{\frac{1}{3} \, \pi}^{\frac{5}{3} \, \pi} {-7 \, \cot\left(x\right) \csc\left(x\right)}\;dx = {\frac{7}{\sin\left(x\right)}}\Bigg\vert_{\frac{1}{3} \, \pi}^{\frac{5}{3} \, \pi} = {-\frac{28}{3} \, \sqrt{3}}
\]

\input{Integral-Concept-0003.HELP.tex}

\begin{multipleChoice}
\choice{The antiderivative is incorrect.}
\choice[correct]{The integrand is not defined over the entire interval.}
\choice{The bounds are evaluated in the wrong order.}
\choice{Nothing is wrong.  The equation is correct, as is.}
\end{multipleChoice}

\end{problem}}%}

\latexProblemContent{
\ifVerboseLocation This is Integration Concept Question 0003. \\ \fi
\begin{problem}

What is wrong with the following equation:

\[
\int_{\frac{1}{2} \, \pi}^{\frac{4}{3} \, \pi} {-6 \, \cot\left(x\right) \csc\left(x\right)}\;dx = {\frac{6}{\sin\left(x\right)}}\Bigg\vert_{\frac{1}{2} \, \pi}^{\frac{4}{3} \, \pi} = {-4 \, \sqrt{3} - 6}
\]

\input{Integral-Concept-0003.HELP.tex}

\begin{multipleChoice}
\choice{The antiderivative is incorrect.}
\choice[correct]{The integrand is not defined over the entire interval.}
\choice{The bounds are evaluated in the wrong order.}
\choice{Nothing is wrong.  The equation is correct, as is.}
\end{multipleChoice}

\end{problem}}%}

\latexProblemContent{
\ifVerboseLocation This is Integration Concept Question 0003. \\ \fi
\begin{problem}

What is wrong with the following equation:

\[
\int_{\frac{1}{3} \, \pi}^{\frac{3}{2} \, \pi} {-\cot\left(x\right) \csc\left(x\right)}\;dx = {\frac{1}{\sin\left(x\right)}}\Bigg\vert_{\frac{1}{3} \, \pi}^{\frac{3}{2} \, \pi} = {-\frac{2}{3} \, \sqrt{3} - 1}
\]

\input{Integral-Concept-0003.HELP.tex}

\begin{multipleChoice}
\choice{The antiderivative is incorrect.}
\choice[correct]{The integrand is not defined over the entire interval.}
\choice{The bounds are evaluated in the wrong order.}
\choice{Nothing is wrong.  The equation is correct, as is.}
\end{multipleChoice}

\end{problem}}%}

\latexProblemContent{
\ifVerboseLocation This is Integration Concept Question 0003. \\ \fi
\begin{problem}

What is wrong with the following equation:

\[
\int_{\frac{3}{4} \, \pi}^{\frac{7}{6} \, \pi} {4 \, \cot\left(x\right) \csc\left(x\right)}\;dx = {-\frac{4}{\sin\left(x\right)}}\Bigg\vert_{\frac{3}{4} \, \pi}^{\frac{7}{6} \, \pi} = {4 \, \sqrt{2} + 8}
\]

\input{Integral-Concept-0003.HELP.tex}

\begin{multipleChoice}
\choice{The antiderivative is incorrect.}
\choice[correct]{The integrand is not defined over the entire interval.}
\choice{The bounds are evaluated in the wrong order.}
\choice{Nothing is wrong.  The equation is correct, as is.}
\end{multipleChoice}

\end{problem}}%}

\latexProblemContent{
\ifVerboseLocation This is Integration Concept Question 0003. \\ \fi
\begin{problem}

What is wrong with the following equation:

\[
\int_{\frac{1}{2} \, \pi}^{\frac{5}{3} \, \pi} {-7 \, \csc\left(x\right)^{2}}\;dx = {\frac{7}{\tan\left(x\right)}}\Bigg\vert_{\frac{1}{2} \, \pi}^{\frac{5}{3} \, \pi} = {-\frac{7}{3} \, \sqrt{3}}
\]

\input{Integral-Concept-0003.HELP.tex}

\begin{multipleChoice}
\choice{The antiderivative is incorrect.}
\choice[correct]{The integrand is not defined over the entire interval.}
\choice{The bounds are evaluated in the wrong order.}
\choice{Nothing is wrong.  The equation is correct, as is.}
\end{multipleChoice}

\end{problem}}%}

\latexProblemContent{
\ifVerboseLocation This is Integration Concept Question 0003. \\ \fi
\begin{problem}

What is wrong with the following equation:

\[
\int_{\frac{3}{4} \, \pi}^{\frac{5}{3} \, \pi} {-4 \, \cot\left(x\right) \csc\left(x\right)}\;dx = {\frac{4}{\sin\left(x\right)}}\Bigg\vert_{\frac{3}{4} \, \pi}^{\frac{5}{3} \, \pi} = {-\frac{8}{3} \, \sqrt{3} - 4 \, \sqrt{2}}
\]

\input{Integral-Concept-0003.HELP.tex}

\begin{multipleChoice}
\choice{The antiderivative is incorrect.}
\choice[correct]{The integrand is not defined over the entire interval.}
\choice{The bounds are evaluated in the wrong order.}
\choice{Nothing is wrong.  The equation is correct, as is.}
\end{multipleChoice}

\end{problem}}%}

\latexProblemContent{
\ifVerboseLocation This is Integration Concept Question 0003. \\ \fi
\begin{problem}

What is wrong with the following equation:

\[
\int_{\frac{3}{4} \, \pi}^{\frac{7}{4} \, \pi} {7 \, \csc\left(x\right)^{2}}\;dx = {-\frac{7}{\tan\left(x\right)}}\Bigg\vert_{\frac{3}{4} \, \pi}^{\frac{7}{4} \, \pi} = {0}
\]

\input{Integral-Concept-0003.HELP.tex}

\begin{multipleChoice}
\choice{The antiderivative is incorrect.}
\choice[correct]{The integrand is not defined over the entire interval.}
\choice{The bounds are evaluated in the wrong order.}
\choice{Nothing is wrong.  The equation is correct, as is.}
\end{multipleChoice}

\end{problem}}%}

\latexProblemContent{
\ifVerboseLocation This is Integration Concept Question 0003. \\ \fi
\begin{problem}

What is wrong with the following equation:

\[
\int_{\frac{2}{3} \, \pi}^{\frac{5}{4} \, \pi} {8 \, \cot\left(x\right) \csc\left(x\right)}\;dx = {-\frac{8}{\sin\left(x\right)}}\Bigg\vert_{\frac{2}{3} \, \pi}^{\frac{5}{4} \, \pi} = {\frac{16}{3} \, \sqrt{3} + 8 \, \sqrt{2}}
\]

\input{Integral-Concept-0003.HELP.tex}

\begin{multipleChoice}
\choice{The antiderivative is incorrect.}
\choice[correct]{The integrand is not defined over the entire interval.}
\choice{The bounds are evaluated in the wrong order.}
\choice{Nothing is wrong.  The equation is correct, as is.}
\end{multipleChoice}

\end{problem}}%}

\latexProblemContent{
\ifVerboseLocation This is Integration Concept Question 0003. \\ \fi
\begin{problem}

What is wrong with the following equation:

\[
\int_{\frac{5}{6} \, \pi}^{\frac{7}{4} \, \pi} {-8 \, \cot\left(x\right) \csc\left(x\right)}\;dx = {\frac{8}{\sin\left(x\right)}}\Bigg\vert_{\frac{5}{6} \, \pi}^{\frac{7}{4} \, \pi} = {-8 \, \sqrt{2} - 16}
\]

\input{Integral-Concept-0003.HELP.tex}

\begin{multipleChoice}
\choice{The antiderivative is incorrect.}
\choice[correct]{The integrand is not defined over the entire interval.}
\choice{The bounds are evaluated in the wrong order.}
\choice{Nothing is wrong.  The equation is correct, as is.}
\end{multipleChoice}

\end{problem}}%}

\latexProblemContent{
\ifVerboseLocation This is Integration Concept Question 0003. \\ \fi
\begin{problem}

What is wrong with the following equation:

\[
\int_{\frac{3}{4} \, \pi}^{\frac{11}{6} \, \pi} {8 \, \csc\left(x\right)^{2}}\;dx = {-\frac{8}{\tan\left(x\right)}}\Bigg\vert_{\frac{3}{4} \, \pi}^{\frac{11}{6} \, \pi} = {8 \, \sqrt{3} - 8}
\]

\input{Integral-Concept-0003.HELP.tex}

\begin{multipleChoice}
\choice{The antiderivative is incorrect.}
\choice[correct]{The integrand is not defined over the entire interval.}
\choice{The bounds are evaluated in the wrong order.}
\choice{Nothing is wrong.  The equation is correct, as is.}
\end{multipleChoice}

\end{problem}}%}

\latexProblemContent{
\ifVerboseLocation This is Integration Concept Question 0003. \\ \fi
\begin{problem}

What is wrong with the following equation:

\[
\int_{\frac{1}{6} \, \pi}^{\frac{3}{2} \, \pi} {-4 \, \cot\left(x\right) \csc\left(x\right)}\;dx = {\frac{4}{\sin\left(x\right)}}\Bigg\vert_{\frac{1}{6} \, \pi}^{\frac{3}{2} \, \pi} = {-12}
\]

\input{Integral-Concept-0003.HELP.tex}

\begin{multipleChoice}
\choice{The antiderivative is incorrect.}
\choice[correct]{The integrand is not defined over the entire interval.}
\choice{The bounds are evaluated in the wrong order.}
\choice{Nothing is wrong.  The equation is correct, as is.}
\end{multipleChoice}

\end{problem}}%}

\latexProblemContent{
\ifVerboseLocation This is Integration Concept Question 0003. \\ \fi
\begin{problem}

What is wrong with the following equation:

\[
\int_{\frac{2}{3} \, \pi}^{\frac{7}{6} \, \pi} {9 \, \csc\left(x\right)^{2}}\;dx = {-\frac{9}{\tan\left(x\right)}}\Bigg\vert_{\frac{2}{3} \, \pi}^{\frac{7}{6} \, \pi} = {-12 \, \sqrt{3}}
\]

\input{Integral-Concept-0003.HELP.tex}

\begin{multipleChoice}
\choice{The antiderivative is incorrect.}
\choice[correct]{The integrand is not defined over the entire interval.}
\choice{The bounds are evaluated in the wrong order.}
\choice{Nothing is wrong.  The equation is correct, as is.}
\end{multipleChoice}

\end{problem}}%}

\latexProblemContent{
\ifVerboseLocation This is Integration Concept Question 0003. \\ \fi
\begin{problem}

What is wrong with the following equation:

\[
\int_{\frac{5}{6} \, \pi}^{\frac{5}{3} \, \pi} {9 \, \csc\left(x\right)^{2}}\;dx = {-\frac{9}{\tan\left(x\right)}}\Bigg\vert_{\frac{5}{6} \, \pi}^{\frac{5}{3} \, \pi} = {-6 \, \sqrt{3}}
\]

\input{Integral-Concept-0003.HELP.tex}

\begin{multipleChoice}
\choice{The antiderivative is incorrect.}
\choice[correct]{The integrand is not defined over the entire interval.}
\choice{The bounds are evaluated in the wrong order.}
\choice{Nothing is wrong.  The equation is correct, as is.}
\end{multipleChoice}

\end{problem}}%}

\latexProblemContent{
\ifVerboseLocation This is Integration Concept Question 0003. \\ \fi
\begin{problem}

What is wrong with the following equation:

\[
\int_{\frac{5}{6} \, \pi}^{\frac{5}{4} \, \pi} {-10 \, \cot\left(x\right) \csc\left(x\right)}\;dx = {\frac{10}{\sin\left(x\right)}}\Bigg\vert_{\frac{5}{6} \, \pi}^{\frac{5}{4} \, \pi} = {-10 \, \sqrt{2} - 20}
\]

\input{Integral-Concept-0003.HELP.tex}

\begin{multipleChoice}
\choice{The antiderivative is incorrect.}
\choice[correct]{The integrand is not defined over the entire interval.}
\choice{The bounds are evaluated in the wrong order.}
\choice{Nothing is wrong.  The equation is correct, as is.}
\end{multipleChoice}

\end{problem}}%}

\latexProblemContent{
\ifVerboseLocation This is Integration Concept Question 0003. \\ \fi
\begin{problem}

What is wrong with the following equation:

\[
\int_{\frac{1}{6} \, \pi}^{\frac{7}{6} \, \pi} {-3 \, \csc\left(x\right)^{2}}\;dx = {\frac{3}{\tan\left(x\right)}}\Bigg\vert_{\frac{1}{6} \, \pi}^{\frac{7}{6} \, \pi} = {0}
\]

\input{Integral-Concept-0003.HELP.tex}

\begin{multipleChoice}
\choice{The antiderivative is incorrect.}
\choice[correct]{The integrand is not defined over the entire interval.}
\choice{The bounds are evaluated in the wrong order.}
\choice{Nothing is wrong.  The equation is correct, as is.}
\end{multipleChoice}

\end{problem}}%}

\latexProblemContent{
\ifVerboseLocation This is Integration Concept Question 0003. \\ \fi
\begin{problem}

What is wrong with the following equation:

\[
\int_{\frac{2}{3} \, \pi}^{\frac{7}{4} \, \pi} {-7 \, \cot\left(x\right) \csc\left(x\right)}\;dx = {\frac{7}{\sin\left(x\right)}}\Bigg\vert_{\frac{2}{3} \, \pi}^{\frac{7}{4} \, \pi} = {-\frac{14}{3} \, \sqrt{3} - 7 \, \sqrt{2}}
\]

\input{Integral-Concept-0003.HELP.tex}

\begin{multipleChoice}
\choice{The antiderivative is incorrect.}
\choice[correct]{The integrand is not defined over the entire interval.}
\choice{The bounds are evaluated in the wrong order.}
\choice{Nothing is wrong.  The equation is correct, as is.}
\end{multipleChoice}

\end{problem}}%}

\latexProblemContent{
\ifVerboseLocation This is Integration Concept Question 0003. \\ \fi
\begin{problem}

What is wrong with the following equation:

\[
\int_{\frac{1}{2} \, \pi}^{\frac{7}{4} \, \pi} {8 \, \cot\left(x\right) \csc\left(x\right)}\;dx = {-\frac{8}{\sin\left(x\right)}}\Bigg\vert_{\frac{1}{2} \, \pi}^{\frac{7}{4} \, \pi} = {8 \, \sqrt{2} + 8}
\]

\input{Integral-Concept-0003.HELP.tex}

\begin{multipleChoice}
\choice{The antiderivative is incorrect.}
\choice[correct]{The integrand is not defined over the entire interval.}
\choice{The bounds are evaluated in the wrong order.}
\choice{Nothing is wrong.  The equation is correct, as is.}
\end{multipleChoice}

\end{problem}}%}

\latexProblemContent{
\ifVerboseLocation This is Integration Concept Question 0003. \\ \fi
\begin{problem}

What is wrong with the following equation:

\[
\int_{\frac{1}{4} \, \pi}^{\frac{7}{6} \, \pi} {4 \, \csc\left(x\right)^{2}}\;dx = {-\frac{4}{\tan\left(x\right)}}\Bigg\vert_{\frac{1}{4} \, \pi}^{\frac{7}{6} \, \pi} = {-4 \, \sqrt{3} + 4}
\]

\input{Integral-Concept-0003.HELP.tex}

\begin{multipleChoice}
\choice{The antiderivative is incorrect.}
\choice[correct]{The integrand is not defined over the entire interval.}
\choice{The bounds are evaluated in the wrong order.}
\choice{Nothing is wrong.  The equation is correct, as is.}
\end{multipleChoice}

\end{problem}}%}

\latexProblemContent{
\ifVerboseLocation This is Integration Concept Question 0003. \\ \fi
\begin{problem}

What is wrong with the following equation:

\[
\int_{\frac{2}{3} \, \pi}^{\frac{5}{3} \, \pi} {7 \, \cot\left(x\right) \csc\left(x\right)}\;dx = {-\frac{7}{\sin\left(x\right)}}\Bigg\vert_{\frac{2}{3} \, \pi}^{\frac{5}{3} \, \pi} = {\frac{28}{3} \, \sqrt{3}}
\]

\input{Integral-Concept-0003.HELP.tex}

\begin{multipleChoice}
\choice{The antiderivative is incorrect.}
\choice[correct]{The integrand is not defined over the entire interval.}
\choice{The bounds are evaluated in the wrong order.}
\choice{Nothing is wrong.  The equation is correct, as is.}
\end{multipleChoice}

\end{problem}}%}

\latexProblemContent{
\ifVerboseLocation This is Integration Concept Question 0003. \\ \fi
\begin{problem}

What is wrong with the following equation:

\[
\int_{\frac{1}{3} \, \pi}^{\frac{7}{4} \, \pi} {-9 \, \cot\left(x\right) \csc\left(x\right)}\;dx = {\frac{9}{\sin\left(x\right)}}\Bigg\vert_{\frac{1}{3} \, \pi}^{\frac{7}{4} \, \pi} = {-6 \, \sqrt{3} - 9 \, \sqrt{2}}
\]

\input{Integral-Concept-0003.HELP.tex}

\begin{multipleChoice}
\choice{The antiderivative is incorrect.}
\choice[correct]{The integrand is not defined over the entire interval.}
\choice{The bounds are evaluated in the wrong order.}
\choice{Nothing is wrong.  The equation is correct, as is.}
\end{multipleChoice}

\end{problem}}%}

\latexProblemContent{
\ifVerboseLocation This is Integration Concept Question 0003. \\ \fi
\begin{problem}

What is wrong with the following equation:

\[
\int_{\frac{1}{3} \, \pi}^{\frac{11}{6} \, \pi} {-9 \, \cot\left(x\right) \csc\left(x\right)}\;dx = {\frac{9}{\sin\left(x\right)}}\Bigg\vert_{\frac{1}{3} \, \pi}^{\frac{11}{6} \, \pi} = {-6 \, \sqrt{3} - 18}
\]

\input{Integral-Concept-0003.HELP.tex}

\begin{multipleChoice}
\choice{The antiderivative is incorrect.}
\choice[correct]{The integrand is not defined over the entire interval.}
\choice{The bounds are evaluated in the wrong order.}
\choice{Nothing is wrong.  The equation is correct, as is.}
\end{multipleChoice}

\end{problem}}%}

\latexProblemContent{
\ifVerboseLocation This is Integration Concept Question 0003. \\ \fi
\begin{problem}

What is wrong with the following equation:

\[
\int_{\frac{2}{3} \, \pi}^{\frac{5}{3} \, \pi} {-10 \, \csc\left(x\right)^{2}}\;dx = {\frac{10}{\tan\left(x\right)}}\Bigg\vert_{\frac{2}{3} \, \pi}^{\frac{5}{3} \, \pi} = {0}
\]

\input{Integral-Concept-0003.HELP.tex}

\begin{multipleChoice}
\choice{The antiderivative is incorrect.}
\choice[correct]{The integrand is not defined over the entire interval.}
\choice{The bounds are evaluated in the wrong order.}
\choice{Nothing is wrong.  The equation is correct, as is.}
\end{multipleChoice}

\end{problem}}%}

\latexProblemContent{
\ifVerboseLocation This is Integration Concept Question 0003. \\ \fi
\begin{problem}

What is wrong with the following equation:

\[
\int_{\frac{3}{4} \, \pi}^{\frac{3}{2} \, \pi} {-6 \, \cot\left(x\right) \csc\left(x\right)}\;dx = {\frac{6}{\sin\left(x\right)}}\Bigg\vert_{\frac{3}{4} \, \pi}^{\frac{3}{2} \, \pi} = {-6 \, \sqrt{2} - 6}
\]

\input{Integral-Concept-0003.HELP.tex}

\begin{multipleChoice}
\choice{The antiderivative is incorrect.}
\choice[correct]{The integrand is not defined over the entire interval.}
\choice{The bounds are evaluated in the wrong order.}
\choice{Nothing is wrong.  The equation is correct, as is.}
\end{multipleChoice}

\end{problem}}%}

\latexProblemContent{
\ifVerboseLocation This is Integration Concept Question 0003. \\ \fi
\begin{problem}

What is wrong with the following equation:

\[
\int_{\frac{2}{3} \, \pi}^{\frac{5}{4} \, \pi} {-8 \, \csc\left(x\right)^{2}}\;dx = {\frac{8}{\tan\left(x\right)}}\Bigg\vert_{\frac{2}{3} \, \pi}^{\frac{5}{4} \, \pi} = {\frac{8}{3} \, \sqrt{3} + 8}
\]

\input{Integral-Concept-0003.HELP.tex}

\begin{multipleChoice}
\choice{The antiderivative is incorrect.}
\choice[correct]{The integrand is not defined over the entire interval.}
\choice{The bounds are evaluated in the wrong order.}
\choice{Nothing is wrong.  The equation is correct, as is.}
\end{multipleChoice}

\end{problem}}%}

\latexProblemContent{
\ifVerboseLocation This is Integration Concept Question 0003. \\ \fi
\begin{problem}

What is wrong with the following equation:

\[
\int_{\frac{1}{3} \, \pi}^{\frac{4}{3} \, \pi} {10 \, \csc\left(x\right)^{2}}\;dx = {-\frac{10}{\tan\left(x\right)}}\Bigg\vert_{\frac{1}{3} \, \pi}^{\frac{4}{3} \, \pi} = {0}
\]

\input{Integral-Concept-0003.HELP.tex}

\begin{multipleChoice}
\choice{The antiderivative is incorrect.}
\choice[correct]{The integrand is not defined over the entire interval.}
\choice{The bounds are evaluated in the wrong order.}
\choice{Nothing is wrong.  The equation is correct, as is.}
\end{multipleChoice}

\end{problem}}%}

\latexProblemContent{
\ifVerboseLocation This is Integration Concept Question 0003. \\ \fi
\begin{problem}

What is wrong with the following equation:

\[
\int_{\frac{1}{4} \, \pi}^{\frac{7}{6} \, \pi} {-3 \, \cot\left(x\right) \csc\left(x\right)}\;dx = {\frac{3}{\sin\left(x\right)}}\Bigg\vert_{\frac{1}{4} \, \pi}^{\frac{7}{6} \, \pi} = {-3 \, \sqrt{2} - 6}
\]

\input{Integral-Concept-0003.HELP.tex}

\begin{multipleChoice}
\choice{The antiderivative is incorrect.}
\choice[correct]{The integrand is not defined over the entire interval.}
\choice{The bounds are evaluated in the wrong order.}
\choice{Nothing is wrong.  The equation is correct, as is.}
\end{multipleChoice}

\end{problem}}%}

\latexProblemContent{
\ifVerboseLocation This is Integration Concept Question 0003. \\ \fi
\begin{problem}

What is wrong with the following equation:

\[
\int_{\frac{3}{4} \, \pi}^{\frac{3}{2} \, \pi} {7 \, \cot\left(x\right) \csc\left(x\right)}\;dx = {-\frac{7}{\sin\left(x\right)}}\Bigg\vert_{\frac{3}{4} \, \pi}^{\frac{3}{2} \, \pi} = {7 \, \sqrt{2} + 7}
\]

\input{Integral-Concept-0003.HELP.tex}

\begin{multipleChoice}
\choice{The antiderivative is incorrect.}
\choice[correct]{The integrand is not defined over the entire interval.}
\choice{The bounds are evaluated in the wrong order.}
\choice{Nothing is wrong.  The equation is correct, as is.}
\end{multipleChoice}

\end{problem}}%}

\latexProblemContent{
\ifVerboseLocation This is Integration Concept Question 0003. \\ \fi
\begin{problem}

What is wrong with the following equation:

\[
\int_{\frac{2}{3} \, \pi}^{\frac{7}{4} \, \pi} {10 \, \cot\left(x\right) \csc\left(x\right)}\;dx = {-\frac{10}{\sin\left(x\right)}}\Bigg\vert_{\frac{2}{3} \, \pi}^{\frac{7}{4} \, \pi} = {\frac{20}{3} \, \sqrt{3} + 10 \, \sqrt{2}}
\]

\input{Integral-Concept-0003.HELP.tex}

\begin{multipleChoice}
\choice{The antiderivative is incorrect.}
\choice[correct]{The integrand is not defined over the entire interval.}
\choice{The bounds are evaluated in the wrong order.}
\choice{Nothing is wrong.  The equation is correct, as is.}
\end{multipleChoice}

\end{problem}}%}

\latexProblemContent{
\ifVerboseLocation This is Integration Concept Question 0003. \\ \fi
\begin{problem}

What is wrong with the following equation:

\[
\int_{\frac{1}{2} \, \pi}^{\frac{7}{6} \, \pi} {-\cot\left(x\right) \csc\left(x\right)}\;dx = {\frac{1}{\sin\left(x\right)}}\Bigg\vert_{\frac{1}{2} \, \pi}^{\frac{7}{6} \, \pi} = {-3}
\]

\input{Integral-Concept-0003.HELP.tex}

\begin{multipleChoice}
\choice{The antiderivative is incorrect.}
\choice[correct]{The integrand is not defined over the entire interval.}
\choice{The bounds are evaluated in the wrong order.}
\choice{Nothing is wrong.  The equation is correct, as is.}
\end{multipleChoice}

\end{problem}}%}

\latexProblemContent{
\ifVerboseLocation This is Integration Concept Question 0003. \\ \fi
\begin{problem}

What is wrong with the following equation:

\[
\int_{\frac{2}{3} \, \pi}^{\frac{7}{4} \, \pi} {-4 \, \cot\left(x\right) \csc\left(x\right)}\;dx = {\frac{4}{\sin\left(x\right)}}\Bigg\vert_{\frac{2}{3} \, \pi}^{\frac{7}{4} \, \pi} = {-\frac{8}{3} \, \sqrt{3} - 4 \, \sqrt{2}}
\]

\input{Integral-Concept-0003.HELP.tex}

\begin{multipleChoice}
\choice{The antiderivative is incorrect.}
\choice[correct]{The integrand is not defined over the entire interval.}
\choice{The bounds are evaluated in the wrong order.}
\choice{Nothing is wrong.  The equation is correct, as is.}
\end{multipleChoice}

\end{problem}}%}

\latexProblemContent{
\ifVerboseLocation This is Integration Concept Question 0003. \\ \fi
\begin{problem}

What is wrong with the following equation:

\[
\int_{\frac{3}{4} \, \pi}^{\frac{5}{4} \, \pi} {-3 \, \csc\left(x\right)^{2}}\;dx = {\frac{3}{\tan\left(x\right)}}\Bigg\vert_{\frac{3}{4} \, \pi}^{\frac{5}{4} \, \pi} = {6}
\]

\input{Integral-Concept-0003.HELP.tex}

\begin{multipleChoice}
\choice{The antiderivative is incorrect.}
\choice[correct]{The integrand is not defined over the entire interval.}
\choice{The bounds are evaluated in the wrong order.}
\choice{Nothing is wrong.  The equation is correct, as is.}
\end{multipleChoice}

\end{problem}}%}

\latexProblemContent{
\ifVerboseLocation This is Integration Concept Question 0003. \\ \fi
\begin{problem}

What is wrong with the following equation:

\[
\int_{\frac{1}{4} \, \pi}^{\frac{5}{3} \, \pi} {7 \, \cot\left(x\right) \csc\left(x\right)}\;dx = {-\frac{7}{\sin\left(x\right)}}\Bigg\vert_{\frac{1}{4} \, \pi}^{\frac{5}{3} \, \pi} = {\frac{14}{3} \, \sqrt{3} + 7 \, \sqrt{2}}
\]

\input{Integral-Concept-0003.HELP.tex}

\begin{multipleChoice}
\choice{The antiderivative is incorrect.}
\choice[correct]{The integrand is not defined over the entire interval.}
\choice{The bounds are evaluated in the wrong order.}
\choice{Nothing is wrong.  The equation is correct, as is.}
\end{multipleChoice}

\end{problem}}%}

\latexProblemContent{
\ifVerboseLocation This is Integration Concept Question 0003. \\ \fi
\begin{problem}

What is wrong with the following equation:

\[
\int_{\frac{1}{6} \, \pi}^{\frac{7}{6} \, \pi} {-7 \, \cot\left(x\right) \csc\left(x\right)}\;dx = {\frac{7}{\sin\left(x\right)}}\Bigg\vert_{\frac{1}{6} \, \pi}^{\frac{7}{6} \, \pi} = {-28}
\]

\input{Integral-Concept-0003.HELP.tex}

\begin{multipleChoice}
\choice{The antiderivative is incorrect.}
\choice[correct]{The integrand is not defined over the entire interval.}
\choice{The bounds are evaluated in the wrong order.}
\choice{Nothing is wrong.  The equation is correct, as is.}
\end{multipleChoice}

\end{problem}}%}

\latexProblemContent{
\ifVerboseLocation This is Integration Concept Question 0003. \\ \fi
\begin{problem}

What is wrong with the following equation:

\[
\int_{\frac{1}{4} \, \pi}^{\frac{7}{6} \, \pi} {7 \, \csc\left(x\right)^{2}}\;dx = {-\frac{7}{\tan\left(x\right)}}\Bigg\vert_{\frac{1}{4} \, \pi}^{\frac{7}{6} \, \pi} = {-7 \, \sqrt{3} + 7}
\]

\input{Integral-Concept-0003.HELP.tex}

\begin{multipleChoice}
\choice{The antiderivative is incorrect.}
\choice[correct]{The integrand is not defined over the entire interval.}
\choice{The bounds are evaluated in the wrong order.}
\choice{Nothing is wrong.  The equation is correct, as is.}
\end{multipleChoice}

\end{problem}}%}

\latexProblemContent{
\ifVerboseLocation This is Integration Concept Question 0003. \\ \fi
\begin{problem}

What is wrong with the following equation:

\[
\int_{\frac{5}{6} \, \pi}^{\frac{3}{2} \, \pi} {6 \, \cot\left(x\right) \csc\left(x\right)}\;dx = {-\frac{6}{\sin\left(x\right)}}\Bigg\vert_{\frac{5}{6} \, \pi}^{\frac{3}{2} \, \pi} = {18}
\]

\input{Integral-Concept-0003.HELP.tex}

\begin{multipleChoice}
\choice{The antiderivative is incorrect.}
\choice[correct]{The integrand is not defined over the entire interval.}
\choice{The bounds are evaluated in the wrong order.}
\choice{Nothing is wrong.  The equation is correct, as is.}
\end{multipleChoice}

\end{problem}}%}

\latexProblemContent{
\ifVerboseLocation This is Integration Concept Question 0003. \\ \fi
\begin{problem}

What is wrong with the following equation:

\[
\int_{\frac{1}{4} \, \pi}^{\frac{3}{2} \, \pi} {-9 \, \csc\left(x\right)^{2}}\;dx = {\frac{9}{\tan\left(x\right)}}\Bigg\vert_{\frac{1}{4} \, \pi}^{\frac{3}{2} \, \pi} = {-9}
\]

\input{Integral-Concept-0003.HELP.tex}

\begin{multipleChoice}
\choice{The antiderivative is incorrect.}
\choice[correct]{The integrand is not defined over the entire interval.}
\choice{The bounds are evaluated in the wrong order.}
\choice{Nothing is wrong.  The equation is correct, as is.}
\end{multipleChoice}

\end{problem}}%}

\latexProblemContent{
\ifVerboseLocation This is Integration Concept Question 0003. \\ \fi
\begin{problem}

What is wrong with the following equation:

\[
\int_{\frac{1}{3} \, \pi}^{\frac{11}{6} \, \pi} {5 \, \csc\left(x\right)^{2}}\;dx = {-\frac{5}{\tan\left(x\right)}}\Bigg\vert_{\frac{1}{3} \, \pi}^{\frac{11}{6} \, \pi} = {\frac{20}{3} \, \sqrt{3}}
\]

\input{Integral-Concept-0003.HELP.tex}

\begin{multipleChoice}
\choice{The antiderivative is incorrect.}
\choice[correct]{The integrand is not defined over the entire interval.}
\choice{The bounds are evaluated in the wrong order.}
\choice{Nothing is wrong.  The equation is correct, as is.}
\end{multipleChoice}

\end{problem}}%}

\latexProblemContent{
\ifVerboseLocation This is Integration Concept Question 0003. \\ \fi
\begin{problem}

What is wrong with the following equation:

\[
\int_{\frac{5}{6} \, \pi}^{\frac{7}{6} \, \pi} {8 \, \csc\left(x\right)^{2}}\;dx = {-\frac{8}{\tan\left(x\right)}}\Bigg\vert_{\frac{5}{6} \, \pi}^{\frac{7}{6} \, \pi} = {-16 \, \sqrt{3}}
\]

\input{Integral-Concept-0003.HELP.tex}

\begin{multipleChoice}
\choice{The antiderivative is incorrect.}
\choice[correct]{The integrand is not defined over the entire interval.}
\choice{The bounds are evaluated in the wrong order.}
\choice{Nothing is wrong.  The equation is correct, as is.}
\end{multipleChoice}

\end{problem}}%}

\latexProblemContent{
\ifVerboseLocation This is Integration Concept Question 0003. \\ \fi
\begin{problem}

What is wrong with the following equation:

\[
\int_{\frac{1}{2} \, \pi}^{\frac{5}{3} \, \pi} {3 \, \cot\left(x\right) \csc\left(x\right)}\;dx = {-\frac{3}{\sin\left(x\right)}}\Bigg\vert_{\frac{1}{2} \, \pi}^{\frac{5}{3} \, \pi} = {2 \, \sqrt{3} + 3}
\]

\input{Integral-Concept-0003.HELP.tex}

\begin{multipleChoice}
\choice{The antiderivative is incorrect.}
\choice[correct]{The integrand is not defined over the entire interval.}
\choice{The bounds are evaluated in the wrong order.}
\choice{Nothing is wrong.  The equation is correct, as is.}
\end{multipleChoice}

\end{problem}}%}

\latexProblemContent{
\ifVerboseLocation This is Integration Concept Question 0003. \\ \fi
\begin{problem}

What is wrong with the following equation:

\[
\int_{\frac{1}{4} \, \pi}^{\frac{7}{6} \, \pi} {9 \, \csc\left(x\right)^{2}}\;dx = {-\frac{9}{\tan\left(x\right)}}\Bigg\vert_{\frac{1}{4} \, \pi}^{\frac{7}{6} \, \pi} = {-9 \, \sqrt{3} + 9}
\]

\input{Integral-Concept-0003.HELP.tex}

\begin{multipleChoice}
\choice{The antiderivative is incorrect.}
\choice[correct]{The integrand is not defined over the entire interval.}
\choice{The bounds are evaluated in the wrong order.}
\choice{Nothing is wrong.  The equation is correct, as is.}
\end{multipleChoice}

\end{problem}}%}

\latexProblemContent{
\ifVerboseLocation This is Integration Concept Question 0003. \\ \fi
\begin{problem}

What is wrong with the following equation:

\[
\int_{\frac{1}{6} \, \pi}^{\frac{7}{4} \, \pi} {-7 \, \cot\left(x\right) \csc\left(x\right)}\;dx = {\frac{7}{\sin\left(x\right)}}\Bigg\vert_{\frac{1}{6} \, \pi}^{\frac{7}{4} \, \pi} = {-7 \, \sqrt{2} - 14}
\]

\input{Integral-Concept-0003.HELP.tex}

\begin{multipleChoice}
\choice{The antiderivative is incorrect.}
\choice[correct]{The integrand is not defined over the entire interval.}
\choice{The bounds are evaluated in the wrong order.}
\choice{Nothing is wrong.  The equation is correct, as is.}
\end{multipleChoice}

\end{problem}}%}

\latexProblemContent{
\ifVerboseLocation This is Integration Concept Question 0003. \\ \fi
\begin{problem}

What is wrong with the following equation:

\[
\int_{\frac{1}{3} \, \pi}^{\frac{7}{4} \, \pi} {-4 \, \cot\left(x\right) \csc\left(x\right)}\;dx = {\frac{4}{\sin\left(x\right)}}\Bigg\vert_{\frac{1}{3} \, \pi}^{\frac{7}{4} \, \pi} = {-\frac{8}{3} \, \sqrt{3} - 4 \, \sqrt{2}}
\]

\input{Integral-Concept-0003.HELP.tex}

\begin{multipleChoice}
\choice{The antiderivative is incorrect.}
\choice[correct]{The integrand is not defined over the entire interval.}
\choice{The bounds are evaluated in the wrong order.}
\choice{Nothing is wrong.  The equation is correct, as is.}
\end{multipleChoice}

\end{problem}}%}

\latexProblemContent{
\ifVerboseLocation This is Integration Concept Question 0003. \\ \fi
\begin{problem}

What is wrong with the following equation:

\[
\int_{\frac{3}{4} \, \pi}^{\frac{5}{4} \, \pi} {-2 \, \cot\left(x\right) \csc\left(x\right)}\;dx = {\frac{2}{\sin\left(x\right)}}\Bigg\vert_{\frac{3}{4} \, \pi}^{\frac{5}{4} \, \pi} = {-4 \, \sqrt{2}}
\]

\input{Integral-Concept-0003.HELP.tex}

\begin{multipleChoice}
\choice{The antiderivative is incorrect.}
\choice[correct]{The integrand is not defined over the entire interval.}
\choice{The bounds are evaluated in the wrong order.}
\choice{Nothing is wrong.  The equation is correct, as is.}
\end{multipleChoice}

\end{problem}}%}

\latexProblemContent{
\ifVerboseLocation This is Integration Concept Question 0003. \\ \fi
\begin{problem}

What is wrong with the following equation:

\[
\int_{\frac{1}{3} \, \pi}^{\frac{5}{4} \, \pi} {-4 \, \cot\left(x\right) \csc\left(x\right)}\;dx = {\frac{4}{\sin\left(x\right)}}\Bigg\vert_{\frac{1}{3} \, \pi}^{\frac{5}{4} \, \pi} = {-\frac{8}{3} \, \sqrt{3} - 4 \, \sqrt{2}}
\]

\input{Integral-Concept-0003.HELP.tex}

\begin{multipleChoice}
\choice{The antiderivative is incorrect.}
\choice[correct]{The integrand is not defined over the entire interval.}
\choice{The bounds are evaluated in the wrong order.}
\choice{Nothing is wrong.  The equation is correct, as is.}
\end{multipleChoice}

\end{problem}}%}

\latexProblemContent{
\ifVerboseLocation This is Integration Concept Question 0003. \\ \fi
\begin{problem}

What is wrong with the following equation:

\[
\int_{\frac{1}{6} \, \pi}^{\frac{5}{3} \, \pi} {-6 \, \cot\left(x\right) \csc\left(x\right)}\;dx = {\frac{6}{\sin\left(x\right)}}\Bigg\vert_{\frac{1}{6} \, \pi}^{\frac{5}{3} \, \pi} = {-4 \, \sqrt{3} - 12}
\]

\input{Integral-Concept-0003.HELP.tex}

\begin{multipleChoice}
\choice{The antiderivative is incorrect.}
\choice[correct]{The integrand is not defined over the entire interval.}
\choice{The bounds are evaluated in the wrong order.}
\choice{Nothing is wrong.  The equation is correct, as is.}
\end{multipleChoice}

\end{problem}}%}

\latexProblemContent{
\ifVerboseLocation This is Integration Concept Question 0003. \\ \fi
\begin{problem}

What is wrong with the following equation:

\[
\int_{\frac{1}{4} \, \pi}^{\frac{4}{3} \, \pi} {-5 \, \csc\left(x\right)^{2}}\;dx = {\frac{5}{\tan\left(x\right)}}\Bigg\vert_{\frac{1}{4} \, \pi}^{\frac{4}{3} \, \pi} = {\frac{5}{3} \, \sqrt{3} - 5}
\]

\input{Integral-Concept-0003.HELP.tex}

\begin{multipleChoice}
\choice{The antiderivative is incorrect.}
\choice[correct]{The integrand is not defined over the entire interval.}
\choice{The bounds are evaluated in the wrong order.}
\choice{Nothing is wrong.  The equation is correct, as is.}
\end{multipleChoice}

\end{problem}}%}

\latexProblemContent{
\ifVerboseLocation This is Integration Concept Question 0003. \\ \fi
\begin{problem}

What is wrong with the following equation:

\[
\int_{\frac{5}{6} \, \pi}^{\frac{4}{3} \, \pi} {7 \, \csc\left(x\right)^{2}}\;dx = {-\frac{7}{\tan\left(x\right)}}\Bigg\vert_{\frac{5}{6} \, \pi}^{\frac{4}{3} \, \pi} = {-\frac{28}{3} \, \sqrt{3}}
\]

\input{Integral-Concept-0003.HELP.tex}

\begin{multipleChoice}
\choice{The antiderivative is incorrect.}
\choice[correct]{The integrand is not defined over the entire interval.}
\choice{The bounds are evaluated in the wrong order.}
\choice{Nothing is wrong.  The equation is correct, as is.}
\end{multipleChoice}

\end{problem}}%}

\latexProblemContent{
\ifVerboseLocation This is Integration Concept Question 0003. \\ \fi
\begin{problem}

What is wrong with the following equation:

\[
\int_{\frac{5}{6} \, \pi}^{\frac{5}{3} \, \pi} {-5 \, \csc\left(x\right)^{2}}\;dx = {\frac{5}{\tan\left(x\right)}}\Bigg\vert_{\frac{5}{6} \, \pi}^{\frac{5}{3} \, \pi} = {\frac{10}{3} \, \sqrt{3}}
\]

\input{Integral-Concept-0003.HELP.tex}

\begin{multipleChoice}
\choice{The antiderivative is incorrect.}
\choice[correct]{The integrand is not defined over the entire interval.}
\choice{The bounds are evaluated in the wrong order.}
\choice{Nothing is wrong.  The equation is correct, as is.}
\end{multipleChoice}

\end{problem}}%}

\latexProblemContent{
\ifVerboseLocation This is Integration Concept Question 0003. \\ \fi
\begin{problem}

What is wrong with the following equation:

\[
\int_{\frac{1}{3} \, \pi}^{\frac{11}{6} \, \pi} {-2 \, \csc\left(x\right)^{2}}\;dx = {\frac{2}{\tan\left(x\right)}}\Bigg\vert_{\frac{1}{3} \, \pi}^{\frac{11}{6} \, \pi} = {-\frac{8}{3} \, \sqrt{3}}
\]

\input{Integral-Concept-0003.HELP.tex}

\begin{multipleChoice}
\choice{The antiderivative is incorrect.}
\choice[correct]{The integrand is not defined over the entire interval.}
\choice{The bounds are evaluated in the wrong order.}
\choice{Nothing is wrong.  The equation is correct, as is.}
\end{multipleChoice}

\end{problem}}%}

\latexProblemContent{
\ifVerboseLocation This is Integration Concept Question 0003. \\ \fi
\begin{problem}

What is wrong with the following equation:

\[
\int_{\frac{1}{2} \, \pi}^{\frac{4}{3} \, \pi} {8 \, \cot\left(x\right) \csc\left(x\right)}\;dx = {-\frac{8}{\sin\left(x\right)}}\Bigg\vert_{\frac{1}{2} \, \pi}^{\frac{4}{3} \, \pi} = {\frac{16}{3} \, \sqrt{3} + 8}
\]

\input{Integral-Concept-0003.HELP.tex}

\begin{multipleChoice}
\choice{The antiderivative is incorrect.}
\choice[correct]{The integrand is not defined over the entire interval.}
\choice{The bounds are evaluated in the wrong order.}
\choice{Nothing is wrong.  The equation is correct, as is.}
\end{multipleChoice}

\end{problem}}%}

\latexProblemContent{
\ifVerboseLocation This is Integration Concept Question 0003. \\ \fi
\begin{problem}

What is wrong with the following equation:

\[
\int_{\frac{1}{2} \, \pi}^{\frac{11}{6} \, \pi} {-4 \, \cot\left(x\right) \csc\left(x\right)}\;dx = {\frac{4}{\sin\left(x\right)}}\Bigg\vert_{\frac{1}{2} \, \pi}^{\frac{11}{6} \, \pi} = {-12}
\]

\input{Integral-Concept-0003.HELP.tex}

\begin{multipleChoice}
\choice{The antiderivative is incorrect.}
\choice[correct]{The integrand is not defined over the entire interval.}
\choice{The bounds are evaluated in the wrong order.}
\choice{Nothing is wrong.  The equation is correct, as is.}
\end{multipleChoice}

\end{problem}}%}

\latexProblemContent{
\ifVerboseLocation This is Integration Concept Question 0003. \\ \fi
\begin{problem}

What is wrong with the following equation:

\[
\int_{\frac{1}{3} \, \pi}^{\frac{7}{4} \, \pi} {-\csc\left(x\right)^{2}}\;dx = {\frac{1}{\tan\left(x\right)}}\Bigg\vert_{\frac{1}{3} \, \pi}^{\frac{7}{4} \, \pi} = {-\frac{1}{3} \, \sqrt{3} - 1}
\]

\input{Integral-Concept-0003.HELP.tex}

\begin{multipleChoice}
\choice{The antiderivative is incorrect.}
\choice[correct]{The integrand is not defined over the entire interval.}
\choice{The bounds are evaluated in the wrong order.}
\choice{Nothing is wrong.  The equation is correct, as is.}
\end{multipleChoice}

\end{problem}}%}

\latexProblemContent{
\ifVerboseLocation This is Integration Concept Question 0003. \\ \fi
\begin{problem}

What is wrong with the following equation:

\[
\int_{\frac{3}{4} \, \pi}^{\frac{4}{3} \, \pi} {2 \, \cot\left(x\right) \csc\left(x\right)}\;dx = {-\frac{2}{\sin\left(x\right)}}\Bigg\vert_{\frac{3}{4} \, \pi}^{\frac{4}{3} \, \pi} = {\frac{4}{3} \, \sqrt{3} + 2 \, \sqrt{2}}
\]

\input{Integral-Concept-0003.HELP.tex}

\begin{multipleChoice}
\choice{The antiderivative is incorrect.}
\choice[correct]{The integrand is not defined over the entire interval.}
\choice{The bounds are evaluated in the wrong order.}
\choice{Nothing is wrong.  The equation is correct, as is.}
\end{multipleChoice}

\end{problem}}%}

\latexProblemContent{
\ifVerboseLocation This is Integration Concept Question 0003. \\ \fi
\begin{problem}

What is wrong with the following equation:

\[
\int_{\frac{1}{4} \, \pi}^{\frac{4}{3} \, \pi} {-10 \, \csc\left(x\right)^{2}}\;dx = {\frac{10}{\tan\left(x\right)}}\Bigg\vert_{\frac{1}{4} \, \pi}^{\frac{4}{3} \, \pi} = {\frac{10}{3} \, \sqrt{3} - 10}
\]

\input{Integral-Concept-0003.HELP.tex}

\begin{multipleChoice}
\choice{The antiderivative is incorrect.}
\choice[correct]{The integrand is not defined over the entire interval.}
\choice{The bounds are evaluated in the wrong order.}
\choice{Nothing is wrong.  The equation is correct, as is.}
\end{multipleChoice}

\end{problem}}%}

\latexProblemContent{
\ifVerboseLocation This is Integration Concept Question 0003. \\ \fi
\begin{problem}

What is wrong with the following equation:

\[
\int_{\frac{1}{3} \, \pi}^{\frac{3}{2} \, \pi} {2 \, \csc\left(x\right)^{2}}\;dx = {-\frac{2}{\tan\left(x\right)}}\Bigg\vert_{\frac{1}{3} \, \pi}^{\frac{3}{2} \, \pi} = {\frac{2}{3} \, \sqrt{3}}
\]

\input{Integral-Concept-0003.HELP.tex}

\begin{multipleChoice}
\choice{The antiderivative is incorrect.}
\choice[correct]{The integrand is not defined over the entire interval.}
\choice{The bounds are evaluated in the wrong order.}
\choice{Nothing is wrong.  The equation is correct, as is.}
\end{multipleChoice}

\end{problem}}%}

\latexProblemContent{
\ifVerboseLocation This is Integration Concept Question 0003. \\ \fi
\begin{problem}

What is wrong with the following equation:

\[
\int_{\frac{3}{4} \, \pi}^{\frac{11}{6} \, \pi} {\csc\left(x\right)^{2}}\;dx = {-\frac{1}{\tan\left(x\right)}}\Bigg\vert_{\frac{3}{4} \, \pi}^{\frac{11}{6} \, \pi} = {\sqrt{3} - 1}
\]

\input{Integral-Concept-0003.HELP.tex}

\begin{multipleChoice}
\choice{The antiderivative is incorrect.}
\choice[correct]{The integrand is not defined over the entire interval.}
\choice{The bounds are evaluated in the wrong order.}
\choice{Nothing is wrong.  The equation is correct, as is.}
\end{multipleChoice}

\end{problem}}%}

\latexProblemContent{
\ifVerboseLocation This is Integration Concept Question 0003. \\ \fi
\begin{problem}

What is wrong with the following equation:

\[
\int_{\frac{1}{2} \, \pi}^{\frac{7}{6} \, \pi} {-\csc\left(x\right)^{2}}\;dx = {\frac{1}{\tan\left(x\right)}}\Bigg\vert_{\frac{1}{2} \, \pi}^{\frac{7}{6} \, \pi} = {\sqrt{3}}
\]

\input{Integral-Concept-0003.HELP.tex}

\begin{multipleChoice}
\choice{The antiderivative is incorrect.}
\choice[correct]{The integrand is not defined over the entire interval.}
\choice{The bounds are evaluated in the wrong order.}
\choice{Nothing is wrong.  The equation is correct, as is.}
\end{multipleChoice}

\end{problem}}%}

\latexProblemContent{
\ifVerboseLocation This is Integration Concept Question 0003. \\ \fi
\begin{problem}

What is wrong with the following equation:

\[
\int_{\frac{1}{3} \, \pi}^{\frac{7}{4} \, \pi} {-3 \, \csc\left(x\right)^{2}}\;dx = {\frac{3}{\tan\left(x\right)}}\Bigg\vert_{\frac{1}{3} \, \pi}^{\frac{7}{4} \, \pi} = {-\sqrt{3} - 3}
\]

\input{Integral-Concept-0003.HELP.tex}

\begin{multipleChoice}
\choice{The antiderivative is incorrect.}
\choice[correct]{The integrand is not defined over the entire interval.}
\choice{The bounds are evaluated in the wrong order.}
\choice{Nothing is wrong.  The equation is correct, as is.}
\end{multipleChoice}

\end{problem}}%}

\latexProblemContent{
\ifVerboseLocation This is Integration Concept Question 0003. \\ \fi
\begin{problem}

What is wrong with the following equation:

\[
\int_{\frac{2}{3} \, \pi}^{\frac{7}{6} \, \pi} {8 \, \cot\left(x\right) \csc\left(x\right)}\;dx = {-\frac{8}{\sin\left(x\right)}}\Bigg\vert_{\frac{2}{3} \, \pi}^{\frac{7}{6} \, \pi} = {\frac{16}{3} \, \sqrt{3} + 16}
\]

\input{Integral-Concept-0003.HELP.tex}

\begin{multipleChoice}
\choice{The antiderivative is incorrect.}
\choice[correct]{The integrand is not defined over the entire interval.}
\choice{The bounds are evaluated in the wrong order.}
\choice{Nothing is wrong.  The equation is correct, as is.}
\end{multipleChoice}

\end{problem}}%}

\latexProblemContent{
\ifVerboseLocation This is Integration Concept Question 0003. \\ \fi
\begin{problem}

What is wrong with the following equation:

\[
\int_{\frac{1}{3} \, \pi}^{\frac{7}{4} \, \pi} {-5 \, \csc\left(x\right)^{2}}\;dx = {\frac{5}{\tan\left(x\right)}}\Bigg\vert_{\frac{1}{3} \, \pi}^{\frac{7}{4} \, \pi} = {-\frac{5}{3} \, \sqrt{3} - 5}
\]

\input{Integral-Concept-0003.HELP.tex}

\begin{multipleChoice}
\choice{The antiderivative is incorrect.}
\choice[correct]{The integrand is not defined over the entire interval.}
\choice{The bounds are evaluated in the wrong order.}
\choice{Nothing is wrong.  The equation is correct, as is.}
\end{multipleChoice}

\end{problem}}%}

\latexProblemContent{
\ifVerboseLocation This is Integration Concept Question 0003. \\ \fi
\begin{problem}

What is wrong with the following equation:

\[
\int_{\frac{5}{6} \, \pi}^{\frac{3}{2} \, \pi} {2 \, \csc\left(x\right)^{2}}\;dx = {-\frac{2}{\tan\left(x\right)}}\Bigg\vert_{\frac{5}{6} \, \pi}^{\frac{3}{2} \, \pi} = {-2 \, \sqrt{3}}
\]

\input{Integral-Concept-0003.HELP.tex}

\begin{multipleChoice}
\choice{The antiderivative is incorrect.}
\choice[correct]{The integrand is not defined over the entire interval.}
\choice{The bounds are evaluated in the wrong order.}
\choice{Nothing is wrong.  The equation is correct, as is.}
\end{multipleChoice}

\end{problem}}%}

\latexProblemContent{
\ifVerboseLocation This is Integration Concept Question 0003. \\ \fi
\begin{problem}

What is wrong with the following equation:

\[
\int_{\frac{2}{3} \, \pi}^{\frac{11}{6} \, \pi} {-4 \, \cot\left(x\right) \csc\left(x\right)}\;dx = {\frac{4}{\sin\left(x\right)}}\Bigg\vert_{\frac{2}{3} \, \pi}^{\frac{11}{6} \, \pi} = {-\frac{8}{3} \, \sqrt{3} - 8}
\]

\input{Integral-Concept-0003.HELP.tex}

\begin{multipleChoice}
\choice{The antiderivative is incorrect.}
\choice[correct]{The integrand is not defined over the entire interval.}
\choice{The bounds are evaluated in the wrong order.}
\choice{Nothing is wrong.  The equation is correct, as is.}
\end{multipleChoice}

\end{problem}}%}

\latexProblemContent{
\ifVerboseLocation This is Integration Concept Question 0003. \\ \fi
\begin{problem}

What is wrong with the following equation:

\[
\int_{\frac{1}{6} \, \pi}^{\frac{3}{2} \, \pi} {9 \, \csc\left(x\right)^{2}}\;dx = {-\frac{9}{\tan\left(x\right)}}\Bigg\vert_{\frac{1}{6} \, \pi}^{\frac{3}{2} \, \pi} = {9 \, \sqrt{3}}
\]

\input{Integral-Concept-0003.HELP.tex}

\begin{multipleChoice}
\choice{The antiderivative is incorrect.}
\choice[correct]{The integrand is not defined over the entire interval.}
\choice{The bounds are evaluated in the wrong order.}
\choice{Nothing is wrong.  The equation is correct, as is.}
\end{multipleChoice}

\end{problem}}%}

\latexProblemContent{
\ifVerboseLocation This is Integration Concept Question 0003. \\ \fi
\begin{problem}

What is wrong with the following equation:

\[
\int_{\frac{5}{6} \, \pi}^{\frac{7}{4} \, \pi} {4 \, \csc\left(x\right)^{2}}\;dx = {-\frac{4}{\tan\left(x\right)}}\Bigg\vert_{\frac{5}{6} \, \pi}^{\frac{7}{4} \, \pi} = {-4 \, \sqrt{3} + 4}
\]

\input{Integral-Concept-0003.HELP.tex}

\begin{multipleChoice}
\choice{The antiderivative is incorrect.}
\choice[correct]{The integrand is not defined over the entire interval.}
\choice{The bounds are evaluated in the wrong order.}
\choice{Nothing is wrong.  The equation is correct, as is.}
\end{multipleChoice}

\end{problem}}%}

\latexProblemContent{
\ifVerboseLocation This is Integration Concept Question 0003. \\ \fi
\begin{problem}

What is wrong with the following equation:

\[
\int_{\frac{1}{6} \, \pi}^{\frac{11}{6} \, \pi} {-10 \, \csc\left(x\right)^{2}}\;dx = {\frac{10}{\tan\left(x\right)}}\Bigg\vert_{\frac{1}{6} \, \pi}^{\frac{11}{6} \, \pi} = {-20 \, \sqrt{3}}
\]

\input{Integral-Concept-0003.HELP.tex}

\begin{multipleChoice}
\choice{The antiderivative is incorrect.}
\choice[correct]{The integrand is not defined over the entire interval.}
\choice{The bounds are evaluated in the wrong order.}
\choice{Nothing is wrong.  The equation is correct, as is.}
\end{multipleChoice}

\end{problem}}%}

\latexProblemContent{
\ifVerboseLocation This is Integration Concept Question 0003. \\ \fi
\begin{problem}

What is wrong with the following equation:

\[
\int_{\frac{3}{4} \, \pi}^{\frac{4}{3} \, \pi} {\csc\left(x\right)^{2}}\;dx = {-\frac{1}{\tan\left(x\right)}}\Bigg\vert_{\frac{3}{4} \, \pi}^{\frac{4}{3} \, \pi} = {-\frac{1}{3} \, \sqrt{3} - 1}
\]

\input{Integral-Concept-0003.HELP.tex}

\begin{multipleChoice}
\choice{The antiderivative is incorrect.}
\choice[correct]{The integrand is not defined over the entire interval.}
\choice{The bounds are evaluated in the wrong order.}
\choice{Nothing is wrong.  The equation is correct, as is.}
\end{multipleChoice}

\end{problem}}%}

\latexProblemContent{
\ifVerboseLocation This is Integration Concept Question 0003. \\ \fi
\begin{problem}

What is wrong with the following equation:

\[
\int_{\frac{3}{4} \, \pi}^{\frac{7}{4} \, \pi} {-\csc\left(x\right)^{2}}\;dx = {\frac{1}{\tan\left(x\right)}}\Bigg\vert_{\frac{3}{4} \, \pi}^{\frac{7}{4} \, \pi} = {0}
\]

\input{Integral-Concept-0003.HELP.tex}

\begin{multipleChoice}
\choice{The antiderivative is incorrect.}
\choice[correct]{The integrand is not defined over the entire interval.}
\choice{The bounds are evaluated in the wrong order.}
\choice{Nothing is wrong.  The equation is correct, as is.}
\end{multipleChoice}

\end{problem}}%}

\latexProblemContent{
\ifVerboseLocation This is Integration Concept Question 0003. \\ \fi
\begin{problem}

What is wrong with the following equation:

\[
\int_{\frac{1}{6} \, \pi}^{\frac{4}{3} \, \pi} {-4 \, \csc\left(x\right)^{2}}\;dx = {\frac{4}{\tan\left(x\right)}}\Bigg\vert_{\frac{1}{6} \, \pi}^{\frac{4}{3} \, \pi} = {-\frac{8}{3} \, \sqrt{3}}
\]

\input{Integral-Concept-0003.HELP.tex}

\begin{multipleChoice}
\choice{The antiderivative is incorrect.}
\choice[correct]{The integrand is not defined over the entire interval.}
\choice{The bounds are evaluated in the wrong order.}
\choice{Nothing is wrong.  The equation is correct, as is.}
\end{multipleChoice}

\end{problem}}%}

\latexProblemContent{
\ifVerboseLocation This is Integration Concept Question 0003. \\ \fi
\begin{problem}

What is wrong with the following equation:

\[
\int_{\frac{1}{2} \, \pi}^{\frac{3}{2} \, \pi} {4 \, \csc\left(x\right)^{2}}\;dx = {-\frac{4}{\tan\left(x\right)}}\Bigg\vert_{\frac{1}{2} \, \pi}^{\frac{3}{2} \, \pi} = {0}
\]

\input{Integral-Concept-0003.HELP.tex}

\begin{multipleChoice}
\choice{The antiderivative is incorrect.}
\choice[correct]{The integrand is not defined over the entire interval.}
\choice{The bounds are evaluated in the wrong order.}
\choice{Nothing is wrong.  The equation is correct, as is.}
\end{multipleChoice}

\end{problem}}%}

\latexProblemContent{
\ifVerboseLocation This is Integration Concept Question 0003. \\ \fi
\begin{problem}

What is wrong with the following equation:

\[
\int_{\frac{1}{2} \, \pi}^{\frac{7}{4} \, \pi} {-\csc\left(x\right)^{2}}\;dx = {\frac{1}{\tan\left(x\right)}}\Bigg\vert_{\frac{1}{2} \, \pi}^{\frac{7}{4} \, \pi} = {-1}
\]

\input{Integral-Concept-0003.HELP.tex}

\begin{multipleChoice}
\choice{The antiderivative is incorrect.}
\choice[correct]{The integrand is not defined over the entire interval.}
\choice{The bounds are evaluated in the wrong order.}
\choice{Nothing is wrong.  The equation is correct, as is.}
\end{multipleChoice}

\end{problem}}%}

\latexProblemContent{
\ifVerboseLocation This is Integration Concept Question 0003. \\ \fi
\begin{problem}

What is wrong with the following equation:

\[
\int_{\frac{3}{4} \, \pi}^{\frac{4}{3} \, \pi} {10 \, \cot\left(x\right) \csc\left(x\right)}\;dx = {-\frac{10}{\sin\left(x\right)}}\Bigg\vert_{\frac{3}{4} \, \pi}^{\frac{4}{3} \, \pi} = {\frac{20}{3} \, \sqrt{3} + 10 \, \sqrt{2}}
\]

\input{Integral-Concept-0003.HELP.tex}

\begin{multipleChoice}
\choice{The antiderivative is incorrect.}
\choice[correct]{The integrand is not defined over the entire interval.}
\choice{The bounds are evaluated in the wrong order.}
\choice{Nothing is wrong.  The equation is correct, as is.}
\end{multipleChoice}

\end{problem}}%}

\latexProblemContent{
\ifVerboseLocation This is Integration Concept Question 0003. \\ \fi
\begin{problem}

What is wrong with the following equation:

\[
\int_{\frac{1}{6} \, \pi}^{\frac{5}{3} \, \pi} {10 \, \cot\left(x\right) \csc\left(x\right)}\;dx = {-\frac{10}{\sin\left(x\right)}}\Bigg\vert_{\frac{1}{6} \, \pi}^{\frac{5}{3} \, \pi} = {\frac{20}{3} \, \sqrt{3} + 20}
\]

\input{Integral-Concept-0003.HELP.tex}

\begin{multipleChoice}
\choice{The antiderivative is incorrect.}
\choice[correct]{The integrand is not defined over the entire interval.}
\choice{The bounds are evaluated in the wrong order.}
\choice{Nothing is wrong.  The equation is correct, as is.}
\end{multipleChoice}

\end{problem}}%}

\latexProblemContent{
\ifVerboseLocation This is Integration Concept Question 0003. \\ \fi
\begin{problem}

What is wrong with the following equation:

\[
\int_{\frac{3}{4} \, \pi}^{\frac{3}{2} \, \pi} {-7 \, \csc\left(x\right)^{2}}\;dx = {\frac{7}{\tan\left(x\right)}}\Bigg\vert_{\frac{3}{4} \, \pi}^{\frac{3}{2} \, \pi} = {7}
\]

\input{Integral-Concept-0003.HELP.tex}

\begin{multipleChoice}
\choice{The antiderivative is incorrect.}
\choice[correct]{The integrand is not defined over the entire interval.}
\choice{The bounds are evaluated in the wrong order.}
\choice{Nothing is wrong.  The equation is correct, as is.}
\end{multipleChoice}

\end{problem}}%}

\latexProblemContent{
\ifVerboseLocation This is Integration Concept Question 0003. \\ \fi
\begin{problem}

What is wrong with the following equation:

\[
\int_{\frac{1}{6} \, \pi}^{\frac{3}{2} \, \pi} {9 \, \cot\left(x\right) \csc\left(x\right)}\;dx = {-\frac{9}{\sin\left(x\right)}}\Bigg\vert_{\frac{1}{6} \, \pi}^{\frac{3}{2} \, \pi} = {27}
\]

\input{Integral-Concept-0003.HELP.tex}

\begin{multipleChoice}
\choice{The antiderivative is incorrect.}
\choice[correct]{The integrand is not defined over the entire interval.}
\choice{The bounds are evaluated in the wrong order.}
\choice{Nothing is wrong.  The equation is correct, as is.}
\end{multipleChoice}

\end{problem}}%}

\latexProblemContent{
\ifVerboseLocation This is Integration Concept Question 0003. \\ \fi
\begin{problem}

What is wrong with the following equation:

\[
\int_{\frac{3}{4} \, \pi}^{\frac{7}{6} \, \pi} {-6 \, \csc\left(x\right)^{2}}\;dx = {\frac{6}{\tan\left(x\right)}}\Bigg\vert_{\frac{3}{4} \, \pi}^{\frac{7}{6} \, \pi} = {6 \, \sqrt{3} + 6}
\]

\input{Integral-Concept-0003.HELP.tex}

\begin{multipleChoice}
\choice{The antiderivative is incorrect.}
\choice[correct]{The integrand is not defined over the entire interval.}
\choice{The bounds are evaluated in the wrong order.}
\choice{Nothing is wrong.  The equation is correct, as is.}
\end{multipleChoice}

\end{problem}}%}

\latexProblemContent{
\ifVerboseLocation This is Integration Concept Question 0003. \\ \fi
\begin{problem}

What is wrong with the following equation:

\[
\int_{\frac{1}{6} \, \pi}^{\frac{3}{2} \, \pi} {4 \, \csc\left(x\right)^{2}}\;dx = {-\frac{4}{\tan\left(x\right)}}\Bigg\vert_{\frac{1}{6} \, \pi}^{\frac{3}{2} \, \pi} = {4 \, \sqrt{3}}
\]

\input{Integral-Concept-0003.HELP.tex}

\begin{multipleChoice}
\choice{The antiderivative is incorrect.}
\choice[correct]{The integrand is not defined over the entire interval.}
\choice{The bounds are evaluated in the wrong order.}
\choice{Nothing is wrong.  The equation is correct, as is.}
\end{multipleChoice}

\end{problem}}%}

\latexProblemContent{
\ifVerboseLocation This is Integration Concept Question 0003. \\ \fi
\begin{problem}

What is wrong with the following equation:

\[
\int_{\frac{1}{4} \, \pi}^{\frac{7}{6} \, \pi} {-6 \, \csc\left(x\right)^{2}}\;dx = {\frac{6}{\tan\left(x\right)}}\Bigg\vert_{\frac{1}{4} \, \pi}^{\frac{7}{6} \, \pi} = {6 \, \sqrt{3} - 6}
\]

\input{Integral-Concept-0003.HELP.tex}

\begin{multipleChoice}
\choice{The antiderivative is incorrect.}
\choice[correct]{The integrand is not defined over the entire interval.}
\choice{The bounds are evaluated in the wrong order.}
\choice{Nothing is wrong.  The equation is correct, as is.}
\end{multipleChoice}

\end{problem}}%}

\latexProblemContent{
\ifVerboseLocation This is Integration Concept Question 0003. \\ \fi
\begin{problem}

What is wrong with the following equation:

\[
\int_{\frac{1}{3} \, \pi}^{\frac{4}{3} \, \pi} {-2 \, \csc\left(x\right)^{2}}\;dx = {\frac{2}{\tan\left(x\right)}}\Bigg\vert_{\frac{1}{3} \, \pi}^{\frac{4}{3} \, \pi} = {0}
\]

\input{Integral-Concept-0003.HELP.tex}

\begin{multipleChoice}
\choice{The antiderivative is incorrect.}
\choice[correct]{The integrand is not defined over the entire interval.}
\choice{The bounds are evaluated in the wrong order.}
\choice{Nothing is wrong.  The equation is correct, as is.}
\end{multipleChoice}

\end{problem}}%}

\latexProblemContent{
\ifVerboseLocation This is Integration Concept Question 0003. \\ \fi
\begin{problem}

What is wrong with the following equation:

\[
\int_{\frac{1}{3} \, \pi}^{\frac{3}{2} \, \pi} {-10 \, \cot\left(x\right) \csc\left(x\right)}\;dx = {\frac{10}{\sin\left(x\right)}}\Bigg\vert_{\frac{1}{3} \, \pi}^{\frac{3}{2} \, \pi} = {-\frac{20}{3} \, \sqrt{3} - 10}
\]

\input{Integral-Concept-0003.HELP.tex}

\begin{multipleChoice}
\choice{The antiderivative is incorrect.}
\choice[correct]{The integrand is not defined over the entire interval.}
\choice{The bounds are evaluated in the wrong order.}
\choice{Nothing is wrong.  The equation is correct, as is.}
\end{multipleChoice}

\end{problem}}%}

\latexProblemContent{
\ifVerboseLocation This is Integration Concept Question 0003. \\ \fi
\begin{problem}

What is wrong with the following equation:

\[
\int_{\frac{5}{6} \, \pi}^{\frac{11}{6} \, \pi} {7 \, \csc\left(x\right)^{2}}\;dx = {-\frac{7}{\tan\left(x\right)}}\Bigg\vert_{\frac{5}{6} \, \pi}^{\frac{11}{6} \, \pi} = {0}
\]

\input{Integral-Concept-0003.HELP.tex}

\begin{multipleChoice}
\choice{The antiderivative is incorrect.}
\choice[correct]{The integrand is not defined over the entire interval.}
\choice{The bounds are evaluated in the wrong order.}
\choice{Nothing is wrong.  The equation is correct, as is.}
\end{multipleChoice}

\end{problem}}%}

\latexProblemContent{
\ifVerboseLocation This is Integration Concept Question 0003. \\ \fi
\begin{problem}

What is wrong with the following equation:

\[
\int_{\frac{1}{4} \, \pi}^{\frac{7}{6} \, \pi} {8 \, \cot\left(x\right) \csc\left(x\right)}\;dx = {-\frac{8}{\sin\left(x\right)}}\Bigg\vert_{\frac{1}{4} \, \pi}^{\frac{7}{6} \, \pi} = {8 \, \sqrt{2} + 16}
\]

\input{Integral-Concept-0003.HELP.tex}

\begin{multipleChoice}
\choice{The antiderivative is incorrect.}
\choice[correct]{The integrand is not defined over the entire interval.}
\choice{The bounds are evaluated in the wrong order.}
\choice{Nothing is wrong.  The equation is correct, as is.}
\end{multipleChoice}

\end{problem}}%}

\latexProblemContent{
\ifVerboseLocation This is Integration Concept Question 0003. \\ \fi
\begin{problem}

What is wrong with the following equation:

\[
\int_{\frac{5}{6} \, \pi}^{\frac{3}{2} \, \pi} {8 \, \cot\left(x\right) \csc\left(x\right)}\;dx = {-\frac{8}{\sin\left(x\right)}}\Bigg\vert_{\frac{5}{6} \, \pi}^{\frac{3}{2} \, \pi} = {24}
\]

\input{Integral-Concept-0003.HELP.tex}

\begin{multipleChoice}
\choice{The antiderivative is incorrect.}
\choice[correct]{The integrand is not defined over the entire interval.}
\choice{The bounds are evaluated in the wrong order.}
\choice{Nothing is wrong.  The equation is correct, as is.}
\end{multipleChoice}

\end{problem}}%}

\latexProblemContent{
\ifVerboseLocation This is Integration Concept Question 0003. \\ \fi
\begin{problem}

What is wrong with the following equation:

\[
\int_{\frac{2}{3} \, \pi}^{\frac{7}{4} \, \pi} {3 \, \csc\left(x\right)^{2}}\;dx = {-\frac{3}{\tan\left(x\right)}}\Bigg\vert_{\frac{2}{3} \, \pi}^{\frac{7}{4} \, \pi} = {-\sqrt{3} + 3}
\]

\input{Integral-Concept-0003.HELP.tex}

\begin{multipleChoice}
\choice{The antiderivative is incorrect.}
\choice[correct]{The integrand is not defined over the entire interval.}
\choice{The bounds are evaluated in the wrong order.}
\choice{Nothing is wrong.  The equation is correct, as is.}
\end{multipleChoice}

\end{problem}}%}

\latexProblemContent{
\ifVerboseLocation This is Integration Concept Question 0003. \\ \fi
\begin{problem}

What is wrong with the following equation:

\[
\int_{\frac{5}{6} \, \pi}^{\frac{5}{3} \, \pi} {-5 \, \cot\left(x\right) \csc\left(x\right)}\;dx = {\frac{5}{\sin\left(x\right)}}\Bigg\vert_{\frac{5}{6} \, \pi}^{\frac{5}{3} \, \pi} = {-\frac{10}{3} \, \sqrt{3} - 10}
\]

\input{Integral-Concept-0003.HELP.tex}

\begin{multipleChoice}
\choice{The antiderivative is incorrect.}
\choice[correct]{The integrand is not defined over the entire interval.}
\choice{The bounds are evaluated in the wrong order.}
\choice{Nothing is wrong.  The equation is correct, as is.}
\end{multipleChoice}

\end{problem}}%}

\latexProblemContent{
\ifVerboseLocation This is Integration Concept Question 0003. \\ \fi
\begin{problem}

What is wrong with the following equation:

\[
\int_{\frac{1}{2} \, \pi}^{\frac{11}{6} \, \pi} {-6 \, \csc\left(x\right)^{2}}\;dx = {\frac{6}{\tan\left(x\right)}}\Bigg\vert_{\frac{1}{2} \, \pi}^{\frac{11}{6} \, \pi} = {-6 \, \sqrt{3}}
\]

\input{Integral-Concept-0003.HELP.tex}

\begin{multipleChoice}
\choice{The antiderivative is incorrect.}
\choice[correct]{The integrand is not defined over the entire interval.}
\choice{The bounds are evaluated in the wrong order.}
\choice{Nothing is wrong.  The equation is correct, as is.}
\end{multipleChoice}

\end{problem}}%}

\latexProblemContent{
\ifVerboseLocation This is Integration Concept Question 0003. \\ \fi
\begin{problem}

What is wrong with the following equation:

\[
\int_{\frac{2}{3} \, \pi}^{\frac{7}{6} \, \pi} {2 \, \csc\left(x\right)^{2}}\;dx = {-\frac{2}{\tan\left(x\right)}}\Bigg\vert_{\frac{2}{3} \, \pi}^{\frac{7}{6} \, \pi} = {-\frac{8}{3} \, \sqrt{3}}
\]

\input{Integral-Concept-0003.HELP.tex}

\begin{multipleChoice}
\choice{The antiderivative is incorrect.}
\choice[correct]{The integrand is not defined over the entire interval.}
\choice{The bounds are evaluated in the wrong order.}
\choice{Nothing is wrong.  The equation is correct, as is.}
\end{multipleChoice}

\end{problem}}%}

\latexProblemContent{
\ifVerboseLocation This is Integration Concept Question 0003. \\ \fi
\begin{problem}

What is wrong with the following equation:

\[
\int_{\frac{1}{3} \, \pi}^{\frac{7}{4} \, \pi} {10 \, \cot\left(x\right) \csc\left(x\right)}\;dx = {-\frac{10}{\sin\left(x\right)}}\Bigg\vert_{\frac{1}{3} \, \pi}^{\frac{7}{4} \, \pi} = {\frac{20}{3} \, \sqrt{3} + 10 \, \sqrt{2}}
\]

\input{Integral-Concept-0003.HELP.tex}

\begin{multipleChoice}
\choice{The antiderivative is incorrect.}
\choice[correct]{The integrand is not defined over the entire interval.}
\choice{The bounds are evaluated in the wrong order.}
\choice{Nothing is wrong.  The equation is correct, as is.}
\end{multipleChoice}

\end{problem}}%}

\latexProblemContent{
\ifVerboseLocation This is Integration Concept Question 0003. \\ \fi
\begin{problem}

What is wrong with the following equation:

\[
\int_{\frac{1}{6} \, \pi}^{\frac{7}{6} \, \pi} {10 \, \csc\left(x\right)^{2}}\;dx = {-\frac{10}{\tan\left(x\right)}}\Bigg\vert_{\frac{1}{6} \, \pi}^{\frac{7}{6} \, \pi} = {0}
\]

\input{Integral-Concept-0003.HELP.tex}

\begin{multipleChoice}
\choice{The antiderivative is incorrect.}
\choice[correct]{The integrand is not defined over the entire interval.}
\choice{The bounds are evaluated in the wrong order.}
\choice{Nothing is wrong.  The equation is correct, as is.}
\end{multipleChoice}

\end{problem}}%}

\latexProblemContent{
\ifVerboseLocation This is Integration Concept Question 0003. \\ \fi
\begin{problem}

What is wrong with the following equation:

\[
\int_{\frac{1}{3} \, \pi}^{\frac{5}{3} \, \pi} {\cot\left(x\right) \csc\left(x\right)}\;dx = {-\frac{1}{\sin\left(x\right)}}\Bigg\vert_{\frac{1}{3} \, \pi}^{\frac{5}{3} \, \pi} = {\frac{4}{3} \, \sqrt{3}}
\]

\input{Integral-Concept-0003.HELP.tex}

\begin{multipleChoice}
\choice{The antiderivative is incorrect.}
\choice[correct]{The integrand is not defined over the entire interval.}
\choice{The bounds are evaluated in the wrong order.}
\choice{Nothing is wrong.  The equation is correct, as is.}
\end{multipleChoice}

\end{problem}}%}

\latexProblemContent{
\ifVerboseLocation This is Integration Concept Question 0003. \\ \fi
\begin{problem}

What is wrong with the following equation:

\[
\int_{\frac{2}{3} \, \pi}^{\frac{5}{3} \, \pi} {3 \, \cot\left(x\right) \csc\left(x\right)}\;dx = {-\frac{3}{\sin\left(x\right)}}\Bigg\vert_{\frac{2}{3} \, \pi}^{\frac{5}{3} \, \pi} = {4 \, \sqrt{3}}
\]

\input{Integral-Concept-0003.HELP.tex}

\begin{multipleChoice}
\choice{The antiderivative is incorrect.}
\choice[correct]{The integrand is not defined over the entire interval.}
\choice{The bounds are evaluated in the wrong order.}
\choice{Nothing is wrong.  The equation is correct, as is.}
\end{multipleChoice}

\end{problem}}%}

\latexProblemContent{
\ifVerboseLocation This is Integration Concept Question 0003. \\ \fi
\begin{problem}

What is wrong with the following equation:

\[
\int_{\frac{1}{6} \, \pi}^{\frac{7}{4} \, \pi} {-8 \, \cot\left(x\right) \csc\left(x\right)}\;dx = {\frac{8}{\sin\left(x\right)}}\Bigg\vert_{\frac{1}{6} \, \pi}^{\frac{7}{4} \, \pi} = {-8 \, \sqrt{2} - 16}
\]

\input{Integral-Concept-0003.HELP.tex}

\begin{multipleChoice}
\choice{The antiderivative is incorrect.}
\choice[correct]{The integrand is not defined over the entire interval.}
\choice{The bounds are evaluated in the wrong order.}
\choice{Nothing is wrong.  The equation is correct, as is.}
\end{multipleChoice}

\end{problem}}%}

\latexProblemContent{
\ifVerboseLocation This is Integration Concept Question 0003. \\ \fi
\begin{problem}

What is wrong with the following equation:

\[
\int_{\frac{1}{4} \, \pi}^{\frac{4}{3} \, \pi} {-2 \, \cot\left(x\right) \csc\left(x\right)}\;dx = {\frac{2}{\sin\left(x\right)}}\Bigg\vert_{\frac{1}{4} \, \pi}^{\frac{4}{3} \, \pi} = {-\frac{4}{3} \, \sqrt{3} - 2 \, \sqrt{2}}
\]

\input{Integral-Concept-0003.HELP.tex}

\begin{multipleChoice}
\choice{The antiderivative is incorrect.}
\choice[correct]{The integrand is not defined over the entire interval.}
\choice{The bounds are evaluated in the wrong order.}
\choice{Nothing is wrong.  The equation is correct, as is.}
\end{multipleChoice}

\end{problem}}%}

\latexProblemContent{
\ifVerboseLocation This is Integration Concept Question 0003. \\ \fi
\begin{problem}

What is wrong with the following equation:

\[
\int_{\frac{3}{4} \, \pi}^{\frac{5}{3} \, \pi} {-7 \, \csc\left(x\right)^{2}}\;dx = {\frac{7}{\tan\left(x\right)}}\Bigg\vert_{\frac{3}{4} \, \pi}^{\frac{5}{3} \, \pi} = {-\frac{7}{3} \, \sqrt{3} + 7}
\]

\input{Integral-Concept-0003.HELP.tex}

\begin{multipleChoice}
\choice{The antiderivative is incorrect.}
\choice[correct]{The integrand is not defined over the entire interval.}
\choice{The bounds are evaluated in the wrong order.}
\choice{Nothing is wrong.  The equation is correct, as is.}
\end{multipleChoice}

\end{problem}}%}

\latexProblemContent{
\ifVerboseLocation This is Integration Concept Question 0003. \\ \fi
\begin{problem}

What is wrong with the following equation:

\[
\int_{\frac{5}{6} \, \pi}^{\frac{3}{2} \, \pi} {6 \, \csc\left(x\right)^{2}}\;dx = {-\frac{6}{\tan\left(x\right)}}\Bigg\vert_{\frac{5}{6} \, \pi}^{\frac{3}{2} \, \pi} = {-6 \, \sqrt{3}}
\]

\input{Integral-Concept-0003.HELP.tex}

\begin{multipleChoice}
\choice{The antiderivative is incorrect.}
\choice[correct]{The integrand is not defined over the entire interval.}
\choice{The bounds are evaluated in the wrong order.}
\choice{Nothing is wrong.  The equation is correct, as is.}
\end{multipleChoice}

\end{problem}}%}

\latexProblemContent{
\ifVerboseLocation This is Integration Concept Question 0003. \\ \fi
\begin{problem}

What is wrong with the following equation:

\[
\int_{\frac{1}{2} \, \pi}^{\frac{11}{6} \, \pi} {5 \, \cot\left(x\right) \csc\left(x\right)}\;dx = {-\frac{5}{\sin\left(x\right)}}\Bigg\vert_{\frac{1}{2} \, \pi}^{\frac{11}{6} \, \pi} = {15}
\]

\input{Integral-Concept-0003.HELP.tex}

\begin{multipleChoice}
\choice{The antiderivative is incorrect.}
\choice[correct]{The integrand is not defined over the entire interval.}
\choice{The bounds are evaluated in the wrong order.}
\choice{Nothing is wrong.  The equation is correct, as is.}
\end{multipleChoice}

\end{problem}}%}

\latexProblemContent{
\ifVerboseLocation This is Integration Concept Question 0003. \\ \fi
\begin{problem}

What is wrong with the following equation:

\[
\int_{\frac{1}{6} \, \pi}^{\frac{7}{4} \, \pi} {-5 \, \csc\left(x\right)^{2}}\;dx = {\frac{5}{\tan\left(x\right)}}\Bigg\vert_{\frac{1}{6} \, \pi}^{\frac{7}{4} \, \pi} = {-5 \, \sqrt{3} - 5}
\]

\input{Integral-Concept-0003.HELP.tex}

\begin{multipleChoice}
\choice{The antiderivative is incorrect.}
\choice[correct]{The integrand is not defined over the entire interval.}
\choice{The bounds are evaluated in the wrong order.}
\choice{Nothing is wrong.  The equation is correct, as is.}
\end{multipleChoice}

\end{problem}}%}

\latexProblemContent{
\ifVerboseLocation This is Integration Concept Question 0003. \\ \fi
\begin{problem}

What is wrong with the following equation:

\[
\int_{\frac{1}{2} \, \pi}^{\frac{5}{4} \, \pi} {-4 \, \csc\left(x\right)^{2}}\;dx = {\frac{4}{\tan\left(x\right)}}\Bigg\vert_{\frac{1}{2} \, \pi}^{\frac{5}{4} \, \pi} = {4}
\]

\input{Integral-Concept-0003.HELP.tex}

\begin{multipleChoice}
\choice{The antiderivative is incorrect.}
\choice[correct]{The integrand is not defined over the entire interval.}
\choice{The bounds are evaluated in the wrong order.}
\choice{Nothing is wrong.  The equation is correct, as is.}
\end{multipleChoice}

\end{problem}}%}

\latexProblemContent{
\ifVerboseLocation This is Integration Concept Question 0003. \\ \fi
\begin{problem}

What is wrong with the following equation:

\[
\int_{\frac{1}{6} \, \pi}^{\frac{5}{3} \, \pi} {-6 \, \csc\left(x\right)^{2}}\;dx = {\frac{6}{\tan\left(x\right)}}\Bigg\vert_{\frac{1}{6} \, \pi}^{\frac{5}{3} \, \pi} = {-8 \, \sqrt{3}}
\]

\input{Integral-Concept-0003.HELP.tex}

\begin{multipleChoice}
\choice{The antiderivative is incorrect.}
\choice[correct]{The integrand is not defined over the entire interval.}
\choice{The bounds are evaluated in the wrong order.}
\choice{Nothing is wrong.  The equation is correct, as is.}
\end{multipleChoice}

\end{problem}}%}

\latexProblemContent{
\ifVerboseLocation This is Integration Concept Question 0003. \\ \fi
\begin{problem}

What is wrong with the following equation:

\[
\int_{\frac{1}{6} \, \pi}^{\frac{5}{3} \, \pi} {\cot\left(x\right) \csc\left(x\right)}\;dx = {-\frac{1}{\sin\left(x\right)}}\Bigg\vert_{\frac{1}{6} \, \pi}^{\frac{5}{3} \, \pi} = {\frac{2}{3} \, \sqrt{3} + 2}
\]

\input{Integral-Concept-0003.HELP.tex}

\begin{multipleChoice}
\choice{The antiderivative is incorrect.}
\choice[correct]{The integrand is not defined over the entire interval.}
\choice{The bounds are evaluated in the wrong order.}
\choice{Nothing is wrong.  The equation is correct, as is.}
\end{multipleChoice}

\end{problem}}%}

\latexProblemContent{
\ifVerboseLocation This is Integration Concept Question 0003. \\ \fi
\begin{problem}

What is wrong with the following equation:

\[
\int_{\frac{2}{3} \, \pi}^{\frac{5}{3} \, \pi} {2 \, \cot\left(x\right) \csc\left(x\right)}\;dx = {-\frac{2}{\sin\left(x\right)}}\Bigg\vert_{\frac{2}{3} \, \pi}^{\frac{5}{3} \, \pi} = {\frac{8}{3} \, \sqrt{3}}
\]

\input{Integral-Concept-0003.HELP.tex}

\begin{multipleChoice}
\choice{The antiderivative is incorrect.}
\choice[correct]{The integrand is not defined over the entire interval.}
\choice{The bounds are evaluated in the wrong order.}
\choice{Nothing is wrong.  The equation is correct, as is.}
\end{multipleChoice}

\end{problem}}%}

\latexProblemContent{
\ifVerboseLocation This is Integration Concept Question 0003. \\ \fi
\begin{problem}

What is wrong with the following equation:

\[
\int_{\frac{5}{6} \, \pi}^{\frac{7}{6} \, \pi} {-2 \, \csc\left(x\right)^{2}}\;dx = {\frac{2}{\tan\left(x\right)}}\Bigg\vert_{\frac{5}{6} \, \pi}^{\frac{7}{6} \, \pi} = {4 \, \sqrt{3}}
\]

\input{Integral-Concept-0003.HELP.tex}

\begin{multipleChoice}
\choice{The antiderivative is incorrect.}
\choice[correct]{The integrand is not defined over the entire interval.}
\choice{The bounds are evaluated in the wrong order.}
\choice{Nothing is wrong.  The equation is correct, as is.}
\end{multipleChoice}

\end{problem}}%}

\latexProblemContent{
\ifVerboseLocation This is Integration Concept Question 0003. \\ \fi
\begin{problem}

What is wrong with the following equation:

\[
\int_{\frac{1}{4} \, \pi}^{\frac{4}{3} \, \pi} {8 \, \cot\left(x\right) \csc\left(x\right)}\;dx = {-\frac{8}{\sin\left(x\right)}}\Bigg\vert_{\frac{1}{4} \, \pi}^{\frac{4}{3} \, \pi} = {\frac{16}{3} \, \sqrt{3} + 8 \, \sqrt{2}}
\]

\input{Integral-Concept-0003.HELP.tex}

\begin{multipleChoice}
\choice{The antiderivative is incorrect.}
\choice[correct]{The integrand is not defined over the entire interval.}
\choice{The bounds are evaluated in the wrong order.}
\choice{Nothing is wrong.  The equation is correct, as is.}
\end{multipleChoice}

\end{problem}}%}

\latexProblemContent{
\ifVerboseLocation This is Integration Concept Question 0003. \\ \fi
\begin{problem}

What is wrong with the following equation:

\[
\int_{\frac{1}{4} \, \pi}^{\frac{7}{6} \, \pi} {-6 \, \cot\left(x\right) \csc\left(x\right)}\;dx = {\frac{6}{\sin\left(x\right)}}\Bigg\vert_{\frac{1}{4} \, \pi}^{\frac{7}{6} \, \pi} = {-6 \, \sqrt{2} - 12}
\]

\input{Integral-Concept-0003.HELP.tex}

\begin{multipleChoice}
\choice{The antiderivative is incorrect.}
\choice[correct]{The integrand is not defined over the entire interval.}
\choice{The bounds are evaluated in the wrong order.}
\choice{Nothing is wrong.  The equation is correct, as is.}
\end{multipleChoice}

\end{problem}}%}

\latexProblemContent{
\ifVerboseLocation This is Integration Concept Question 0003. \\ \fi
\begin{problem}

What is wrong with the following equation:

\[
\int_{\frac{3}{4} \, \pi}^{\frac{5}{4} \, \pi} {-3 \, \cot\left(x\right) \csc\left(x\right)}\;dx = {\frac{3}{\sin\left(x\right)}}\Bigg\vert_{\frac{3}{4} \, \pi}^{\frac{5}{4} \, \pi} = {-6 \, \sqrt{2}}
\]

\input{Integral-Concept-0003.HELP.tex}

\begin{multipleChoice}
\choice{The antiderivative is incorrect.}
\choice[correct]{The integrand is not defined over the entire interval.}
\choice{The bounds are evaluated in the wrong order.}
\choice{Nothing is wrong.  The equation is correct, as is.}
\end{multipleChoice}

\end{problem}}%}

\latexProblemContent{
\ifVerboseLocation This is Integration Concept Question 0003. \\ \fi
\begin{problem}

What is wrong with the following equation:

\[
\int_{\frac{2}{3} \, \pi}^{\frac{11}{6} \, \pi} {-6 \, \cot\left(x\right) \csc\left(x\right)}\;dx = {\frac{6}{\sin\left(x\right)}}\Bigg\vert_{\frac{2}{3} \, \pi}^{\frac{11}{6} \, \pi} = {-4 \, \sqrt{3} - 12}
\]

\input{Integral-Concept-0003.HELP.tex}

\begin{multipleChoice}
\choice{The antiderivative is incorrect.}
\choice[correct]{The integrand is not defined over the entire interval.}
\choice{The bounds are evaluated in the wrong order.}
\choice{Nothing is wrong.  The equation is correct, as is.}
\end{multipleChoice}

\end{problem}}%}

\latexProblemContent{
\ifVerboseLocation This is Integration Concept Question 0003. \\ \fi
\begin{problem}

What is wrong with the following equation:

\[
\int_{\frac{1}{2} \, \pi}^{\frac{5}{4} \, \pi} {-\cot\left(x\right) \csc\left(x\right)}\;dx = {\frac{1}{\sin\left(x\right)}}\Bigg\vert_{\frac{1}{2} \, \pi}^{\frac{5}{4} \, \pi} = {-\sqrt{2} - 1}
\]

\input{Integral-Concept-0003.HELP.tex}

\begin{multipleChoice}
\choice{The antiderivative is incorrect.}
\choice[correct]{The integrand is not defined over the entire interval.}
\choice{The bounds are evaluated in the wrong order.}
\choice{Nothing is wrong.  The equation is correct, as is.}
\end{multipleChoice}

\end{problem}}%}

\latexProblemContent{
\ifVerboseLocation This is Integration Concept Question 0003. \\ \fi
\begin{problem}

What is wrong with the following equation:

\[
\int_{\frac{3}{4} \, \pi}^{\frac{7}{6} \, \pi} {-8 \, \cot\left(x\right) \csc\left(x\right)}\;dx = {\frac{8}{\sin\left(x\right)}}\Bigg\vert_{\frac{3}{4} \, \pi}^{\frac{7}{6} \, \pi} = {-8 \, \sqrt{2} - 16}
\]

\input{Integral-Concept-0003.HELP.tex}

\begin{multipleChoice}
\choice{The antiderivative is incorrect.}
\choice[correct]{The integrand is not defined over the entire interval.}
\choice{The bounds are evaluated in the wrong order.}
\choice{Nothing is wrong.  The equation is correct, as is.}
\end{multipleChoice}

\end{problem}}%}

\latexProblemContent{
\ifVerboseLocation This is Integration Concept Question 0003. \\ \fi
\begin{problem}

What is wrong with the following equation:

\[
\int_{\frac{1}{2} \, \pi}^{\frac{11}{6} \, \pi} {-8 \, \cot\left(x\right) \csc\left(x\right)}\;dx = {\frac{8}{\sin\left(x\right)}}\Bigg\vert_{\frac{1}{2} \, \pi}^{\frac{11}{6} \, \pi} = {-24}
\]

\input{Integral-Concept-0003.HELP.tex}

\begin{multipleChoice}
\choice{The antiderivative is incorrect.}
\choice[correct]{The integrand is not defined over the entire interval.}
\choice{The bounds are evaluated in the wrong order.}
\choice{Nothing is wrong.  The equation is correct, as is.}
\end{multipleChoice}

\end{problem}}%}

\latexProblemContent{
\ifVerboseLocation This is Integration Concept Question 0003. \\ \fi
\begin{problem}

What is wrong with the following equation:

\[
\int_{\frac{1}{4} \, \pi}^{\frac{5}{4} \, \pi} {\cot\left(x\right) \csc\left(x\right)}\;dx = {-\frac{1}{\sin\left(x\right)}}\Bigg\vert_{\frac{1}{4} \, \pi}^{\frac{5}{4} \, \pi} = {2 \, \sqrt{2}}
\]

\input{Integral-Concept-0003.HELP.tex}

\begin{multipleChoice}
\choice{The antiderivative is incorrect.}
\choice[correct]{The integrand is not defined over the entire interval.}
\choice{The bounds are evaluated in the wrong order.}
\choice{Nothing is wrong.  The equation is correct, as is.}
\end{multipleChoice}

\end{problem}}%}

\latexProblemContent{
\ifVerboseLocation This is Integration Concept Question 0003. \\ \fi
\begin{problem}

What is wrong with the following equation:

\[
\int_{\frac{1}{4} \, \pi}^{\frac{7}{4} \, \pi} {-8 \, \csc\left(x\right)^{2}}\;dx = {\frac{8}{\tan\left(x\right)}}\Bigg\vert_{\frac{1}{4} \, \pi}^{\frac{7}{4} \, \pi} = {-16}
\]

\input{Integral-Concept-0003.HELP.tex}

\begin{multipleChoice}
\choice{The antiderivative is incorrect.}
\choice[correct]{The integrand is not defined over the entire interval.}
\choice{The bounds are evaluated in the wrong order.}
\choice{Nothing is wrong.  The equation is correct, as is.}
\end{multipleChoice}

\end{problem}}%}

\latexProblemContent{
\ifVerboseLocation This is Integration Concept Question 0003. \\ \fi
\begin{problem}

What is wrong with the following equation:

\[
\int_{\frac{1}{2} \, \pi}^{\frac{7}{4} \, \pi} {-5 \, \csc\left(x\right)^{2}}\;dx = {\frac{5}{\tan\left(x\right)}}\Bigg\vert_{\frac{1}{2} \, \pi}^{\frac{7}{4} \, \pi} = {-5}
\]

\input{Integral-Concept-0003.HELP.tex}

\begin{multipleChoice}
\choice{The antiderivative is incorrect.}
\choice[correct]{The integrand is not defined over the entire interval.}
\choice{The bounds are evaluated in the wrong order.}
\choice{Nothing is wrong.  The equation is correct, as is.}
\end{multipleChoice}

\end{problem}}%}

\latexProblemContent{
\ifVerboseLocation This is Integration Concept Question 0003. \\ \fi
\begin{problem}

What is wrong with the following equation:

\[
\int_{\frac{1}{6} \, \pi}^{\frac{7}{6} \, \pi} {-4 \, \csc\left(x\right)^{2}}\;dx = {\frac{4}{\tan\left(x\right)}}\Bigg\vert_{\frac{1}{6} \, \pi}^{\frac{7}{6} \, \pi} = {0}
\]

\input{Integral-Concept-0003.HELP.tex}

\begin{multipleChoice}
\choice{The antiderivative is incorrect.}
\choice[correct]{The integrand is not defined over the entire interval.}
\choice{The bounds are evaluated in the wrong order.}
\choice{Nothing is wrong.  The equation is correct, as is.}
\end{multipleChoice}

\end{problem}}%}

\latexProblemContent{
\ifVerboseLocation This is Integration Concept Question 0003. \\ \fi
\begin{problem}

What is wrong with the following equation:

\[
\int_{\frac{5}{6} \, \pi}^{\frac{4}{3} \, \pi} {5 \, \cot\left(x\right) \csc\left(x\right)}\;dx = {-\frac{5}{\sin\left(x\right)}}\Bigg\vert_{\frac{5}{6} \, \pi}^{\frac{4}{3} \, \pi} = {\frac{10}{3} \, \sqrt{3} + 10}
\]

\input{Integral-Concept-0003.HELP.tex}

\begin{multipleChoice}
\choice{The antiderivative is incorrect.}
\choice[correct]{The integrand is not defined over the entire interval.}
\choice{The bounds are evaluated in the wrong order.}
\choice{Nothing is wrong.  The equation is correct, as is.}
\end{multipleChoice}

\end{problem}}%}

\latexProblemContent{
\ifVerboseLocation This is Integration Concept Question 0003. \\ \fi
\begin{problem}

What is wrong with the following equation:

\[
\int_{\frac{1}{2} \, \pi}^{\frac{4}{3} \, \pi} {-3 \, \csc\left(x\right)^{2}}\;dx = {\frac{3}{\tan\left(x\right)}}\Bigg\vert_{\frac{1}{2} \, \pi}^{\frac{4}{3} \, \pi} = {\sqrt{3}}
\]

\input{Integral-Concept-0003.HELP.tex}

\begin{multipleChoice}
\choice{The antiderivative is incorrect.}
\choice[correct]{The integrand is not defined over the entire interval.}
\choice{The bounds are evaluated in the wrong order.}
\choice{Nothing is wrong.  The equation is correct, as is.}
\end{multipleChoice}

\end{problem}}%}

\latexProblemContent{
\ifVerboseLocation This is Integration Concept Question 0003. \\ \fi
\begin{problem}

What is wrong with the following equation:

\[
\int_{\frac{3}{4} \, \pi}^{\frac{5}{4} \, \pi} {-7 \, \csc\left(x\right)^{2}}\;dx = {\frac{7}{\tan\left(x\right)}}\Bigg\vert_{\frac{3}{4} \, \pi}^{\frac{5}{4} \, \pi} = {14}
\]

\input{Integral-Concept-0003.HELP.tex}

\begin{multipleChoice}
\choice{The antiderivative is incorrect.}
\choice[correct]{The integrand is not defined over the entire interval.}
\choice{The bounds are evaluated in the wrong order.}
\choice{Nothing is wrong.  The equation is correct, as is.}
\end{multipleChoice}

\end{problem}}%}

\latexProblemContent{
\ifVerboseLocation This is Integration Concept Question 0003. \\ \fi
\begin{problem}

What is wrong with the following equation:

\[
\int_{\frac{1}{6} \, \pi}^{\frac{11}{6} \, \pi} {\cot\left(x\right) \csc\left(x\right)}\;dx = {-\frac{1}{\sin\left(x\right)}}\Bigg\vert_{\frac{1}{6} \, \pi}^{\frac{11}{6} \, \pi} = {4}
\]

\input{Integral-Concept-0003.HELP.tex}

\begin{multipleChoice}
\choice{The antiderivative is incorrect.}
\choice[correct]{The integrand is not defined over the entire interval.}
\choice{The bounds are evaluated in the wrong order.}
\choice{Nothing is wrong.  The equation is correct, as is.}
\end{multipleChoice}

\end{problem}}%}

\latexProblemContent{
\ifVerboseLocation This is Integration Concept Question 0003. \\ \fi
\begin{problem}

What is wrong with the following equation:

\[
\int_{\frac{2}{3} \, \pi}^{\frac{3}{2} \, \pi} {-8 \, \cot\left(x\right) \csc\left(x\right)}\;dx = {\frac{8}{\sin\left(x\right)}}\Bigg\vert_{\frac{2}{3} \, \pi}^{\frac{3}{2} \, \pi} = {-\frac{16}{3} \, \sqrt{3} - 8}
\]

\input{Integral-Concept-0003.HELP.tex}

\begin{multipleChoice}
\choice{The antiderivative is incorrect.}
\choice[correct]{The integrand is not defined over the entire interval.}
\choice{The bounds are evaluated in the wrong order.}
\choice{Nothing is wrong.  The equation is correct, as is.}
\end{multipleChoice}

\end{problem}}%}

\latexProblemContent{
\ifVerboseLocation This is Integration Concept Question 0003. \\ \fi
\begin{problem}

What is wrong with the following equation:

\[
\int_{\frac{1}{2} \, \pi}^{\frac{11}{6} \, \pi} {-3 \, \csc\left(x\right)^{2}}\;dx = {\frac{3}{\tan\left(x\right)}}\Bigg\vert_{\frac{1}{2} \, \pi}^{\frac{11}{6} \, \pi} = {-3 \, \sqrt{3}}
\]

\input{Integral-Concept-0003.HELP.tex}

\begin{multipleChoice}
\choice{The antiderivative is incorrect.}
\choice[correct]{The integrand is not defined over the entire interval.}
\choice{The bounds are evaluated in the wrong order.}
\choice{Nothing is wrong.  The equation is correct, as is.}
\end{multipleChoice}

\end{problem}}%}

\latexProblemContent{
\ifVerboseLocation This is Integration Concept Question 0003. \\ \fi
\begin{problem}

What is wrong with the following equation:

\[
\int_{\frac{1}{2} \, \pi}^{\frac{5}{4} \, \pi} {9 \, \cot\left(x\right) \csc\left(x\right)}\;dx = {-\frac{9}{\sin\left(x\right)}}\Bigg\vert_{\frac{1}{2} \, \pi}^{\frac{5}{4} \, \pi} = {9 \, \sqrt{2} + 9}
\]

\input{Integral-Concept-0003.HELP.tex}

\begin{multipleChoice}
\choice{The antiderivative is incorrect.}
\choice[correct]{The integrand is not defined over the entire interval.}
\choice{The bounds are evaluated in the wrong order.}
\choice{Nothing is wrong.  The equation is correct, as is.}
\end{multipleChoice}

\end{problem}}%}

\latexProblemContent{
\ifVerboseLocation This is Integration Concept Question 0003. \\ \fi
\begin{problem}

What is wrong with the following equation:

\[
\int_{\frac{5}{6} \, \pi}^{\frac{7}{4} \, \pi} {8 \, \cot\left(x\right) \csc\left(x\right)}\;dx = {-\frac{8}{\sin\left(x\right)}}\Bigg\vert_{\frac{5}{6} \, \pi}^{\frac{7}{4} \, \pi} = {8 \, \sqrt{2} + 16}
\]

\input{Integral-Concept-0003.HELP.tex}

\begin{multipleChoice}
\choice{The antiderivative is incorrect.}
\choice[correct]{The integrand is not defined over the entire interval.}
\choice{The bounds are evaluated in the wrong order.}
\choice{Nothing is wrong.  The equation is correct, as is.}
\end{multipleChoice}

\end{problem}}%}

\latexProblemContent{
\ifVerboseLocation This is Integration Concept Question 0003. \\ \fi
\begin{problem}

What is wrong with the following equation:

\[
\int_{\frac{5}{6} \, \pi}^{\frac{11}{6} \, \pi} {2 \, \cot\left(x\right) \csc\left(x\right)}\;dx = {-\frac{2}{\sin\left(x\right)}}\Bigg\vert_{\frac{5}{6} \, \pi}^{\frac{11}{6} \, \pi} = {8}
\]

\input{Integral-Concept-0003.HELP.tex}

\begin{multipleChoice}
\choice{The antiderivative is incorrect.}
\choice[correct]{The integrand is not defined over the entire interval.}
\choice{The bounds are evaluated in the wrong order.}
\choice{Nothing is wrong.  The equation is correct, as is.}
\end{multipleChoice}

\end{problem}}%}

\latexProblemContent{
\ifVerboseLocation This is Integration Concept Question 0003. \\ \fi
\begin{problem}

What is wrong with the following equation:

\[
\int_{\frac{3}{4} \, \pi}^{\frac{3}{2} \, \pi} {3 \, \csc\left(x\right)^{2}}\;dx = {-\frac{3}{\tan\left(x\right)}}\Bigg\vert_{\frac{3}{4} \, \pi}^{\frac{3}{2} \, \pi} = {-3}
\]

\input{Integral-Concept-0003.HELP.tex}

\begin{multipleChoice}
\choice{The antiderivative is incorrect.}
\choice[correct]{The integrand is not defined over the entire interval.}
\choice{The bounds are evaluated in the wrong order.}
\choice{Nothing is wrong.  The equation is correct, as is.}
\end{multipleChoice}

\end{problem}}%}

\latexProblemContent{
\ifVerboseLocation This is Integration Concept Question 0003. \\ \fi
\begin{problem}

What is wrong with the following equation:

\[
\int_{\frac{3}{4} \, \pi}^{\frac{7}{4} \, \pi} {6 \, \cot\left(x\right) \csc\left(x\right)}\;dx = {-\frac{6}{\sin\left(x\right)}}\Bigg\vert_{\frac{3}{4} \, \pi}^{\frac{7}{4} \, \pi} = {12 \, \sqrt{2}}
\]

\input{Integral-Concept-0003.HELP.tex}

\begin{multipleChoice}
\choice{The antiderivative is incorrect.}
\choice[correct]{The integrand is not defined over the entire interval.}
\choice{The bounds are evaluated in the wrong order.}
\choice{Nothing is wrong.  The equation is correct, as is.}
\end{multipleChoice}

\end{problem}}%}

\latexProblemContent{
\ifVerboseLocation This is Integration Concept Question 0003. \\ \fi
\begin{problem}

What is wrong with the following equation:

\[
\int_{\frac{1}{4} \, \pi}^{\frac{11}{6} \, \pi} {6 \, \cot\left(x\right) \csc\left(x\right)}\;dx = {-\frac{6}{\sin\left(x\right)}}\Bigg\vert_{\frac{1}{4} \, \pi}^{\frac{11}{6} \, \pi} = {6 \, \sqrt{2} + 12}
\]

\input{Integral-Concept-0003.HELP.tex}

\begin{multipleChoice}
\choice{The antiderivative is incorrect.}
\choice[correct]{The integrand is not defined over the entire interval.}
\choice{The bounds are evaluated in the wrong order.}
\choice{Nothing is wrong.  The equation is correct, as is.}
\end{multipleChoice}

\end{problem}}%}

\latexProblemContent{
\ifVerboseLocation This is Integration Concept Question 0003. \\ \fi
\begin{problem}

What is wrong with the following equation:

\[
\int_{\frac{1}{2} \, \pi}^{\frac{5}{4} \, \pi} {7 \, \csc\left(x\right)^{2}}\;dx = {-\frac{7}{\tan\left(x\right)}}\Bigg\vert_{\frac{1}{2} \, \pi}^{\frac{5}{4} \, \pi} = {-7}
\]

\input{Integral-Concept-0003.HELP.tex}

\begin{multipleChoice}
\choice{The antiderivative is incorrect.}
\choice[correct]{The integrand is not defined over the entire interval.}
\choice{The bounds are evaluated in the wrong order.}
\choice{Nothing is wrong.  The equation is correct, as is.}
\end{multipleChoice}

\end{problem}}%}

\latexProblemContent{
\ifVerboseLocation This is Integration Concept Question 0003. \\ \fi
\begin{problem}

What is wrong with the following equation:

\[
\int_{\frac{3}{4} \, \pi}^{\frac{11}{6} \, \pi} {-7 \, \cot\left(x\right) \csc\left(x\right)}\;dx = {\frac{7}{\sin\left(x\right)}}\Bigg\vert_{\frac{3}{4} \, \pi}^{\frac{11}{6} \, \pi} = {-7 \, \sqrt{2} - 14}
\]

\input{Integral-Concept-0003.HELP.tex}

\begin{multipleChoice}
\choice{The antiderivative is incorrect.}
\choice[correct]{The integrand is not defined over the entire interval.}
\choice{The bounds are evaluated in the wrong order.}
\choice{Nothing is wrong.  The equation is correct, as is.}
\end{multipleChoice}

\end{problem}}%}

\latexProblemContent{
\ifVerboseLocation This is Integration Concept Question 0003. \\ \fi
\begin{problem}

What is wrong with the following equation:

\[
\int_{\frac{1}{3} \, \pi}^{\frac{3}{2} \, \pi} {-5 \, \csc\left(x\right)^{2}}\;dx = {\frac{5}{\tan\left(x\right)}}\Bigg\vert_{\frac{1}{3} \, \pi}^{\frac{3}{2} \, \pi} = {-\frac{5}{3} \, \sqrt{3}}
\]

\input{Integral-Concept-0003.HELP.tex}

\begin{multipleChoice}
\choice{The antiderivative is incorrect.}
\choice[correct]{The integrand is not defined over the entire interval.}
\choice{The bounds are evaluated in the wrong order.}
\choice{Nothing is wrong.  The equation is correct, as is.}
\end{multipleChoice}

\end{problem}}%}

\latexProblemContent{
\ifVerboseLocation This is Integration Concept Question 0003. \\ \fi
\begin{problem}

What is wrong with the following equation:

\[
\int_{\frac{2}{3} \, \pi}^{\frac{11}{6} \, \pi} {4 \, \csc\left(x\right)^{2}}\;dx = {-\frac{4}{\tan\left(x\right)}}\Bigg\vert_{\frac{2}{3} \, \pi}^{\frac{11}{6} \, \pi} = {\frac{8}{3} \, \sqrt{3}}
\]

\input{Integral-Concept-0003.HELP.tex}

\begin{multipleChoice}
\choice{The antiderivative is incorrect.}
\choice[correct]{The integrand is not defined over the entire interval.}
\choice{The bounds are evaluated in the wrong order.}
\choice{Nothing is wrong.  The equation is correct, as is.}
\end{multipleChoice}

\end{problem}}%}

\latexProblemContent{
\ifVerboseLocation This is Integration Concept Question 0003. \\ \fi
\begin{problem}

What is wrong with the following equation:

\[
\int_{\frac{1}{3} \, \pi}^{\frac{11}{6} \, \pi} {-7 \, \csc\left(x\right)^{2}}\;dx = {\frac{7}{\tan\left(x\right)}}\Bigg\vert_{\frac{1}{3} \, \pi}^{\frac{11}{6} \, \pi} = {-\frac{28}{3} \, \sqrt{3}}
\]

\input{Integral-Concept-0003.HELP.tex}

\begin{multipleChoice}
\choice{The antiderivative is incorrect.}
\choice[correct]{The integrand is not defined over the entire interval.}
\choice{The bounds are evaluated in the wrong order.}
\choice{Nothing is wrong.  The equation is correct, as is.}
\end{multipleChoice}

\end{problem}}%}

\latexProblemContent{
\ifVerboseLocation This is Integration Concept Question 0003. \\ \fi
\begin{problem}

What is wrong with the following equation:

\[
\int_{\frac{1}{4} \, \pi}^{\frac{5}{4} \, \pi} {-3 \, \csc\left(x\right)^{2}}\;dx = {\frac{3}{\tan\left(x\right)}}\Bigg\vert_{\frac{1}{4} \, \pi}^{\frac{5}{4} \, \pi} = {0}
\]

\input{Integral-Concept-0003.HELP.tex}

\begin{multipleChoice}
\choice{The antiderivative is incorrect.}
\choice[correct]{The integrand is not defined over the entire interval.}
\choice{The bounds are evaluated in the wrong order.}
\choice{Nothing is wrong.  The equation is correct, as is.}
\end{multipleChoice}

\end{problem}}%}

\latexProblemContent{
\ifVerboseLocation This is Integration Concept Question 0003. \\ \fi
\begin{problem}

What is wrong with the following equation:

\[
\int_{\frac{1}{2} \, \pi}^{\frac{7}{4} \, \pi} {10 \, \cot\left(x\right) \csc\left(x\right)}\;dx = {-\frac{10}{\sin\left(x\right)}}\Bigg\vert_{\frac{1}{2} \, \pi}^{\frac{7}{4} \, \pi} = {10 \, \sqrt{2} + 10}
\]

\input{Integral-Concept-0003.HELP.tex}

\begin{multipleChoice}
\choice{The antiderivative is incorrect.}
\choice[correct]{The integrand is not defined over the entire interval.}
\choice{The bounds are evaluated in the wrong order.}
\choice{Nothing is wrong.  The equation is correct, as is.}
\end{multipleChoice}

\end{problem}}%}

\latexProblemContent{
\ifVerboseLocation This is Integration Concept Question 0003. \\ \fi
\begin{problem}

What is wrong with the following equation:

\[
\int_{\frac{1}{3} \, \pi}^{\frac{4}{3} \, \pi} {9 \, \cot\left(x\right) \csc\left(x\right)}\;dx = {-\frac{9}{\sin\left(x\right)}}\Bigg\vert_{\frac{1}{3} \, \pi}^{\frac{4}{3} \, \pi} = {12 \, \sqrt{3}}
\]

\input{Integral-Concept-0003.HELP.tex}

\begin{multipleChoice}
\choice{The antiderivative is incorrect.}
\choice[correct]{The integrand is not defined over the entire interval.}
\choice{The bounds are evaluated in the wrong order.}
\choice{Nothing is wrong.  The equation is correct, as is.}
\end{multipleChoice}

\end{problem}}%}

\latexProblemContent{
\ifVerboseLocation This is Integration Concept Question 0003. \\ \fi
\begin{problem}

What is wrong with the following equation:

\[
\int_{\frac{1}{3} \, \pi}^{\frac{4}{3} \, \pi} {2 \, \cot\left(x\right) \csc\left(x\right)}\;dx = {-\frac{2}{\sin\left(x\right)}}\Bigg\vert_{\frac{1}{3} \, \pi}^{\frac{4}{3} \, \pi} = {\frac{8}{3} \, \sqrt{3}}
\]

\input{Integral-Concept-0003.HELP.tex}

\begin{multipleChoice}
\choice{The antiderivative is incorrect.}
\choice[correct]{The integrand is not defined over the entire interval.}
\choice{The bounds are evaluated in the wrong order.}
\choice{Nothing is wrong.  The equation is correct, as is.}
\end{multipleChoice}

\end{problem}}%}

\latexProblemContent{
\ifVerboseLocation This is Integration Concept Question 0003. \\ \fi
\begin{problem}

What is wrong with the following equation:

\[
\int_{\frac{1}{3} \, \pi}^{\frac{11}{6} \, \pi} {2 \, \csc\left(x\right)^{2}}\;dx = {-\frac{2}{\tan\left(x\right)}}\Bigg\vert_{\frac{1}{3} \, \pi}^{\frac{11}{6} \, \pi} = {\frac{8}{3} \, \sqrt{3}}
\]

\input{Integral-Concept-0003.HELP.tex}

\begin{multipleChoice}
\choice{The antiderivative is incorrect.}
\choice[correct]{The integrand is not defined over the entire interval.}
\choice{The bounds are evaluated in the wrong order.}
\choice{Nothing is wrong.  The equation is correct, as is.}
\end{multipleChoice}

\end{problem}}%}

\latexProblemContent{
\ifVerboseLocation This is Integration Concept Question 0003. \\ \fi
\begin{problem}

What is wrong with the following equation:

\[
\int_{\frac{3}{4} \, \pi}^{\frac{3}{2} \, \pi} {-10 \, \csc\left(x\right)^{2}}\;dx = {\frac{10}{\tan\left(x\right)}}\Bigg\vert_{\frac{3}{4} \, \pi}^{\frac{3}{2} \, \pi} = {10}
\]

\input{Integral-Concept-0003.HELP.tex}

\begin{multipleChoice}
\choice{The antiderivative is incorrect.}
\choice[correct]{The integrand is not defined over the entire interval.}
\choice{The bounds are evaluated in the wrong order.}
\choice{Nothing is wrong.  The equation is correct, as is.}
\end{multipleChoice}

\end{problem}}%}

\latexProblemContent{
\ifVerboseLocation This is Integration Concept Question 0003. \\ \fi
\begin{problem}

What is wrong with the following equation:

\[
\int_{\frac{1}{6} \, \pi}^{\frac{7}{6} \, \pi} {-10 \, \cot\left(x\right) \csc\left(x\right)}\;dx = {\frac{10}{\sin\left(x\right)}}\Bigg\vert_{\frac{1}{6} \, \pi}^{\frac{7}{6} \, \pi} = {-40}
\]

\input{Integral-Concept-0003.HELP.tex}

\begin{multipleChoice}
\choice{The antiderivative is incorrect.}
\choice[correct]{The integrand is not defined over the entire interval.}
\choice{The bounds are evaluated in the wrong order.}
\choice{Nothing is wrong.  The equation is correct, as is.}
\end{multipleChoice}

\end{problem}}%}

\latexProblemContent{
\ifVerboseLocation This is Integration Concept Question 0003. \\ \fi
\begin{problem}

What is wrong with the following equation:

\[
\int_{\frac{1}{6} \, \pi}^{\frac{7}{6} \, \pi} {7 \, \cot\left(x\right) \csc\left(x\right)}\;dx = {-\frac{7}{\sin\left(x\right)}}\Bigg\vert_{\frac{1}{6} \, \pi}^{\frac{7}{6} \, \pi} = {28}
\]

\input{Integral-Concept-0003.HELP.tex}

\begin{multipleChoice}
\choice{The antiderivative is incorrect.}
\choice[correct]{The integrand is not defined over the entire interval.}
\choice{The bounds are evaluated in the wrong order.}
\choice{Nothing is wrong.  The equation is correct, as is.}
\end{multipleChoice}

\end{problem}}%}

\latexProblemContent{
\ifVerboseLocation This is Integration Concept Question 0003. \\ \fi
\begin{problem}

What is wrong with the following equation:

\[
\int_{\frac{1}{4} \, \pi}^{\frac{4}{3} \, \pi} {2 \, \csc\left(x\right)^{2}}\;dx = {-\frac{2}{\tan\left(x\right)}}\Bigg\vert_{\frac{1}{4} \, \pi}^{\frac{4}{3} \, \pi} = {-\frac{2}{3} \, \sqrt{3} + 2}
\]

\input{Integral-Concept-0003.HELP.tex}

\begin{multipleChoice}
\choice{The antiderivative is incorrect.}
\choice[correct]{The integrand is not defined over the entire interval.}
\choice{The bounds are evaluated in the wrong order.}
\choice{Nothing is wrong.  The equation is correct, as is.}
\end{multipleChoice}

\end{problem}}%}

\latexProblemContent{
\ifVerboseLocation This is Integration Concept Question 0003. \\ \fi
\begin{problem}

What is wrong with the following equation:

\[
\int_{\frac{5}{6} \, \pi}^{\frac{7}{4} \, \pi} {-4 \, \csc\left(x\right)^{2}}\;dx = {\frac{4}{\tan\left(x\right)}}\Bigg\vert_{\frac{5}{6} \, \pi}^{\frac{7}{4} \, \pi} = {4 \, \sqrt{3} - 4}
\]

\input{Integral-Concept-0003.HELP.tex}

\begin{multipleChoice}
\choice{The antiderivative is incorrect.}
\choice[correct]{The integrand is not defined over the entire interval.}
\choice{The bounds are evaluated in the wrong order.}
\choice{Nothing is wrong.  The equation is correct, as is.}
\end{multipleChoice}

\end{problem}}%}

\latexProblemContent{
\ifVerboseLocation This is Integration Concept Question 0003. \\ \fi
\begin{problem}

What is wrong with the following equation:

\[
\int_{\frac{3}{4} \, \pi}^{\frac{11}{6} \, \pi} {10 \, \cot\left(x\right) \csc\left(x\right)}\;dx = {-\frac{10}{\sin\left(x\right)}}\Bigg\vert_{\frac{3}{4} \, \pi}^{\frac{11}{6} \, \pi} = {10 \, \sqrt{2} + 20}
\]

\input{Integral-Concept-0003.HELP.tex}

\begin{multipleChoice}
\choice{The antiderivative is incorrect.}
\choice[correct]{The integrand is not defined over the entire interval.}
\choice{The bounds are evaluated in the wrong order.}
\choice{Nothing is wrong.  The equation is correct, as is.}
\end{multipleChoice}

\end{problem}}%}

\latexProblemContent{
\ifVerboseLocation This is Integration Concept Question 0003. \\ \fi
\begin{problem}

What is wrong with the following equation:

\[
\int_{\frac{3}{4} \, \pi}^{\frac{5}{4} \, \pi} {\cot\left(x\right) \csc\left(x\right)}\;dx = {-\frac{1}{\sin\left(x\right)}}\Bigg\vert_{\frac{3}{4} \, \pi}^{\frac{5}{4} \, \pi} = {2 \, \sqrt{2}}
\]

\input{Integral-Concept-0003.HELP.tex}

\begin{multipleChoice}
\choice{The antiderivative is incorrect.}
\choice[correct]{The integrand is not defined over the entire interval.}
\choice{The bounds are evaluated in the wrong order.}
\choice{Nothing is wrong.  The equation is correct, as is.}
\end{multipleChoice}

\end{problem}}%}

\latexProblemContent{
\ifVerboseLocation This is Integration Concept Question 0003. \\ \fi
\begin{problem}

What is wrong with the following equation:

\[
\int_{\frac{5}{6} \, \pi}^{\frac{4}{3} \, \pi} {-10 \, \csc\left(x\right)^{2}}\;dx = {\frac{10}{\tan\left(x\right)}}\Bigg\vert_{\frac{5}{6} \, \pi}^{\frac{4}{3} \, \pi} = {\frac{40}{3} \, \sqrt{3}}
\]

\input{Integral-Concept-0003.HELP.tex}

\begin{multipleChoice}
\choice{The antiderivative is incorrect.}
\choice[correct]{The integrand is not defined over the entire interval.}
\choice{The bounds are evaluated in the wrong order.}
\choice{Nothing is wrong.  The equation is correct, as is.}
\end{multipleChoice}

\end{problem}}%}

\latexProblemContent{
\ifVerboseLocation This is Integration Concept Question 0003. \\ \fi
\begin{problem}

What is wrong with the following equation:

\[
\int_{\frac{5}{6} \, \pi}^{\frac{4}{3} \, \pi} {10 \, \csc\left(x\right)^{2}}\;dx = {-\frac{10}{\tan\left(x\right)}}\Bigg\vert_{\frac{5}{6} \, \pi}^{\frac{4}{3} \, \pi} = {-\frac{40}{3} \, \sqrt{3}}
\]

\input{Integral-Concept-0003.HELP.tex}

\begin{multipleChoice}
\choice{The antiderivative is incorrect.}
\choice[correct]{The integrand is not defined over the entire interval.}
\choice{The bounds are evaluated in the wrong order.}
\choice{Nothing is wrong.  The equation is correct, as is.}
\end{multipleChoice}

\end{problem}}%}

\latexProblemContent{
\ifVerboseLocation This is Integration Concept Question 0003. \\ \fi
\begin{problem}

What is wrong with the following equation:

\[
\int_{\frac{1}{3} \, \pi}^{\frac{11}{6} \, \pi} {\cot\left(x\right) \csc\left(x\right)}\;dx = {-\frac{1}{\sin\left(x\right)}}\Bigg\vert_{\frac{1}{3} \, \pi}^{\frac{11}{6} \, \pi} = {\frac{2}{3} \, \sqrt{3} + 2}
\]

\input{Integral-Concept-0003.HELP.tex}

\begin{multipleChoice}
\choice{The antiderivative is incorrect.}
\choice[correct]{The integrand is not defined over the entire interval.}
\choice{The bounds are evaluated in the wrong order.}
\choice{Nothing is wrong.  The equation is correct, as is.}
\end{multipleChoice}

\end{problem}}%}

\latexProblemContent{
\ifVerboseLocation This is Integration Concept Question 0003. \\ \fi
\begin{problem}

What is wrong with the following equation:

\[
\int_{\frac{3}{4} \, \pi}^{\frac{5}{4} \, \pi} {-5 \, \csc\left(x\right)^{2}}\;dx = {\frac{5}{\tan\left(x\right)}}\Bigg\vert_{\frac{3}{4} \, \pi}^{\frac{5}{4} \, \pi} = {10}
\]

\input{Integral-Concept-0003.HELP.tex}

\begin{multipleChoice}
\choice{The antiderivative is incorrect.}
\choice[correct]{The integrand is not defined over the entire interval.}
\choice{The bounds are evaluated in the wrong order.}
\choice{Nothing is wrong.  The equation is correct, as is.}
\end{multipleChoice}

\end{problem}}%}

\latexProblemContent{
\ifVerboseLocation This is Integration Concept Question 0003. \\ \fi
\begin{problem}

What is wrong with the following equation:

\[
\int_{\frac{5}{6} \, \pi}^{\frac{5}{3} \, \pi} {7 \, \cot\left(x\right) \csc\left(x\right)}\;dx = {-\frac{7}{\sin\left(x\right)}}\Bigg\vert_{\frac{5}{6} \, \pi}^{\frac{5}{3} \, \pi} = {\frac{14}{3} \, \sqrt{3} + 14}
\]

\input{Integral-Concept-0003.HELP.tex}

\begin{multipleChoice}
\choice{The antiderivative is incorrect.}
\choice[correct]{The integrand is not defined over the entire interval.}
\choice{The bounds are evaluated in the wrong order.}
\choice{Nothing is wrong.  The equation is correct, as is.}
\end{multipleChoice}

\end{problem}}%}

\latexProblemContent{
\ifVerboseLocation This is Integration Concept Question 0003. \\ \fi
\begin{problem}

What is wrong with the following equation:

\[
\int_{\frac{1}{4} \, \pi}^{\frac{4}{3} \, \pi} {-\csc\left(x\right)^{2}}\;dx = {\frac{1}{\tan\left(x\right)}}\Bigg\vert_{\frac{1}{4} \, \pi}^{\frac{4}{3} \, \pi} = {\frac{1}{3} \, \sqrt{3} - 1}
\]

\input{Integral-Concept-0003.HELP.tex}

\begin{multipleChoice}
\choice{The antiderivative is incorrect.}
\choice[correct]{The integrand is not defined over the entire interval.}
\choice{The bounds are evaluated in the wrong order.}
\choice{Nothing is wrong.  The equation is correct, as is.}
\end{multipleChoice}

\end{problem}}%}

\latexProblemContent{
\ifVerboseLocation This is Integration Concept Question 0003. \\ \fi
\begin{problem}

What is wrong with the following equation:

\[
\int_{\frac{3}{4} \, \pi}^{\frac{5}{4} \, \pi} {10 \, \csc\left(x\right)^{2}}\;dx = {-\frac{10}{\tan\left(x\right)}}\Bigg\vert_{\frac{3}{4} \, \pi}^{\frac{5}{4} \, \pi} = {-20}
\]

\input{Integral-Concept-0003.HELP.tex}

\begin{multipleChoice}
\choice{The antiderivative is incorrect.}
\choice[correct]{The integrand is not defined over the entire interval.}
\choice{The bounds are evaluated in the wrong order.}
\choice{Nothing is wrong.  The equation is correct, as is.}
\end{multipleChoice}

\end{problem}}%}

\latexProblemContent{
\ifVerboseLocation This is Integration Concept Question 0003. \\ \fi
\begin{problem}

What is wrong with the following equation:

\[
\int_{\frac{1}{6} \, \pi}^{\frac{5}{3} \, \pi} {-8 \, \cot\left(x\right) \csc\left(x\right)}\;dx = {\frac{8}{\sin\left(x\right)}}\Bigg\vert_{\frac{1}{6} \, \pi}^{\frac{5}{3} \, \pi} = {-\frac{16}{3} \, \sqrt{3} - 16}
\]

\input{Integral-Concept-0003.HELP.tex}

\begin{multipleChoice}
\choice{The antiderivative is incorrect.}
\choice[correct]{The integrand is not defined over the entire interval.}
\choice{The bounds are evaluated in the wrong order.}
\choice{Nothing is wrong.  The equation is correct, as is.}
\end{multipleChoice}

\end{problem}}%}

\latexProblemContent{
\ifVerboseLocation This is Integration Concept Question 0003. \\ \fi
\begin{problem}

What is wrong with the following equation:

\[
\int_{\frac{2}{3} \, \pi}^{\frac{5}{4} \, \pi} {-10 \, \csc\left(x\right)^{2}}\;dx = {\frac{10}{\tan\left(x\right)}}\Bigg\vert_{\frac{2}{3} \, \pi}^{\frac{5}{4} \, \pi} = {\frac{10}{3} \, \sqrt{3} + 10}
\]

\input{Integral-Concept-0003.HELP.tex}

\begin{multipleChoice}
\choice{The antiderivative is incorrect.}
\choice[correct]{The integrand is not defined over the entire interval.}
\choice{The bounds are evaluated in the wrong order.}
\choice{Nothing is wrong.  The equation is correct, as is.}
\end{multipleChoice}

\end{problem}}%}

\latexProblemContent{
\ifVerboseLocation This is Integration Concept Question 0003. \\ \fi
\begin{problem}

What is wrong with the following equation:

\[
\int_{\frac{5}{6} \, \pi}^{\frac{3}{2} \, \pi} {4 \, \cot\left(x\right) \csc\left(x\right)}\;dx = {-\frac{4}{\sin\left(x\right)}}\Bigg\vert_{\frac{5}{6} \, \pi}^{\frac{3}{2} \, \pi} = {12}
\]

\input{Integral-Concept-0003.HELP.tex}

\begin{multipleChoice}
\choice{The antiderivative is incorrect.}
\choice[correct]{The integrand is not defined over the entire interval.}
\choice{The bounds are evaluated in the wrong order.}
\choice{Nothing is wrong.  The equation is correct, as is.}
\end{multipleChoice}

\end{problem}}%}

\latexProblemContent{
\ifVerboseLocation This is Integration Concept Question 0003. \\ \fi
\begin{problem}

What is wrong with the following equation:

\[
\int_{\frac{1}{6} \, \pi}^{\frac{5}{4} \, \pi} {8 \, \csc\left(x\right)^{2}}\;dx = {-\frac{8}{\tan\left(x\right)}}\Bigg\vert_{\frac{1}{6} \, \pi}^{\frac{5}{4} \, \pi} = {8 \, \sqrt{3} - 8}
\]

\input{Integral-Concept-0003.HELP.tex}

\begin{multipleChoice}
\choice{The antiderivative is incorrect.}
\choice[correct]{The integrand is not defined over the entire interval.}
\choice{The bounds are evaluated in the wrong order.}
\choice{Nothing is wrong.  The equation is correct, as is.}
\end{multipleChoice}

\end{problem}}%}

\latexProblemContent{
\ifVerboseLocation This is Integration Concept Question 0003. \\ \fi
\begin{problem}

What is wrong with the following equation:

\[
\int_{\frac{1}{4} \, \pi}^{\frac{7}{4} \, \pi} {4 \, \cot\left(x\right) \csc\left(x\right)}\;dx = {-\frac{4}{\sin\left(x\right)}}\Bigg\vert_{\frac{1}{4} \, \pi}^{\frac{7}{4} \, \pi} = {8 \, \sqrt{2}}
\]

\input{Integral-Concept-0003.HELP.tex}

\begin{multipleChoice}
\choice{The antiderivative is incorrect.}
\choice[correct]{The integrand is not defined over the entire interval.}
\choice{The bounds are evaluated in the wrong order.}
\choice{Nothing is wrong.  The equation is correct, as is.}
\end{multipleChoice}

\end{problem}}%}

\latexProblemContent{
\ifVerboseLocation This is Integration Concept Question 0003. \\ \fi
\begin{problem}

What is wrong with the following equation:

\[
\int_{\frac{2}{3} \, \pi}^{\frac{5}{4} \, \pi} {5 \, \csc\left(x\right)^{2}}\;dx = {-\frac{5}{\tan\left(x\right)}}\Bigg\vert_{\frac{2}{3} \, \pi}^{\frac{5}{4} \, \pi} = {-\frac{5}{3} \, \sqrt{3} - 5}
\]

\input{Integral-Concept-0003.HELP.tex}

\begin{multipleChoice}
\choice{The antiderivative is incorrect.}
\choice[correct]{The integrand is not defined over the entire interval.}
\choice{The bounds are evaluated in the wrong order.}
\choice{Nothing is wrong.  The equation is correct, as is.}
\end{multipleChoice}

\end{problem}}%}

\latexProblemContent{
\ifVerboseLocation This is Integration Concept Question 0003. \\ \fi
\begin{problem}

What is wrong with the following equation:

\[
\int_{\frac{2}{3} \, \pi}^{\frac{7}{4} \, \pi} {8 \, \csc\left(x\right)^{2}}\;dx = {-\frac{8}{\tan\left(x\right)}}\Bigg\vert_{\frac{2}{3} \, \pi}^{\frac{7}{4} \, \pi} = {-\frac{8}{3} \, \sqrt{3} + 8}
\]

\input{Integral-Concept-0003.HELP.tex}

\begin{multipleChoice}
\choice{The antiderivative is incorrect.}
\choice[correct]{The integrand is not defined over the entire interval.}
\choice{The bounds are evaluated in the wrong order.}
\choice{Nothing is wrong.  The equation is correct, as is.}
\end{multipleChoice}

\end{problem}}%}

\latexProblemContent{
\ifVerboseLocation This is Integration Concept Question 0003. \\ \fi
\begin{problem}

What is wrong with the following equation:

\[
\int_{\frac{5}{6} \, \pi}^{\frac{3}{2} \, \pi} {-\csc\left(x\right)^{2}}\;dx = {\frac{1}{\tan\left(x\right)}}\Bigg\vert_{\frac{5}{6} \, \pi}^{\frac{3}{2} \, \pi} = {\sqrt{3}}
\]

\input{Integral-Concept-0003.HELP.tex}

\begin{multipleChoice}
\choice{The antiderivative is incorrect.}
\choice[correct]{The integrand is not defined over the entire interval.}
\choice{The bounds are evaluated in the wrong order.}
\choice{Nothing is wrong.  The equation is correct, as is.}
\end{multipleChoice}

\end{problem}}%}

\latexProblemContent{
\ifVerboseLocation This is Integration Concept Question 0003. \\ \fi
\begin{problem}

What is wrong with the following equation:

\[
\int_{\frac{1}{2} \, \pi}^{\frac{3}{2} \, \pi} {6 \, \cot\left(x\right) \csc\left(x\right)}\;dx = {-\frac{6}{\sin\left(x\right)}}\Bigg\vert_{\frac{1}{2} \, \pi}^{\frac{3}{2} \, \pi} = {12}
\]

\input{Integral-Concept-0003.HELP.tex}

\begin{multipleChoice}
\choice{The antiderivative is incorrect.}
\choice[correct]{The integrand is not defined over the entire interval.}
\choice{The bounds are evaluated in the wrong order.}
\choice{Nothing is wrong.  The equation is correct, as is.}
\end{multipleChoice}

\end{problem}}%}

\latexProblemContent{
\ifVerboseLocation This is Integration Concept Question 0003. \\ \fi
\begin{problem}

What is wrong with the following equation:

\[
\int_{\frac{1}{4} \, \pi}^{\frac{7}{4} \, \pi} {-7 \, \csc\left(x\right)^{2}}\;dx = {\frac{7}{\tan\left(x\right)}}\Bigg\vert_{\frac{1}{4} \, \pi}^{\frac{7}{4} \, \pi} = {-14}
\]

\input{Integral-Concept-0003.HELP.tex}

\begin{multipleChoice}
\choice{The antiderivative is incorrect.}
\choice[correct]{The integrand is not defined over the entire interval.}
\choice{The bounds are evaluated in the wrong order.}
\choice{Nothing is wrong.  The equation is correct, as is.}
\end{multipleChoice}

\end{problem}}%}

\latexProblemContent{
\ifVerboseLocation This is Integration Concept Question 0003. \\ \fi
\begin{problem}

What is wrong with the following equation:

\[
\int_{\frac{5}{6} \, \pi}^{\frac{7}{6} \, \pi} {-8 \, \cot\left(x\right) \csc\left(x\right)}\;dx = {\frac{8}{\sin\left(x\right)}}\Bigg\vert_{\frac{5}{6} \, \pi}^{\frac{7}{6} \, \pi} = {-32}
\]

\input{Integral-Concept-0003.HELP.tex}

\begin{multipleChoice}
\choice{The antiderivative is incorrect.}
\choice[correct]{The integrand is not defined over the entire interval.}
\choice{The bounds are evaluated in the wrong order.}
\choice{Nothing is wrong.  The equation is correct, as is.}
\end{multipleChoice}

\end{problem}}%}

\latexProblemContent{
\ifVerboseLocation This is Integration Concept Question 0003. \\ \fi
\begin{problem}

What is wrong with the following equation:

\[
\int_{\frac{1}{6} \, \pi}^{\frac{4}{3} \, \pi} {-10 \, \cot\left(x\right) \csc\left(x\right)}\;dx = {\frac{10}{\sin\left(x\right)}}\Bigg\vert_{\frac{1}{6} \, \pi}^{\frac{4}{3} \, \pi} = {-\frac{20}{3} \, \sqrt{3} - 20}
\]

\input{Integral-Concept-0003.HELP.tex}

\begin{multipleChoice}
\choice{The antiderivative is incorrect.}
\choice[correct]{The integrand is not defined over the entire interval.}
\choice{The bounds are evaluated in the wrong order.}
\choice{Nothing is wrong.  The equation is correct, as is.}
\end{multipleChoice}

\end{problem}}%}

\latexProblemContent{
\ifVerboseLocation This is Integration Concept Question 0003. \\ \fi
\begin{problem}

What is wrong with the following equation:

\[
\int_{\frac{1}{3} \, \pi}^{\frac{7}{6} \, \pi} {6 \, \csc\left(x\right)^{2}}\;dx = {-\frac{6}{\tan\left(x\right)}}\Bigg\vert_{\frac{1}{3} \, \pi}^{\frac{7}{6} \, \pi} = {-4 \, \sqrt{3}}
\]

\input{Integral-Concept-0003.HELP.tex}

\begin{multipleChoice}
\choice{The antiderivative is incorrect.}
\choice[correct]{The integrand is not defined over the entire interval.}
\choice{The bounds are evaluated in the wrong order.}
\choice{Nothing is wrong.  The equation is correct, as is.}
\end{multipleChoice}

\end{problem}}%}

\latexProblemContent{
\ifVerboseLocation This is Integration Concept Question 0003. \\ \fi
\begin{problem}

What is wrong with the following equation:

\[
\int_{\frac{1}{2} \, \pi}^{\frac{11}{6} \, \pi} {7 \, \csc\left(x\right)^{2}}\;dx = {-\frac{7}{\tan\left(x\right)}}\Bigg\vert_{\frac{1}{2} \, \pi}^{\frac{11}{6} \, \pi} = {7 \, \sqrt{3}}
\]

\input{Integral-Concept-0003.HELP.tex}

\begin{multipleChoice}
\choice{The antiderivative is incorrect.}
\choice[correct]{The integrand is not defined over the entire interval.}
\choice{The bounds are evaluated in the wrong order.}
\choice{Nothing is wrong.  The equation is correct, as is.}
\end{multipleChoice}

\end{problem}}%}

\latexProblemContent{
\ifVerboseLocation This is Integration Concept Question 0003. \\ \fi
\begin{problem}

What is wrong with the following equation:

\[
\int_{\frac{1}{2} \, \pi}^{\frac{4}{3} \, \pi} {9 \, \cot\left(x\right) \csc\left(x\right)}\;dx = {-\frac{9}{\sin\left(x\right)}}\Bigg\vert_{\frac{1}{2} \, \pi}^{\frac{4}{3} \, \pi} = {6 \, \sqrt{3} + 9}
\]

\input{Integral-Concept-0003.HELP.tex}

\begin{multipleChoice}
\choice{The antiderivative is incorrect.}
\choice[correct]{The integrand is not defined over the entire interval.}
\choice{The bounds are evaluated in the wrong order.}
\choice{Nothing is wrong.  The equation is correct, as is.}
\end{multipleChoice}

\end{problem}}%}

\latexProblemContent{
\ifVerboseLocation This is Integration Concept Question 0003. \\ \fi
\begin{problem}

What is wrong with the following equation:

\[
\int_{\frac{1}{6} \, \pi}^{\frac{7}{4} \, \pi} {3 \, \csc\left(x\right)^{2}}\;dx = {-\frac{3}{\tan\left(x\right)}}\Bigg\vert_{\frac{1}{6} \, \pi}^{\frac{7}{4} \, \pi} = {3 \, \sqrt{3} + 3}
\]

\input{Integral-Concept-0003.HELP.tex}

\begin{multipleChoice}
\choice{The antiderivative is incorrect.}
\choice[correct]{The integrand is not defined over the entire interval.}
\choice{The bounds are evaluated in the wrong order.}
\choice{Nothing is wrong.  The equation is correct, as is.}
\end{multipleChoice}

\end{problem}}%}

\latexProblemContent{
\ifVerboseLocation This is Integration Concept Question 0003. \\ \fi
\begin{problem}

What is wrong with the following equation:

\[
\int_{\frac{5}{6} \, \pi}^{\frac{7}{4} \, \pi} {-7 \, \csc\left(x\right)^{2}}\;dx = {\frac{7}{\tan\left(x\right)}}\Bigg\vert_{\frac{5}{6} \, \pi}^{\frac{7}{4} \, \pi} = {7 \, \sqrt{3} - 7}
\]

\input{Integral-Concept-0003.HELP.tex}

\begin{multipleChoice}
\choice{The antiderivative is incorrect.}
\choice[correct]{The integrand is not defined over the entire interval.}
\choice{The bounds are evaluated in the wrong order.}
\choice{Nothing is wrong.  The equation is correct, as is.}
\end{multipleChoice}

\end{problem}}%}

\latexProblemContent{
\ifVerboseLocation This is Integration Concept Question 0003. \\ \fi
\begin{problem}

What is wrong with the following equation:

\[
\int_{\frac{1}{3} \, \pi}^{\frac{7}{4} \, \pi} {10 \, \csc\left(x\right)^{2}}\;dx = {-\frac{10}{\tan\left(x\right)}}\Bigg\vert_{\frac{1}{3} \, \pi}^{\frac{7}{4} \, \pi} = {\frac{10}{3} \, \sqrt{3} + 10}
\]

\input{Integral-Concept-0003.HELP.tex}

\begin{multipleChoice}
\choice{The antiderivative is incorrect.}
\choice[correct]{The integrand is not defined over the entire interval.}
\choice{The bounds are evaluated in the wrong order.}
\choice{Nothing is wrong.  The equation is correct, as is.}
\end{multipleChoice}

\end{problem}}%}

\latexProblemContent{
\ifVerboseLocation This is Integration Concept Question 0003. \\ \fi
\begin{problem}

What is wrong with the following equation:

\[
\int_{\frac{3}{4} \, \pi}^{\frac{4}{3} \, \pi} {-4 \, \csc\left(x\right)^{2}}\;dx = {\frac{4}{\tan\left(x\right)}}\Bigg\vert_{\frac{3}{4} \, \pi}^{\frac{4}{3} \, \pi} = {\frac{4}{3} \, \sqrt{3} + 4}
\]

\input{Integral-Concept-0003.HELP.tex}

\begin{multipleChoice}
\choice{The antiderivative is incorrect.}
\choice[correct]{The integrand is not defined over the entire interval.}
\choice{The bounds are evaluated in the wrong order.}
\choice{Nothing is wrong.  The equation is correct, as is.}
\end{multipleChoice}

\end{problem}}%}

\latexProblemContent{
\ifVerboseLocation This is Integration Concept Question 0003. \\ \fi
\begin{problem}

What is wrong with the following equation:

\[
\int_{\frac{3}{4} \, \pi}^{\frac{5}{3} \, \pi} {7 \, \csc\left(x\right)^{2}}\;dx = {-\frac{7}{\tan\left(x\right)}}\Bigg\vert_{\frac{3}{4} \, \pi}^{\frac{5}{3} \, \pi} = {\frac{7}{3} \, \sqrt{3} - 7}
\]

\input{Integral-Concept-0003.HELP.tex}

\begin{multipleChoice}
\choice{The antiderivative is incorrect.}
\choice[correct]{The integrand is not defined over the entire interval.}
\choice{The bounds are evaluated in the wrong order.}
\choice{Nothing is wrong.  The equation is correct, as is.}
\end{multipleChoice}

\end{problem}}%}

\latexProblemContent{
\ifVerboseLocation This is Integration Concept Question 0003. \\ \fi
\begin{problem}

What is wrong with the following equation:

\[
\int_{\frac{1}{3} \, \pi}^{\frac{5}{3} \, \pi} {\csc\left(x\right)^{2}}\;dx = {-\frac{1}{\tan\left(x\right)}}\Bigg\vert_{\frac{1}{3} \, \pi}^{\frac{5}{3} \, \pi} = {\frac{2}{3} \, \sqrt{3}}
\]

\input{Integral-Concept-0003.HELP.tex}

\begin{multipleChoice}
\choice{The antiderivative is incorrect.}
\choice[correct]{The integrand is not defined over the entire interval.}
\choice{The bounds are evaluated in the wrong order.}
\choice{Nothing is wrong.  The equation is correct, as is.}
\end{multipleChoice}

\end{problem}}%}

\latexProblemContent{
\ifVerboseLocation This is Integration Concept Question 0003. \\ \fi
\begin{problem}

What is wrong with the following equation:

\[
\int_{\frac{1}{2} \, \pi}^{\frac{5}{4} \, \pi} {3 \, \cot\left(x\right) \csc\left(x\right)}\;dx = {-\frac{3}{\sin\left(x\right)}}\Bigg\vert_{\frac{1}{2} \, \pi}^{\frac{5}{4} \, \pi} = {3 \, \sqrt{2} + 3}
\]

\input{Integral-Concept-0003.HELP.tex}

\begin{multipleChoice}
\choice{The antiderivative is incorrect.}
\choice[correct]{The integrand is not defined over the entire interval.}
\choice{The bounds are evaluated in the wrong order.}
\choice{Nothing is wrong.  The equation is correct, as is.}
\end{multipleChoice}

\end{problem}}%}

\latexProblemContent{
\ifVerboseLocation This is Integration Concept Question 0003. \\ \fi
\begin{problem}

What is wrong with the following equation:

\[
\int_{\frac{5}{6} \, \pi}^{\frac{11}{6} \, \pi} {-9 \, \cot\left(x\right) \csc\left(x\right)}\;dx = {\frac{9}{\sin\left(x\right)}}\Bigg\vert_{\frac{5}{6} \, \pi}^{\frac{11}{6} \, \pi} = {-36}
\]

\input{Integral-Concept-0003.HELP.tex}

\begin{multipleChoice}
\choice{The antiderivative is incorrect.}
\choice[correct]{The integrand is not defined over the entire interval.}
\choice{The bounds are evaluated in the wrong order.}
\choice{Nothing is wrong.  The equation is correct, as is.}
\end{multipleChoice}

\end{problem}}%}

\latexProblemContent{
\ifVerboseLocation This is Integration Concept Question 0003. \\ \fi
\begin{problem}

What is wrong with the following equation:

\[
\int_{\frac{1}{6} \, \pi}^{\frac{5}{3} \, \pi} {-4 \, \csc\left(x\right)^{2}}\;dx = {\frac{4}{\tan\left(x\right)}}\Bigg\vert_{\frac{1}{6} \, \pi}^{\frac{5}{3} \, \pi} = {-\frac{16}{3} \, \sqrt{3}}
\]

\input{Integral-Concept-0003.HELP.tex}

\begin{multipleChoice}
\choice{The antiderivative is incorrect.}
\choice[correct]{The integrand is not defined over the entire interval.}
\choice{The bounds are evaluated in the wrong order.}
\choice{Nothing is wrong.  The equation is correct, as is.}
\end{multipleChoice}

\end{problem}}%}

\latexProblemContent{
\ifVerboseLocation This is Integration Concept Question 0003. \\ \fi
\begin{problem}

What is wrong with the following equation:

\[
\int_{\frac{1}{4} \, \pi}^{\frac{3}{2} \, \pi} {-\cot\left(x\right) \csc\left(x\right)}\;dx = {\frac{1}{\sin\left(x\right)}}\Bigg\vert_{\frac{1}{4} \, \pi}^{\frac{3}{2} \, \pi} = {-\sqrt{2} - 1}
\]

\input{Integral-Concept-0003.HELP.tex}

\begin{multipleChoice}
\choice{The antiderivative is incorrect.}
\choice[correct]{The integrand is not defined over the entire interval.}
\choice{The bounds are evaluated in the wrong order.}
\choice{Nothing is wrong.  The equation is correct, as is.}
\end{multipleChoice}

\end{problem}}%}

\latexProblemContent{
\ifVerboseLocation This is Integration Concept Question 0003. \\ \fi
\begin{problem}

What is wrong with the following equation:

\[
\int_{\frac{1}{3} \, \pi}^{\frac{7}{6} \, \pi} {9 \, \cot\left(x\right) \csc\left(x\right)}\;dx = {-\frac{9}{\sin\left(x\right)}}\Bigg\vert_{\frac{1}{3} \, \pi}^{\frac{7}{6} \, \pi} = {6 \, \sqrt{3} + 18}
\]

\input{Integral-Concept-0003.HELP.tex}

\begin{multipleChoice}
\choice{The antiderivative is incorrect.}
\choice[correct]{The integrand is not defined over the entire interval.}
\choice{The bounds are evaluated in the wrong order.}
\choice{Nothing is wrong.  The equation is correct, as is.}
\end{multipleChoice}

\end{problem}}%}

\latexProblemContent{
\ifVerboseLocation This is Integration Concept Question 0003. \\ \fi
\begin{problem}

What is wrong with the following equation:

\[
\int_{\frac{1}{3} \, \pi}^{\frac{11}{6} \, \pi} {7 \, \cot\left(x\right) \csc\left(x\right)}\;dx = {-\frac{7}{\sin\left(x\right)}}\Bigg\vert_{\frac{1}{3} \, \pi}^{\frac{11}{6} \, \pi} = {\frac{14}{3} \, \sqrt{3} + 14}
\]

\input{Integral-Concept-0003.HELP.tex}

\begin{multipleChoice}
\choice{The antiderivative is incorrect.}
\choice[correct]{The integrand is not defined over the entire interval.}
\choice{The bounds are evaluated in the wrong order.}
\choice{Nothing is wrong.  The equation is correct, as is.}
\end{multipleChoice}

\end{problem}}%}

\latexProblemContent{
\ifVerboseLocation This is Integration Concept Question 0003. \\ \fi
\begin{problem}

What is wrong with the following equation:

\[
\int_{\frac{1}{3} \, \pi}^{\frac{3}{2} \, \pi} {5 \, \cot\left(x\right) \csc\left(x\right)}\;dx = {-\frac{5}{\sin\left(x\right)}}\Bigg\vert_{\frac{1}{3} \, \pi}^{\frac{3}{2} \, \pi} = {\frac{10}{3} \, \sqrt{3} + 5}
\]

\input{Integral-Concept-0003.HELP.tex}

\begin{multipleChoice}
\choice{The antiderivative is incorrect.}
\choice[correct]{The integrand is not defined over the entire interval.}
\choice{The bounds are evaluated in the wrong order.}
\choice{Nothing is wrong.  The equation is correct, as is.}
\end{multipleChoice}

\end{problem}}%}

\latexProblemContent{
\ifVerboseLocation This is Integration Concept Question 0003. \\ \fi
\begin{problem}

What is wrong with the following equation:

\[
\int_{\frac{1}{4} \, \pi}^{\frac{7}{6} \, \pi} {6 \, \csc\left(x\right)^{2}}\;dx = {-\frac{6}{\tan\left(x\right)}}\Bigg\vert_{\frac{1}{4} \, \pi}^{\frac{7}{6} \, \pi} = {-6 \, \sqrt{3} + 6}
\]

\input{Integral-Concept-0003.HELP.tex}

\begin{multipleChoice}
\choice{The antiderivative is incorrect.}
\choice[correct]{The integrand is not defined over the entire interval.}
\choice{The bounds are evaluated in the wrong order.}
\choice{Nothing is wrong.  The equation is correct, as is.}
\end{multipleChoice}

\end{problem}}%}

\latexProblemContent{
\ifVerboseLocation This is Integration Concept Question 0003. \\ \fi
\begin{problem}

What is wrong with the following equation:

\[
\int_{\frac{1}{6} \, \pi}^{\frac{7}{6} \, \pi} {-5 \, \csc\left(x\right)^{2}}\;dx = {\frac{5}{\tan\left(x\right)}}\Bigg\vert_{\frac{1}{6} \, \pi}^{\frac{7}{6} \, \pi} = {0}
\]

\input{Integral-Concept-0003.HELP.tex}

\begin{multipleChoice}
\choice{The antiderivative is incorrect.}
\choice[correct]{The integrand is not defined over the entire interval.}
\choice{The bounds are evaluated in the wrong order.}
\choice{Nothing is wrong.  The equation is correct, as is.}
\end{multipleChoice}

\end{problem}}%}

\latexProblemContent{
\ifVerboseLocation This is Integration Concept Question 0003. \\ \fi
\begin{problem}

What is wrong with the following equation:

\[
\int_{\frac{1}{2} \, \pi}^{\frac{11}{6} \, \pi} {-4 \, \csc\left(x\right)^{2}}\;dx = {\frac{4}{\tan\left(x\right)}}\Bigg\vert_{\frac{1}{2} \, \pi}^{\frac{11}{6} \, \pi} = {-4 \, \sqrt{3}}
\]

\input{Integral-Concept-0003.HELP.tex}

\begin{multipleChoice}
\choice{The antiderivative is incorrect.}
\choice[correct]{The integrand is not defined over the entire interval.}
\choice{The bounds are evaluated in the wrong order.}
\choice{Nothing is wrong.  The equation is correct, as is.}
\end{multipleChoice}

\end{problem}}%}

\latexProblemContent{
\ifVerboseLocation This is Integration Concept Question 0003. \\ \fi
\begin{problem}

What is wrong with the following equation:

\[
\int_{\frac{3}{4} \, \pi}^{\frac{7}{4} \, \pi} {-3 \, \csc\left(x\right)^{2}}\;dx = {\frac{3}{\tan\left(x\right)}}\Bigg\vert_{\frac{3}{4} \, \pi}^{\frac{7}{4} \, \pi} = {0}
\]

\input{Integral-Concept-0003.HELP.tex}

\begin{multipleChoice}
\choice{The antiderivative is incorrect.}
\choice[correct]{The integrand is not defined over the entire interval.}
\choice{The bounds are evaluated in the wrong order.}
\choice{Nothing is wrong.  The equation is correct, as is.}
\end{multipleChoice}

\end{problem}}%}

\latexProblemContent{
\ifVerboseLocation This is Integration Concept Question 0003. \\ \fi
\begin{problem}

What is wrong with the following equation:

\[
\int_{\frac{5}{6} \, \pi}^{\frac{7}{4} \, \pi} {\cot\left(x\right) \csc\left(x\right)}\;dx = {-\frac{1}{\sin\left(x\right)}}\Bigg\vert_{\frac{5}{6} \, \pi}^{\frac{7}{4} \, \pi} = {\sqrt{2} + 2}
\]

\input{Integral-Concept-0003.HELP.tex}

\begin{multipleChoice}
\choice{The antiderivative is incorrect.}
\choice[correct]{The integrand is not defined over the entire interval.}
\choice{The bounds are evaluated in the wrong order.}
\choice{Nothing is wrong.  The equation is correct, as is.}
\end{multipleChoice}

\end{problem}}%}

\latexProblemContent{
\ifVerboseLocation This is Integration Concept Question 0003. \\ \fi
\begin{problem}

What is wrong with the following equation:

\[
\int_{\frac{1}{2} \, \pi}^{\frac{7}{6} \, \pi} {9 \, \cot\left(x\right) \csc\left(x\right)}\;dx = {-\frac{9}{\sin\left(x\right)}}\Bigg\vert_{\frac{1}{2} \, \pi}^{\frac{7}{6} \, \pi} = {27}
\]

\input{Integral-Concept-0003.HELP.tex}

\begin{multipleChoice}
\choice{The antiderivative is incorrect.}
\choice[correct]{The integrand is not defined over the entire interval.}
\choice{The bounds are evaluated in the wrong order.}
\choice{Nothing is wrong.  The equation is correct, as is.}
\end{multipleChoice}

\end{problem}}%}

\latexProblemContent{
\ifVerboseLocation This is Integration Concept Question 0003. \\ \fi
\begin{problem}

What is wrong with the following equation:

\[
\int_{\frac{1}{3} \, \pi}^{\frac{3}{2} \, \pi} {8 \, \cot\left(x\right) \csc\left(x\right)}\;dx = {-\frac{8}{\sin\left(x\right)}}\Bigg\vert_{\frac{1}{3} \, \pi}^{\frac{3}{2} \, \pi} = {\frac{16}{3} \, \sqrt{3} + 8}
\]

\input{Integral-Concept-0003.HELP.tex}

\begin{multipleChoice}
\choice{The antiderivative is incorrect.}
\choice[correct]{The integrand is not defined over the entire interval.}
\choice{The bounds are evaluated in the wrong order.}
\choice{Nothing is wrong.  The equation is correct, as is.}
\end{multipleChoice}

\end{problem}}%}

\latexProblemContent{
\ifVerboseLocation This is Integration Concept Question 0003. \\ \fi
\begin{problem}

What is wrong with the following equation:

\[
\int_{\frac{3}{4} \, \pi}^{\frac{11}{6} \, \pi} {6 \, \cot\left(x\right) \csc\left(x\right)}\;dx = {-\frac{6}{\sin\left(x\right)}}\Bigg\vert_{\frac{3}{4} \, \pi}^{\frac{11}{6} \, \pi} = {6 \, \sqrt{2} + 12}
\]

\input{Integral-Concept-0003.HELP.tex}

\begin{multipleChoice}
\choice{The antiderivative is incorrect.}
\choice[correct]{The integrand is not defined over the entire interval.}
\choice{The bounds are evaluated in the wrong order.}
\choice{Nothing is wrong.  The equation is correct, as is.}
\end{multipleChoice}

\end{problem}}%}

\latexProblemContent{
\ifVerboseLocation This is Integration Concept Question 0003. \\ \fi
\begin{problem}

What is wrong with the following equation:

\[
\int_{\frac{1}{4} \, \pi}^{\frac{5}{3} \, \pi} {7 \, \csc\left(x\right)^{2}}\;dx = {-\frac{7}{\tan\left(x\right)}}\Bigg\vert_{\frac{1}{4} \, \pi}^{\frac{5}{3} \, \pi} = {\frac{7}{3} \, \sqrt{3} + 7}
\]

\input{Integral-Concept-0003.HELP.tex}

\begin{multipleChoice}
\choice{The antiderivative is incorrect.}
\choice[correct]{The integrand is not defined over the entire interval.}
\choice{The bounds are evaluated in the wrong order.}
\choice{Nothing is wrong.  The equation is correct, as is.}
\end{multipleChoice}

\end{problem}}%}

\latexProblemContent{
\ifVerboseLocation This is Integration Concept Question 0003. \\ \fi
\begin{problem}

What is wrong with the following equation:

\[
\int_{\frac{1}{6} \, \pi}^{\frac{3}{2} \, \pi} {8 \, \cot\left(x\right) \csc\left(x\right)}\;dx = {-\frac{8}{\sin\left(x\right)}}\Bigg\vert_{\frac{1}{6} \, \pi}^{\frac{3}{2} \, \pi} = {24}
\]

\input{Integral-Concept-0003.HELP.tex}

\begin{multipleChoice}
\choice{The antiderivative is incorrect.}
\choice[correct]{The integrand is not defined over the entire interval.}
\choice{The bounds are evaluated in the wrong order.}
\choice{Nothing is wrong.  The equation is correct, as is.}
\end{multipleChoice}

\end{problem}}%}

\latexProblemContent{
\ifVerboseLocation This is Integration Concept Question 0003. \\ \fi
\begin{problem}

What is wrong with the following equation:

\[
\int_{\frac{5}{6} \, \pi}^{\frac{4}{3} \, \pi} {-\cot\left(x\right) \csc\left(x\right)}\;dx = {\frac{1}{\sin\left(x\right)}}\Bigg\vert_{\frac{5}{6} \, \pi}^{\frac{4}{3} \, \pi} = {-\frac{2}{3} \, \sqrt{3} - 2}
\]

\input{Integral-Concept-0003.HELP.tex}

\begin{multipleChoice}
\choice{The antiderivative is incorrect.}
\choice[correct]{The integrand is not defined over the entire interval.}
\choice{The bounds are evaluated in the wrong order.}
\choice{Nothing is wrong.  The equation is correct, as is.}
\end{multipleChoice}

\end{problem}}%}

\latexProblemContent{
\ifVerboseLocation This is Integration Concept Question 0003. \\ \fi
\begin{problem}

What is wrong with the following equation:

\[
\int_{\frac{5}{6} \, \pi}^{\frac{4}{3} \, \pi} {7 \, \cot\left(x\right) \csc\left(x\right)}\;dx = {-\frac{7}{\sin\left(x\right)}}\Bigg\vert_{\frac{5}{6} \, \pi}^{\frac{4}{3} \, \pi} = {\frac{14}{3} \, \sqrt{3} + 14}
\]

\input{Integral-Concept-0003.HELP.tex}

\begin{multipleChoice}
\choice{The antiderivative is incorrect.}
\choice[correct]{The integrand is not defined over the entire interval.}
\choice{The bounds are evaluated in the wrong order.}
\choice{Nothing is wrong.  The equation is correct, as is.}
\end{multipleChoice}

\end{problem}}%}

\latexProblemContent{
\ifVerboseLocation This is Integration Concept Question 0003. \\ \fi
\begin{problem}

What is wrong with the following equation:

\[
\int_{\frac{1}{4} \, \pi}^{\frac{4}{3} \, \pi} {-5 \, \cot\left(x\right) \csc\left(x\right)}\;dx = {\frac{5}{\sin\left(x\right)}}\Bigg\vert_{\frac{1}{4} \, \pi}^{\frac{4}{3} \, \pi} = {-\frac{10}{3} \, \sqrt{3} - 5 \, \sqrt{2}}
\]

\input{Integral-Concept-0003.HELP.tex}

\begin{multipleChoice}
\choice{The antiderivative is incorrect.}
\choice[correct]{The integrand is not defined over the entire interval.}
\choice{The bounds are evaluated in the wrong order.}
\choice{Nothing is wrong.  The equation is correct, as is.}
\end{multipleChoice}

\end{problem}}%}

\latexProblemContent{
\ifVerboseLocation This is Integration Concept Question 0003. \\ \fi
\begin{problem}

What is wrong with the following equation:

\[
\int_{\frac{1}{6} \, \pi}^{\frac{4}{3} \, \pi} {\csc\left(x\right)^{2}}\;dx = {-\frac{1}{\tan\left(x\right)}}\Bigg\vert_{\frac{1}{6} \, \pi}^{\frac{4}{3} \, \pi} = {\frac{2}{3} \, \sqrt{3}}
\]

\input{Integral-Concept-0003.HELP.tex}

\begin{multipleChoice}
\choice{The antiderivative is incorrect.}
\choice[correct]{The integrand is not defined over the entire interval.}
\choice{The bounds are evaluated in the wrong order.}
\choice{Nothing is wrong.  The equation is correct, as is.}
\end{multipleChoice}

\end{problem}}%}

\latexProblemContent{
\ifVerboseLocation This is Integration Concept Question 0003. \\ \fi
\begin{problem}

What is wrong with the following equation:

\[
\int_{\frac{3}{4} \, \pi}^{\frac{7}{6} \, \pi} {-4 \, \cot\left(x\right) \csc\left(x\right)}\;dx = {\frac{4}{\sin\left(x\right)}}\Bigg\vert_{\frac{3}{4} \, \pi}^{\frac{7}{6} \, \pi} = {-4 \, \sqrt{2} - 8}
\]

\input{Integral-Concept-0003.HELP.tex}

\begin{multipleChoice}
\choice{The antiderivative is incorrect.}
\choice[correct]{The integrand is not defined over the entire interval.}
\choice{The bounds are evaluated in the wrong order.}
\choice{Nothing is wrong.  The equation is correct, as is.}
\end{multipleChoice}

\end{problem}}%}

\latexProblemContent{
\ifVerboseLocation This is Integration Concept Question 0003. \\ \fi
\begin{problem}

What is wrong with the following equation:

\[
\int_{\frac{1}{6} \, \pi}^{\frac{7}{6} \, \pi} {-7 \, \csc\left(x\right)^{2}}\;dx = {\frac{7}{\tan\left(x\right)}}\Bigg\vert_{\frac{1}{6} \, \pi}^{\frac{7}{6} \, \pi} = {0}
\]

\input{Integral-Concept-0003.HELP.tex}

\begin{multipleChoice}
\choice{The antiderivative is incorrect.}
\choice[correct]{The integrand is not defined over the entire interval.}
\choice{The bounds are evaluated in the wrong order.}
\choice{Nothing is wrong.  The equation is correct, as is.}
\end{multipleChoice}

\end{problem}}%}

\latexProblemContent{
\ifVerboseLocation This is Integration Concept Question 0003. \\ \fi
\begin{problem}

What is wrong with the following equation:

\[
\int_{\frac{1}{6} \, \pi}^{\frac{7}{4} \, \pi} {-\csc\left(x\right)^{2}}\;dx = {\frac{1}{\tan\left(x\right)}}\Bigg\vert_{\frac{1}{6} \, \pi}^{\frac{7}{4} \, \pi} = {-\sqrt{3} - 1}
\]

\input{Integral-Concept-0003.HELP.tex}

\begin{multipleChoice}
\choice{The antiderivative is incorrect.}
\choice[correct]{The integrand is not defined over the entire interval.}
\choice{The bounds are evaluated in the wrong order.}
\choice{Nothing is wrong.  The equation is correct, as is.}
\end{multipleChoice}

\end{problem}}%}

\latexProblemContent{
\ifVerboseLocation This is Integration Concept Question 0003. \\ \fi
\begin{problem}

What is wrong with the following equation:

\[
\int_{\frac{1}{4} \, \pi}^{\frac{4}{3} \, \pi} {-9 \, \cot\left(x\right) \csc\left(x\right)}\;dx = {\frac{9}{\sin\left(x\right)}}\Bigg\vert_{\frac{1}{4} \, \pi}^{\frac{4}{3} \, \pi} = {-6 \, \sqrt{3} - 9 \, \sqrt{2}}
\]

\input{Integral-Concept-0003.HELP.tex}

\begin{multipleChoice}
\choice{The antiderivative is incorrect.}
\choice[correct]{The integrand is not defined over the entire interval.}
\choice{The bounds are evaluated in the wrong order.}
\choice{Nothing is wrong.  The equation is correct, as is.}
\end{multipleChoice}

\end{problem}}%}

\latexProblemContent{
\ifVerboseLocation This is Integration Concept Question 0003. \\ \fi
\begin{problem}

What is wrong with the following equation:

\[
\int_{\frac{2}{3} \, \pi}^{\frac{3}{2} \, \pi} {-7 \, \csc\left(x\right)^{2}}\;dx = {\frac{7}{\tan\left(x\right)}}\Bigg\vert_{\frac{2}{3} \, \pi}^{\frac{3}{2} \, \pi} = {\frac{7}{3} \, \sqrt{3}}
\]

\input{Integral-Concept-0003.HELP.tex}

\begin{multipleChoice}
\choice{The antiderivative is incorrect.}
\choice[correct]{The integrand is not defined over the entire interval.}
\choice{The bounds are evaluated in the wrong order.}
\choice{Nothing is wrong.  The equation is correct, as is.}
\end{multipleChoice}

\end{problem}}%}

\latexProblemContent{
\ifVerboseLocation This is Integration Concept Question 0003. \\ \fi
\begin{problem}

What is wrong with the following equation:

\[
\int_{\frac{1}{3} \, \pi}^{\frac{5}{4} \, \pi} {9 \, \csc\left(x\right)^{2}}\;dx = {-\frac{9}{\tan\left(x\right)}}\Bigg\vert_{\frac{1}{3} \, \pi}^{\frac{5}{4} \, \pi} = {3 \, \sqrt{3} - 9}
\]

\input{Integral-Concept-0003.HELP.tex}

\begin{multipleChoice}
\choice{The antiderivative is incorrect.}
\choice[correct]{The integrand is not defined over the entire interval.}
\choice{The bounds are evaluated in the wrong order.}
\choice{Nothing is wrong.  The equation is correct, as is.}
\end{multipleChoice}

\end{problem}}%}

\latexProblemContent{
\ifVerboseLocation This is Integration Concept Question 0003. \\ \fi
\begin{problem}

What is wrong with the following equation:

\[
\int_{\frac{1}{2} \, \pi}^{\frac{3}{2} \, \pi} {5 \, \cot\left(x\right) \csc\left(x\right)}\;dx = {-\frac{5}{\sin\left(x\right)}}\Bigg\vert_{\frac{1}{2} \, \pi}^{\frac{3}{2} \, \pi} = {10}
\]

\input{Integral-Concept-0003.HELP.tex}

\begin{multipleChoice}
\choice{The antiderivative is incorrect.}
\choice[correct]{The integrand is not defined over the entire interval.}
\choice{The bounds are evaluated in the wrong order.}
\choice{Nothing is wrong.  The equation is correct, as is.}
\end{multipleChoice}

\end{problem}}%}

\latexProblemContent{
\ifVerboseLocation This is Integration Concept Question 0003. \\ \fi
\begin{problem}

What is wrong with the following equation:

\[
\int_{\frac{3}{4} \, \pi}^{\frac{5}{3} \, \pi} {6 \, \cot\left(x\right) \csc\left(x\right)}\;dx = {-\frac{6}{\sin\left(x\right)}}\Bigg\vert_{\frac{3}{4} \, \pi}^{\frac{5}{3} \, \pi} = {4 \, \sqrt{3} + 6 \, \sqrt{2}}
\]

\input{Integral-Concept-0003.HELP.tex}

\begin{multipleChoice}
\choice{The antiderivative is incorrect.}
\choice[correct]{The integrand is not defined over the entire interval.}
\choice{The bounds are evaluated in the wrong order.}
\choice{Nothing is wrong.  The equation is correct, as is.}
\end{multipleChoice}

\end{problem}}%}

\latexProblemContent{
\ifVerboseLocation This is Integration Concept Question 0003. \\ \fi
\begin{problem}

What is wrong with the following equation:

\[
\int_{\frac{1}{2} \, \pi}^{\frac{4}{3} \, \pi} {10 \, \cot\left(x\right) \csc\left(x\right)}\;dx = {-\frac{10}{\sin\left(x\right)}}\Bigg\vert_{\frac{1}{2} \, \pi}^{\frac{4}{3} \, \pi} = {\frac{20}{3} \, \sqrt{3} + 10}
\]

\input{Integral-Concept-0003.HELP.tex}

\begin{multipleChoice}
\choice{The antiderivative is incorrect.}
\choice[correct]{The integrand is not defined over the entire interval.}
\choice{The bounds are evaluated in the wrong order.}
\choice{Nothing is wrong.  The equation is correct, as is.}
\end{multipleChoice}

\end{problem}}%}

\latexProblemContent{
\ifVerboseLocation This is Integration Concept Question 0003. \\ \fi
\begin{problem}

What is wrong with the following equation:

\[
\int_{\frac{1}{3} \, \pi}^{\frac{4}{3} \, \pi} {-9 \, \cot\left(x\right) \csc\left(x\right)}\;dx = {\frac{9}{\sin\left(x\right)}}\Bigg\vert_{\frac{1}{3} \, \pi}^{\frac{4}{3} \, \pi} = {-12 \, \sqrt{3}}
\]

\input{Integral-Concept-0003.HELP.tex}

\begin{multipleChoice}
\choice{The antiderivative is incorrect.}
\choice[correct]{The integrand is not defined over the entire interval.}
\choice{The bounds are evaluated in the wrong order.}
\choice{Nothing is wrong.  The equation is correct, as is.}
\end{multipleChoice}

\end{problem}}%}

\latexProblemContent{
\ifVerboseLocation This is Integration Concept Question 0003. \\ \fi
\begin{problem}

What is wrong with the following equation:

\[
\int_{\frac{2}{3} \, \pi}^{\frac{11}{6} \, \pi} {-9 \, \cot\left(x\right) \csc\left(x\right)}\;dx = {\frac{9}{\sin\left(x\right)}}\Bigg\vert_{\frac{2}{3} \, \pi}^{\frac{11}{6} \, \pi} = {-6 \, \sqrt{3} - 18}
\]

\input{Integral-Concept-0003.HELP.tex}

\begin{multipleChoice}
\choice{The antiderivative is incorrect.}
\choice[correct]{The integrand is not defined over the entire interval.}
\choice{The bounds are evaluated in the wrong order.}
\choice{Nothing is wrong.  The equation is correct, as is.}
\end{multipleChoice}

\end{problem}}%}

\latexProblemContent{
\ifVerboseLocation This is Integration Concept Question 0003. \\ \fi
\begin{problem}

What is wrong with the following equation:

\[
\int_{\frac{1}{2} \, \pi}^{\frac{11}{6} \, \pi} {6 \, \csc\left(x\right)^{2}}\;dx = {-\frac{6}{\tan\left(x\right)}}\Bigg\vert_{\frac{1}{2} \, \pi}^{\frac{11}{6} \, \pi} = {6 \, \sqrt{3}}
\]

\input{Integral-Concept-0003.HELP.tex}

\begin{multipleChoice}
\choice{The antiderivative is incorrect.}
\choice[correct]{The integrand is not defined over the entire interval.}
\choice{The bounds are evaluated in the wrong order.}
\choice{Nothing is wrong.  The equation is correct, as is.}
\end{multipleChoice}

\end{problem}}%}

\latexProblemContent{
\ifVerboseLocation This is Integration Concept Question 0003. \\ \fi
\begin{problem}

What is wrong with the following equation:

\[
\int_{\frac{1}{3} \, \pi}^{\frac{5}{3} \, \pi} {9 \, \csc\left(x\right)^{2}}\;dx = {-\frac{9}{\tan\left(x\right)}}\Bigg\vert_{\frac{1}{3} \, \pi}^{\frac{5}{3} \, \pi} = {6 \, \sqrt{3}}
\]

\input{Integral-Concept-0003.HELP.tex}

\begin{multipleChoice}
\choice{The antiderivative is incorrect.}
\choice[correct]{The integrand is not defined over the entire interval.}
\choice{The bounds are evaluated in the wrong order.}
\choice{Nothing is wrong.  The equation is correct, as is.}
\end{multipleChoice}

\end{problem}}%}

\latexProblemContent{
\ifVerboseLocation This is Integration Concept Question 0003. \\ \fi
\begin{problem}

What is wrong with the following equation:

\[
\int_{\frac{1}{4} \, \pi}^{\frac{7}{4} \, \pi} {-6 \, \cot\left(x\right) \csc\left(x\right)}\;dx = {\frac{6}{\sin\left(x\right)}}\Bigg\vert_{\frac{1}{4} \, \pi}^{\frac{7}{4} \, \pi} = {-12 \, \sqrt{2}}
\]

\input{Integral-Concept-0003.HELP.tex}

\begin{multipleChoice}
\choice{The antiderivative is incorrect.}
\choice[correct]{The integrand is not defined over the entire interval.}
\choice{The bounds are evaluated in the wrong order.}
\choice{Nothing is wrong.  The equation is correct, as is.}
\end{multipleChoice}

\end{problem}}%}

\latexProblemContent{
\ifVerboseLocation This is Integration Concept Question 0003. \\ \fi
\begin{problem}

What is wrong with the following equation:

\[
\int_{\frac{2}{3} \, \pi}^{\frac{4}{3} \, \pi} {5 \, \csc\left(x\right)^{2}}\;dx = {-\frac{5}{\tan\left(x\right)}}\Bigg\vert_{\frac{2}{3} \, \pi}^{\frac{4}{3} \, \pi} = {-\frac{10}{3} \, \sqrt{3}}
\]

\input{Integral-Concept-0003.HELP.tex}

\begin{multipleChoice}
\choice{The antiderivative is incorrect.}
\choice[correct]{The integrand is not defined over the entire interval.}
\choice{The bounds are evaluated in the wrong order.}
\choice{Nothing is wrong.  The equation is correct, as is.}
\end{multipleChoice}

\end{problem}}%}

\latexProblemContent{
\ifVerboseLocation This is Integration Concept Question 0003. \\ \fi
\begin{problem}

What is wrong with the following equation:

\[
\int_{\frac{2}{3} \, \pi}^{\frac{11}{6} \, \pi} {-3 \, \csc\left(x\right)^{2}}\;dx = {\frac{3}{\tan\left(x\right)}}\Bigg\vert_{\frac{2}{3} \, \pi}^{\frac{11}{6} \, \pi} = {-2 \, \sqrt{3}}
\]

\input{Integral-Concept-0003.HELP.tex}

\begin{multipleChoice}
\choice{The antiderivative is incorrect.}
\choice[correct]{The integrand is not defined over the entire interval.}
\choice{The bounds are evaluated in the wrong order.}
\choice{Nothing is wrong.  The equation is correct, as is.}
\end{multipleChoice}

\end{problem}}%}

\latexProblemContent{
\ifVerboseLocation This is Integration Concept Question 0003. \\ \fi
\begin{problem}

What is wrong with the following equation:

\[
\int_{\frac{1}{4} \, \pi}^{\frac{3}{2} \, \pi} {\cot\left(x\right) \csc\left(x\right)}\;dx = {-\frac{1}{\sin\left(x\right)}}\Bigg\vert_{\frac{1}{4} \, \pi}^{\frac{3}{2} \, \pi} = {\sqrt{2} + 1}
\]

\input{Integral-Concept-0003.HELP.tex}

\begin{multipleChoice}
\choice{The antiderivative is incorrect.}
\choice[correct]{The integrand is not defined over the entire interval.}
\choice{The bounds are evaluated in the wrong order.}
\choice{Nothing is wrong.  The equation is correct, as is.}
\end{multipleChoice}

\end{problem}}%}

\latexProblemContent{
\ifVerboseLocation This is Integration Concept Question 0003. \\ \fi
\begin{problem}

What is wrong with the following equation:

\[
\int_{\frac{3}{4} \, \pi}^{\frac{7}{4} \, \pi} {8 \, \csc\left(x\right)^{2}}\;dx = {-\frac{8}{\tan\left(x\right)}}\Bigg\vert_{\frac{3}{4} \, \pi}^{\frac{7}{4} \, \pi} = {0}
\]

\input{Integral-Concept-0003.HELP.tex}

\begin{multipleChoice}
\choice{The antiderivative is incorrect.}
\choice[correct]{The integrand is not defined over the entire interval.}
\choice{The bounds are evaluated in the wrong order.}
\choice{Nothing is wrong.  The equation is correct, as is.}
\end{multipleChoice}

\end{problem}}%}

\latexProblemContent{
\ifVerboseLocation This is Integration Concept Question 0003. \\ \fi
\begin{problem}

What is wrong with the following equation:

\[
\int_{\frac{1}{6} \, \pi}^{\frac{7}{4} \, \pi} {-4 \, \cot\left(x\right) \csc\left(x\right)}\;dx = {\frac{4}{\sin\left(x\right)}}\Bigg\vert_{\frac{1}{6} \, \pi}^{\frac{7}{4} \, \pi} = {-4 \, \sqrt{2} - 8}
\]

\input{Integral-Concept-0003.HELP.tex}

\begin{multipleChoice}
\choice{The antiderivative is incorrect.}
\choice[correct]{The integrand is not defined over the entire interval.}
\choice{The bounds are evaluated in the wrong order.}
\choice{Nothing is wrong.  The equation is correct, as is.}
\end{multipleChoice}

\end{problem}}%}

\latexProblemContent{
\ifVerboseLocation This is Integration Concept Question 0003. \\ \fi
\begin{problem}

What is wrong with the following equation:

\[
\int_{\frac{2}{3} \, \pi}^{\frac{5}{3} \, \pi} {-7 \, \csc\left(x\right)^{2}}\;dx = {\frac{7}{\tan\left(x\right)}}\Bigg\vert_{\frac{2}{3} \, \pi}^{\frac{5}{3} \, \pi} = {0}
\]

\input{Integral-Concept-0003.HELP.tex}

\begin{multipleChoice}
\choice{The antiderivative is incorrect.}
\choice[correct]{The integrand is not defined over the entire interval.}
\choice{The bounds are evaluated in the wrong order.}
\choice{Nothing is wrong.  The equation is correct, as is.}
\end{multipleChoice}

\end{problem}}%}

\latexProblemContent{
\ifVerboseLocation This is Integration Concept Question 0003. \\ \fi
\begin{problem}

What is wrong with the following equation:

\[
\int_{\frac{2}{3} \, \pi}^{\frac{4}{3} \, \pi} {4 \, \csc\left(x\right)^{2}}\;dx = {-\frac{4}{\tan\left(x\right)}}\Bigg\vert_{\frac{2}{3} \, \pi}^{\frac{4}{3} \, \pi} = {-\frac{8}{3} \, \sqrt{3}}
\]

\input{Integral-Concept-0003.HELP.tex}

\begin{multipleChoice}
\choice{The antiderivative is incorrect.}
\choice[correct]{The integrand is not defined over the entire interval.}
\choice{The bounds are evaluated in the wrong order.}
\choice{Nothing is wrong.  The equation is correct, as is.}
\end{multipleChoice}

\end{problem}}%}

\latexProblemContent{
\ifVerboseLocation This is Integration Concept Question 0003. \\ \fi
\begin{problem}

What is wrong with the following equation:

\[
\int_{\frac{5}{6} \, \pi}^{\frac{7}{6} \, \pi} {4 \, \csc\left(x\right)^{2}}\;dx = {-\frac{4}{\tan\left(x\right)}}\Bigg\vert_{\frac{5}{6} \, \pi}^{\frac{7}{6} \, \pi} = {-8 \, \sqrt{3}}
\]

\input{Integral-Concept-0003.HELP.tex}

\begin{multipleChoice}
\choice{The antiderivative is incorrect.}
\choice[correct]{The integrand is not defined over the entire interval.}
\choice{The bounds are evaluated in the wrong order.}
\choice{Nothing is wrong.  The equation is correct, as is.}
\end{multipleChoice}

\end{problem}}%}

\latexProblemContent{
\ifVerboseLocation This is Integration Concept Question 0003. \\ \fi
\begin{problem}

What is wrong with the following equation:

\[
\int_{\frac{5}{6} \, \pi}^{\frac{11}{6} \, \pi} {-5 \, \cot\left(x\right) \csc\left(x\right)}\;dx = {\frac{5}{\sin\left(x\right)}}\Bigg\vert_{\frac{5}{6} \, \pi}^{\frac{11}{6} \, \pi} = {-20}
\]

\input{Integral-Concept-0003.HELP.tex}

\begin{multipleChoice}
\choice{The antiderivative is incorrect.}
\choice[correct]{The integrand is not defined over the entire interval.}
\choice{The bounds are evaluated in the wrong order.}
\choice{Nothing is wrong.  The equation is correct, as is.}
\end{multipleChoice}

\end{problem}}%}

\latexProblemContent{
\ifVerboseLocation This is Integration Concept Question 0003. \\ \fi
\begin{problem}

What is wrong with the following equation:

\[
\int_{\frac{1}{6} \, \pi}^{\frac{11}{6} \, \pi} {2 \, \csc\left(x\right)^{2}}\;dx = {-\frac{2}{\tan\left(x\right)}}\Bigg\vert_{\frac{1}{6} \, \pi}^{\frac{11}{6} \, \pi} = {4 \, \sqrt{3}}
\]

\input{Integral-Concept-0003.HELP.tex}

\begin{multipleChoice}
\choice{The antiderivative is incorrect.}
\choice[correct]{The integrand is not defined over the entire interval.}
\choice{The bounds are evaluated in the wrong order.}
\choice{Nothing is wrong.  The equation is correct, as is.}
\end{multipleChoice}

\end{problem}}%}

\latexProblemContent{
\ifVerboseLocation This is Integration Concept Question 0003. \\ \fi
\begin{problem}

What is wrong with the following equation:

\[
\int_{\frac{1}{6} \, \pi}^{\frac{4}{3} \, \pi} {-10 \, \csc\left(x\right)^{2}}\;dx = {\frac{10}{\tan\left(x\right)}}\Bigg\vert_{\frac{1}{6} \, \pi}^{\frac{4}{3} \, \pi} = {-\frac{20}{3} \, \sqrt{3}}
\]

\input{Integral-Concept-0003.HELP.tex}

\begin{multipleChoice}
\choice{The antiderivative is incorrect.}
\choice[correct]{The integrand is not defined over the entire interval.}
\choice{The bounds are evaluated in the wrong order.}
\choice{Nothing is wrong.  The equation is correct, as is.}
\end{multipleChoice}

\end{problem}}%}

\latexProblemContent{
\ifVerboseLocation This is Integration Concept Question 0003. \\ \fi
\begin{problem}

What is wrong with the following equation:

\[
\int_{\frac{5}{6} \, \pi}^{\frac{5}{4} \, \pi} {8 \, \csc\left(x\right)^{2}}\;dx = {-\frac{8}{\tan\left(x\right)}}\Bigg\vert_{\frac{5}{6} \, \pi}^{\frac{5}{4} \, \pi} = {-8 \, \sqrt{3} - 8}
\]

\input{Integral-Concept-0003.HELP.tex}

\begin{multipleChoice}
\choice{The antiderivative is incorrect.}
\choice[correct]{The integrand is not defined over the entire interval.}
\choice{The bounds are evaluated in the wrong order.}
\choice{Nothing is wrong.  The equation is correct, as is.}
\end{multipleChoice}

\end{problem}}%}

\latexProblemContent{
\ifVerboseLocation This is Integration Concept Question 0003. \\ \fi
\begin{problem}

What is wrong with the following equation:

\[
\int_{\frac{2}{3} \, \pi}^{\frac{11}{6} \, \pi} {8 \, \csc\left(x\right)^{2}}\;dx = {-\frac{8}{\tan\left(x\right)}}\Bigg\vert_{\frac{2}{3} \, \pi}^{\frac{11}{6} \, \pi} = {\frac{16}{3} \, \sqrt{3}}
\]

\input{Integral-Concept-0003.HELP.tex}

\begin{multipleChoice}
\choice{The antiderivative is incorrect.}
\choice[correct]{The integrand is not defined over the entire interval.}
\choice{The bounds are evaluated in the wrong order.}
\choice{Nothing is wrong.  The equation is correct, as is.}
\end{multipleChoice}

\end{problem}}%}

\latexProblemContent{
\ifVerboseLocation This is Integration Concept Question 0003. \\ \fi
\begin{problem}

What is wrong with the following equation:

\[
\int_{\frac{1}{4} \, \pi}^{\frac{7}{4} \, \pi} {6 \, \csc\left(x\right)^{2}}\;dx = {-\frac{6}{\tan\left(x\right)}}\Bigg\vert_{\frac{1}{4} \, \pi}^{\frac{7}{4} \, \pi} = {12}
\]

\input{Integral-Concept-0003.HELP.tex}

\begin{multipleChoice}
\choice{The antiderivative is incorrect.}
\choice[correct]{The integrand is not defined over the entire interval.}
\choice{The bounds are evaluated in the wrong order.}
\choice{Nothing is wrong.  The equation is correct, as is.}
\end{multipleChoice}

\end{problem}}%}

\latexProblemContent{
\ifVerboseLocation This is Integration Concept Question 0003. \\ \fi
\begin{problem}

What is wrong with the following equation:

\[
\int_{\frac{1}{6} \, \pi}^{\frac{4}{3} \, \pi} {-5 \, \cot\left(x\right) \csc\left(x\right)}\;dx = {\frac{5}{\sin\left(x\right)}}\Bigg\vert_{\frac{1}{6} \, \pi}^{\frac{4}{3} \, \pi} = {-\frac{10}{3} \, \sqrt{3} - 10}
\]

\input{Integral-Concept-0003.HELP.tex}

\begin{multipleChoice}
\choice{The antiderivative is incorrect.}
\choice[correct]{The integrand is not defined over the entire interval.}
\choice{The bounds are evaluated in the wrong order.}
\choice{Nothing is wrong.  The equation is correct, as is.}
\end{multipleChoice}

\end{problem}}%}

\latexProblemContent{
\ifVerboseLocation This is Integration Concept Question 0003. \\ \fi
\begin{problem}

What is wrong with the following equation:

\[
\int_{\frac{2}{3} \, \pi}^{\frac{7}{4} \, \pi} {-9 \, \csc\left(x\right)^{2}}\;dx = {\frac{9}{\tan\left(x\right)}}\Bigg\vert_{\frac{2}{3} \, \pi}^{\frac{7}{4} \, \pi} = {3 \, \sqrt{3} - 9}
\]

\input{Integral-Concept-0003.HELP.tex}

\begin{multipleChoice}
\choice{The antiderivative is incorrect.}
\choice[correct]{The integrand is not defined over the entire interval.}
\choice{The bounds are evaluated in the wrong order.}
\choice{Nothing is wrong.  The equation is correct, as is.}
\end{multipleChoice}

\end{problem}}%}

\latexProblemContent{
\ifVerboseLocation This is Integration Concept Question 0003. \\ \fi
\begin{problem}

What is wrong with the following equation:

\[
\int_{\frac{1}{4} \, \pi}^{\frac{7}{6} \, \pi} {-7 \, \cot\left(x\right) \csc\left(x\right)}\;dx = {\frac{7}{\sin\left(x\right)}}\Bigg\vert_{\frac{1}{4} \, \pi}^{\frac{7}{6} \, \pi} = {-7 \, \sqrt{2} - 14}
\]

\input{Integral-Concept-0003.HELP.tex}

\begin{multipleChoice}
\choice{The antiderivative is incorrect.}
\choice[correct]{The integrand is not defined over the entire interval.}
\choice{The bounds are evaluated in the wrong order.}
\choice{Nothing is wrong.  The equation is correct, as is.}
\end{multipleChoice}

\end{problem}}%}

\latexProblemContent{
\ifVerboseLocation This is Integration Concept Question 0003. \\ \fi
\begin{problem}

What is wrong with the following equation:

\[
\int_{\frac{1}{4} \, \pi}^{\frac{7}{6} \, \pi} {-\cot\left(x\right) \csc\left(x\right)}\;dx = {\frac{1}{\sin\left(x\right)}}\Bigg\vert_{\frac{1}{4} \, \pi}^{\frac{7}{6} \, \pi} = {-\sqrt{2} - 2}
\]

\input{Integral-Concept-0003.HELP.tex}

\begin{multipleChoice}
\choice{The antiderivative is incorrect.}
\choice[correct]{The integrand is not defined over the entire interval.}
\choice{The bounds are evaluated in the wrong order.}
\choice{Nothing is wrong.  The equation is correct, as is.}
\end{multipleChoice}

\end{problem}}%}

\latexProblemContent{
\ifVerboseLocation This is Integration Concept Question 0003. \\ \fi
\begin{problem}

What is wrong with the following equation:

\[
\int_{\frac{1}{2} \, \pi}^{\frac{7}{6} \, \pi} {-6 \, \csc\left(x\right)^{2}}\;dx = {\frac{6}{\tan\left(x\right)}}\Bigg\vert_{\frac{1}{2} \, \pi}^{\frac{7}{6} \, \pi} = {6 \, \sqrt{3}}
\]

\input{Integral-Concept-0003.HELP.tex}

\begin{multipleChoice}
\choice{The antiderivative is incorrect.}
\choice[correct]{The integrand is not defined over the entire interval.}
\choice{The bounds are evaluated in the wrong order.}
\choice{Nothing is wrong.  The equation is correct, as is.}
\end{multipleChoice}

\end{problem}}%}

\latexProblemContent{
\ifVerboseLocation This is Integration Concept Question 0003. \\ \fi
\begin{problem}

What is wrong with the following equation:

\[
\int_{\frac{1}{3} \, \pi}^{\frac{7}{4} \, \pi} {9 \, \cot\left(x\right) \csc\left(x\right)}\;dx = {-\frac{9}{\sin\left(x\right)}}\Bigg\vert_{\frac{1}{3} \, \pi}^{\frac{7}{4} \, \pi} = {6 \, \sqrt{3} + 9 \, \sqrt{2}}
\]

\input{Integral-Concept-0003.HELP.tex}

\begin{multipleChoice}
\choice{The antiderivative is incorrect.}
\choice[correct]{The integrand is not defined over the entire interval.}
\choice{The bounds are evaluated in the wrong order.}
\choice{Nothing is wrong.  The equation is correct, as is.}
\end{multipleChoice}

\end{problem}}%}

\latexProblemContent{
\ifVerboseLocation This is Integration Concept Question 0003. \\ \fi
\begin{problem}

What is wrong with the following equation:

\[
\int_{\frac{1}{6} \, \pi}^{\frac{5}{4} \, \pi} {-\csc\left(x\right)^{2}}\;dx = {\frac{1}{\tan\left(x\right)}}\Bigg\vert_{\frac{1}{6} \, \pi}^{\frac{5}{4} \, \pi} = {-\sqrt{3} + 1}
\]

\input{Integral-Concept-0003.HELP.tex}

\begin{multipleChoice}
\choice{The antiderivative is incorrect.}
\choice[correct]{The integrand is not defined over the entire interval.}
\choice{The bounds are evaluated in the wrong order.}
\choice{Nothing is wrong.  The equation is correct, as is.}
\end{multipleChoice}

\end{problem}}%}

\latexProblemContent{
\ifVerboseLocation This is Integration Concept Question 0003. \\ \fi
\begin{problem}

What is wrong with the following equation:

\[
\int_{\frac{1}{2} \, \pi}^{\frac{7}{6} \, \pi} {-8 \, \csc\left(x\right)^{2}}\;dx = {\frac{8}{\tan\left(x\right)}}\Bigg\vert_{\frac{1}{2} \, \pi}^{\frac{7}{6} \, \pi} = {8 \, \sqrt{3}}
\]

\input{Integral-Concept-0003.HELP.tex}

\begin{multipleChoice}
\choice{The antiderivative is incorrect.}
\choice[correct]{The integrand is not defined over the entire interval.}
\choice{The bounds are evaluated in the wrong order.}
\choice{Nothing is wrong.  The equation is correct, as is.}
\end{multipleChoice}

\end{problem}}%}

\latexProblemContent{
\ifVerboseLocation This is Integration Concept Question 0003. \\ \fi
\begin{problem}

What is wrong with the following equation:

\[
\int_{\frac{3}{4} \, \pi}^{\frac{7}{6} \, \pi} {\cot\left(x\right) \csc\left(x\right)}\;dx = {-\frac{1}{\sin\left(x\right)}}\Bigg\vert_{\frac{3}{4} \, \pi}^{\frac{7}{6} \, \pi} = {\sqrt{2} + 2}
\]

\input{Integral-Concept-0003.HELP.tex}

\begin{multipleChoice}
\choice{The antiderivative is incorrect.}
\choice[correct]{The integrand is not defined over the entire interval.}
\choice{The bounds are evaluated in the wrong order.}
\choice{Nothing is wrong.  The equation is correct, as is.}
\end{multipleChoice}

\end{problem}}%}

\latexProblemContent{
\ifVerboseLocation This is Integration Concept Question 0003. \\ \fi
\begin{problem}

What is wrong with the following equation:

\[
\int_{\frac{1}{4} \, \pi}^{\frac{5}{4} \, \pi} {-\csc\left(x\right)^{2}}\;dx = {\frac{1}{\tan\left(x\right)}}\Bigg\vert_{\frac{1}{4} \, \pi}^{\frac{5}{4} \, \pi} = {0}
\]

\input{Integral-Concept-0003.HELP.tex}

\begin{multipleChoice}
\choice{The antiderivative is incorrect.}
\choice[correct]{The integrand is not defined over the entire interval.}
\choice{The bounds are evaluated in the wrong order.}
\choice{Nothing is wrong.  The equation is correct, as is.}
\end{multipleChoice}

\end{problem}}%}

\latexProblemContent{
\ifVerboseLocation This is Integration Concept Question 0003. \\ \fi
\begin{problem}

What is wrong with the following equation:

\[
\int_{\frac{2}{3} \, \pi}^{\frac{11}{6} \, \pi} {9 \, \csc\left(x\right)^{2}}\;dx = {-\frac{9}{\tan\left(x\right)}}\Bigg\vert_{\frac{2}{3} \, \pi}^{\frac{11}{6} \, \pi} = {6 \, \sqrt{3}}
\]

\input{Integral-Concept-0003.HELP.tex}

\begin{multipleChoice}
\choice{The antiderivative is incorrect.}
\choice[correct]{The integrand is not defined over the entire interval.}
\choice{The bounds are evaluated in the wrong order.}
\choice{Nothing is wrong.  The equation is correct, as is.}
\end{multipleChoice}

\end{problem}}%}

\latexProblemContent{
\ifVerboseLocation This is Integration Concept Question 0003. \\ \fi
\begin{problem}

What is wrong with the following equation:

\[
\int_{\frac{1}{2} \, \pi}^{\frac{4}{3} \, \pi} {-8 \, \cot\left(x\right) \csc\left(x\right)}\;dx = {\frac{8}{\sin\left(x\right)}}\Bigg\vert_{\frac{1}{2} \, \pi}^{\frac{4}{3} \, \pi} = {-\frac{16}{3} \, \sqrt{3} - 8}
\]

\input{Integral-Concept-0003.HELP.tex}

\begin{multipleChoice}
\choice{The antiderivative is incorrect.}
\choice[correct]{The integrand is not defined over the entire interval.}
\choice{The bounds are evaluated in the wrong order.}
\choice{Nothing is wrong.  The equation is correct, as is.}
\end{multipleChoice}

\end{problem}}%}

\latexProblemContent{
\ifVerboseLocation This is Integration Concept Question 0003. \\ \fi
\begin{problem}

What is wrong with the following equation:

\[
\int_{\frac{1}{2} \, \pi}^{\frac{5}{4} \, \pi} {8 \, \csc\left(x\right)^{2}}\;dx = {-\frac{8}{\tan\left(x\right)}}\Bigg\vert_{\frac{1}{2} \, \pi}^{\frac{5}{4} \, \pi} = {-8}
\]

\input{Integral-Concept-0003.HELP.tex}

\begin{multipleChoice}
\choice{The antiderivative is incorrect.}
\choice[correct]{The integrand is not defined over the entire interval.}
\choice{The bounds are evaluated in the wrong order.}
\choice{Nothing is wrong.  The equation is correct, as is.}
\end{multipleChoice}

\end{problem}}%}

\latexProblemContent{
\ifVerboseLocation This is Integration Concept Question 0003. \\ \fi
\begin{problem}

What is wrong with the following equation:

\[
\int_{\frac{3}{4} \, \pi}^{\frac{4}{3} \, \pi} {10 \, \csc\left(x\right)^{2}}\;dx = {-\frac{10}{\tan\left(x\right)}}\Bigg\vert_{\frac{3}{4} \, \pi}^{\frac{4}{3} \, \pi} = {-\frac{10}{3} \, \sqrt{3} - 10}
\]

\input{Integral-Concept-0003.HELP.tex}

\begin{multipleChoice}
\choice{The antiderivative is incorrect.}
\choice[correct]{The integrand is not defined over the entire interval.}
\choice{The bounds are evaluated in the wrong order.}
\choice{Nothing is wrong.  The equation is correct, as is.}
\end{multipleChoice}

\end{problem}}%}

\latexProblemContent{
\ifVerboseLocation This is Integration Concept Question 0003. \\ \fi
\begin{problem}

What is wrong with the following equation:

\[
\int_{\frac{1}{4} \, \pi}^{\frac{4}{3} \, \pi} {7 \, \cot\left(x\right) \csc\left(x\right)}\;dx = {-\frac{7}{\sin\left(x\right)}}\Bigg\vert_{\frac{1}{4} \, \pi}^{\frac{4}{3} \, \pi} = {\frac{14}{3} \, \sqrt{3} + 7 \, \sqrt{2}}
\]

\input{Integral-Concept-0003.HELP.tex}

\begin{multipleChoice}
\choice{The antiderivative is incorrect.}
\choice[correct]{The integrand is not defined over the entire interval.}
\choice{The bounds are evaluated in the wrong order.}
\choice{Nothing is wrong.  The equation is correct, as is.}
\end{multipleChoice}

\end{problem}}%}

\latexProblemContent{
\ifVerboseLocation This is Integration Concept Question 0003. \\ \fi
\begin{problem}

What is wrong with the following equation:

\[
\int_{\frac{3}{4} \, \pi}^{\frac{5}{4} \, \pi} {-6 \, \cot\left(x\right) \csc\left(x\right)}\;dx = {\frac{6}{\sin\left(x\right)}}\Bigg\vert_{\frac{3}{4} \, \pi}^{\frac{5}{4} \, \pi} = {-12 \, \sqrt{2}}
\]

\input{Integral-Concept-0003.HELP.tex}

\begin{multipleChoice}
\choice{The antiderivative is incorrect.}
\choice[correct]{The integrand is not defined over the entire interval.}
\choice{The bounds are evaluated in the wrong order.}
\choice{Nothing is wrong.  The equation is correct, as is.}
\end{multipleChoice}

\end{problem}}%}

\latexProblemContent{
\ifVerboseLocation This is Integration Concept Question 0003. \\ \fi
\begin{problem}

What is wrong with the following equation:

\[
\int_{\frac{1}{2} \, \pi}^{\frac{5}{4} \, \pi} {-2 \, \cot\left(x\right) \csc\left(x\right)}\;dx = {\frac{2}{\sin\left(x\right)}}\Bigg\vert_{\frac{1}{2} \, \pi}^{\frac{5}{4} \, \pi} = {-2 \, \sqrt{2} - 2}
\]

\input{Integral-Concept-0003.HELP.tex}

\begin{multipleChoice}
\choice{The antiderivative is incorrect.}
\choice[correct]{The integrand is not defined over the entire interval.}
\choice{The bounds are evaluated in the wrong order.}
\choice{Nothing is wrong.  The equation is correct, as is.}
\end{multipleChoice}

\end{problem}}%}

\latexProblemContent{
\ifVerboseLocation This is Integration Concept Question 0003. \\ \fi
\begin{problem}

What is wrong with the following equation:

\[
\int_{\frac{1}{3} \, \pi}^{\frac{11}{6} \, \pi} {-9 \, \csc\left(x\right)^{2}}\;dx = {\frac{9}{\tan\left(x\right)}}\Bigg\vert_{\frac{1}{3} \, \pi}^{\frac{11}{6} \, \pi} = {-12 \, \sqrt{3}}
\]

\input{Integral-Concept-0003.HELP.tex}

\begin{multipleChoice}
\choice{The antiderivative is incorrect.}
\choice[correct]{The integrand is not defined over the entire interval.}
\choice{The bounds are evaluated in the wrong order.}
\choice{Nothing is wrong.  The equation is correct, as is.}
\end{multipleChoice}

\end{problem}}%}

\latexProblemContent{
\ifVerboseLocation This is Integration Concept Question 0003. \\ \fi
\begin{problem}

What is wrong with the following equation:

\[
\int_{\frac{1}{3} \, \pi}^{\frac{7}{6} \, \pi} {-2 \, \csc\left(x\right)^{2}}\;dx = {\frac{2}{\tan\left(x\right)}}\Bigg\vert_{\frac{1}{3} \, \pi}^{\frac{7}{6} \, \pi} = {\frac{4}{3} \, \sqrt{3}}
\]

\input{Integral-Concept-0003.HELP.tex}

\begin{multipleChoice}
\choice{The antiderivative is incorrect.}
\choice[correct]{The integrand is not defined over the entire interval.}
\choice{The bounds are evaluated in the wrong order.}
\choice{Nothing is wrong.  The equation is correct, as is.}
\end{multipleChoice}

\end{problem}}%}

\latexProblemContent{
\ifVerboseLocation This is Integration Concept Question 0003. \\ \fi
\begin{problem}

What is wrong with the following equation:

\[
\int_{\frac{1}{4} \, \pi}^{\frac{7}{6} \, \pi} {10 \, \cot\left(x\right) \csc\left(x\right)}\;dx = {-\frac{10}{\sin\left(x\right)}}\Bigg\vert_{\frac{1}{4} \, \pi}^{\frac{7}{6} \, \pi} = {10 \, \sqrt{2} + 20}
\]

\input{Integral-Concept-0003.HELP.tex}

\begin{multipleChoice}
\choice{The antiderivative is incorrect.}
\choice[correct]{The integrand is not defined over the entire interval.}
\choice{The bounds are evaluated in the wrong order.}
\choice{Nothing is wrong.  The equation is correct, as is.}
\end{multipleChoice}

\end{problem}}%}

\latexProblemContent{
\ifVerboseLocation This is Integration Concept Question 0003. \\ \fi
\begin{problem}

What is wrong with the following equation:

\[
\int_{\frac{1}{4} \, \pi}^{\frac{3}{2} \, \pi} {5 \, \cot\left(x\right) \csc\left(x\right)}\;dx = {-\frac{5}{\sin\left(x\right)}}\Bigg\vert_{\frac{1}{4} \, \pi}^{\frac{3}{2} \, \pi} = {5 \, \sqrt{2} + 5}
\]

\input{Integral-Concept-0003.HELP.tex}

\begin{multipleChoice}
\choice{The antiderivative is incorrect.}
\choice[correct]{The integrand is not defined over the entire interval.}
\choice{The bounds are evaluated in the wrong order.}
\choice{Nothing is wrong.  The equation is correct, as is.}
\end{multipleChoice}

\end{problem}}%}

\latexProblemContent{
\ifVerboseLocation This is Integration Concept Question 0003. \\ \fi
\begin{problem}

What is wrong with the following equation:

\[
\int_{\frac{1}{2} \, \pi}^{\frac{5}{4} \, \pi} {-10 \, \csc\left(x\right)^{2}}\;dx = {\frac{10}{\tan\left(x\right)}}\Bigg\vert_{\frac{1}{2} \, \pi}^{\frac{5}{4} \, \pi} = {10}
\]

\input{Integral-Concept-0003.HELP.tex}

\begin{multipleChoice}
\choice{The antiderivative is incorrect.}
\choice[correct]{The integrand is not defined over the entire interval.}
\choice{The bounds are evaluated in the wrong order.}
\choice{Nothing is wrong.  The equation is correct, as is.}
\end{multipleChoice}

\end{problem}}%}

\latexProblemContent{
\ifVerboseLocation This is Integration Concept Question 0003. \\ \fi
\begin{problem}

What is wrong with the following equation:

\[
\int_{\frac{1}{3} \, \pi}^{\frac{5}{4} \, \pi} {-4 \, \csc\left(x\right)^{2}}\;dx = {\frac{4}{\tan\left(x\right)}}\Bigg\vert_{\frac{1}{3} \, \pi}^{\frac{5}{4} \, \pi} = {-\frac{4}{3} \, \sqrt{3} + 4}
\]

\input{Integral-Concept-0003.HELP.tex}

\begin{multipleChoice}
\choice{The antiderivative is incorrect.}
\choice[correct]{The integrand is not defined over the entire interval.}
\choice{The bounds are evaluated in the wrong order.}
\choice{Nothing is wrong.  The equation is correct, as is.}
\end{multipleChoice}

\end{problem}}%}

\latexProblemContent{
\ifVerboseLocation This is Integration Concept Question 0003. \\ \fi
\begin{problem}

What is wrong with the following equation:

\[
\int_{\frac{1}{4} \, \pi}^{\frac{5}{4} \, \pi} {-7 \, \csc\left(x\right)^{2}}\;dx = {\frac{7}{\tan\left(x\right)}}\Bigg\vert_{\frac{1}{4} \, \pi}^{\frac{5}{4} \, \pi} = {0}
\]

\input{Integral-Concept-0003.HELP.tex}

\begin{multipleChoice}
\choice{The antiderivative is incorrect.}
\choice[correct]{The integrand is not defined over the entire interval.}
\choice{The bounds are evaluated in the wrong order.}
\choice{Nothing is wrong.  The equation is correct, as is.}
\end{multipleChoice}

\end{problem}}%}

\latexProblemContent{
\ifVerboseLocation This is Integration Concept Question 0003. \\ \fi
\begin{problem}

What is wrong with the following equation:

\[
\int_{\frac{1}{4} \, \pi}^{\frac{7}{4} \, \pi} {-5 \, \cot\left(x\right) \csc\left(x\right)}\;dx = {\frac{5}{\sin\left(x\right)}}\Bigg\vert_{\frac{1}{4} \, \pi}^{\frac{7}{4} \, \pi} = {-10 \, \sqrt{2}}
\]

\input{Integral-Concept-0003.HELP.tex}

\begin{multipleChoice}
\choice{The antiderivative is incorrect.}
\choice[correct]{The integrand is not defined over the entire interval.}
\choice{The bounds are evaluated in the wrong order.}
\choice{Nothing is wrong.  The equation is correct, as is.}
\end{multipleChoice}

\end{problem}}%}

\latexProblemContent{
\ifVerboseLocation This is Integration Concept Question 0003. \\ \fi
\begin{problem}

What is wrong with the following equation:

\[
\int_{\frac{1}{6} \, \pi}^{\frac{5}{4} \, \pi} {-\cot\left(x\right) \csc\left(x\right)}\;dx = {\frac{1}{\sin\left(x\right)}}\Bigg\vert_{\frac{1}{6} \, \pi}^{\frac{5}{4} \, \pi} = {-\sqrt{2} - 2}
\]

\input{Integral-Concept-0003.HELP.tex}

\begin{multipleChoice}
\choice{The antiderivative is incorrect.}
\choice[correct]{The integrand is not defined over the entire interval.}
\choice{The bounds are evaluated in the wrong order.}
\choice{Nothing is wrong.  The equation is correct, as is.}
\end{multipleChoice}

\end{problem}}%}

\latexProblemContent{
\ifVerboseLocation This is Integration Concept Question 0003. \\ \fi
\begin{problem}

What is wrong with the following equation:

\[
\int_{\frac{3}{4} \, \pi}^{\frac{11}{6} \, \pi} {4 \, \csc\left(x\right)^{2}}\;dx = {-\frac{4}{\tan\left(x\right)}}\Bigg\vert_{\frac{3}{4} \, \pi}^{\frac{11}{6} \, \pi} = {4 \, \sqrt{3} - 4}
\]

\input{Integral-Concept-0003.HELP.tex}

\begin{multipleChoice}
\choice{The antiderivative is incorrect.}
\choice[correct]{The integrand is not defined over the entire interval.}
\choice{The bounds are evaluated in the wrong order.}
\choice{Nothing is wrong.  The equation is correct, as is.}
\end{multipleChoice}

\end{problem}}%}

\latexProblemContent{
\ifVerboseLocation This is Integration Concept Question 0003. \\ \fi
\begin{problem}

What is wrong with the following equation:

\[
\int_{\frac{3}{4} \, \pi}^{\frac{11}{6} \, \pi} {-3 \, \cot\left(x\right) \csc\left(x\right)}\;dx = {\frac{3}{\sin\left(x\right)}}\Bigg\vert_{\frac{3}{4} \, \pi}^{\frac{11}{6} \, \pi} = {-3 \, \sqrt{2} - 6}
\]

\input{Integral-Concept-0003.HELP.tex}

\begin{multipleChoice}
\choice{The antiderivative is incorrect.}
\choice[correct]{The integrand is not defined over the entire interval.}
\choice{The bounds are evaluated in the wrong order.}
\choice{Nothing is wrong.  The equation is correct, as is.}
\end{multipleChoice}

\end{problem}}%}

\latexProblemContent{
\ifVerboseLocation This is Integration Concept Question 0003. \\ \fi
\begin{problem}

What is wrong with the following equation:

\[
\int_{\frac{1}{4} \, \pi}^{\frac{7}{4} \, \pi} {-10 \, \cot\left(x\right) \csc\left(x\right)}\;dx = {\frac{10}{\sin\left(x\right)}}\Bigg\vert_{\frac{1}{4} \, \pi}^{\frac{7}{4} \, \pi} = {-20 \, \sqrt{2}}
\]

\input{Integral-Concept-0003.HELP.tex}

\begin{multipleChoice}
\choice{The antiderivative is incorrect.}
\choice[correct]{The integrand is not defined over the entire interval.}
\choice{The bounds are evaluated in the wrong order.}
\choice{Nothing is wrong.  The equation is correct, as is.}
\end{multipleChoice}

\end{problem}}%}

\latexProblemContent{
\ifVerboseLocation This is Integration Concept Question 0003. \\ \fi
\begin{problem}

What is wrong with the following equation:

\[
\int_{\frac{5}{6} \, \pi}^{\frac{5}{4} \, \pi} {7 \, \cot\left(x\right) \csc\left(x\right)}\;dx = {-\frac{7}{\sin\left(x\right)}}\Bigg\vert_{\frac{5}{6} \, \pi}^{\frac{5}{4} \, \pi} = {7 \, \sqrt{2} + 14}
\]

\input{Integral-Concept-0003.HELP.tex}

\begin{multipleChoice}
\choice{The antiderivative is incorrect.}
\choice[correct]{The integrand is not defined over the entire interval.}
\choice{The bounds are evaluated in the wrong order.}
\choice{Nothing is wrong.  The equation is correct, as is.}
\end{multipleChoice}

\end{problem}}%}

\latexProblemContent{
\ifVerboseLocation This is Integration Concept Question 0003. \\ \fi
\begin{problem}

What is wrong with the following equation:

\[
\int_{\frac{1}{6} \, \pi}^{\frac{7}{6} \, \pi} {-8 \, \csc\left(x\right)^{2}}\;dx = {\frac{8}{\tan\left(x\right)}}\Bigg\vert_{\frac{1}{6} \, \pi}^{\frac{7}{6} \, \pi} = {0}
\]

\input{Integral-Concept-0003.HELP.tex}

\begin{multipleChoice}
\choice{The antiderivative is incorrect.}
\choice[correct]{The integrand is not defined over the entire interval.}
\choice{The bounds are evaluated in the wrong order.}
\choice{Nothing is wrong.  The equation is correct, as is.}
\end{multipleChoice}

\end{problem}}%}

\latexProblemContent{
\ifVerboseLocation This is Integration Concept Question 0003. \\ \fi
\begin{problem}

What is wrong with the following equation:

\[
\int_{\frac{5}{6} \, \pi}^{\frac{7}{4} \, \pi} {2 \, \csc\left(x\right)^{2}}\;dx = {-\frac{2}{\tan\left(x\right)}}\Bigg\vert_{\frac{5}{6} \, \pi}^{\frac{7}{4} \, \pi} = {-2 \, \sqrt{3} + 2}
\]

\input{Integral-Concept-0003.HELP.tex}

\begin{multipleChoice}
\choice{The antiderivative is incorrect.}
\choice[correct]{The integrand is not defined over the entire interval.}
\choice{The bounds are evaluated in the wrong order.}
\choice{Nothing is wrong.  The equation is correct, as is.}
\end{multipleChoice}

\end{problem}}%}

\latexProblemContent{
\ifVerboseLocation This is Integration Concept Question 0003. \\ \fi
\begin{problem}

What is wrong with the following equation:

\[
\int_{\frac{1}{4} \, \pi}^{\frac{11}{6} \, \pi} {8 \, \csc\left(x\right)^{2}}\;dx = {-\frac{8}{\tan\left(x\right)}}\Bigg\vert_{\frac{1}{4} \, \pi}^{\frac{11}{6} \, \pi} = {8 \, \sqrt{3} + 8}
\]

\input{Integral-Concept-0003.HELP.tex}

\begin{multipleChoice}
\choice{The antiderivative is incorrect.}
\choice[correct]{The integrand is not defined over the entire interval.}
\choice{The bounds are evaluated in the wrong order.}
\choice{Nothing is wrong.  The equation is correct, as is.}
\end{multipleChoice}

\end{problem}}%}

\latexProblemContent{
\ifVerboseLocation This is Integration Concept Question 0003. \\ \fi
\begin{problem}

What is wrong with the following equation:

\[
\int_{\frac{5}{6} \, \pi}^{\frac{4}{3} \, \pi} {8 \, \cot\left(x\right) \csc\left(x\right)}\;dx = {-\frac{8}{\sin\left(x\right)}}\Bigg\vert_{\frac{5}{6} \, \pi}^{\frac{4}{3} \, \pi} = {\frac{16}{3} \, \sqrt{3} + 16}
\]

\input{Integral-Concept-0003.HELP.tex}

\begin{multipleChoice}
\choice{The antiderivative is incorrect.}
\choice[correct]{The integrand is not defined over the entire interval.}
\choice{The bounds are evaluated in the wrong order.}
\choice{Nothing is wrong.  The equation is correct, as is.}
\end{multipleChoice}

\end{problem}}%}

\latexProblemContent{
\ifVerboseLocation This is Integration Concept Question 0003. \\ \fi
\begin{problem}

What is wrong with the following equation:

\[
\int_{\frac{2}{3} \, \pi}^{\frac{3}{2} \, \pi} {-6 \, \cot\left(x\right) \csc\left(x\right)}\;dx = {\frac{6}{\sin\left(x\right)}}\Bigg\vert_{\frac{2}{3} \, \pi}^{\frac{3}{2} \, \pi} = {-4 \, \sqrt{3} - 6}
\]

\input{Integral-Concept-0003.HELP.tex}

\begin{multipleChoice}
\choice{The antiderivative is incorrect.}
\choice[correct]{The integrand is not defined over the entire interval.}
\choice{The bounds are evaluated in the wrong order.}
\choice{Nothing is wrong.  The equation is correct, as is.}
\end{multipleChoice}

\end{problem}}%}

\latexProblemContent{
\ifVerboseLocation This is Integration Concept Question 0003. \\ \fi
\begin{problem}

What is wrong with the following equation:

\[
\int_{\frac{1}{6} \, \pi}^{\frac{4}{3} \, \pi} {4 \, \cot\left(x\right) \csc\left(x\right)}\;dx = {-\frac{4}{\sin\left(x\right)}}\Bigg\vert_{\frac{1}{6} \, \pi}^{\frac{4}{3} \, \pi} = {\frac{8}{3} \, \sqrt{3} + 8}
\]

\input{Integral-Concept-0003.HELP.tex}

\begin{multipleChoice}
\choice{The antiderivative is incorrect.}
\choice[correct]{The integrand is not defined over the entire interval.}
\choice{The bounds are evaluated in the wrong order.}
\choice{Nothing is wrong.  The equation is correct, as is.}
\end{multipleChoice}

\end{problem}}%}

\latexProblemContent{
\ifVerboseLocation This is Integration Concept Question 0003. \\ \fi
\begin{problem}

What is wrong with the following equation:

\[
\int_{\frac{1}{2} \, \pi}^{\frac{7}{4} \, \pi} {-8 \, \cot\left(x\right) \csc\left(x\right)}\;dx = {\frac{8}{\sin\left(x\right)}}\Bigg\vert_{\frac{1}{2} \, \pi}^{\frac{7}{4} \, \pi} = {-8 \, \sqrt{2} - 8}
\]

\input{Integral-Concept-0003.HELP.tex}

\begin{multipleChoice}
\choice{The antiderivative is incorrect.}
\choice[correct]{The integrand is not defined over the entire interval.}
\choice{The bounds are evaluated in the wrong order.}
\choice{Nothing is wrong.  The equation is correct, as is.}
\end{multipleChoice}

\end{problem}}%}

\latexProblemContent{
\ifVerboseLocation This is Integration Concept Question 0003. \\ \fi
\begin{problem}

What is wrong with the following equation:

\[
\int_{\frac{1}{3} \, \pi}^{\frac{7}{4} \, \pi} {-3 \, \cot\left(x\right) \csc\left(x\right)}\;dx = {\frac{3}{\sin\left(x\right)}}\Bigg\vert_{\frac{1}{3} \, \pi}^{\frac{7}{4} \, \pi} = {-2 \, \sqrt{3} - 3 \, \sqrt{2}}
\]

\input{Integral-Concept-0003.HELP.tex}

\begin{multipleChoice}
\choice{The antiderivative is incorrect.}
\choice[correct]{The integrand is not defined over the entire interval.}
\choice{The bounds are evaluated in the wrong order.}
\choice{Nothing is wrong.  The equation is correct, as is.}
\end{multipleChoice}

\end{problem}}%}

\latexProblemContent{
\ifVerboseLocation This is Integration Concept Question 0003. \\ \fi
\begin{problem}

What is wrong with the following equation:

\[
\int_{\frac{2}{3} \, \pi}^{\frac{5}{3} \, \pi} {-2 \, \cot\left(x\right) \csc\left(x\right)}\;dx = {\frac{2}{\sin\left(x\right)}}\Bigg\vert_{\frac{2}{3} \, \pi}^{\frac{5}{3} \, \pi} = {-\frac{8}{3} \, \sqrt{3}}
\]

\input{Integral-Concept-0003.HELP.tex}

\begin{multipleChoice}
\choice{The antiderivative is incorrect.}
\choice[correct]{The integrand is not defined over the entire interval.}
\choice{The bounds are evaluated in the wrong order.}
\choice{Nothing is wrong.  The equation is correct, as is.}
\end{multipleChoice}

\end{problem}}%}

\latexProblemContent{
\ifVerboseLocation This is Integration Concept Question 0003. \\ \fi
\begin{problem}

What is wrong with the following equation:

\[
\int_{\frac{1}{2} \, \pi}^{\frac{5}{3} \, \pi} {-\cot\left(x\right) \csc\left(x\right)}\;dx = {\frac{1}{\sin\left(x\right)}}\Bigg\vert_{\frac{1}{2} \, \pi}^{\frac{5}{3} \, \pi} = {-\frac{2}{3} \, \sqrt{3} - 1}
\]

\input{Integral-Concept-0003.HELP.tex}

\begin{multipleChoice}
\choice{The antiderivative is incorrect.}
\choice[correct]{The integrand is not defined over the entire interval.}
\choice{The bounds are evaluated in the wrong order.}
\choice{Nothing is wrong.  The equation is correct, as is.}
\end{multipleChoice}

\end{problem}}%}

\latexProblemContent{
\ifVerboseLocation This is Integration Concept Question 0003. \\ \fi
\begin{problem}

What is wrong with the following equation:

\[
\int_{\frac{1}{6} \, \pi}^{\frac{4}{3} \, \pi} {-5 \, \csc\left(x\right)^{2}}\;dx = {\frac{5}{\tan\left(x\right)}}\Bigg\vert_{\frac{1}{6} \, \pi}^{\frac{4}{3} \, \pi} = {-\frac{10}{3} \, \sqrt{3}}
\]

\input{Integral-Concept-0003.HELP.tex}

\begin{multipleChoice}
\choice{The antiderivative is incorrect.}
\choice[correct]{The integrand is not defined over the entire interval.}
\choice{The bounds are evaluated in the wrong order.}
\choice{Nothing is wrong.  The equation is correct, as is.}
\end{multipleChoice}

\end{problem}}%}

\latexProblemContent{
\ifVerboseLocation This is Integration Concept Question 0003. \\ \fi
\begin{problem}

What is wrong with the following equation:

\[
\int_{\frac{5}{6} \, \pi}^{\frac{7}{6} \, \pi} {4 \, \cot\left(x\right) \csc\left(x\right)}\;dx = {-\frac{4}{\sin\left(x\right)}}\Bigg\vert_{\frac{5}{6} \, \pi}^{\frac{7}{6} \, \pi} = {16}
\]

\input{Integral-Concept-0003.HELP.tex}

\begin{multipleChoice}
\choice{The antiderivative is incorrect.}
\choice[correct]{The integrand is not defined over the entire interval.}
\choice{The bounds are evaluated in the wrong order.}
\choice{Nothing is wrong.  The equation is correct, as is.}
\end{multipleChoice}

\end{problem}}%}

\latexProblemContent{
\ifVerboseLocation This is Integration Concept Question 0003. \\ \fi
\begin{problem}

What is wrong with the following equation:

\[
\int_{\frac{1}{3} \, \pi}^{\frac{11}{6} \, \pi} {-5 \, \csc\left(x\right)^{2}}\;dx = {\frac{5}{\tan\left(x\right)}}\Bigg\vert_{\frac{1}{3} \, \pi}^{\frac{11}{6} \, \pi} = {-\frac{20}{3} \, \sqrt{3}}
\]

\input{Integral-Concept-0003.HELP.tex}

\begin{multipleChoice}
\choice{The antiderivative is incorrect.}
\choice[correct]{The integrand is not defined over the entire interval.}
\choice{The bounds are evaluated in the wrong order.}
\choice{Nothing is wrong.  The equation is correct, as is.}
\end{multipleChoice}

\end{problem}}%}

\latexProblemContent{
\ifVerboseLocation This is Integration Concept Question 0003. \\ \fi
\begin{problem}

What is wrong with the following equation:

\[
\int_{\frac{1}{3} \, \pi}^{\frac{5}{4} \, \pi} {-2 \, \cot\left(x\right) \csc\left(x\right)}\;dx = {\frac{2}{\sin\left(x\right)}}\Bigg\vert_{\frac{1}{3} \, \pi}^{\frac{5}{4} \, \pi} = {-\frac{4}{3} \, \sqrt{3} - 2 \, \sqrt{2}}
\]

\input{Integral-Concept-0003.HELP.tex}

\begin{multipleChoice}
\choice{The antiderivative is incorrect.}
\choice[correct]{The integrand is not defined over the entire interval.}
\choice{The bounds are evaluated in the wrong order.}
\choice{Nothing is wrong.  The equation is correct, as is.}
\end{multipleChoice}

\end{problem}}%}

\latexProblemContent{
\ifVerboseLocation This is Integration Concept Question 0003. \\ \fi
\begin{problem}

What is wrong with the following equation:

\[
\int_{\frac{1}{6} \, \pi}^{\frac{5}{4} \, \pi} {6 \, \cot\left(x\right) \csc\left(x\right)}\;dx = {-\frac{6}{\sin\left(x\right)}}\Bigg\vert_{\frac{1}{6} \, \pi}^{\frac{5}{4} \, \pi} = {6 \, \sqrt{2} + 12}
\]

\input{Integral-Concept-0003.HELP.tex}

\begin{multipleChoice}
\choice{The antiderivative is incorrect.}
\choice[correct]{The integrand is not defined over the entire interval.}
\choice{The bounds are evaluated in the wrong order.}
\choice{Nothing is wrong.  The equation is correct, as is.}
\end{multipleChoice}

\end{problem}}%}

\latexProblemContent{
\ifVerboseLocation This is Integration Concept Question 0003. \\ \fi
\begin{problem}

What is wrong with the following equation:

\[
\int_{\frac{1}{2} \, \pi}^{\frac{3}{2} \, \pi} {-10 \, \csc\left(x\right)^{2}}\;dx = {\frac{10}{\tan\left(x\right)}}\Bigg\vert_{\frac{1}{2} \, \pi}^{\frac{3}{2} \, \pi} = {0}
\]

\input{Integral-Concept-0003.HELP.tex}

\begin{multipleChoice}
\choice{The antiderivative is incorrect.}
\choice[correct]{The integrand is not defined over the entire interval.}
\choice{The bounds are evaluated in the wrong order.}
\choice{Nothing is wrong.  The equation is correct, as is.}
\end{multipleChoice}

\end{problem}}%}

\latexProblemContent{
\ifVerboseLocation This is Integration Concept Question 0003. \\ \fi
\begin{problem}

What is wrong with the following equation:

\[
\int_{\frac{1}{2} \, \pi}^{\frac{7}{4} \, \pi} {6 \, \cot\left(x\right) \csc\left(x\right)}\;dx = {-\frac{6}{\sin\left(x\right)}}\Bigg\vert_{\frac{1}{2} \, \pi}^{\frac{7}{4} \, \pi} = {6 \, \sqrt{2} + 6}
\]

\input{Integral-Concept-0003.HELP.tex}

\begin{multipleChoice}
\choice{The antiderivative is incorrect.}
\choice[correct]{The integrand is not defined over the entire interval.}
\choice{The bounds are evaluated in the wrong order.}
\choice{Nothing is wrong.  The equation is correct, as is.}
\end{multipleChoice}

\end{problem}}%}

\latexProblemContent{
\ifVerboseLocation This is Integration Concept Question 0003. \\ \fi
\begin{problem}

What is wrong with the following equation:

\[
\int_{\frac{1}{6} \, \pi}^{\frac{5}{4} \, \pi} {-7 \, \cot\left(x\right) \csc\left(x\right)}\;dx = {\frac{7}{\sin\left(x\right)}}\Bigg\vert_{\frac{1}{6} \, \pi}^{\frac{5}{4} \, \pi} = {-7 \, \sqrt{2} - 14}
\]

\input{Integral-Concept-0003.HELP.tex}

\begin{multipleChoice}
\choice{The antiderivative is incorrect.}
\choice[correct]{The integrand is not defined over the entire interval.}
\choice{The bounds are evaluated in the wrong order.}
\choice{Nothing is wrong.  The equation is correct, as is.}
\end{multipleChoice}

\end{problem}}%}

\latexProblemContent{
\ifVerboseLocation This is Integration Concept Question 0003. \\ \fi
\begin{problem}

What is wrong with the following equation:

\[
\int_{\frac{2}{3} \, \pi}^{\frac{11}{6} \, \pi} {-9 \, \csc\left(x\right)^{2}}\;dx = {\frac{9}{\tan\left(x\right)}}\Bigg\vert_{\frac{2}{3} \, \pi}^{\frac{11}{6} \, \pi} = {-6 \, \sqrt{3}}
\]

\input{Integral-Concept-0003.HELP.tex}

\begin{multipleChoice}
\choice{The antiderivative is incorrect.}
\choice[correct]{The integrand is not defined over the entire interval.}
\choice{The bounds are evaluated in the wrong order.}
\choice{Nothing is wrong.  The equation is correct, as is.}
\end{multipleChoice}

\end{problem}}%}

\latexProblemContent{
\ifVerboseLocation This is Integration Concept Question 0003. \\ \fi
\begin{problem}

What is wrong with the following equation:

\[
\int_{\frac{2}{3} \, \pi}^{\frac{11}{6} \, \pi} {-8 \, \cot\left(x\right) \csc\left(x\right)}\;dx = {\frac{8}{\sin\left(x\right)}}\Bigg\vert_{\frac{2}{3} \, \pi}^{\frac{11}{6} \, \pi} = {-\frac{16}{3} \, \sqrt{3} - 16}
\]

\input{Integral-Concept-0003.HELP.tex}

\begin{multipleChoice}
\choice{The antiderivative is incorrect.}
\choice[correct]{The integrand is not defined over the entire interval.}
\choice{The bounds are evaluated in the wrong order.}
\choice{Nothing is wrong.  The equation is correct, as is.}
\end{multipleChoice}

\end{problem}}%}

\latexProblemContent{
\ifVerboseLocation This is Integration Concept Question 0003. \\ \fi
\begin{problem}

What is wrong with the following equation:

\[
\int_{\frac{5}{6} \, \pi}^{\frac{7}{6} \, \pi} {-10 \, \csc\left(x\right)^{2}}\;dx = {\frac{10}{\tan\left(x\right)}}\Bigg\vert_{\frac{5}{6} \, \pi}^{\frac{7}{6} \, \pi} = {20 \, \sqrt{3}}
\]

\input{Integral-Concept-0003.HELP.tex}

\begin{multipleChoice}
\choice{The antiderivative is incorrect.}
\choice[correct]{The integrand is not defined over the entire interval.}
\choice{The bounds are evaluated in the wrong order.}
\choice{Nothing is wrong.  The equation is correct, as is.}
\end{multipleChoice}

\end{problem}}%}

\latexProblemContent{
\ifVerboseLocation This is Integration Concept Question 0003. \\ \fi
\begin{problem}

What is wrong with the following equation:

\[
\int_{\frac{1}{2} \, \pi}^{\frac{5}{4} \, \pi} {7 \, \cot\left(x\right) \csc\left(x\right)}\;dx = {-\frac{7}{\sin\left(x\right)}}\Bigg\vert_{\frac{1}{2} \, \pi}^{\frac{5}{4} \, \pi} = {7 \, \sqrt{2} + 7}
\]

\input{Integral-Concept-0003.HELP.tex}

\begin{multipleChoice}
\choice{The antiderivative is incorrect.}
\choice[correct]{The integrand is not defined over the entire interval.}
\choice{The bounds are evaluated in the wrong order.}
\choice{Nothing is wrong.  The equation is correct, as is.}
\end{multipleChoice}

\end{problem}}%}

\latexProblemContent{
\ifVerboseLocation This is Integration Concept Question 0003. \\ \fi
\begin{problem}

What is wrong with the following equation:

\[
\int_{\frac{5}{6} \, \pi}^{\frac{4}{3} \, \pi} {-5 \, \cot\left(x\right) \csc\left(x\right)}\;dx = {\frac{5}{\sin\left(x\right)}}\Bigg\vert_{\frac{5}{6} \, \pi}^{\frac{4}{3} \, \pi} = {-\frac{10}{3} \, \sqrt{3} - 10}
\]

\input{Integral-Concept-0003.HELP.tex}

\begin{multipleChoice}
\choice{The antiderivative is incorrect.}
\choice[correct]{The integrand is not defined over the entire interval.}
\choice{The bounds are evaluated in the wrong order.}
\choice{Nothing is wrong.  The equation is correct, as is.}
\end{multipleChoice}

\end{problem}}%}

\latexProblemContent{
\ifVerboseLocation This is Integration Concept Question 0003. \\ \fi
\begin{problem}

What is wrong with the following equation:

\[
\int_{\frac{1}{2} \, \pi}^{\frac{11}{6} \, \pi} {-9 \, \csc\left(x\right)^{2}}\;dx = {\frac{9}{\tan\left(x\right)}}\Bigg\vert_{\frac{1}{2} \, \pi}^{\frac{11}{6} \, \pi} = {-9 \, \sqrt{3}}
\]

\input{Integral-Concept-0003.HELP.tex}

\begin{multipleChoice}
\choice{The antiderivative is incorrect.}
\choice[correct]{The integrand is not defined over the entire interval.}
\choice{The bounds are evaluated in the wrong order.}
\choice{Nothing is wrong.  The equation is correct, as is.}
\end{multipleChoice}

\end{problem}}%}

\latexProblemContent{
\ifVerboseLocation This is Integration Concept Question 0003. \\ \fi
\begin{problem}

What is wrong with the following equation:

\[
\int_{\frac{3}{4} \, \pi}^{\frac{7}{6} \, \pi} {-8 \, \csc\left(x\right)^{2}}\;dx = {\frac{8}{\tan\left(x\right)}}\Bigg\vert_{\frac{3}{4} \, \pi}^{\frac{7}{6} \, \pi} = {8 \, \sqrt{3} + 8}
\]

\input{Integral-Concept-0003.HELP.tex}

\begin{multipleChoice}
\choice{The antiderivative is incorrect.}
\choice[correct]{The integrand is not defined over the entire interval.}
\choice{The bounds are evaluated in the wrong order.}
\choice{Nothing is wrong.  The equation is correct, as is.}
\end{multipleChoice}

\end{problem}}%}

\latexProblemContent{
\ifVerboseLocation This is Integration Concept Question 0003. \\ \fi
\begin{problem}

What is wrong with the following equation:

\[
\int_{\frac{5}{6} \, \pi}^{\frac{5}{4} \, \pi} {5 \, \csc\left(x\right)^{2}}\;dx = {-\frac{5}{\tan\left(x\right)}}\Bigg\vert_{\frac{5}{6} \, \pi}^{\frac{5}{4} \, \pi} = {-5 \, \sqrt{3} - 5}
\]

\input{Integral-Concept-0003.HELP.tex}

\begin{multipleChoice}
\choice{The antiderivative is incorrect.}
\choice[correct]{The integrand is not defined over the entire interval.}
\choice{The bounds are evaluated in the wrong order.}
\choice{Nothing is wrong.  The equation is correct, as is.}
\end{multipleChoice}

\end{problem}}%}

\latexProblemContent{
\ifVerboseLocation This is Integration Concept Question 0003. \\ \fi
\begin{problem}

What is wrong with the following equation:

\[
\int_{\frac{1}{2} \, \pi}^{\frac{7}{4} \, \pi} {-2 \, \csc\left(x\right)^{2}}\;dx = {\frac{2}{\tan\left(x\right)}}\Bigg\vert_{\frac{1}{2} \, \pi}^{\frac{7}{4} \, \pi} = {-2}
\]

\input{Integral-Concept-0003.HELP.tex}

\begin{multipleChoice}
\choice{The antiderivative is incorrect.}
\choice[correct]{The integrand is not defined over the entire interval.}
\choice{The bounds are evaluated in the wrong order.}
\choice{Nothing is wrong.  The equation is correct, as is.}
\end{multipleChoice}

\end{problem}}%}

\latexProblemContent{
\ifVerboseLocation This is Integration Concept Question 0003. \\ \fi
\begin{problem}

What is wrong with the following equation:

\[
\int_{\frac{1}{2} \, \pi}^{\frac{3}{2} \, \pi} {2 \, \cot\left(x\right) \csc\left(x\right)}\;dx = {-\frac{2}{\sin\left(x\right)}}\Bigg\vert_{\frac{1}{2} \, \pi}^{\frac{3}{2} \, \pi} = {4}
\]

\input{Integral-Concept-0003.HELP.tex}

\begin{multipleChoice}
\choice{The antiderivative is incorrect.}
\choice[correct]{The integrand is not defined over the entire interval.}
\choice{The bounds are evaluated in the wrong order.}
\choice{Nothing is wrong.  The equation is correct, as is.}
\end{multipleChoice}

\end{problem}}%}

\latexProblemContent{
\ifVerboseLocation This is Integration Concept Question 0003. \\ \fi
\begin{problem}

What is wrong with the following equation:

\[
\int_{\frac{1}{2} \, \pi}^{\frac{3}{2} \, \pi} {7 \, \cot\left(x\right) \csc\left(x\right)}\;dx = {-\frac{7}{\sin\left(x\right)}}\Bigg\vert_{\frac{1}{2} \, \pi}^{\frac{3}{2} \, \pi} = {14}
\]

\input{Integral-Concept-0003.HELP.tex}

\begin{multipleChoice}
\choice{The antiderivative is incorrect.}
\choice[correct]{The integrand is not defined over the entire interval.}
\choice{The bounds are evaluated in the wrong order.}
\choice{Nothing is wrong.  The equation is correct, as is.}
\end{multipleChoice}

\end{problem}}%}

\latexProblemContent{
\ifVerboseLocation This is Integration Concept Question 0003. \\ \fi
\begin{problem}

What is wrong with the following equation:

\[
\int_{\frac{1}{3} \, \pi}^{\frac{11}{6} \, \pi} {-10 \, \csc\left(x\right)^{2}}\;dx = {\frac{10}{\tan\left(x\right)}}\Bigg\vert_{\frac{1}{3} \, \pi}^{\frac{11}{6} \, \pi} = {-\frac{40}{3} \, \sqrt{3}}
\]

\input{Integral-Concept-0003.HELP.tex}

\begin{multipleChoice}
\choice{The antiderivative is incorrect.}
\choice[correct]{The integrand is not defined over the entire interval.}
\choice{The bounds are evaluated in the wrong order.}
\choice{Nothing is wrong.  The equation is correct, as is.}
\end{multipleChoice}

\end{problem}}%}

\latexProblemContent{
\ifVerboseLocation This is Integration Concept Question 0003. \\ \fi
\begin{problem}

What is wrong with the following equation:

\[
\int_{\frac{1}{3} \, \pi}^{\frac{7}{6} \, \pi} {5 \, \csc\left(x\right)^{2}}\;dx = {-\frac{5}{\tan\left(x\right)}}\Bigg\vert_{\frac{1}{3} \, \pi}^{\frac{7}{6} \, \pi} = {-\frac{10}{3} \, \sqrt{3}}
\]

\input{Integral-Concept-0003.HELP.tex}

\begin{multipleChoice}
\choice{The antiderivative is incorrect.}
\choice[correct]{The integrand is not defined over the entire interval.}
\choice{The bounds are evaluated in the wrong order.}
\choice{Nothing is wrong.  The equation is correct, as is.}
\end{multipleChoice}

\end{problem}}%}

\latexProblemContent{
\ifVerboseLocation This is Integration Concept Question 0003. \\ \fi
\begin{problem}

What is wrong with the following equation:

\[
\int_{\frac{1}{2} \, \pi}^{\frac{3}{2} \, \pi} {3 \, \csc\left(x\right)^{2}}\;dx = {-\frac{3}{\tan\left(x\right)}}\Bigg\vert_{\frac{1}{2} \, \pi}^{\frac{3}{2} \, \pi} = {0}
\]

\input{Integral-Concept-0003.HELP.tex}

\begin{multipleChoice}
\choice{The antiderivative is incorrect.}
\choice[correct]{The integrand is not defined over the entire interval.}
\choice{The bounds are evaluated in the wrong order.}
\choice{Nothing is wrong.  The equation is correct, as is.}
\end{multipleChoice}

\end{problem}}%}

\latexProblemContent{
\ifVerboseLocation This is Integration Concept Question 0003. \\ \fi
\begin{problem}

What is wrong with the following equation:

\[
\int_{\frac{2}{3} \, \pi}^{\frac{3}{2} \, \pi} {2 \, \cot\left(x\right) \csc\left(x\right)}\;dx = {-\frac{2}{\sin\left(x\right)}}\Bigg\vert_{\frac{2}{3} \, \pi}^{\frac{3}{2} \, \pi} = {\frac{4}{3} \, \sqrt{3} + 2}
\]

\input{Integral-Concept-0003.HELP.tex}

\begin{multipleChoice}
\choice{The antiderivative is incorrect.}
\choice[correct]{The integrand is not defined over the entire interval.}
\choice{The bounds are evaluated in the wrong order.}
\choice{Nothing is wrong.  The equation is correct, as is.}
\end{multipleChoice}

\end{problem}}%}

\latexProblemContent{
\ifVerboseLocation This is Integration Concept Question 0003. \\ \fi
\begin{problem}

What is wrong with the following equation:

\[
\int_{\frac{1}{3} \, \pi}^{\frac{5}{4} \, \pi} {-\csc\left(x\right)^{2}}\;dx = {\frac{1}{\tan\left(x\right)}}\Bigg\vert_{\frac{1}{3} \, \pi}^{\frac{5}{4} \, \pi} = {-\frac{1}{3} \, \sqrt{3} + 1}
\]

\input{Integral-Concept-0003.HELP.tex}

\begin{multipleChoice}
\choice{The antiderivative is incorrect.}
\choice[correct]{The integrand is not defined over the entire interval.}
\choice{The bounds are evaluated in the wrong order.}
\choice{Nothing is wrong.  The equation is correct, as is.}
\end{multipleChoice}

\end{problem}}%}

\latexProblemContent{
\ifVerboseLocation This is Integration Concept Question 0003. \\ \fi
\begin{problem}

What is wrong with the following equation:

\[
\int_{\frac{1}{3} \, \pi}^{\frac{3}{2} \, \pi} {-9 \, \cot\left(x\right) \csc\left(x\right)}\;dx = {\frac{9}{\sin\left(x\right)}}\Bigg\vert_{\frac{1}{3} \, \pi}^{\frac{3}{2} \, \pi} = {-6 \, \sqrt{3} - 9}
\]

\input{Integral-Concept-0003.HELP.tex}

\begin{multipleChoice}
\choice{The antiderivative is incorrect.}
\choice[correct]{The integrand is not defined over the entire interval.}
\choice{The bounds are evaluated in the wrong order.}
\choice{Nothing is wrong.  The equation is correct, as is.}
\end{multipleChoice}

\end{problem}}%}

\latexProblemContent{
\ifVerboseLocation This is Integration Concept Question 0003. \\ \fi
\begin{problem}

What is wrong with the following equation:

\[
\int_{\frac{3}{4} \, \pi}^{\frac{11}{6} \, \pi} {2 \, \cot\left(x\right) \csc\left(x\right)}\;dx = {-\frac{2}{\sin\left(x\right)}}\Bigg\vert_{\frac{3}{4} \, \pi}^{\frac{11}{6} \, \pi} = {2 \, \sqrt{2} + 4}
\]

\input{Integral-Concept-0003.HELP.tex}

\begin{multipleChoice}
\choice{The antiderivative is incorrect.}
\choice[correct]{The integrand is not defined over the entire interval.}
\choice{The bounds are evaluated in the wrong order.}
\choice{Nothing is wrong.  The equation is correct, as is.}
\end{multipleChoice}

\end{problem}}%}

\latexProblemContent{
\ifVerboseLocation This is Integration Concept Question 0003. \\ \fi
\begin{problem}

What is wrong with the following equation:

\[
\int_{\frac{3}{4} \, \pi}^{\frac{5}{3} \, \pi} {-10 \, \cot\left(x\right) \csc\left(x\right)}\;dx = {\frac{10}{\sin\left(x\right)}}\Bigg\vert_{\frac{3}{4} \, \pi}^{\frac{5}{3} \, \pi} = {-\frac{20}{3} \, \sqrt{3} - 10 \, \sqrt{2}}
\]

\input{Integral-Concept-0003.HELP.tex}

\begin{multipleChoice}
\choice{The antiderivative is incorrect.}
\choice[correct]{The integrand is not defined over the entire interval.}
\choice{The bounds are evaluated in the wrong order.}
\choice{Nothing is wrong.  The equation is correct, as is.}
\end{multipleChoice}

\end{problem}}%}

\latexProblemContent{
\ifVerboseLocation This is Integration Concept Question 0003. \\ \fi
\begin{problem}

What is wrong with the following equation:

\[
\int_{\frac{1}{6} \, \pi}^{\frac{7}{4} \, \pi} {3 \, \cot\left(x\right) \csc\left(x\right)}\;dx = {-\frac{3}{\sin\left(x\right)}}\Bigg\vert_{\frac{1}{6} \, \pi}^{\frac{7}{4} \, \pi} = {3 \, \sqrt{2} + 6}
\]

\input{Integral-Concept-0003.HELP.tex}

\begin{multipleChoice}
\choice{The antiderivative is incorrect.}
\choice[correct]{The integrand is not defined over the entire interval.}
\choice{The bounds are evaluated in the wrong order.}
\choice{Nothing is wrong.  The equation is correct, as is.}
\end{multipleChoice}

\end{problem}}%}

\latexProblemContent{
\ifVerboseLocation This is Integration Concept Question 0003. \\ \fi
\begin{problem}

What is wrong with the following equation:

\[
\int_{\frac{5}{6} \, \pi}^{\frac{7}{6} \, \pi} {-3 \, \csc\left(x\right)^{2}}\;dx = {\frac{3}{\tan\left(x\right)}}\Bigg\vert_{\frac{5}{6} \, \pi}^{\frac{7}{6} \, \pi} = {6 \, \sqrt{3}}
\]

\input{Integral-Concept-0003.HELP.tex}

\begin{multipleChoice}
\choice{The antiderivative is incorrect.}
\choice[correct]{The integrand is not defined over the entire interval.}
\choice{The bounds are evaluated in the wrong order.}
\choice{Nothing is wrong.  The equation is correct, as is.}
\end{multipleChoice}

\end{problem}}%}

\latexProblemContent{
\ifVerboseLocation This is Integration Concept Question 0003. \\ \fi
\begin{problem}

What is wrong with the following equation:

\[
\int_{\frac{1}{4} \, \pi}^{\frac{7}{6} \, \pi} {-5 \, \cot\left(x\right) \csc\left(x\right)}\;dx = {\frac{5}{\sin\left(x\right)}}\Bigg\vert_{\frac{1}{4} \, \pi}^{\frac{7}{6} \, \pi} = {-5 \, \sqrt{2} - 10}
\]

\input{Integral-Concept-0003.HELP.tex}

\begin{multipleChoice}
\choice{The antiderivative is incorrect.}
\choice[correct]{The integrand is not defined over the entire interval.}
\choice{The bounds are evaluated in the wrong order.}
\choice{Nothing is wrong.  The equation is correct, as is.}
\end{multipleChoice}

\end{problem}}%}

\latexProblemContent{
\ifVerboseLocation This is Integration Concept Question 0003. \\ \fi
\begin{problem}

What is wrong with the following equation:

\[
\int_{\frac{1}{6} \, \pi}^{\frac{3}{2} \, \pi} {-6 \, \csc\left(x\right)^{2}}\;dx = {\frac{6}{\tan\left(x\right)}}\Bigg\vert_{\frac{1}{6} \, \pi}^{\frac{3}{2} \, \pi} = {-6 \, \sqrt{3}}
\]

\input{Integral-Concept-0003.HELP.tex}

\begin{multipleChoice}
\choice{The antiderivative is incorrect.}
\choice[correct]{The integrand is not defined over the entire interval.}
\choice{The bounds are evaluated in the wrong order.}
\choice{Nothing is wrong.  The equation is correct, as is.}
\end{multipleChoice}

\end{problem}}%}

\latexProblemContent{
\ifVerboseLocation This is Integration Concept Question 0003. \\ \fi
\begin{problem}

What is wrong with the following equation:

\[
\int_{\frac{3}{4} \, \pi}^{\frac{11}{6} \, \pi} {-\csc\left(x\right)^{2}}\;dx = {\frac{1}{\tan\left(x\right)}}\Bigg\vert_{\frac{3}{4} \, \pi}^{\frac{11}{6} \, \pi} = {-\sqrt{3} + 1}
\]

\input{Integral-Concept-0003.HELP.tex}

\begin{multipleChoice}
\choice{The antiderivative is incorrect.}
\choice[correct]{The integrand is not defined over the entire interval.}
\choice{The bounds are evaluated in the wrong order.}
\choice{Nothing is wrong.  The equation is correct, as is.}
\end{multipleChoice}

\end{problem}}%}

\latexProblemContent{
\ifVerboseLocation This is Integration Concept Question 0003. \\ \fi
\begin{problem}

What is wrong with the following equation:

\[
\int_{\frac{1}{4} \, \pi}^{\frac{3}{2} \, \pi} {4 \, \cot\left(x\right) \csc\left(x\right)}\;dx = {-\frac{4}{\sin\left(x\right)}}\Bigg\vert_{\frac{1}{4} \, \pi}^{\frac{3}{2} \, \pi} = {4 \, \sqrt{2} + 4}
\]

\input{Integral-Concept-0003.HELP.tex}

\begin{multipleChoice}
\choice{The antiderivative is incorrect.}
\choice[correct]{The integrand is not defined over the entire interval.}
\choice{The bounds are evaluated in the wrong order.}
\choice{Nothing is wrong.  The equation is correct, as is.}
\end{multipleChoice}

\end{problem}}%}

\latexProblemContent{
\ifVerboseLocation This is Integration Concept Question 0003. \\ \fi
\begin{problem}

What is wrong with the following equation:

\[
\int_{\frac{1}{6} \, \pi}^{\frac{7}{4} \, \pi} {10 \, \cot\left(x\right) \csc\left(x\right)}\;dx = {-\frac{10}{\sin\left(x\right)}}\Bigg\vert_{\frac{1}{6} \, \pi}^{\frac{7}{4} \, \pi} = {10 \, \sqrt{2} + 20}
\]

\input{Integral-Concept-0003.HELP.tex}

\begin{multipleChoice}
\choice{The antiderivative is incorrect.}
\choice[correct]{The integrand is not defined over the entire interval.}
\choice{The bounds are evaluated in the wrong order.}
\choice{Nothing is wrong.  The equation is correct, as is.}
\end{multipleChoice}

\end{problem}}%}

\latexProblemContent{
\ifVerboseLocation This is Integration Concept Question 0003. \\ \fi
\begin{problem}

What is wrong with the following equation:

\[
\int_{\frac{1}{6} \, \pi}^{\frac{4}{3} \, \pi} {-9 \, \cot\left(x\right) \csc\left(x\right)}\;dx = {\frac{9}{\sin\left(x\right)}}\Bigg\vert_{\frac{1}{6} \, \pi}^{\frac{4}{3} \, \pi} = {-6 \, \sqrt{3} - 18}
\]

\input{Integral-Concept-0003.HELP.tex}

\begin{multipleChoice}
\choice{The antiderivative is incorrect.}
\choice[correct]{The integrand is not defined over the entire interval.}
\choice{The bounds are evaluated in the wrong order.}
\choice{Nothing is wrong.  The equation is correct, as is.}
\end{multipleChoice}

\end{problem}}%}

\latexProblemContent{
\ifVerboseLocation This is Integration Concept Question 0003. \\ \fi
\begin{problem}

What is wrong with the following equation:

\[
\int_{\frac{3}{4} \, \pi}^{\frac{11}{6} \, \pi} {9 \, \csc\left(x\right)^{2}}\;dx = {-\frac{9}{\tan\left(x\right)}}\Bigg\vert_{\frac{3}{4} \, \pi}^{\frac{11}{6} \, \pi} = {9 \, \sqrt{3} - 9}
\]

\input{Integral-Concept-0003.HELP.tex}

\begin{multipleChoice}
\choice{The antiderivative is incorrect.}
\choice[correct]{The integrand is not defined over the entire interval.}
\choice{The bounds are evaluated in the wrong order.}
\choice{Nothing is wrong.  The equation is correct, as is.}
\end{multipleChoice}

\end{problem}}%}

\latexProblemContent{
\ifVerboseLocation This is Integration Concept Question 0003. \\ \fi
\begin{problem}

What is wrong with the following equation:

\[
\int_{\frac{5}{6} \, \pi}^{\frac{7}{6} \, \pi} {-4 \, \cot\left(x\right) \csc\left(x\right)}\;dx = {\frac{4}{\sin\left(x\right)}}\Bigg\vert_{\frac{5}{6} \, \pi}^{\frac{7}{6} \, \pi} = {-16}
\]

\input{Integral-Concept-0003.HELP.tex}

\begin{multipleChoice}
\choice{The antiderivative is incorrect.}
\choice[correct]{The integrand is not defined over the entire interval.}
\choice{The bounds are evaluated in the wrong order.}
\choice{Nothing is wrong.  The equation is correct, as is.}
\end{multipleChoice}

\end{problem}}%}

\latexProblemContent{
\ifVerboseLocation This is Integration Concept Question 0003. \\ \fi
\begin{problem}

What is wrong with the following equation:

\[
\int_{\frac{1}{4} \, \pi}^{\frac{7}{6} \, \pi} {\cot\left(x\right) \csc\left(x\right)}\;dx = {-\frac{1}{\sin\left(x\right)}}\Bigg\vert_{\frac{1}{4} \, \pi}^{\frac{7}{6} \, \pi} = {\sqrt{2} + 2}
\]

\input{Integral-Concept-0003.HELP.tex}

\begin{multipleChoice}
\choice{The antiderivative is incorrect.}
\choice[correct]{The integrand is not defined over the entire interval.}
\choice{The bounds are evaluated in the wrong order.}
\choice{Nothing is wrong.  The equation is correct, as is.}
\end{multipleChoice}

\end{problem}}%}

\latexProblemContent{
\ifVerboseLocation This is Integration Concept Question 0003. \\ \fi
\begin{problem}

What is wrong with the following equation:

\[
\int_{\frac{1}{4} \, \pi}^{\frac{7}{6} \, \pi} {3 \, \csc\left(x\right)^{2}}\;dx = {-\frac{3}{\tan\left(x\right)}}\Bigg\vert_{\frac{1}{4} \, \pi}^{\frac{7}{6} \, \pi} = {-3 \, \sqrt{3} + 3}
\]

\input{Integral-Concept-0003.HELP.tex}

\begin{multipleChoice}
\choice{The antiderivative is incorrect.}
\choice[correct]{The integrand is not defined over the entire interval.}
\choice{The bounds are evaluated in the wrong order.}
\choice{Nothing is wrong.  The equation is correct, as is.}
\end{multipleChoice}

\end{problem}}%}

\latexProblemContent{
\ifVerboseLocation This is Integration Concept Question 0003. \\ \fi
\begin{problem}

What is wrong with the following equation:

\[
\int_{\frac{1}{4} \, \pi}^{\frac{5}{3} \, \pi} {-9 \, \csc\left(x\right)^{2}}\;dx = {\frac{9}{\tan\left(x\right)}}\Bigg\vert_{\frac{1}{4} \, \pi}^{\frac{5}{3} \, \pi} = {-3 \, \sqrt{3} - 9}
\]

\input{Integral-Concept-0003.HELP.tex}

\begin{multipleChoice}
\choice{The antiderivative is incorrect.}
\choice[correct]{The integrand is not defined over the entire interval.}
\choice{The bounds are evaluated in the wrong order.}
\choice{Nothing is wrong.  The equation is correct, as is.}
\end{multipleChoice}

\end{problem}}%}

\latexProblemContent{
\ifVerboseLocation This is Integration Concept Question 0003. \\ \fi
\begin{problem}

What is wrong with the following equation:

\[
\int_{\frac{5}{6} \, \pi}^{\frac{5}{4} \, \pi} {-3 \, \csc\left(x\right)^{2}}\;dx = {\frac{3}{\tan\left(x\right)}}\Bigg\vert_{\frac{5}{6} \, \pi}^{\frac{5}{4} \, \pi} = {3 \, \sqrt{3} + 3}
\]

\input{Integral-Concept-0003.HELP.tex}

\begin{multipleChoice}
\choice{The antiderivative is incorrect.}
\choice[correct]{The integrand is not defined over the entire interval.}
\choice{The bounds are evaluated in the wrong order.}
\choice{Nothing is wrong.  The equation is correct, as is.}
\end{multipleChoice}

\end{problem}}%}

\latexProblemContent{
\ifVerboseLocation This is Integration Concept Question 0003. \\ \fi
\begin{problem}

What is wrong with the following equation:

\[
\int_{\frac{1}{4} \, \pi}^{\frac{3}{2} \, \pi} {-3 \, \csc\left(x\right)^{2}}\;dx = {\frac{3}{\tan\left(x\right)}}\Bigg\vert_{\frac{1}{4} \, \pi}^{\frac{3}{2} \, \pi} = {-3}
\]

\input{Integral-Concept-0003.HELP.tex}

\begin{multipleChoice}
\choice{The antiderivative is incorrect.}
\choice[correct]{The integrand is not defined over the entire interval.}
\choice{The bounds are evaluated in the wrong order.}
\choice{Nothing is wrong.  The equation is correct, as is.}
\end{multipleChoice}

\end{problem}}%}

\latexProblemContent{
\ifVerboseLocation This is Integration Concept Question 0003. \\ \fi
\begin{problem}

What is wrong with the following equation:

\[
\int_{\frac{1}{3} \, \pi}^{\frac{3}{2} \, \pi} {5 \, \csc\left(x\right)^{2}}\;dx = {-\frac{5}{\tan\left(x\right)}}\Bigg\vert_{\frac{1}{3} \, \pi}^{\frac{3}{2} \, \pi} = {\frac{5}{3} \, \sqrt{3}}
\]

\input{Integral-Concept-0003.HELP.tex}

\begin{multipleChoice}
\choice{The antiderivative is incorrect.}
\choice[correct]{The integrand is not defined over the entire interval.}
\choice{The bounds are evaluated in the wrong order.}
\choice{Nothing is wrong.  The equation is correct, as is.}
\end{multipleChoice}

\end{problem}}%}

\latexProblemContent{
\ifVerboseLocation This is Integration Concept Question 0003. \\ \fi
\begin{problem}

What is wrong with the following equation:

\[
\int_{\frac{1}{4} \, \pi}^{\frac{11}{6} \, \pi} {2 \, \cot\left(x\right) \csc\left(x\right)}\;dx = {-\frac{2}{\sin\left(x\right)}}\Bigg\vert_{\frac{1}{4} \, \pi}^{\frac{11}{6} \, \pi} = {2 \, \sqrt{2} + 4}
\]

\input{Integral-Concept-0003.HELP.tex}

\begin{multipleChoice}
\choice{The antiderivative is incorrect.}
\choice[correct]{The integrand is not defined over the entire interval.}
\choice{The bounds are evaluated in the wrong order.}
\choice{Nothing is wrong.  The equation is correct, as is.}
\end{multipleChoice}

\end{problem}}%}

\latexProblemContent{
\ifVerboseLocation This is Integration Concept Question 0003. \\ \fi
\begin{problem}

What is wrong with the following equation:

\[
\int_{\frac{3}{4} \, \pi}^{\frac{4}{3} \, \pi} {6 \, \csc\left(x\right)^{2}}\;dx = {-\frac{6}{\tan\left(x\right)}}\Bigg\vert_{\frac{3}{4} \, \pi}^{\frac{4}{3} \, \pi} = {-2 \, \sqrt{3} - 6}
\]

\input{Integral-Concept-0003.HELP.tex}

\begin{multipleChoice}
\choice{The antiderivative is incorrect.}
\choice[correct]{The integrand is not defined over the entire interval.}
\choice{The bounds are evaluated in the wrong order.}
\choice{Nothing is wrong.  The equation is correct, as is.}
\end{multipleChoice}

\end{problem}}%}

\latexProblemContent{
\ifVerboseLocation This is Integration Concept Question 0003. \\ \fi
\begin{problem}

What is wrong with the following equation:

\[
\int_{\frac{3}{4} \, \pi}^{\frac{7}{4} \, \pi} {-4 \, \csc\left(x\right)^{2}}\;dx = {\frac{4}{\tan\left(x\right)}}\Bigg\vert_{\frac{3}{4} \, \pi}^{\frac{7}{4} \, \pi} = {0}
\]

\input{Integral-Concept-0003.HELP.tex}

\begin{multipleChoice}
\choice{The antiderivative is incorrect.}
\choice[correct]{The integrand is not defined over the entire interval.}
\choice{The bounds are evaluated in the wrong order.}
\choice{Nothing is wrong.  The equation is correct, as is.}
\end{multipleChoice}

\end{problem}}%}

\latexProblemContent{
\ifVerboseLocation This is Integration Concept Question 0003. \\ \fi
\begin{problem}

What is wrong with the following equation:

\[
\int_{\frac{1}{3} \, \pi}^{\frac{7}{4} \, \pi} {5 \, \cot\left(x\right) \csc\left(x\right)}\;dx = {-\frac{5}{\sin\left(x\right)}}\Bigg\vert_{\frac{1}{3} \, \pi}^{\frac{7}{4} \, \pi} = {\frac{10}{3} \, \sqrt{3} + 5 \, \sqrt{2}}
\]

\input{Integral-Concept-0003.HELP.tex}

\begin{multipleChoice}
\choice{The antiderivative is incorrect.}
\choice[correct]{The integrand is not defined over the entire interval.}
\choice{The bounds are evaluated in the wrong order.}
\choice{Nothing is wrong.  The equation is correct, as is.}
\end{multipleChoice}

\end{problem}}%}

\latexProblemContent{
\ifVerboseLocation This is Integration Concept Question 0003. \\ \fi
\begin{problem}

What is wrong with the following equation:

\[
\int_{\frac{1}{2} \, \pi}^{\frac{11}{6} \, \pi} {-5 \, \csc\left(x\right)^{2}}\;dx = {\frac{5}{\tan\left(x\right)}}\Bigg\vert_{\frac{1}{2} \, \pi}^{\frac{11}{6} \, \pi} = {-5 \, \sqrt{3}}
\]

\input{Integral-Concept-0003.HELP.tex}

\begin{multipleChoice}
\choice{The antiderivative is incorrect.}
\choice[correct]{The integrand is not defined over the entire interval.}
\choice{The bounds are evaluated in the wrong order.}
\choice{Nothing is wrong.  The equation is correct, as is.}
\end{multipleChoice}

\end{problem}}%}

\latexProblemContent{
\ifVerboseLocation This is Integration Concept Question 0003. \\ \fi
\begin{problem}

What is wrong with the following equation:

\[
\int_{\frac{1}{2} \, \pi}^{\frac{7}{6} \, \pi} {-5 \, \csc\left(x\right)^{2}}\;dx = {\frac{5}{\tan\left(x\right)}}\Bigg\vert_{\frac{1}{2} \, \pi}^{\frac{7}{6} \, \pi} = {5 \, \sqrt{3}}
\]

\input{Integral-Concept-0003.HELP.tex}

\begin{multipleChoice}
\choice{The antiderivative is incorrect.}
\choice[correct]{The integrand is not defined over the entire interval.}
\choice{The bounds are evaluated in the wrong order.}
\choice{Nothing is wrong.  The equation is correct, as is.}
\end{multipleChoice}

\end{problem}}%}

\latexProblemContent{
\ifVerboseLocation This is Integration Concept Question 0003. \\ \fi
\begin{problem}

What is wrong with the following equation:

\[
\int_{\frac{2}{3} \, \pi}^{\frac{5}{3} \, \pi} {-2 \, \csc\left(x\right)^{2}}\;dx = {\frac{2}{\tan\left(x\right)}}\Bigg\vert_{\frac{2}{3} \, \pi}^{\frac{5}{3} \, \pi} = {0}
\]

\input{Integral-Concept-0003.HELP.tex}

\begin{multipleChoice}
\choice{The antiderivative is incorrect.}
\choice[correct]{The integrand is not defined over the entire interval.}
\choice{The bounds are evaluated in the wrong order.}
\choice{Nothing is wrong.  The equation is correct, as is.}
\end{multipleChoice}

\end{problem}}%}

\latexProblemContent{
\ifVerboseLocation This is Integration Concept Question 0003. \\ \fi
\begin{problem}

What is wrong with the following equation:

\[
\int_{\frac{5}{6} \, \pi}^{\frac{11}{6} \, \pi} {10 \, \cot\left(x\right) \csc\left(x\right)}\;dx = {-\frac{10}{\sin\left(x\right)}}\Bigg\vert_{\frac{5}{6} \, \pi}^{\frac{11}{6} \, \pi} = {40}
\]

\input{Integral-Concept-0003.HELP.tex}

\begin{multipleChoice}
\choice{The antiderivative is incorrect.}
\choice[correct]{The integrand is not defined over the entire interval.}
\choice{The bounds are evaluated in the wrong order.}
\choice{Nothing is wrong.  The equation is correct, as is.}
\end{multipleChoice}

\end{problem}}%}

\latexProblemContent{
\ifVerboseLocation This is Integration Concept Question 0003. \\ \fi
\begin{problem}

What is wrong with the following equation:

\[
\int_{\frac{1}{6} \, \pi}^{\frac{11}{6} \, \pi} {10 \, \csc\left(x\right)^{2}}\;dx = {-\frac{10}{\tan\left(x\right)}}\Bigg\vert_{\frac{1}{6} \, \pi}^{\frac{11}{6} \, \pi} = {20 \, \sqrt{3}}
\]

\input{Integral-Concept-0003.HELP.tex}

\begin{multipleChoice}
\choice{The antiderivative is incorrect.}
\choice[correct]{The integrand is not defined over the entire interval.}
\choice{The bounds are evaluated in the wrong order.}
\choice{Nothing is wrong.  The equation is correct, as is.}
\end{multipleChoice}

\end{problem}}%}

\latexProblemContent{
\ifVerboseLocation This is Integration Concept Question 0003. \\ \fi
\begin{problem}

What is wrong with the following equation:

\[
\int_{\frac{1}{2} \, \pi}^{\frac{5}{4} \, \pi} {-8 \, \cot\left(x\right) \csc\left(x\right)}\;dx = {\frac{8}{\sin\left(x\right)}}\Bigg\vert_{\frac{1}{2} \, \pi}^{\frac{5}{4} \, \pi} = {-8 \, \sqrt{2} - 8}
\]

\input{Integral-Concept-0003.HELP.tex}

\begin{multipleChoice}
\choice{The antiderivative is incorrect.}
\choice[correct]{The integrand is not defined over the entire interval.}
\choice{The bounds are evaluated in the wrong order.}
\choice{Nothing is wrong.  The equation is correct, as is.}
\end{multipleChoice}

\end{problem}}%}

\latexProblemContent{
\ifVerboseLocation This is Integration Concept Question 0003. \\ \fi
\begin{problem}

What is wrong with the following equation:

\[
\int_{\frac{1}{6} \, \pi}^{\frac{7}{4} \, \pi} {4 \, \cot\left(x\right) \csc\left(x\right)}\;dx = {-\frac{4}{\sin\left(x\right)}}\Bigg\vert_{\frac{1}{6} \, \pi}^{\frac{7}{4} \, \pi} = {4 \, \sqrt{2} + 8}
\]

\input{Integral-Concept-0003.HELP.tex}

\begin{multipleChoice}
\choice{The antiderivative is incorrect.}
\choice[correct]{The integrand is not defined over the entire interval.}
\choice{The bounds are evaluated in the wrong order.}
\choice{Nothing is wrong.  The equation is correct, as is.}
\end{multipleChoice}

\end{problem}}%}

\latexProblemContent{
\ifVerboseLocation This is Integration Concept Question 0003. \\ \fi
\begin{problem}

What is wrong with the following equation:

\[
\int_{\frac{3}{4} \, \pi}^{\frac{3}{2} \, \pi} {5 \, \cot\left(x\right) \csc\left(x\right)}\;dx = {-\frac{5}{\sin\left(x\right)}}\Bigg\vert_{\frac{3}{4} \, \pi}^{\frac{3}{2} \, \pi} = {5 \, \sqrt{2} + 5}
\]

\input{Integral-Concept-0003.HELP.tex}

\begin{multipleChoice}
\choice{The antiderivative is incorrect.}
\choice[correct]{The integrand is not defined over the entire interval.}
\choice{The bounds are evaluated in the wrong order.}
\choice{Nothing is wrong.  The equation is correct, as is.}
\end{multipleChoice}

\end{problem}}%}

\latexProblemContent{
\ifVerboseLocation This is Integration Concept Question 0003. \\ \fi
\begin{problem}

What is wrong with the following equation:

\[
\int_{\frac{3}{4} \, \pi}^{\frac{7}{4} \, \pi} {-10 \, \cot\left(x\right) \csc\left(x\right)}\;dx = {\frac{10}{\sin\left(x\right)}}\Bigg\vert_{\frac{3}{4} \, \pi}^{\frac{7}{4} \, \pi} = {-20 \, \sqrt{2}}
\]

\input{Integral-Concept-0003.HELP.tex}

\begin{multipleChoice}
\choice{The antiderivative is incorrect.}
\choice[correct]{The integrand is not defined over the entire interval.}
\choice{The bounds are evaluated in the wrong order.}
\choice{Nothing is wrong.  The equation is correct, as is.}
\end{multipleChoice}

\end{problem}}%}

\latexProblemContent{
\ifVerboseLocation This is Integration Concept Question 0003. \\ \fi
\begin{problem}

What is wrong with the following equation:

\[
\int_{\frac{2}{3} \, \pi}^{\frac{5}{3} \, \pi} {8 \, \cot\left(x\right) \csc\left(x\right)}\;dx = {-\frac{8}{\sin\left(x\right)}}\Bigg\vert_{\frac{2}{3} \, \pi}^{\frac{5}{3} \, \pi} = {\frac{32}{3} \, \sqrt{3}}
\]

\input{Integral-Concept-0003.HELP.tex}

\begin{multipleChoice}
\choice{The antiderivative is incorrect.}
\choice[correct]{The integrand is not defined over the entire interval.}
\choice{The bounds are evaluated in the wrong order.}
\choice{Nothing is wrong.  The equation is correct, as is.}
\end{multipleChoice}

\end{problem}}%}

\latexProblemContent{
\ifVerboseLocation This is Integration Concept Question 0003. \\ \fi
\begin{problem}

What is wrong with the following equation:

\[
\int_{\frac{1}{3} \, \pi}^{\frac{4}{3} \, \pi} {7 \, \cot\left(x\right) \csc\left(x\right)}\;dx = {-\frac{7}{\sin\left(x\right)}}\Bigg\vert_{\frac{1}{3} \, \pi}^{\frac{4}{3} \, \pi} = {\frac{28}{3} \, \sqrt{3}}
\]

\input{Integral-Concept-0003.HELP.tex}

\begin{multipleChoice}
\choice{The antiderivative is incorrect.}
\choice[correct]{The integrand is not defined over the entire interval.}
\choice{The bounds are evaluated in the wrong order.}
\choice{Nothing is wrong.  The equation is correct, as is.}
\end{multipleChoice}

\end{problem}}%}

\latexProblemContent{
\ifVerboseLocation This is Integration Concept Question 0003. \\ \fi
\begin{problem}

What is wrong with the following equation:

\[
\int_{\frac{1}{3} \, \pi}^{\frac{7}{6} \, \pi} {9 \, \csc\left(x\right)^{2}}\;dx = {-\frac{9}{\tan\left(x\right)}}\Bigg\vert_{\frac{1}{3} \, \pi}^{\frac{7}{6} \, \pi} = {-6 \, \sqrt{3}}
\]

\input{Integral-Concept-0003.HELP.tex}

\begin{multipleChoice}
\choice{The antiderivative is incorrect.}
\choice[correct]{The integrand is not defined over the entire interval.}
\choice{The bounds are evaluated in the wrong order.}
\choice{Nothing is wrong.  The equation is correct, as is.}
\end{multipleChoice}

\end{problem}}%}

\latexProblemContent{
\ifVerboseLocation This is Integration Concept Question 0003. \\ \fi
\begin{problem}

What is wrong with the following equation:

\[
\int_{\frac{1}{4} \, \pi}^{\frac{11}{6} \, \pi} {10 \, \csc\left(x\right)^{2}}\;dx = {-\frac{10}{\tan\left(x\right)}}\Bigg\vert_{\frac{1}{4} \, \pi}^{\frac{11}{6} \, \pi} = {10 \, \sqrt{3} + 10}
\]

\input{Integral-Concept-0003.HELP.tex}

\begin{multipleChoice}
\choice{The antiderivative is incorrect.}
\choice[correct]{The integrand is not defined over the entire interval.}
\choice{The bounds are evaluated in the wrong order.}
\choice{Nothing is wrong.  The equation is correct, as is.}
\end{multipleChoice}

\end{problem}}%}

\latexProblemContent{
\ifVerboseLocation This is Integration Concept Question 0003. \\ \fi
\begin{problem}

What is wrong with the following equation:

\[
\int_{\frac{5}{6} \, \pi}^{\frac{5}{4} \, \pi} {-7 \, \csc\left(x\right)^{2}}\;dx = {\frac{7}{\tan\left(x\right)}}\Bigg\vert_{\frac{5}{6} \, \pi}^{\frac{5}{4} \, \pi} = {7 \, \sqrt{3} + 7}
\]

\input{Integral-Concept-0003.HELP.tex}

\begin{multipleChoice}
\choice{The antiderivative is incorrect.}
\choice[correct]{The integrand is not defined over the entire interval.}
\choice{The bounds are evaluated in the wrong order.}
\choice{Nothing is wrong.  The equation is correct, as is.}
\end{multipleChoice}

\end{problem}}%}

