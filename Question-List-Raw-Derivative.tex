%%%%%%%%%%%%%%%%%%%%%%%%%%%%%%%%%%%%%%%%%%%%%%%%%%%%%%%%%%%%%%%%%%%%%%%%%%%%%%%
%%%%%%%%%%%%%%%%%%%%%%%%%%%%%%%%%%%%%%%%%%%%%%%%%%%%%%%%%%%%%%%%%%%%%%%%%%%%%%%
%%%%%%%%%%%%%%%%%%%                                     %%%%%%%%%%%%%%%%%%%%%%%
%%%%%%%%%%%%%%%%%%%             Computation             %%%%%%%%%%%%%%%%%%%%%%%
%%%%%%%%%%%%%%%%%%%                                     %%%%%%%%%%%%%%%%%%%%%%%
%%%%%%%%%%%%%%%%%%%%%%%%%%%%%%%%%%%%%%%%%%%%%%%%%%%%%%%%%%%%%%%%%%%%%%%%%%%%%%%
%%%%%%%%%%%%%%%%%%%%%%%%%%%%%%%%%%%%%%%%%%%%%%%%%%%%%%%%%%%%%%%%%%%%%%%%%%%%%%%



%%%%%%%%%%%%%%%%%%%%%%%%%%%%%%%%%%%%%%%%%%%%%%%%%%%%%%%%%%%%%%%%%%%%%%%%%%%%%%%
%%%%%%%%%%%%%%%%%%%         MAC2311: Calculus 1         %%%%%%%%%%%%%%%%%%%%%%%
%%%%%%%%%%%%%%%%%%%%%%%%%%%%%%%%%%%%%%%%%%%%%%%%%%%%%%%%%%%%%%%%%%%%%%%%%%%%%%%


%%%%%%%%%%%%%%%%%%%%%%%
%\tagged{Ans@ShortAns, Type@Compute, Topic@Derivative, Sub@Poly, Sub@Power-Rule, File@0001}{
\begin{sagesilent}
a = NonZeroInt(-5,5)
b = NonZeroInt(-5,5,[0,a])
c = NonZeroInt(-5,5)
    
vright=[(x-a),expand((x-a)*(x-b)),expand((x-a)*(x-b)*(x-c)),1]

first = Integer(randint(0,3))
second = Integer(randint(0,2))

Fone=vright[first]
Ftwo=vright[second]
F=expand(Fone*Ftwo)

Ans=derivative(F,x)

\end{sagesilent}

\latexProblemContent{
\ifVerboseLocation This is Derivative Compute Question 0001. \\ \fi
\begin{problem}

Compute the following derivative:

\input{Derivative-Compute-0001.HELP.tex}

\[
    \dfrac{d}{dx}\left(\sage{F}\right)
    =
    \answer{\sage{Ans}}
\]
\end{problem}}%}
%%%%%%%%%%%%%%%%%%%%%%


%%%%%%%%%%%%%%%%%%%%%%%
%\tagged{Ans@ShortAns, Type@Compute, Topic@Derivative, Func@Trig, File@0002}{
\begin{sagesilent}
w0=SR.wild(0)
def TrigSimp(f):
    ftemp = f
    ftempsec = ftemp.substitute(tan(w0)^2+1 == sec(w0)^2)
    ftempcsc = ftempsec.substitute(cot(w0)^2+1 == csc(w0)^2)
    ffinal = ftempcsc
    return simplify(ffinal)

a = NonZeroInt(-5,5)
b = NonZeroInt(-5,5, [0,a])
    
vleft=[a*sin(b*x*pi), a*cos(b*x*pi), a*tan(b*x*pi), a*sin(b*x*pi/2), a*cos(b*x*pi/2), a*tan(b*x*pi/2), a*sin(b*x*pi/6), a*cos(b*x*pi/6), a*tan(b*x*pi/6), a*sin(b*x*pi/3), a*cos(b*x*pi/3), a*tan(b*x*pi/3)]

first = Integer(randint(0,11))

Fone=vleft[first]

ans=TrigSimp(derivative(Fone,x))
\end{sagesilent}

\latexProblemContent{
\ifVerboseLocation This is Derivative Compute Question 0002. \\ \fi
\begin{problem}

Compute the following derivative:

\input{Derivative-Compute-0002.HELP.tex}

\[
    \dfrac{d}{dx}\left(\sage{Fone}\right)
    =
    \answer{\sage{ans}}
\]
\end{problem}}%}
%%%%%%%%%%%%%%%%%%%%%%

%%%%%%%%%%%%%%%%%%%%%%%
%\tagged{Ans@ShortAns, Type@Compute, Topic@Derivative, Func@Trig, Sub@Product-Rule, File@0003}{
\begin{sagesilent}
w0=SR.wild(0)
def TrigSimp(f):
    ftemp = f
    ftempsec = ftemp.substitute(tan(w0)^2+1 == sec(w0)^2)
    ftempcsc = ftempsec.substitute(cot(w0)^2+1 == csc(w0)^2)
    ffinal = ftempcsc
    return simplify(ffinal)

a = NonZeroInt(-5,5)
b = NonZeroInt(-5,5, [0,a])
c = NonZeroInt(-5,5)   
d=NonZeroInt(-5,5, [0,c])
    
vleft=[a*sin(b*x*pi), a*cos(b*x*pi), a*tan(b*x*pi), a*sin(b*x*pi/2), a*cos(b*x*pi/2), a*tan(b*x*pi/2), a*sin(b*x*pi/6), a*cos(b*x*pi/6), a*tan(b*x*pi/6), a*sin(b*x*pi/3), a*cos(b*x*pi/3), a*tan(b*x*pi/3)]

vright=[c*sin(d*x*pi), c*cos(d*x*pi), c*tan(d*x*pi), c*sin(d*x*pi/2), c*cos(d*x*pi/2), c*tan(d*x*pi/2), c*sin(d*x*pi/6), c*cos(d*x*pi/6), c*tan(d*x*pi/6), c*sin(d*x*pi/3), c*cos(d*x*pi/3), c*tan(d*x*pi/3)]

first = Integer(randint(0,11))
second = Integer(randint(0,11))

Fone=vleft[first]
Ftwo=vright[second]
F=expand(Fone*Ftwo)

Ans=TrigSimp(derivative(F,x))

\end{sagesilent}

\latexProblemContent{
\ifVerboseLocation This is Derivative Compute Question 0003. \\ \fi
\begin{problem}

Compute the following derivative:

\input{Derivative-Compute-0003.HELP.tex}

\[
    \dfrac{d}{dx}\left(\sage{F}\right)
    =
    \answer{\sage{Ans}}
\]

\end{problem}}%}
%%%%%%%%%%%%%%%%%%%%%%

%%%%%%%%%%%%%%%%%%%%%%%
%\tagged{Ans@ShortAns, Type@Compute, Topic@Derivative, Sub@Rational, Sub@Quotient-Rule, File@0004}{
\begin{sagesilent}

a = NonZeroInt(-5,5)
b = NonZeroInt(-5,5, [0,a])
c = NonZeroInt(-5,5, [0,a])
d=NonZeroInt(-5,5, [0,a])
e=NonZeroInt(-5,5, [0,a])
f=NonZeroInt(-5,5)

vleft=[(x-a),expand((x-a)*(x-b)),expand((x-a)*(x-b)*(x-c)),1]
vright=[(x-d),expand((x-d)*(x-e)),expand((x-d)*(x-e)*(x-f))]

first = Integer(randint(0,3))
second = Integer(randint(0,2))

Fone=vleft[first]
Ftwo=vright[second]
F=Fone/Ftwo

fBotDer = derivative(Ftwo,x)
fTopDer = derivative(Fone,x)

FTop = expand(Ftwo*fTopDer-Fone*fBotDer)
FBot = Ftwo*Ftwo

Ans = FTop/FBot

\end{sagesilent}

\latexProblemContent{
\ifVerboseLocation This is Derivative Compute Question 0004. \\ \fi
\begin{problem}

Compute the following derivative:

\input{Derivative-Compute-0004.HELP.tex}

\[
    \dfrac{d}{dx}\left(\sage{F}\right)
    =
    \answer{\sage{Ans}}
\]
\end{problem}}%}
%%%%%%%%%%%%%%%%%%%%%%


%%%%%%%%%%%%%%%%%%%%%%%
%\tagged{Ans@MultiAns, Type@Compute, Topic@Derivative, Sub@FormalDef, Sub@DifferenceQuotient, Sub@Poly, File@0005}{
\begin{sagesilent}
var('h')
a = NonZeroInt(-5,5)
b = NonZeroInt(-4,4)
p = Integer(randint(0,3))
if p == 0:
    b = NonZeroInt(1,4)

v = [
    a*sqrt(x),
    a/x, 
    a*x^2, 
    a*x^3
]

F = v[p]
Fone = F(b+h)
Ftwo = F(b)
DQ = (F(b+h)-F(b)) / h

Ans=limit(DQ,h=0)

\end{sagesilent}

\latexProblemContent{
\ifVerboseLocation This is Derivative Compute Question 0005. \\ \fi
\begin{problem}

Compute the following limit definition of derivative.
\[
    \lim_{h\to0}\frac{\sage{Fone-Ftwo}}{h}
    =
    \answer{\sage{Ans}}
\]
What function is being differentiated? (Assume there is no horizontal translation)
\[
    f(x)
    =
    \answer{\sage{F}}
\]
At what $x-$value are you computing the derivative (given your previous answer)?
\[
    x
    =
    \answer{\sage{b}}
\]

\input{Derivative-Compute-0005.HELP.tex}


\end{problem}}%}
%%%%%%%%%%%%%%%%%%%%%%


%%%%%%%%%%%%%%%%%%%%%%%
%\tagged{Ans@ShortAns, Type@Compute, Topic@Derivative, Sub@Log, File@0006}{
\begin{sagesilent}
a = NonZeroInt(-10,10)
c = RandInt(-15,15)
g = a*log(x)+c
Ans = derivative(g,x)

\end{sagesilent}

\latexProblemContent{
\ifVerboseLocation This is Derivative Compute Question 0006. \\ \fi
\begin{problem}

Compute the following derivative:

\input{Derivative-Compute-0006.HELP.tex}

\[
    \frac{d}{dx}\left(\sage{g}\right)
    =
    \answer{\sage{Ans}}
\] 
\end{problem}}%}
%%%%%%%%%%%%%%%%%%%%%%


%%%%%%%%%%%%%%%%%%%%%%%
%\tagged{Ans@ShortAns, Type@Compute, Topic@Derivative, Sub@Log, Sub@Chain-Rule, File@0007}{
\begin{sagesilent}
a = NonZeroInt(-5,5)
b = NonZeroInt(-5,5)
c = NonZeroInt(-5,5)
f = a*log(b*x+c)
Ans = derivative(f,x)
\end{sagesilent}

\latexProblemContent{
\ifVerboseLocation This is Derivative Compute Question 0007. \\ \fi
\begin{problem}

Compute the following derivative:

\input{Derivative-Compute-0007.HELP.tex}

\[
    \dfrac{d}{dx}\left(\sage{f}\right)
    =
    \answer{\sage{Ans}}
\] 
\end{problem}}%}
%%%%%%%%%%%%%%%%%%%%%%

%%%%%%%%%%%%%%%%%%%%%%%
%\tagged{Ans@ShortAns, Type@Compute, Topic@Derivative, Sub@Exp, File@0008}{
\begin{sagesilent}
a = NonZeroInt(-5,5)
b = Integer(randint(-5,5))
g=a*exp(b+x)
Ans = derivative(g,x)
\end{sagesilent}

\latexProblemContent{
\ifVerboseLocation This is Derivative Compute Question 0008. \\ \fi
\begin{problem}

Compute the following derivative:

\input{Derivative-Compute-0008.HELP.tex}

\[
    \dfrac{d}{dx}\left(\sage{g}\right)
    =
    \answer{\sage{Ans}}
\]
\end{problem}}%}
%%%%%%%%%%%%%%%%%%%%%%


%%%%%%%%%%%%%%%%%%%%%%%
%\tagged{Ans@ShortAns, Type@Compute, Topic@Derivative, Sub@Exp, Sub@Chain-Rule, File@0009}{
\begin{sagesilent}
a = NonZeroInt(-5,5)
b = Integer(randint(-5,5))
c = NonZeroInt(-5,5)
f=a*exp(c*x+b)
Ans = derivative(f,x)
\end{sagesilent}

\latexProblemContent{
\ifVerboseLocation This is Derivative Compute Question 0009. \\ \fi
\begin{problem}

Compute the following derivative:

\input{Derivative-Compute-0009.HELP.tex}

\[
    \dfrac{d}{dx}\left(\sage{f}\right)
    =
    \answer{\sage{Ans}}
\] 
\end{problem}}%}
%%%%%%%%%%%%%%%%%%%%%%


%%%%%%%%%%%%%%%%%%%%%%%
%\tagged{Ans@ShortAns, Type@Compute, Topic@Derivative, Sub@Product-Rule, File@0010}{
\begin{sagesilent}
a = NonZeroInt(-5,5)
b = NonZeroInt(-5,5)

p=Integer(randint(0,8))
q=Integer(randint(0,8))
while p==q:
    q=Integer(randint(0,8))

v = [
    sqrt(x), 
    log(x), 
    exp(x), 
    x^2, 
    x^3, 
    x^4, 
    (x-b), 
    sin(x), 
    cos(x)
]


Fone=v[p]
Ftwo=v[q]
F=a*Fone*Ftwo

Ans = derivative(F,x)
\end{sagesilent}

\latexProblemContent{
\ifVerboseLocation This is Derivative Compute Question 0010. \\ \fi
\begin{problem}

Compute the following derivative:

\input{Derivative-Compute-0010.HELP.tex}

\[
    \dfrac{d}{dx}\left(\sage{F}\right)
    =
    \answer{\sage{Ans}}
\]
\end{problem}}%}
%%%%%%%%%%%%%%%%%%%%%%

%%%%%%%%%%%%%%%%%%%%%%%
%\tagged{Ans@ShortAns, Type@Compute, Topic@Derivative, Sub@Quotient-Rule, File@0011}{
\begin{sagesilent}
a = NonZeroInt(-5,5)
b = NonZeroInt(-5,5)

p=Integer(randint(0,8))
q=Integer(randint(0,8))
while p==q:
    q=Integer(randint(0,8))

v = [
    sqrt(x), 
    log(x), 
    exp(x), 
    x^2, 
    x^3, 
    x^4, 
    (x-b), 
    sin(x), 
    cos(x)
]

Fone=v[p]
Ftwo=v[q]
F=a*Fone/Ftwo

Ans = derivative(F,x)
\end{sagesilent}

\latexProblemContent{
\ifVerboseLocation This is Derivative Compute Question 0011. \\ \fi
\begin{problem}

Compute the following derivative:

\input{Derivative-Compute-0011.HELP.tex}

\[
    \dfrac{d}{dx}\left(\sage{F}\right)
    =
    \answer{\sage{Ans}}
\] 
\end{problem}}%}
%%%%%%%%%%%%%%%%%%%%%%


%%%%%%%%%%%%%%%%%%%%%%%
%\tagged{Ans@ShortAns, Type@Compute, Topic@Derivative, Sub@Chain-Rule, File@0012}{
\begin{sagesilent}
a = NonZeroInt(-5,5)
b = NonZeroInt(-5,5)

funcVec = [
    sqrt(x), 
    log(x), 
    exp(x), 
    x^2, 
    x^3, 
    x^4, 
    (x-b), 
    sin(x), 
    cos(x)
]

funcOne = choice(funcVec)
funcTwo = choice(funcVec)

while funcOne == funcTwo:
    # This will ensure the functions are different.
    funcTwo = choice(funcVec)

func = a * funcOne(x = funcTwo)

ans = derivative(func,x)
\end{sagesilent}

\latexProblemContent{
\ifVerboseLocation This is Derivative Compute Question 0012. \\ \fi
\begin{problem}

Compute the following derivative:

\input{Derivative-Compute-0012.HELP.tex}

\[
    \frac{d}{dx}\left(\sage{func}\right)
    =
    \answer{\sage{ans}}
\]
\end{problem}}%}
%%%%%%%%%%%%%%%%%%%%%%


%%%%%%%%%%%%%%%%%%%%%%%
%\tagged{Ans@ShortAns, Type@Compute, Topic@Derivative, Sub@Tan-Line, Sub@Poly, File@0013}{
\begin{sagesilent}

slope = 100
y_val = 100

while abs(slope) >= 100 or abs(y_val) >= 100:
    x_val = RandInt(-9,9) # x value of point at which we take the tangent line

    deg = RandInt(2,3)
    poly = RandPoly(deg)

    y_val = poly.subs(x = x_val)

    slope = derivative(poly, x).subs(x = x_val)
    ans = slope * (x - x_val) + y_val

\end{sagesilent}

\latexProblemContent{
\ifVerboseLocation This is Derivative Compute Question 0013. \\ \fi
\begin{problem}

Find the equation of the line tangent to $f(x)=\sage{poly}$ at the point $(\sage{x_val}, \sage{y_val})$.

\input{Derivative-Compute-0013.HELP.tex}

\[
    y
    =
    \answer{\sage{ans}}
\] 
\end{problem}}%}
%%%%%%%%%%%%%%%%%%%%%%




%%%%%%%%%%%%%%%%%%%%%%%
%\tagged{Ans@ShortAns, Type@Compute, Topic@Derivative, Sub@Tan-Line, Func@Trig, File@0014}{
\begin{sagesilent}
a = RandAng(0,pi)
b = NonZeroInt(-3,3)
p=Integer(randint(0,1))
v=[b*sin(x), b*cos(x)]
F=v[p]

if p==0:
    Ans=b*cos(a)*x-a*b*cos(a)+b*sin(a)
else:
    Ans = -b*sin(a)*x+a*b*sin(a)+b*cos(a)
\end{sagesilent}

\latexProblemContent{
\ifVerboseLocation This is Derivative Compute Question 0014. \\ \fi
\begin{problem}

Find the equation of the line tangent to $f(x)=\sage{F}$ at $x=\sage{a}$.

\input{Derivative-Compute-0014.HELP.tex}

\[y=\answer{\sage{Ans}}\] 
\end{problem}}%}
%%%%%%%%%%%%%%%%%%%%%%

%%%%%%%%%%%%%%%%%%%%%%%
%\tagged{Ans@ShortAns, Type@Compute, Topic@Derivative, Sub@Tan-Line, Func@Log, File@0015}{
\begin{sagesilent}
a = RandInt(1,10)
b = NonZeroInt(-8,8)
c = RandInt(-3,3)
d = RandInt(-3,3)

base = RandInt(2,8)

v=[b*log(x), b*exp(x), b*base^x]
p = 0#RandInt(0,2)

F=v[p]

if p == 0:
    tar = abs(c) + 1
else:
    tar = c

derv(x) = derivative(F(x),x)
slope = derv(tar)

Ans = slope*(x-tar) + (F(tar))



\end{sagesilent}

\latexProblemContent{
\ifVerboseLocation This is Derivative Compute Question 0015. \\ \fi
\begin{problem}

Find the equation of the line tangent to $f(x)=\sage{F}$ at $x=\sage{tar}$.

\input{Derivative-Compute-0015.HELP.tex}

\[y=\answer{\sage{Ans}}\] 
\end{problem}}%}
%%%%%%%%%%%%%%%%%%%%%%



%%%%%%%%%%%%%%%%%%%%%%%
%\tagged{Ans@ShortAns, Type@Compute, Topic@Derivative, Sub@Implicit, Sub@Poly, File@0016}{
\begin{sagesilent}
def FixDeriv(f):
    ftemp=f
    ftempderiv = ftemp.substitute(diff(w) == dy/dx)
    #ftempderivv = ftempderiv.substitute(diff(diff(y)) == (d^2*y)/(d*x^2))
    ffinal = ftempderiv
    return ffinal

def RegY(f):
    ftemp=f
    ftempnew = ftemp.substitute(w(x) == y)
    ffinal=ftempnew
    return ffinal


a = NonZeroInt(-5,5)
b = RandInt(1,10)
p=Integer(randint(0,8))
w=function('w',x)
v=[x*w, x^2*w, x^3*w, x*w^2, x*w^3, x^2*w^2, x^2*w^3, x^3*w^2, x^3*w^3]

F=b*v[p]
f=diff(F, x)
g=solve(diff(F),diff(w))
f2=g[0]
f3=FixDeriv(f2)
f4=RegY(f3)
f5=f4.rhs()
G=RegY(F)
\end{sagesilent}

\latexProblemContent{
\ifVerboseLocation This is Derivative Compute Question 0016. \\ \fi
\begin{problem}

Implicitly derive the expression $\sage{G}=\sage{a}$ and solve for $\dfrac{dy}{dx}$.

\input{Derivative-Compute-0016.HELP.tex}

\[\dfrac{dy}{dx}=\answer{\sage{f5}}\] 
\end{problem}}%}
%%%%%%%%%%%%%%%%%%%%%%


%%%%%%%%%%%%%%%%%%%%%%%
%\tagged{Ans@ShortAns, Type@Compute, Topic@Derivative, Sub@Implicit, Func@Trig, Sub@Poly, File@0017}{
\begin{sagesilent}
def FixDeriv(f):
    ftemp=f
    ftempderiv = ftemp.substitute(diff(w) == dy/dx)
    #ftempderivv = ftempderiv.substitute(diff(diff(y)) == (d^2*y)/(d*x^2))
    ffinal = ftempderiv
    return ffinal

def RegY(f):
    ftemp=f
    ftempnew = ftemp.substitute(w(x) == y)
    ffinal=ftempnew
    return ffinal


a = NonZeroInt(-5,5)
b = RandInt(1,10)
p=RandInt(0,7)
w=function('w',x)
v=[sin(x*w), sin(x^2*w), cos(x*w), sin(x*w^2), cos(x^2*w), cos(x*w^2), sin(x^2*w^2), cos(x^2*w^2)]

F=b*v[p]
f=diff(F, x)
g=solve(diff(F),diff(w))
f2=g[0]
f3=FixDeriv(f2)
f4=RegY(f3)
f5=f4.rhs()
G=RegY(F)
\end{sagesilent}

\latexProblemContent{
\ifVerboseLocation This is Derivative Compute Question 0017. \\ \fi
\begin{problem}

Implicitly derive the expression $\sage{G}=\sage{a}$ and solve for $\dfrac{dy}{dx}$.

\input{Derivative-Compute-0017.HELP.tex}

\[\dfrac{dy}{dx}=\answer{\sage{f5}}\] 
\end{problem}}%}
%%%%%%%%%%%%%%%%%%%%%%


%%%%%%%%%%%%%%%%%%%%%%%
%\tagged{Ans@ShortAns, Type@Compute, Topic@Derivative, Sub@Implicit, Func@Trig, Sub@Poly, File@0018}{
\begin{sagesilent}
def FixDeriv(f):
    ftemp=f
    ftempderiv = ftemp.substitute(diff(w) == dy/dx)
    #ftempderivv = ftempderiv.substitute(diff(diff(y)) == (d^2*y)/(d*x^2))
    ffinal = ftempderiv
    return ffinal

def RegY(f):
    ftemp=f
    ftempnew = ftemp.substitute(w(x) == y)
    ffinal=ftempnew
    return ffinal


a = NonZeroInt(-5,5)
b = RandInt(1,10)
p=Integer(randint(0,7))
w=function('w',x)
v=[x*sin(w), x^2*sin(w), x*cos(w), x*sin(w^2), x^2*cos(w), x*cos(w^2), x^2*sin(w^2), x^2*cos(w^2)]

F=b*v[p]
f=diff(F, x)
g=solve(diff(F),diff(w))
f2=g[0]
f3=FixDeriv(f2)
f4=RegY(f3)
f5=f4.rhs()
G=RegY(F)
\end{sagesilent}

\latexProblemContent{
\ifVerboseLocation This is Derivative Compute Question 0018. \\ \fi
\begin{problem}

Implicitly derive the expression $\sage{G}=\sage{a}$ and solve for $\dfrac{dy}{dx}$.

\input{Derivative-Compute-0018.HELP.tex}

\[\dfrac{dy}{dx}=\answer{\sage{f5}}\] 
\end{problem}}%}
%%%%%%%%%%%%%%%%%%%%%%


%%%%%%%%%%%%%%%%%%%%%%%
%\tagged{Ans@ShortAns, Type@Compute, Topic@Derivative, Sub@Implicit, Func@Trig, Sub@Poly, File@0019}{
\begin{sagesilent}
def FixDeriv(f):
    ftemp=f
    ftempderiv = ftemp.substitute(diff(w) == dy/dx)
    #ftempderivv = ftempderiv.substitute(diff(diff(y)) == (d^2*y)/(d*x^2))
    ffinal = ftempderiv
    return ffinal

def RegY(f):
    ftemp=f
    ftempnew = ftemp.substitute(w(x) == y)
    ffinal=ftempnew
    return ffinal


a = NonZeroInt(-3,3)
b = RandInt(1,10)
p=Integer(randint(0,7))
w=function('w',x)
v=[w*sin(x), w^2*sin(x), w*cos(x), w*sin(x^2), w^2*cos(x), w*cos(x^2), w^2*sin(x^2), w^2*cos(x^2)]

F=b*v[p]
f=diff(F, x)
g=solve(diff(F),diff(w))
f2=g[0]
f3=FixDeriv(f2)
f4=RegY(f3)
f5=f4.rhs()
G=RegY(F)
\end{sagesilent}

\latexProblemContent{
\ifVerboseLocation This is Derivative Compute Question 0019. \\ \fi
\begin{problem}

Implicitly derive the expression $\sage{G}=\sage{a}$ and solve for $\dfrac{dy}{dx}$.

\input{Derivative-Compute-0019.HELP.tex}

\[\dfrac{dy}{dx}=\answer{\sage{f5}}\] 
\end{problem}}%}
%%%%%%%%%%%%%%%%%%%%%%


%%%%%%%%%%%%%%%%%%%%%%%
%\tagged{Ans@ShortAns, Type@Compute, Topic@Derivative, Sub@Implicit, Sub@Exp, Sub@Poly, File@0020}{
\begin{sagesilent}
def FixDeriv(f):
    ftemp=f
    ftempderiv = ftemp.substitute(diff(w) == dy/dx)
    #ftempderivv = ftempderiv.substitute(diff(diff(y)) == (d^2*y)/(d*x^2))
    ffinal = ftempderiv
    return ffinal

def RegY(f):
    ftemp=f
    ftempnew = ftemp.substitute(w(x) == y)
    ffinal=ftempnew
    return ffinal


a = NonZeroInt(-5,5)
b = RandInt(1,10)
p=Integer(randint(0,8))
w=function('w',x)
v=[exp(x*w), exp(x^2*w), exp(x*w^2), exp(x^2*w^2), exp(x^2*w^3), exp(x^3*w^2), exp(x^3*w^3), exp(x*w^3), exp(x^3*w)]

F=b*v[p]
f=diff(F, x)
g=solve(diff(F),diff(w))
f2=g[0]
f3=FixDeriv(f2)
f4=RegY(f3)
f5=f4.rhs()
G=RegY(F)
\end{sagesilent}

\latexProblemContent{
\ifVerboseLocation This is Derivative Compute Question 0020. \\ \fi
\begin{problem}

Implicitly derive the expression $\sage{G}=\sage{a}$ and solve for $\dfrac{dy}{dx}$.

\input{Derivative-Compute-0020.HELP.tex}

\[\dfrac{dy}{dx}=\answer{\sage{f5}}\] 
\end{problem}}%}
%%%%%%%%%%%%%%%%%%%%%%


%%%%%%%%%%%%%%%%%%%%%%%
%\tagged{Ans@ShortAns, Type@Compute, Topic@Derivative, Sub@Implicit, Sub@Exp, Sub@Poly, File@0021}{
\begin{sagesilent}
def FixDeriv(f):
    ftemp=f
    ftempderiv = ftemp.substitute(diff(w) == dy/dx)
    #ftempderivv = ftempderiv.substitute(diff(diff(y)) == (d^2*y)/(d*x^2))
    ffinal = ftempderiv
    return ffinal

def RegY(f):
    ftemp=f
    ftempnew = ftemp.substitute(w(x) == y)
    ffinal=ftempnew
    return ffinal


a = NonZeroInt(-20,20)
b = RandInt(1,10)

p=Integer(randint(0,8))
w=function('w',x)
v=[x*exp(x*w), x*exp(x^2*w), x*exp(x*w^2), x*exp(x^2*w^2), x*exp(x^2*w^3), x*exp(x^3*w^2), x*exp(x^3*w^3), x*exp(x*w^3), x*exp(x^3*w)]

F=b*v[p]
f=diff(F, x)
g=solve(diff(F),diff(w))
f2=g[0]
f3=FixDeriv(f2)
f4=RegY(f3)
f5=f4.rhs()
G=RegY(F)
\end{sagesilent}

\latexProblemContent{
\ifVerboseLocation This is Derivative Compute Question 0021. \\ \fi
\begin{problem}

Implicitly derive the expression $\sage{G}=\sage{a}$ and solve for $\dfrac{dy}{dx}$.

\input{Derivative-Compute-0021.HELP.tex}

\[\dfrac{dy}{dx}=\answer{\sage{f5}}\] 
\end{problem}}%}
%%%%%%%%%%%%%%%%%%%%%%


%%%%%%%%%%%%%%%%%%%%%%%
%\tagged{Ans@ShortAns, Type@Compute, Topic@Derivative, Sub@Implicit, Sub@Exp, Sub@Poly, File@0022}{
\begin{sagesilent}
def FixDeriv(f):
    ftemp=f
    ftempderiv = ftemp.substitute(diff(w) == dy/dx)
    #ftempderivv = ftempderiv.substitute(diff(diff(y)) == (d^2*y)/(d*x^2))
    ffinal = ftempderiv
    return ffinal

def RegY(f):
    ftemp=f
    ftempnew = ftemp.substitute(w(x) == y)
    ffinal=ftempnew
    return ffinal


a = NonZeroInt(-5,5)
b = RandInt(1,10)
p=Integer(randint(0,8))
w=function('w',x)
v=[w*exp(x*w), w*exp(x^2*w), w*exp(x*w^2), w*exp(x^2*w^2), w*exp(x^2*w^3), w*exp(x^3*w^2), w*exp(x^3*w^3), w*exp(x*w^3), w*exp(x^3*w)]

F=b*v[p]
f=diff(F, x)
g=solve(diff(F),diff(w))
f2=g[0]
f3=FixDeriv(f2)
f4=RegY(f3)
f5=f4.rhs()
G=RegY(F)
\end{sagesilent}

\latexProblemContent{
\ifVerboseLocation This is Derivative Compute Question 0022. \\ \fi
\begin{problem}

Implicitly derive the expression $\sage{G}=\sage{a}$ and solve for $\dfrac{dy}{dx}$.

\input{Derivative-Compute-0022.HELP.tex}

\[\dfrac{dy}{dx}=\answer{\sage{f5}}\] 
\end{problem}}%}
%%%%%%%%%%%%%%%%%%%%%%


%%%%%%%%%%%%%%%%%%%%%%%
%\tagged{Ans@ShortAns, Type@Compute, Topic@Derivative, Sub@Implicit, Sub@Exp, Sub@Poly, File@0023}{
\begin{sagesilent}
def FixDeriv(f):
    ftemp=f
    ftempderiv = ftemp.substitute(diff(w) == dy/dx)
    #ftempderivv = ftempderiv.substitute(diff(diff(y)) == (d^2*y)/(d*x^2))
    ffinal = ftempderiv
    return ffinal

def RegY(f):
    ftemp=f
    ftempnew = ftemp.substitute(w(x) == y)
    ffinal=ftempnew
    return ffinal


a = NonZeroInt(-5,5)
b = RandInt(1,10)
p=RandInt(0,8)
w=function('w',x)
v=[x*exp(w), x^2*exp(w), x*exp(w^2), x^2*exp(w^2), x^2*exp(w^3), x^3*exp(w^2), x^3*exp(w^3), x*exp(w^3), x^3*exp(w)]

F=b*v[p]
f=diff(F, x)
g=solve(diff(F),diff(w))
f2=g[0]
f3=FixDeriv(f2)
f4=RegY(f3)
f5=f4.rhs()
G=RegY(F)
\end{sagesilent}

\latexProblemContent{
\ifVerboseLocation This is Derivative Compute Question 0023. \\ \fi
\begin{problem}

Implicitly derive the expression $\sage{G}=\sage{a}$ and solve for $\dfrac{dy}{dx}$.

\input{Derivative-Compute-0023.HELP.tex}

\[\dfrac{dy}{dx}=\answer{\sage{f5}}\] 
\end{problem}}%}
%%%%%%%%%%%%%%%%%%%%%%

%%%%%%%%%%%%%%%%%%%%%%%
%\tagged{Ans@ShortAns, Type@Compute, Topic@Derivative, Sub@Implicit, Sub@Log, Sub@Poly, File@0024}{
\begin{sagesilent}
def FixDeriv(f):
    ftemp=f
    ftempderiv = ftemp.substitute(diff(w) == dy/dx)
    #ftempderivv = ftempderiv.substitute(diff(diff(y)) == (d^2*y)/(d*x^2))
    ffinal = ftempderiv
    return ffinal

def RegY(f):
    ftemp=f
    ftempnew = ftemp.substitute(w(x) == y)
    ffinal=ftempnew
    return ffinal


a = RandInt(-10,10)
b = RandInt(1,10)
p=Integer(randint(0,8))
w=function('w',x)
v=[log(x*w), log(x^2*w), log(x*w^2), log(x^2*w^2), log(x^2*w^3), log(x^3*w^2), log(x^3*w^3), log(x*w^3), log(x^3*w)]

F=b*v[p]
f=diff(F, x)
g=solve(diff(F),diff(w))
f2=g[0]
f3=FixDeriv(f2)
f4=RegY(f3)
f5=f4.rhs()
G=RegY(F)
\end{sagesilent}

\latexProblemContent{
\ifVerboseLocation This is Derivative Compute Question 0024. \\ \fi
\begin{problem}

Implicitly derive the expression $\sage{G}=\sage{a}$ and solve for $\dfrac{dy}{dx}$.

\input{Derivative-Compute-0024.HELP.tex}

\[\dfrac{dy}{dx}=\answer{\sage{f5}}\] 
\end{problem}}%}
%%%%%%%%%%%%%%%%%%%%%%


%%%%%%%%%%%%%%%%%%%%%%%
%\tagged{Ans@ShortAns, Type@Compute, Topic@Derivative, Sub@Domain, Sub@Rational, File@0025}{
\begin{sagesilent}
a = NonZeroInt(-25,25)
num = RandInt(1,10)
pwr = RandInt(1,5)
F(x) = num/(x-a)^pwr

\end{sagesilent}

\latexProblemContent{
\ifVerboseLocation This is Derivative Compute Question 0025. \\ \fi
\begin{problem}

Find each interval over which $f(x)=\sage{F(x)}$ is differentiable.

\input{Derivative-Compute-0025.HELP.tex}

\[\answer{(-\infty,\sage{a})\cup(\sage{a},\infty)}\]

\end{problem}}%}
%%%%%%%%%%%%%%%%%%%%%%


%%%%%%%%%%%%%%%%%%%%%%%
%\tagged{Ans@ShortAns, Type@Compute, Topic@Derivative, Sub@Domain, Sub@Rational, File@0026}{
\begin{sagesilent}
a = NonZeroInt(-5,5)
b = NonZeroInt(-5,5,[0,a])
p=Integer(randint(0,2))
v=[1/((x-a)*(x-b)), 1/((x-a)^2*(x-b)), 1/((x-a)*(x-b)^2)]
F=v[p]
\end{sagesilent}

\latexProblemContent{
\ifVerboseLocation This is Derivative Compute Question 0026. \\ \fi
\begin{problem}

Find each interval over which $f(x)=\sage{F}$ is differentiable.

\input{Derivative-Compute-0026.HELP.tex}

\[\answer{(-\infty,\sage{a})\cup(\sage{a},\sage{b})\cup(\sage{b},\infty)}\]
\end{problem}}%}
%%%%%%%%%%%%%%%%%%%%%%

%%%%%%%%%%%%%%%%%%%%%%%
%\tagged{Ans@ShortAns, Type@Compute, Topic@Derivative, Sub@Poly, Sub@Power-Rule, File@0027}{
\begin{sagesilent}
a = NonZeroInt(-5,5)
b = NonZeroInt(-5,5, [0,a])
p=Integer(randint(0,5))
v=[x^2, x^3, x^4, expand((x-a)*(x-b)), expand((x-a)^2*(x-b)), expand((x-a)*(x-b)^2)]
F=v[p]
Ans=diff(F,x)
\end{sagesilent}

\latexProblemContent{
\ifVerboseLocation This is Derivative Compute Question 0027. \\ \fi
\begin{problem}

Compute the following derivative:

\input{Derivative-Compute-0027.HELP.tex}

\[\dfrac{d}{dx}\left(\sage{F}\right)=\answer{\sage{Ans}}\]
\end{problem}}%}
%%%%%%%%%%%%%%%%%%%%%%

%%% SO FAR 27-Compute %%%%%%%%


%%%%%%%%%%%%%%%%%%%%%%%
%\tagged{Ans@ShortAns, Type@Compute, Topic@Derivative, Sub@Arctrig, File@0028}{
\begin{sagesilent}
a = NonZeroInt(-5,5)
b = NonZeroInt(-5,5, [0,a])
p=Integer(randint(0,2))
v=[a*arcsin(b*x), a*arccos(b*x), a*arctan(b*x)]
F=v[p]
Ans=diff(F,x)
\end{sagesilent}

\latexProblemContent{
\ifVerboseLocation This is Derivative Compute Question 0028. \\ \fi
\begin{problem}

Compute the following derivative:

\input{Derivative-Compute-0028.HELP.tex}

\[\dfrac{d}{dx}\left(\sage{F}\right)=\answer{\sage{Ans}}\]
\end{problem}}%}
%%%%%%%%%%%%%%%%%%%%%%



%%%%%%%%%%%%%%%%%%%%%%%
%\tagged{Ans@ShortAns, Type@Compute, Topic@Derivative, Sub@Related-Rates, File@0029}{
\begin{sagesilent}
unitvec = ['cm', 'in', 'mm', 'line']
descvec = ['', '', '', ' (a line is 1/12 of an inch)']
p = RandInt(0,3)
unit = unitvec[p]
unitdesc = descvec[p]

R = RandInt(2,10)
V = 4/3*R^3

dR = RandInt(1,10)
dV = 4*R^2*dR

Ans = dR
\end{sagesilent}

\latexProblemContent{
\ifVerboseLocation This is Derivative Compute Question 0029. \\ \fi
\begin{problem}

A snowball is melting at a rate of $\sage{dV}\pi$ $\sage{unit}^3/s\sage{unitdesc}$.  At what rate is the radius decreasing when the volume of the snowball is $\sage{V}\pi$ $\sage{unit}^3$?\\[0.5in]

\input{Derivative-Compute-0029.HELP.tex}

The radius is decreasing at $\answer{\sage{Ans}}$ $\sage{unit}/s$.
\end{problem}}%}
%%%%%%%%%%%%%%%%%%%%%%




%%%%%%%%%%%%%%%%%%%%%%%
%\tagged{Ans@ShortAns, Type@Compute, Topic@Derivative, Sub@Related-Rates, File@0030}{
\begin{sagesilent}

# Set up Pythagorean triples by having one entry in each vector.
v1 = [3, 5, 8]
v2 = [4, 12, 15]
v3 = [5, 13, 17]

# Choose a triple
p = RandInt(0,2)

# Assign base vectors
a1 = v1[p]# Horizontal vector
b1 = v2[p]# Vertical Vector
c1 = v3[p]# Vector Sum

# Scale the vectors, we can do so independently as long as they form (mathematically) similar triangles.
speedscale = RandInt(3,15)*10
distscale = RandInt(1,8)

# Speeds based on the Pythagorean triple.
Sa2 = a1*speedscale
Sb2 = b1*speedscale
Sc2 = c1*speedscale

# Distances based on the Pythagorean triple.
Da2 = a1*distscale
Db2 = b1*distscale
Dc2 = c1*distscale

Ans = Sc2
\end{sagesilent}

\latexProblemContent{
\ifVerboseLocation This is Derivative Compute Question 0030. \\ \fi
\begin{problem}

An experimental plane, currently overhead, is flying at $\sage{Sa2}$ mph at an altitude of $\sage{Db2}$ thousand feet.  How fast is the plane's distance from you increasing at
the moment when the plane is flying over a point on the ground $\sage{Da2}$ thousand feet from you?\\[0.5in]

\input{Derivative-Compute-0030.HELP.tex}

The distance between you and the plane is increasing by $\answer{\sage{Ans}}$ mph.
\end{problem}}%}
%%%%%%%%%%%%%%%%%%%%%%


%%%%%%%%%%%%%%%%%%%%%%%
%\tagged{Ans@ShortAns, Type@Compute, Topic@Derivative, Sub@Related-Rates, File@0031}{
\begin{sagesilent}
unitvec = ['miles', 'kilometers', 'hundred meters', 'thousand feet']
speedvec = ['miles per hour', 'kilometers per hour', 'meters per second', 'feet per second']
unitchoice = RandInt(0,3)
unit = unitvec[unitchoice]
speed = speedvec[unitchoice]

# Set up Pythagorean triples by having one entry in each vector.
v1 = [3, 5, 8]
v2 = [4, 12, 15]
v3 = [5, 13, 17]

# Choose a triple
p = RandInt(0,2)

a1 = v1[p]
a2 = v2[p]
a3 = v3[p]

speedscale = RandInt(2,10)/(p+1)
distscale = RandInt(1,15)

Da = a1*distscale
Db = a2*distscale
Dc = a3*distscale

A = speedscale*a2
B = speedscale*a1
Ans=((Da*A)+(Db*B))/Dc

\end{sagesilent}

\latexProblemContent{
\ifVerboseLocation This is Derivative Compute Question 0031. \\ \fi
\begin{problem}

A road running north to south crosses a road going east to west at the point $P$.  Cyclist $A$ is riding north along the first road, and cyclist $B$ is riding east along the second road.  At a particular time, cyclist $A$ is $\sage{Da}$ $\sage{unit}$ to the north of $P$ and traveling at $\sage{A}$ $\sage{speed}$, while cyclist $B$ is $\sage{Db}$ $\sage{unit}$ to the east of $P$ and traveling at $\sage{B}$ $\sage{speed}$.  How fast is the distance between the two cyclists changing? \\[0.5in]

\input{Derivative-Compute-0031.HELP.tex}

The distance between the two cyclists is increasing by $\answer{\sage{Ans}}$ $\sage{speed}$.
\end{problem}}%}
%%%%%%%%%%%%%%%%%%%%%%


%%%%%%%%%%%%% 31 Compute %%%%%%%%%%%%%%%



%%%%%%%%%%%%%%%%%%%%%%%
%\tagged{Ans@ShortAns, Type@Compute, Topic@Derivative, Sub@Related-Rates, File@0032}{
\begin{sagesilent}

unitvec = ['meters', 'feet', 'yards']
unitchoice = RandInt(0,2)
unit = unitvec[unitchoice]

# Set up Pythagorean triples by having one entry in each vector.
v1 = [3, 5, 8]
v2 = [4, 12, 15]
v3 = [5, 13, 17]

# Choose a triple
p = RandInt(0,2)

scale = RandInt(1,4)

A = scale*v3[p]
B = scale*v1[p]
Ans=v1[p]/v2[p]
\end{sagesilent}

\latexProblemContent{
\ifVerboseLocation This is Derivative Compute Question 0032. \\ \fi
\begin{problem}

A swing consists of a board at the end of a $\sage{A}$ $\sage{unit}$ long rope.  Think of the board as a point $P$ at the end of the rope, and let $Q$ be the point of attachment at the other end.  Suppose that the swing is directly below $Q$ at time $t=0$, and is being pushed by someone who walks at $\sage{B}$ $\sage{unit}$ per second from left to right.  What is the angular speed of the rope in rad/s after 1 sec? \\[0.5in]

\input{Derivative-Compute-0032.HELP.tex}

The angular speed of the rope is $\answer{\sage{Ans}}$ rad/s.
\end{problem}}%}
%%%%%%%%%%%%%%%%%%%%%%



%%%%%%%%%%%%%%%%%%%%%%%
%\tagged{Ans@ShortAns, Type@Compute, Topic@Derivative, Sub@Related-Rates, File@0033}{
\begin{sagesilent}
a = NonZeroInt(10,16)
b = NonZeroInt(4,6)
B = a-b
c = NonZeroInt(3,9)
Ans=b
\end{sagesilent}

\latexProblemContent{
\ifVerboseLocation This is Derivative Compute Question 0033. \\ \fi
\begin{problem}

It is night. Someone who is $\sage{b}$ feet tall is walking away from a street light at a rate of $\sage{B}$ feet per second.  The street light is $\sage{a}$ feet tall.  The person casts a shadow on the ground in front of them. How fast is the length of the shadow growing when the person is $\sage{c}$ feet from the street light? \\[0.5in]

\input{Derivative-Compute-0033.HELP.tex}

The length of the shadow is growing at a rate of $\answer{\sage{Ans}}$ ft/s.
\end{problem}}%}
%%%%%%%%%%%%%%%%%%%%%%


%%%%%%%%%%% 33 - Compute %%%%%%%%%%


%%%%%%%%%%%%%%%%%%%%%%%
%\tagged{Ans@ShortAns, Type@Compute, Topic@Derivative, Sub@Differential, Sub@Rational, File@0034}{
\begin{sagesilent}
a = NonZeroInt(-5,5)
b = NonZeroInt(-5,5)
c = NonZeroInt(-5,5)

funcVec = [
    a/(x-b), 
    a/(x-b)^2, 
    a/expand((x-b)*(x-c))
]

F = choice(funcVec)
f = diff(F, x)

ans = f
\end{sagesilent}

\latexProblemContent{
\ifVerboseLocation This is Derivative Compute Question 0034. \\ \fi
\begin{problem}

Compute the differential of the function $y=\sage{F}$.

\input{Derivative-Compute-0034.HELP.tex}

\[
    dy
    =
    \answer{\left(\sage{ans}\right)\,dx}
\]
\end{problem}}%}
%%%%%%%%%%%%%%%%%%%%%%

%%%%%%%%%%%%%%%%%%%%%%%
%\tagged{Ans@ShortAns, Type@Compute, Topic@Derivative, Sub@Differential, File@0035}{
\begin{sagesilent}
a = NonZeroInt(-5,5)
b = NonZeroInt(-5,5)

p=Integer(randint(0,8))

v = [
    sqrt(x), 
    log(x), 
    exp(x), 
    x^2, 
    x^3, 
    x^4, 
    (x-b), 
    sin(x), 
    cos(x)
]

F=a*v[p]
f=diff(F, x)
ans=f
\end{sagesilent}

\latexProblemContent{
\ifVerboseLocation This is Derivative Compute Question 0035. \\ \fi
\begin{problem}

Compute the differential of the function $y=\sage{F}$.

\input{Derivative-Compute-0035.HELP.tex}

\[
    dy
    =
    \answer{\left(\sage{ans}\right)\,dx}
\]
\end{problem}}%}
%%%%%%%%%%%%%%%%%%%%%%

%%%%%%%%%%%%%%%%%%%%%%%
%\tagged{Ans@ShortAns, Type@Compute, Topic@Derivative, Sub@Differential, Sub@Chain-Rule, File@0036}{
\begin{sagesilent}
a = NonZeroInt(-5,5)
b = NonZeroInt(-5,5)
c = NonZeroInt(-5,5)

p=Integer(randint(0,8))
q=Integer(randint(0,8))

v = [
    sqrt(x), 
    log(x), 
    exp(x), 
    x^2, 
    x^3, 
    x^4, 
    (x-b), 
    sin(x), 
    cos(x)
]

F=v[p]
G=v[q]
H=F(G)
f=diff(H, x)
ans=f
\end{sagesilent}

\latexProblemContent{
\ifVerboseLocation This is Derivative Compute Question 0036. \\ \fi
\begin{problem}

Compute the differential of the function $y=\sage{H}$.

\input{Derivative-Compute-0036.HELP.tex}

\[
    dy
    =
    \answer{\left(\sage{ans}\right)\,dx}
\]
\end{problem}}%}
%%%%%%%%%%%%%%%%%%%%%%


%%%%%%%%%%%%%%%%%%%%%%%
%\tagged{Ans@MultiAns, Type@Compute, Topic@Derivative, Sub@Poly, Sub@Multi-Deriv, File@0037}{
\begin{sagesilent}
a = NonZeroInt(-5,5)
b = NonZeroInt(-7,7)
p=Integer(randint(0,2))
v=[a*(x-b), a*expand((x-b)^2), a*expand((x-b)^3)]
F=v[p]
f=diff(F, x)
g=diff(f,x)
\end{sagesilent}

\latexProblemContent{
\ifVerboseLocation This is Derivative Compute Question 0037. \\ \fi
\begin{problem}

Compute the first and second derivatives for the function\\ $f(x)=\sage{F}$.\\

\input{Derivative-Compute-0037.HELP.tex}

\[f'(x)=\answer{\sage{f}}\]

\[f''(x)=\answer{\sage{g}}\] 
\end{problem}}%}
%%%%%%%%%%%%%%%%%%%%%%




%%%%%%%%%%%%%%%%%%%%%%%
%\tagged{Ans@MultiAns, Type@Compute, Topic@Derivative, Func@Trig, Sub@Multi-Deriv, File@0038}{
\begin{sagesilent}
a = NonZeroInt(-10,10)
b = NonZeroInt(-5,5)
p=Integer(randint(0,5))

vtrig=[a*sin(b*x), a*cos(b*x), a*tan(b*x), a*sin(x)+b*cos(x), a*sin(x)^2, a*cos(x)^2]

F=vtrig[p]
f=diff(F, x)
g=diff(f,x)
\end{sagesilent}

\latexProblemContent{
\ifVerboseLocation This is Derivative Compute Question 0038. \\ \fi
\begin{problem}

Compute the first and second derivatives for the function\\ $f(x)=\sage{F}$.\\

\input{Derivative-Compute-0038.HELP.tex}

\[f'(x)=\answer{\sage{f}}\]

\[f''(x)=\answer{\sage{g}}\] 
\end{problem}}%}
%%%%%%%%%%%%%%%%%%%%%%


%%%%%%%%%%%%%%%%%%%%%%%
%\tagged{Ans@MultiAns, Type@Compute, Topic@Derivative, Sub@Product-Rule, Sub@Multi-Deriv, File@0039}{
\begin{sagesilent}
b = NonZeroInt(-5,5)
p=Integer(randint(0,11))
q=Integer(randint(0,11))
v=[sqrt(x), log(x), exp(x), x^2, x^3, x^4, (x-b), sin(x), cos(x), 1/x, 1/x^2, 1/x^3]

Fone=v[p]
Ftwo=v[q]
G=Fone*Ftwo
f=diff(G, x)
g=diff(f,x)
\end{sagesilent}

\latexProblemContent{
\ifVerboseLocation This is Derivative Compute Question 0039. \\ \fi
\begin{problem}

Compute the first and second derivatives for the function\\ $f(x)=\sage{G}$.\\

\input{Derivative-Compute-0039.HELP.tex}

\[f'(x)=\answer{\sage{f}}\]

\[f''(x)=\answer{\sage{g}}\] 
\end{problem}}%}
%%%%%%%%%%%%%%%%%%%%%%


%%%%%%%%%%%%%%%%%%%%%%%
%\tagged{Ans@MultiAns, Type@Compute, Topic@Derivative, Sub@Chain-Rule, Sub@Multi-Deriv, File@0040}{
\begin{sagesilent}
b = NonZeroInt(-5,5)
p=Integer(randint(0,11))
q=Integer(randint(0,11))
v=[sqrt(x), log(x), exp(x), x^2, x^3, x^4, (x-b), sin(x), cos(x), 1/x, 1/x^2, 1/x^3]

Fone=v[p]
Ftwo=v[q]
G=Fone(Ftwo)
f=diff(G, x)
g=diff(f,x)
\end{sagesilent}

\latexProblemContent{
\ifVerboseLocation This is Derivative Compute Question 0040. \\ \fi
\begin{problem}

Compute the first and second derivatives for the function\\ $f(x)=\sage{G}$.\\

\input{Derivative-Compute-0040.HELP.tex}

\[f'(x)=\answer{\sage{f}}\]

\[f''(x)=\answer{\sage{g}}\] 
\end{problem}}%}
%%%%%%%%%%%%%%%%%%%%%%


%%%%%%%%%%%%%%%%%%%%%%%
%\tagged{Ans@ShortAns, Type@Compute, Topic@Derivative, Sub@Poly, Sub@Critical-Number, File@0041}{
\begin{sagesilent}
a = NonZeroInt(-4,4)
b = RandInt(-20,20)
p=Integer(randint(0,2))
vpoly=[(x-a), expand((x-a)^2), expand((x-a)^3)]
F=vpoly[p]
f=integral(F,x) + b
\end{sagesilent}

\latexProblemContent{
\ifVerboseLocation This is Derivative Compute Question 0041. \\ \fi
\begin{problem}

Find the critical numbers for the function $f(x)=\sage{f}$.\\

\input{Derivative-Compute-0041.HELP.tex}

\[x=\answer{\sage{a}}\]
\end{problem}}%}
%%%%%%%%%%%%%%%%%%%%%%


%%%%%%%%%%%%%%%%%%%%%%%
%\tagged{Ans@ShortAns, Type@Compute, Topic@Derivative, Sub@Poly, Sub@Critical-Number, File@0042}{
\begin{sagesilent}
a = NonZeroInt(-4,4)
b = RandInt(-20,20)
p=Integer(randint(0,2))
vpoly=[expand(x*(x-a)), expand(x*(x-a)^2), expand(x*(x-a)^3)]
F=vpoly[p]
f=integral(F,x) + b
\end{sagesilent}

\latexProblemContent{
\ifVerboseLocation This is Derivative Compute Question 0042. \\ \fi
\begin{problem}

Find the critical numbers for the function $f(x)=\sage{f}$.\\

\input{Derivative-Compute-0042.HELP.tex}

\[x=\answer{0,\sage{a}}\]
\end{problem}}%}
%%%%%%%%%%%%%%%%%%%%%%


%%%%%%%%%%%%%%%%%%%%%%%
%\tagged{Ans@ShortAns, Type@Compute, Topic@Derivative, Sub@Poly, Sub@Critical-Number, File@0043}{
\begin{sagesilent}
a = NonZeroInt(-4,4)
b = NonZeroInt(-4,4)
c = RandInt(-20,20)
p=Integer(randint(0,1))
vpoly=[expand((x-a)*(x-b)), (x-a)^2*(x-b)]
F=vpoly[p]
f=integral(F,x) + c
\end{sagesilent}

\latexProblemContent{
\ifVerboseLocation This is Derivative Compute Question 0043. \\ \fi
\begin{problem}

Find the critical numbers for the function $f(x)=\sage{f}$.\\

\input{Derivative-Compute-0043.HELP.tex}

\[x=\answer{\sage{b},\sage{a}}\]
\end{problem}}%}
%%%%%%%%%%%%%%%%%%%%%%

%%%%%%%%%%%%%%%%%%%%%%%
%\tagged{Ans@ShortAns, Type@Compute, Topic@Derivative, Sub@Optimization, File@0044}{
\begin{sagesilent}
a = NonZeroInt(1,10)
A=a^2
\end{sagesilent}

\latexProblemContent{
\ifVerboseLocation This is Derivative Compute Question 0044. \\ \fi
\begin{problem}

Find two positive numbers whose product is $\sage{A}$ and whose sum is a minimum.

\input{Derivative-Compute-0044.HELP.tex}

\[\mbox{The two numbers are}\; \answer{\sage{a}, \sage{a}}\]
\end{problem}}%}
%%%%%%%%%%%%%%%%%%%%%%

%%%%%%%%%%%%%%%%%%%%%%%
%\tagged{Ans@ShortAns, Type@Compute, Topic@Derivative, Sub@Optimization, File@0045}{
\begin{sagesilent}
a = NonZeroInt(1,15)
b = NonZeroInt(1,15,[a])   
A=a^2
B=b^2
Area=2*a*b
\end{sagesilent}

\latexProblemContent{
\ifVerboseLocation This is Derivative Compute Question 0045. \\ \fi
\begin{problem}

Find the area of the largest rectangle that can be inscribed in the ellipse $\dfrac{x^2}{\sage{A}}+\dfrac{y^2}{\sage{B}}=1$

\input{Derivative-Compute-0045.HELP.tex}

\[\mbox{The largest area is }\; \answer{\sage{Area}}\]
\end{problem}}%}
%%%%%%%%%%%%%%%%%%%%%%


%%%%%%%%%%%%%%%%%%%%%%%
%\tagged{Ans@ShortAns, Type@Compute, Topic@Derivative, Sub@Optimization, File@0046}{
\begin{sagesilent}
a = NonZeroInt(1,8)
b = NonZeroInt(1,8)
A=a*b
Area=A^2/(2*(4 + 3*pi))
\end{sagesilent}

\latexProblemContent{
\ifVerboseLocation This is Derivative Compute Question 0046. \\ \fi
\begin{problem}

A Norman window has the shape of a rectangle surmounted by a semicircle (i.e. the diameter of the semicircle is equal to the width of the rectangle).  If the perimeter of the window is $\sage{A}$ ft, what area will allow in the greatest amount of light?

\input{Derivative-Compute-0046.HELP.tex}

\[\mbox{The largest possible area is}\; \answer{\sage{Area}}\]
\end{problem}}%}
%%%%%%%%%%%%%%%%%%%%%%



%%%%%%%%%%%%%%%%%%%%%%%
%\tagged{Ans@ShortAns, Type@Compute, Topic@Derivative, Sub@Optimization, File@0047}{
\begin{sagesilent}
a = NonZeroInt(-15,15)
b = NonZeroInt(-15,15)   
X=(-a*b)/(a^2+1)
Y=b/(a^2+1)
f=a*x+b
\end{sagesilent}

\latexProblemContent{
\ifVerboseLocation This is Derivative Compute Question 0047. \\ \fi
\begin{problem}

Find the point on the line $y=\sage{f}$ that is closest to the origin.

\input{Derivative-Compute-0047.HELP.tex}

\[\mbox{The closest point to the origin is}\; \answer{\left(\sage{X},\sage{Y}\right)}\]
\end{problem}}%}
%%%%%%%%%%%%%%%%%%%%%%

%%%%%%%%%%%%%%%%%%%%%%%
%\tagged{Ans@ShortAns, Type@Compute, Topic@Derivative, Func@Trig, File@0048}{
\begin{sagesilent}
b = NonZeroInt(-20,20)
c = RandInt(-20,20)   
p=Integer(randint(0,5))
vtrig=[b*sin(x), b*cos(x), b*tan(x), b*cot(x), b*sec(x), b*csc(x)]
F=vtrig[p] + c
f=diff(F,x)
\end{sagesilent}

\latexProblemContent{
\ifVerboseLocation This is Derivative Compute Question 0048. \\ \fi
\begin{problem}

Compute the following derivative:

\input{Derivative-Compute-0048.HELP.tex}

\[\dfrac{d}{dx}\left(\sage{F}\right) = \answer{\sage{f}}\]
\end{problem}}%}
%%%%%%%%%%%%%%%%%%%%%%

%%%%%%%%%%%%%%%%%%%%%%%
%\tagged{Ans@ShortAns, Type@Compute, Topic@Derivative, Func@Trig, Sub@LHopital, File@0049}{
\begin{sagesilent}
a = NonZeroInt(-8,8)
b = NonZeroInt(-8,8)
p=Integer(randint(0,3))
v=[sin(a*x)/tan(b*x), sin(a*x)/cos(b*x), tan(a*x)/sin(b*x), tan(a*x)/cos(b*x)]
F=v[p]
Ans=limit(F,x=0)
\end{sagesilent}

\latexProblemContent{
\ifVerboseLocation This is Derivative Compute Question 0049. \\ \fi
\begin{problem}

Find the limit.  Use L'H$\hat{o}$pital's rule where appropriate.

\input{Derivative-Compute-0049.HELP.tex}

\[\lim\limits_{x\to0} \sage{F}=\answer{\sage{Ans}}\]
\end{problem}}%}
%%%%%%%%%%%%%%%%%%%%%%

%%%%%%%%%%%%%%%%%%%%%%%
%\tagged{Ans@ShortAns, Type@Compute, Topic@Derivative, Func@Trig, Sub@LHopital, File@0050}{
\begin{sagesilent}
a = NonZeroInt(-10,10)
b = NonZeroInt(-10,10)
c = RandInt(-20,20)
F=(1+a/x)^(b*x) + c
Ans=limit(F,x=infinity)
\end{sagesilent}

\latexProblemContent{
\ifVerboseLocation This is Derivative Compute Question 0050. \\ \fi
\begin{problem}

Find the limit.  Use L'H$\hat{o}$pital's rule where appropriate.

\input{Derivative-Compute-0050.HELP.tex}

\[\lim\limits_{x\to\infty} \sage{F}=\answer{\sage{Ans}}\]
\end{problem}}%}
%%%%%%%%%%%%%%%%%%%%%%


%%%%%%%%%%%%%%%%%%%%%%%
%\tagged{Ans@ShortAns, Type@Compute, Topic@Derivative, Sub@Exp, Sub@LHopital, File@0051}{
\begin{sagesilent}
a = NonZeroInt(-8,8)
b = NonZeroInt(-8,8)
c = NonZeroInt(-8,8)
p=Integer(randint(0,2))
v=[(x-b),(x-b)^2,(x-b)^3]
F=a*v[p]*exp(-x+c)
Ans=limit(F,x=infinity)
\end{sagesilent}

\latexProblemContent{
\ifVerboseLocation This is Derivative Compute Question 0051. \\ \fi
\begin{problem}

Find the limit.  Use L'H$\hat{o}$pital's rule where appropriate.

\input{Derivative-Compute-0051.HELP.tex}

\[\lim\limits_{x\to\infty} \sage{F}=\answer{\sage{Ans}}\]
\end{problem}}%}
%%%%%%%%%%%%%%%%%%%%%%

%%%%%%%%%%%%%%%%%%%%%%%
%\tagged{Ans@MultiAns, Type@Compute, Topic@Derivative, Sub@Poly, Sub@Theorems-MVT, File@0052}{
\begin{sagesilent}
a = NonZeroInt(-8,8)
b = NonZeroInt(-5,5)
c = NonZeroInt(-5,5)

F=expand(a*(x-b)*(x-c))

vert = (c+b)/(2)
shift = RandInt(1,4)
Lpt = vert - shift
Rpt = vert + shift
\end{sagesilent}

\latexProblemContent{
\ifVerboseLocation This is Derivative Compute Question 0052. \\ \fi
\begin{problem}

Does the function $\sage{F}$ satisfy the conditions of the Mean Value Theorem (MVT) over the interval $[\sage{Lpt},\sage{Rpt}]$?

\input{Derivative-Compute-0052.HELP.tex}

\begin{multipleChoice}
\choice{No, the function is not continuous over the closed interval $[\sage{Lpt},\sage{Rpt}]$}
\choice{No, the function is not differentiable over the open interval $[\sage{Lpt},\sage{Rpt}]$}
\choice{No, the function is not defined over the interval $[\sage{Lpt},\sage{Rpt}]$}
\choice[correct]{Yes, all the conditions are met}
\end{multipleChoice}

If the conditions are met, then find all numbers $c$ that satisfy the conclusion of the MVT over the interval $[\sage{Lpt},\sage{Rpt}]$.  If the conditions are not met, then write ``NONE.''

\[c = \answer{\sage{vert}}\]
\end{problem}}%}
%%%%%%%%%%%%%%%%%%%%%%



%%%%%%%%%%%%%%%%%%%%%%%
%\tagged{Ans@MultiAns, Type@Compute, Topic@Derivative, Sub@Exp, Sub@Theorems-MVT, File@0053}{
\begin{sagesilent}
a = NonZeroInt(-10,10)
b = NonZeroInt(-10,10)
c = NonZeroInt(1,5)
f=a*exp(b*x)
Ans=1/b*log((exp(b*c)-1)/(b*c))
\end{sagesilent}

\latexProblemContent{
\ifVerboseLocation This is Derivative Compute Question 0053. \\ \fi
\begin{problem}

Does the function $\sage{f}$ satisfy the conditions of the Mean Value Theorem (MVT) over the interval $[0,\sage{c}]$?

\input{Derivative-Compute-0053.HELP.tex}

\begin{multipleChoice}
\choice{No, the function is not continuous over the closed interval $[0,\sage{c}]$}
\choice{No, the function is not differentiable over the open interval $(0,\sage{c})$}
\choice{No, the function is not defined over the interval $[0,\sage{c}]$}
\choice[correct]{Yes, all the conditions are met}
\end{multipleChoice}

If the conditions are met, then find all numbers $c$ that satisfy the conclusion of the MVT.  If the conditions are not met, then write ``NONE.''

\[c = \answer{\sage{Ans}}\]
\end{problem}}%}
%%%%%%%%%%%%%%%%%%%%%%


%%%%%%%%%%%%%%%%%%%%%%%
%\tagged{Ans@MultiAns, Type@Compute, Topic@Derivative, Sub@Arctrig, Sub@Theorems-MVT, File@0054}{
\begin{sagesilent}
a = NonZeroInt(-10,10)
b = NonZeroInt(-10,10)
c = NonZeroInt(1,5)
f=a*arctan(b*x)

Ans= 1/b * sqrt( (c*b) / (arctan(b*c)) - 1 )


\end{sagesilent}

\latexProblemContent{
\ifVerboseLocation This is Derivative Compute Question 0054. \\ \fi
\begin{problem}

Does the function $\sage{f}$ satisfy the conditions of the Mean Value Theorem (MVT) over the interval $[0,\sage{c}]$?

\input{Derivative-Compute-0054.HELP.tex}

\begin{multipleChoice}
\choice{No, the function is not continuous over the closed interval $[0,\sage{c}]$}
\choice{No, the function is not differentiable over the open interval $(0,\sage{c})$}
\choice{No, the function is not defined over the interval $[0,\sage{c}]$}
\choice[correct]{Yes, all the conditions are met}
\end{multipleChoice}

If the conditions are met, then find all numbers $c$ that satisfy the conclusion of the MVT.  If the conditions are not met, then write ``NONE.''

\[c = \answer{\sage{Ans}}\]
\end{problem}}%}
%%%%%%%%%%%%%%%%%%%%%%




%\tagged{Ans@Short, Type@Compute, Topic@Derivative, Sub@Poly, Sub@Inflection, File@0055}{

\begin{sagesilent}
a = NonZeroInt(-10,10)
b = NonZeroInt(-10,10)
c = NonZeroInt(-10,10)
f=x^3+a*x^2+b*x+c
Ans=-a/3
\end{sagesilent}

\latexProblemContent{
\ifVerboseLocation This is Derivative Compute Question 0055. \\ \fi
\begin{problem}

Find all inflection points for the function $f(x)=\sage{f}$.  If there are no inflection points enter ``NONE.''

\input{Derivative-Compute-0055.HELP.tex}

\[\mbox{Inflection points at } x=\answer{\sage{Ans}}\]
\end{problem}}%}
%%%%%%%%%%%%%%%%%%%%%%



%%%%%%%%%%%%%%%%%%%%%%%
%\tagged{Ans@ShortAns, Type@Compute, Topic@Derivative, Sub@Domain, Sub@Inflection, File@0056}{
\begin{sagesilent}
a = NonZeroInt(-10,10)
b = RandInt(-20,20)
f = (sqrt(x+a)) + b
\end{sagesilent}

\latexProblemContent{
\ifVerboseLocation This is Derivative Compute Question 0056. \\ \fi
\begin{problem}

Find all inflection points for the function $f(x)=x\sage{f}$.  If there are no inflection points enter ``NONE.''

\input{Derivative-Compute-0056.HELP.tex}

\[
    \mbox{Inflection points at } x
    =
    \answer{NONE}
\]
\end{problem}}%}
%%%%%%%%%%%%%%%%%%%%%%

%%%%%%%%%%%%%%%%%%%%%%%
%\tagged{Ans@ShortAns, Type@Compute, Topic@Derivative, Sub@Inflection, File@0057}{
\begin{sagesilent}
a = NonZeroInt(-10,10)
b = RandInt(-20, 20)
c = ISP(b)
f = (x+a)
Ans = a/2
\end{sagesilent}

\latexProblemContent{
\ifVerboseLocation This is Derivative Compute Question 0057. \\ \fi
\begin{problem}

Find all inflection points for the function $f(x)=x^{\frac{1}{3}}(\sage{f}) \sage{c}$.

\input{Derivative-Compute-0057.HELP.tex}

\[
    \mbox{Inflection points at } x
    =
    \answer{0,\sage{Ans}}
\]
\end{problem}}%}
%%%%%%%%%%%%%%%%%%%%%%

%%%%%%%%%%%%%%%%%%%%%%%
%\tagged{Ans@ShortAns, Type@Compute, Topic@Derivative, Sub@Inflection, File@0058}{
\begin{sagesilent}
a = NonZeroInt(-10,10)
b = NonZeroInt(1,10)
c = RandInt(-20,20)
f = a*exp(b/x) + c
ans = -b/2
\end{sagesilent}

\latexProblemContent{
\ifVerboseLocation This is Derivative Compute Question 0058. \\ \fi
\begin{problem}

Find all inflection points for the function $f(x)=\sage{f}$.

\input{Derivative-Compute-0058.HELP.tex}

\[
    \text{Inflection points at } x
    =
    \answer{\sage{ans}}
\]
\end{problem}}%}
%%%%%%%%%%%%%%%%%%%%%%

%\tagged{Ans@ShortAns, Type@Compute, Topic@Derivative, Sub@Rational, Sub@Quotient-Rule, File@0059}{

\begin{sagesilent}

c1 = RandInt(-5,5)
c2 = RandInt(-5,5)
c3 = RandInt(-5,5)

f1(x) = expand((x-c1)*(x-c2))
f2(x) = (x-c3)*(x+c3)

fFinal(x) = f1(x)/f2(x)

ans = derivative(fFinal(x))

\end{sagesilent}

\latexProblemContent{
\ifVerboseLocation This is Derivative Compute Question 0059. \\ \fi
\begin{problem}

Calculate the derivative of the following function:

\input{Derivative-Compute-0059.HELP.tex}

$f(x) = \sage{fFinal(x)}$ 

$f'(x) = \answer{\sage{ans}}$
\end{problem}}%}

%\tagged{Ans@ShortAns, Type@Compute, Topic@Derivative, Sub@Rational, Sub@Quotient-Rule, File@0060}{

\begin{sagesilent}

c1 = RandInt(-10,10)
c2 = RandInt(-10,10)
c3 = RandInt(-10,10)

f1(x) = factor((x-c1)*(x-c2))
f2(x) = expand((x-c3)*(x+c3))

fFinal(x) = f1(x)/f2(x)

ans = derivative(fFinal(x))

\end{sagesilent}

\latexProblemContent{
\ifVerboseLocation This is Derivative Compute Question 0060. \\ \fi
\begin{problem}

Calculate the derivative of the following function:

\input{Derivative-Compute-0060.HELP.tex}

$f(x) = \sage{fFinal(x)}$

$f'(x) = \answer{\sage{ans}}$
\end{problem}}%}

%\tagged{Ans@ShortAns, Type@Compute, Topic@Derivative, Sub@Rational, Sub@Quotient-Rule, File@0061}{

\begin{sagesilent}

funcvec = [x, x^2, x^3, sqrt(x), x^(1/3)]

pick1 = RandInt(0, 4)
pick2 = RandInt(0, 4)
pick3 = RandInt(0, 4)
pick4 = RandInt(0, 4)

f1 = funcvec[pick1]
f2 = funcvec[pick2]
f3 = funcvec[pick3]
f4 = funcvec[pick4]

c1 = RandInt(-5,5)
c2 = RandInt(-5,5)
c3 = RandInt(-5,5)

ftop(x) = (f1(x)-c1)*(f2(x)-c2)
fbot(x) = (f3(x)-c3)*(f4(x)+c3)

fFinal(x) = ftop(x)/fbot(x)

fcalc(x) = expand(ftop(x))/expand(fbot(x))

ans = derivative(fcalc(x))

\end{sagesilent}

\latexProblemContent{
\ifVerboseLocation This is Derivative Compute Question 0061. \\ \fi
\begin{problem}

Calculate the derivative of the following function:

\input{Derivative-Compute-0061.HELP.tex}

$f(x) = \sage{fFinal(x)}$

$f'(x) = \answer{\sage{ans}}$
\end{problem}}%}

%\tagged{Ans@ShortAns, Type@Compute, Topic@Derivative, Sub@Rational, Sub@ProductRule, File@0062}{

\begin{sagesilent}

funcvec = [x, x^2, x^3, sqrt(x), x^(1/3)]

pick1 = RandInt(0, 4)
pick2 = RandInt(0, 4)
pick3 = RandInt(0, 4)
pick4 = RandInt(0, 4)

f1 = funcvec[pick1]
f2 = funcvec[pick2]
f3 = funcvec[pick3]
f4 = funcvec[pick4]

c1 = RandInt(-5,5)
c2 = RandInt(-5,5)
c3 = RandInt(-5,5)

ftop(x) = expand((f1(x)-c1)*(f2(x)-c2))
fbot(x) = expand((f3(x)-c3)*(f4(x)+c3))

fFinal(x) = ftop(x)/fbot(x)

ans = derivative(fFinal(x))

\end{sagesilent}

\latexProblemContent{
\ifVerboseLocation This is Derivative Compute Question 0062. \\ \fi
\begin{problem}

Calculate the derivative of the following function:

\input{Derivative-Compute-0062.HELP.tex}

$f(x) = \sage{fFinal(x)}$

$f'(x) = \answer{\sage{ans}}$
\end{problem}}%}



%\tagged{Ans@ShortAns, Type@Compute, Topic@Derivative, Sub@Rational, Sub@Quotient-Rule, File@0063}{

\begin{sagesilent}

funcvec = [e^x, x, x^2, x^3, sqrt(x), x^(1/3)]

pick1 = RandInt(0, 4)
pick2 = RandInt(0, 4)
pick3 = RandInt(0, 4)
pick4 = RandInt(0, 4)

f1 = funcvec[pick1]
f2 = funcvec[pick2]
f3 = funcvec[pick3]
f4 = funcvec[pick4]

c1 = RandInt(-5,5)
c2 = RandInt(-5,5)
c3 = RandInt(-5,5)
c4 = RandInt(-20,20)
c5 = RandInt(-20,20)

fleft(x) = (f1(x)-c1)*(f1(x)+c1)
fright(x) = (f3(x)-c3)*(f3(x)+c3)

fFinal(x) = fleft(x)*fright(x)

ans = derivative(expand(fFinal(x)))

\end{sagesilent}

\latexProblemContent{
\ifVerboseLocation This is Derivative Compute Question 0063. \\ \fi
\begin{problem}

Calculate the derivative of the following function:

\input{Derivative-Compute-0063.HELP.tex}

$f(x) = \sage{fFinal(x)}$

$f'(x) = \answer{\sage{ans}}$
\end{problem}}%}


%\tagged{Ans@ShortAns, Type@Compute, Topic@Derivative, Sub@Poly, Func@Trig,  Sub@Log, Sub@Exp, Sub@Radical, File@0064}{


\begin{sagesilent}
a1 = NonZeroInt(-5,5)
a2 = RandInt(-5,5)
a3 = RandInt(-5,5)
a4 = RandInt(-5,5)
b1 = RandInt(-5,5)
b2 = RandInt(-5,5)
b3 = RandInt(-5,5)
b4 = RandInt(-5,5)
pwr = RandInt(2,10)

funcv = [expand((a1*x+b1)*(a2*x+b2)*(a3*x+b3)*(a4*x+b4)), sin(x), tan(x), sec(x), csc(x), cot(x), cos(x), e^x, ln(x), x^(1/pwr)]

p = RandInt(0,9)

f(x) = funcv[p]

ans = simplify(derivative(f(x),x))


\end{sagesilent}

\latexProblemContent{
\ifVerboseLocation This is Derivative Compute Question 0064 \fi
\begin{problem}
Compute the following derivative,
\input{Derivative-Compute-0064.HELP.tex}

$\frac{d}{dx} \sage{f(x)} = \answer{\sage{ans}}$


\end{problem}}%}




%\tagged{Ans@ShortAns, Type@Compute, Topic@Derivative, Sub@Poly, Sub@Motion, File@0065}{


\begin{sagesilent}
hpitch = RandInt(48,66)
vinit = RandInt(40,70)


f(t) = -16*t^2+vinit*t+hpitch

determ = vinit^2 - 4*(-16)*(hpitch)

ans = (vinit + determ^(1/2))/32

\end{sagesilent}

\latexProblemContent{
\ifVerboseLocation This is Derivative Compute Question 0065 \fi
\begin{problem}
\input{Derivative-Compute-0065.HELP.tex}
A baseball pitcher is practicing out on the field. He releases the ball at shoulder level which is $\sage{hpitch}$ inches. The height of the ball as it passes through the air can be modeled by;
\[\sage{f(t)}\]

At what time does the ball hit the ground? $t = \answer[tolerance=.2]{\sage{ans}}$ seconds.



\end{problem}}%}




%\tagged{Ans@ShortAns, Type@Compute, Topic@Derivative, Sub@Poly, Sub@Motion, File@0066}{


\begin{sagesilent}
hpitch = RandInt(48,66)
vinit = RandInt(40,70)
hcatch = RandInt(48,66)

f(t) = -16*t^2+vinit*t+hpitch

ans = (-vinit - (vinit^2+4*16*((hpitch - hcatch)/12))^(1/2) ) / (-2*16)

\end{sagesilent}

\latexProblemContent{
\ifVerboseLocation This is Derivative Compute Question 0066 \fi
\begin{problem}
\input{Derivative-Compute-0066.HELP.tex}
A baseball pitcher is practicing out on the field with a partner. He releases the ball at shoulder level which is $\sage{hpitch}$ inches and his partner catches the ball at the height of $\sage{hcatch}$ inches. The height of the ball as it passes through the air can be modeled by;
\[
    \sage{f(t)}
\]

At what time does the partner catch the ball? $t = \answer[tolerance=.2]{\sage{ans}}$ seconds.



\end{problem}}%}


%\tagged{Ans@ShortAns, Type@Compute, Topic@Derivative, Sub@Poly, Sub@Motion, File@0067}{


\begin{sagesilent}
hpitch = RandInt(48,66)
vinit = RandInt(40,70)
hhit = RandInt(25,50)
vhit = RandInt(40,70)

f(t) = -16*t^2+vinit*t+hpitch
g(t) = -16*t^2+vhit*t+hhit


ans = (-vinit - (vinit^2+4*16*((hpitch - hhit)/12))^(1/2) ) / (-2*16)

ans2 = (-vinit - (vinit^2+4*16*((hhit)/12))^(1/2) ) / (-2*16) + ans

\end{sagesilent}

\latexProblemContent{
\ifVerboseLocation This is Derivative Compute Question 0067 \fi
\begin{problem}
\input{Derivative-Compute-0067.HELP.tex}
A baseball pitcher and batter are practicing out on the field. The pitcher releases the ball at shoulder level which is $\sage{hpitch}$ inches. The batter hits the ball when it is at a height of $\sage{hhit}$. The height of the ball as it passes through the air can be modeled by;
\[
    \sage{f(t)}
\]

At what time does the batter hit the ball? $t = \answer[tolerance=.2]{\sage{ans}}$ seconds.

After the ball is struck the height of the ball is modeled by the function

\[
    \sage{g(t)}
\]

When does the ball hit the ground (after the batter strikes it)? $t = \answer[tolerance=0.2]{\sage{ans2}}$ seconds.


\end{problem}}%}






%%%%%%%%%%%%%%%%%%%%%%%%%%%%%%%%%%%%%%%%%%%%%%%%%%%%%%%%%%%%%%%%%%%%%%%%%%%%%%%
%
%
%
%
%
%
%
%
%%%%%%%%%%%%%%%%%%%%%%%%%%%%%%%%%%%%%%%%%%%%%%%%%%%%%%%%%%%%%%%%%%%%%%%%%%%%%%%
%%%%%%%%%%%%%%%%%%%%%%%%%%%%%%%%%%%%%%%%%%%%%%%%%%%%%%%%%%%%%%%%%%%%%%%%%%%%%%%
%%%%%%%%%%%%%%%%%%%                                     %%%%%%%%%%%%%%%%%%%%%%%
%%%%%%%%%%%%%%%%%%%             Concept                 %%%%%%%%%%%%%%%%%%%%%%%
%%%%%%%%%%%%%%%%%%%                                     %%%%%%%%%%%%%%%%%%%%%%%
%%%%%%%%%%%%%%%%%%%%%%%%%%%%%%%%%%%%%%%%%%%%%%%%%%%%%%%%%%%%%%%%%%%%%%%%%%%%%%%
%%%%%%%%%%%%%%%%%%%%%%%%%%%%%%%%%%%%%%%%%%%%%%%%%%%%%%%%%%%%%%%%%%%%%%%%%%%%%%%
%
%
%
%
%
%
%
%
%
%%%%%%%%%%%%%%%%%%%%%%%%%%%%%%%%%%%%%%%%%%%%%%%%%%%%%%%%%%%%%%%%%%%%%%%%%%%%%%%
%
%
%
%
%%%%%%%%%%%%%%%%%%%%%%%%%%%%%%%%%%%%%%%%%%%%%%%%%%%%%%%%%%%%%%%%%%%%%%%%%%%%%%%
%%%%%%%%%%%%%%%%%%%             Calc 1 Concept          %%%%%%%%%%%%%%%%%%%%%%%
%%%%%%%%%%%%%%%%%%%%%%%%%%%%%%%%%%%%%%%%%%%%%%%%%%%%%%%%%%%%%%%%%%%%%%%%%%%%%%%

%%%%%%%%%%%%%%%%%%%%%%%
%\tagged{Ans@MC, Type@Concept, Topic@Derivative, Sub@Theorems-EVT, File@0001}{
\begin{sagesilent}
a = NonZeroInt(1,5)
b=a-1
c=a+1
p=Integer(randint(1,4))
f=log(c-x)
g=(x-b)^p-1
\end{sagesilent}

\latexProblemContent{
\ifVerboseLocation This is Derivative Concept Question 0001. \\ \fi
\begin{problem}

Does the Extreme Value Theorem hold for the function 
\[
    f(x)
    =
    \left\lbrace\begin{array}{ll}
    \sage{f}\; , & x<\sage{a}\\[3pt]
    \sage{g}\; , & x\geq\sage{a}
    \end{array}\right.
\]
 over the interval $[\sage{a-2},\sage{a+3}]$?

\input{Derivative-Concept-0001.HELP.tex}

\begin{multipleChoice}
\choice[correct]{Yes}
\choice{No}
\end{multipleChoice}

\end{problem}}%}
%%%%%%%%%%%%%%%%%%%%%%


%%%%%%%%%%%%%%%%%%%%%%%
%\tagged{Ans@MultiAns, Type@Concept, Topic@Derivative, Sub@Theorems-EVT, File@0002}{
\begin{sagesilent}
a = NonZeroInt(1,5)
b=a-1
c=a+1
p=Integer(randint(1,4))
f=log(c-x)
g=(x-b)^p-1
\end{sagesilent}

\latexProblemContent{
\ifVerboseLocation This is Derivative Concept Question 0002. \\ \fi
\begin{problem}

Does the Extreme Value Theorem hold for the function \[f(x)=\left\lbrace\begin{array}{ll}
\sage{f}\; , & x<\sage{a}\\[3pt]
\sage{g}\; , & x\geq\sage{a}
\end{array}\right.\]
 over the interval $[\sage{a-3},\sage{a+5})$?

\input{Derivative-Concept-0002.HELP.tex}

\begin{multipleChoice}
\choice{Yes}
\choice[correct]{No}
\end{multipleChoice}

\begin{problem}
If not, which condition fails?

\begin{multipleChoice}
\choice{$f$ is not continuous on the interval}
\choice[correct]{The interval is not closed}
\choice{$f$ is not differentiable on the interval}
\choice{Nothing fails; the theorem holds}
\end{multipleChoice}
\end{problem}
\end{problem}}%}
%%%%%%%%%%%%%%%%%%%%%%

%%%%%%%%%%%%%%%%%%%%%%%
%\tagged{Ans@MultiAns, Type@Concept, Topic@Derivative, Sub@Theorems-EVT, File@0003}{
\begin{sagesilent}
a = NonZeroInt(-10,10)
b=a-1
c=a+1
p=NonZeroInt(-10,10)
f=log(c-x)
g=(x-b)^p

r = randint(1,20)
q = randint(1,20)
intstart = a-r
intend = a + q
\end{sagesilent}

\latexProblemContent{
\ifVerboseLocation This is Derivative Concept Question 0003. \\ \fi
\begin{problem}

Does the Extreme Value Theorem hold for the function 
\[
    f(x)
    =
    \left\lbrace\begin{array}{ll}
    \sage{f}\; , & x<\sage{a}\\[3pt]
    \sage{g}\; , & x\geq\sage{a}
    \end{array}\right.
\]
 over the interval $[\sage{intstart},\sage{intend}]$?

\input{Derivative-Concept-0003.HELP.tex}
 
\begin{multipleChoice}
\choice{Yes}
\choice[correct]{No}
\end{multipleChoice}

If not, which condition fails?

\begin{multipleChoice}
\choice[correct]{$f$ is not continuous on the interval}
\choice{The interval is not closed}
\choice{$f$ is not differentiable on the interval}
\choice{Nothing fails; the theorem holds}
\end{multipleChoice}
\end{problem}
}%}
%%%%%%%%%%%%%%%%%%%%%%



%%%%%%%%%%%%%%%%%%%%%%%
%\tagged{Ans@MC, Type@Concept, Topic@Derivative, Sub@Theorems-EVT, File@0004}{
\begin{sagesilent}
r = RandInt(1,3)
pwr = 2*r+1
base = RandInt(2,5)
funcvec = [x, x^3, log(x), x^(1/pwr), base^x, e^x]

coef1 = RandInt(1,10)
coef2 = RandInt(1,10)
coef3 = RandInt(-5,-1)

choice1 = RandInt(0,5)
if choice1 > 3:
    choice2 = RandInt(4,5)
else:
    choice2 = RandInt(1,3)

pivot = RandInt(1,5)

if choice1 > 3:
    shift = pivot
else:
    shift = pivot - 1


fT = coef1 * funcvec[choice1] + coef3
gT = coef2 * funcvec[choice2]

f(x) = fT(x-shift)
gTT(x) = gT(x-shift)

coef4 = f(pivot) - gTT(pivot)

g(x) = gTT(x) + coef4

startT = RandInt (-10,-1)
endT = RandInt(1, 10)

start = pivot + startT
end = pivot + endT

\end{sagesilent}

\latexProblemContent{
\ifVerboseLocation This is Derivative Concept Question 0004. \\ \fi
\begin{problem}

Does the Extreme Value Theorem hold for the function 
\[
    f(x)
    =
    \left\lbrace\begin{array}{ll}
    \sage{f(x)}\; , & x<\sage{pivot}\\[3pt]
    \sage{g(x)}\; , & x\geq\sage{pivot}
    \end{array}\right.
\]
 over the interval $[\sage{start},\sage{end}]$?

\input{Derivative-Concept-0004.HELP.tex}

 \begin{multipleChoice}
 \choice[correct]{Yes}
 \choice{No}
 \end{multipleChoice}

\end{problem}}%}
%%%%%%%%%%%%%%%%%%%%%%



%%%%%%%%%%%%%%%%%%%%%%%
%\tagged{Ans@MC, Type@Concept, Topic@Derivative, Sub@Theorems-IVT, File@0005}{
\begin{sagesilent}
r = RandInt(1,3)
pwr = 2*r+1
base = RandInt(2,5)
funcvec = [x, x^3, log(x), x^(1/pwr), base^x, e^x]

coef1 = RandInt(1,10)
coef2 = RandInt(1,10)
coef3 = RandInt(-5,-1)

choice1 = RandInt(0,5)
if choice1 > 3:
    choice2 = RandInt(4,5)
else:
    choice2 = RandInt(1,3)

pivot = RandInt(1,5)

if choice1 > 3:
    shift = pivot
else:
    shift = pivot - 1


fT = coef1 * funcvec[choice1] + coef3
gT = coef2 * funcvec[choice2]

f(x) = fT(x-shift)
gTT(x) = gT(x-shift)

coef4 = f(pivot) - gTT(pivot)

g(x) = gTT(x) + coef4

startT = RandInt (-10,-1)
endT = RandInt(1, 10)

start = pivot + startT
end = pivot + endT

\end{sagesilent}

\latexProblemContent{
\ifVerboseLocation This is Derivative Concept Question 0005. \\ \fi
\begin{problem}

Does the Intermediate Value Theorem hold for the function 
\[
    f(x)
    =
    \left\lbrace\begin{array}{ll}
    \sage{f(x)}\; , & x<\sage{pivot}\\[3pt]
    \sage{g(x)}\; , & x\geq\sage{pivot}
    \end{array}\right.
\]
 over the interval $[\sage{start},\sage{end}]$?

\input{Derivative-Concept-0005.HELP.tex}

 \begin{multipleChoice}
 \choice[correct]{Yes}
 \choice{No}
 \end{multipleChoice}

\end{problem}}%}
%%%%%%%%%%%%%%%%%%%%%%




%%%%%%%%%%%%%%%%%%%%%%%
%\tagged{Ans@MC, Type@Concept, Topic@Derivative, Sub@Theorems-EVT, File@0006}{
\begin{sagesilent}
r = RandInt(1,3)
pwr = 2*r+1
base = RandInt(2,5)
funcvec = [x, x^3, log(x), x^(1/pwr), base^x, e^x]

coef1 = RandInt(1,10)
coef2 = RandInt(1,10)
coef3 = RandInt(-5,-1)

choice1 = RandInt(0,5)
if choice1 > 3:
    choice2 = RandInt(4,5)
else:
    choice2 = RandInt(1,3)

pivot = RandInt(1,5)

if choice1 > 3:
    shift = pivot
else:
    shift = pivot - 1


fT = coef1 * funcvec[choice1] + coef3
gT = coef2 * funcvec[choice2]

f(x) = fT(x-shift)
gTT(x) = gT(x-shift)

coef4c = f(pivot) - gTT(pivot)

coef4 = NonZeroInt(-10,10,[coef4c])

g(x) = gTT(x) + coef4

startT = RandInt (-10,-1)
endT = RandInt(1, 10)

start = pivot + startT
end = pivot + endT

\end{sagesilent}

\latexProblemContent{
\ifVerboseLocation This is Derivative Concept Question 0006. \\ \fi
\begin{problem}

Does the Extreme Value Theorem hold for the function 
\[
    f(x)
    =
    \left\lbrace\begin{array}{ll}
    \sage{f(x)}\; , & x<\sage{pivot}\\[3pt]
    \sage{g(x)}\; , & x\geq\sage{pivot}
    \end{array}\right.
\]
 over the entire interval $[\sage{start},\sage{end}]$?

\input{Derivative-Concept-0006.HELP.tex}

 \begin{multipleChoice}
 \choice{Yes}
 \choice[correct]{No}
 \end{multipleChoice}

\end{problem}}%}
%%%%%%%%%%%%%%%%%%%%%%



%%%%%%%%%%%%%%%%%%%%%%%
%\tagged{Ans@MC, Type@Concept, Topic@Derivative, Sub@Theorems-IVT, File@0007}{
\begin{sagesilent}
r = RandInt(1,3)
pwr = 2*r+1
base = RandInt(2,5)
funcvec = [x, x^3, log(x), x^(1/pwr), base^x, e^x]

coef1 = RandInt(1,10)
coef2 = RandInt(1,10)
coef3 = RandInt(-5,-1)

choice1 = RandInt(0,5)
if choice1 > 3:
    choice2 = RandInt(4,5)
else:
    choice2 = RandInt(1,3)

pivot = RandInt(1,5)

if choice1 > 3:
    shift = pivot
else:
    shift = pivot - 1


fT = coef1 * funcvec[choice1] + coef3
gT = coef2 * funcvec[choice2]

f(x) = fT(x-shift)
gTT(x) = gT(x-shift)

coef4c = f(pivot) - gTT(pivot)

coef4 = NonZeroInt(-10,10,[coef4c])

g(x) = gTT(x) + coef4

startT = RandInt (-10,-1)
endT = RandInt(1, 10)

start = pivot + startT
end = pivot + endT

\end{sagesilent}

\latexProblemContent{
\ifVerboseLocation This is Derivative Concept Question 0007. \\ \fi
\begin{problem}

Does the Intermediate Value Theorem hold for the function 
\[
    f(x)
    =
    \left\lbrace\begin{array}{ll}
    \sage{f(x)}\; , & x<\sage{pivot}\\[3pt]
    \sage{g(x)}\; , & x\geq\sage{pivot}
    \end{array}\right.
\]
 over the entire interval $[\sage{start},\sage{end}]$?

\input{Derivative-Concept-0007.HELP.tex}

 \begin{multipleChoice}
 \choice{Yes}
 \choice[correct]{No}
 \end{multipleChoice}

\end{problem}}%}
%%%%%%%%%%%%%%%%%%%%%%


%%%%%%%%%%%%%%%%%%%%%%%
%\tagged{Ans@MC, Type@Concept, Topic@Derivative, Sub@Theorems-MVT, File@0008}{
\begin{sagesilent}
r = RandInt(1,3)
pwr = 2*r+1
base = RandInt(2,5)
funcvec = [x, x^3, log(x), x^(1/pwr), base^x, e^x]

coef1 = RandInt(1,10)
coef2 = RandInt(1,10)
coef3 = RandInt(-5,-1)

choice1 = RandInt(0,5)
if choice1 > 3:
    choice2 = RandInt(4,5)
else:
    choice2 = RandInt(1,3)

pivot = RandInt(1,5)

if choice1 > 3:
    shift = pivot
else:
    shift = pivot - 1


fT = coef1 * funcvec[choice1] + coef3
gT = coef2 * funcvec[choice2]

f(x) = fT(x-shift)
gTT(x) = gT(x-shift)

coef4c = f(pivot) - gTT(pivot)

coef4 = NonZeroInt(-10,10,[coef4c])

g(x) = gTT(x) + coef4

startT = RandInt (-10,-1)
endT = RandInt(1, 10)

start = pivot + startT
end = pivot + endT

\end{sagesilent}

\latexProblemContent{
\ifVerboseLocation This is Derivative Concept Question 0008. \\ \fi
\begin{problem}

Does the Mean Value Theorem hold for the function 
\[
    f(x)
    =
    \left\lbrace\begin{array}{ll}
    \sage{f(x)}\; , & x<\sage{pivot}\\[3pt]
    \sage{g(x)}\; , & x\geq\sage{pivot}
    \end{array}\right.
\]
 over the entire interval $[\sage{start},\sage{end}]$?

\input{Derivative-Concept-0008.HELP.tex}

 \begin{multipleChoice}
 \choice{Yes}
 \choice[correct]{No}
 \end{multipleChoice}

\end{problem}}%}
%%%%%%%%%%%%%%%%%%%%%%


%%%%%%%%%%%%%%%%%%%%%%%
%\tagged{Ans@ShortAns, Type@Compute, Topic@Derivative, Func@Radical, Sub@Chain-Rule, File@1001}{
\begin{sagesilent}

cons = RandInt(-10,10)
coef = 2 * NonZeroInt(-5,5)

inner = coef * x + cons

func = sqrt(inner)
ans = coef / (2 * sqrt(inner))

\end{sagesilent}

\latexProblemContent{
\ifVerboseLocation This is Derivative Compute Question 1001. \\ \fi
\begin{problem}

Compute the derivative.

\input{Derivative-Compute-1001.HELP.tex}

\[
    \frac{d}{dx}\left(\sqrt{\sage{inner}}\right)
    =
    \answer{\sage{ans}}
\] 
\end{problem}}%}
%%%%%%%%%%%%%%%%%%%%%%



%%%%%%%%%%%%%%%%%%%%%%%
%\tagged{Ans@ShortAns, Type@Compute, Topic@Derivative, Func@Poly, Sub@Domain, File@1002}{
\begin{sagesilent}

funcCoef = NonZeroInt(-10,10)
funcCons = RandInt(-10,10)

func = funcCoef * x + funcCons

deriv = funcCoef

funcDomainLeft = -infinity
funcDomainRight = infinity

derivDomainLeft = -infinity
derivDomainRight = infinity
\end{sagesilent}

\latexProblemContent{
\ifVerboseLocation This is Derivative Compute Question 1002. \\ \fi
\begin{problem}

Find the derivative of the funciton using the definition of the derivative.

\[
    f(x)
    =
    \sage{func}
\]

\input{Derivative-Compute-1002.HELP.tex}

\[
    f'(x)
    =
    \answer{\sage{deriv}}
\]

\begin{problem}

State the domain of the function. (Enter your answer using interval notation.)
\[
    \left(\answer{\sage{funcDomainLeft}}
    \,,\,
    \answer{\sage{funcDomainRight}}\right)
\] 

\begin{problem}

State the domain of its derivative. (Enter your answer using interval notation.)
\[
    \left(\answer{\sage{derivDomainLeft}}
    \,,\,
    \answer{\sage{derivDomainRight}}\right)
\] 
\end{problem}
\end{problem}
\end{problem}}%}
%%%%%%%%%%%%%%%%%%%%%%

%%%%%%%%%%%%%%%%%%%%%%%
%\tagged{Ans@ShortAns, Type@Compute, Topic@Derivative, Func@Poly, Sub@Domain, File@1003}{
\begin{sagesilent}

coef = NonZeroInt(-10,10)
n = RandInt(0,3)
pwr = 2 * n + 1

func = coef / sqrt(x^pwr)

deriv = coef * (-pwr/2) / sqrt(x^(pwr+2))

funcDomainLeft = 0
funcDomainRight = infinity

derivDomainLeft = 0
derivDomainRight = infinity
\end{sagesilent}

\latexProblemContent{
\ifVerboseLocation This is Derivative Compute Question 1003. \\ \fi
\begin{problem}

Find the derivative of the funciton using the definition of the derivative.

\[
    f(x)
    =
    \frac{\sage{coef}}{\sqrt{x^{\sage{pwr}}}}
\]

\input{Derivative-Compute-1003.HELP.tex}

\[
    f'(x)
    =
    \answer{\sage{deriv}}
\]

\begin{problem}

State the domain of the function. (Enter your answer using interval notation.)
\[
    \left(\answer{\sage{funcDomainLeft}}
    \,,\,
    \answer{\sage{funcDomainRight}}\right)
\] 

\begin{problem}

State the domain of its derivative. (Enter your answer using interval notation.)
\[
    \left(\answer{\sage{derivDomainLeft}}
    \,,\,
    \answer{\sage{derivDomainRight}}\right)
\] 
\end{problem}
\end{problem}
\end{problem}}%}
%%%%%%%%%%%%%%%%%%%%%%