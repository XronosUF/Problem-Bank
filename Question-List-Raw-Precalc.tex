%%%%%%%%%%%%%%%%%%%%%%%%%%
%%%%%%%%%%%%%%%%%%%%%%%%%%



%%%%%%%%%%%%%%%%%%%%%%%
%%\tagged{Ans@ShortAns, Type@Compute, Topic@Precalc, Sub@Notation, File@0001}{
\begin{sagesilent}
# Define variables and constants/exponents
a=RandInt(1,5)
b=RandInt(1,10)
c=NonZeroInt(-5,5,[b])
p=RandInt(0,3)

#Define the function and values
v=[b*x+c, b*x^2+c, sqrt(b*x), (x+c)/(x+b)]
F=v[p]
F1=F(a)
F2=F(a+1)
F3=F(a+2)
\end{sagesilent}
 
\latexProblemContent{
\ifVerboseLocation This is Precalc Compute Question 0001. \\ \fi
\begin{problem}
Evaluate the function $f(x)=\sage{F}$ at the given values.

\input{Precalc-Compute-0001.HELP.tex}

\begin{tabular}{|l|l|}
\hline
$x$ & $f(x)$\\
\hline
$\sage{a}$ & $\sage{F1}$\\
\hline
$\sage{a+1}$ & $\sage{F2}$\\
\hline
$\sage{a+2}$ & $\sage{F3}$\\
\hline
\end{tabular}

\end{problem}}%}
%%%%%%%%%%%%%%%%%%%%%%

%%%%%%%%%%%%%%%%%%%%%%%
%%\tagged{Ans@ShortAns, Type@Compute, Topic@Precalc, Sub@Domain, Sub@Poly, File@0002}{
\begin{sagesilent}
# Define variables and constants/exponents
b=RandInt(1,10)
c=RandInt(-5,5)
p=RandInt(0,2)

#Define the function and values
v=[b*x+c, b*x^2+c, b*x^3+c]
F=v[p]   
\end{sagesilent}
 
\latexProblemContent{
\ifVerboseLocation This is Precalc Compute Question 0002. \\ \fi
\begin{problem}
Compute the domain of the function $\sage{F}$.

\input{Precalc-Compute-0002.HELP.tex}

\[\left(\answer{-\infty},\answer{\infty}\right)\]

\end{problem}}%}
%%%%%%%%%%%%%%%%%%%%%%

%%%%%%%%%%%%%%%%%%%%%%%
%%\tagged{Ans@ShortAns, Type@Compute, Topic@Precalc, Sub@Domain, Sub@Radical, File@0003}{
\begin{sagesilent}
# Define variables and constants/exponents
b=RandInt(1,10)
c=NonZeroInt(-5,5)
p=Integer(randint(0,1))

#Define the function and values
v=[sqrt(b*x), sqrt(b*x+c)]
F=v[p]
if p==0:
   D1=0
   D2=infinity
else:
   D1=-c/b
   D2=infinity   
\end{sagesilent}
 
\latexProblemContent{
\ifVerboseLocation This is Precalc Compute Question 0003. \\ \fi
\begin{problem}
Compute the domain of the function $\sage{F}$.

\input{Precalc-Compute-0003.HELP.tex}

\[\left(\answer{\sage{D1}},\answer{\sage{D2}}\right)\]

\end{problem}}%}
%%%%%%%%%%%%%%%%%%%%%%

%%%%%%%%%%%%%%%%%%%%%%%
%%\tagged{Ans@ShortAns, Type@Compute, Topic@Precalc, Sub@Domain, Sub@Rational, File@0004}{
\begin{sagesilent}
# Define variables and constants/exponents
b=RandInt(1,10)
c=RandInt(-5,5)

#Define the function and values
F=(x+c)/(x+b)
\end{sagesilent}
 
\latexProblemContent{
\ifVerboseLocation This is Precalc Compute Question 0004. \\ \fi
\begin{problem}
Compute the domain of the function $\sage{F}$.

\input{Precalc-Compute-0004.HELP.tex}

\[\left(\answer{-\infty},\answer{\sage{-b}}\right)\bigcup\left(\answer{\sage{-b}},\answer{\infty}\right)\]

\end{problem}}%}
%%%%%%%%%%%%%%%%%%%%%%

%%%%%%%%%%%%%%%%%%%%%%%%
%%%\tagged{Ans@MC, Type@Compute, Topic@Precalc, Sub@Domain, Sub@Relation, File@0005}{
%\begin{sagesilent}
%# Define variables and constants/exponents
%a=RandInt(-5,5)
%b = a-1
%c = a+1
%d = a+2
%e = a+3
%q = a-2
%\end{sagesilent}
% 
%\latexProblemContent{
%\ifVerboseLocation This is Precalc Compute Question 0005. \\ \fi
%\begin{problem}
%Which of the following is \underline{not} in the domain of the function:
%\[\lbrace(\sage{q},\sage{b}), (\sage{b},\sage{a}), (\sage{a},\sage{c}), (\sage{c},\sage{d}), (\sage{d},\sage{e})\rbrace\]
%
%\input{Precalc-Compute-0005.HELP.tex}
%%\answer{$\sage{e}$}
%
%\begin{multipleChoice}
%\choice{$\sage{b}$}
%\choice{$\sage{d}$}
%\choice{$\sage{a}$}
%\choice[correct]{$\sage{e}$}
%\choice{$\sage{c}$}
%\end{multipleChoice}
%
%\end{problem}}%}
%%%%%%%%%%%%%%%%%%%%%%%

%%%%%%%%%%%%%%%%%%%%%%%
%%\tagged{Ans@ShortAns, Type@Compute, Topic@Precalc, Sub@Avg-RoC, File@0006}{
\begin{sagesilent}
# Define variables and constants/exponents
a=RandInt(1,5)
c=RandInt(-5,5)
d=RandInt(a+1,10)
b=NonZeroInt(1,10,[-a,-d,c])
p=RandInt(0,4)

#Choosing the function
v=[b*x+c, b*x^2+c, b*x^3+c, sqrt(b*x), (x+c)/(x+b)]
F=v[p]

#Computing the slope
Ans=(F(d)-F(a))/(d-a)
\end{sagesilent}
 
\latexProblemContent{
\ifVerboseLocation This is Precalc Compute Question 0006. \\ \fi
\begin{problem}
Compute the average rate of change of the function $\sage{F}$ on the interval $[\sage{a},\sage{d}]$.

\input{Precalc-Compute-0006.HELP.tex}

\[\answer{\sage{Ans}}\]

\end{problem}}%}
%%%%%%%%%%%%%%%%%%%%%%

%%%%%%%%%%%%%%%%%%%%%%%
%%\tagged{Ans@ShortAns, Type@Compute, Topic@Precalc, Sub@Composition, File@0007}{
\begin{sagesilent}
# Define variables and constants/exponents
b=RandInt(1,10)
c=RandInt(-5,5)
p=RandInt(0,4)
q=NonZeroInt(0,4,[p])

#Choosing the function
v=[b*x+c, b*x^2+c, b*x^3+c, sqrt(b*x), (x+c)/(x+b)]
F=v[p]
G=v[q]

#Computing the Composites
A1=F-G
A2=G/F
\end{sagesilent}
 
\latexProblemContent{
\ifVerboseLocation This is Precalc Compute Question 0007. \\ \fi
\begin{problem}
Given the functions $f(x)=\sage{F}$ and $g(x)=\sage{G}$, compute the following:

\input{Precalc-Compute-0007.HELP.tex}

\begin{itemize}
\item $f(x)-g(x)=\answer{\sage{A1}}$
\item $\frac{g(x)}{f(x)}=\answer{\sage{A2}}$
\end{itemize}

\end{problem}}%}
%%%%%%%%%%%%%%%%%%%%%%


%%%%%%%%%%%%%%%%%%%%%%%
%%\tagged{Ans@ShortAns, Type@Compute, Topic@Precalc, Sub@Composition, File@0008}{
\begin{sagesilent}
# Define variables and constants/exponents
b=RandInt(1,10)
c=RandInt(-5,5)
p=RandInt(0,4)
q=NonZeroInt(0,4,[p])

#Choosing the function
v=[b*x+c, b*x^2+c, b*x^3+c, sqrt(b*x), (x+c)/(x+b)]
F=v[p]
G=v[q]

#Computing the Composites
A3=G-F
A4=F*G
\end{sagesilent}
 
\latexProblemContent{
\ifVerboseLocation This is Precalc Compute Question 0008. \\ \fi
\begin{problem}
Given the functions $f(x)=\sage{F}$ and $g(x)=\sage{G}$, compute the following:

\input{Precalc-Compute-0008.HELP.tex}

\begin{itemize}
\item $g(x)-f(x)=\answer{\sage{A3}}$
\item $f(x)g(x)=\answer{\sage{A4}}$
\end{itemize}

\end{problem}}%}
%%%%%%%%%%%%%%%%%%%%%%

%%%%%%%%%%%%%%%%%%%%%%%
%%\tagged{Ans@ShortAns, Type@Compute, Topic@Precalc, Sub@Composition, File@0009}{
\begin{sagesilent}
# Define variables and constants/exponents
b=RandInt(1,10)
c=NonZeroInt(-5,5,[b])
p=RandInt(0,4)
q=NonZeroInt(0,4,[p])

#Choosing the function
v=[b*x+c, b*x^2+c, b*x^3+c, sqrt(b*x), (x+c)/(x+b)]
F=v[p]
G=v[q]

#Computing the Composites
A1=F(G)
A2=G(F)
\end{sagesilent}
 
\latexProblemContent{
\ifVerboseLocation This is Precalc Compute Question 0009. \\ \fi
\begin{problem}
Given the functions $f(x)=\sage{F}$ and $g(x)=\sage{G}$, compute the following compositions:

\input{Precalc-Compute-0009.HELP.tex}

\begin{itemize}
\item $f(g(x))=\answer{\sage{A1}}$
\item $g(f(x))=\answer{\sage{A2}}$
\end{itemize}

\end{problem}}%}
%%%%%%%%%%%%%%%%%%%%%%

%%%%%%%%%%%%%%%%%%%%%%%%
%%%\tagged{Ans@ShortAns, Type@Compute, Topic@Precalc, Sub@Composition, File@0010}{
%\begin{sagesilent}
%# Define variables and constants/exponents
%a=RandInt(-2,5)
%p=RandInt(0,3)
%q=RandInt(0,3)
%
%#Choosing the function
%v=[a-1, a, a+1, a+2]
%B=v[p]
%C=v[q]
%
%#Computing the slope
%vg=[a-2, C, 3*a, 2*a-1]
%vf=[2*a, B, 2*a+3, a-4]
%A1=vf[q]
%A2=vg[p]
%
%B1 = a-1
%B2 = a-2
%B3 = a+1
%B4 = a+2
%B5 = a-4
%C1 = 2*a
%C2 = 2*a+3
%C3 = 3*a
%C4 = 2*a-1
%
%\end{sagesilent}
% 
%\latexProblemContent{
%\ifVerboseLocation This is Precalc Compute Question 0010. \\ \fi
%\begin{problem}
%Given the following table, evaluate $f(g(\sage{a}))$ and $g(f(\sage{a}))$.
%
%\begin{tabular}{c|c|c}
%$x$ & $f(x)$ & $g(x)$\\
%\hline
%$\sage{B1}$  & $\sage{C1}$ & $\sage{B2}$ \\
%$\sage{a}$  & $\sage{B}$ & $\sage{C}$ \\
%$\sage{B3}$  & $\sage{C2}$ & $\sage{C3}$ \\
%$\sage{B4}$  & $\sage{B5}$ & $\sage{C4}$ 
%\end{tabular}
%
%\input{Precalc-Compute-0010.HELP.tex}
%
%\[
%f(g(\sage{a}))=\answer{\sage{A1}} \qquad\qquad g(f(\sage{a}))=\answer{\sage{A2}}
%\]
%
%\end{problem}}%}
%%%%%%%%%%%%%%%%%%%%%%%




%%%%%%%%%%%%%%%%%%%%%%%
%%\tagged{Ans@ShortAns, Type@Compute, Topic@Precalc, Sub@Composition, File@0011}{
\begin{sagesilent}
# Define variables and constants/exponents
a = 0
b=0
F(x) = x
G(x) = x
while F(a) == -b or G(a) == -b:
    a=RandInt(1,5)
    b=RandInt(1,4)
    c=NonZeroInt(-5,5,[b])
    p=RandInt(0,4)
    q=NonZeroInt(0,4,[p])
    
    #Choosing the function
    v=[b*x+c, b*x^2+c, b*x^3+c, sqrt(b*x), (x+c)/(x+b)]
    F=v[p]
    G=v[q]

#Computing the Composites
A1=F(G(a))
A2=G(F(a))
\end{sagesilent}
 
\latexProblemContent{
\ifVerboseLocation This is Precalc Compute Question 0011. \\ \fi
\begin{problem}
Given the functions $f(x)=\sage{F}$ and $g(x)=\sage{G}$, compute the following compositions:

\input{Precalc-Compute-0011.HELP.tex}

\begin{itemize}
\item $f(g(\sage{a}))=\answer{\sage{A1}}$
\item $g(f(\sage{a}))=\answer{\sage{A2}}$
\end{itemize}

\end{problem}}%}
%%%%%%%%%%%%%%%%%%%%%%


%%%%%%%%%%%%%%%%%%%%%%%
%%\tagged{Ans@ShortAns, Type@Compute, Topic@Precalc, Sub@Domain, Sub@Composition, File@0012}{
\begin{sagesilent}
# Define variables and constants/exponents
a=RandInt(1,5)
b=RandInt(1,10)

#Choosing the function
F=(x-a)^(1/2)
G=(b-x)^(1/2)
\end{sagesilent}
 
\latexProblemContent{
\ifVerboseLocation This is Precalc Compute Question 0012. \\ \fi
\begin{problem}
Given the functions $f(x)=\sage{F}$ and $g(x)=\sage{G}$, find the domain of the function $(f\circ g)(x)$.

\input{Precalc-Compute-0012.HELP.tex}

\[\left(\answer{-\infty}, \answer{\sage{b-a^2}}\right]\]
\end{problem}}%}
%%%%%%%%%%%%%%%%%%%%%%

%%%%%%%%%%%%%%%%%%%%%%%
%%\tagged{Ans@ShortAns, Type@Compute, Topic@Precalc, Sub@Domain, Sub@Composition, File@0013}{
\begin{sagesilent}
# Define variables and constants/exponents
a=RandInt(2,5)
b=RandInt(1,10)
while a^2<b:
   b=RandInt(1,10)

#Choosing the function
F=1/(x-a)
G=(x+b)^(1/2)
\end{sagesilent}
 
\latexProblemContent{
\ifVerboseLocation This is Precalc Compute Question 0013. \\ \fi
\begin{problem}
Given the functions $f(x)=\sage{F}$ and $g(x)=\sage{G}$, find the domain of the function $(f\circ g)(x)$.

\input{Precalc-Compute-0013.HELP.tex}

\[\left[\answer{\sage{-b}}, \answer{\sage{a^2-b}}\right)\bigcup\left(\answer{\sage{a^2-b}},\answer{\infty}\right)\]
\end{problem}}%}
%%%%%%%%%%%%%%%%%%%%%%


































