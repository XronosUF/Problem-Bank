%%%%%%%%%%%%%%%%%%%%%%%%%%
%%%%%%%%%%%%%%%%%%%%%%%%%%



%%%%%%%%%%%%%%%%%%%%%%%
%%\tagged{Ans@ShortAns, Type@Compute, Topic@Precalc, Sub@Notation, File@0001}{
\begin{sagesilent}
# Define variables and constants/exponents
a=RandInt(1,5)
b=RandInt(1,4)
c=NonZeroInt(-5,5,[b])
p=RandInt(0,3)

#Define the function and values
v=[b*x+c, b*x^2+c, (b*x)^(1/2), (x+c)/(x+b)]
F=v[p]
F1=F(a)
F2=F(a+1)
F3=F(a+2)
\end{sagesilent}
 
\latexProblemContent{
\ifVerboseLocation This is Precalc Compute Question 0001. \\ \fi
\begin{problem}
Evaluate the function $f(x)=\sage{F}$ at the given values.

\input{Precalc-Compute-0001.HELP.tex}

\begin{tabular}{|l|l|}
\hline
$x$ & $f(x)$\\
\hline
$\sage{a}$ & $\answer{\sage{F1}}$\\
\hline
$\sage{a+1}$ & $\answer{\sage{F2}}$\\
\hline
$\sage{a+2}$ & $\answer{\sage{F3}}$\\
\hline
\end{tabular}

\end{problem}}%}
%%%%%%%%%%%%%%%%%%%%%%

%%%%%%%%%%%%%%%%%%%%%%%
%%\tagged{Ans@ShortAns, Type@Compute, Topic@Precalc, Sub@Domain, Sub@Poly, File@0002}{
\begin{sagesilent}
# Define variables and constants/exponents
b=RandInt(1,4)
c=RandInt(-5,5)
p=RandInt(0,2)

#Define the function and values
v=[b*x+c, b*x^2+c, b*x^3+c]
F=v[p]   
\end{sagesilent}
 
\latexProblemContent{
\ifVerboseLocation This is Precalc Compute Question 0002. \\ \fi
\begin{problem}
Compute the domain of the function $\sage{F}$.

\input{Precalc-Compute-0002.HELP.tex}

\[\left(\answer{-\infty},\answer{\infty}\right)\]

\end{problem}}%}
%%%%%%%%%%%%%%%%%%%%%%

%%%%%%%%%%%%%%%%%%%%%%%
%%\tagged{Ans@ShortAns, Type@Compute, Topic@Precalc, Sub@Domain, Sub@Radical, File@0003}{
\begin{sagesilent}
# Define variables and constants/exponents
b=RandInt(1,4)
c=NonZeroInt(-5,5)
p=Integer(randint(0,1))

#Define the function and values
v=[(b*x)^(1/2), (b*x+c)^(1/2)]
F=v[p]
if p==0:
   D1=0
   D2=infinity
else:
   D1=-c/b
   D2=infinity   
\end{sagesilent}
 
\latexProblemContent{
\ifVerboseLocation This is Precalc Compute Question 0003. \\ \fi
\begin{problem}
Compute the domain of the function $\sage{F}$.

\input{Precalc-Compute-0003.HELP.tex}

\[\left(\answer{\sage{D1}},\answer{\sage{D2}}\right)\]

\end{problem}}%}
%%%%%%%%%%%%%%%%%%%%%%

%%%%%%%%%%%%%%%%%%%%%%%
%%\tagged{Ans@ShortAns, Type@Compute, Topic@Precalc, Sub@Domain, Sub@Rational, File@0004}{
\begin{sagesilent}
# Define variables and constants/exponents
b=RandInt(1,10)
c=RandInt(-5,5)

#Define the function and values
F=(x+c)/(x+b)
\end{sagesilent}
 
\latexProblemContent{
\ifVerboseLocation This is Precalc Compute Question 0004. \\ \fi
\begin{problem}
Compute the domain of the function $\sage{F}$.

\input{Precalc-Compute-0004.HELP.tex}

\[\left(\answer{-\infty},\answer{\sage{-b}}\right)\bigcup\left(\answer{\sage{-b}},\answer{\infty}\right)\]

\end{problem}}%}
%%%%%%%%%%%%%%%%%%%%%%

%%%%%%%%%%%%%%%%%%%%%%%%
%%%\tagged{Ans@MC, Type@Compute, Topic@Precalc, Sub@Domain, Sub@Relation, File@0005}{
%\begin{sagesilent}
%# Define variables and constants/exponents
%a=RandInt(-5,5)
%b = a-1
%c = a+1
%d = a+2
%e = a+3
%q = a-2
%\end{sagesilent}
% 
%\latexProblemContent{
%\ifVerboseLocation This is Precalc Compute Question 0005. \\ \fi
%\begin{problem}
%Which of the following is \underline{not} in the domain of the function:
%\[\lbrace(\sage{q},\sage{b}), (\sage{b},\sage{a}), (\sage{a},\sage{c}), (\sage{c},\sage{d}), (\sage{d},\sage{e})\rbrace\]
%
%\input{Precalc-Compute-0005.HELP.tex}
%%\answer{$\sage{e}$}
%
%\begin{multipleChoice}
%\choice{$\sage{b}$}
%\choice{$\sage{d}$}
%\choice{$\sage{a}$}
%\choice[correct]{$\sage{e}$}
%\choice{$\sage{c}$}
%\end{multipleChoice}
%
%\end{problem}}%}
%%%%%%%%%%%%%%%%%%%%%%%

%%%%%%%%%%%%%%%%%%%%%%%
%%\tagged{Ans@ShortAns, Type@Compute, Topic@Precalc, Sub@Avg-RoC, File@0006}{
\begin{sagesilent}
# Define variables and constants/exponents
a=RandInt(1,5)
c=RandInt(-5,5)
d=RandInt(a+1,10)
b=NonZeroInt(1,4,[-a,-d,c])
p=RandInt(0,4)

#Choosing the function
v=[b*x+c, b*x^2+c, b*x^3+c, (b*x)^(1/2), (x+c)/(x+b)]
F=v[p]

#Computing the slope
Ans=(F(d)-F(a))/(d-a)
\end{sagesilent}
 
\latexProblemContent{
\ifVerboseLocation This is Precalc Compute Question 0006. \\ \fi
\begin{problem}
Compute the average rate of change of the function $\sage{F}$ on the interval $[\sage{a},\sage{d}]$.

\input{Precalc-Compute-0006.HELP.tex}

\[\answer{\sage{Ans}}\]

\end{problem}}%}
%%%%%%%%%%%%%%%%%%%%%%

%%%%%%%%%%%%%%%%%%%%%%%
%%\tagged{Ans@ShortAns, Type@Compute, Topic@Precalc, Sub@Composition, File@0007}{
\begin{sagesilent}
# Define variables and constants/exponents
b=RandInt(1,4)
c=RandInt(-5,5)
p=RandInt(0,4)
q=NonZeroInt(0,4,[p])

#Choosing the function
v=[b*x+c, b*x^2+c, b*x^3+c, (b*x)^(1/2), (x+c)/(x+b)]
F=v[p]
G=v[q]

#Computing the Composites
A1=F-G
A2=G/F
\end{sagesilent}
 
\latexProblemContent{
\ifVerboseLocation This is Precalc Compute Question 0007. \\ \fi
\begin{problem}
Given the functions $f(x)=\sage{F}$ and $g(x)=\sage{G}$, compute the following:

\input{Precalc-Compute-0007.HELP.tex}

\begin{itemize}
\item $f(x)-g(x)=\answer{\sage{A1}}$
\item $\frac{g(x)}{f(x)}=\answer{\sage{A2}}$
\end{itemize}

\end{problem}}%}
%%%%%%%%%%%%%%%%%%%%%%


%%%%%%%%%%%%%%%%%%%%%%%
%%\tagged{Ans@ShortAns, Type@Compute, Topic@Precalc, Sub@Composition, File@0008}{
\begin{sagesilent}
# Define variables and constants/exponents
b=RandInt(1,4)
c=RandInt(-5,5)
p=RandInt(0,4)
q=NonZeroInt(0,4,[p])

#Choosing the function
v=[b*x+c, b*x^2+c, b*x^3+c, (b*x)^(1/2), (x+c)/(x+b)]
F=v[p]
G=v[q]

#Computing the Composites
A3=G-F
A4=F*G
\end{sagesilent}
 
\latexProblemContent{
\ifVerboseLocation This is Precalc Compute Question 0008. \\ \fi
\begin{problem}
Given the functions $f(x)=\sage{F}$ and $g(x)=\sage{G}$, compute the following:

\input{Precalc-Compute-0008.HELP.tex}

\begin{itemize}
\item $g(x)-f(x)=\answer{\sage{A3}}$
\item $f(x)g(x)=\answer{\sage{A4}}$
\end{itemize}

\end{problem}}%}
%%%%%%%%%%%%%%%%%%%%%%

%%%%%%%%%%%%%%%%%%%%%%%
%%\tagged{Ans@ShortAns, Type@Compute, Topic@Precalc, Sub@Composition, File@0009}{
\begin{sagesilent}
# Define variables and constants/exponents
b=RandInt(1,4)
c=NonZeroInt(-5,5,[b])
p=RandInt(0,4)
q=NonZeroInt(0,4,[p])

#Choosing the function
v=[b*x+c, b*x^2+c, b*x^3+c, (b*x)^(1/2), (x+c)/(x+b)]
F=v[p]
G=v[q]

#Computing the Composites
A1=F(G)
A2=G(F)
\end{sagesilent}
 
\latexProblemContent{
\ifVerboseLocation This is Precalc Compute Question 0009. \\ \fi
\begin{problem}
Given the functions $f(x)=\sage{F}$ and $g(x)=\sage{G}$, compute the following compositions:

\input{Precalc-Compute-0009.HELP.tex}

\begin{itemize}
\item $f(g(x))=\answer{\sage{A1}}$
\item $g(f(x))=\answer{\sage{A2}}$
\end{itemize}

\end{problem}}%}
%%%%%%%%%%%%%%%%%%%%%%

%%%%%%%%%%%%%%%%%%%%%%%%
%%%\tagged{Ans@ShortAns, Type@Compute, Topic@Precalc, Sub@Composition, File@0010}{
%\begin{sagesilent}
%# Define variables and constants/exponents
%a=RandInt(-2,5)
%p=RandInt(0,3)
%q=RandInt(0,3)
%
%#Choosing the function
%v=[a-1, a, a+1, a+2]
%B=v[p]
%C=v[q]
%
%#Computing the slope
%vg=[a-2, C, 3*a, 2*a-1]
%vf=[2*a, B, 2*a+3, a-4]
%A1=vf[q]
%A2=vg[p]
%
%B1 = a-1
%B2 = a-2
%B3 = a+1
%B4 = a+2
%B5 = a-4
%C1 = 2*a
%C2 = 2*a+3
%C3 = 3*a
%C4 = 2*a-1
%
%\end{sagesilent}
% 
%\latexProblemContent{
%\ifVerboseLocation This is Precalc Compute Question 0010. \\ \fi
%\begin{problem}
%Given the following table, evaluate $f(g(\sage{a}))$ and $g(f(\sage{a}))$.
%
%\begin{tabular}{c|c|c}
%$x$ & $f(x)$ & $g(x)$\\
%\hline
%$\sage{B1}$  & $\sage{C1}$ & $\sage{B2}$ \\
%$\sage{a}$  & $\sage{B}$ & $\sage{C}$ \\
%$\sage{B3}$  & $\sage{C2}$ & $\sage{C3}$ \\
%$\sage{B4}$  & $\sage{B5}$ & $\sage{C4}$ 
%\end{tabular}
%
%\input{Precalc-Compute-0010.HELP.tex}
%
%\[
%f(g(\sage{a}))=\answer{\sage{A1}} \qquad\qquad g(f(\sage{a}))=\answer{\sage{A2}}
%\]
%
%\end{problem}}%}
%%%%%%%%%%%%%%%%%%%%%%%




%%%%%%%%%%%%%%%%%%%%%%%
%%\tagged{Ans@ShortAns, Type@Compute, Topic@Precalc, Sub@Composition, File@0011}{
\begin{sagesilent}
# Define variables and constants/exponents
a = 0
b=0
F(x) = x
G(x) = x
while F(a) == -b or G(a) == -b:
    a=RandInt(1,5)
    b=RandInt(1,4)
    c=NonZeroInt(-5,5,[b])
    p=RandInt(0,4)
    q=NonZeroInt(0,4,[p])
    
    #Choosing the function
    v=[b*x+c, b*x^2+c, b*x^3+c, (b*x)^(1/2), (x+c)/(x+b)]
    F=v[p]
    G=v[q]

#Computing the Composites
A1=F(G(a))
A2=G(F(a))
\end{sagesilent}
 
\latexProblemContent{
\ifVerboseLocation This is Precalc Compute Question 0011. \\ \fi
\begin{problem}
Given the functions $f(x)=\sage{F}$ and $g(x)=\sage{G}$, compute the following compositions:

\input{Precalc-Compute-0011.HELP.tex}

\begin{itemize}
\item $f(g(\sage{a}))=\answer{\sage{A1}}$
\item $g(f(\sage{a}))=\answer{\sage{A2}}$
\end{itemize}

\end{problem}}%}
%%%%%%%%%%%%%%%%%%%%%%


%%%%%%%%%%%%%%%%%%%%%%%
%%\tagged{Ans@ShortAns, Type@Compute, Topic@Precalc, Sub@Domain, Sub@Composition, File@0012}{
\begin{sagesilent}
# Define variables and constants/exponents
a=RandInt(1,5)
b=RandInt(1,10)

#Choosing the function
F=(x-a)^(1/2)
G=(b-x)^(1/2)
\end{sagesilent}
 
\latexProblemContent{
\ifVerboseLocation This is Precalc Compute Question 0012. \\ \fi
\begin{problem}
Given the functions $f(x)=\sage{F}$ and $g(x)=\sage{G}$, find the domain of the function $(f\circ g)(x)$.

\input{Precalc-Compute-0012.HELP.tex}

\[\left(\answer{-\infty}, \answer{\sage{b-a^2}}\right]\]
\end{problem}}%}
%%%%%%%%%%%%%%%%%%%%%%

%%%%%%%%%%%%%%%%%%%%%%%
%%\tagged{Ans@ShortAns, Type@Compute, Topic@Precalc, Sub@Domain, Sub@Composition, File@0013}{
\begin{sagesilent}
# Define variables and constants/exponents
a=RandInt(2,5)
b=RandInt(1,10)
while a^2<b:
   b=RandInt(1,10)

#Choosing the function
F=1/(x-a)
G=(x+b)^(1/2)
\end{sagesilent}
 
\latexProblemContent{
\ifVerboseLocation This is Precalc Compute Question 0013. \\ \fi
\begin{problem}
Given the functions $f(x)=\sage{F}$ and $g(x)=\sage{G}$, find the domain of the function $(f\circ g)(x)$.

\input{Precalc-Compute-0013.HELP.tex}

\[\left[\answer{\sage{-b}}, \answer{\sage{a^2-b}}\right)\bigcup\left(\answer{\sage{a^2-b}},\answer{\infty}\right)\]
\end{problem}}%}
%%%%%%%%%%%%%%%%%%%%%%


%%%%%%%%%%%%%%%%%%%%%%%%
%%%\tagged{Ans@ShortAns, Type@Compute, Topic@Precalc, Sub@Translation, File@0014}{
%\begin{sagesilent}
%# Define variables and constants/exponents
%a=NonZeroInt(-4,4)
%b=RandInt(-4,4)
%p=RandInt(-4,4)
%A=RandInt(-10,10)
%B=RandInt(-10,10)
%C=RandInt(-10,10)
%D=RandInt(-10,10)
%
%#Choosing the function
%g=a*x+b
%A1=g(A)
%A2=g(B)
%A3=g(C)
%A4=g(D)
%
%\end{sagesilent}
% 
%\latexProblemContent{
%\ifVerboseLocation This is Precalc Compute Question 0014. \\ \fi
%\begin{problem}
%Given the table for $f(x)$, compute the values for the translated function $g(x)=\sage{sgn(a)*abs(a)}f(x)\sage{sgn(b)*abs(b)}$.
%
%\input{Precalc-Compute-0014.HELP.tex}
%
%\begin{tabular}{|l|l|l|l|l|}
%\hline
%$x$     & $\sage{p}$ & $\sage{p+1}$ & $\sage{p+2}$ & $\sage{p+3}$ \\[5pt]
%\hline
%$f(x)$  & $\sage{A}$ & $\sage{B}$ & $\sage{C}$ & $\sage{D}$ \\[5pt]
%\hline
%$g(x)$  & $\answer{\sage{A1}}$ & $\answer{\sage{A2}}$ & $\answer{\sage{A3}}$ & $\answer{\sage{A4}}$ \\
%\hline
%\end{tabular}
%
%
%\end{problem}}%}
%%%%%%%%%%%%%%%%%%%%%%%

%%%%%%%%%%%%%%%%%%%%%%%%
%%%\tagged{Ans@ShortAns, Type@Compute, Topic@Precalc, Sub@Translation, File@0014}{
%\begin{sagesilent}
%# Define variables and constants/exponents
%a=NonZeroInt(-4,4)
%b=RandInt(-4,4)
%p=RandInt(-4,4)
%A=RandInt(-10,10)
%B=RandInt(-10,10)
%C=RandInt(-10,10)
%D=RandInt(-10,10)
%
%#Choosing the function
%g=a*x+d
%A1=g(A)
%A2=g(B)
%A3=g(C)
%A4=g(D)
%
%\end{sagesilent}
% 
%\latexProblemContent{
%\ifVerboseLocation This is Precalc Compute Question 0014. \\ \fi
%\begin{problem}
%Given the table for $f(x)$, compute the values for the translated function $g(x)=\sage{sgn(a)*abs(a)}f(x)\sage{sgn(d)*abs(d)}$.
%
%\input{Precalc-Compute-0014.HELP.tex}
%
%\[\begin{array}{|l|l|l|}
%\hline
%x     & f(x)  & g(x)  \\[5pt]
%\hline
%\sage{p}   & \sage{A} & \answer{\sage{A1}}\\[5pt] 
%\sage{p+1} & \sage{B} & \answer{\sage{A2}}\\[5pt]
%\sage{p+2} & \sage{C} & \answer{\sage{A3}}\\[5pt]
%\sage{p+3} & \sage{D} & \answer{\sage{A4}}\\[5pt]
%\hline
%\end{array}\]
%
%\end{problem}}%}
%%%%%%%%%%%%%%%%%%%%%%%



%%%%%%%%%%%%%%%%%%%%%%%
%%\tagged{Ans@ShortAns, Type@Compute, Topic@Precalc, Sub@Translation, File@0015}{
\begin{sagesilent}
# Define variables and constants/exponents
a=NonZeroInt(-4,4,[0,1])
b=NonZeroInt(-4,4, [0,1])
c=RandInt(-4,4)
d=RandInt(-4,4)
p=RandInt(-4,4)
q=RandInt(-4,4)

#Choosing the function
fy=a*x+d
fx=(x-c)/b
A1=fx(p)
A2=fy(q)
X=b*x+c
\end{sagesilent}
 
\latexProblemContent{
\ifVerboseLocation This is Precalc Compute Question 0015. \\ \fi
\begin{problem}
Given $(\sage{p},\sage{q})$ is a point on the graph of $f(x)$, find a corresponding point on the translated function $g(x)=\sage{sgn(a)*abs(a)}f(\sage{X})\sage{ISP(d)}$.

\input{Precalc-Compute-0015.HELP.tex}

\[\left(\answer{\sage{A1}},\answer{\sage{A2}}\right)\]

\end{problem}}%}
%%%%%%%%%%%%%%%%%%%%%%

%%%%%%%%%%%%%%%%%%%%%%%
%%\tagged{Ans@ShortAns, Type@Compute, Topic@Precalc, Sub@AbsoluteValue, File@0016}{
\begin{sagesilent}
# Define variables and constants/exponents
a=RandInt(1,10)
b=RandInt(-4,4)

#Choosing the function
f=x-b

\end{sagesilent}
 
\latexProblemContent{
\ifVerboseLocation This is Precalc Compute Question 0016. \\ \fi
\begin{problem}
Describe all values of $x$ which are within a distance of $\sage{a}$ from the number $\sage{b}$.

What is the correct inequality to use for this statement?

\input{Precalc-Compute-0016.HELP.tex}

\begin{problem}
\[\Bigg\vert\answer{\sage{f}}\Bigg\vert <\answer{\sage{a}}\]
\end{problem}

\begin{multipleChoice}
\choice[correct]{$|\ast|< \#$}
\choice{$|\ast|\leq \#$}
\choice{$|\ast|\geq \#$}
\choice{$|\ast|> \#$}
\end{multipleChoice}
\end{problem}}%}
%%%%%%%%%%%%%%%%%%%%%%

%%%%%%%%%%%%%%%%%%%%%%%
%%\tagged{Ans@ShortAns, Type@Compute, Topic@Precalc, Sub@AbsoluteValue, Sub@Translation, File@0017}{
\begin{sagesilent}
# Define variables and constants/exponents
a=RandInt(-10,10)
b=RandInt(-10,10)
c=NonZeroInt(-10,10,[a])
d=RandInt(-10,10)

#Choosing the function
Ans=(d-b)/abs(c-a)
f=abs(x-a)+b

\end{sagesilent}
 
\latexProblemContent{
\ifVerboseLocation This is Precalc Compute Question 0017. \\ \fi
\begin{problem}
Find the value of $a$ to complete the equation $f(x) = a\sage{f}$, given that $(\sage{c},\sage{d})$ is a point on the graph of $f(x)$.

\input{Precalc-Compute-0017.HELP.tex}

\[a = \answer{\sage{Ans}}\]

\end{problem}}%}
%%%%%%%%%%%%%%%%%%%%%%

%%%%%%%%%%%%%%%%%%%%%%%
%%\tagged{Ans@ShortAns, Type@Compute, Topic@Precalc, Sub@AbsoluteValue, Sub@Zeros, File@0018}{
\begin{sagesilent}
# Define variables and constants/exponents
a=RandInt(-10,10)
b=RandInt(-10,-1)

#Choosing the function
f=abs(x-a)+b

#Compute Zeros
Ans=solve(f,x, to_poly_solve=True)
A1=Ans[0].rhs()
A2=Ans[1].rhs()
\end{sagesilent}
 
\latexProblemContent{
\ifVerboseLocation This is Precalc Compute Question 0018. \\ \fi
\begin{problem}
Find the zeros of the absolute value equation $f(x)=\sage{f}$.

\input{Precalc-Compute-0018.HELP.tex}

\[\mbox{Smaller }x-\mbox{value}:\answer{\sage{A1}}\qquad\qquad\qquad\mbox{Larger }x-\mbox{value}:\answer{\sage{A2}}\]

\end{problem}}%}
%%%%%%%%%%%%%%%%%%%%%%


%%%%%%%%%%%%%%%%%%%%%%%
%%\tagged{Ans@ShortAns, Type@Compute, Topic@Precalc, Sub@Inverse, File@0019}{
\begin{sagesilent}
# Define variables and constants/exponents
a=NonZeroInt(-10,10)
b=RandInt(-10,10)
c=NonZeroInt(-10,10,[b])

#Choosing the function
f=a/(x-b)
g=(a+c*x)/x

#Compute Zeros
A1=f(g)
A2=g(f)
\end{sagesilent}
 
\latexProblemContent{
\ifVerboseLocation This is Precalc Compute Question 0019. \\ \fi
\begin{problem}
Are $f(x)=\sage{f}$ and $g(x)=\sage{g}$ inverses of each other?\\
(Hint: Check $f(g(x))$ and $g(f(x))$)

\input{Precalc-Compute-0019.HELP.tex}

\begin{problem}
What are $f(g(x))$ and $g(f(x))$?

\[f(g(x)) = \answer{\sage{A1}}\]

\[g(f(x)) = \answer{\sage{A2}}\]
\end{problem}

\begin{multipleChoice}
\choice{Yes}
\choice[correct]{No}
\end{multipleChoice}

\end{problem}}%}
%%%%%%%%%%%%%%%%%%%%%%

%%%%%%%%%%%%%%%%%%%%%%%
%%\tagged{Ans@ShortAns, Type@Compute, Topic@Precalc, Sub@Inverse, File@0020}{
\begin{sagesilent}
# Define variables and constants/exponents
a=NonZeroInt(-10,10)
b=RandInt(-10,10)

#Choosing the function
f=a/(x-b)
g=(a+b*x)/x
\end{sagesilent}
 
\latexProblemContent{
\ifVerboseLocation This is Precalc Compute Question 0020. \\ \fi
\begin{problem}
Are $f(x)=\sage{f}$ and $g(x)=\sage{g}$ inverses of each other?\\
(Hint: Check $f(g(x))$ and $g(f(x))$)

\input{Precalc-Compute-0020.HELP.tex}

\begin{problem}
What are $f(g(x))$ and $g(f(x))$?

\[f(g(x)) = \answer{x}\]

\[g(f(x)) = \answer{x}\]
\end{problem}

\begin{multipleChoice}
\choice[correct]{Yes}
\choice{No}
\end{multipleChoice}

\end{problem}}%}
%%%%%%%%%%%%%%%%%%%%%%

%%%%%%%%%%%%%%%%%%%%%%%
%%\tagged{Ans@ShortAns, Type@Compute, Topic@Precalc, Sub@Inverse, Sub@Domain, File@0021}{
\begin{sagesilent}
# Define variables and constants/exponents
a=RandInt(-10,10)
b=RandInt(-10,10)

\end{sagesilent}
 
\latexProblemContent{
\ifVerboseLocation This is Precalc Compute Question 0021. \\ \fi
\begin{problem}
Given that $f(x)$ has domain $[\sage{a},\infty)$ and range $(-\infty,\sage{b}]$, determine the domain of $f^{-1}$.

\input{Precalc-Compute-0021.HELP.tex}

\[\mbox{Domain}: (\answer{-\infty}, \answer{\sage{b}}]\qquad\qquad\mbox{Range}: [\answer{\sage{a}}, \answer{\infty})\]

\end{problem}}%}
%%%%%%%%%%%%%%%%%%%%%%

%%%%%%%%%%%%%%%%%%%%%%%
%%\tagged{Ans@ShortAns, Type@Compute, Topic@Precalc, Sub@Inverse, Sub@Linear, File@0022}{
\begin{sagesilent}
# Define variables and constants/exponents
a=NonZeroInt(-10,10)
b=RandInt(-10,10)

#Choosing the function
f=a*x+b
g=(x-b)/a
\end{sagesilent}
 
\latexProblemContent{
\ifVerboseLocation This is Precalc Compute Question 0022. \\ \fi
\begin{problem}
Find the inverse of the function $f(x)=\sage{f}$.

\input{Precalc-Compute-0022.HELP.tex}

\begin{problem}
What is the domain of $f^{-1}(x)$?
\[(\answer{-\infty},\answer{\infty})\]
\end{problem}

\[f^{-1}(x) = \answer{\sage{g}}\]

\end{problem}}%}
%%%%%%%%%%%%%%%%%%%%%%

%%%%%%%%%%%%%%%%%%%%%%%
%%\tagged{Ans@ShortAns, Type@Compute, Topic@Precalc, Sub@Inverse, Sub@Rational, File@0023}{
\begin{sagesilent}
# Define variables and constants/exponents
a=NonZeroInt(-10,10)
b=RandInt(-10,10)
c=RandInt(-10,10)

#Choosing the function
f=a/(x-b)+c
g=a/(x-c)+b
\end{sagesilent}
 
\latexProblemContent{
\ifVerboseLocation This is Precalc Compute Question 0023. \\ \fi
\begin{problem}
Find the inverse of the function $f(x)=\sage{f}$.

\input{Precalc-Compute-0023.HELP.tex}

\begin{problem}
What is the domain of $f^{-1}(x)$?
\[(\answer{-\infty},\answer{\sage{c}})\bigcup(\answer{\sage{c}},\answer{\infty})\]
\end{problem}

\[f^{-1}(x) = \answer{\sage{g}}\]

\end{problem}}%}
%%%%%%%%%%%%%%%%%%%%%%

%%%%%%%%%%%%%%%%%%%%%%%
%%\tagged{Ans@ShortAns, Type@Compute, Topic@Precalc, Sub@Inverse, Sub@Radical, File@0024}{
\begin{sagesilent}
# Define variables and constants/exponents
a=NonZeroInt(-10,10)
b=RandInt(-10,10)

#Choosing the function
f=a+(x-b)^(1/2)
g=(x-a)^2+b
\end{sagesilent}
 
\latexProblemContent{
\ifVerboseLocation This is Precalc Compute Question 0024. \\ \fi
\begin{problem}
Find the inverse of the function $f(x)=\sage{f}$.

\input{Precalc-Compute-0024.HELP.tex}

\begin{problem}
What is the domain of $f^{-1}(x)$?
\[[\answer{\sage{a}},\answer{\infty})\]
\end{problem}

\[f^{-1}(x) = \answer{\sage{g}}\]

\end{problem}}%}
%%%%%%%%%%%%%%%%%%%%%%

%%%%%%%%%%%%%%%%%%%%%%%
%%\tagged{Ans@ShortAns, Type@Compute, Topic@Precalc, Sub@Linear, File@0025}{
\begin{sagesilent}
# Define variables and constants/exponents
a=RandInt(-10,10)
b=RandInt(-10,10)
c=NonZeroInt(-10,10,[a])
d=RandInt(-10,10)

#Choosing the function
m=(d-b)/(c-a)
f=m*(x-a)+b
\end{sagesilent}
 
\latexProblemContent{
\ifVerboseLocation This is Precalc Compute Question 0025. \\ \fi
\begin{problem}
Given the points $(\sage{a},\sage{b})$ and $(\sage{c},\sage{d})$, answer the following.  

The graph of this line is...

\input{Precalc-Compute-0025.HELP.tex}

\begin{problem}
Compute the equation of the line containing the two points.
\[y=\answer{\sage{f}}\]
\end{problem}

\begin{multipleChoice}
\choice{a horizontal line}
\choice{a vertical line}
\choice[correct]{neither}
\end{multipleChoice}

\end{problem}}%}
%%%%%%%%%%%%%%%%%%%%%%

%%%%%%%%%%%%%%%%%%%%%%%
%%\tagged{Ans@ShortAns, Type@Compute, Topic@Precalc, Sub@Linear, Sub@Horizontal, File@0026}{
\begin{sagesilent}
# Define variables and constants/exponents
a=RandInt(-10,10)
b=RandInt(-10,10)
c=RandInt(-10,10)

\end{sagesilent}
 
\latexProblemContent{
\ifVerboseLocation This is Precalc Compute Question 0026. \\ \fi
\begin{problem}
Given the points $(\sage{a},\sage{b})$ and $(\sage{c},\sage{b})$, answer the following.  

The graph of this line is...

\input{Precalc-Compute-0026.HELP.tex}

\begin{problem}
Compute the equation of the line containing the two points.
\[y=\answer{\sage{b}}\]
\end{problem}

\begin{multipleChoice}
\choice[correct]{a horizontal line}
\choice{a vertical line}
\choice{neither}
\end{multipleChoice}

\end{problem}}%}
%%%%%%%%%%%%%%%%%%%%%%

%%%%%%%%%%%%%%%%%%%%%%%
%%\tagged{Ans@ShortAns, Type@Compute, Topic@Precalc, Sub@Linear, Sub@Vertical, File@0027}{
\begin{sagesilent}
# Define variables and constants/exponents
a=RandInt(-10,10)
b=RandInt(-10,10)
c=RandInt(-10,10)

\end{sagesilent}
 
\latexProblemContent{
\ifVerboseLocation This is Precalc Compute Question 0027. \\ \fi
\begin{problem}
Given the points $(\sage{a},\sage{b})$ and $(\sage{a},\sage{c})$, answer the following.  

The graph of this line is...

\input{Precalc-Compute-0027.HELP.tex}

\begin{problem}
Compute the equation of the line containing the two points.
\[x=\answer{\sage{a}}\]
\end{problem}

\begin{multipleChoice}
\choice{a horizontal line}
\choice[correct]{a vertical line}
\choice{neither}
\end{multipleChoice}

\end{problem}}%}
%%%%%%%%%%%%%%%%%%%%%%

%%%%%%%%%%%%%%%%%%%%%%%
%%\tagged{Ans@ShortAns, Type@Compute, Topic@Precalc, Sub@Linear, Sub@Parallel, File@0028}{
\begin{sagesilent}
# Define variables and constants/exponents
a=RandInt(-10,10)
b=RandInt(-10,10)
c=NonZeroInt(-10,10)
d=RandInt(-10,10)

#Choosing the function
f=c*x+d
g=c*(x-a)+b
\end{sagesilent}
 
\latexProblemContent{
\ifVerboseLocation This is Precalc Compute Question 0028. \\ \fi
\begin{problem}
Find the equation of the line through the point $(\sage{a},\sage{b})$, which is parallel to the line $y=\sage{f}$.

\input{Precalc-Compute-0028.HELP.tex}

\[y=\answer{\sage{g}}\]

\end{problem}}%}
%%%%%%%%%%%%%%%%%%%%%%

%%%%%%%%%%%%%%%%%%%%%%%
%%\tagged{Ans@ShortAns, Type@Compute, Topic@Precalc, Sub@Linear, Sub@Perpendicular, File@0029}{
\begin{sagesilent}
# Define variables and constants/exponents
a=RandInt(-10,10)
b=RandInt(-10,10)
c=NonZeroInt(-10,10)
d=RandInt(-10,10)

#Choosing the function
f=c*x+d
g=-1/c*(x-a)+b
\end{sagesilent}
 
\latexProblemContent{
\ifVerboseLocation This is Precalc Compute Question 0029. \\ \fi
\begin{problem}
Find the equation of the line through the point $(\sage{a},\sage{b})$, which is perpendicular to the line $y=\sage{f}$.

\input{Precalc-Compute-0029.HELP.tex}

\[y=\answer{\sage{g}}\]

\end{problem}}%}
%%%%%%%%%%%%%%%%%%%%%%

%%%%%%%%%%%%%%%%%%%%%%%
%%\tagged{Ans@ShortAns, Type@Compute, Topic@Precalc, Sub@Linear, Sub@Model, File@0030}{
\begin{sagesilent}
# Define variables and constants/exponents
var('t')
a=RandInt(0,10)
b=RandInt(a+3,13)
c=RandInt(b+3,20)
d=RandInt(1,8)
e=RandInt(1,4)
g=RandInt(c+5,25)

#Slope of pop function
SLOPE=d*100

#Years
year1=2000+a
year2=2000+b
year3=2000+c
year4=2000+g

#Pop
pop1=5000*e+(d*100)
POPFUNC=SLOPE*(t-a)+pop1
pop2=POPFUNC(b)
pop3=POPFUNC(c)
pop4=POPFUNC(g)
\end{sagesilent}
 
\latexProblemContent{
\ifVerboseLocation This is Precalc Compute Question 0030. \\ \fi
\begin{problem}
A town's population grows linearly.  In $\sage{year1}$ the population was $\sage{pop1}$.  By $\sage{year2}$ the population had grown to $\sage{pop2}$.  Assuming this trend continues...

Write the equation of the population $P$ in terms of the number of years $t$ after $2000$.

\input{Precalc-Compute-0030.HELP.tex}

\begin{problem}
Predict the population in $\sage{year3}$

\begin{problem}
When will the population reach $\sage{pop4}$?\\
(Remember $t$ is years after $2000$)
\[\answer{\sage{year4}}\]
\end{problem}

\[\mbox{Population}\approx \answer{\sage{pop3}}\]
\end{problem}

\[P=\answer{\sage{POPFUNC}}\]
\end{problem}}%}
%%%%%%%%%%%%%%%%%%%%%%

%%%%%%%%%%%%%%%%%%%%%%%
%%\tagged{Ans@ShortAns, Type@Compute, Topic@Precalc, Sub@Linear, Sub@Model, File@0031}{
\begin{sagesilent}
# Define variables and constants/exponents
a=RandInt(1,8)
b=RandInt(1,20)
c=RandInt(1,10)
d=RandInt(c+1,20)

#Initial values
C1=500*b
C2=round(a*25*0.01, ndigits=3)	
item1=20*c
item2=10*d

#Costs
COST=C2*x+C1
cost1=COST(item1)
cost2=COST(item2)
\end{sagesilent}
 
\latexProblemContent{
\ifVerboseLocation This is Precalc Compute Question 0031. \\ \fi
\begin{problem}
A company manufactures a particular item.  The fixed cost for opening the factory is $\$\sage{C1}$, and it costs $\$\sage{C2}$ to produce each item.\\

Write a linear model that represents the cost $C$ in terms of the number of items produced $x$.

\input{Precalc-Compute-0031.HELP.tex}

\begin{problem}
What is the cost of producing $\sage{item1}$ items?

\begin{problem}
How many items will they need to make to incur a cost of $\$\sage{cost2}$?

\[\answer{\sage{item2}}\]
\end{problem}

\[\answer{\sage{cost1}}\]
\end{problem}

\[C=\answer{\sage{COST}}\]
\end{problem}}%}
%%%%%%%%%%%%%%%%%%%%%%

%%%%%%%%%%%%%%%%%%%%%%%
%%\tagged{Ans@ShortAns, Type@Compute, Topic@Precalc, Sub@Quadratic, Sub@Vertex, File@0032}{
\begin{sagesilent}
# Define variables and constants/exponents
a=RandInt(1,3)
b=RandInt(-3,3)
c=RandInt(-3,3)

# Define the function
f=x-b
g=expand(a*(x-b)^2+c)
\end{sagesilent}
 
\latexProblemContent{
\ifVerboseLocation This is Precalc Compute Question 0032. \\ \fi
\begin{problem}
Rewrite the equation $\sage{g}$ in vertex form.

\input{Precalc-Compute-0032.HELP.tex}

\begin{problem}
What is the vertex of the parabola?

\[\mbox{Vertex}:(\answer{\sage{b}},\answer{\sage{c}})\]
\end{problem}

\[f(x) = \answer{\sage{a}(\sage{f})^2\sage{ISP(c)}}\]

\end{problem}}%}
%%%%%%%%%%%%%%%%%%%%%%

%%%%%%%%%%%%%%%%%%%%%%%
%%\tagged{Ans@ShortAns, Type@Compute, Topic@Precalc, Sub@Quadratic, Sub@Zeros, File@0033}{
\begin{sagesilent}
# Define variables and constants/exponents
a=RandInt(1,3)
b=RandInt(-3,3)
c=RandInt(-3,3)

# Define the function
f=expand((a*x+b)*(x+c))
#Ans=solve(f,x)
a1=-b/a#=Ans[0].rhs()
a2=-c#Ans[1].rhs()
A1=min(a1,a2)
A2=max(a1,a2)
\end{sagesilent}
 
\latexProblemContent{
\ifVerboseLocation This is Precalc Compute Question 0033. \\ \fi
\begin{problem}
Solve the equation $\sage{f}=0$.

\input{Precalc-Compute-0033.HELP.tex}

\[\mbox{Smaller $x-$value}\,\answer{\sage{A1}}\qquad\qquad\mbox{Larger $x-$value}\, \answer{\sage{A2}}\]

\end{problem}}%}
%%%%%%%%%%%%%%%%%%%%%%

%%%%%%%%%%%%%%%%%%%%%%%
%%\tagged{Ans@ShortAns, Type@Compute, Topic@Precalc, Sub@Quadratic, Sub@Zeros, File@0034}{
\begin{sagesilent}
# Define variables and constants/exponents
a=RandInt(1,3)
b=RandInt(-3,3)
c=RandInt(-3,3)

# Define the function
f=a*x^2+b*x+c
#Ans=solve(f,x)
a1=(-b+(b^2-4*a*c)^(1/2))/(2*a)#Ans[0].rhs()
a2=(-b-(b^2-4*a*c)^(1/2))/(2*a)#=Ans[1].rhs()
A1=min(a1,a2)
A2=max(a1,a2)
\end{sagesilent}
 
\latexProblemContent{
\ifVerboseLocation This is Precalc Compute Question 0034. \\ \fi
\begin{problem}
Solve the equation $\sage{f}=0$.

\input{Precalc-Compute-0034.HELP.tex}

\[\mbox{Smaller $x-$value}\,\answer{\sage{A1}}\qquad\qquad\mbox{Larger $x-$value}\, \answer{\sage{A2}}\]

\end{problem}}%}
%%%%%%%%%%%%%%%%%%%%%%

%%%%%%%%%%%%%%%%%%%%%%%
%%\tagged{Ans@ShortAns, Type@Compute, Topic@Precalc, Sub@Quadratic, Sub@Vertex, File@0035}{
\begin{sagesilent}
# Define variables and constants/exponents
a=RandInt(1,3)
b=RandInt(-3,3)
c=RandInt(-3,3)

# Define the function
f=expand(a*(x-b)^2+c)
\end{sagesilent}
 
\latexProblemContent{
\ifVerboseLocation This is Precalc Compute Question 0035. \\ \fi
\begin{problem}
Complete the square for $f(x) = \sage{f}$.

\input{Precalc-Compute-0035.HELP.tex}

\begin{problem}

Identify the vertex of $f(x)$.

\begin{problem}

Determine the axis of symmetry.

\[x=\answer{\sage{b}}\]

\end{problem}

\[\mbox{Vertex}:\qquad \left(\answer{\sage{b}},\answer{\sage{c}}\right)\]

\end{problem}

\[f(x) = \answer{\sage{f}}\]
\end{problem}}%}
%%%%%%%%%%%%%%%%%%%%%%

%%%%%%%%%%%%%%%%%%%%%%%
%%\tagged{Ans@ShortAns, Type@Compute, Topic@Precalc, Sub@Quadratic, Sub@Complex, File@0036}{
\begin{sagesilent}
# Define variables and constants/exponents
a=RandInt(1,3)
b=RandInt(-3,3)
c=RandInt(-3,3)
d=RandInt(5,20)

# Define the function
A=abs(a*b*c)+10*d
Ans=I^A
\end{sagesilent}
 
\latexProblemContent{
\ifVerboseLocation This is Precalc Compute Question 0036. \\ \fi
\begin{problem}
Evaluate the power of $i$:

\input{Precalc-Compute-0036.HELP.tex}

\[i^{\sage{A}} = \answer{\sage{Ans}}\]

\end{problem}}%}
%%%%%%%%%%%%%%%%%%%%%%

%%%%%%%%%%%%%%%%%%%%%%%
%%\tagged{Ans@ShortAns, Type@Compute, Topic@Precalc, Sub@Power, Sub@Behavior, File@0037}{
\begin{sagesilent}
# Define variables and constants/exponents
a=NonZeroInt(-10,10)
p=NonZeroInt(-10,10)

# Define the function
f=a*x^p
A1=limit(f,x=infinity)
A2=limit(f,x=-infinity)
\end{sagesilent}
 
\latexProblemContent{
\ifVerboseLocation This is Precalc Compute Question 0037. \\ \fi
\begin{problem}
Determine the end behavior of the power function $f(x) = \sage{f}$.

\input{Precalc-Compute-0037.HELP.tex}

\[\mbox{as}\; x\to \infty, f(x)\to\answer{\sage{A1}}\]
\[\mbox{as}\; x\to -\infty, f(x)\to\answer{\sage{A2}}\]

\end{problem}}%}
%%%%%%%%%%%%%%%%%%%%%%

%%%%%%%%%%%%%%%%%%%%%%%
%%\tagged{Ans@ShortAns, Type@Compute, Topic@Precalc, Sub@Poly, Sub@Behavior, File@0038}{
\begin{sagesilent}
# Define variables and constants/exponents
a=NonZeroInt(-10,10)
b=RandInt(-10,10)
c=RandInt(-10,10)
d=RandInt(-10,10)
k=RandInt(-10,10)
p=RandInt(1,10)
q=NonZeroInt(1,10,[p])
r=NonZeroInt(1,10,[p,q])
s=NonZeroInt(1,10,[p,q,r])
pick=RandInt(0,3)

# Define the function
v=[a*x^p+k, a*x^p+b*x^q+k, a*x^p+b*x^q+c*x^r+k, a*x^p+b*x^q+c*x^r+d*x^s+k]
f=v[pick]

#Define Answers
DEG=f.degree(x)
COEFF=f.leading_coefficient(x)
A1=limit(f,x=infinity)
A2=limit(f,x=-infinity)
\end{sagesilent}
 
\latexProblemContent{
\ifVerboseLocation This is Precalc Compute Question 0038. \\ \fi
\begin{problem}
Determine the degree of the polynomial $f(x) = \sage{f}$.

\input{Precalc-Compute-0038.HELP.tex}

\begin{problem}

Determine the leading coefficient of the polynomial.

\begin{problem}

Determine the end behavior of the polynomial.

\[\mbox{as}\; x\to \infty, f(x)\to\answer{\sage{A1}}\]
\[\mbox{as}\; x\to -\infty, f(x)\to\answer{\sage{A2}}\]

\end{problem}

\[\mbox{Leading Coefficient}\;=\; \answer{\sage{COEFF}}\]

\end{problem}

\[\mbox{Degree}\;=\;\answer{\sage{DEG}}\]

\end{problem}}%}
%%%%%%%%%%%%%%%%%%%%%%

%%%%%%%%%%%%%%%%%%%%%%%
%%\tagged{Ans@ShortAns, Type@Compute, Topic@Precalc, Sub@Poly, Sub@Zeros, File@0039}{
\begin{sagesilent}
# Define variables and constants/exponents
a=NonZeroInt(-10,10)
b=RandInt(-10,10)
c=RandInt(-10,10)
p=RandInt(0,1)

# Define the function
v=[expand((x-a)*(x-b)*(x-c)), expand(a*x^2*(x-b)*(x-c))]
w=[(x-a)*(x-b)*(x-c), a*x^2*(x-b)*(x-c)]
f=v[p]
F=w[p]

#Define Answers
if p==0:
   m1=min(a,b)
   m2=min(m1,c)
   M1=max(a,b)
   M2=max(M1,c)
   mid=a+b+c
   Mid=mid-M2-m2
if p==1:
   m1=min(b,c)
   m2=min(0,m1)
   M1=max(b,c)
   M2=max(0,M1)
   mid=b+c
   Mid=mid-M2-m2
\end{sagesilent}
 
\latexProblemContent{
\ifVerboseLocation This is Precalc Compute Question 0039. \\ \fi
\begin{problem}
Factor the polynomial $f(x) = \sage{f}$.

\input{Precalc-Compute-0039.HELP.tex}

\begin{problem}

List the zeros of the polynomial.  

\[\mbox{Smallest zero}:\,\answer{\sage{m2}} \qquad\qquad \mbox{Middle Zero}:\, \answer{\sage{Mid}}\qquad\qquad \mbox{Largest zero}:\,\answer{\sage{M2}}\]

\end{problem}

\[\answer{\sage{F}}\]

\end{problem}}%}
%%%%%%%%%%%%%%%%%%%%%%

%%%%%%%%%%%%%%%%%%%%%%%
%%\tagged{Ans@ShortAns, Type@Compute, Topic@Precalc, Sub@Poly, Sub@Zeros, Sub@Multiplicity, File@0040}{
\begin{sagesilent}
# Define variables and constants/exponents
a=NonZeroInt(-10,10)
b=RandInt(-10,10)
c=RandInt(-10,10)
p=RandInt(1,3)
P=2*p
q=RandInt(1,3)
Q=2*q+1
pick=RandInt(0,2)

# Define the function
v=[(x-a)^P*(x-b)^Q*(x-c), (x-a)^Q*(x-b)^P*(x-c), (x-a)*(x-b)^P*(x-c)^Q]
f=v[pick]

#Define Answers
if pick==0:
   Aeven=a
   BODD=b
   Codd=c
if pick==1:
   Aeven=b
   BODD=a
   Codd=c
if pick==2:
   Aeven=b
   BODD=c
   Codd=a
\end{sagesilent}
 
\latexProblemContent{
\ifVerboseLocation This is Precalc Compute Question 0040. \\ \fi
\begin{problem}
Identify the multiplicity of the zeros of the polynomial $f(x) = \sage{f}$

\input{Precalc-Compute-0040.HELP.tex}

\begin{problem}

For the zero $x=\sage{Aeven}$, the graph ...

\begin{multipleChoice}
\choice{crosses like a line}
\choice[correct]{bounces like a parabola}
\choice{crosses like an S-shape}
\end{multipleChoice}

\end{problem}

\begin{problem}

For the zero $x=\sage{BODD}$, the graph ...

\begin{multipleChoice}
\choice{crosses like a line}
\choice{bounces like a parabola}
\choice[correct]{crosses like an S-shape}
\end{multipleChoice}

\end{problem}

\begin{problem}

For the zero $x=\sage{Codd}$, the graph ...

\begin{multipleChoice}
\choice[correct]{crosses like a line}
\choice{bounces like a parabola}
\choice{crosses like an S-shape}
\end{multipleChoice}

\end{problem}

\[\mbox{Mult. of }x=\sage{Aeven}:\,\answer{\sage{P}} \qquad\qquad \mbox{Mult. of }x=\sage{Codd}:\, \answer{1}\qquad\qquad \mbox{Mult. of }x=\sage{BODD}:\,\answer{\sage{Q}}\]

\end{problem}}%}
%%%%%%%%%%%%%%%%%%%%%%

%%%%%%%%%%%%%%%%%%%%%%%
%%\tagged{Ans@ShortAns, Type@Compute, Topic@Precalc, Sub@Poly, Sub@PolyDiv, File@0041}{
\begin{sagesilent}
# Define variables and constants/exponents
a=NonZeroInt(-3,3)
b=RandInt(-3,3)
c=NonZeroInt(-3,3)
d=NonZeroInt(-3,3)
e=RandInt(-3,3)
k1=RandInt(-10,10)
k2=RandInt(-10,10)

p=RandInt(1,2)
q=RandInt(0,p-1)
r1=RandInt(2,3)
r2=RandInt(1,2)
r3=RandInt(0,1)
pick=RandInt(0,1)
const=RandInt(-5,5)
# Define the function
v=[x-d, expand((x-d)*(x-e))]
g=v[pick]
Quot=c*x^r1+d*x^r2+e*x^r3+k2
if pick==0:
   Rem=const
if pick==1:
   Rem=x-const
f=(expand(g*Quot)+Rem).simplify()


\end{sagesilent}
 
\latexProblemContent{
\ifVerboseLocation This is Precalc Compute Question 0041. \\ \fi
\begin{problem}
What is the remainder when $\sage{f}$ is divided by $\sage{g}$?

\input{Precalc-Compute-0041.HELP.tex}

\[\mbox{Reminder:}\;\answer{\sage{Rem}}\]

\end{problem}}%}
%%%%%%%%%%%%%%%%%%%%%%

%%%%%%%%%%%%%%%%%%%%%%%
%%\tagged{Ans@ShortAns, Type@Compute, Topic@Precalc, Sub@Poly, Sub@PolyDiv, File@0042}{
\begin{sagesilent}
# Define variables and constants/exponents
a=NonZeroInt(-3,3)
b=RandInt(-3,3)
c=NonZeroInt(-3,3)
d=NonZeroInt(-3,3)
pick=RandInt(0,1)

# Define the function
v=[expand((x-a)*(x-b)*(x-c)), expand((x-a)*(x-b)*(x-c)*(x-d))]
w=[expand((x-b)*(x-c)), expand((x-b)*(x-c)*(x-d))]
f=v[pick]
g=w[pick]

\end{sagesilent}
 
\latexProblemContent{
\ifVerboseLocation This is Precalc Compute Question 0042. \\ \fi
\begin{problem}
Use Synthetic division to divide $\sage{f}$ given the $\sage{a}$ is a zero of the polynomial.

\input{Precalc-Compute-0042.HELP.tex}

\[\answer{\sage{g}}\]

\end{problem}}%}
%%%%%%%%%%%%%%%%%%%%%%

%%%%%%%%%%%%%%%%%%%%%%%
%%\tagged{Ans@ShortAns, Type@Compute, Topic@Precalc, Sub@Poly, Sub@PolyDiv, File@0043}{
\begin{sagesilent}
# Define variables and constants/exponents
a=NonZeroInt(-3,3)
b=RandInt(-3,3)
c=NonZeroInt(-3,3)
d=NonZeroInt(-3,3)
pick=RandInt(0,1)

# Define the function
v=[expand((x-a)*(x-b)*(x-c)), expand((x-a)*(x-b)*(x-c)*(x-d))]
w=[(x-a)*(x-b)*(x-c), (x-a)*(x-b)*(x-c)*(x-d)]
f=v[pick]
g=w[pick]

\end{sagesilent}
 
\latexProblemContent{
\ifVerboseLocation This is Precalc Compute Question 0043. \\ \fi
\begin{problem}
Write the polynomial $f(x) = \sage{f}$ in factored form.

\input{Precalc-Compute-0043.HELP.tex}

\[\answer{\sage{g}}\]

\end{problem}}%}
%%%%%%%%%%%%%%%%%%%%%%

%%%%%%%%%%%%%%%%%%%%%%%
%%\tagged{Ans@ShortAns, Type@Compute, Topic@Precalc, Sub@Poly, Sub@PolyDiv, File@0044}{
\begin{sagesilent}
# Define variables and constants/exponents
a=NonZeroInt(-3,3)
b=RandInt(-3,3)
c=NonZeroInt(-3,3)

# Define the function
Im=a+b*I
Im2=a-b*I
f=expand((x-Im)*(x-Im2)*(x-c))

\end{sagesilent}
 
\latexProblemContent{
\ifVerboseLocation This is Precalc Compute Question 0044. \\ \fi
\begin{problem}
Given $\sage{Im}$ is a complex root of the polynomial $f(x) = \sage{f}$, determine the other complex root.

\input{Precalc-Compute-0044.HELP.tex}

\begin{problem}

Determine the other real root using polynomial division.

\[\answer{\sage{c}}\]

\end{problem}

\[\answer{\sage{Im2}}\]

\end{problem}}%}
%%%%%%%%%%%%%%%%%%%%%%


%%%%%%%%%%%%%%%%%%%%%%%
%%\tagged{Ans@ShortAns, Type@Compute, Topic@Precalc, Sub@Rational, Sub@VerticalAsymptotes, Sub@HorizontalAsymptotes, Sub@Holes, File@0045}{
\begin{sagesilent}
# Define variables and constants/exponents
a=RandInt(-5,5)
b=NonZeroInt(-5,5, [a])
c=NonZeroInt(-5,5, [a,b])

# Define the function
f=(expand((x-a)*(x-b)))/(expand((x-c)*(x-b)))
hole=(b-a)/(b-c)

\end{sagesilent}
 
\latexProblemContent{
\ifVerboseLocation This is Precalc Compute Question 0045. \\ \fi
\begin{problem}

Consider the rational function $f(x) = \sage{f}$.  Identify any vertical asymptotes.

\input{Precalc-Compute-0045.HELP.tex}

\begin{feedback}[attempt]
Remember that vertical asymptotes come from the denominator being $0$.
\end{feedback}

%\begin{feedback}[VA=$\sage{b}$]
%That factor canceled out, which would lead to a hole.
%\end{feedback}

\begin{feedback}[correct]
Remember that vertical asymptotes are barriers that the graph cannot cross.
\end{feedback}

\begin{problem}

Are there any holes?  If so, give the coordinates of the hole.  If not, enter ``NONE''.

\begin{feedback}[attempt]
Remember that holes occur from cancellation of factors in the numerator and the denominator.
\end{feedback}

%\begin{feedback}[VA=$\sage{c}$]
%That factor did not cancel out, which would lead to a vertical asymptote.
%\end{feedback}

\begin{feedback}[correct]
Remember that holes are removable discontinuities in the graph.
\end{feedback}

\begin{problem}

Identify any horizontal asymptotes.  If none, enter ``NONE''.

\[y=\answer[format=integer,id=HA]{1}\]

\begin{feedback}[attempt]
Remember that horizontal asymptotes you need to compare the degree of the numerator and the denominator.
\end{feedback}

\begin{feedback}[HA=0]
This would happen if the degree of the denominator was higher.
\end{feedback}

\begin{feedback}[HA>1]
This would happen if the degree of the numerator was higher.
\end{feedback}

\begin{feedback}[correct]
The horizontal asymptote is describing the long-run behavior of the function.
\end{feedback}


\end{problem}

\[(\answer[format=integer]{\sage{b}}, \answer{\sage{hole}})\]

\end{problem}

\[x=\answer[format=integer,id=VA]{\sage{c}}\]

\end{problem}}%}
%%%%%%%%%%%%%%%%%%%%%%

%%%%%%%%%%%%%%%%%%%%%%%
%%\tagged{Ans@ShortAns, Type@Compute, Topic@Precalc, Sub@Rational, Sub@VerticalAsymptotes, Sub@HorizontalAsymptotes, File@0046}{
\begin{sagesilent}
# Define variables and constants/exponents
a=RandInt(-5,5)
b=NonZeroInt(-5,5, [a])
c=NonZeroInt(-5,5, [a,b])
d=NonZeroInt(-5,5, [a,b,c])

# Define the function
f=(expand((x-a)*(x-b)))/(expand((x-c)*(x-d)))

\end{sagesilent}
 
\latexProblemContent{
\ifVerboseLocation This is Precalc Compute Question 0046. \\ \fi
\begin{problem}

Consider the rational function $f(x) = \sage{f}$.  Identify any vertical asymptotes.

\input{Precalc-Compute-0046.HELP.tex}

\begin{feedback}[attempt]
Remember that vertical asymptotes come from the denominator being $0$.
\end{feedback}

%\begin{feedback}[VA=$\sage{b}$]
%That factor canceled out, which would lead to a hole.
%\end{feedback}

\begin{feedback}[correct]
Remember that vertical asymptotes are barriers that the graph cannot cross.
\end{feedback}

\begin{problem}

Are there any holes?  If so, give the coordinates of the hole.  If not, enter ``NONE''.

\begin{feedback}[attempt]
Remember that holes occur from cancellation of factors in the numerator and the denominator.
\end{feedback}

%\begin{feedback}[VA=$\sage{c}$]
%That factor did not cancel out, which would lead to a vertical asymptote.
%\end{feedback}

\begin{feedback}[correct]
Remember that holes are removable discontinuities in the graph.
\end{feedback}

\begin{problem}

Identify any horizontal asymptotes.  If none, enter ``NONE''.

\[y=\answer[format=integer,id=HA]{1}\]

\begin{feedback}[attempt]
Remember that horizontal asymptotes you need to compare the degree of the numerator and the denominator.
\end{feedback}

\begin{feedback}[HA=0]
This would happen if the degree of the denominator was higher.
\end{feedback}

\begin{feedback}[HA>1]
This would happen if the degree of the numerator was higher.
\end{feedback}

\begin{feedback}[correct]
The horizontal asymptote is describing the long-run behavior of the function.
\end{feedback}


\end{problem}

\[(\answer[format=string]{NONE}, \answer[format=string]{NONE})\]

\end{problem}

\[x=\answer[format=integer]{\sage{c}}\qquad\qquad x=\answer[format=integer]{\sage{d}}\]

\end{problem}}%}
%%%%%%%%%%%%%%%%%%%%%%

%%%%%%%%%%%%%%%%%%%%%%%
%%\tagged{Ans@ShortAns, Type@Compute, Topic@Precalc, Sub@Rational, Sub@VerticalAsymptotes, Sub@HorizontalAsymptotes, Sub@Holes, File@0047}{
\begin{sagesilent}
# Define variables and constants/exponents
a=RandInt(-5,5)
b=NonZeroInt(-5,5, [a])
c=NonZeroInt(-5,5, [a,b])

# Define the function
f=(expand((x-a)*(x-b)^2))/(expand((x-c)*(x-b)))

\end{sagesilent}
 
\latexProblemContent{
\ifVerboseLocation This is Precalc Compute Question 0047. \\ \fi
\begin{problem}

Consider the rational function $f(x) = \sage{f}$.  Identify any vertical asymptotes.

\input{Precalc-Compute-0047.HELP.tex}

\begin{feedback}[attempt]
Remember that vertical asymptotes come from the denominator being $0$.
\end{feedback}

%\begin{feedback}[VA=$\sage{b}$]
%That factor canceled out, which would lead to a hole.
%\end{feedback}

\begin{feedback}[correct]
Remember that vertical asymptotes are barriers that the graph cannot cross.
\end{feedback}

\begin{problem}

Are there any holes?  If so, give the coordinates of the hole.  If not, enter ``NONE''.

\begin{feedback}[attempt]
Remember that holes occur from cancellation of factors in the numerator and the denominator.
\end{feedback}

%\begin{feedback}[VA=$\sage{c}$]
%That factor did not cancel out, which would lead to a vertical asymptote.
%\end{feedback}

\begin{feedback}[correct]
Remember that holes are removable discontinuities in the graph.
\end{feedback}

\begin{problem}

Identify any horizontal asymptotes.  If none, enter ``NONE''.

\[y=\answer[format=string,id=HA]{NONE}\]

\begin{feedback}[attempt]
Remember that horizontal asymptotes you need to compare the degree of the numerator and the denominator.
\end{feedback}

\begin{feedback}[HA=0]
This would happen if the degree of the denominator was higher.
\end{feedback}

\begin{feedback}[HA=1]
This would happen if the degree of the numerator equaled the degree of the denominator.
\end{feedback}

\begin{feedback}[correct]
The horizontal asymptote is describing the long-run behavior of the function.
\end{feedback}


\end{problem}

\[(\answer[format=integer]{\sage{b}}, \answer{0})\]

\end{problem}

\[x=\answer[format=integer,id=VA]{\sage{c}}\]

\end{problem}}%}
%%%%%%%%%%%%%%%%%%%%%%

%%%%%%%%%%%%%%%%%%%%%%%
%%\tagged{Ans@ShortAns, Type@Compute, Topic@Precalc, Sub@Rational, Sub@VerticalAsymptotes, Sub@HorizontalAsymptotes, File@0048}{
\begin{sagesilent}
# Define variables and constants/exponents
a=RandInt(-5,5)
b=NonZeroInt(-5,5, [a])
c=NonZeroInt(-5,5, [a,b])

# Define the function
f=(expand((x-a)*(x-b)))/(expand((x-c)*(x-b)^2))

\end{sagesilent}
 
\latexProblemContent{
\ifVerboseLocation This is Precalc Compute Question 0048. \\ \fi
\begin{problem}

Consider the rational function $f(x) = \sage{f}$.  Identify any vertical asymptotes.

\input{Precalc-Compute-0048.HELP.tex}

\begin{feedback}[attempt]
Remember that vertical asymptotes come from the denominator being $0$.
\end{feedback}

%\begin{feedback}[VA=$\sage{b}$]
%That factor canceled out, which would lead to a hole.
%\end{feedback}

\begin{feedback}[correct]
Remember that vertical asymptotes are barriers that the graph cannot cross.
\end{feedback}

\begin{problem}

Are there any holes?  If so, give the coordinates of the hole.  If not, enter ``NONE''.

\begin{feedback}[attempt]
Remember that holes occur from cancellation of factors in the numerator and the denominator.
\end{feedback}

%\begin{feedback}[VA=$\sage{c}$]
%That factor did not cancel out, which would lead to a vertical asymptote.
%\end{feedback}

\begin{feedback}[correct]
Remember that holes are removable discontinuities in the graph.
\end{feedback}

\begin{problem}

Identify any horizontal asymptotes.  If none, enter ``NONE''.

\[y=\answer[format=integer,id=HA]{0}\]

\begin{feedback}[attempt]
Remember that horizontal asymptotes you need to compare the degree of the numerator and the denominator.
\end{feedback}

\begin{feedback}[HA>1]
This would happen if the degree of the numerator was higher.
\end{feedback}

\begin{feedback}[HA=1]
This would happen if the degree of the numerator equaled the degree of the denominator.
\end{feedback}

\begin{feedback}[correct]
The horizontal asymptote is describing the long-run behavior of the function.
\end{feedback}


\end{problem}

\[(\answer[format=string]{NONE}, \answer[format=string]{NONE})\]

\end{problem}

%  \begin{validator}[a^2+b^2==$\sage{square}$]
\[x=\answer[format=integer,id=a]{\sage{c}}\qquad\qquad x=\answer[format=integer,id=b]{\sage{b}}\]
%  \end{validator}


\end{problem}}%}
%%%%%%%%%%%%%%%%%%%%%%

%%%%%%%%%%%%%%%%%%%%%%%
%%\tagged{Ans@ShortAns, Type@Compute, Topic@Precalc, Sub@Rational, Sub@Intercept, File@0049}{
\begin{sagesilent}
# Define variables and constants/exponents
a=RandInt(-5,5)
b=NonZeroInt(-5,5, [a])
c=NonZeroInt(-5,5, [0,a,b])

# Define the function
f=(expand((x-a)*(x-b)))/(expand((x-c)*(x-b)))
yint=(a)/(c)
\end{sagesilent}
 
\latexProblemContent{
\ifVerboseLocation This is Precalc Compute Question 0049. \\ \fi
\begin{problem}

Consider the rational function $f(x) = \sage{f}$.  What is the $y-$intercept?

\input{Precalc-Compute-0049.HELP.tex}

\begin{feedback}[attempt]
To find the $y-$intercept, set $x=0$.
\end{feedback}

\begin{feedback}[correct]
The $y-$intercept is where the graph of the function crosses the $y-$axis.
\end{feedback}

\begin{problem}

What are the $x-$intercept(s)?

\begin{feedback}[attempt]
For rational functions, you need only set the numerator equal to $0$ and solve for $x$.  And don't forget to make sure the values are in the domain!
\end{feedback}

\begin{feedback}[correct]
An $x-$intercept is where the graph of the function crosses the $x-$axis.
\end{feedback}

\[x=\answer{\sage{a}}\]

\end{problem}

\[y=\answer{\sage{yint}}\]

\end{problem}}%}
%%%%%%%%%%%%%%%%%%%%%%

%%%%%%%%%%%%%%%%%%%%%%%
%%\tagged{Ans@ShortAns, Type@Compute, Topic@Precalc, Sub@Quadratic, Sub@Inverse, Sub@DomainRestricted, File@0050}{
\begin{sagesilent}
# Define variables and constants/exponents
a=NonZeroInt(-5,5)
b=NonZeroInt(-5,5)
c=NonZeroInt(-5,5, [b])

# Define the function
f=expand(a*(x-b)^2+c)
finv=((x-c)/a)^(1/2)+b
\end{sagesilent}
 
\latexProblemContent{
\ifVerboseLocation This is Precalc Compute Question 0050. \\ \fi
\begin{problem}

Given $f(x) = \sage{f}$, which of the following is a possible restricted domain to make $f(x)$ one-to-one?

\input{Precalc-Compute-0050.HELP.tex}

\begin{feedback}[attempt]
Completing the square makes it clear!
\end{feedback}

\begin{feedback}[correct]
Identifying the axis of symmetry is the goal.
\end{feedback}

\begin{problem}

What is the inverse of $f(x)$ under this restricted domain?

\begin{feedback}
Don't forget about the restricted domain!
\end{feedback}

\[f^{-1}(x) = \answer{\sage{finv}}\]

\end{problem}

\begin{multipleChoice}
\choice{$[0,\infty)$}
\choice{$(-\infty,\sage{b^2}]$}
\choice{$[\sage{b^2},\infty)$}
\choice{$(-\infty,\sage{c}]$}
\choice[correct]{$(-\infty,\sage{b}]$}
\end{multipleChoice}

\end{problem}}%}
%%%%%%%%%%%%%%%%%%%%%%

%%%%%%%%%%%%%%%%%%%%%%%
%%\tagged{Ans@ShortAns, Type@Compute, Topic@Precalc, Sub@Quadratic, Sub@Inverse, Sub@DomainRestricted, File@0051}{
\begin{sagesilent}
# Define variables and constants/exponents
a=NonZeroInt(-5,5)
b=NonZeroInt(-5,5)
c=NonZeroInt(-5,5, [b])

# Define the function
f=expand(a*(x-b)^2+c)
finv=-((x-c)/a)^(1/2)+b
\end{sagesilent}
 
\latexProblemContent{
\ifVerboseLocation This is Precalc Compute Question 0051. \\ \fi
\begin{problem}

Given $f(x) = \sage{f}$, which of the following is a possible restricted domain to make $f(x)$ one-to-one?

\input{Precalc-Compute-0051.HELP.tex}

\begin{feedback}[attempt]
Completing the square makes it clear!
\end{feedback}

\begin{feedback}[correct]
Identifying the axis of symmetry is the goal.
\end{feedback}

\begin{problem}

What is the inverse of $f(x)$ under this restricted domain?

\begin{feedback}
Don't forget about the restricted domain!
\end{feedback}

\[f^{-1}(x) = \answer{\sage{finv}}\]

\end{problem}

\begin{multipleChoice}
\choice{$[0,\infty)$}
\choice{$[\sage{b^2},\infty)$}
\choice[correct]{$[\sage{b},\infty)$}
\choice{$(-\infty,\sage{c}]$}
\choice{$(-\infty,\sage{b^2}]$}
\end{multipleChoice}

\end{problem}}%}
%%%%%%%%%%%%%%%%%%%%%%

%%%%%%%%%%%%%%%%%%%%%%%
%%\tagged{Ans@ShortAns, Type@Compute, Topic@Precalc, Sub@Exp, File@0052}{
\begin{sagesilent}
# Define variables and constants/exponents
a=NonZeroInt(1,5)
b=RandInt(2,5)
c=RandInt(-3,3)
d=NonZeroInt(-3,3, [c])

# Define the function
f=a*b^x
f1=f(c)
f2=f(d)
\end{sagesilent}
 
\latexProblemContent{
\ifVerboseLocation This is Precalc Compute Question 0052. \\ \fi
\begin{problem}

Given $\left(\sage{c},\sage{f1}\right)$ and $\left(\sage{d},\sage{f2}\right)$ are points on an exponential function, find the equation of the function.

\input{Precalc-Compute-0052.HELP.tex}

\begin{feedback}[attempt]
To find the equation, substitute both points and then find the growth factor and the $y-$intercept.
\end{feedback}

\begin{problem}
Is this a growth or decay function?  Enter in `Growth' or `Decay'

\begin{feedback}[attempt]
Remember to check if the growth factor is $>$ or $<$ $1$.
\end{feedback}

\[\answer[format=string,id=Ans]{Growth}\]

\end{problem}

\[f(x) = \answer{\sage{f}}\]
\end{problem}}%}
%%%%%%%%%%%%%%%%%%%%%%

%%%%%%%%%%%%%%%%%%%%%%%
%%\tagged{Ans@ShortAns, Type@Compute, Topic@Precalc, Sub@Exp, File@0053}{
\begin{sagesilent}
# Define variables and constants/exponents
a=NonZeroInt(1,5)
b=RandInt(2,5)
c=RandInt(-3,3)
d=NonZeroInt(-3,3, [c])

# Define the function
f=a*(1/b)^x
f1=f(c)
f2=f(d)
\end{sagesilent}
 
\latexProblemContent{
\ifVerboseLocation This is Precalc Compute Question 0053. \\ \fi
\begin{problem}

Given $\left(\sage{c},\sage{f1}\right)$ and $\left(\sage{d},\sage{f2}\right)$ are points on an exponential function, find the equation of the function.

\input{Precalc-Compute-0053.HELP.tex}

\begin{feedback}[attempt]
To find the equation, substitute both points and then find the growth factor and the $y-$intercept.
\end{feedback}

\begin{problem}
Is this a growth or decay function?  Enter in `Growth' or `Decay'

\begin{feedback}[attempt]
Remember to check if the growth factor is $>$ or $<$ $1$.
\end{feedback}

\[\answer[format=string,id=Ans]{Decay}\]

\end{problem}

\[f(x) = \answer{\sage{f}}\]
\end{problem}}%}
%%%%%%%%%%%%%%%%%%%%%%

%%%%%%%%%%%%%%%%%%%%%%%
%%\tagged{Ans@ShortAns, Type@Compute, Topic@Precalc, Sub@Exp, Sub@Interest, File@0054}{
\begin{sagesilent}
# Define variables and constants/exponents
var('t')
a=RandInt(1,5)
b=RandInt(2,5)
r=RandInt(4,6)
pick=RandInt(0,3)

# Define the function
P=b*1000
rate=r*0.01
v=['semiannually', 'quarterly', 'monthly', 'yearly']
w=[2,4,12,1]
C=w[pick]
comp=v[pick]
Amt=P*(1+(rate)/C)^(C*t)
A1=Amt(a)
\end{sagesilent}
 
\latexProblemContent{
\ifVerboseLocation This is Precalc Compute Question 0054. \\ \fi
\begin{problem}

Suppose we invest $\sage{P}$ into an account paying $\sage{r}$ percent interest, compounded $\sage{comp}$.  Write the equation for the amount of money in the account after $t$ years.

\input{Precalc-Compute-0054.HELP.tex}

\begin{problem}
What will be the value of the account in $\sage{a}$ years?

\[A(\sage{a}) = \answer{\sage{A1}}\]
\end{problem}

\[A(t) = \answer{\sage{Amt}}\]
\end{problem}}%}
%%%%%%%%%%%%%%%%%%%%%%

%%%%%%%%%%%%%%%%%%%%%%%
%%\tagged{Ans@ShortAns, Type@Compute, Topic@Precalc, Sub@Exp, Sub@Interest, File@0055}{
\begin{sagesilent}
# Define variables and constants/exponents
var('t')
a=RandInt(1,5)
b=RandInt(2,5)
r=NonZeroInt(4,6)

# Define the function
P=b*1000
rate=r*0.01
Amt=P*exp((rate)*t)
A1=Amt(a)
\end{sagesilent}
 
\latexProblemContent{
\ifVerboseLocation This is Precalc Compute Question 0055. \\ \fi
\begin{problem}

Suppose we invest $\sage{P}$ into an account paying $\sage{r}$ percent interest, compounded continuously.  Write the equation for the amount of money in the account after $t$ years.

\input{Precalc-Compute-0055.HELP.tex}

\begin{problem}
What will be the value of the account in $\sage{a}$ years?

\[A(\sage{a}) = \answer{\sage{A1}}\]
\end{problem}

\[A(t) = \answer{\sage{Amt}}\]
\end{problem}}%}
%%%%%%%%%%%%%%%%%%%%%%

%%%%%%%%%%%%%%%%%%%%%%%
%%\tagged{Ans@ShortAns, Type@Compute, Topic@Precalc, Sub@Exp, Sub@Application, File@0056}{
\begin{sagesilent}
# Define variables and constants/exponents
var('t')
a=NonZeroInt(1,8)
b=RandInt(2,8)
r=NonZeroInt(3,20)

# Define the function
P=b*100
rate=r*0.01
Amt=P*exp(-1*(rate)*t)
A1=Amt(a)
\end{sagesilent}
 
\latexProblemContent{
\ifVerboseLocation This is Precalc Compute Question 0056. \\ \fi
\begin{problem}

A certain radioactive substance decays at a rate of $\sage{r}$ percent per hour.  Write an equation to model the decay of $\sage{P}$ mg of the substance after $t$ hours.

\input{Precalc-Compute-0056.HELP.tex}

\begin{problem}
How much of $\sage{P}$ mg of the substance will remain after $\sage{a}$ hours?

\[A(\sage{a}) = \answer{\sage{A1}}\]
\end{problem}

\[A(t) = \answer{\sage{Amt}}\]
\end{problem}}%}
%%%%%%%%%%%%%%%%%%%%%%




































