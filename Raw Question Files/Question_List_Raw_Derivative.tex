%%%%%%  This is the raw list of questions before processing %%%%%%


%%%%%%%%%%%%%%%%%%%%%%%%%%
%%%%%%%%%%%%%%%%%%%%%%%%%%


%%%%%%%%%%%%%%%%%%%%%%%
%%\tagged{Cat@One, Cat@Two, Cat@Three, Cat@Four, Cat@Five, Ans@ShortAns, Type@Compute, Topic@Derivative, Sub@Poly, Sub@Power_Rule}{
\begin{sagesilent}
a = NonZeroInt(-5,5)
b = NonZeroInt(-5,5, [0,a])
c = NonZeroInt(-5,5)
   
vright=[(x-a),expand((x-a)*(x-b)),expand((x-a)*(x-b)*(x-c)),1]

first = Integer(randint(0,3))
second = Integer(randint(0,2))

Fone=vright[first]
Ftwo=vright[second]
F=expand(Fone*Ftwo)

Ans=derivative(F,x)

\end{sagesilent}

\latexProblemContent{
\begin{problem}

Compute the following derivative:

\input{2311_Compute_Derivative_0001.HELP.tex}

\[\dfrac{d}{dx}\left(\sage{F}\right)=\answer{\sage{Ans}}\]
\end{problem}}%}
%%%%%%%%%%%%%%%%%%%%%%


%%%%%%%%%%%%%%%%%%%%%%%
%%\tagged{Cat@One, Cat@Two, Cat@Three, Cat@Four, Cat@Five, Ans@ShortAns, Type@Compute, Topic@Derivative, Sub@Trig}{
\begin{sagesilent}
w0=SR.wild(0)
def TrigSimp(f):
   ftemp = f
   ftempsec = ftemp.substitute(tan(w0)^2+1 == sec(w0)^2)
   ftempcsc = ftempsec.substitute(cot(w0)^2+1 == csc(w0)^2)
   ffinal = ftempcsc
   return simplify(ffinal)

a = NonZeroInt(-5,5)
b = NonZeroInt(-5,5, [0,a])
   
vleft=[a*sin(b*x*pi), a*cos(b*x*pi), a*tan(b*x*pi), a*sin(b*x*pi/2), a*cos(b*x*pi/2), a*tan(b*x*pi/2), a*sin(b*x*pi/6), a*cos(b*x*pi/6), a*tan(b*x*pi/6), a*sin(b*x*pi/3), a*cos(b*x*pi/3), a*tan(b*x*pi/3)]

first = Integer(randint(0,11))

Fone=vleft[first]

Ans=TrigSimp(derivative(Fone,x))
\end{sagesilent}

\latexProblemContent{
\begin{problem}

Compute the following derivative:

\input{2311_Compute_Derivative_0002.HELP.tex}

\[\dfrac{d}{dx}\left(\sage{Fone}\right)=\answer{\sage{Ans}}\]
\end{problem}}%}
%%%%%%%%%%%%%%%%%%%%%%

%%%%%%%%%%%%%%%%%%%%%%%
%%\tagged{Cat@One, Cat@Two, Cat@Three, Cat@Four, Cat@Five, Ans@ShortAns, Type@Compute, Topic@Derivative, Sub@Trig, Sub@Product_Rule}{
\begin{sagesilent}
w0=SR.wild(0)
def TrigSimp(f):
   ftemp = f
   ftempsec = ftemp.substitute(tan(w0)^2+1 == sec(w0)^2)
   ftempcsc = ftempsec.substitute(cot(w0)^2+1 == csc(w0)^2)
   ffinal = ftempcsc
   return simplify(ffinal)

a = NonZeroInt(-5,5)
b = NonZeroInt(-5,5, [0,a])
c = NonZeroInt(-5,5)   
d=NonZeroInt(-5,5, [0,c])
   
vleft=[a*sin(b*x*pi), a*cos(b*x*pi), a*tan(b*x*pi), a*sin(b*x*pi/2), a*cos(b*x*pi/2), a*tan(b*x*pi/2), a*sin(b*x*pi/6), a*cos(b*x*pi/6), a*tan(b*x*pi/6), a*sin(b*x*pi/3), a*cos(b*x*pi/3), a*tan(b*x*pi/3)]

vright=[c*sin(d*x*pi), c*cos(d*x*pi), c*tan(d*x*pi), c*sin(d*x*pi/2), c*cos(d*x*pi/2), c*tan(d*x*pi/2), c*sin(d*x*pi/6), c*cos(d*x*pi/6), c*tan(d*x*pi/6), c*sin(d*x*pi/3), c*cos(d*x*pi/3), c*tan(d*x*pi/3)]

first = Integer(randint(0,11))
second = Integer(randint(0,11))

Fone=vleft[first]
Ftwo=vright[second]
F=expand(Fone*Ftwo)

Ans=TrigSimp(derivative(F,x))

\end{sagesilent}

\latexProblemContent{
\begin{problem}

Compute the following derivative:

\input{2311_Compute_Derivative_0003.HELP.tex}

\[\dfrac{d}{dx}\left(\sage{F}\right)=\answer{\sage{Ans}}\]

\end{problem}}%}
%%%%%%%%%%%%%%%%%%%%%%


%%%%%%%%%%%%%%%%%%%%%%%
%%\tagged{Cat@One, Cat@Two, Cat@Three, Cat@Four, Cat@Five, Ans@ShortAns, Type@Compute, Topic@Derivative, Sub@Rational, Sub@Quotient_Rule}{
\begin{sagesilent}

a = NonZeroInt(-5,5)
b = NonZeroInt(-5,5, [0,a])
c = NonZeroInt(-5,5, [0,a])
d=NonZeroInt(-5,5, [0,a])
e=NonZeroInt(-5,5, [0,a])
f=NonZeroInt(-5,5)

vleft=[(x-a),expand((x-a)*(x-b)),expand((x-a)*(x-b)*(x-c)),1]
vright=[(x-d),expand((x-d)*(x-e)),expand((x-d)*(x-e)*(x-f))]

first = Integer(randint(0,3))
second = Integer(randint(0,2))

Fone=vleft[first]
Ftwo=vright[second]
F=Fone/Ftwo

Fnum=(derivative(F,x)).numerator()
Fden=(derivative(F,x)).denominator()
Ans=Fnum/Fden
\end{sagesilent}

\latexProblemContent{
\begin{problem}

Compute the following derivative:

\input{2311_Compute_Derivative_0004.HELP.tex}

\[\dfrac{d}{dx}\left(\sage{F}\right)=\answer{\sage{Ans}}\]
\end{problem}}%}
%%%%%%%%%%%%%%%%%%%%%%


%%%%%%%%%%%%%%%%%%%%%%%
%%\tagged{Cat@One, Cat@Two, Cat@Three, Cat@Four, Cat@Five, Ans@MultiAns, Type@Compute, Topic@Derivative, Sub@FormalDef, Sub@DifferenceQuotient, Sub@Poly}{
\begin{sagesilent}
var('h')
a = NonZeroInt(-5,5)
b = NonZeroInt(-4,4)
p=Integer(randint(0,3))
if p==0:
   b = NonZeroInt(1,4)

v=[a*sqrt(x),a/x, a*x^2, a*x^3]

F=v[p]
Fone=F(b+h)
Ftwo=F(b)
DQ=(F(b+h)-F(b))/h

Ans=limit(DQ,h=0)

\end{sagesilent}

\latexProblemContent{
\begin{problem}

Compute the following limit definition of derivative.
\[\lim_{h\to0}\frac{\sage{Fone-Ftwo}}{h}=\answer{\sage{Ans}}\]
What function is being differentiated?
\[f(x)=\answer{\sage{F}}\]
At what $x-$value are you computing the derivative?
\[x=\answer{\sage{b}}\]

\input{2311_Compute_Derivative_0005.HELP.tex}


\end{problem}}%}
%%%%%%%%%%%%%%%%%%%%%%


%%%%%%%%%%%%%%%%%%%%%%%
%%\tagged{Cat@One, Cat@Two, Cat@Three, Cat@Four, Cat@Five, Ans@ShortAns, Type@Compute, Topic@Derivative, Sub@Log}{
\begin{sagesilent}
a = NonZeroInt(-5,5)
g=a*log(x)
Ans = derivative(g,x)

\end{sagesilent}

\latexProblemContent{
\begin{problem}

Compute the following derivative:

\input{2311_Compute_Derivative_0006.HELP.tex}

\[\dfrac{d}{dx}\left(\sage{g}\right)=\answer{\sage{Ans}}\] 
\end{problem}}%}
%%%%%%%%%%%%%%%%%%%%%%


%%%%%%%%%%%%%%%%%%%%%%%
%%\tagged{Cat@One, Cat@Two, Cat@Three, Cat@Four, Cat@Five, Ans@ShortAns, Type@Compute, Topic@Derivative, Sub@Log, Sub@Chain_Rule}{
\begin{sagesilent}
a = NonZeroInt(-5,5)
b = NonZeroInt(-5,5)
c = NonZeroInt(-5,5)
f=a*log(b*x+c)
Ans = derivative(f,x)
\end{sagesilent}

\latexProblemContent{
\begin{problem}

Compute the following derivative:

\input{2311_Compute_Derivative_0007.HELP.tex}

\[\dfrac{d}{dx}\left(\sage{f}\right)=\answer{\sage{Ans}}\] 
\end{problem}}%}
%%%%%%%%%%%%%%%%%%%%%%

%%%%%%%%%%%%%%%%%%%%%%%
%%\tagged{Cat@One, Cat@Two, Cat@Three, Cat@Four, Cat@Five, Ans@ShortAns, Type@Compute, Topic@Derivative, Sub@Exp}{
\begin{sagesilent}
a = NonZeroInt(-5,5)
b = Integer(randint(-5,5))
g=a*exp(b+x)
Ans = derivative(g,x)
\end{sagesilent}

\latexProblemContent{
\begin{problem}

Compute the following derivative:

\input{2311_Compute_Derivative_0008.HELP.tex}

\[\dfrac{d}{dx}\left(\sage{g}\right)=\answer{\sage{Ans}}\] 
\end{problem}}%}
%%%%%%%%%%%%%%%%%%%%%%


%%%%%%%%%%%%%%%%%%%%%%%
%%\tagged{Cat@One, Cat@Two, Cat@Three, Cat@Four, Cat@Five, Ans@ShortAns, Type@Compute, Topic@Derivative, Sub@Exp, Sub@Chain_Rule}{
\begin{sagesilent}
a = NonZeroInt(-5,5)
b = Integer(randint(-5,5))
c = NonZeroInt(-5,5)
f=a*exp(c*x+b)
Ans = derivative(f,x)
\end{sagesilent}

\latexProblemContent{
\begin{problem}

Compute the following derivative:

\input{2311_Compute_Derivative_0009.HELP.tex}

\[\dfrac{d}{dx}\left(\sage{f}\right)=\answer{\sage{Ans}}\] 
\end{problem}}%}
%%%%%%%%%%%%%%%%%%%%%%


%%%%%%%%%%%%%%%%%%%%%%%
%%\tagged{Cat@One, Cat@Two, Cat@Three, Cat@Four, Cat@Five, Ans@ShortAns, Type@Compute, Topic@Derivative, Sub@Product_Rule}{
\begin{sagesilent}
a = NonZeroInt(-5,5)
b = NonZeroInt(-5,5)

p=Integer(randint(0,8))
q=Integer(randint(0,8))
while p==q:
   q=Integer(randint(0,8))

v=[sqrt(x), log(x), exp(x), x^2, x^3, x^4, (x-b), sin(x), cos(x)]

Fone=v[p]
Ftwo=v[q]
F=a*Fone*Ftwo

Ans = derivative(F,x)
\end{sagesilent}

\latexProblemContent{
\begin{problem}

Compute the following derivative:

\input{2311_Compute_Derivative_0010.HELP.tex}

\[\dfrac{d}{dx}\left(\sage{F}\right)=\answer{\sage{Ans}}\] 
\end{problem}}%}
%%%%%%%%%%%%%%%%%%%%%%

%%%%%%%%%%%%%%%%%%%%%%%
%%\tagged{Cat@One, Cat@Two, Cat@Three, Cat@Four, Cat@Five, Ans@ShortAns, Type@Compute, Topic@Derivative, Sub@Quotient_Rule}{
\begin{sagesilent}
a = NonZeroInt(-5,5)
b = NonZeroInt(-5,5)

p=Integer(randint(0,8))
q=Integer(randint(0,8))
while p==q:
   q=Integer(randint(0,8))

v=[sqrt(x), log(x), exp(x), x^2, x^3, x^4, (x-b), sin(x), cos(x)]

Fone=v[p]
Ftwo=v[q]
F=a*Fone/Ftwo

Ans = derivative(F,x)
\end{sagesilent}

\latexProblemContent{
\begin{problem}

Compute the following derivative:

\input{2311_Compute_Derivative_0011.HELP.tex}

\[\dfrac{d}{dx}\left(\sage{F}\right)=\answer{\sage{Ans}}\] 
\end{problem}}%}
%%%%%%%%%%%%%%%%%%%%%%


%%%%%%%%%%%%%%%%%%%%%%%
%%\tagged{Cat@One, Cat@Two, Cat@Three, Cat@Four, Cat@Five, Ans@ShortAns, Type@Compute, Topic@Derivative, Sub@Chain_Rule}{
\begin{sagesilent}
a = NonZeroInt(-5,5)
b = NonZeroInt(-5,5)

p=Integer(randint(0,8))
q=Integer(randint(0,8))
while p==q:
   q=Integer(randint(0,8))

v=[sqrt(x), log(x), exp(x), x^2, x^3, x^4, (x-b), sin(x), cos(x)]

Fone=v[p]
Ftwo=v[q]
F=a*Fone(Ftwo)

Ans = derivative(F,x)
\end{sagesilent}

\latexProblemContent{
\begin{problem}

Compute the following derivative:

\input{2311_Compute_Derivative_0012.HELP.tex}

\[\dfrac{d}{dx}\left(\sage{F}\right)=\answer{\sage{Ans}}\] 
\end{problem}}%}
%%%%%%%%%%%%%%%%%%%%%%


%%%%%%%%%%%%%%%%%%%%%%%
%%\tagged{Cat@One, Cat@Two, Cat@Three, Cat@Four, Cat@Five, Ans@ShortAns, Type@Compute, Topic@Derivative, Sub@Tan_Line, Sub@Poly}{
\begin{sagesilent}
a = NonZeroInt(-3,3)
b = NonZeroInt(-3,3)
p=Integer(randint(0,1))
v=[expand((x-b)^2), expand((x-b)^3)]
F=v[p]

if p==0:
   Ans=2*(a-b)*x-2*a*(a-b)+(a-b)^2
else:
   Ans = 3*(a-b)^2*x-3*a*(a-b)^2+(a-b)^3
\end{sagesilent}

\latexProblemContent{
\begin{problem}

Find the equation of the line tangent to $f(x)=\sage{F}$ at $x=\sage{a}$.

\input{2311_Compute_Derivative_0013.HELP.tex}

\[y=\answer{\sage{Ans}}\] 
\end{problem}}%}
%%%%%%%%%%%%%%%%%%%%%%


%%%%%%%%%%%%%%%%%%%%%%%
%%\tagged{Cat@One, Cat@Two, Cat@Three, Cat@Four, Cat@Five, Ans@ShortAns, Type@Compute, Topic@Derivative, Sub@Tan_Line, Sub@Trig}{
\begin{sagesilent}
a = RandAng(0,pi)
b = NonZeroInt(-3,3)
p=Integer(randint(0,1))
v=[b*sin(x), b*cos(x)]
F=v[p]

if p==0:
   Ans=b*cos(a)*x-a*b*cos(a)+b*sin(a)
else:
   Ans = -b*sin(a)*x+a*b*sin(a)+b*cos(a)
\end{sagesilent}

\latexProblemContent{
\begin{problem}

Find the equation of the line tangent to $f(x)=\sage{F}$ at $x=\sage{a}$.

\input{2311_Compute_Derivative_0014.HELP.tex}

\[y=\answer{\sage{Ans}}\] 
\end{problem}}%}
%%%%%%%%%%%%%%%%%%%%%%

%%%%%%%%%%%%%%%%%%%%%%%
%%\tagged{Cat@One, Cat@Two, Cat@Three, Cat@Four, Cat@Five, Ans@ShortAns, Type@Compute, Topic@Derivative, Sub@Tan_Line, Sub@Exp, Sub@Log}{
\begin{sagesilent}
a = NonZeroInt(1,5)
b = NonZeroInt(-3,3)
c = Integer(randint(-5,5))
v=[b*log(x), b*exp(x)]

F=v[p]

if p==0:
   tar=a   
   Ans = b/a*x-b+b*log(a)
else:
   tar=c
   Ans = b*exp(c)*x-c*b*exp(c)+b*exp(c)
\end{sagesilent}

\latexProblemContent{
\begin{problem}

Find the equation of the line tangent to $f(x)=\sage{F}$ at $x=\sage{tar}$.

\input{2311_Compute_Derivative_0015.HELP.tex}

\[y=\answer{\sage{Ans}}\] 
\end{problem}}%}
%%%%%%%%%%%%%%%%%%%%%%


%%%%%%%%%%%%%%%%%%%%%%%
%%\tagged{Cat@One, Cat@Two, Cat@Three, Cat@Four, Cat@Five, Ans@ShortAns, Type@Compute, Topic@Derivative, Sub@Implicit, Sub@Poly}{
\begin{sagesilent}
def FixDeriv(f):
   ftemp=f
   ftempderiv = ftemp.substitute(diff(w) == dy/dx)
   #ftempderivv = ftempderiv.substitute(diff(diff(y)) == (d^2*y)/(d*x^2))
   ffinal = ftempderiv
   return ffinal

def RegY(f):
   ftemp=f
   ftempnew = ftemp.substitute(w(x) == y)
   ffinal=ftempnew
   return ffinal


a = NonZeroInt(-3,3)
p=Integer(randint(0,8))
w=function('w',x)
v=[x*w, x^2*w, x^3*w, x*w^2, x*w^3, x^2*w^2, x^2*w^3, x^3*w^2, x^3*w^3]

F=v[p]
f=diff(F, x)
g=solve(diff(F),diff(w))
f2=g[0]
f3=FixDeriv(f2)
f4=RegY(f3)
G=RegY(F)
\end{sagesilent}

\latexProblemContent{
\begin{problem}

Implicitly derive the expression $\sage{G}=\sage{a}$ and solve for $\dfrac{dy}{dx}$.

\input{2311_Compute_Derivative_0016.HELP.tex}

\[\answer{\sage{f4}}\] 
\end{problem}}%}
%%%%%%%%%%%%%%%%%%%%%%


%%%%%%%%%%%%%%%%%%%%%%%
%%\tagged{Cat@One, Cat@Two, Cat@Three, Cat@Four, Cat@Five, Ans@ShortAns, Type@Compute, Topic@Derivative, Sub@Implicit, Sub@Trig, Sub@Poly}{
\begin{sagesilent}
def FixDeriv(f):
   ftemp=f
   ftempderiv = ftemp.substitute(diff(w) == dy/dx)
   #ftempderivv = ftempderiv.substitute(diff(diff(y)) == (d^2*y)/(d*x^2))
   ffinal = ftempderiv
   return ffinal

def RegY(f):
   ftemp=f
   ftempnew = ftemp.substitute(w(x) == y)
   ffinal=ftempnew
   return ffinal


a = NonZeroInt(-3,3)
p=Integer(randint(0,7))
w=function('w',x)
v=[sin(x*w), sin(x^2*w), cos(x*w), sin(x*w^2), cos(x^2*w), cos(x*w^2), sin(x^2*w^2), cos(x^2*w^2)]

F=v[p]
f=diff(F, x)
g=solve(diff(F),diff(w))
f2=g[0]
f3=FixDeriv(f2)
f4=RegY(f3)
G=RegY(F)
\end{sagesilent}

\latexProblemContent{
\begin{problem}

Implicitly derive the expression $\sage{G}=\sage{a}$ and solve for $\dfrac{dy}{dx}$.

\input{2311_Compute_Derivative_0017.HELP.tex}

\[\answer{\sage{f4}}\] 
\end{problem}}%}
%%%%%%%%%%%%%%%%%%%%%%


%%%%%%%%%%%%%%%%%%%%%%%
%%\tagged{Cat@One, Cat@Two, Cat@Three, Cat@Four, Cat@Five, Ans@ShortAns, Type@Compute, Topic@Derivative, Sub@Implicit, Sub@Trig, Sub@Poly}{
\begin{sagesilent}
def FixDeriv(f):
   ftemp=f
   ftempderiv = ftemp.substitute(diff(w) == dy/dx)
   #ftempderivv = ftempderiv.substitute(diff(diff(y)) == (d^2*y)/(d*x^2))
   ffinal = ftempderiv
   return ffinal

def RegY(f):
   ftemp=f
   ftempnew = ftemp.substitute(w(x) == y)
   ffinal=ftempnew
   return ffinal


a = NonZeroInt(-3,3)
p=Integer(randint(0,7))
w=function('w',x)
v=[x*sin(w), x^2*sin(w), x*cos(w), x*sin(w^2), x^2*cos(w), x*cos(w^2), x^2*sin(w^2), x^2*cos(w^2)]

F=v[p]
f=diff(F, x)
g=solve(diff(F),diff(w))
f2=g[0]
f3=FixDeriv(f2)
f4=RegY(f3)
G=RegY(F)
\end{sagesilent}

\latexProblemContent{
\begin{problem}

Implicitly derive the expression $\sage{G}=\sage{a}$ and solve for $\dfrac{dy}{dx}$.

\input{2311_Compute_Derivative_0018.HELP.tex}

\[\answer{\sage{f4}}\] 
\end{problem}}%}
%%%%%%%%%%%%%%%%%%%%%%


%%%%%%%%%%%%%%%%%%%%%%%
%%\tagged{Cat@One, Cat@Two, Cat@Three, Cat@Four, Cat@Five, Ans@ShortAns, Type@Compute, Topic@Derivative, Sub@Implicit, Sub@Trig, Sub@Poly}{
\begin{sagesilent}
def FixDeriv(f):
   ftemp=f
   ftempderiv = ftemp.substitute(diff(w) == dy/dx)
   #ftempderivv = ftempderiv.substitute(diff(diff(y)) == (d^2*y)/(d*x^2))
   ffinal = ftempderiv
   return ffinal

def RegY(f):
   ftemp=f
   ftempnew = ftemp.substitute(w(x) == y)
   ffinal=ftempnew
   return ffinal


a = NonZeroInt(-3,3)
p=Integer(randint(0,7))
w=function('w',x)
v=[w*sin(x), w^2*sin(x), w*cos(x), w*sin(x^2), w^2*cos(x), w*cos(x^2), w^2*sin(x^2), w^2*cos(x^2)]

F=v[p]
f=diff(F, x)
g=solve(diff(F),diff(w))
f2=g[0]
f3=FixDeriv(f2)
f4=RegY(f3)
G=RegY(F)
\end{sagesilent}

\latexProblemContent{
\begin{problem}

Implicitly derive the expression $\sage{G}=\sage{a}$ and solve for $\dfrac{dy}{dx}$.

\input{2311_Compute_Derivative_0019.HELP.tex}

\[\answer{\sage{f4}}\] 
\end{problem}}%}
%%%%%%%%%%%%%%%%%%%%%%


%%%%%%%%%%%%%%%%%%%%%%%
%%\tagged{Cat@One, Cat@Two, Cat@Three, Cat@Four, Cat@Five, Ans@ShortAns, Type@Compute, Topic@Derivative, Sub@Implicit, Sub@Exp, Sub@Poly}{
\begin{sagesilent}
def FixDeriv(f):
   ftemp=f
   ftempderiv = ftemp.substitute(diff(w) == dy/dx)
   #ftempderivv = ftempderiv.substitute(diff(diff(y)) == (d^2*y)/(d*x^2))
   ffinal = ftempderiv
   return ffinal

def RegY(f):
   ftemp=f
   ftempnew = ftemp.substitute(w(x) == y)
   ffinal=ftempnew
   return ffinal


a = NonZeroInt(-3,3)
p=Integer(randint(0,8))
w=function('w',x)
v=[exp(x*w), exp(x^2*w), exp(x*w^2), exp(x^2*w^2), exp(x^2*w^3), exp(x^3*w^2), exp(x^3*w^3), exp(x*w^3), exp(x^3*w)]

F=v[p]
f=diff(F, x)
g=solve(diff(F),diff(w))
f2=g[0]
f3=FixDeriv(f2)
f4=RegY(f3)
G=RegY(F)
\end{sagesilent}

\latexProblemContent{
\begin{problem}

Implicitly derive the expression $\sage{G}=\sage{a}$ and solve for $\dfrac{dy}{dx}$.

\input{2311_Compute_Derivative_0020.HELP.tex}

\[\answer{\sage{f4}}\] 
\end{problem}}%}
%%%%%%%%%%%%%%%%%%%%%%


%%%%%%%%%%%%%%%%%%%%%%%
%%\tagged{Cat@One, Cat@Two, Cat@Three, Cat@Four, Cat@Five, Ans@ShortAns, Type@Compute, Topic@Derivative, Sub@Implicit, Sub@Exp, Sub@Poly}{
\begin{sagesilent}
def FixDeriv(f):
   ftemp=f
   ftempderiv = ftemp.substitute(diff(w) == dy/dx)
   #ftempderivv = ftempderiv.substitute(diff(diff(y)) == (d^2*y)/(d*x^2))
   ffinal = ftempderiv
   return ffinal

def RegY(f):
   ftemp=f
   ftempnew = ftemp.substitute(w(x) == y)
   ffinal=ftempnew
   return ffinal


a = NonZeroInt(-3,3)
p=Integer(randint(0,8))
w=function('w',x)
v=[x*exp(x*w), x*exp(x^2*w), x*exp(x*w^2), x*exp(x^2*w^2), x*exp(x^2*w^3), x*exp(x^3*w^2), x*exp(x^3*w^3), x*exp(x*w^3), x*exp(x^3*w)]

F=v[p]
f=diff(F, x)
g=solve(diff(F),diff(w))
f2=g[0]
f3=FixDeriv(f2)
f4=RegY(f3)
G=RegY(F)
\end{sagesilent}

\latexProblemContent{
\begin{problem}

Implicitly derive the expression $\sage{G}=\sage{a}$ and solve for $\dfrac{dy}{dx}$.

\input{2311_Compute_Derivative_0021.HELP.tex}

\[\answer{\sage{f4}}\] 
\end{problem}}%}
%%%%%%%%%%%%%%%%%%%%%%


%%%%%%%%%%%%%%%%%%%%%%%
%%\tagged{Cat@One, Cat@Two, Cat@Three, Cat@Four, Cat@Five, Ans@ShortAns, Type@Compute, Topic@Derivative, Sub@Implicit, Sub@Exp, Sub@Poly}{
\begin{sagesilent}
def FixDeriv(f):
   ftemp=f
   ftempderiv = ftemp.substitute(diff(w) == dy/dx)
   #ftempderivv = ftempderiv.substitute(diff(diff(y)) == (d^2*y)/(d*x^2))
   ffinal = ftempderiv
   return ffinal

def RegY(f):
   ftemp=f
   ftempnew = ftemp.substitute(w(x) == y)
   ffinal=ftempnew
   return ffinal


a = NonZeroInt(-3,3)
p=Integer(randint(0,8))
w=function('w',x)
v=[w*exp(x*w), w*exp(x^2*w), w*exp(x*w^2), w*exp(x^2*w^2), w*exp(x^2*w^3), w*exp(x^3*w^2), w*exp(x^3*w^3), w*exp(x*w^3), w*exp(x^3*w)]

F=v[p]
f=diff(F, x)
g=solve(diff(F),diff(w))
f2=g[0]
f3=FixDeriv(f2)
f4=RegY(f3)
G=RegY(F)
\end{sagesilent}

\latexProblemContent{
\begin{problem}

Implicitly derive the expression $\sage{G}=\sage{a}$ and solve for $\dfrac{dy}{dx}$.

\input{2311_Compute_Derivative_0022.HELP.tex}

\[\answer{\sage{f4}}\] 
\end{problem}}%}
%%%%%%%%%%%%%%%%%%%%%%


%%%%%%%%%%%%%%%%%%%%%%%
%%\tagged{Cat@One, Cat@Two, Cat@Three, Cat@Four, Cat@Five, Ans@ShortAns, Type@Compute, Topic@Derivative, Sub@Implicit, Sub@Exp, Sub@Poly}{
\begin{sagesilent}
def FixDeriv(f):
   ftemp=f
   ftempderiv = ftemp.substitute(diff(w) == dy/dx)
   #ftempderivv = ftempderiv.substitute(diff(diff(y)) == (d^2*y)/(d*x^2))
   ffinal = ftempderiv
   return ffinal

def RegY(f):
   ftemp=f
   ftempnew = ftemp.substitute(w(x) == y)
   ffinal=ftempnew
   return ffinal


a = NonZeroInt(-3,3)
p=Integer(randint(0,8))
w=function('w',x)
v=[x*exp(w), x^2*exp(w), x*exp(w^2), x^2*exp(w^2), x^2*exp(w^3), x^3*exp(w^2), x^3*exp(w^3), x*exp(w^3), x^3*exp(w)]

F=v[p]
f=diff(F, x)
g=solve(diff(F),diff(w))
f2=g[0]
f3=FixDeriv(f2)
f4=RegY(f3)
G=RegY(F)
\end{sagesilent}

\latexProblemContent{
\begin{problem}

Implicitly derive the expression $\sage{G}=\sage{a}$ and solve for $\dfrac{dy}{dx}$.

\input{2311_Compute_Derivative_0023.HELP.tex}

\[\answer{\sage{f4}}\] 
\end{problem}}%}
%%%%%%%%%%%%%%%%%%%%%%

%%%%%%%%%%%%%%%%%%%%%%%
%%\tagged{Cat@One, Cat@Two, Cat@Three, Cat@Four, Cat@Five, Ans@ShortAns, Type@Compute, Topic@Derivative, Sub@Implicit, Sub@Log, Sub@Poly}{
\begin{sagesilent}
def FixDeriv(f):
   ftemp=f
   ftempderiv = ftemp.substitute(diff(w) == dy/dx)
   #ftempderivv = ftempderiv.substitute(diff(diff(y)) == (d^2*y)/(d*x^2))
   ffinal = ftempderiv
   return ffinal

def RegY(f):
   ftemp=f
   ftempnew = ftemp.substitute(w(x) == y)
   ffinal=ftempnew
   return ffinal


a = NonZeroInt(-3,3)
p=Integer(randint(0,8))
w=function('w',x)
v=[log(x*w), log(x^2*w), log(x*w^2), log(x^2*w^2), log(x^2*w^3), log(x^3*w^2), log(x^3*w^3), log(x*w^3), log(x^3*w)]

F=v[p]
f=diff(F, x)
g=solve(diff(F),diff(w))
f2=g[0]
f3=FixDeriv(f2)
f4=RegY(f3)
G=RegY(F)
\end{sagesilent}

\latexProblemContent{
\begin{problem}

Implicitly derive the expression $\sage{G}=\sage{a}$ and solve for $\dfrac{dy}{dx}$.

\input{2311_Compute_Derivative_0024.HELP.tex}

\[\answer{\sage{f4}}\] 
\end{problem}}%}
%%%%%%%%%%%%%%%%%%%%%%

%%% SO FAR 24-Compute %%%%%%%%

%%%%%%%%%%%%%%%%%%%%%%%
%%\tagged{Cat@One, Cat@Two, Cat@Three, Cat@Four, Cat@Five, Ans@ShortAns, Type@Compute, Topic@Derivative, Sub@Domain, Sub@Rational}{
\begin{sagesilent}
a = NonZeroInt(-5,5)
p=Integer(randint(0,2))
v=[1/(x-a), 1/(x-a)^2, 1/(x-a)^3]
F=v[p]
\end{sagesilent}

\latexProblemContent{
\begin{problem}

Find each interval over which $f(x)=\sage{F}$ is differentiable.

\input{2311_Compute_Derivative_0025.HELP.tex}

\[\answer{(-\infty,\sage{a})\cup(\sage{a},\infty)}\]

\end{problem}}%}
%%%%%%%%%%%%%%%%%%%%%%

%%%%%%%%%%%%%%%%%%%%%%%
%%\tagged{Cat@One, Cat@Two, Cat@Three, Cat@Four, Cat@Five, Ans@ShortAns, Type@Compute, Topic@Derivative, Sub@Domain, Sub@Rational}{
\begin{sagesilent}
a = NonZeroInt(-5,5)
b = NonZeroInt(-5,5,[0,a])
p=Integer(randint(0,2))
v=[1/((x-a)*(x-b)), 1/((x-a)^2*(x-b)), 1/((x-a)*(x-b)^2)]
F=v[p]
\end{sagesilent}

\latexProblemContent{
\begin{problem}

Find each interval over which $f(x)=\sage{F}$ is differentiable.

\input{2311_Compute_Derivative_0026.HELP.tex}

\[\answer{(-\infty,\sage{a})\cup(\sage{a},\sage{b})\cup(\sage{b},\infty)}\]
\end{problem}}%}
%%%%%%%%%%%%%%%%%%%%%%

%%%%%%%%%%%%%%%%%%%%%%%
%%\tagged{Cat@One, Cat@Two, Cat@Three, Cat@Four, Cat@Five, Ans@ShortAns, Type@Compute, Topic@Derivative, Sub@Poly, Sub@Power_Rule}{
\begin{sagesilent}
a = NonZeroInt(-5,5)
b = NonZeroInt(-5,5, [0,a])
p=Integer(randint(0,5))
v=[x^2, x^3, x^4, expand((x-a)*(x-b)), expand((x-a)^2*(x-b)), expand((x-a)*(x-b)^2)]
F=v[p]
Ans=diff(F,x)
\end{sagesilent}

\latexProblemContent{
\begin{problem}

Compute the following derivative:

\input{2311_Compute_Derivative_0027.HELP.tex}

\[\dfrac{d}{dx}\left(\sage{F}\right)=\answer{\sage{Ans}}\]
\end{problem}}%}
%%%%%%%%%%%%%%%%%%%%%%

%%% SO FAR 27-Compute %%%%%%%%


%%%%%%%%%%%%%%%%%%%%%%%
%%\tagged{Cat@One, Cat@Two, Cat@Three, Cat@Four, Cat@Five, Ans@ShortAns, Type@Compute, Topic@Derivative, Sub@Arctrig}{
\begin{sagesilent}
a = NonZeroInt(-5,5)
b = NonZeroInt(-5,5, [0,a])
p=Integer(randint(0,2))
v=[a*arcsin(b*x), a*arccos(b*x), a*arctan(b*x)]
F=v[p]
Ans=diff(F,x)
\end{sagesilent}

\latexProblemContent{
\begin{problem}

Compute the following derivative:

\input{2311_Compute_Derivative_0028.HELP.tex}

\[\dfrac{d}{dx}\left(\sage{F}\right)=\answer{\sage{Ans}}\]
\end{problem}}%}
%%%%%%%%%%%%%%%%%%%%%%


%%%%%%%%%%%%%%%%%%%%%%%
%%\tagged{Cat@One, Cat@Two, Cat@Three, Cat@Four, Cat@Five, Ans@ShortAns, Type@Compute, Topic@Derivative, Sub@Related_Rates}{
\begin{sagesilent}
a = NonZeroInt(1,72)
A=a*pi
Ans=a/36
\end{sagesilent}

\latexProblemContent{
\begin{problem}

A snowball is melting at a rate of $\sage{A}$ $cm^3/s$.  At what rate is the radius decreasing when the volume of the snowball is $36\,\pi$ $cm^3$?\\[0.5in]

\input{2311_Compute_Derivative_0029.HELP.tex}

The radius is decreasing at $\answer{\sage{Ans}}$ $cm/s$.
\end{problem}}%}
%%%%%%%%%%%%%%%%%%%%%%


%%%%%%%%%%%%%%%%%%%%%%%
%%\tagged{Cat@One, Cat@Two, Cat@Three, Cat@Four, Cat@Five, Ans@ShortAns, Type@Compute, Topic@Derivative, Sub@Related_Rates}{
\begin{sagesilent}
a = NonZeroInt(1,6)
A=a*100
Ans=a*20
\end{sagesilent}

\latexProblemContent{
\begin{problem}

A plane, currently overhead, is flying directly away from you at $\sage{A}$ mph at an altitude of $3$ miles.  How fast is the plane's distance from you increasing at
the moment when the plane is flying over a point on the ground $4$ miles from you?\\[0.5in]

\input{2311_Compute_Derivative_0030.HELP.tex}

The distance between you and the plane is increasing by $\answer{\sage{Ans}}$ mph.
\end{problem}}%}
%%%%%%%%%%%%%%%%%%%%%%


%%%%%%%%%%%%%%%%%%%%%%%
%%\tagged{Cat@One, Cat@Two, Cat@Three, Cat@Four, Cat@Five, Ans@ShortAns, Type@Compute, Topic@Derivative, Sub@Related_Rates}{
\begin{sagesilent}
a = NonZeroInt(2,10)
A = a*4
B = a*3
Ans=(3*A+4*B)/5
\end{sagesilent}

\latexProblemContent{
\begin{problem}

A road running north to south crosses a road going east to west at the point $P$.  Cyclist $A$ is riding north along the first road, and cyclist $B$ is riding east along the second road.  At a particular time, cyclist $A$ is $3$ kilometers to the north of $P$ and traveling at $\sage{A}$ kph, while cyclist $B$ is $4$ kilometers to the east of $P$ and traveling at $\sage{B}$ kph.  How fast is the distance between the two cyclists changing? \\[0.5in]

\input{2311_Compute_Derivative_0031.HELP.tex}

The distance between the two cyclists is increasing by $\answer{\sage{Ans}}$ kph.
\end{problem}}%}
%%%%%%%%%%%%%%%%%%%%%%

%%%%%%%%%%%%% 31 Compute %%%%%%%%%%%%%%%


%%%%%%%%%%%%%%%%%%%%%%%
%%\tagged{Cat@One, Cat@Two, Cat@Three, Cat@Four, Cat@Five, Ans@ShortAns, Type@Compute, Topic@Derivative, Sub@Related_Rates}{
\begin{sagesilent}
a = NonZeroInt(1,4)
A = a*5
B = a*3
Ans=3/4
\end{sagesilent}

\latexProblemContent{
\begin{problem}

A swing consists of a board at the end of a $\sage{A}$ ft long rope.  Think of the board as a point $P$ at the end of the rope, and let $Q$ be the point of attachment at the other end.  Suppose that the swing is directly below $Q$ at time $t=0$, and is being pushed by someone who walks at $\sage{B}$ ft/s from left to right.  What is the angular speed of the rope in rad/s after 1 sec? \\[0.5in]

\input{2311_Compute_Derivative_0032.HELP.tex}

The angular speed of the rope is $\answer{\sage{Ans}}$ rad/s.
\end{problem}}%}
%%%%%%%%%%%%%%%%%%%%%%

%%%%%%%%%%%%%%%%%%%%%%%
%%\tagged{Cat@One, Cat@Two, Cat@Three, Cat@Four, Cat@Five, Ans@ShortAns, Type@Compute, Topic@Derivative, Sub@Related_Rates}{
\begin{sagesilent}
a = NonZeroInt(10,16)
b = NonZeroInt(4,6)
B = a-b
c = NonZeroInt(3,9)
Ans=b
\end{sagesilent}

\latexProblemContent{
\begin{problem}

It is night. Someone who is $\sage{b}$ feet tall is walking away from a street light at a rate of $\sage{B}$ feet per second.  The street light is $\sage{a}$ feet tall.  The person casts a shadow on the ground in front of them. How fast is the length of the shadow growing when the person is $\sage{c}$ feet from the street light? \\[0.5in]

\input{2311_Compute_Derivative_0033.HELP.tex}

The length of the shadow is growing at a rate of $\answer{\sage{Ans}}$ ft/s.
\end{problem}}%}
%%%%%%%%%%%%%%%%%%%%%%


%%%%%%%%%%% 33 - Compute %%%%%%%%%%


%%%%%%%%%%%%%%%%%%%%%%%
%%\tagged{Cat@One, Cat@Two, Cat@Three, Cat@Four, Cat@Five, Ans@ShortAns, Type@Compute, Topic@Derivative, Sub@Differential, Sub@Rational}{
\begin{sagesilent}
a = NonZeroInt(-5,5)
b = NonZeroInt(-5,5)
c = NonZeroInt(-5,5)

p=Integer(randint(0,2))

v=[a/(x-b), a/(x-b)^2, a/expand((x-b)*(x-c))]
F=v[p]
f=diff(F, x)
Ans=f
\end{sagesilent}

\latexProblemContent{
\begin{problem}

Compute the differential of the function $y=\sage{F}$.

\input{2311_Compute_Derivative_0034.HELP.tex}

\[dy=\answer{\sage{Ans}\,dx}\]
\end{problem}}%}
%%%%%%%%%%%%%%%%%%%%%%

%%%%%%%%%%%%%%%%%%%%%%%
%%\tagged{Cat@One, Cat@Two, Cat@Three, Cat@Four, Cat@Five, Ans@ShortAns, Type@Compute, Topic@Derivative, Sub@Differential}{
\begin{sagesilent}
a = NonZeroInt(-5,5)
b = NonZeroInt(-5,5)
c = NonZeroInt(-5,5)

p=Integer(randint(0,8))

v=[a*sqrt(x)+c, a*log(x)+c, a*exp(x)+c, a*x^2+c, a*x^3+c, a*x^4+c, (x-b), a*sin(x)+c, a*cos(x)+c]
F=v[p]
f=diff(F, x)
Ans=f
\end{sagesilent}

\latexProblemContent{
\begin{problem}

Compute the differential of the function $y=\sage{F}$.

\input{2311_Compute_Derivative_0035.HELP.tex}

\[dy=\answer{\sage{Ans}\,dx}\]
\end{problem}}%}
%%%%%%%%%%%%%%%%%%%%%%

%%%%%%%%%%%%%%%%%%%%%%%
%%\tagged{Cat@One, Cat@Two, Cat@Three, Cat@Four, Cat@Five, Ans@ShortAns, Type@Compute, Topic@Derivative, Sub@Differential, Sub@Chain_Rule}{
\begin{sagesilent}
a = NonZeroInt(-5,5)
b = NonZeroInt(-5,5)
c = NonZeroInt(-5,5)

p=Integer(randint(0,8))
q=Integer(randint(0,8))

v=[sqrt(x), log(x), exp(x), x^2, x^3, x^4, (x-b), sin(x), cos(x)]
F=v[p]
G=v[q]
H=F(G)
f=diff(H, x)
Ans=f
\end{sagesilent}

\latexProblemContent{
\begin{problem}

Compute the differential of the function $y=\sage{H}$.

\input{2311_Compute_Derivative_0036.HELP.tex}

\[dy=\answer{\sage{Ans}\,dx}\]
\end{problem}}%}
%%%%%%%%%%%%%%%%%%%%%%

%%%%%%%% 1-Concept  -----  36-Compute %%%%%%%%

%%%%%%%%%%%%%%%%%%%%%%%
%%\tagged{Cat@One, Cat@Two, Cat@Three, Cat@Four, Cat@Five, Ans@MC, Type@Concept, Topic@Derivative, Sub@Theorems_EVT}{
\begin{sagesilent}
a = NonZeroInt(-15,15)
b=a-1
c=a+1
p=Integer(randint(1,4))
f=log(c-x)
g=(x-b)^p-1
l=NonZeroInt(2,10)
lower=a-l
u=NonZeroInt(3,10)
upper=a+u
\end{sagesilent}

\latexProblemContent{
\begin{problem}

Does the Extreme Value Theorem hold for the function \[f(x)=\left\{\begin{array}{ll}
\sage{f}\; , & x<\sage{a}\\[3pt]
\sage{g}\; , & x\geq\sage{a}
\end{array}\right.\]
 over the interval $[\sage{lower},\sage{upper}]$?

\input{2311_Concept_Derivative_0001.HELP.tex}

 \begin{multipleChoice}
 \choice[correct]{Yes}
 \choice{No}
 \end{multipleChoice}

\end{problem}}%}
%%%%%%%%%%%%%%%%%%%%%%


%%%%%%%%%%%%%%%%%%%%%%%
%%\tagged{Cat@One, Cat@Two, Cat@Three, Cat@Four, Cat@Five, Ans@MultiAns, Type@Concept, Topic@Derivative, Sub@Theorems_EVT}{
\begin{sagesilent}
a = NonZeroInt(-15,15)
b=a-1
c=a+1
p=Integer(randint(1,4))
f=log(c-x)
g=(x-b)^p-1
l=NonZeroInt(2,10)
lower=a-l
u=NonZeroInt(3,10)
upper=a+u
\end{sagesilent}

\latexProblemContent{
\begin{problem}

Does the Extreme Value Theorem hold for the function \[f(x)=\left\{\begin{array}{ll}
\sage{f}\; , & x<\sage{a}\\[3pt]
\sage{g}\; , & x\geq\sage{a}
\end{array}\right.\]
 over the interval $[\sage{lower},\sage{upper})$?

\input{2311_Concept_Derivative_0002.HELP.tex}

 \begin{multipleChoice}
 \choice{Yes}
 \choice[correct]{No}
 \end{multipleChoice}

If not, which condition fails?

\begin{multipleChoice}
\choice{$f$ is not continuous on the interval}
\choice[correct]{The interval is not closed}
\choice{$f$ is not differentiable on the interval}
\choice{Nothing fails; the theorem holds}
\end{multipleChoice}
\end{problem}}%}
%%%%%%%%%%%%%%%%%%%%%%

%%%%%%%%%%%%%%%%%%%%%%%
%%\tagged{Cat@One, Cat@Two, Cat@Three, Cat@Four, Cat@Five, Ans@MultiAns, Type@Concept, Topic@Derivative, Sub@Theorems_EVT}{
\begin{sagesilent}
a = NonZeroInt(-10,10)
b=a-1
c=a+1
p=NonZeroInt(-10,10)
f=log(c-x)
g=(x-b)^p

r = randint(1,20)
q = randint(1,20)
intstart = a-r
intend = a + q
\end{sagesilent}

\latexProblemContent{
\begin{problem}

Does the Extreme Value Theorem hold for the function \[f(x)=\left\{\begin{array}{ll}
\sage{f}\; , & x<\sage{a}\\[3pt]
\sage{g}\; , & x\geq\sage{a}
\end{array}\right.\]
 over the interval $[\sage{intstart},\sage{intend}]$?

\input{2311_Concept_Derivative_0003.HELP.tex}
 
 \begin{multipleChoice}
 \choice{Yes}
 \choice[correct]{No}
 \end{multipleChoice}

If not, which condition fails?

\begin{multipleChoice}
\choice[correct]{$f$ is not continuous on the interval}
\choice{The interval is not closed}
\choice{$f$ is not differentiable on the interval}
\choice{Nothing fails; the theorem holds}
\end{multipleChoice}
\end{problem}}%}
%%%%%%%%%%%%%%%%%%%%%%

%%%%%%%%%%% 3 Concept & 36 Compute %%%%%%%%%%%%%


%%%%%%%%%%%%%%%%%%%%%%%
%%\tagged{Cat@One, Cat@Two, Cat@Three, Cat@Four, Cat@Five, Ans@MultiAns, Type@Compute, Topic@Derivative, Sub@Poly, Sub@Multi_Deriv}{
\begin{sagesilent}
a = NonZeroInt(-5,5)
b = NonZeroInt(-7,7)
p=Integer(randint(0,2))
v=[a*(x-b), a*expand((x-b)^2), a*expand((x-b)^3)]
F=v[p]
f=diff(F, x)
g=diff(f,x)
\end{sagesilent}

\latexProblemContent{
\begin{problem}

Compute the first and second derivatives for the function\\ $f(x)=\sage{F}$.\\

\input{2311_Compute_Derivative_0037.HELP.tex}

\[f'(x)=\answer{\sage{f}}\]

\[f''(x)=\answer{\sage{g}}\] 
\end{problem}}%}
%%%%%%%%%%%%%%%%%%%%%%




%%%%%%%%%%%%%%%%%%%%%%%
%%\tagged{Cat@One, Cat@Two, Cat@Three, Cat@Four, Cat@Five, Ans@MultiAns, Type@Compute, Topic@Derivative, Sub@Trig, Sub@Multi_Deriv}{
\begin{sagesilent}
a = NonZeroInt(-10,10)
b = NonZeroInt(-5,5)
p=Integer(randint(0,5))

vtrig=[a*sin(b*x), a*cos(b*x), a*tan(b*x), a*sin(x)+b*cos(x), a*sin(x)^2, a*cos(x)^2]

F=vtrig[p]
f=diff(F, x)
g=diff(f,x)
\end{sagesilent}

\latexProblemContent{
\begin{problem}

Compute the first and second derivatives for the function\\ $f(x)=\sage{F}$.\\

\input{2311_Compute_Derivative_0038.HELP.tex}

\[f'(x)=\answer{\sage{f}}\]

\[f''(x)=\answer{\sage{g}}\] 
\end{problem}}%}
%%%%%%%%%%%%%%%%%%%%%%


%%%%%%%%% 3 Concept - 38 Compute %%%%%%%%%%%

%%%%%%%%%%%%%%%%%%%%%%%
%%\tagged{Cat@One, Cat@Two, Cat@Three, Cat@Four, Cat@Five, Ans@MultiAns, Type@Compute, Topic@Derivative, Sub@Product_Rule, Sub@Multi_Deriv}{
\begin{sagesilent}
a = NonZeroInt(-5,5)
b = NonZeroInt(-5,5)
c = NonZeroInt(-5,5)
p=Integer(randint(0,11))
q=Integer(randint(0,11))
v=[a*sqrt(x)+c, a*log(x)+c, a*exp(x)+c, a*x^2+c, a*x^3+c, a*x^4+c, (x-b), a*sin(x)+c, a*cos(x)+c, a/x+c, a/x^2+c, a/x^3+c]

Fone=v[p]
Ftwo=v[q]
G=Fone*Ftwo
f=diff(G, x)
g=diff(f,x)
\end{sagesilent}

\latexProblemContent{
\begin{problem}

Compute the first and second derivatives for the function\\ $f(x)=\sage{G}$.\\

\input{2311_Compute_Derivative_0039.HELP.tex}

\[f'(x)=\answer{\sage{f}}\]

\[f''(x)=\answer{\sage{g}}\] 
\end{problem}}%}
%%%%%%%%%%%%%%%%%%%%%%


%%%%%%%%%%%%%%%%%%%%%%%
%%\tagged{Cat@One, Cat@Two, Cat@Three, Cat@Four, Cat@Five, Ans@MultiAns, Type@Compute, Topic@Derivative, Sub@Chain_Rule, Sub@Multi_Deriv}{
\begin{sagesilent}
b = NonZeroInt(-5,5)
p=Integer(randint(0,11))
q=Integer(randint(0,11))
c = RandInt(-20,20)
v=[sqrt(x), log(x), exp(x), x^2, x^3, x^4, (x-b), sin(x), cos(x), 1/x, 1/x^2, 1/x^3]

Fone=v[p]
Ftwo=v[q]
G=Fone(Ftwo) + c
f=diff(G, x)
g=diff(f,x)
\end{sagesilent}

\latexProblemContent{
\begin{problem}

Compute the first and second derivatives for the function\\ $f(x)=\sage{G}$.\\

\input{2311_Compute_Derivative_0040.HELP.tex}

\[f'(x)=\answer{\sage{f}}\]

\[f''(x)=\answer{\sage{g}}\] 
\end{problem}}%}
%%%%%%%%%%%%%%%%%%%%%%

%%%%%%%%%%%%% 3 Concept  40 Compute %%%%%%%%%%%%


%%%%%%%%%%%%%%%%%%%%%%%
%%\tagged{Cat@One, Cat@Two, Cat@Three, Cat@Four, Cat@Five, Ans@ShortAns, Type@Compute, Topic@Derivative, Sub@Poly, Sub@Critical_Number}{
\begin{sagesilent}
a = NonZeroInt(-4,4)
b = RandInt(-20,20)
p=Integer(randint(0,2))
vpoly=[(x-a), expand((x-a)^2), expand((x-a)^3)]
F=vpoly[p]
f=integral(F,x) + b
\end{sagesilent}

\latexProblemContent{
\begin{problem}

Find the critical numbers for the function $f(x)=\sage{f}$.\\

\input{2311_Compute_Derivative_0041.HELP.tex}

\[x=\answer{\sage{a}}\]
\end{problem}}%}
%%%%%%%%%%%%%%%%%%%%%%


%%%%%%%%%%%%%%%%%%%%%%%
%%\tagged{Cat@One, Cat@Two, Cat@Three, Cat@Four, Cat@Five, Ans@ShortAns, Type@Compute, Topic@Derivative, Sub@Poly, Sub@Critical_Number}{
\begin{sagesilent}
a = NonZeroInt(-4,4)
b = RandInt(-20,20)
p=Integer(randint(0,2))
vpoly=[expand(x*(x-a)), expand(x*(x-a)^2), expand(x*(x-a)^3)]
F=vpoly[p]
f=integral(F,x) + b
\end{sagesilent}

\latexProblemContent{
\begin{problem}

Find the critical numbers for the function $f(x)=\sage{f}$.\\

\input{2311_Compute_Derivative_0042.HELP.tex}

\[x=\answer{0,\sage{a}}\]
\end{problem}}%}
%%%%%%%%%%%%%%%%%%%%%%


%%%%%%%%%%%%%%%%%%%%%%%
%%\tagged{Cat@One, Cat@Two, Cat@Three, Cat@Four, Cat@Five, Ans@ShortAns, Type@Compute, Topic@Derivative, Sub@Poly, Sub@Critical_Number}{
\begin{sagesilent}
a = NonZeroInt(-4,4)
b = NonZeroInt(-4,4)
c = RandInt(-20,20)
p=Integer(randint(0,1))
vpoly=[expand((x-a)*(x-b)), (x-a)^2*(x-b)]
F=vpoly[p]
f=integral(F,x) + c
\end{sagesilent}

\latexProblemContent{
\begin{problem}

Find the critical numbers for the function $f(x)=\sage{f}$.\\

\input{2311_Compute_Derivative_0043.HELP.tex}

\[x=\answer{\sage{b},\sage{a}}\]
\end{problem}}%}
%%%%%%%%%%%%%%%%%%%%%%

%%%%%%%%%%%%%%%%%%%%%%%
%%\tagged{Cat@One, Cat@Two, Cat@Three, Cat@Four, Cat@Five, Ans@ShortAns, Type@Compute, Topic@Derivative, Sub@Optimization}{
\begin{sagesilent}
a = NonZeroInt(1,10)
A=a^2
\end{sagesilent}

\latexProblemContent{
\begin{problem}

Find two positive numbers whose product is $\sage{A}$ and whose sum is a minimum.

\input{2311_Compute_Derivative_0044.HELP.tex}

\[\mbox{The two numbers are\;} \answer{\sage{a}, \sage{a}}\]
\end{problem}}%}
%%%%%%%%%%%%%%%%%%%%%%

%%%%%%%%%%%%%%%%%%%%%%%
%%\tagged{Cat@One, Cat@Two, Cat@Three, Cat@Four, Cat@Five, Ans@ShortAns, Type@Compute, Topic@Derivative, Sub@Optimization}{
\begin{sagesilent}
a = NonZeroInt(1,15)
b = NonZeroInt(1,15,[a])   
A=a^2
B=b^2
Area=2*a*b
\end{sagesilent}

\latexProblemContent{
\begin{problem}

Find the area of the largest rectangle that can be inscribed in the ellipse $\dfrac{x^2}{\sage{A}}+\dfrac{y^2}{\sage{B}}=1$

\input{2311_Compute_Derivative_0045.HELP.tex}

\[\mbox{The largest area is \;} \answer{\sage{Area}}\]
\end{problem}}%}
%%%%%%%%%%%%%%%%%%%%%%

%%%%%%%%%%%%%%%%%%%%%%%
%%\tagged{Cat@One, Cat@Two, Cat@Three, Cat@Four, Cat@Five, Ans@ShortAns, Type@Compute, Topic@Derivative, Sub@Optimization}{
\begin{sagesilent}
a = NonZeroInt(1,8)
b = NonZeroInt(1,8)
A=a*b
Area=A^2/(2*(pi+4))
\end{sagesilent}

\latexProblemContent{
\begin{problem}

A Norman window has the shape of a rectangle surmounted by a semicircle (i.e. the diameter of the semicircle is equal to the width of the rectangle).  If the perimeter of the window is $\sage{A}$ ft, what area will allow in the greatest amount of light?

\input{2311_Compute_Derivative_0046.HELP.tex}

\[\mbox{The largest possible area is\;} \answer{\sage{Area}}\]
\end{problem}}%}
%%%%%%%%%%%%%%%%%%%%%%

%%%%%%%%%%%%%%%%%%%%%%%
%%\tagged{Cat@One, Cat@Two, Cat@Three, Cat@Four, Cat@Five, Ans@ShortAns, Type@Compute, Topic@Derivative, Sub@Optimization}{
\begin{sagesilent}
a = NonZeroInt(-15,15)
b = NonZeroInt(-15,15)   
X=(-a*b)/(a^2+1)
Y=b/(a^2+1)
f=a*x+b
\end{sagesilent}

\latexProblemContent{
\begin{problem}

Find the point on the line $y=\sage{f}$ that is closest to the origin.

\input{2311_Compute_Derivative_0047.HELP.tex}

\[\mbox{The closest point to the origin is}\; \answer{\left(\sage{X},\sage{Y}\right)}\]
\end{problem}}%}
%%%%%%%%%%%%%%%%%%%%%%

%%%%%%%%%%%%%% 3 Concept  46 Compute %%%%%%%%%%

%%%%%%%%%%%%%%%%%%%%%%%
%%\tagged{Cat@One, Cat@Two, Cat@Three, Cat@Four, Cat@Five, Ans@ShortAns, Type@Compute, Topic@Derivative, Sub@Trig}{
\begin{sagesilent}
b = NonZeroInt(-20,20)
c = RandInt(-20,20)   
p=Integer(randint(0,5))
vtrig=[b*sin(x), b*cos(x), b*tan(x), b*cot(x), b*sec(x), b*csc(x)]
F=vtrig[p] + c
f=diff(F,x)
\end{sagesilent}

\latexProblemContent{
\begin{problem}

Compute the following derivative:

\input{2311_Compute_Derivative_0048.HELP.tex}

\[\dfrac{d}{dx}\left(\sage{F}\right) = \answer{\sage{f}}\]
\end{problem}}%}
%%%%%%%%%%%%%%%%%%%%%%

%%%%%%%%%%%%%%%%%%%%%%%
%%\tagged{Cat@One, Cat@Two, Cat@Three, Cat@Four, Cat@Five, Ans@ShortAns, Type@Compute, Topic@Derivative, Sub@Trig, Sub@LHopital}{
\begin{sagesilent}
a = NonZeroInt(-8,8)
b = NonZeroInt(-8,8)
p=Integer(randint(0,3))
v=[sin(a*x)/tan(b*x), sin(a*x)/cos(b*x), tan(a*x)/sin(b*x), tan(a*x)/cos(b*x)]
F=v[p]
Ans=limit(F,x=0)
\end{sagesilent}

\latexProblemContent{
\begin{problem}

Find the limit.  Use L'H$\hat{o}$pital's rule where appropriate.

\input{2311_Compute_Derivative_0049.HELP.tex}

\[\lim\limits_{x\to0} \sage{F}=\answer{\sage{Ans}}\]
\end{problem}}%}
%%%%%%%%%%%%%%%%%%%%%%

%%%%%%%%%%%%%%%%%%%%%%%
%%\tagged{Cat@One, Cat@Two, Cat@Three, Cat@Four, Cat@Five, Ans@ShortAns, Type@Compute, Topic@Derivative, Sub@Trig, Sub@LHopital}{
\begin{sagesilent}
a = NonZeroInt(-10,10)
b = NonZeroInt(-10,10)
c = RandInt(-20,20)
F=(1+a/x)^(b*x) + c
Ans=limit(F,x=infinity)
\end{sagesilent}

\latexProblemContent{
\begin{problem}

Find the limit.  Use L'H$\hat{o}$pital's rule where appropriate.

\input{2311_Compute_Derivative_0050.HELP.tex}

\[\lim\limits_{x\to\infty} \sage{F}=\answer{\sage{Ans}}\]
\end{problem}}%}
%%%%%%%%%%%%%%%%%%%%%%


%%%%%%%%%%%%%%%%%%%%%%%
%%\tagged{Cat@One, Cat@Two, Cat@Three, Cat@Four, Cat@Five, Ans@ShortAns, Type@Compute, Topic@Derivative, Sub@Exp, Sub@LHopital}{
\begin{sagesilent}
a = NonZeroInt(-8,8)
b = NonZeroInt(-8,8)
c = NonZeroInt(-8,8)
p=Integer(randint(0,2))
v=[(x-b),(x-b)^2,(x-b)^3]
F=a*v[p]*exp(-x+c)
Ans=limit(F,x=infinity)
\end{sagesilent}

\latexProblemContent{
\begin{problem}

Find the limit.  Use L'H$\hat{o}$pital's rule where appropriate.

\input{2311_Compute_Derivative_0051.HELP.tex}

\[\lim\limits_{x\to\infty} \sage{F}=\answer{\sage{Ans}}\]
\end{problem}}%}
%%%%%%%%%%%%%%%%%%%%%%


%%%%%%%%%%%%%%%%%%%%%%%
%%\tagged{Cat@One, Cat@Two, Cat@Three, Cat@Four, Cat@Five, Ans@MultiAns, Type@Compute, Topic@Derivative, Sub@Poly, Sub@Theorems_MVT}{
\begin{sagesilent}
a = NonZeroInt(-8,8)
b = NonZeroInt(-5,5)
c = NonZeroInt(-5,5)
F=expand(a*(x-b)*(x-c))
\end{sagesilent}

\latexProblemContent{
\begin{problem}

Does the function $\sage{F}$ satisfy the conditions of the Mean Value Theorem (MVT) over the interval $[-1,1]$?

\input{2311_Compute_Derivative_0052.HELP.tex}

\begin{multipleChoice}
\choice No, the function is not continuous over the closed interval $[-1,1]$
\choice No, the function is not differentiable over the open interval $(-1,1)$
\choice No, the function is not defined over the interval $[-1,1]$
\choice[correct] Yes, all the conditions are met
\end{multipleChoice}

If the conditions are met, then find all numbers $c$ that satisfy the conclusion of the MVT.  If the conditions are not met, then write ``NONE.''

\[c = \answer{0}\]
\end{problem}}%}
%%%%%%%%%%%%%%%%%%%%%%


%%%%%%%%%% 3 Concept    51 Compute  %%%%%%%%%%%%


%%%%%%%%%%%%%%%%%%%%%%%
%%\tagged{Cat@One, Cat@Two, Cat@Three, Cat@Four, Cat@Five, Ans@MultiAns, Type@Compute, Topic@Derivative, Sub@Exp, Sub@Theorems_MVT}{
\begin{sagesilent}
a = NonZeroInt(-10,10)
b = NonZeroInt(-10,10)
c = NonZeroInt(1,5)
f=a*exp(b*x)
Ans=1/b*log((exp(b*c)-1)/(b*c))
\end{sagesilent}

\latexProblemContent{
\begin{problem}

Does the function $\sage{f}$ satisfy the conditions of the Mean Value Theorem (MVT) over the interval $[0,\sage{c}]$?

\input{2311_Compute_Derivative_0053.HELP.tex}

\begin{multipleChoice}
\choice No, the function is not continuous over the closed interval $[0,\sage{c}]$
\choice No, the function is not differentiable over the open interval $(0,\sage{c})$
\choice No, the function is not defined over the interval $[0,\sage{c}]$
\choice[correct] Yes, all the conditions are met
\end{multipleChoice}

If the conditions are met, then find all numbers $c$ that satisfy the conclusion of the MVT.  If the conditions are not met, then write ``NONE.''

\[c = \answer{\sage{Ans}}\]
\end{problem}}%}
%%%%%%%%%%%%%%%%%%%%%%

%%%%%%%%%%%%%%%%%%%%%%%
%%\tagged{Cat@One, Cat@Two, Cat@Three, Cat@Four, Cat@Five, Ans@MultiAns, Type@Compute, Topic@Derivative, Sub@Arctrig, Sub@Theorems_MVT}{
\begin{sagesilent}
a = NonZeroInt(-10,10)
b = NonZeroInt(-10,10)
c = NonZeroInt(1,5)
f=a*arctan(b*x)
Ans=sqrt((c-arctan(b*c))/(b^2*arctan(b*c)))
\end{sagesilent}

\latexProblemContent{
\begin{problem}

Does the function $\sage{f}$ satisfy the conditions of the Mean Value Theorem (MVT) over the interval $[0,\sage{c}]$?

\input{2311_Compute_Derivative_0054.HELP.tex}

\begin{multipleChoice}
\choice No, the function is not continuous over the closed interval $[0,\sage{c}]$
\choice No, the function is not differentiable over the open interval $(0,\sage{c})$
\choice No, the function is not defined over the interval $[0,\sage{c}]$
\choice[correct] Yes, all the conditions are met
\end{multipleChoice}

If the conditions are met, then find all numbers $c$ that satisfy the conclusion of the MVT.  If the conditions are not met, then write ``NONE.''

\[c = \answer{\sage{Ans}}\]
\end{problem}}%}
%%%%%%%%%%%%%%%%%%%%%%

\begin{sagesilent}
a = NonZeroInt(-10,10)
b = NonZeroInt(-10,10)
c = NonZeroInt(-10,10)
f=x^3+a*x^2+b*x+c
Ans=-a/3
\end{sagesilent}

\latexProblemContent{
\begin{problem}

Find all inflection points for the function $f(x)=\sage{f}$.  If there are no inflection points enter ``NONE.''

\input{2311_Compute_Derivative_0055.HELP.tex}

\[\mbox{Inflection points at\;} x=\answer{\sage{Ans}}\]
\end{problem}}%}
%%%%%%%%%%%%%%%%%%%%%%



%%%%%%%%%%%%%%%%%%%%%%%
%%\tagged{Cat@One, Cat@Two, Cat@Three, Cat@Four, Cat@Five, Ans@ShortAns, Type@Compute, Topic@Derivative, Sub@Domain, Sub@Inflection}{
\begin{sagesilent}
a = NonZeroInt(-10,10)
b = RandInt(-20,20)
f=(sqrt(x+a)) + b
\end{sagesilent}

\latexProblemContent{
\begin{problem}

Find all inflection points for the function $f(x)=x\sage{f}$.  If there are no inflection points enter ``NONE.''

\input{2311_Compute_Derivative_0056.HELP.tex}

\[\mbox{Inflection points at\;} x=\answer{NONE}\]
\end{problem}}%}
%%%%%%%%%%%%%%%%%%%%%%

%%%%%%%%%%%%%%%%%%%%%%%
%%\tagged{Cat@One, Cat@Two, Cat@Three, Cat@Four, Cat@Five, Ans@ShortAns, Type@Compute, Topic@Derivative, Sub@Inflection}{
\begin{sagesilent}
a = NonZeroInt(-10,10)
b = RandInt(-20, 20)
c = ISP(b)
f=(x+a)
Ans=a/2
\end{sagesilent}

\latexProblemContent{
\begin{problem}

Find all inflection points for the function $f(x)=x^{\frac{1}{3}}(\sage{f}) \sage{c}$.

\input{2311_Compute_Derivative_0057.HELP.tex}

\[\mbox{Inflection points at\;} x=\answer{0,\sage{Ans}}\]
\end{problem}}%}
%%%%%%%%%%%%%%%%%%%%%%

\begin{sagesilent}
a = NonZeroInt(-10,10)
b = NonZeroInt(1,10)
c = RandInt(-20,20)
f=a*exp(b/x) + c
Ans=-b/2
\end{sagesilent}

\latexProblemContent{
\begin{problem}

Find all inflection points for the function $f(x)=\sage{f}$.

\input{2311_Compute_Derivative_0058.HELP.tex}

\[\mbox{Inflection points at\;} x=\answer{\sage{Ans}}\]
\end{problem}}%}
%%%%%%%%%%%%%%%%%%%%%%




\begin{sagesilent}
a = NonZeroInt(-10,10)
b = NonZeroInt(1,10)
c = RandInt(-20,20)
f=a*exp(b/x) + c
Ans=-b/2
\end{sagesilent}

\latexProblemContent{
\begin{problem}

Find all inflection points for the function $f(x)=\sage{f}$.

\input{2311_Compute_Derivative_0058.HELP.tex}

\[\mbox{Inflection points at\;} x=\answer{\sage{Ans}}\]
\end{problem}}%}
%%%%%%%%%%%%%%%%%%%%%%




\begin{sagesilent}
a = NonZeroInt(-10,10)
b = NonZeroInt(1,10)
c = RandInt(-20,20)
f=a*exp(b/x) + c
Ans=-b/2
\end{sagesilent}

\latexProblemContent{
\begin{problem}

Find all inflection points for the function $f(x)=\sage{f}$.

\input{2311_Compute_Derivative_0058.HELP.tex}

\[\mbox{Inflection points at\;} x=\answer{\sage{Ans}}\]
\end{problem}}%}
%%%%%%%%%%%%%%%%%%%%%%




\begin{sagesilent}
a = NonZeroInt(-10,10)
b = NonZeroInt(1,10)
c = RandInt(-20,20)
f=a*exp(b/x) + c
Ans=-b/2
\end{sagesilent}

\latexProblemContent{
\begin{problem}

Find all inflection points for the function $f(x)=\sage{f}$.

\input{2311_Compute_Derivative_0058.HELP.tex}

\[\mbox{Inflection points at\;} x=\answer{\sage{Ans}}\]
\end{problem}}%}
%%%%%%%%%%%%%%%%%%%%%%




\begin{sagesilent}
a = NonZeroInt(-10,10)
b = NonZeroInt(1,10)
c = RandInt(-20,20)
f=a*exp(b/x) + c
Ans=-b/2
\end{sagesilent}

\latexProblemContent{
\begin{problem}

Find all inflection points for the function $f(x)=\sage{f}$.

\input{2311_Compute_Derivative_0058.HELP.tex}

\[\mbox{Inflection points at\;} x=\answer{\sage{Ans}}\]
\end{problem}}%}
%%%%%%%%%%%%%%%%%%%%%%




\begin{sagesilent}
a = NonZeroInt(-10,10)
b = NonZeroInt(1,10)
c = RandInt(-20,20)
f=a*exp(b/x) + c
Ans=-b/2
\end{sagesilent}

\latexProblemContent{
\begin{problem}

Find all inflection points for the function $f(x)=\sage{f}$.

\input{2311_Compute_Derivative_0058.HELP.tex}

\[\mbox{Inflection points at\;} x=\answer{\sage{Ans}}\]
\end{problem}}%}
%%%%%%%%%%%%%%%%%%%%%%




\begin{sagesilent}
a = NonZeroInt(-10,10)
b = NonZeroInt(1,10)
c = RandInt(-20,20)
f=a*exp(b/x) + c
Ans=-b/2
\end{sagesilent}

\latexProblemContent{
\begin{problem}

Find all inflection points for the function $f(x)=\sage{f}$.

\input{2311_Compute_Derivative_0058.HELP.tex}

\[\mbox{Inflection points at\;} x=\answer{\sage{Ans}}\]
\end{problem}}%}
%%%%%%%%%%%%%%%%%%%%%%




\begin{sagesilent}
a = NonZeroInt(-10,10)
b = NonZeroInt(1,10)
c = RandInt(-20,20)
f=a*exp(b/x) + c
Ans=-b/2
\end{sagesilent}

\latexProblemContent{
\begin{problem}

Find all inflection points for the function $f(x)=\sage{f}$.

\input{2311_Compute_Derivative_0058.HELP.tex}

\[\mbox{Inflection points at\;} x=\answer{\sage{Ans}}\]
\end{problem}}%}
%%%%%%%%%%%%%%%%%%%%%%




\begin{sagesilent}
a = NonZeroInt(-10,10)
b = NonZeroInt(1,10)
c = RandInt(-20,20)
f=a*exp(b/x) + c
Ans=-b/2
\end{sagesilent}

\latexProblemContent{
\begin{problem}

Find all inflection points for the function $f(x)=\sage{f}$.

\input{2311_Compute_Derivative_0058.HELP.tex}

\[\mbox{Inflection points at\;} x=\answer{\sage{Ans}}\]
\end{problem}}%}
%%%%%%%%%%%%%%%%%%%%%%




\begin{sagesilent}
a = NonZeroInt(-10,10)
b = NonZeroInt(1,10)
c = RandInt(-20,20)
f=a*exp(b/x) + c
Ans=-b/2
\end{sagesilent}

\latexProblemContent{
\begin{problem}

Find all inflection points for the function $f(x)=\sage{f}$.

\input{2311_Compute_Derivative_0058.HELP.tex}

\[\mbox{Inflection points at\;} x=\answer{\sage{Ans}}\]
\end{problem}}%}
%%%%%%%%%%%%%%%%%%%%%%




\begin{sagesilent}
a = NonZeroInt(-10,10)
b = NonZeroInt(1,10)
c = RandInt(-20,20)
f=a*exp(b/x) + c
Ans=-b/2
\end{sagesilent}

\latexProblemContent{
\begin{problem}

Find all inflection points for the function $f(x)=\sage{f}$.

\input{2311_Compute_Derivative_0058.HELP.tex}

\[\mbox{Inflection points at\;} x=\answer{\sage{Ans}}\]
\end{problem}}%}
%%%%%%%%%%%%%%%%%%%%%%




\begin{sagesilent}
a = NonZeroInt(-10,10)
b = NonZeroInt(1,10)
c = RandInt(-20,20)
f=a*exp(b/x) + c
Ans=-b/2
\end{sagesilent}

\latexProblemContent{
\begin{problem}

Find all inflection points for the function $f(x)=\sage{f}$.

\input{2311_Compute_Derivative_0058.HELP.tex}

\[\mbox{Inflection points at\;} x=\answer{\sage{Ans}}\]
\end{problem}}%}
%%%%%%%%%%%%%%%%%%%%%%




\begin{sagesilent}
a = NonZeroInt(-10,10)
b = NonZeroInt(1,10)
c = RandInt(-20,20)
f=a*exp(b/x) + c
Ans=-b/2
\end{sagesilent}

\latexProblemContent{
\begin{problem}

Find all inflection points for the function $f(x)=\sage{f}$.

\input{2311_Compute_Derivative_0058.HELP.tex}

\[\mbox{Inflection points at\;} x=\answer{\sage{Ans}}\]
\end{problem}}%}
%%%%%%%%%%%%%%%%%%%%%%




\begin{sagesilent}
a = NonZeroInt(-10,10)
b = NonZeroInt(1,10)
c = RandInt(-20,20)
f=a*exp(b/x) + c
Ans=-b/2
\end{sagesilent}

\latexProblemContent{
\begin{problem}

Find all inflection points for the function $f(x)=\sage{f}$.

\input{2311_Compute_Derivative_0058.HELP.tex}

\[\mbox{Inflection points at\;} x=\answer{\sage{Ans}}\]
\end{problem}}%}
%%%%%%%%%%%%%%%%%%%%%%




\begin{sagesilent}
a = NonZeroInt(-10,10)
b = NonZeroInt(1,10)
c = RandInt(-20,20)
f=a*exp(b/x) + c
Ans=-b/2
\end{sagesilent}

\latexProblemContent{
\begin{problem}

Find all inflection points for the function $f(x)=\sage{f}$.

\input{2311_Compute_Derivative_0058.HELP.tex}

\[\mbox{Inflection points at\;} x=\answer{\sage{Ans}}\]
\end{problem}}%}
%%%%%%%%%%%%%%%%%%%%%%




\begin{sagesilent}
a = NonZeroInt(-10,10)
b = NonZeroInt(1,10)
c = RandInt(-20,20)
f=a*exp(b/x) + c
Ans=-b/2
\end{sagesilent}

\latexProblemContent{
\begin{problem}

Find all inflection points for the function $f(x)=\sage{f}$.

\input{2311_Compute_Derivative_0058.HELP.tex}

\[\mbox{Inflection points at\;} x=\answer{\sage{Ans}}\]
\end{problem}}%}
%%%%%%%%%%%%%%%%%%%%%%




\begin{sagesilent}
a = NonZeroInt(-10,10)
b = NonZeroInt(1,10)
c = RandInt(-20,20)
f=a*exp(b/x) + c
Ans=-b/2
\end{sagesilent}

\latexProblemContent{
\begin{problem}

Find all inflection points for the function $f(x)=\sage{f}$.

\input{2311_Compute_Derivative_0058.HELP.tex}

\[\mbox{Inflection points at\;} x=\answer{\sage{Ans}}\]
\end{problem}}%}
%%%%%%%%%%%%%%%%%%%%%%




\begin{sagesilent}
a = NonZeroInt(-10,10)
b = NonZeroInt(1,10)
c = RandInt(-20,20)
f=a*exp(b/x) + c
Ans=-b/2
\end{sagesilent}

\latexProblemContent{
\begin{problem}

Find all inflection points for the function $f(x)=\sage{f}$.

\input{2311_Compute_Derivative_0058.HELP.tex}

\[\mbox{Inflection points at\;} x=\answer{\sage{Ans}}\]
\end{problem}}%}
%%%%%%%%%%%%%%%%%%%%%%




\begin{sagesilent}
a = NonZeroInt(-10,10)
b = NonZeroInt(1,10)
c = RandInt(-20,20)
f=a*exp(b/x) + c
Ans=-b/2
\end{sagesilent}

\latexProblemContent{
\begin{problem}

Find all inflection points for the function $f(x)=\sage{f}$.

\input{2311_Compute_Derivative_0058.HELP.tex}

\[\mbox{Inflection points at\;} x=\answer{\sage{Ans}}\]
\end{problem}}%}
%%%%%%%%%%%%%%%%%%%%%%




\begin{sagesilent}
a = NonZeroInt(-10,10)
b = NonZeroInt(1,10)
c = RandInt(-20,20)
f=a*exp(b/x) + c
Ans=-b/2
\end{sagesilent}

\latexProblemContent{
\begin{problem}

Find all inflection points for the function $f(x)=\sage{f}$.

\input{2311_Compute_Derivative_0058.HELP.tex}

\[\mbox{Inflection points at\;} x=\answer{\sage{Ans}}\]
\end{problem}}%}
%%%%%%%%%%%%%%%%%%%%%%




\begin{sagesilent}
a = NonZeroInt(-10,10)
b = NonZeroInt(1,10)
c = RandInt(-20,20)
f=a*exp(b/x) + c
Ans=-b/2
\end{sagesilent}

\latexProblemContent{
\begin{problem}

Find all inflection points for the function $f(x)=\sage{f}$.

\input{2311_Compute_Derivative_0058.HELP.tex}

\[\mbox{Inflection points at\;} x=\answer{\sage{Ans}}\]
\end{problem}}%}
%%%%%%%%%%%%%%%%%%%%%%




\begin{sagesilent}
a = NonZeroInt(-10,10)
b = NonZeroInt(1,10)
c = RandInt(-20,20)
f=a*exp(b/x) + c
Ans=-b/2
\end{sagesilent}

\latexProblemContent{
\begin{problem}

Find all inflection points for the function $f(x)=\sage{f}$.

\input{2311_Compute_Derivative_0058.HELP.tex}

\[\mbox{Inflection points at\;} x=\answer{\sage{Ans}}\]
\end{problem}}%}
%%%%%%%%%%%%%%%%%%%%%%




\begin{sagesilent}
a = NonZeroInt(-10,10)
b = NonZeroInt(1,10)
c = RandInt(-20,20)
f=a*exp(b/x) + c
Ans=-b/2
\end{sagesilent}

\latexProblemContent{
\begin{problem}

Find all inflection points for the function $f(x)=\sage{f}$.

\input{2311_Compute_Derivative_0058.HELP.tex}

\[\mbox{Inflection points at\;} x=\answer{\sage{Ans}}\]
\end{problem}}%}
%%%%%%%%%%%%%%%%%%%%%%




\begin{sagesilent}
a = NonZeroInt(-10,10)
b = NonZeroInt(1,10)
c = RandInt(-20,20)
f=a*exp(b/x) + c
Ans=-b/2
\end{sagesilent}

\latexProblemContent{
\begin{problem}

Find all inflection points for the function $f(x)=\sage{f}$.

\input{2311_Compute_Derivative_0058.HELP.tex}

\[\mbox{Inflection points at\;} x=\answer{\sage{Ans}}\]
\end{problem}}%}
%%%%%%%%%%%%%%%%%%%%%%




\begin{sagesilent}
a = NonZeroInt(-10,10)
b = NonZeroInt(1,10)
c = RandInt(-20,20)
f=a*exp(b/x) + c
Ans=-b/2
\end{sagesilent}

\latexProblemContent{
\begin{problem}

Find all inflection points for the function $f(x)=\sage{f}$.

\input{2311_Compute_Derivative_0058.HELP.tex}

\[\mbox{Inflection points at\;} x=\answer{\sage{Ans}}\]
\end{problem}}%}
%%%%%%%%%%%%%%%%%%%%%%




\begin{sagesilent}
a = NonZeroInt(-10,10)
b = NonZeroInt(1,10)
c = RandInt(-20,20)
f=a*exp(b/x) + c
Ans=-b/2
\end{sagesilent}

\latexProblemContent{
\begin{problem}

Find all inflection points for the function $f(x)=\sage{f}$.

\input{2311_Compute_Derivative_0058.HELP.tex}

\[\mbox{Inflection points at\;} x=\answer{\sage{Ans}}\]
\end{problem}}%}
%%%%%%%%%%%%%%%%%%%%%%




\begin{sagesilent}
a = NonZeroInt(-10,10)
b = NonZeroInt(1,10)
c = RandInt(-20,20)
f=a*exp(b/x) + c
Ans=-b/2
\end{sagesilent}

\latexProblemContent{
\begin{problem}

Find all inflection points for the function $f(x)=\sage{f}$.

\input{2311_Compute_Derivative_0058.HELP.tex}

\[\mbox{Inflection points at\;} x=\answer{\sage{Ans}}\]
\end{problem}}%}
%%%%%%%%%%%%%%%%%%%%%%




\begin{sagesilent}
a = NonZeroInt(-10,10)
b = NonZeroInt(1,10)
c = RandInt(-20,20)
f=a*exp(b/x) + c
Ans=-b/2
\end{sagesilent}

\latexProblemContent{
\begin{problem}

Find all inflection points for the function $f(x)=\sage{f}$.

\input{2311_Compute_Derivative_0058.HELP.tex}

\[\mbox{Inflection points at\;} x=\answer{\sage{Ans}}\]
\end{problem}}%}
%%%%%%%%%%%%%%%%%%%%%%




\begin{sagesilent}
a = NonZeroInt(-10,10)
b = NonZeroInt(1,10)
c = RandInt(-20,20)
f=a*exp(b/x) + c
Ans=-b/2
\end{sagesilent}

\latexProblemContent{
\begin{problem}

Find all inflection points for the function $f(x)=\sage{f}$.

\input{2311_Compute_Derivative_0058.HELP.tex}

\[\mbox{Inflection points at\;} x=\answer{\sage{Ans}}\]
\end{problem}}%}
%%%%%%%%%%%%%%%%%%%%%%




\begin{sagesilent}
a = NonZeroInt(-10,10)
b = NonZeroInt(1,10)
c = RandInt(-20,20)
f=a*exp(b/x) + c
Ans=-b/2
\end{sagesilent}

\latexProblemContent{
\begin{problem}

Find all inflection points for the function $f(x)=\sage{f}$.

\input{2311_Compute_Derivative_0058.HELP.tex}

\[\mbox{Inflection points at\;} x=\answer{\sage{Ans}}\]
\end{problem}}%}
%%%%%%%%%%%%%%%%%%%%%%




\begin{sagesilent}
a = NonZeroInt(-10,10)
b = NonZeroInt(1,10)
c = RandInt(-20,20)
f=a*exp(b/x) + c
Ans=-b/2
\end{sagesilent}

\latexProblemContent{
\begin{problem}

Find all inflection points for the function $f(x)=\sage{f}$.

\input{2311_Compute_Derivative_0058.HELP.tex}

\[\mbox{Inflection points at\;} x=\answer{\sage{Ans}}\]
\end{problem}}%}
%%%%%%%%%%%%%%%%%%%%%%




\begin{sagesilent}
a = NonZeroInt(-10,10)
b = NonZeroInt(1,10)
c = RandInt(-20,20)
f=a*exp(b/x) + c
Ans=-b/2
\end{sagesilent}

\latexProblemContent{
\begin{problem}

Find all inflection points for the function $f(x)=\sage{f}$.

\input{2311_Compute_Derivative_0058.HELP.tex}

\[\mbox{Inflection points at\;} x=\answer{\sage{Ans}}\]
\end{problem}}%}
%%%%%%%%%%%%%%%%%%%%%%




\begin{sagesilent}
a = NonZeroInt(-10,10)
b = NonZeroInt(1,10)
c = RandInt(-20,20)
f=a*exp(b/x) + c
Ans=-b/2
\end{sagesilent}

\latexProblemContent{
\begin{problem}

Find all inflection points for the function $f(x)=\sage{f}$.

\input{2311_Compute_Derivative_0058.HELP.tex}

\[\mbox{Inflection points at\;} x=\answer{\sage{Ans}}\]
\end{problem}}%}
%%%%%%%%%%%%%%%%%%%%%%




\begin{sagesilent}
a = NonZeroInt(-10,10)
b = NonZeroInt(1,10)
c = RandInt(-20,20)
f=a*exp(b/x) + c
Ans=-b/2
\end{sagesilent}

\latexProblemContent{
\begin{problem}

Find all inflection points for the function $f(x)=\sage{f}$.

\input{2311_Compute_Derivative_0058.HELP.tex}

\[\mbox{Inflection points at\;} x=\answer{\sage{Ans}}\]
\end{problem}}%}
%%%%%%%%%%%%%%%%%%%%%%




\begin{sagesilent}
a = NonZeroInt(-10,10)
b = NonZeroInt(1,10)
c = RandInt(-20,20)
f=a*exp(b/x) + c
Ans=-b/2
\end{sagesilent}

\latexProblemContent{
\begin{problem}

Find all inflection points for the function $f(x)=\sage{f}$.

\input{2311_Compute_Derivative_0058.HELP.tex}

\[\mbox{Inflection points at\;} x=\answer{\sage{Ans}}\]
\end{problem}}%}
%%%%%%%%%%%%%%%%%%%%%%




\begin{sagesilent}
a = NonZeroInt(-10,10)
b = NonZeroInt(1,10)
c = RandInt(-20,20)
f=a*exp(b/x) + c
Ans=-b/2
\end{sagesilent}

\latexProblemContent{
\begin{problem}

Find all inflection points for the function $f(x)=\sage{f}$.

\input{2311_Compute_Derivative_0058.HELP.tex}

\[\mbox{Inflection points at\;} x=\answer{\sage{Ans}}\]
\end{problem}}%}
%%%%%%%%%%%%%%%%%%%%%%




\begin{sagesilent}
a = NonZeroInt(-10,10)
b = NonZeroInt(1,10)
c = RandInt(-20,20)
f=a*exp(b/x) + c
Ans=-b/2
\end{sagesilent}

\latexProblemContent{
\begin{problem}

Find all inflection points for the function $f(x)=\sage{f}$.

\input{2311_Compute_Derivative_0058.HELP.tex}

\[\mbox{Inflection points at\;} x=\answer{\sage{Ans}}\]
\end{problem}}%}
%%%%%%%%%%%%%%%%%%%%%%




\begin{sagesilent}
a = NonZeroInt(-10,10)
b = NonZeroInt(1,10)
c = RandInt(-20,20)
f=a*exp(b/x) + c
Ans=-b/2
\end{sagesilent}

\latexProblemContent{
\begin{problem}

Find all inflection points for the function $f(x)=\sage{f}$.

\input{2311_Compute_Derivative_0058.HELP.tex}

\[\mbox{Inflection points at\;} x=\answer{\sage{Ans}}\]
\end{problem}}%}
%%%%%%%%%%%%%%%%%%%%%%




\begin{sagesilent}
a = NonZeroInt(-10,10)
b = NonZeroInt(1,10)
c = RandInt(-20,20)
f=a*exp(b/x) + c
Ans=-b/2
\end{sagesilent}

\latexProblemContent{
\begin{problem}

Find all inflection points for the function $f(x)=\sage{f}$.

\input{2311_Compute_Derivative_0058.HELP.tex}

\[\mbox{Inflection points at\;} x=\answer{\sage{Ans}}\]
\end{problem}}%}
%%%%%%%%%%%%%%%%%%%%%%




\begin{sagesilent}
a = NonZeroInt(-10,10)
b = NonZeroInt(1,10)
c = RandInt(-20,20)
f=a*exp(b/x) + c
Ans=-b/2
\end{sagesilent}

\latexProblemContent{
\begin{problem}

Find all inflection points for the function $f(x)=\sage{f}$.

\input{2311_Compute_Derivative_0058.HELP.tex}

\[\mbox{Inflection points at\;} x=\answer{\sage{Ans}}\]
\end{problem}}%}
%%%%%%%%%%%%%%%%%%%%%%




\begin{sagesilent}
a = NonZeroInt(-10,10)
b = NonZeroInt(1,10)
c = RandInt(-20,20)
f=a*exp(b/x) + c
Ans=-b/2
\end{sagesilent}

\latexProblemContent{
\begin{problem}

Find all inflection points for the function $f(x)=\sage{f}$.

\input{2311_Compute_Derivative_0058.HELP.tex}

\[\mbox{Inflection points at\;} x=\answer{\sage{Ans}}\]
\end{problem}}%}
%%%%%%%%%%%%%%%%%%%%%%




\begin{sagesilent}
a = NonZeroInt(-10,10)
b = NonZeroInt(1,10)
c = RandInt(-20,20)
f=a*exp(b/x) + c
Ans=-b/2
\end{sagesilent}

\latexProblemContent{
\begin{problem}

Find all inflection points for the function $f(x)=\sage{f}$.

\input{2311_Compute_Derivative_0058.HELP.tex}

\[\mbox{Inflection points at\;} x=\answer{\sage{Ans}}\]
\end{problem}}%}
%%%%%%%%%%%%%%%%%%%%%%




\begin{sagesilent}
a = NonZeroInt(-10,10)
b = NonZeroInt(1,10)
c = RandInt(-20,20)
f=a*exp(b/x) + c
Ans=-b/2
\end{sagesilent}

\latexProblemContent{
\begin{problem}

Find all inflection points for the function $f(x)=\sage{f}$.

\input{2311_Compute_Derivative_0058.HELP.tex}

\[\mbox{Inflection points at\;} x=\answer{\sage{Ans}}\]
\end{problem}}%}
%%%%%%%%%%%%%%%%%%%%%%




\begin{sagesilent}
a = NonZeroInt(-10,10)
b = NonZeroInt(1,10)
c = RandInt(-20,20)
f=a*exp(b/x) + c
Ans=-b/2
\end{sagesilent}

\latexProblemContent{
\begin{problem}

Find all inflection points for the function $f(x)=\sage{f}$.

\input{2311_Compute_Derivative_0058.HELP.tex}

\[\mbox{Inflection points at\;} x=\answer{\sage{Ans}}\]
\end{problem}}%}
%%%%%%%%%%%%%%%%%%%%%%




\begin{sagesilent}
a = NonZeroInt(-10,10)
b = NonZeroInt(1,10)
c = RandInt(-20,20)
f=a*exp(b/x) + c
Ans=-b/2
\end{sagesilent}

\latexProblemContent{
\begin{problem}

Find all inflection points for the function $f(x)=\sage{f}$.

\input{2311_Compute_Derivative_0058.HELP.tex}

\[\mbox{Inflection points at\;} x=\answer{\sage{Ans}}\]
\end{problem}}%}
%%%%%%%%%%%%%%%%%%%%%%




\begin{sagesilent}
a = NonZeroInt(-10,10)
b = NonZeroInt(1,10)
c = RandInt(-20,20)
f=a*exp(b/x) + c
Ans=-b/2
\end{sagesilent}

\latexProblemContent{
\begin{problem}

Find all inflection points for the function $f(x)=\sage{f}$.

\input{2311_Compute_Derivative_0058.HELP.tex}

\[\mbox{Inflection points at\;} x=\answer{\sage{Ans}}\]
\end{problem}}%}
%%%%%%%%%%%%%%%%%%%%%%




\begin{sagesilent}
a = NonZeroInt(-10,10)
b = NonZeroInt(1,10)
c = RandInt(-20,20)
f=a*exp(b/x) + c
Ans=-b/2
\end{sagesilent}

\latexProblemContent{
\begin{problem}

Find all inflection points for the function $f(x)=\sage{f}$.

\input{2311_Compute_Derivative_0058.HELP.tex}

\[\mbox{Inflection points at\;} x=\answer{\sage{Ans}}\]
\end{problem}}%}
%%%%%%%%%%%%%%%%%%%%%%




\begin{sagesilent}
a = NonZeroInt(-10,10)
b = NonZeroInt(1,10)
c = RandInt(-20,20)
f=a*exp(b/x) + c
Ans=-b/2
\end{sagesilent}

\latexProblemContent{
\begin{problem}

Find all inflection points for the function $f(x)=\sage{f}$.

\input{2311_Compute_Derivative_0058.HELP.tex}

\[\mbox{Inflection points at\;} x=\answer{\sage{Ans}}\]
\end{problem}}%}
%%%%%%%%%%%%%%%%%%%%%%




\begin{sagesilent}
a = NonZeroInt(-10,10)
b = NonZeroInt(1,10)
c = RandInt(-20,20)
f=a*exp(b/x) + c
Ans=-b/2
\end{sagesilent}

\latexProblemContent{
\begin{problem}

Find all inflection points for the function $f(x)=\sage{f}$.

\input{2311_Compute_Derivative_0058.HELP.tex}

\[\mbox{Inflection points at\;} x=\answer{\sage{Ans}}\]
\end{problem}}%}
%%%%%%%%%%%%%%%%%%%%%%




\begin{sagesilent}
a = NonZeroInt(-10,10)
b = NonZeroInt(1,10)
c = RandInt(-20,20)
f=a*exp(b/x) + c
Ans=-b/2
\end{sagesilent}

\latexProblemContent{
\begin{problem}

Find all inflection points for the function $f(x)=\sage{f}$.

\input{2311_Compute_Derivative_0058.HELP.tex}

\[\mbox{Inflection points at\;} x=\answer{\sage{Ans}}\]
\end{problem}}%}
%%%%%%%%%%%%%%%%%%%%%%




\begin{sagesilent}
a = NonZeroInt(-10,10)
b = NonZeroInt(1,10)
c = RandInt(-20,20)
f=a*exp(b/x) + c
Ans=-b/2
\end{sagesilent}

\latexProblemContent{
\begin{problem}

Find all inflection points for the function $f(x)=\sage{f}$.

\input{2311_Compute_Derivative_0058.HELP.tex}

\[\mbox{Inflection points at\;} x=\answer{\sage{Ans}}\]
\end{problem}}%}
%%%%%%%%%%%%%%%%%%%%%%




\begin{sagesilent}
a = NonZeroInt(-10,10)
b = NonZeroInt(1,10)
c = RandInt(-20,20)
f=a*exp(b/x) + c
Ans=-b/2
\end{sagesilent}

\latexProblemContent{
\begin{problem}

Find all inflection points for the function $f(x)=\sage{f}$.

\input{2311_Compute_Derivative_0058.HELP.tex}

\[\mbox{Inflection points at\;} x=\answer{\sage{Ans}}\]
\end{problem}}%}
%%%%%%%%%%%%%%%%%%%%%%




\begin{sagesilent}
a = NonZeroInt(-10,10)
b = NonZeroInt(1,10)
c = RandInt(-20,20)
f=a*exp(b/x) + c
Ans=-b/2
\end{sagesilent}

\latexProblemContent{
\begin{problem}

Find all inflection points for the function $f(x)=\sage{f}$.

\input{2311_Compute_Derivative_0058.HELP.tex}

\[\mbox{Inflection points at\;} x=\answer{\sage{Ans}}\]
\end{problem}}%}
%%%%%%%%%%%%%%%%%%%%%%




\begin{sagesilent}
a = NonZeroInt(-10,10)
b = NonZeroInt(1,10)
c = RandInt(-20,20)
f=a*exp(b/x) + c
Ans=-b/2
\end{sagesilent}

\latexProblemContent{
\begin{problem}

Find all inflection points for the function $f(x)=\sage{f}$.

\input{2311_Compute_Derivative_0058.HELP.tex}

\[\mbox{Inflection points at\;} x=\answer{\sage{Ans}}\]
\end{problem}}%}
%%%%%%%%%%%%%%%%%%%%%%




\begin{sagesilent}
a = NonZeroInt(-10,10)
b = NonZeroInt(1,10)
c = RandInt(-20,20)
f=a*exp(b/x) + c
Ans=-b/2
\end{sagesilent}

\latexProblemContent{
\begin{problem}

Find all inflection points for the function $f(x)=\sage{f}$.

\input{2311_Compute_Derivative_0058.HELP.tex}

\[\mbox{Inflection points at\;} x=\answer{\sage{Ans}}\]
\end{problem}}%}
%%%%%%%%%%%%%%%%%%%%%%




\begin{sagesilent}
a = NonZeroInt(-10,10)
b = NonZeroInt(1,10)
c = RandInt(-20,20)
f=a*exp(b/x) + c
Ans=-b/2
\end{sagesilent}

\latexProblemContent{
\begin{problem}

Find all inflection points for the function $f(x)=\sage{f}$.

\input{2311_Compute_Derivative_0058.HELP.tex}

\[\mbox{Inflection points at\;} x=\answer{\sage{Ans}}\]
\end{problem}}%}
%%%%%%%%%%%%%%%%%%%%%%




\begin{sagesilent}
a = NonZeroInt(-10,10)
b = NonZeroInt(1,10)
c = RandInt(-20,20)
f=a*exp(b/x) + c
Ans=-b/2
\end{sagesilent}

\latexProblemContent{
\begin{problem}

Find all inflection points for the function $f(x)=\sage{f}$.

\input{2311_Compute_Derivative_0058.HELP.tex}

\[\mbox{Inflection points at\;} x=\answer{\sage{Ans}}\]
\end{problem}}%}
%%%%%%%%%%%%%%%%%%%%%%




\begin{sagesilent}
a = NonZeroInt(-10,10)
b = NonZeroInt(1,10)
c = RandInt(-20,20)
f=a*exp(b/x) + c
Ans=-b/2
\end{sagesilent}

\latexProblemContent{
\begin{problem}

Find all inflection points for the function $f(x)=\sage{f}$.

\input{2311_Compute_Derivative_0058.HELP.tex}

\[\mbox{Inflection points at\;} x=\answer{\sage{Ans}}\]
\end{problem}}%}
%%%%%%%%%%%%%%%%%%%%%%




\begin{sagesilent}
a = NonZeroInt(-10,10)
b = NonZeroInt(1,10)
c = RandInt(-20,20)
f=a*exp(b/x) + c
Ans=-b/2
\end{sagesilent}

\latexProblemContent{
\begin{problem}

Find all inflection points for the function $f(x)=\sage{f}$.

\input{2311_Compute_Derivative_0058.HELP.tex}

\[\mbox{Inflection points at\;} x=\answer{\sage{Ans}}\]
\end{problem}}%}
%%%%%%%%%%%%%%%%%%%%%%




\begin{sagesilent}
a = NonZeroInt(-10,10)
b = NonZeroInt(1,10)
c = RandInt(-20,20)
f=a*exp(b/x) + c
Ans=-b/2
\end{sagesilent}

\latexProblemContent{
\begin{problem}

Find all inflection points for the function $f(x)=\sage{f}$.

\input{2311_Compute_Derivative_0058.HELP.tex}

\[\mbox{Inflection points at\;} x=\answer{\sage{Ans}}\]
\end{problem}}%}
%%%%%%%%%%%%%%%%%%%%%%




\begin{sagesilent}
a = NonZeroInt(-10,10)
b = NonZeroInt(1,10)
c = RandInt(-20,20)
f=a*exp(b/x) + c
Ans=-b/2
\end{sagesilent}

\latexProblemContent{
\begin{problem}

Find all inflection points for the function $f(x)=\sage{f}$.

\input{2311_Compute_Derivative_0058.HELP.tex}

\[\mbox{Inflection points at\;} x=\answer{\sage{Ans}}\]
\end{problem}}%}
%%%%%%%%%%%%%%%%%%%%%%




\begin{sagesilent}
a = NonZeroInt(-10,10)
b = NonZeroInt(1,10)
c = RandInt(-20,20)
f=a*exp(b/x) + c
Ans=-b/2
\end{sagesilent}

\latexProblemContent{
\begin{problem}

Find all inflection points for the function $f(x)=\sage{f}$.

\input{2311_Compute_Derivative_0058.HELP.tex}

\[\mbox{Inflection points at\;} x=\answer{\sage{Ans}}\]
\end{problem}}%}
%%%%%%%%%%%%%%%%%%%%%%




\begin{sagesilent}
a = NonZeroInt(-10,10)
b = NonZeroInt(1,10)
c = RandInt(-20,20)
f=a*exp(b/x) + c
Ans=-b/2
\end{sagesilent}

\latexProblemContent{
\begin{problem}

Find all inflection points for the function $f(x)=\sage{f}$.

\input{2311_Compute_Derivative_0058.HELP.tex}

\[\mbox{Inflection points at\;} x=\answer{\sage{Ans}}\]
\end{problem}}%}
%%%%%%%%%%%%%%%%%%%%%%




\begin{sagesilent}
a = NonZeroInt(-10,10)
b = NonZeroInt(1,10)
c = RandInt(-20,20)
f=a*exp(b/x) + c
Ans=-b/2
\end{sagesilent}

\latexProblemContent{
\begin{problem}

Find all inflection points for the function $f(x)=\sage{f}$.

\input{2311_Compute_Derivative_0058.HELP.tex}

\[\mbox{Inflection points at\;} x=\answer{\sage{Ans}}\]
\end{problem}}%}
%%%%%%%%%%%%%%%%%%%%%%




\begin{sagesilent}
a = NonZeroInt(-10,10)
b = NonZeroInt(1,10)
c = RandInt(-20,20)
f=a*exp(b/x) + c
Ans=-b/2
\end{sagesilent}

\latexProblemContent{
\begin{problem}

Find all inflection points for the function $f(x)=\sage{f}$.

\input{2311_Compute_Derivative_0058.HELP.tex}

\[\mbox{Inflection points at\;} x=\answer{\sage{Ans}}\]
\end{problem}}%}
%%%%%%%%%%%%%%%%%%%%%%




\begin{sagesilent}
a = NonZeroInt(-10,10)
b = NonZeroInt(1,10)
c = RandInt(-20,20)
f=a*exp(b/x) + c
Ans=-b/2
\end{sagesilent}

\latexProblemContent{
\begin{problem}

Find all inflection points for the function $f(x)=\sage{f}$.

\input{2311_Compute_Derivative_0058.HELP.tex}

\[\mbox{Inflection points at\;} x=\answer{\sage{Ans}}\]
\end{problem}}%}
%%%%%%%%%%%%%%%%%%%%%%




\begin{sagesilent}
a = NonZeroInt(-10,10)
b = NonZeroInt(1,10)
c = RandInt(-20,20)
f=a*exp(b/x) + c
Ans=-b/2
\end{sagesilent}

\latexProblemContent{
\begin{problem}

Find all inflection points for the function $f(x)=\sage{f}$.

\input{2311_Compute_Derivative_0058.HELP.tex}

\[\mbox{Inflection points at\;} x=\answer{\sage{Ans}}\]
\end{problem}}%}
%%%%%%%%%%%%%%%%%%%%%%




\begin{sagesilent}
a = NonZeroInt(-10,10)
b = NonZeroInt(1,10)
c = RandInt(-20,20)
f=a*exp(b/x) + c
Ans=-b/2
\end{sagesilent}

\latexProblemContent{
\begin{problem}

Find all inflection points for the function $f(x)=\sage{f}$.

\input{2311_Compute_Derivative_0058.HELP.tex}

\[\mbox{Inflection points at\;} x=\answer{\sage{Ans}}\]
\end{problem}}%}
%%%%%%%%%%%%%%%%%%%%%%




\begin{sagesilent}
a = NonZeroInt(-10,10)
b = NonZeroInt(1,10)
c = RandInt(-20,20)
f=a*exp(b/x) + c
Ans=-b/2
\end{sagesilent}

\latexProblemContent{
\begin{problem}

Find all inflection points for the function $f(x)=\sage{f}$.

\input{2311_Compute_Derivative_0058.HELP.tex}

\[\mbox{Inflection points at\;} x=\answer{\sage{Ans}}\]
\end{problem}}%}
%%%%%%%%%%%%%%%%%%%%%%




\begin{sagesilent}
a = NonZeroInt(-10,10)
b = NonZeroInt(1,10)
c = RandInt(-20,20)
f=a*exp(b/x) + c
Ans=-b/2
\end{sagesilent}

\latexProblemContent{
\begin{problem}

Find all inflection points for the function $f(x)=\sage{f}$.

\input{2311_Compute_Derivative_0058.HELP.tex}

\[\mbox{Inflection points at\;} x=\answer{\sage{Ans}}\]
\end{problem}}%}
%%%%%%%%%%%%%%%%%%%%%%




\begin{sagesilent}
a = NonZeroInt(-10,10)
b = NonZeroInt(1,10)
c = RandInt(-20,20)
f=a*exp(b/x) + c
Ans=-b/2
\end{sagesilent}

\latexProblemContent{
\begin{problem}

Find all inflection points for the function $f(x)=\sage{f}$.

\input{2311_Compute_Derivative_0058.HELP.tex}

\[\mbox{Inflection points at\;} x=\answer{\sage{Ans}}\]
\end{problem}}%}
%%%%%%%%%%%%%%%%%%%%%%




\begin{sagesilent}
a = NonZeroInt(-10,10)
b = NonZeroInt(1,10)
c = RandInt(-20,20)
f=a*exp(b/x) + c
Ans=-b/2
\end{sagesilent}

\latexProblemContent{
\begin{problem}

Find all inflection points for the function $f(x)=\sage{f}$.

\input{2311_Compute_Derivative_0058.HELP.tex}

\[\mbox{Inflection points at\;} x=\answer{\sage{Ans}}\]
\end{problem}}%}
%%%%%%%%%%%%%%%%%%%%%%




\begin{sagesilent}
a = NonZeroInt(-10,10)
b = NonZeroInt(1,10)
c = RandInt(-20,20)
f=a*exp(b/x) + c
Ans=-b/2
\end{sagesilent}

\latexProblemContent{
\begin{problem}

Find all inflection points for the function $f(x)=\sage{f}$.

\input{2311_Compute_Derivative_0058.HELP.tex}

\[\mbox{Inflection points at\;} x=\answer{\sage{Ans}}\]
\end{problem}}%}
%%%%%%%%%%%%%%%%%%%%%%




\begin{sagesilent}
a = NonZeroInt(-10,10)
b = NonZeroInt(1,10)
c = RandInt(-20,20)
f=a*exp(b/x) + c
Ans=-b/2
\end{sagesilent}

\latexProblemContent{
\begin{problem}

Find all inflection points for the function $f(x)=\sage{f}$.

\input{2311_Compute_Derivative_0058.HELP.tex}

\[\mbox{Inflection points at\;} x=\answer{\sage{Ans}}\]
\end{problem}}%}
%%%%%%%%%%%%%%%%%%%%%%




\begin{sagesilent}
a = NonZeroInt(-10,10)
b = NonZeroInt(1,10)
c = RandInt(-20,20)
f=a*exp(b/x) + c
Ans=-b/2
\end{sagesilent}

\latexProblemContent{
\begin{problem}

Find all inflection points for the function $f(x)=\sage{f}$.

\input{2311_Compute_Derivative_0058.HELP.tex}

\[\mbox{Inflection points at\;} x=\answer{\sage{Ans}}\]
\end{problem}}%}
%%%%%%%%%%%%%%%%%%%%%%




\begin{sagesilent}
a = NonZeroInt(-10,10)
b = NonZeroInt(1,10)
c = RandInt(-20,20)
f=a*exp(b/x) + c
Ans=-b/2
\end{sagesilent}

\latexProblemContent{
\begin{problem}

Find all inflection points for the function $f(x)=\sage{f}$.

\input{2311_Compute_Derivative_0058.HELP.tex}

\[\mbox{Inflection points at\;} x=\answer{\sage{Ans}}\]
\end{problem}}%}
%%%%%%%%%%%%%%%%%%%%%%




\begin{sagesilent}
a = NonZeroInt(-10,10)
b = NonZeroInt(1,10)
c = RandInt(-20,20)
f=a*exp(b/x) + c
Ans=-b/2
\end{sagesilent}

\latexProblemContent{
\begin{problem}

Find all inflection points for the function $f(x)=\sage{f}$.

\input{2311_Compute_Derivative_0058.HELP.tex}

\[\mbox{Inflection points at\;} x=\answer{\sage{Ans}}\]
\end{problem}}%}
%%%%%%%%%%%%%%%%%%%%%%




\begin{sagesilent}
a = NonZeroInt(-10,10)
b = NonZeroInt(1,10)
c = RandInt(-20,20)
f=a*exp(b/x) + c
Ans=-b/2
\end{sagesilent}

\latexProblemContent{
\begin{problem}

Find all inflection points for the function $f(x)=\sage{f}$.

\input{2311_Compute_Derivative_0058.HELP.tex}

\[\mbox{Inflection points at\;} x=\answer{\sage{Ans}}\]
\end{problem}}%}
%%%%%%%%%%%%%%%%%%%%%%




\begin{sagesilent}
a = NonZeroInt(-10,10)
b = NonZeroInt(1,10)
c = RandInt(-20,20)
f=a*exp(b/x) + c
Ans=-b/2
\end{sagesilent}

\latexProblemContent{
\begin{problem}

Find all inflection points for the function $f(x)=\sage{f}$.

\input{2311_Compute_Derivative_0058.HELP.tex}

\[\mbox{Inflection points at\;} x=\answer{\sage{Ans}}\]
\end{problem}}%}
%%%%%%%%%%%%%%%%%%%%%%




\begin{sagesilent}
a = NonZeroInt(-10,10)
b = NonZeroInt(1,10)
c = RandInt(-20,20)
f=a*exp(b/x) + c
Ans=-b/2
\end{sagesilent}

\latexProblemContent{
\begin{problem}

Find all inflection points for the function $f(x)=\sage{f}$.

\input{2311_Compute_Derivative_0058.HELP.tex}

\[\mbox{Inflection points at\;} x=\answer{\sage{Ans}}\]
\end{problem}}%}
%%%%%%%%%%%%%%%%%%%%%%




\begin{sagesilent}
a = NonZeroInt(-10,10)
b = NonZeroInt(1,10)
c = RandInt(-20,20)
f=a*exp(b/x) + c
Ans=-b/2
\end{sagesilent}

\latexProblemContent{
\begin{problem}

Find all inflection points for the function $f(x)=\sage{f}$.

\input{2311_Compute_Derivative_0058.HELP.tex}

\[\mbox{Inflection points at\;} x=\answer{\sage{Ans}}\]
\end{problem}}%}
%%%%%%%%%%%%%%%%%%%%%%




\begin{sagesilent}
a = NonZeroInt(-10,10)
b = NonZeroInt(1,10)
c = RandInt(-20,20)
f=a*exp(b/x) + c
Ans=-b/2
\end{sagesilent}

\latexProblemContent{
\begin{problem}

Find all inflection points for the function $f(x)=\sage{f}$.

\input{2311_Compute_Derivative_0058.HELP.tex}

\[\mbox{Inflection points at\;} x=\answer{\sage{Ans}}\]
\end{problem}}%}
%%%%%%%%%%%%%%%%%%%%%%




\begin{sagesilent}
a = NonZeroInt(-10,10)
b = NonZeroInt(1,10)
c = RandInt(-20,20)
f=a*exp(b/x) + c
Ans=-b/2
\end{sagesilent}

\latexProblemContent{
\begin{problem}

Find all inflection points for the function $f(x)=\sage{f}$.

\input{2311_Compute_Derivative_0058.HELP.tex}

\[\mbox{Inflection points at\;} x=\answer{\sage{Ans}}\]
\end{problem}}%}
%%%%%%%%%%%%%%%%%%%%%%




\begin{sagesilent}
a = NonZeroInt(-10,10)
b = NonZeroInt(1,10)
c = RandInt(-20,20)
f=a*exp(b/x) + c
Ans=-b/2
\end{sagesilent}

\latexProblemContent{
\begin{problem}

Find all inflection points for the function $f(x)=\sage{f}$.

\input{2311_Compute_Derivative_0058.HELP.tex}

\[\mbox{Inflection points at\;} x=\answer{\sage{Ans}}\]
\end{problem}}%}
%%%%%%%%%%%%%%%%%%%%%%




\begin{sagesilent}
a = NonZeroInt(-10,10)
b = NonZeroInt(1,10)
c = RandInt(-20,20)
f=a*exp(b/x) + c
Ans=-b/2
\end{sagesilent}

\latexProblemContent{
\begin{problem}

Find all inflection points for the function $f(x)=\sage{f}$.

\input{2311_Compute_Derivative_0058.HELP.tex}

\[\mbox{Inflection points at\;} x=\answer{\sage{Ans}}\]
\end{problem}}%}
%%%%%%%%%%%%%%%%%%%%%%




\begin{sagesilent}
a = NonZeroInt(-10,10)
b = NonZeroInt(1,10)
c = RandInt(-20,20)
f=a*exp(b/x) + c
Ans=-b/2
\end{sagesilent}

\latexProblemContent{
\begin{problem}

Find all inflection points for the function $f(x)=\sage{f}$.

\input{2311_Compute_Derivative_0058.HELP.tex}

\[\mbox{Inflection points at\;} x=\answer{\sage{Ans}}\]
\end{problem}}%}
%%%%%%%%%%%%%%%%%%%%%%




\begin{sagesilent}
a = NonZeroInt(-10,10)
b = NonZeroInt(1,10)
c = RandInt(-20,20)
f=a*exp(b/x) + c
Ans=-b/2
\end{sagesilent}

\latexProblemContent{
\begin{problem}

Find all inflection points for the function $f(x)=\sage{f}$.

\input{2311_Compute_Derivative_0058.HELP.tex}

\[\mbox{Inflection points at\;} x=\answer{\sage{Ans}}\]
\end{problem}}%}
%%%%%%%%%%%%%%%%%%%%%%




\begin{sagesilent}
a = NonZeroInt(-10,10)
b = NonZeroInt(1,10)
c = RandInt(-20,20)
f=a*exp(b/x) + c
Ans=-b/2
\end{sagesilent}

\latexProblemContent{
\begin{problem}

Find all inflection points for the function $f(x)=\sage{f}$.

\input{2311_Compute_Derivative_0058.HELP.tex}

\[\mbox{Inflection points at\;} x=\answer{\sage{Ans}}\]
\end{problem}}%}
%%%%%%%%%%%%%%%%%%%%%%




\begin{sagesilent}
a = NonZeroInt(-10,10)
b = NonZeroInt(1,10)
c = RandInt(-20,20)
f=a*exp(b/x) + c
Ans=-b/2
\end{sagesilent}

\latexProblemContent{
\begin{problem}

Find all inflection points for the function $f(x)=\sage{f}$.

\input{2311_Compute_Derivative_0058.HELP.tex}

\[\mbox{Inflection points at\;} x=\answer{\sage{Ans}}\]
\end{problem}}%}
%%%%%%%%%%%%%%%%%%%%%%




\begin{sagesilent}
a = NonZeroInt(-10,10)
b = NonZeroInt(1,10)
c = RandInt(-20,20)
f=a*exp(b/x) + c
Ans=-b/2
\end{sagesilent}

\latexProblemContent{
\begin{problem}

Find all inflection points for the function $f(x)=\sage{f}$.

\input{2311_Compute_Derivative_0058.HELP.tex}

\[\mbox{Inflection points at\;} x=\answer{\sage{Ans}}\]
\end{problem}}%}
%%%%%%%%%%%%%%%%%%%%%%




\begin{sagesilent}
a = NonZeroInt(-10,10)
b = NonZeroInt(1,10)
c = RandInt(-20,20)
f=a*exp(b/x) + c
Ans=-b/2
\end{sagesilent}

\latexProblemContent{
\begin{problem}

Find all inflection points for the function $f(x)=\sage{f}$.

\input{2311_Compute_Derivative_0058.HELP.tex}

\[\mbox{Inflection points at\;} x=\answer{\sage{Ans}}\]
\end{problem}}%}
%%%%%%%%%%%%%%%%%%%%%%




\begin{sagesilent}
a = NonZeroInt(-10,10)
b = NonZeroInt(1,10)
c = RandInt(-20,20)
f=a*exp(b/x) + c
Ans=-b/2
\end{sagesilent}

\latexProblemContent{
\begin{problem}

Find all inflection points for the function $f(x)=\sage{f}$.

\input{2311_Compute_Derivative_0058.HELP.tex}

\[\mbox{Inflection points at\;} x=\answer{\sage{Ans}}\]
\end{problem}}%}
%%%%%%%%%%%%%%%%%%%%%%




\begin{sagesilent}
a = NonZeroInt(-10,10)
b = NonZeroInt(1,10)
c = RandInt(-20,20)
f=a*exp(b/x) + c
Ans=-b/2
\end{sagesilent}

\latexProblemContent{
\begin{problem}

Find all inflection points for the function $f(x)=\sage{f}$.

\input{2311_Compute_Derivative_0058.HELP.tex}

\[\mbox{Inflection points at\;} x=\answer{\sage{Ans}}\]
\end{problem}}%}
%%%%%%%%%%%%%%%%%%%%%%




\begin{sagesilent}
a = NonZeroInt(-10,10)
b = NonZeroInt(1,10)
c = RandInt(-20,20)
f=a*exp(b/x) + c
Ans=-b/2
\end{sagesilent}

\latexProblemContent{
\begin{problem}

Find all inflection points for the function $f(x)=\sage{f}$.

\input{2311_Compute_Derivative_0058.HELP.tex}

\[\mbox{Inflection points at\;} x=\answer{\sage{Ans}}\]
\end{problem}}%}
%%%%%%%%%%%%%%%%%%%%%%




\begin{sagesilent}
a = NonZeroInt(-10,10)
b = NonZeroInt(1,10)
c = RandInt(-20,20)
f=a*exp(b/x) + c
Ans=-b/2
\end{sagesilent}

\latexProblemContent{
\begin{problem}

Find all inflection points for the function $f(x)=\sage{f}$.

\input{2311_Compute_Derivative_0058.HELP.tex}

\[\mbox{Inflection points at\;} x=\answer{\sage{Ans}}\]
\end{problem}}%}
%%%%%%%%%%%%%%%%%%%%%%




\begin{sagesilent}
a = NonZeroInt(-10,10)
b = NonZeroInt(1,10)
c = RandInt(-20,20)
f=a*exp(b/x) + c
Ans=-b/2
\end{sagesilent}

\latexProblemContent{
\begin{problem}

Find all inflection points for the function $f(x)=\sage{f}$.

\input{2311_Compute_Derivative_0058.HELP.tex}

\[\mbox{Inflection points at\;} x=\answer{\sage{Ans}}\]
\end{problem}}%}
%%%%%%%%%%%%%%%%%%%%%%




\begin{sagesilent}
a = NonZeroInt(-10,10)
b = NonZeroInt(1,10)
c = RandInt(-20,20)
f=a*exp(b/x) + c
Ans=-b/2
\end{sagesilent}

\latexProblemContent{
\begin{problem}

Find all inflection points for the function $f(x)=\sage{f}$.

\input{2311_Compute_Derivative_0058.HELP.tex}

\[\mbox{Inflection points at\;} x=\answer{\sage{Ans}}\]
\end{problem}}%}
%%%%%%%%%%%%%%%%%%%%%%




\begin{sagesilent}
a = NonZeroInt(-10,10)
b = NonZeroInt(1,10)
c = RandInt(-20,20)
f=a*exp(b/x) + c
Ans=-b/2
\end{sagesilent}

\latexProblemContent{
\begin{problem}

Find all inflection points for the function $f(x)=\sage{f}$.

\input{2311_Compute_Derivative_0058.HELP.tex}

\[\mbox{Inflection points at\;} x=\answer{\sage{Ans}}\]
\end{problem}}%}
%%%%%%%%%%%%%%%%%%%%%%




\begin{sagesilent}
a = NonZeroInt(-10,10)
b = NonZeroInt(1,10)
c = RandInt(-20,20)
f=a*exp(b/x) + c
Ans=-b/2
\end{sagesilent}

\latexProblemContent{
\begin{problem}

Find all inflection points for the function $f(x)=\sage{f}$.

\input{2311_Compute_Derivative_0058.HELP.tex}

\[\mbox{Inflection points at\;} x=\answer{\sage{Ans}}\]
\end{problem}}%}
%%%%%%%%%%%%%%%%%%%%%%




\begin{sagesilent}
a = NonZeroInt(-10,10)
b = NonZeroInt(1,10)
c = RandInt(-20,20)
f=a*exp(b/x) + c
Ans=-b/2
\end{sagesilent}

\latexProblemContent{
\begin{problem}

Find all inflection points for the function $f(x)=\sage{f}$.

\input{2311_Compute_Derivative_0058.HELP.tex}

\[\mbox{Inflection points at\;} x=\answer{\sage{Ans}}\]
\end{problem}}%}
%%%%%%%%%%%%%%%%%%%%%%




\begin{sagesilent}
a = NonZeroInt(-10,10)
b = NonZeroInt(1,10)
c = RandInt(-20,20)
f=a*exp(b/x) + c
Ans=-b/2
\end{sagesilent}

\latexProblemContent{
\begin{problem}

Find all inflection points for the function $f(x)=\sage{f}$.

\input{2311_Compute_Derivative_0058.HELP.tex}

\[\mbox{Inflection points at\;} x=\answer{\sage{Ans}}\]
\end{problem}}%}
%%%%%%%%%%%%%%%%%%%%%%




\begin{sagesilent}
a = NonZeroInt(-10,10)
b = NonZeroInt(1,10)
c = RandInt(-20,20)
f=a*exp(b/x) + c
Ans=-b/2
\end{sagesilent}

\latexProblemContent{
\begin{problem}

Find all inflection points for the function $f(x)=\sage{f}$.

\input{2311_Compute_Derivative_0058.HELP.tex}

\[\mbox{Inflection points at\;} x=\answer{\sage{Ans}}\]
\end{problem}}%}
%%%%%%%%%%%%%%%%%%%%%%




\begin{sagesilent}
a = NonZeroInt(-10,10)
b = NonZeroInt(1,10)
c = RandInt(-20,20)
f=a*exp(b/x) + c
Ans=-b/2
\end{sagesilent}

\latexProblemContent{
\begin{problem}

Find all inflection points for the function $f(x)=\sage{f}$.

\input{2311_Compute_Derivative_0058.HELP.tex}

\[\mbox{Inflection points at\;} x=\answer{\sage{Ans}}\]
\end{problem}}%}
%%%%%%%%%%%%%%%%%%%%%%




\begin{sagesilent}
a = NonZeroInt(-10,10)
b = NonZeroInt(1,10)
c = RandInt(-20,20)
f=a*exp(b/x) + c
Ans=-b/2
\end{sagesilent}

\latexProblemContent{
\begin{problem}

Find all inflection points for the function $f(x)=\sage{f}$.

\input{2311_Compute_Derivative_0058.HELP.tex}

\[\mbox{Inflection points at\;} x=\answer{\sage{Ans}}\]
\end{problem}}%}
%%%%%%%%%%%%%%%%%%%%%%




\begin{sagesilent}
a = NonZeroInt(-10,10)
b = NonZeroInt(1,10)
c = RandInt(-20,20)
f=a*exp(b/x) + c
Ans=-b/2
\end{sagesilent}

\latexProblemContent{
\begin{problem}

Find all inflection points for the function $f(x)=\sage{f}$.

\input{2311_Compute_Derivative_0058.HELP.tex}

\[\mbox{Inflection points at\;} x=\answer{\sage{Ans}}\]
\end{problem}}%}
%%%%%%%%%%%%%%%%%%%%%%




\begin{sagesilent}
a = NonZeroInt(-10,10)
b = NonZeroInt(1,10)
c = RandInt(-20,20)
f=a*exp(b/x) + c
Ans=-b/2
\end{sagesilent}

\latexProblemContent{
\begin{problem}

Find all inflection points for the function $f(x)=\sage{f}$.

\input{2311_Compute_Derivative_0058.HELP.tex}

\[\mbox{Inflection points at\;} x=\answer{\sage{Ans}}\]
\end{problem}}%}
%%%%%%%%%%%%%%%%%%%%%%




\begin{sagesilent}
a = NonZeroInt(-10,10)
b = NonZeroInt(1,10)
c = RandInt(-20,20)
f=a*exp(b/x) + c
Ans=-b/2
\end{sagesilent}

\latexProblemContent{
\begin{problem}

Find all inflection points for the function $f(x)=\sage{f}$.

\input{2311_Compute_Derivative_0058.HELP.tex}

\[\mbox{Inflection points at\;} x=\answer{\sage{Ans}}\]
\end{problem}}%}
%%%%%%%%%%%%%%%%%%%%%%




\begin{sagesilent}
a = NonZeroInt(-10,10)
b = NonZeroInt(1,10)
c = RandInt(-20,20)
f=a*exp(b/x) + c
Ans=-b/2
\end{sagesilent}

\latexProblemContent{
\begin{problem}

Find all inflection points for the function $f(x)=\sage{f}$.

\input{2311_Compute_Derivative_0058.HELP.tex}

\[\mbox{Inflection points at\;} x=\answer{\sage{Ans}}\]
\end{problem}}%}
%%%%%%%%%%%%%%%%%%%%%%




\begin{sagesilent}
a = NonZeroInt(-10,10)
b = NonZeroInt(1,10)
c = RandInt(-20,20)
f=a*exp(b/x) + c
Ans=-b/2
\end{sagesilent}

\latexProblemContent{
\begin{problem}

Find all inflection points for the function $f(x)=\sage{f}$.

\input{2311_Compute_Derivative_0058.HELP.tex}

\[\mbox{Inflection points at\;} x=\answer{\sage{Ans}}\]
\end{problem}}%}
%%%%%%%%%%%%%%%%%%%%%%




\begin{sagesilent}
a = NonZeroInt(-10,10)
b = NonZeroInt(1,10)
c = RandInt(-20,20)
f=a*exp(b/x) + c
Ans=-b/2
\end{sagesilent}

\latexProblemContent{
\begin{problem}

Find all inflection points for the function $f(x)=\sage{f}$.

\input{2311_Compute_Derivative_0058.HELP.tex}

\[\mbox{Inflection points at\;} x=\answer{\sage{Ans}}\]
\end{problem}}%}
%%%%%%%%%%%%%%%%%%%%%%




\begin{sagesilent}
a = NonZeroInt(-10,10)
b = NonZeroInt(1,10)
c = RandInt(-20,20)
f=a*exp(b/x) + c
Ans=-b/2
\end{sagesilent}

\latexProblemContent{
\begin{problem}

Find all inflection points for the function $f(x)=\sage{f}$.

\input{2311_Compute_Derivative_0058.HELP.tex}

\[\mbox{Inflection points at\;} x=\answer{\sage{Ans}}\]
\end{problem}}%}
%%%%%%%%%%%%%%%%%%%%%%




\begin{sagesilent}
a = NonZeroInt(-10,10)
b = NonZeroInt(1,10)
c = RandInt(-20,20)
f=a*exp(b/x) + c
Ans=-b/2
\end{sagesilent}

\latexProblemContent{
\begin{problem}

Find all inflection points for the function $f(x)=\sage{f}$.

\input{2311_Compute_Derivative_0058.HELP.tex}

\[\mbox{Inflection points at\;} x=\answer{\sage{Ans}}\]
\end{problem}}%}
%%%%%%%%%%%%%%%%%%%%%%




%%%%%%%% 3 Concept   57 Compute  %%%%%%%%%%%


























