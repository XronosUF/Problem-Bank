%%%%%%  This is the raw list of questions before processing %%%%%%


%%%%%%%%%%%%%%%%%%%%%%%%%%
%%%%%%%%%%%%%%%%%%%%%%%%%%


%%%%%%%%%%%%%%%%%%%%%%%
%%\tagged{Cat@One, Cat@Two, Cat@Three, Cat@Four, Cat@Five, Ans@ShortAns, Type@Compute, Topic@Integral, Sub@Poly, Sub@Riemann}{
\begin{sagesilent}
# Build the Interval
a = NonZeroInt(-8,8) 				# Left end point
n = NonZeroInt(3,7)					# Number of rectangles
b = a + n				# Fix right endpoint based on left endpoint and number of rec.
M = max( abs(a), abs(b)) 			# Max val useful for some functions
m = min( abs(a), abs(b))			# Min val useful for some functions

# Build the function
v = [ x - a, x^2 - m^2, -x^2 + M^2, b - x]# Make a vector of functions
p = RandInt(0,3)					# Select a function
f = v[p]							# Assign function

v2 = ['right', 'left']				# Vector of endpoints. (Note, we can add more!)
p2 = RandInt(0,1)					# Pick Endpoint
Endpoint = v2[p2]					#Initialize Endpoint

# Calculate the area
Area = 0							# Initialize Area
for j in range(n):					# Note range(n) = 0, 1, 2, ..., n-1.
    if p2 == 0:						# Right Endpoint case
        Xpt = j + 1					# Shift to right x anchor point by adding width
    if p2 == 1:						# Left Endpoint case
        Xpt = j						# Use left x anchor point.
# Case not implemented    if p2 == 2:						# Midpoint case
# Case not implemented        Xpt = j + 0.5				# Shift half a width
    Area = Area + f(a+Xpt)			# Exploit that rectangles are width 1
   
Ans=Area
\end{sagesilent}

\latexProblemContent{
\begin{problem}

Estimate the area under the graph of $f(x)=\sage{f}$ from $x=\sage{a}$ to $x=\sage{b}$ using $\sage{n}$ rectangles and $\sage{Endpoint}$ endpoints.

\input{2311_Compute_Integral_0001.HELP.tex}

\[\mbox{Area}\approx\answer{\sage{Ans}}\]
\end{problem}}%}
%%%%%%%%%%%%%%%%%%%%%%


%%%%%%%%%%%%%%%%%%%%%%%
%%\tagged{Cat@One, Cat@Two, Cat@Three, Cat@Four, Cat@Five, Ans@ShortAns, Type@Compute, Topic@Integral, Sub@Poly, Sub@Riemann}{
\begin{sagesilent}
# Build the Interval
Parity = (-1)^RandInt(0,1)			# Determine if you are on negative or pos side.
LeftEnd = NonZeroInt(1,3)			# Left end point
n = NonZeroInt(3,5)					# Number of rectangles
RightEnd = LeftEnd + n				# Determine Right Endpoint
Delx = (RightEnd - LeftEnd)/n		# Width of each rectangle.

# Build the function
v = [ 1/(x), 1/(x^2), 1/(-x^2), 1/(x)]# Make a vector of functions
p = RandInt(0,3)					# Select a function
f = v[p]							# Assign function

v2 = ['right', 'left']				# Vector of endpoints. (Note, we can add more!)
p2 = RandInt(0,1)					# Pick Endpoint
Endpoint = v2[p2]					# Initialize Endpoint

# Calculate the area
Area = 0							# Initialize Area
for j in range(n):					# Note range(n) = 0, 1, 2, ..., n-1.
    if p2 == 0:						# Right Endpoint case
        Xpt = j + 1					# Shift to right x anchor point by adding width
    if p2 == 1:						# Left Endpoint case
        Xpt = j						# Use left x anchor point.
# Case not implemented    if p2 == 2:					# Midpoint case
# Case not implemented        Xpt = j + 0.5				# Shift half a width
    Area = Area + f(LeftEnd+(Delx*Xpt))	# Exploit that rectangles are width 1
   
Ans=abs(Area)
\end{sagesilent}

\latexProblemContent{
\begin{problem}

Estimate the area under the graph of $f(x)=\sage{f}$ from $x=\sage{LeftEnd}$ to $x=\sage{RightEnd}$ using $\sage{n}$ rectangles and $\sage{Endpoint}$ endpoints.

\input{2311_Compute_Integral_0002.HELP.tex}

\[\mbox{Area}\approx\answer{\sage{Ans}}\]
\end{problem}}%}
%%%%%%%%%%%%%%%%%%%%%%




%%%%%%%%%%%%%%%%%%%%%%%
%%\tagged{Cat@One, Cat@Two, Cat@Three, Cat@Four, Cat@Five, Ans@ShortAns, Type@Compute, Topic@Integral, Sub@Rational, Sub@Riemann}{
\begin{sagesilent}
# Build the Interval
a = NonZeroInt(0,8) 				# Left end point
n = NonZeroInt(3,7)					# Number of rectangles
b = a + n				# Fix right endpoint based on left endpoint and number of rec.
M = max( abs(a), abs(b)) 			# Max val useful for some functions
m = min( abs(a), abs(b))			# Min val useful for some functions

# Build the function
v = [ sqrt(x - a), sqrt(x^2 - m^2), sqrt(-x^2 + M^2), sqrt(b - x)]# Make a vector of functions
p = RandInt(0,3)					# Select a function
f = v[p]							# Assign function

v2 = ['right', 'left']				# Vector of endpoints. (Note, we can add more!)
p2 = RandInt(0,1)					# Pick Endpoint
Endpoint = v2[p2]					# Initialize Endpoint

# Calculate the area
Area = 0							# Initialize Area
for j in range(n):					# Note range(n) = 0, 1, 2, ..., n-1.
    if p2 == 0:						# Right Endpoint case
        Xpt = j + 1					# Shift to right x anchor point by adding width
    if p2 == 1:						# Left Endpoint case
        Xpt = j						# Use left x anchor point.
# Case not implemented    if p2 == 2:					# Midpoint case
# Case not implemented        Xpt = j + 0.5				# Shift half a width
    Area = Area + f(a+(Xpt))	# Exploit that rectangles are width 1
   
Ans=abs(Area)
\end{sagesilent}

\latexProblemContent{
\begin{problem}

Estimate the area under the graph of $f(x)=\sage{f}$ from $x=\sage{a}$ to $x=\sage{b}$ using $\sage{n}$ rectangles and $\sage{Endpoint}$ endpoints.

\input{2311_Compute_Integral_0003.HELP.tex}

\[\mbox{Area}\approx\answer{\sage{Ans}}\]
\end{problem}}%}
%%%%%%%%%%%%%%%%%%%%%%

%%%%%%%% 10 Compute (Riemann Sums problems ^) %%%%%%%%%%%%%



%%%%%%%%%%%%%%%%%%%%%%%
%%\tagged{Cat@One, Cat@Two, Cat@Three, Cat@Four, Cat@Five, Ans@ShortAns, Type@Compute, Topic@Integral, Sub@Definite}{
\begin{sagesilent}
# Build Interval
a = NonZeroInt(-10,10)
Width = RandInt(3,10)
b = a + Width
r = RandInt(1, 5)

# Build Function
v = [abs(x), x, sqrt(r^2 - x^2)]
p = RandInt(0,2)
f = v[p]

if p==2:
   Area = (1/2)*pi*r^2
   a = -r
   b = r
else:
   c1 = NonZeroInt(-5,5)
   c2 = RandInt(-5,5)
   f(x) = f(c1*x + c2)
   g(x) = abs(f(x))
   Area = integral(g(x), x, a, b)

\end{sagesilent}

\latexProblemContent{
\begin{problem}

Evaluate the definite integral by interpreting it in terms of areas.  

\input{2311_Compute_Integral_004.HELP.tex}

\[\int_{\sage{a}}^{\sage{b}} \sage{f(x)}\;dx= \answer{\sage{Area}}\]
\end{problem}}%}
%%%%%%%%%%%%%%%%%%%%%%




%%%%%%%%%%%%%%%%%%%%%%%
%%\tagged{Cat@One, Cat@Two, Cat@Three, Cat@Four, Cat@Five, Ans@ShortAns, Type@Compute, Topic@Integral, Sub@Definite, \Sub@Piecewise}{
\begin{sagesilent}
a = NonZeroInt(-5,5)
b=Integer(randint(a+3,a+8))
c = NonZeroInt(1,5)
m1=NonZeroInt(1,4)
m2=NonZeroInt(1,4)
d=(-2*m1*a-2*c)/(b-a)
l=Integer(randint(a-5,a))
u=Integer(randint(a+1,b+5))

f1=m1*x+c
f2=d*(x-a)+(m1*a+c)
f3=m2*(x-b)-(m1*a+c)
f = piecewise([([-20,a],f1), ((a,b), f2), ([b,20], f3)])

Ans=integrate(f,x,l,u)
\end{sagesilent}

\latexProblemContent{
\begin{problem}

Given the piecewise function 
\[f(x)=\left\{\begin{array}{ll}
\sage{f1}\; , & x\leq\sage{a}\\[3pt]
\sage{f2}\; , & \sage{a}< x< \sage{b}\\[3pt]
\sage{f3}\; , & x\geq\sage{b}
\end{array} \right.\]
evaluate the following definite integral by interpreting it in terms of areas.  

\input{2311_Compute_Integral_0005.HELP.tex}

\[\int_{\sage{l}}^{\sage{u}} f(x)\;dx= \answer{\sage{Ans}}\]  
\end{problem}}%}
%%%%%%%%%%%%%%%%%%%%%%


%%%%%%%%%%%%%%%%%%%%%%%
%%\tagged{Cat@One, Cat@Two, Cat@Three, Cat@Four, Cat@Five, Ans@ShortAns, Type@Compute, Topic@Integral, Sub@Definite, Sub@Theorems_FTC}{
\begin{sagesilent}
a = NonZeroInt(1,5)
b = NonZeroInt(-10,10)
c = NonZeroInt(1,5)

p=Integer(randint(0,11))
v=[sqrt(x-b), log(x-c), exp(x-b), (x-b)^2, (x-b)^3, (x-b)^4, (x-b), sin(x-b), cos(x-b), 1/(x-b), 1/(x-b)^2, 1/(x-b)^3]
f=v[p]
Ans=f(t)
\end{sagesilent}

\latexProblemContent{
\begin{problem}

Use the Fundamental Theorem of Calculus to find the derivative of the function.
\[g(t)=\int_{\sage{a}}^{t} \sage{f}\;dx\]

\input{2311_Compute_Integral_0006.HELP.tex}

\[\dfrac{d}{dt}(g(t))=\answer{\sage{Ans}}\]
\end{problem}}%}
%%%%%%%%%%%%%%%%%%%%%%

%%%%%%%%%%%%%%%%%%%%%%%
%%\tagged{Cat@One, Cat@Two, Cat@Three, Cat@Four, Cat@Five, Ans@ShortAns, Type@Compute, Topic@Integral, Sub@Definite, Sub@Theorems_FTC}{
\begin{sagesilent}
a = NonZeroInt(1,5)
b = NonZeroInt(-10,10)
c = NonZeroInt(1,5)

p=Integer(randint(0,11))
q=Integer(randint(0,11))
v1=[sqrt(x-b), log(x-c), exp(x-b), (x-b)^2, (x-b)^3, (x-b)^4, (x-b), sin(x-b), cos(x-b), 1/(x-b), 1/(x-b)^2, 1/(x-b)^3]
v2=[sqrt(x), log(x), exp(x), (x)^2, (x)^3, (x)^4, (x-b), sin(x), cos(x), 1/(x), 1/(x)^2, 1/(x)^3]
f=v1[p]
g=v2[p]
h=f*g
Ans=f(t)*g(t)
\end{sagesilent}

\latexProblemContent{
\begin{problem}

Use the Fundamental Theorem of Calculus to find the derivative of the function.
\[g(t)=\int_{\sage{a}}^{t} \sage{h}\;dx\]

\input{2311_Compute_Integral_0007.HELP.tex}

\[\dfrac{d}{dt}(g(t))=\answer{\sage{Ans}}\]
\end{problem}}%}
%%%%%%%%%%%%%%%%%%%%%%


%%%%%%%%%%%%%%%%%%%%%%%
%%\tagged{Cat@One, Cat@Two, Cat@Three, Cat@Four, Cat@Five, Ans@ShortAns, Type@Compute, Topic@Integral, Sub@Definite, Sub@Theorems_FTC}{
\begin{sagesilent}
a = NonZeroInt(1,5)
b = NonZeroInt(-10,10)
c = NonZeroInt(1,5)

p=Integer(randint(0,11))
q=Integer(randint(0,11))
v1=[sqrt(x-b), log(x-c), exp(x-b), (x-b)^2, (x-b)^3, (x-b)^4, (x-b), sin(x-b), cos(x-b), 1/(x-b), 1/(x-b)^2, 1/(x-b)^3]
v2=[sqrt(x), log(x), exp(x), (x)^2, (x)^3, (x)^4, (x-b), sin(x), cos(x), 1/(x), 1/(x)^2, 1/(x)^3]
f=v1[p]
g=v2[p]
h=g(f)
Ans=g(f(t))
\end{sagesilent}

\latexProblemContent{
\begin{problem}

Use the Fundamental Theorem of Calculus to find the derivative of the function.
\[g(t)=\int_{\sage{a}}^{t} \sage{h}\;dx\]

\input{2311_Compute_Integral_0008.HELP.tex}

\[\dfrac{d}{dt}(g(t))=\answer{\sage{Ans}}\]
\end{problem}}%}
%%%%%%%%%%%%%%%%%%%%%%

%%%%%%%%%%%%%%%%%%%%%%%
%%\tagged{Cat@One, Cat@Two, Cat@Three, Cat@Four, Cat@Five, Ans@ShortAns, Type@Compute, Topic@Integral, Sub@Definite, Sub@Theorems_FTC, Sub@Chain_Rule}{
\begin{sagesilent}
a = NonZeroInt(1,5)
b = NonZeroInt(-10,10)
c = NonZeroInt(1,5)

p=Integer(randint(0,11))
q=Integer(randint(0,11))
v1=[sqrt(x-b), log(x-c), exp(x-b), (x-b)^2, (x-b)^3, (x-b)^4, (x-b), sin(x-b), cos(x-b), 1/(x-b), 1/(x-b)^2, 1/(x-b)^3]
v2=[sqrt(x), log(x), exp(x), (x)^2, (x)^3, (x)^4, (x-b), sin(x), cos(x), 1/(x), 1/(x)^2, 1/(x)^3]
f=v1[p]
g=v2[p]

Ans=(diff(g,x))(t)*f(t)

\end{sagesilent}

\latexProblemContent{
\begin{problem}

Use the Fundamental Theorem of Calculus to find the derivative of the function.
\[g(t)=\int_{\sage{a}}^{\sage{g(t)}} \sage{f}\;dx\]

\input{2311_Compute_Integral_0009.HELP.tex}

\[\dfrac{d}{dt}(g(t))=\answer{\sage{Ans}}\]
\end{problem}}%}
%%%%%%%%%%%%%%%%%%%%%%

%%%%%%%%%%%% 18 Compute %%%%%%%%%%%%

%%%%%%%%%%%%%%%%%%%%%%%
%%\tagged{Cat@One, Cat@Two, Cat@Three, Cat@Four, Cat@Five, Ans@ShortAns, Type@Compute, Topic@Integral, Sub@Definite, Sub@Theorems_FTC, Sub@Poly}{
\begin{sagesilent}
a = NonZeroInt(-5,5)
b = Integer(randint(-8,8))
c = NonZeroInt(1,5)
l=Integer(randint(-10,5))
u=Integer(randint(l,12))

p=Integer(randint(0,4))
vpoly=[(x-a), expand((x-a)^2), expand((x-a)^3), expand((x-a)*(x-b)), expand((x-a)^2*(x-b))]
F=vpoly[p]
Ans=integrate(F,x,l,u)
\end{sagesilent}

\latexProblemContent{
\begin{problem}

Use the Fundamental Theorem of Calculus to evaluate the integral.

\input{2311_Compute_Integral_0010.HELP.tex}

\[\int_{\sage{l}}^{\sage{u}} \sage{F}\;dx=\answer{\sage{Ans}}\]
\end{problem}}%}
%%%%%%%%%%%%%%%%%%%%%%

%%%%%%%%%%%%%%%%%%%%%%%
%%\tagged{Cat@One, Cat@Two, Cat@Three, Cat@Four, Cat@Five, Ans@ShortAns, Type@Compute, Topic@Integral, Sub@Definite, Sub@Theorems_FTC, Sub@Poly}{
\begin{sagesilent}
a = NonZeroInt(-5,5)
b = Integer(randint(-8,8))
c = NonZeroInt(1,5)
l=Integer(randint(-10,5))
u=Integer(randint(l,12))

F=(x-a)/(b*sqrt(x))
Ans=integrate(F,x,l,u)
\end{sagesilent}a-3

\latexProblemContent{
\begin{problem}

Use the Fundamental Theorem of Calculus to evaluate the integral.

\input{2311_Compute_Integral_0011.HELP.tex}

\[\int_{\sage{l}}^{\sage{u}} \sage{F}\;dx=\answer{\sage{Ans}}\]
\end{problem}}%}
%%%%%%%%%%%%%%%%%%%%%%

%%%%%%%%%%%%%%%%%%%%%%%
%%\tagged{Cat@One, Cat@Two, Cat@Three, Cat@Four, Cat@Five, Ans@ShortAns, Type@Compute, Topic@Integral, Sub@Definite, Sub@Theorems_FTC, Sub@Trig}{
\begin{sagesilent} 
b = NonZeroInt(-10,10)
l = RandAng(0,pi)
u = RandAng(l,2*pi)
p=Integer(randint(0,1))
vtrig=[b*sin(x), b*cos(x)]

F=vtrig[p]
Ans=integrate(F,x,l,u)
\end{sagesilent}

\latexProblemContent{
\begin{problem}

Use the Fundamental Theorem of Calculus to evaluate the integral.

\input{2311_Compute_Integral_0012.HELP.tex}

\[\int_{\sage{l}}^{\sage{u}} \sage{F}\;dx=\answer{\sage{Ans}}\]
\end{problem}}%}
%%%%%%%%%%%%%%%%%%%%%%

%%%%%%%%%%%%%%%%%%%%%%%
%%\tagged{Cat@One, Cat@Two, Cat@Three, Cat@Four, Cat@Five, Ans@ShortAns, Type@Compute, Topic@Integral, Sub@Definite, Sub@Theorems_FTC, Sub@Rational}{
\begin{sagesilent}
a = NonZeroInt(-10,10)

l=NonZeroInt(-10,5)  
if l<0:
   while l==-1:
      l=NonZeroInt(-10,5)
   u=NonZeroInt(l+1,-1)
elif l>0:
   u=NonZeroInt(l+1,12)

p=Integer(randint(0,3))
v=[a/x,a/x^2, a/x^3, a/x^4]
F=v[p]
Ans=integrate(F,x,l,u)
\end{sagesilent}

\latexProblemContent{
\begin{problem}

Use the Fundamental Theorem of Calculus to evaluate the integral.

\input{2311_Compute_Integral_0013.HELP.tex}

\[\int_{\sage{l}}^{\sage{u}} \sage{F}\;dx=\answer{\sage{Ans}}\]
\end{problem}}%}
%%%%%%%%%%%%%%%%%%%%%%

%%%%%%%%%%%% 22 Compute %%%%%%%%%%%%%%%

%%%%%%%%%%%%%%%%%%%%%%%
%%\tagged{Cat@One, Cat@Two, Cat@Three, Cat@Four, Cat@Five, Ans@MC, Type@Concept, Topic@Integral, Sub@Definite, Sub@Theorems_FTC, Sub@Rational}{
\begin{sagesilent}
a = NonZeroInt(-10,10)
l=NonZeroInt(-6,-1)  
u=NonZeroInt(1,6)

p=Integer(randint(0,3))
v=[a/x,a/x^2, a/x^3, a/x^4]
F=v[p]
f=integrate(F,x)
Ans=f(u)-f(l)
\end{sagesilent}

\latexProblemContent{
\begin{problem}

What is wrong with the following equation:

\[\int_{\sage{l}}^{\sage{u}} \sage{F}\;dx = \sage{f}\Bigg\vert_{\sage{l}}^{\sage{u}} = \sage{Ans}\]

\input{2311_Concept_Integral_0001.HELP.tex}

\begin{multipleChoice}
\choice The bounds are evaluated in the wrong order.
\choice The antiderivative is incorrect.
\choice[correct] The integrand is not defined over the entire interval.
\choice Nothing is wrong.  The equation is correct, as is.
\end{multipleChoice}

\end{problem}}%}
%%%%%%%%%%%%%%%%%%%%%%


%%%%%%%%%%% 22 Compute   1 Concept %%%%%%%%%%%%%

%%%%%%%%%%%%%%%%%%%%%%%
%%\tagged{Cat@One, Cat@Two, Cat@Three, Cat@Four, Cat@Five, Ans@MC, Type@Concept, Topic@Integral, Sub@Definite, Sub@Theorems_FTC, Sub@Trig}{
\begin{sagesilent}
b = NonZeroInt(-10,10)
l = RandAng(0,pi/3)
u = RandAng(2*pi/3,pi)

p=Integer(randint(0,1))
vt=[b*sec(x)*tan(x), b*sec(x)^2]
F=vt[p]
f=integrate(F,x)
Ans=f(u)-f(l)
\end{sagesilent}

\latexProblemContent{
\begin{problem}

What is wrong with the following equation:

\[\int_{\sage{l}}^{\sage{u}} \sage{F}\;dx = \sage{f}\Bigg\vert_{\sage{l}}^{\sage{u}} = \sage{Ans}\]

\input{2311_Concept_Integral_0002.HELP.tex}

\begin{multipleChoice}
\choice The antiderivative is incorrect.
\choice[correct] The integrand is not defined over the entire interval.
\choice The bounds are evaluated in the wrong order.
\choice Nothing is wrong.  The equation is correct, as is.
\end{multipleChoice}

\end{problem}}%}
%%%%%%%%%%%%%%%%%%%%%%



%%%%%%%%%%%%%%%%%%%%%%%
%%\tagged{Cat@One, Cat@Two, Cat@Three, Cat@Four, Cat@Five, Ans@MC, Type@Concept, Topic@Integral, Sub@Definite, Sub@Theorems_FTC, Sub@Trig}{
\begin{sagesilent}
b = NonZeroInt(-10,10)
l = RandAng(pi/6,5*pi/6)
u = RandAng(7*pi/6,11*pi/6)

p=Integer(randint(0,1))
vt=[b*csc(x)^2, b*csc(x)*cot(x)]
F=vt[p]
f=integrate(F,x)
Ans=f(u)-f(l)
\end{sagesilent}

\latexProblemContent{
\begin{problem}

What is wrong with the following equation:

\[\int_{\sage{l}}^{\sage{u}} \sage{F}\;dx = \sage{f}\Bigg\vert_{\sage{l}}^{\sage{u}} = \sage{Ans}\]

\input{2311_Concept_Integral_0003.HELP.tex}

\begin{multipleChoice}
\choice The antiderivative is incorrect.
\choice[correct] The integrand is not defined over the entire interval.
\choice The bounds are evaluated in the wrong order.
\choice Nothing is wrong.  The equation is correct, as is.
\end{multipleChoice}

\end{problem}}%}
%%%%%%%%%%%%%%%%%%%%%%



%%%%%%%%%% 3 Concept   13 Compute %%%%%%%%%%


%%%%%%%%%%%%%%%%%%%%%%%
%%\tagged{Cat@One, Cat@Two, Cat@Three, Cat@Four, Cat@Five, Ans@ShortAns, Type@Compute, Topic@Integral, Sub@Indefinite}{
\begin{sagesilent}
a = NonZeroInt(-10,10)
b = NonZeroInt(-10,10)
c = Integer(randint(-15,15))
p=Integer(randint(0,10))
v2=[sqrt(x), exp(x), (x)^2, (x)^3, (x)^4, (x-b), sin(x), cos(x), 1/(x), 1/(x)^2, 1/(x)^3]
F=v2[p]
G=a*F+c
Ans=integrate(G,x)
\end{sagesilent}

\latexProblemContent{
\begin{problem}

Compute the indefinite integral:

\input{2311_Compute_Integral_0014.HELP.tex}

\[\int\;\sage{G}\;dx = \answer{\sage{Ans}+C}\]
\end{problem}}%}
%%%%%%%%%%%%%%%%%%%%%%

%%%%%%%%%%%%%%%%%%%%%%%
%%\tagged{Cat@One, Cat@Two, Cat@Three, Cat@Four, Cat@Five, Ans@ShortAns, Type@Compute, Topic@Integral, Sub@Indefinite}{
\begin{sagesilent}
a = NonZeroInt(-10,10)
b = NonZeroInt(-10,10)
p=Integer(randint(0,10))
q=Integer(randint(0,10))
v2=[sqrt(x), exp(x), (x)^2, (x)^3, (x)^4, (x-b), sin(x), cos(x), 1/(x), 1/(x)^2, 1/(x)^3]
F=v2[p]
G=v2[q]
Ans=integrate(a*(F+G),x)
\end{sagesilent}

\latexProblemContent{
\begin{problem}

Compute the indefinite integral:

\input{2311_Compute_Integral_0015.HELP.tex}

\[\int\;\sage{a*(F+G)}\;dx = \answer{\sage{Ans}+C}\]
\end{problem}}%}
%%%%%%%%%%%%%%%%%%%%%%



%%%%%%%%%%%%%%%%%%%%%%%
%%\tagged{Cat@One, Cat@Two, Cat@Three, Cat@Four, Cat@Five, Ans@ShortAns, Type@Compute, Topic@Integral, Sub@Indefinite, Sub@Sub_u}{
\begin{sagesilent}
a = NonZeroInt(-10,10)
b = NonZeroInt(-10,10)
c = NonZeroInt(-10,10)
p=Integer(randint(0,11))
v2=[sqrt(x), exp(x), log(x), (x)^2, (x)^3, (x)^4, (x-b), sin(x), cos(x), 1/(x), 1/(x)^2, 1/(x)^3]
F=v2[p]+c
G=diff(F,x)
Ans=integrate(a*G*F,x)
\end{sagesilent}

\latexProblemContent{
\begin{problem}

Compute the indefinite integral:

\input{2311_Compute_Integral_0016.HELP.tex}

\[\int\;\sage{a*G*F}\;dx = \answer{\sage{Ans}+C}\]
\end{problem}}%}
%%%%%%%%%%%%%%%%%%%%%%

%%%%%%%%%%%%%%%%%%%%%%%
%%\tagged{Cat@One, Cat@Two, Cat@Three, Cat@Four, Cat@Five, Ans@ShortAns, Type@Compute, Topic@Integral, Sub@Indefinite, Sub@Sub_u}{
\begin{sagesilent}
a = NonZeroInt(-10,10)
b = NonZeroInt(-10,10)
p=Integer(randint(0,11))
q=Integer(randint(0,11))
v2=[sqrt(x), exp(x), log(x), (x)^2, (x)^3, (x)^4, (x-b), sin(x), cos(x), 1/(x), 1/(x)^2, 1/(x)^3]
F=v2[p]
G=v2[q]
f=diff(F,x)
Ans=integrate(a*G(F)*f,x)
\end{sagesilent}

\latexProblemContent{
\begin{problem}

Compute the indefinite integral:

\input{2311_Compute_Integral_0017.HELP.tex}

\[\int\;\sage{a*G(F)*f}\;dx = \answer{\sage{Ans}+C}\]
\end{problem}}%}
%%%%%%%%%%%%%%%%%%%%%%


%%%%%%%%%%%%%%%%%%%%%%%
%%\tagged{Cat@One, Cat@Two, Cat@Three, Cat@Four, Cat@Five, Ans@ShortAns, Type@Compute, Topic@Integral, Sub@Indefinite, Sub@Sub_u}{
\begin{sagesilent}
a = NonZeroInt(-10,10)
b = NonZeroInt(-10,10)
c = NonZeroInt(1,5)
p=Integer(randint(0,11))
q=Integer(randint(0,3))
v1=[sqrt(x-b), log(x-c), exp(x-b), (x-b)^2, (x-b)^3, (x-b)^4, (x-b), sin(x-b), cos(x-b), 1/(x-b), 1/(x-b)^2, 1/(x-b)^3]
v=[a/x,a/x^2, a/x^3, a/x^4]
F=v1[p]
G=v[q]
f=diff(F,x)
Ans=integrate(G(F)*f,x)
\end{sagesilent}

\latexProblemContent{
\begin{problem}

Compute the indefinite integral:

\input{2311_Compute_Integral_0018.HELP.tex}

\[\int\;\sage{G(F)*f}\;dx = \answer{\sage{Ans}+C}\]
\end{problem}}%}
%%%%%%%%%%%%%%%%%%%%%%


%%%%%%%%%%%%%%%%%%%%%%%
%%\tagged{Cat@One, Cat@Two, Cat@Three, Cat@Four, Cat@Five, Ans@ShortAns, Type@Compute, Topic@Integral, Sub@Indefinite, Sub@Sub_u, Sub@Arctrig}{
\begin{sagesilent}
a = NonZeroInt(-10,10)
b = NonZeroInt(-10,10)
p=Integer(randint(0,1))
v=[a*arcsin(b*x), a*arccos(b*x), a*arctan(b*x)]
F=v[p]
f=diff(F,x)
Ans=integrate(F*f,x)
\end{sagesilent}

\latexProblemContent{
\begin{problem}

Compute the indefinite integral:

\input{2311_Compute_Integral_0019.HELP.tex}

\[\int\;\sage{F*f}\;dx = \answer{\sage{Ans}+C}\]
\end{problem}}%}
%%%%%%%%%%%%%%%%%%%%%%



%%%%%%%%%  3 Concept    19 Compute %%%%%%%%%%%%%










































