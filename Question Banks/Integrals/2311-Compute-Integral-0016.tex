%%%%%%%%%%%%%%%%%%%%%%%
%%\tagged{Cat@One, Cat@Two, Cat@Three, Cat@Four, Cat@Five, Ans@ShortAns, Type@Compute, Topic@Integral, Sub@Indefinite, Sub@Sub_u}{

\latexProblemContent{
\begin{problem}

Compute the indefinite integral:

\input{2311-Compute-Integral-0016.HELP.tex}

\[\int\;{-7 \, {\left(e^{x} + 6\right)} e^{x}}\;dx = \answer{{-\frac{7}{2} \, {\left(e^{x} + 6\right)}^{2}}+C}\]
\end{problem}}%}

%%%%%%%%%%%%%%%%%%%%%%


\latexProblemContent{
\begin{problem}

Compute the indefinite integral:

\input{2311-Compute-Integral-0016.HELP.tex}

\[\int\;{-28 \, {\left(x^{4} + 2\right)} x^{3}}\;dx = \answer{{-\frac{7}{2} \, {\left(x^{4} + 2\right)}^{2}}+C}\]
\end{problem}}%}

%%%%%%%%%%%%%%%%%%%%%%


\latexProblemContent{
\begin{problem}

Compute the indefinite integral:

\input{2311-Compute-Integral-0016.HELP.tex}

\[\int\;{-\frac{2 \, {\left(\sqrt{x} + 5\right)}}{\sqrt{x}}}\;dx = \answer{{-2 \, {\left(\sqrt{x} + 5\right)}^{2}}+C}\]
\end{problem}}%}

%%%%%%%%%%%%%%%%%%%%%%


\latexProblemContent{
\begin{problem}

Compute the indefinite integral:

\input{2311-Compute-Integral-0016.HELP.tex}

\[\int\;{-5 \, x - 40}\;dx = \answer{{-\frac{5}{2} \, x^{2} - 40 \, x}+C}\]
\end{problem}}%}

%%%%%%%%%%%%%%%%%%%%%%


\latexProblemContent{
\begin{problem}

Compute the indefinite integral:

\input{2311-Compute-Integral-0016.HELP.tex}

\[\int\;{-{\left(\sin\left(x\right) + 2\right)} \cos\left(x\right)}\;dx = \answer{{-\frac{1}{2} \, {\left(\sin\left(x\right) + 2\right)}^{2}}+C}\]
\end{problem}}%}

%%%%%%%%%%%%%%%%%%%%%%


\latexProblemContent{
\begin{problem}

Compute the indefinite integral:

\input{2311-Compute-Integral-0016.HELP.tex}

\[\int\;{18 \, {\left(x^{3} - 3\right)} x^{2}}\;dx = \answer{{3 \, {\left(x^{3} - 3\right)}^{2}}+C}\]
\end{problem}}%}

%%%%%%%%%%%%%%%%%%%%%%


\latexProblemContent{
\begin{problem}

Compute the indefinite integral:

\input{2311-Compute-Integral-0016.HELP.tex}

\[\int\;{6 \, x - 12}\;dx = \answer{{3 \, x^{2} - 12 \, x}+C}\]
\end{problem}}%}

%%%%%%%%%%%%%%%%%%%%%%


\latexProblemContent{
\begin{problem}

Compute the indefinite integral:

\input{2311-Compute-Integral-0016.HELP.tex}

\[\int\;{-\frac{15 \, {\left(\frac{1}{x^{3}} - 5\right)}}{x^{4}}}\;dx = \answer{{\frac{5}{2} \, {\left(\frac{1}{x^{3}} - 5\right)}^{2}}+C}\]
\end{problem}}%}

%%%%%%%%%%%%%%%%%%%%%%


\latexProblemContent{
\begin{problem}

Compute the indefinite integral:

\input{2311-Compute-Integral-0016.HELP.tex}

\[\int\;{\frac{16 \, {\left(\frac{1}{x^{2}} + 2\right)}}{x^{3}}}\;dx = \answer{{-4 \, {\left(\frac{1}{x^{2}} + 2\right)}^{2}}+C}\]
\end{problem}}%}

%%%%%%%%%%%%%%%%%%%%%%


\latexProblemContent{
\begin{problem}

Compute the indefinite integral:

\input{2311-Compute-Integral-0016.HELP.tex}

\[\int\;{\frac{18 \, {\left(\frac{1}{x^{2}} - 8\right)}}{x^{3}}}\;dx = \answer{{-\frac{9}{2} \, {\left(\frac{1}{x^{2}} - 8\right)}^{2}}+C}\]
\end{problem}}%}

%%%%%%%%%%%%%%%%%%%%%%


\latexProblemContent{
\begin{problem}

Compute the indefinite integral:

\input{2311-Compute-Integral-0016.HELP.tex}

\[\int\;{-12 \, {\left(x^{3} + 1\right)} x^{2}}\;dx = \answer{{-2 \, {\left(x^{3} + 1\right)}^{2}}+C}\]
\end{problem}}%}

%%%%%%%%%%%%%%%%%%%%%%


\latexProblemContent{
\begin{problem}

Compute the indefinite integral:

\input{2311-Compute-Integral-0016.HELP.tex}

\[\int\;{-\frac{18 \, {\left(\frac{1}{x^{2}} + 2\right)}}{x^{3}}}\;dx = \answer{{\frac{9}{2} \, {\left(\frac{1}{x^{2}} + 2\right)}^{2}}+C}\]
\end{problem}}%}

%%%%%%%%%%%%%%%%%%%%%%


\latexProblemContent{
\begin{problem}

Compute the indefinite integral:

\input{2311-Compute-Integral-0016.HELP.tex}

\[\int\;{16 \, {\left(x^{4} + 3\right)} x^{3}}\;dx = \answer{{2 \, {\left(x^{4} + 3\right)}^{2}}+C}\]
\end{problem}}%}

%%%%%%%%%%%%%%%%%%%%%%


\latexProblemContent{
\begin{problem}

Compute the indefinite integral:

\input{2311-Compute-Integral-0016.HELP.tex}

\[\int\;{\frac{2 \, {\left(\sqrt{x} - 5\right)}}{\sqrt{x}}}\;dx = \answer{{2 \, {\left(\sqrt{x} - 5\right)}^{2}}+C}\]
\end{problem}}%}

%%%%%%%%%%%%%%%%%%%%%%


\latexProblemContent{
\begin{problem}

Compute the indefinite integral:

\input{2311-Compute-Integral-0016.HELP.tex}

\[\int\;{\frac{4 \, {\left(\frac{1}{x^{2}} - 8\right)}}{x^{3}}}\;dx = \answer{{-{\left(\frac{1}{x^{2}} - 8\right)}^{2}}+C}\]
\end{problem}}%}

%%%%%%%%%%%%%%%%%%%%%%


\latexProblemContent{
\begin{problem}

Compute the indefinite integral:

\input{2311-Compute-Integral-0016.HELP.tex}

\[\int\;{-\frac{9 \, {\left(\frac{1}{x^{3}} + 5\right)}}{x^{4}}}\;dx = \answer{{\frac{3}{2} \, {\left(\frac{1}{x^{3}} + 5\right)}^{2}}+C}\]
\end{problem}}%}

%%%%%%%%%%%%%%%%%%%%%%


\latexProblemContent{
\begin{problem}

Compute the indefinite integral:

\input{2311-Compute-Integral-0016.HELP.tex}

\[\int\;{-9 \, {\left(\sin\left(x\right) - 5\right)} \cos\left(x\right)}\;dx = \answer{{-\frac{9}{2} \, {\left(\sin\left(x\right) - 5\right)}^{2}}+C}\]
\end{problem}}%}

%%%%%%%%%%%%%%%%%%%%%%


\latexProblemContent{
\begin{problem}

Compute the indefinite integral:

\input{2311-Compute-Integral-0016.HELP.tex}

\[\int\;{-4 \, {\left(x^{4} - 3\right)} x^{3}}\;dx = \answer{{-\frac{1}{2} \, {\left(x^{4} - 3\right)}^{2}}+C}\]
\end{problem}}%}

%%%%%%%%%%%%%%%%%%%%%%


\latexProblemContent{
\begin{problem}

Compute the indefinite integral:

\input{2311-Compute-Integral-0016.HELP.tex}

\[\int\;{-\frac{30 \, {\left(\frac{1}{x^{3}} + 5\right)}}{x^{4}}}\;dx = \answer{{5 \, {\left(\frac{1}{x^{3}} + 5\right)}^{2}}+C}\]
\end{problem}}%}

%%%%%%%%%%%%%%%%%%%%%%


\latexProblemContent{
\begin{problem}

Compute the indefinite integral:

\input{2311-Compute-Integral-0016.HELP.tex}

\[\int\;{\frac{3 \, {\left(\frac{1}{x} + 9\right)}}{x^{2}}}\;dx = \answer{{-\frac{3}{2} \, {\left(\frac{1}{x} + 9\right)}^{2}}+C}\]
\end{problem}}%}

%%%%%%%%%%%%%%%%%%%%%%


\latexProblemContent{
\begin{problem}

Compute the indefinite integral:

\input{2311-Compute-Integral-0016.HELP.tex}

\[\int\;{4 \, x}\;dx = \answer{{2 \, x^{2}}+C}\]
\end{problem}}%}

%%%%%%%%%%%%%%%%%%%%%%


\latexProblemContent{
\begin{problem}

Compute the indefinite integral:

\input{2311-Compute-Integral-0016.HELP.tex}

\[\int\;{-10 \, {\left(x^{2} - 8\right)} x}\;dx = \answer{{-\frac{5}{2} \, {\left(x^{2} - 8\right)}^{2}}+C}\]
\end{problem}}%}

%%%%%%%%%%%%%%%%%%%%%%


\latexProblemContent{
\begin{problem}

Compute the indefinite integral:

\input{2311-Compute-Integral-0016.HELP.tex}

\[\int\;{\frac{3 \, {\left(\sqrt{x} + 6\right)}}{2 \, \sqrt{x}}}\;dx = \answer{{\frac{3}{2} \, {\left(\sqrt{x} + 6\right)}^{2}}+C}\]
\end{problem}}%}

%%%%%%%%%%%%%%%%%%%%%%


\latexProblemContent{
\begin{problem}

Compute the indefinite integral:

\input{2311-Compute-Integral-0016.HELP.tex}

\[\int\;{-9 \, {\left(\cos\left(x\right) - 7\right)} \sin\left(x\right)}\;dx = \answer{{\frac{9}{2} \, {\left(\cos\left(x\right) - 7\right)}^{2}}+C}\]
\end{problem}}%}

%%%%%%%%%%%%%%%%%%%%%%


\latexProblemContent{
\begin{problem}

Compute the indefinite integral:

\input{2311-Compute-Integral-0016.HELP.tex}

\[\int\;{-4 \, {\left(\sin\left(x\right) + 7\right)} \cos\left(x\right)}\;dx = \answer{{-2 \, {\left(\sin\left(x\right) + 7\right)}^{2}}+C}\]
\end{problem}}%}

%%%%%%%%%%%%%%%%%%%%%%


\latexProblemContent{
\begin{problem}

Compute the indefinite integral:

\input{2311-Compute-Integral-0016.HELP.tex}

\[\int\;{-\frac{3 \, {\left(\log\left(x\right) + 5\right)}}{x}}\;dx = \answer{{-\frac{3}{2} \, {\left(\log\left(x\right) + 5\right)}^{2}}+C}\]
\end{problem}}%}

%%%%%%%%%%%%%%%%%%%%%%


\latexProblemContent{
\begin{problem}

Compute the indefinite integral:

\input{2311-Compute-Integral-0016.HELP.tex}

\[\int\;{-7 \, {\left(\cos\left(x\right) + 5\right)} \sin\left(x\right)}\;dx = \answer{{\frac{7}{2} \, {\left(\cos\left(x\right) + 5\right)}^{2}}+C}\]
\end{problem}}%}

%%%%%%%%%%%%%%%%%%%%%%


\latexProblemContent{
\begin{problem}

Compute the indefinite integral:

\input{2311-Compute-Integral-0016.HELP.tex}

\[\int\;{-8 \, {\left(\cos\left(x\right) - 7\right)} \sin\left(x\right)}\;dx = \answer{{4 \, {\left(\cos\left(x\right) - 7\right)}^{2}}+C}\]
\end{problem}}%}

%%%%%%%%%%%%%%%%%%%%%%


\latexProblemContent{
\begin{problem}

Compute the indefinite integral:

\input{2311-Compute-Integral-0016.HELP.tex}

\[\int\;{\frac{9 \, {\left(\frac{1}{x^{3}} - 5\right)}}{x^{4}}}\;dx = \answer{{-\frac{3}{2} \, {\left(\frac{1}{x^{3}} - 5\right)}^{2}}+C}\]
\end{problem}}%}

%%%%%%%%%%%%%%%%%%%%%%


\latexProblemContent{
\begin{problem}

Compute the indefinite integral:

\input{2311-Compute-Integral-0016.HELP.tex}

\[\int\;{-12 \, {\left(x^{2} - 3\right)} x}\;dx = \answer{{-3 \, {\left(x^{2} - 3\right)}^{2}}+C}\]
\end{problem}}%}

%%%%%%%%%%%%%%%%%%%%%%


\latexProblemContent{
\begin{problem}

Compute the indefinite integral:

\input{2311-Compute-Integral-0016.HELP.tex}

\[\int\;{-6 \, {\left(x^{2} + 7\right)} x}\;dx = \answer{{-\frac{3}{2} \, {\left(x^{2} + 7\right)}^{2}}+C}\]
\end{problem}}%}

%%%%%%%%%%%%%%%%%%%%%%


\latexProblemContent{
\begin{problem}

Compute the indefinite integral:

\input{2311-Compute-Integral-0016.HELP.tex}

\[\int\;{-4 \, {\left(\cos\left(x\right) - 4\right)} \sin\left(x\right)}\;dx = \answer{{2 \, {\left(\cos\left(x\right) - 4\right)}^{2}}+C}\]
\end{problem}}%}

%%%%%%%%%%%%%%%%%%%%%%


\latexProblemContent{
\begin{problem}

Compute the indefinite integral:

\input{2311-Compute-Integral-0016.HELP.tex}

\[\int\;{-32 \, {\left(x^{4} + 2\right)} x^{3}}\;dx = \answer{{-4 \, {\left(x^{4} + 2\right)}^{2}}+C}\]
\end{problem}}%}

%%%%%%%%%%%%%%%%%%%%%%


\latexProblemContent{
\begin{problem}

Compute the indefinite integral:

\input{2311-Compute-Integral-0016.HELP.tex}

\[\int\;{-6 \, x + 18}\;dx = \answer{{-3 \, x^{2} + 18 \, x}+C}\]
\end{problem}}%}

%%%%%%%%%%%%%%%%%%%%%%


\latexProblemContent{
\begin{problem}

Compute the indefinite integral:

\input{2311-Compute-Integral-0016.HELP.tex}

\[\int\;{\frac{21 \, {\left(\frac{1}{x^{3}} - 3\right)}}{x^{4}}}\;dx = \answer{{-\frac{7}{2} \, {\left(\frac{1}{x^{3}} - 3\right)}^{2}}+C}\]
\end{problem}}%}

%%%%%%%%%%%%%%%%%%%%%%


\latexProblemContent{
\begin{problem}

Compute the indefinite integral:

\input{2311-Compute-Integral-0016.HELP.tex}

\[\int\;{12 \, {\left(x^{3} + 7\right)} x^{2}}\;dx = \answer{{2 \, {\left(x^{3} + 7\right)}^{2}}+C}\]
\end{problem}}%}

%%%%%%%%%%%%%%%%%%%%%%


\latexProblemContent{
\begin{problem}

Compute the indefinite integral:

\input{2311-Compute-Integral-0016.HELP.tex}

\[\int\;{\frac{2 \, {\left(\sqrt{x} + 2\right)}}{\sqrt{x}}}\;dx = \answer{{2 \, {\left(\sqrt{x} + 2\right)}^{2}}+C}\]
\end{problem}}%}

%%%%%%%%%%%%%%%%%%%%%%


\latexProblemContent{
\begin{problem}

Compute the indefinite integral:

\input{2311-Compute-Integral-0016.HELP.tex}

\[\int\;{-\frac{27 \, {\left(\frac{1}{x^{3}} - 7\right)}}{x^{4}}}\;dx = \answer{{\frac{9}{2} \, {\left(\frac{1}{x^{3}} - 7\right)}^{2}}+C}\]
\end{problem}}%}

%%%%%%%%%%%%%%%%%%%%%%


\latexProblemContent{
\begin{problem}

Compute the indefinite integral:

\input{2311-Compute-Integral-0016.HELP.tex}

\[\int\;{4 \, {\left(e^{x} + 5\right)} e^{x}}\;dx = \answer{{2 \, {\left(e^{x} + 5\right)}^{2}}+C}\]
\end{problem}}%}

%%%%%%%%%%%%%%%%%%%%%%


\latexProblemContent{
\begin{problem}

Compute the indefinite integral:

\input{2311-Compute-Integral-0016.HELP.tex}

\[\int\;{-\frac{6 \, {\left(\frac{1}{x} - 7\right)}}{x^{2}}}\;dx = \answer{{3 \, {\left(\frac{1}{x} - 7\right)}^{2}}+C}\]
\end{problem}}%}

%%%%%%%%%%%%%%%%%%%%%%


\latexProblemContent{
\begin{problem}

Compute the indefinite integral:

\input{2311-Compute-Integral-0016.HELP.tex}

\[\int\;{3 \, {\left(x^{3} + 1\right)} x^{2}}\;dx = \answer{{\frac{1}{2} \, {\left(x^{3} + 1\right)}^{2}}+C}\]
\end{problem}}%}

%%%%%%%%%%%%%%%%%%%%%%


\latexProblemContent{
\begin{problem}

Compute the indefinite integral:

\input{2311-Compute-Integral-0016.HELP.tex}

\[\int\;{6 \, {\left(\sin\left(x\right) - 3\right)} \cos\left(x\right)}\;dx = \answer{{3 \, {\left(\sin\left(x\right) - 3\right)}^{2}}+C}\]
\end{problem}}%}

%%%%%%%%%%%%%%%%%%%%%%


\latexProblemContent{
\begin{problem}

Compute the indefinite integral:

\input{2311-Compute-Integral-0016.HELP.tex}

\[\int\;{-\frac{9 \, {\left(\sqrt{x} - 8\right)}}{2 \, \sqrt{x}}}\;dx = \answer{{-\frac{9}{2} \, {\left(\sqrt{x} - 8\right)}^{2}}+C}\]
\end{problem}}%}

%%%%%%%%%%%%%%%%%%%%%%


\latexProblemContent{
\begin{problem}

Compute the indefinite integral:

\input{2311-Compute-Integral-0016.HELP.tex}

\[\int\;{-\frac{4 \, {\left(\frac{1}{x^{2}} - 9\right)}}{x^{3}}}\;dx = \answer{{{\left(\frac{1}{x^{2}} - 9\right)}^{2}}+C}\]
\end{problem}}%}

%%%%%%%%%%%%%%%%%%%%%%


\latexProblemContent{
\begin{problem}

Compute the indefinite integral:

\input{2311-Compute-Integral-0016.HELP.tex}

\[\int\;{-5 \, {\left(\sin\left(x\right) + 7\right)} \cos\left(x\right)}\;dx = \answer{{-\frac{5}{2} \, {\left(\sin\left(x\right) + 7\right)}^{2}}+C}\]
\end{problem}}%}

%%%%%%%%%%%%%%%%%%%%%%


\latexProblemContent{
\begin{problem}

Compute the indefinite integral:

\input{2311-Compute-Integral-0016.HELP.tex}

\[\int\;{\frac{10 \, {\left(\log\left(x\right) + 10\right)}}{x}}\;dx = \answer{{5 \, {\left(\log\left(x\right) + 10\right)}^{2}}+C}\]
\end{problem}}%}

%%%%%%%%%%%%%%%%%%%%%%


\latexProblemContent{
\begin{problem}

Compute the indefinite integral:

\input{2311-Compute-Integral-0016.HELP.tex}

\[\int\;{4 \, x + 16}\;dx = \answer{{2 \, x^{2} + 16 \, x}+C}\]
\end{problem}}%}

%%%%%%%%%%%%%%%%%%%%%%


\latexProblemContent{
\begin{problem}

Compute the indefinite integral:

\input{2311-Compute-Integral-0016.HELP.tex}

\[\int\;{\frac{9 \, {\left(\frac{1}{x^{3}} + 9\right)}}{x^{4}}}\;dx = \answer{{-\frac{3}{2} \, {\left(\frac{1}{x^{3}} + 9\right)}^{2}}+C}\]
\end{problem}}%}

%%%%%%%%%%%%%%%%%%%%%%


\latexProblemContent{
\begin{problem}

Compute the indefinite integral:

\input{2311-Compute-Integral-0016.HELP.tex}

\[\int\;{24 \, {\left(x^{3} - 9\right)} x^{2}}\;dx = \answer{{4 \, {\left(x^{3} - 9\right)}^{2}}+C}\]
\end{problem}}%}

%%%%%%%%%%%%%%%%%%%%%%


\latexProblemContent{
\begin{problem}

Compute the indefinite integral:

\input{2311-Compute-Integral-0016.HELP.tex}

\[\int\;{40 \, {\left(x^{4} - 8\right)} x^{3}}\;dx = \answer{{5 \, {\left(x^{4} - 8\right)}^{2}}+C}\]
\end{problem}}%}

%%%%%%%%%%%%%%%%%%%%%%


\latexProblemContent{
\begin{problem}

Compute the indefinite integral:

\input{2311-Compute-Integral-0016.HELP.tex}

\[\int\;{40 \, {\left(x^{4} + 6\right)} x^{3}}\;dx = \answer{{5 \, {\left(x^{4} + 6\right)}^{2}}+C}\]
\end{problem}}%}

%%%%%%%%%%%%%%%%%%%%%%


\latexProblemContent{
\begin{problem}

Compute the indefinite integral:

\input{2311-Compute-Integral-0016.HELP.tex}

\[\int\;{6 \, x + 6}\;dx = \answer{{3 \, x^{2} + 6 \, x}+C}\]
\end{problem}}%}

%%%%%%%%%%%%%%%%%%%%%%


\latexProblemContent{
\begin{problem}

Compute the indefinite integral:

\input{2311-Compute-Integral-0016.HELP.tex}

\[\int\;{30 \, {\left(x^{3} + 7\right)} x^{2}}\;dx = \answer{{5 \, {\left(x^{3} + 7\right)}^{2}}+C}\]
\end{problem}}%}

%%%%%%%%%%%%%%%%%%%%%%


\latexProblemContent{
\begin{problem}

Compute the indefinite integral:

\input{2311-Compute-Integral-0016.HELP.tex}

\[\int\;{\frac{8 \, {\left(\frac{1}{x} - 5\right)}}{x^{2}}}\;dx = \answer{{-4 \, {\left(\frac{1}{x} - 5\right)}^{2}}+C}\]
\end{problem}}%}

%%%%%%%%%%%%%%%%%%%%%%


\latexProblemContent{
\begin{problem}

Compute the indefinite integral:

\input{2311-Compute-Integral-0016.HELP.tex}

\[\int\;{-\frac{15 \, {\left(\frac{1}{x^{3}} + 8\right)}}{x^{4}}}\;dx = \answer{{\frac{5}{2} \, {\left(\frac{1}{x^{3}} + 8\right)}^{2}}+C}\]
\end{problem}}%}

%%%%%%%%%%%%%%%%%%%%%%


\latexProblemContent{
\begin{problem}

Compute the indefinite integral:

\input{2311-Compute-Integral-0016.HELP.tex}

\[\int\;{-\frac{14 \, {\left(\frac{1}{x^{2}} + 8\right)}}{x^{3}}}\;dx = \answer{{\frac{7}{2} \, {\left(\frac{1}{x^{2}} + 8\right)}^{2}}+C}\]
\end{problem}}%}

%%%%%%%%%%%%%%%%%%%%%%


\latexProblemContent{
\begin{problem}

Compute the indefinite integral:

\input{2311-Compute-Integral-0016.HELP.tex}

\[\int\;{9 \, {\left(\sin\left(x\right) + 4\right)} \cos\left(x\right)}\;dx = \answer{{\frac{9}{2} \, {\left(\sin\left(x\right) + 4\right)}^{2}}+C}\]
\end{problem}}%}

%%%%%%%%%%%%%%%%%%%%%%


\latexProblemContent{
\begin{problem}

Compute the indefinite integral:

\input{2311-Compute-Integral-0016.HELP.tex}

\[\int\;{-2 \, {\left(e^{x} + 6\right)} e^{x}}\;dx = \answer{{-{\left(e^{x} + 6\right)}^{2}}+C}\]
\end{problem}}%}

%%%%%%%%%%%%%%%%%%%%%%


\latexProblemContent{
\begin{problem}

Compute the indefinite integral:

\input{2311-Compute-Integral-0016.HELP.tex}

\[\int\;{\frac{8 \, {\left(\frac{1}{x} + 8\right)}}{x^{2}}}\;dx = \answer{{-4 \, {\left(\frac{1}{x} + 8\right)}^{2}}+C}\]
\end{problem}}%}

%%%%%%%%%%%%%%%%%%%%%%


\latexProblemContent{
\begin{problem}

Compute the indefinite integral:

\input{2311-Compute-Integral-0016.HELP.tex}

\[\int\;{-\frac{8 \, {\left(\frac{1}{x} - 3\right)}}{x^{2}}}\;dx = \answer{{4 \, {\left(\frac{1}{x} - 3\right)}^{2}}+C}\]
\end{problem}}%}

%%%%%%%%%%%%%%%%%%%%%%


\latexProblemContent{
\begin{problem}

Compute the indefinite integral:

\input{2311-Compute-Integral-0016.HELP.tex}

\[\int\;{\frac{2 \, {\left(\log\left(x\right) - 2\right)}}{x}}\;dx = \answer{{{\left(\log\left(x\right) - 2\right)}^{2}}+C}\]
\end{problem}}%}

%%%%%%%%%%%%%%%%%%%%%%


\latexProblemContent{
\begin{problem}

Compute the indefinite integral:

\input{2311-Compute-Integral-0016.HELP.tex}

\[\int\;{\frac{5 \, {\left(\sqrt{x} + 3\right)}}{\sqrt{x}}}\;dx = \answer{{5 \, {\left(\sqrt{x} + 3\right)}^{2}}+C}\]
\end{problem}}%}

%%%%%%%%%%%%%%%%%%%%%%


\latexProblemContent{
\begin{problem}

Compute the indefinite integral:

\input{2311-Compute-Integral-0016.HELP.tex}

\[\int\;{-4 \, {\left(e^{x} - 6\right)} e^{x}}\;dx = \answer{{-2 \, {\left(e^{x} - 6\right)}^{2}}+C}\]
\end{problem}}%}

%%%%%%%%%%%%%%%%%%%%%%


\latexProblemContent{
\begin{problem}

Compute the indefinite integral:

\input{2311-Compute-Integral-0016.HELP.tex}

\[\int\;{\frac{9 \, {\left(\sqrt{x} - 10\right)}}{2 \, \sqrt{x}}}\;dx = \answer{{\frac{9}{2} \, {\left(\sqrt{x} - 10\right)}^{2}}+C}\]
\end{problem}}%}

%%%%%%%%%%%%%%%%%%%%%%


\latexProblemContent{
\begin{problem}

Compute the indefinite integral:

\input{2311-Compute-Integral-0016.HELP.tex}

\[\int\;{-21 \, {\left(x^{3} - 3\right)} x^{2}}\;dx = \answer{{-\frac{7}{2} \, {\left(x^{3} - 3\right)}^{2}}+C}\]
\end{problem}}%}

%%%%%%%%%%%%%%%%%%%%%%


\latexProblemContent{
\begin{problem}

Compute the indefinite integral:

\input{2311-Compute-Integral-0016.HELP.tex}

\[\int\;{-\frac{\log\left(x\right) - 4}{x}}\;dx = \answer{{-\frac{1}{2} \, {\left(\log\left(x\right) - 4\right)}^{2}}+C}\]
\end{problem}}%}

%%%%%%%%%%%%%%%%%%%%%%


\latexProblemContent{
\begin{problem}

Compute the indefinite integral:

\input{2311-Compute-Integral-0016.HELP.tex}

\[\int\;{12 \, {\left(x^{4} + 4\right)} x^{3}}\;dx = \answer{{\frac{3}{2} \, {\left(x^{4} + 4\right)}^{2}}+C}\]
\end{problem}}%}

%%%%%%%%%%%%%%%%%%%%%%


\latexProblemContent{
\begin{problem}

Compute the indefinite integral:

\input{2311-Compute-Integral-0016.HELP.tex}

\[\int\;{-12 \, {\left(x^{2} - 6\right)} x}\;dx = \answer{{-3 \, {\left(x^{2} - 6\right)}^{2}}+C}\]
\end{problem}}%}

%%%%%%%%%%%%%%%%%%%%%%


\latexProblemContent{
\begin{problem}

Compute the indefinite integral:

\input{2311-Compute-Integral-0016.HELP.tex}

\[\int\;{-12 \, {\left(x^{4} + 8\right)} x^{3}}\;dx = \answer{{-\frac{3}{2} \, {\left(x^{4} + 8\right)}^{2}}+C}\]
\end{problem}}%}

%%%%%%%%%%%%%%%%%%%%%%


\latexProblemContent{
\begin{problem}

Compute the indefinite integral:

\input{2311-Compute-Integral-0016.HELP.tex}

\[\int\;{{\left(e^{x} - 7\right)} e^{x}}\;dx = \answer{{\frac{1}{2} \, {\left(e^{x} - 7\right)}^{2}}+C}\]
\end{problem}}%}

%%%%%%%%%%%%%%%%%%%%%%


\latexProblemContent{
\begin{problem}

Compute the indefinite integral:

\input{2311-Compute-Integral-0016.HELP.tex}

\[\int\;{-\frac{7 \, {\left(\sqrt{x} - 5\right)}}{2 \, \sqrt{x}}}\;dx = \answer{{-\frac{7}{2} \, {\left(\sqrt{x} - 5\right)}^{2}}+C}\]
\end{problem}}%}

%%%%%%%%%%%%%%%%%%%%%%


\latexProblemContent{
\begin{problem}

Compute the indefinite integral:

\input{2311-Compute-Integral-0016.HELP.tex}

\[\int\;{-6 \, {\left(e^{x} + 6\right)} e^{x}}\;dx = \answer{{-3 \, {\left(e^{x} + 6\right)}^{2}}+C}\]
\end{problem}}%}

%%%%%%%%%%%%%%%%%%%%%%


\latexProblemContent{
\begin{problem}

Compute the indefinite integral:

\input{2311-Compute-Integral-0016.HELP.tex}

\[\int\;{\frac{4 \, {\left(\log\left(x\right) - 2\right)}}{x}}\;dx = \answer{{2 \, {\left(\log\left(x\right) - 2\right)}^{2}}+C}\]
\end{problem}}%}

%%%%%%%%%%%%%%%%%%%%%%


\latexProblemContent{
\begin{problem}

Compute the indefinite integral:

\input{2311-Compute-Integral-0016.HELP.tex}

\[\int\;{-9 \, {\left(\cos\left(x\right) + 2\right)} \sin\left(x\right)}\;dx = \answer{{\frac{9}{2} \, {\left(\cos\left(x\right) + 2\right)}^{2}}+C}\]
\end{problem}}%}

%%%%%%%%%%%%%%%%%%%%%%


\latexProblemContent{
\begin{problem}

Compute the indefinite integral:

\input{2311-Compute-Integral-0016.HELP.tex}

\[\int\;{-\frac{7 \, {\left(\sqrt{x} + 1\right)}}{2 \, \sqrt{x}}}\;dx = \answer{{-\frac{7}{2} \, {\left(\sqrt{x} + 1\right)}^{2}}+C}\]
\end{problem}}%}

%%%%%%%%%%%%%%%%%%%%%%


\latexProblemContent{
\begin{problem}

Compute the indefinite integral:

\input{2311-Compute-Integral-0016.HELP.tex}

\[\int\;{-\frac{6 \, {\left(\frac{1}{x} + 5\right)}}{x^{2}}}\;dx = \answer{{3 \, {\left(\frac{1}{x} + 5\right)}^{2}}+C}\]
\end{problem}}%}

%%%%%%%%%%%%%%%%%%%%%%


\latexProblemContent{
\begin{problem}

Compute the indefinite integral:

\input{2311-Compute-Integral-0016.HELP.tex}

\[\int\;{\frac{2 \, {\left(\frac{1}{x} + 6\right)}}{x^{2}}}\;dx = \answer{{-{\left(\frac{1}{x} + 6\right)}^{2}}+C}\]
\end{problem}}%}

%%%%%%%%%%%%%%%%%%%%%%


\latexProblemContent{
\begin{problem}

Compute the indefinite integral:

\input{2311-Compute-Integral-0016.HELP.tex}

\[\int\;{8 \, {\left(e^{x} + 4\right)} e^{x}}\;dx = \answer{{4 \, {\left(e^{x} + 4\right)}^{2}}+C}\]
\end{problem}}%}

%%%%%%%%%%%%%%%%%%%%%%


\latexProblemContent{
\begin{problem}

Compute the indefinite integral:

\input{2311-Compute-Integral-0016.HELP.tex}

\[\int\;{-8 \, {\left(\cos\left(x\right) - 3\right)} \sin\left(x\right)}\;dx = \answer{{4 \, {\left(\cos\left(x\right) - 3\right)}^{2}}+C}\]
\end{problem}}%}

%%%%%%%%%%%%%%%%%%%%%%


\latexProblemContent{
\begin{problem}

Compute the indefinite integral:

\input{2311-Compute-Integral-0016.HELP.tex}

\[\int\;{-40 \, {\left(x^{4} - 7\right)} x^{3}}\;dx = \answer{{-5 \, {\left(x^{4} - 7\right)}^{2}}+C}\]
\end{problem}}%}

%%%%%%%%%%%%%%%%%%%%%%


\latexProblemContent{
\begin{problem}

Compute the indefinite integral:

\input{2311-Compute-Integral-0016.HELP.tex}

\[\int\;{-21 \, {\left(x^{3} + 8\right)} x^{2}}\;dx = \answer{{-\frac{7}{2} \, {\left(x^{3} + 8\right)}^{2}}+C}\]
\end{problem}}%}

%%%%%%%%%%%%%%%%%%%%%%


\latexProblemContent{
\begin{problem}

Compute the indefinite integral:

\input{2311-Compute-Integral-0016.HELP.tex}

\[\int\;{4 \, {\left(x^{2} - 10\right)} x}\;dx = \answer{{{\left(x^{2} - 10\right)}^{2}}+C}\]
\end{problem}}%}

%%%%%%%%%%%%%%%%%%%%%%


\latexProblemContent{
\begin{problem}

Compute the indefinite integral:

\input{2311-Compute-Integral-0016.HELP.tex}

\[\int\;{-\frac{21 \, {\left(\frac{1}{x^{3}} - 5\right)}}{x^{4}}}\;dx = \answer{{\frac{7}{2} \, {\left(\frac{1}{x^{3}} - 5\right)}^{2}}+C}\]
\end{problem}}%}

%%%%%%%%%%%%%%%%%%%%%%


\latexProblemContent{
\begin{problem}

Compute the indefinite integral:

\input{2311-Compute-Integral-0016.HELP.tex}

\[\int\;{24 \, {\left(x^{4} - 5\right)} x^{3}}\;dx = \answer{{3 \, {\left(x^{4} - 5\right)}^{2}}+C}\]
\end{problem}}%}

%%%%%%%%%%%%%%%%%%%%%%


\latexProblemContent{
\begin{problem}

Compute the indefinite integral:

\input{2311-Compute-Integral-0016.HELP.tex}

\[\int\;{-32 \, {\left(x^{4} + 7\right)} x^{3}}\;dx = \answer{{-4 \, {\left(x^{4} + 7\right)}^{2}}+C}\]
\end{problem}}%}

%%%%%%%%%%%%%%%%%%%%%%


\latexProblemContent{
\begin{problem}

Compute the indefinite integral:

\input{2311-Compute-Integral-0016.HELP.tex}

\[\int\;{10 \, {\left(x^{2} + 9\right)} x}\;dx = \answer{{\frac{5}{2} \, {\left(x^{2} + 9\right)}^{2}}+C}\]
\end{problem}}%}

%%%%%%%%%%%%%%%%%%%%%%


\latexProblemContent{
\begin{problem}

Compute the indefinite integral:

\input{2311-Compute-Integral-0016.HELP.tex}

\[\int\;{-5 \, {\left(\cos\left(x\right) - 2\right)} \sin\left(x\right)}\;dx = \answer{{\frac{5}{2} \, {\left(\cos\left(x\right) - 2\right)}^{2}}+C}\]
\end{problem}}%}

%%%%%%%%%%%%%%%%%%%%%%


\latexProblemContent{
\begin{problem}

Compute the indefinite integral:

\input{2311-Compute-Integral-0016.HELP.tex}

\[\int\;{\frac{27 \, {\left(\frac{1}{x^{3}} - 10\right)}}{x^{4}}}\;dx = \answer{{-\frac{9}{2} \, {\left(\frac{1}{x^{3}} - 10\right)}^{2}}+C}\]
\end{problem}}%}

%%%%%%%%%%%%%%%%%%%%%%


\latexProblemContent{
\begin{problem}

Compute the indefinite integral:

\input{2311-Compute-Integral-0016.HELP.tex}

\[\int\;{-\frac{4 \, {\left(\frac{1}{x} + 4\right)}}{x^{2}}}\;dx = \answer{{2 \, {\left(\frac{1}{x} + 4\right)}^{2}}+C}\]
\end{problem}}%}

%%%%%%%%%%%%%%%%%%%%%%


\latexProblemContent{
\begin{problem}

Compute the indefinite integral:

\input{2311-Compute-Integral-0016.HELP.tex}

\[\int\;{\frac{5 \, {\left(\frac{1}{x} + 1\right)}}{x^{2}}}\;dx = \answer{{-\frac{5}{2} \, {\left(\frac{1}{x} + 1\right)}^{2}}+C}\]
\end{problem}}%}

%%%%%%%%%%%%%%%%%%%%%%


\latexProblemContent{
\begin{problem}

Compute the indefinite integral:

\input{2311-Compute-Integral-0016.HELP.tex}

\[\int\;{-6 \, {\left(\sin\left(x\right) + 10\right)} \cos\left(x\right)}\;dx = \answer{{-3 \, {\left(\sin\left(x\right) + 10\right)}^{2}}+C}\]
\end{problem}}%}

%%%%%%%%%%%%%%%%%%%%%%


\latexProblemContent{
\begin{problem}

Compute the indefinite integral:

\input{2311-Compute-Integral-0016.HELP.tex}

\[\int\;{8 \, {\left(\cos\left(x\right) + 9\right)} \sin\left(x\right)}\;dx = \answer{{-4 \, {\left(\cos\left(x\right) + 9\right)}^{2}}+C}\]
\end{problem}}%}

%%%%%%%%%%%%%%%%%%%%%%


\latexProblemContent{
\begin{problem}

Compute the indefinite integral:

\input{2311-Compute-Integral-0016.HELP.tex}

\[\int\;{-\frac{3 \, {\left(\sqrt{x} - 8\right)}}{2 \, \sqrt{x}}}\;dx = \answer{{-\frac{3}{2} \, {\left(\sqrt{x} - 8\right)}^{2}}+C}\]
\end{problem}}%}

%%%%%%%%%%%%%%%%%%%%%%


\latexProblemContent{
\begin{problem}

Compute the indefinite integral:

\input{2311-Compute-Integral-0016.HELP.tex}

\[\int\;{{\left(\sin\left(x\right) + 5\right)} \cos\left(x\right)}\;dx = \answer{{\frac{1}{2} \, {\left(\sin\left(x\right) + 5\right)}^{2}}+C}\]
\end{problem}}%}

%%%%%%%%%%%%%%%%%%%%%%


\latexProblemContent{
\begin{problem}

Compute the indefinite integral:

\input{2311-Compute-Integral-0016.HELP.tex}

\[\int\;{-5 \, x - 30}\;dx = \answer{{-\frac{5}{2} \, x^{2} - 30 \, x}+C}\]
\end{problem}}%}

%%%%%%%%%%%%%%%%%%%%%%


\latexProblemContent{
\begin{problem}

Compute the indefinite integral:

\input{2311-Compute-Integral-0016.HELP.tex}

\[\int\;{\frac{6 \, {\left(\frac{1}{x^{2}} - 4\right)}}{x^{3}}}\;dx = \answer{{-\frac{3}{2} \, {\left(\frac{1}{x^{2}} - 4\right)}^{2}}+C}\]
\end{problem}}%}

%%%%%%%%%%%%%%%%%%%%%%


\latexProblemContent{
\begin{problem}

Compute the indefinite integral:

\input{2311-Compute-Integral-0016.HELP.tex}

\[\int\;{{\left(e^{x} + 2\right)} e^{x}}\;dx = \answer{{\frac{1}{2} \, {\left(e^{x} + 2\right)}^{2}}+C}\]
\end{problem}}%}

%%%%%%%%%%%%%%%%%%%%%%


\latexProblemContent{
\begin{problem}

Compute the indefinite integral:

\input{2311-Compute-Integral-0016.HELP.tex}

\[\int\;{-5 \, {\left(\sin\left(x\right) + 3\right)} \cos\left(x\right)}\;dx = \answer{{-\frac{5}{2} \, {\left(\sin\left(x\right) + 3\right)}^{2}}+C}\]
\end{problem}}%}

%%%%%%%%%%%%%%%%%%%%%%


\latexProblemContent{
\begin{problem}

Compute the indefinite integral:

\input{2311-Compute-Integral-0016.HELP.tex}

\[\int\;{-18 \, {\left(x^{2} + 2\right)} x}\;dx = \answer{{-\frac{9}{2} \, {\left(x^{2} + 2\right)}^{2}}+C}\]
\end{problem}}%}

%%%%%%%%%%%%%%%%%%%%%%


\latexProblemContent{
\begin{problem}

Compute the indefinite integral:

\input{2311-Compute-Integral-0016.HELP.tex}

\[\int\;{28 \, {\left(x^{4} - 9\right)} x^{3}}\;dx = \answer{{\frac{7}{2} \, {\left(x^{4} - 9\right)}^{2}}+C}\]
\end{problem}}%}

%%%%%%%%%%%%%%%%%%%%%%


\latexProblemContent{
\begin{problem}

Compute the indefinite integral:

\input{2311-Compute-Integral-0016.HELP.tex}

\[\int\;{-8 \, {\left(\cos\left(x\right) + 1\right)} \sin\left(x\right)}\;dx = \answer{{4 \, {\left(\cos\left(x\right) + 1\right)}^{2}}+C}\]
\end{problem}}%}

%%%%%%%%%%%%%%%%%%%%%%


\latexProblemContent{
\begin{problem}

Compute the indefinite integral:

\input{2311-Compute-Integral-0016.HELP.tex}

\[\int\;{\frac{4 \, {\left(\log\left(x\right) - 5\right)}}{x}}\;dx = \answer{{2 \, {\left(\log\left(x\right) - 5\right)}^{2}}+C}\]
\end{problem}}%}

%%%%%%%%%%%%%%%%%%%%%%


\latexProblemContent{
\begin{problem}

Compute the indefinite integral:

\input{2311-Compute-Integral-0016.HELP.tex}

\[\int\;{-6 \, {\left(x^{2} - 5\right)} x}\;dx = \answer{{-\frac{3}{2} \, {\left(x^{2} - 5\right)}^{2}}+C}\]
\end{problem}}%}

%%%%%%%%%%%%%%%%%%%%%%


\latexProblemContent{
\begin{problem}

Compute the indefinite integral:

\input{2311-Compute-Integral-0016.HELP.tex}

\[\int\;{-\frac{6 \, {\left(\frac{1}{x} + 8\right)}}{x^{2}}}\;dx = \answer{{3 \, {\left(\frac{1}{x} + 8\right)}^{2}}+C}\]
\end{problem}}%}

%%%%%%%%%%%%%%%%%%%%%%


\latexProblemContent{
\begin{problem}

Compute the indefinite integral:

\input{2311-Compute-Integral-0016.HELP.tex}

\[\int\;{-3 \, {\left(e^{x} - 2\right)} e^{x}}\;dx = \answer{{-\frac{3}{2} \, {\left(e^{x} - 2\right)}^{2}}+C}\]
\end{problem}}%}

%%%%%%%%%%%%%%%%%%%%%%


\latexProblemContent{
\begin{problem}

Compute the indefinite integral:

\input{2311-Compute-Integral-0016.HELP.tex}

\[\int\;{5 \, {\left(\cos\left(x\right) + 9\right)} \sin\left(x\right)}\;dx = \answer{{-\frac{5}{2} \, {\left(\cos\left(x\right) + 9\right)}^{2}}+C}\]
\end{problem}}%}

%%%%%%%%%%%%%%%%%%%%%%


\latexProblemContent{
\begin{problem}

Compute the indefinite integral:

\input{2311-Compute-Integral-0016.HELP.tex}

\[\int\;{-16 \, {\left(x^{4} + 8\right)} x^{3}}\;dx = \answer{{-2 \, {\left(x^{4} + 8\right)}^{2}}+C}\]
\end{problem}}%}

%%%%%%%%%%%%%%%%%%%%%%


\latexProblemContent{
\begin{problem}

Compute the indefinite integral:

\input{2311-Compute-Integral-0016.HELP.tex}

\[\int\;{-27 \, {\left(x^{3} - 2\right)} x^{2}}\;dx = \answer{{-\frac{9}{2} \, {\left(x^{3} - 2\right)}^{2}}+C}\]
\end{problem}}%}

%%%%%%%%%%%%%%%%%%%%%%


\latexProblemContent{
\begin{problem}

Compute the indefinite integral:

\input{2311-Compute-Integral-0016.HELP.tex}

\[\int\;{36 \, {\left(x^{4} + 6\right)} x^{3}}\;dx = \answer{{\frac{9}{2} \, {\left(x^{4} + 6\right)}^{2}}+C}\]
\end{problem}}%}

%%%%%%%%%%%%%%%%%%%%%%


\latexProblemContent{
\begin{problem}

Compute the indefinite integral:

\input{2311-Compute-Integral-0016.HELP.tex}

\[\int\;{-\frac{3 \, {\left(\frac{1}{x^{3}} + 7\right)}}{x^{4}}}\;dx = \answer{{\frac{1}{2} \, {\left(\frac{1}{x^{3}} + 7\right)}^{2}}+C}\]
\end{problem}}%}

%%%%%%%%%%%%%%%%%%%%%%


\latexProblemContent{
\begin{problem}

Compute the indefinite integral:

\input{2311-Compute-Integral-0016.HELP.tex}

\[\int\;{24 \, {\left(x^{3} + 4\right)} x^{2}}\;dx = \answer{{4 \, {\left(x^{3} + 4\right)}^{2}}+C}\]
\end{problem}}%}

%%%%%%%%%%%%%%%%%%%%%%


\latexProblemContent{
\begin{problem}

Compute the indefinite integral:

\input{2311-Compute-Integral-0016.HELP.tex}

\[\int\;{9 \, x + 63}\;dx = \answer{{\frac{9}{2} \, x^{2} + 63 \, x}+C}\]
\end{problem}}%}

%%%%%%%%%%%%%%%%%%%%%%


\latexProblemContent{
\begin{problem}

Compute the indefinite integral:

\input{2311-Compute-Integral-0016.HELP.tex}

\[\int\;{-3 \, {\left(\cos\left(x\right) - 6\right)} \sin\left(x\right)}\;dx = \answer{{\frac{3}{2} \, {\left(\cos\left(x\right) - 6\right)}^{2}}+C}\]
\end{problem}}%}

%%%%%%%%%%%%%%%%%%%%%%


\latexProblemContent{
\begin{problem}

Compute the indefinite integral:

\input{2311-Compute-Integral-0016.HELP.tex}

\[\int\;{16 \, {\left(x^{2} - 10\right)} x}\;dx = \answer{{4 \, {\left(x^{2} - 10\right)}^{2}}+C}\]
\end{problem}}%}

%%%%%%%%%%%%%%%%%%%%%%


\latexProblemContent{
\begin{problem}

Compute the indefinite integral:

\input{2311-Compute-Integral-0016.HELP.tex}

\[\int\;{-10 \, {\left(\cos\left(x\right) + 8\right)} \sin\left(x\right)}\;dx = \answer{{5 \, {\left(\cos\left(x\right) + 8\right)}^{2}}+C}\]
\end{problem}}%}

%%%%%%%%%%%%%%%%%%%%%%


\latexProblemContent{
\begin{problem}

Compute the indefinite integral:

\input{2311-Compute-Integral-0016.HELP.tex}

\[\int\;{-6 \, x + 72}\;dx = \answer{{-3 \, x^{2} + 72 \, x}+C}\]
\end{problem}}%}

%%%%%%%%%%%%%%%%%%%%%%


\latexProblemContent{
\begin{problem}

Compute the indefinite integral:

\input{2311-Compute-Integral-0016.HELP.tex}

\[\int\;{\frac{7 \, {\left(\sqrt{x} + 4\right)}}{2 \, \sqrt{x}}}\;dx = \answer{{\frac{7}{2} \, {\left(\sqrt{x} + 4\right)}^{2}}+C}\]
\end{problem}}%}

%%%%%%%%%%%%%%%%%%%%%%


\latexProblemContent{
\begin{problem}

Compute the indefinite integral:

\input{2311-Compute-Integral-0016.HELP.tex}

\[\int\;{6 \, {\left(x^{3} - 1\right)} x^{2}}\;dx = \answer{{{\left(x^{3} - 1\right)}^{2}}+C}\]
\end{problem}}%}

%%%%%%%%%%%%%%%%%%%%%%


\latexProblemContent{
\begin{problem}

Compute the indefinite integral:

\input{2311-Compute-Integral-0016.HELP.tex}

\[\int\;{-\frac{2 \, {\left(\sqrt{x} - 1\right)}}{\sqrt{x}}}\;dx = \answer{{-2 \, {\left(\sqrt{x} - 1\right)}^{2}}+C}\]
\end{problem}}%}

%%%%%%%%%%%%%%%%%%%%%%


\latexProblemContent{
\begin{problem}

Compute the indefinite integral:

\input{2311-Compute-Integral-0016.HELP.tex}

\[\int\;{\frac{5 \, {\left(\sqrt{x} - 1\right)}}{\sqrt{x}}}\;dx = \answer{{5 \, {\left(\sqrt{x} - 1\right)}^{2}}+C}\]
\end{problem}}%}

%%%%%%%%%%%%%%%%%%%%%%


\latexProblemContent{
\begin{problem}

Compute the indefinite integral:

\input{2311-Compute-Integral-0016.HELP.tex}

\[\int\;{-14 \, {\left(x^{2} - 4\right)} x}\;dx = \answer{{-\frac{7}{2} \, {\left(x^{2} - 4\right)}^{2}}+C}\]
\end{problem}}%}

%%%%%%%%%%%%%%%%%%%%%%


\latexProblemContent{
\begin{problem}

Compute the indefinite integral:

\input{2311-Compute-Integral-0016.HELP.tex}

\[\int\;{\frac{4 \, {\left(\frac{1}{x^{2}} - 7\right)}}{x^{3}}}\;dx = \answer{{-{\left(\frac{1}{x^{2}} - 7\right)}^{2}}+C}\]
\end{problem}}%}

%%%%%%%%%%%%%%%%%%%%%%


\latexProblemContent{
\begin{problem}

Compute the indefinite integral:

\input{2311-Compute-Integral-0016.HELP.tex}

\[\int\;{-16 \, {\left(x^{2} - 2\right)} x}\;dx = \answer{{-4 \, {\left(x^{2} - 2\right)}^{2}}+C}\]
\end{problem}}%}

%%%%%%%%%%%%%%%%%%%%%%


\latexProblemContent{
\begin{problem}

Compute the indefinite integral:

\input{2311-Compute-Integral-0016.HELP.tex}

\[\int\;{15 \, {\left(x^{3} - 6\right)} x^{2}}\;dx = \answer{{\frac{5}{2} \, {\left(x^{3} - 6\right)}^{2}}+C}\]
\end{problem}}%}

%%%%%%%%%%%%%%%%%%%%%%


\latexProblemContent{
\begin{problem}

Compute the indefinite integral:

\input{2311-Compute-Integral-0016.HELP.tex}

\[\int\;{\frac{8 \, {\left(\frac{1}{x} + 7\right)}}{x^{2}}}\;dx = \answer{{-4 \, {\left(\frac{1}{x} + 7\right)}^{2}}+C}\]
\end{problem}}%}

%%%%%%%%%%%%%%%%%%%%%%


\latexProblemContent{
\begin{problem}

Compute the indefinite integral:

\input{2311-Compute-Integral-0016.HELP.tex}

\[\int\;{6 \, {\left(x^{3} - 3\right)} x^{2}}\;dx = \answer{{{\left(x^{3} - 3\right)}^{2}}+C}\]
\end{problem}}%}

%%%%%%%%%%%%%%%%%%%%%%


\latexProblemContent{
\begin{problem}

Compute the indefinite integral:

\input{2311-Compute-Integral-0016.HELP.tex}

\[\int\;{\frac{12 \, {\left(\frac{1}{x^{2}} + 8\right)}}{x^{3}}}\;dx = \answer{{-3 \, {\left(\frac{1}{x^{2}} + 8\right)}^{2}}+C}\]
\end{problem}}%}

%%%%%%%%%%%%%%%%%%%%%%


\latexProblemContent{
\begin{problem}

Compute the indefinite integral:

\input{2311-Compute-Integral-0016.HELP.tex}

\[\int\;{3 \, {\left(x^{3} + 3\right)} x^{2}}\;dx = \answer{{\frac{1}{2} \, {\left(x^{3} + 3\right)}^{2}}+C}\]
\end{problem}}%}

%%%%%%%%%%%%%%%%%%%%%%


\latexProblemContent{
\begin{problem}

Compute the indefinite integral:

\input{2311-Compute-Integral-0016.HELP.tex}

\[\int\;{\frac{7 \, {\left(\frac{1}{x} + 6\right)}}{x^{2}}}\;dx = \answer{{-\frac{7}{2} \, {\left(\frac{1}{x} + 6\right)}^{2}}+C}\]
\end{problem}}%}

%%%%%%%%%%%%%%%%%%%%%%


%%%%%%%%%%%%%%%%%%%%%%


\latexProblemContent{
\begin{problem}

Compute the indefinite integral:

\input{2311-Compute-Integral-0016.HELP.tex}

\[\int\;{10 \, {\left(\sin\left(x\right) + 10\right)} \cos\left(x\right)}\;dx = \answer{{5 \, {\left(\sin\left(x\right) + 10\right)}^{2}}+C}\]
\end{problem}}%}

%%%%%%%%%%%%%%%%%%%%%%


\latexProblemContent{
\begin{problem}

Compute the indefinite integral:

\input{2311-Compute-Integral-0016.HELP.tex}

\[\int\;{-\frac{9 \, {\left(\frac{1}{x} + 3\right)}}{x^{2}}}\;dx = \answer{{\frac{9}{2} \, {\left(\frac{1}{x} + 3\right)}^{2}}+C}\]
\end{problem}}%}

%%%%%%%%%%%%%%%%%%%%%%


\latexProblemContent{
\begin{problem}

Compute the indefinite integral:

\input{2311-Compute-Integral-0016.HELP.tex}

\[\int\;{\frac{9 \, {\left(\log\left(x\right) - 3\right)}}{x}}\;dx = \answer{{\frac{9}{2} \, {\left(\log\left(x\right) - 3\right)}^{2}}+C}\]
\end{problem}}%}

%%%%%%%%%%%%%%%%%%%%%%


\latexProblemContent{
\begin{problem}

Compute the indefinite integral:

\input{2311-Compute-Integral-0016.HELP.tex}

\[\int\;{5 \, {\left(\cos\left(x\right) + 4\right)} \sin\left(x\right)}\;dx = \answer{{-\frac{5}{2} \, {\left(\cos\left(x\right) + 4\right)}^{2}}+C}\]
\end{problem}}%}

%%%%%%%%%%%%%%%%%%%%%%


\latexProblemContent{
\begin{problem}

Compute the indefinite integral:

\input{2311-Compute-Integral-0016.HELP.tex}

\[\int\;{\frac{4 \, {\left(\frac{1}{x} + 4\right)}}{x^{2}}}\;dx = \answer{{-2 \, {\left(\frac{1}{x} + 4\right)}^{2}}+C}\]
\end{problem}}%}

%%%%%%%%%%%%%%%%%%%%%%


\latexProblemContent{
\begin{problem}

Compute the indefinite integral:

\input{2311-Compute-Integral-0016.HELP.tex}

\[\int\;{-\frac{14 \, {\left(\frac{1}{x^{2}} - 5\right)}}{x^{3}}}\;dx = \answer{{\frac{7}{2} \, {\left(\frac{1}{x^{2}} - 5\right)}^{2}}+C}\]
\end{problem}}%}

%%%%%%%%%%%%%%%%%%%%%%


\latexProblemContent{
\begin{problem}

Compute the indefinite integral:

\input{2311-Compute-Integral-0016.HELP.tex}

\[\int\;{4 \, {\left(e^{x} - 3\right)} e^{x}}\;dx = \answer{{2 \, {\left(e^{x} - 3\right)}^{2}}+C}\]
\end{problem}}%}

%%%%%%%%%%%%%%%%%%%%%%


\latexProblemContent{
\begin{problem}

Compute the indefinite integral:

\input{2311-Compute-Integral-0016.HELP.tex}

\[\int\;{-3 \, {\left(\cos\left(x\right) + 8\right)} \sin\left(x\right)}\;dx = \answer{{\frac{3}{2} \, {\left(\cos\left(x\right) + 8\right)}^{2}}+C}\]
\end{problem}}%}

%%%%%%%%%%%%%%%%%%%%%%


\latexProblemContent{
\begin{problem}

Compute the indefinite integral:

\input{2311-Compute-Integral-0016.HELP.tex}

\[\int\;{-8 \, {\left(\cos\left(x\right) + 7\right)} \sin\left(x\right)}\;dx = \answer{{4 \, {\left(\cos\left(x\right) + 7\right)}^{2}}+C}\]
\end{problem}}%}

%%%%%%%%%%%%%%%%%%%%%%


\latexProblemContent{
\begin{problem}

Compute the indefinite integral:

\input{2311-Compute-Integral-0016.HELP.tex}

\[\int\;{-4 \, {\left(x^{2} + 2\right)} x}\;dx = \answer{{-{\left(x^{2} + 2\right)}^{2}}+C}\]
\end{problem}}%}

%%%%%%%%%%%%%%%%%%%%%%


\latexProblemContent{
\begin{problem}

Compute the indefinite integral:

\input{2311-Compute-Integral-0016.HELP.tex}

\[\int\;{4 \, {\left(\cos\left(x\right) - 1\right)} \sin\left(x\right)}\;dx = \answer{{-2 \, {\left(\cos\left(x\right) - 1\right)}^{2}}+C}\]
\end{problem}}%}

%%%%%%%%%%%%%%%%%%%%%%


\latexProblemContent{
\begin{problem}

Compute the indefinite integral:

\input{2311-Compute-Integral-0016.HELP.tex}

\[\int\;{\frac{3 \, {\left(\log\left(x\right) + 1\right)}}{x}}\;dx = \answer{{\frac{3}{2} \, {\left(\log\left(x\right) + 1\right)}^{2}}+C}\]
\end{problem}}%}

%%%%%%%%%%%%%%%%%%%%%%


\latexProblemContent{
\begin{problem}

Compute the indefinite integral:

\input{2311-Compute-Integral-0016.HELP.tex}

\[\int\;{-\frac{4 \, {\left(\frac{1}{x^{2}} - 1\right)}}{x^{3}}}\;dx = \answer{{{\left(\frac{1}{x^{2}} - 1\right)}^{2}}+C}\]
\end{problem}}%}

%%%%%%%%%%%%%%%%%%%%%%


\latexProblemContent{
\begin{problem}

Compute the indefinite integral:

\input{2311-Compute-Integral-0016.HELP.tex}

\[\int\;{\frac{6 \, {\left(\frac{1}{x^{3}} - 8\right)}}{x^{4}}}\;dx = \answer{{-{\left(\frac{1}{x^{3}} - 8\right)}^{2}}+C}\]
\end{problem}}%}

%%%%%%%%%%%%%%%%%%%%%%


\latexProblemContent{
\begin{problem}

Compute the indefinite integral:

\input{2311-Compute-Integral-0016.HELP.tex}

\[\int\;{-5 \, {\left(\sin\left(x\right) + 6\right)} \cos\left(x\right)}\;dx = \answer{{-\frac{5}{2} \, {\left(\sin\left(x\right) + 6\right)}^{2}}+C}\]
\end{problem}}%}

%%%%%%%%%%%%%%%%%%%%%%


\latexProblemContent{
\begin{problem}

Compute the indefinite integral:

\input{2311-Compute-Integral-0016.HELP.tex}

\[\int\;{8 \, x - 144}\;dx = \answer{{4 \, x^{2} - 144 \, x}+C}\]
\end{problem}}%}

%%%%%%%%%%%%%%%%%%%%%%


\latexProblemContent{
\begin{problem}

Compute the indefinite integral:

\input{2311-Compute-Integral-0016.HELP.tex}

\[\int\;{-\frac{5 \, {\left(\frac{1}{x} - 1\right)}}{x^{2}}}\;dx = \answer{{\frac{5}{2} \, {\left(\frac{1}{x} - 1\right)}^{2}}+C}\]
\end{problem}}%}

%%%%%%%%%%%%%%%%%%%%%%


\latexProblemContent{
\begin{problem}

Compute the indefinite integral:

\input{2311-Compute-Integral-0016.HELP.tex}

\[\int\;{-x - 7}\;dx = \answer{{-\frac{1}{2} \, x^{2} - 7 \, x}+C}\]
\end{problem}}%}

%%%%%%%%%%%%%%%%%%%%%%


\latexProblemContent{
\begin{problem}

Compute the indefinite integral:

\input{2311-Compute-Integral-0016.HELP.tex}

\[\int\;{-10 \, {\left(x^{2} + 7\right)} x}\;dx = \answer{{-\frac{5}{2} \, {\left(x^{2} + 7\right)}^{2}}+C}\]
\end{problem}}%}

%%%%%%%%%%%%%%%%%%%%%%


\latexProblemContent{
\begin{problem}

Compute the indefinite integral:

\input{2311-Compute-Integral-0016.HELP.tex}

\[\int\;{7 \, {\left(\cos\left(x\right) - 9\right)} \sin\left(x\right)}\;dx = \answer{{-\frac{7}{2} \, {\left(\cos\left(x\right) - 9\right)}^{2}}+C}\]
\end{problem}}%}

%%%%%%%%%%%%%%%%%%%%%%


\latexProblemContent{
\begin{problem}

Compute the indefinite integral:

\input{2311-Compute-Integral-0016.HELP.tex}

\[\int\;{\frac{14 \, {\left(\frac{1}{x^{2}} + 8\right)}}{x^{3}}}\;dx = \answer{{-\frac{7}{2} \, {\left(\frac{1}{x^{2}} + 8\right)}^{2}}+C}\]
\end{problem}}%}

%%%%%%%%%%%%%%%%%%%%%%


\latexProblemContent{
\begin{problem}

Compute the indefinite integral:

\input{2311-Compute-Integral-0016.HELP.tex}

\[\int\;{\frac{4 \, {\left(\frac{1}{x^{2}} - 6\right)}}{x^{3}}}\;dx = \answer{{-{\left(\frac{1}{x^{2}} - 6\right)}^{2}}+C}\]
\end{problem}}%}

%%%%%%%%%%%%%%%%%%%%%%


\latexProblemContent{
\begin{problem}

Compute the indefinite integral:

\input{2311-Compute-Integral-0016.HELP.tex}

\[\int\;{7 \, {\left(\cos\left(x\right) + 1\right)} \sin\left(x\right)}\;dx = \answer{{-\frac{7}{2} \, {\left(\cos\left(x\right) + 1\right)}^{2}}+C}\]
\end{problem}}%}

%%%%%%%%%%%%%%%%%%%%%%


\latexProblemContent{
\begin{problem}

Compute the indefinite integral:

\input{2311-Compute-Integral-0016.HELP.tex}

\[\int\;{-\frac{2 \, {\left(\frac{1}{x} + 7\right)}}{x^{2}}}\;dx = \answer{{{\left(\frac{1}{x} + 7\right)}^{2}}+C}\]
\end{problem}}%}

%%%%%%%%%%%%%%%%%%%%%%


\latexProblemContent{
\begin{problem}

Compute the indefinite integral:

\input{2311-Compute-Integral-0016.HELP.tex}

\[\int\;{27 \, {\left(x^{3} + 5\right)} x^{2}}\;dx = \answer{{\frac{9}{2} \, {\left(x^{3} + 5\right)}^{2}}+C}\]
\end{problem}}%}

%%%%%%%%%%%%%%%%%%%%%%


%%%%%%%%%%%%%%%%%%%%%%


\latexProblemContent{
\begin{problem}

Compute the indefinite integral:

\input{2311-Compute-Integral-0016.HELP.tex}

\[\int\;{\frac{7 \, {\left(\log\left(x\right) - 2\right)}}{x}}\;dx = \answer{{\frac{7}{2} \, {\left(\log\left(x\right) - 2\right)}^{2}}+C}\]
\end{problem}}%}

%%%%%%%%%%%%%%%%%%%%%%


\latexProblemContent{
\begin{problem}

Compute the indefinite integral:

\input{2311-Compute-Integral-0016.HELP.tex}

\[\int\;{-\frac{30 \, {\left(\frac{1}{x^{3}} + 9\right)}}{x^{4}}}\;dx = \answer{{5 \, {\left(\frac{1}{x^{3}} + 9\right)}^{2}}+C}\]
\end{problem}}%}

%%%%%%%%%%%%%%%%%%%%%%


\latexProblemContent{
\begin{problem}

Compute the indefinite integral:

\input{2311-Compute-Integral-0016.HELP.tex}

\[\int\;{-8 \, {\left(\cos\left(x\right) + 8\right)} \sin\left(x\right)}\;dx = \answer{{4 \, {\left(\cos\left(x\right) + 8\right)}^{2}}+C}\]
\end{problem}}%}

%%%%%%%%%%%%%%%%%%%%%%


\latexProblemContent{
\begin{problem}

Compute the indefinite integral:

\input{2311-Compute-Integral-0016.HELP.tex}

\[\int\;{4 \, {\left(\cos\left(x\right) - 9\right)} \sin\left(x\right)}\;dx = \answer{{-2 \, {\left(\cos\left(x\right) - 9\right)}^{2}}+C}\]
\end{problem}}%}

%%%%%%%%%%%%%%%%%%%%%%


\latexProblemContent{
\begin{problem}

Compute the indefinite integral:

\input{2311-Compute-Integral-0016.HELP.tex}

\[\int\;{-27 \, {\left(x^{3} + 3\right)} x^{2}}\;dx = \answer{{-\frac{9}{2} \, {\left(x^{3} + 3\right)}^{2}}+C}\]
\end{problem}}%}

%%%%%%%%%%%%%%%%%%%%%%


\latexProblemContent{
\begin{problem}

Compute the indefinite integral:

\input{2311-Compute-Integral-0016.HELP.tex}

\[\int\;{-3 \, {\left(x^{3} + 5\right)} x^{2}}\;dx = \answer{{-\frac{1}{2} \, {\left(x^{3} + 5\right)}^{2}}+C}\]
\end{problem}}%}

%%%%%%%%%%%%%%%%%%%%%%


\latexProblemContent{
\begin{problem}

Compute the indefinite integral:

\input{2311-Compute-Integral-0016.HELP.tex}

\[\int\;{-3 \, {\left(x^{3} - 3\right)} x^{2}}\;dx = \answer{{-\frac{1}{2} \, {\left(x^{3} - 3\right)}^{2}}+C}\]
\end{problem}}%}

%%%%%%%%%%%%%%%%%%%%%%


\latexProblemContent{
\begin{problem}

Compute the indefinite integral:

\input{2311-Compute-Integral-0016.HELP.tex}

\[\int\;{\frac{3 \, {\left(\sqrt{x} + 10\right)}}{2 \, \sqrt{x}}}\;dx = \answer{{\frac{3}{2} \, {\left(\sqrt{x} + 10\right)}^{2}}+C}\]
\end{problem}}%}

%%%%%%%%%%%%%%%%%%%%%%


\latexProblemContent{
\begin{problem}

Compute the indefinite integral:

\input{2311-Compute-Integral-0016.HELP.tex}

\[\int\;{-\frac{16 \, {\left(\frac{1}{x^{2}} - 7\right)}}{x^{3}}}\;dx = \answer{{4 \, {\left(\frac{1}{x^{2}} - 7\right)}^{2}}+C}\]
\end{problem}}%}

%%%%%%%%%%%%%%%%%%%%%%


\latexProblemContent{
\begin{problem}

Compute the indefinite integral:

\input{2311-Compute-Integral-0016.HELP.tex}

\[\int\;{-4 \, {\left(x^{4} + 6\right)} x^{3}}\;dx = \answer{{-\frac{1}{2} \, {\left(x^{4} + 6\right)}^{2}}+C}\]
\end{problem}}%}

%%%%%%%%%%%%%%%%%%%%%%


\latexProblemContent{
\begin{problem}

Compute the indefinite integral:

\input{2311-Compute-Integral-0016.HELP.tex}

\[\int\;{\frac{3 \, {\left(\log\left(x\right) + 2\right)}}{x}}\;dx = \answer{{\frac{3}{2} \, {\left(\log\left(x\right) + 2\right)}^{2}}+C}\]
\end{problem}}%}

%%%%%%%%%%%%%%%%%%%%%%


\latexProblemContent{
\begin{problem}

Compute the indefinite integral:

\input{2311-Compute-Integral-0016.HELP.tex}

\[\int\;{10 \, x - 60}\;dx = \answer{{5 \, x^{2} - 60 \, x}+C}\]
\end{problem}}%}

%%%%%%%%%%%%%%%%%%%%%%


\latexProblemContent{
\begin{problem}

Compute the indefinite integral:

\input{2311-Compute-Integral-0016.HELP.tex}

\[\int\;{-\frac{3 \, {\left(\frac{1}{x} + 4\right)}}{x^{2}}}\;dx = \answer{{\frac{3}{2} \, {\left(\frac{1}{x} + 4\right)}^{2}}+C}\]
\end{problem}}%}

%%%%%%%%%%%%%%%%%%%%%%


\latexProblemContent{
\begin{problem}

Compute the indefinite integral:

\input{2311-Compute-Integral-0016.HELP.tex}

\[\int\;{-{\left(e^{x} + 8\right)} e^{x}}\;dx = \answer{{-\frac{1}{2} \, {\left(e^{x} + 8\right)}^{2}}+C}\]
\end{problem}}%}

%%%%%%%%%%%%%%%%%%%%%%


\latexProblemContent{
\begin{problem}

Compute the indefinite integral:

\input{2311-Compute-Integral-0016.HELP.tex}

\[\int\;{4 \, {\left(x^{4} - 10\right)} x^{3}}\;dx = \answer{{\frac{1}{2} \, {\left(x^{4} - 10\right)}^{2}}+C}\]
\end{problem}}%}

%%%%%%%%%%%%%%%%%%%%%%


\latexProblemContent{
\begin{problem}

Compute the indefinite integral:

\input{2311-Compute-Integral-0016.HELP.tex}

\[\int\;{\frac{24 \, {\left(\frac{1}{x^{3}} + 2\right)}}{x^{4}}}\;dx = \answer{{-4 \, {\left(\frac{1}{x^{3}} + 2\right)}^{2}}+C}\]
\end{problem}}%}

%%%%%%%%%%%%%%%%%%%%%%


\latexProblemContent{
\begin{problem}

Compute the indefinite integral:

\input{2311-Compute-Integral-0016.HELP.tex}

\[\int\;{-24 \, {\left(x^{3} + 8\right)} x^{2}}\;dx = \answer{{-4 \, {\left(x^{3} + 8\right)}^{2}}+C}\]
\end{problem}}%}

%%%%%%%%%%%%%%%%%%%%%%


\latexProblemContent{
\begin{problem}

Compute the indefinite integral:

\input{2311-Compute-Integral-0016.HELP.tex}

\[\int\;{9 \, {\left(\sin\left(x\right) + 7\right)} \cos\left(x\right)}\;dx = \answer{{\frac{9}{2} \, {\left(\sin\left(x\right) + 7\right)}^{2}}+C}\]
\end{problem}}%}

%%%%%%%%%%%%%%%%%%%%%%


\latexProblemContent{
\begin{problem}

Compute the indefinite integral:

\input{2311-Compute-Integral-0016.HELP.tex}

\[\int\;{-\frac{4 \, {\left(\frac{1}{x^{2}} - 3\right)}}{x^{3}}}\;dx = \answer{{{\left(\frac{1}{x^{2}} - 3\right)}^{2}}+C}\]
\end{problem}}%}

%%%%%%%%%%%%%%%%%%%%%%


\latexProblemContent{
\begin{problem}

Compute the indefinite integral:

\input{2311-Compute-Integral-0016.HELP.tex}

\[\int\;{\frac{27 \, {\left(\frac{1}{x^{3}} - 1\right)}}{x^{4}}}\;dx = \answer{{-\frac{9}{2} \, {\left(\frac{1}{x^{3}} - 1\right)}^{2}}+C}\]
\end{problem}}%}

%%%%%%%%%%%%%%%%%%%%%%


\latexProblemContent{
\begin{problem}

Compute the indefinite integral:

\input{2311-Compute-Integral-0016.HELP.tex}

\[\int\;{\frac{18 \, {\left(\frac{1}{x^{2}} + 1\right)}}{x^{3}}}\;dx = \answer{{-\frac{9}{2} \, {\left(\frac{1}{x^{2}} + 1\right)}^{2}}+C}\]
\end{problem}}%}

%%%%%%%%%%%%%%%%%%%%%%


\latexProblemContent{
\begin{problem}

Compute the indefinite integral:

\input{2311-Compute-Integral-0016.HELP.tex}

\[\int\;{-\frac{3 \, {\left(\frac{1}{x} - 10\right)}}{x^{2}}}\;dx = \answer{{\frac{3}{2} \, {\left(\frac{1}{x} - 10\right)}^{2}}+C}\]
\end{problem}}%}

%%%%%%%%%%%%%%%%%%%%%%


\latexProblemContent{
\begin{problem}

Compute the indefinite integral:

\input{2311-Compute-Integral-0016.HELP.tex}

\[\int\;{-\frac{8 \, {\left(\frac{1}{x} + 1\right)}}{x^{2}}}\;dx = \answer{{4 \, {\left(\frac{1}{x} + 1\right)}^{2}}+C}\]
\end{problem}}%}

%%%%%%%%%%%%%%%%%%%%%%


\latexProblemContent{
\begin{problem}

Compute the indefinite integral:

\input{2311-Compute-Integral-0016.HELP.tex}

\[\int\;{9 \, {\left(e^{x} + 10\right)} e^{x}}\;dx = \answer{{\frac{9}{2} \, {\left(e^{x} + 10\right)}^{2}}+C}\]
\end{problem}}%}

%%%%%%%%%%%%%%%%%%%%%%


\latexProblemContent{
\begin{problem}

Compute the indefinite integral:

\input{2311-Compute-Integral-0016.HELP.tex}

\[\int\;{-21 \, {\left(x^{3} + 6\right)} x^{2}}\;dx = \answer{{-\frac{7}{2} \, {\left(x^{3} + 6\right)}^{2}}+C}\]
\end{problem}}%}

%%%%%%%%%%%%%%%%%%%%%%


\latexProblemContent{
\begin{problem}

Compute the indefinite integral:

\input{2311-Compute-Integral-0016.HELP.tex}

\[\int\;{-\frac{6 \, {\left(\frac{1}{x} - 8\right)}}{x^{2}}}\;dx = \answer{{3 \, {\left(\frac{1}{x} - 8\right)}^{2}}+C}\]
\end{problem}}%}

%%%%%%%%%%%%%%%%%%%%%%


\latexProblemContent{
\begin{problem}

Compute the indefinite integral:

\input{2311-Compute-Integral-0016.HELP.tex}

\[\int\;{-8 \, {\left(\sin\left(x\right) - 7\right)} \cos\left(x\right)}\;dx = \answer{{-4 \, {\left(\sin\left(x\right) - 7\right)}^{2}}+C}\]
\end{problem}}%}

%%%%%%%%%%%%%%%%%%%%%%


\latexProblemContent{
\begin{problem}

Compute the indefinite integral:

\input{2311-Compute-Integral-0016.HELP.tex}

\[\int\;{-6 \, x + 42}\;dx = \answer{{-3 \, x^{2} + 42 \, x}+C}\]
\end{problem}}%}

%%%%%%%%%%%%%%%%%%%%%%


\latexProblemContent{
\begin{problem}

Compute the indefinite integral:

\input{2311-Compute-Integral-0016.HELP.tex}

\[\int\;{9 \, x + 9}\;dx = \answer{{\frac{9}{2} \, x^{2} + 9 \, x}+C}\]
\end{problem}}%}

%%%%%%%%%%%%%%%%%%%%%%


\latexProblemContent{
\begin{problem}

Compute the indefinite integral:

\input{2311-Compute-Integral-0016.HELP.tex}

\[\int\;{-\frac{7 \, {\left(\log\left(x\right) - 8\right)}}{x}}\;dx = \answer{{-\frac{7}{2} \, {\left(\log\left(x\right) - 8\right)}^{2}}+C}\]
\end{problem}}%}

%%%%%%%%%%%%%%%%%%%%%%


\latexProblemContent{
\begin{problem}

Compute the indefinite integral:

\input{2311-Compute-Integral-0016.HELP.tex}

\[\int\;{\frac{10 \, {\left(\frac{1}{x^{2}} - 1\right)}}{x^{3}}}\;dx = \answer{{-\frac{5}{2} \, {\left(\frac{1}{x^{2}} - 1\right)}^{2}}+C}\]
\end{problem}}%}

%%%%%%%%%%%%%%%%%%%%%%


\latexProblemContent{
\begin{problem}

Compute the indefinite integral:

\input{2311-Compute-Integral-0016.HELP.tex}

\[\int\;{20 \, {\left(x^{4} + 8\right)} x^{3}}\;dx = \answer{{\frac{5}{2} \, {\left(x^{4} + 8\right)}^{2}}+C}\]
\end{problem}}%}

%%%%%%%%%%%%%%%%%%%%%%


\latexProblemContent{
\begin{problem}

Compute the indefinite integral:

\input{2311-Compute-Integral-0016.HELP.tex}

\[\int\;{6 \, {\left(e^{x} + 2\right)} e^{x}}\;dx = \answer{{3 \, {\left(e^{x} + 2\right)}^{2}}+C}\]
\end{problem}}%}

%%%%%%%%%%%%%%%%%%%%%%


\latexProblemContent{
\begin{problem}

Compute the indefinite integral:

\input{2311-Compute-Integral-0016.HELP.tex}

\[\int\;{-\frac{3 \, {\left(\frac{1}{x^{3}} + 8\right)}}{x^{4}}}\;dx = \answer{{\frac{1}{2} \, {\left(\frac{1}{x^{3}} + 8\right)}^{2}}+C}\]
\end{problem}}%}

%%%%%%%%%%%%%%%%%%%%%%


\latexProblemContent{
\begin{problem}

Compute the indefinite integral:

\input{2311-Compute-Integral-0016.HELP.tex}

\[\int\;{\frac{10 \, {\left(\frac{1}{x^{2}} + 7\right)}}{x^{3}}}\;dx = \answer{{-\frac{5}{2} \, {\left(\frac{1}{x^{2}} + 7\right)}^{2}}+C}\]
\end{problem}}%}

%%%%%%%%%%%%%%%%%%%%%%


\latexProblemContent{
\begin{problem}

Compute the indefinite integral:

\input{2311-Compute-Integral-0016.HELP.tex}

\[\int\;{\frac{4 \, {\left(\sqrt{x} - 3\right)}}{\sqrt{x}}}\;dx = \answer{{4 \, {\left(\sqrt{x} - 3\right)}^{2}}+C}\]
\end{problem}}%}

%%%%%%%%%%%%%%%%%%%%%%


\latexProblemContent{
\begin{problem}

Compute the indefinite integral:

\input{2311-Compute-Integral-0016.HELP.tex}

\[\int\;{\frac{7 \, {\left(\frac{1}{x} + 8\right)}}{x^{2}}}\;dx = \answer{{-\frac{7}{2} \, {\left(\frac{1}{x} + 8\right)}^{2}}+C}\]
\end{problem}}%}

%%%%%%%%%%%%%%%%%%%%%%


\latexProblemContent{
\begin{problem}

Compute the indefinite integral:

\input{2311-Compute-Integral-0016.HELP.tex}

\[\int\;{-30 \, {\left(x^{3} + 10\right)} x^{2}}\;dx = \answer{{-5 \, {\left(x^{3} + 10\right)}^{2}}+C}\]
\end{problem}}%}

%%%%%%%%%%%%%%%%%%%%%%


\latexProblemContent{
\begin{problem}

Compute the indefinite integral:

\input{2311-Compute-Integral-0016.HELP.tex}

\[\int\;{-\frac{5 \, {\left(\log\left(x\right) + 7\right)}}{x}}\;dx = \answer{{-\frac{5}{2} \, {\left(\log\left(x\right) + 7\right)}^{2}}+C}\]
\end{problem}}%}

%%%%%%%%%%%%%%%%%%%%%%


\latexProblemContent{
\begin{problem}

Compute the indefinite integral:

\input{2311-Compute-Integral-0016.HELP.tex}

\[\int\;{36 \, {\left(x^{4} - 8\right)} x^{3}}\;dx = \answer{{\frac{9}{2} \, {\left(x^{4} - 8\right)}^{2}}+C}\]
\end{problem}}%}

%%%%%%%%%%%%%%%%%%%%%%


\latexProblemContent{
\begin{problem}

Compute the indefinite integral:

\input{2311-Compute-Integral-0016.HELP.tex}

\[\int\;{-\frac{12 \, {\left(\frac{1}{x^{2}} + 8\right)}}{x^{3}}}\;dx = \answer{{3 \, {\left(\frac{1}{x^{2}} + 8\right)}^{2}}+C}\]
\end{problem}}%}

%%%%%%%%%%%%%%%%%%%%%%


\latexProblemContent{
\begin{problem}

Compute the indefinite integral:

\input{2311-Compute-Integral-0016.HELP.tex}

\[\int\;{12 \, {\left(x^{2} - 4\right)} x}\;dx = \answer{{3 \, {\left(x^{2} - 4\right)}^{2}}+C}\]
\end{problem}}%}

%%%%%%%%%%%%%%%%%%%%%%


\latexProblemContent{
\begin{problem}

Compute the indefinite integral:

\input{2311-Compute-Integral-0016.HELP.tex}

\[\int\;{\frac{6 \, {\left(\frac{1}{x^{2}} - 2\right)}}{x^{3}}}\;dx = \answer{{-\frac{3}{2} \, {\left(\frac{1}{x^{2}} - 2\right)}^{2}}+C}\]
\end{problem}}%}

%%%%%%%%%%%%%%%%%%%%%%


\latexProblemContent{
\begin{problem}

Compute the indefinite integral:

\input{2311-Compute-Integral-0016.HELP.tex}

\[\int\;{-16 \, {\left(x^{2} - 5\right)} x}\;dx = \answer{{-4 \, {\left(x^{2} - 5\right)}^{2}}+C}\]
\end{problem}}%}

%%%%%%%%%%%%%%%%%%%%%%


\latexProblemContent{
\begin{problem}

Compute the indefinite integral:

\input{2311-Compute-Integral-0016.HELP.tex}

\[\int\;{\frac{8 \, {\left(\frac{1}{x} + 5\right)}}{x^{2}}}\;dx = \answer{{-4 \, {\left(\frac{1}{x} + 5\right)}^{2}}+C}\]
\end{problem}}%}

%%%%%%%%%%%%%%%%%%%%%%


\latexProblemContent{
\begin{problem}

Compute the indefinite integral:

\input{2311-Compute-Integral-0016.HELP.tex}

\[\int\;{-2 \, {\left(x^{2} + 10\right)} x}\;dx = \answer{{-\frac{1}{2} \, {\left(x^{2} + 10\right)}^{2}}+C}\]
\end{problem}}%}

%%%%%%%%%%%%%%%%%%%%%%


\latexProblemContent{
\begin{problem}

Compute the indefinite integral:

\input{2311-Compute-Integral-0016.HELP.tex}

\[\int\;{4 \, {\left(e^{x} - 2\right)} e^{x}}\;dx = \answer{{2 \, {\left(e^{x} - 2\right)}^{2}}+C}\]
\end{problem}}%}

%%%%%%%%%%%%%%%%%%%%%%


\latexProblemContent{
\begin{problem}

Compute the indefinite integral:

\input{2311-Compute-Integral-0016.HELP.tex}

\[\int\;{\frac{2 \, {\left(\log\left(x\right) - 6\right)}}{x}}\;dx = \answer{{{\left(\log\left(x\right) - 6\right)}^{2}}+C}\]
\end{problem}}%}

%%%%%%%%%%%%%%%%%%%%%%


\latexProblemContent{
\begin{problem}

Compute the indefinite integral:

\input{2311-Compute-Integral-0016.HELP.tex}

\[\int\;{-\frac{\sqrt{x} - 1}{\sqrt{x}}}\;dx = \answer{{-{\left(\sqrt{x} - 1\right)}^{2}}+C}\]
\end{problem}}%}

%%%%%%%%%%%%%%%%%%%%%%


%%%%%%%%%%%%%%%%%%%%%%


\latexProblemContent{
\begin{problem}

Compute the indefinite integral:

\input{2311-Compute-Integral-0016.HELP.tex}

\[\int\;{-8 \, {\left(e^{x} - 5\right)} e^{x}}\;dx = \answer{{-4 \, {\left(e^{x} - 5\right)}^{2}}+C}\]
\end{problem}}%}

%%%%%%%%%%%%%%%%%%%%%%


\latexProblemContent{
\begin{problem}

Compute the indefinite integral:

\input{2311-Compute-Integral-0016.HELP.tex}

\[\int\;{-3 \, {\left(\cos\left(x\right) - 10\right)} \sin\left(x\right)}\;dx = \answer{{\frac{3}{2} \, {\left(\cos\left(x\right) - 10\right)}^{2}}+C}\]
\end{problem}}%}

%%%%%%%%%%%%%%%%%%%%%%


\latexProblemContent{
\begin{problem}

Compute the indefinite integral:

\input{2311-Compute-Integral-0016.HELP.tex}

\[\int\;{\frac{10 \, {\left(\frac{1}{x} + 10\right)}}{x^{2}}}\;dx = \answer{{-5 \, {\left(\frac{1}{x} + 10\right)}^{2}}+C}\]
\end{problem}}%}

%%%%%%%%%%%%%%%%%%%%%%


\latexProblemContent{
\begin{problem}

Compute the indefinite integral:

\input{2311-Compute-Integral-0016.HELP.tex}

\[\int\;{21 \, {\left(x^{3} - 6\right)} x^{2}}\;dx = \answer{{\frac{7}{2} \, {\left(x^{3} - 6\right)}^{2}}+C}\]
\end{problem}}%}

%%%%%%%%%%%%%%%%%%%%%%


\latexProblemContent{
\begin{problem}

Compute the indefinite integral:

\input{2311-Compute-Integral-0016.HELP.tex}

\[\int\;{10 \, {\left(\cos\left(x\right) - 4\right)} \sin\left(x\right)}\;dx = \answer{{-5 \, {\left(\cos\left(x\right) - 4\right)}^{2}}+C}\]
\end{problem}}%}

%%%%%%%%%%%%%%%%%%%%%%


\latexProblemContent{
\begin{problem}

Compute the indefinite integral:

\input{2311-Compute-Integral-0016.HELP.tex}

\[\int\;{-6 \, x - 18}\;dx = \answer{{-3 \, x^{2} - 18 \, x}+C}\]
\end{problem}}%}

%%%%%%%%%%%%%%%%%%%%%%


\latexProblemContent{
\begin{problem}

Compute the indefinite integral:

\input{2311-Compute-Integral-0016.HELP.tex}

\[\int\;{-\frac{3 \, {\left(\frac{1}{x^{3}} - 3\right)}}{x^{4}}}\;dx = \answer{{\frac{1}{2} \, {\left(\frac{1}{x^{3}} - 3\right)}^{2}}+C}\]
\end{problem}}%}

%%%%%%%%%%%%%%%%%%%%%%


\latexProblemContent{
\begin{problem}

Compute the indefinite integral:

\input{2311-Compute-Integral-0016.HELP.tex}

\[\int\;{-2 \, {\left(\cos\left(x\right) - 6\right)} \sin\left(x\right)}\;dx = \answer{{{\left(\cos\left(x\right) - 6\right)}^{2}}+C}\]
\end{problem}}%}

%%%%%%%%%%%%%%%%%%%%%%


\latexProblemContent{
\begin{problem}

Compute the indefinite integral:

\input{2311-Compute-Integral-0016.HELP.tex}

\[\int\;{8 \, x - 24}\;dx = \answer{{4 \, x^{2} - 24 \, x}+C}\]
\end{problem}}%}

%%%%%%%%%%%%%%%%%%%%%%


\latexProblemContent{
\begin{problem}

Compute the indefinite integral:

\input{2311-Compute-Integral-0016.HELP.tex}

\[\int\;{{\left(\sin\left(x\right) + 7\right)} \cos\left(x\right)}\;dx = \answer{{\frac{1}{2} \, {\left(\sin\left(x\right) + 7\right)}^{2}}+C}\]
\end{problem}}%}

%%%%%%%%%%%%%%%%%%%%%%


\latexProblemContent{
\begin{problem}

Compute the indefinite integral:

\input{2311-Compute-Integral-0016.HELP.tex}

\[\int\;{-\frac{20 \, {\left(\frac{1}{x^{2}} - 7\right)}}{x^{3}}}\;dx = \answer{{5 \, {\left(\frac{1}{x^{2}} - 7\right)}^{2}}+C}\]
\end{problem}}%}

%%%%%%%%%%%%%%%%%%%%%%


\latexProblemContent{
\begin{problem}

Compute the indefinite integral:

\input{2311-Compute-Integral-0016.HELP.tex}

\[\int\;{-10 \, {\left(\sin\left(x\right) + 8\right)} \cos\left(x\right)}\;dx = \answer{{-5 \, {\left(\sin\left(x\right) + 8\right)}^{2}}+C}\]
\end{problem}}%}

%%%%%%%%%%%%%%%%%%%%%%


\latexProblemContent{
\begin{problem}

Compute the indefinite integral:

\input{2311-Compute-Integral-0016.HELP.tex}

\[\int\;{-7 \, x - 35}\;dx = \answer{{-\frac{7}{2} \, x^{2} - 35 \, x}+C}\]
\end{problem}}%}

%%%%%%%%%%%%%%%%%%%%%%


\latexProblemContent{
\begin{problem}

Compute the indefinite integral:

\input{2311-Compute-Integral-0016.HELP.tex}

\[\int\;{-4 \, {\left(e^{x} - 1\right)} e^{x}}\;dx = \answer{{-2 \, {\left(e^{x} - 1\right)}^{2}}+C}\]
\end{problem}}%}

%%%%%%%%%%%%%%%%%%%%%%


\latexProblemContent{
\begin{problem}

Compute the indefinite integral:

\input{2311-Compute-Integral-0016.HELP.tex}

\[\int\;{-\frac{2 \, {\left(\frac{1}{x^{2}} + 10\right)}}{x^{3}}}\;dx = \answer{{\frac{1}{2} \, {\left(\frac{1}{x^{2}} + 10\right)}^{2}}+C}\]
\end{problem}}%}

%%%%%%%%%%%%%%%%%%%%%%


\latexProblemContent{
\begin{problem}

Compute the indefinite integral:

\input{2311-Compute-Integral-0016.HELP.tex}

\[\int\;{-\frac{3 \, {\left(\log\left(x\right) - 4\right)}}{x}}\;dx = \answer{{-\frac{3}{2} \, {\left(\log\left(x\right) - 4\right)}^{2}}+C}\]
\end{problem}}%}

%%%%%%%%%%%%%%%%%%%%%%


\latexProblemContent{
\begin{problem}

Compute the indefinite integral:

\input{2311-Compute-Integral-0016.HELP.tex}

\[\int\;{\frac{10 \, {\left(\frac{1}{x} - 3\right)}}{x^{2}}}\;dx = \answer{{-5 \, {\left(\frac{1}{x} - 3\right)}^{2}}+C}\]
\end{problem}}%}

%%%%%%%%%%%%%%%%%%%%%%


\latexProblemContent{
\begin{problem}

Compute the indefinite integral:

\input{2311-Compute-Integral-0016.HELP.tex}

\[\int\;{-5 \, {\left(e^{x} - 1\right)} e^{x}}\;dx = \answer{{-\frac{5}{2} \, {\left(e^{x} - 1\right)}^{2}}+C}\]
\end{problem}}%}

%%%%%%%%%%%%%%%%%%%%%%


\latexProblemContent{
\begin{problem}

Compute the indefinite integral:

\input{2311-Compute-Integral-0016.HELP.tex}

\[\int\;{-\frac{3 \, {\left(\log\left(x\right) + 6\right)}}{x}}\;dx = \answer{{-\frac{3}{2} \, {\left(\log\left(x\right) + 6\right)}^{2}}+C}\]
\end{problem}}%}

%%%%%%%%%%%%%%%%%%%%%%


\latexProblemContent{
\begin{problem}

Compute the indefinite integral:

\input{2311-Compute-Integral-0016.HELP.tex}

\[\int\;{\frac{3 \, {\left(\sqrt{x} + 5\right)}}{\sqrt{x}}}\;dx = \answer{{3 \, {\left(\sqrt{x} + 5\right)}^{2}}+C}\]
\end{problem}}%}

%%%%%%%%%%%%%%%%%%%%%%


\latexProblemContent{
\begin{problem}

Compute the indefinite integral:

\input{2311-Compute-Integral-0016.HELP.tex}

\[\int\;{-12 \, {\left(x^{3} - 5\right)} x^{2}}\;dx = \answer{{-2 \, {\left(x^{3} - 5\right)}^{2}}+C}\]
\end{problem}}%}

%%%%%%%%%%%%%%%%%%%%%%


\latexProblemContent{
\begin{problem}

Compute the indefinite integral:

\input{2311-Compute-Integral-0016.HELP.tex}

\[\int\;{21 \, {\left(x^{3} - 2\right)} x^{2}}\;dx = \answer{{\frac{7}{2} \, {\left(x^{3} - 2\right)}^{2}}+C}\]
\end{problem}}%}

%%%%%%%%%%%%%%%%%%%%%%


\latexProblemContent{
\begin{problem}

Compute the indefinite integral:

\input{2311-Compute-Integral-0016.HELP.tex}

\[\int\;{-\frac{4 \, {\left(\frac{1}{x^{2}} + 3\right)}}{x^{3}}}\;dx = \answer{{{\left(\frac{1}{x^{2}} + 3\right)}^{2}}+C}\]
\end{problem}}%}

%%%%%%%%%%%%%%%%%%%%%%


\latexProblemContent{
\begin{problem}

Compute the indefinite integral:

\input{2311-Compute-Integral-0016.HELP.tex}

\[\int\;{\frac{5 \, {\left(\sqrt{x} + 7\right)}}{\sqrt{x}}}\;dx = \answer{{5 \, {\left(\sqrt{x} + 7\right)}^{2}}+C}\]
\end{problem}}%}

%%%%%%%%%%%%%%%%%%%%%%


\latexProblemContent{
\begin{problem}

Compute the indefinite integral:

\input{2311-Compute-Integral-0016.HELP.tex}

\[\int\;{-18 \, {\left(x^{3} + 2\right)} x^{2}}\;dx = \answer{{-3 \, {\left(x^{3} + 2\right)}^{2}}+C}\]
\end{problem}}%}

%%%%%%%%%%%%%%%%%%%%%%


\latexProblemContent{
\begin{problem}

Compute the indefinite integral:

\input{2311-Compute-Integral-0016.HELP.tex}

\[\int\;{8 \, {\left(\cos\left(x\right) - 4\right)} \sin\left(x\right)}\;dx = \answer{{-4 \, {\left(\cos\left(x\right) - 4\right)}^{2}}+C}\]
\end{problem}}%}

%%%%%%%%%%%%%%%%%%%%%%


\latexProblemContent{
\begin{problem}

Compute the indefinite integral:

\input{2311-Compute-Integral-0016.HELP.tex}

\[\int\;{\frac{\log\left(x\right) - 7}{x}}\;dx = \answer{{\frac{1}{2} \, {\left(\log\left(x\right) - 7\right)}^{2}}+C}\]
\end{problem}}%}

%%%%%%%%%%%%%%%%%%%%%%


\latexProblemContent{
\begin{problem}

Compute the indefinite integral:

\input{2311-Compute-Integral-0016.HELP.tex}

\[\int\;{\frac{14 \, {\left(\frac{1}{x^{2}} - 1\right)}}{x^{3}}}\;dx = \answer{{-\frac{7}{2} \, {\left(\frac{1}{x^{2}} - 1\right)}^{2}}+C}\]
\end{problem}}%}

%%%%%%%%%%%%%%%%%%%%%%


\latexProblemContent{
\begin{problem}

Compute the indefinite integral:

\input{2311-Compute-Integral-0016.HELP.tex}

\[\int\;{30 \, {\left(x^{3} - 9\right)} x^{2}}\;dx = \answer{{5 \, {\left(x^{3} - 9\right)}^{2}}+C}\]
\end{problem}}%}

%%%%%%%%%%%%%%%%%%%%%%


\latexProblemContent{
\begin{problem}

Compute the indefinite integral:

\input{2311-Compute-Integral-0016.HELP.tex}

\[\int\;{-\frac{2 \, {\left(\frac{1}{x^{2}} + 5\right)}}{x^{3}}}\;dx = \answer{{\frac{1}{2} \, {\left(\frac{1}{x^{2}} + 5\right)}^{2}}+C}\]
\end{problem}}%}

%%%%%%%%%%%%%%%%%%%%%%


\latexProblemContent{
\begin{problem}

Compute the indefinite integral:

\input{2311-Compute-Integral-0016.HELP.tex}

\[\int\;{2 \, {\left(x^{2} - 9\right)} x}\;dx = \answer{{\frac{1}{2} \, {\left(x^{2} - 9\right)}^{2}}+C}\]
\end{problem}}%}

%%%%%%%%%%%%%%%%%%%%%%


\latexProblemContent{
\begin{problem}

Compute the indefinite integral:

\input{2311-Compute-Integral-0016.HELP.tex}

\[\int\;{-\frac{18 \, {\left(\frac{1}{x^{3}} - 1\right)}}{x^{4}}}\;dx = \answer{{3 \, {\left(\frac{1}{x^{3}} - 1\right)}^{2}}+C}\]
\end{problem}}%}

%%%%%%%%%%%%%%%%%%%%%%


\latexProblemContent{
\begin{problem}

Compute the indefinite integral:

\input{2311-Compute-Integral-0016.HELP.tex}

\[\int\;{4 \, {\left(x^{2} + 2\right)} x}\;dx = \answer{{{\left(x^{2} + 2\right)}^{2}}+C}\]
\end{problem}}%}

%%%%%%%%%%%%%%%%%%%%%%


\latexProblemContent{
\begin{problem}

Compute the indefinite integral:

\input{2311-Compute-Integral-0016.HELP.tex}

\[\int\;{\frac{20 \, {\left(\frac{1}{x^{2}} + 9\right)}}{x^{3}}}\;dx = \answer{{-5 \, {\left(\frac{1}{x^{2}} + 9\right)}^{2}}+C}\]
\end{problem}}%}

%%%%%%%%%%%%%%%%%%%%%%


\latexProblemContent{
\begin{problem}

Compute the indefinite integral:

\input{2311-Compute-Integral-0016.HELP.tex}

\[\int\;{-{\left(\cos\left(x\right) + 9\right)} \sin\left(x\right)}\;dx = \answer{{\frac{1}{2} \, {\left(\cos\left(x\right) + 9\right)}^{2}}+C}\]
\end{problem}}%}

%%%%%%%%%%%%%%%%%%%%%%


\latexProblemContent{
\begin{problem}

Compute the indefinite integral:

\input{2311-Compute-Integral-0016.HELP.tex}

\[\int\;{-x + 4}\;dx = \answer{{-\frac{1}{2} \, x^{2} + 4 \, x}+C}\]
\end{problem}}%}

%%%%%%%%%%%%%%%%%%%%%%


\latexProblemContent{
\begin{problem}

Compute the indefinite integral:

\input{2311-Compute-Integral-0016.HELP.tex}

\[\int\;{-\frac{2 \, {\left(\frac{1}{x} + 9\right)}}{x^{2}}}\;dx = \answer{{{\left(\frac{1}{x} + 9\right)}^{2}}+C}\]
\end{problem}}%}

%%%%%%%%%%%%%%%%%%%%%%


\latexProblemContent{
\begin{problem}

Compute the indefinite integral:

\input{2311-Compute-Integral-0016.HELP.tex}

\[\int\;{-8 \, {\left(\cos\left(x\right) + 10\right)} \sin\left(x\right)}\;dx = \answer{{4 \, {\left(\cos\left(x\right) + 10\right)}^{2}}+C}\]
\end{problem}}%}

%%%%%%%%%%%%%%%%%%%%%%


\latexProblemContent{
\begin{problem}

Compute the indefinite integral:

\input{2311-Compute-Integral-0016.HELP.tex}

\[\int\;{30 \, {\left(x^{3} + 10\right)} x^{2}}\;dx = \answer{{5 \, {\left(x^{3} + 10\right)}^{2}}+C}\]
\end{problem}}%}

%%%%%%%%%%%%%%%%%%%%%%


\latexProblemContent{
\begin{problem}

Compute the indefinite integral:

\input{2311-Compute-Integral-0016.HELP.tex}

\[\int\;{20 \, {\left(x^{2} - 5\right)} x}\;dx = \answer{{5 \, {\left(x^{2} - 5\right)}^{2}}+C}\]
\end{problem}}%}

%%%%%%%%%%%%%%%%%%%%%%


\latexProblemContent{
\begin{problem}

Compute the indefinite integral:

\input{2311-Compute-Integral-0016.HELP.tex}

\[\int\;{-\frac{12 \, {\left(\frac{1}{x^{2}} + 3\right)}}{x^{3}}}\;dx = \answer{{3 \, {\left(\frac{1}{x^{2}} + 3\right)}^{2}}+C}\]
\end{problem}}%}

%%%%%%%%%%%%%%%%%%%%%%


\latexProblemContent{
\begin{problem}

Compute the indefinite integral:

\input{2311-Compute-Integral-0016.HELP.tex}

\[\int\;{-\frac{27 \, {\left(\frac{1}{x^{3}} - 1\right)}}{x^{4}}}\;dx = \answer{{\frac{9}{2} \, {\left(\frac{1}{x^{3}} - 1\right)}^{2}}+C}\]
\end{problem}}%}

%%%%%%%%%%%%%%%%%%%%%%


\latexProblemContent{
\begin{problem}

Compute the indefinite integral:

\input{2311-Compute-Integral-0016.HELP.tex}

\[\int\;{6 \, {\left(e^{x} - 4\right)} e^{x}}\;dx = \answer{{3 \, {\left(e^{x} - 4\right)}^{2}}+C}\]
\end{problem}}%}

%%%%%%%%%%%%%%%%%%%%%%


\latexProblemContent{
\begin{problem}

Compute the indefinite integral:

\input{2311-Compute-Integral-0016.HELP.tex}

\[\int\;{7 \, {\left(\sin\left(x\right) + 3\right)} \cos\left(x\right)}\;dx = \answer{{\frac{7}{2} \, {\left(\sin\left(x\right) + 3\right)}^{2}}+C}\]
\end{problem}}%}

%%%%%%%%%%%%%%%%%%%%%%


\latexProblemContent{
\begin{problem}

Compute the indefinite integral:

\input{2311-Compute-Integral-0016.HELP.tex}

\[\int\;{2 \, {\left(e^{x} + 10\right)} e^{x}}\;dx = \answer{{{\left(e^{x} + 10\right)}^{2}}+C}\]
\end{problem}}%}

%%%%%%%%%%%%%%%%%%%%%%


\latexProblemContent{
\begin{problem}

Compute the indefinite integral:

\input{2311-Compute-Integral-0016.HELP.tex}

\[\int\;{6 \, {\left(\sin\left(x\right) + 2\right)} \cos\left(x\right)}\;dx = \answer{{3 \, {\left(\sin\left(x\right) + 2\right)}^{2}}+C}\]
\end{problem}}%}

%%%%%%%%%%%%%%%%%%%%%%


\latexProblemContent{
\begin{problem}

Compute the indefinite integral:

\input{2311-Compute-Integral-0016.HELP.tex}

\[\int\;{\frac{2 \, {\left(\frac{1}{x^{2}} + 10\right)}}{x^{3}}}\;dx = \answer{{-\frac{1}{2} \, {\left(\frac{1}{x^{2}} + 10\right)}^{2}}+C}\]
\end{problem}}%}

%%%%%%%%%%%%%%%%%%%%%%


\latexProblemContent{
\begin{problem}

Compute the indefinite integral:

\input{2311-Compute-Integral-0016.HELP.tex}

\[\int\;{\frac{4 \, {\left(\log\left(x\right) + 2\right)}}{x}}\;dx = \answer{{2 \, {\left(\log\left(x\right) + 2\right)}^{2}}+C}\]
\end{problem}}%}

%%%%%%%%%%%%%%%%%%%%%%


\latexProblemContent{
\begin{problem}

Compute the indefinite integral:

\input{2311-Compute-Integral-0016.HELP.tex}

\[\int\;{-\frac{\sqrt{x} + 1}{\sqrt{x}}}\;dx = \answer{{-{\left(\sqrt{x} + 1\right)}^{2}}+C}\]
\end{problem}}%}

%%%%%%%%%%%%%%%%%%%%%%


%%%%%%%%%%%%%%%%%%%%%%


\latexProblemContent{
\begin{problem}

Compute the indefinite integral:

\input{2311-Compute-Integral-0016.HELP.tex}

\[\int\;{10 \, x + 20}\;dx = \answer{{5 \, x^{2} + 20 \, x}+C}\]
\end{problem}}%}

%%%%%%%%%%%%%%%%%%%%%%


\latexProblemContent{
\begin{problem}

Compute the indefinite integral:

\input{2311-Compute-Integral-0016.HELP.tex}

\[\int\;{\frac{4 \, {\left(\frac{1}{x} - 8\right)}}{x^{2}}}\;dx = \answer{{-2 \, {\left(\frac{1}{x} - 8\right)}^{2}}+C}\]
\end{problem}}%}

%%%%%%%%%%%%%%%%%%%%%%


\latexProblemContent{
\begin{problem}

Compute the indefinite integral:

\input{2311-Compute-Integral-0016.HELP.tex}

\[\int\;{-\frac{7 \, {\left(\log\left(x\right) - 4\right)}}{x}}\;dx = \answer{{-\frac{7}{2} \, {\left(\log\left(x\right) - 4\right)}^{2}}+C}\]
\end{problem}}%}

%%%%%%%%%%%%%%%%%%%%%%


\latexProblemContent{
\begin{problem}

Compute the indefinite integral:

\input{2311-Compute-Integral-0016.HELP.tex}

\[\int\;{8 \, {\left(\cos\left(x\right) + 10\right)} \sin\left(x\right)}\;dx = \answer{{-4 \, {\left(\cos\left(x\right) + 10\right)}^{2}}+C}\]
\end{problem}}%}

%%%%%%%%%%%%%%%%%%%%%%


\latexProblemContent{
\begin{problem}

Compute the indefinite integral:

\input{2311-Compute-Integral-0016.HELP.tex}

\[\int\;{\frac{8 \, {\left(\frac{1}{x} + 9\right)}}{x^{2}}}\;dx = \answer{{-4 \, {\left(\frac{1}{x} + 9\right)}^{2}}+C}\]
\end{problem}}%}

%%%%%%%%%%%%%%%%%%%%%%


\latexProblemContent{
\begin{problem}

Compute the indefinite integral:

\input{2311-Compute-Integral-0016.HELP.tex}

\[\int\;{-2 \, {\left(\sin\left(x\right) + 4\right)} \cos\left(x\right)}\;dx = \answer{{-{\left(\sin\left(x\right) + 4\right)}^{2}}+C}\]
\end{problem}}%}

%%%%%%%%%%%%%%%%%%%%%%


\latexProblemContent{
\begin{problem}

Compute the indefinite integral:

\input{2311-Compute-Integral-0016.HELP.tex}

\[\int\;{\frac{5 \, {\left(\sqrt{x} + 6\right)}}{\sqrt{x}}}\;dx = \answer{{5 \, {\left(\sqrt{x} + 6\right)}^{2}}+C}\]
\end{problem}}%}

%%%%%%%%%%%%%%%%%%%%%%


\latexProblemContent{
\begin{problem}

Compute the indefinite integral:

\input{2311-Compute-Integral-0016.HELP.tex}

\[\int\;{-6 \, {\left(\cos\left(x\right) - 4\right)} \sin\left(x\right)}\;dx = \answer{{3 \, {\left(\cos\left(x\right) - 4\right)}^{2}}+C}\]
\end{problem}}%}

%%%%%%%%%%%%%%%%%%%%%%


\latexProblemContent{
\begin{problem}

Compute the indefinite integral:

\input{2311-Compute-Integral-0016.HELP.tex}

\[\int\;{10 \, x + 150}\;dx = \answer{{5 \, x^{2} + 150 \, x}+C}\]
\end{problem}}%}

%%%%%%%%%%%%%%%%%%%%%%


\latexProblemContent{
\begin{problem}

Compute the indefinite integral:

\input{2311-Compute-Integral-0016.HELP.tex}

\[\int\;{-\frac{7 \, {\left(\sqrt{x} + 3\right)}}{2 \, \sqrt{x}}}\;dx = \answer{{-\frac{7}{2} \, {\left(\sqrt{x} + 3\right)}^{2}}+C}\]
\end{problem}}%}

%%%%%%%%%%%%%%%%%%%%%%


%%%%%%%%%%%%%%%%%%%%%%


\latexProblemContent{
\begin{problem}

Compute the indefinite integral:

\input{2311-Compute-Integral-0016.HELP.tex}

\[\int\;{\frac{16 \, {\left(\frac{1}{x^{2}} - 9\right)}}{x^{3}}}\;dx = \answer{{-4 \, {\left(\frac{1}{x^{2}} - 9\right)}^{2}}+C}\]
\end{problem}}%}

%%%%%%%%%%%%%%%%%%%%%%


\latexProblemContent{
\begin{problem}

Compute the indefinite integral:

\input{2311-Compute-Integral-0016.HELP.tex}

\[\int\;{-20 \, {\left(x^{2} - 5\right)} x}\;dx = \answer{{-5 \, {\left(x^{2} - 5\right)}^{2}}+C}\]
\end{problem}}%}

%%%%%%%%%%%%%%%%%%%%%%


\latexProblemContent{
\begin{problem}

Compute the indefinite integral:

\input{2311-Compute-Integral-0016.HELP.tex}

\[\int\;{-\frac{12 \, {\left(\frac{1}{x^{3}} + 9\right)}}{x^{4}}}\;dx = \answer{{2 \, {\left(\frac{1}{x^{3}} + 9\right)}^{2}}+C}\]
\end{problem}}%}

%%%%%%%%%%%%%%%%%%%%%%


\latexProblemContent{
\begin{problem}

Compute the indefinite integral:

\input{2311-Compute-Integral-0016.HELP.tex}

\[\int\;{\frac{6 \, {\left(\log\left(x\right) - 10\right)}}{x}}\;dx = \answer{{3 \, {\left(\log\left(x\right) - 10\right)}^{2}}+C}\]
\end{problem}}%}

%%%%%%%%%%%%%%%%%%%%%%


\latexProblemContent{
\begin{problem}

Compute the indefinite integral:

\input{2311-Compute-Integral-0016.HELP.tex}

\[\int\;{-\frac{9 \, {\left(\log\left(x\right) + 4\right)}}{x}}\;dx = \answer{{-\frac{9}{2} \, {\left(\log\left(x\right) + 4\right)}^{2}}+C}\]
\end{problem}}%}

%%%%%%%%%%%%%%%%%%%%%%


\latexProblemContent{
\begin{problem}

Compute the indefinite integral:

\input{2311-Compute-Integral-0016.HELP.tex}

\[\int\;{-\frac{30 \, {\left(\frac{1}{x^{3}} + 8\right)}}{x^{4}}}\;dx = \answer{{5 \, {\left(\frac{1}{x^{3}} + 8\right)}^{2}}+C}\]
\end{problem}}%}

%%%%%%%%%%%%%%%%%%%%%%


\latexProblemContent{
\begin{problem}

Compute the indefinite integral:

\input{2311-Compute-Integral-0016.HELP.tex}

\[\int\;{-\frac{3 \, {\left(\frac{1}{x^{3}} - 5\right)}}{x^{4}}}\;dx = \answer{{\frac{1}{2} \, {\left(\frac{1}{x^{3}} - 5\right)}^{2}}+C}\]
\end{problem}}%}

%%%%%%%%%%%%%%%%%%%%%%


\latexProblemContent{
\begin{problem}

Compute the indefinite integral:

\input{2311-Compute-Integral-0016.HELP.tex}

\[\int\;{5 \, {\left(\cos\left(x\right) + 5\right)} \sin\left(x\right)}\;dx = \answer{{-\frac{5}{2} \, {\left(\cos\left(x\right) + 5\right)}^{2}}+C}\]
\end{problem}}%}

%%%%%%%%%%%%%%%%%%%%%%


\latexProblemContent{
\begin{problem}

Compute the indefinite integral:

\input{2311-Compute-Integral-0016.HELP.tex}

\[\int\;{\frac{2 \, {\left(\frac{1}{x} - 3\right)}}{x^{2}}}\;dx = \answer{{-{\left(\frac{1}{x} - 3\right)}^{2}}+C}\]
\end{problem}}%}

%%%%%%%%%%%%%%%%%%%%%%


\latexProblemContent{
\begin{problem}

Compute the indefinite integral:

\input{2311-Compute-Integral-0016.HELP.tex}

\[\int\;{-\frac{5 \, {\left(\sqrt{x} - 2\right)}}{\sqrt{x}}}\;dx = \answer{{-5 \, {\left(\sqrt{x} - 2\right)}^{2}}+C}\]
\end{problem}}%}

%%%%%%%%%%%%%%%%%%%%%%


\latexProblemContent{
\begin{problem}

Compute the indefinite integral:

\input{2311-Compute-Integral-0016.HELP.tex}

\[\int\;{8 \, x - 96}\;dx = \answer{{4 \, x^{2} - 96 \, x}+C}\]
\end{problem}}%}

%%%%%%%%%%%%%%%%%%%%%%


\latexProblemContent{
\begin{problem}

Compute the indefinite integral:

\input{2311-Compute-Integral-0016.HELP.tex}

\[\int\;{-18 \, {\left(x^{3} - 6\right)} x^{2}}\;dx = \answer{{-3 \, {\left(x^{3} - 6\right)}^{2}}+C}\]
\end{problem}}%}

%%%%%%%%%%%%%%%%%%%%%%


\latexProblemContent{
\begin{problem}

Compute the indefinite integral:

\input{2311-Compute-Integral-0016.HELP.tex}

\[\int\;{8 \, {\left(\cos\left(x\right) + 4\right)} \sin\left(x\right)}\;dx = \answer{{-4 \, {\left(\cos\left(x\right) + 4\right)}^{2}}+C}\]
\end{problem}}%}

%%%%%%%%%%%%%%%%%%%%%%


\latexProblemContent{
\begin{problem}

Compute the indefinite integral:

\input{2311-Compute-Integral-0016.HELP.tex}

\[\int\;{-\frac{4 \, {\left(\frac{1}{x} - 7\right)}}{x^{2}}}\;dx = \answer{{2 \, {\left(\frac{1}{x} - 7\right)}^{2}}+C}\]
\end{problem}}%}

%%%%%%%%%%%%%%%%%%%%%%


\latexProblemContent{
\begin{problem}

Compute the indefinite integral:

\input{2311-Compute-Integral-0016.HELP.tex}

\[\int\;{-\frac{10 \, {\left(\frac{1}{x} - 3\right)}}{x^{2}}}\;dx = \answer{{5 \, {\left(\frac{1}{x} - 3\right)}^{2}}+C}\]
\end{problem}}%}

%%%%%%%%%%%%%%%%%%%%%%


\latexProblemContent{
\begin{problem}

Compute the indefinite integral:

\input{2311-Compute-Integral-0016.HELP.tex}

\[\int\;{\frac{5 \, {\left(\sqrt{x} + 9\right)}}{2 \, \sqrt{x}}}\;dx = \answer{{\frac{5}{2} \, {\left(\sqrt{x} + 9\right)}^{2}}+C}\]
\end{problem}}%}

%%%%%%%%%%%%%%%%%%%%%%


\latexProblemContent{
\begin{problem}

Compute the indefinite integral:

\input{2311-Compute-Integral-0016.HELP.tex}

\[\int\;{-\frac{5 \, {\left(\log\left(x\right) + 3\right)}}{x}}\;dx = \answer{{-\frac{5}{2} \, {\left(\log\left(x\right) + 3\right)}^{2}}+C}\]
\end{problem}}%}

%%%%%%%%%%%%%%%%%%%%%%


\latexProblemContent{
\begin{problem}

Compute the indefinite integral:

\input{2311-Compute-Integral-0016.HELP.tex}

\[\int\;{-\frac{2 \, {\left(\sqrt{x} + 4\right)}}{\sqrt{x}}}\;dx = \answer{{-2 \, {\left(\sqrt{x} + 4\right)}^{2}}+C}\]
\end{problem}}%}

%%%%%%%%%%%%%%%%%%%%%%


\latexProblemContent{
\begin{problem}

Compute the indefinite integral:

\input{2311-Compute-Integral-0016.HELP.tex}

\[\int\;{-\frac{4 \, {\left(\frac{1}{x^{2}} + 9\right)}}{x^{3}}}\;dx = \answer{{{\left(\frac{1}{x^{2}} + 9\right)}^{2}}+C}\]
\end{problem}}%}

%%%%%%%%%%%%%%%%%%%%%%


\latexProblemContent{
\begin{problem}

Compute the indefinite integral:

\input{2311-Compute-Integral-0016.HELP.tex}

\[\int\;{-5 \, {\left(\cos\left(x\right) + 10\right)} \sin\left(x\right)}\;dx = \answer{{\frac{5}{2} \, {\left(\cos\left(x\right) + 10\right)}^{2}}+C}\]
\end{problem}}%}

%%%%%%%%%%%%%%%%%%%%%%


\latexProblemContent{
\begin{problem}

Compute the indefinite integral:

\input{2311-Compute-Integral-0016.HELP.tex}

\[\int\;{5 \, {\left(e^{x} - 8\right)} e^{x}}\;dx = \answer{{\frac{5}{2} \, {\left(e^{x} - 8\right)}^{2}}+C}\]
\end{problem}}%}

%%%%%%%%%%%%%%%%%%%%%%


\latexProblemContent{
\begin{problem}

Compute the indefinite integral:

\input{2311-Compute-Integral-0016.HELP.tex}

\[\int\;{-6 \, {\left(x^{3} - 10\right)} x^{2}}\;dx = \answer{{-{\left(x^{3} - 10\right)}^{2}}+C}\]
\end{problem}}%}

%%%%%%%%%%%%%%%%%%%%%%


\latexProblemContent{
\begin{problem}

Compute the indefinite integral:

\input{2311-Compute-Integral-0016.HELP.tex}

\[\int\;{10 \, {\left(\sin\left(x\right) - 4\right)} \cos\left(x\right)}\;dx = \answer{{5 \, {\left(\sin\left(x\right) - 4\right)}^{2}}+C}\]
\end{problem}}%}

%%%%%%%%%%%%%%%%%%%%%%


\latexProblemContent{
\begin{problem}

Compute the indefinite integral:

\input{2311-Compute-Integral-0016.HELP.tex}

\[\int\;{-30 \, {\left(x^{3} - 2\right)} x^{2}}\;dx = \answer{{-5 \, {\left(x^{3} - 2\right)}^{2}}+C}\]
\end{problem}}%}

%%%%%%%%%%%%%%%%%%%%%%


\latexProblemContent{
\begin{problem}

Compute the indefinite integral:

\input{2311-Compute-Integral-0016.HELP.tex}

\[\int\;{\frac{8 \, {\left(\frac{1}{x^{2}} - 10\right)}}{x^{3}}}\;dx = \answer{{-2 \, {\left(\frac{1}{x^{2}} - 10\right)}^{2}}+C}\]
\end{problem}}%}

%%%%%%%%%%%%%%%%%%%%%%


\latexProblemContent{
\begin{problem}

Compute the indefinite integral:

\input{2311-Compute-Integral-0016.HELP.tex}

\[\int\;{-12 \, {\left(x^{2} - 4\right)} x}\;dx = \answer{{-3 \, {\left(x^{2} - 4\right)}^{2}}+C}\]
\end{problem}}%}

%%%%%%%%%%%%%%%%%%%%%%


\latexProblemContent{
\begin{problem}

Compute the indefinite integral:

\input{2311-Compute-Integral-0016.HELP.tex}

\[\int\;{-\frac{5 \, {\left(\log\left(x\right) + 4\right)}}{x}}\;dx = \answer{{-\frac{5}{2} \, {\left(\log\left(x\right) + 4\right)}^{2}}+C}\]
\end{problem}}%}

%%%%%%%%%%%%%%%%%%%%%%


\latexProblemContent{
\begin{problem}

Compute the indefinite integral:

\input{2311-Compute-Integral-0016.HELP.tex}

\[\int\;{-\frac{9 \, {\left(\sqrt{x} + 9\right)}}{2 \, \sqrt{x}}}\;dx = \answer{{-\frac{9}{2} \, {\left(\sqrt{x} + 9\right)}^{2}}+C}\]
\end{problem}}%}

%%%%%%%%%%%%%%%%%%%%%%


\latexProblemContent{
\begin{problem}

Compute the indefinite integral:

\input{2311-Compute-Integral-0016.HELP.tex}

\[\int\;{-6 \, {\left(x^{3} - 9\right)} x^{2}}\;dx = \answer{{-{\left(x^{3} - 9\right)}^{2}}+C}\]
\end{problem}}%}

%%%%%%%%%%%%%%%%%%%%%%


\latexProblemContent{
\begin{problem}

Compute the indefinite integral:

\input{2311-Compute-Integral-0016.HELP.tex}

\[\int\;{18 \, {\left(x^{2} - 9\right)} x}\;dx = \answer{{\frac{9}{2} \, {\left(x^{2} - 9\right)}^{2}}+C}\]
\end{problem}}%}

%%%%%%%%%%%%%%%%%%%%%%


\latexProblemContent{
\begin{problem}

Compute the indefinite integral:

\input{2311-Compute-Integral-0016.HELP.tex}

\[\int\;{\frac{12 \, {\left(\frac{1}{x^{2}} - 8\right)}}{x^{3}}}\;dx = \answer{{-3 \, {\left(\frac{1}{x^{2}} - 8\right)}^{2}}+C}\]
\end{problem}}%}

%%%%%%%%%%%%%%%%%%%%%%


\latexProblemContent{
\begin{problem}

Compute the indefinite integral:

\input{2311-Compute-Integral-0016.HELP.tex}

\[\int\;{9 \, x + 45}\;dx = \answer{{\frac{9}{2} \, x^{2} + 45 \, x}+C}\]
\end{problem}}%}

%%%%%%%%%%%%%%%%%%%%%%


\latexProblemContent{
\begin{problem}

Compute the indefinite integral:

\input{2311-Compute-Integral-0016.HELP.tex}

\[\int\;{9 \, {\left(\sin\left(x\right) + 2\right)} \cos\left(x\right)}\;dx = \answer{{\frac{9}{2} \, {\left(\sin\left(x\right) + 2\right)}^{2}}+C}\]
\end{problem}}%}

%%%%%%%%%%%%%%%%%%%%%%


\latexProblemContent{
\begin{problem}

Compute the indefinite integral:

\input{2311-Compute-Integral-0016.HELP.tex}

\[\int\;{10 \, {\left(\cos\left(x\right) + 3\right)} \sin\left(x\right)}\;dx = \answer{{-5 \, {\left(\cos\left(x\right) + 3\right)}^{2}}+C}\]
\end{problem}}%}

%%%%%%%%%%%%%%%%%%%%%%


\latexProblemContent{
\begin{problem}

Compute the indefinite integral:

\input{2311-Compute-Integral-0016.HELP.tex}

\[\int\;{-6 \, x - 102}\;dx = \answer{{-3 \, x^{2} - 102 \, x}+C}\]
\end{problem}}%}

%%%%%%%%%%%%%%%%%%%%%%


%%%%%%%%%%%%%%%%%%%%%%


\latexProblemContent{
\begin{problem}

Compute the indefinite integral:

\input{2311-Compute-Integral-0016.HELP.tex}

\[\int\;{2 \, {\left(\cos\left(x\right) + 1\right)} \sin\left(x\right)}\;dx = \answer{{-{\left(\cos\left(x\right) + 1\right)}^{2}}+C}\]
\end{problem}}%}

%%%%%%%%%%%%%%%%%%%%%%


\latexProblemContent{
\begin{problem}

Compute the indefinite integral:

\input{2311-Compute-Integral-0016.HELP.tex}

\[\int\;{6 \, {\left(x^{3} + 10\right)} x^{2}}\;dx = \answer{{{\left(x^{3} + 10\right)}^{2}}+C}\]
\end{problem}}%}

%%%%%%%%%%%%%%%%%%%%%%


%%%%%%%%%%%%%%%%%%%%%%


\latexProblemContent{
\begin{problem}

Compute the indefinite integral:

\input{2311-Compute-Integral-0016.HELP.tex}

\[\int\;{\frac{7 \, {\left(\sqrt{x} + 10\right)}}{2 \, \sqrt{x}}}\;dx = \answer{{\frac{7}{2} \, {\left(\sqrt{x} + 10\right)}^{2}}+C}\]
\end{problem}}%}

%%%%%%%%%%%%%%%%%%%%%%


\latexProblemContent{
\begin{problem}

Compute the indefinite integral:

\input{2311-Compute-Integral-0016.HELP.tex}

\[\int\;{7 \, {\left(e^{x} + 5\right)} e^{x}}\;dx = \answer{{\frac{7}{2} \, {\left(e^{x} + 5\right)}^{2}}+C}\]
\end{problem}}%}

%%%%%%%%%%%%%%%%%%%%%%


\latexProblemContent{
\begin{problem}

Compute the indefinite integral:

\input{2311-Compute-Integral-0016.HELP.tex}

\[\int\;{\frac{6 \, {\left(\frac{1}{x} + 1\right)}}{x^{2}}}\;dx = \answer{{-3 \, {\left(\frac{1}{x} + 1\right)}^{2}}+C}\]
\end{problem}}%}

%%%%%%%%%%%%%%%%%%%%%%


\latexProblemContent{
\begin{problem}

Compute the indefinite integral:

\input{2311-Compute-Integral-0016.HELP.tex}

\[\int\;{-6 \, {\left(x^{3} + 10\right)} x^{2}}\;dx = \answer{{-{\left(x^{3} + 10\right)}^{2}}+C}\]
\end{problem}}%}

%%%%%%%%%%%%%%%%%%%%%%


\latexProblemContent{
\begin{problem}

Compute the indefinite integral:

\input{2311-Compute-Integral-0016.HELP.tex}

\[\int\;{-36 \, {\left(x^{4} + 7\right)} x^{3}}\;dx = \answer{{-\frac{9}{2} \, {\left(x^{4} + 7\right)}^{2}}+C}\]
\end{problem}}%}

%%%%%%%%%%%%%%%%%%%%%%


\latexProblemContent{
\begin{problem}

Compute the indefinite integral:

\input{2311-Compute-Integral-0016.HELP.tex}

\[\int\;{-10 \, x + 110}\;dx = \answer{{-5 \, x^{2} + 110 \, x}+C}\]
\end{problem}}%}

%%%%%%%%%%%%%%%%%%%%%%


\latexProblemContent{
\begin{problem}

Compute the indefinite integral:

\input{2311-Compute-Integral-0016.HELP.tex}

\[\int\;{-4 \, {\left(\cos\left(x\right) - 6\right)} \sin\left(x\right)}\;dx = \answer{{2 \, {\left(\cos\left(x\right) - 6\right)}^{2}}+C}\]
\end{problem}}%}

%%%%%%%%%%%%%%%%%%%%%%


\latexProblemContent{
\begin{problem}

Compute the indefinite integral:

\input{2311-Compute-Integral-0016.HELP.tex}

\[\int\;{{\left(\sin\left(x\right) + 6\right)} \cos\left(x\right)}\;dx = \answer{{\frac{1}{2} \, {\left(\sin\left(x\right) + 6\right)}^{2}}+C}\]
\end{problem}}%}

%%%%%%%%%%%%%%%%%%%%%%


\latexProblemContent{
\begin{problem}

Compute the indefinite integral:

\input{2311-Compute-Integral-0016.HELP.tex}

\[\int\;{8 \, {\left(x^{4} - 9\right)} x^{3}}\;dx = \answer{{{\left(x^{4} - 9\right)}^{2}}+C}\]
\end{problem}}%}

%%%%%%%%%%%%%%%%%%%%%%


\latexProblemContent{
\begin{problem}

Compute the indefinite integral:

\input{2311-Compute-Integral-0016.HELP.tex}

\[\int\;{-8 \, {\left(x^{4} + 2\right)} x^{3}}\;dx = \answer{{-{\left(x^{4} + 2\right)}^{2}}+C}\]
\end{problem}}%}

%%%%%%%%%%%%%%%%%%%%%%


\latexProblemContent{
\begin{problem}

Compute the indefinite integral:

\input{2311-Compute-Integral-0016.HELP.tex}

\[\int\;{-\frac{2 \, {\left(\frac{1}{x} + 8\right)}}{x^{2}}}\;dx = \answer{{{\left(\frac{1}{x} + 8\right)}^{2}}+C}\]
\end{problem}}%}

%%%%%%%%%%%%%%%%%%%%%%


%%%%%%%%%%%%%%%%%%%%%%


\latexProblemContent{
\begin{problem}

Compute the indefinite integral:

\input{2311-Compute-Integral-0016.HELP.tex}

\[\int\;{2 \, {\left(\cos\left(x\right) - 9\right)} \sin\left(x\right)}\;dx = \answer{{-{\left(\cos\left(x\right) - 9\right)}^{2}}+C}\]
\end{problem}}%}

%%%%%%%%%%%%%%%%%%%%%%


\latexProblemContent{
\begin{problem}

Compute the indefinite integral:

\input{2311-Compute-Integral-0016.HELP.tex}

\[\int\;{-\frac{9 \, {\left(\frac{1}{x} + 1\right)}}{x^{2}}}\;dx = \answer{{\frac{9}{2} \, {\left(\frac{1}{x} + 1\right)}^{2}}+C}\]
\end{problem}}%}

%%%%%%%%%%%%%%%%%%%%%%


\latexProblemContent{
\begin{problem}

Compute the indefinite integral:

\input{2311-Compute-Integral-0016.HELP.tex}

\[\int\;{5 \, {\left(\cos\left(x\right) - 7\right)} \sin\left(x\right)}\;dx = \answer{{-\frac{5}{2} \, {\left(\cos\left(x\right) - 7\right)}^{2}}+C}\]
\end{problem}}%}

%%%%%%%%%%%%%%%%%%%%%%


\latexProblemContent{
\begin{problem}

Compute the indefinite integral:

\input{2311-Compute-Integral-0016.HELP.tex}

\[\int\;{\frac{2 \, {\left(\log\left(x\right) - 5\right)}}{x}}\;dx = \answer{{{\left(\log\left(x\right) - 5\right)}^{2}}+C}\]
\end{problem}}%}

%%%%%%%%%%%%%%%%%%%%%%


\latexProblemContent{
\begin{problem}

Compute the indefinite integral:

\input{2311-Compute-Integral-0016.HELP.tex}

\[\int\;{\frac{2 \, {\left(\log\left(x\right) + 8\right)}}{x}}\;dx = \answer{{{\left(\log\left(x\right) + 8\right)}^{2}}+C}\]
\end{problem}}%}

%%%%%%%%%%%%%%%%%%%%%%


\latexProblemContent{
\begin{problem}

Compute the indefinite integral:

\input{2311-Compute-Integral-0016.HELP.tex}

\[\int\;{5 \, {\left(e^{x} + 2\right)} e^{x}}\;dx = \answer{{\frac{5}{2} \, {\left(e^{x} + 2\right)}^{2}}+C}\]
\end{problem}}%}

%%%%%%%%%%%%%%%%%%%%%%


\latexProblemContent{
\begin{problem}

Compute the indefinite integral:

\input{2311-Compute-Integral-0016.HELP.tex}

\[\int\;{3 \, {\left(\sin\left(x\right) - 10\right)} \cos\left(x\right)}\;dx = \answer{{\frac{3}{2} \, {\left(\sin\left(x\right) - 10\right)}^{2}}+C}\]
\end{problem}}%}

%%%%%%%%%%%%%%%%%%%%%%


%%%%%%%%%%%%%%%%%%%%%%


\latexProblemContent{
\begin{problem}

Compute the indefinite integral:

\input{2311-Compute-Integral-0016.HELP.tex}

\[\int\;{-\frac{6 \, {\left(\frac{1}{x^{2}} + 9\right)}}{x^{3}}}\;dx = \answer{{\frac{3}{2} \, {\left(\frac{1}{x^{2}} + 9\right)}^{2}}+C}\]
\end{problem}}%}

%%%%%%%%%%%%%%%%%%%%%%


\latexProblemContent{
\begin{problem}

Compute the indefinite integral:

\input{2311-Compute-Integral-0016.HELP.tex}

\[\int\;{30 \, {\left(x^{3} - 5\right)} x^{2}}\;dx = \answer{{5 \, {\left(x^{3} - 5\right)}^{2}}+C}\]
\end{problem}}%}

%%%%%%%%%%%%%%%%%%%%%%


\latexProblemContent{
\begin{problem}

Compute the indefinite integral:

\input{2311-Compute-Integral-0016.HELP.tex}

\[\int\;{-21 \, {\left(x^{3} - 7\right)} x^{2}}\;dx = \answer{{-\frac{7}{2} \, {\left(x^{3} - 7\right)}^{2}}+C}\]
\end{problem}}%}

%%%%%%%%%%%%%%%%%%%%%%


\latexProblemContent{
\begin{problem}

Compute the indefinite integral:

\input{2311-Compute-Integral-0016.HELP.tex}

\[\int\;{28 \, {\left(x^{4} + 9\right)} x^{3}}\;dx = \answer{{\frac{7}{2} \, {\left(x^{4} + 9\right)}^{2}}+C}\]
\end{problem}}%}

%%%%%%%%%%%%%%%%%%%%%%


\latexProblemContent{
\begin{problem}

Compute the indefinite integral:

\input{2311-Compute-Integral-0016.HELP.tex}

\[\int\;{\frac{8 \, {\left(\frac{1}{x^{2}} - 8\right)}}{x^{3}}}\;dx = \answer{{-2 \, {\left(\frac{1}{x^{2}} - 8\right)}^{2}}+C}\]
\end{problem}}%}

%%%%%%%%%%%%%%%%%%%%%%


\latexProblemContent{
\begin{problem}

Compute the indefinite integral:

\input{2311-Compute-Integral-0016.HELP.tex}

\[\int\;{9 \, {\left(\cos\left(x\right) + 8\right)} \sin\left(x\right)}\;dx = \answer{{-\frac{9}{2} \, {\left(\cos\left(x\right) + 8\right)}^{2}}+C}\]
\end{problem}}%}

%%%%%%%%%%%%%%%%%%%%%%


\latexProblemContent{
\begin{problem}

Compute the indefinite integral:

\input{2311-Compute-Integral-0016.HELP.tex}

\[\int\;{\frac{9 \, {\left(\log\left(x\right) + 2\right)}}{x}}\;dx = \answer{{\frac{9}{2} \, {\left(\log\left(x\right) + 2\right)}^{2}}+C}\]
\end{problem}}%}

%%%%%%%%%%%%%%%%%%%%%%


\latexProblemContent{
\begin{problem}

Compute the indefinite integral:

\input{2311-Compute-Integral-0016.HELP.tex}

\[\int\;{12 \, {\left(x^{4} + 6\right)} x^{3}}\;dx = \answer{{\frac{3}{2} \, {\left(x^{4} + 6\right)}^{2}}+C}\]
\end{problem}}%}

%%%%%%%%%%%%%%%%%%%%%%


\latexProblemContent{
\begin{problem}

Compute the indefinite integral:

\input{2311-Compute-Integral-0016.HELP.tex}

\[\int\;{9 \, x - 9}\;dx = \answer{{\frac{9}{2} \, x^{2} - 9 \, x}+C}\]
\end{problem}}%}

%%%%%%%%%%%%%%%%%%%%%%


%%%%%%%%%%%%%%%%%%%%%%


\latexProblemContent{
\begin{problem}

Compute the indefinite integral:

\input{2311-Compute-Integral-0016.HELP.tex}

\[\int\;{-\frac{2 \, {\left(\log\left(x\right) + 6\right)}}{x}}\;dx = \answer{{-{\left(\log\left(x\right) + 6\right)}^{2}}+C}\]
\end{problem}}%}

%%%%%%%%%%%%%%%%%%%%%%


\latexProblemContent{
\begin{problem}

Compute the indefinite integral:

\input{2311-Compute-Integral-0016.HELP.tex}

\[\int\;{\frac{2 \, {\left(\frac{1}{x^{2}} + 3\right)}}{x^{3}}}\;dx = \answer{{-\frac{1}{2} \, {\left(\frac{1}{x^{2}} + 3\right)}^{2}}+C}\]
\end{problem}}%}

%%%%%%%%%%%%%%%%%%%%%%


\latexProblemContent{
\begin{problem}

Compute the indefinite integral:

\input{2311-Compute-Integral-0016.HELP.tex}

\[\int\;{\frac{\sqrt{x} - 3}{2 \, \sqrt{x}}}\;dx = \answer{{\frac{1}{2} \, {\left(\sqrt{x} - 3\right)}^{2}}+C}\]
\end{problem}}%}

%%%%%%%%%%%%%%%%%%%%%%


\latexProblemContent{
\begin{problem}

Compute the indefinite integral:

\input{2311-Compute-Integral-0016.HELP.tex}

\[\int\;{\frac{3 \, {\left(\frac{1}{x} - 5\right)}}{x^{2}}}\;dx = \answer{{-\frac{3}{2} \, {\left(\frac{1}{x} - 5\right)}^{2}}+C}\]
\end{problem}}%}

%%%%%%%%%%%%%%%%%%%%%%


\latexProblemContent{
\begin{problem}

Compute the indefinite integral:

\input{2311-Compute-Integral-0016.HELP.tex}

\[\int\;{8 \, {\left(e^{x} + 7\right)} e^{x}}\;dx = \answer{{4 \, {\left(e^{x} + 7\right)}^{2}}+C}\]
\end{problem}}%}

%%%%%%%%%%%%%%%%%%%%%%


\latexProblemContent{
\begin{problem}

Compute the indefinite integral:

\input{2311-Compute-Integral-0016.HELP.tex}

\[\int\;{-4 \, x - 8}\;dx = \answer{{-2 \, x^{2} - 8 \, x}+C}\]
\end{problem}}%}

%%%%%%%%%%%%%%%%%%%%%%


\latexProblemContent{
\begin{problem}

Compute the indefinite integral:

\input{2311-Compute-Integral-0016.HELP.tex}

\[\int\;{-18 \, {\left(x^{2} + 1\right)} x}\;dx = \answer{{-\frac{9}{2} \, {\left(x^{2} + 1\right)}^{2}}+C}\]
\end{problem}}%}

%%%%%%%%%%%%%%%%%%%%%%


%%%%%%%%%%%%%%%%%%%%%%


\latexProblemContent{
\begin{problem}

Compute the indefinite integral:

\input{2311-Compute-Integral-0016.HELP.tex}

\[\int\;{-2 \, x - 16}\;dx = \answer{{-x^{2} - 16 \, x}+C}\]
\end{problem}}%}

%%%%%%%%%%%%%%%%%%%%%%


\latexProblemContent{
\begin{problem}

Compute the indefinite integral:

\input{2311-Compute-Integral-0016.HELP.tex}

\[\int\;{8 \, {\left(e^{x} + 2\right)} e^{x}}\;dx = \answer{{4 \, {\left(e^{x} + 2\right)}^{2}}+C}\]
\end{problem}}%}

%%%%%%%%%%%%%%%%%%%%%%


\latexProblemContent{
\begin{problem}

Compute the indefinite integral:

\input{2311-Compute-Integral-0016.HELP.tex}

\[\int\;{-8 \, {\left(\cos\left(x\right) + 4\right)} \sin\left(x\right)}\;dx = \answer{{4 \, {\left(\cos\left(x\right) + 4\right)}^{2}}+C}\]
\end{problem}}%}

%%%%%%%%%%%%%%%%%%%%%%


\latexProblemContent{
\begin{problem}

Compute the indefinite integral:

\input{2311-Compute-Integral-0016.HELP.tex}

\[\int\;{\frac{8 \, {\left(\log\left(x\right) - 6\right)}}{x}}\;dx = \answer{{4 \, {\left(\log\left(x\right) - 6\right)}^{2}}+C}\]
\end{problem}}%}

%%%%%%%%%%%%%%%%%%%%%%


%%%%%%%%%%%%%%%%%%%%%%


\latexProblemContent{
\begin{problem}

Compute the indefinite integral:

\input{2311-Compute-Integral-0016.HELP.tex}

\[\int\;{3 \, {\left(\cos\left(x\right) - 4\right)} \sin\left(x\right)}\;dx = \answer{{-\frac{3}{2} \, {\left(\cos\left(x\right) - 4\right)}^{2}}+C}\]
\end{problem}}%}

%%%%%%%%%%%%%%%%%%%%%%


%%%%%%%%%%%%%%%%%%%%%%


%%%%%%%%%%%%%%%%%%%%%%


\latexProblemContent{
\begin{problem}

Compute the indefinite integral:

\input{2311-Compute-Integral-0016.HELP.tex}

\[\int\;{27 \, {\left(x^{3} - 8\right)} x^{2}}\;dx = \answer{{\frac{9}{2} \, {\left(x^{3} - 8\right)}^{2}}+C}\]
\end{problem}}%}

%%%%%%%%%%%%%%%%%%%%%%


\latexProblemContent{
\begin{problem}

Compute the indefinite integral:

\input{2311-Compute-Integral-0016.HELP.tex}

\[\int\;{\frac{7 \, {\left(\frac{1}{x} + 5\right)}}{x^{2}}}\;dx = \answer{{-\frac{7}{2} \, {\left(\frac{1}{x} + 5\right)}^{2}}+C}\]
\end{problem}}%}

%%%%%%%%%%%%%%%%%%%%%%


\latexProblemContent{
\begin{problem}

Compute the indefinite integral:

\input{2311-Compute-Integral-0016.HELP.tex}

\[\int\;{\frac{9 \, {\left(\sqrt{x} + 3\right)}}{2 \, \sqrt{x}}}\;dx = \answer{{\frac{9}{2} \, {\left(\sqrt{x} + 3\right)}^{2}}+C}\]
\end{problem}}%}

%%%%%%%%%%%%%%%%%%%%%%


\latexProblemContent{
\begin{problem}

Compute the indefinite integral:

\input{2311-Compute-Integral-0016.HELP.tex}

\[\int\;{9 \, {\left(x^{3} - 8\right)} x^{2}}\;dx = \answer{{\frac{3}{2} \, {\left(x^{3} - 8\right)}^{2}}+C}\]
\end{problem}}%}

%%%%%%%%%%%%%%%%%%%%%%


%%%%%%%%%%%%%%%%%%%%%%


\latexProblemContent{
\begin{problem}

Compute the indefinite integral:

\input{2311-Compute-Integral-0016.HELP.tex}

\[\int\;{-\frac{10 \, {\left(\log\left(x\right) - 7\right)}}{x}}\;dx = \answer{{-5 \, {\left(\log\left(x\right) - 7\right)}^{2}}+C}\]
\end{problem}}%}

%%%%%%%%%%%%%%%%%%%%%%


\latexProblemContent{
\begin{problem}

Compute the indefinite integral:

\input{2311-Compute-Integral-0016.HELP.tex}

\[\int\;{5 \, {\left(e^{x} - 5\right)} e^{x}}\;dx = \answer{{\frac{5}{2} \, {\left(e^{x} - 5\right)}^{2}}+C}\]
\end{problem}}%}

%%%%%%%%%%%%%%%%%%%%%%


\latexProblemContent{
\begin{problem}

Compute the indefinite integral:

\input{2311-Compute-Integral-0016.HELP.tex}

\[\int\;{-2 \, {\left(x^{2} - 3\right)} x}\;dx = \answer{{-\frac{1}{2} \, {\left(x^{2} - 3\right)}^{2}}+C}\]
\end{problem}}%}

%%%%%%%%%%%%%%%%%%%%%%


\latexProblemContent{
\begin{problem}

Compute the indefinite integral:

\input{2311-Compute-Integral-0016.HELP.tex}

\[\int\;{-10 \, {\left(\cos\left(x\right) - 3\right)} \sin\left(x\right)}\;dx = \answer{{5 \, {\left(\cos\left(x\right) - 3\right)}^{2}}+C}\]
\end{problem}}%}

%%%%%%%%%%%%%%%%%%%%%%


\latexProblemContent{
\begin{problem}

Compute the indefinite integral:

\input{2311-Compute-Integral-0016.HELP.tex}

\[\int\;{\frac{18 \, {\left(\frac{1}{x^{2}} + 3\right)}}{x^{3}}}\;dx = \answer{{-\frac{9}{2} \, {\left(\frac{1}{x^{2}} + 3\right)}^{2}}+C}\]
\end{problem}}%}

%%%%%%%%%%%%%%%%%%%%%%


\latexProblemContent{
\begin{problem}

Compute the indefinite integral:

\input{2311-Compute-Integral-0016.HELP.tex}

\[\int\;{-\frac{10 \, {\left(\frac{1}{x} - 5\right)}}{x^{2}}}\;dx = \answer{{5 \, {\left(\frac{1}{x} - 5\right)}^{2}}+C}\]
\end{problem}}%}

%%%%%%%%%%%%%%%%%%%%%%


\latexProblemContent{
\begin{problem}

Compute the indefinite integral:

\input{2311-Compute-Integral-0016.HELP.tex}

\[\int\;{-6 \, {\left(\sin\left(x\right) + 6\right)} \cos\left(x\right)}\;dx = \answer{{-3 \, {\left(\sin\left(x\right) + 6\right)}^{2}}+C}\]
\end{problem}}%}

%%%%%%%%%%%%%%%%%%%%%%


\latexProblemContent{
\begin{problem}

Compute the indefinite integral:

\input{2311-Compute-Integral-0016.HELP.tex}

\[\int\;{\frac{2 \, {\left(\frac{1}{x} + 7\right)}}{x^{2}}}\;dx = \answer{{-{\left(\frac{1}{x} + 7\right)}^{2}}+C}\]
\end{problem}}%}

%%%%%%%%%%%%%%%%%%%%%%


\latexProblemContent{
\begin{problem}

Compute the indefinite integral:

\input{2311-Compute-Integral-0016.HELP.tex}

\[\int\;{\frac{12 \, {\left(\frac{1}{x^{2}} + 2\right)}}{x^{3}}}\;dx = \answer{{-3 \, {\left(\frac{1}{x^{2}} + 2\right)}^{2}}+C}\]
\end{problem}}%}

%%%%%%%%%%%%%%%%%%%%%%


\latexProblemContent{
\begin{problem}

Compute the indefinite integral:

\input{2311-Compute-Integral-0016.HELP.tex}

\[\int\;{\frac{10 \, {\left(\frac{1}{x^{2}} + 8\right)}}{x^{3}}}\;dx = \answer{{-\frac{5}{2} \, {\left(\frac{1}{x^{2}} + 8\right)}^{2}}+C}\]
\end{problem}}%}

%%%%%%%%%%%%%%%%%%%%%%


\latexProblemContent{
\begin{problem}

Compute the indefinite integral:

\input{2311-Compute-Integral-0016.HELP.tex}

\[\int\;{-\frac{2 \, {\left(\log\left(x\right) - 4\right)}}{x}}\;dx = \answer{{-{\left(\log\left(x\right) - 4\right)}^{2}}+C}\]
\end{problem}}%}

%%%%%%%%%%%%%%%%%%%%%%


\latexProblemContent{
\begin{problem}

Compute the indefinite integral:

\input{2311-Compute-Integral-0016.HELP.tex}

\[\int\;{-9 \, x + 162}\;dx = \answer{{-\frac{9}{2} \, x^{2} + 162 \, x}+C}\]
\end{problem}}%}

%%%%%%%%%%%%%%%%%%%%%%


%%%%%%%%%%%%%%%%%%%%%%


\latexProblemContent{
\begin{problem}

Compute the indefinite integral:

\input{2311-Compute-Integral-0016.HELP.tex}

\[\int\;{10 \, {\left(e^{x} + 6\right)} e^{x}}\;dx = \answer{{5 \, {\left(e^{x} + 6\right)}^{2}}+C}\]
\end{problem}}%}

%%%%%%%%%%%%%%%%%%%%%%


\latexProblemContent{
\begin{problem}

Compute the indefinite integral:

\input{2311-Compute-Integral-0016.HELP.tex}

\[\int\;{-27 \, {\left(x^{3} + 1\right)} x^{2}}\;dx = \answer{{-\frac{9}{2} \, {\left(x^{3} + 1\right)}^{2}}+C}\]
\end{problem}}%}

%%%%%%%%%%%%%%%%%%%%%%


\latexProblemContent{
\begin{problem}

Compute the indefinite integral:

\input{2311-Compute-Integral-0016.HELP.tex}

\[\int\;{-7 \, {\left(e^{x} - 3\right)} e^{x}}\;dx = \answer{{-\frac{7}{2} \, {\left(e^{x} - 3\right)}^{2}}+C}\]
\end{problem}}%}

%%%%%%%%%%%%%%%%%%%%%%


\latexProblemContent{
\begin{problem}

Compute the indefinite integral:

\input{2311-Compute-Integral-0016.HELP.tex}

\[\int\;{\frac{6 \, {\left(\log\left(x\right) + 3\right)}}{x}}\;dx = \answer{{3 \, {\left(\log\left(x\right) + 3\right)}^{2}}+C}\]
\end{problem}}%}

%%%%%%%%%%%%%%%%%%%%%%


%%%%%%%%%%%%%%%%%%%%%%


\latexProblemContent{
\begin{problem}

Compute the indefinite integral:

\input{2311-Compute-Integral-0016.HELP.tex}

\[\int\;{7 \, {\left(\cos\left(x\right) - 2\right)} \sin\left(x\right)}\;dx = \answer{{-\frac{7}{2} \, {\left(\cos\left(x\right) - 2\right)}^{2}}+C}\]
\end{problem}}%}

%%%%%%%%%%%%%%%%%%%%%%


\latexProblemContent{
\begin{problem}

Compute the indefinite integral:

\input{2311-Compute-Integral-0016.HELP.tex}

\[\int\;{40 \, {\left(x^{4} - 9\right)} x^{3}}\;dx = \answer{{5 \, {\left(x^{4} - 9\right)}^{2}}+C}\]
\end{problem}}%}

%%%%%%%%%%%%%%%%%%%%%%


\latexProblemContent{
\begin{problem}

Compute the indefinite integral:

\input{2311-Compute-Integral-0016.HELP.tex}

\[\int\;{-9 \, {\left(x^{3} - 7\right)} x^{2}}\;dx = \answer{{-\frac{3}{2} \, {\left(x^{3} - 7\right)}^{2}}+C}\]
\end{problem}}%}

%%%%%%%%%%%%%%%%%%%%%%


\latexProblemContent{
\begin{problem}

Compute the indefinite integral:

\input{2311-Compute-Integral-0016.HELP.tex}

\[\int\;{\frac{7 \, {\left(\frac{1}{x} + 7\right)}}{x^{2}}}\;dx = \answer{{-\frac{7}{2} \, {\left(\frac{1}{x} + 7\right)}^{2}}+C}\]
\end{problem}}%}

%%%%%%%%%%%%%%%%%%%%%%


\latexProblemContent{
\begin{problem}

Compute the indefinite integral:

\input{2311-Compute-Integral-0016.HELP.tex}

\[\int\;{-32 \, {\left(x^{4} - 2\right)} x^{3}}\;dx = \answer{{-4 \, {\left(x^{4} - 2\right)}^{2}}+C}\]
\end{problem}}%}

%%%%%%%%%%%%%%%%%%%%%%


\latexProblemContent{
\begin{problem}

Compute the indefinite integral:

\input{2311-Compute-Integral-0016.HELP.tex}

\[\int\;{7 \, x - 49}\;dx = \answer{{\frac{7}{2} \, x^{2} - 49 \, x}+C}\]
\end{problem}}%}

%%%%%%%%%%%%%%%%%%%%%%


\latexProblemContent{
\begin{problem}

Compute the indefinite integral:

\input{2311-Compute-Integral-0016.HELP.tex}

\[\int\;{-40 \, {\left(x^{4} - 5\right)} x^{3}}\;dx = \answer{{-5 \, {\left(x^{4} - 5\right)}^{2}}+C}\]
\end{problem}}%}

%%%%%%%%%%%%%%%%%%%%%%


\latexProblemContent{
\begin{problem}

Compute the indefinite integral:

\input{2311-Compute-Integral-0016.HELP.tex}

\[\int\;{-\frac{10 \, {\left(\frac{1}{x} - 6\right)}}{x^{2}}}\;dx = \answer{{5 \, {\left(\frac{1}{x} - 6\right)}^{2}}+C}\]
\end{problem}}%}

%%%%%%%%%%%%%%%%%%%%%%


\latexProblemContent{
\begin{problem}

Compute the indefinite integral:

\input{2311-Compute-Integral-0016.HELP.tex}

\[\int\;{-\frac{2 \, {\left(\frac{1}{x^{2}} - 8\right)}}{x^{3}}}\;dx = \answer{{\frac{1}{2} \, {\left(\frac{1}{x^{2}} - 8\right)}^{2}}+C}\]
\end{problem}}%}

%%%%%%%%%%%%%%%%%%%%%%


\latexProblemContent{
\begin{problem}

Compute the indefinite integral:

\input{2311-Compute-Integral-0016.HELP.tex}

\[\int\;{-6 \, {\left(e^{x} + 8\right)} e^{x}}\;dx = \answer{{-3 \, {\left(e^{x} + 8\right)}^{2}}+C}\]
\end{problem}}%}

%%%%%%%%%%%%%%%%%%%%%%


\latexProblemContent{
\begin{problem}

Compute the indefinite integral:

\input{2311-Compute-Integral-0016.HELP.tex}

\[\int\;{\frac{3 \, {\left(\sqrt{x} - 1\right)}}{2 \, \sqrt{x}}}\;dx = \answer{{\frac{3}{2} \, {\left(\sqrt{x} - 1\right)}^{2}}+C}\]
\end{problem}}%}

%%%%%%%%%%%%%%%%%%%%%%


\latexProblemContent{
\begin{problem}

Compute the indefinite integral:

\input{2311-Compute-Integral-0016.HELP.tex}

\[\int\;{-\frac{12 \, {\left(\frac{1}{x^{2}} - 10\right)}}{x^{3}}}\;dx = \answer{{3 \, {\left(\frac{1}{x^{2}} - 10\right)}^{2}}+C}\]
\end{problem}}%}

%%%%%%%%%%%%%%%%%%%%%%


\latexProblemContent{
\begin{problem}

Compute the indefinite integral:

\input{2311-Compute-Integral-0016.HELP.tex}

\[\int\;{-\frac{30 \, {\left(\frac{1}{x^{3}} - 2\right)}}{x^{4}}}\;dx = \answer{{5 \, {\left(\frac{1}{x^{3}} - 2\right)}^{2}}+C}\]
\end{problem}}%}

%%%%%%%%%%%%%%%%%%%%%%


\latexProblemContent{
\begin{problem}

Compute the indefinite integral:

\input{2311-Compute-Integral-0016.HELP.tex}

\[\int\;{4 \, {\left(\sin\left(x\right) + 1\right)} \cos\left(x\right)}\;dx = \answer{{2 \, {\left(\sin\left(x\right) + 1\right)}^{2}}+C}\]
\end{problem}}%}

%%%%%%%%%%%%%%%%%%%%%%


\latexProblemContent{
\begin{problem}

Compute the indefinite integral:

\input{2311-Compute-Integral-0016.HELP.tex}

\[\int\;{18 \, {\left(x^{3} - 10\right)} x^{2}}\;dx = \answer{{3 \, {\left(x^{3} - 10\right)}^{2}}+C}\]
\end{problem}}%}

%%%%%%%%%%%%%%%%%%%%%%


\latexProblemContent{
\begin{problem}

Compute the indefinite integral:

\input{2311-Compute-Integral-0016.HELP.tex}

\[\int\;{-2 \, {\left(x^{2} + 9\right)} x}\;dx = \answer{{-\frac{1}{2} \, {\left(x^{2} + 9\right)}^{2}}+C}\]
\end{problem}}%}

%%%%%%%%%%%%%%%%%%%%%%


\latexProblemContent{
\begin{problem}

Compute the indefinite integral:

\input{2311-Compute-Integral-0016.HELP.tex}

\[\int\;{\frac{8 \, {\left(\frac{1}{x} + 1\right)}}{x^{2}}}\;dx = \answer{{-4 \, {\left(\frac{1}{x} + 1\right)}^{2}}+C}\]
\end{problem}}%}

%%%%%%%%%%%%%%%%%%%%%%


\latexProblemContent{
\begin{problem}

Compute the indefinite integral:

\input{2311-Compute-Integral-0016.HELP.tex}

\[\int\;{-36 \, {\left(x^{4} - 10\right)} x^{3}}\;dx = \answer{{-\frac{9}{2} \, {\left(x^{4} - 10\right)}^{2}}+C}\]
\end{problem}}%}

%%%%%%%%%%%%%%%%%%%%%%


\latexProblemContent{
\begin{problem}

Compute the indefinite integral:

\input{2311-Compute-Integral-0016.HELP.tex}

\[\int\;{-18 \, {\left(x^{3} + 1\right)} x^{2}}\;dx = \answer{{-3 \, {\left(x^{3} + 1\right)}^{2}}+C}\]
\end{problem}}%}

%%%%%%%%%%%%%%%%%%%%%%


\latexProblemContent{
\begin{problem}

Compute the indefinite integral:

\input{2311-Compute-Integral-0016.HELP.tex}

\[\int\;{16 \, {\left(x^{2} + 5\right)} x}\;dx = \answer{{4 \, {\left(x^{2} + 5\right)}^{2}}+C}\]
\end{problem}}%}

%%%%%%%%%%%%%%%%%%%%%%


\latexProblemContent{
\begin{problem}

Compute the indefinite integral:

\input{2311-Compute-Integral-0016.HELP.tex}

\[\int\;{-10 \, {\left(e^{x} - 8\right)} e^{x}}\;dx = \answer{{-5 \, {\left(e^{x} - 8\right)}^{2}}+C}\]
\end{problem}}%}

%%%%%%%%%%%%%%%%%%%%%%


\latexProblemContent{
\begin{problem}

Compute the indefinite integral:

\input{2311-Compute-Integral-0016.HELP.tex}

\[\int\;{\frac{5 \, {\left(\sqrt{x} + 7\right)}}{2 \, \sqrt{x}}}\;dx = \answer{{\frac{5}{2} \, {\left(\sqrt{x} + 7\right)}^{2}}+C}\]
\end{problem}}%}

%%%%%%%%%%%%%%%%%%%%%%


\latexProblemContent{
\begin{problem}

Compute the indefinite integral:

\input{2311-Compute-Integral-0016.HELP.tex}

\[\int\;{-10 \, x}\;dx = \answer{{-5 \, x^{2}}+C}\]
\end{problem}}%}

%%%%%%%%%%%%%%%%%%%%%%


\latexProblemContent{
\begin{problem}

Compute the indefinite integral:

\input{2311-Compute-Integral-0016.HELP.tex}

\[\int\;{-\frac{5 \, {\left(\sqrt{x} - 10\right)}}{\sqrt{x}}}\;dx = \answer{{-5 \, {\left(\sqrt{x} - 10\right)}^{2}}+C}\]
\end{problem}}%}

%%%%%%%%%%%%%%%%%%%%%%


\latexProblemContent{
\begin{problem}

Compute the indefinite integral:

\input{2311-Compute-Integral-0016.HELP.tex}

\[\int\;{6 \, {\left(x^{3} + 9\right)} x^{2}}\;dx = \answer{{{\left(x^{3} + 9\right)}^{2}}+C}\]
\end{problem}}%}

%%%%%%%%%%%%%%%%%%%%%%


\latexProblemContent{
\begin{problem}

Compute the indefinite integral:

\input{2311-Compute-Integral-0016.HELP.tex}

\[\int\;{\frac{8 \, {\left(\frac{1}{x^{2}} - 4\right)}}{x^{3}}}\;dx = \answer{{-2 \, {\left(\frac{1}{x^{2}} - 4\right)}^{2}}+C}\]
\end{problem}}%}

%%%%%%%%%%%%%%%%%%%%%%


%%%%%%%%%%%%%%%%%%%%%%


\latexProblemContent{
\begin{problem}

Compute the indefinite integral:

\input{2311-Compute-Integral-0016.HELP.tex}

\[\int\;{\frac{5 \, {\left(\frac{1}{x} + 5\right)}}{x^{2}}}\;dx = \answer{{-\frac{5}{2} \, {\left(\frac{1}{x} + 5\right)}^{2}}+C}\]
\end{problem}}%}

%%%%%%%%%%%%%%%%%%%%%%


%%%%%%%%%%%%%%%%%%%%%%


\latexProblemContent{
\begin{problem}

Compute the indefinite integral:

\input{2311-Compute-Integral-0016.HELP.tex}

\[\int\;{\frac{12 \, {\left(\frac{1}{x^{2}} - 5\right)}}{x^{3}}}\;dx = \answer{{-3 \, {\left(\frac{1}{x^{2}} - 5\right)}^{2}}+C}\]
\end{problem}}%}

%%%%%%%%%%%%%%%%%%%%%%


\latexProblemContent{
\begin{problem}

Compute the indefinite integral:

\input{2311-Compute-Integral-0016.HELP.tex}

\[\int\;{-{\left(\cos\left(x\right) + 2\right)} \sin\left(x\right)}\;dx = \answer{{\frac{1}{2} \, {\left(\cos\left(x\right) + 2\right)}^{2}}+C}\]
\end{problem}}%}

%%%%%%%%%%%%%%%%%%%%%%


\latexProblemContent{
\begin{problem}

Compute the indefinite integral:

\input{2311-Compute-Integral-0016.HELP.tex}

\[\int\;{-\frac{\log\left(x\right) + 8}{x}}\;dx = \answer{{-\frac{1}{2} \, {\left(\log\left(x\right) + 8\right)}^{2}}+C}\]
\end{problem}}%}

%%%%%%%%%%%%%%%%%%%%%%


\latexProblemContent{
\begin{problem}

Compute the indefinite integral:

\input{2311-Compute-Integral-0016.HELP.tex}

\[\int\;{-\frac{7 \, {\left(\log\left(x\right) + 9\right)}}{x}}\;dx = \answer{{-\frac{7}{2} \, {\left(\log\left(x\right) + 9\right)}^{2}}+C}\]
\end{problem}}%}

%%%%%%%%%%%%%%%%%%%%%%


\latexProblemContent{
\begin{problem}

Compute the indefinite integral:

\input{2311-Compute-Integral-0016.HELP.tex}

\[\int\;{-12 \, {\left(x^{2} + 1\right)} x}\;dx = \answer{{-3 \, {\left(x^{2} + 1\right)}^{2}}+C}\]
\end{problem}}%}

%%%%%%%%%%%%%%%%%%%%%%


\latexProblemContent{
\begin{problem}

Compute the indefinite integral:

\input{2311-Compute-Integral-0016.HELP.tex}

\[\int\;{-4 \, {\left(\cos\left(x\right) + 5\right)} \sin\left(x\right)}\;dx = \answer{{2 \, {\left(\cos\left(x\right) + 5\right)}^{2}}+C}\]
\end{problem}}%}

%%%%%%%%%%%%%%%%%%%%%%


\latexProblemContent{
\begin{problem}

Compute the indefinite integral:

\input{2311-Compute-Integral-0016.HELP.tex}

\[\int\;{-8 \, x - 104}\;dx = \answer{{-4 \, x^{2} - 104 \, x}+C}\]
\end{problem}}%}

%%%%%%%%%%%%%%%%%%%%%%


\latexProblemContent{
\begin{problem}

Compute the indefinite integral:

\input{2311-Compute-Integral-0016.HELP.tex}

\[\int\;{6 \, {\left(\sin\left(x\right) - 8\right)} \cos\left(x\right)}\;dx = \answer{{3 \, {\left(\sin\left(x\right) - 8\right)}^{2}}+C}\]
\end{problem}}%}

%%%%%%%%%%%%%%%%%%%%%%


\latexProblemContent{
\begin{problem}

Compute the indefinite integral:

\input{2311-Compute-Integral-0016.HELP.tex}

\[\int\;{-\frac{5 \, {\left(\sqrt{x} + 1\right)}}{\sqrt{x}}}\;dx = \answer{{-5 \, {\left(\sqrt{x} + 1\right)}^{2}}+C}\]
\end{problem}}%}

%%%%%%%%%%%%%%%%%%%%%%


\latexProblemContent{
\begin{problem}

Compute the indefinite integral:

\input{2311-Compute-Integral-0016.HELP.tex}

\[\int\;{9 \, {\left(\sin\left(x\right) - 7\right)} \cos\left(x\right)}\;dx = \answer{{\frac{9}{2} \, {\left(\sin\left(x\right) - 7\right)}^{2}}+C}\]
\end{problem}}%}

%%%%%%%%%%%%%%%%%%%%%%


%%%%%%%%%%%%%%%%%%%%%%


\latexProblemContent{
\begin{problem}

Compute the indefinite integral:

\input{2311-Compute-Integral-0016.HELP.tex}

\[\int\;{6 \, {\left(\cos\left(x\right) + 9\right)} \sin\left(x\right)}\;dx = \answer{{-3 \, {\left(\cos\left(x\right) + 9\right)}^{2}}+C}\]
\end{problem}}%}

%%%%%%%%%%%%%%%%%%%%%%


\latexProblemContent{
\begin{problem}

Compute the indefinite integral:

\input{2311-Compute-Integral-0016.HELP.tex}

\[\int\;{2 \, {\left(\cos\left(x\right) - 8\right)} \sin\left(x\right)}\;dx = \answer{{-{\left(\cos\left(x\right) - 8\right)}^{2}}+C}\]
\end{problem}}%}

%%%%%%%%%%%%%%%%%%%%%%


\latexProblemContent{
\begin{problem}

Compute the indefinite integral:

\input{2311-Compute-Integral-0016.HELP.tex}

\[\int\;{6 \, {\left(x^{3} - 2\right)} x^{2}}\;dx = \answer{{{\left(x^{3} - 2\right)}^{2}}+C}\]
\end{problem}}%}

%%%%%%%%%%%%%%%%%%%%%%


\latexProblemContent{
\begin{problem}

Compute the indefinite integral:

\input{2311-Compute-Integral-0016.HELP.tex}

\[\int\;{-\frac{3 \, {\left(\sqrt{x} + 5\right)}}{2 \, \sqrt{x}}}\;dx = \answer{{-\frac{3}{2} \, {\left(\sqrt{x} + 5\right)}^{2}}+C}\]
\end{problem}}%}

%%%%%%%%%%%%%%%%%%%%%%


\latexProblemContent{
\begin{problem}

Compute the indefinite integral:

\input{2311-Compute-Integral-0016.HELP.tex}

\[\int\;{-20 \, {\left(x^{4} + 7\right)} x^{3}}\;dx = \answer{{-\frac{5}{2} \, {\left(x^{4} + 7\right)}^{2}}+C}\]
\end{problem}}%}

%%%%%%%%%%%%%%%%%%%%%%


\latexProblemContent{
\begin{problem}

Compute the indefinite integral:

\input{2311-Compute-Integral-0016.HELP.tex}

\[\int\;{-36 \, {\left(x^{4} + 3\right)} x^{3}}\;dx = \answer{{-\frac{9}{2} \, {\left(x^{4} + 3\right)}^{2}}+C}\]
\end{problem}}%}

%%%%%%%%%%%%%%%%%%%%%%


\latexProblemContent{
\begin{problem}

Compute the indefinite integral:

\input{2311-Compute-Integral-0016.HELP.tex}

\[\int\;{-5 \, {\left(\sin\left(x\right) + 5\right)} \cos\left(x\right)}\;dx = \answer{{-\frac{5}{2} \, {\left(\sin\left(x\right) + 5\right)}^{2}}+C}\]
\end{problem}}%}

%%%%%%%%%%%%%%%%%%%%%%


\latexProblemContent{
\begin{problem}

Compute the indefinite integral:

\input{2311-Compute-Integral-0016.HELP.tex}

\[\int\;{10 \, {\left(\sin\left(x\right) - 5\right)} \cos\left(x\right)}\;dx = \answer{{5 \, {\left(\sin\left(x\right) - 5\right)}^{2}}+C}\]
\end{problem}}%}

%%%%%%%%%%%%%%%%%%%%%%


\latexProblemContent{
\begin{problem}

Compute the indefinite integral:

\input{2311-Compute-Integral-0016.HELP.tex}

\[\int\;{9 \, {\left(\sin\left(x\right) + 1\right)} \cos\left(x\right)}\;dx = \answer{{\frac{9}{2} \, {\left(\sin\left(x\right) + 1\right)}^{2}}+C}\]
\end{problem}}%}

%%%%%%%%%%%%%%%%%%%%%%


\latexProblemContent{
\begin{problem}

Compute the indefinite integral:

\input{2311-Compute-Integral-0016.HELP.tex}

\[\int\;{2 \, {\left(x^{2} + 4\right)} x}\;dx = \answer{{\frac{1}{2} \, {\left(x^{2} + 4\right)}^{2}}+C}\]
\end{problem}}%}

%%%%%%%%%%%%%%%%%%%%%%


\latexProblemContent{
\begin{problem}

Compute the indefinite integral:

\input{2311-Compute-Integral-0016.HELP.tex}

\[\int\;{\frac{10 \, {\left(\frac{1}{x} + 8\right)}}{x^{2}}}\;dx = \answer{{-5 \, {\left(\frac{1}{x} + 8\right)}^{2}}+C}\]
\end{problem}}%}

%%%%%%%%%%%%%%%%%%%%%%


\latexProblemContent{
\begin{problem}

Compute the indefinite integral:

\input{2311-Compute-Integral-0016.HELP.tex}

\[\int\;{-16 \, {\left(x^{4} - 5\right)} x^{3}}\;dx = \answer{{-2 \, {\left(x^{4} - 5\right)}^{2}}+C}\]
\end{problem}}%}

%%%%%%%%%%%%%%%%%%%%%%


\latexProblemContent{
\begin{problem}

Compute the indefinite integral:

\input{2311-Compute-Integral-0016.HELP.tex}

\[\int\;{-\frac{7 \, {\left(\sqrt{x} - 6\right)}}{2 \, \sqrt{x}}}\;dx = \answer{{-\frac{7}{2} \, {\left(\sqrt{x} - 6\right)}^{2}}+C}\]
\end{problem}}%}

%%%%%%%%%%%%%%%%%%%%%%


\latexProblemContent{
\begin{problem}

Compute the indefinite integral:

\input{2311-Compute-Integral-0016.HELP.tex}

\[\int\;{-\frac{9 \, {\left(\log\left(x\right) + 1\right)}}{x}}\;dx = \answer{{-\frac{9}{2} \, {\left(\log\left(x\right) + 1\right)}^{2}}+C}\]
\end{problem}}%}

%%%%%%%%%%%%%%%%%%%%%%


\latexProblemContent{
\begin{problem}

Compute the indefinite integral:

\input{2311-Compute-Integral-0016.HELP.tex}

\[\int\;{40 \, {\left(x^{4} + 8\right)} x^{3}}\;dx = \answer{{5 \, {\left(x^{4} + 8\right)}^{2}}+C}\]
\end{problem}}%}

%%%%%%%%%%%%%%%%%%%%%%


\latexProblemContent{
\begin{problem}

Compute the indefinite integral:

\input{2311-Compute-Integral-0016.HELP.tex}

\[\int\;{2 \, {\left(\sin\left(x\right) - 5\right)} \cos\left(x\right)}\;dx = \answer{{{\left(\sin\left(x\right) - 5\right)}^{2}}+C}\]
\end{problem}}%}

%%%%%%%%%%%%%%%%%%%%%%


\latexProblemContent{
\begin{problem}

Compute the indefinite integral:

\input{2311-Compute-Integral-0016.HELP.tex}

\[\int\;{-\frac{8 \, {\left(\frac{1}{x} + 8\right)}}{x^{2}}}\;dx = \answer{{4 \, {\left(\frac{1}{x} + 8\right)}^{2}}+C}\]
\end{problem}}%}

%%%%%%%%%%%%%%%%%%%%%%


\latexProblemContent{
\begin{problem}

Compute the indefinite integral:

\input{2311-Compute-Integral-0016.HELP.tex}

\[\int\;{36 \, {\left(x^{4} + 10\right)} x^{3}}\;dx = \answer{{\frac{9}{2} \, {\left(x^{4} + 10\right)}^{2}}+C}\]
\end{problem}}%}

%%%%%%%%%%%%%%%%%%%%%%


\latexProblemContent{
\begin{problem}

Compute the indefinite integral:

\input{2311-Compute-Integral-0016.HELP.tex}

\[\int\;{-\frac{6 \, {\left(\frac{1}{x} + 2\right)}}{x^{2}}}\;dx = \answer{{3 \, {\left(\frac{1}{x} + 2\right)}^{2}}+C}\]
\end{problem}}%}

%%%%%%%%%%%%%%%%%%%%%%


\latexProblemContent{
\begin{problem}

Compute the indefinite integral:

\input{2311-Compute-Integral-0016.HELP.tex}

\[\int\;{\frac{21 \, {\left(\frac{1}{x^{3}} + 3\right)}}{x^{4}}}\;dx = \answer{{-\frac{7}{2} \, {\left(\frac{1}{x^{3}} + 3\right)}^{2}}+C}\]
\end{problem}}%}

%%%%%%%%%%%%%%%%%%%%%%


\latexProblemContent{
\begin{problem}

Compute the indefinite integral:

\input{2311-Compute-Integral-0016.HELP.tex}

\[\int\;{\frac{\frac{1}{x} + 4}{x^{2}}}\;dx = \answer{{-\frac{1}{2} \, {\left(\frac{1}{x} + 4\right)}^{2}}+C}\]
\end{problem}}%}

%%%%%%%%%%%%%%%%%%%%%%


\latexProblemContent{
\begin{problem}

Compute the indefinite integral:

\input{2311-Compute-Integral-0016.HELP.tex}

\[\int\;{24 \, {\left(x^{4} - 2\right)} x^{3}}\;dx = \answer{{3 \, {\left(x^{4} - 2\right)}^{2}}+C}\]
\end{problem}}%}

%%%%%%%%%%%%%%%%%%%%%%


\latexProblemContent{
\begin{problem}

Compute the indefinite integral:

\input{2311-Compute-Integral-0016.HELP.tex}

\[\int\;{-8 \, {\left(\sin\left(x\right) - 5\right)} \cos\left(x\right)}\;dx = \answer{{-4 \, {\left(\sin\left(x\right) - 5\right)}^{2}}+C}\]
\end{problem}}%}

%%%%%%%%%%%%%%%%%%%%%%


\latexProblemContent{
\begin{problem}

Compute the indefinite integral:

\input{2311-Compute-Integral-0016.HELP.tex}

\[\int\;{20 \, {\left(x^{2} - 2\right)} x}\;dx = \answer{{5 \, {\left(x^{2} - 2\right)}^{2}}+C}\]
\end{problem}}%}

%%%%%%%%%%%%%%%%%%%%%%


\latexProblemContent{
\begin{problem}

Compute the indefinite integral:

\input{2311-Compute-Integral-0016.HELP.tex}

\[\int\;{\frac{4 \, {\left(\sqrt{x} + 5\right)}}{\sqrt{x}}}\;dx = \answer{{4 \, {\left(\sqrt{x} + 5\right)}^{2}}+C}\]
\end{problem}}%}

%%%%%%%%%%%%%%%%%%%%%%


\latexProblemContent{
\begin{problem}

Compute the indefinite integral:

\input{2311-Compute-Integral-0016.HELP.tex}

\[\int\;{\frac{5 \, {\left(\sqrt{x} - 8\right)}}{\sqrt{x}}}\;dx = \answer{{5 \, {\left(\sqrt{x} - 8\right)}^{2}}+C}\]
\end{problem}}%}

%%%%%%%%%%%%%%%%%%%%%%


\latexProblemContent{
\begin{problem}

Compute the indefinite integral:

\input{2311-Compute-Integral-0016.HELP.tex}

\[\int\;{{\left(\sin\left(x\right) - 5\right)} \cos\left(x\right)}\;dx = \answer{{\frac{1}{2} \, {\left(\sin\left(x\right) - 5\right)}^{2}}+C}\]
\end{problem}}%}

%%%%%%%%%%%%%%%%%%%%%%


\latexProblemContent{
\begin{problem}

Compute the indefinite integral:

\input{2311-Compute-Integral-0016.HELP.tex}

\[\int\;{40 \, {\left(x^{4} + 7\right)} x^{3}}\;dx = \answer{{5 \, {\left(x^{4} + 7\right)}^{2}}+C}\]
\end{problem}}%}

%%%%%%%%%%%%%%%%%%%%%%


\latexProblemContent{
\begin{problem}

Compute the indefinite integral:

\input{2311-Compute-Integral-0016.HELP.tex}

\[\int\;{-\frac{\sqrt{x} + 1}{2 \, \sqrt{x}}}\;dx = \answer{{-\frac{1}{2} \, {\left(\sqrt{x} + 1\right)}^{2}}+C}\]
\end{problem}}%}

%%%%%%%%%%%%%%%%%%%%%%


\latexProblemContent{
\begin{problem}

Compute the indefinite integral:

\input{2311-Compute-Integral-0016.HELP.tex}

\[\int\;{-\frac{15 \, {\left(\frac{1}{x^{3}} - 3\right)}}{x^{4}}}\;dx = \answer{{\frac{5}{2} \, {\left(\frac{1}{x^{3}} - 3\right)}^{2}}+C}\]
\end{problem}}%}

%%%%%%%%%%%%%%%%%%%%%%


\latexProblemContent{
\begin{problem}

Compute the indefinite integral:

\input{2311-Compute-Integral-0016.HELP.tex}

\[\int\;{-10 \, {\left(\cos\left(x\right) - 7\right)} \sin\left(x\right)}\;dx = \answer{{5 \, {\left(\cos\left(x\right) - 7\right)}^{2}}+C}\]
\end{problem}}%}

%%%%%%%%%%%%%%%%%%%%%%


\latexProblemContent{
\begin{problem}

Compute the indefinite integral:

\input{2311-Compute-Integral-0016.HELP.tex}

\[\int\;{-\frac{4 \, {\left(\log\left(x\right) - 7\right)}}{x}}\;dx = \answer{{-2 \, {\left(\log\left(x\right) - 7\right)}^{2}}+C}\]
\end{problem}}%}

%%%%%%%%%%%%%%%%%%%%%%


\latexProblemContent{
\begin{problem}

Compute the indefinite integral:

\input{2311-Compute-Integral-0016.HELP.tex}

\[\int\;{-8 \, {\left(x^{2} + 6\right)} x}\;dx = \answer{{-2 \, {\left(x^{2} + 6\right)}^{2}}+C}\]
\end{problem}}%}

%%%%%%%%%%%%%%%%%%%%%%


\latexProblemContent{
\begin{problem}

Compute the indefinite integral:

\input{2311-Compute-Integral-0016.HELP.tex}

\[\int\;{7 \, x + 84}\;dx = \answer{{\frac{7}{2} \, x^{2} + 84 \, x}+C}\]
\end{problem}}%}

%%%%%%%%%%%%%%%%%%%%%%


\latexProblemContent{
\begin{problem}

Compute the indefinite integral:

\input{2311-Compute-Integral-0016.HELP.tex}

\[\int\;{\frac{8 \, {\left(\frac{1}{x^{2}} - 1\right)}}{x^{3}}}\;dx = \answer{{-2 \, {\left(\frac{1}{x^{2}} - 1\right)}^{2}}+C}\]
\end{problem}}%}

%%%%%%%%%%%%%%%%%%%%%%


%%%%%%%%%%%%%%%%%%%%%%


\latexProblemContent{
\begin{problem}

Compute the indefinite integral:

\input{2311-Compute-Integral-0016.HELP.tex}

\[\int\;{-\frac{3 \, {\left(\sqrt{x} + 9\right)}}{2 \, \sqrt{x}}}\;dx = \answer{{-\frac{3}{2} \, {\left(\sqrt{x} + 9\right)}^{2}}+C}\]
\end{problem}}%}

%%%%%%%%%%%%%%%%%%%%%%


\latexProblemContent{
\begin{problem}

Compute the indefinite integral:

\input{2311-Compute-Integral-0016.HELP.tex}

\[\int\;{-18 \, {\left(x^{3} - 8\right)} x^{2}}\;dx = \answer{{-3 \, {\left(x^{3} - 8\right)}^{2}}+C}\]
\end{problem}}%}

%%%%%%%%%%%%%%%%%%%%%%


\latexProblemContent{
\begin{problem}

Compute the indefinite integral:

\input{2311-Compute-Integral-0016.HELP.tex}

\[\int\;{-8 \, {\left(\sin\left(x\right) - 8\right)} \cos\left(x\right)}\;dx = \answer{{-4 \, {\left(\sin\left(x\right) - 8\right)}^{2}}+C}\]
\end{problem}}%}

%%%%%%%%%%%%%%%%%%%%%%


\latexProblemContent{
\begin{problem}

Compute the indefinite integral:

\input{2311-Compute-Integral-0016.HELP.tex}

\[\int\;{2 \, x - 4}\;dx = \answer{{x^{2} - 4 \, x}+C}\]
\end{problem}}%}

%%%%%%%%%%%%%%%%%%%%%%


\latexProblemContent{
\begin{problem}

Compute the indefinite integral:

\input{2311-Compute-Integral-0016.HELP.tex}

\[\int\;{-\frac{12 \, {\left(\frac{1}{x^{3}} + 3\right)}}{x^{4}}}\;dx = \answer{{2 \, {\left(\frac{1}{x^{3}} + 3\right)}^{2}}+C}\]
\end{problem}}%}

%%%%%%%%%%%%%%%%%%%%%%


\latexProblemContent{
\begin{problem}

Compute the indefinite integral:

\input{2311-Compute-Integral-0016.HELP.tex}

\[\int\;{2 \, {\left(e^{x} - 9\right)} e^{x}}\;dx = \answer{{{\left(e^{x} - 9\right)}^{2}}+C}\]
\end{problem}}%}

%%%%%%%%%%%%%%%%%%%%%%


\latexProblemContent{
\begin{problem}

Compute the indefinite integral:

\input{2311-Compute-Integral-0016.HELP.tex}

\[\int\;{-5 \, x - 20}\;dx = \answer{{-\frac{5}{2} \, x^{2} - 20 \, x}+C}\]
\end{problem}}%}

%%%%%%%%%%%%%%%%%%%%%%


\latexProblemContent{
\begin{problem}

Compute the indefinite integral:

\input{2311-Compute-Integral-0016.HELP.tex}

\[\int\;{\frac{10 \, {\left(\log\left(x\right) - 7\right)}}{x}}\;dx = \answer{{5 \, {\left(\log\left(x\right) - 7\right)}^{2}}+C}\]
\end{problem}}%}

%%%%%%%%%%%%%%%%%%%%%%


\latexProblemContent{
\begin{problem}

Compute the indefinite integral:

\input{2311-Compute-Integral-0016.HELP.tex}

\[\int\;{-12 \, {\left(x^{4} - 1\right)} x^{3}}\;dx = \answer{{-\frac{3}{2} \, {\left(x^{4} - 1\right)}^{2}}+C}\]
\end{problem}}%}

%%%%%%%%%%%%%%%%%%%%%%


\latexProblemContent{
\begin{problem}

Compute the indefinite integral:

\input{2311-Compute-Integral-0016.HELP.tex}

\[\int\;{4 \, {\left(x^{2} + 8\right)} x}\;dx = \answer{{{\left(x^{2} + 8\right)}^{2}}+C}\]
\end{problem}}%}

%%%%%%%%%%%%%%%%%%%%%%


\latexProblemContent{
\begin{problem}

Compute the indefinite integral:

\input{2311-Compute-Integral-0016.HELP.tex}

\[\int\;{-\frac{10 \, {\left(\frac{1}{x^{2}} - 4\right)}}{x^{3}}}\;dx = \answer{{\frac{5}{2} \, {\left(\frac{1}{x^{2}} - 4\right)}^{2}}+C}\]
\end{problem}}%}

%%%%%%%%%%%%%%%%%%%%%%


\latexProblemContent{
\begin{problem}

Compute the indefinite integral:

\input{2311-Compute-Integral-0016.HELP.tex}

\[\int\;{9 \, {\left(\sin\left(x\right) + 6\right)} \cos\left(x\right)}\;dx = \answer{{\frac{9}{2} \, {\left(\sin\left(x\right) + 6\right)}^{2}}+C}\]
\end{problem}}%}

%%%%%%%%%%%%%%%%%%%%%%


\latexProblemContent{
\begin{problem}

Compute the indefinite integral:

\input{2311-Compute-Integral-0016.HELP.tex}

\[\int\;{4 \, {\left(e^{x} + 1\right)} e^{x}}\;dx = \answer{{2 \, {\left(e^{x} + 1\right)}^{2}}+C}\]
\end{problem}}%}

%%%%%%%%%%%%%%%%%%%%%%


%%%%%%%%%%%%%%%%%%%%%%


\latexProblemContent{
\begin{problem}

Compute the indefinite integral:

\input{2311-Compute-Integral-0016.HELP.tex}

\[\int\;{-9 \, {\left(x^{3} - 6\right)} x^{2}}\;dx = \answer{{-\frac{3}{2} \, {\left(x^{3} - 6\right)}^{2}}+C}\]
\end{problem}}%}

%%%%%%%%%%%%%%%%%%%%%%


\latexProblemContent{
\begin{problem}

Compute the indefinite integral:

\input{2311-Compute-Integral-0016.HELP.tex}

\[\int\;{4 \, {\left(x^{4} - 2\right)} x^{3}}\;dx = \answer{{\frac{1}{2} \, {\left(x^{4} - 2\right)}^{2}}+C}\]
\end{problem}}%}

%%%%%%%%%%%%%%%%%%%%%%


\latexProblemContent{
\begin{problem}

Compute the indefinite integral:

\input{2311-Compute-Integral-0016.HELP.tex}

\[\int\;{\frac{10 \, {\left(\frac{1}{x^{2}} - 7\right)}}{x^{3}}}\;dx = \answer{{-\frac{5}{2} \, {\left(\frac{1}{x^{2}} - 7\right)}^{2}}+C}\]
\end{problem}}%}

%%%%%%%%%%%%%%%%%%%%%%


\latexProblemContent{
\begin{problem}

Compute the indefinite integral:

\input{2311-Compute-Integral-0016.HELP.tex}

\[\int\;{-\frac{2 \, {\left(\sqrt{x} + 1\right)}}{\sqrt{x}}}\;dx = \answer{{-2 \, {\left(\sqrt{x} + 1\right)}^{2}}+C}\]
\end{problem}}%}

%%%%%%%%%%%%%%%%%%%%%%


\latexProblemContent{
\begin{problem}

Compute the indefinite integral:

\input{2311-Compute-Integral-0016.HELP.tex}

\[\int\;{\frac{7 \, {\left(\sqrt{x} + 6\right)}}{2 \, \sqrt{x}}}\;dx = \answer{{\frac{7}{2} \, {\left(\sqrt{x} + 6\right)}^{2}}+C}\]
\end{problem}}%}

%%%%%%%%%%%%%%%%%%%%%%


\latexProblemContent{
\begin{problem}

Compute the indefinite integral:

\input{2311-Compute-Integral-0016.HELP.tex}

\[\int\;{-\frac{4 \, {\left(\frac{1}{x^{2}} + 5\right)}}{x^{3}}}\;dx = \answer{{{\left(\frac{1}{x^{2}} + 5\right)}^{2}}+C}\]
\end{problem}}%}

%%%%%%%%%%%%%%%%%%%%%%


\latexProblemContent{
\begin{problem}

Compute the indefinite integral:

\input{2311-Compute-Integral-0016.HELP.tex}

\[\int\;{-8 \, {\left(\sin\left(x\right) + 4\right)} \cos\left(x\right)}\;dx = \answer{{-4 \, {\left(\sin\left(x\right) + 4\right)}^{2}}+C}\]
\end{problem}}%}

%%%%%%%%%%%%%%%%%%%%%%


\latexProblemContent{
\begin{problem}

Compute the indefinite integral:

\input{2311-Compute-Integral-0016.HELP.tex}

\[\int\;{\frac{\sqrt{x} + 10}{2 \, \sqrt{x}}}\;dx = \answer{{\frac{1}{2} \, {\left(\sqrt{x} + 10\right)}^{2}}+C}\]
\end{problem}}%}

%%%%%%%%%%%%%%%%%%%%%%


\latexProblemContent{
\begin{problem}

Compute the indefinite integral:

\input{2311-Compute-Integral-0016.HELP.tex}

\[\int\;{3 \, {\left(x^{3} + 10\right)} x^{2}}\;dx = \answer{{\frac{1}{2} \, {\left(x^{3} + 10\right)}^{2}}+C}\]
\end{problem}}%}

%%%%%%%%%%%%%%%%%%%%%%


\latexProblemContent{
\begin{problem}

Compute the indefinite integral:

\input{2311-Compute-Integral-0016.HELP.tex}

\[\int\;{{\left(\sin\left(x\right) - 2\right)} \cos\left(x\right)}\;dx = \answer{{\frac{1}{2} \, {\left(\sin\left(x\right) - 2\right)}^{2}}+C}\]
\end{problem}}%}

%%%%%%%%%%%%%%%%%%%%%%


\latexProblemContent{
\begin{problem}

Compute the indefinite integral:

\input{2311-Compute-Integral-0016.HELP.tex}

\[\int\;{-4 \, x + 4}\;dx = \answer{{-2 \, x^{2} + 4 \, x}+C}\]
\end{problem}}%}

%%%%%%%%%%%%%%%%%%%%%%


\latexProblemContent{
\begin{problem}

Compute the indefinite integral:

\input{2311-Compute-Integral-0016.HELP.tex}

\[\int\;{\frac{16 \, {\left(\frac{1}{x^{2}} - 4\right)}}{x^{3}}}\;dx = \answer{{-4 \, {\left(\frac{1}{x^{2}} - 4\right)}^{2}}+C}\]
\end{problem}}%}

%%%%%%%%%%%%%%%%%%%%%%


%%%%%%%%%%%%%%%%%%%%%%


\latexProblemContent{
\begin{problem}

Compute the indefinite integral:

\input{2311-Compute-Integral-0016.HELP.tex}

\[\int\;{20 \, {\left(x^{4} + 3\right)} x^{3}}\;dx = \answer{{\frac{5}{2} \, {\left(x^{4} + 3\right)}^{2}}+C}\]
\end{problem}}%}

%%%%%%%%%%%%%%%%%%%%%%


\latexProblemContent{
\begin{problem}

Compute the indefinite integral:

\input{2311-Compute-Integral-0016.HELP.tex}

\[\int\;{-\frac{4 \, {\left(\frac{1}{x^{2}} - 8\right)}}{x^{3}}}\;dx = \answer{{{\left(\frac{1}{x^{2}} - 8\right)}^{2}}+C}\]
\end{problem}}%}

%%%%%%%%%%%%%%%%%%%%%%


\latexProblemContent{
\begin{problem}

Compute the indefinite integral:

\input{2311-Compute-Integral-0016.HELP.tex}

\[\int\;{-20 \, {\left(x^{2} + 6\right)} x}\;dx = \answer{{-5 \, {\left(x^{2} + 6\right)}^{2}}+C}\]
\end{problem}}%}

%%%%%%%%%%%%%%%%%%%%%%


\latexProblemContent{
\begin{problem}

Compute the indefinite integral:

\input{2311-Compute-Integral-0016.HELP.tex}

\[\int\;{-\frac{5 \, {\left(\frac{1}{x} + 6\right)}}{x^{2}}}\;dx = \answer{{\frac{5}{2} \, {\left(\frac{1}{x} + 6\right)}^{2}}+C}\]
\end{problem}}%}

%%%%%%%%%%%%%%%%%%%%%%


\latexProblemContent{
\begin{problem}

Compute the indefinite integral:

\input{2311-Compute-Integral-0016.HELP.tex}

\[\int\;{-16 \, {\left(x^{2} + 5\right)} x}\;dx = \answer{{-4 \, {\left(x^{2} + 5\right)}^{2}}+C}\]
\end{problem}}%}

%%%%%%%%%%%%%%%%%%%%%%


\latexProblemContent{
\begin{problem}

Compute the indefinite integral:

\input{2311-Compute-Integral-0016.HELP.tex}

\[\int\;{-\frac{10 \, {\left(\frac{1}{x^{2}} - 6\right)}}{x^{3}}}\;dx = \answer{{\frac{5}{2} \, {\left(\frac{1}{x^{2}} - 6\right)}^{2}}+C}\]
\end{problem}}%}

%%%%%%%%%%%%%%%%%%%%%%


\latexProblemContent{
\begin{problem}

Compute the indefinite integral:

\input{2311-Compute-Integral-0016.HELP.tex}

\[\int\;{-\frac{24 \, {\left(\frac{1}{x^{3}} - 10\right)}}{x^{4}}}\;dx = \answer{{4 \, {\left(\frac{1}{x^{3}} - 10\right)}^{2}}+C}\]
\end{problem}}%}

%%%%%%%%%%%%%%%%%%%%%%


\latexProblemContent{
\begin{problem}

Compute the indefinite integral:

\input{2311-Compute-Integral-0016.HELP.tex}

\[\int\;{\frac{6 \, {\left(\log\left(x\right) + 1\right)}}{x}}\;dx = \answer{{3 \, {\left(\log\left(x\right) + 1\right)}^{2}}+C}\]
\end{problem}}%}

%%%%%%%%%%%%%%%%%%%%%%


\latexProblemContent{
\begin{problem}

Compute the indefinite integral:

\input{2311-Compute-Integral-0016.HELP.tex}

\[\int\;{-\frac{10 \, {\left(\frac{1}{x^{2}} + 5\right)}}{x^{3}}}\;dx = \answer{{\frac{5}{2} \, {\left(\frac{1}{x^{2}} + 5\right)}^{2}}+C}\]
\end{problem}}%}

%%%%%%%%%%%%%%%%%%%%%%


\latexProblemContent{
\begin{problem}

Compute the indefinite integral:

\input{2311-Compute-Integral-0016.HELP.tex}

\[\int\;{-\frac{9 \, {\left(\frac{1}{x} - 4\right)}}{x^{2}}}\;dx = \answer{{\frac{9}{2} \, {\left(\frac{1}{x} - 4\right)}^{2}}+C}\]
\end{problem}}%}

%%%%%%%%%%%%%%%%%%%%%%


\latexProblemContent{
\begin{problem}

Compute the indefinite integral:

\input{2311-Compute-Integral-0016.HELP.tex}

\[\int\;{-\frac{7 \, {\left(\frac{1}{x} + 2\right)}}{x^{2}}}\;dx = \answer{{\frac{7}{2} \, {\left(\frac{1}{x} + 2\right)}^{2}}+C}\]
\end{problem}}%}

%%%%%%%%%%%%%%%%%%%%%%


\latexProblemContent{
\begin{problem}

Compute the indefinite integral:

\input{2311-Compute-Integral-0016.HELP.tex}

\[\int\;{-28 \, {\left(x^{4} + 10\right)} x^{3}}\;dx = \answer{{-\frac{7}{2} \, {\left(x^{4} + 10\right)}^{2}}+C}\]
\end{problem}}%}

%%%%%%%%%%%%%%%%%%%%%%


\latexProblemContent{
\begin{problem}

Compute the indefinite integral:

\input{2311-Compute-Integral-0016.HELP.tex}

\[\int\;{-10 \, {\left(e^{x} - 9\right)} e^{x}}\;dx = \answer{{-5 \, {\left(e^{x} - 9\right)}^{2}}+C}\]
\end{problem}}%}

%%%%%%%%%%%%%%%%%%%%%%


\latexProblemContent{
\begin{problem}

Compute the indefinite integral:

\input{2311-Compute-Integral-0016.HELP.tex}

\[\int\;{\frac{6 \, {\left(\frac{1}{x^{2}} + 1\right)}}{x^{3}}}\;dx = \answer{{-\frac{3}{2} \, {\left(\frac{1}{x^{2}} + 1\right)}^{2}}+C}\]
\end{problem}}%}

%%%%%%%%%%%%%%%%%%%%%%


\latexProblemContent{
\begin{problem}

Compute the indefinite integral:

\input{2311-Compute-Integral-0016.HELP.tex}

\[\int\;{-\frac{15 \, {\left(\frac{1}{x^{3}} + 3\right)}}{x^{4}}}\;dx = \answer{{\frac{5}{2} \, {\left(\frac{1}{x^{3}} + 3\right)}^{2}}+C}\]
\end{problem}}%}

%%%%%%%%%%%%%%%%%%%%%%


\latexProblemContent{
\begin{problem}

Compute the indefinite integral:

\input{2311-Compute-Integral-0016.HELP.tex}

\[\int\;{12 \, {\left(x^{2} - 1\right)} x}\;dx = \answer{{3 \, {\left(x^{2} - 1\right)}^{2}}+C}\]
\end{problem}}%}

%%%%%%%%%%%%%%%%%%%%%%


\latexProblemContent{
\begin{problem}

Compute the indefinite integral:

\input{2311-Compute-Integral-0016.HELP.tex}

\[\int\;{10 \, {\left(e^{x} - 6\right)} e^{x}}\;dx = \answer{{5 \, {\left(e^{x} - 6\right)}^{2}}+C}\]
\end{problem}}%}

%%%%%%%%%%%%%%%%%%%%%%


\latexProblemContent{
\begin{problem}

Compute the indefinite integral:

\input{2311-Compute-Integral-0016.HELP.tex}

\[\int\;{\frac{10 \, {\left(\frac{1}{x^{2}} - 9\right)}}{x^{3}}}\;dx = \answer{{-\frac{5}{2} \, {\left(\frac{1}{x^{2}} - 9\right)}^{2}}+C}\]
\end{problem}}%}

%%%%%%%%%%%%%%%%%%%%%%


\latexProblemContent{
\begin{problem}

Compute the indefinite integral:

\input{2311-Compute-Integral-0016.HELP.tex}

\[\int\;{\frac{7 \, {\left(\log\left(x\right) + 8\right)}}{x}}\;dx = \answer{{\frac{7}{2} \, {\left(\log\left(x\right) + 8\right)}^{2}}+C}\]
\end{problem}}%}

%%%%%%%%%%%%%%%%%%%%%%


\latexProblemContent{
\begin{problem}

Compute the indefinite integral:

\input{2311-Compute-Integral-0016.HELP.tex}

\[\int\;{\frac{2 \, {\left(\sqrt{x} + 1\right)}}{\sqrt{x}}}\;dx = \answer{{2 \, {\left(\sqrt{x} + 1\right)}^{2}}+C}\]
\end{problem}}%}

%%%%%%%%%%%%%%%%%%%%%%


\latexProblemContent{
\begin{problem}

Compute the indefinite integral:

\input{2311-Compute-Integral-0016.HELP.tex}

\[\int\;{-27 \, {\left(x^{3} + 5\right)} x^{2}}\;dx = \answer{{-\frac{9}{2} \, {\left(x^{3} + 5\right)}^{2}}+C}\]
\end{problem}}%}

%%%%%%%%%%%%%%%%%%%%%%


\latexProblemContent{
\begin{problem}

Compute the indefinite integral:

\input{2311-Compute-Integral-0016.HELP.tex}

\[\int\;{-\frac{4 \, {\left(\sqrt{x} - 7\right)}}{\sqrt{x}}}\;dx = \answer{{-4 \, {\left(\sqrt{x} - 7\right)}^{2}}+C}\]
\end{problem}}%}

%%%%%%%%%%%%%%%%%%%%%%


\latexProblemContent{
\begin{problem}

Compute the indefinite integral:

\input{2311-Compute-Integral-0016.HELP.tex}

\[\int\;{-20 \, {\left(x^{2} - 8\right)} x}\;dx = \answer{{-5 \, {\left(x^{2} - 8\right)}^{2}}+C}\]
\end{problem}}%}

%%%%%%%%%%%%%%%%%%%%%%


\latexProblemContent{
\begin{problem}

Compute the indefinite integral:

\input{2311-Compute-Integral-0016.HELP.tex}

\[\int\;{-\frac{7 \, {\left(\sqrt{x} - 3\right)}}{2 \, \sqrt{x}}}\;dx = \answer{{-\frac{7}{2} \, {\left(\sqrt{x} - 3\right)}^{2}}+C}\]
\end{problem}}%}

%%%%%%%%%%%%%%%%%%%%%%


\latexProblemContent{
\begin{problem}

Compute the indefinite integral:

\input{2311-Compute-Integral-0016.HELP.tex}

\[\int\;{-9 \, {\left(e^{x} + 5\right)} e^{x}}\;dx = \answer{{-\frac{9}{2} \, {\left(e^{x} + 5\right)}^{2}}+C}\]
\end{problem}}%}

%%%%%%%%%%%%%%%%%%%%%%


\latexProblemContent{
\begin{problem}

Compute the indefinite integral:

\input{2311-Compute-Integral-0016.HELP.tex}

\[\int\;{3 \, {\left(\cos\left(x\right) - 8\right)} \sin\left(x\right)}\;dx = \answer{{-\frac{3}{2} \, {\left(\cos\left(x\right) - 8\right)}^{2}}+C}\]
\end{problem}}%}

%%%%%%%%%%%%%%%%%%%%%%


\latexProblemContent{
\begin{problem}

Compute the indefinite integral:

\input{2311-Compute-Integral-0016.HELP.tex}

\[\int\;{\frac{6 \, {\left(\log\left(x\right) - 2\right)}}{x}}\;dx = \answer{{3 \, {\left(\log\left(x\right) - 2\right)}^{2}}+C}\]
\end{problem}}%}

%%%%%%%%%%%%%%%%%%%%%%


\latexProblemContent{
\begin{problem}

Compute the indefinite integral:

\input{2311-Compute-Integral-0016.HELP.tex}

\[\int\;{10 \, {\left(x^{2} - 1\right)} x}\;dx = \answer{{\frac{5}{2} \, {\left(x^{2} - 1\right)}^{2}}+C}\]
\end{problem}}%}

%%%%%%%%%%%%%%%%%%%%%%


\latexProblemContent{
\begin{problem}

Compute the indefinite integral:

\input{2311-Compute-Integral-0016.HELP.tex}

\[\int\;{10 \, {\left(\sin\left(x\right) + 3\right)} \cos\left(x\right)}\;dx = \answer{{5 \, {\left(\sin\left(x\right) + 3\right)}^{2}}+C}\]
\end{problem}}%}

%%%%%%%%%%%%%%%%%%%%%%


\latexProblemContent{
\begin{problem}

Compute the indefinite integral:

\input{2311-Compute-Integral-0016.HELP.tex}

\[\int\;{\frac{6 \, {\left(\frac{1}{x} - 6\right)}}{x^{2}}}\;dx = \answer{{-3 \, {\left(\frac{1}{x} - 6\right)}^{2}}+C}\]
\end{problem}}%}

%%%%%%%%%%%%%%%%%%%%%%


\latexProblemContent{
\begin{problem}

Compute the indefinite integral:

\input{2311-Compute-Integral-0016.HELP.tex}

\[\int\;{-\frac{\sqrt{x} + 6}{\sqrt{x}}}\;dx = \answer{{-{\left(\sqrt{x} + 6\right)}^{2}}+C}\]
\end{problem}}%}

%%%%%%%%%%%%%%%%%%%%%%


\latexProblemContent{
\begin{problem}

Compute the indefinite integral:

\input{2311-Compute-Integral-0016.HELP.tex}

\[\int\;{2 \, {\left(\sin\left(x\right) + 6\right)} \cos\left(x\right)}\;dx = \answer{{{\left(\sin\left(x\right) + 6\right)}^{2}}+C}\]
\end{problem}}%}

%%%%%%%%%%%%%%%%%%%%%%


\latexProblemContent{
\begin{problem}

Compute the indefinite integral:

\input{2311-Compute-Integral-0016.HELP.tex}

\[\int\;{2 \, {\left(\sin\left(x\right) + 5\right)} \cos\left(x\right)}\;dx = \answer{{{\left(\sin\left(x\right) + 5\right)}^{2}}+C}\]
\end{problem}}%}

%%%%%%%%%%%%%%%%%%%%%%


\latexProblemContent{
\begin{problem}

Compute the indefinite integral:

\input{2311-Compute-Integral-0016.HELP.tex}

\[\int\;{-10 \, {\left(x^{2} + 5\right)} x}\;dx = \answer{{-\frac{5}{2} \, {\left(x^{2} + 5\right)}^{2}}+C}\]
\end{problem}}%}

%%%%%%%%%%%%%%%%%%%%%%


\latexProblemContent{
\begin{problem}

Compute the indefinite integral:

\input{2311-Compute-Integral-0016.HELP.tex}

\[\int\;{-36 \, {\left(x^{4} + 1\right)} x^{3}}\;dx = \answer{{-\frac{9}{2} \, {\left(x^{4} + 1\right)}^{2}}+C}\]
\end{problem}}%}

%%%%%%%%%%%%%%%%%%%%%%


\latexProblemContent{
\begin{problem}

Compute the indefinite integral:

\input{2311-Compute-Integral-0016.HELP.tex}

\[\int\;{-4 \, {\left(x^{2} + 5\right)} x}\;dx = \answer{{-{\left(x^{2} + 5\right)}^{2}}+C}\]
\end{problem}}%}

%%%%%%%%%%%%%%%%%%%%%%


\latexProblemContent{
\begin{problem}

Compute the indefinite integral:

\input{2311-Compute-Integral-0016.HELP.tex}

\[\int\;{-6 \, {\left(\sin\left(x\right) - 4\right)} \cos\left(x\right)}\;dx = \answer{{-3 \, {\left(\sin\left(x\right) - 4\right)}^{2}}+C}\]
\end{problem}}%}

%%%%%%%%%%%%%%%%%%%%%%


\latexProblemContent{
\begin{problem}

Compute the indefinite integral:

\input{2311-Compute-Integral-0016.HELP.tex}

\[\int\;{-\frac{2 \, {\left(\log\left(x\right) + 4\right)}}{x}}\;dx = \answer{{-{\left(\log\left(x\right) + 4\right)}^{2}}+C}\]
\end{problem}}%}

%%%%%%%%%%%%%%%%%%%%%%


\latexProblemContent{
\begin{problem}

Compute the indefinite integral:

\input{2311-Compute-Integral-0016.HELP.tex}

\[\int\;{-\frac{5 \, {\left(\frac{1}{x} - 4\right)}}{x^{2}}}\;dx = \answer{{\frac{5}{2} \, {\left(\frac{1}{x} - 4\right)}^{2}}+C}\]
\end{problem}}%}

%%%%%%%%%%%%%%%%%%%%%%


\latexProblemContent{
\begin{problem}

Compute the indefinite integral:

\input{2311-Compute-Integral-0016.HELP.tex}

\[\int\;{-\frac{18 \, {\left(\frac{1}{x^{3}} + 8\right)}}{x^{4}}}\;dx = \answer{{3 \, {\left(\frac{1}{x^{3}} + 8\right)}^{2}}+C}\]
\end{problem}}%}

%%%%%%%%%%%%%%%%%%%%%%


\latexProblemContent{
\begin{problem}

Compute the indefinite integral:

\input{2311-Compute-Integral-0016.HELP.tex}

\[\int\;{-10 \, {\left(\sin\left(x\right) - 9\right)} \cos\left(x\right)}\;dx = \answer{{-5 \, {\left(\sin\left(x\right) - 9\right)}^{2}}+C}\]
\end{problem}}%}

%%%%%%%%%%%%%%%%%%%%%%


\latexProblemContent{
\begin{problem}

Compute the indefinite integral:

\input{2311-Compute-Integral-0016.HELP.tex}

\[\int\;{-10 \, x - 10}\;dx = \answer{{-5 \, x^{2} - 10 \, x}+C}\]
\end{problem}}%}

%%%%%%%%%%%%%%%%%%%%%%


\latexProblemContent{
\begin{problem}

Compute the indefinite integral:

\input{2311-Compute-Integral-0016.HELP.tex}

\[\int\;{\frac{12 \, {\left(\frac{1}{x^{3}} + 7\right)}}{x^{4}}}\;dx = \answer{{-2 \, {\left(\frac{1}{x^{3}} + 7\right)}^{2}}+C}\]
\end{problem}}%}

%%%%%%%%%%%%%%%%%%%%%%


%%%%%%%%%%%%%%%%%%%%%%


%%%%%%%%%%%%%%%%%%%%%%


\latexProblemContent{
\begin{problem}

Compute the indefinite integral:

\input{2311-Compute-Integral-0016.HELP.tex}

\[\int\;{-18 \, {\left(x^{3} + 9\right)} x^{2}}\;dx = \answer{{-3 \, {\left(x^{3} + 9\right)}^{2}}+C}\]
\end{problem}}%}

%%%%%%%%%%%%%%%%%%%%%%


\latexProblemContent{
\begin{problem}

Compute the indefinite integral:

\input{2311-Compute-Integral-0016.HELP.tex}

\[\int\;{-21 \, {\left(x^{3} - 1\right)} x^{2}}\;dx = \answer{{-\frac{7}{2} \, {\left(x^{3} - 1\right)}^{2}}+C}\]
\end{problem}}%}

%%%%%%%%%%%%%%%%%%%%%%


\latexProblemContent{
\begin{problem}

Compute the indefinite integral:

\input{2311-Compute-Integral-0016.HELP.tex}

\[\int\;{10 \, {\left(\cos\left(x\right) - 6\right)} \sin\left(x\right)}\;dx = \answer{{-5 \, {\left(\cos\left(x\right) - 6\right)}^{2}}+C}\]
\end{problem}}%}

%%%%%%%%%%%%%%%%%%%%%%


\latexProblemContent{
\begin{problem}

Compute the indefinite integral:

\input{2311-Compute-Integral-0016.HELP.tex}

\[\int\;{-10 \, {\left(\sin\left(x\right) + 6\right)} \cos\left(x\right)}\;dx = \answer{{-5 \, {\left(\sin\left(x\right) + 6\right)}^{2}}+C}\]
\end{problem}}%}

%%%%%%%%%%%%%%%%%%%%%%


\latexProblemContent{
\begin{problem}

Compute the indefinite integral:

\input{2311-Compute-Integral-0016.HELP.tex}

\[\int\;{-3 \, x - 27}\;dx = \answer{{-\frac{3}{2} \, x^{2} - 27 \, x}+C}\]
\end{problem}}%}

%%%%%%%%%%%%%%%%%%%%%%


\latexProblemContent{
\begin{problem}

Compute the indefinite integral:

\input{2311-Compute-Integral-0016.HELP.tex}

\[\int\;{-14 \, {\left(x^{2} - 10\right)} x}\;dx = \answer{{-\frac{7}{2} \, {\left(x^{2} - 10\right)}^{2}}+C}\]
\end{problem}}%}

%%%%%%%%%%%%%%%%%%%%%%


\latexProblemContent{
\begin{problem}

Compute the indefinite integral:

\input{2311-Compute-Integral-0016.HELP.tex}

\[\int\;{30 \, {\left(x^{3} - 6\right)} x^{2}}\;dx = \answer{{5 \, {\left(x^{3} - 6\right)}^{2}}+C}\]
\end{problem}}%}

%%%%%%%%%%%%%%%%%%%%%%


\latexProblemContent{
\begin{problem}

Compute the indefinite integral:

\input{2311-Compute-Integral-0016.HELP.tex}

\[\int\;{-10 \, x - 90}\;dx = \answer{{-5 \, x^{2} - 90 \, x}+C}\]
\end{problem}}%}

%%%%%%%%%%%%%%%%%%%%%%


%%%%%%%%%%%%%%%%%%%%%%


\latexProblemContent{
\begin{problem}

Compute the indefinite integral:

\input{2311-Compute-Integral-0016.HELP.tex}

\[\int\;{-\frac{\sqrt{x} + 8}{\sqrt{x}}}\;dx = \answer{{-{\left(\sqrt{x} + 8\right)}^{2}}+C}\]
\end{problem}}%}

%%%%%%%%%%%%%%%%%%%%%%


\latexProblemContent{
\begin{problem}

Compute the indefinite integral:

\input{2311-Compute-Integral-0016.HELP.tex}

\[\int\;{-{\left(e^{x} + 7\right)} e^{x}}\;dx = \answer{{-\frac{1}{2} \, {\left(e^{x} + 7\right)}^{2}}+C}\]
\end{problem}}%}

%%%%%%%%%%%%%%%%%%%%%%


\latexProblemContent{
\begin{problem}

Compute the indefinite integral:

\input{2311-Compute-Integral-0016.HELP.tex}

\[\int\;{18 \, {\left(x^{2} - 3\right)} x}\;dx = \answer{{\frac{9}{2} \, {\left(x^{2} - 3\right)}^{2}}+C}\]
\end{problem}}%}

%%%%%%%%%%%%%%%%%%%%%%


\latexProblemContent{
\begin{problem}

Compute the indefinite integral:

\input{2311-Compute-Integral-0016.HELP.tex}

\[\int\;{-3 \, {\left(\sin\left(x\right) + 6\right)} \cos\left(x\right)}\;dx = \answer{{-\frac{3}{2} \, {\left(\sin\left(x\right) + 6\right)}^{2}}+C}\]
\end{problem}}%}

%%%%%%%%%%%%%%%%%%%%%%


\latexProblemContent{
\begin{problem}

Compute the indefinite integral:

\input{2311-Compute-Integral-0016.HELP.tex}

\[\int\;{-16 \, {\left(x^{4} + 6\right)} x^{3}}\;dx = \answer{{-2 \, {\left(x^{4} + 6\right)}^{2}}+C}\]
\end{problem}}%}

%%%%%%%%%%%%%%%%%%%%%%


\latexProblemContent{
\begin{problem}

Compute the indefinite integral:

\input{2311-Compute-Integral-0016.HELP.tex}

\[\int\;{-8 \, {\left(\cos\left(x\right) - 9\right)} \sin\left(x\right)}\;dx = \answer{{4 \, {\left(\cos\left(x\right) - 9\right)}^{2}}+C}\]
\end{problem}}%}

%%%%%%%%%%%%%%%%%%%%%%


\latexProblemContent{
\begin{problem}

Compute the indefinite integral:

\input{2311-Compute-Integral-0016.HELP.tex}

\[\int\;{\frac{5 \, {\left(\sqrt{x} + 4\right)}}{2 \, \sqrt{x}}}\;dx = \answer{{\frac{5}{2} \, {\left(\sqrt{x} + 4\right)}^{2}}+C}\]
\end{problem}}%}

%%%%%%%%%%%%%%%%%%%%%%


\latexProblemContent{
\begin{problem}

Compute the indefinite integral:

\input{2311-Compute-Integral-0016.HELP.tex}

\[\int\;{9 \, {\left(\sin\left(x\right) - 5\right)} \cos\left(x\right)}\;dx = \answer{{\frac{9}{2} \, {\left(\sin\left(x\right) - 5\right)}^{2}}+C}\]
\end{problem}}%}

%%%%%%%%%%%%%%%%%%%%%%


\latexProblemContent{
\begin{problem}

Compute the indefinite integral:

\input{2311-Compute-Integral-0016.HELP.tex}

\[\int\;{4 \, {\left(\cos\left(x\right) + 2\right)} \sin\left(x\right)}\;dx = \answer{{-2 \, {\left(\cos\left(x\right) + 2\right)}^{2}}+C}\]
\end{problem}}%}

%%%%%%%%%%%%%%%%%%%%%%


\latexProblemContent{
\begin{problem}

Compute the indefinite integral:

\input{2311-Compute-Integral-0016.HELP.tex}

\[\int\;{2 \, {\left(x^{2} + 8\right)} x}\;dx = \answer{{\frac{1}{2} \, {\left(x^{2} + 8\right)}^{2}}+C}\]
\end{problem}}%}

%%%%%%%%%%%%%%%%%%%%%%


\latexProblemContent{
\begin{problem}

Compute the indefinite integral:

\input{2311-Compute-Integral-0016.HELP.tex}

\[\int\;{\frac{4 \, {\left(\frac{1}{x} + 7\right)}}{x^{2}}}\;dx = \answer{{-2 \, {\left(\frac{1}{x} + 7\right)}^{2}}+C}\]
\end{problem}}%}

%%%%%%%%%%%%%%%%%%%%%%


\latexProblemContent{
\begin{problem}

Compute the indefinite integral:

\input{2311-Compute-Integral-0016.HELP.tex}

\[\int\;{-3 \, {\left(\sin\left(x\right) + 10\right)} \cos\left(x\right)}\;dx = \answer{{-\frac{3}{2} \, {\left(\sin\left(x\right) + 10\right)}^{2}}+C}\]
\end{problem}}%}

%%%%%%%%%%%%%%%%%%%%%%


%%%%%%%%%%%%%%%%%%%%%%


%%%%%%%%%%%%%%%%%%%%%%


\latexProblemContent{
\begin{problem}

Compute the indefinite integral:

\input{2311-Compute-Integral-0016.HELP.tex}

\[\int\;{-x}\;dx = \answer{{-\frac{1}{2} \, x^{2}}+C}\]
\end{problem}}%}

%%%%%%%%%%%%%%%%%%%%%%


\latexProblemContent{
\begin{problem}

Compute the indefinite integral:

\input{2311-Compute-Integral-0016.HELP.tex}

\[\int\;{18 \, {\left(x^{3} + 8\right)} x^{2}}\;dx = \answer{{3 \, {\left(x^{3} + 8\right)}^{2}}+C}\]
\end{problem}}%}

%%%%%%%%%%%%%%%%%%%%%%


\latexProblemContent{
\begin{problem}

Compute the indefinite integral:

\input{2311-Compute-Integral-0016.HELP.tex}

\[\int\;{\frac{8 \, {\left(\frac{1}{x^{2}} + 9\right)}}{x^{3}}}\;dx = \answer{{-2 \, {\left(\frac{1}{x^{2}} + 9\right)}^{2}}+C}\]
\end{problem}}%}

%%%%%%%%%%%%%%%%%%%%%%


\latexProblemContent{
\begin{problem}

Compute the indefinite integral:

\input{2311-Compute-Integral-0016.HELP.tex}

\[\int\;{6 \, {\left(x^{3} + 1\right)} x^{2}}\;dx = \answer{{{\left(x^{3} + 1\right)}^{2}}+C}\]
\end{problem}}%}

%%%%%%%%%%%%%%%%%%%%%%


\latexProblemContent{
\begin{problem}

Compute the indefinite integral:

\input{2311-Compute-Integral-0016.HELP.tex}

\[\int\;{6 \, x - 54}\;dx = \answer{{3 \, x^{2} - 54 \, x}+C}\]
\end{problem}}%}

%%%%%%%%%%%%%%%%%%%%%%


\latexProblemContent{
\begin{problem}

Compute the indefinite integral:

\input{2311-Compute-Integral-0016.HELP.tex}

\[\int\;{-10 \, {\left(e^{x} + 4\right)} e^{x}}\;dx = \answer{{-5 \, {\left(e^{x} + 4\right)}^{2}}+C}\]
\end{problem}}%}

%%%%%%%%%%%%%%%%%%%%%%


\latexProblemContent{
\begin{problem}

Compute the indefinite integral:

\input{2311-Compute-Integral-0016.HELP.tex}

\[\int\;{\frac{8 \, {\left(\log\left(x\right) - 9\right)}}{x}}\;dx = \answer{{4 \, {\left(\log\left(x\right) - 9\right)}^{2}}+C}\]
\end{problem}}%}

%%%%%%%%%%%%%%%%%%%%%%


\latexProblemContent{
\begin{problem}

Compute the indefinite integral:

\input{2311-Compute-Integral-0016.HELP.tex}

\[\int\;{-16 \, {\left(x^{2} + 8\right)} x}\;dx = \answer{{-4 \, {\left(x^{2} + 8\right)}^{2}}+C}\]
\end{problem}}%}

%%%%%%%%%%%%%%%%%%%%%%


\latexProblemContent{
\begin{problem}

Compute the indefinite integral:

\input{2311-Compute-Integral-0016.HELP.tex}

\[\int\;{21 \, {\left(x^{3} + 3\right)} x^{2}}\;dx = \answer{{\frac{7}{2} \, {\left(x^{3} + 3\right)}^{2}}+C}\]
\end{problem}}%}

%%%%%%%%%%%%%%%%%%%%%%


\latexProblemContent{
\begin{problem}

Compute the indefinite integral:

\input{2311-Compute-Integral-0016.HELP.tex}

\[\int\;{-10 \, {\left(e^{x} + 1\right)} e^{x}}\;dx = \answer{{-5 \, {\left(e^{x} + 1\right)}^{2}}+C}\]
\end{problem}}%}

%%%%%%%%%%%%%%%%%%%%%%


\latexProblemContent{
\begin{problem}

Compute the indefinite integral:

\input{2311-Compute-Integral-0016.HELP.tex}

\[\int\;{-4 \, {\left(x^{2} + 1\right)} x}\;dx = \answer{{-{\left(x^{2} + 1\right)}^{2}}+C}\]
\end{problem}}%}

%%%%%%%%%%%%%%%%%%%%%%


\latexProblemContent{
\begin{problem}

Compute the indefinite integral:

\input{2311-Compute-Integral-0016.HELP.tex}

\[\int\;{-\frac{2 \, {\left(\frac{1}{x} - 9\right)}}{x^{2}}}\;dx = \answer{{{\left(\frac{1}{x} - 9\right)}^{2}}+C}\]
\end{problem}}%}

%%%%%%%%%%%%%%%%%%%%%%


\latexProblemContent{
\begin{problem}

Compute the indefinite integral:

\input{2311-Compute-Integral-0016.HELP.tex}

\[\int\;{-\frac{7 \, {\left(\sqrt{x} + 5\right)}}{2 \, \sqrt{x}}}\;dx = \answer{{-\frac{7}{2} \, {\left(\sqrt{x} + 5\right)}^{2}}+C}\]
\end{problem}}%}

%%%%%%%%%%%%%%%%%%%%%%


\latexProblemContent{
\begin{problem}

Compute the indefinite integral:

\input{2311-Compute-Integral-0016.HELP.tex}

\[\int\;{-9 \, {\left(e^{x} - 9\right)} e^{x}}\;dx = \answer{{-\frac{9}{2} \, {\left(e^{x} - 9\right)}^{2}}+C}\]
\end{problem}}%}

%%%%%%%%%%%%%%%%%%%%%%


\latexProblemContent{
\begin{problem}

Compute the indefinite integral:

\input{2311-Compute-Integral-0016.HELP.tex}

\[\int\;{-\frac{6 \, {\left(\frac{1}{x^{2}} + 10\right)}}{x^{3}}}\;dx = \answer{{\frac{3}{2} \, {\left(\frac{1}{x^{2}} + 10\right)}^{2}}+C}\]
\end{problem}}%}

%%%%%%%%%%%%%%%%%%%%%%


\latexProblemContent{
\begin{problem}

Compute the indefinite integral:

\input{2311-Compute-Integral-0016.HELP.tex}

\[\int\;{\frac{5 \, {\left(\sqrt{x} - 2\right)}}{\sqrt{x}}}\;dx = \answer{{5 \, {\left(\sqrt{x} - 2\right)}^{2}}+C}\]
\end{problem}}%}

%%%%%%%%%%%%%%%%%%%%%%


\latexProblemContent{
\begin{problem}

Compute the indefinite integral:

\input{2311-Compute-Integral-0016.HELP.tex}

\[\int\;{9 \, {\left(\sin\left(x\right) - 10\right)} \cos\left(x\right)}\;dx = \answer{{\frac{9}{2} \, {\left(\sin\left(x\right) - 10\right)}^{2}}+C}\]
\end{problem}}%}

%%%%%%%%%%%%%%%%%%%%%%


\latexProblemContent{
\begin{problem}

Compute the indefinite integral:

\input{2311-Compute-Integral-0016.HELP.tex}

\[\int\;{-18 \, {\left(x^{2} - 5\right)} x}\;dx = \answer{{-\frac{9}{2} \, {\left(x^{2} - 5\right)}^{2}}+C}\]
\end{problem}}%}

%%%%%%%%%%%%%%%%%%%%%%


\latexProblemContent{
\begin{problem}

Compute the indefinite integral:

\input{2311-Compute-Integral-0016.HELP.tex}

\[\int\;{-\frac{3 \, {\left(\sqrt{x} + 2\right)}}{2 \, \sqrt{x}}}\;dx = \answer{{-\frac{3}{2} \, {\left(\sqrt{x} + 2\right)}^{2}}+C}\]
\end{problem}}%}

%%%%%%%%%%%%%%%%%%%%%%


\latexProblemContent{
\begin{problem}

Compute the indefinite integral:

\input{2311-Compute-Integral-0016.HELP.tex}

\[\int\;{\frac{6 \, {\left(\frac{1}{x^{2}} - 9\right)}}{x^{3}}}\;dx = \answer{{-\frac{3}{2} \, {\left(\frac{1}{x^{2}} - 9\right)}^{2}}+C}\]
\end{problem}}%}

%%%%%%%%%%%%%%%%%%%%%%


%%%%%%%%%%%%%%%%%%%%%%


\latexProblemContent{
\begin{problem}

Compute the indefinite integral:

\input{2311-Compute-Integral-0016.HELP.tex}

\[\int\;{-32 \, {\left(x^{4} - 3\right)} x^{3}}\;dx = \answer{{-4 \, {\left(x^{4} - 3\right)}^{2}}+C}\]
\end{problem}}%}

%%%%%%%%%%%%%%%%%%%%%%


\latexProblemContent{
\begin{problem}

Compute the indefinite integral:

\input{2311-Compute-Integral-0016.HELP.tex}

\[\int\;{\frac{2 \, {\left(\frac{1}{x} + 9\right)}}{x^{2}}}\;dx = \answer{{-{\left(\frac{1}{x} + 9\right)}^{2}}+C}\]
\end{problem}}%}

%%%%%%%%%%%%%%%%%%%%%%


\latexProblemContent{
\begin{problem}

Compute the indefinite integral:

\input{2311-Compute-Integral-0016.HELP.tex}

\[\int\;{-6 \, x - 12}\;dx = \answer{{-3 \, x^{2} - 12 \, x}+C}\]
\end{problem}}%}

%%%%%%%%%%%%%%%%%%%%%%


\latexProblemContent{
\begin{problem}

Compute the indefinite integral:

\input{2311-Compute-Integral-0016.HELP.tex}

\[\int\;{-8 \, {\left(x^{4} + 5\right)} x^{3}}\;dx = \answer{{-{\left(x^{4} + 5\right)}^{2}}+C}\]
\end{problem}}%}

%%%%%%%%%%%%%%%%%%%%%%


\latexProblemContent{
\begin{problem}

Compute the indefinite integral:

\input{2311-Compute-Integral-0016.HELP.tex}

\[\int\;{-7 \, {\left(\cos\left(x\right) - 9\right)} \sin\left(x\right)}\;dx = \answer{{\frac{7}{2} \, {\left(\cos\left(x\right) - 9\right)}^{2}}+C}\]
\end{problem}}%}

%%%%%%%%%%%%%%%%%%%%%%


\latexProblemContent{
\begin{problem}

Compute the indefinite integral:

\input{2311-Compute-Integral-0016.HELP.tex}

\[\int\;{\frac{6 \, {\left(\frac{1}{x^{2}} + 10\right)}}{x^{3}}}\;dx = \answer{{-\frac{3}{2} \, {\left(\frac{1}{x^{2}} + 10\right)}^{2}}+C}\]
\end{problem}}%}

%%%%%%%%%%%%%%%%%%%%%%


\latexProblemContent{
\begin{problem}

Compute the indefinite integral:

\input{2311-Compute-Integral-0016.HELP.tex}

\[\int\;{14 \, {\left(x^{2} - 7\right)} x}\;dx = \answer{{\frac{7}{2} \, {\left(x^{2} - 7\right)}^{2}}+C}\]
\end{problem}}%}

%%%%%%%%%%%%%%%%%%%%%%


\latexProblemContent{
\begin{problem}

Compute the indefinite integral:

\input{2311-Compute-Integral-0016.HELP.tex}

\[\int\;{-4 \, {\left(e^{x} + 1\right)} e^{x}}\;dx = \answer{{-2 \, {\left(e^{x} + 1\right)}^{2}}+C}\]
\end{problem}}%}

%%%%%%%%%%%%%%%%%%%%%%


\latexProblemContent{
\begin{problem}

Compute the indefinite integral:

\input{2311-Compute-Integral-0016.HELP.tex}

\[\int\;{8 \, {\left(x^{2} - 1\right)} x}\;dx = \answer{{2 \, {\left(x^{2} - 1\right)}^{2}}+C}\]
\end{problem}}%}

%%%%%%%%%%%%%%%%%%%%%%


\latexProblemContent{
\begin{problem}

Compute the indefinite integral:

\input{2311-Compute-Integral-0016.HELP.tex}

\[\int\;{\frac{5 \, {\left(\log\left(x\right) - 1\right)}}{x}}\;dx = \answer{{\frac{5}{2} \, {\left(\log\left(x\right) - 1\right)}^{2}}+C}\]
\end{problem}}%}

%%%%%%%%%%%%%%%%%%%%%%


\latexProblemContent{
\begin{problem}

Compute the indefinite integral:

\input{2311-Compute-Integral-0016.HELP.tex}

\[\int\;{5 \, {\left(\sin\left(x\right) - 9\right)} \cos\left(x\right)}\;dx = \answer{{\frac{5}{2} \, {\left(\sin\left(x\right) - 9\right)}^{2}}+C}\]
\end{problem}}%}

%%%%%%%%%%%%%%%%%%%%%%


\latexProblemContent{
\begin{problem}

Compute the indefinite integral:

\input{2311-Compute-Integral-0016.HELP.tex}

\[\int\;{\frac{12 \, {\left(\frac{1}{x^{3}} - 8\right)}}{x^{4}}}\;dx = \answer{{-2 \, {\left(\frac{1}{x^{3}} - 8\right)}^{2}}+C}\]
\end{problem}}%}

%%%%%%%%%%%%%%%%%%%%%%


\latexProblemContent{
\begin{problem}

Compute the indefinite integral:

\input{2311-Compute-Integral-0016.HELP.tex}

\[\int\;{-12 \, {\left(x^{4} + 10\right)} x^{3}}\;dx = \answer{{-\frac{3}{2} \, {\left(x^{4} + 10\right)}^{2}}+C}\]
\end{problem}}%}

%%%%%%%%%%%%%%%%%%%%%%


\latexProblemContent{
\begin{problem}

Compute the indefinite integral:

\input{2311-Compute-Integral-0016.HELP.tex}

\[\int\;{28 \, {\left(x^{4} - 10\right)} x^{3}}\;dx = \answer{{\frac{7}{2} \, {\left(x^{4} - 10\right)}^{2}}+C}\]
\end{problem}}%}

%%%%%%%%%%%%%%%%%%%%%%


\latexProblemContent{
\begin{problem}

Compute the indefinite integral:

\input{2311-Compute-Integral-0016.HELP.tex}

\[\int\;{\frac{4 \, {\left(\frac{1}{x} + 9\right)}}{x^{2}}}\;dx = \answer{{-2 \, {\left(\frac{1}{x} + 9\right)}^{2}}+C}\]
\end{problem}}%}

%%%%%%%%%%%%%%%%%%%%%%


\latexProblemContent{
\begin{problem}

Compute the indefinite integral:

\input{2311-Compute-Integral-0016.HELP.tex}

\[\int\;{-21 \, {\left(x^{3} + 9\right)} x^{2}}\;dx = \answer{{-\frac{7}{2} \, {\left(x^{3} + 9\right)}^{2}}+C}\]
\end{problem}}%}

%%%%%%%%%%%%%%%%%%%%%%


\latexProblemContent{
\begin{problem}

Compute the indefinite integral:

\input{2311-Compute-Integral-0016.HELP.tex}

\[\int\;{4 \, {\left(\sin\left(x\right) + 6\right)} \cos\left(x\right)}\;dx = \answer{{2 \, {\left(\sin\left(x\right) + 6\right)}^{2}}+C}\]
\end{problem}}%}

%%%%%%%%%%%%%%%%%%%%%%


\latexProblemContent{
\begin{problem}

Compute the indefinite integral:

\input{2311-Compute-Integral-0016.HELP.tex}

\[\int\;{-{\left(\sin\left(x\right) - 4\right)} \cos\left(x\right)}\;dx = \answer{{-\frac{1}{2} \, {\left(\sin\left(x\right) - 4\right)}^{2}}+C}\]
\end{problem}}%}

%%%%%%%%%%%%%%%%%%%%%%


\latexProblemContent{
\begin{problem}

Compute the indefinite integral:

\input{2311-Compute-Integral-0016.HELP.tex}

\[\int\;{4 \, {\left(x^{2} + 5\right)} x}\;dx = \answer{{{\left(x^{2} + 5\right)}^{2}}+C}\]
\end{problem}}%}

%%%%%%%%%%%%%%%%%%%%%%


\latexProblemContent{
\begin{problem}

Compute the indefinite integral:

\input{2311-Compute-Integral-0016.HELP.tex}

\[\int\;{\frac{10 \, {\left(\log\left(x\right) - 9\right)}}{x}}\;dx = \answer{{5 \, {\left(\log\left(x\right) - 9\right)}^{2}}+C}\]
\end{problem}}%}

%%%%%%%%%%%%%%%%%%%%%%


\latexProblemContent{
\begin{problem}

Compute the indefinite integral:

\input{2311-Compute-Integral-0016.HELP.tex}

\[\int\;{-2 \, {\left(\sin\left(x\right) + 5\right)} \cos\left(x\right)}\;dx = \answer{{-{\left(\sin\left(x\right) + 5\right)}^{2}}+C}\]
\end{problem}}%}

%%%%%%%%%%%%%%%%%%%%%%


\latexProblemContent{
\begin{problem}

Compute the indefinite integral:

\input{2311-Compute-Integral-0016.HELP.tex}

\[\int\;{-6 \, {\left(e^{x} - 7\right)} e^{x}}\;dx = \answer{{-3 \, {\left(e^{x} - 7\right)}^{2}}+C}\]
\end{problem}}%}

%%%%%%%%%%%%%%%%%%%%%%


\latexProblemContent{
\begin{problem}

Compute the indefinite integral:

\input{2311-Compute-Integral-0016.HELP.tex}

\[\int\;{\frac{5 \, {\left(\frac{1}{x} - 6\right)}}{x^{2}}}\;dx = \answer{{-\frac{5}{2} \, {\left(\frac{1}{x} - 6\right)}^{2}}+C}\]
\end{problem}}%}

%%%%%%%%%%%%%%%%%%%%%%


%%%%%%%%%%%%%%%%%%%%%%


\latexProblemContent{
\begin{problem}

Compute the indefinite integral:

\input{2311-Compute-Integral-0016.HELP.tex}

\[\int\;{-8 \, {\left(e^{x} + 8\right)} e^{x}}\;dx = \answer{{-4 \, {\left(e^{x} + 8\right)}^{2}}+C}\]
\end{problem}}%}

%%%%%%%%%%%%%%%%%%%%%%


%%%%%%%%%%%%%%%%%%%%%%


%%%%%%%%%%%%%%%%%%%%%%


\latexProblemContent{
\begin{problem}

Compute the indefinite integral:

\input{2311-Compute-Integral-0016.HELP.tex}

\[\int\;{\frac{4 \, {\left(\frac{1}{x} - 2\right)}}{x^{2}}}\;dx = \answer{{-2 \, {\left(\frac{1}{x} - 2\right)}^{2}}+C}\]
\end{problem}}%}

%%%%%%%%%%%%%%%%%%%%%%


\latexProblemContent{
\begin{problem}

Compute the indefinite integral:

\input{2311-Compute-Integral-0016.HELP.tex}

\[\int\;{-7 \, x + 70}\;dx = \answer{{-\frac{7}{2} \, x^{2} + 70 \, x}+C}\]
\end{problem}}%}

%%%%%%%%%%%%%%%%%%%%%%


\latexProblemContent{
\begin{problem}

Compute the indefinite integral:

\input{2311-Compute-Integral-0016.HELP.tex}

\[\int\;{-24 \, {\left(x^{3} + 9\right)} x^{2}}\;dx = \answer{{-4 \, {\left(x^{3} + 9\right)}^{2}}+C}\]
\end{problem}}%}

%%%%%%%%%%%%%%%%%%%%%%


\latexProblemContent{
\begin{problem}

Compute the indefinite integral:

\input{2311-Compute-Integral-0016.HELP.tex}

\[\int\;{\frac{\frac{1}{x} + 7}{x^{2}}}\;dx = \answer{{-\frac{1}{2} \, {\left(\frac{1}{x} + 7\right)}^{2}}+C}\]
\end{problem}}%}

%%%%%%%%%%%%%%%%%%%%%%


\latexProblemContent{
\begin{problem}

Compute the indefinite integral:

\input{2311-Compute-Integral-0016.HELP.tex}

\[\int\;{2 \, x - 6}\;dx = \answer{{x^{2} - 6 \, x}+C}\]
\end{problem}}%}

%%%%%%%%%%%%%%%%%%%%%%


\latexProblemContent{
\begin{problem}

Compute the indefinite integral:

\input{2311-Compute-Integral-0016.HELP.tex}

\[\int\;{-\frac{4 \, {\left(\log\left(x\right) + 6\right)}}{x}}\;dx = \answer{{-2 \, {\left(\log\left(x\right) + 6\right)}^{2}}+C}\]
\end{problem}}%}

%%%%%%%%%%%%%%%%%%%%%%

