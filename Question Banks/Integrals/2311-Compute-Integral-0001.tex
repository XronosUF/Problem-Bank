%%%%%%%%%%%%%%%%%%%%%%%
%%\tagged{Cat@One, Cat@Two, Cat@Three, Cat@Four, Cat@Five, Ans@ShortAns, Type@Compute, Topic@Integral, Sub@Poly, Sub@Riemann}{

\latexProblemContent{
\begin{problem}

Estimate the area under the graph of $f(x)={-x - 2}$ from $x={-8}$ to $x={-2}$ using ${6}$ rectangles and ${\text{right}}$ endpoints.

\input{2311-Compute-Integral-0001.HELP.tex}

\[\mbox{Area}\approx\answer{15}\]
\end{problem}}%}

%%%%%%%%%%%%%%%%%%%%%%





\latexProblemContent{
\begin{problem}

Estimate the area under the graph of $f(x)={-x^{2} + 9}$ from $x={-1}$ to $x={3}$ using ${4}$ rectangles and ${\text{left}}$ endpoints.

\input{2311-Compute-Integral-0001.HELP.tex}

\[\mbox{Area}\approx\answer{30}\]
\end{problem}}%}

%%%%%%%%%%%%%%%%%%%%%%





\latexProblemContent{
\begin{problem}

Estimate the area under the graph of $f(x)={x^{2} - 25}$ from $x={-8}$ to $x={-5}$ using ${3}$ rectangles and ${\text{left}}$ endpoints.

\input{2311-Compute-Integral-0001.HELP.tex}

\[\mbox{Area}\approx\answer{74}\]
\end{problem}}%}

%%%%%%%%%%%%%%%%%%%%%%





\latexProblemContent{
\begin{problem}

Estimate the area under the graph of $f(x)={x^{2} - 16}$ from $x={-7}$ to $x={-4}$ using ${3}$ rectangles and ${\text{right}}$ endpoints.

\input{2311-Compute-Integral-0001.HELP.tex}

\[\mbox{Area}\approx\answer{29}\]
\end{problem}}%}

%%%%%%%%%%%%%%%%%%%%%%





\latexProblemContent{
\begin{problem}

Estimate the area under the graph of $f(x)={-x^{2} + 196}$ from $x={8}$ to $x={14}$ using ${6}$ rectangles and ${\text{right}}$ endpoints.

\input{2311-Compute-Integral-0001.HELP.tex}

\[\mbox{Area}\approx\answer{365}\]
\end{problem}}%}

%%%%%%%%%%%%%%%%%%%%%%





\latexProblemContent{
\begin{problem}

Estimate the area under the graph of $f(x)={x + 1}$ from $x={-1}$ to $x={6}$ using ${7}$ rectangles and ${\text{left}}$ endpoints.

\input{2311-Compute-Integral-0001.HELP.tex}

\[\mbox{Area}\approx\answer{21}\]
\end{problem}}%}

%%%%%%%%%%%%%%%%%%%%%%





\latexProblemContent{
\begin{problem}

Estimate the area under the graph of $f(x)={x - 5}$ from $x={5}$ to $x={8}$ using ${3}$ rectangles and ${\text{right}}$ endpoints.

\input{2311-Compute-Integral-0001.HELP.tex}

\[\mbox{Area}\approx\answer{6}\]
\end{problem}}%}

%%%%%%%%%%%%%%%%%%%%%%





\latexProblemContent{
\begin{problem}

Estimate the area under the graph of $f(x)={-x}$ from $x={-6}$ to $x={0}$ using ${6}$ rectangles and ${\text{left}}$ endpoints.

\input{2311-Compute-Integral-0001.HELP.tex}

\[\mbox{Area}\approx\answer{21}\]
\end{problem}}%}

%%%%%%%%%%%%%%%%%%%%%%





\latexProblemContent{
\begin{problem}

Estimate the area under the graph of $f(x)={x + 4}$ from $x={-4}$ to $x={2}$ using ${6}$ rectangles and ${\text{left}}$ endpoints.

\input{2311-Compute-Integral-0001.HELP.tex}

\[\mbox{Area}\approx\answer{15}\]
\end{problem}}%}

%%%%%%%%%%%%%%%%%%%%%%





\latexProblemContent{
\begin{problem}

Estimate the area under the graph of $f(x)={-x^{2} + 144}$ from $x={5}$ to $x={12}$ using ${7}$ rectangles and ${\text{right}}$ endpoints.

\input{2311-Compute-Integral-0001.HELP.tex}

\[\mbox{Area}\approx\answer{413}\]
\end{problem}}%}

%%%%%%%%%%%%%%%%%%%%%%





\latexProblemContent{
\begin{problem}

Estimate the area under the graph of $f(x)={-x - 1}$ from $x={-5}$ to $x={-1}$ using ${4}$ rectangles and ${\text{left}}$ endpoints.

\input{2311-Compute-Integral-0001.HELP.tex}

\[\mbox{Area}\approx\answer{10}\]
\end{problem}}%}

%%%%%%%%%%%%%%%%%%%%%%





\latexProblemContent{
\begin{problem}

Estimate the area under the graph of $f(x)={x + 8}$ from $x={-8}$ to $x={-1}$ using ${7}$ rectangles and ${\text{left}}$ endpoints.

\input{2311-Compute-Integral-0001.HELP.tex}

\[\mbox{Area}\approx\answer{21}\]
\end{problem}}%}

%%%%%%%%%%%%%%%%%%%%%%





\latexProblemContent{
\begin{problem}

Estimate the area under the graph of $f(x)={x^{2} - 16}$ from $x={4}$ to $x={11}$ using ${7}$ rectangles and ${\text{right}}$ endpoints.

\input{2311-Compute-Integral-0001.HELP.tex}

\[\mbox{Area}\approx\answer{364}\]
\end{problem}}%}

%%%%%%%%%%%%%%%%%%%%%%





\latexProblemContent{
\begin{problem}

Estimate the area under the graph of $f(x)={-x^{2} + 64}$ from $x={1}$ to $x={8}$ using ${7}$ rectangles and ${\text{right}}$ endpoints.

\input{2311-Compute-Integral-0001.HELP.tex}

\[\mbox{Area}\approx\answer{245}\]
\end{problem}}%}

%%%%%%%%%%%%%%%%%%%%%%





\latexProblemContent{
\begin{problem}

Estimate the area under the graph of $f(x)={-x^{2} + 16}$ from $x={-4}$ to $x={-1}$ using ${3}$ rectangles and ${\text{left}}$ endpoints.

\input{2311-Compute-Integral-0001.HELP.tex}

\[\mbox{Area}\approx\answer{19}\]
\end{problem}}%}

%%%%%%%%%%%%%%%%%%%%%%





\latexProblemContent{
\begin{problem}

Estimate the area under the graph of $f(x)={x^{2} - 16}$ from $x={4}$ to $x={7}$ using ${3}$ rectangles and ${\text{left}}$ endpoints.

\input{2311-Compute-Integral-0001.HELP.tex}

\[\mbox{Area}\approx\answer{29}\]
\end{problem}}%}

%%%%%%%%%%%%%%%%%%%%%%





\latexProblemContent{
\begin{problem}

Estimate the area under the graph of $f(x)={-x^{2} + 36}$ from $x={-6}$ to $x={-3}$ using ${3}$ rectangles and ${\text{right}}$ endpoints.

\input{2311-Compute-Integral-0001.HELP.tex}

\[\mbox{Area}\approx\answer{58}\]
\end{problem}}%}

%%%%%%%%%%%%%%%%%%%%%%





\latexProblemContent{
\begin{problem}

Estimate the area under the graph of $f(x)={-x + 11}$ from $x={8}$ to $x={11}$ using ${3}$ rectangles and ${\text{right}}$ endpoints.

\input{2311-Compute-Integral-0001.HELP.tex}

\[\mbox{Area}\approx\answer{3}\]
\end{problem}}%}

%%%%%%%%%%%%%%%%%%%%%%





\latexProblemContent{
\begin{problem}

Estimate the area under the graph of $f(x)={x^{2} - 4}$ from $x={2}$ to $x={6}$ using ${4}$ rectangles and ${\text{right}}$ endpoints.

\input{2311-Compute-Integral-0001.HELP.tex}

\[\mbox{Area}\approx\answer{70}\]
\end{problem}}%}

%%%%%%%%%%%%%%%%%%%%%%





\latexProblemContent{
\begin{problem}

Estimate the area under the graph of $f(x)={x^{2} - 16}$ from $x={4}$ to $x={9}$ using ${5}$ rectangles and ${\text{right}}$ endpoints.

\input{2311-Compute-Integral-0001.HELP.tex}

\[\mbox{Area}\approx\answer{175}\]
\end{problem}}%}

%%%%%%%%%%%%%%%%%%%%%%





\latexProblemContent{
\begin{problem}

Estimate the area under the graph of $f(x)={-x + 8}$ from $x={3}$ to $x={8}$ using ${5}$ rectangles and ${\text{left}}$ endpoints.

\input{2311-Compute-Integral-0001.HELP.tex}

\[\mbox{Area}\approx\answer{15}\]
\end{problem}}%}

%%%%%%%%%%%%%%%%%%%%%%





\latexProblemContent{
\begin{problem}

Estimate the area under the graph of $f(x)={-x^{2} + 36}$ from $x={2}$ to $x={6}$ using ${4}$ rectangles and ${\text{right}}$ endpoints.

\input{2311-Compute-Integral-0001.HELP.tex}

\[\mbox{Area}\approx\answer{58}\]
\end{problem}}%}

%%%%%%%%%%%%%%%%%%%%%%





\latexProblemContent{
\begin{problem}

Estimate the area under the graph of $f(x)={-x^{2} + 64}$ from $x={-8}$ to $x={-2}$ using ${6}$ rectangles and ${\text{right}}$ endpoints.

\input{2311-Compute-Integral-0001.HELP.tex}

\[\mbox{Area}\approx\answer{245}\]
\end{problem}}%}

%%%%%%%%%%%%%%%%%%%%%%





\latexProblemContent{
\begin{problem}

Estimate the area under the graph of $f(x)={-x^{2} + 25}$ from $x={-5}$ to $x={-2}$ using ${3}$ rectangles and ${\text{right}}$ endpoints.

\input{2311-Compute-Integral-0001.HELP.tex}

\[\mbox{Area}\approx\answer{46}\]
\end{problem}}%}

%%%%%%%%%%%%%%%%%%%%%%





\latexProblemContent{
\begin{problem}

Estimate the area under the graph of $f(x)={-x + 9}$ from $x={6}$ to $x={9}$ using ${3}$ rectangles and ${\text{left}}$ endpoints.

\input{2311-Compute-Integral-0001.HELP.tex}

\[\mbox{Area}\approx\answer{6}\]
\end{problem}}%}

%%%%%%%%%%%%%%%%%%%%%%





\latexProblemContent{
\begin{problem}

Estimate the area under the graph of $f(x)={-x}$ from $x={-3}$ to $x={0}$ using ${3}$ rectangles and ${\text{left}}$ endpoints.

\input{2311-Compute-Integral-0001.HELP.tex}

\[\mbox{Area}\approx\answer{6}\]
\end{problem}}%}

%%%%%%%%%%%%%%%%%%%%%%





\latexProblemContent{
\begin{problem}

Estimate the area under the graph of $f(x)={x^{2} - 4}$ from $x={-5}$ to $x={-2}$ using ${3}$ rectangles and ${\text{right}}$ endpoints.

\input{2311-Compute-Integral-0001.HELP.tex}

\[\mbox{Area}\approx\answer{17}\]
\end{problem}}%}

%%%%%%%%%%%%%%%%%%%%%%





\latexProblemContent{
\begin{problem}

Estimate the area under the graph of $f(x)={-x + 1}$ from $x={-3}$ to $x={1}$ using ${4}$ rectangles and ${\text{left}}$ endpoints.

\input{2311-Compute-Integral-0001.HELP.tex}

\[\mbox{Area}\approx\answer{10}\]
\end{problem}}%}

%%%%%%%%%%%%%%%%%%%%%%





\latexProblemContent{
\begin{problem}

Estimate the area under the graph of $f(x)={-x + 9}$ from $x={5}$ to $x={9}$ using ${4}$ rectangles and ${\text{right}}$ endpoints.

\input{2311-Compute-Integral-0001.HELP.tex}

\[\mbox{Area}\approx\answer{6}\]
\end{problem}}%}

%%%%%%%%%%%%%%%%%%%%%%





\latexProblemContent{
\begin{problem}

Estimate the area under the graph of $f(x)={-x^{2} + 64}$ from $x={4}$ to $x={8}$ using ${4}$ rectangles and ${\text{right}}$ endpoints.

\input{2311-Compute-Integral-0001.HELP.tex}

\[\mbox{Area}\approx\answer{82}\]
\end{problem}}%}

%%%%%%%%%%%%%%%%%%%%%%





\latexProblemContent{
\begin{problem}

Estimate the area under the graph of $f(x)={x^{2} - 1}$ from $x={-5}$ to $x={-1}$ using ${4}$ rectangles and ${\text{left}}$ endpoints.

\input{2311-Compute-Integral-0001.HELP.tex}

\[\mbox{Area}\approx\answer{50}\]
\end{problem}}%}

%%%%%%%%%%%%%%%%%%%%%%





\latexProblemContent{
\begin{problem}

Estimate the area under the graph of $f(x)={x + 4}$ from $x={-4}$ to $x={-1}$ using ${3}$ rectangles and ${\text{left}}$ endpoints.

\input{2311-Compute-Integral-0001.HELP.tex}

\[\mbox{Area}\approx\answer{3}\]
\end{problem}}%}

%%%%%%%%%%%%%%%%%%%%%%





\latexProblemContent{
\begin{problem}

Estimate the area under the graph of $f(x)={-x^{2} + 25}$ from $x={-5}$ to $x={2}$ using ${7}$ rectangles and ${\text{left}}$ endpoints.

\input{2311-Compute-Integral-0001.HELP.tex}

\[\mbox{Area}\approx\answer{119}\]
\end{problem}}%}

%%%%%%%%%%%%%%%%%%%%%%





\latexProblemContent{
\begin{problem}

Estimate the area under the graph of $f(x)={x^{2} - 4}$ from $x={-2}$ to $x={5}$ using ${7}$ rectangles and ${\text{right}}$ endpoints.

\input{2311-Compute-Integral-0001.HELP.tex}

\[\mbox{Area}\approx\answer{28}\]
\end{problem}}%}

%%%%%%%%%%%%%%%%%%%%%%





\latexProblemContent{
\begin{problem}

Estimate the area under the graph of $f(x)={-x + 2}$ from $x={-5}$ to $x={2}$ using ${7}$ rectangles and ${\text{left}}$ endpoints.

\input{2311-Compute-Integral-0001.HELP.tex}

\[\mbox{Area}\approx\answer{28}\]
\end{problem}}%}

%%%%%%%%%%%%%%%%%%%%%%





\latexProblemContent{
\begin{problem}

Estimate the area under the graph of $f(x)={x^{2} - 4}$ from $x={-3}$ to $x={2}$ using ${5}$ rectangles and ${\text{left}}$ endpoints.

\input{2311-Compute-Integral-0001.HELP.tex}

\[\mbox{Area}\approx\answer{-5}\]
\end{problem}}%}

%%%%%%%%%%%%%%%%%%%%%%





\latexProblemContent{
\begin{problem}

Estimate the area under the graph of $f(x)={-x^{2} + 81}$ from $x={3}$ to $x={9}$ using ${6}$ rectangles and ${\text{left}}$ endpoints.

\input{2311-Compute-Integral-0001.HELP.tex}

\[\mbox{Area}\approx\answer{287}\]
\end{problem}}%}

%%%%%%%%%%%%%%%%%%%%%%





\latexProblemContent{
\begin{problem}

Estimate the area under the graph of $f(x)={x - 3}$ from $x={3}$ to $x={9}$ using ${6}$ rectangles and ${\text{right}}$ endpoints.

\input{2311-Compute-Integral-0001.HELP.tex}

\[\mbox{Area}\approx\answer{21}\]
\end{problem}}%}

%%%%%%%%%%%%%%%%%%%%%%





\latexProblemContent{
\begin{problem}

Estimate the area under the graph of $f(x)={x + 2}$ from $x={-2}$ to $x={5}$ using ${7}$ rectangles and ${\text{left}}$ endpoints.

\input{2311-Compute-Integral-0001.HELP.tex}

\[\mbox{Area}\approx\answer{21}\]
\end{problem}}%}

%%%%%%%%%%%%%%%%%%%%%%





\latexProblemContent{
\begin{problem}

Estimate the area under the graph of $f(x)={x + 2}$ from $x={-2}$ to $x={3}$ using ${5}$ rectangles and ${\text{right}}$ endpoints.

\input{2311-Compute-Integral-0001.HELP.tex}

\[\mbox{Area}\approx\answer{15}\]
\end{problem}}%}

%%%%%%%%%%%%%%%%%%%%%%





\latexProblemContent{
\begin{problem}

Estimate the area under the graph of $f(x)={-x - 5}$ from $x={-8}$ to $x={-5}$ using ${3}$ rectangles and ${\text{right}}$ endpoints.

\input{2311-Compute-Integral-0001.HELP.tex}

\[\mbox{Area}\approx\answer{3}\]
\end{problem}}%}

%%%%%%%%%%%%%%%%%%%%%%





\latexProblemContent{
\begin{problem}

Estimate the area under the graph of $f(x)={x + 5}$ from $x={-5}$ to $x={-1}$ using ${4}$ rectangles and ${\text{right}}$ endpoints.

\input{2311-Compute-Integral-0001.HELP.tex}

\[\mbox{Area}\approx\answer{10}\]
\end{problem}}%}

%%%%%%%%%%%%%%%%%%%%%%





\latexProblemContent{
\begin{problem}

Estimate the area under the graph of $f(x)={-x + 2}$ from $x={-4}$ to $x={2}$ using ${6}$ rectangles and ${\text{right}}$ endpoints.

\input{2311-Compute-Integral-0001.HELP.tex}

\[\mbox{Area}\approx\answer{15}\]
\end{problem}}%}

%%%%%%%%%%%%%%%%%%%%%%





\latexProblemContent{
\begin{problem}

Estimate the area under the graph of $f(x)={x - 6}$ from $x={6}$ to $x={13}$ using ${7}$ rectangles and ${\text{right}}$ endpoints.

\input{2311-Compute-Integral-0001.HELP.tex}

\[\mbox{Area}\approx\answer{28}\]
\end{problem}}%}

%%%%%%%%%%%%%%%%%%%%%%





\latexProblemContent{
\begin{problem}

Estimate the area under the graph of $f(x)={-x^{2} + 49}$ from $x={-7}$ to $x={-4}$ using ${3}$ rectangles and ${\text{left}}$ endpoints.

\input{2311-Compute-Integral-0001.HELP.tex}

\[\mbox{Area}\approx\answer{37}\]
\end{problem}}%}

%%%%%%%%%%%%%%%%%%%%%%





\latexProblemContent{
\begin{problem}

Estimate the area under the graph of $f(x)={x + 2}$ from $x={-2}$ to $x={5}$ using ${7}$ rectangles and ${\text{right}}$ endpoints.

\input{2311-Compute-Integral-0001.HELP.tex}

\[\mbox{Area}\approx\answer{28}\]
\end{problem}}%}

%%%%%%%%%%%%%%%%%%%%%%





\latexProblemContent{
\begin{problem}

Estimate the area under the graph of $f(x)={x + 8}$ from $x={-8}$ to $x={-4}$ using ${4}$ rectangles and ${\text{right}}$ endpoints.

\input{2311-Compute-Integral-0001.HELP.tex}

\[\mbox{Area}\approx\answer{10}\]
\end{problem}}%}

%%%%%%%%%%%%%%%%%%%%%%





\latexProblemContent{
\begin{problem}

Estimate the area under the graph of $f(x)={-x^{2} + 49}$ from $x={-7}$ to $x={-1}$ using ${6}$ rectangles and ${\text{right}}$ endpoints.

\input{2311-Compute-Integral-0001.HELP.tex}

\[\mbox{Area}\approx\answer{203}\]
\end{problem}}%}

%%%%%%%%%%%%%%%%%%%%%%





\latexProblemContent{
\begin{problem}

Estimate the area under the graph of $f(x)={x^{2} - 16}$ from $x={4}$ to $x={11}$ using ${7}$ rectangles and ${\text{left}}$ endpoints.

\input{2311-Compute-Integral-0001.HELP.tex}

\[\mbox{Area}\approx\answer{259}\]
\end{problem}}%}

%%%%%%%%%%%%%%%%%%%%%%





\latexProblemContent{
\begin{problem}

Estimate the area under the graph of $f(x)={x^{2} - 1}$ from $x={-8}$ to $x={-1}$ using ${7}$ rectangles and ${\text{right}}$ endpoints.

\input{2311-Compute-Integral-0001.HELP.tex}

\[\mbox{Area}\approx\answer{133}\]
\end{problem}}%}

%%%%%%%%%%%%%%%%%%%%%%





\latexProblemContent{
\begin{problem}

Estimate the area under the graph of $f(x)={-x + 8}$ from $x={2}$ to $x={8}$ using ${6}$ rectangles and ${\text{left}}$ endpoints.

\input{2311-Compute-Integral-0001.HELP.tex}

\[\mbox{Area}\approx\answer{21}\]
\end{problem}}%}

%%%%%%%%%%%%%%%%%%%%%%





\latexProblemContent{
\begin{problem}

Estimate the area under the graph of $f(x)={-x^{2} + 16}$ from $x={-2}$ to $x={4}$ using ${6}$ rectangles and ${\text{left}}$ endpoints.

\input{2311-Compute-Integral-0001.HELP.tex}

\[\mbox{Area}\approx\answer{77}\]
\end{problem}}%}

%%%%%%%%%%%%%%%%%%%%%%





\latexProblemContent{
\begin{problem}

Estimate the area under the graph of $f(x)={-x + 11}$ from $x={8}$ to $x={11}$ using ${3}$ rectangles and ${\text{left}}$ endpoints.

\input{2311-Compute-Integral-0001.HELP.tex}

\[\mbox{Area}\approx\answer{6}\]
\end{problem}}%}

%%%%%%%%%%%%%%%%%%%%%%





\latexProblemContent{
\begin{problem}

Estimate the area under the graph of $f(x)={-x^{2} + 9}$ from $x={-2}$ to $x={3}$ using ${5}$ rectangles and ${\text{left}}$ endpoints.

\input{2311-Compute-Integral-0001.HELP.tex}

\[\mbox{Area}\approx\answer{35}\]
\end{problem}}%}

%%%%%%%%%%%%%%%%%%%%%%





\latexProblemContent{
\begin{problem}

Estimate the area under the graph of $f(x)={-x + 11}$ from $x={5}$ to $x={11}$ using ${6}$ rectangles and ${\text{left}}$ endpoints.

\input{2311-Compute-Integral-0001.HELP.tex}

\[\mbox{Area}\approx\answer{21}\]
\end{problem}}%}

%%%%%%%%%%%%%%%%%%%%%%





\latexProblemContent{
\begin{problem}

Estimate the area under the graph of $f(x)={x^{2} - 25}$ from $x={5}$ to $x={9}$ using ${4}$ rectangles and ${\text{left}}$ endpoints.

\input{2311-Compute-Integral-0001.HELP.tex}

\[\mbox{Area}\approx\answer{74}\]
\end{problem}}%}

%%%%%%%%%%%%%%%%%%%%%%





\latexProblemContent{
\begin{problem}

Estimate the area under the graph of $f(x)={x^{2} - 36}$ from $x={6}$ to $x={11}$ using ${5}$ rectangles and ${\text{right}}$ endpoints.

\input{2311-Compute-Integral-0001.HELP.tex}

\[\mbox{Area}\approx\answer{235}\]
\end{problem}}%}

%%%%%%%%%%%%%%%%%%%%%%





\latexProblemContent{
\begin{problem}

Estimate the area under the graph of $f(x)={x^{2} - 4}$ from $x={-8}$ to $x={-2}$ using ${6}$ rectangles and ${\text{right}}$ endpoints.

\input{2311-Compute-Integral-0001.HELP.tex}

\[\mbox{Area}\approx\answer{115}\]
\end{problem}}%}

%%%%%%%%%%%%%%%%%%%%%%





\latexProblemContent{
\begin{problem}

Estimate the area under the graph of $f(x)={-x^{2} + 64}$ from $x={-8}$ to $x={-5}$ using ${3}$ rectangles and ${\text{left}}$ endpoints.

\input{2311-Compute-Integral-0001.HELP.tex}

\[\mbox{Area}\approx\answer{43}\]
\end{problem}}%}

%%%%%%%%%%%%%%%%%%%%%%





\latexProblemContent{
\begin{problem}

Estimate the area under the graph of $f(x)={-x^{2} + 64}$ from $x={5}$ to $x={8}$ using ${3}$ rectangles and ${\text{left}}$ endpoints.

\input{2311-Compute-Integral-0001.HELP.tex}

\[\mbox{Area}\approx\answer{82}\]
\end{problem}}%}

%%%%%%%%%%%%%%%%%%%%%%





\latexProblemContent{
\begin{problem}

Estimate the area under the graph of $f(x)={-x^{2} + 64}$ from $x={-8}$ to $x={-4}$ using ${4}$ rectangles and ${\text{left}}$ endpoints.

\input{2311-Compute-Integral-0001.HELP.tex}

\[\mbox{Area}\approx\answer{82}\]
\end{problem}}%}

%%%%%%%%%%%%%%%%%%%%%%





%%%%%%%%%%%%%%%%%%%%%%





\latexProblemContent{
\begin{problem}

Estimate the area under the graph of $f(x)={x + 5}$ from $x={-5}$ to $x={2}$ using ${7}$ rectangles and ${\text{left}}$ endpoints.

\input{2311-Compute-Integral-0001.HELP.tex}

\[\mbox{Area}\approx\answer{21}\]
\end{problem}}%}

%%%%%%%%%%%%%%%%%%%%%%





\latexProblemContent{
\begin{problem}

Estimate the area under the graph of $f(x)={-x + 8}$ from $x={4}$ to $x={8}$ using ${4}$ rectangles and ${\text{right}}$ endpoints.

\input{2311-Compute-Integral-0001.HELP.tex}

\[\mbox{Area}\approx\answer{6}\]
\end{problem}}%}

%%%%%%%%%%%%%%%%%%%%%%





\latexProblemContent{
\begin{problem}

Estimate the area under the graph of $f(x)={x^{2} - 1}$ from $x={-1}$ to $x={5}$ using ${6}$ rectangles and ${\text{right}}$ endpoints.

\input{2311-Compute-Integral-0001.HELP.tex}

\[\mbox{Area}\approx\answer{49}\]
\end{problem}}%}

%%%%%%%%%%%%%%%%%%%%%%





\latexProblemContent{
\begin{problem}

Estimate the area under the graph of $f(x)={x^{2} - 4}$ from $x={-5}$ to $x={2}$ using ${7}$ rectangles and ${\text{left}}$ endpoints.

\input{2311-Compute-Integral-0001.HELP.tex}

\[\mbox{Area}\approx\answer{28}\]
\end{problem}}%}

%%%%%%%%%%%%%%%%%%%%%%





\latexProblemContent{
\begin{problem}

Estimate the area under the graph of $f(x)={x^{2} - 9}$ from $x={3}$ to $x={6}$ using ${3}$ rectangles and ${\text{right}}$ endpoints.

\input{2311-Compute-Integral-0001.HELP.tex}

\[\mbox{Area}\approx\answer{50}\]
\end{problem}}%}

%%%%%%%%%%%%%%%%%%%%%%





\latexProblemContent{
\begin{problem}

Estimate the area under the graph of $f(x)={-x^{2} + 25}$ from $x={-5}$ to $x={0}$ using ${5}$ rectangles and ${\text{right}}$ endpoints.

\input{2311-Compute-Integral-0001.HELP.tex}

\[\mbox{Area}\approx\answer{95}\]
\end{problem}}%}

%%%%%%%%%%%%%%%%%%%%%%





\latexProblemContent{
\begin{problem}

Estimate the area under the graph of $f(x)={x^{2} - 1}$ from $x={-4}$ to $x={-1}$ using ${3}$ rectangles and ${\text{right}}$ endpoints.

\input{2311-Compute-Integral-0001.HELP.tex}

\[\mbox{Area}\approx\answer{11}\]
\end{problem}}%}

%%%%%%%%%%%%%%%%%%%%%%





\latexProblemContent{
\begin{problem}

Estimate the area under the graph of $f(x)={-x^{2} + 121}$ from $x={8}$ to $x={11}$ using ${3}$ rectangles and ${\text{left}}$ endpoints.

\input{2311-Compute-Integral-0001.HELP.tex}

\[\mbox{Area}\approx\answer{118}\]
\end{problem}}%}

%%%%%%%%%%%%%%%%%%%%%%





\latexProblemContent{
\begin{problem}

Estimate the area under the graph of $f(x)={-x^{2} + 169}$ from $x={6}$ to $x={13}$ using ${7}$ rectangles and ${\text{right}}$ endpoints.

\input{2311-Compute-Integral-0001.HELP.tex}

\[\mbox{Area}\approx\answer{455}\]
\end{problem}}%}

%%%%%%%%%%%%%%%%%%%%%%





\latexProblemContent{
\begin{problem}

Estimate the area under the graph of $f(x)={x^{2} - 4}$ from $x={-2}$ to $x={3}$ using ${5}$ rectangles and ${\text{right}}$ endpoints.

\input{2311-Compute-Integral-0001.HELP.tex}

\[\mbox{Area}\approx\answer{-5}\]
\end{problem}}%}

%%%%%%%%%%%%%%%%%%%%%%





\latexProblemContent{
\begin{problem}

Estimate the area under the graph of $f(x)={-x^{2} + 169}$ from $x={7}$ to $x={13}$ using ${6}$ rectangles and ${\text{right}}$ endpoints.

\input{2311-Compute-Integral-0001.HELP.tex}

\[\mbox{Area}\approx\answer{335}\]
\end{problem}}%}

%%%%%%%%%%%%%%%%%%%%%%





\latexProblemContent{
\begin{problem}

Estimate the area under the graph of $f(x)={-x^{2} + 49}$ from $x={-7}$ to $x={0}$ using ${7}$ rectangles and ${\text{right}}$ endpoints.

\input{2311-Compute-Integral-0001.HELP.tex}

\[\mbox{Area}\approx\answer{252}\]
\end{problem}}%}

%%%%%%%%%%%%%%%%%%%%%%





\latexProblemContent{
\begin{problem}

Estimate the area under the graph of $f(x)={x + 6}$ from $x={-6}$ to $x={-1}$ using ${5}$ rectangles and ${\text{left}}$ endpoints.

\input{2311-Compute-Integral-0001.HELP.tex}

\[\mbox{Area}\approx\answer{10}\]
\end{problem}}%}

%%%%%%%%%%%%%%%%%%%%%%





%%%%%%%%%%%%%%%%%%%%%%





\latexProblemContent{
\begin{problem}

Estimate the area under the graph of $f(x)={-x^{2} + 64}$ from $x={-8}$ to $x={-1}$ using ${7}$ rectangles and ${\text{left}}$ endpoints.

\input{2311-Compute-Integral-0001.HELP.tex}

\[\mbox{Area}\approx\answer{245}\]
\end{problem}}%}

%%%%%%%%%%%%%%%%%%%%%%





%%%%%%%%%%%%%%%%%%%%%%





\latexProblemContent{
\begin{problem}

Estimate the area under the graph of $f(x)={-x^{2} + 49}$ from $x={1}$ to $x={7}$ using ${6}$ rectangles and ${\text{left}}$ endpoints.

\input{2311-Compute-Integral-0001.HELP.tex}

\[\mbox{Area}\approx\answer{203}\]
\end{problem}}%}

%%%%%%%%%%%%%%%%%%%%%%





\latexProblemContent{
\begin{problem}

Estimate the area under the graph of $f(x)={-x + 5}$ from $x={1}$ to $x={5}$ using ${4}$ rectangles and ${\text{right}}$ endpoints.

\input{2311-Compute-Integral-0001.HELP.tex}

\[\mbox{Area}\approx\answer{6}\]
\end{problem}}%}

%%%%%%%%%%%%%%%%%%%%%%





\latexProblemContent{
\begin{problem}

Estimate the area under the graph of $f(x)={-x^{2} + 9}$ from $x={-3}$ to $x={3}$ using ${6}$ rectangles and ${\text{left}}$ endpoints.

\input{2311-Compute-Integral-0001.HELP.tex}

\[\mbox{Area}\approx\answer{35}\]
\end{problem}}%}

%%%%%%%%%%%%%%%%%%%%%%





\latexProblemContent{
\begin{problem}

Estimate the area under the graph of $f(x)={-x - 4}$ from $x={-7}$ to $x={-4}$ using ${3}$ rectangles and ${\text{left}}$ endpoints.

\input{2311-Compute-Integral-0001.HELP.tex}

\[\mbox{Area}\approx\answer{6}\]
\end{problem}}%}

%%%%%%%%%%%%%%%%%%%%%%





\latexProblemContent{
\begin{problem}

Estimate the area under the graph of $f(x)={-x^{2} + 49}$ from $x={-7}$ to $x={-2}$ using ${5}$ rectangles and ${\text{right}}$ endpoints.

\input{2311-Compute-Integral-0001.HELP.tex}

\[\mbox{Area}\approx\answer{155}\]
\end{problem}}%}

%%%%%%%%%%%%%%%%%%%%%%





\latexProblemContent{
\begin{problem}

Estimate the area under the graph of $f(x)={x - 7}$ from $x={7}$ to $x={14}$ using ${7}$ rectangles and ${\text{right}}$ endpoints.

\input{2311-Compute-Integral-0001.HELP.tex}

\[\mbox{Area}\approx\answer{28}\]
\end{problem}}%}

%%%%%%%%%%%%%%%%%%%%%%





\latexProblemContent{
\begin{problem}

Estimate the area under the graph of $f(x)={-x + 13}$ from $x={7}$ to $x={13}$ using ${6}$ rectangles and ${\text{right}}$ endpoints.

\input{2311-Compute-Integral-0001.HELP.tex}

\[\mbox{Area}\approx\answer{15}\]
\end{problem}}%}

%%%%%%%%%%%%%%%%%%%%%%





\latexProblemContent{
\begin{problem}

Estimate the area under the graph of $f(x)={x + 7}$ from $x={-7}$ to $x={-4}$ using ${3}$ rectangles and ${\text{right}}$ endpoints.

\input{2311-Compute-Integral-0001.HELP.tex}

\[\mbox{Area}\approx\answer{6}\]
\end{problem}}%}

%%%%%%%%%%%%%%%%%%%%%%





%%%%%%%%%%%%%%%%%%%%%%





\latexProblemContent{
\begin{problem}

Estimate the area under the graph of $f(x)={x^{2} - 4}$ from $x={2}$ to $x={9}$ using ${7}$ rectangles and ${\text{left}}$ endpoints.

\input{2311-Compute-Integral-0001.HELP.tex}

\[\mbox{Area}\approx\answer{175}\]
\end{problem}}%}

%%%%%%%%%%%%%%%%%%%%%%





%%%%%%%%%%%%%%%%%%%%%%





\latexProblemContent{
\begin{problem}

Estimate the area under the graph of $f(x)={x - 8}$ from $x={8}$ to $x={13}$ using ${5}$ rectangles and ${\text{right}}$ endpoints.

\input{2311-Compute-Integral-0001.HELP.tex}

\[\mbox{Area}\approx\answer{15}\]
\end{problem}}%}

%%%%%%%%%%%%%%%%%%%%%%





\latexProblemContent{
\begin{problem}

Estimate the area under the graph of $f(x)={-x + 9}$ from $x={3}$ to $x={9}$ using ${6}$ rectangles and ${\text{left}}$ endpoints.

\input{2311-Compute-Integral-0001.HELP.tex}

\[\mbox{Area}\approx\answer{21}\]
\end{problem}}%}

%%%%%%%%%%%%%%%%%%%%%%





\latexProblemContent{
\begin{problem}

Estimate the area under the graph of $f(x)={-x^{2} + 100}$ from $x={3}$ to $x={10}$ using ${7}$ rectangles and ${\text{right}}$ endpoints.

\input{2311-Compute-Integral-0001.HELP.tex}

\[\mbox{Area}\approx\answer{329}\]
\end{problem}}%}

%%%%%%%%%%%%%%%%%%%%%%





\latexProblemContent{
\begin{problem}

Estimate the area under the graph of $f(x)={x^{2} - 1}$ from $x={-5}$ to $x={1}$ using ${6}$ rectangles and ${\text{left}}$ endpoints.

\input{2311-Compute-Integral-0001.HELP.tex}

\[\mbox{Area}\approx\answer{49}\]
\end{problem}}%}

%%%%%%%%%%%%%%%%%%%%%%





\latexProblemContent{
\begin{problem}

Estimate the area under the graph of $f(x)={-x^{2} + 25}$ from $x={-5}$ to $x={2}$ using ${7}$ rectangles and ${\text{right}}$ endpoints.

\input{2311-Compute-Integral-0001.HELP.tex}

\[\mbox{Area}\approx\answer{140}\]
\end{problem}}%}

%%%%%%%%%%%%%%%%%%%%%%





\latexProblemContent{
\begin{problem}

Estimate the area under the graph of $f(x)={-x^{2} + 64}$ from $x={-8}$ to $x={-3}$ using ${5}$ rectangles and ${\text{right}}$ endpoints.

\input{2311-Compute-Integral-0001.HELP.tex}

\[\mbox{Area}\approx\answer{185}\]
\end{problem}}%}

%%%%%%%%%%%%%%%%%%%%%%





%%%%%%%%%%%%%%%%%%%%%%





\latexProblemContent{
\begin{problem}

Estimate the area under the graph of $f(x)={-x^{2} + 64}$ from $x={1}$ to $x={8}$ using ${7}$ rectangles and ${\text{left}}$ endpoints.

\input{2311-Compute-Integral-0001.HELP.tex}

\[\mbox{Area}\approx\answer{308}\]
\end{problem}}%}

%%%%%%%%%%%%%%%%%%%%%%





\latexProblemContent{
\begin{problem}

Estimate the area under the graph of $f(x)={-x + 10}$ from $x={5}$ to $x={10}$ using ${5}$ rectangles and ${\text{right}}$ endpoints.

\input{2311-Compute-Integral-0001.HELP.tex}

\[\mbox{Area}\approx\answer{10}\]
\end{problem}}%}

%%%%%%%%%%%%%%%%%%%%%%





\latexProblemContent{
\begin{problem}

Estimate the area under the graph of $f(x)={-x - 1}$ from $x={-6}$ to $x={-1}$ using ${5}$ rectangles and ${\text{right}}$ endpoints.

\input{2311-Compute-Integral-0001.HELP.tex}

\[\mbox{Area}\approx\answer{10}\]
\end{problem}}%}

%%%%%%%%%%%%%%%%%%%%%%





%%%%%%%%%%%%%%%%%%%%%%





\latexProblemContent{
\begin{problem}

Estimate the area under the graph of $f(x)={-x + 10}$ from $x={6}$ to $x={10}$ using ${4}$ rectangles and ${\text{left}}$ endpoints.

\input{2311-Compute-Integral-0001.HELP.tex}

\[\mbox{Area}\approx\answer{10}\]
\end{problem}}%}

%%%%%%%%%%%%%%%%%%%%%%





\latexProblemContent{
\begin{problem}

Estimate the area under the graph of $f(x)={-x + 1}$ from $x={-4}$ to $x={1}$ using ${5}$ rectangles and ${\text{left}}$ endpoints.

\input{2311-Compute-Integral-0001.HELP.tex}

\[\mbox{Area}\approx\answer{15}\]
\end{problem}}%}

%%%%%%%%%%%%%%%%%%%%%%





\latexProblemContent{
\begin{problem}

Estimate the area under the graph of $f(x)={-x + 1}$ from $x={-6}$ to $x={1}$ using ${7}$ rectangles and ${\text{left}}$ endpoints.

\input{2311-Compute-Integral-0001.HELP.tex}

\[\mbox{Area}\approx\answer{28}\]
\end{problem}}%}

%%%%%%%%%%%%%%%%%%%%%%





\latexProblemContent{
\begin{problem}

Estimate the area under the graph of $f(x)={-x}$ from $x={-3}$ to $x={0}$ using ${3}$ rectangles and ${\text{right}}$ endpoints.

\input{2311-Compute-Integral-0001.HELP.tex}

\[\mbox{Area}\approx\answer{3}\]
\end{problem}}%}

%%%%%%%%%%%%%%%%%%%%%%





\latexProblemContent{
\begin{problem}

Estimate the area under the graph of $f(x)={x^{2} - 25}$ from $x={5}$ to $x={11}$ using ${6}$ rectangles and ${\text{left}}$ endpoints.

\input{2311-Compute-Integral-0001.HELP.tex}

\[\mbox{Area}\approx\answer{205}\]
\end{problem}}%}

%%%%%%%%%%%%%%%%%%%%%%





%%%%%%%%%%%%%%%%%%%%%%





\latexProblemContent{
\begin{problem}

Estimate the area under the graph of $f(x)={-x^{2} + 16}$ from $x={-1}$ to $x={4}$ using ${5}$ rectangles and ${\text{left}}$ endpoints.

\input{2311-Compute-Integral-0001.HELP.tex}

\[\mbox{Area}\approx\answer{65}\]
\end{problem}}%}

%%%%%%%%%%%%%%%%%%%%%%





\latexProblemContent{
\begin{problem}

Estimate the area under the graph of $f(x)={-x^{2} + 16}$ from $x={-3}$ to $x={4}$ using ${7}$ rectangles and ${\text{left}}$ endpoints.

\input{2311-Compute-Integral-0001.HELP.tex}

\[\mbox{Area}\approx\answer{84}\]
\end{problem}}%}

%%%%%%%%%%%%%%%%%%%%%%





\latexProblemContent{
\begin{problem}

Estimate the area under the graph of $f(x)={x^{2} - 9}$ from $x={-7}$ to $x={-3}$ using ${4}$ rectangles and ${\text{left}}$ endpoints.

\input{2311-Compute-Integral-0001.HELP.tex}

\[\mbox{Area}\approx\answer{90}\]
\end{problem}}%}

%%%%%%%%%%%%%%%%%%%%%%





\latexProblemContent{
\begin{problem}

Estimate the area under the graph of $f(x)={x^{2} - 9}$ from $x={3}$ to $x={10}$ using ${7}$ rectangles and ${\text{right}}$ endpoints.

\input{2311-Compute-Integral-0001.HELP.tex}

\[\mbox{Area}\approx\answer{308}\]
\end{problem}}%}

%%%%%%%%%%%%%%%%%%%%%%





\latexProblemContent{
\begin{problem}

Estimate the area under the graph of $f(x)={x + 3}$ from $x={-3}$ to $x={4}$ using ${7}$ rectangles and ${\text{left}}$ endpoints.

\input{2311-Compute-Integral-0001.HELP.tex}

\[\mbox{Area}\approx\answer{21}\]
\end{problem}}%}

%%%%%%%%%%%%%%%%%%%%%%





%%%%%%%%%%%%%%%%%%%%%%





\latexProblemContent{
\begin{problem}

Estimate the area under the graph of $f(x)={-x - 1}$ from $x={-7}$ to $x={-1}$ using ${6}$ rectangles and ${\text{right}}$ endpoints.

\input{2311-Compute-Integral-0001.HELP.tex}

\[\mbox{Area}\approx\answer{15}\]
\end{problem}}%}

%%%%%%%%%%%%%%%%%%%%%%





\latexProblemContent{
\begin{problem}

Estimate the area under the graph of $f(x)={-x^{2} + 100}$ from $x={6}$ to $x={10}$ using ${4}$ rectangles and ${\text{left}}$ endpoints.

\input{2311-Compute-Integral-0001.HELP.tex}

\[\mbox{Area}\approx\answer{170}\]
\end{problem}}%}

%%%%%%%%%%%%%%%%%%%%%%





\latexProblemContent{
\begin{problem}

Estimate the area under the graph of $f(x)={-x + 13}$ from $x={6}$ to $x={13}$ using ${7}$ rectangles and ${\text{right}}$ endpoints.

\input{2311-Compute-Integral-0001.HELP.tex}

\[\mbox{Area}\approx\answer{21}\]
\end{problem}}%}

%%%%%%%%%%%%%%%%%%%%%%





%%%%%%%%%%%%%%%%%%%%%%





\latexProblemContent{
\begin{problem}

Estimate the area under the graph of $f(x)={x^{2} - 64}$ from $x={8}$ to $x={14}$ using ${6}$ rectangles and ${\text{left}}$ endpoints.

\input{2311-Compute-Integral-0001.HELP.tex}

\[\mbox{Area}\approx\answer{295}\]
\end{problem}}%}

%%%%%%%%%%%%%%%%%%%%%%





\latexProblemContent{
\begin{problem}

Estimate the area under the graph of $f(x)={x^{2} - 1}$ from $x={-5}$ to $x={1}$ using ${6}$ rectangles and ${\text{right}}$ endpoints.

\input{2311-Compute-Integral-0001.HELP.tex}

\[\mbox{Area}\approx\answer{25}\]
\end{problem}}%}

%%%%%%%%%%%%%%%%%%%%%%





\latexProblemContent{
\begin{problem}

Estimate the area under the graph of $f(x)={-x - 3}$ from $x={-6}$ to $x={-3}$ using ${3}$ rectangles and ${\text{left}}$ endpoints.

\input{2311-Compute-Integral-0001.HELP.tex}

\[\mbox{Area}\approx\answer{6}\]
\end{problem}}%}

%%%%%%%%%%%%%%%%%%%%%%





\latexProblemContent{
\begin{problem}

Estimate the area under the graph of $f(x)={x + 3}$ from $x={-3}$ to $x={3}$ using ${6}$ rectangles and ${\text{right}}$ endpoints.

\input{2311-Compute-Integral-0001.HELP.tex}

\[\mbox{Area}\approx\answer{21}\]
\end{problem}}%}

%%%%%%%%%%%%%%%%%%%%%%





\latexProblemContent{
\begin{problem}

Estimate the area under the graph of $f(x)={x + 7}$ from $x={-7}$ to $x={-1}$ using ${6}$ rectangles and ${\text{left}}$ endpoints.

\input{2311-Compute-Integral-0001.HELP.tex}

\[\mbox{Area}\approx\answer{15}\]
\end{problem}}%}

%%%%%%%%%%%%%%%%%%%%%%





%%%%%%%%%%%%%%%%%%%%%%





\latexProblemContent{
\begin{problem}

Estimate the area under the graph of $f(x)={x^{2} - 36}$ from $x={6}$ to $x={10}$ using ${4}$ rectangles and ${\text{right}}$ endpoints.

\input{2311-Compute-Integral-0001.HELP.tex}

\[\mbox{Area}\approx\answer{150}\]
\end{problem}}%}

%%%%%%%%%%%%%%%%%%%%%%





\latexProblemContent{
\begin{problem}

Estimate the area under the graph of $f(x)={-x^{2} + 36}$ from $x={-6}$ to $x={-2}$ using ${4}$ rectangles and ${\text{left}}$ endpoints.

\input{2311-Compute-Integral-0001.HELP.tex}

\[\mbox{Area}\approx\answer{58}\]
\end{problem}}%}

%%%%%%%%%%%%%%%%%%%%%%





\latexProblemContent{
\begin{problem}

Estimate the area under the graph of $f(x)={-x^{2} + 25}$ from $x={-5}$ to $x={-1}$ using ${4}$ rectangles and ${\text{left}}$ endpoints.

\input{2311-Compute-Integral-0001.HELP.tex}

\[\mbox{Area}\approx\answer{46}\]
\end{problem}}%}

%%%%%%%%%%%%%%%%%%%%%%





\latexProblemContent{
\begin{problem}

Estimate the area under the graph of $f(x)={x^{2} - 1}$ from $x={1}$ to $x={6}$ using ${5}$ rectangles and ${\text{left}}$ endpoints.

\input{2311-Compute-Integral-0001.HELP.tex}

\[\mbox{Area}\approx\answer{50}\]
\end{problem}}%}

%%%%%%%%%%%%%%%%%%%%%%





\latexProblemContent{
\begin{problem}

Estimate the area under the graph of $f(x)={-x^{2} + 81}$ from $x={5}$ to $x={9}$ using ${4}$ rectangles and ${\text{right}}$ endpoints.

\input{2311-Compute-Integral-0001.HELP.tex}

\[\mbox{Area}\approx\answer{94}\]
\end{problem}}%}

%%%%%%%%%%%%%%%%%%%%%%





\latexProblemContent{
\begin{problem}

Estimate the area under the graph of $f(x)={-x - 4}$ from $x={-8}$ to $x={-4}$ using ${4}$ rectangles and ${\text{right}}$ endpoints.

\input{2311-Compute-Integral-0001.HELP.tex}

\[\mbox{Area}\approx\answer{6}\]
\end{problem}}%}

%%%%%%%%%%%%%%%%%%%%%%





\latexProblemContent{
\begin{problem}

Estimate the area under the graph of $f(x)={-x + 4}$ from $x={1}$ to $x={4}$ using ${3}$ rectangles and ${\text{left}}$ endpoints.

\input{2311-Compute-Integral-0001.HELP.tex}

\[\mbox{Area}\approx\answer{6}\]
\end{problem}}%}

%%%%%%%%%%%%%%%%%%%%%%





\latexProblemContent{
\begin{problem}

Estimate the area under the graph of $f(x)={x + 3}$ from $x={-3}$ to $x={2}$ using ${5}$ rectangles and ${\text{right}}$ endpoints.

\input{2311-Compute-Integral-0001.HELP.tex}

\[\mbox{Area}\approx\answer{15}\]
\end{problem}}%}

%%%%%%%%%%%%%%%%%%%%%%





\latexProblemContent{
\begin{problem}

Estimate the area under the graph of $f(x)={x - 5}$ from $x={5}$ to $x={9}$ using ${4}$ rectangles and ${\text{right}}$ endpoints.

\input{2311-Compute-Integral-0001.HELP.tex}

\[\mbox{Area}\approx\answer{10}\]
\end{problem}}%}

%%%%%%%%%%%%%%%%%%%%%%





\latexProblemContent{
\begin{problem}

Estimate the area under the graph of $f(x)={-x^{2} + 169}$ from $x={7}$ to $x={13}$ using ${6}$ rectangles and ${\text{left}}$ endpoints.

\input{2311-Compute-Integral-0001.HELP.tex}

\[\mbox{Area}\approx\answer{455}\]
\end{problem}}%}

%%%%%%%%%%%%%%%%%%%%%%





\latexProblemContent{
\begin{problem}

Estimate the area under the graph of $f(x)={-x^{2} + 9}$ from $x={-3}$ to $x={2}$ using ${5}$ rectangles and ${\text{left}}$ endpoints.

\input{2311-Compute-Integral-0001.HELP.tex}

\[\mbox{Area}\approx\answer{30}\]
\end{problem}}%}

%%%%%%%%%%%%%%%%%%%%%%





%%%%%%%%%%%%%%%%%%%%%%





\latexProblemContent{
\begin{problem}

Estimate the area under the graph of $f(x)={-x^{2} + 49}$ from $x={3}$ to $x={7}$ using ${4}$ rectangles and ${\text{left}}$ endpoints.

\input{2311-Compute-Integral-0001.HELP.tex}

\[\mbox{Area}\approx\answer{110}\]
\end{problem}}%}

%%%%%%%%%%%%%%%%%%%%%%





\latexProblemContent{
\begin{problem}

Estimate the area under the graph of $f(x)={x^{2} - 1}$ from $x={1}$ to $x={7}$ using ${6}$ rectangles and ${\text{left}}$ endpoints.

\input{2311-Compute-Integral-0001.HELP.tex}

\[\mbox{Area}\approx\answer{85}\]
\end{problem}}%}

%%%%%%%%%%%%%%%%%%%%%%





\latexProblemContent{
\begin{problem}

Estimate the area under the graph of $f(x)={-x + 6}$ from $x={1}$ to $x={6}$ using ${5}$ rectangles and ${\text{right}}$ endpoints.

\input{2311-Compute-Integral-0001.HELP.tex}

\[\mbox{Area}\approx\answer{10}\]
\end{problem}}%}

%%%%%%%%%%%%%%%%%%%%%%





%%%%%%%%%%%%%%%%%%%%%%





\latexProblemContent{
\begin{problem}

Estimate the area under the graph of $f(x)={x - 8}$ from $x={8}$ to $x={14}$ using ${6}$ rectangles and ${\text{right}}$ endpoints.

\input{2311-Compute-Integral-0001.HELP.tex}

\[\mbox{Area}\approx\answer{21}\]
\end{problem}}%}

%%%%%%%%%%%%%%%%%%%%%%





\latexProblemContent{
\begin{problem}

Estimate the area under the graph of $f(x)={x - 1}$ from $x={1}$ to $x={4}$ using ${3}$ rectangles and ${\text{right}}$ endpoints.

\input{2311-Compute-Integral-0001.HELP.tex}

\[\mbox{Area}\approx\answer{6}\]
\end{problem}}%}

%%%%%%%%%%%%%%%%%%%%%%





\latexProblemContent{
\begin{problem}

Estimate the area under the graph of $f(x)={-x + 7}$ from $x={2}$ to $x={7}$ using ${5}$ rectangles and ${\text{left}}$ endpoints.

\input{2311-Compute-Integral-0001.HELP.tex}

\[\mbox{Area}\approx\answer{15}\]
\end{problem}}%}

%%%%%%%%%%%%%%%%%%%%%%





\latexProblemContent{
\begin{problem}

Estimate the area under the graph of $f(x)={-x + 14}$ from $x={7}$ to $x={14}$ using ${7}$ rectangles and ${\text{left}}$ endpoints.

\input{2311-Compute-Integral-0001.HELP.tex}

\[\mbox{Area}\approx\answer{28}\]
\end{problem}}%}

%%%%%%%%%%%%%%%%%%%%%%





\latexProblemContent{
\begin{problem}

Estimate the area under the graph of $f(x)={-x^{2} + 100}$ from $x={5}$ to $x={10}$ using ${5}$ rectangles and ${\text{left}}$ endpoints.

\input{2311-Compute-Integral-0001.HELP.tex}

\[\mbox{Area}\approx\answer{245}\]
\end{problem}}%}

%%%%%%%%%%%%%%%%%%%%%%





\latexProblemContent{
\begin{problem}

Estimate the area under the graph of $f(x)={-x^{2} + 25}$ from $x={-2}$ to $x={5}$ using ${7}$ rectangles and ${\text{left}}$ endpoints.

\input{2311-Compute-Integral-0001.HELP.tex}

\[\mbox{Area}\approx\answer{140}\]
\end{problem}}%}

%%%%%%%%%%%%%%%%%%%%%%





%%%%%%%%%%%%%%%%%%%%%%





\latexProblemContent{
\begin{problem}

Estimate the area under the graph of $f(x)={x^{2} - 36}$ from $x={6}$ to $x={10}$ using ${4}$ rectangles and ${\text{left}}$ endpoints.

\input{2311-Compute-Integral-0001.HELP.tex}

\[\mbox{Area}\approx\answer{86}\]
\end{problem}}%}

%%%%%%%%%%%%%%%%%%%%%%





\latexProblemContent{
\begin{problem}

Estimate the area under the graph of $f(x)={-x^{2} + 100}$ from $x={4}$ to $x={10}$ using ${6}$ rectangles and ${\text{right}}$ endpoints.

\input{2311-Compute-Integral-0001.HELP.tex}

\[\mbox{Area}\approx\answer{245}\]
\end{problem}}%}

%%%%%%%%%%%%%%%%%%%%%%





%%%%%%%%%%%%%%%%%%%%%%





%%%%%%%%%%%%%%%%%%%%%%





\latexProblemContent{
\begin{problem}

Estimate the area under the graph of $f(x)={x^{2} - 9}$ from $x={-6}$ to $x={-3}$ using ${3}$ rectangles and ${\text{right}}$ endpoints.

\input{2311-Compute-Integral-0001.HELP.tex}

\[\mbox{Area}\approx\answer{23}\]
\end{problem}}%}

%%%%%%%%%%%%%%%%%%%%%%





\latexProblemContent{
\begin{problem}

Estimate the area under the graph of $f(x)={-x}$ from $x={-7}$ to $x={0}$ using ${7}$ rectangles and ${\text{right}}$ endpoints.

\input{2311-Compute-Integral-0001.HELP.tex}

\[\mbox{Area}\approx\answer{21}\]
\end{problem}}%}

%%%%%%%%%%%%%%%%%%%%%%





%%%%%%%%%%%%%%%%%%%%%%





\latexProblemContent{
\begin{problem}

Estimate the area under the graph of $f(x)={-x^{2} + 64}$ from $x={5}$ to $x={8}$ using ${3}$ rectangles and ${\text{right}}$ endpoints.

\input{2311-Compute-Integral-0001.HELP.tex}

\[\mbox{Area}\approx\answer{43}\]
\end{problem}}%}

%%%%%%%%%%%%%%%%%%%%%%





\latexProblemContent{
\begin{problem}

Estimate the area under the graph of $f(x)={x - 7}$ from $x={7}$ to $x={11}$ using ${4}$ rectangles and ${\text{left}}$ endpoints.

\input{2311-Compute-Integral-0001.HELP.tex}

\[\mbox{Area}\approx\answer{6}\]
\end{problem}}%}

%%%%%%%%%%%%%%%%%%%%%%





\latexProblemContent{
\begin{problem}

Estimate the area under the graph of $f(x)={x^{2} - 1}$ from $x={-3}$ to $x={1}$ using ${4}$ rectangles and ${\text{right}}$ endpoints.

\input{2311-Compute-Integral-0001.HELP.tex}

\[\mbox{Area}\approx\answer{2}\]
\end{problem}}%}

%%%%%%%%%%%%%%%%%%%%%%





\latexProblemContent{
\begin{problem}

Estimate the area under the graph of $f(x)={x + 5}$ from $x={-5}$ to $x={0}$ using ${5}$ rectangles and ${\text{right}}$ endpoints.

\input{2311-Compute-Integral-0001.HELP.tex}

\[\mbox{Area}\approx\answer{15}\]
\end{problem}}%}

%%%%%%%%%%%%%%%%%%%%%%





\latexProblemContent{
\begin{problem}

Estimate the area under the graph of $f(x)={x^{2} - 49}$ from $x={7}$ to $x={13}$ using ${6}$ rectangles and ${\text{right}}$ endpoints.

\input{2311-Compute-Integral-0001.HELP.tex}

\[\mbox{Area}\approx\answer{385}\]
\end{problem}}%}

%%%%%%%%%%%%%%%%%%%%%%





\latexProblemContent{
\begin{problem}

Estimate the area under the graph of $f(x)={-x + 6}$ from $x={-1}$ to $x={6}$ using ${7}$ rectangles and ${\text{left}}$ endpoints.

\input{2311-Compute-Integral-0001.HELP.tex}

\[\mbox{Area}\approx\answer{28}\]
\end{problem}}%}

%%%%%%%%%%%%%%%%%%%%%%





%%%%%%%%%%%%%%%%%%%%%%





\latexProblemContent{
\begin{problem}

Estimate the area under the graph of $f(x)={-x^{2} + 25}$ from $x={-2}$ to $x={5}$ using ${7}$ rectangles and ${\text{right}}$ endpoints.

\input{2311-Compute-Integral-0001.HELP.tex}

\[\mbox{Area}\approx\answer{119}\]
\end{problem}}%}

%%%%%%%%%%%%%%%%%%%%%%





\latexProblemContent{
\begin{problem}

Estimate the area under the graph of $f(x)={x^{2} - 49}$ from $x={7}$ to $x={11}$ using ${4}$ rectangles and ${\text{left}}$ endpoints.

\input{2311-Compute-Integral-0001.HELP.tex}

\[\mbox{Area}\approx\answer{98}\]
\end{problem}}%}

%%%%%%%%%%%%%%%%%%%%%%





\latexProblemContent{
\begin{problem}

Estimate the area under the graph of $f(x)={x - 2}$ from $x={2}$ to $x={7}$ using ${5}$ rectangles and ${\text{right}}$ endpoints.

\input{2311-Compute-Integral-0001.HELP.tex}

\[\mbox{Area}\approx\answer{15}\]
\end{problem}}%}

%%%%%%%%%%%%%%%%%%%%%%





\latexProblemContent{
\begin{problem}

Estimate the area under the graph of $f(x)={-x + 5}$ from $x={2}$ to $x={5}$ using ${3}$ rectangles and ${\text{left}}$ endpoints.

\input{2311-Compute-Integral-0001.HELP.tex}

\[\mbox{Area}\approx\answer{6}\]
\end{problem}}%}

%%%%%%%%%%%%%%%%%%%%%%





\latexProblemContent{
\begin{problem}

Estimate the area under the graph of $f(x)={-x^{2} + 9}$ from $x={-3}$ to $x={0}$ using ${3}$ rectangles and ${\text{right}}$ endpoints.

\input{2311-Compute-Integral-0001.HELP.tex}

\[\mbox{Area}\approx\answer{22}\]
\end{problem}}%}

%%%%%%%%%%%%%%%%%%%%%%





\latexProblemContent{
\begin{problem}

Estimate the area under the graph of $f(x)={x^{2} - 36}$ from $x={6}$ to $x={13}$ using ${7}$ rectangles and ${\text{left}}$ endpoints.

\input{2311-Compute-Integral-0001.HELP.tex}

\[\mbox{Area}\approx\answer{343}\]
\end{problem}}%}

%%%%%%%%%%%%%%%%%%%%%%





\latexProblemContent{
\begin{problem}

Estimate the area under the graph of $f(x)={-x - 4}$ from $x={-8}$ to $x={-4}$ using ${4}$ rectangles and ${\text{left}}$ endpoints.

\input{2311-Compute-Integral-0001.HELP.tex}

\[\mbox{Area}\approx\answer{10}\]
\end{problem}}%}

%%%%%%%%%%%%%%%%%%%%%%





\latexProblemContent{
\begin{problem}

Estimate the area under the graph of $f(x)={-x + 2}$ from $x={-1}$ to $x={2}$ using ${3}$ rectangles and ${\text{left}}$ endpoints.

\input{2311-Compute-Integral-0001.HELP.tex}

\[\mbox{Area}\approx\answer{6}\]
\end{problem}}%}

%%%%%%%%%%%%%%%%%%%%%%





%%%%%%%%%%%%%%%%%%%%%%





\latexProblemContent{
\begin{problem}

Estimate the area under the graph of $f(x)={x^{2} - 9}$ from $x={3}$ to $x={8}$ using ${5}$ rectangles and ${\text{right}}$ endpoints.

\input{2311-Compute-Integral-0001.HELP.tex}

\[\mbox{Area}\approx\answer{145}\]
\end{problem}}%}

%%%%%%%%%%%%%%%%%%%%%%





%%%%%%%%%%%%%%%%%%%%%%





\latexProblemContent{
\begin{problem}

Estimate the area under the graph of $f(x)={x - 8}$ from $x={8}$ to $x={13}$ using ${5}$ rectangles and ${\text{left}}$ endpoints.

\input{2311-Compute-Integral-0001.HELP.tex}

\[\mbox{Area}\approx\answer{10}\]
\end{problem}}%}

%%%%%%%%%%%%%%%%%%%%%%





\latexProblemContent{
\begin{problem}

Estimate the area under the graph of $f(x)={x^{2}}$ from $x={-6}$ to $x={0}$ using ${6}$ rectangles and ${\text{left}}$ endpoints.

\input{2311-Compute-Integral-0001.HELP.tex}

\[\mbox{Area}\approx\answer{91}\]
\end{problem}}%}

%%%%%%%%%%%%%%%%%%%%%%





\latexProblemContent{
\begin{problem}

Estimate the area under the graph of $f(x)={-x + 13}$ from $x={8}$ to $x={13}$ using ${5}$ rectangles and ${\text{right}}$ endpoints.

\input{2311-Compute-Integral-0001.HELP.tex}

\[\mbox{Area}\approx\answer{10}\]
\end{problem}}%}

%%%%%%%%%%%%%%%%%%%%%%





\latexProblemContent{
\begin{problem}

Estimate the area under the graph of $f(x)={x^{2} - 16}$ from $x={4}$ to $x={8}$ using ${4}$ rectangles and ${\text{left}}$ endpoints.

\input{2311-Compute-Integral-0001.HELP.tex}

\[\mbox{Area}\approx\answer{62}\]
\end{problem}}%}

%%%%%%%%%%%%%%%%%%%%%%





%%%%%%%%%%%%%%%%%%%%%%





\latexProblemContent{
\begin{problem}

Estimate the area under the graph of $f(x)={-x - 2}$ from $x={-8}$ to $x={-2}$ using ${6}$ rectangles and ${\text{left}}$ endpoints.

\input{2311-Compute-Integral-0001.HELP.tex}

\[\mbox{Area}\approx\answer{21}\]
\end{problem}}%}

%%%%%%%%%%%%%%%%%%%%%%





\latexProblemContent{
\begin{problem}

Estimate the area under the graph of $f(x)={x + 7}$ from $x={-7}$ to $x={-3}$ using ${4}$ rectangles and ${\text{right}}$ endpoints.

\input{2311-Compute-Integral-0001.HELP.tex}

\[\mbox{Area}\approx\answer{10}\]
\end{problem}}%}

%%%%%%%%%%%%%%%%%%%%%%





\latexProblemContent{
\begin{problem}

Estimate the area under the graph of $f(x)={x^{2} - 9}$ from $x={-3}$ to $x={4}$ using ${7}$ rectangles and ${\text{left}}$ endpoints.

\input{2311-Compute-Integral-0001.HELP.tex}

\[\mbox{Area}\approx\answer{-35}\]
\end{problem}}%}

%%%%%%%%%%%%%%%%%%%%%%





\latexProblemContent{
\begin{problem}

Estimate the area under the graph of $f(x)={-x^{2} + 81}$ from $x={5}$ to $x={9}$ using ${4}$ rectangles and ${\text{left}}$ endpoints.

\input{2311-Compute-Integral-0001.HELP.tex}

\[\mbox{Area}\approx\answer{150}\]
\end{problem}}%}

%%%%%%%%%%%%%%%%%%%%%%





\latexProblemContent{
\begin{problem}

Estimate the area under the graph of $f(x)={x + 6}$ from $x={-6}$ to $x={0}$ using ${6}$ rectangles and ${\text{left}}$ endpoints.

\input{2311-Compute-Integral-0001.HELP.tex}

\[\mbox{Area}\approx\answer{15}\]
\end{problem}}%}

%%%%%%%%%%%%%%%%%%%%%%





\latexProblemContent{
\begin{problem}

Estimate the area under the graph of $f(x)={-x - 3}$ from $x={-7}$ to $x={-3}$ using ${4}$ rectangles and ${\text{left}}$ endpoints.

\input{2311-Compute-Integral-0001.HELP.tex}

\[\mbox{Area}\approx\answer{10}\]
\end{problem}}%}

%%%%%%%%%%%%%%%%%%%%%%





%%%%%%%%%%%%%%%%%%%%%%





\latexProblemContent{
\begin{problem}

Estimate the area under the graph of $f(x)={x^{2} - 49}$ from $x={7}$ to $x={12}$ using ${5}$ rectangles and ${\text{right}}$ endpoints.

\input{2311-Compute-Integral-0001.HELP.tex}

\[\mbox{Area}\approx\answer{265}\]
\end{problem}}%}

%%%%%%%%%%%%%%%%%%%%%%





\latexProblemContent{
\begin{problem}

Estimate the area under the graph of $f(x)={x^{2} - 1}$ from $x={-1}$ to $x={5}$ using ${6}$ rectangles and ${\text{left}}$ endpoints.

\input{2311-Compute-Integral-0001.HELP.tex}

\[\mbox{Area}\approx\answer{25}\]
\end{problem}}%}

%%%%%%%%%%%%%%%%%%%%%%





\latexProblemContent{
\begin{problem}

Estimate the area under the graph of $f(x)={x^{2} - 9}$ from $x={-4}$ to $x={3}$ using ${7}$ rectangles and ${\text{left}}$ endpoints.

\input{2311-Compute-Integral-0001.HELP.tex}

\[\mbox{Area}\approx\answer{-28}\]
\end{problem}}%}

%%%%%%%%%%%%%%%%%%%%%%





%%%%%%%%%%%%%%%%%%%%%%





%%%%%%%%%%%%%%%%%%%%%%





\latexProblemContent{
\begin{problem}

Estimate the area under the graph of $f(x)={x^{2} - 1}$ from $x={1}$ to $x={5}$ using ${4}$ rectangles and ${\text{left}}$ endpoints.

\input{2311-Compute-Integral-0001.HELP.tex}

\[\mbox{Area}\approx\answer{26}\]
\end{problem}}%}

%%%%%%%%%%%%%%%%%%%%%%





\latexProblemContent{
\begin{problem}

Estimate the area under the graph of $f(x)={-x + 14}$ from $x={8}$ to $x={14}$ using ${6}$ rectangles and ${\text{right}}$ endpoints.

\input{2311-Compute-Integral-0001.HELP.tex}

\[\mbox{Area}\approx\answer{15}\]
\end{problem}}%}

%%%%%%%%%%%%%%%%%%%%%%





\latexProblemContent{
\begin{problem}

Estimate the area under the graph of $f(x)={-x^{2} + 64}$ from $x={2}$ to $x={8}$ using ${6}$ rectangles and ${\text{right}}$ endpoints.

\input{2311-Compute-Integral-0001.HELP.tex}

\[\mbox{Area}\approx\answer{185}\]
\end{problem}}%}

%%%%%%%%%%%%%%%%%%%%%%





\latexProblemContent{
\begin{problem}

Estimate the area under the graph of $f(x)={-x + 4}$ from $x={-3}$ to $x={4}$ using ${7}$ rectangles and ${\text{left}}$ endpoints.

\input{2311-Compute-Integral-0001.HELP.tex}

\[\mbox{Area}\approx\answer{28}\]
\end{problem}}%}

%%%%%%%%%%%%%%%%%%%%%%





\latexProblemContent{
\begin{problem}

Estimate the area under the graph of $f(x)={-x + 4}$ from $x={-3}$ to $x={4}$ using ${7}$ rectangles and ${\text{right}}$ endpoints.

\input{2311-Compute-Integral-0001.HELP.tex}

\[\mbox{Area}\approx\answer{21}\]
\end{problem}}%}

%%%%%%%%%%%%%%%%%%%%%%





\latexProblemContent{
\begin{problem}

Estimate the area under the graph of $f(x)={-x^{2} + 64}$ from $x={2}$ to $x={8}$ using ${6}$ rectangles and ${\text{left}}$ endpoints.

\input{2311-Compute-Integral-0001.HELP.tex}

\[\mbox{Area}\approx\answer{245}\]
\end{problem}}%}

%%%%%%%%%%%%%%%%%%%%%%





\latexProblemContent{
\begin{problem}

Estimate the area under the graph of $f(x)={x^{2} - 4}$ from $x={-5}$ to $x={-2}$ using ${3}$ rectangles and ${\text{left}}$ endpoints.

\input{2311-Compute-Integral-0001.HELP.tex}

\[\mbox{Area}\approx\answer{38}\]
\end{problem}}%}

%%%%%%%%%%%%%%%%%%%%%%





\latexProblemContent{
\begin{problem}

Estimate the area under the graph of $f(x)={x^{2} - 4}$ from $x={2}$ to $x={5}$ using ${3}$ rectangles and ${\text{right}}$ endpoints.

\input{2311-Compute-Integral-0001.HELP.tex}

\[\mbox{Area}\approx\answer{38}\]
\end{problem}}%}

%%%%%%%%%%%%%%%%%%%%%%





\latexProblemContent{
\begin{problem}

Estimate the area under the graph of $f(x)={x^{2} - 4}$ from $x={-2}$ to $x={4}$ using ${6}$ rectangles and ${\text{right}}$ endpoints.

\input{2311-Compute-Integral-0001.HELP.tex}

\[\mbox{Area}\approx\answer{7}\]
\end{problem}}%}

%%%%%%%%%%%%%%%%%%%%%%





%%%%%%%%%%%%%%%%%%%%%%





\latexProblemContent{
\begin{problem}

Estimate the area under the graph of $f(x)={x - 5}$ from $x={5}$ to $x={11}$ using ${6}$ rectangles and ${\text{right}}$ endpoints.

\input{2311-Compute-Integral-0001.HELP.tex}

\[\mbox{Area}\approx\answer{21}\]
\end{problem}}%}

%%%%%%%%%%%%%%%%%%%%%%





\latexProblemContent{
\begin{problem}

Estimate the area under the graph of $f(x)={-x^{2} + 121}$ from $x={7}$ to $x={11}$ using ${4}$ rectangles and ${\text{left}}$ endpoints.

\input{2311-Compute-Integral-0001.HELP.tex}

\[\mbox{Area}\approx\answer{190}\]
\end{problem}}%}

%%%%%%%%%%%%%%%%%%%%%%





\latexProblemContent{
\begin{problem}

Estimate the area under the graph of $f(x)={x + 2}$ from $x={-2}$ to $x={2}$ using ${4}$ rectangles and ${\text{left}}$ endpoints.

\input{2311-Compute-Integral-0001.HELP.tex}

\[\mbox{Area}\approx\answer{6}\]
\end{problem}}%}

%%%%%%%%%%%%%%%%%%%%%%





%%%%%%%%%%%%%%%%%%%%%%





%%%%%%%%%%%%%%%%%%%%%%





\latexProblemContent{
\begin{problem}

Estimate the area under the graph of $f(x)={-x + 8}$ from $x={5}$ to $x={8}$ using ${3}$ rectangles and ${\text{right}}$ endpoints.

\input{2311-Compute-Integral-0001.HELP.tex}

\[\mbox{Area}\approx\answer{3}\]
\end{problem}}%}

%%%%%%%%%%%%%%%%%%%%%%





\latexProblemContent{
\begin{problem}

Estimate the area under the graph of $f(x)={x^{2} - 9}$ from $x={-3}$ to $x={4}$ using ${7}$ rectangles and ${\text{right}}$ endpoints.

\input{2311-Compute-Integral-0001.HELP.tex}

\[\mbox{Area}\approx\answer{-28}\]
\end{problem}}%}

%%%%%%%%%%%%%%%%%%%%%%





\latexProblemContent{
\begin{problem}

Estimate the area under the graph of $f(x)={-x}$ from $x={-4}$ to $x={0}$ using ${4}$ rectangles and ${\text{left}}$ endpoints.

\input{2311-Compute-Integral-0001.HELP.tex}

\[\mbox{Area}\approx\answer{10}\]
\end{problem}}%}

%%%%%%%%%%%%%%%%%%%%%%





\latexProblemContent{
\begin{problem}

Estimate the area under the graph of $f(x)={x + 6}$ from $x={-6}$ to $x={1}$ using ${7}$ rectangles and ${\text{right}}$ endpoints.

\input{2311-Compute-Integral-0001.HELP.tex}

\[\mbox{Area}\approx\answer{28}\]
\end{problem}}%}

%%%%%%%%%%%%%%%%%%%%%%





\latexProblemContent{
\begin{problem}

Estimate the area under the graph of $f(x)={-x + 7}$ from $x={1}$ to $x={7}$ using ${6}$ rectangles and ${\text{left}}$ endpoints.

\input{2311-Compute-Integral-0001.HELP.tex}

\[\mbox{Area}\approx\answer{21}\]
\end{problem}}%}

%%%%%%%%%%%%%%%%%%%%%%





%%%%%%%%%%%%%%%%%%%%%%





%%%%%%%%%%%%%%%%%%%%%%





\latexProblemContent{
\begin{problem}

Estimate the area under the graph of $f(x)={-x + 9}$ from $x={6}$ to $x={9}$ using ${3}$ rectangles and ${\text{right}}$ endpoints.

\input{2311-Compute-Integral-0001.HELP.tex}

\[\mbox{Area}\approx\answer{3}\]
\end{problem}}%}

%%%%%%%%%%%%%%%%%%%%%%





\latexProblemContent{
\begin{problem}

Estimate the area under the graph of $f(x)={x + 2}$ from $x={-2}$ to $x={1}$ using ${3}$ rectangles and ${\text{right}}$ endpoints.

\input{2311-Compute-Integral-0001.HELP.tex}

\[\mbox{Area}\approx\answer{6}\]
\end{problem}}%}

%%%%%%%%%%%%%%%%%%%%%%





\latexProblemContent{
\begin{problem}

Estimate the area under the graph of $f(x)={x - 4}$ from $x={4}$ to $x={8}$ using ${4}$ rectangles and ${\text{left}}$ endpoints.

\input{2311-Compute-Integral-0001.HELP.tex}

\[\mbox{Area}\approx\answer{6}\]
\end{problem}}%}

%%%%%%%%%%%%%%%%%%%%%%





\latexProblemContent{
\begin{problem}

Estimate the area under the graph of $f(x)={x - 7}$ from $x={7}$ to $x={11}$ using ${4}$ rectangles and ${\text{right}}$ endpoints.

\input{2311-Compute-Integral-0001.HELP.tex}

\[\mbox{Area}\approx\answer{10}\]
\end{problem}}%}

%%%%%%%%%%%%%%%%%%%%%%





\latexProblemContent{
\begin{problem}

Estimate the area under the graph of $f(x)={x^{2} - 36}$ from $x={6}$ to $x={13}$ using ${7}$ rectangles and ${\text{right}}$ endpoints.

\input{2311-Compute-Integral-0001.HELP.tex}

\[\mbox{Area}\approx\answer{476}\]
\end{problem}}%}

%%%%%%%%%%%%%%%%%%%%%%





\latexProblemContent{
\begin{problem}

Estimate the area under the graph of $f(x)={x^{2} - 9}$ from $x={3}$ to $x={9}$ using ${6}$ rectangles and ${\text{right}}$ endpoints.

\input{2311-Compute-Integral-0001.HELP.tex}

\[\mbox{Area}\approx\answer{217}\]
\end{problem}}%}

%%%%%%%%%%%%%%%%%%%%%%





%%%%%%%%%%%%%%%%%%%%%%





%%%%%%%%%%%%%%%%%%%%%%





%%%%%%%%%%%%%%%%%%%%%%





%%%%%%%%%%%%%%%%%%%%%%





\latexProblemContent{
\begin{problem}

Estimate the area under the graph of $f(x)={x^{2} - 9}$ from $x={-8}$ to $x={-3}$ using ${5}$ rectangles and ${\text{left}}$ endpoints.

\input{2311-Compute-Integral-0001.HELP.tex}

\[\mbox{Area}\approx\answer{145}\]
\end{problem}}%}

%%%%%%%%%%%%%%%%%%%%%%





%%%%%%%%%%%%%%%%%%%%%%





%%%%%%%%%%%%%%%%%%%%%%





\latexProblemContent{
\begin{problem}

Estimate the area under the graph of $f(x)={x^{2} - 1}$ from $x={-1}$ to $x={3}$ using ${4}$ rectangles and ${\text{right}}$ endpoints.

\input{2311-Compute-Integral-0001.HELP.tex}

\[\mbox{Area}\approx\answer{10}\]
\end{problem}}%}

%%%%%%%%%%%%%%%%%%%%%%





\latexProblemContent{
\begin{problem}

Estimate the area under the graph of $f(x)={x^{2} - 1}$ from $x={-6}$ to $x={1}$ using ${7}$ rectangles and ${\text{left}}$ endpoints.

\input{2311-Compute-Integral-0001.HELP.tex}

\[\mbox{Area}\approx\answer{84}\]
\end{problem}}%}

%%%%%%%%%%%%%%%%%%%%%%





%%%%%%%%%%%%%%%%%%%%%%





\latexProblemContent{
\begin{problem}

Estimate the area under the graph of $f(x)={-x + 9}$ from $x={4}$ to $x={9}$ using ${5}$ rectangles and ${\text{left}}$ endpoints.

\input{2311-Compute-Integral-0001.HELP.tex}

\[\mbox{Area}\approx\answer{15}\]
\end{problem}}%}

%%%%%%%%%%%%%%%%%%%%%%





\latexProblemContent{
\begin{problem}

Estimate the area under the graph of $f(x)={-x + 11}$ from $x={6}$ to $x={11}$ using ${5}$ rectangles and ${\text{right}}$ endpoints.

\input{2311-Compute-Integral-0001.HELP.tex}

\[\mbox{Area}\approx\answer{10}\]
\end{problem}}%}

%%%%%%%%%%%%%%%%%%%%%%





\latexProblemContent{
\begin{problem}

Estimate the area under the graph of $f(x)={-x^{2} + 81}$ from $x={6}$ to $x={9}$ using ${3}$ rectangles and ${\text{left}}$ endpoints.

\input{2311-Compute-Integral-0001.HELP.tex}

\[\mbox{Area}\approx\answer{94}\]
\end{problem}}%}

%%%%%%%%%%%%%%%%%%%%%%





\latexProblemContent{
\begin{problem}

Estimate the area under the graph of $f(x)={x^{2} - 1}$ from $x={-3}$ to $x={1}$ using ${4}$ rectangles and ${\text{left}}$ endpoints.

\input{2311-Compute-Integral-0001.HELP.tex}

\[\mbox{Area}\approx\answer{10}\]
\end{problem}}%}

%%%%%%%%%%%%%%%%%%%%%%





\latexProblemContent{
\begin{problem}

Estimate the area under the graph of $f(x)={-x^{2} + 196}$ from $x={7}$ to $x={14}$ using ${7}$ rectangles and ${\text{right}}$ endpoints.

\input{2311-Compute-Integral-0001.HELP.tex}

\[\mbox{Area}\approx\answer{497}\]
\end{problem}}%}

%%%%%%%%%%%%%%%%%%%%%%





\latexProblemContent{
\begin{problem}

Estimate the area under the graph of $f(x)={-x^{2} + 16}$ from $x={-4}$ to $x={1}$ using ${5}$ rectangles and ${\text{right}}$ endpoints.

\input{2311-Compute-Integral-0001.HELP.tex}

\[\mbox{Area}\approx\answer{65}\]
\end{problem}}%}

%%%%%%%%%%%%%%%%%%%%%%





%%%%%%%%%%%%%%%%%%%%%%





%%%%%%%%%%%%%%%%%%%%%%





\latexProblemContent{
\begin{problem}

Estimate the area under the graph of $f(x)={x^{2} - 4}$ from $x={-2}$ to $x={2}$ using ${4}$ rectangles and ${\text{left}}$ endpoints.

\input{2311-Compute-Integral-0001.HELP.tex}

\[\mbox{Area}\approx\answer{-10}\]
\end{problem}}%}

%%%%%%%%%%%%%%%%%%%%%%





\latexProblemContent{
\begin{problem}

Estimate the area under the graph of $f(x)={x^{2} - 9}$ from $x={-3}$ to $x={3}$ using ${6}$ rectangles and ${\text{left}}$ endpoints.

\input{2311-Compute-Integral-0001.HELP.tex}

\[\mbox{Area}\approx\answer{-35}\]
\end{problem}}%}

%%%%%%%%%%%%%%%%%%%%%%





\latexProblemContent{
\begin{problem}

Estimate the area under the graph of $f(x)={x^{2}}$ from $x={-4}$ to $x={0}$ using ${4}$ rectangles and ${\text{left}}$ endpoints.

\input{2311-Compute-Integral-0001.HELP.tex}

\[\mbox{Area}\approx\answer{30}\]
\end{problem}}%}

%%%%%%%%%%%%%%%%%%%%%%





%%%%%%%%%%%%%%%%%%%%%%





\latexProblemContent{
\begin{problem}

Estimate the area under the graph of $f(x)={-x - 5}$ from $x={-8}$ to $x={-5}$ using ${3}$ rectangles and ${\text{left}}$ endpoints.

\input{2311-Compute-Integral-0001.HELP.tex}

\[\mbox{Area}\approx\answer{6}\]
\end{problem}}%}

%%%%%%%%%%%%%%%%%%%%%%





\latexProblemContent{
\begin{problem}

Estimate the area under the graph of $f(x)={x - 8}$ from $x={8}$ to $x={12}$ using ${4}$ rectangles and ${\text{right}}$ endpoints.

\input{2311-Compute-Integral-0001.HELP.tex}

\[\mbox{Area}\approx\answer{10}\]
\end{problem}}%}

%%%%%%%%%%%%%%%%%%%%%%





\latexProblemContent{
\begin{problem}

Estimate the area under the graph of $f(x)={x^{2} - 1}$ from $x={-6}$ to $x={-1}$ using ${5}$ rectangles and ${\text{left}}$ endpoints.

\input{2311-Compute-Integral-0001.HELP.tex}

\[\mbox{Area}\approx\answer{85}\]
\end{problem}}%}

%%%%%%%%%%%%%%%%%%%%%%





\latexProblemContent{
\begin{problem}

Estimate the area under the graph of $f(x)={x + 5}$ from $x={-5}$ to $x={0}$ using ${5}$ rectangles and ${\text{left}}$ endpoints.

\input{2311-Compute-Integral-0001.HELP.tex}

\[\mbox{Area}\approx\answer{10}\]
\end{problem}}%}

%%%%%%%%%%%%%%%%%%%%%%





\latexProblemContent{
\begin{problem}

Estimate the area under the graph of $f(x)={-x^{2} + 100}$ from $x={6}$ to $x={10}$ using ${4}$ rectangles and ${\text{right}}$ endpoints.

\input{2311-Compute-Integral-0001.HELP.tex}

\[\mbox{Area}\approx\answer{106}\]
\end{problem}}%}

%%%%%%%%%%%%%%%%%%%%%%





%%%%%%%%%%%%%%%%%%%%%%





\latexProblemContent{
\begin{problem}

Estimate the area under the graph of $f(x)={x^{2} - 36}$ from $x={6}$ to $x={12}$ using ${6}$ rectangles and ${\text{left}}$ endpoints.

\input{2311-Compute-Integral-0001.HELP.tex}

\[\mbox{Area}\approx\answer{235}\]
\end{problem}}%}

%%%%%%%%%%%%%%%%%%%%%%





%%%%%%%%%%%%%%%%%%%%%%





%%%%%%%%%%%%%%%%%%%%%%





\latexProblemContent{
\begin{problem}

Estimate the area under the graph of $f(x)={-x^{2} + 49}$ from $x={-7}$ to $x={-1}$ using ${6}$ rectangles and ${\text{left}}$ endpoints.

\input{2311-Compute-Integral-0001.HELP.tex}

\[\mbox{Area}\approx\answer{155}\]
\end{problem}}%}

%%%%%%%%%%%%%%%%%%%%%%





%%%%%%%%%%%%%%%%%%%%%%





\latexProblemContent{
\begin{problem}

Estimate the area under the graph of $f(x)={-x + 6}$ from $x={3}$ to $x={6}$ using ${3}$ rectangles and ${\text{right}}$ endpoints.

\input{2311-Compute-Integral-0001.HELP.tex}

\[\mbox{Area}\approx\answer{3}\]
\end{problem}}%}

%%%%%%%%%%%%%%%%%%%%%%





%%%%%%%%%%%%%%%%%%%%%%





\latexProblemContent{
\begin{problem}

Estimate the area under the graph of $f(x)={x^{2} - 49}$ from $x={7}$ to $x={10}$ using ${3}$ rectangles and ${\text{left}}$ endpoints.

\input{2311-Compute-Integral-0001.HELP.tex}

\[\mbox{Area}\approx\answer{47}\]
\end{problem}}%}

%%%%%%%%%%%%%%%%%%%%%%





\latexProblemContent{
\begin{problem}

Estimate the area under the graph of $f(x)={x^{2} - 64}$ from $x={8}$ to $x={11}$ using ${3}$ rectangles and ${\text{right}}$ endpoints.

\input{2311-Compute-Integral-0001.HELP.tex}

\[\mbox{Area}\approx\answer{110}\]
\end{problem}}%}

%%%%%%%%%%%%%%%%%%%%%%





\latexProblemContent{
\begin{problem}

Estimate the area under the graph of $f(x)={-x + 2}$ from $x={-2}$ to $x={2}$ using ${4}$ rectangles and ${\text{left}}$ endpoints.

\input{2311-Compute-Integral-0001.HELP.tex}

\[\mbox{Area}\approx\answer{10}\]
\end{problem}}%}

%%%%%%%%%%%%%%%%%%%%%%





\latexProblemContent{
\begin{problem}

Estimate the area under the graph of $f(x)={-x - 3}$ from $x={-8}$ to $x={-3}$ using ${5}$ rectangles and ${\text{left}}$ endpoints.

\input{2311-Compute-Integral-0001.HELP.tex}

\[\mbox{Area}\approx\answer{15}\]
\end{problem}}%}

%%%%%%%%%%%%%%%%%%%%%%





\latexProblemContent{
\begin{problem}

Estimate the area under the graph of $f(x)={x - 4}$ from $x={4}$ to $x={10}$ using ${6}$ rectangles and ${\text{right}}$ endpoints.

\input{2311-Compute-Integral-0001.HELP.tex}

\[\mbox{Area}\approx\answer{21}\]
\end{problem}}%}

%%%%%%%%%%%%%%%%%%%%%%





%%%%%%%%%%%%%%%%%%%%%%





%%%%%%%%%%%%%%%%%%%%%%





\latexProblemContent{
\begin{problem}

Estimate the area under the graph of $f(x)={x + 7}$ from $x={-7}$ to $x={-2}$ using ${5}$ rectangles and ${\text{right}}$ endpoints.

\input{2311-Compute-Integral-0001.HELP.tex}

\[\mbox{Area}\approx\answer{15}\]
\end{problem}}%}

%%%%%%%%%%%%%%%%%%%%%%





%%%%%%%%%%%%%%%%%%%%%%





\latexProblemContent{
\begin{problem}

Estimate the area under the graph of $f(x)={-x^{2} + 25}$ from $x={-5}$ to $x={-2}$ using ${3}$ rectangles and ${\text{left}}$ endpoints.

\input{2311-Compute-Integral-0001.HELP.tex}

\[\mbox{Area}\approx\answer{25}\]
\end{problem}}%}

%%%%%%%%%%%%%%%%%%%%%%





\latexProblemContent{
\begin{problem}

Estimate the area under the graph of $f(x)={-x + 10}$ from $x={3}$ to $x={10}$ using ${7}$ rectangles and ${\text{right}}$ endpoints.

\input{2311-Compute-Integral-0001.HELP.tex}

\[\mbox{Area}\approx\answer{21}\]
\end{problem}}%}

%%%%%%%%%%%%%%%%%%%%%%





\latexProblemContent{
\begin{problem}

Estimate the area under the graph of $f(x)={x^{2} - 9}$ from $x={-8}$ to $x={-3}$ using ${5}$ rectangles and ${\text{right}}$ endpoints.

\input{2311-Compute-Integral-0001.HELP.tex}

\[\mbox{Area}\approx\answer{90}\]
\end{problem}}%}

%%%%%%%%%%%%%%%%%%%%%%





\latexProblemContent{
\begin{problem}

Estimate the area under the graph of $f(x)={x + 5}$ from $x={-5}$ to $x={-2}$ using ${3}$ rectangles and ${\text{right}}$ endpoints.

\input{2311-Compute-Integral-0001.HELP.tex}

\[\mbox{Area}\approx\answer{6}\]
\end{problem}}%}

%%%%%%%%%%%%%%%%%%%%%%





\latexProblemContent{
\begin{problem}

Estimate the area under the graph of $f(x)={-x - 2}$ from $x={-7}$ to $x={-2}$ using ${5}$ rectangles and ${\text{right}}$ endpoints.

\input{2311-Compute-Integral-0001.HELP.tex}

\[\mbox{Area}\approx\answer{10}\]
\end{problem}}%}

%%%%%%%%%%%%%%%%%%%%%%





\latexProblemContent{
\begin{problem}

Estimate the area under the graph of $f(x)={-x^{2} + 36}$ from $x={3}$ to $x={6}$ using ${3}$ rectangles and ${\text{left}}$ endpoints.

\input{2311-Compute-Integral-0001.HELP.tex}

\[\mbox{Area}\approx\answer{58}\]
\end{problem}}%}

%%%%%%%%%%%%%%%%%%%%%%





\latexProblemContent{
\begin{problem}

Estimate the area under the graph of $f(x)={-x^{2} + 81}$ from $x={2}$ to $x={9}$ using ${7}$ rectangles and ${\text{right}}$ endpoints.

\input{2311-Compute-Integral-0001.HELP.tex}

\[\mbox{Area}\approx\answer{287}\]
\end{problem}}%}

%%%%%%%%%%%%%%%%%%%%%%





\latexProblemContent{
\begin{problem}

Estimate the area under the graph of $f(x)={x - 3}$ from $x={3}$ to $x={6}$ using ${3}$ rectangles and ${\text{right}}$ endpoints.

\input{2311-Compute-Integral-0001.HELP.tex}

\[\mbox{Area}\approx\answer{6}\]
\end{problem}}%}

%%%%%%%%%%%%%%%%%%%%%%





\latexProblemContent{
\begin{problem}

Estimate the area under the graph of $f(x)={x^{2} - 4}$ from $x={2}$ to $x={9}$ using ${7}$ rectangles and ${\text{right}}$ endpoints.

\input{2311-Compute-Integral-0001.HELP.tex}

\[\mbox{Area}\approx\answer{252}\]
\end{problem}}%}

%%%%%%%%%%%%%%%%%%%%%%





\latexProblemContent{
\begin{problem}

Estimate the area under the graph of $f(x)={x^{2} - 1}$ from $x={-4}$ to $x={-1}$ using ${3}$ rectangles and ${\text{left}}$ endpoints.

\input{2311-Compute-Integral-0001.HELP.tex}

\[\mbox{Area}\approx\answer{26}\]
\end{problem}}%}

%%%%%%%%%%%%%%%%%%%%%%





%%%%%%%%%%%%%%%%%%%%%%





\latexProblemContent{
\begin{problem}

Estimate the area under the graph of $f(x)={x^{2} - 49}$ from $x={7}$ to $x={14}$ using ${7}$ rectangles and ${\text{right}}$ endpoints.

\input{2311-Compute-Integral-0001.HELP.tex}

\[\mbox{Area}\approx\answer{532}\]
\end{problem}}%}

%%%%%%%%%%%%%%%%%%%%%%





\latexProblemContent{
\begin{problem}

Estimate the area under the graph of $f(x)={-x^{2} + 100}$ from $x={4}$ to $x={10}$ using ${6}$ rectangles and ${\text{left}}$ endpoints.

\input{2311-Compute-Integral-0001.HELP.tex}

\[\mbox{Area}\approx\answer{329}\]
\end{problem}}%}

%%%%%%%%%%%%%%%%%%%%%%





%%%%%%%%%%%%%%%%%%%%%%





%%%%%%%%%%%%%%%%%%%%%%





\latexProblemContent{
\begin{problem}

Estimate the area under the graph of $f(x)={-x + 3}$ from $x={-1}$ to $x={3}$ using ${4}$ rectangles and ${\text{right}}$ endpoints.

\input{2311-Compute-Integral-0001.HELP.tex}

\[\mbox{Area}\approx\answer{6}\]
\end{problem}}%}

%%%%%%%%%%%%%%%%%%%%%%





\latexProblemContent{
\begin{problem}

Estimate the area under the graph of $f(x)={x^{2} - 16}$ from $x={4}$ to $x={7}$ using ${3}$ rectangles and ${\text{right}}$ endpoints.

\input{2311-Compute-Integral-0001.HELP.tex}

\[\mbox{Area}\approx\answer{62}\]
\end{problem}}%}

%%%%%%%%%%%%%%%%%%%%%%





\latexProblemContent{
\begin{problem}

Estimate the area under the graph of $f(x)={-x + 1}$ from $x={-3}$ to $x={1}$ using ${4}$ rectangles and ${\text{right}}$ endpoints.

\input{2311-Compute-Integral-0001.HELP.tex}

\[\mbox{Area}\approx\answer{6}\]
\end{problem}}%}

%%%%%%%%%%%%%%%%%%%%%%





\latexProblemContent{
\begin{problem}

Estimate the area under the graph of $f(x)={x + 5}$ from $x={-5}$ to $x={1}$ using ${6}$ rectangles and ${\text{right}}$ endpoints.

\input{2311-Compute-Integral-0001.HELP.tex}

\[\mbox{Area}\approx\answer{21}\]
\end{problem}}%}

%%%%%%%%%%%%%%%%%%%%%%





\latexProblemContent{
\begin{problem}

Estimate the area under the graph of $f(x)={-x + 5}$ from $x={-2}$ to $x={5}$ using ${7}$ rectangles and ${\text{left}}$ endpoints.

\input{2311-Compute-Integral-0001.HELP.tex}

\[\mbox{Area}\approx\answer{28}\]
\end{problem}}%}

%%%%%%%%%%%%%%%%%%%%%%





%%%%%%%%%%%%%%%%%%%%%%





\latexProblemContent{
\begin{problem}

Estimate the area under the graph of $f(x)={x^{2} - 9}$ from $x={3}$ to $x={8}$ using ${5}$ rectangles and ${\text{left}}$ endpoints.

\input{2311-Compute-Integral-0001.HELP.tex}

\[\mbox{Area}\approx\answer{90}\]
\end{problem}}%}

%%%%%%%%%%%%%%%%%%%%%%





\latexProblemContent{
\begin{problem}

Estimate the area under the graph of $f(x)={-x^{2} + 36}$ from $x={-6}$ to $x={-3}$ using ${3}$ rectangles and ${\text{left}}$ endpoints.

\input{2311-Compute-Integral-0001.HELP.tex}

\[\mbox{Area}\approx\answer{31}\]
\end{problem}}%}

%%%%%%%%%%%%%%%%%%%%%%





%%%%%%%%%%%%%%%%%%%%%%





\latexProblemContent{
\begin{problem}

Estimate the area under the graph of $f(x)={-x + 10}$ from $x={7}$ to $x={10}$ using ${3}$ rectangles and ${\text{right}}$ endpoints.

\input{2311-Compute-Integral-0001.HELP.tex}

\[\mbox{Area}\approx\answer{3}\]
\end{problem}}%}

%%%%%%%%%%%%%%%%%%%%%%





\latexProblemContent{
\begin{problem}

Estimate the area under the graph of $f(x)={x^{2} - 25}$ from $x={5}$ to $x={8}$ using ${3}$ rectangles and ${\text{left}}$ endpoints.

\input{2311-Compute-Integral-0001.HELP.tex}

\[\mbox{Area}\approx\answer{35}\]
\end{problem}}%}

%%%%%%%%%%%%%%%%%%%%%%





%%%%%%%%%%%%%%%%%%%%%%





\latexProblemContent{
\begin{problem}

Estimate the area under the graph of $f(x)={x + 5}$ from $x={-5}$ to $x={1}$ using ${6}$ rectangles and ${\text{left}}$ endpoints.

\input{2311-Compute-Integral-0001.HELP.tex}

\[\mbox{Area}\approx\answer{15}\]
\end{problem}}%}

%%%%%%%%%%%%%%%%%%%%%%





\latexProblemContent{
\begin{problem}

Estimate the area under the graph of $f(x)={-x + 2}$ from $x={-1}$ to $x={2}$ using ${3}$ rectangles and ${\text{right}}$ endpoints.

\input{2311-Compute-Integral-0001.HELP.tex}

\[\mbox{Area}\approx\answer{3}\]
\end{problem}}%}

%%%%%%%%%%%%%%%%%%%%%%





\latexProblemContent{
\begin{problem}

Estimate the area under the graph of $f(x)={-x^{2} + 81}$ from $x={2}$ to $x={9}$ using ${7}$ rectangles and ${\text{left}}$ endpoints.

\input{2311-Compute-Integral-0001.HELP.tex}

\[\mbox{Area}\approx\answer{364}\]
\end{problem}}%}

%%%%%%%%%%%%%%%%%%%%%%





\latexProblemContent{
\begin{problem}

Estimate the area under the graph of $f(x)={x^{2} - 1}$ from $x={-5}$ to $x={-1}$ using ${4}$ rectangles and ${\text{right}}$ endpoints.

\input{2311-Compute-Integral-0001.HELP.tex}

\[\mbox{Area}\approx\answer{26}\]
\end{problem}}%}

%%%%%%%%%%%%%%%%%%%%%%





\latexProblemContent{
\begin{problem}

Estimate the area under the graph of $f(x)={x^{2} - 64}$ from $x={8}$ to $x={11}$ using ${3}$ rectangles and ${\text{left}}$ endpoints.

\input{2311-Compute-Integral-0001.HELP.tex}

\[\mbox{Area}\approx\answer{53}\]
\end{problem}}%}

%%%%%%%%%%%%%%%%%%%%%%





\latexProblemContent{
\begin{problem}

Estimate the area under the graph of $f(x)={x + 7}$ from $x={-7}$ to $x={-4}$ using ${3}$ rectangles and ${\text{left}}$ endpoints.

\input{2311-Compute-Integral-0001.HELP.tex}

\[\mbox{Area}\approx\answer{3}\]
\end{problem}}%}

%%%%%%%%%%%%%%%%%%%%%%





\latexProblemContent{
\begin{problem}

Estimate the area under the graph of $f(x)={x - 6}$ from $x={6}$ to $x={12}$ using ${6}$ rectangles and ${\text{right}}$ endpoints.

\input{2311-Compute-Integral-0001.HELP.tex}

\[\mbox{Area}\approx\answer{21}\]
\end{problem}}%}

%%%%%%%%%%%%%%%%%%%%%%





\latexProblemContent{
\begin{problem}

Estimate the area under the graph of $f(x)={x - 1}$ from $x={1}$ to $x={5}$ using ${4}$ rectangles and ${\text{left}}$ endpoints.

\input{2311-Compute-Integral-0001.HELP.tex}

\[\mbox{Area}\approx\answer{6}\]
\end{problem}}%}

%%%%%%%%%%%%%%%%%%%%%%





\latexProblemContent{
\begin{problem}

Estimate the area under the graph of $f(x)={x + 4}$ from $x={-4}$ to $x={3}$ using ${7}$ rectangles and ${\text{left}}$ endpoints.

\input{2311-Compute-Integral-0001.HELP.tex}

\[\mbox{Area}\approx\answer{21}\]
\end{problem}}%}

%%%%%%%%%%%%%%%%%%%%%%





%%%%%%%%%%%%%%%%%%%%%%





\latexProblemContent{
\begin{problem}

Estimate the area under the graph of $f(x)={-x^{2} + 64}$ from $x={-8}$ to $x={-2}$ using ${6}$ rectangles and ${\text{left}}$ endpoints.

\input{2311-Compute-Integral-0001.HELP.tex}

\[\mbox{Area}\approx\answer{185}\]
\end{problem}}%}

%%%%%%%%%%%%%%%%%%%%%%





%%%%%%%%%%%%%%%%%%%%%%





\latexProblemContent{
\begin{problem}

Estimate the area under the graph of $f(x)={-x^{2} + 121}$ from $x={6}$ to $x={11}$ using ${5}$ rectangles and ${\text{right}}$ endpoints.

\input{2311-Compute-Integral-0001.HELP.tex}

\[\mbox{Area}\approx\answer{190}\]
\end{problem}}%}

%%%%%%%%%%%%%%%%%%%%%%





%%%%%%%%%%%%%%%%%%%%%%





%%%%%%%%%%%%%%%%%%%%%%





\latexProblemContent{
\begin{problem}

Estimate the area under the graph of $f(x)={-x^{2} + 144}$ from $x={7}$ to $x={12}$ using ${5}$ rectangles and ${\text{left}}$ endpoints.

\input{2311-Compute-Integral-0001.HELP.tex}

\[\mbox{Area}\approx\answer{305}\]
\end{problem}}%}

%%%%%%%%%%%%%%%%%%%%%%





%%%%%%%%%%%%%%%%%%%%%%





%%%%%%%%%%%%%%%%%%%%%%





%%%%%%%%%%%%%%%%%%%%%%





\latexProblemContent{
\begin{problem}

Estimate the area under the graph of $f(x)={x - 3}$ from $x={3}$ to $x={7}$ using ${4}$ rectangles and ${\text{left}}$ endpoints.

\input{2311-Compute-Integral-0001.HELP.tex}

\[\mbox{Area}\approx\answer{6}\]
\end{problem}}%}

%%%%%%%%%%%%%%%%%%%%%%





%%%%%%%%%%%%%%%%%%%%%%





\latexProblemContent{
\begin{problem}

Estimate the area under the graph of $f(x)={x + 6}$ from $x={-6}$ to $x={-3}$ using ${3}$ rectangles and ${\text{left}}$ endpoints.

\input{2311-Compute-Integral-0001.HELP.tex}

\[\mbox{Area}\approx\answer{3}\]
\end{problem}}%}

%%%%%%%%%%%%%%%%%%%%%%





%%%%%%%%%%%%%%%%%%%%%%





\latexProblemContent{
\begin{problem}

Estimate the area under the graph of $f(x)={x - 1}$ from $x={1}$ to $x={6}$ using ${5}$ rectangles and ${\text{right}}$ endpoints.

\input{2311-Compute-Integral-0001.HELP.tex}

\[\mbox{Area}\approx\answer{15}\]
\end{problem}}%}

%%%%%%%%%%%%%%%%%%%%%%





\latexProblemContent{
\begin{problem}

Estimate the area under the graph of $f(x)={-x^{2} + 196}$ from $x={8}$ to $x={14}$ using ${6}$ rectangles and ${\text{left}}$ endpoints.

\input{2311-Compute-Integral-0001.HELP.tex}

\[\mbox{Area}\approx\answer{497}\]
\end{problem}}%}

%%%%%%%%%%%%%%%%%%%%%%





\latexProblemContent{
\begin{problem}

Estimate the area under the graph of $f(x)={x + 1}$ from $x={-1}$ to $x={5}$ using ${6}$ rectangles and ${\text{right}}$ endpoints.

\input{2311-Compute-Integral-0001.HELP.tex}

\[\mbox{Area}\approx\answer{21}\]
\end{problem}}%}

%%%%%%%%%%%%%%%%%%%%%%





%%%%%%%%%%%%%%%%%%%%%%





%%%%%%%%%%%%%%%%%%%%%%





%%%%%%%%%%%%%%%%%%%%%%





\latexProblemContent{
\begin{problem}

Estimate the area under the graph of $f(x)={x^{2} - 1}$ from $x={1}$ to $x={4}$ using ${3}$ rectangles and ${\text{left}}$ endpoints.

\input{2311-Compute-Integral-0001.HELP.tex}

\[\mbox{Area}\approx\answer{11}\]
\end{problem}}%}

%%%%%%%%%%%%%%%%%%%%%%





\latexProblemContent{
\begin{problem}

Estimate the area under the graph of $f(x)={-x^{2} + 81}$ from $x={4}$ to $x={9}$ using ${5}$ rectangles and ${\text{left}}$ endpoints.

\input{2311-Compute-Integral-0001.HELP.tex}

\[\mbox{Area}\approx\answer{215}\]
\end{problem}}%}

%%%%%%%%%%%%%%%%%%%%%%





\latexProblemContent{
\begin{problem}

Estimate the area under the graph of $f(x)={-x + 2}$ from $x={-2}$ to $x={2}$ using ${4}$ rectangles and ${\text{right}}$ endpoints.

\input{2311-Compute-Integral-0001.HELP.tex}

\[\mbox{Area}\approx\answer{6}\]
\end{problem}}%}

%%%%%%%%%%%%%%%%%%%%%%





\latexProblemContent{
\begin{problem}

Estimate the area under the graph of $f(x)={x^{2} - 25}$ from $x={5}$ to $x={10}$ using ${5}$ rectangles and ${\text{left}}$ endpoints.

\input{2311-Compute-Integral-0001.HELP.tex}

\[\mbox{Area}\approx\answer{130}\]
\end{problem}}%}

%%%%%%%%%%%%%%%%%%%%%%





\latexProblemContent{
\begin{problem}

Estimate the area under the graph of $f(x)={x - 4}$ from $x={4}$ to $x={10}$ using ${6}$ rectangles and ${\text{left}}$ endpoints.

\input{2311-Compute-Integral-0001.HELP.tex}

\[\mbox{Area}\approx\answer{15}\]
\end{problem}}%}

%%%%%%%%%%%%%%%%%%%%%%





\latexProblemContent{
\begin{problem}

Estimate the area under the graph of $f(x)={-x^{2} + 225}$ from $x={8}$ to $x={15}$ using ${7}$ rectangles and ${\text{left}}$ endpoints.

\input{2311-Compute-Integral-0001.HELP.tex}

\[\mbox{Area}\approx\answer{700}\]
\end{problem}}%}

%%%%%%%%%%%%%%%%%%%%%%





\latexProblemContent{
\begin{problem}

Estimate the area under the graph of $f(x)={-x + 6}$ from $x={2}$ to $x={6}$ using ${4}$ rectangles and ${\text{left}}$ endpoints.

\input{2311-Compute-Integral-0001.HELP.tex}

\[\mbox{Area}\approx\answer{10}\]
\end{problem}}%}

%%%%%%%%%%%%%%%%%%%%%%





%%%%%%%%%%%%%%%%%%%%%%





\latexProblemContent{
\begin{problem}

Estimate the area under the graph of $f(x)={-x^{2} + 9}$ from $x={-3}$ to $x={2}$ using ${5}$ rectangles and ${\text{right}}$ endpoints.

\input{2311-Compute-Integral-0001.HELP.tex}

\[\mbox{Area}\approx\answer{35}\]
\end{problem}}%}

%%%%%%%%%%%%%%%%%%%%%%





\latexProblemContent{
\begin{problem}

Estimate the area under the graph of $f(x)={-x^{2} + 25}$ from $x={-5}$ to $x={1}$ using ${6}$ rectangles and ${\text{left}}$ endpoints.

\input{2311-Compute-Integral-0001.HELP.tex}

\[\mbox{Area}\approx\answer{95}\]
\end{problem}}%}

%%%%%%%%%%%%%%%%%%%%%%





\latexProblemContent{
\begin{problem}

Estimate the area under the graph of $f(x)={x^{2}}$ from $x={-6}$ to $x={0}$ using ${6}$ rectangles and ${\text{right}}$ endpoints.

\input{2311-Compute-Integral-0001.HELP.tex}

\[\mbox{Area}\approx\answer{55}\]
\end{problem}}%}

%%%%%%%%%%%%%%%%%%%%%%





%%%%%%%%%%%%%%%%%%%%%%





\latexProblemContent{
\begin{problem}

Estimate the area under the graph of $f(x)={-x + 8}$ from $x={1}$ to $x={8}$ using ${7}$ rectangles and ${\text{left}}$ endpoints.

\input{2311-Compute-Integral-0001.HELP.tex}

\[\mbox{Area}\approx\answer{28}\]
\end{problem}}%}

%%%%%%%%%%%%%%%%%%%%%%





%%%%%%%%%%%%%%%%%%%%%%





\latexProblemContent{
\begin{problem}

Estimate the area under the graph of $f(x)={-x^{2} + 225}$ from $x={8}$ to $x={15}$ using ${7}$ rectangles and ${\text{right}}$ endpoints.

\input{2311-Compute-Integral-0001.HELP.tex}

\[\mbox{Area}\approx\answer{539}\]
\end{problem}}%}

%%%%%%%%%%%%%%%%%%%%%%





\latexProblemContent{
\begin{problem}

Estimate the area under the graph of $f(x)={-x + 4}$ from $x={-1}$ to $x={4}$ using ${5}$ rectangles and ${\text{left}}$ endpoints.

\input{2311-Compute-Integral-0001.HELP.tex}

\[\mbox{Area}\approx\answer{15}\]
\end{problem}}%}

%%%%%%%%%%%%%%%%%%%%%%





%%%%%%%%%%%%%%%%%%%%%%





\latexProblemContent{
\begin{problem}

Estimate the area under the graph of $f(x)={x + 6}$ from $x={-6}$ to $x={-2}$ using ${4}$ rectangles and ${\text{right}}$ endpoints.

\input{2311-Compute-Integral-0001.HELP.tex}

\[\mbox{Area}\approx\answer{10}\]
\end{problem}}%}

%%%%%%%%%%%%%%%%%%%%%%





\latexProblemContent{
\begin{problem}

Estimate the area under the graph of $f(x)={-x + 5}$ from $x={-1}$ to $x={5}$ using ${6}$ rectangles and ${\text{left}}$ endpoints.

\input{2311-Compute-Integral-0001.HELP.tex}

\[\mbox{Area}\approx\answer{21}\]
\end{problem}}%}

%%%%%%%%%%%%%%%%%%%%%%





%%%%%%%%%%%%%%%%%%%%%%





\latexProblemContent{
\begin{problem}

Estimate the area under the graph of $f(x)={-x + 9}$ from $x={4}$ to $x={9}$ using ${5}$ rectangles and ${\text{right}}$ endpoints.

\input{2311-Compute-Integral-0001.HELP.tex}

\[\mbox{Area}\approx\answer{10}\]
\end{problem}}%}

%%%%%%%%%%%%%%%%%%%%%%





%%%%%%%%%%%%%%%%%%%%%%





%%%%%%%%%%%%%%%%%%%%%%





\latexProblemContent{
\begin{problem}

Estimate the area under the graph of $f(x)={-x^{2} + 100}$ from $x={5}$ to $x={10}$ using ${5}$ rectangles and ${\text{right}}$ endpoints.

\input{2311-Compute-Integral-0001.HELP.tex}

\[\mbox{Area}\approx\answer{170}\]
\end{problem}}%}

%%%%%%%%%%%%%%%%%%%%%%





\latexProblemContent{
\begin{problem}

Estimate the area under the graph of $f(x)={-x + 15}$ from $x={8}$ to $x={15}$ using ${7}$ rectangles and ${\text{left}}$ endpoints.

\input{2311-Compute-Integral-0001.HELP.tex}

\[\mbox{Area}\approx\answer{28}\]
\end{problem}}%}

%%%%%%%%%%%%%%%%%%%%%%





\latexProblemContent{
\begin{problem}

Estimate the area under the graph of $f(x)={-x + 8}$ from $x={2}$ to $x={8}$ using ${6}$ rectangles and ${\text{right}}$ endpoints.

\input{2311-Compute-Integral-0001.HELP.tex}

\[\mbox{Area}\approx\answer{15}\]
\end{problem}}%}

%%%%%%%%%%%%%%%%%%%%%%





\latexProblemContent{
\begin{problem}

Estimate the area under the graph of $f(x)={x - 7}$ from $x={7}$ to $x={12}$ using ${5}$ rectangles and ${\text{right}}$ endpoints.

\input{2311-Compute-Integral-0001.HELP.tex}

\[\mbox{Area}\approx\answer{15}\]
\end{problem}}%}

%%%%%%%%%%%%%%%%%%%%%%





\latexProblemContent{
\begin{problem}

Estimate the area under the graph of $f(x)={-x + 10}$ from $x={4}$ to $x={10}$ using ${6}$ rectangles and ${\text{left}}$ endpoints.

\input{2311-Compute-Integral-0001.HELP.tex}

\[\mbox{Area}\approx\answer{21}\]
\end{problem}}%}

%%%%%%%%%%%%%%%%%%%%%%





%%%%%%%%%%%%%%%%%%%%%%





\latexProblemContent{
\begin{problem}

Estimate the area under the graph of $f(x)={x + 8}$ from $x={-8}$ to $x={-5}$ using ${3}$ rectangles and ${\text{left}}$ endpoints.

\input{2311-Compute-Integral-0001.HELP.tex}

\[\mbox{Area}\approx\answer{3}\]
\end{problem}}%}

%%%%%%%%%%%%%%%%%%%%%%





\latexProblemContent{
\begin{problem}

Estimate the area under the graph of $f(x)={x^{2} - 36}$ from $x={6}$ to $x={12}$ using ${6}$ rectangles and ${\text{right}}$ endpoints.

\input{2311-Compute-Integral-0001.HELP.tex}

\[\mbox{Area}\approx\answer{343}\]
\end{problem}}%}

%%%%%%%%%%%%%%%%%%%%%%





\latexProblemContent{
\begin{problem}

Estimate the area under the graph of $f(x)={-x + 6}$ from $x={2}$ to $x={6}$ using ${4}$ rectangles and ${\text{right}}$ endpoints.

\input{2311-Compute-Integral-0001.HELP.tex}

\[\mbox{Area}\approx\answer{6}\]
\end{problem}}%}

%%%%%%%%%%%%%%%%%%%%%%





%%%%%%%%%%%%%%%%%%%%%%





\latexProblemContent{
\begin{problem}

Estimate the area under the graph of $f(x)={x - 4}$ from $x={4}$ to $x={11}$ using ${7}$ rectangles and ${\text{left}}$ endpoints.

\input{2311-Compute-Integral-0001.HELP.tex}

\[\mbox{Area}\approx\answer{21}\]
\end{problem}}%}

%%%%%%%%%%%%%%%%%%%%%%





\latexProblemContent{
\begin{problem}

Estimate the area under the graph of $f(x)={x - 7}$ from $x={7}$ to $x={13}$ using ${6}$ rectangles and ${\text{right}}$ endpoints.

\input{2311-Compute-Integral-0001.HELP.tex}

\[\mbox{Area}\approx\answer{21}\]
\end{problem}}%}

%%%%%%%%%%%%%%%%%%%%%%





\latexProblemContent{
\begin{problem}

Estimate the area under the graph of $f(x)={-x^{2} + 49}$ from $x={2}$ to $x={7}$ using ${5}$ rectangles and ${\text{right}}$ endpoints.

\input{2311-Compute-Integral-0001.HELP.tex}

\[\mbox{Area}\approx\answer{110}\]
\end{problem}}%}

%%%%%%%%%%%%%%%%%%%%%%





%%%%%%%%%%%%%%%%%%%%%%





%%%%%%%%%%%%%%%%%%%%%%





\latexProblemContent{
\begin{problem}

Estimate the area under the graph of $f(x)={x^{2} - 64}$ from $x={8}$ to $x={15}$ using ${7}$ rectangles and ${\text{left}}$ endpoints.

\input{2311-Compute-Integral-0001.HELP.tex}

\[\mbox{Area}\approx\answer{427}\]
\end{problem}}%}

%%%%%%%%%%%%%%%%%%%%%%





\latexProblemContent{
\begin{problem}

Estimate the area under the graph of $f(x)={x^{2} - 16}$ from $x={4}$ to $x={10}$ using ${6}$ rectangles and ${\text{left}}$ endpoints.

\input{2311-Compute-Integral-0001.HELP.tex}

\[\mbox{Area}\approx\answer{175}\]
\end{problem}}%}

%%%%%%%%%%%%%%%%%%%%%%





\latexProblemContent{
\begin{problem}

Estimate the area under the graph of $f(x)={x^{2}}$ from $x={-5}$ to $x={0}$ using ${5}$ rectangles and ${\text{left}}$ endpoints.

\input{2311-Compute-Integral-0001.HELP.tex}

\[\mbox{Area}\approx\answer{55}\]
\end{problem}}%}

%%%%%%%%%%%%%%%%%%%%%%





%%%%%%%%%%%%%%%%%%%%%%





%%%%%%%%%%%%%%%%%%%%%%





\latexProblemContent{
\begin{problem}

Estimate the area under the graph of $f(x)={-x^{2} + 144}$ from $x={8}$ to $x={12}$ using ${4}$ rectangles and ${\text{right}}$ endpoints.

\input{2311-Compute-Integral-0001.HELP.tex}

\[\mbox{Area}\approx\answer{130}\]
\end{problem}}%}

%%%%%%%%%%%%%%%%%%%%%%





\latexProblemContent{
\begin{problem}

Estimate the area under the graph of $f(x)={x^{2} - 64}$ from $x={8}$ to $x={15}$ using ${7}$ rectangles and ${\text{right}}$ endpoints.

\input{2311-Compute-Integral-0001.HELP.tex}

\[\mbox{Area}\approx\answer{588}\]
\end{problem}}%}

%%%%%%%%%%%%%%%%%%%%%%





\latexProblemContent{
\begin{problem}

Estimate the area under the graph of $f(x)={x^{2} - 4}$ from $x={-6}$ to $x={-2}$ using ${4}$ rectangles and ${\text{left}}$ endpoints.

\input{2311-Compute-Integral-0001.HELP.tex}

\[\mbox{Area}\approx\answer{70}\]
\end{problem}}%}

%%%%%%%%%%%%%%%%%%%%%%





\latexProblemContent{
\begin{problem}

Estimate the area under the graph of $f(x)={x^{2} - 4}$ from $x={-2}$ to $x={5}$ using ${7}$ rectangles and ${\text{left}}$ endpoints.

\input{2311-Compute-Integral-0001.HELP.tex}

\[\mbox{Area}\approx\answer{7}\]
\end{problem}}%}

%%%%%%%%%%%%%%%%%%%%%%





\latexProblemContent{
\begin{problem}

Estimate the area under the graph of $f(x)={-x - 3}$ from $x={-6}$ to $x={-3}$ using ${3}$ rectangles and ${\text{right}}$ endpoints.

\input{2311-Compute-Integral-0001.HELP.tex}

\[\mbox{Area}\approx\answer{3}\]
\end{problem}}%}

%%%%%%%%%%%%%%%%%%%%%%





%%%%%%%%%%%%%%%%%%%%%%





%%%%%%%%%%%%%%%%%%%%%%





\latexProblemContent{
\begin{problem}

Estimate the area under the graph of $f(x)={-x^{2} + 25}$ from $x={-5}$ to $x={0}$ using ${5}$ rectangles and ${\text{left}}$ endpoints.

\input{2311-Compute-Integral-0001.HELP.tex}

\[\mbox{Area}\approx\answer{70}\]
\end{problem}}%}

%%%%%%%%%%%%%%%%%%%%%%





%%%%%%%%%%%%%%%%%%%%%%





\latexProblemContent{
\begin{problem}

Estimate the area under the graph of $f(x)={-x^{2} + 16}$ from $x={-4}$ to $x={2}$ using ${6}$ rectangles and ${\text{right}}$ endpoints.

\input{2311-Compute-Integral-0001.HELP.tex}

\[\mbox{Area}\approx\answer{77}\]
\end{problem}}%}

%%%%%%%%%%%%%%%%%%%%%%





\latexProblemContent{
\begin{problem}

Estimate the area under the graph of $f(x)={-x^{2} + 4}$ from $x={-2}$ to $x={2}$ using ${4}$ rectangles and ${\text{left}}$ endpoints.

\input{2311-Compute-Integral-0001.HELP.tex}

\[\mbox{Area}\approx\answer{10}\]
\end{problem}}%}

%%%%%%%%%%%%%%%%%%%%%%





\latexProblemContent{
\begin{problem}

Estimate the area under the graph of $f(x)={x - 2}$ from $x={2}$ to $x={8}$ using ${6}$ rectangles and ${\text{right}}$ endpoints.

\input{2311-Compute-Integral-0001.HELP.tex}

\[\mbox{Area}\approx\answer{21}\]
\end{problem}}%}

%%%%%%%%%%%%%%%%%%%%%%





%%%%%%%%%%%%%%%%%%%%%%





%%%%%%%%%%%%%%%%%%%%%%





%%%%%%%%%%%%%%%%%%%%%%





\latexProblemContent{
\begin{problem}

Estimate the area under the graph of $f(x)={x^{2} - 1}$ from $x={-7}$ to $x={-1}$ using ${6}$ rectangles and ${\text{right}}$ endpoints.

\input{2311-Compute-Integral-0001.HELP.tex}

\[\mbox{Area}\approx\answer{85}\]
\end{problem}}%}

%%%%%%%%%%%%%%%%%%%%%%





%%%%%%%%%%%%%%%%%%%%%%





\latexProblemContent{
\begin{problem}

Estimate the area under the graph of $f(x)={x^{2}}$ from $x={-4}$ to $x={0}$ using ${4}$ rectangles and ${\text{right}}$ endpoints.

\input{2311-Compute-Integral-0001.HELP.tex}

\[\mbox{Area}\approx\answer{14}\]
\end{problem}}%}

%%%%%%%%%%%%%%%%%%%%%%





\latexProblemContent{
\begin{problem}

Estimate the area under the graph of $f(x)={x - 2}$ from $x={2}$ to $x={9}$ using ${7}$ rectangles and ${\text{right}}$ endpoints.

\input{2311-Compute-Integral-0001.HELP.tex}

\[\mbox{Area}\approx\answer{28}\]
\end{problem}}%}

%%%%%%%%%%%%%%%%%%%%%%





\latexProblemContent{
\begin{problem}

Estimate the area under the graph of $f(x)={x + 1}$ from $x={-1}$ to $x={3}$ using ${4}$ rectangles and ${\text{left}}$ endpoints.

\input{2311-Compute-Integral-0001.HELP.tex}

\[\mbox{Area}\approx\answer{6}\]
\end{problem}}%}

%%%%%%%%%%%%%%%%%%%%%%





%%%%%%%%%%%%%%%%%%%%%%





\latexProblemContent{
\begin{problem}

Estimate the area under the graph of $f(x)={x - 4}$ from $x={4}$ to $x={9}$ using ${5}$ rectangles and ${\text{left}}$ endpoints.

\input{2311-Compute-Integral-0001.HELP.tex}

\[\mbox{Area}\approx\answer{10}\]
\end{problem}}%}

%%%%%%%%%%%%%%%%%%%%%%





\latexProblemContent{
\begin{problem}

Estimate the area under the graph of $f(x)={x - 6}$ from $x={6}$ to $x={11}$ using ${5}$ rectangles and ${\text{right}}$ endpoints.

\input{2311-Compute-Integral-0001.HELP.tex}

\[\mbox{Area}\approx\answer{15}\]
\end{problem}}%}

%%%%%%%%%%%%%%%%%%%%%%





\latexProblemContent{
\begin{problem}

Estimate the area under the graph of $f(x)={-x^{2} + 169}$ from $x={8}$ to $x={13}$ using ${5}$ rectangles and ${\text{left}}$ endpoints.

\input{2311-Compute-Integral-0001.HELP.tex}

\[\mbox{Area}\approx\answer{335}\]
\end{problem}}%}

%%%%%%%%%%%%%%%%%%%%%%





%%%%%%%%%%%%%%%%%%%%%%





\latexProblemContent{
\begin{problem}

Estimate the area under the graph of $f(x)={x + 6}$ from $x={-6}$ to $x={0}$ using ${6}$ rectangles and ${\text{right}}$ endpoints.

\input{2311-Compute-Integral-0001.HELP.tex}

\[\mbox{Area}\approx\answer{21}\]
\end{problem}}%}

%%%%%%%%%%%%%%%%%%%%%%





%%%%%%%%%%%%%%%%%%%%%%





%%%%%%%%%%%%%%%%%%%%%%





\latexProblemContent{
\begin{problem}

Estimate the area under the graph of $f(x)={x^{2} - 16}$ from $x={-8}$ to $x={-4}$ using ${4}$ rectangles and ${\text{left}}$ endpoints.

\input{2311-Compute-Integral-0001.HELP.tex}

\[\mbox{Area}\approx\answer{110}\]
\end{problem}}%}

%%%%%%%%%%%%%%%%%%%%%%





\latexProblemContent{
\begin{problem}

Estimate the area under the graph of $f(x)={-x - 1}$ from $x={-8}$ to $x={-1}$ using ${7}$ rectangles and ${\text{left}}$ endpoints.

\input{2311-Compute-Integral-0001.HELP.tex}

\[\mbox{Area}\approx\answer{28}\]
\end{problem}}%}

%%%%%%%%%%%%%%%%%%%%%%





\latexProblemContent{
\begin{problem}

Estimate the area under the graph of $f(x)={x - 3}$ from $x={3}$ to $x={7}$ using ${4}$ rectangles and ${\text{right}}$ endpoints.

\input{2311-Compute-Integral-0001.HELP.tex}

\[\mbox{Area}\approx\answer{10}\]
\end{problem}}%}

%%%%%%%%%%%%%%%%%%%%%%





\latexProblemContent{
\begin{problem}

Estimate the area under the graph of $f(x)={-x^{2} + 49}$ from $x={-7}$ to $x={-3}$ using ${4}$ rectangles and ${\text{left}}$ endpoints.

\input{2311-Compute-Integral-0001.HELP.tex}

\[\mbox{Area}\approx\answer{70}\]
\end{problem}}%}

%%%%%%%%%%%%%%%%%%%%%%





%%%%%%%%%%%%%%%%%%%%%%





\latexProblemContent{
\begin{problem}

Estimate the area under the graph of $f(x)={-x + 5}$ from $x={2}$ to $x={5}$ using ${3}$ rectangles and ${\text{right}}$ endpoints.

\input{2311-Compute-Integral-0001.HELP.tex}

\[\mbox{Area}\approx\answer{3}\]
\end{problem}}%}

%%%%%%%%%%%%%%%%%%%%%%





\latexProblemContent{
\begin{problem}

Estimate the area under the graph of $f(x)={x + 6}$ from $x={-6}$ to $x={-3}$ using ${3}$ rectangles and ${\text{right}}$ endpoints.

\input{2311-Compute-Integral-0001.HELP.tex}

\[\mbox{Area}\approx\answer{6}\]
\end{problem}}%}

%%%%%%%%%%%%%%%%%%%%%%





\latexProblemContent{
\begin{problem}

Estimate the area under the graph of $f(x)={-x + 12}$ from $x={6}$ to $x={12}$ using ${6}$ rectangles and ${\text{right}}$ endpoints.

\input{2311-Compute-Integral-0001.HELP.tex}

\[\mbox{Area}\approx\answer{15}\]
\end{problem}}%}

%%%%%%%%%%%%%%%%%%%%%%





%%%%%%%%%%%%%%%%%%%%%%





\latexProblemContent{
\begin{problem}

Estimate the area under the graph of $f(x)={-x^{2} + 25}$ from $x={-1}$ to $x={5}$ using ${6}$ rectangles and ${\text{right}}$ endpoints.

\input{2311-Compute-Integral-0001.HELP.tex}

\[\mbox{Area}\approx\answer{95}\]
\end{problem}}%}

%%%%%%%%%%%%%%%%%%%%%%





\latexProblemContent{
\begin{problem}

Estimate the area under the graph of $f(x)={-x + 8}$ from $x={4}$ to $x={8}$ using ${4}$ rectangles and ${\text{left}}$ endpoints.

\input{2311-Compute-Integral-0001.HELP.tex}

\[\mbox{Area}\approx\answer{10}\]
\end{problem}}%}

%%%%%%%%%%%%%%%%%%%%%%





%%%%%%%%%%%%%%%%%%%%%%





%%%%%%%%%%%%%%%%%%%%%%





%%%%%%%%%%%%%%%%%%%%%%





%%%%%%%%%%%%%%%%%%%%%%





%%%%%%%%%%%%%%%%%%%%%%





%%%%%%%%%%%%%%%%%%%%%%





\latexProblemContent{
\begin{problem}

Estimate the area under the graph of $f(x)={-x - 2}$ from $x={-5}$ to $x={-2}$ using ${3}$ rectangles and ${\text{left}}$ endpoints.

\input{2311-Compute-Integral-0001.HELP.tex}

\[\mbox{Area}\approx\answer{6}\]
\end{problem}}%}

%%%%%%%%%%%%%%%%%%%%%%




