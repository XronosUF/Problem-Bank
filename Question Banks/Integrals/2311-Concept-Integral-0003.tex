%%%%%%%%%%%%%%%%%%%%%%%
%%\tagged{Cat@One, Cat@Two, Cat@Three, Cat@Four, Cat@Five, Ans@MC, Type@Concept, Topic@Integral, Sub@Definite, Sub@Theorems_FTC, Sub@Trig}{

\latexProblemContent{
\begin{problem}

What is wrong with the following equation:

\[\int_{\frac{1}{4} \, \pi}^{\frac{5}{3} \, \pi} {-4 \, \cot\left(x\right) \csc\left(x\right)}\;dx = {\frac{4}{\sin\left(x\right)}}\Bigg\vert_{\frac{1}{4} \, \pi}^{\frac{5}{3} \, \pi} = {-\frac{8}{3} \, \sqrt{3} - 4 \, \sqrt{2}}\]

\input{2311_Concept_Integral_0003.HELP.tex}

\begin{multipleChoice}
\choice The antiderivative is incorrect.
\choice[correct] The integrand is not defined over the entire interval.
\choice The bounds are evaluated in the wrong order.
\choice Nothing is wrong.  The equation is correct, as is.
\end{multipleChoice}

\end{problem}}%}

%%%%%%%%%%%%%%%%%%%%%%


\latexProblemContent{
\begin{problem}

What is wrong with the following equation:

\[\int_{\frac{3}{4} \, \pi}^{\frac{4}{3} \, \pi} {-9 \, \csc\left(x\right)^{2}}\;dx = {\frac{9}{\tan\left(x\right)}}\Bigg\vert_{\frac{3}{4} \, \pi}^{\frac{4}{3} \, \pi} = {3 \, \sqrt{3} + 9}\]

\input{2311_Concept_Integral_0003.HELP.tex}

\begin{multipleChoice}
\choice The antiderivative is incorrect.
\choice[correct] The integrand is not defined over the entire interval.
\choice The bounds are evaluated in the wrong order.
\choice Nothing is wrong.  The equation is correct, as is.
\end{multipleChoice}

\end{problem}}%}

%%%%%%%%%%%%%%%%%%%%%%


\latexProblemContent{
\begin{problem}

What is wrong with the following equation:

\[\int_{\frac{1}{2} \, \pi}^{\frac{7}{4} \, \pi} {-15 \, \csc\left(x\right)^{2}}\;dx = {\frac{15}{\tan\left(x\right)}}\Bigg\vert_{\frac{1}{2} \, \pi}^{\frac{7}{4} \, \pi} = {-15}\]

\input{2311_Concept_Integral_0003.HELP.tex}

\begin{multipleChoice}
\choice The antiderivative is incorrect.
\choice[correct] The integrand is not defined over the entire interval.
\choice The bounds are evaluated in the wrong order.
\choice Nothing is wrong.  The equation is correct, as is.
\end{multipleChoice}

\end{problem}}%}

%%%%%%%%%%%%%%%%%%%%%%


\latexProblemContent{
\begin{problem}

What is wrong with the following equation:

\[\int_{\frac{2}{3} \, \pi}^{\frac{4}{3} \, \pi} {-12 \, \csc\left(x\right)^{2}}\;dx = {\frac{12}{\tan\left(x\right)}}\Bigg\vert_{\frac{2}{3} \, \pi}^{\frac{4}{3} \, \pi} = {8 \, \sqrt{3}}\]

\input{2311_Concept_Integral_0003.HELP.tex}

\begin{multipleChoice}
\choice The antiderivative is incorrect.
\choice[correct] The integrand is not defined over the entire interval.
\choice The bounds are evaluated in the wrong order.
\choice Nothing is wrong.  The equation is correct, as is.
\end{multipleChoice}

\end{problem}}%}

%%%%%%%%%%%%%%%%%%%%%%


\latexProblemContent{
\begin{problem}

What is wrong with the following equation:

\[\int_{\frac{2}{3} \, \pi}^{\frac{7}{4} \, \pi} {3 \, \csc\left(x\right)^{2}}\;dx = {-\frac{3}{\tan\left(x\right)}}\Bigg\vert_{\frac{2}{3} \, \pi}^{\frac{7}{4} \, \pi} = {-\sqrt{3} + 3}\]

\input{2311_Concept_Integral_0003.HELP.tex}

\begin{multipleChoice}
\choice The antiderivative is incorrect.
\choice[correct] The integrand is not defined over the entire interval.
\choice The bounds are evaluated in the wrong order.
\choice Nothing is wrong.  The equation is correct, as is.
\end{multipleChoice}

\end{problem}}%}

%%%%%%%%%%%%%%%%%%%%%%


\latexProblemContent{
\begin{problem}

What is wrong with the following equation:

\[\int_{\frac{3}{4} \, \pi}^{\frac{5}{4} \, \pi} {5 \, \cot\left(x\right) \csc\left(x\right)}\;dx = {-\frac{5}{\sin\left(x\right)}}\Bigg\vert_{\frac{3}{4} \, \pi}^{\frac{5}{4} \, \pi} = {10 \, \sqrt{2}}\]

\input{2311_Concept_Integral_0003.HELP.tex}

\begin{multipleChoice}
\choice The antiderivative is incorrect.
\choice[correct] The integrand is not defined over the entire interval.
\choice The bounds are evaluated in the wrong order.
\choice Nothing is wrong.  The equation is correct, as is.
\end{multipleChoice}

\end{problem}}%}

%%%%%%%%%%%%%%%%%%%%%%


\latexProblemContent{
\begin{problem}

What is wrong with the following equation:

\[\int_{\frac{1}{2} \, \pi}^{\frac{4}{3} \, \pi} {9 \, \cot\left(x\right) \csc\left(x\right)}\;dx = {-\frac{9}{\sin\left(x\right)}}\Bigg\vert_{\frac{1}{2} \, \pi}^{\frac{4}{3} \, \pi} = {6 \, \sqrt{3} + 9}\]

\input{2311_Concept_Integral_0003.HELP.tex}

\begin{multipleChoice}
\choice The antiderivative is incorrect.
\choice[correct] The integrand is not defined over the entire interval.
\choice The bounds are evaluated in the wrong order.
\choice Nothing is wrong.  The equation is correct, as is.
\end{multipleChoice}

\end{problem}}%}

%%%%%%%%%%%%%%%%%%%%%%


\latexProblemContent{
\begin{problem}

What is wrong with the following equation:

\[\int_{\frac{1}{3} \, \pi}^{\frac{3}{2} \, \pi} {3 \, \csc\left(x\right)^{2}}\;dx = {-\frac{3}{\tan\left(x\right)}}\Bigg\vert_{\frac{1}{3} \, \pi}^{\frac{3}{2} \, \pi} = {\sqrt{3}}\]

\input{2311_Concept_Integral_0003.HELP.tex}

\begin{multipleChoice}
\choice The antiderivative is incorrect.
\choice[correct] The integrand is not defined over the entire interval.
\choice The bounds are evaluated in the wrong order.
\choice Nothing is wrong.  The equation is correct, as is.
\end{multipleChoice}

\end{problem}}%}

%%%%%%%%%%%%%%%%%%%%%%


\latexProblemContent{
\begin{problem}

What is wrong with the following equation:

\[\int_{\frac{1}{3} \, \pi}^{\frac{5}{3} \, \pi} {12 \, \cot\left(x\right) \csc\left(x\right)}\;dx = {-\frac{12}{\sin\left(x\right)}}\Bigg\vert_{\frac{1}{3} \, \pi}^{\frac{5}{3} \, \pi} = {16 \, \sqrt{3}}\]

\input{2311_Concept_Integral_0003.HELP.tex}

\begin{multipleChoice}
\choice The antiderivative is incorrect.
\choice[correct] The integrand is not defined over the entire interval.
\choice The bounds are evaluated in the wrong order.
\choice Nothing is wrong.  The equation is correct, as is.
\end{multipleChoice}

\end{problem}}%}

%%%%%%%%%%%%%%%%%%%%%%


\latexProblemContent{
\begin{problem}

What is wrong with the following equation:

\[\int_{\frac{1}{2} \, \pi}^{\frac{3}{2} \, \pi} {13 \, \cot\left(x\right) \csc\left(x\right)}\;dx = {-\frac{13}{\sin\left(x\right)}}\Bigg\vert_{\frac{1}{2} \, \pi}^{\frac{3}{2} \, \pi} = {26}\]

\input{2311_Concept_Integral_0003.HELP.tex}

\begin{multipleChoice}
\choice The antiderivative is incorrect.
\choice[correct] The integrand is not defined over the entire interval.
\choice The bounds are evaluated in the wrong order.
\choice Nothing is wrong.  The equation is correct, as is.
\end{multipleChoice}

\end{problem}}%}

%%%%%%%%%%%%%%%%%%%%%%


\latexProblemContent{
\begin{problem}

What is wrong with the following equation:

\[\int_{\frac{1}{3} \, \pi}^{\frac{4}{3} \, \pi} {9 \, \cot\left(x\right) \csc\left(x\right)}\;dx = {-\frac{9}{\sin\left(x\right)}}\Bigg\vert_{\frac{1}{3} \, \pi}^{\frac{4}{3} \, \pi} = {12 \, \sqrt{3}}\]

\input{2311_Concept_Integral_0003.HELP.tex}

\begin{multipleChoice}
\choice The antiderivative is incorrect.
\choice[correct] The integrand is not defined over the entire interval.
\choice The bounds are evaluated in the wrong order.
\choice Nothing is wrong.  The equation is correct, as is.
\end{multipleChoice}

\end{problem}}%}

%%%%%%%%%%%%%%%%%%%%%%


\latexProblemContent{
\begin{problem}

What is wrong with the following equation:

\[\int_{\frac{1}{4} \, \pi}^{\frac{5}{4} \, \pi} {-7 \, \csc\left(x\right)^{2}}\;dx = {\frac{7}{\tan\left(x\right)}}\Bigg\vert_{\frac{1}{4} \, \pi}^{\frac{5}{4} \, \pi} = {0}\]

\input{2311_Concept_Integral_0003.HELP.tex}

\begin{multipleChoice}
\choice The antiderivative is incorrect.
\choice[correct] The integrand is not defined over the entire interval.
\choice The bounds are evaluated in the wrong order.
\choice Nothing is wrong.  The equation is correct, as is.
\end{multipleChoice}

\end{problem}}%}

%%%%%%%%%%%%%%%%%%%%%%


\latexProblemContent{
\begin{problem}

What is wrong with the following equation:

\[\int_{\frac{1}{4} \, \pi}^{\frac{7}{4} \, \pi} {4 \, \csc\left(x\right)^{2}}\;dx = {-\frac{4}{\tan\left(x\right)}}\Bigg\vert_{\frac{1}{4} \, \pi}^{\frac{7}{4} \, \pi} = {8}\]

\input{2311_Concept_Integral_0003.HELP.tex}

\begin{multipleChoice}
\choice The antiderivative is incorrect.
\choice[correct] The integrand is not defined over the entire interval.
\choice The bounds are evaluated in the wrong order.
\choice Nothing is wrong.  The equation is correct, as is.
\end{multipleChoice}

\end{problem}}%}

%%%%%%%%%%%%%%%%%%%%%%


\latexProblemContent{
\begin{problem}

What is wrong with the following equation:

\[\int_{\frac{1}{4} \, \pi}^{\frac{4}{3} \, \pi} {-3 \, \csc\left(x\right)^{2}}\;dx = {\frac{3}{\tan\left(x\right)}}\Bigg\vert_{\frac{1}{4} \, \pi}^{\frac{4}{3} \, \pi} = {\sqrt{3} - 3}\]

\input{2311_Concept_Integral_0003.HELP.tex}

\begin{multipleChoice}
\choice The antiderivative is incorrect.
\choice[correct] The integrand is not defined over the entire interval.
\choice The bounds are evaluated in the wrong order.
\choice Nothing is wrong.  The equation is correct, as is.
\end{multipleChoice}

\end{problem}}%}

%%%%%%%%%%%%%%%%%%%%%%


\latexProblemContent{
\begin{problem}

What is wrong with the following equation:

\[\int_{\frac{3}{4} \, \pi}^{\frac{5}{4} \, \pi} {6 \, \csc\left(x\right)^{2}}\;dx = {-\frac{6}{\tan\left(x\right)}}\Bigg\vert_{\frac{3}{4} \, \pi}^{\frac{5}{4} \, \pi} = {-12}\]

\input{2311_Concept_Integral_0003.HELP.tex}

\begin{multipleChoice}
\choice The antiderivative is incorrect.
\choice[correct] The integrand is not defined over the entire interval.
\choice The bounds are evaluated in the wrong order.
\choice Nothing is wrong.  The equation is correct, as is.
\end{multipleChoice}

\end{problem}}%}

%%%%%%%%%%%%%%%%%%%%%%


\latexProblemContent{
\begin{problem}

What is wrong with the following equation:

\[\int_{\frac{1}{4} \, \pi}^{\frac{4}{3} \, \pi} {-8 \, \cot\left(x\right) \csc\left(x\right)}\;dx = {\frac{8}{\sin\left(x\right)}}\Bigg\vert_{\frac{1}{4} \, \pi}^{\frac{4}{3} \, \pi} = {-\frac{16}{3} \, \sqrt{3} - 8 \, \sqrt{2}}\]

\input{2311_Concept_Integral_0003.HELP.tex}

\begin{multipleChoice}
\choice The antiderivative is incorrect.
\choice[correct] The integrand is not defined over the entire interval.
\choice The bounds are evaluated in the wrong order.
\choice Nothing is wrong.  The equation is correct, as is.
\end{multipleChoice}

\end{problem}}%}

%%%%%%%%%%%%%%%%%%%%%%


\latexProblemContent{
\begin{problem}

What is wrong with the following equation:

\[\int_{\frac{1}{4} \, \pi}^{\frac{4}{3} \, \pi} {-7 \, \cot\left(x\right) \csc\left(x\right)}\;dx = {\frac{7}{\sin\left(x\right)}}\Bigg\vert_{\frac{1}{4} \, \pi}^{\frac{4}{3} \, \pi} = {-\frac{14}{3} \, \sqrt{3} - 7 \, \sqrt{2}}\]

\input{2311_Concept_Integral_0003.HELP.tex}

\begin{multipleChoice}
\choice The antiderivative is incorrect.
\choice[correct] The integrand is not defined over the entire interval.
\choice The bounds are evaluated in the wrong order.
\choice Nothing is wrong.  The equation is correct, as is.
\end{multipleChoice}

\end{problem}}%}

%%%%%%%%%%%%%%%%%%%%%%


\latexProblemContent{
\begin{problem}

What is wrong with the following equation:

\[\int_{\frac{1}{3} \, \pi}^{\frac{4}{3} \, \pi} {5 \, \cot\left(x\right) \csc\left(x\right)}\;dx = {-\frac{5}{\sin\left(x\right)}}\Bigg\vert_{\frac{1}{3} \, \pi}^{\frac{4}{3} \, \pi} = {\frac{20}{3} \, \sqrt{3}}\]

\input{2311_Concept_Integral_0003.HELP.tex}

\begin{multipleChoice}
\choice The antiderivative is incorrect.
\choice[correct] The integrand is not defined over the entire interval.
\choice The bounds are evaluated in the wrong order.
\choice Nothing is wrong.  The equation is correct, as is.
\end{multipleChoice}

\end{problem}}%}

%%%%%%%%%%%%%%%%%%%%%%


\latexProblemContent{
\begin{problem}

What is wrong with the following equation:

\[\int_{\frac{3}{4} \, \pi}^{\frac{5}{3} \, \pi} {-13 \, \csc\left(x\right)^{2}}\;dx = {\frac{13}{\tan\left(x\right)}}\Bigg\vert_{\frac{3}{4} \, \pi}^{\frac{5}{3} \, \pi} = {-\frac{13}{3} \, \sqrt{3} + 13}\]

\input{2311_Concept_Integral_0003.HELP.tex}

\begin{multipleChoice}
\choice The antiderivative is incorrect.
\choice[correct] The integrand is not defined over the entire interval.
\choice The bounds are evaluated in the wrong order.
\choice Nothing is wrong.  The equation is correct, as is.
\end{multipleChoice}

\end{problem}}%}

%%%%%%%%%%%%%%%%%%%%%%


\latexProblemContent{
\begin{problem}

What is wrong with the following equation:

\[\int_{\frac{3}{4} \, \pi}^{\frac{7}{4} \, \pi} {-8 \, \csc\left(x\right)^{2}}\;dx = {\frac{8}{\tan\left(x\right)}}\Bigg\vert_{\frac{3}{4} \, \pi}^{\frac{7}{4} \, \pi} = {0}\]

\input{2311_Concept_Integral_0003.HELP.tex}

\begin{multipleChoice}
\choice The antiderivative is incorrect.
\choice[correct] The integrand is not defined over the entire interval.
\choice The bounds are evaluated in the wrong order.
\choice Nothing is wrong.  The equation is correct, as is.
\end{multipleChoice}

\end{problem}}%}

%%%%%%%%%%%%%%%%%%%%%%


%%%%%%%%%%%%%%%%%%%%%%


\latexProblemContent{
\begin{problem}

What is wrong with the following equation:

\[\int_{\frac{1}{3} \, \pi}^{\frac{7}{4} \, \pi} {-\csc\left(x\right)^{2}}\;dx = {\frac{1}{\tan\left(x\right)}}\Bigg\vert_{\frac{1}{3} \, \pi}^{\frac{7}{4} \, \pi} = {-\frac{1}{3} \, \sqrt{3} - 1}\]

\input{2311_Concept_Integral_0003.HELP.tex}

\begin{multipleChoice}
\choice The antiderivative is incorrect.
\choice[correct] The integrand is not defined over the entire interval.
\choice The bounds are evaluated in the wrong order.
\choice Nothing is wrong.  The equation is correct, as is.
\end{multipleChoice}

\end{problem}}%}

%%%%%%%%%%%%%%%%%%%%%%


\latexProblemContent{
\begin{problem}

What is wrong with the following equation:

\[\int_{\frac{1}{3} \, \pi}^{\frac{4}{3} \, \pi} {3 \, \cot\left(x\right) \csc\left(x\right)}\;dx = {-\frac{3}{\sin\left(x\right)}}\Bigg\vert_{\frac{1}{3} \, \pi}^{\frac{4}{3} \, \pi} = {4 \, \sqrt{3}}\]

\input{2311_Concept_Integral_0003.HELP.tex}

\begin{multipleChoice}
\choice The antiderivative is incorrect.
\choice[correct] The integrand is not defined over the entire interval.
\choice The bounds are evaluated in the wrong order.
\choice Nothing is wrong.  The equation is correct, as is.
\end{multipleChoice}

\end{problem}}%}

%%%%%%%%%%%%%%%%%%%%%%


\latexProblemContent{
\begin{problem}

What is wrong with the following equation:

\[\int_{\frac{3}{4} \, \pi}^{\frac{4}{3} \, \pi} {4 \, \cot\left(x\right) \csc\left(x\right)}\;dx = {-\frac{4}{\sin\left(x\right)}}\Bigg\vert_{\frac{3}{4} \, \pi}^{\frac{4}{3} \, \pi} = {\frac{8}{3} \, \sqrt{3} + 4 \, \sqrt{2}}\]

\input{2311_Concept_Integral_0003.HELP.tex}

\begin{multipleChoice}
\choice The antiderivative is incorrect.
\choice[correct] The integrand is not defined over the entire interval.
\choice The bounds are evaluated in the wrong order.
\choice Nothing is wrong.  The equation is correct, as is.
\end{multipleChoice}

\end{problem}}%}

%%%%%%%%%%%%%%%%%%%%%%


\latexProblemContent{
\begin{problem}

What is wrong with the following equation:

\[\int_{\frac{1}{2} \, \pi}^{\frac{5}{4} \, \pi} {3 \, \csc\left(x\right)^{2}}\;dx = {-\frac{3}{\tan\left(x\right)}}\Bigg\vert_{\frac{1}{2} \, \pi}^{\frac{5}{4} \, \pi} = {-3}\]

\input{2311_Concept_Integral_0003.HELP.tex}

\begin{multipleChoice}
\choice The antiderivative is incorrect.
\choice[correct] The integrand is not defined over the entire interval.
\choice The bounds are evaluated in the wrong order.
\choice Nothing is wrong.  The equation is correct, as is.
\end{multipleChoice}

\end{problem}}%}

%%%%%%%%%%%%%%%%%%%%%%


\latexProblemContent{
\begin{problem}

What is wrong with the following equation:

\[\int_{\frac{1}{2} \, \pi}^{\frac{3}{2} \, \pi} {15 \, \csc\left(x\right)^{2}}\;dx = {-\frac{15}{\tan\left(x\right)}}\Bigg\vert_{\frac{1}{2} \, \pi}^{\frac{3}{2} \, \pi} = {0}\]

\input{2311_Concept_Integral_0003.HELP.tex}

\begin{multipleChoice}
\choice The antiderivative is incorrect.
\choice[correct] The integrand is not defined over the entire interval.
\choice The bounds are evaluated in the wrong order.
\choice Nothing is wrong.  The equation is correct, as is.
\end{multipleChoice}

\end{problem}}%}

%%%%%%%%%%%%%%%%%%%%%%


\latexProblemContent{
\begin{problem}

What is wrong with the following equation:

\[\int_{\frac{3}{4} \, \pi}^{\frac{7}{4} \, \pi} {3 \, \cot\left(x\right) \csc\left(x\right)}\;dx = {-\frac{3}{\sin\left(x\right)}}\Bigg\vert_{\frac{3}{4} \, \pi}^{\frac{7}{4} \, \pi} = {6 \, \sqrt{2}}\]

\input{2311_Concept_Integral_0003.HELP.tex}

\begin{multipleChoice}
\choice The antiderivative is incorrect.
\choice[correct] The integrand is not defined over the entire interval.
\choice The bounds are evaluated in the wrong order.
\choice Nothing is wrong.  The equation is correct, as is.
\end{multipleChoice}

\end{problem}}%}

%%%%%%%%%%%%%%%%%%%%%%


\latexProblemContent{
\begin{problem}

What is wrong with the following equation:

\[\int_{\frac{1}{3} \, \pi}^{\frac{4}{3} \, \pi} {2 \, \cot\left(x\right) \csc\left(x\right)}\;dx = {-\frac{2}{\sin\left(x\right)}}\Bigg\vert_{\frac{1}{3} \, \pi}^{\frac{4}{3} \, \pi} = {\frac{8}{3} \, \sqrt{3}}\]

\input{2311_Concept_Integral_0003.HELP.tex}

\begin{multipleChoice}
\choice The antiderivative is incorrect.
\choice[correct] The integrand is not defined over the entire interval.
\choice The bounds are evaluated in the wrong order.
\choice Nothing is wrong.  The equation is correct, as is.
\end{multipleChoice}

\end{problem}}%}

%%%%%%%%%%%%%%%%%%%%%%


\latexProblemContent{
\begin{problem}

What is wrong with the following equation:

\[\int_{\frac{3}{4} \, \pi}^{\frac{4}{3} \, \pi} {-5 \, \cot\left(x\right) \csc\left(x\right)}\;dx = {\frac{5}{\sin\left(x\right)}}\Bigg\vert_{\frac{3}{4} \, \pi}^{\frac{4}{3} \, \pi} = {-\frac{10}{3} \, \sqrt{3} - 5 \, \sqrt{2}}\]

\input{2311_Concept_Integral_0003.HELP.tex}

\begin{multipleChoice}
\choice The antiderivative is incorrect.
\choice[correct] The integrand is not defined over the entire interval.
\choice The bounds are evaluated in the wrong order.
\choice Nothing is wrong.  The equation is correct, as is.
\end{multipleChoice}

\end{problem}}%}

%%%%%%%%%%%%%%%%%%%%%%


\latexProblemContent{
\begin{problem}

What is wrong with the following equation:

\[\int_{\frac{2}{3} \, \pi}^{\frac{5}{3} \, \pi} {6 \, \csc\left(x\right)^{2}}\;dx = {-\frac{6}{\tan\left(x\right)}}\Bigg\vert_{\frac{2}{3} \, \pi}^{\frac{5}{3} \, \pi} = {0}\]

\input{2311_Concept_Integral_0003.HELP.tex}

\begin{multipleChoice}
\choice The antiderivative is incorrect.
\choice[correct] The integrand is not defined over the entire interval.
\choice The bounds are evaluated in the wrong order.
\choice Nothing is wrong.  The equation is correct, as is.
\end{multipleChoice}

\end{problem}}%}

%%%%%%%%%%%%%%%%%%%%%%


\latexProblemContent{
\begin{problem}

What is wrong with the following equation:

\[\int_{\frac{1}{4} \, \pi}^{\frac{5}{4} \, \pi} {-14 \, \cot\left(x\right) \csc\left(x\right)}\;dx = {\frac{14}{\sin\left(x\right)}}\Bigg\vert_{\frac{1}{4} \, \pi}^{\frac{5}{4} \, \pi} = {-28 \, \sqrt{2}}\]

\input{2311_Concept_Integral_0003.HELP.tex}

\begin{multipleChoice}
\choice The antiderivative is incorrect.
\choice[correct] The integrand is not defined over the entire interval.
\choice The bounds are evaluated in the wrong order.
\choice Nothing is wrong.  The equation is correct, as is.
\end{multipleChoice}

\end{problem}}%}

%%%%%%%%%%%%%%%%%%%%%%


\latexProblemContent{
\begin{problem}

What is wrong with the following equation:

\[\int_{\frac{1}{3} \, \pi}^{\frac{3}{2} \, \pi} {-3 \, \cot\left(x\right) \csc\left(x\right)}\;dx = {\frac{3}{\sin\left(x\right)}}\Bigg\vert_{\frac{1}{3} \, \pi}^{\frac{3}{2} \, \pi} = {-2 \, \sqrt{3} - 3}\]

\input{2311_Concept_Integral_0003.HELP.tex}

\begin{multipleChoice}
\choice The antiderivative is incorrect.
\choice[correct] The integrand is not defined over the entire interval.
\choice The bounds are evaluated in the wrong order.
\choice Nothing is wrong.  The equation is correct, as is.
\end{multipleChoice}

\end{problem}}%}

%%%%%%%%%%%%%%%%%%%%%%


\latexProblemContent{
\begin{problem}

What is wrong with the following equation:

\[\int_{\frac{1}{2} \, \pi}^{\frac{3}{2} \, \pi} {10 \, \csc\left(x\right)^{2}}\;dx = {-\frac{10}{\tan\left(x\right)}}\Bigg\vert_{\frac{1}{2} \, \pi}^{\frac{3}{2} \, \pi} = {0}\]

\input{2311_Concept_Integral_0003.HELP.tex}

\begin{multipleChoice}
\choice The antiderivative is incorrect.
\choice[correct] The integrand is not defined over the entire interval.
\choice The bounds are evaluated in the wrong order.
\choice Nothing is wrong.  The equation is correct, as is.
\end{multipleChoice}

\end{problem}}%}

%%%%%%%%%%%%%%%%%%%%%%


\latexProblemContent{
\begin{problem}

What is wrong with the following equation:

\[\int_{\frac{1}{3} \, \pi}^{\frac{5}{4} \, \pi} {-10 \, \cot\left(x\right) \csc\left(x\right)}\;dx = {\frac{10}{\sin\left(x\right)}}\Bigg\vert_{\frac{1}{3} \, \pi}^{\frac{5}{4} \, \pi} = {-\frac{20}{3} \, \sqrt{3} - 10 \, \sqrt{2}}\]

\input{2311_Concept_Integral_0003.HELP.tex}

\begin{multipleChoice}
\choice The antiderivative is incorrect.
\choice[correct] The integrand is not defined over the entire interval.
\choice The bounds are evaluated in the wrong order.
\choice Nothing is wrong.  The equation is correct, as is.
\end{multipleChoice}

\end{problem}}%}

%%%%%%%%%%%%%%%%%%%%%%


\latexProblemContent{
\begin{problem}

What is wrong with the following equation:

\[\int_{\frac{1}{4} \, \pi}^{\frac{5}{4} \, \pi} {7 \, \csc\left(x\right)^{2}}\;dx = {-\frac{7}{\tan\left(x\right)}}\Bigg\vert_{\frac{1}{4} \, \pi}^{\frac{5}{4} \, \pi} = {0}\]

\input{2311_Concept_Integral_0003.HELP.tex}

\begin{multipleChoice}
\choice The antiderivative is incorrect.
\choice[correct] The integrand is not defined over the entire interval.
\choice The bounds are evaluated in the wrong order.
\choice Nothing is wrong.  The equation is correct, as is.
\end{multipleChoice}

\end{problem}}%}

%%%%%%%%%%%%%%%%%%%%%%


\latexProblemContent{
\begin{problem}

What is wrong with the following equation:

\[\int_{\frac{2}{3} \, \pi}^{\frac{4}{3} \, \pi} {-13 \, \cot\left(x\right) \csc\left(x\right)}\;dx = {\frac{13}{\sin\left(x\right)}}\Bigg\vert_{\frac{2}{3} \, \pi}^{\frac{4}{3} \, \pi} = {-\frac{52}{3} \, \sqrt{3}}\]

\input{2311_Concept_Integral_0003.HELP.tex}

\begin{multipleChoice}
\choice The antiderivative is incorrect.
\choice[correct] The integrand is not defined over the entire interval.
\choice The bounds are evaluated in the wrong order.
\choice Nothing is wrong.  The equation is correct, as is.
\end{multipleChoice}

\end{problem}}%}

%%%%%%%%%%%%%%%%%%%%%%


\latexProblemContent{
\begin{problem}

What is wrong with the following equation:

\[\int_{\frac{3}{4} \, \pi}^{\frac{3}{2} \, \pi} {12 \, \csc\left(x\right)^{2}}\;dx = {-\frac{12}{\tan\left(x\right)}}\Bigg\vert_{\frac{3}{4} \, \pi}^{\frac{3}{2} \, \pi} = {-12}\]

\input{2311_Concept_Integral_0003.HELP.tex}

\begin{multipleChoice}
\choice The antiderivative is incorrect.
\choice[correct] The integrand is not defined over the entire interval.
\choice The bounds are evaluated in the wrong order.
\choice Nothing is wrong.  The equation is correct, as is.
\end{multipleChoice}

\end{problem}}%}

%%%%%%%%%%%%%%%%%%%%%%


\latexProblemContent{
\begin{problem}

What is wrong with the following equation:

\[\int_{\frac{1}{4} \, \pi}^{\frac{4}{3} \, \pi} {-6 \, \cot\left(x\right) \csc\left(x\right)}\;dx = {\frac{6}{\sin\left(x\right)}}\Bigg\vert_{\frac{1}{4} \, \pi}^{\frac{4}{3} \, \pi} = {-4 \, \sqrt{3} - 6 \, \sqrt{2}}\]

\input{2311_Concept_Integral_0003.HELP.tex}

\begin{multipleChoice}
\choice The antiderivative is incorrect.
\choice[correct] The integrand is not defined over the entire interval.
\choice The bounds are evaluated in the wrong order.
\choice Nothing is wrong.  The equation is correct, as is.
\end{multipleChoice}

\end{problem}}%}

%%%%%%%%%%%%%%%%%%%%%%


\latexProblemContent{
\begin{problem}

What is wrong with the following equation:

\[\int_{\frac{1}{2} \, \pi}^{\frac{7}{4} \, \pi} {-10 \, \cot\left(x\right) \csc\left(x\right)}\;dx = {\frac{10}{\sin\left(x\right)}}\Bigg\vert_{\frac{1}{2} \, \pi}^{\frac{7}{4} \, \pi} = {-10 \, \sqrt{2} - 10}\]

\input{2311_Concept_Integral_0003.HELP.tex}

\begin{multipleChoice}
\choice The antiderivative is incorrect.
\choice[correct] The integrand is not defined over the entire interval.
\choice The bounds are evaluated in the wrong order.
\choice Nothing is wrong.  The equation is correct, as is.
\end{multipleChoice}

\end{problem}}%}

%%%%%%%%%%%%%%%%%%%%%%


\latexProblemContent{
\begin{problem}

What is wrong with the following equation:

\[\int_{\frac{2}{3} \, \pi}^{\frac{7}{4} \, \pi} {-8 \, \cot\left(x\right) \csc\left(x\right)}\;dx = {\frac{8}{\sin\left(x\right)}}\Bigg\vert_{\frac{2}{3} \, \pi}^{\frac{7}{4} \, \pi} = {-\frac{16}{3} \, \sqrt{3} - 8 \, \sqrt{2}}\]

\input{2311_Concept_Integral_0003.HELP.tex}

\begin{multipleChoice}
\choice The antiderivative is incorrect.
\choice[correct] The integrand is not defined over the entire interval.
\choice The bounds are evaluated in the wrong order.
\choice Nothing is wrong.  The equation is correct, as is.
\end{multipleChoice}

\end{problem}}%}

%%%%%%%%%%%%%%%%%%%%%%


\latexProblemContent{
\begin{problem}

What is wrong with the following equation:

\[\int_{\frac{1}{4} \, \pi}^{\frac{3}{2} \, \pi} {7 \, \csc\left(x\right)^{2}}\;dx = {-\frac{7}{\tan\left(x\right)}}\Bigg\vert_{\frac{1}{4} \, \pi}^{\frac{3}{2} \, \pi} = {7}\]

\input{2311_Concept_Integral_0003.HELP.tex}

\begin{multipleChoice}
\choice The antiderivative is incorrect.
\choice[correct] The integrand is not defined over the entire interval.
\choice The bounds are evaluated in the wrong order.
\choice Nothing is wrong.  The equation is correct, as is.
\end{multipleChoice}

\end{problem}}%}

%%%%%%%%%%%%%%%%%%%%%%


\latexProblemContent{
\begin{problem}

What is wrong with the following equation:

\[\int_{\frac{1}{2} \, \pi}^{\frac{5}{4} \, \pi} {9 \, \csc\left(x\right)^{2}}\;dx = {-\frac{9}{\tan\left(x\right)}}\Bigg\vert_{\frac{1}{2} \, \pi}^{\frac{5}{4} \, \pi} = {-9}\]

\input{2311_Concept_Integral_0003.HELP.tex}

\begin{multipleChoice}
\choice The antiderivative is incorrect.
\choice[correct] The integrand is not defined over the entire interval.
\choice The bounds are evaluated in the wrong order.
\choice Nothing is wrong.  The equation is correct, as is.
\end{multipleChoice}

\end{problem}}%}

%%%%%%%%%%%%%%%%%%%%%%


\latexProblemContent{
\begin{problem}

What is wrong with the following equation:

\[\int_{\frac{1}{3} \, \pi}^{\frac{5}{4} \, \pi} {-15 \, \csc\left(x\right)^{2}}\;dx = {\frac{15}{\tan\left(x\right)}}\Bigg\vert_{\frac{1}{3} \, \pi}^{\frac{5}{4} \, \pi} = {-5 \, \sqrt{3} + 15}\]

\input{2311_Concept_Integral_0003.HELP.tex}

\begin{multipleChoice}
\choice The antiderivative is incorrect.
\choice[correct] The integrand is not defined over the entire interval.
\choice The bounds are evaluated in the wrong order.
\choice Nothing is wrong.  The equation is correct, as is.
\end{multipleChoice}

\end{problem}}%}

%%%%%%%%%%%%%%%%%%%%%%


\latexProblemContent{
\begin{problem}

What is wrong with the following equation:

\[\int_{\frac{1}{2} \, \pi}^{\frac{7}{4} \, \pi} {7 \, \cot\left(x\right) \csc\left(x\right)}\;dx = {-\frac{7}{\sin\left(x\right)}}\Bigg\vert_{\frac{1}{2} \, \pi}^{\frac{7}{4} \, \pi} = {7 \, \sqrt{2} + 7}\]

\input{2311_Concept_Integral_0003.HELP.tex}

\begin{multipleChoice}
\choice The antiderivative is incorrect.
\choice[correct] The integrand is not defined over the entire interval.
\choice The bounds are evaluated in the wrong order.
\choice Nothing is wrong.  The equation is correct, as is.
\end{multipleChoice}

\end{problem}}%}

%%%%%%%%%%%%%%%%%%%%%%


\latexProblemContent{
\begin{problem}

What is wrong with the following equation:

\[\int_{\frac{3}{4} \, \pi}^{\frac{4}{3} \, \pi} {3 \, \cot\left(x\right) \csc\left(x\right)}\;dx = {-\frac{3}{\sin\left(x\right)}}\Bigg\vert_{\frac{3}{4} \, \pi}^{\frac{4}{3} \, \pi} = {2 \, \sqrt{3} + 3 \, \sqrt{2}}\]

\input{2311_Concept_Integral_0003.HELP.tex}

\begin{multipleChoice}
\choice The antiderivative is incorrect.
\choice[correct] The integrand is not defined over the entire interval.
\choice The bounds are evaluated in the wrong order.
\choice Nothing is wrong.  The equation is correct, as is.
\end{multipleChoice}

\end{problem}}%}

%%%%%%%%%%%%%%%%%%%%%%


\latexProblemContent{
\begin{problem}

What is wrong with the following equation:

\[\int_{\frac{2}{3} \, \pi}^{\frac{5}{4} \, \pi} {-\csc\left(x\right)^{2}}\;dx = {\frac{1}{\tan\left(x\right)}}\Bigg\vert_{\frac{2}{3} \, \pi}^{\frac{5}{4} \, \pi} = {\frac{1}{3} \, \sqrt{3} + 1}\]

\input{2311_Concept_Integral_0003.HELP.tex}

\begin{multipleChoice}
\choice The antiderivative is incorrect.
\choice[correct] The integrand is not defined over the entire interval.
\choice The bounds are evaluated in the wrong order.
\choice Nothing is wrong.  The equation is correct, as is.
\end{multipleChoice}

\end{problem}}%}

%%%%%%%%%%%%%%%%%%%%%%


\latexProblemContent{
\begin{problem}

What is wrong with the following equation:

\[\int_{\frac{2}{3} \, \pi}^{\frac{7}{4} \, \pi} {-8 \, \csc\left(x\right)^{2}}\;dx = {\frac{8}{\tan\left(x\right)}}\Bigg\vert_{\frac{2}{3} \, \pi}^{\frac{7}{4} \, \pi} = {\frac{8}{3} \, \sqrt{3} - 8}\]

\input{2311_Concept_Integral_0003.HELP.tex}

\begin{multipleChoice}
\choice The antiderivative is incorrect.
\choice[correct] The integrand is not defined over the entire interval.
\choice The bounds are evaluated in the wrong order.
\choice Nothing is wrong.  The equation is correct, as is.
\end{multipleChoice}

\end{problem}}%}

%%%%%%%%%%%%%%%%%%%%%%


\latexProblemContent{
\begin{problem}

What is wrong with the following equation:

\[\int_{\frac{1}{4} \, \pi}^{\frac{5}{4} \, \pi} {-14 \, \csc\left(x\right)^{2}}\;dx = {\frac{14}{\tan\left(x\right)}}\Bigg\vert_{\frac{1}{4} \, \pi}^{\frac{5}{4} \, \pi} = {0}\]

\input{2311_Concept_Integral_0003.HELP.tex}

\begin{multipleChoice}
\choice The antiderivative is incorrect.
\choice[correct] The integrand is not defined over the entire interval.
\choice The bounds are evaluated in the wrong order.
\choice Nothing is wrong.  The equation is correct, as is.
\end{multipleChoice}

\end{problem}}%}

%%%%%%%%%%%%%%%%%%%%%%


\latexProblemContent{
\begin{problem}

What is wrong with the following equation:

\[\int_{\frac{2}{3} \, \pi}^{\frac{3}{2} \, \pi} {\csc\left(x\right)^{2}}\;dx = {-\frac{1}{\tan\left(x\right)}}\Bigg\vert_{\frac{2}{3} \, \pi}^{\frac{3}{2} \, \pi} = {-\frac{1}{3} \, \sqrt{3}}\]

\input{2311_Concept_Integral_0003.HELP.tex}

\begin{multipleChoice}
\choice The antiderivative is incorrect.
\choice[correct] The integrand is not defined over the entire interval.
\choice The bounds are evaluated in the wrong order.
\choice Nothing is wrong.  The equation is correct, as is.
\end{multipleChoice}

\end{problem}}%}

%%%%%%%%%%%%%%%%%%%%%%


%%%%%%%%%%%%%%%%%%%%%%


\latexProblemContent{
\begin{problem}

What is wrong with the following equation:

\[\int_{\frac{1}{2} \, \pi}^{\frac{5}{4} \, \pi} {11 \, \csc\left(x\right)^{2}}\;dx = {-\frac{11}{\tan\left(x\right)}}\Bigg\vert_{\frac{1}{2} \, \pi}^{\frac{5}{4} \, \pi} = {-11}\]

\input{2311_Concept_Integral_0003.HELP.tex}

\begin{multipleChoice}
\choice The antiderivative is incorrect.
\choice[correct] The integrand is not defined over the entire interval.
\choice The bounds are evaluated in the wrong order.
\choice Nothing is wrong.  The equation is correct, as is.
\end{multipleChoice}

\end{problem}}%}

%%%%%%%%%%%%%%%%%%%%%%


\latexProblemContent{
\begin{problem}

What is wrong with the following equation:

\[\int_{\frac{3}{4} \, \pi}^{\frac{7}{4} \, \pi} {2 \, \cot\left(x\right) \csc\left(x\right)}\;dx = {-\frac{2}{\sin\left(x\right)}}\Bigg\vert_{\frac{3}{4} \, \pi}^{\frac{7}{4} \, \pi} = {4 \, \sqrt{2}}\]

\input{2311_Concept_Integral_0003.HELP.tex}

\begin{multipleChoice}
\choice The antiderivative is incorrect.
\choice[correct] The integrand is not defined over the entire interval.
\choice The bounds are evaluated in the wrong order.
\choice Nothing is wrong.  The equation is correct, as is.
\end{multipleChoice}

\end{problem}}%}

%%%%%%%%%%%%%%%%%%%%%%


\latexProblemContent{
\begin{problem}

What is wrong with the following equation:

\[\int_{\frac{3}{4} \, \pi}^{\frac{5}{3} \, \pi} {9 \, \csc\left(x\right)^{2}}\;dx = {-\frac{9}{\tan\left(x\right)}}\Bigg\vert_{\frac{3}{4} \, \pi}^{\frac{5}{3} \, \pi} = {3 \, \sqrt{3} - 9}\]

\input{2311_Concept_Integral_0003.HELP.tex}

\begin{multipleChoice}
\choice The antiderivative is incorrect.
\choice[correct] The integrand is not defined over the entire interval.
\choice The bounds are evaluated in the wrong order.
\choice Nothing is wrong.  The equation is correct, as is.
\end{multipleChoice}

\end{problem}}%}

%%%%%%%%%%%%%%%%%%%%%%


\latexProblemContent{
\begin{problem}

What is wrong with the following equation:

\[\int_{\frac{1}{3} \, \pi}^{\frac{4}{3} \, \pi} {11 \, \cot\left(x\right) \csc\left(x\right)}\;dx = {-\frac{11}{\sin\left(x\right)}}\Bigg\vert_{\frac{1}{3} \, \pi}^{\frac{4}{3} \, \pi} = {\frac{44}{3} \, \sqrt{3}}\]

\input{2311_Concept_Integral_0003.HELP.tex}

\begin{multipleChoice}
\choice The antiderivative is incorrect.
\choice[correct] The integrand is not defined over the entire interval.
\choice The bounds are evaluated in the wrong order.
\choice Nothing is wrong.  The equation is correct, as is.
\end{multipleChoice}

\end{problem}}%}

%%%%%%%%%%%%%%%%%%%%%%


\latexProblemContent{
\begin{problem}

What is wrong with the following equation:

\[\int_{\frac{1}{2} \, \pi}^{\frac{5}{4} \, \pi} {\csc\left(x\right)^{2}}\;dx = {-\frac{1}{\tan\left(x\right)}}\Bigg\vert_{\frac{1}{2} \, \pi}^{\frac{5}{4} \, \pi} = {-1}\]

\input{2311_Concept_Integral_0003.HELP.tex}

\begin{multipleChoice}
\choice The antiderivative is incorrect.
\choice[correct] The integrand is not defined over the entire interval.
\choice The bounds are evaluated in the wrong order.
\choice Nothing is wrong.  The equation is correct, as is.
\end{multipleChoice}

\end{problem}}%}

%%%%%%%%%%%%%%%%%%%%%%


\latexProblemContent{
\begin{problem}

What is wrong with the following equation:

\[\int_{\frac{1}{2} \, \pi}^{\frac{4}{3} \, \pi} {-10 \, \cot\left(x\right) \csc\left(x\right)}\;dx = {\frac{10}{\sin\left(x\right)}}\Bigg\vert_{\frac{1}{2} \, \pi}^{\frac{4}{3} \, \pi} = {-\frac{20}{3} \, \sqrt{3} - 10}\]

\input{2311_Concept_Integral_0003.HELP.tex}

\begin{multipleChoice}
\choice The antiderivative is incorrect.
\choice[correct] The integrand is not defined over the entire interval.
\choice The bounds are evaluated in the wrong order.
\choice Nothing is wrong.  The equation is correct, as is.
\end{multipleChoice}

\end{problem}}%}

%%%%%%%%%%%%%%%%%%%%%%


\latexProblemContent{
\begin{problem}

What is wrong with the following equation:

\[\int_{\frac{1}{3} \, \pi}^{\frac{3}{2} \, \pi} {2 \, \csc\left(x\right)^{2}}\;dx = {-\frac{2}{\tan\left(x\right)}}\Bigg\vert_{\frac{1}{3} \, \pi}^{\frac{3}{2} \, \pi} = {\frac{2}{3} \, \sqrt{3}}\]

\input{2311_Concept_Integral_0003.HELP.tex}

\begin{multipleChoice}
\choice The antiderivative is incorrect.
\choice[correct] The integrand is not defined over the entire interval.
\choice The bounds are evaluated in the wrong order.
\choice Nothing is wrong.  The equation is correct, as is.
\end{multipleChoice}

\end{problem}}%}

%%%%%%%%%%%%%%%%%%%%%%


\latexProblemContent{
\begin{problem}

What is wrong with the following equation:

\[\int_{\frac{1}{3} \, \pi}^{\frac{3}{2} \, \pi} {6 \, \csc\left(x\right)^{2}}\;dx = {-\frac{6}{\tan\left(x\right)}}\Bigg\vert_{\frac{1}{3} \, \pi}^{\frac{3}{2} \, \pi} = {2 \, \sqrt{3}}\]

\input{2311_Concept_Integral_0003.HELP.tex}

\begin{multipleChoice}
\choice The antiderivative is incorrect.
\choice[correct] The integrand is not defined over the entire interval.
\choice The bounds are evaluated in the wrong order.
\choice Nothing is wrong.  The equation is correct, as is.
\end{multipleChoice}

\end{problem}}%}

%%%%%%%%%%%%%%%%%%%%%%


\latexProblemContent{
\begin{problem}

What is wrong with the following equation:

\[\int_{\frac{3}{4} \, \pi}^{\frac{3}{2} \, \pi} {2 \, \cot\left(x\right) \csc\left(x\right)}\;dx = {-\frac{2}{\sin\left(x\right)}}\Bigg\vert_{\frac{3}{4} \, \pi}^{\frac{3}{2} \, \pi} = {2 \, \sqrt{2} + 2}\]

\input{2311_Concept_Integral_0003.HELP.tex}

\begin{multipleChoice}
\choice The antiderivative is incorrect.
\choice[correct] The integrand is not defined over the entire interval.
\choice The bounds are evaluated in the wrong order.
\choice Nothing is wrong.  The equation is correct, as is.
\end{multipleChoice}

\end{problem}}%}

%%%%%%%%%%%%%%%%%%%%%%


\latexProblemContent{
\begin{problem}

What is wrong with the following equation:

\[\int_{\frac{3}{4} \, \pi}^{\frac{4}{3} \, \pi} {-12 \, \csc\left(x\right)^{2}}\;dx = {\frac{12}{\tan\left(x\right)}}\Bigg\vert_{\frac{3}{4} \, \pi}^{\frac{4}{3} \, \pi} = {4 \, \sqrt{3} + 12}\]

\input{2311_Concept_Integral_0003.HELP.tex}

\begin{multipleChoice}
\choice The antiderivative is incorrect.
\choice[correct] The integrand is not defined over the entire interval.
\choice The bounds are evaluated in the wrong order.
\choice Nothing is wrong.  The equation is correct, as is.
\end{multipleChoice}

\end{problem}}%}

%%%%%%%%%%%%%%%%%%%%%%


\latexProblemContent{
\begin{problem}

What is wrong with the following equation:

\[\int_{\frac{3}{4} \, \pi}^{\frac{5}{3} \, \pi} {\cot\left(x\right) \csc\left(x\right)}\;dx = {-\frac{1}{\sin\left(x\right)}}\Bigg\vert_{\frac{3}{4} \, \pi}^{\frac{5}{3} \, \pi} = {\frac{2}{3} \, \sqrt{3} + \sqrt{2}}\]

\input{2311_Concept_Integral_0003.HELP.tex}

\begin{multipleChoice}
\choice The antiderivative is incorrect.
\choice[correct] The integrand is not defined over the entire interval.
\choice The bounds are evaluated in the wrong order.
\choice Nothing is wrong.  The equation is correct, as is.
\end{multipleChoice}

\end{problem}}%}

%%%%%%%%%%%%%%%%%%%%%%


\latexProblemContent{
\begin{problem}

What is wrong with the following equation:

\[\int_{\frac{2}{3} \, \pi}^{\frac{5}{3} \, \pi} {13 \, \csc\left(x\right)^{2}}\;dx = {-\frac{13}{\tan\left(x\right)}}\Bigg\vert_{\frac{2}{3} \, \pi}^{\frac{5}{3} \, \pi} = {0}\]

\input{2311_Concept_Integral_0003.HELP.tex}

\begin{multipleChoice}
\choice The antiderivative is incorrect.
\choice[correct] The integrand is not defined over the entire interval.
\choice The bounds are evaluated in the wrong order.
\choice Nothing is wrong.  The equation is correct, as is.
\end{multipleChoice}

\end{problem}}%}

%%%%%%%%%%%%%%%%%%%%%%


\latexProblemContent{
\begin{problem}

What is wrong with the following equation:

\[\int_{\frac{1}{2} \, \pi}^{\frac{4}{3} \, \pi} {-7 \, \csc\left(x\right)^{2}}\;dx = {\frac{7}{\tan\left(x\right)}}\Bigg\vert_{\frac{1}{2} \, \pi}^{\frac{4}{3} \, \pi} = {\frac{7}{3} \, \sqrt{3}}\]

\input{2311_Concept_Integral_0003.HELP.tex}

\begin{multipleChoice}
\choice The antiderivative is incorrect.
\choice[correct] The integrand is not defined over the entire interval.
\choice The bounds are evaluated in the wrong order.
\choice Nothing is wrong.  The equation is correct, as is.
\end{multipleChoice}

\end{problem}}%}

%%%%%%%%%%%%%%%%%%%%%%


\latexProblemContent{
\begin{problem}

What is wrong with the following equation:

\[\int_{\frac{1}{3} \, \pi}^{\frac{5}{4} \, \pi} {-2 \, \cot\left(x\right) \csc\left(x\right)}\;dx = {\frac{2}{\sin\left(x\right)}}\Bigg\vert_{\frac{1}{3} \, \pi}^{\frac{5}{4} \, \pi} = {-\frac{4}{3} \, \sqrt{3} - 2 \, \sqrt{2}}\]

\input{2311_Concept_Integral_0003.HELP.tex}

\begin{multipleChoice}
\choice The antiderivative is incorrect.
\choice[correct] The integrand is not defined over the entire interval.
\choice The bounds are evaluated in the wrong order.
\choice Nothing is wrong.  The equation is correct, as is.
\end{multipleChoice}

\end{problem}}%}

%%%%%%%%%%%%%%%%%%%%%%


\latexProblemContent{
\begin{problem}

What is wrong with the following equation:

\[\int_{\frac{1}{4} \, \pi}^{\frac{3}{2} \, \pi} {-10 \, \cot\left(x\right) \csc\left(x\right)}\;dx = {\frac{10}{\sin\left(x\right)}}\Bigg\vert_{\frac{1}{4} \, \pi}^{\frac{3}{2} \, \pi} = {-10 \, \sqrt{2} - 10}\]

\input{2311_Concept_Integral_0003.HELP.tex}

\begin{multipleChoice}
\choice The antiderivative is incorrect.
\choice[correct] The integrand is not defined over the entire interval.
\choice The bounds are evaluated in the wrong order.
\choice Nothing is wrong.  The equation is correct, as is.
\end{multipleChoice}

\end{problem}}%}

%%%%%%%%%%%%%%%%%%%%%%


\latexProblemContent{
\begin{problem}

What is wrong with the following equation:

\[\int_{\frac{1}{4} \, \pi}^{\frac{7}{4} \, \pi} {5 \, \csc\left(x\right)^{2}}\;dx = {-\frac{5}{\tan\left(x\right)}}\Bigg\vert_{\frac{1}{4} \, \pi}^{\frac{7}{4} \, \pi} = {10}\]

\input{2311_Concept_Integral_0003.HELP.tex}

\begin{multipleChoice}
\choice The antiderivative is incorrect.
\choice[correct] The integrand is not defined over the entire interval.
\choice The bounds are evaluated in the wrong order.
\choice Nothing is wrong.  The equation is correct, as is.
\end{multipleChoice}

\end{problem}}%}

%%%%%%%%%%%%%%%%%%%%%%


\latexProblemContent{
\begin{problem}

What is wrong with the following equation:

\[\int_{\frac{1}{4} \, \pi}^{\frac{4}{3} \, \pi} {7 \, \csc\left(x\right)^{2}}\;dx = {-\frac{7}{\tan\left(x\right)}}\Bigg\vert_{\frac{1}{4} \, \pi}^{\frac{4}{3} \, \pi} = {-\frac{7}{3} \, \sqrt{3} + 7}\]

\input{2311_Concept_Integral_0003.HELP.tex}

\begin{multipleChoice}
\choice The antiderivative is incorrect.
\choice[correct] The integrand is not defined over the entire interval.
\choice The bounds are evaluated in the wrong order.
\choice Nothing is wrong.  The equation is correct, as is.
\end{multipleChoice}

\end{problem}}%}

%%%%%%%%%%%%%%%%%%%%%%


\latexProblemContent{
\begin{problem}

What is wrong with the following equation:

\[\int_{\frac{1}{4} \, \pi}^{\frac{4}{3} \, \pi} {2 \, \cot\left(x\right) \csc\left(x\right)}\;dx = {-\frac{2}{\sin\left(x\right)}}\Bigg\vert_{\frac{1}{4} \, \pi}^{\frac{4}{3} \, \pi} = {\frac{4}{3} \, \sqrt{3} + 2 \, \sqrt{2}}\]

\input{2311_Concept_Integral_0003.HELP.tex}

\begin{multipleChoice}
\choice The antiderivative is incorrect.
\choice[correct] The integrand is not defined over the entire interval.
\choice The bounds are evaluated in the wrong order.
\choice Nothing is wrong.  The equation is correct, as is.
\end{multipleChoice}

\end{problem}}%}

%%%%%%%%%%%%%%%%%%%%%%


\latexProblemContent{
\begin{problem}

What is wrong with the following equation:

\[\int_{\frac{3}{4} \, \pi}^{\frac{5}{4} \, \pi} {-8 \, \csc\left(x\right)^{2}}\;dx = {\frac{8}{\tan\left(x\right)}}\Bigg\vert_{\frac{3}{4} \, \pi}^{\frac{5}{4} \, \pi} = {16}\]

\input{2311_Concept_Integral_0003.HELP.tex}

\begin{multipleChoice}
\choice The antiderivative is incorrect.
\choice[correct] The integrand is not defined over the entire interval.
\choice The bounds are evaluated in the wrong order.
\choice Nothing is wrong.  The equation is correct, as is.
\end{multipleChoice}

\end{problem}}%}

%%%%%%%%%%%%%%%%%%%%%%


\latexProblemContent{
\begin{problem}

What is wrong with the following equation:

\[\int_{\frac{2}{3} \, \pi}^{\frac{5}{3} \, \pi} {15 \, \cot\left(x\right) \csc\left(x\right)}\;dx = {-\frac{15}{\sin\left(x\right)}}\Bigg\vert_{\frac{2}{3} \, \pi}^{\frac{5}{3} \, \pi} = {20 \, \sqrt{3}}\]

\input{2311_Concept_Integral_0003.HELP.tex}

\begin{multipleChoice}
\choice The antiderivative is incorrect.
\choice[correct] The integrand is not defined over the entire interval.
\choice The bounds are evaluated in the wrong order.
\choice Nothing is wrong.  The equation is correct, as is.
\end{multipleChoice}

\end{problem}}%}

%%%%%%%%%%%%%%%%%%%%%%


\latexProblemContent{
\begin{problem}

What is wrong with the following equation:

\[\int_{\frac{1}{4} \, \pi}^{\frac{5}{4} \, \pi} {-6 \, \cot\left(x\right) \csc\left(x\right)}\;dx = {\frac{6}{\sin\left(x\right)}}\Bigg\vert_{\frac{1}{4} \, \pi}^{\frac{5}{4} \, \pi} = {-12 \, \sqrt{2}}\]

\input{2311_Concept_Integral_0003.HELP.tex}

\begin{multipleChoice}
\choice The antiderivative is incorrect.
\choice[correct] The integrand is not defined over the entire interval.
\choice The bounds are evaluated in the wrong order.
\choice Nothing is wrong.  The equation is correct, as is.
\end{multipleChoice}

\end{problem}}%}

%%%%%%%%%%%%%%%%%%%%%%


\latexProblemContent{
\begin{problem}

What is wrong with the following equation:

\[\int_{\frac{2}{3} \, \pi}^{\frac{5}{4} \, \pi} {-8 \, \cot\left(x\right) \csc\left(x\right)}\;dx = {\frac{8}{\sin\left(x\right)}}\Bigg\vert_{\frac{2}{3} \, \pi}^{\frac{5}{4} \, \pi} = {-\frac{16}{3} \, \sqrt{3} - 8 \, \sqrt{2}}\]

\input{2311_Concept_Integral_0003.HELP.tex}

\begin{multipleChoice}
\choice The antiderivative is incorrect.
\choice[correct] The integrand is not defined over the entire interval.
\choice The bounds are evaluated in the wrong order.
\choice Nothing is wrong.  The equation is correct, as is.
\end{multipleChoice}

\end{problem}}%}

%%%%%%%%%%%%%%%%%%%%%%


\latexProblemContent{
\begin{problem}

What is wrong with the following equation:

\[\int_{\frac{1}{4} \, \pi}^{\frac{7}{4} \, \pi} {15 \, \cot\left(x\right) \csc\left(x\right)}\;dx = {-\frac{15}{\sin\left(x\right)}}\Bigg\vert_{\frac{1}{4} \, \pi}^{\frac{7}{4} \, \pi} = {30 \, \sqrt{2}}\]

\input{2311_Concept_Integral_0003.HELP.tex}

\begin{multipleChoice}
\choice The antiderivative is incorrect.
\choice[correct] The integrand is not defined over the entire interval.
\choice The bounds are evaluated in the wrong order.
\choice Nothing is wrong.  The equation is correct, as is.
\end{multipleChoice}

\end{problem}}%}

%%%%%%%%%%%%%%%%%%%%%%


\latexProblemContent{
\begin{problem}

What is wrong with the following equation:

\[\int_{\frac{2}{3} \, \pi}^{\frac{4}{3} \, \pi} {3 \, \csc\left(x\right)^{2}}\;dx = {-\frac{3}{\tan\left(x\right)}}\Bigg\vert_{\frac{2}{3} \, \pi}^{\frac{4}{3} \, \pi} = {-2 \, \sqrt{3}}\]

\input{2311_Concept_Integral_0003.HELP.tex}

\begin{multipleChoice}
\choice The antiderivative is incorrect.
\choice[correct] The integrand is not defined over the entire interval.
\choice The bounds are evaluated in the wrong order.
\choice Nothing is wrong.  The equation is correct, as is.
\end{multipleChoice}

\end{problem}}%}

%%%%%%%%%%%%%%%%%%%%%%


\latexProblemContent{
\begin{problem}

What is wrong with the following equation:

\[\int_{\frac{1}{2} \, \pi}^{\frac{3}{2} \, \pi} {-\cot\left(x\right) \csc\left(x\right)}\;dx = {\frac{1}{\sin\left(x\right)}}\Bigg\vert_{\frac{1}{2} \, \pi}^{\frac{3}{2} \, \pi} = {-2}\]

\input{2311_Concept_Integral_0003.HELP.tex}

\begin{multipleChoice}
\choice The antiderivative is incorrect.
\choice[correct] The integrand is not defined over the entire interval.
\choice The bounds are evaluated in the wrong order.
\choice Nothing is wrong.  The equation is correct, as is.
\end{multipleChoice}

\end{problem}}%}

%%%%%%%%%%%%%%%%%%%%%%


\latexProblemContent{
\begin{problem}

What is wrong with the following equation:

\[\int_{\frac{3}{4} \, \pi}^{\frac{7}{4} \, \pi} {2 \, \csc\left(x\right)^{2}}\;dx = {-\frac{2}{\tan\left(x\right)}}\Bigg\vert_{\frac{3}{4} \, \pi}^{\frac{7}{4} \, \pi} = {0}\]

\input{2311_Concept_Integral_0003.HELP.tex}

\begin{multipleChoice}
\choice The antiderivative is incorrect.
\choice[correct] The integrand is not defined over the entire interval.
\choice The bounds are evaluated in the wrong order.
\choice Nothing is wrong.  The equation is correct, as is.
\end{multipleChoice}

\end{problem}}%}

%%%%%%%%%%%%%%%%%%%%%%


\latexProblemContent{
\begin{problem}

What is wrong with the following equation:

\[\int_{\frac{3}{4} \, \pi}^{\frac{5}{4} \, \pi} {-5 \, \csc\left(x\right)^{2}}\;dx = {\frac{5}{\tan\left(x\right)}}\Bigg\vert_{\frac{3}{4} \, \pi}^{\frac{5}{4} \, \pi} = {10}\]

\input{2311_Concept_Integral_0003.HELP.tex}

\begin{multipleChoice}
\choice The antiderivative is incorrect.
\choice[correct] The integrand is not defined over the entire interval.
\choice The bounds are evaluated in the wrong order.
\choice Nothing is wrong.  The equation is correct, as is.
\end{multipleChoice}

\end{problem}}%}

%%%%%%%%%%%%%%%%%%%%%%


\latexProblemContent{
\begin{problem}

What is wrong with the following equation:

\[\int_{\frac{1}{2} \, \pi}^{\frac{7}{4} \, \pi} {9 \, \csc\left(x\right)^{2}}\;dx = {-\frac{9}{\tan\left(x\right)}}\Bigg\vert_{\frac{1}{2} \, \pi}^{\frac{7}{4} \, \pi} = {9}\]

\input{2311_Concept_Integral_0003.HELP.tex}

\begin{multipleChoice}
\choice The antiderivative is incorrect.
\choice[correct] The integrand is not defined over the entire interval.
\choice The bounds are evaluated in the wrong order.
\choice Nothing is wrong.  The equation is correct, as is.
\end{multipleChoice}

\end{problem}}%}

%%%%%%%%%%%%%%%%%%%%%%


\latexProblemContent{
\begin{problem}

What is wrong with the following equation:

\[\int_{\frac{1}{3} \, \pi}^{\frac{5}{4} \, \pi} {12 \, \csc\left(x\right)^{2}}\;dx = {-\frac{12}{\tan\left(x\right)}}\Bigg\vert_{\frac{1}{3} \, \pi}^{\frac{5}{4} \, \pi} = {4 \, \sqrt{3} - 12}\]

\input{2311_Concept_Integral_0003.HELP.tex}

\begin{multipleChoice}
\choice The antiderivative is incorrect.
\choice[correct] The integrand is not defined over the entire interval.
\choice The bounds are evaluated in the wrong order.
\choice Nothing is wrong.  The equation is correct, as is.
\end{multipleChoice}

\end{problem}}%}

%%%%%%%%%%%%%%%%%%%%%%


\latexProblemContent{
\begin{problem}

What is wrong with the following equation:

\[\int_{\frac{1}{3} \, \pi}^{\frac{5}{4} \, \pi} {-8 \, \cot\left(x\right) \csc\left(x\right)}\;dx = {\frac{8}{\sin\left(x\right)}}\Bigg\vert_{\frac{1}{3} \, \pi}^{\frac{5}{4} \, \pi} = {-\frac{16}{3} \, \sqrt{3} - 8 \, \sqrt{2}}\]

\input{2311_Concept_Integral_0003.HELP.tex}

\begin{multipleChoice}
\choice The antiderivative is incorrect.
\choice[correct] The integrand is not defined over the entire interval.
\choice The bounds are evaluated in the wrong order.
\choice Nothing is wrong.  The equation is correct, as is.
\end{multipleChoice}

\end{problem}}%}

%%%%%%%%%%%%%%%%%%%%%%


\latexProblemContent{
\begin{problem}

What is wrong with the following equation:

\[\int_{\frac{1}{3} \, \pi}^{\frac{5}{3} \, \pi} {-9 \, \csc\left(x\right)^{2}}\;dx = {\frac{9}{\tan\left(x\right)}}\Bigg\vert_{\frac{1}{3} \, \pi}^{\frac{5}{3} \, \pi} = {-6 \, \sqrt{3}}\]

\input{2311_Concept_Integral_0003.HELP.tex}

\begin{multipleChoice}
\choice The antiderivative is incorrect.
\choice[correct] The integrand is not defined over the entire interval.
\choice The bounds are evaluated in the wrong order.
\choice Nothing is wrong.  The equation is correct, as is.
\end{multipleChoice}

\end{problem}}%}

%%%%%%%%%%%%%%%%%%%%%%


\latexProblemContent{
\begin{problem}

What is wrong with the following equation:

\[\int_{\frac{1}{3} \, \pi}^{\frac{7}{4} \, \pi} {5 \, \csc\left(x\right)^{2}}\;dx = {-\frac{5}{\tan\left(x\right)}}\Bigg\vert_{\frac{1}{3} \, \pi}^{\frac{7}{4} \, \pi} = {\frac{5}{3} \, \sqrt{3} + 5}\]

\input{2311_Concept_Integral_0003.HELP.tex}

\begin{multipleChoice}
\choice The antiderivative is incorrect.
\choice[correct] The integrand is not defined over the entire interval.
\choice The bounds are evaluated in the wrong order.
\choice Nothing is wrong.  The equation is correct, as is.
\end{multipleChoice}

\end{problem}}%}

%%%%%%%%%%%%%%%%%%%%%%


\latexProblemContent{
\begin{problem}

What is wrong with the following equation:

\[\int_{\frac{2}{3} \, \pi}^{\frac{7}{4} \, \pi} {13 \, \cot\left(x\right) \csc\left(x\right)}\;dx = {-\frac{13}{\sin\left(x\right)}}\Bigg\vert_{\frac{2}{3} \, \pi}^{\frac{7}{4} \, \pi} = {\frac{26}{3} \, \sqrt{3} + 13 \, \sqrt{2}}\]

\input{2311_Concept_Integral_0003.HELP.tex}

\begin{multipleChoice}
\choice The antiderivative is incorrect.
\choice[correct] The integrand is not defined over the entire interval.
\choice The bounds are evaluated in the wrong order.
\choice Nothing is wrong.  The equation is correct, as is.
\end{multipleChoice}

\end{problem}}%}

%%%%%%%%%%%%%%%%%%%%%%


%%%%%%%%%%%%%%%%%%%%%%


\latexProblemContent{
\begin{problem}

What is wrong with the following equation:

\[\int_{\frac{3}{4} \, \pi}^{\frac{3}{2} \, \pi} {-14 \, \cot\left(x\right) \csc\left(x\right)}\;dx = {\frac{14}{\sin\left(x\right)}}\Bigg\vert_{\frac{3}{4} \, \pi}^{\frac{3}{2} \, \pi} = {-14 \, \sqrt{2} - 14}\]

\input{2311_Concept_Integral_0003.HELP.tex}

\begin{multipleChoice}
\choice The antiderivative is incorrect.
\choice[correct] The integrand is not defined over the entire interval.
\choice The bounds are evaluated in the wrong order.
\choice Nothing is wrong.  The equation is correct, as is.
\end{multipleChoice}

\end{problem}}%}

%%%%%%%%%%%%%%%%%%%%%%


\latexProblemContent{
\begin{problem}

What is wrong with the following equation:

\[\int_{\frac{2}{3} \, \pi}^{\frac{3}{2} \, \pi} {4 \, \csc\left(x\right)^{2}}\;dx = {-\frac{4}{\tan\left(x\right)}}\Bigg\vert_{\frac{2}{3} \, \pi}^{\frac{3}{2} \, \pi} = {-\frac{4}{3} \, \sqrt{3}}\]

\input{2311_Concept_Integral_0003.HELP.tex}

\begin{multipleChoice}
\choice The antiderivative is incorrect.
\choice[correct] The integrand is not defined over the entire interval.
\choice The bounds are evaluated in the wrong order.
\choice Nothing is wrong.  The equation is correct, as is.
\end{multipleChoice}

\end{problem}}%}

%%%%%%%%%%%%%%%%%%%%%%


\latexProblemContent{
\begin{problem}

What is wrong with the following equation:

\[\int_{\frac{2}{3} \, \pi}^{\frac{5}{4} \, \pi} {8 \, \cot\left(x\right) \csc\left(x\right)}\;dx = {-\frac{8}{\sin\left(x\right)}}\Bigg\vert_{\frac{2}{3} \, \pi}^{\frac{5}{4} \, \pi} = {\frac{16}{3} \, \sqrt{3} + 8 \, \sqrt{2}}\]

\input{2311_Concept_Integral_0003.HELP.tex}

\begin{multipleChoice}
\choice The antiderivative is incorrect.
\choice[correct] The integrand is not defined over the entire interval.
\choice The bounds are evaluated in the wrong order.
\choice Nothing is wrong.  The equation is correct, as is.
\end{multipleChoice}

\end{problem}}%}

%%%%%%%%%%%%%%%%%%%%%%


\latexProblemContent{
\begin{problem}

What is wrong with the following equation:

\[\int_{\frac{1}{2} \, \pi}^{\frac{5}{3} \, \pi} {-5 \, \cot\left(x\right) \csc\left(x\right)}\;dx = {\frac{5}{\sin\left(x\right)}}\Bigg\vert_{\frac{1}{2} \, \pi}^{\frac{5}{3} \, \pi} = {-\frac{10}{3} \, \sqrt{3} - 5}\]

\input{2311_Concept_Integral_0003.HELP.tex}

\begin{multipleChoice}
\choice The antiderivative is incorrect.
\choice[correct] The integrand is not defined over the entire interval.
\choice The bounds are evaluated in the wrong order.
\choice Nothing is wrong.  The equation is correct, as is.
\end{multipleChoice}

\end{problem}}%}

%%%%%%%%%%%%%%%%%%%%%%


\latexProblemContent{
\begin{problem}

What is wrong with the following equation:

\[\int_{\frac{3}{4} \, \pi}^{\frac{4}{3} \, \pi} {6 \, \cot\left(x\right) \csc\left(x\right)}\;dx = {-\frac{6}{\sin\left(x\right)}}\Bigg\vert_{\frac{3}{4} \, \pi}^{\frac{4}{3} \, \pi} = {4 \, \sqrt{3} + 6 \, \sqrt{2}}\]

\input{2311_Concept_Integral_0003.HELP.tex}

\begin{multipleChoice}
\choice The antiderivative is incorrect.
\choice[correct] The integrand is not defined over the entire interval.
\choice The bounds are evaluated in the wrong order.
\choice Nothing is wrong.  The equation is correct, as is.
\end{multipleChoice}

\end{problem}}%}

%%%%%%%%%%%%%%%%%%%%%%


\latexProblemContent{
\begin{problem}

What is wrong with the following equation:

\[\int_{\frac{3}{4} \, \pi}^{\frac{7}{4} \, \pi} {15 \, \csc\left(x\right)^{2}}\;dx = {-\frac{15}{\tan\left(x\right)}}\Bigg\vert_{\frac{3}{4} \, \pi}^{\frac{7}{4} \, \pi} = {0}\]

\input{2311_Concept_Integral_0003.HELP.tex}

\begin{multipleChoice}
\choice The antiderivative is incorrect.
\choice[correct] The integrand is not defined over the entire interval.
\choice The bounds are evaluated in the wrong order.
\choice Nothing is wrong.  The equation is correct, as is.
\end{multipleChoice}

\end{problem}}%}

%%%%%%%%%%%%%%%%%%%%%%


\latexProblemContent{
\begin{problem}

What is wrong with the following equation:

\[\int_{\frac{1}{2} \, \pi}^{\frac{4}{3} \, \pi} {2 \, \csc\left(x\right)^{2}}\;dx = {-\frac{2}{\tan\left(x\right)}}\Bigg\vert_{\frac{1}{2} \, \pi}^{\frac{4}{3} \, \pi} = {-\frac{2}{3} \, \sqrt{3}}\]

\input{2311_Concept_Integral_0003.HELP.tex}

\begin{multipleChoice}
\choice The antiderivative is incorrect.
\choice[correct] The integrand is not defined over the entire interval.
\choice The bounds are evaluated in the wrong order.
\choice Nothing is wrong.  The equation is correct, as is.
\end{multipleChoice}

\end{problem}}%}

%%%%%%%%%%%%%%%%%%%%%%


%%%%%%%%%%%%%%%%%%%%%%


\latexProblemContent{
\begin{problem}

What is wrong with the following equation:

\[\int_{\frac{1}{2} \, \pi}^{\frac{4}{3} \, \pi} {4 \, \csc\left(x\right)^{2}}\;dx = {-\frac{4}{\tan\left(x\right)}}\Bigg\vert_{\frac{1}{2} \, \pi}^{\frac{4}{3} \, \pi} = {-\frac{4}{3} \, \sqrt{3}}\]

\input{2311_Concept_Integral_0003.HELP.tex}

\begin{multipleChoice}
\choice The antiderivative is incorrect.
\choice[correct] The integrand is not defined over the entire interval.
\choice The bounds are evaluated in the wrong order.
\choice Nothing is wrong.  The equation is correct, as is.
\end{multipleChoice}

\end{problem}}%}

%%%%%%%%%%%%%%%%%%%%%%


\latexProblemContent{
\begin{problem}

What is wrong with the following equation:

\[\int_{\frac{1}{4} \, \pi}^{\frac{5}{3} \, \pi} {-14 \, \csc\left(x\right)^{2}}\;dx = {\frac{14}{\tan\left(x\right)}}\Bigg\vert_{\frac{1}{4} \, \pi}^{\frac{5}{3} \, \pi} = {-\frac{14}{3} \, \sqrt{3} - 14}\]

\input{2311_Concept_Integral_0003.HELP.tex}

\begin{multipleChoice}
\choice The antiderivative is incorrect.
\choice[correct] The integrand is not defined over the entire interval.
\choice The bounds are evaluated in the wrong order.
\choice Nothing is wrong.  The equation is correct, as is.
\end{multipleChoice}

\end{problem}}%}

%%%%%%%%%%%%%%%%%%%%%%


\latexProblemContent{
\begin{problem}

What is wrong with the following equation:

\[\int_{\frac{1}{2} \, \pi}^{\frac{5}{4} \, \pi} {-8 \, \cot\left(x\right) \csc\left(x\right)}\;dx = {\frac{8}{\sin\left(x\right)}}\Bigg\vert_{\frac{1}{2} \, \pi}^{\frac{5}{4} \, \pi} = {-8 \, \sqrt{2} - 8}\]

\input{2311_Concept_Integral_0003.HELP.tex}

\begin{multipleChoice}
\choice The antiderivative is incorrect.
\choice[correct] The integrand is not defined over the entire interval.
\choice The bounds are evaluated in the wrong order.
\choice Nothing is wrong.  The equation is correct, as is.
\end{multipleChoice}

\end{problem}}%}

%%%%%%%%%%%%%%%%%%%%%%


\latexProblemContent{
\begin{problem}

What is wrong with the following equation:

\[\int_{\frac{3}{4} \, \pi}^{\frac{5}{3} \, \pi} {-8 \, \cot\left(x\right) \csc\left(x\right)}\;dx = {\frac{8}{\sin\left(x\right)}}\Bigg\vert_{\frac{3}{4} \, \pi}^{\frac{5}{3} \, \pi} = {-\frac{16}{3} \, \sqrt{3} - 8 \, \sqrt{2}}\]

\input{2311_Concept_Integral_0003.HELP.tex}

\begin{multipleChoice}
\choice The antiderivative is incorrect.
\choice[correct] The integrand is not defined over the entire interval.
\choice The bounds are evaluated in the wrong order.
\choice Nothing is wrong.  The equation is correct, as is.
\end{multipleChoice}

\end{problem}}%}

%%%%%%%%%%%%%%%%%%%%%%


\latexProblemContent{
\begin{problem}

What is wrong with the following equation:

\[\int_{\frac{1}{2} \, \pi}^{\frac{4}{3} \, \pi} {15 \, \csc\left(x\right)^{2}}\;dx = {-\frac{15}{\tan\left(x\right)}}\Bigg\vert_{\frac{1}{2} \, \pi}^{\frac{4}{3} \, \pi} = {-5 \, \sqrt{3}}\]

\input{2311_Concept_Integral_0003.HELP.tex}

\begin{multipleChoice}
\choice The antiderivative is incorrect.
\choice[correct] The integrand is not defined over the entire interval.
\choice The bounds are evaluated in the wrong order.
\choice Nothing is wrong.  The equation is correct, as is.
\end{multipleChoice}

\end{problem}}%}

%%%%%%%%%%%%%%%%%%%%%%


\latexProblemContent{
\begin{problem}

What is wrong with the following equation:

\[\int_{\frac{1}{4} \, \pi}^{\frac{7}{4} \, \pi} {-\csc\left(x\right)^{2}}\;dx = {\frac{1}{\tan\left(x\right)}}\Bigg\vert_{\frac{1}{4} \, \pi}^{\frac{7}{4} \, \pi} = {-2}\]

\input{2311_Concept_Integral_0003.HELP.tex}

\begin{multipleChoice}
\choice The antiderivative is incorrect.
\choice[correct] The integrand is not defined over the entire interval.
\choice The bounds are evaluated in the wrong order.
\choice Nothing is wrong.  The equation is correct, as is.
\end{multipleChoice}

\end{problem}}%}

%%%%%%%%%%%%%%%%%%%%%%


\latexProblemContent{
\begin{problem}

What is wrong with the following equation:

\[\int_{\frac{2}{3} \, \pi}^{\frac{3}{2} \, \pi} {-6 \, \csc\left(x\right)^{2}}\;dx = {\frac{6}{\tan\left(x\right)}}\Bigg\vert_{\frac{2}{3} \, \pi}^{\frac{3}{2} \, \pi} = {2 \, \sqrt{3}}\]

\input{2311_Concept_Integral_0003.HELP.tex}

\begin{multipleChoice}
\choice The antiderivative is incorrect.
\choice[correct] The integrand is not defined over the entire interval.
\choice The bounds are evaluated in the wrong order.
\choice Nothing is wrong.  The equation is correct, as is.
\end{multipleChoice}

\end{problem}}%}

%%%%%%%%%%%%%%%%%%%%%%


\latexProblemContent{
\begin{problem}

What is wrong with the following equation:

\[\int_{\frac{1}{3} \, \pi}^{\frac{5}{4} \, \pi} {4 \, \cot\left(x\right) \csc\left(x\right)}\;dx = {-\frac{4}{\sin\left(x\right)}}\Bigg\vert_{\frac{1}{3} \, \pi}^{\frac{5}{4} \, \pi} = {\frac{8}{3} \, \sqrt{3} + 4 \, \sqrt{2}}\]

\input{2311_Concept_Integral_0003.HELP.tex}

\begin{multipleChoice}
\choice The antiderivative is incorrect.
\choice[correct] The integrand is not defined over the entire interval.
\choice The bounds are evaluated in the wrong order.
\choice Nothing is wrong.  The equation is correct, as is.
\end{multipleChoice}

\end{problem}}%}

%%%%%%%%%%%%%%%%%%%%%%


\latexProblemContent{
\begin{problem}

What is wrong with the following equation:

\[\int_{\frac{1}{4} \, \pi}^{\frac{4}{3} \, \pi} {-2 \, \csc\left(x\right)^{2}}\;dx = {\frac{2}{\tan\left(x\right)}}\Bigg\vert_{\frac{1}{4} \, \pi}^{\frac{4}{3} \, \pi} = {\frac{2}{3} \, \sqrt{3} - 2}\]

\input{2311_Concept_Integral_0003.HELP.tex}

\begin{multipleChoice}
\choice The antiderivative is incorrect.
\choice[correct] The integrand is not defined over the entire interval.
\choice The bounds are evaluated in the wrong order.
\choice Nothing is wrong.  The equation is correct, as is.
\end{multipleChoice}

\end{problem}}%}

%%%%%%%%%%%%%%%%%%%%%%


\latexProblemContent{
\begin{problem}

What is wrong with the following equation:

\[\int_{\frac{1}{3} \, \pi}^{\frac{3}{2} \, \pi} {9 \, \cot\left(x\right) \csc\left(x\right)}\;dx = {-\frac{9}{\sin\left(x\right)}}\Bigg\vert_{\frac{1}{3} \, \pi}^{\frac{3}{2} \, \pi} = {6 \, \sqrt{3} + 9}\]

\input{2311_Concept_Integral_0003.HELP.tex}

\begin{multipleChoice}
\choice The antiderivative is incorrect.
\choice[correct] The integrand is not defined over the entire interval.
\choice The bounds are evaluated in the wrong order.
\choice Nothing is wrong.  The equation is correct, as is.
\end{multipleChoice}

\end{problem}}%}

%%%%%%%%%%%%%%%%%%%%%%


\latexProblemContent{
\begin{problem}

What is wrong with the following equation:

\[\int_{\frac{1}{4} \, \pi}^{\frac{3}{2} \, \pi} {-15 \, \cot\left(x\right) \csc\left(x\right)}\;dx = {\frac{15}{\sin\left(x\right)}}\Bigg\vert_{\frac{1}{4} \, \pi}^{\frac{3}{2} \, \pi} = {-15 \, \sqrt{2} - 15}\]

\input{2311_Concept_Integral_0003.HELP.tex}

\begin{multipleChoice}
\choice The antiderivative is incorrect.
\choice[correct] The integrand is not defined over the entire interval.
\choice The bounds are evaluated in the wrong order.
\choice Nothing is wrong.  The equation is correct, as is.
\end{multipleChoice}

\end{problem}}%}

%%%%%%%%%%%%%%%%%%%%%%


\latexProblemContent{
\begin{problem}

What is wrong with the following equation:

\[\int_{\frac{2}{3} \, \pi}^{\frac{3}{2} \, \pi} {5 \, \csc\left(x\right)^{2}}\;dx = {-\frac{5}{\tan\left(x\right)}}\Bigg\vert_{\frac{2}{3} \, \pi}^{\frac{3}{2} \, \pi} = {-\frac{5}{3} \, \sqrt{3}}\]

\input{2311_Concept_Integral_0003.HELP.tex}

\begin{multipleChoice}
\choice The antiderivative is incorrect.
\choice[correct] The integrand is not defined over the entire interval.
\choice The bounds are evaluated in the wrong order.
\choice Nothing is wrong.  The equation is correct, as is.
\end{multipleChoice}

\end{problem}}%}

%%%%%%%%%%%%%%%%%%%%%%


\latexProblemContent{
\begin{problem}

What is wrong with the following equation:

\[\int_{\frac{3}{4} \, \pi}^{\frac{3}{2} \, \pi} {-7 \, \cot\left(x\right) \csc\left(x\right)}\;dx = {\frac{7}{\sin\left(x\right)}}\Bigg\vert_{\frac{3}{4} \, \pi}^{\frac{3}{2} \, \pi} = {-7 \, \sqrt{2} - 7}\]

\input{2311_Concept_Integral_0003.HELP.tex}

\begin{multipleChoice}
\choice The antiderivative is incorrect.
\choice[correct] The integrand is not defined over the entire interval.
\choice The bounds are evaluated in the wrong order.
\choice Nothing is wrong.  The equation is correct, as is.
\end{multipleChoice}

\end{problem}}%}

%%%%%%%%%%%%%%%%%%%%%%


\latexProblemContent{
\begin{problem}

What is wrong with the following equation:

\[\int_{\frac{1}{3} \, \pi}^{\frac{5}{4} \, \pi} {13 \, \cot\left(x\right) \csc\left(x\right)}\;dx = {-\frac{13}{\sin\left(x\right)}}\Bigg\vert_{\frac{1}{3} \, \pi}^{\frac{5}{4} \, \pi} = {\frac{26}{3} \, \sqrt{3} + 13 \, \sqrt{2}}\]

\input{2311_Concept_Integral_0003.HELP.tex}

\begin{multipleChoice}
\choice The antiderivative is incorrect.
\choice[correct] The integrand is not defined over the entire interval.
\choice The bounds are evaluated in the wrong order.
\choice Nothing is wrong.  The equation is correct, as is.
\end{multipleChoice}

\end{problem}}%}

%%%%%%%%%%%%%%%%%%%%%%


\latexProblemContent{
\begin{problem}

What is wrong with the following equation:

\[\int_{\frac{3}{4} \, \pi}^{\frac{5}{3} \, \pi} {-\cot\left(x\right) \csc\left(x\right)}\;dx = {\frac{1}{\sin\left(x\right)}}\Bigg\vert_{\frac{3}{4} \, \pi}^{\frac{5}{3} \, \pi} = {-\frac{2}{3} \, \sqrt{3} - \sqrt{2}}\]

\input{2311_Concept_Integral_0003.HELP.tex}

\begin{multipleChoice}
\choice The antiderivative is incorrect.
\choice[correct] The integrand is not defined over the entire interval.
\choice The bounds are evaluated in the wrong order.
\choice Nothing is wrong.  The equation is correct, as is.
\end{multipleChoice}

\end{problem}}%}

%%%%%%%%%%%%%%%%%%%%%%


\latexProblemContent{
\begin{problem}

What is wrong with the following equation:

\[\int_{\frac{1}{3} \, \pi}^{\frac{5}{3} \, \pi} {11 \, \cot\left(x\right) \csc\left(x\right)}\;dx = {-\frac{11}{\sin\left(x\right)}}\Bigg\vert_{\frac{1}{3} \, \pi}^{\frac{5}{3} \, \pi} = {\frac{44}{3} \, \sqrt{3}}\]

\input{2311_Concept_Integral_0003.HELP.tex}

\begin{multipleChoice}
\choice The antiderivative is incorrect.
\choice[correct] The integrand is not defined over the entire interval.
\choice The bounds are evaluated in the wrong order.
\choice Nothing is wrong.  The equation is correct, as is.
\end{multipleChoice}

\end{problem}}%}

%%%%%%%%%%%%%%%%%%%%%%


\latexProblemContent{
\begin{problem}

What is wrong with the following equation:

\[\int_{\frac{1}{3} \, \pi}^{\frac{5}{4} \, \pi} {14 \, \cot\left(x\right) \csc\left(x\right)}\;dx = {-\frac{14}{\sin\left(x\right)}}\Bigg\vert_{\frac{1}{3} \, \pi}^{\frac{5}{4} \, \pi} = {\frac{28}{3} \, \sqrt{3} + 14 \, \sqrt{2}}\]

\input{2311_Concept_Integral_0003.HELP.tex}

\begin{multipleChoice}
\choice The antiderivative is incorrect.
\choice[correct] The integrand is not defined over the entire interval.
\choice The bounds are evaluated in the wrong order.
\choice Nothing is wrong.  The equation is correct, as is.
\end{multipleChoice}

\end{problem}}%}

%%%%%%%%%%%%%%%%%%%%%%


\latexProblemContent{
\begin{problem}

What is wrong with the following equation:

\[\int_{\frac{1}{4} \, \pi}^{\frac{5}{4} \, \pi} {2 \, \cot\left(x\right) \csc\left(x\right)}\;dx = {-\frac{2}{\sin\left(x\right)}}\Bigg\vert_{\frac{1}{4} \, \pi}^{\frac{5}{4} \, \pi} = {4 \, \sqrt{2}}\]

\input{2311_Concept_Integral_0003.HELP.tex}

\begin{multipleChoice}
\choice The antiderivative is incorrect.
\choice[correct] The integrand is not defined over the entire interval.
\choice The bounds are evaluated in the wrong order.
\choice Nothing is wrong.  The equation is correct, as is.
\end{multipleChoice}

\end{problem}}%}

%%%%%%%%%%%%%%%%%%%%%%


\latexProblemContent{
\begin{problem}

What is wrong with the following equation:

\[\int_{\frac{1}{2} \, \pi}^{\frac{5}{4} \, \pi} {-3 \, \csc\left(x\right)^{2}}\;dx = {\frac{3}{\tan\left(x\right)}}\Bigg\vert_{\frac{1}{2} \, \pi}^{\frac{5}{4} \, \pi} = {3}\]

\input{2311_Concept_Integral_0003.HELP.tex}

\begin{multipleChoice}
\choice The antiderivative is incorrect.
\choice[correct] The integrand is not defined over the entire interval.
\choice The bounds are evaluated in the wrong order.
\choice Nothing is wrong.  The equation is correct, as is.
\end{multipleChoice}

\end{problem}}%}

%%%%%%%%%%%%%%%%%%%%%%


\latexProblemContent{
\begin{problem}

What is wrong with the following equation:

\[\int_{\frac{1}{2} \, \pi}^{\frac{5}{3} \, \pi} {4 \, \csc\left(x\right)^{2}}\;dx = {-\frac{4}{\tan\left(x\right)}}\Bigg\vert_{\frac{1}{2} \, \pi}^{\frac{5}{3} \, \pi} = {\frac{4}{3} \, \sqrt{3}}\]

\input{2311_Concept_Integral_0003.HELP.tex}

\begin{multipleChoice}
\choice The antiderivative is incorrect.
\choice[correct] The integrand is not defined over the entire interval.
\choice The bounds are evaluated in the wrong order.
\choice Nothing is wrong.  The equation is correct, as is.
\end{multipleChoice}

\end{problem}}%}

%%%%%%%%%%%%%%%%%%%%%%


\latexProblemContent{
\begin{problem}

What is wrong with the following equation:

\[\int_{\frac{1}{3} \, \pi}^{\frac{3}{2} \, \pi} {-2 \, \csc\left(x\right)^{2}}\;dx = {\frac{2}{\tan\left(x\right)}}\Bigg\vert_{\frac{1}{3} \, \pi}^{\frac{3}{2} \, \pi} = {-\frac{2}{3} \, \sqrt{3}}\]

\input{2311_Concept_Integral_0003.HELP.tex}

\begin{multipleChoice}
\choice The antiderivative is incorrect.
\choice[correct] The integrand is not defined over the entire interval.
\choice The bounds are evaluated in the wrong order.
\choice Nothing is wrong.  The equation is correct, as is.
\end{multipleChoice}

\end{problem}}%}

%%%%%%%%%%%%%%%%%%%%%%


\latexProblemContent{
\begin{problem}

What is wrong with the following equation:

\[\int_{\frac{3}{4} \, \pi}^{\frac{7}{4} \, \pi} {-14 \, \cot\left(x\right) \csc\left(x\right)}\;dx = {\frac{14}{\sin\left(x\right)}}\Bigg\vert_{\frac{3}{4} \, \pi}^{\frac{7}{4} \, \pi} = {-28 \, \sqrt{2}}\]

\input{2311_Concept_Integral_0003.HELP.tex}

\begin{multipleChoice}
\choice The antiderivative is incorrect.
\choice[correct] The integrand is not defined over the entire interval.
\choice The bounds are evaluated in the wrong order.
\choice Nothing is wrong.  The equation is correct, as is.
\end{multipleChoice}

\end{problem}}%}

%%%%%%%%%%%%%%%%%%%%%%


\latexProblemContent{
\begin{problem}

What is wrong with the following equation:

\[\int_{\frac{2}{3} \, \pi}^{\frac{5}{3} \, \pi} {-3 \, \cot\left(x\right) \csc\left(x\right)}\;dx = {\frac{3}{\sin\left(x\right)}}\Bigg\vert_{\frac{2}{3} \, \pi}^{\frac{5}{3} \, \pi} = {-4 \, \sqrt{3}}\]

\input{2311_Concept_Integral_0003.HELP.tex}

\begin{multipleChoice}
\choice The antiderivative is incorrect.
\choice[correct] The integrand is not defined over the entire interval.
\choice The bounds are evaluated in the wrong order.
\choice Nothing is wrong.  The equation is correct, as is.
\end{multipleChoice}

\end{problem}}%}

%%%%%%%%%%%%%%%%%%%%%%


\latexProblemContent{
\begin{problem}

What is wrong with the following equation:

\[\int_{\frac{1}{3} \, \pi}^{\frac{5}{3} \, \pi} {10 \, \cot\left(x\right) \csc\left(x\right)}\;dx = {-\frac{10}{\sin\left(x\right)}}\Bigg\vert_{\frac{1}{3} \, \pi}^{\frac{5}{3} \, \pi} = {\frac{40}{3} \, \sqrt{3}}\]

\input{2311_Concept_Integral_0003.HELP.tex}

\begin{multipleChoice}
\choice The antiderivative is incorrect.
\choice[correct] The integrand is not defined over the entire interval.
\choice The bounds are evaluated in the wrong order.
\choice Nothing is wrong.  The equation is correct, as is.
\end{multipleChoice}

\end{problem}}%}

%%%%%%%%%%%%%%%%%%%%%%


\latexProblemContent{
\begin{problem}

What is wrong with the following equation:

\[\int_{\frac{1}{4} \, \pi}^{\frac{5}{3} \, \pi} {7 \, \cot\left(x\right) \csc\left(x\right)}\;dx = {-\frac{7}{\sin\left(x\right)}}\Bigg\vert_{\frac{1}{4} \, \pi}^{\frac{5}{3} \, \pi} = {\frac{14}{3} \, \sqrt{3} + 7 \, \sqrt{2}}\]

\input{2311_Concept_Integral_0003.HELP.tex}

\begin{multipleChoice}
\choice The antiderivative is incorrect.
\choice[correct] The integrand is not defined over the entire interval.
\choice The bounds are evaluated in the wrong order.
\choice Nothing is wrong.  The equation is correct, as is.
\end{multipleChoice}

\end{problem}}%}

%%%%%%%%%%%%%%%%%%%%%%


\latexProblemContent{
\begin{problem}

What is wrong with the following equation:

\[\int_{\frac{1}{2} \, \pi}^{\frac{7}{4} \, \pi} {-8 \, \cot\left(x\right) \csc\left(x\right)}\;dx = {\frac{8}{\sin\left(x\right)}}\Bigg\vert_{\frac{1}{2} \, \pi}^{\frac{7}{4} \, \pi} = {-8 \, \sqrt{2} - 8}\]

\input{2311_Concept_Integral_0003.HELP.tex}

\begin{multipleChoice}
\choice The antiderivative is incorrect.
\choice[correct] The integrand is not defined over the entire interval.
\choice The bounds are evaluated in the wrong order.
\choice Nothing is wrong.  The equation is correct, as is.
\end{multipleChoice}

\end{problem}}%}

%%%%%%%%%%%%%%%%%%%%%%


\latexProblemContent{
\begin{problem}

What is wrong with the following equation:

\[\int_{\frac{2}{3} \, \pi}^{\frac{4}{3} \, \pi} {-2 \, \csc\left(x\right)^{2}}\;dx = {\frac{2}{\tan\left(x\right)}}\Bigg\vert_{\frac{2}{3} \, \pi}^{\frac{4}{3} \, \pi} = {\frac{4}{3} \, \sqrt{3}}\]

\input{2311_Concept_Integral_0003.HELP.tex}

\begin{multipleChoice}
\choice The antiderivative is incorrect.
\choice[correct] The integrand is not defined over the entire interval.
\choice The bounds are evaluated in the wrong order.
\choice Nothing is wrong.  The equation is correct, as is.
\end{multipleChoice}

\end{problem}}%}

%%%%%%%%%%%%%%%%%%%%%%


\latexProblemContent{
\begin{problem}

What is wrong with the following equation:

\[\int_{\frac{1}{4} \, \pi}^{\frac{5}{3} \, \pi} {-6 \, \cot\left(x\right) \csc\left(x\right)}\;dx = {\frac{6}{\sin\left(x\right)}}\Bigg\vert_{\frac{1}{4} \, \pi}^{\frac{5}{3} \, \pi} = {-4 \, \sqrt{3} - 6 \, \sqrt{2}}\]

\input{2311_Concept_Integral_0003.HELP.tex}

\begin{multipleChoice}
\choice The antiderivative is incorrect.
\choice[correct] The integrand is not defined over the entire interval.
\choice The bounds are evaluated in the wrong order.
\choice Nothing is wrong.  The equation is correct, as is.
\end{multipleChoice}

\end{problem}}%}

%%%%%%%%%%%%%%%%%%%%%%


\latexProblemContent{
\begin{problem}

What is wrong with the following equation:

\[\int_{\frac{1}{3} \, \pi}^{\frac{7}{4} \, \pi} {-7 \, \csc\left(x\right)^{2}}\;dx = {\frac{7}{\tan\left(x\right)}}\Bigg\vert_{\frac{1}{3} \, \pi}^{\frac{7}{4} \, \pi} = {-\frac{7}{3} \, \sqrt{3} - 7}\]

\input{2311_Concept_Integral_0003.HELP.tex}

\begin{multipleChoice}
\choice The antiderivative is incorrect.
\choice[correct] The integrand is not defined over the entire interval.
\choice The bounds are evaluated in the wrong order.
\choice Nothing is wrong.  The equation is correct, as is.
\end{multipleChoice}

\end{problem}}%}

%%%%%%%%%%%%%%%%%%%%%%


\latexProblemContent{
\begin{problem}

What is wrong with the following equation:

\[\int_{\frac{2}{3} \, \pi}^{\frac{3}{2} \, \pi} {-3 \, \cot\left(x\right) \csc\left(x\right)}\;dx = {\frac{3}{\sin\left(x\right)}}\Bigg\vert_{\frac{2}{3} \, \pi}^{\frac{3}{2} \, \pi} = {-2 \, \sqrt{3} - 3}\]

\input{2311_Concept_Integral_0003.HELP.tex}

\begin{multipleChoice}
\choice The antiderivative is incorrect.
\choice[correct] The integrand is not defined over the entire interval.
\choice The bounds are evaluated in the wrong order.
\choice Nothing is wrong.  The equation is correct, as is.
\end{multipleChoice}

\end{problem}}%}

%%%%%%%%%%%%%%%%%%%%%%


\latexProblemContent{
\begin{problem}

What is wrong with the following equation:

\[\int_{\frac{1}{2} \, \pi}^{\frac{7}{4} \, \pi} {-5 \, \csc\left(x\right)^{2}}\;dx = {\frac{5}{\tan\left(x\right)}}\Bigg\vert_{\frac{1}{2} \, \pi}^{\frac{7}{4} \, \pi} = {-5}\]

\input{2311_Concept_Integral_0003.HELP.tex}

\begin{multipleChoice}
\choice The antiderivative is incorrect.
\choice[correct] The integrand is not defined over the entire interval.
\choice The bounds are evaluated in the wrong order.
\choice Nothing is wrong.  The equation is correct, as is.
\end{multipleChoice}

\end{problem}}%}

%%%%%%%%%%%%%%%%%%%%%%


\latexProblemContent{
\begin{problem}

What is wrong with the following equation:

\[\int_{\frac{1}{3} \, \pi}^{\frac{4}{3} \, \pi} {-10 \, \cot\left(x\right) \csc\left(x\right)}\;dx = {\frac{10}{\sin\left(x\right)}}\Bigg\vert_{\frac{1}{3} \, \pi}^{\frac{4}{3} \, \pi} = {-\frac{40}{3} \, \sqrt{3}}\]

\input{2311_Concept_Integral_0003.HELP.tex}

\begin{multipleChoice}
\choice The antiderivative is incorrect.
\choice[correct] The integrand is not defined over the entire interval.
\choice The bounds are evaluated in the wrong order.
\choice Nothing is wrong.  The equation is correct, as is.
\end{multipleChoice}

\end{problem}}%}

%%%%%%%%%%%%%%%%%%%%%%


%%%%%%%%%%%%%%%%%%%%%%


\latexProblemContent{
\begin{problem}

What is wrong with the following equation:

\[\int_{\frac{2}{3} \, \pi}^{\frac{7}{4} \, \pi} {-13 \, \csc\left(x\right)^{2}}\;dx = {\frac{13}{\tan\left(x\right)}}\Bigg\vert_{\frac{2}{3} \, \pi}^{\frac{7}{4} \, \pi} = {\frac{13}{3} \, \sqrt{3} - 13}\]

\input{2311_Concept_Integral_0003.HELP.tex}

\begin{multipleChoice}
\choice The antiderivative is incorrect.
\choice[correct] The integrand is not defined over the entire interval.
\choice The bounds are evaluated in the wrong order.
\choice Nothing is wrong.  The equation is correct, as is.
\end{multipleChoice}

\end{problem}}%}

%%%%%%%%%%%%%%%%%%%%%%


\latexProblemContent{
\begin{problem}

What is wrong with the following equation:

\[\int_{\frac{1}{3} \, \pi}^{\frac{3}{2} \, \pi} {-15 \, \csc\left(x\right)^{2}}\;dx = {\frac{15}{\tan\left(x\right)}}\Bigg\vert_{\frac{1}{3} \, \pi}^{\frac{3}{2} \, \pi} = {-5 \, \sqrt{3}}\]

\input{2311_Concept_Integral_0003.HELP.tex}

\begin{multipleChoice}
\choice The antiderivative is incorrect.
\choice[correct] The integrand is not defined over the entire interval.
\choice The bounds are evaluated in the wrong order.
\choice Nothing is wrong.  The equation is correct, as is.
\end{multipleChoice}

\end{problem}}%}

%%%%%%%%%%%%%%%%%%%%%%


\latexProblemContent{
\begin{problem}

What is wrong with the following equation:

\[\int_{\frac{1}{3} \, \pi}^{\frac{4}{3} \, \pi} {-6 \, \csc\left(x\right)^{2}}\;dx = {\frac{6}{\tan\left(x\right)}}\Bigg\vert_{\frac{1}{3} \, \pi}^{\frac{4}{3} \, \pi} = {0}\]

\input{2311_Concept_Integral_0003.HELP.tex}

\begin{multipleChoice}
\choice The antiderivative is incorrect.
\choice[correct] The integrand is not defined over the entire interval.
\choice The bounds are evaluated in the wrong order.
\choice Nothing is wrong.  The equation is correct, as is.
\end{multipleChoice}

\end{problem}}%}

%%%%%%%%%%%%%%%%%%%%%%


\latexProblemContent{
\begin{problem}

What is wrong with the following equation:

\[\int_{\frac{1}{3} \, \pi}^{\frac{7}{4} \, \pi} {11 \, \cot\left(x\right) \csc\left(x\right)}\;dx = {-\frac{11}{\sin\left(x\right)}}\Bigg\vert_{\frac{1}{3} \, \pi}^{\frac{7}{4} \, \pi} = {\frac{22}{3} \, \sqrt{3} + 11 \, \sqrt{2}}\]

\input{2311_Concept_Integral_0003.HELP.tex}

\begin{multipleChoice}
\choice The antiderivative is incorrect.
\choice[correct] The integrand is not defined over the entire interval.
\choice The bounds are evaluated in the wrong order.
\choice Nothing is wrong.  The equation is correct, as is.
\end{multipleChoice}

\end{problem}}%}

%%%%%%%%%%%%%%%%%%%%%%


\latexProblemContent{
\begin{problem}

What is wrong with the following equation:

\[\int_{\frac{1}{3} \, \pi}^{\frac{5}{3} \, \pi} {-\cot\left(x\right) \csc\left(x\right)}\;dx = {\frac{1}{\sin\left(x\right)}}\Bigg\vert_{\frac{1}{3} \, \pi}^{\frac{5}{3} \, \pi} = {-\frac{4}{3} \, \sqrt{3}}\]

\input{2311_Concept_Integral_0003.HELP.tex}

\begin{multipleChoice}
\choice The antiderivative is incorrect.
\choice[correct] The integrand is not defined over the entire interval.
\choice The bounds are evaluated in the wrong order.
\choice Nothing is wrong.  The equation is correct, as is.
\end{multipleChoice}

\end{problem}}%}

%%%%%%%%%%%%%%%%%%%%%%


\latexProblemContent{
\begin{problem}

What is wrong with the following equation:

\[\int_{\frac{3}{4} \, \pi}^{\frac{5}{3} \, \pi} {6 \, \csc\left(x\right)^{2}}\;dx = {-\frac{6}{\tan\left(x\right)}}\Bigg\vert_{\frac{3}{4} \, \pi}^{\frac{5}{3} \, \pi} = {2 \, \sqrt{3} - 6}\]

\input{2311_Concept_Integral_0003.HELP.tex}

\begin{multipleChoice}
\choice The antiderivative is incorrect.
\choice[correct] The integrand is not defined over the entire interval.
\choice The bounds are evaluated in the wrong order.
\choice Nothing is wrong.  The equation is correct, as is.
\end{multipleChoice}

\end{problem}}%}

%%%%%%%%%%%%%%%%%%%%%%


\latexProblemContent{
\begin{problem}

What is wrong with the following equation:

\[\int_{\frac{1}{4} \, \pi}^{\frac{3}{2} \, \pi} {15 \, \cot\left(x\right) \csc\left(x\right)}\;dx = {-\frac{15}{\sin\left(x\right)}}\Bigg\vert_{\frac{1}{4} \, \pi}^{\frac{3}{2} \, \pi} = {15 \, \sqrt{2} + 15}\]

\input{2311_Concept_Integral_0003.HELP.tex}

\begin{multipleChoice}
\choice The antiderivative is incorrect.
\choice[correct] The integrand is not defined over the entire interval.
\choice The bounds are evaluated in the wrong order.
\choice Nothing is wrong.  The equation is correct, as is.
\end{multipleChoice}

\end{problem}}%}

%%%%%%%%%%%%%%%%%%%%%%


\latexProblemContent{
\begin{problem}

What is wrong with the following equation:

\[\int_{\frac{3}{4} \, \pi}^{\frac{4}{3} \, \pi} {-11 \, \cot\left(x\right) \csc\left(x\right)}\;dx = {\frac{11}{\sin\left(x\right)}}\Bigg\vert_{\frac{3}{4} \, \pi}^{\frac{4}{3} \, \pi} = {-\frac{22}{3} \, \sqrt{3} - 11 \, \sqrt{2}}\]

\input{2311_Concept_Integral_0003.HELP.tex}

\begin{multipleChoice}
\choice The antiderivative is incorrect.
\choice[correct] The integrand is not defined over the entire interval.
\choice The bounds are evaluated in the wrong order.
\choice Nothing is wrong.  The equation is correct, as is.
\end{multipleChoice}

\end{problem}}%}

%%%%%%%%%%%%%%%%%%%%%%


\latexProblemContent{
\begin{problem}

What is wrong with the following equation:

\[\int_{\frac{2}{3} \, \pi}^{\frac{3}{2} \, \pi} {-\cot\left(x\right) \csc\left(x\right)}\;dx = {\frac{1}{\sin\left(x\right)}}\Bigg\vert_{\frac{2}{3} \, \pi}^{\frac{3}{2} \, \pi} = {-\frac{2}{3} \, \sqrt{3} - 1}\]

\input{2311_Concept_Integral_0003.HELP.tex}

\begin{multipleChoice}
\choice The antiderivative is incorrect.
\choice[correct] The integrand is not defined over the entire interval.
\choice The bounds are evaluated in the wrong order.
\choice Nothing is wrong.  The equation is correct, as is.
\end{multipleChoice}

\end{problem}}%}

%%%%%%%%%%%%%%%%%%%%%%


\latexProblemContent{
\begin{problem}

What is wrong with the following equation:

\[\int_{\frac{3}{4} \, \pi}^{\frac{7}{4} \, \pi} {5 \, \cot\left(x\right) \csc\left(x\right)}\;dx = {-\frac{5}{\sin\left(x\right)}}\Bigg\vert_{\frac{3}{4} \, \pi}^{\frac{7}{4} \, \pi} = {10 \, \sqrt{2}}\]

\input{2311_Concept_Integral_0003.HELP.tex}

\begin{multipleChoice}
\choice The antiderivative is incorrect.
\choice[correct] The integrand is not defined over the entire interval.
\choice The bounds are evaluated in the wrong order.
\choice Nothing is wrong.  The equation is correct, as is.
\end{multipleChoice}

\end{problem}}%}

%%%%%%%%%%%%%%%%%%%%%%


\latexProblemContent{
\begin{problem}

What is wrong with the following equation:

\[\int_{\frac{1}{2} \, \pi}^{\frac{3}{2} \, \pi} {\csc\left(x\right)^{2}}\;dx = {-\frac{1}{\tan\left(x\right)}}\Bigg\vert_{\frac{1}{2} \, \pi}^{\frac{3}{2} \, \pi} = {0}\]

\input{2311_Concept_Integral_0003.HELP.tex}

\begin{multipleChoice}
\choice The antiderivative is incorrect.
\choice[correct] The integrand is not defined over the entire interval.
\choice The bounds are evaluated in the wrong order.
\choice Nothing is wrong.  The equation is correct, as is.
\end{multipleChoice}

\end{problem}}%}

%%%%%%%%%%%%%%%%%%%%%%


\latexProblemContent{
\begin{problem}

What is wrong with the following equation:

\[\int_{\frac{1}{3} \, \pi}^{\frac{5}{4} \, \pi} {6 \, \csc\left(x\right)^{2}}\;dx = {-\frac{6}{\tan\left(x\right)}}\Bigg\vert_{\frac{1}{3} \, \pi}^{\frac{5}{4} \, \pi} = {2 \, \sqrt{3} - 6}\]

\input{2311_Concept_Integral_0003.HELP.tex}

\begin{multipleChoice}
\choice The antiderivative is incorrect.
\choice[correct] The integrand is not defined over the entire interval.
\choice The bounds are evaluated in the wrong order.
\choice Nothing is wrong.  The equation is correct, as is.
\end{multipleChoice}

\end{problem}}%}

%%%%%%%%%%%%%%%%%%%%%%


\latexProblemContent{
\begin{problem}

What is wrong with the following equation:

\[\int_{\frac{1}{3} \, \pi}^{\frac{7}{4} \, \pi} {-3 \, \cot\left(x\right) \csc\left(x\right)}\;dx = {\frac{3}{\sin\left(x\right)}}\Bigg\vert_{\frac{1}{3} \, \pi}^{\frac{7}{4} \, \pi} = {-2 \, \sqrt{3} - 3 \, \sqrt{2}}\]

\input{2311_Concept_Integral_0003.HELP.tex}

\begin{multipleChoice}
\choice The antiderivative is incorrect.
\choice[correct] The integrand is not defined over the entire interval.
\choice The bounds are evaluated in the wrong order.
\choice Nothing is wrong.  The equation is correct, as is.
\end{multipleChoice}

\end{problem}}%}

%%%%%%%%%%%%%%%%%%%%%%


\latexProblemContent{
\begin{problem}

What is wrong with the following equation:

\[\int_{\frac{1}{4} \, \pi}^{\frac{7}{4} \, \pi} {-7 \, \cot\left(x\right) \csc\left(x\right)}\;dx = {\frac{7}{\sin\left(x\right)}}\Bigg\vert_{\frac{1}{4} \, \pi}^{\frac{7}{4} \, \pi} = {-14 \, \sqrt{2}}\]

\input{2311_Concept_Integral_0003.HELP.tex}

\begin{multipleChoice}
\choice The antiderivative is incorrect.
\choice[correct] The integrand is not defined over the entire interval.
\choice The bounds are evaluated in the wrong order.
\choice Nothing is wrong.  The equation is correct, as is.
\end{multipleChoice}

\end{problem}}%}

%%%%%%%%%%%%%%%%%%%%%%


%%%%%%%%%%%%%%%%%%%%%%


\latexProblemContent{
\begin{problem}

What is wrong with the following equation:

\[\int_{\frac{1}{4} \, \pi}^{\frac{7}{4} \, \pi} {-11 \, \csc\left(x\right)^{2}}\;dx = {\frac{11}{\tan\left(x\right)}}\Bigg\vert_{\frac{1}{4} \, \pi}^{\frac{7}{4} \, \pi} = {-22}\]

\input{2311_Concept_Integral_0003.HELP.tex}

\begin{multipleChoice}
\choice The antiderivative is incorrect.
\choice[correct] The integrand is not defined over the entire interval.
\choice The bounds are evaluated in the wrong order.
\choice Nothing is wrong.  The equation is correct, as is.
\end{multipleChoice}

\end{problem}}%}

%%%%%%%%%%%%%%%%%%%%%%


\latexProblemContent{
\begin{problem}

What is wrong with the following equation:

\[\int_{\frac{1}{2} \, \pi}^{\frac{5}{3} \, \pi} {-15 \, \cot\left(x\right) \csc\left(x\right)}\;dx = {\frac{15}{\sin\left(x\right)}}\Bigg\vert_{\frac{1}{2} \, \pi}^{\frac{5}{3} \, \pi} = {-10 \, \sqrt{3} - 15}\]

\input{2311_Concept_Integral_0003.HELP.tex}

\begin{multipleChoice}
\choice The antiderivative is incorrect.
\choice[correct] The integrand is not defined over the entire interval.
\choice The bounds are evaluated in the wrong order.
\choice Nothing is wrong.  The equation is correct, as is.
\end{multipleChoice}

\end{problem}}%}

%%%%%%%%%%%%%%%%%%%%%%


\latexProblemContent{
\begin{problem}

What is wrong with the following equation:

\[\int_{\frac{1}{3} \, \pi}^{\frac{7}{4} \, \pi} {-14 \, \cot\left(x\right) \csc\left(x\right)}\;dx = {\frac{14}{\sin\left(x\right)}}\Bigg\vert_{\frac{1}{3} \, \pi}^{\frac{7}{4} \, \pi} = {-\frac{28}{3} \, \sqrt{3} - 14 \, \sqrt{2}}\]

\input{2311_Concept_Integral_0003.HELP.tex}

\begin{multipleChoice}
\choice The antiderivative is incorrect.
\choice[correct] The integrand is not defined over the entire interval.
\choice The bounds are evaluated in the wrong order.
\choice Nothing is wrong.  The equation is correct, as is.
\end{multipleChoice}

\end{problem}}%}

%%%%%%%%%%%%%%%%%%%%%%


\latexProblemContent{
\begin{problem}

What is wrong with the following equation:

\[\int_{\frac{2}{3} \, \pi}^{\frac{5}{3} \, \pi} {13 \, \cot\left(x\right) \csc\left(x\right)}\;dx = {-\frac{13}{\sin\left(x\right)}}\Bigg\vert_{\frac{2}{3} \, \pi}^{\frac{5}{3} \, \pi} = {\frac{52}{3} \, \sqrt{3}}\]

\input{2311_Concept_Integral_0003.HELP.tex}

\begin{multipleChoice}
\choice The antiderivative is incorrect.
\choice[correct] The integrand is not defined over the entire interval.
\choice The bounds are evaluated in the wrong order.
\choice Nothing is wrong.  The equation is correct, as is.
\end{multipleChoice}

\end{problem}}%}

%%%%%%%%%%%%%%%%%%%%%%


\latexProblemContent{
\begin{problem}

What is wrong with the following equation:

\[\int_{\frac{1}{4} \, \pi}^{\frac{4}{3} \, \pi} {-10 \, \csc\left(x\right)^{2}}\;dx = {\frac{10}{\tan\left(x\right)}}\Bigg\vert_{\frac{1}{4} \, \pi}^{\frac{4}{3} \, \pi} = {\frac{10}{3} \, \sqrt{3} - 10}\]

\input{2311_Concept_Integral_0003.HELP.tex}

\begin{multipleChoice}
\choice The antiderivative is incorrect.
\choice[correct] The integrand is not defined over the entire interval.
\choice The bounds are evaluated in the wrong order.
\choice Nothing is wrong.  The equation is correct, as is.
\end{multipleChoice}

\end{problem}}%}

%%%%%%%%%%%%%%%%%%%%%%


\latexProblemContent{
\begin{problem}

What is wrong with the following equation:

\[\int_{\frac{1}{2} \, \pi}^{\frac{4}{3} \, \pi} {-11 \, \csc\left(x\right)^{2}}\;dx = {\frac{11}{\tan\left(x\right)}}\Bigg\vert_{\frac{1}{2} \, \pi}^{\frac{4}{3} \, \pi} = {\frac{11}{3} \, \sqrt{3}}\]

\input{2311_Concept_Integral_0003.HELP.tex}

\begin{multipleChoice}
\choice The antiderivative is incorrect.
\choice[correct] The integrand is not defined over the entire interval.
\choice The bounds are evaluated in the wrong order.
\choice Nothing is wrong.  The equation is correct, as is.
\end{multipleChoice}

\end{problem}}%}

%%%%%%%%%%%%%%%%%%%%%%


\latexProblemContent{
\begin{problem}

What is wrong with the following equation:

\[\int_{\frac{1}{4} \, \pi}^{\frac{5}{4} \, \pi} {-9 \, \csc\left(x\right)^{2}}\;dx = {\frac{9}{\tan\left(x\right)}}\Bigg\vert_{\frac{1}{4} \, \pi}^{\frac{5}{4} \, \pi} = {0}\]

\input{2311_Concept_Integral_0003.HELP.tex}

\begin{multipleChoice}
\choice The antiderivative is incorrect.
\choice[correct] The integrand is not defined over the entire interval.
\choice The bounds are evaluated in the wrong order.
\choice Nothing is wrong.  The equation is correct, as is.
\end{multipleChoice}

\end{problem}}%}

%%%%%%%%%%%%%%%%%%%%%%


\latexProblemContent{
\begin{problem}

What is wrong with the following equation:

\[\int_{\frac{1}{2} \, \pi}^{\frac{5}{4} \, \pi} {4 \, \cot\left(x\right) \csc\left(x\right)}\;dx = {-\frac{4}{\sin\left(x\right)}}\Bigg\vert_{\frac{1}{2} \, \pi}^{\frac{5}{4} \, \pi} = {4 \, \sqrt{2} + 4}\]

\input{2311_Concept_Integral_0003.HELP.tex}

\begin{multipleChoice}
\choice The antiderivative is incorrect.
\choice[correct] The integrand is not defined over the entire interval.
\choice The bounds are evaluated in the wrong order.
\choice Nothing is wrong.  The equation is correct, as is.
\end{multipleChoice}

\end{problem}}%}

%%%%%%%%%%%%%%%%%%%%%%


\latexProblemContent{
\begin{problem}

What is wrong with the following equation:

\[\int_{\frac{1}{3} \, \pi}^{\frac{5}{4} \, \pi} {8 \, \cot\left(x\right) \csc\left(x\right)}\;dx = {-\frac{8}{\sin\left(x\right)}}\Bigg\vert_{\frac{1}{3} \, \pi}^{\frac{5}{4} \, \pi} = {\frac{16}{3} \, \sqrt{3} + 8 \, \sqrt{2}}\]

\input{2311_Concept_Integral_0003.HELP.tex}

\begin{multipleChoice}
\choice The antiderivative is incorrect.
\choice[correct] The integrand is not defined over the entire interval.
\choice The bounds are evaluated in the wrong order.
\choice Nothing is wrong.  The equation is correct, as is.
\end{multipleChoice}

\end{problem}}%}

%%%%%%%%%%%%%%%%%%%%%%


\latexProblemContent{
\begin{problem}

What is wrong with the following equation:

\[\int_{\frac{1}{2} \, \pi}^{\frac{4}{3} \, \pi} {13 \, \cot\left(x\right) \csc\left(x\right)}\;dx = {-\frac{13}{\sin\left(x\right)}}\Bigg\vert_{\frac{1}{2} \, \pi}^{\frac{4}{3} \, \pi} = {\frac{26}{3} \, \sqrt{3} + 13}\]

\input{2311_Concept_Integral_0003.HELP.tex}

\begin{multipleChoice}
\choice The antiderivative is incorrect.
\choice[correct] The integrand is not defined over the entire interval.
\choice The bounds are evaluated in the wrong order.
\choice Nothing is wrong.  The equation is correct, as is.
\end{multipleChoice}

\end{problem}}%}

%%%%%%%%%%%%%%%%%%%%%%


\latexProblemContent{
\begin{problem}

What is wrong with the following equation:

\[\int_{\frac{2}{3} \, \pi}^{\frac{5}{3} \, \pi} {-7 \, \cot\left(x\right) \csc\left(x\right)}\;dx = {\frac{7}{\sin\left(x\right)}}\Bigg\vert_{\frac{2}{3} \, \pi}^{\frac{5}{3} \, \pi} = {-\frac{28}{3} \, \sqrt{3}}\]

\input{2311_Concept_Integral_0003.HELP.tex}

\begin{multipleChoice}
\choice The antiderivative is incorrect.
\choice[correct] The integrand is not defined over the entire interval.
\choice The bounds are evaluated in the wrong order.
\choice Nothing is wrong.  The equation is correct, as is.
\end{multipleChoice}

\end{problem}}%}

%%%%%%%%%%%%%%%%%%%%%%


\latexProblemContent{
\begin{problem}

What is wrong with the following equation:

\[\int_{\frac{1}{2} \, \pi}^{\frac{5}{3} \, \pi} {10 \, \csc\left(x\right)^{2}}\;dx = {-\frac{10}{\tan\left(x\right)}}\Bigg\vert_{\frac{1}{2} \, \pi}^{\frac{5}{3} \, \pi} = {\frac{10}{3} \, \sqrt{3}}\]

\input{2311_Concept_Integral_0003.HELP.tex}

\begin{multipleChoice}
\choice The antiderivative is incorrect.
\choice[correct] The integrand is not defined over the entire interval.
\choice The bounds are evaluated in the wrong order.
\choice Nothing is wrong.  The equation is correct, as is.
\end{multipleChoice}

\end{problem}}%}

%%%%%%%%%%%%%%%%%%%%%%


\latexProblemContent{
\begin{problem}

What is wrong with the following equation:

\[\int_{\frac{1}{4} \, \pi}^{\frac{5}{3} \, \pi} {8 \, \csc\left(x\right)^{2}}\;dx = {-\frac{8}{\tan\left(x\right)}}\Bigg\vert_{\frac{1}{4} \, \pi}^{\frac{5}{3} \, \pi} = {\frac{8}{3} \, \sqrt{3} + 8}\]

\input{2311_Concept_Integral_0003.HELP.tex}

\begin{multipleChoice}
\choice The antiderivative is incorrect.
\choice[correct] The integrand is not defined over the entire interval.
\choice The bounds are evaluated in the wrong order.
\choice Nothing is wrong.  The equation is correct, as is.
\end{multipleChoice}

\end{problem}}%}

%%%%%%%%%%%%%%%%%%%%%%


\latexProblemContent{
\begin{problem}

What is wrong with the following equation:

\[\int_{\frac{1}{3} \, \pi}^{\frac{3}{2} \, \pi} {9 \, \csc\left(x\right)^{2}}\;dx = {-\frac{9}{\tan\left(x\right)}}\Bigg\vert_{\frac{1}{3} \, \pi}^{\frac{3}{2} \, \pi} = {3 \, \sqrt{3}}\]

\input{2311_Concept_Integral_0003.HELP.tex}

\begin{multipleChoice}
\choice The antiderivative is incorrect.
\choice[correct] The integrand is not defined over the entire interval.
\choice The bounds are evaluated in the wrong order.
\choice Nothing is wrong.  The equation is correct, as is.
\end{multipleChoice}

\end{problem}}%}

%%%%%%%%%%%%%%%%%%%%%%


\latexProblemContent{
\begin{problem}

What is wrong with the following equation:

\[\int_{\frac{1}{2} \, \pi}^{\frac{5}{4} \, \pi} {8 \, \cot\left(x\right) \csc\left(x\right)}\;dx = {-\frac{8}{\sin\left(x\right)}}\Bigg\vert_{\frac{1}{2} \, \pi}^{\frac{5}{4} \, \pi} = {8 \, \sqrt{2} + 8}\]

\input{2311_Concept_Integral_0003.HELP.tex}

\begin{multipleChoice}
\choice The antiderivative is incorrect.
\choice[correct] The integrand is not defined over the entire interval.
\choice The bounds are evaluated in the wrong order.
\choice Nothing is wrong.  The equation is correct, as is.
\end{multipleChoice}

\end{problem}}%}

%%%%%%%%%%%%%%%%%%%%%%


\latexProblemContent{
\begin{problem}

What is wrong with the following equation:

\[\int_{\frac{2}{3} \, \pi}^{\frac{3}{2} \, \pi} {-11 \, \cot\left(x\right) \csc\left(x\right)}\;dx = {\frac{11}{\sin\left(x\right)}}\Bigg\vert_{\frac{2}{3} \, \pi}^{\frac{3}{2} \, \pi} = {-\frac{22}{3} \, \sqrt{3} - 11}\]

\input{2311_Concept_Integral_0003.HELP.tex}

\begin{multipleChoice}
\choice The antiderivative is incorrect.
\choice[correct] The integrand is not defined over the entire interval.
\choice The bounds are evaluated in the wrong order.
\choice Nothing is wrong.  The equation is correct, as is.
\end{multipleChoice}

\end{problem}}%}

%%%%%%%%%%%%%%%%%%%%%%


\latexProblemContent{
\begin{problem}

What is wrong with the following equation:

\[\int_{\frac{3}{4} \, \pi}^{\frac{4}{3} \, \pi} {2 \, \csc\left(x\right)^{2}}\;dx = {-\frac{2}{\tan\left(x\right)}}\Bigg\vert_{\frac{3}{4} \, \pi}^{\frac{4}{3} \, \pi} = {-\frac{2}{3} \, \sqrt{3} - 2}\]

\input{2311_Concept_Integral_0003.HELP.tex}

\begin{multipleChoice}
\choice The antiderivative is incorrect.
\choice[correct] The integrand is not defined over the entire interval.
\choice The bounds are evaluated in the wrong order.
\choice Nothing is wrong.  The equation is correct, as is.
\end{multipleChoice}

\end{problem}}%}

%%%%%%%%%%%%%%%%%%%%%%


\latexProblemContent{
\begin{problem}

What is wrong with the following equation:

\[\int_{\frac{1}{2} \, \pi}^{\frac{5}{3} \, \pi} {-12 \, \cot\left(x\right) \csc\left(x\right)}\;dx = {\frac{12}{\sin\left(x\right)}}\Bigg\vert_{\frac{1}{2} \, \pi}^{\frac{5}{3} \, \pi} = {-8 \, \sqrt{3} - 12}\]

\input{2311_Concept_Integral_0003.HELP.tex}

\begin{multipleChoice}
\choice The antiderivative is incorrect.
\choice[correct] The integrand is not defined over the entire interval.
\choice The bounds are evaluated in the wrong order.
\choice Nothing is wrong.  The equation is correct, as is.
\end{multipleChoice}

\end{problem}}%}

%%%%%%%%%%%%%%%%%%%%%%


\latexProblemContent{
\begin{problem}

What is wrong with the following equation:

\[\int_{\frac{3}{4} \, \pi}^{\frac{5}{3} \, \pi} {-4 \, \csc\left(x\right)^{2}}\;dx = {\frac{4}{\tan\left(x\right)}}\Bigg\vert_{\frac{3}{4} \, \pi}^{\frac{5}{3} \, \pi} = {-\frac{4}{3} \, \sqrt{3} + 4}\]

\input{2311_Concept_Integral_0003.HELP.tex}

\begin{multipleChoice}
\choice The antiderivative is incorrect.
\choice[correct] The integrand is not defined over the entire interval.
\choice The bounds are evaluated in the wrong order.
\choice Nothing is wrong.  The equation is correct, as is.
\end{multipleChoice}

\end{problem}}%}

%%%%%%%%%%%%%%%%%%%%%%


\latexProblemContent{
\begin{problem}

What is wrong with the following equation:

\[\int_{\frac{1}{2} \, \pi}^{\frac{5}{4} \, \pi} {-4 \, \csc\left(x\right)^{2}}\;dx = {\frac{4}{\tan\left(x\right)}}\Bigg\vert_{\frac{1}{2} \, \pi}^{\frac{5}{4} \, \pi} = {4}\]

\input{2311_Concept_Integral_0003.HELP.tex}

\begin{multipleChoice}
\choice The antiderivative is incorrect.
\choice[correct] The integrand is not defined over the entire interval.
\choice The bounds are evaluated in the wrong order.
\choice Nothing is wrong.  The equation is correct, as is.
\end{multipleChoice}

\end{problem}}%}

%%%%%%%%%%%%%%%%%%%%%%


\latexProblemContent{
\begin{problem}

What is wrong with the following equation:

\[\int_{\frac{1}{2} \, \pi}^{\frac{5}{4} \, \pi} {\cot\left(x\right) \csc\left(x\right)}\;dx = {-\frac{1}{\sin\left(x\right)}}\Bigg\vert_{\frac{1}{2} \, \pi}^{\frac{5}{4} \, \pi} = {\sqrt{2} + 1}\]

\input{2311_Concept_Integral_0003.HELP.tex}

\begin{multipleChoice}
\choice The antiderivative is incorrect.
\choice[correct] The integrand is not defined over the entire interval.
\choice The bounds are evaluated in the wrong order.
\choice Nothing is wrong.  The equation is correct, as is.
\end{multipleChoice}

\end{problem}}%}

%%%%%%%%%%%%%%%%%%%%%%


\latexProblemContent{
\begin{problem}

What is wrong with the following equation:

\[\int_{\frac{2}{3} \, \pi}^{\frac{7}{4} \, \pi} {12 \, \cot\left(x\right) \csc\left(x\right)}\;dx = {-\frac{12}{\sin\left(x\right)}}\Bigg\vert_{\frac{2}{3} \, \pi}^{\frac{7}{4} \, \pi} = {8 \, \sqrt{3} + 12 \, \sqrt{2}}\]

\input{2311_Concept_Integral_0003.HELP.tex}

\begin{multipleChoice}
\choice The antiderivative is incorrect.
\choice[correct] The integrand is not defined over the entire interval.
\choice The bounds are evaluated in the wrong order.
\choice Nothing is wrong.  The equation is correct, as is.
\end{multipleChoice}

\end{problem}}%}

%%%%%%%%%%%%%%%%%%%%%%


%%%%%%%%%%%%%%%%%%%%%%


%%%%%%%%%%%%%%%%%%%%%%


\latexProblemContent{
\begin{problem}

What is wrong with the following equation:

\[\int_{\frac{1}{2} \, \pi}^{\frac{5}{3} \, \pi} {8 \, \csc\left(x\right)^{2}}\;dx = {-\frac{8}{\tan\left(x\right)}}\Bigg\vert_{\frac{1}{2} \, \pi}^{\frac{5}{3} \, \pi} = {\frac{8}{3} \, \sqrt{3}}\]

\input{2311_Concept_Integral_0003.HELP.tex}

\begin{multipleChoice}
\choice The antiderivative is incorrect.
\choice[correct] The integrand is not defined over the entire interval.
\choice The bounds are evaluated in the wrong order.
\choice Nothing is wrong.  The equation is correct, as is.
\end{multipleChoice}

\end{problem}}%}

%%%%%%%%%%%%%%%%%%%%%%


\latexProblemContent{
\begin{problem}

What is wrong with the following equation:

\[\int_{\frac{1}{4} \, \pi}^{\frac{5}{3} \, \pi} {3 \, \cot\left(x\right) \csc\left(x\right)}\;dx = {-\frac{3}{\sin\left(x\right)}}\Bigg\vert_{\frac{1}{4} \, \pi}^{\frac{5}{3} \, \pi} = {2 \, \sqrt{3} + 3 \, \sqrt{2}}\]

\input{2311_Concept_Integral_0003.HELP.tex}

\begin{multipleChoice}
\choice The antiderivative is incorrect.
\choice[correct] The integrand is not defined over the entire interval.
\choice The bounds are evaluated in the wrong order.
\choice Nothing is wrong.  The equation is correct, as is.
\end{multipleChoice}

\end{problem}}%}

%%%%%%%%%%%%%%%%%%%%%%


\latexProblemContent{
\begin{problem}

What is wrong with the following equation:

\[\int_{\frac{1}{2} \, \pi}^{\frac{5}{4} \, \pi} {7 \, \cot\left(x\right) \csc\left(x\right)}\;dx = {-\frac{7}{\sin\left(x\right)}}\Bigg\vert_{\frac{1}{2} \, \pi}^{\frac{5}{4} \, \pi} = {7 \, \sqrt{2} + 7}\]

\input{2311_Concept_Integral_0003.HELP.tex}

\begin{multipleChoice}
\choice The antiderivative is incorrect.
\choice[correct] The integrand is not defined over the entire interval.
\choice The bounds are evaluated in the wrong order.
\choice Nothing is wrong.  The equation is correct, as is.
\end{multipleChoice}

\end{problem}}%}

%%%%%%%%%%%%%%%%%%%%%%


\latexProblemContent{
\begin{problem}

What is wrong with the following equation:

\[\int_{\frac{1}{2} \, \pi}^{\frac{3}{2} \, \pi} {13 \, \csc\left(x\right)^{2}}\;dx = {-\frac{13}{\tan\left(x\right)}}\Bigg\vert_{\frac{1}{2} \, \pi}^{\frac{3}{2} \, \pi} = {0}\]

\input{2311_Concept_Integral_0003.HELP.tex}

\begin{multipleChoice}
\choice The antiderivative is incorrect.
\choice[correct] The integrand is not defined over the entire interval.
\choice The bounds are evaluated in the wrong order.
\choice Nothing is wrong.  The equation is correct, as is.
\end{multipleChoice}

\end{problem}}%}

%%%%%%%%%%%%%%%%%%%%%%


\latexProblemContent{
\begin{problem}

What is wrong with the following equation:

\[\int_{\frac{1}{4} \, \pi}^{\frac{5}{4} \, \pi} {-15 \, \csc\left(x\right)^{2}}\;dx = {\frac{15}{\tan\left(x\right)}}\Bigg\vert_{\frac{1}{4} \, \pi}^{\frac{5}{4} \, \pi} = {0}\]

\input{2311_Concept_Integral_0003.HELP.tex}

\begin{multipleChoice}
\choice The antiderivative is incorrect.
\choice[correct] The integrand is not defined over the entire interval.
\choice The bounds are evaluated in the wrong order.
\choice Nothing is wrong.  The equation is correct, as is.
\end{multipleChoice}

\end{problem}}%}

%%%%%%%%%%%%%%%%%%%%%%


\latexProblemContent{
\begin{problem}

What is wrong with the following equation:

\[\int_{\frac{1}{2} \, \pi}^{\frac{4}{3} \, \pi} {15 \, \cot\left(x\right) \csc\left(x\right)}\;dx = {-\frac{15}{\sin\left(x\right)}}\Bigg\vert_{\frac{1}{2} \, \pi}^{\frac{4}{3} \, \pi} = {10 \, \sqrt{3} + 15}\]

\input{2311_Concept_Integral_0003.HELP.tex}

\begin{multipleChoice}
\choice The antiderivative is incorrect.
\choice[correct] The integrand is not defined over the entire interval.
\choice The bounds are evaluated in the wrong order.
\choice Nothing is wrong.  The equation is correct, as is.
\end{multipleChoice}

\end{problem}}%}

%%%%%%%%%%%%%%%%%%%%%%


\latexProblemContent{
\begin{problem}

What is wrong with the following equation:

\[\int_{\frac{3}{4} \, \pi}^{\frac{5}{4} \, \pi} {-2 \, \csc\left(x\right)^{2}}\;dx = {\frac{2}{\tan\left(x\right)}}\Bigg\vert_{\frac{3}{4} \, \pi}^{\frac{5}{4} \, \pi} = {4}\]

\input{2311_Concept_Integral_0003.HELP.tex}

\begin{multipleChoice}
\choice The antiderivative is incorrect.
\choice[correct] The integrand is not defined over the entire interval.
\choice The bounds are evaluated in the wrong order.
\choice Nothing is wrong.  The equation is correct, as is.
\end{multipleChoice}

\end{problem}}%}

%%%%%%%%%%%%%%%%%%%%%%


\latexProblemContent{
\begin{problem}

What is wrong with the following equation:

\[\int_{\frac{1}{4} \, \pi}^{\frac{5}{3} \, \pi} {13 \, \csc\left(x\right)^{2}}\;dx = {-\frac{13}{\tan\left(x\right)}}\Bigg\vert_{\frac{1}{4} \, \pi}^{\frac{5}{3} \, \pi} = {\frac{13}{3} \, \sqrt{3} + 13}\]

\input{2311_Concept_Integral_0003.HELP.tex}

\begin{multipleChoice}
\choice The antiderivative is incorrect.
\choice[correct] The integrand is not defined over the entire interval.
\choice The bounds are evaluated in the wrong order.
\choice Nothing is wrong.  The equation is correct, as is.
\end{multipleChoice}

\end{problem}}%}

%%%%%%%%%%%%%%%%%%%%%%


\latexProblemContent{
\begin{problem}

What is wrong with the following equation:

\[\int_{\frac{1}{3} \, \pi}^{\frac{3}{2} \, \pi} {2 \, \cot\left(x\right) \csc\left(x\right)}\;dx = {-\frac{2}{\sin\left(x\right)}}\Bigg\vert_{\frac{1}{3} \, \pi}^{\frac{3}{2} \, \pi} = {\frac{4}{3} \, \sqrt{3} + 2}\]

\input{2311_Concept_Integral_0003.HELP.tex}

\begin{multipleChoice}
\choice The antiderivative is incorrect.
\choice[correct] The integrand is not defined over the entire interval.
\choice The bounds are evaluated in the wrong order.
\choice Nothing is wrong.  The equation is correct, as is.
\end{multipleChoice}

\end{problem}}%}

%%%%%%%%%%%%%%%%%%%%%%


\latexProblemContent{
\begin{problem}

What is wrong with the following equation:

\[\int_{\frac{2}{3} \, \pi}^{\frac{5}{4} \, \pi} {-11 \, \cot\left(x\right) \csc\left(x\right)}\;dx = {\frac{11}{\sin\left(x\right)}}\Bigg\vert_{\frac{2}{3} \, \pi}^{\frac{5}{4} \, \pi} = {-\frac{22}{3} \, \sqrt{3} - 11 \, \sqrt{2}}\]

\input{2311_Concept_Integral_0003.HELP.tex}

\begin{multipleChoice}
\choice The antiderivative is incorrect.
\choice[correct] The integrand is not defined over the entire interval.
\choice The bounds are evaluated in the wrong order.
\choice Nothing is wrong.  The equation is correct, as is.
\end{multipleChoice}

\end{problem}}%}

%%%%%%%%%%%%%%%%%%%%%%


\latexProblemContent{
\begin{problem}

What is wrong with the following equation:

\[\int_{\frac{1}{4} \, \pi}^{\frac{7}{4} \, \pi} {6 \, \cot\left(x\right) \csc\left(x\right)}\;dx = {-\frac{6}{\sin\left(x\right)}}\Bigg\vert_{\frac{1}{4} \, \pi}^{\frac{7}{4} \, \pi} = {12 \, \sqrt{2}}\]

\input{2311_Concept_Integral_0003.HELP.tex}

\begin{multipleChoice}
\choice The antiderivative is incorrect.
\choice[correct] The integrand is not defined over the entire interval.
\choice The bounds are evaluated in the wrong order.
\choice Nothing is wrong.  The equation is correct, as is.
\end{multipleChoice}

\end{problem}}%}

%%%%%%%%%%%%%%%%%%%%%%


\latexProblemContent{
\begin{problem}

What is wrong with the following equation:

\[\int_{\frac{2}{3} \, \pi}^{\frac{3}{2} \, \pi} {-13 \, \csc\left(x\right)^{2}}\;dx = {\frac{13}{\tan\left(x\right)}}\Bigg\vert_{\frac{2}{3} \, \pi}^{\frac{3}{2} \, \pi} = {\frac{13}{3} \, \sqrt{3}}\]

\input{2311_Concept_Integral_0003.HELP.tex}

\begin{multipleChoice}
\choice The antiderivative is incorrect.
\choice[correct] The integrand is not defined over the entire interval.
\choice The bounds are evaluated in the wrong order.
\choice Nothing is wrong.  The equation is correct, as is.
\end{multipleChoice}

\end{problem}}%}

%%%%%%%%%%%%%%%%%%%%%%


\latexProblemContent{
\begin{problem}

What is wrong with the following equation:

\[\int_{\frac{1}{3} \, \pi}^{\frac{5}{4} \, \pi} {-5 \, \csc\left(x\right)^{2}}\;dx = {\frac{5}{\tan\left(x\right)}}\Bigg\vert_{\frac{1}{3} \, \pi}^{\frac{5}{4} \, \pi} = {-\frac{5}{3} \, \sqrt{3} + 5}\]

\input{2311_Concept_Integral_0003.HELP.tex}

\begin{multipleChoice}
\choice The antiderivative is incorrect.
\choice[correct] The integrand is not defined over the entire interval.
\choice The bounds are evaluated in the wrong order.
\choice Nothing is wrong.  The equation is correct, as is.
\end{multipleChoice}

\end{problem}}%}

%%%%%%%%%%%%%%%%%%%%%%


\latexProblemContent{
\begin{problem}

What is wrong with the following equation:

\[\int_{\frac{1}{2} \, \pi}^{\frac{3}{2} \, \pi} {6 \, \cot\left(x\right) \csc\left(x\right)}\;dx = {-\frac{6}{\sin\left(x\right)}}\Bigg\vert_{\frac{1}{2} \, \pi}^{\frac{3}{2} \, \pi} = {12}\]

\input{2311_Concept_Integral_0003.HELP.tex}

\begin{multipleChoice}
\choice The antiderivative is incorrect.
\choice[correct] The integrand is not defined over the entire interval.
\choice The bounds are evaluated in the wrong order.
\choice Nothing is wrong.  The equation is correct, as is.
\end{multipleChoice}

\end{problem}}%}

%%%%%%%%%%%%%%%%%%%%%%


\latexProblemContent{
\begin{problem}

What is wrong with the following equation:

\[\int_{\frac{1}{3} \, \pi}^{\frac{3}{2} \, \pi} {-\csc\left(x\right)^{2}}\;dx = {\frac{1}{\tan\left(x\right)}}\Bigg\vert_{\frac{1}{3} \, \pi}^{\frac{3}{2} \, \pi} = {-\frac{1}{3} \, \sqrt{3}}\]

\input{2311_Concept_Integral_0003.HELP.tex}

\begin{multipleChoice}
\choice The antiderivative is incorrect.
\choice[correct] The integrand is not defined over the entire interval.
\choice The bounds are evaluated in the wrong order.
\choice Nothing is wrong.  The equation is correct, as is.
\end{multipleChoice}

\end{problem}}%}

%%%%%%%%%%%%%%%%%%%%%%


\latexProblemContent{
\begin{problem}

What is wrong with the following equation:

\[\int_{\frac{2}{3} \, \pi}^{\frac{3}{2} \, \pi} {2 \, \csc\left(x\right)^{2}}\;dx = {-\frac{2}{\tan\left(x\right)}}\Bigg\vert_{\frac{2}{3} \, \pi}^{\frac{3}{2} \, \pi} = {-\frac{2}{3} \, \sqrt{3}}\]

\input{2311_Concept_Integral_0003.HELP.tex}

\begin{multipleChoice}
\choice The antiderivative is incorrect.
\choice[correct] The integrand is not defined over the entire interval.
\choice The bounds are evaluated in the wrong order.
\choice Nothing is wrong.  The equation is correct, as is.
\end{multipleChoice}

\end{problem}}%}

%%%%%%%%%%%%%%%%%%%%%%


\latexProblemContent{
\begin{problem}

What is wrong with the following equation:

\[\int_{\frac{1}{3} \, \pi}^{\frac{5}{4} \, \pi} {-3 \, \cot\left(x\right) \csc\left(x\right)}\;dx = {\frac{3}{\sin\left(x\right)}}\Bigg\vert_{\frac{1}{3} \, \pi}^{\frac{5}{4} \, \pi} = {-2 \, \sqrt{3} - 3 \, \sqrt{2}}\]

\input{2311_Concept_Integral_0003.HELP.tex}

\begin{multipleChoice}
\choice The antiderivative is incorrect.
\choice[correct] The integrand is not defined over the entire interval.
\choice The bounds are evaluated in the wrong order.
\choice Nothing is wrong.  The equation is correct, as is.
\end{multipleChoice}

\end{problem}}%}

%%%%%%%%%%%%%%%%%%%%%%


\latexProblemContent{
\begin{problem}

What is wrong with the following equation:

\[\int_{\frac{1}{3} \, \pi}^{\frac{4}{3} \, \pi} {-\csc\left(x\right)^{2}}\;dx = {\frac{1}{\tan\left(x\right)}}\Bigg\vert_{\frac{1}{3} \, \pi}^{\frac{4}{3} \, \pi} = {0}\]

\input{2311_Concept_Integral_0003.HELP.tex}

\begin{multipleChoice}
\choice The antiderivative is incorrect.
\choice[correct] The integrand is not defined over the entire interval.
\choice The bounds are evaluated in the wrong order.
\choice Nothing is wrong.  The equation is correct, as is.
\end{multipleChoice}

\end{problem}}%}

%%%%%%%%%%%%%%%%%%%%%%


\latexProblemContent{
\begin{problem}

What is wrong with the following equation:

\[\int_{\frac{2}{3} \, \pi}^{\frac{5}{4} \, \pi} {11 \, \csc\left(x\right)^{2}}\;dx = {-\frac{11}{\tan\left(x\right)}}\Bigg\vert_{\frac{2}{3} \, \pi}^{\frac{5}{4} \, \pi} = {-\frac{11}{3} \, \sqrt{3} - 11}\]

\input{2311_Concept_Integral_0003.HELP.tex}

\begin{multipleChoice}
\choice The antiderivative is incorrect.
\choice[correct] The integrand is not defined over the entire interval.
\choice The bounds are evaluated in the wrong order.
\choice Nothing is wrong.  The equation is correct, as is.
\end{multipleChoice}

\end{problem}}%}

%%%%%%%%%%%%%%%%%%%%%%


\latexProblemContent{
\begin{problem}

What is wrong with the following equation:

\[\int_{\frac{3}{4} \, \pi}^{\frac{4}{3} \, \pi} {11 \, \csc\left(x\right)^{2}}\;dx = {-\frac{11}{\tan\left(x\right)}}\Bigg\vert_{\frac{3}{4} \, \pi}^{\frac{4}{3} \, \pi} = {-\frac{11}{3} \, \sqrt{3} - 11}\]

\input{2311_Concept_Integral_0003.HELP.tex}

\begin{multipleChoice}
\choice The antiderivative is incorrect.
\choice[correct] The integrand is not defined over the entire interval.
\choice The bounds are evaluated in the wrong order.
\choice Nothing is wrong.  The equation is correct, as is.
\end{multipleChoice}

\end{problem}}%}

%%%%%%%%%%%%%%%%%%%%%%


\latexProblemContent{
\begin{problem}

What is wrong with the following equation:

\[\int_{\frac{2}{3} \, \pi}^{\frac{7}{4} \, \pi} {-12 \, \csc\left(x\right)^{2}}\;dx = {\frac{12}{\tan\left(x\right)}}\Bigg\vert_{\frac{2}{3} \, \pi}^{\frac{7}{4} \, \pi} = {4 \, \sqrt{3} - 12}\]

\input{2311_Concept_Integral_0003.HELP.tex}

\begin{multipleChoice}
\choice The antiderivative is incorrect.
\choice[correct] The integrand is not defined over the entire interval.
\choice The bounds are evaluated in the wrong order.
\choice Nothing is wrong.  The equation is correct, as is.
\end{multipleChoice}

\end{problem}}%}

%%%%%%%%%%%%%%%%%%%%%%


\latexProblemContent{
\begin{problem}

What is wrong with the following equation:

\[\int_{\frac{2}{3} \, \pi}^{\frac{3}{2} \, \pi} {10 \, \csc\left(x\right)^{2}}\;dx = {-\frac{10}{\tan\left(x\right)}}\Bigg\vert_{\frac{2}{3} \, \pi}^{\frac{3}{2} \, \pi} = {-\frac{10}{3} \, \sqrt{3}}\]

\input{2311_Concept_Integral_0003.HELP.tex}

\begin{multipleChoice}
\choice The antiderivative is incorrect.
\choice[correct] The integrand is not defined over the entire interval.
\choice The bounds are evaluated in the wrong order.
\choice Nothing is wrong.  The equation is correct, as is.
\end{multipleChoice}

\end{problem}}%}

%%%%%%%%%%%%%%%%%%%%%%


\latexProblemContent{
\begin{problem}

What is wrong with the following equation:

\[\int_{\frac{1}{4} \, \pi}^{\frac{5}{4} \, \pi} {15 \, \cot\left(x\right) \csc\left(x\right)}\;dx = {-\frac{15}{\sin\left(x\right)}}\Bigg\vert_{\frac{1}{4} \, \pi}^{\frac{5}{4} \, \pi} = {30 \, \sqrt{2}}\]

\input{2311_Concept_Integral_0003.HELP.tex}

\begin{multipleChoice}
\choice The antiderivative is incorrect.
\choice[correct] The integrand is not defined over the entire interval.
\choice The bounds are evaluated in the wrong order.
\choice Nothing is wrong.  The equation is correct, as is.
\end{multipleChoice}

\end{problem}}%}

%%%%%%%%%%%%%%%%%%%%%%


\latexProblemContent{
\begin{problem}

What is wrong with the following equation:

\[\int_{\frac{1}{4} \, \pi}^{\frac{5}{4} \, \pi} {4 \, \csc\left(x\right)^{2}}\;dx = {-\frac{4}{\tan\left(x\right)}}\Bigg\vert_{\frac{1}{4} \, \pi}^{\frac{5}{4} \, \pi} = {0}\]

\input{2311_Concept_Integral_0003.HELP.tex}

\begin{multipleChoice}
\choice The antiderivative is incorrect.
\choice[correct] The integrand is not defined over the entire interval.
\choice The bounds are evaluated in the wrong order.
\choice Nothing is wrong.  The equation is correct, as is.
\end{multipleChoice}

\end{problem}}%}

%%%%%%%%%%%%%%%%%%%%%%


\latexProblemContent{
\begin{problem}

What is wrong with the following equation:

\[\int_{\frac{3}{4} \, \pi}^{\frac{3}{2} \, \pi} {6 \, \csc\left(x\right)^{2}}\;dx = {-\frac{6}{\tan\left(x\right)}}\Bigg\vert_{\frac{3}{4} \, \pi}^{\frac{3}{2} \, \pi} = {-6}\]

\input{2311_Concept_Integral_0003.HELP.tex}

\begin{multipleChoice}
\choice The antiderivative is incorrect.
\choice[correct] The integrand is not defined over the entire interval.
\choice The bounds are evaluated in the wrong order.
\choice Nothing is wrong.  The equation is correct, as is.
\end{multipleChoice}

\end{problem}}%}

%%%%%%%%%%%%%%%%%%%%%%


\latexProblemContent{
\begin{problem}

What is wrong with the following equation:

\[\int_{\frac{1}{2} \, \pi}^{\frac{5}{3} \, \pi} {10 \, \cot\left(x\right) \csc\left(x\right)}\;dx = {-\frac{10}{\sin\left(x\right)}}\Bigg\vert_{\frac{1}{2} \, \pi}^{\frac{5}{3} \, \pi} = {\frac{20}{3} \, \sqrt{3} + 10}\]

\input{2311_Concept_Integral_0003.HELP.tex}

\begin{multipleChoice}
\choice The antiderivative is incorrect.
\choice[correct] The integrand is not defined over the entire interval.
\choice The bounds are evaluated in the wrong order.
\choice Nothing is wrong.  The equation is correct, as is.
\end{multipleChoice}

\end{problem}}%}

%%%%%%%%%%%%%%%%%%%%%%


\latexProblemContent{
\begin{problem}

What is wrong with the following equation:

\[\int_{\frac{1}{4} \, \pi}^{\frac{4}{3} \, \pi} {-13 \, \csc\left(x\right)^{2}}\;dx = {\frac{13}{\tan\left(x\right)}}\Bigg\vert_{\frac{1}{4} \, \pi}^{\frac{4}{3} \, \pi} = {\frac{13}{3} \, \sqrt{3} - 13}\]

\input{2311_Concept_Integral_0003.HELP.tex}

\begin{multipleChoice}
\choice The antiderivative is incorrect.
\choice[correct] The integrand is not defined over the entire interval.
\choice The bounds are evaluated in the wrong order.
\choice Nothing is wrong.  The equation is correct, as is.
\end{multipleChoice}

\end{problem}}%}

%%%%%%%%%%%%%%%%%%%%%%


\latexProblemContent{
\begin{problem}

What is wrong with the following equation:

\[\int_{\frac{1}{3} \, \pi}^{\frac{4}{3} \, \pi} {7 \, \cot\left(x\right) \csc\left(x\right)}\;dx = {-\frac{7}{\sin\left(x\right)}}\Bigg\vert_{\frac{1}{3} \, \pi}^{\frac{4}{3} \, \pi} = {\frac{28}{3} \, \sqrt{3}}\]

\input{2311_Concept_Integral_0003.HELP.tex}

\begin{multipleChoice}
\choice The antiderivative is incorrect.
\choice[correct] The integrand is not defined over the entire interval.
\choice The bounds are evaluated in the wrong order.
\choice Nothing is wrong.  The equation is correct, as is.
\end{multipleChoice}

\end{problem}}%}

%%%%%%%%%%%%%%%%%%%%%%


\latexProblemContent{
\begin{problem}

What is wrong with the following equation:

\[\int_{\frac{1}{2} \, \pi}^{\frac{5}{4} \, \pi} {13 \, \csc\left(x\right)^{2}}\;dx = {-\frac{13}{\tan\left(x\right)}}\Bigg\vert_{\frac{1}{2} \, \pi}^{\frac{5}{4} \, \pi} = {-13}\]

\input{2311_Concept_Integral_0003.HELP.tex}

\begin{multipleChoice}
\choice The antiderivative is incorrect.
\choice[correct] The integrand is not defined over the entire interval.
\choice The bounds are evaluated in the wrong order.
\choice Nothing is wrong.  The equation is correct, as is.
\end{multipleChoice}

\end{problem}}%}

%%%%%%%%%%%%%%%%%%%%%%


\latexProblemContent{
\begin{problem}

What is wrong with the following equation:

\[\int_{\frac{2}{3} \, \pi}^{\frac{4}{3} \, \pi} {7 \, \cot\left(x\right) \csc\left(x\right)}\;dx = {-\frac{7}{\sin\left(x\right)}}\Bigg\vert_{\frac{2}{3} \, \pi}^{\frac{4}{3} \, \pi} = {\frac{28}{3} \, \sqrt{3}}\]

\input{2311_Concept_Integral_0003.HELP.tex}

\begin{multipleChoice}
\choice The antiderivative is incorrect.
\choice[correct] The integrand is not defined over the entire interval.
\choice The bounds are evaluated in the wrong order.
\choice Nothing is wrong.  The equation is correct, as is.
\end{multipleChoice}

\end{problem}}%}

%%%%%%%%%%%%%%%%%%%%%%


\latexProblemContent{
\begin{problem}

What is wrong with the following equation:

\[\int_{\frac{1}{2} \, \pi}^{\frac{5}{4} \, \pi} {5 \, \csc\left(x\right)^{2}}\;dx = {-\frac{5}{\tan\left(x\right)}}\Bigg\vert_{\frac{1}{2} \, \pi}^{\frac{5}{4} \, \pi} = {-5}\]

\input{2311_Concept_Integral_0003.HELP.tex}

\begin{multipleChoice}
\choice The antiderivative is incorrect.
\choice[correct] The integrand is not defined over the entire interval.
\choice The bounds are evaluated in the wrong order.
\choice Nothing is wrong.  The equation is correct, as is.
\end{multipleChoice}

\end{problem}}%}

%%%%%%%%%%%%%%%%%%%%%%


\latexProblemContent{
\begin{problem}

What is wrong with the following equation:

\[\int_{\frac{2}{3} \, \pi}^{\frac{3}{2} \, \pi} {-10 \, \csc\left(x\right)^{2}}\;dx = {\frac{10}{\tan\left(x\right)}}\Bigg\vert_{\frac{2}{3} \, \pi}^{\frac{3}{2} \, \pi} = {\frac{10}{3} \, \sqrt{3}}\]

\input{2311_Concept_Integral_0003.HELP.tex}

\begin{multipleChoice}
\choice The antiderivative is incorrect.
\choice[correct] The integrand is not defined over the entire interval.
\choice The bounds are evaluated in the wrong order.
\choice Nothing is wrong.  The equation is correct, as is.
\end{multipleChoice}

\end{problem}}%}

%%%%%%%%%%%%%%%%%%%%%%


\latexProblemContent{
\begin{problem}

What is wrong with the following equation:

\[\int_{\frac{2}{3} \, \pi}^{\frac{7}{4} \, \pi} {14 \, \cot\left(x\right) \csc\left(x\right)}\;dx = {-\frac{14}{\sin\left(x\right)}}\Bigg\vert_{\frac{2}{3} \, \pi}^{\frac{7}{4} \, \pi} = {\frac{28}{3} \, \sqrt{3} + 14 \, \sqrt{2}}\]

\input{2311_Concept_Integral_0003.HELP.tex}

\begin{multipleChoice}
\choice The antiderivative is incorrect.
\choice[correct] The integrand is not defined over the entire interval.
\choice The bounds are evaluated in the wrong order.
\choice Nothing is wrong.  The equation is correct, as is.
\end{multipleChoice}

\end{problem}}%}

%%%%%%%%%%%%%%%%%%%%%%


%%%%%%%%%%%%%%%%%%%%%%


\latexProblemContent{
\begin{problem}

What is wrong with the following equation:

\[\int_{\frac{1}{4} \, \pi}^{\frac{5}{4} \, \pi} {-15 \, \cot\left(x\right) \csc\left(x\right)}\;dx = {\frac{15}{\sin\left(x\right)}}\Bigg\vert_{\frac{1}{4} \, \pi}^{\frac{5}{4} \, \pi} = {-30 \, \sqrt{2}}\]

\input{2311_Concept_Integral_0003.HELP.tex}

\begin{multipleChoice}
\choice The antiderivative is incorrect.
\choice[correct] The integrand is not defined over the entire interval.
\choice The bounds are evaluated in the wrong order.
\choice Nothing is wrong.  The equation is correct, as is.
\end{multipleChoice}

\end{problem}}%}

%%%%%%%%%%%%%%%%%%%%%%


\latexProblemContent{
\begin{problem}

What is wrong with the following equation:

\[\int_{\frac{1}{2} \, \pi}^{\frac{3}{2} \, \pi} {-7 \, \csc\left(x\right)^{2}}\;dx = {\frac{7}{\tan\left(x\right)}}\Bigg\vert_{\frac{1}{2} \, \pi}^{\frac{3}{2} \, \pi} = {0}\]

\input{2311_Concept_Integral_0003.HELP.tex}

\begin{multipleChoice}
\choice The antiderivative is incorrect.
\choice[correct] The integrand is not defined over the entire interval.
\choice The bounds are evaluated in the wrong order.
\choice Nothing is wrong.  The equation is correct, as is.
\end{multipleChoice}

\end{problem}}%}

%%%%%%%%%%%%%%%%%%%%%%


\latexProblemContent{
\begin{problem}

What is wrong with the following equation:

\[\int_{\frac{2}{3} \, \pi}^{\frac{7}{4} \, \pi} {11 \, \csc\left(x\right)^{2}}\;dx = {-\frac{11}{\tan\left(x\right)}}\Bigg\vert_{\frac{2}{3} \, \pi}^{\frac{7}{4} \, \pi} = {-\frac{11}{3} \, \sqrt{3} + 11}\]

\input{2311_Concept_Integral_0003.HELP.tex}

\begin{multipleChoice}
\choice The antiderivative is incorrect.
\choice[correct] The integrand is not defined over the entire interval.
\choice The bounds are evaluated in the wrong order.
\choice Nothing is wrong.  The equation is correct, as is.
\end{multipleChoice}

\end{problem}}%}

%%%%%%%%%%%%%%%%%%%%%%


%%%%%%%%%%%%%%%%%%%%%%


\latexProblemContent{
\begin{problem}

What is wrong with the following equation:

\[\int_{\frac{1}{4} \, \pi}^{\frac{3}{2} \, \pi} {-4 \, \cot\left(x\right) \csc\left(x\right)}\;dx = {\frac{4}{\sin\left(x\right)}}\Bigg\vert_{\frac{1}{4} \, \pi}^{\frac{3}{2} \, \pi} = {-4 \, \sqrt{2} - 4}\]

\input{2311_Concept_Integral_0003.HELP.tex}

\begin{multipleChoice}
\choice The antiderivative is incorrect.
\choice[correct] The integrand is not defined over the entire interval.
\choice The bounds are evaluated in the wrong order.
\choice Nothing is wrong.  The equation is correct, as is.
\end{multipleChoice}

\end{problem}}%}

%%%%%%%%%%%%%%%%%%%%%%


%%%%%%%%%%%%%%%%%%%%%%


%%%%%%%%%%%%%%%%%%%%%%


\latexProblemContent{
\begin{problem}

What is wrong with the following equation:

\[\int_{\frac{3}{4} \, \pi}^{\frac{3}{2} \, \pi} {-15 \, \cot\left(x\right) \csc\left(x\right)}\;dx = {\frac{15}{\sin\left(x\right)}}\Bigg\vert_{\frac{3}{4} \, \pi}^{\frac{3}{2} \, \pi} = {-15 \, \sqrt{2} - 15}\]

\input{2311_Concept_Integral_0003.HELP.tex}

\begin{multipleChoice}
\choice The antiderivative is incorrect.
\choice[correct] The integrand is not defined over the entire interval.
\choice The bounds are evaluated in the wrong order.
\choice Nothing is wrong.  The equation is correct, as is.
\end{multipleChoice}

\end{problem}}%}

%%%%%%%%%%%%%%%%%%%%%%


\latexProblemContent{
\begin{problem}

What is wrong with the following equation:

\[\int_{\frac{2}{3} \, \pi}^{\frac{5}{4} \, \pi} {-3 \, \csc\left(x\right)^{2}}\;dx = {\frac{3}{\tan\left(x\right)}}\Bigg\vert_{\frac{2}{3} \, \pi}^{\frac{5}{4} \, \pi} = {\sqrt{3} + 3}\]

\input{2311_Concept_Integral_0003.HELP.tex}

\begin{multipleChoice}
\choice The antiderivative is incorrect.
\choice[correct] The integrand is not defined over the entire interval.
\choice The bounds are evaluated in the wrong order.
\choice Nothing is wrong.  The equation is correct, as is.
\end{multipleChoice}

\end{problem}}%}

%%%%%%%%%%%%%%%%%%%%%%


\latexProblemContent{
\begin{problem}

What is wrong with the following equation:

\[\int_{\frac{1}{4} \, \pi}^{\frac{5}{3} \, \pi} {-15 \, \csc\left(x\right)^{2}}\;dx = {\frac{15}{\tan\left(x\right)}}\Bigg\vert_{\frac{1}{4} \, \pi}^{\frac{5}{3} \, \pi} = {-5 \, \sqrt{3} - 15}\]

\input{2311_Concept_Integral_0003.HELP.tex}

\begin{multipleChoice}
\choice The antiderivative is incorrect.
\choice[correct] The integrand is not defined over the entire interval.
\choice The bounds are evaluated in the wrong order.
\choice Nothing is wrong.  The equation is correct, as is.
\end{multipleChoice}

\end{problem}}%}

%%%%%%%%%%%%%%%%%%%%%%


\latexProblemContent{
\begin{problem}

What is wrong with the following equation:

\[\int_{\frac{1}{4} \, \pi}^{\frac{5}{3} \, \pi} {-9 \, \cot\left(x\right) \csc\left(x\right)}\;dx = {\frac{9}{\sin\left(x\right)}}\Bigg\vert_{\frac{1}{4} \, \pi}^{\frac{5}{3} \, \pi} = {-6 \, \sqrt{3} - 9 \, \sqrt{2}}\]

\input{2311_Concept_Integral_0003.HELP.tex}

\begin{multipleChoice}
\choice The antiderivative is incorrect.
\choice[correct] The integrand is not defined over the entire interval.
\choice The bounds are evaluated in the wrong order.
\choice Nothing is wrong.  The equation is correct, as is.
\end{multipleChoice}

\end{problem}}%}

%%%%%%%%%%%%%%%%%%%%%%


\latexProblemContent{
\begin{problem}

What is wrong with the following equation:

\[\int_{\frac{3}{4} \, \pi}^{\frac{7}{4} \, \pi} {-3 \, \csc\left(x\right)^{2}}\;dx = {\frac{3}{\tan\left(x\right)}}\Bigg\vert_{\frac{3}{4} \, \pi}^{\frac{7}{4} \, \pi} = {0}\]

\input{2311_Concept_Integral_0003.HELP.tex}

\begin{multipleChoice}
\choice The antiderivative is incorrect.
\choice[correct] The integrand is not defined over the entire interval.
\choice The bounds are evaluated in the wrong order.
\choice Nothing is wrong.  The equation is correct, as is.
\end{multipleChoice}

\end{problem}}%}

%%%%%%%%%%%%%%%%%%%%%%


%%%%%%%%%%%%%%%%%%%%%%


\latexProblemContent{
\begin{problem}

What is wrong with the following equation:

\[\int_{\frac{1}{4} \, \pi}^{\frac{3}{2} \, \pi} {-11 \, \csc\left(x\right)^{2}}\;dx = {\frac{11}{\tan\left(x\right)}}\Bigg\vert_{\frac{1}{4} \, \pi}^{\frac{3}{2} \, \pi} = {-11}\]

\input{2311_Concept_Integral_0003.HELP.tex}

\begin{multipleChoice}
\choice The antiderivative is incorrect.
\choice[correct] The integrand is not defined over the entire interval.
\choice The bounds are evaluated in the wrong order.
\choice Nothing is wrong.  The equation is correct, as is.
\end{multipleChoice}

\end{problem}}%}

%%%%%%%%%%%%%%%%%%%%%%


\latexProblemContent{
\begin{problem}

What is wrong with the following equation:

\[\int_{\frac{3}{4} \, \pi}^{\frac{3}{2} \, \pi} {6 \, \cot\left(x\right) \csc\left(x\right)}\;dx = {-\frac{6}{\sin\left(x\right)}}\Bigg\vert_{\frac{3}{4} \, \pi}^{\frac{3}{2} \, \pi} = {6 \, \sqrt{2} + 6}\]

\input{2311_Concept_Integral_0003.HELP.tex}

\begin{multipleChoice}
\choice The antiderivative is incorrect.
\choice[correct] The integrand is not defined over the entire interval.
\choice The bounds are evaluated in the wrong order.
\choice Nothing is wrong.  The equation is correct, as is.
\end{multipleChoice}

\end{problem}}%}

%%%%%%%%%%%%%%%%%%%%%%


\latexProblemContent{
\begin{problem}

What is wrong with the following equation:

\[\int_{\frac{1}{3} \, \pi}^{\frac{5}{3} \, \pi} {5 \, \cot\left(x\right) \csc\left(x\right)}\;dx = {-\frac{5}{\sin\left(x\right)}}\Bigg\vert_{\frac{1}{3} \, \pi}^{\frac{5}{3} \, \pi} = {\frac{20}{3} \, \sqrt{3}}\]

\input{2311_Concept_Integral_0003.HELP.tex}

\begin{multipleChoice}
\choice The antiderivative is incorrect.
\choice[correct] The integrand is not defined over the entire interval.
\choice The bounds are evaluated in the wrong order.
\choice Nothing is wrong.  The equation is correct, as is.
\end{multipleChoice}

\end{problem}}%}

%%%%%%%%%%%%%%%%%%%%%%


\latexProblemContent{
\begin{problem}

What is wrong with the following equation:

\[\int_{\frac{1}{4} \, \pi}^{\frac{5}{4} \, \pi} {-5 \, \csc\left(x\right)^{2}}\;dx = {\frac{5}{\tan\left(x\right)}}\Bigg\vert_{\frac{1}{4} \, \pi}^{\frac{5}{4} \, \pi} = {0}\]

\input{2311_Concept_Integral_0003.HELP.tex}

\begin{multipleChoice}
\choice The antiderivative is incorrect.
\choice[correct] The integrand is not defined over the entire interval.
\choice The bounds are evaluated in the wrong order.
\choice Nothing is wrong.  The equation is correct, as is.
\end{multipleChoice}

\end{problem}}%}

%%%%%%%%%%%%%%%%%%%%%%


%%%%%%%%%%%%%%%%%%%%%%


\latexProblemContent{
\begin{problem}

What is wrong with the following equation:

\[\int_{\frac{1}{3} \, \pi}^{\frac{5}{3} \, \pi} {-14 \, \cot\left(x\right) \csc\left(x\right)}\;dx = {\frac{14}{\sin\left(x\right)}}\Bigg\vert_{\frac{1}{3} \, \pi}^{\frac{5}{3} \, \pi} = {-\frac{56}{3} \, \sqrt{3}}\]

\input{2311_Concept_Integral_0003.HELP.tex}

\begin{multipleChoice}
\choice The antiderivative is incorrect.
\choice[correct] The integrand is not defined over the entire interval.
\choice The bounds are evaluated in the wrong order.
\choice Nothing is wrong.  The equation is correct, as is.
\end{multipleChoice}

\end{problem}}%}

%%%%%%%%%%%%%%%%%%%%%%


%%%%%%%%%%%%%%%%%%%%%%


%%%%%%%%%%%%%%%%%%%%%%


\latexProblemContent{
\begin{problem}

What is wrong with the following equation:

\[\int_{\frac{1}{3} \, \pi}^{\frac{5}{3} \, \pi} {-2 \, \cot\left(x\right) \csc\left(x\right)}\;dx = {\frac{2}{\sin\left(x\right)}}\Bigg\vert_{\frac{1}{3} \, \pi}^{\frac{5}{3} \, \pi} = {-\frac{8}{3} \, \sqrt{3}}\]

\input{2311_Concept_Integral_0003.HELP.tex}

\begin{multipleChoice}
\choice The antiderivative is incorrect.
\choice[correct] The integrand is not defined over the entire interval.
\choice The bounds are evaluated in the wrong order.
\choice Nothing is wrong.  The equation is correct, as is.
\end{multipleChoice}

\end{problem}}%}

%%%%%%%%%%%%%%%%%%%%%%


\latexProblemContent{
\begin{problem}

What is wrong with the following equation:

\[\int_{\frac{1}{3} \, \pi}^{\frac{3}{2} \, \pi} {-10 \, \cot\left(x\right) \csc\left(x\right)}\;dx = {\frac{10}{\sin\left(x\right)}}\Bigg\vert_{\frac{1}{3} \, \pi}^{\frac{3}{2} \, \pi} = {-\frac{20}{3} \, \sqrt{3} - 10}\]

\input{2311_Concept_Integral_0003.HELP.tex}

\begin{multipleChoice}
\choice The antiderivative is incorrect.
\choice[correct] The integrand is not defined over the entire interval.
\choice The bounds are evaluated in the wrong order.
\choice Nothing is wrong.  The equation is correct, as is.
\end{multipleChoice}

\end{problem}}%}

%%%%%%%%%%%%%%%%%%%%%%


\latexProblemContent{
\begin{problem}

What is wrong with the following equation:

\[\int_{\frac{3}{4} \, \pi}^{\frac{5}{3} \, \pi} {10 \, \cot\left(x\right) \csc\left(x\right)}\;dx = {-\frac{10}{\sin\left(x\right)}}\Bigg\vert_{\frac{3}{4} \, \pi}^{\frac{5}{3} \, \pi} = {\frac{20}{3} \, \sqrt{3} + 10 \, \sqrt{2}}\]

\input{2311_Concept_Integral_0003.HELP.tex}

\begin{multipleChoice}
\choice The antiderivative is incorrect.
\choice[correct] The integrand is not defined over the entire interval.
\choice The bounds are evaluated in the wrong order.
\choice Nothing is wrong.  The equation is correct, as is.
\end{multipleChoice}

\end{problem}}%}

%%%%%%%%%%%%%%%%%%%%%%


\latexProblemContent{
\begin{problem}

What is wrong with the following equation:

\[\int_{\frac{2}{3} \, \pi}^{\frac{5}{4} \, \pi} {7 \, \csc\left(x\right)^{2}}\;dx = {-\frac{7}{\tan\left(x\right)}}\Bigg\vert_{\frac{2}{3} \, \pi}^{\frac{5}{4} \, \pi} = {-\frac{7}{3} \, \sqrt{3} - 7}\]

\input{2311_Concept_Integral_0003.HELP.tex}

\begin{multipleChoice}
\choice The antiderivative is incorrect.
\choice[correct] The integrand is not defined over the entire interval.
\choice The bounds are evaluated in the wrong order.
\choice Nothing is wrong.  The equation is correct, as is.
\end{multipleChoice}

\end{problem}}%}

%%%%%%%%%%%%%%%%%%%%%%


\latexProblemContent{
\begin{problem}

What is wrong with the following equation:

\[\int_{\frac{2}{3} \, \pi}^{\frac{5}{3} \, \pi} {-8 \, \csc\left(x\right)^{2}}\;dx = {\frac{8}{\tan\left(x\right)}}\Bigg\vert_{\frac{2}{3} \, \pi}^{\frac{5}{3} \, \pi} = {0}\]

\input{2311_Concept_Integral_0003.HELP.tex}

\begin{multipleChoice}
\choice The antiderivative is incorrect.
\choice[correct] The integrand is not defined over the entire interval.
\choice The bounds are evaluated in the wrong order.
\choice Nothing is wrong.  The equation is correct, as is.
\end{multipleChoice}

\end{problem}}%}

%%%%%%%%%%%%%%%%%%%%%%


%%%%%%%%%%%%%%%%%%%%%%


%%%%%%%%%%%%%%%%%%%%%%


\latexProblemContent{
\begin{problem}

What is wrong with the following equation:

\[\int_{\frac{1}{4} \, \pi}^{\frac{7}{4} \, \pi} {9 \, \cot\left(x\right) \csc\left(x\right)}\;dx = {-\frac{9}{\sin\left(x\right)}}\Bigg\vert_{\frac{1}{4} \, \pi}^{\frac{7}{4} \, \pi} = {18 \, \sqrt{2}}\]

\input{2311_Concept_Integral_0003.HELP.tex}

\begin{multipleChoice}
\choice The antiderivative is incorrect.
\choice[correct] The integrand is not defined over the entire interval.
\choice The bounds are evaluated in the wrong order.
\choice Nothing is wrong.  The equation is correct, as is.
\end{multipleChoice}

\end{problem}}%}

%%%%%%%%%%%%%%%%%%%%%%


\latexProblemContent{
\begin{problem}

What is wrong with the following equation:

\[\int_{\frac{1}{4} \, \pi}^{\frac{7}{4} \, \pi} {-15 \, \cot\left(x\right) \csc\left(x\right)}\;dx = {\frac{15}{\sin\left(x\right)}}\Bigg\vert_{\frac{1}{4} \, \pi}^{\frac{7}{4} \, \pi} = {-30 \, \sqrt{2}}\]

\input{2311_Concept_Integral_0003.HELP.tex}

\begin{multipleChoice}
\choice The antiderivative is incorrect.
\choice[correct] The integrand is not defined over the entire interval.
\choice The bounds are evaluated in the wrong order.
\choice Nothing is wrong.  The equation is correct, as is.
\end{multipleChoice}

\end{problem}}%}

%%%%%%%%%%%%%%%%%%%%%%


\latexProblemContent{
\begin{problem}

What is wrong with the following equation:

\[\int_{\frac{2}{3} \, \pi}^{\frac{3}{2} \, \pi} {-4 \, \cot\left(x\right) \csc\left(x\right)}\;dx = {\frac{4}{\sin\left(x\right)}}\Bigg\vert_{\frac{2}{3} \, \pi}^{\frac{3}{2} \, \pi} = {-\frac{8}{3} \, \sqrt{3} - 4}\]

\input{2311_Concept_Integral_0003.HELP.tex}

\begin{multipleChoice}
\choice The antiderivative is incorrect.
\choice[correct] The integrand is not defined over the entire interval.
\choice The bounds are evaluated in the wrong order.
\choice Nothing is wrong.  The equation is correct, as is.
\end{multipleChoice}

\end{problem}}%}

%%%%%%%%%%%%%%%%%%%%%%


\latexProblemContent{
\begin{problem}

What is wrong with the following equation:

\[\int_{\frac{2}{3} \, \pi}^{\frac{5}{3} \, \pi} {14 \, \csc\left(x\right)^{2}}\;dx = {-\frac{14}{\tan\left(x\right)}}\Bigg\vert_{\frac{2}{3} \, \pi}^{\frac{5}{3} \, \pi} = {0}\]

\input{2311_Concept_Integral_0003.HELP.tex}

\begin{multipleChoice}
\choice The antiderivative is incorrect.
\choice[correct] The integrand is not defined over the entire interval.
\choice The bounds are evaluated in the wrong order.
\choice Nothing is wrong.  The equation is correct, as is.
\end{multipleChoice}

\end{problem}}%}

%%%%%%%%%%%%%%%%%%%%%%


%%%%%%%%%%%%%%%%%%%%%%


\latexProblemContent{
\begin{problem}

What is wrong with the following equation:

\[\int_{\frac{1}{2} \, \pi}^{\frac{3}{2} \, \pi} {-7 \, \cot\left(x\right) \csc\left(x\right)}\;dx = {\frac{7}{\sin\left(x\right)}}\Bigg\vert_{\frac{1}{2} \, \pi}^{\frac{3}{2} \, \pi} = {-14}\]

\input{2311_Concept_Integral_0003.HELP.tex}

\begin{multipleChoice}
\choice The antiderivative is incorrect.
\choice[correct] The integrand is not defined over the entire interval.
\choice The bounds are evaluated in the wrong order.
\choice Nothing is wrong.  The equation is correct, as is.
\end{multipleChoice}

\end{problem}}%}

%%%%%%%%%%%%%%%%%%%%%%


\latexProblemContent{
\begin{problem}

What is wrong with the following equation:

\[\int_{\frac{1}{2} \, \pi}^{\frac{3}{2} \, \pi} {-12 \, \cot\left(x\right) \csc\left(x\right)}\;dx = {\frac{12}{\sin\left(x\right)}}\Bigg\vert_{\frac{1}{2} \, \pi}^{\frac{3}{2} \, \pi} = {-24}\]

\input{2311_Concept_Integral_0003.HELP.tex}

\begin{multipleChoice}
\choice The antiderivative is incorrect.
\choice[correct] The integrand is not defined over the entire interval.
\choice The bounds are evaluated in the wrong order.
\choice Nothing is wrong.  The equation is correct, as is.
\end{multipleChoice}

\end{problem}}%}

%%%%%%%%%%%%%%%%%%%%%%


\latexProblemContent{
\begin{problem}

What is wrong with the following equation:

\[\int_{\frac{3}{4} \, \pi}^{\frac{3}{2} \, \pi} {-3 \, \csc\left(x\right)^{2}}\;dx = {\frac{3}{\tan\left(x\right)}}\Bigg\vert_{\frac{3}{4} \, \pi}^{\frac{3}{2} \, \pi} = {3}\]

\input{2311_Concept_Integral_0003.HELP.tex}

\begin{multipleChoice}
\choice The antiderivative is incorrect.
\choice[correct] The integrand is not defined over the entire interval.
\choice The bounds are evaluated in the wrong order.
\choice Nothing is wrong.  The equation is correct, as is.
\end{multipleChoice}

\end{problem}}%}

%%%%%%%%%%%%%%%%%%%%%%


\latexProblemContent{
\begin{problem}

What is wrong with the following equation:

\[\int_{\frac{1}{3} \, \pi}^{\frac{4}{3} \, \pi} {3 \, \csc\left(x\right)^{2}}\;dx = {-\frac{3}{\tan\left(x\right)}}\Bigg\vert_{\frac{1}{3} \, \pi}^{\frac{4}{3} \, \pi} = {0}\]

\input{2311_Concept_Integral_0003.HELP.tex}

\begin{multipleChoice}
\choice The antiderivative is incorrect.
\choice[correct] The integrand is not defined over the entire interval.
\choice The bounds are evaluated in the wrong order.
\choice Nothing is wrong.  The equation is correct, as is.
\end{multipleChoice}

\end{problem}}%}

%%%%%%%%%%%%%%%%%%%%%%


\latexProblemContent{
\begin{problem}

What is wrong with the following equation:

\[\int_{\frac{2}{3} \, \pi}^{\frac{7}{4} \, \pi} {-4 \, \cot\left(x\right) \csc\left(x\right)}\;dx = {\frac{4}{\sin\left(x\right)}}\Bigg\vert_{\frac{2}{3} \, \pi}^{\frac{7}{4} \, \pi} = {-\frac{8}{3} \, \sqrt{3} - 4 \, \sqrt{2}}\]

\input{2311_Concept_Integral_0003.HELP.tex}

\begin{multipleChoice}
\choice The antiderivative is incorrect.
\choice[correct] The integrand is not defined over the entire interval.
\choice The bounds are evaluated in the wrong order.
\choice Nothing is wrong.  The equation is correct, as is.
\end{multipleChoice}

\end{problem}}%}

%%%%%%%%%%%%%%%%%%%%%%


\latexProblemContent{
\begin{problem}

What is wrong with the following equation:

\[\int_{\frac{1}{2} \, \pi}^{\frac{5}{3} \, \pi} {-2 \, \cot\left(x\right) \csc\left(x\right)}\;dx = {\frac{2}{\sin\left(x\right)}}\Bigg\vert_{\frac{1}{2} \, \pi}^{\frac{5}{3} \, \pi} = {-\frac{4}{3} \, \sqrt{3} - 2}\]

\input{2311_Concept_Integral_0003.HELP.tex}

\begin{multipleChoice}
\choice The antiderivative is incorrect.
\choice[correct] The integrand is not defined over the entire interval.
\choice The bounds are evaluated in the wrong order.
\choice Nothing is wrong.  The equation is correct, as is.
\end{multipleChoice}

\end{problem}}%}

%%%%%%%%%%%%%%%%%%%%%%


\latexProblemContent{
\begin{problem}

What is wrong with the following equation:

\[\int_{\frac{2}{3} \, \pi}^{\frac{3}{2} \, \pi} {8 \, \cot\left(x\right) \csc\left(x\right)}\;dx = {-\frac{8}{\sin\left(x\right)}}\Bigg\vert_{\frac{2}{3} \, \pi}^{\frac{3}{2} \, \pi} = {\frac{16}{3} \, \sqrt{3} + 8}\]

\input{2311_Concept_Integral_0003.HELP.tex}

\begin{multipleChoice}
\choice The antiderivative is incorrect.
\choice[correct] The integrand is not defined over the entire interval.
\choice The bounds are evaluated in the wrong order.
\choice Nothing is wrong.  The equation is correct, as is.
\end{multipleChoice}

\end{problem}}%}

%%%%%%%%%%%%%%%%%%%%%%


%%%%%%%%%%%%%%%%%%%%%%


\latexProblemContent{
\begin{problem}

What is wrong with the following equation:

\[\int_{\frac{1}{4} \, \pi}^{\frac{3}{2} \, \pi} {6 \, \csc\left(x\right)^{2}}\;dx = {-\frac{6}{\tan\left(x\right)}}\Bigg\vert_{\frac{1}{4} \, \pi}^{\frac{3}{2} \, \pi} = {6}\]

\input{2311_Concept_Integral_0003.HELP.tex}

\begin{multipleChoice}
\choice The antiderivative is incorrect.
\choice[correct] The integrand is not defined over the entire interval.
\choice The bounds are evaluated in the wrong order.
\choice Nothing is wrong.  The equation is correct, as is.
\end{multipleChoice}

\end{problem}}%}

%%%%%%%%%%%%%%%%%%%%%%


\latexProblemContent{
\begin{problem}

What is wrong with the following equation:

\[\int_{\frac{2}{3} \, \pi}^{\frac{5}{3} \, \pi} {8 \, \cot\left(x\right) \csc\left(x\right)}\;dx = {-\frac{8}{\sin\left(x\right)}}\Bigg\vert_{\frac{2}{3} \, \pi}^{\frac{5}{3} \, \pi} = {\frac{32}{3} \, \sqrt{3}}\]

\input{2311_Concept_Integral_0003.HELP.tex}

\begin{multipleChoice}
\choice The antiderivative is incorrect.
\choice[correct] The integrand is not defined over the entire interval.
\choice The bounds are evaluated in the wrong order.
\choice Nothing is wrong.  The equation is correct, as is.
\end{multipleChoice}

\end{problem}}%}

%%%%%%%%%%%%%%%%%%%%%%


%%%%%%%%%%%%%%%%%%%%%%


%%%%%%%%%%%%%%%%%%%%%%


\latexProblemContent{
\begin{problem}

What is wrong with the following equation:

\[\int_{\frac{1}{2} \, \pi}^{\frac{4}{3} \, \pi} {14 \, \cot\left(x\right) \csc\left(x\right)}\;dx = {-\frac{14}{\sin\left(x\right)}}\Bigg\vert_{\frac{1}{2} \, \pi}^{\frac{4}{3} \, \pi} = {\frac{28}{3} \, \sqrt{3} + 14}\]

\input{2311_Concept_Integral_0003.HELP.tex}

\begin{multipleChoice}
\choice The antiderivative is incorrect.
\choice[correct] The integrand is not defined over the entire interval.
\choice The bounds are evaluated in the wrong order.
\choice Nothing is wrong.  The equation is correct, as is.
\end{multipleChoice}

\end{problem}}%}

%%%%%%%%%%%%%%%%%%%%%%


%%%%%%%%%%%%%%%%%%%%%%


\latexProblemContent{
\begin{problem}

What is wrong with the following equation:

\[\int_{\frac{3}{4} \, \pi}^{\frac{4}{3} \, \pi} {-3 \, \cot\left(x\right) \csc\left(x\right)}\;dx = {\frac{3}{\sin\left(x\right)}}\Bigg\vert_{\frac{3}{4} \, \pi}^{\frac{4}{3} \, \pi} = {-2 \, \sqrt{3} - 3 \, \sqrt{2}}\]

\input{2311_Concept_Integral_0003.HELP.tex}

\begin{multipleChoice}
\choice The antiderivative is incorrect.
\choice[correct] The integrand is not defined over the entire interval.
\choice The bounds are evaluated in the wrong order.
\choice Nothing is wrong.  The equation is correct, as is.
\end{multipleChoice}

\end{problem}}%}

%%%%%%%%%%%%%%%%%%%%%%


\latexProblemContent{
\begin{problem}

What is wrong with the following equation:

\[\int_{\frac{3}{4} \, \pi}^{\frac{5}{4} \, \pi} {-4 \, \cot\left(x\right) \csc\left(x\right)}\;dx = {\frac{4}{\sin\left(x\right)}}\Bigg\vert_{\frac{3}{4} \, \pi}^{\frac{5}{4} \, \pi} = {-8 \, \sqrt{2}}\]

\input{2311_Concept_Integral_0003.HELP.tex}

\begin{multipleChoice}
\choice The antiderivative is incorrect.
\choice[correct] The integrand is not defined over the entire interval.
\choice The bounds are evaluated in the wrong order.
\choice Nothing is wrong.  The equation is correct, as is.
\end{multipleChoice}

\end{problem}}%}

%%%%%%%%%%%%%%%%%%%%%%


\latexProblemContent{
\begin{problem}

What is wrong with the following equation:

\[\int_{\frac{2}{3} \, \pi}^{\frac{7}{4} \, \pi} {2 \, \csc\left(x\right)^{2}}\;dx = {-\frac{2}{\tan\left(x\right)}}\Bigg\vert_{\frac{2}{3} \, \pi}^{\frac{7}{4} \, \pi} = {-\frac{2}{3} \, \sqrt{3} + 2}\]

\input{2311_Concept_Integral_0003.HELP.tex}

\begin{multipleChoice}
\choice The antiderivative is incorrect.
\choice[correct] The integrand is not defined over the entire interval.
\choice The bounds are evaluated in the wrong order.
\choice Nothing is wrong.  The equation is correct, as is.
\end{multipleChoice}

\end{problem}}%}

%%%%%%%%%%%%%%%%%%%%%%


\latexProblemContent{
\begin{problem}

What is wrong with the following equation:

\[\int_{\frac{1}{3} \, \pi}^{\frac{4}{3} \, \pi} {-2 \, \cot\left(x\right) \csc\left(x\right)}\;dx = {\frac{2}{\sin\left(x\right)}}\Bigg\vert_{\frac{1}{3} \, \pi}^{\frac{4}{3} \, \pi} = {-\frac{8}{3} \, \sqrt{3}}\]

\input{2311_Concept_Integral_0003.HELP.tex}

\begin{multipleChoice}
\choice The antiderivative is incorrect.
\choice[correct] The integrand is not defined over the entire interval.
\choice The bounds are evaluated in the wrong order.
\choice Nothing is wrong.  The equation is correct, as is.
\end{multipleChoice}

\end{problem}}%}

%%%%%%%%%%%%%%%%%%%%%%


\latexProblemContent{
\begin{problem}

What is wrong with the following equation:

\[\int_{\frac{1}{2} \, \pi}^{\frac{5}{4} \, \pi} {-8 \, \csc\left(x\right)^{2}}\;dx = {\frac{8}{\tan\left(x\right)}}\Bigg\vert_{\frac{1}{2} \, \pi}^{\frac{5}{4} \, \pi} = {8}\]

\input{2311_Concept_Integral_0003.HELP.tex}

\begin{multipleChoice}
\choice The antiderivative is incorrect.
\choice[correct] The integrand is not defined over the entire interval.
\choice The bounds are evaluated in the wrong order.
\choice Nothing is wrong.  The equation is correct, as is.
\end{multipleChoice}

\end{problem}}%}

%%%%%%%%%%%%%%%%%%%%%%


\latexProblemContent{
\begin{problem}

What is wrong with the following equation:

\[\int_{\frac{1}{4} \, \pi}^{\frac{4}{3} \, \pi} {11 \, \csc\left(x\right)^{2}}\;dx = {-\frac{11}{\tan\left(x\right)}}\Bigg\vert_{\frac{1}{4} \, \pi}^{\frac{4}{3} \, \pi} = {-\frac{11}{3} \, \sqrt{3} + 11}\]

\input{2311_Concept_Integral_0003.HELP.tex}

\begin{multipleChoice}
\choice The antiderivative is incorrect.
\choice[correct] The integrand is not defined over the entire interval.
\choice The bounds are evaluated in the wrong order.
\choice Nothing is wrong.  The equation is correct, as is.
\end{multipleChoice}

\end{problem}}%}

%%%%%%%%%%%%%%%%%%%%%%


\latexProblemContent{
\begin{problem}

What is wrong with the following equation:

\[\int_{\frac{1}{3} \, \pi}^{\frac{7}{4} \, \pi} {-7 \, \cot\left(x\right) \csc\left(x\right)}\;dx = {\frac{7}{\sin\left(x\right)}}\Bigg\vert_{\frac{1}{3} \, \pi}^{\frac{7}{4} \, \pi} = {-\frac{14}{3} \, \sqrt{3} - 7 \, \sqrt{2}}\]

\input{2311_Concept_Integral_0003.HELP.tex}

\begin{multipleChoice}
\choice The antiderivative is incorrect.
\choice[correct] The integrand is not defined over the entire interval.
\choice The bounds are evaluated in the wrong order.
\choice Nothing is wrong.  The equation is correct, as is.
\end{multipleChoice}

\end{problem}}%}

%%%%%%%%%%%%%%%%%%%%%%


%%%%%%%%%%%%%%%%%%%%%%


\latexProblemContent{
\begin{problem}

What is wrong with the following equation:

\[\int_{\frac{1}{2} \, \pi}^{\frac{3}{2} \, \pi} {-3 \, \csc\left(x\right)^{2}}\;dx = {\frac{3}{\tan\left(x\right)}}\Bigg\vert_{\frac{1}{2} \, \pi}^{\frac{3}{2} \, \pi} = {0}\]

\input{2311_Concept_Integral_0003.HELP.tex}

\begin{multipleChoice}
\choice The antiderivative is incorrect.
\choice[correct] The integrand is not defined over the entire interval.
\choice The bounds are evaluated in the wrong order.
\choice Nothing is wrong.  The equation is correct, as is.
\end{multipleChoice}

\end{problem}}%}

%%%%%%%%%%%%%%%%%%%%%%


\latexProblemContent{
\begin{problem}

What is wrong with the following equation:

\[\int_{\frac{3}{4} \, \pi}^{\frac{5}{4} \, \pi} {2 \, \cot\left(x\right) \csc\left(x\right)}\;dx = {-\frac{2}{\sin\left(x\right)}}\Bigg\vert_{\frac{3}{4} \, \pi}^{\frac{5}{4} \, \pi} = {4 \, \sqrt{2}}\]

\input{2311_Concept_Integral_0003.HELP.tex}

\begin{multipleChoice}
\choice The antiderivative is incorrect.
\choice[correct] The integrand is not defined over the entire interval.
\choice The bounds are evaluated in the wrong order.
\choice Nothing is wrong.  The equation is correct, as is.
\end{multipleChoice}

\end{problem}}%}

%%%%%%%%%%%%%%%%%%%%%%


%%%%%%%%%%%%%%%%%%%%%%


\latexProblemContent{
\begin{problem}

What is wrong with the following equation:

\[\int_{\frac{1}{2} \, \pi}^{\frac{7}{4} \, \pi} {-8 \, \csc\left(x\right)^{2}}\;dx = {\frac{8}{\tan\left(x\right)}}\Bigg\vert_{\frac{1}{2} \, \pi}^{\frac{7}{4} \, \pi} = {-8}\]

\input{2311_Concept_Integral_0003.HELP.tex}

\begin{multipleChoice}
\choice The antiderivative is incorrect.
\choice[correct] The integrand is not defined over the entire interval.
\choice The bounds are evaluated in the wrong order.
\choice Nothing is wrong.  The equation is correct, as is.
\end{multipleChoice}

\end{problem}}%}

%%%%%%%%%%%%%%%%%%%%%%


\latexProblemContent{
\begin{problem}

What is wrong with the following equation:

\[\int_{\frac{1}{4} \, \pi}^{\frac{4}{3} \, \pi} {12 \, \cot\left(x\right) \csc\left(x\right)}\;dx = {-\frac{12}{\sin\left(x\right)}}\Bigg\vert_{\frac{1}{4} \, \pi}^{\frac{4}{3} \, \pi} = {8 \, \sqrt{3} + 12 \, \sqrt{2}}\]

\input{2311_Concept_Integral_0003.HELP.tex}

\begin{multipleChoice}
\choice The antiderivative is incorrect.
\choice[correct] The integrand is not defined over the entire interval.
\choice The bounds are evaluated in the wrong order.
\choice Nothing is wrong.  The equation is correct, as is.
\end{multipleChoice}

\end{problem}}%}

%%%%%%%%%%%%%%%%%%%%%%


\latexProblemContent{
\begin{problem}

What is wrong with the following equation:

\[\int_{\frac{3}{4} \, \pi}^{\frac{5}{3} \, \pi} {14 \, \csc\left(x\right)^{2}}\;dx = {-\frac{14}{\tan\left(x\right)}}\Bigg\vert_{\frac{3}{4} \, \pi}^{\frac{5}{3} \, \pi} = {\frac{14}{3} \, \sqrt{3} - 14}\]

\input{2311_Concept_Integral_0003.HELP.tex}

\begin{multipleChoice}
\choice The antiderivative is incorrect.
\choice[correct] The integrand is not defined over the entire interval.
\choice The bounds are evaluated in the wrong order.
\choice Nothing is wrong.  The equation is correct, as is.
\end{multipleChoice}

\end{problem}}%}

%%%%%%%%%%%%%%%%%%%%%%


\latexProblemContent{
\begin{problem}

What is wrong with the following equation:

\[\int_{\frac{2}{3} \, \pi}^{\frac{4}{3} \, \pi} {-5 \, \cot\left(x\right) \csc\left(x\right)}\;dx = {\frac{5}{\sin\left(x\right)}}\Bigg\vert_{\frac{2}{3} \, \pi}^{\frac{4}{3} \, \pi} = {-\frac{20}{3} \, \sqrt{3}}\]

\input{2311_Concept_Integral_0003.HELP.tex}

\begin{multipleChoice}
\choice The antiderivative is incorrect.
\choice[correct] The integrand is not defined over the entire interval.
\choice The bounds are evaluated in the wrong order.
\choice Nothing is wrong.  The equation is correct, as is.
\end{multipleChoice}

\end{problem}}%}

%%%%%%%%%%%%%%%%%%%%%%


\latexProblemContent{
\begin{problem}

What is wrong with the following equation:

\[\int_{\frac{1}{4} \, \pi}^{\frac{5}{3} \, \pi} {5 \, \csc\left(x\right)^{2}}\;dx = {-\frac{5}{\tan\left(x\right)}}\Bigg\vert_{\frac{1}{4} \, \pi}^{\frac{5}{3} \, \pi} = {\frac{5}{3} \, \sqrt{3} + 5}\]

\input{2311_Concept_Integral_0003.HELP.tex}

\begin{multipleChoice}
\choice The antiderivative is incorrect.
\choice[correct] The integrand is not defined over the entire interval.
\choice The bounds are evaluated in the wrong order.
\choice Nothing is wrong.  The equation is correct, as is.
\end{multipleChoice}

\end{problem}}%}

%%%%%%%%%%%%%%%%%%%%%%


\latexProblemContent{
\begin{problem}

What is wrong with the following equation:

\[\int_{\frac{2}{3} \, \pi}^{\frac{4}{3} \, \pi} {-8 \, \cot\left(x\right) \csc\left(x\right)}\;dx = {\frac{8}{\sin\left(x\right)}}\Bigg\vert_{\frac{2}{3} \, \pi}^{\frac{4}{3} \, \pi} = {-\frac{32}{3} \, \sqrt{3}}\]

\input{2311_Concept_Integral_0003.HELP.tex}

\begin{multipleChoice}
\choice The antiderivative is incorrect.
\choice[correct] The integrand is not defined over the entire interval.
\choice The bounds are evaluated in the wrong order.
\choice Nothing is wrong.  The equation is correct, as is.
\end{multipleChoice}

\end{problem}}%}

%%%%%%%%%%%%%%%%%%%%%%


%%%%%%%%%%%%%%%%%%%%%%


\latexProblemContent{
\begin{problem}

What is wrong with the following equation:

\[\int_{\frac{2}{3} \, \pi}^{\frac{4}{3} \, \pi} {\csc\left(x\right)^{2}}\;dx = {-\frac{1}{\tan\left(x\right)}}\Bigg\vert_{\frac{2}{3} \, \pi}^{\frac{4}{3} \, \pi} = {-\frac{2}{3} \, \sqrt{3}}\]

\input{2311_Concept_Integral_0003.HELP.tex}

\begin{multipleChoice}
\choice The antiderivative is incorrect.
\choice[correct] The integrand is not defined over the entire interval.
\choice The bounds are evaluated in the wrong order.
\choice Nothing is wrong.  The equation is correct, as is.
\end{multipleChoice}

\end{problem}}%}

%%%%%%%%%%%%%%%%%%%%%%


\latexProblemContent{
\begin{problem}

What is wrong with the following equation:

\[\int_{\frac{3}{4} \, \pi}^{\frac{5}{3} \, \pi} {-11 \, \cot\left(x\right) \csc\left(x\right)}\;dx = {\frac{11}{\sin\left(x\right)}}\Bigg\vert_{\frac{3}{4} \, \pi}^{\frac{5}{3} \, \pi} = {-\frac{22}{3} \, \sqrt{3} - 11 \, \sqrt{2}}\]

\input{2311_Concept_Integral_0003.HELP.tex}

\begin{multipleChoice}
\choice The antiderivative is incorrect.
\choice[correct] The integrand is not defined over the entire interval.
\choice The bounds are evaluated in the wrong order.
\choice Nothing is wrong.  The equation is correct, as is.
\end{multipleChoice}

\end{problem}}%}

%%%%%%%%%%%%%%%%%%%%%%


\latexProblemContent{
\begin{problem}

What is wrong with the following equation:

\[\int_{\frac{2}{3} \, \pi}^{\frac{4}{3} \, \pi} {5 \, \cot\left(x\right) \csc\left(x\right)}\;dx = {-\frac{5}{\sin\left(x\right)}}\Bigg\vert_{\frac{2}{3} \, \pi}^{\frac{4}{3} \, \pi} = {\frac{20}{3} \, \sqrt{3}}\]

\input{2311_Concept_Integral_0003.HELP.tex}

\begin{multipleChoice}
\choice The antiderivative is incorrect.
\choice[correct] The integrand is not defined over the entire interval.
\choice The bounds are evaluated in the wrong order.
\choice Nothing is wrong.  The equation is correct, as is.
\end{multipleChoice}

\end{problem}}%}

%%%%%%%%%%%%%%%%%%%%%%


\latexProblemContent{
\begin{problem}

What is wrong with the following equation:

\[\int_{\frac{3}{4} \, \pi}^{\frac{4}{3} \, \pi} {12 \, \csc\left(x\right)^{2}}\;dx = {-\frac{12}{\tan\left(x\right)}}\Bigg\vert_{\frac{3}{4} \, \pi}^{\frac{4}{3} \, \pi} = {-4 \, \sqrt{3} - 12}\]

\input{2311_Concept_Integral_0003.HELP.tex}

\begin{multipleChoice}
\choice The antiderivative is incorrect.
\choice[correct] The integrand is not defined over the entire interval.
\choice The bounds are evaluated in the wrong order.
\choice Nothing is wrong.  The equation is correct, as is.
\end{multipleChoice}

\end{problem}}%}

%%%%%%%%%%%%%%%%%%%%%%


\latexProblemContent{
\begin{problem}

What is wrong with the following equation:

\[\int_{\frac{1}{4} \, \pi}^{\frac{3}{2} \, \pi} {-14 \, \cot\left(x\right) \csc\left(x\right)}\;dx = {\frac{14}{\sin\left(x\right)}}\Bigg\vert_{\frac{1}{4} \, \pi}^{\frac{3}{2} \, \pi} = {-14 \, \sqrt{2} - 14}\]

\input{2311_Concept_Integral_0003.HELP.tex}

\begin{multipleChoice}
\choice The antiderivative is incorrect.
\choice[correct] The integrand is not defined over the entire interval.
\choice The bounds are evaluated in the wrong order.
\choice Nothing is wrong.  The equation is correct, as is.
\end{multipleChoice}

\end{problem}}%}

%%%%%%%%%%%%%%%%%%%%%%


\latexProblemContent{
\begin{problem}

What is wrong with the following equation:

\[\int_{\frac{3}{4} \, \pi}^{\frac{7}{4} \, \pi} {-11 \, \cot\left(x\right) \csc\left(x\right)}\;dx = {\frac{11}{\sin\left(x\right)}}\Bigg\vert_{\frac{3}{4} \, \pi}^{\frac{7}{4} \, \pi} = {-22 \, \sqrt{2}}\]

\input{2311_Concept_Integral_0003.HELP.tex}

\begin{multipleChoice}
\choice The antiderivative is incorrect.
\choice[correct] The integrand is not defined over the entire interval.
\choice The bounds are evaluated in the wrong order.
\choice Nothing is wrong.  The equation is correct, as is.
\end{multipleChoice}

\end{problem}}%}

%%%%%%%%%%%%%%%%%%%%%%


\latexProblemContent{
\begin{problem}

What is wrong with the following equation:

\[\int_{\frac{1}{3} \, \pi}^{\frac{5}{3} \, \pi} {-13 \, \cot\left(x\right) \csc\left(x\right)}\;dx = {\frac{13}{\sin\left(x\right)}}\Bigg\vert_{\frac{1}{3} \, \pi}^{\frac{5}{3} \, \pi} = {-\frac{52}{3} \, \sqrt{3}}\]

\input{2311_Concept_Integral_0003.HELP.tex}

\begin{multipleChoice}
\choice The antiderivative is incorrect.
\choice[correct] The integrand is not defined over the entire interval.
\choice The bounds are evaluated in the wrong order.
\choice Nothing is wrong.  The equation is correct, as is.
\end{multipleChoice}

\end{problem}}%}

%%%%%%%%%%%%%%%%%%%%%%


\latexProblemContent{
\begin{problem}

What is wrong with the following equation:

\[\int_{\frac{1}{4} \, \pi}^{\frac{5}{4} \, \pi} {-4 \, \csc\left(x\right)^{2}}\;dx = {\frac{4}{\tan\left(x\right)}}\Bigg\vert_{\frac{1}{4} \, \pi}^{\frac{5}{4} \, \pi} = {0}\]

\input{2311_Concept_Integral_0003.HELP.tex}

\begin{multipleChoice}
\choice The antiderivative is incorrect.
\choice[correct] The integrand is not defined over the entire interval.
\choice The bounds are evaluated in the wrong order.
\choice Nothing is wrong.  The equation is correct, as is.
\end{multipleChoice}

\end{problem}}%}

%%%%%%%%%%%%%%%%%%%%%%


\latexProblemContent{
\begin{problem}

What is wrong with the following equation:

\[\int_{\frac{1}{3} \, \pi}^{\frac{5}{3} \, \pi} {-13 \, \csc\left(x\right)^{2}}\;dx = {\frac{13}{\tan\left(x\right)}}\Bigg\vert_{\frac{1}{3} \, \pi}^{\frac{5}{3} \, \pi} = {-\frac{26}{3} \, \sqrt{3}}\]

\input{2311_Concept_Integral_0003.HELP.tex}

\begin{multipleChoice}
\choice The antiderivative is incorrect.
\choice[correct] The integrand is not defined over the entire interval.
\choice The bounds are evaluated in the wrong order.
\choice Nothing is wrong.  The equation is correct, as is.
\end{multipleChoice}

\end{problem}}%}

%%%%%%%%%%%%%%%%%%%%%%


\latexProblemContent{
\begin{problem}

What is wrong with the following equation:

\[\int_{\frac{3}{4} \, \pi}^{\frac{4}{3} \, \pi} {-6 \, \cot\left(x\right) \csc\left(x\right)}\;dx = {\frac{6}{\sin\left(x\right)}}\Bigg\vert_{\frac{3}{4} \, \pi}^{\frac{4}{3} \, \pi} = {-4 \, \sqrt{3} - 6 \, \sqrt{2}}\]

\input{2311_Concept_Integral_0003.HELP.tex}

\begin{multipleChoice}
\choice The antiderivative is incorrect.
\choice[correct] The integrand is not defined over the entire interval.
\choice The bounds are evaluated in the wrong order.
\choice Nothing is wrong.  The equation is correct, as is.
\end{multipleChoice}

\end{problem}}%}

%%%%%%%%%%%%%%%%%%%%%%


\latexProblemContent{
\begin{problem}

What is wrong with the following equation:

\[\int_{\frac{1}{4} \, \pi}^{\frac{4}{3} \, \pi} {4 \, \cot\left(x\right) \csc\left(x\right)}\;dx = {-\frac{4}{\sin\left(x\right)}}\Bigg\vert_{\frac{1}{4} \, \pi}^{\frac{4}{3} \, \pi} = {\frac{8}{3} \, \sqrt{3} + 4 \, \sqrt{2}}\]

\input{2311_Concept_Integral_0003.HELP.tex}

\begin{multipleChoice}
\choice The antiderivative is incorrect.
\choice[correct] The integrand is not defined over the entire interval.
\choice The bounds are evaluated in the wrong order.
\choice Nothing is wrong.  The equation is correct, as is.
\end{multipleChoice}

\end{problem}}%}

%%%%%%%%%%%%%%%%%%%%%%


%%%%%%%%%%%%%%%%%%%%%%


\latexProblemContent{
\begin{problem}

What is wrong with the following equation:

\[\int_{\frac{3}{4} \, \pi}^{\frac{7}{4} \, \pi} {-7 \, \csc\left(x\right)^{2}}\;dx = {\frac{7}{\tan\left(x\right)}}\Bigg\vert_{\frac{3}{4} \, \pi}^{\frac{7}{4} \, \pi} = {0}\]

\input{2311_Concept_Integral_0003.HELP.tex}

\begin{multipleChoice}
\choice The antiderivative is incorrect.
\choice[correct] The integrand is not defined over the entire interval.
\choice The bounds are evaluated in the wrong order.
\choice Nothing is wrong.  The equation is correct, as is.
\end{multipleChoice}

\end{problem}}%}

%%%%%%%%%%%%%%%%%%%%%%


\latexProblemContent{
\begin{problem}

What is wrong with the following equation:

\[\int_{\frac{1}{3} \, \pi}^{\frac{4}{3} \, \pi} {12 \, \csc\left(x\right)^{2}}\;dx = {-\frac{12}{\tan\left(x\right)}}\Bigg\vert_{\frac{1}{3} \, \pi}^{\frac{4}{3} \, \pi} = {0}\]

\input{2311_Concept_Integral_0003.HELP.tex}

\begin{multipleChoice}
\choice The antiderivative is incorrect.
\choice[correct] The integrand is not defined over the entire interval.
\choice The bounds are evaluated in the wrong order.
\choice Nothing is wrong.  The equation is correct, as is.
\end{multipleChoice}

\end{problem}}%}

%%%%%%%%%%%%%%%%%%%%%%


\latexProblemContent{
\begin{problem}

What is wrong with the following equation:

\[\int_{\frac{3}{4} \, \pi}^{\frac{7}{4} \, \pi} {-14 \, \csc\left(x\right)^{2}}\;dx = {\frac{14}{\tan\left(x\right)}}\Bigg\vert_{\frac{3}{4} \, \pi}^{\frac{7}{4} \, \pi} = {0}\]

\input{2311_Concept_Integral_0003.HELP.tex}

\begin{multipleChoice}
\choice The antiderivative is incorrect.
\choice[correct] The integrand is not defined over the entire interval.
\choice The bounds are evaluated in the wrong order.
\choice Nothing is wrong.  The equation is correct, as is.
\end{multipleChoice}

\end{problem}}%}

%%%%%%%%%%%%%%%%%%%%%%


\latexProblemContent{
\begin{problem}

What is wrong with the following equation:

\[\int_{\frac{1}{3} \, \pi}^{\frac{4}{3} \, \pi} {-12 \, \cot\left(x\right) \csc\left(x\right)}\;dx = {\frac{12}{\sin\left(x\right)}}\Bigg\vert_{\frac{1}{3} \, \pi}^{\frac{4}{3} \, \pi} = {-16 \, \sqrt{3}}\]

\input{2311_Concept_Integral_0003.HELP.tex}

\begin{multipleChoice}
\choice The antiderivative is incorrect.
\choice[correct] The integrand is not defined over the entire interval.
\choice The bounds are evaluated in the wrong order.
\choice Nothing is wrong.  The equation is correct, as is.
\end{multipleChoice}

\end{problem}}%}

%%%%%%%%%%%%%%%%%%%%%%


\latexProblemContent{
\begin{problem}

What is wrong with the following equation:

\[\int_{\frac{1}{4} \, \pi}^{\frac{5}{4} \, \pi} {14 \, \csc\left(x\right)^{2}}\;dx = {-\frac{14}{\tan\left(x\right)}}\Bigg\vert_{\frac{1}{4} \, \pi}^{\frac{5}{4} \, \pi} = {0}\]

\input{2311_Concept_Integral_0003.HELP.tex}

\begin{multipleChoice}
\choice The antiderivative is incorrect.
\choice[correct] The integrand is not defined over the entire interval.
\choice The bounds are evaluated in the wrong order.
\choice Nothing is wrong.  The equation is correct, as is.
\end{multipleChoice}

\end{problem}}%}

%%%%%%%%%%%%%%%%%%%%%%


\latexProblemContent{
\begin{problem}

What is wrong with the following equation:

\[\int_{\frac{3}{4} \, \pi}^{\frac{3}{2} \, \pi} {3 \, \cot\left(x\right) \csc\left(x\right)}\;dx = {-\frac{3}{\sin\left(x\right)}}\Bigg\vert_{\frac{3}{4} \, \pi}^{\frac{3}{2} \, \pi} = {3 \, \sqrt{2} + 3}\]

\input{2311_Concept_Integral_0003.HELP.tex}

\begin{multipleChoice}
\choice The antiderivative is incorrect.
\choice[correct] The integrand is not defined over the entire interval.
\choice The bounds are evaluated in the wrong order.
\choice Nothing is wrong.  The equation is correct, as is.
\end{multipleChoice}

\end{problem}}%}

%%%%%%%%%%%%%%%%%%%%%%


\latexProblemContent{
\begin{problem}

What is wrong with the following equation:

\[\int_{\frac{2}{3} \, \pi}^{\frac{7}{4} \, \pi} {10 \, \cot\left(x\right) \csc\left(x\right)}\;dx = {-\frac{10}{\sin\left(x\right)}}\Bigg\vert_{\frac{2}{3} \, \pi}^{\frac{7}{4} \, \pi} = {\frac{20}{3} \, \sqrt{3} + 10 \, \sqrt{2}}\]

\input{2311_Concept_Integral_0003.HELP.tex}

\begin{multipleChoice}
\choice The antiderivative is incorrect.
\choice[correct] The integrand is not defined over the entire interval.
\choice The bounds are evaluated in the wrong order.
\choice Nothing is wrong.  The equation is correct, as is.
\end{multipleChoice}

\end{problem}}%}

%%%%%%%%%%%%%%%%%%%%%%


\latexProblemContent{
\begin{problem}

What is wrong with the following equation:

\[\int_{\frac{1}{4} \, \pi}^{\frac{5}{3} \, \pi} {5 \, \cot\left(x\right) \csc\left(x\right)}\;dx = {-\frac{5}{\sin\left(x\right)}}\Bigg\vert_{\frac{1}{4} \, \pi}^{\frac{5}{3} \, \pi} = {\frac{10}{3} \, \sqrt{3} + 5 \, \sqrt{2}}\]

\input{2311_Concept_Integral_0003.HELP.tex}

\begin{multipleChoice}
\choice The antiderivative is incorrect.
\choice[correct] The integrand is not defined over the entire interval.
\choice The bounds are evaluated in the wrong order.
\choice Nothing is wrong.  The equation is correct, as is.
\end{multipleChoice}

\end{problem}}%}

%%%%%%%%%%%%%%%%%%%%%%


\latexProblemContent{
\begin{problem}

What is wrong with the following equation:

\[\int_{\frac{1}{2} \, \pi}^{\frac{4}{3} \, \pi} {-7 \, \cot\left(x\right) \csc\left(x\right)}\;dx = {\frac{7}{\sin\left(x\right)}}\Bigg\vert_{\frac{1}{2} \, \pi}^{\frac{4}{3} \, \pi} = {-\frac{14}{3} \, \sqrt{3} - 7}\]

\input{2311_Concept_Integral_0003.HELP.tex}

\begin{multipleChoice}
\choice The antiderivative is incorrect.
\choice[correct] The integrand is not defined over the entire interval.
\choice The bounds are evaluated in the wrong order.
\choice Nothing is wrong.  The equation is correct, as is.
\end{multipleChoice}

\end{problem}}%}

%%%%%%%%%%%%%%%%%%%%%%


\latexProblemContent{
\begin{problem}

What is wrong with the following equation:

\[\int_{\frac{1}{3} \, \pi}^{\frac{4}{3} \, \pi} {-7 \, \csc\left(x\right)^{2}}\;dx = {\frac{7}{\tan\left(x\right)}}\Bigg\vert_{\frac{1}{3} \, \pi}^{\frac{4}{3} \, \pi} = {0}\]

\input{2311_Concept_Integral_0003.HELP.tex}

\begin{multipleChoice}
\choice The antiderivative is incorrect.
\choice[correct] The integrand is not defined over the entire interval.
\choice The bounds are evaluated in the wrong order.
\choice Nothing is wrong.  The equation is correct, as is.
\end{multipleChoice}

\end{problem}}%}

%%%%%%%%%%%%%%%%%%%%%%


\latexProblemContent{
\begin{problem}

What is wrong with the following equation:

\[\int_{\frac{1}{3} \, \pi}^{\frac{7}{4} \, \pi} {12 \, \csc\left(x\right)^{2}}\;dx = {-\frac{12}{\tan\left(x\right)}}\Bigg\vert_{\frac{1}{3} \, \pi}^{\frac{7}{4} \, \pi} = {4 \, \sqrt{3} + 12}\]

\input{2311_Concept_Integral_0003.HELP.tex}

\begin{multipleChoice}
\choice The antiderivative is incorrect.
\choice[correct] The integrand is not defined over the entire interval.
\choice The bounds are evaluated in the wrong order.
\choice Nothing is wrong.  The equation is correct, as is.
\end{multipleChoice}

\end{problem}}%}

%%%%%%%%%%%%%%%%%%%%%%


%%%%%%%%%%%%%%%%%%%%%%


\latexProblemContent{
\begin{problem}

What is wrong with the following equation:

\[\int_{\frac{2}{3} \, \pi}^{\frac{5}{3} \, \pi} {-2 \, \csc\left(x\right)^{2}}\;dx = {\frac{2}{\tan\left(x\right)}}\Bigg\vert_{\frac{2}{3} \, \pi}^{\frac{5}{3} \, \pi} = {0}\]

\input{2311_Concept_Integral_0003.HELP.tex}

\begin{multipleChoice}
\choice The antiderivative is incorrect.
\choice[correct] The integrand is not defined over the entire interval.
\choice The bounds are evaluated in the wrong order.
\choice Nothing is wrong.  The equation is correct, as is.
\end{multipleChoice}

\end{problem}}%}

%%%%%%%%%%%%%%%%%%%%%%


\latexProblemContent{
\begin{problem}

What is wrong with the following equation:

\[\int_{\frac{1}{4} \, \pi}^{\frac{3}{2} \, \pi} {\cot\left(x\right) \csc\left(x\right)}\;dx = {-\frac{1}{\sin\left(x\right)}}\Bigg\vert_{\frac{1}{4} \, \pi}^{\frac{3}{2} \, \pi} = {\sqrt{2} + 1}\]

\input{2311_Concept_Integral_0003.HELP.tex}

\begin{multipleChoice}
\choice The antiderivative is incorrect.
\choice[correct] The integrand is not defined over the entire interval.
\choice The bounds are evaluated in the wrong order.
\choice Nothing is wrong.  The equation is correct, as is.
\end{multipleChoice}

\end{problem}}%}

%%%%%%%%%%%%%%%%%%%%%%


%%%%%%%%%%%%%%%%%%%%%%


\latexProblemContent{
\begin{problem}

What is wrong with the following equation:

\[\int_{\frac{1}{2} \, \pi}^{\frac{3}{2} \, \pi} {2 \, \cot\left(x\right) \csc\left(x\right)}\;dx = {-\frac{2}{\sin\left(x\right)}}\Bigg\vert_{\frac{1}{2} \, \pi}^{\frac{3}{2} \, \pi} = {4}\]

\input{2311_Concept_Integral_0003.HELP.tex}

\begin{multipleChoice}
\choice The antiderivative is incorrect.
\choice[correct] The integrand is not defined over the entire interval.
\choice The bounds are evaluated in the wrong order.
\choice Nothing is wrong.  The equation is correct, as is.
\end{multipleChoice}

\end{problem}}%}

%%%%%%%%%%%%%%%%%%%%%%


\latexProblemContent{
\begin{problem}

What is wrong with the following equation:

\[\int_{\frac{1}{4} \, \pi}^{\frac{3}{2} \, \pi} {3 \, \cot\left(x\right) \csc\left(x\right)}\;dx = {-\frac{3}{\sin\left(x\right)}}\Bigg\vert_{\frac{1}{4} \, \pi}^{\frac{3}{2} \, \pi} = {3 \, \sqrt{2} + 3}\]

\input{2311_Concept_Integral_0003.HELP.tex}

\begin{multipleChoice}
\choice The antiderivative is incorrect.
\choice[correct] The integrand is not defined over the entire interval.
\choice The bounds are evaluated in the wrong order.
\choice Nothing is wrong.  The equation is correct, as is.
\end{multipleChoice}

\end{problem}}%}

%%%%%%%%%%%%%%%%%%%%%%


\latexProblemContent{
\begin{problem}

What is wrong with the following equation:

\[\int_{\frac{1}{2} \, \pi}^{\frac{4}{3} \, \pi} {-11 \, \cot\left(x\right) \csc\left(x\right)}\;dx = {\frac{11}{\sin\left(x\right)}}\Bigg\vert_{\frac{1}{2} \, \pi}^{\frac{4}{3} \, \pi} = {-\frac{22}{3} \, \sqrt{3} - 11}\]

\input{2311_Concept_Integral_0003.HELP.tex}

\begin{multipleChoice}
\choice The antiderivative is incorrect.
\choice[correct] The integrand is not defined over the entire interval.
\choice The bounds are evaluated in the wrong order.
\choice Nothing is wrong.  The equation is correct, as is.
\end{multipleChoice}

\end{problem}}%}

%%%%%%%%%%%%%%%%%%%%%%


\latexProblemContent{
\begin{problem}

What is wrong with the following equation:

\[\int_{\frac{1}{4} \, \pi}^{\frac{7}{4} \, \pi} {-8 \, \csc\left(x\right)^{2}}\;dx = {\frac{8}{\tan\left(x\right)}}\Bigg\vert_{\frac{1}{4} \, \pi}^{\frac{7}{4} \, \pi} = {-16}\]

\input{2311_Concept_Integral_0003.HELP.tex}

\begin{multipleChoice}
\choice The antiderivative is incorrect.
\choice[correct] The integrand is not defined over the entire interval.
\choice The bounds are evaluated in the wrong order.
\choice Nothing is wrong.  The equation is correct, as is.
\end{multipleChoice}

\end{problem}}%}

%%%%%%%%%%%%%%%%%%%%%%


\latexProblemContent{
\begin{problem}

What is wrong with the following equation:

\[\int_{\frac{3}{4} \, \pi}^{\frac{4}{3} \, \pi} {3 \, \csc\left(x\right)^{2}}\;dx = {-\frac{3}{\tan\left(x\right)}}\Bigg\vert_{\frac{3}{4} \, \pi}^{\frac{4}{3} \, \pi} = {-\sqrt{3} - 3}\]

\input{2311_Concept_Integral_0003.HELP.tex}

\begin{multipleChoice}
\choice The antiderivative is incorrect.
\choice[correct] The integrand is not defined over the entire interval.
\choice The bounds are evaluated in the wrong order.
\choice Nothing is wrong.  The equation is correct, as is.
\end{multipleChoice}

\end{problem}}%}

%%%%%%%%%%%%%%%%%%%%%%


\latexProblemContent{
\begin{problem}

What is wrong with the following equation:

\[\int_{\frac{1}{3} \, \pi}^{\frac{5}{4} \, \pi} {10 \, \cot\left(x\right) \csc\left(x\right)}\;dx = {-\frac{10}{\sin\left(x\right)}}\Bigg\vert_{\frac{1}{3} \, \pi}^{\frac{5}{4} \, \pi} = {\frac{20}{3} \, \sqrt{3} + 10 \, \sqrt{2}}\]

\input{2311_Concept_Integral_0003.HELP.tex}

\begin{multipleChoice}
\choice The antiderivative is incorrect.
\choice[correct] The integrand is not defined over the entire interval.
\choice The bounds are evaluated in the wrong order.
\choice Nothing is wrong.  The equation is correct, as is.
\end{multipleChoice}

\end{problem}}%}

%%%%%%%%%%%%%%%%%%%%%%


\latexProblemContent{
\begin{problem}

What is wrong with the following equation:

\[\int_{\frac{1}{4} \, \pi}^{\frac{5}{4} \, \pi} {-12 \, \csc\left(x\right)^{2}}\;dx = {\frac{12}{\tan\left(x\right)}}\Bigg\vert_{\frac{1}{4} \, \pi}^{\frac{5}{4} \, \pi} = {0}\]

\input{2311_Concept_Integral_0003.HELP.tex}

\begin{multipleChoice}
\choice The antiderivative is incorrect.
\choice[correct] The integrand is not defined over the entire interval.
\choice The bounds are evaluated in the wrong order.
\choice Nothing is wrong.  The equation is correct, as is.
\end{multipleChoice}

\end{problem}}%}

%%%%%%%%%%%%%%%%%%%%%%


%%%%%%%%%%%%%%%%%%%%%%


\latexProblemContent{
\begin{problem}

What is wrong with the following equation:

\[\int_{\frac{1}{2} \, \pi}^{\frac{7}{4} \, \pi} {5 \, \csc\left(x\right)^{2}}\;dx = {-\frac{5}{\tan\left(x\right)}}\Bigg\vert_{\frac{1}{2} \, \pi}^{\frac{7}{4} \, \pi} = {5}\]

\input{2311_Concept_Integral_0003.HELP.tex}

\begin{multipleChoice}
\choice The antiderivative is incorrect.
\choice[correct] The integrand is not defined over the entire interval.
\choice The bounds are evaluated in the wrong order.
\choice Nothing is wrong.  The equation is correct, as is.
\end{multipleChoice}

\end{problem}}%}

%%%%%%%%%%%%%%%%%%%%%%


\latexProblemContent{
\begin{problem}

What is wrong with the following equation:

\[\int_{\frac{1}{3} \, \pi}^{\frac{7}{4} \, \pi} {4 \, \cot\left(x\right) \csc\left(x\right)}\;dx = {-\frac{4}{\sin\left(x\right)}}\Bigg\vert_{\frac{1}{3} \, \pi}^{\frac{7}{4} \, \pi} = {\frac{8}{3} \, \sqrt{3} + 4 \, \sqrt{2}}\]

\input{2311_Concept_Integral_0003.HELP.tex}

\begin{multipleChoice}
\choice The antiderivative is incorrect.
\choice[correct] The integrand is not defined over the entire interval.
\choice The bounds are evaluated in the wrong order.
\choice Nothing is wrong.  The equation is correct, as is.
\end{multipleChoice}

\end{problem}}%}

%%%%%%%%%%%%%%%%%%%%%%


\latexProblemContent{
\begin{problem}

What is wrong with the following equation:

\[\int_{\frac{3}{4} \, \pi}^{\frac{5}{4} \, \pi} {14 \, \cot\left(x\right) \csc\left(x\right)}\;dx = {-\frac{14}{\sin\left(x\right)}}\Bigg\vert_{\frac{3}{4} \, \pi}^{\frac{5}{4} \, \pi} = {28 \, \sqrt{2}}\]

\input{2311_Concept_Integral_0003.HELP.tex}

\begin{multipleChoice}
\choice The antiderivative is incorrect.
\choice[correct] The integrand is not defined over the entire interval.
\choice The bounds are evaluated in the wrong order.
\choice Nothing is wrong.  The equation is correct, as is.
\end{multipleChoice}

\end{problem}}%}

%%%%%%%%%%%%%%%%%%%%%%


\latexProblemContent{
\begin{problem}

What is wrong with the following equation:

\[\int_{\frac{1}{2} \, \pi}^{\frac{4}{3} \, \pi} {-\csc\left(x\right)^{2}}\;dx = {\frac{1}{\tan\left(x\right)}}\Bigg\vert_{\frac{1}{2} \, \pi}^{\frac{4}{3} \, \pi} = {\frac{1}{3} \, \sqrt{3}}\]

\input{2311_Concept_Integral_0003.HELP.tex}

\begin{multipleChoice}
\choice The antiderivative is incorrect.
\choice[correct] The integrand is not defined over the entire interval.
\choice The bounds are evaluated in the wrong order.
\choice Nothing is wrong.  The equation is correct, as is.
\end{multipleChoice}

\end{problem}}%}

%%%%%%%%%%%%%%%%%%%%%%


\latexProblemContent{
\begin{problem}

What is wrong with the following equation:

\[\int_{\frac{3}{4} \, \pi}^{\frac{5}{3} \, \pi} {4 \, \cot\left(x\right) \csc\left(x\right)}\;dx = {-\frac{4}{\sin\left(x\right)}}\Bigg\vert_{\frac{3}{4} \, \pi}^{\frac{5}{3} \, \pi} = {\frac{8}{3} \, \sqrt{3} + 4 \, \sqrt{2}}\]

\input{2311_Concept_Integral_0003.HELP.tex}

\begin{multipleChoice}
\choice The antiderivative is incorrect.
\choice[correct] The integrand is not defined over the entire interval.
\choice The bounds are evaluated in the wrong order.
\choice Nothing is wrong.  The equation is correct, as is.
\end{multipleChoice}

\end{problem}}%}

%%%%%%%%%%%%%%%%%%%%%%


\latexProblemContent{
\begin{problem}

What is wrong with the following equation:

\[\int_{\frac{1}{2} \, \pi}^{\frac{7}{4} \, \pi} {-3 \, \cot\left(x\right) \csc\left(x\right)}\;dx = {\frac{3}{\sin\left(x\right)}}\Bigg\vert_{\frac{1}{2} \, \pi}^{\frac{7}{4} \, \pi} = {-3 \, \sqrt{2} - 3}\]

\input{2311_Concept_Integral_0003.HELP.tex}

\begin{multipleChoice}
\choice The antiderivative is incorrect.
\choice[correct] The integrand is not defined over the entire interval.
\choice The bounds are evaluated in the wrong order.
\choice Nothing is wrong.  The equation is correct, as is.
\end{multipleChoice}

\end{problem}}%}

%%%%%%%%%%%%%%%%%%%%%%


%%%%%%%%%%%%%%%%%%%%%%


%%%%%%%%%%%%%%%%%%%%%%


\latexProblemContent{
\begin{problem}

What is wrong with the following equation:

\[\int_{\frac{1}{4} \, \pi}^{\frac{7}{4} \, \pi} {9 \, \csc\left(x\right)^{2}}\;dx = {-\frac{9}{\tan\left(x\right)}}\Bigg\vert_{\frac{1}{4} \, \pi}^{\frac{7}{4} \, \pi} = {18}\]

\input{2311_Concept_Integral_0003.HELP.tex}

\begin{multipleChoice}
\choice The antiderivative is incorrect.
\choice[correct] The integrand is not defined over the entire interval.
\choice The bounds are evaluated in the wrong order.
\choice Nothing is wrong.  The equation is correct, as is.
\end{multipleChoice}

\end{problem}}%}

%%%%%%%%%%%%%%%%%%%%%%


%%%%%%%%%%%%%%%%%%%%%%


\latexProblemContent{
\begin{problem}

What is wrong with the following equation:

\[\int_{\frac{1}{2} \, \pi}^{\frac{7}{4} \, \pi} {\csc\left(x\right)^{2}}\;dx = {-\frac{1}{\tan\left(x\right)}}\Bigg\vert_{\frac{1}{2} \, \pi}^{\frac{7}{4} \, \pi} = {1}\]

\input{2311_Concept_Integral_0003.HELP.tex}

\begin{multipleChoice}
\choice The antiderivative is incorrect.
\choice[correct] The integrand is not defined over the entire interval.
\choice The bounds are evaluated in the wrong order.
\choice Nothing is wrong.  The equation is correct, as is.
\end{multipleChoice}

\end{problem}}%}

%%%%%%%%%%%%%%%%%%%%%%


\latexProblemContent{
\begin{problem}

What is wrong with the following equation:

\[\int_{\frac{1}{3} \, \pi}^{\frac{3}{2} \, \pi} {11 \, \csc\left(x\right)^{2}}\;dx = {-\frac{11}{\tan\left(x\right)}}\Bigg\vert_{\frac{1}{3} \, \pi}^{\frac{3}{2} \, \pi} = {\frac{11}{3} \, \sqrt{3}}\]

\input{2311_Concept_Integral_0003.HELP.tex}

\begin{multipleChoice}
\choice The antiderivative is incorrect.
\choice[correct] The integrand is not defined over the entire interval.
\choice The bounds are evaluated in the wrong order.
\choice Nothing is wrong.  The equation is correct, as is.
\end{multipleChoice}

\end{problem}}%}

%%%%%%%%%%%%%%%%%%%%%%


\latexProblemContent{
\begin{problem}

What is wrong with the following equation:

\[\int_{\frac{1}{3} \, \pi}^{\frac{3}{2} \, \pi} {\csc\left(x\right)^{2}}\;dx = {-\frac{1}{\tan\left(x\right)}}\Bigg\vert_{\frac{1}{3} \, \pi}^{\frac{3}{2} \, \pi} = {\frac{1}{3} \, \sqrt{3}}\]

\input{2311_Concept_Integral_0003.HELP.tex}

\begin{multipleChoice}
\choice The antiderivative is incorrect.
\choice[correct] The integrand is not defined over the entire interval.
\choice The bounds are evaluated in the wrong order.
\choice Nothing is wrong.  The equation is correct, as is.
\end{multipleChoice}

\end{problem}}%}

%%%%%%%%%%%%%%%%%%%%%%


%%%%%%%%%%%%%%%%%%%%%%


\latexProblemContent{
\begin{problem}

What is wrong with the following equation:

\[\int_{\frac{1}{4} \, \pi}^{\frac{4}{3} \, \pi} {\cot\left(x\right) \csc\left(x\right)}\;dx = {-\frac{1}{\sin\left(x\right)}}\Bigg\vert_{\frac{1}{4} \, \pi}^{\frac{4}{3} \, \pi} = {\frac{2}{3} \, \sqrt{3} + \sqrt{2}}\]

\input{2311_Concept_Integral_0003.HELP.tex}

\begin{multipleChoice}
\choice The antiderivative is incorrect.
\choice[correct] The integrand is not defined over the entire interval.
\choice The bounds are evaluated in the wrong order.
\choice Nothing is wrong.  The equation is correct, as is.
\end{multipleChoice}

\end{problem}}%}

%%%%%%%%%%%%%%%%%%%%%%


%%%%%%%%%%%%%%%%%%%%%%


\latexProblemContent{
\begin{problem}

What is wrong with the following equation:

\[\int_{\frac{1}{3} \, \pi}^{\frac{5}{4} \, \pi} {-4 \, \csc\left(x\right)^{2}}\;dx = {\frac{4}{\tan\left(x\right)}}\Bigg\vert_{\frac{1}{3} \, \pi}^{\frac{5}{4} \, \pi} = {-\frac{4}{3} \, \sqrt{3} + 4}\]

\input{2311_Concept_Integral_0003.HELP.tex}

\begin{multipleChoice}
\choice The antiderivative is incorrect.
\choice[correct] The integrand is not defined over the entire interval.
\choice The bounds are evaluated in the wrong order.
\choice Nothing is wrong.  The equation is correct, as is.
\end{multipleChoice}

\end{problem}}%}

%%%%%%%%%%%%%%%%%%%%%%


\latexProblemContent{
\begin{problem}

What is wrong with the following equation:

\[\int_{\frac{3}{4} \, \pi}^{\frac{7}{4} \, \pi} {-15 \, \csc\left(x\right)^{2}}\;dx = {\frac{15}{\tan\left(x\right)}}\Bigg\vert_{\frac{3}{4} \, \pi}^{\frac{7}{4} \, \pi} = {0}\]

\input{2311_Concept_Integral_0003.HELP.tex}

\begin{multipleChoice}
\choice The antiderivative is incorrect.
\choice[correct] The integrand is not defined over the entire interval.
\choice The bounds are evaluated in the wrong order.
\choice Nothing is wrong.  The equation is correct, as is.
\end{multipleChoice}

\end{problem}}%}

%%%%%%%%%%%%%%%%%%%%%%


\latexProblemContent{
\begin{problem}

What is wrong with the following equation:

\[\int_{\frac{2}{3} \, \pi}^{\frac{3}{2} \, \pi} {7 \, \cot\left(x\right) \csc\left(x\right)}\;dx = {-\frac{7}{\sin\left(x\right)}}\Bigg\vert_{\frac{2}{3} \, \pi}^{\frac{3}{2} \, \pi} = {\frac{14}{3} \, \sqrt{3} + 7}\]

\input{2311_Concept_Integral_0003.HELP.tex}

\begin{multipleChoice}
\choice The antiderivative is incorrect.
\choice[correct] The integrand is not defined over the entire interval.
\choice The bounds are evaluated in the wrong order.
\choice Nothing is wrong.  The equation is correct, as is.
\end{multipleChoice}

\end{problem}}%}

%%%%%%%%%%%%%%%%%%%%%%


%%%%%%%%%%%%%%%%%%%%%%


\latexProblemContent{
\begin{problem}

What is wrong with the following equation:

\[\int_{\frac{3}{4} \, \pi}^{\frac{5}{3} \, \pi} {-4 \, \cot\left(x\right) \csc\left(x\right)}\;dx = {\frac{4}{\sin\left(x\right)}}\Bigg\vert_{\frac{3}{4} \, \pi}^{\frac{5}{3} \, \pi} = {-\frac{8}{3} \, \sqrt{3} - 4 \, \sqrt{2}}\]

\input{2311_Concept_Integral_0003.HELP.tex}

\begin{multipleChoice}
\choice The antiderivative is incorrect.
\choice[correct] The integrand is not defined over the entire interval.
\choice The bounds are evaluated in the wrong order.
\choice Nothing is wrong.  The equation is correct, as is.
\end{multipleChoice}

\end{problem}}%}

%%%%%%%%%%%%%%%%%%%%%%


\latexProblemContent{
\begin{problem}

What is wrong with the following equation:

\[\int_{\frac{3}{4} \, \pi}^{\frac{4}{3} \, \pi} {-5 \, \csc\left(x\right)^{2}}\;dx = {\frac{5}{\tan\left(x\right)}}\Bigg\vert_{\frac{3}{4} \, \pi}^{\frac{4}{3} \, \pi} = {\frac{5}{3} \, \sqrt{3} + 5}\]

\input{2311_Concept_Integral_0003.HELP.tex}

\begin{multipleChoice}
\choice The antiderivative is incorrect.
\choice[correct] The integrand is not defined over the entire interval.
\choice The bounds are evaluated in the wrong order.
\choice Nothing is wrong.  The equation is correct, as is.
\end{multipleChoice}

\end{problem}}%}

%%%%%%%%%%%%%%%%%%%%%%


\latexProblemContent{
\begin{problem}

What is wrong with the following equation:

\[\int_{\frac{1}{2} \, \pi}^{\frac{4}{3} \, \pi} {\cot\left(x\right) \csc\left(x\right)}\;dx = {-\frac{1}{\sin\left(x\right)}}\Bigg\vert_{\frac{1}{2} \, \pi}^{\frac{4}{3} \, \pi} = {\frac{2}{3} \, \sqrt{3} + 1}\]

\input{2311_Concept_Integral_0003.HELP.tex}

\begin{multipleChoice}
\choice The antiderivative is incorrect.
\choice[correct] The integrand is not defined over the entire interval.
\choice The bounds are evaluated in the wrong order.
\choice Nothing is wrong.  The equation is correct, as is.
\end{multipleChoice}

\end{problem}}%}

%%%%%%%%%%%%%%%%%%%%%%


\latexProblemContent{
\begin{problem}

What is wrong with the following equation:

\[\int_{\frac{2}{3} \, \pi}^{\frac{4}{3} \, \pi} {11 \, \cot\left(x\right) \csc\left(x\right)}\;dx = {-\frac{11}{\sin\left(x\right)}}\Bigg\vert_{\frac{2}{3} \, \pi}^{\frac{4}{3} \, \pi} = {\frac{44}{3} \, \sqrt{3}}\]

\input{2311_Concept_Integral_0003.HELP.tex}

\begin{multipleChoice}
\choice The antiderivative is incorrect.
\choice[correct] The integrand is not defined over the entire interval.
\choice The bounds are evaluated in the wrong order.
\choice Nothing is wrong.  The equation is correct, as is.
\end{multipleChoice}

\end{problem}}%}

%%%%%%%%%%%%%%%%%%%%%%


\latexProblemContent{
\begin{problem}

What is wrong with the following equation:

\[\int_{\frac{3}{4} \, \pi}^{\frac{3}{2} \, \pi} {12 \, \cot\left(x\right) \csc\left(x\right)}\;dx = {-\frac{12}{\sin\left(x\right)}}\Bigg\vert_{\frac{3}{4} \, \pi}^{\frac{3}{2} \, \pi} = {12 \, \sqrt{2} + 12}\]

\input{2311_Concept_Integral_0003.HELP.tex}

\begin{multipleChoice}
\choice The antiderivative is incorrect.
\choice[correct] The integrand is not defined over the entire interval.
\choice The bounds are evaluated in the wrong order.
\choice Nothing is wrong.  The equation is correct, as is.
\end{multipleChoice}

\end{problem}}%}

%%%%%%%%%%%%%%%%%%%%%%


\latexProblemContent{
\begin{problem}

What is wrong with the following equation:

\[\int_{\frac{1}{4} \, \pi}^{\frac{3}{2} \, \pi} {-13 \, \csc\left(x\right)^{2}}\;dx = {\frac{13}{\tan\left(x\right)}}\Bigg\vert_{\frac{1}{4} \, \pi}^{\frac{3}{2} \, \pi} = {-13}\]

\input{2311_Concept_Integral_0003.HELP.tex}

\begin{multipleChoice}
\choice The antiderivative is incorrect.
\choice[correct] The integrand is not defined over the entire interval.
\choice The bounds are evaluated in the wrong order.
\choice Nothing is wrong.  The equation is correct, as is.
\end{multipleChoice}

\end{problem}}%}

%%%%%%%%%%%%%%%%%%%%%%


\latexProblemContent{
\begin{problem}

What is wrong with the following equation:

\[\int_{\frac{2}{3} \, \pi}^{\frac{7}{4} \, \pi} {15 \, \cot\left(x\right) \csc\left(x\right)}\;dx = {-\frac{15}{\sin\left(x\right)}}\Bigg\vert_{\frac{2}{3} \, \pi}^{\frac{7}{4} \, \pi} = {10 \, \sqrt{3} + 15 \, \sqrt{2}}\]

\input{2311_Concept_Integral_0003.HELP.tex}

\begin{multipleChoice}
\choice The antiderivative is incorrect.
\choice[correct] The integrand is not defined over the entire interval.
\choice The bounds are evaluated in the wrong order.
\choice Nothing is wrong.  The equation is correct, as is.
\end{multipleChoice}

\end{problem}}%}

%%%%%%%%%%%%%%%%%%%%%%


\latexProblemContent{
\begin{problem}

What is wrong with the following equation:

\[\int_{\frac{1}{4} \, \pi}^{\frac{7}{4} \, \pi} {-13 \, \csc\left(x\right)^{2}}\;dx = {\frac{13}{\tan\left(x\right)}}\Bigg\vert_{\frac{1}{4} \, \pi}^{\frac{7}{4} \, \pi} = {-26}\]

\input{2311_Concept_Integral_0003.HELP.tex}

\begin{multipleChoice}
\choice The antiderivative is incorrect.
\choice[correct] The integrand is not defined over the entire interval.
\choice The bounds are evaluated in the wrong order.
\choice Nothing is wrong.  The equation is correct, as is.
\end{multipleChoice}

\end{problem}}%}

%%%%%%%%%%%%%%%%%%%%%%


\latexProblemContent{
\begin{problem}

What is wrong with the following equation:

\[\int_{\frac{1}{2} \, \pi}^{\frac{7}{4} \, \pi} {-3 \, \csc\left(x\right)^{2}}\;dx = {\frac{3}{\tan\left(x\right)}}\Bigg\vert_{\frac{1}{2} \, \pi}^{\frac{7}{4} \, \pi} = {-3}\]

\input{2311_Concept_Integral_0003.HELP.tex}

\begin{multipleChoice}
\choice The antiderivative is incorrect.
\choice[correct] The integrand is not defined over the entire interval.
\choice The bounds are evaluated in the wrong order.
\choice Nothing is wrong.  The equation is correct, as is.
\end{multipleChoice}

\end{problem}}%}

%%%%%%%%%%%%%%%%%%%%%%


\latexProblemContent{
\begin{problem}

What is wrong with the following equation:

\[\int_{\frac{2}{3} \, \pi}^{\frac{3}{2} \, \pi} {14 \, \csc\left(x\right)^{2}}\;dx = {-\frac{14}{\tan\left(x\right)}}\Bigg\vert_{\frac{2}{3} \, \pi}^{\frac{3}{2} \, \pi} = {-\frac{14}{3} \, \sqrt{3}}\]

\input{2311_Concept_Integral_0003.HELP.tex}

\begin{multipleChoice}
\choice The antiderivative is incorrect.
\choice[correct] The integrand is not defined over the entire interval.
\choice The bounds are evaluated in the wrong order.
\choice Nothing is wrong.  The equation is correct, as is.
\end{multipleChoice}

\end{problem}}%}

%%%%%%%%%%%%%%%%%%%%%%


%%%%%%%%%%%%%%%%%%%%%%


\latexProblemContent{
\begin{problem}

What is wrong with the following equation:

\[\int_{\frac{1}{3} \, \pi}^{\frac{3}{2} \, \pi} {8 \, \csc\left(x\right)^{2}}\;dx = {-\frac{8}{\tan\left(x\right)}}\Bigg\vert_{\frac{1}{3} \, \pi}^{\frac{3}{2} \, \pi} = {\frac{8}{3} \, \sqrt{3}}\]

\input{2311_Concept_Integral_0003.HELP.tex}

\begin{multipleChoice}
\choice The antiderivative is incorrect.
\choice[correct] The integrand is not defined over the entire interval.
\choice The bounds are evaluated in the wrong order.
\choice Nothing is wrong.  The equation is correct, as is.
\end{multipleChoice}

\end{problem}}%}

%%%%%%%%%%%%%%%%%%%%%%


\latexProblemContent{
\begin{problem}

What is wrong with the following equation:

\[\int_{\frac{1}{4} \, \pi}^{\frac{7}{4} \, \pi} {8 \, \csc\left(x\right)^{2}}\;dx = {-\frac{8}{\tan\left(x\right)}}\Bigg\vert_{\frac{1}{4} \, \pi}^{\frac{7}{4} \, \pi} = {16}\]

\input{2311_Concept_Integral_0003.HELP.tex}

\begin{multipleChoice}
\choice The antiderivative is incorrect.
\choice[correct] The integrand is not defined over the entire interval.
\choice The bounds are evaluated in the wrong order.
\choice Nothing is wrong.  The equation is correct, as is.
\end{multipleChoice}

\end{problem}}%}

%%%%%%%%%%%%%%%%%%%%%%


%%%%%%%%%%%%%%%%%%%%%%


\latexProblemContent{
\begin{problem}

What is wrong with the following equation:

\[\int_{\frac{1}{2} \, \pi}^{\frac{3}{2} \, \pi} {-9 \, \cot\left(x\right) \csc\left(x\right)}\;dx = {\frac{9}{\sin\left(x\right)}}\Bigg\vert_{\frac{1}{2} \, \pi}^{\frac{3}{2} \, \pi} = {-18}\]

\input{2311_Concept_Integral_0003.HELP.tex}

\begin{multipleChoice}
\choice The antiderivative is incorrect.
\choice[correct] The integrand is not defined over the entire interval.
\choice The bounds are evaluated in the wrong order.
\choice Nothing is wrong.  The equation is correct, as is.
\end{multipleChoice}

\end{problem}}%}

%%%%%%%%%%%%%%%%%%%%%%


\latexProblemContent{
\begin{problem}

What is wrong with the following equation:

\[\int_{\frac{1}{2} \, \pi}^{\frac{5}{4} \, \pi} {-2 \, \csc\left(x\right)^{2}}\;dx = {\frac{2}{\tan\left(x\right)}}\Bigg\vert_{\frac{1}{2} \, \pi}^{\frac{5}{4} \, \pi} = {2}\]

\input{2311_Concept_Integral_0003.HELP.tex}

\begin{multipleChoice}
\choice The antiderivative is incorrect.
\choice[correct] The integrand is not defined over the entire interval.
\choice The bounds are evaluated in the wrong order.
\choice Nothing is wrong.  The equation is correct, as is.
\end{multipleChoice}

\end{problem}}%}

%%%%%%%%%%%%%%%%%%%%%%


%%%%%%%%%%%%%%%%%%%%%%


\latexProblemContent{
\begin{problem}

What is wrong with the following equation:

\[\int_{\frac{1}{4} \, \pi}^{\frac{5}{4} \, \pi} {-12 \, \cot\left(x\right) \csc\left(x\right)}\;dx = {\frac{12}{\sin\left(x\right)}}\Bigg\vert_{\frac{1}{4} \, \pi}^{\frac{5}{4} \, \pi} = {-24 \, \sqrt{2}}\]

\input{2311_Concept_Integral_0003.HELP.tex}

\begin{multipleChoice}
\choice The antiderivative is incorrect.
\choice[correct] The integrand is not defined over the entire interval.
\choice The bounds are evaluated in the wrong order.
\choice Nothing is wrong.  The equation is correct, as is.
\end{multipleChoice}

\end{problem}}%}

%%%%%%%%%%%%%%%%%%%%%%


%%%%%%%%%%%%%%%%%%%%%%


\latexProblemContent{
\begin{problem}

What is wrong with the following equation:

\[\int_{\frac{1}{3} \, \pi}^{\frac{5}{3} \, \pi} {4 \, \csc\left(x\right)^{2}}\;dx = {-\frac{4}{\tan\left(x\right)}}\Bigg\vert_{\frac{1}{3} \, \pi}^{\frac{5}{3} \, \pi} = {\frac{8}{3} \, \sqrt{3}}\]

\input{2311_Concept_Integral_0003.HELP.tex}

\begin{multipleChoice}
\choice The antiderivative is incorrect.
\choice[correct] The integrand is not defined over the entire interval.
\choice The bounds are evaluated in the wrong order.
\choice Nothing is wrong.  The equation is correct, as is.
\end{multipleChoice}

\end{problem}}%}

%%%%%%%%%%%%%%%%%%%%%%


\latexProblemContent{
\begin{problem}

What is wrong with the following equation:

\[\int_{\frac{3}{4} \, \pi}^{\frac{5}{3} \, \pi} {-3 \, \cot\left(x\right) \csc\left(x\right)}\;dx = {\frac{3}{\sin\left(x\right)}}\Bigg\vert_{\frac{3}{4} \, \pi}^{\frac{5}{3} \, \pi} = {-2 \, \sqrt{3} - 3 \, \sqrt{2}}\]

\input{2311_Concept_Integral_0003.HELP.tex}

\begin{multipleChoice}
\choice The antiderivative is incorrect.
\choice[correct] The integrand is not defined over the entire interval.
\choice The bounds are evaluated in the wrong order.
\choice Nothing is wrong.  The equation is correct, as is.
\end{multipleChoice}

\end{problem}}%}

%%%%%%%%%%%%%%%%%%%%%%


\latexProblemContent{
\begin{problem}

What is wrong with the following equation:

\[\int_{\frac{1}{3} \, \pi}^{\frac{3}{2} \, \pi} {-8 \, \cot\left(x\right) \csc\left(x\right)}\;dx = {\frac{8}{\sin\left(x\right)}}\Bigg\vert_{\frac{1}{3} \, \pi}^{\frac{3}{2} \, \pi} = {-\frac{16}{3} \, \sqrt{3} - 8}\]

\input{2311_Concept_Integral_0003.HELP.tex}

\begin{multipleChoice}
\choice The antiderivative is incorrect.
\choice[correct] The integrand is not defined over the entire interval.
\choice The bounds are evaluated in the wrong order.
\choice Nothing is wrong.  The equation is correct, as is.
\end{multipleChoice}

\end{problem}}%}

%%%%%%%%%%%%%%%%%%%%%%


\latexProblemContent{
\begin{problem}

What is wrong with the following equation:

\[\int_{\frac{2}{3} \, \pi}^{\frac{3}{2} \, \pi} {11 \, \csc\left(x\right)^{2}}\;dx = {-\frac{11}{\tan\left(x\right)}}\Bigg\vert_{\frac{2}{3} \, \pi}^{\frac{3}{2} \, \pi} = {-\frac{11}{3} \, \sqrt{3}}\]

\input{2311_Concept_Integral_0003.HELP.tex}

\begin{multipleChoice}
\choice The antiderivative is incorrect.
\choice[correct] The integrand is not defined over the entire interval.
\choice The bounds are evaluated in the wrong order.
\choice Nothing is wrong.  The equation is correct, as is.
\end{multipleChoice}

\end{problem}}%}

%%%%%%%%%%%%%%%%%%%%%%


%%%%%%%%%%%%%%%%%%%%%%


%%%%%%%%%%%%%%%%%%%%%%


%%%%%%%%%%%%%%%%%%%%%%


\latexProblemContent{
\begin{problem}

What is wrong with the following equation:

\[\int_{\frac{3}{4} \, \pi}^{\frac{4}{3} \, \pi} {8 \, \cot\left(x\right) \csc\left(x\right)}\;dx = {-\frac{8}{\sin\left(x\right)}}\Bigg\vert_{\frac{3}{4} \, \pi}^{\frac{4}{3} \, \pi} = {\frac{16}{3} \, \sqrt{3} + 8 \, \sqrt{2}}\]

\input{2311_Concept_Integral_0003.HELP.tex}

\begin{multipleChoice}
\choice The antiderivative is incorrect.
\choice[correct] The integrand is not defined over the entire interval.
\choice The bounds are evaluated in the wrong order.
\choice Nothing is wrong.  The equation is correct, as is.
\end{multipleChoice}

\end{problem}}%}

%%%%%%%%%%%%%%%%%%%%%%


\latexProblemContent{
\begin{problem}

What is wrong with the following equation:

\[\int_{\frac{2}{3} \, \pi}^{\frac{3}{2} \, \pi} {-11 \, \csc\left(x\right)^{2}}\;dx = {\frac{11}{\tan\left(x\right)}}\Bigg\vert_{\frac{2}{3} \, \pi}^{\frac{3}{2} \, \pi} = {\frac{11}{3} \, \sqrt{3}}\]

\input{2311_Concept_Integral_0003.HELP.tex}

\begin{multipleChoice}
\choice The antiderivative is incorrect.
\choice[correct] The integrand is not defined over the entire interval.
\choice The bounds are evaluated in the wrong order.
\choice Nothing is wrong.  The equation is correct, as is.
\end{multipleChoice}

\end{problem}}%}

%%%%%%%%%%%%%%%%%%%%%%


%%%%%%%%%%%%%%%%%%%%%%


\latexProblemContent{
\begin{problem}

What is wrong with the following equation:

\[\int_{\frac{2}{3} \, \pi}^{\frac{3}{2} \, \pi} {-12 \, \cot\left(x\right) \csc\left(x\right)}\;dx = {\frac{12}{\sin\left(x\right)}}\Bigg\vert_{\frac{2}{3} \, \pi}^{\frac{3}{2} \, \pi} = {-8 \, \sqrt{3} - 12}\]

\input{2311_Concept_Integral_0003.HELP.tex}

\begin{multipleChoice}
\choice The antiderivative is incorrect.
\choice[correct] The integrand is not defined over the entire interval.
\choice The bounds are evaluated in the wrong order.
\choice Nothing is wrong.  The equation is correct, as is.
\end{multipleChoice}

\end{problem}}%}

%%%%%%%%%%%%%%%%%%%%%%


\latexProblemContent{
\begin{problem}

What is wrong with the following equation:

\[\int_{\frac{3}{4} \, \pi}^{\frac{5}{3} \, \pi} {-12 \, \cot\left(x\right) \csc\left(x\right)}\;dx = {\frac{12}{\sin\left(x\right)}}\Bigg\vert_{\frac{3}{4} \, \pi}^{\frac{5}{3} \, \pi} = {-8 \, \sqrt{3} - 12 \, \sqrt{2}}\]

\input{2311_Concept_Integral_0003.HELP.tex}

\begin{multipleChoice}
\choice The antiderivative is incorrect.
\choice[correct] The integrand is not defined over the entire interval.
\choice The bounds are evaluated in the wrong order.
\choice Nothing is wrong.  The equation is correct, as is.
\end{multipleChoice}

\end{problem}}%}

%%%%%%%%%%%%%%%%%%%%%%


\latexProblemContent{
\begin{problem}

What is wrong with the following equation:

\[\int_{\frac{1}{2} \, \pi}^{\frac{5}{4} \, \pi} {-10 \, \csc\left(x\right)^{2}}\;dx = {\frac{10}{\tan\left(x\right)}}\Bigg\vert_{\frac{1}{2} \, \pi}^{\frac{5}{4} \, \pi} = {10}\]

\input{2311_Concept_Integral_0003.HELP.tex}

\begin{multipleChoice}
\choice The antiderivative is incorrect.
\choice[correct] The integrand is not defined over the entire interval.
\choice The bounds are evaluated in the wrong order.
\choice Nothing is wrong.  The equation is correct, as is.
\end{multipleChoice}

\end{problem}}%}

%%%%%%%%%%%%%%%%%%%%%%


\latexProblemContent{
\begin{problem}

What is wrong with the following equation:

\[\int_{\frac{1}{2} \, \pi}^{\frac{5}{4} \, \pi} {-10 \, \cot\left(x\right) \csc\left(x\right)}\;dx = {\frac{10}{\sin\left(x\right)}}\Bigg\vert_{\frac{1}{2} \, \pi}^{\frac{5}{4} \, \pi} = {-10 \, \sqrt{2} - 10}\]

\input{2311_Concept_Integral_0003.HELP.tex}

\begin{multipleChoice}
\choice The antiderivative is incorrect.
\choice[correct] The integrand is not defined over the entire interval.
\choice The bounds are evaluated in the wrong order.
\choice Nothing is wrong.  The equation is correct, as is.
\end{multipleChoice}

\end{problem}}%}

%%%%%%%%%%%%%%%%%%%%%%


\latexProblemContent{
\begin{problem}

What is wrong with the following equation:

\[\int_{\frac{1}{3} \, \pi}^{\frac{7}{4} \, \pi} {-6 \, \csc\left(x\right)^{2}}\;dx = {\frac{6}{\tan\left(x\right)}}\Bigg\vert_{\frac{1}{3} \, \pi}^{\frac{7}{4} \, \pi} = {-2 \, \sqrt{3} - 6}\]

\input{2311_Concept_Integral_0003.HELP.tex}

\begin{multipleChoice}
\choice The antiderivative is incorrect.
\choice[correct] The integrand is not defined over the entire interval.
\choice The bounds are evaluated in the wrong order.
\choice Nothing is wrong.  The equation is correct, as is.
\end{multipleChoice}

\end{problem}}%}

%%%%%%%%%%%%%%%%%%%%%%


\latexProblemContent{
\begin{problem}

What is wrong with the following equation:

\[\int_{\frac{1}{4} \, \pi}^{\frac{5}{4} \, \pi} {-10 \, \csc\left(x\right)^{2}}\;dx = {\frac{10}{\tan\left(x\right)}}\Bigg\vert_{\frac{1}{4} \, \pi}^{\frac{5}{4} \, \pi} = {0}\]

\input{2311_Concept_Integral_0003.HELP.tex}

\begin{multipleChoice}
\choice The antiderivative is incorrect.
\choice[correct] The integrand is not defined over the entire interval.
\choice The bounds are evaluated in the wrong order.
\choice Nothing is wrong.  The equation is correct, as is.
\end{multipleChoice}

\end{problem}}%}

%%%%%%%%%%%%%%%%%%%%%%


%%%%%%%%%%%%%%%%%%%%%%


\latexProblemContent{
\begin{problem}

What is wrong with the following equation:

\[\int_{\frac{2}{3} \, \pi}^{\frac{3}{2} \, \pi} {-2 \, \cot\left(x\right) \csc\left(x\right)}\;dx = {\frac{2}{\sin\left(x\right)}}\Bigg\vert_{\frac{2}{3} \, \pi}^{\frac{3}{2} \, \pi} = {-\frac{4}{3} \, \sqrt{3} - 2}\]

\input{2311_Concept_Integral_0003.HELP.tex}

\begin{multipleChoice}
\choice The antiderivative is incorrect.
\choice[correct] The integrand is not defined over the entire interval.
\choice The bounds are evaluated in the wrong order.
\choice Nothing is wrong.  The equation is correct, as is.
\end{multipleChoice}

\end{problem}}%}

%%%%%%%%%%%%%%%%%%%%%%


\latexProblemContent{
\begin{problem}

What is wrong with the following equation:

\[\int_{\frac{3}{4} \, \pi}^{\frac{5}{4} \, \pi} {-\csc\left(x\right)^{2}}\;dx = {\frac{1}{\tan\left(x\right)}}\Bigg\vert_{\frac{3}{4} \, \pi}^{\frac{5}{4} \, \pi} = {2}\]

\input{2311_Concept_Integral_0003.HELP.tex}

\begin{multipleChoice}
\choice The antiderivative is incorrect.
\choice[correct] The integrand is not defined over the entire interval.
\choice The bounds are evaluated in the wrong order.
\choice Nothing is wrong.  The equation is correct, as is.
\end{multipleChoice}

\end{problem}}%}

%%%%%%%%%%%%%%%%%%%%%%


\latexProblemContent{
\begin{problem}

What is wrong with the following equation:

\[\int_{\frac{3}{4} \, \pi}^{\frac{5}{3} \, \pi} {8 \, \csc\left(x\right)^{2}}\;dx = {-\frac{8}{\tan\left(x\right)}}\Bigg\vert_{\frac{3}{4} \, \pi}^{\frac{5}{3} \, \pi} = {\frac{8}{3} \, \sqrt{3} - 8}\]

\input{2311_Concept_Integral_0003.HELP.tex}

\begin{multipleChoice}
\choice The antiderivative is incorrect.
\choice[correct] The integrand is not defined over the entire interval.
\choice The bounds are evaluated in the wrong order.
\choice Nothing is wrong.  The equation is correct, as is.
\end{multipleChoice}

\end{problem}}%}

%%%%%%%%%%%%%%%%%%%%%%


\latexProblemContent{
\begin{problem}

What is wrong with the following equation:

\[\int_{\frac{1}{4} \, \pi}^{\frac{5}{3} \, \pi} {-3 \, \cot\left(x\right) \csc\left(x\right)}\;dx = {\frac{3}{\sin\left(x\right)}}\Bigg\vert_{\frac{1}{4} \, \pi}^{\frac{5}{3} \, \pi} = {-2 \, \sqrt{3} - 3 \, \sqrt{2}}\]

\input{2311_Concept_Integral_0003.HELP.tex}

\begin{multipleChoice}
\choice The antiderivative is incorrect.
\choice[correct] The integrand is not defined over the entire interval.
\choice The bounds are evaluated in the wrong order.
\choice Nothing is wrong.  The equation is correct, as is.
\end{multipleChoice}

\end{problem}}%}

%%%%%%%%%%%%%%%%%%%%%%


\latexProblemContent{
\begin{problem}

What is wrong with the following equation:

\[\int_{\frac{3}{4} \, \pi}^{\frac{5}{4} \, \pi} {-3 \, \cot\left(x\right) \csc\left(x\right)}\;dx = {\frac{3}{\sin\left(x\right)}}\Bigg\vert_{\frac{3}{4} \, \pi}^{\frac{5}{4} \, \pi} = {-6 \, \sqrt{2}}\]

\input{2311_Concept_Integral_0003.HELP.tex}

\begin{multipleChoice}
\choice The antiderivative is incorrect.
\choice[correct] The integrand is not defined over the entire interval.
\choice The bounds are evaluated in the wrong order.
\choice Nothing is wrong.  The equation is correct, as is.
\end{multipleChoice}

\end{problem}}%}

%%%%%%%%%%%%%%%%%%%%%%


%%%%%%%%%%%%%%%%%%%%%%


\latexProblemContent{
\begin{problem}

What is wrong with the following equation:

\[\int_{\frac{1}{4} \, \pi}^{\frac{4}{3} \, \pi} {3 \, \cot\left(x\right) \csc\left(x\right)}\;dx = {-\frac{3}{\sin\left(x\right)}}\Bigg\vert_{\frac{1}{4} \, \pi}^{\frac{4}{3} \, \pi} = {2 \, \sqrt{3} + 3 \, \sqrt{2}}\]

\input{2311_Concept_Integral_0003.HELP.tex}

\begin{multipleChoice}
\choice The antiderivative is incorrect.
\choice[correct] The integrand is not defined over the entire interval.
\choice The bounds are evaluated in the wrong order.
\choice Nothing is wrong.  The equation is correct, as is.
\end{multipleChoice}

\end{problem}}%}

%%%%%%%%%%%%%%%%%%%%%%


\latexProblemContent{
\begin{problem}

What is wrong with the following equation:

\[\int_{\frac{1}{4} \, \pi}^{\frac{7}{4} \, \pi} {11 \, \cot\left(x\right) \csc\left(x\right)}\;dx = {-\frac{11}{\sin\left(x\right)}}\Bigg\vert_{\frac{1}{4} \, \pi}^{\frac{7}{4} \, \pi} = {22 \, \sqrt{2}}\]

\input{2311_Concept_Integral_0003.HELP.tex}

\begin{multipleChoice}
\choice The antiderivative is incorrect.
\choice[correct] The integrand is not defined over the entire interval.
\choice The bounds are evaluated in the wrong order.
\choice Nothing is wrong.  The equation is correct, as is.
\end{multipleChoice}

\end{problem}}%}

%%%%%%%%%%%%%%%%%%%%%%


\latexProblemContent{
\begin{problem}

What is wrong with the following equation:

\[\int_{\frac{1}{2} \, \pi}^{\frac{5}{4} \, \pi} {5 \, \cot\left(x\right) \csc\left(x\right)}\;dx = {-\frac{5}{\sin\left(x\right)}}\Bigg\vert_{\frac{1}{2} \, \pi}^{\frac{5}{4} \, \pi} = {5 \, \sqrt{2} + 5}\]

\input{2311_Concept_Integral_0003.HELP.tex}

\begin{multipleChoice}
\choice The antiderivative is incorrect.
\choice[correct] The integrand is not defined over the entire interval.
\choice The bounds are evaluated in the wrong order.
\choice Nothing is wrong.  The equation is correct, as is.
\end{multipleChoice}

\end{problem}}%}

%%%%%%%%%%%%%%%%%%%%%%


\latexProblemContent{
\begin{problem}

What is wrong with the following equation:

\[\int_{\frac{1}{4} \, \pi}^{\frac{3}{2} \, \pi} {-5 \, \cot\left(x\right) \csc\left(x\right)}\;dx = {\frac{5}{\sin\left(x\right)}}\Bigg\vert_{\frac{1}{4} \, \pi}^{\frac{3}{2} \, \pi} = {-5 \, \sqrt{2} - 5}\]

\input{2311_Concept_Integral_0003.HELP.tex}

\begin{multipleChoice}
\choice The antiderivative is incorrect.
\choice[correct] The integrand is not defined over the entire interval.
\choice The bounds are evaluated in the wrong order.
\choice Nothing is wrong.  The equation is correct, as is.
\end{multipleChoice}

\end{problem}}%}

%%%%%%%%%%%%%%%%%%%%%%


\latexProblemContent{
\begin{problem}

What is wrong with the following equation:

\[\int_{\frac{1}{3} \, \pi}^{\frac{4}{3} \, \pi} {10 \, \cot\left(x\right) \csc\left(x\right)}\;dx = {-\frac{10}{\sin\left(x\right)}}\Bigg\vert_{\frac{1}{3} \, \pi}^{\frac{4}{3} \, \pi} = {\frac{40}{3} \, \sqrt{3}}\]

\input{2311_Concept_Integral_0003.HELP.tex}

\begin{multipleChoice}
\choice The antiderivative is incorrect.
\choice[correct] The integrand is not defined over the entire interval.
\choice The bounds are evaluated in the wrong order.
\choice Nothing is wrong.  The equation is correct, as is.
\end{multipleChoice}

\end{problem}}%}

%%%%%%%%%%%%%%%%%%%%%%


\latexProblemContent{
\begin{problem}

What is wrong with the following equation:

\[\int_{\frac{1}{4} \, \pi}^{\frac{4}{3} \, \pi} {10 \, \csc\left(x\right)^{2}}\;dx = {-\frac{10}{\tan\left(x\right)}}\Bigg\vert_{\frac{1}{4} \, \pi}^{\frac{4}{3} \, \pi} = {-\frac{10}{3} \, \sqrt{3} + 10}\]

\input{2311_Concept_Integral_0003.HELP.tex}

\begin{multipleChoice}
\choice The antiderivative is incorrect.
\choice[correct] The integrand is not defined over the entire interval.
\choice The bounds are evaluated in the wrong order.
\choice Nothing is wrong.  The equation is correct, as is.
\end{multipleChoice}

\end{problem}}%}

%%%%%%%%%%%%%%%%%%%%%%


\latexProblemContent{
\begin{problem}

What is wrong with the following equation:

\[\int_{\frac{1}{4} \, \pi}^{\frac{7}{4} \, \pi} {10 \, \csc\left(x\right)^{2}}\;dx = {-\frac{10}{\tan\left(x\right)}}\Bigg\vert_{\frac{1}{4} \, \pi}^{\frac{7}{4} \, \pi} = {20}\]

\input{2311_Concept_Integral_0003.HELP.tex}

\begin{multipleChoice}
\choice The antiderivative is incorrect.
\choice[correct] The integrand is not defined over the entire interval.
\choice The bounds are evaluated in the wrong order.
\choice Nothing is wrong.  The equation is correct, as is.
\end{multipleChoice}

\end{problem}}%}

%%%%%%%%%%%%%%%%%%%%%%


\latexProblemContent{
\begin{problem}

What is wrong with the following equation:

\[\int_{\frac{1}{4} \, \pi}^{\frac{4}{3} \, \pi} {-11 \, \csc\left(x\right)^{2}}\;dx = {\frac{11}{\tan\left(x\right)}}\Bigg\vert_{\frac{1}{4} \, \pi}^{\frac{4}{3} \, \pi} = {\frac{11}{3} \, \sqrt{3} - 11}\]

\input{2311_Concept_Integral_0003.HELP.tex}

\begin{multipleChoice}
\choice The antiderivative is incorrect.
\choice[correct] The integrand is not defined over the entire interval.
\choice The bounds are evaluated in the wrong order.
\choice Nothing is wrong.  The equation is correct, as is.
\end{multipleChoice}

\end{problem}}%}

%%%%%%%%%%%%%%%%%%%%%%


\latexProblemContent{
\begin{problem}

What is wrong with the following equation:

\[\int_{\frac{1}{2} \, \pi}^{\frac{3}{2} \, \pi} {-\csc\left(x\right)^{2}}\;dx = {\frac{1}{\tan\left(x\right)}}\Bigg\vert_{\frac{1}{2} \, \pi}^{\frac{3}{2} \, \pi} = {0}\]

\input{2311_Concept_Integral_0003.HELP.tex}

\begin{multipleChoice}
\choice The antiderivative is incorrect.
\choice[correct] The integrand is not defined over the entire interval.
\choice The bounds are evaluated in the wrong order.
\choice Nothing is wrong.  The equation is correct, as is.
\end{multipleChoice}

\end{problem}}%}

%%%%%%%%%%%%%%%%%%%%%%


\latexProblemContent{
\begin{problem}

What is wrong with the following equation:

\[\int_{\frac{1}{4} \, \pi}^{\frac{5}{4} \, \pi} {6 \, \cot\left(x\right) \csc\left(x\right)}\;dx = {-\frac{6}{\sin\left(x\right)}}\Bigg\vert_{\frac{1}{4} \, \pi}^{\frac{5}{4} \, \pi} = {12 \, \sqrt{2}}\]

\input{2311_Concept_Integral_0003.HELP.tex}

\begin{multipleChoice}
\choice The antiderivative is incorrect.
\choice[correct] The integrand is not defined over the entire interval.
\choice The bounds are evaluated in the wrong order.
\choice Nothing is wrong.  The equation is correct, as is.
\end{multipleChoice}

\end{problem}}%}

%%%%%%%%%%%%%%%%%%%%%%


\latexProblemContent{
\begin{problem}

What is wrong with the following equation:

\[\int_{\frac{1}{4} \, \pi}^{\frac{5}{4} \, \pi} {-3 \, \csc\left(x\right)^{2}}\;dx = {\frac{3}{\tan\left(x\right)}}\Bigg\vert_{\frac{1}{4} \, \pi}^{\frac{5}{4} \, \pi} = {0}\]

\input{2311_Concept_Integral_0003.HELP.tex}

\begin{multipleChoice}
\choice The antiderivative is incorrect.
\choice[correct] The integrand is not defined over the entire interval.
\choice The bounds are evaluated in the wrong order.
\choice Nothing is wrong.  The equation is correct, as is.
\end{multipleChoice}

\end{problem}}%}

%%%%%%%%%%%%%%%%%%%%%%


%%%%%%%%%%%%%%%%%%%%%%


\latexProblemContent{
\begin{problem}

What is wrong with the following equation:

\[\int_{\frac{3}{4} \, \pi}^{\frac{3}{2} \, \pi} {11 \, \csc\left(x\right)^{2}}\;dx = {-\frac{11}{\tan\left(x\right)}}\Bigg\vert_{\frac{3}{4} \, \pi}^{\frac{3}{2} \, \pi} = {-11}\]

\input{2311_Concept_Integral_0003.HELP.tex}

\begin{multipleChoice}
\choice The antiderivative is incorrect.
\choice[correct] The integrand is not defined over the entire interval.
\choice The bounds are evaluated in the wrong order.
\choice Nothing is wrong.  The equation is correct, as is.
\end{multipleChoice}

\end{problem}}%}

%%%%%%%%%%%%%%%%%%%%%%


%%%%%%%%%%%%%%%%%%%%%%


\latexProblemContent{
\begin{problem}

What is wrong with the following equation:

\[\int_{\frac{1}{2} \, \pi}^{\frac{7}{4} \, \pi} {7 \, \csc\left(x\right)^{2}}\;dx = {-\frac{7}{\tan\left(x\right)}}\Bigg\vert_{\frac{1}{2} \, \pi}^{\frac{7}{4} \, \pi} = {7}\]

\input{2311_Concept_Integral_0003.HELP.tex}

\begin{multipleChoice}
\choice The antiderivative is incorrect.
\choice[correct] The integrand is not defined over the entire interval.
\choice The bounds are evaluated in the wrong order.
\choice Nothing is wrong.  The equation is correct, as is.
\end{multipleChoice}

\end{problem}}%}

%%%%%%%%%%%%%%%%%%%%%%


\latexProblemContent{
\begin{problem}

What is wrong with the following equation:

\[\int_{\frac{1}{2} \, \pi}^{\frac{3}{2} \, \pi} {7 \, \cot\left(x\right) \csc\left(x\right)}\;dx = {-\frac{7}{\sin\left(x\right)}}\Bigg\vert_{\frac{1}{2} \, \pi}^{\frac{3}{2} \, \pi} = {14}\]

\input{2311_Concept_Integral_0003.HELP.tex}

\begin{multipleChoice}
\choice The antiderivative is incorrect.
\choice[correct] The integrand is not defined over the entire interval.
\choice The bounds are evaluated in the wrong order.
\choice Nothing is wrong.  The equation is correct, as is.
\end{multipleChoice}

\end{problem}}%}

%%%%%%%%%%%%%%%%%%%%%%


\latexProblemContent{
\begin{problem}

What is wrong with the following equation:

\[\int_{\frac{2}{3} \, \pi}^{\frac{4}{3} \, \pi} {-12 \, \cot\left(x\right) \csc\left(x\right)}\;dx = {\frac{12}{\sin\left(x\right)}}\Bigg\vert_{\frac{2}{3} \, \pi}^{\frac{4}{3} \, \pi} = {-16 \, \sqrt{3}}\]

\input{2311_Concept_Integral_0003.HELP.tex}

\begin{multipleChoice}
\choice The antiderivative is incorrect.
\choice[correct] The integrand is not defined over the entire interval.
\choice The bounds are evaluated in the wrong order.
\choice Nothing is wrong.  The equation is correct, as is.
\end{multipleChoice}

\end{problem}}%}

%%%%%%%%%%%%%%%%%%%%%%


\latexProblemContent{
\begin{problem}

What is wrong with the following equation:

\[\int_{\frac{1}{4} \, \pi}^{\frac{7}{4} \, \pi} {14 \, \csc\left(x\right)^{2}}\;dx = {-\frac{14}{\tan\left(x\right)}}\Bigg\vert_{\frac{1}{4} \, \pi}^{\frac{7}{4} \, \pi} = {28}\]

\input{2311_Concept_Integral_0003.HELP.tex}

\begin{multipleChoice}
\choice The antiderivative is incorrect.
\choice[correct] The integrand is not defined over the entire interval.
\choice The bounds are evaluated in the wrong order.
\choice Nothing is wrong.  The equation is correct, as is.
\end{multipleChoice}

\end{problem}}%}

%%%%%%%%%%%%%%%%%%%%%%


\latexProblemContent{
\begin{problem}

What is wrong with the following equation:

\[\int_{\frac{3}{4} \, \pi}^{\frac{5}{3} \, \pi} {13 \, \csc\left(x\right)^{2}}\;dx = {-\frac{13}{\tan\left(x\right)}}\Bigg\vert_{\frac{3}{4} \, \pi}^{\frac{5}{3} \, \pi} = {\frac{13}{3} \, \sqrt{3} - 13}\]

\input{2311_Concept_Integral_0003.HELP.tex}

\begin{multipleChoice}
\choice The antiderivative is incorrect.
\choice[correct] The integrand is not defined over the entire interval.
\choice The bounds are evaluated in the wrong order.
\choice Nothing is wrong.  The equation is correct, as is.
\end{multipleChoice}

\end{problem}}%}

%%%%%%%%%%%%%%%%%%%%%%


\latexProblemContent{
\begin{problem}

What is wrong with the following equation:

\[\int_{\frac{1}{3} \, \pi}^{\frac{3}{2} \, \pi} {7 \, \cot\left(x\right) \csc\left(x\right)}\;dx = {-\frac{7}{\sin\left(x\right)}}\Bigg\vert_{\frac{1}{3} \, \pi}^{\frac{3}{2} \, \pi} = {\frac{14}{3} \, \sqrt{3} + 7}\]

\input{2311_Concept_Integral_0003.HELP.tex}

\begin{multipleChoice}
\choice The antiderivative is incorrect.
\choice[correct] The integrand is not defined over the entire interval.
\choice The bounds are evaluated in the wrong order.
\choice Nothing is wrong.  The equation is correct, as is.
\end{multipleChoice}

\end{problem}}%}

%%%%%%%%%%%%%%%%%%%%%%


%%%%%%%%%%%%%%%%%%%%%%


\latexProblemContent{
\begin{problem}

What is wrong with the following equation:

\[\int_{\frac{1}{3} \, \pi}^{\frac{5}{3} \, \pi} {8 \, \cot\left(x\right) \csc\left(x\right)}\;dx = {-\frac{8}{\sin\left(x\right)}}\Bigg\vert_{\frac{1}{3} \, \pi}^{\frac{5}{3} \, \pi} = {\frac{32}{3} \, \sqrt{3}}\]

\input{2311_Concept_Integral_0003.HELP.tex}

\begin{multipleChoice}
\choice The antiderivative is incorrect.
\choice[correct] The integrand is not defined over the entire interval.
\choice The bounds are evaluated in the wrong order.
\choice Nothing is wrong.  The equation is correct, as is.
\end{multipleChoice}

\end{problem}}%}

%%%%%%%%%%%%%%%%%%%%%%


\latexProblemContent{
\begin{problem}

What is wrong with the following equation:

\[\int_{\frac{1}{3} \, \pi}^{\frac{5}{4} \, \pi} {15 \, \csc\left(x\right)^{2}}\;dx = {-\frac{15}{\tan\left(x\right)}}\Bigg\vert_{\frac{1}{3} \, \pi}^{\frac{5}{4} \, \pi} = {5 \, \sqrt{3} - 15}\]

\input{2311_Concept_Integral_0003.HELP.tex}

\begin{multipleChoice}
\choice The antiderivative is incorrect.
\choice[correct] The integrand is not defined over the entire interval.
\choice The bounds are evaluated in the wrong order.
\choice Nothing is wrong.  The equation is correct, as is.
\end{multipleChoice}

\end{problem}}%}

%%%%%%%%%%%%%%%%%%%%%%


\latexProblemContent{
\begin{problem}

What is wrong with the following equation:

\[\int_{\frac{1}{2} \, \pi}^{\frac{4}{3} \, \pi} {7 \, \csc\left(x\right)^{2}}\;dx = {-\frac{7}{\tan\left(x\right)}}\Bigg\vert_{\frac{1}{2} \, \pi}^{\frac{4}{3} \, \pi} = {-\frac{7}{3} \, \sqrt{3}}\]

\input{2311_Concept_Integral_0003.HELP.tex}

\begin{multipleChoice}
\choice The antiderivative is incorrect.
\choice[correct] The integrand is not defined over the entire interval.
\choice The bounds are evaluated in the wrong order.
\choice Nothing is wrong.  The equation is correct, as is.
\end{multipleChoice}

\end{problem}}%}

%%%%%%%%%%%%%%%%%%%%%%


\latexProblemContent{
\begin{problem}

What is wrong with the following equation:

\[\int_{\frac{1}{4} \, \pi}^{\frac{5}{3} \, \pi} {14 \, \cot\left(x\right) \csc\left(x\right)}\;dx = {-\frac{14}{\sin\left(x\right)}}\Bigg\vert_{\frac{1}{4} \, \pi}^{\frac{5}{3} \, \pi} = {\frac{28}{3} \, \sqrt{3} + 14 \, \sqrt{2}}\]

\input{2311_Concept_Integral_0003.HELP.tex}

\begin{multipleChoice}
\choice The antiderivative is incorrect.
\choice[correct] The integrand is not defined over the entire interval.
\choice The bounds are evaluated in the wrong order.
\choice Nothing is wrong.  The equation is correct, as is.
\end{multipleChoice}

\end{problem}}%}

%%%%%%%%%%%%%%%%%%%%%%


\latexProblemContent{
\begin{problem}

What is wrong with the following equation:

\[\int_{\frac{3}{4} \, \pi}^{\frac{7}{4} \, \pi} {-6 \, \cot\left(x\right) \csc\left(x\right)}\;dx = {\frac{6}{\sin\left(x\right)}}\Bigg\vert_{\frac{3}{4} \, \pi}^{\frac{7}{4} \, \pi} = {-12 \, \sqrt{2}}\]

\input{2311_Concept_Integral_0003.HELP.tex}

\begin{multipleChoice}
\choice The antiderivative is incorrect.
\choice[correct] The integrand is not defined over the entire interval.
\choice The bounds are evaluated in the wrong order.
\choice Nothing is wrong.  The equation is correct, as is.
\end{multipleChoice}

\end{problem}}%}

%%%%%%%%%%%%%%%%%%%%%%


\latexProblemContent{
\begin{problem}

What is wrong with the following equation:

\[\int_{\frac{1}{2} \, \pi}^{\frac{7}{4} \, \pi} {15 \, \cot\left(x\right) \csc\left(x\right)}\;dx = {-\frac{15}{\sin\left(x\right)}}\Bigg\vert_{\frac{1}{2} \, \pi}^{\frac{7}{4} \, \pi} = {15 \, \sqrt{2} + 15}\]

\input{2311_Concept_Integral_0003.HELP.tex}

\begin{multipleChoice}
\choice The antiderivative is incorrect.
\choice[correct] The integrand is not defined over the entire interval.
\choice The bounds are evaluated in the wrong order.
\choice Nothing is wrong.  The equation is correct, as is.
\end{multipleChoice}

\end{problem}}%}

%%%%%%%%%%%%%%%%%%%%%%


\latexProblemContent{
\begin{problem}

What is wrong with the following equation:

\[\int_{\frac{1}{4} \, \pi}^{\frac{4}{3} \, \pi} {-4 \, \csc\left(x\right)^{2}}\;dx = {\frac{4}{\tan\left(x\right)}}\Bigg\vert_{\frac{1}{4} \, \pi}^{\frac{4}{3} \, \pi} = {\frac{4}{3} \, \sqrt{3} - 4}\]

\input{2311_Concept_Integral_0003.HELP.tex}

\begin{multipleChoice}
\choice The antiderivative is incorrect.
\choice[correct] The integrand is not defined over the entire interval.
\choice The bounds are evaluated in the wrong order.
\choice Nothing is wrong.  The equation is correct, as is.
\end{multipleChoice}

\end{problem}}%}

%%%%%%%%%%%%%%%%%%%%%%


\latexProblemContent{
\begin{problem}

What is wrong with the following equation:

\[\int_{\frac{3}{4} \, \pi}^{\frac{4}{3} \, \pi} {-10 \, \cot\left(x\right) \csc\left(x\right)}\;dx = {\frac{10}{\sin\left(x\right)}}\Bigg\vert_{\frac{3}{4} \, \pi}^{\frac{4}{3} \, \pi} = {-\frac{20}{3} \, \sqrt{3} - 10 \, \sqrt{2}}\]

\input{2311_Concept_Integral_0003.HELP.tex}

\begin{multipleChoice}
\choice The antiderivative is incorrect.
\choice[correct] The integrand is not defined over the entire interval.
\choice The bounds are evaluated in the wrong order.
\choice Nothing is wrong.  The equation is correct, as is.
\end{multipleChoice}

\end{problem}}%}

%%%%%%%%%%%%%%%%%%%%%%


\latexProblemContent{
\begin{problem}

What is wrong with the following equation:

\[\int_{\frac{2}{3} \, \pi}^{\frac{5}{3} \, \pi} {\csc\left(x\right)^{2}}\;dx = {-\frac{1}{\tan\left(x\right)}}\Bigg\vert_{\frac{2}{3} \, \pi}^{\frac{5}{3} \, \pi} = {0}\]

\input{2311_Concept_Integral_0003.HELP.tex}

\begin{multipleChoice}
\choice The antiderivative is incorrect.
\choice[correct] The integrand is not defined over the entire interval.
\choice The bounds are evaluated in the wrong order.
\choice Nothing is wrong.  The equation is correct, as is.
\end{multipleChoice}

\end{problem}}%}

%%%%%%%%%%%%%%%%%%%%%%


%%%%%%%%%%%%%%%%%%%%%%


\latexProblemContent{
\begin{problem}

What is wrong with the following equation:

\[\int_{\frac{3}{4} \, \pi}^{\frac{5}{3} \, \pi} {-10 \, \cot\left(x\right) \csc\left(x\right)}\;dx = {\frac{10}{\sin\left(x\right)}}\Bigg\vert_{\frac{3}{4} \, \pi}^{\frac{5}{3} \, \pi} = {-\frac{20}{3} \, \sqrt{3} - 10 \, \sqrt{2}}\]

\input{2311_Concept_Integral_0003.HELP.tex}

\begin{multipleChoice}
\choice The antiderivative is incorrect.
\choice[correct] The integrand is not defined over the entire interval.
\choice The bounds are evaluated in the wrong order.
\choice Nothing is wrong.  The equation is correct, as is.
\end{multipleChoice}

\end{problem}}%}

%%%%%%%%%%%%%%%%%%%%%%


\latexProblemContent{
\begin{problem}

What is wrong with the following equation:

\[\int_{\frac{1}{4} \, \pi}^{\frac{3}{2} \, \pi} {15 \, \csc\left(x\right)^{2}}\;dx = {-\frac{15}{\tan\left(x\right)}}\Bigg\vert_{\frac{1}{4} \, \pi}^{\frac{3}{2} \, \pi} = {15}\]

\input{2311_Concept_Integral_0003.HELP.tex}

\begin{multipleChoice}
\choice The antiderivative is incorrect.
\choice[correct] The integrand is not defined over the entire interval.
\choice The bounds are evaluated in the wrong order.
\choice Nothing is wrong.  The equation is correct, as is.
\end{multipleChoice}

\end{problem}}%}

%%%%%%%%%%%%%%%%%%%%%%


\latexProblemContent{
\begin{problem}

What is wrong with the following equation:

\[\int_{\frac{1}{3} \, \pi}^{\frac{3}{2} \, \pi} {13 \, \cot\left(x\right) \csc\left(x\right)}\;dx = {-\frac{13}{\sin\left(x\right)}}\Bigg\vert_{\frac{1}{3} \, \pi}^{\frac{3}{2} \, \pi} = {\frac{26}{3} \, \sqrt{3} + 13}\]

\input{2311_Concept_Integral_0003.HELP.tex}

\begin{multipleChoice}
\choice The antiderivative is incorrect.
\choice[correct] The integrand is not defined over the entire interval.
\choice The bounds are evaluated in the wrong order.
\choice Nothing is wrong.  The equation is correct, as is.
\end{multipleChoice}

\end{problem}}%}

%%%%%%%%%%%%%%%%%%%%%%


\latexProblemContent{
\begin{problem}

What is wrong with the following equation:

\[\int_{\frac{1}{2} \, \pi}^{\frac{3}{2} \, \pi} {5 \, \cot\left(x\right) \csc\left(x\right)}\;dx = {-\frac{5}{\sin\left(x\right)}}\Bigg\vert_{\frac{1}{2} \, \pi}^{\frac{3}{2} \, \pi} = {10}\]

\input{2311_Concept_Integral_0003.HELP.tex}

\begin{multipleChoice}
\choice The antiderivative is incorrect.
\choice[correct] The integrand is not defined over the entire interval.
\choice The bounds are evaluated in the wrong order.
\choice Nothing is wrong.  The equation is correct, as is.
\end{multipleChoice}

\end{problem}}%}

%%%%%%%%%%%%%%%%%%%%%%


%%%%%%%%%%%%%%%%%%%%%%


\latexProblemContent{
\begin{problem}

What is wrong with the following equation:

\[\int_{\frac{3}{4} \, \pi}^{\frac{4}{3} \, \pi} {15 \, \csc\left(x\right)^{2}}\;dx = {-\frac{15}{\tan\left(x\right)}}\Bigg\vert_{\frac{3}{4} \, \pi}^{\frac{4}{3} \, \pi} = {-5 \, \sqrt{3} - 15}\]

\input{2311_Concept_Integral_0003.HELP.tex}

\begin{multipleChoice}
\choice The antiderivative is incorrect.
\choice[correct] The integrand is not defined over the entire interval.
\choice The bounds are evaluated in the wrong order.
\choice Nothing is wrong.  The equation is correct, as is.
\end{multipleChoice}

\end{problem}}%}

%%%%%%%%%%%%%%%%%%%%%%


%%%%%%%%%%%%%%%%%%%%%%


\latexProblemContent{
\begin{problem}

What is wrong with the following equation:

\[\int_{\frac{1}{2} \, \pi}^{\frac{3}{2} \, \pi} {-12 \, \csc\left(x\right)^{2}}\;dx = {\frac{12}{\tan\left(x\right)}}\Bigg\vert_{\frac{1}{2} \, \pi}^{\frac{3}{2} \, \pi} = {0}\]

\input{2311_Concept_Integral_0003.HELP.tex}

\begin{multipleChoice}
\choice The antiderivative is incorrect.
\choice[correct] The integrand is not defined over the entire interval.
\choice The bounds are evaluated in the wrong order.
\choice Nothing is wrong.  The equation is correct, as is.
\end{multipleChoice}

\end{problem}}%}

%%%%%%%%%%%%%%%%%%%%%%


\latexProblemContent{
\begin{problem}

What is wrong with the following equation:

\[\int_{\frac{3}{4} \, \pi}^{\frac{7}{4} \, \pi} {-5 \, \csc\left(x\right)^{2}}\;dx = {\frac{5}{\tan\left(x\right)}}\Bigg\vert_{\frac{3}{4} \, \pi}^{\frac{7}{4} \, \pi} = {0}\]

\input{2311_Concept_Integral_0003.HELP.tex}

\begin{multipleChoice}
\choice The antiderivative is incorrect.
\choice[correct] The integrand is not defined over the entire interval.
\choice The bounds are evaluated in the wrong order.
\choice Nothing is wrong.  The equation is correct, as is.
\end{multipleChoice}

\end{problem}}%}

%%%%%%%%%%%%%%%%%%%%%%


\latexProblemContent{
\begin{problem}

What is wrong with the following equation:

\[\int_{\frac{1}{2} \, \pi}^{\frac{5}{3} \, \pi} {6 \, \cot\left(x\right) \csc\left(x\right)}\;dx = {-\frac{6}{\sin\left(x\right)}}\Bigg\vert_{\frac{1}{2} \, \pi}^{\frac{5}{3} \, \pi} = {4 \, \sqrt{3} + 6}\]

\input{2311_Concept_Integral_0003.HELP.tex}

\begin{multipleChoice}
\choice The antiderivative is incorrect.
\choice[correct] The integrand is not defined over the entire interval.
\choice The bounds are evaluated in the wrong order.
\choice Nothing is wrong.  The equation is correct, as is.
\end{multipleChoice}

\end{problem}}%}

%%%%%%%%%%%%%%%%%%%%%%


\latexProblemContent{
\begin{problem}

What is wrong with the following equation:

\[\int_{\frac{1}{3} \, \pi}^{\frac{5}{4} \, \pi} {7 \, \csc\left(x\right)^{2}}\;dx = {-\frac{7}{\tan\left(x\right)}}\Bigg\vert_{\frac{1}{3} \, \pi}^{\frac{5}{4} \, \pi} = {\frac{7}{3} \, \sqrt{3} - 7}\]

\input{2311_Concept_Integral_0003.HELP.tex}

\begin{multipleChoice}
\choice The antiderivative is incorrect.
\choice[correct] The integrand is not defined over the entire interval.
\choice The bounds are evaluated in the wrong order.
\choice Nothing is wrong.  The equation is correct, as is.
\end{multipleChoice}

\end{problem}}%}

%%%%%%%%%%%%%%%%%%%%%%


%%%%%%%%%%%%%%%%%%%%%%


\latexProblemContent{
\begin{problem}

What is wrong with the following equation:

\[\int_{\frac{1}{4} \, \pi}^{\frac{3}{2} \, \pi} {-\csc\left(x\right)^{2}}\;dx = {\frac{1}{\tan\left(x\right)}}\Bigg\vert_{\frac{1}{4} \, \pi}^{\frac{3}{2} \, \pi} = {-1}\]

\input{2311_Concept_Integral_0003.HELP.tex}

\begin{multipleChoice}
\choice The antiderivative is incorrect.
\choice[correct] The integrand is not defined over the entire interval.
\choice The bounds are evaluated in the wrong order.
\choice Nothing is wrong.  The equation is correct, as is.
\end{multipleChoice}

\end{problem}}%}

%%%%%%%%%%%%%%%%%%%%%%


%%%%%%%%%%%%%%%%%%%%%%


\latexProblemContent{
\begin{problem}

What is wrong with the following equation:

\[\int_{\frac{1}{2} \, \pi}^{\frac{5}{4} \, \pi} {8 \, \csc\left(x\right)^{2}}\;dx = {-\frac{8}{\tan\left(x\right)}}\Bigg\vert_{\frac{1}{2} \, \pi}^{\frac{5}{4} \, \pi} = {-8}\]

\input{2311_Concept_Integral_0003.HELP.tex}

\begin{multipleChoice}
\choice The antiderivative is incorrect.
\choice[correct] The integrand is not defined over the entire interval.
\choice The bounds are evaluated in the wrong order.
\choice Nothing is wrong.  The equation is correct, as is.
\end{multipleChoice}

\end{problem}}%}

%%%%%%%%%%%%%%%%%%%%%%


%%%%%%%%%%%%%%%%%%%%%%


\latexProblemContent{
\begin{problem}

What is wrong with the following equation:

\[\int_{\frac{3}{4} \, \pi}^{\frac{3}{2} \, \pi} {15 \, \csc\left(x\right)^{2}}\;dx = {-\frac{15}{\tan\left(x\right)}}\Bigg\vert_{\frac{3}{4} \, \pi}^{\frac{3}{2} \, \pi} = {-15}\]

\input{2311_Concept_Integral_0003.HELP.tex}

\begin{multipleChoice}
\choice The antiderivative is incorrect.
\choice[correct] The integrand is not defined over the entire interval.
\choice The bounds are evaluated in the wrong order.
\choice Nothing is wrong.  The equation is correct, as is.
\end{multipleChoice}

\end{problem}}%}

%%%%%%%%%%%%%%%%%%%%%%


\latexProblemContent{
\begin{problem}

What is wrong with the following equation:

\[\int_{\frac{2}{3} \, \pi}^{\frac{3}{2} \, \pi} {9 \, \csc\left(x\right)^{2}}\;dx = {-\frac{9}{\tan\left(x\right)}}\Bigg\vert_{\frac{2}{3} \, \pi}^{\frac{3}{2} \, \pi} = {-3 \, \sqrt{3}}\]

\input{2311_Concept_Integral_0003.HELP.tex}

\begin{multipleChoice}
\choice The antiderivative is incorrect.
\choice[correct] The integrand is not defined over the entire interval.
\choice The bounds are evaluated in the wrong order.
\choice Nothing is wrong.  The equation is correct, as is.
\end{multipleChoice}

\end{problem}}%}

%%%%%%%%%%%%%%%%%%%%%%


\latexProblemContent{
\begin{problem}

What is wrong with the following equation:

\[\int_{\frac{1}{2} \, \pi}^{\frac{7}{4} \, \pi} {-13 \, \csc\left(x\right)^{2}}\;dx = {\frac{13}{\tan\left(x\right)}}\Bigg\vert_{\frac{1}{2} \, \pi}^{\frac{7}{4} \, \pi} = {-13}\]

\input{2311_Concept_Integral_0003.HELP.tex}

\begin{multipleChoice}
\choice The antiderivative is incorrect.
\choice[correct] The integrand is not defined over the entire interval.
\choice The bounds are evaluated in the wrong order.
\choice Nothing is wrong.  The equation is correct, as is.
\end{multipleChoice}

\end{problem}}%}

%%%%%%%%%%%%%%%%%%%%%%


\latexProblemContent{
\begin{problem}

What is wrong with the following equation:

\[\int_{\frac{1}{4} \, \pi}^{\frac{5}{4} \, \pi} {13 \, \csc\left(x\right)^{2}}\;dx = {-\frac{13}{\tan\left(x\right)}}\Bigg\vert_{\frac{1}{4} \, \pi}^{\frac{5}{4} \, \pi} = {0}\]

\input{2311_Concept_Integral_0003.HELP.tex}

\begin{multipleChoice}
\choice The antiderivative is incorrect.
\choice[correct] The integrand is not defined over the entire interval.
\choice The bounds are evaluated in the wrong order.
\choice Nothing is wrong.  The equation is correct, as is.
\end{multipleChoice}

\end{problem}}%}

%%%%%%%%%%%%%%%%%%%%%%


\latexProblemContent{
\begin{problem}

What is wrong with the following equation:

\[\int_{\frac{3}{4} \, \pi}^{\frac{5}{3} \, \pi} {-9 \, \csc\left(x\right)^{2}}\;dx = {\frac{9}{\tan\left(x\right)}}\Bigg\vert_{\frac{3}{4} \, \pi}^{\frac{5}{3} \, \pi} = {-3 \, \sqrt{3} + 9}\]

\input{2311_Concept_Integral_0003.HELP.tex}

\begin{multipleChoice}
\choice The antiderivative is incorrect.
\choice[correct] The integrand is not defined over the entire interval.
\choice The bounds are evaluated in the wrong order.
\choice Nothing is wrong.  The equation is correct, as is.
\end{multipleChoice}

\end{problem}}%}

%%%%%%%%%%%%%%%%%%%%%%


\latexProblemContent{
\begin{problem}

What is wrong with the following equation:

\[\int_{\frac{1}{2} \, \pi}^{\frac{4}{3} \, \pi} {-13 \, \cot\left(x\right) \csc\left(x\right)}\;dx = {\frac{13}{\sin\left(x\right)}}\Bigg\vert_{\frac{1}{2} \, \pi}^{\frac{4}{3} \, \pi} = {-\frac{26}{3} \, \sqrt{3} - 13}\]

\input{2311_Concept_Integral_0003.HELP.tex}

\begin{multipleChoice}
\choice The antiderivative is incorrect.
\choice[correct] The integrand is not defined over the entire interval.
\choice The bounds are evaluated in the wrong order.
\choice Nothing is wrong.  The equation is correct, as is.
\end{multipleChoice}

\end{problem}}%}

%%%%%%%%%%%%%%%%%%%%%%


\latexProblemContent{
\begin{problem}

What is wrong with the following equation:

\[\int_{\frac{1}{3} \, \pi}^{\frac{5}{4} \, \pi} {-9 \, \csc\left(x\right)^{2}}\;dx = {\frac{9}{\tan\left(x\right)}}\Bigg\vert_{\frac{1}{3} \, \pi}^{\frac{5}{4} \, \pi} = {-3 \, \sqrt{3} + 9}\]

\input{2311_Concept_Integral_0003.HELP.tex}

\begin{multipleChoice}
\choice The antiderivative is incorrect.
\choice[correct] The integrand is not defined over the entire interval.
\choice The bounds are evaluated in the wrong order.
\choice Nothing is wrong.  The equation is correct, as is.
\end{multipleChoice}

\end{problem}}%}

%%%%%%%%%%%%%%%%%%%%%%


\latexProblemContent{
\begin{problem}

What is wrong with the following equation:

\[\int_{\frac{1}{2} \, \pi}^{\frac{5}{3} \, \pi} {6 \, \csc\left(x\right)^{2}}\;dx = {-\frac{6}{\tan\left(x\right)}}\Bigg\vert_{\frac{1}{2} \, \pi}^{\frac{5}{3} \, \pi} = {2 \, \sqrt{3}}\]

\input{2311_Concept_Integral_0003.HELP.tex}

\begin{multipleChoice}
\choice The antiderivative is incorrect.
\choice[correct] The integrand is not defined over the entire interval.
\choice The bounds are evaluated in the wrong order.
\choice Nothing is wrong.  The equation is correct, as is.
\end{multipleChoice}

\end{problem}}%}

%%%%%%%%%%%%%%%%%%%%%%


\latexProblemContent{
\begin{problem}

What is wrong with the following equation:

\[\int_{\frac{1}{4} \, \pi}^{\frac{3}{2} \, \pi} {5 \, \cot\left(x\right) \csc\left(x\right)}\;dx = {-\frac{5}{\sin\left(x\right)}}\Bigg\vert_{\frac{1}{4} \, \pi}^{\frac{3}{2} \, \pi} = {5 \, \sqrt{2} + 5}\]

\input{2311_Concept_Integral_0003.HELP.tex}

\begin{multipleChoice}
\choice The antiderivative is incorrect.
\choice[correct] The integrand is not defined over the entire interval.
\choice The bounds are evaluated in the wrong order.
\choice Nothing is wrong.  The equation is correct, as is.
\end{multipleChoice}

\end{problem}}%}

%%%%%%%%%%%%%%%%%%%%%%


%%%%%%%%%%%%%%%%%%%%%%


\latexProblemContent{
\begin{problem}

What is wrong with the following equation:

\[\int_{\frac{3}{4} \, \pi}^{\frac{5}{4} \, \pi} {-7 \, \csc\left(x\right)^{2}}\;dx = {\frac{7}{\tan\left(x\right)}}\Bigg\vert_{\frac{3}{4} \, \pi}^{\frac{5}{4} \, \pi} = {14}\]

\input{2311_Concept_Integral_0003.HELP.tex}

\begin{multipleChoice}
\choice The antiderivative is incorrect.
\choice[correct] The integrand is not defined over the entire interval.
\choice The bounds are evaluated in the wrong order.
\choice Nothing is wrong.  The equation is correct, as is.
\end{multipleChoice}

\end{problem}}%}

%%%%%%%%%%%%%%%%%%%%%%


\latexProblemContent{
\begin{problem}

What is wrong with the following equation:

\[\int_{\frac{1}{2} \, \pi}^{\frac{5}{3} \, \pi} {3 \, \cot\left(x\right) \csc\left(x\right)}\;dx = {-\frac{3}{\sin\left(x\right)}}\Bigg\vert_{\frac{1}{2} \, \pi}^{\frac{5}{3} \, \pi} = {2 \, \sqrt{3} + 3}\]

\input{2311_Concept_Integral_0003.HELP.tex}

\begin{multipleChoice}
\choice The antiderivative is incorrect.
\choice[correct] The integrand is not defined over the entire interval.
\choice The bounds are evaluated in the wrong order.
\choice Nothing is wrong.  The equation is correct, as is.
\end{multipleChoice}

\end{problem}}%}

%%%%%%%%%%%%%%%%%%%%%%


%%%%%%%%%%%%%%%%%%%%%%


\latexProblemContent{
\begin{problem}

What is wrong with the following equation:

\[\int_{\frac{1}{3} \, \pi}^{\frac{3}{2} \, \pi} {-\cot\left(x\right) \csc\left(x\right)}\;dx = {\frac{1}{\sin\left(x\right)}}\Bigg\vert_{\frac{1}{3} \, \pi}^{\frac{3}{2} \, \pi} = {-\frac{2}{3} \, \sqrt{3} - 1}\]

\input{2311_Concept_Integral_0003.HELP.tex}

\begin{multipleChoice}
\choice The antiderivative is incorrect.
\choice[correct] The integrand is not defined over the entire interval.
\choice The bounds are evaluated in the wrong order.
\choice Nothing is wrong.  The equation is correct, as is.
\end{multipleChoice}

\end{problem}}%}

%%%%%%%%%%%%%%%%%%%%%%


\latexProblemContent{
\begin{problem}

What is wrong with the following equation:

\[\int_{\frac{1}{3} \, \pi}^{\frac{7}{4} \, \pi} {-13 \, \cot\left(x\right) \csc\left(x\right)}\;dx = {\frac{13}{\sin\left(x\right)}}\Bigg\vert_{\frac{1}{3} \, \pi}^{\frac{7}{4} \, \pi} = {-\frac{26}{3} \, \sqrt{3} - 13 \, \sqrt{2}}\]

\input{2311_Concept_Integral_0003.HELP.tex}

\begin{multipleChoice}
\choice The antiderivative is incorrect.
\choice[correct] The integrand is not defined over the entire interval.
\choice The bounds are evaluated in the wrong order.
\choice Nothing is wrong.  The equation is correct, as is.
\end{multipleChoice}

\end{problem}}%}

%%%%%%%%%%%%%%%%%%%%%%


\latexProblemContent{
\begin{problem}

What is wrong with the following equation:

\[\int_{\frac{2}{3} \, \pi}^{\frac{7}{4} \, \pi} {4 \, \cot\left(x\right) \csc\left(x\right)}\;dx = {-\frac{4}{\sin\left(x\right)}}\Bigg\vert_{\frac{2}{3} \, \pi}^{\frac{7}{4} \, \pi} = {\frac{8}{3} \, \sqrt{3} + 4 \, \sqrt{2}}\]

\input{2311_Concept_Integral_0003.HELP.tex}

\begin{multipleChoice}
\choice The antiderivative is incorrect.
\choice[correct] The integrand is not defined over the entire interval.
\choice The bounds are evaluated in the wrong order.
\choice Nothing is wrong.  The equation is correct, as is.
\end{multipleChoice}

\end{problem}}%}

%%%%%%%%%%%%%%%%%%%%%%


\latexProblemContent{
\begin{problem}

What is wrong with the following equation:

\[\int_{\frac{1}{2} \, \pi}^{\frac{7}{4} \, \pi} {-2 \, \csc\left(x\right)^{2}}\;dx = {\frac{2}{\tan\left(x\right)}}\Bigg\vert_{\frac{1}{2} \, \pi}^{\frac{7}{4} \, \pi} = {-2}\]

\input{2311_Concept_Integral_0003.HELP.tex}

\begin{multipleChoice}
\choice The antiderivative is incorrect.
\choice[correct] The integrand is not defined over the entire interval.
\choice The bounds are evaluated in the wrong order.
\choice Nothing is wrong.  The equation is correct, as is.
\end{multipleChoice}

\end{problem}}%}

%%%%%%%%%%%%%%%%%%%%%%


\latexProblemContent{
\begin{problem}

What is wrong with the following equation:

\[\int_{\frac{1}{4} \, \pi}^{\frac{4}{3} \, \pi} {-2 \, \cot\left(x\right) \csc\left(x\right)}\;dx = {\frac{2}{\sin\left(x\right)}}\Bigg\vert_{\frac{1}{4} \, \pi}^{\frac{4}{3} \, \pi} = {-\frac{4}{3} \, \sqrt{3} - 2 \, \sqrt{2}}\]

\input{2311_Concept_Integral_0003.HELP.tex}

\begin{multipleChoice}
\choice The antiderivative is incorrect.
\choice[correct] The integrand is not defined over the entire interval.
\choice The bounds are evaluated in the wrong order.
\choice Nothing is wrong.  The equation is correct, as is.
\end{multipleChoice}

\end{problem}}%}

%%%%%%%%%%%%%%%%%%%%%%


\latexProblemContent{
\begin{problem}

What is wrong with the following equation:

\[\int_{\frac{1}{4} \, \pi}^{\frac{3}{2} \, \pi} {2 \, \csc\left(x\right)^{2}}\;dx = {-\frac{2}{\tan\left(x\right)}}\Bigg\vert_{\frac{1}{4} \, \pi}^{\frac{3}{2} \, \pi} = {2}\]

\input{2311_Concept_Integral_0003.HELP.tex}

\begin{multipleChoice}
\choice The antiderivative is incorrect.
\choice[correct] The integrand is not defined over the entire interval.
\choice The bounds are evaluated in the wrong order.
\choice Nothing is wrong.  The equation is correct, as is.
\end{multipleChoice}

\end{problem}}%}

%%%%%%%%%%%%%%%%%%%%%%


\latexProblemContent{
\begin{problem}

What is wrong with the following equation:

\[\int_{\frac{1}{4} \, \pi}^{\frac{4}{3} \, \pi} {8 \, \cot\left(x\right) \csc\left(x\right)}\;dx = {-\frac{8}{\sin\left(x\right)}}\Bigg\vert_{\frac{1}{4} \, \pi}^{\frac{4}{3} \, \pi} = {\frac{16}{3} \, \sqrt{3} + 8 \, \sqrt{2}}\]

\input{2311_Concept_Integral_0003.HELP.tex}

\begin{multipleChoice}
\choice The antiderivative is incorrect.
\choice[correct] The integrand is not defined over the entire interval.
\choice The bounds are evaluated in the wrong order.
\choice Nothing is wrong.  The equation is correct, as is.
\end{multipleChoice}

\end{problem}}%}

%%%%%%%%%%%%%%%%%%%%%%


\latexProblemContent{
\begin{problem}

What is wrong with the following equation:

\[\int_{\frac{1}{2} \, \pi}^{\frac{7}{4} \, \pi} {-7 \, \csc\left(x\right)^{2}}\;dx = {\frac{7}{\tan\left(x\right)}}\Bigg\vert_{\frac{1}{2} \, \pi}^{\frac{7}{4} \, \pi} = {-7}\]

\input{2311_Concept_Integral_0003.HELP.tex}

\begin{multipleChoice}
\choice The antiderivative is incorrect.
\choice[correct] The integrand is not defined over the entire interval.
\choice The bounds are evaluated in the wrong order.
\choice Nothing is wrong.  The equation is correct, as is.
\end{multipleChoice}

\end{problem}}%}

%%%%%%%%%%%%%%%%%%%%%%


%%%%%%%%%%%%%%%%%%%%%%


\latexProblemContent{
\begin{problem}

What is wrong with the following equation:

\[\int_{\frac{1}{4} \, \pi}^{\frac{3}{2} \, \pi} {4 \, \cot\left(x\right) \csc\left(x\right)}\;dx = {-\frac{4}{\sin\left(x\right)}}\Bigg\vert_{\frac{1}{4} \, \pi}^{\frac{3}{2} \, \pi} = {4 \, \sqrt{2} + 4}\]

\input{2311_Concept_Integral_0003.HELP.tex}

\begin{multipleChoice}
\choice The antiderivative is incorrect.
\choice[correct] The integrand is not defined over the entire interval.
\choice The bounds are evaluated in the wrong order.
\choice Nothing is wrong.  The equation is correct, as is.
\end{multipleChoice}

\end{problem}}%}

%%%%%%%%%%%%%%%%%%%%%%


\latexProblemContent{
\begin{problem}

What is wrong with the following equation:

\[\int_{\frac{1}{3} \, \pi}^{\frac{4}{3} \, \pi} {-6 \, \cot\left(x\right) \csc\left(x\right)}\;dx = {\frac{6}{\sin\left(x\right)}}\Bigg\vert_{\frac{1}{3} \, \pi}^{\frac{4}{3} \, \pi} = {-8 \, \sqrt{3}}\]

\input{2311_Concept_Integral_0003.HELP.tex}

\begin{multipleChoice}
\choice The antiderivative is incorrect.
\choice[correct] The integrand is not defined over the entire interval.
\choice The bounds are evaluated in the wrong order.
\choice Nothing is wrong.  The equation is correct, as is.
\end{multipleChoice}

\end{problem}}%}

%%%%%%%%%%%%%%%%%%%%%%


\latexProblemContent{
\begin{problem}

What is wrong with the following equation:

\[\int_{\frac{1}{3} \, \pi}^{\frac{5}{4} \, \pi} {-10 \, \csc\left(x\right)^{2}}\;dx = {\frac{10}{\tan\left(x\right)}}\Bigg\vert_{\frac{1}{3} \, \pi}^{\frac{5}{4} \, \pi} = {-\frac{10}{3} \, \sqrt{3} + 10}\]

\input{2311_Concept_Integral_0003.HELP.tex}

\begin{multipleChoice}
\choice The antiderivative is incorrect.
\choice[correct] The integrand is not defined over the entire interval.
\choice The bounds are evaluated in the wrong order.
\choice Nothing is wrong.  The equation is correct, as is.
\end{multipleChoice}

\end{problem}}%}

%%%%%%%%%%%%%%%%%%%%%%


%%%%%%%%%%%%%%%%%%%%%%


\latexProblemContent{
\begin{problem}

What is wrong with the following equation:

\[\int_{\frac{2}{3} \, \pi}^{\frac{4}{3} \, \pi} {-3 \, \csc\left(x\right)^{2}}\;dx = {\frac{3}{\tan\left(x\right)}}\Bigg\vert_{\frac{2}{3} \, \pi}^{\frac{4}{3} \, \pi} = {2 \, \sqrt{3}}\]

\input{2311_Concept_Integral_0003.HELP.tex}

\begin{multipleChoice}
\choice The antiderivative is incorrect.
\choice[correct] The integrand is not defined over the entire interval.
\choice The bounds are evaluated in the wrong order.
\choice Nothing is wrong.  The equation is correct, as is.
\end{multipleChoice}

\end{problem}}%}

%%%%%%%%%%%%%%%%%%%%%%


\latexProblemContent{
\begin{problem}

What is wrong with the following equation:

\[\int_{\frac{2}{3} \, \pi}^{\frac{5}{4} \, \pi} {12 \, \cot\left(x\right) \csc\left(x\right)}\;dx = {-\frac{12}{\sin\left(x\right)}}\Bigg\vert_{\frac{2}{3} \, \pi}^{\frac{5}{4} \, \pi} = {8 \, \sqrt{3} + 12 \, \sqrt{2}}\]

\input{2311_Concept_Integral_0003.HELP.tex}

\begin{multipleChoice}
\choice The antiderivative is incorrect.
\choice[correct] The integrand is not defined over the entire interval.
\choice The bounds are evaluated in the wrong order.
\choice Nothing is wrong.  The equation is correct, as is.
\end{multipleChoice}

\end{problem}}%}

%%%%%%%%%%%%%%%%%%%%%%


%%%%%%%%%%%%%%%%%%%%%%


\latexProblemContent{
\begin{problem}

What is wrong with the following equation:

\[\int_{\frac{1}{3} \, \pi}^{\frac{4}{3} \, \pi} {-9 \, \csc\left(x\right)^{2}}\;dx = {\frac{9}{\tan\left(x\right)}}\Bigg\vert_{\frac{1}{3} \, \pi}^{\frac{4}{3} \, \pi} = {0}\]

\input{2311_Concept_Integral_0003.HELP.tex}

\begin{multipleChoice}
\choice The antiderivative is incorrect.
\choice[correct] The integrand is not defined over the entire interval.
\choice The bounds are evaluated in the wrong order.
\choice Nothing is wrong.  The equation is correct, as is.
\end{multipleChoice}

\end{problem}}%}

%%%%%%%%%%%%%%%%%%%%%%


\latexProblemContent{
\begin{problem}

What is wrong with the following equation:

\[\int_{\frac{3}{4} \, \pi}^{\frac{7}{4} \, \pi} {-10 \, \cot\left(x\right) \csc\left(x\right)}\;dx = {\frac{10}{\sin\left(x\right)}}\Bigg\vert_{\frac{3}{4} \, \pi}^{\frac{7}{4} \, \pi} = {-20 \, \sqrt{2}}\]

\input{2311_Concept_Integral_0003.HELP.tex}

\begin{multipleChoice}
\choice The antiderivative is incorrect.
\choice[correct] The integrand is not defined over the entire interval.
\choice The bounds are evaluated in the wrong order.
\choice Nothing is wrong.  The equation is correct, as is.
\end{multipleChoice}

\end{problem}}%}

%%%%%%%%%%%%%%%%%%%%%%


\latexProblemContent{
\begin{problem}

What is wrong with the following equation:

\[\int_{\frac{2}{3} \, \pi}^{\frac{5}{4} \, \pi} {5 \, \csc\left(x\right)^{2}}\;dx = {-\frac{5}{\tan\left(x\right)}}\Bigg\vert_{\frac{2}{3} \, \pi}^{\frac{5}{4} \, \pi} = {-\frac{5}{3} \, \sqrt{3} - 5}\]

\input{2311_Concept_Integral_0003.HELP.tex}

\begin{multipleChoice}
\choice The antiderivative is incorrect.
\choice[correct] The integrand is not defined over the entire interval.
\choice The bounds are evaluated in the wrong order.
\choice Nothing is wrong.  The equation is correct, as is.
\end{multipleChoice}

\end{problem}}%}

%%%%%%%%%%%%%%%%%%%%%%


%%%%%%%%%%%%%%%%%%%%%%


\latexProblemContent{
\begin{problem}

What is wrong with the following equation:

\[\int_{\frac{1}{2} \, \pi}^{\frac{4}{3} \, \pi} {-4 \, \csc\left(x\right)^{2}}\;dx = {\frac{4}{\tan\left(x\right)}}\Bigg\vert_{\frac{1}{2} \, \pi}^{\frac{4}{3} \, \pi} = {\frac{4}{3} \, \sqrt{3}}\]

\input{2311_Concept_Integral_0003.HELP.tex}

\begin{multipleChoice}
\choice The antiderivative is incorrect.
\choice[correct] The integrand is not defined over the entire interval.
\choice The bounds are evaluated in the wrong order.
\choice Nothing is wrong.  The equation is correct, as is.
\end{multipleChoice}

\end{problem}}%}

%%%%%%%%%%%%%%%%%%%%%%


\latexProblemContent{
\begin{problem}

What is wrong with the following equation:

\[\int_{\frac{2}{3} \, \pi}^{\frac{4}{3} \, \pi} {\cot\left(x\right) \csc\left(x\right)}\;dx = {-\frac{1}{\sin\left(x\right)}}\Bigg\vert_{\frac{2}{3} \, \pi}^{\frac{4}{3} \, \pi} = {\frac{4}{3} \, \sqrt{3}}\]

\input{2311_Concept_Integral_0003.HELP.tex}

\begin{multipleChoice}
\choice The antiderivative is incorrect.
\choice[correct] The integrand is not defined over the entire interval.
\choice The bounds are evaluated in the wrong order.
\choice Nothing is wrong.  The equation is correct, as is.
\end{multipleChoice}

\end{problem}}%}

%%%%%%%%%%%%%%%%%%%%%%


\latexProblemContent{
\begin{problem}

What is wrong with the following equation:

\[\int_{\frac{1}{4} \, \pi}^{\frac{5}{3} \, \pi} {-14 \, \cot\left(x\right) \csc\left(x\right)}\;dx = {\frac{14}{\sin\left(x\right)}}\Bigg\vert_{\frac{1}{4} \, \pi}^{\frac{5}{3} \, \pi} = {-\frac{28}{3} \, \sqrt{3} - 14 \, \sqrt{2}}\]

\input{2311_Concept_Integral_0003.HELP.tex}

\begin{multipleChoice}
\choice The antiderivative is incorrect.
\choice[correct] The integrand is not defined over the entire interval.
\choice The bounds are evaluated in the wrong order.
\choice Nothing is wrong.  The equation is correct, as is.
\end{multipleChoice}

\end{problem}}%}

%%%%%%%%%%%%%%%%%%%%%%


\latexProblemContent{
\begin{problem}

What is wrong with the following equation:

\[\int_{\frac{1}{4} \, \pi}^{\frac{4}{3} \, \pi} {13 \, \cot\left(x\right) \csc\left(x\right)}\;dx = {-\frac{13}{\sin\left(x\right)}}\Bigg\vert_{\frac{1}{4} \, \pi}^{\frac{4}{3} \, \pi} = {\frac{26}{3} \, \sqrt{3} + 13 \, \sqrt{2}}\]

\input{2311_Concept_Integral_0003.HELP.tex}

\begin{multipleChoice}
\choice The antiderivative is incorrect.
\choice[correct] The integrand is not defined over the entire interval.
\choice The bounds are evaluated in the wrong order.
\choice Nothing is wrong.  The equation is correct, as is.
\end{multipleChoice}

\end{problem}}%}

%%%%%%%%%%%%%%%%%%%%%%


%%%%%%%%%%%%%%%%%%%%%%


\latexProblemContent{
\begin{problem}

What is wrong with the following equation:

\[\int_{\frac{1}{2} \, \pi}^{\frac{5}{3} \, \pi} {12 \, \cot\left(x\right) \csc\left(x\right)}\;dx = {-\frac{12}{\sin\left(x\right)}}\Bigg\vert_{\frac{1}{2} \, \pi}^{\frac{5}{3} \, \pi} = {8 \, \sqrt{3} + 12}\]

\input{2311_Concept_Integral_0003.HELP.tex}

\begin{multipleChoice}
\choice The antiderivative is incorrect.
\choice[correct] The integrand is not defined over the entire interval.
\choice The bounds are evaluated in the wrong order.
\choice Nothing is wrong.  The equation is correct, as is.
\end{multipleChoice}

\end{problem}}%}

%%%%%%%%%%%%%%%%%%%%%%


\latexProblemContent{
\begin{problem}

What is wrong with the following equation:

\[\int_{\frac{2}{3} \, \pi}^{\frac{4}{3} \, \pi} {-15 \, \cot\left(x\right) \csc\left(x\right)}\;dx = {\frac{15}{\sin\left(x\right)}}\Bigg\vert_{\frac{2}{3} \, \pi}^{\frac{4}{3} \, \pi} = {-20 \, \sqrt{3}}\]

\input{2311_Concept_Integral_0003.HELP.tex}

\begin{multipleChoice}
\choice The antiderivative is incorrect.
\choice[correct] The integrand is not defined over the entire interval.
\choice The bounds are evaluated in the wrong order.
\choice Nothing is wrong.  The equation is correct, as is.
\end{multipleChoice}

\end{problem}}%}

%%%%%%%%%%%%%%%%%%%%%%


\latexProblemContent{
\begin{problem}

What is wrong with the following equation:

\[\int_{\frac{3}{4} \, \pi}^{\frac{7}{4} \, \pi} {-\csc\left(x\right)^{2}}\;dx = {\frac{1}{\tan\left(x\right)}}\Bigg\vert_{\frac{3}{4} \, \pi}^{\frac{7}{4} \, \pi} = {0}\]

\input{2311_Concept_Integral_0003.HELP.tex}

\begin{multipleChoice}
\choice The antiderivative is incorrect.
\choice[correct] The integrand is not defined over the entire interval.
\choice The bounds are evaluated in the wrong order.
\choice Nothing is wrong.  The equation is correct, as is.
\end{multipleChoice}

\end{problem}}%}

%%%%%%%%%%%%%%%%%%%%%%


\latexProblemContent{
\begin{problem}

What is wrong with the following equation:

\[\int_{\frac{1}{2} \, \pi}^{\frac{4}{3} \, \pi} {-6 \, \csc\left(x\right)^{2}}\;dx = {\frac{6}{\tan\left(x\right)}}\Bigg\vert_{\frac{1}{2} \, \pi}^{\frac{4}{3} \, \pi} = {2 \, \sqrt{3}}\]

\input{2311_Concept_Integral_0003.HELP.tex}

\begin{multipleChoice}
\choice The antiderivative is incorrect.
\choice[correct] The integrand is not defined over the entire interval.
\choice The bounds are evaluated in the wrong order.
\choice Nothing is wrong.  The equation is correct, as is.
\end{multipleChoice}

\end{problem}}%}

%%%%%%%%%%%%%%%%%%%%%%


%%%%%%%%%%%%%%%%%%%%%%


\latexProblemContent{
\begin{problem}

What is wrong with the following equation:

\[\int_{\frac{1}{3} \, \pi}^{\frac{5}{4} \, \pi} {-14 \, \cot\left(x\right) \csc\left(x\right)}\;dx = {\frac{14}{\sin\left(x\right)}}\Bigg\vert_{\frac{1}{3} \, \pi}^{\frac{5}{4} \, \pi} = {-\frac{28}{3} \, \sqrt{3} - 14 \, \sqrt{2}}\]

\input{2311_Concept_Integral_0003.HELP.tex}

\begin{multipleChoice}
\choice The antiderivative is incorrect.
\choice[correct] The integrand is not defined over the entire interval.
\choice The bounds are evaluated in the wrong order.
\choice Nothing is wrong.  The equation is correct, as is.
\end{multipleChoice}

\end{problem}}%}

%%%%%%%%%%%%%%%%%%%%%%


\latexProblemContent{
\begin{problem}

What is wrong with the following equation:

\[\int_{\frac{1}{4} \, \pi}^{\frac{7}{4} \, \pi} {-3 \, \cot\left(x\right) \csc\left(x\right)}\;dx = {\frac{3}{\sin\left(x\right)}}\Bigg\vert_{\frac{1}{4} \, \pi}^{\frac{7}{4} \, \pi} = {-6 \, \sqrt{2}}\]

\input{2311_Concept_Integral_0003.HELP.tex}

\begin{multipleChoice}
\choice The antiderivative is incorrect.
\choice[correct] The integrand is not defined over the entire interval.
\choice The bounds are evaluated in the wrong order.
\choice Nothing is wrong.  The equation is correct, as is.
\end{multipleChoice}

\end{problem}}%}

%%%%%%%%%%%%%%%%%%%%%%


\latexProblemContent{
\begin{problem}

What is wrong with the following equation:

\[\int_{\frac{2}{3} \, \pi}^{\frac{3}{2} \, \pi} {-5 \, \csc\left(x\right)^{2}}\;dx = {\frac{5}{\tan\left(x\right)}}\Bigg\vert_{\frac{2}{3} \, \pi}^{\frac{3}{2} \, \pi} = {\frac{5}{3} \, \sqrt{3}}\]

\input{2311_Concept_Integral_0003.HELP.tex}

\begin{multipleChoice}
\choice The antiderivative is incorrect.
\choice[correct] The integrand is not defined over the entire interval.
\choice The bounds are evaluated in the wrong order.
\choice Nothing is wrong.  The equation is correct, as is.
\end{multipleChoice}

\end{problem}}%}

%%%%%%%%%%%%%%%%%%%%%%


\latexProblemContent{
\begin{problem}

What is wrong with the following equation:

\[\int_{\frac{1}{3} \, \pi}^{\frac{5}{3} \, \pi} {-6 \, \cot\left(x\right) \csc\left(x\right)}\;dx = {\frac{6}{\sin\left(x\right)}}\Bigg\vert_{\frac{1}{3} \, \pi}^{\frac{5}{3} \, \pi} = {-8 \, \sqrt{3}}\]

\input{2311_Concept_Integral_0003.HELP.tex}

\begin{multipleChoice}
\choice The antiderivative is incorrect.
\choice[correct] The integrand is not defined over the entire interval.
\choice The bounds are evaluated in the wrong order.
\choice Nothing is wrong.  The equation is correct, as is.
\end{multipleChoice}

\end{problem}}%}

%%%%%%%%%%%%%%%%%%%%%%


\latexProblemContent{
\begin{problem}

What is wrong with the following equation:

\[\int_{\frac{1}{3} \, \pi}^{\frac{4}{3} \, \pi} {-13 \, \csc\left(x\right)^{2}}\;dx = {\frac{13}{\tan\left(x\right)}}\Bigg\vert_{\frac{1}{3} \, \pi}^{\frac{4}{3} \, \pi} = {0}\]

\input{2311_Concept_Integral_0003.HELP.tex}

\begin{multipleChoice}
\choice The antiderivative is incorrect.
\choice[correct] The integrand is not defined over the entire interval.
\choice The bounds are evaluated in the wrong order.
\choice Nothing is wrong.  The equation is correct, as is.
\end{multipleChoice}

\end{problem}}%}

%%%%%%%%%%%%%%%%%%%%%%


\latexProblemContent{
\begin{problem}

What is wrong with the following equation:

\[\int_{\frac{1}{2} \, \pi}^{\frac{5}{3} \, \pi} {-13 \, \csc\left(x\right)^{2}}\;dx = {\frac{13}{\tan\left(x\right)}}\Bigg\vert_{\frac{1}{2} \, \pi}^{\frac{5}{3} \, \pi} = {-\frac{13}{3} \, \sqrt{3}}\]

\input{2311_Concept_Integral_0003.HELP.tex}

\begin{multipleChoice}
\choice The antiderivative is incorrect.
\choice[correct] The integrand is not defined over the entire interval.
\choice The bounds are evaluated in the wrong order.
\choice Nothing is wrong.  The equation is correct, as is.
\end{multipleChoice}

\end{problem}}%}

%%%%%%%%%%%%%%%%%%%%%%


\latexProblemContent{
\begin{problem}

What is wrong with the following equation:

\[\int_{\frac{3}{4} \, \pi}^{\frac{3}{2} \, \pi} {-5 \, \csc\left(x\right)^{2}}\;dx = {\frac{5}{\tan\left(x\right)}}\Bigg\vert_{\frac{3}{4} \, \pi}^{\frac{3}{2} \, \pi} = {5}\]

\input{2311_Concept_Integral_0003.HELP.tex}

\begin{multipleChoice}
\choice The antiderivative is incorrect.
\choice[correct] The integrand is not defined over the entire interval.
\choice The bounds are evaluated in the wrong order.
\choice Nothing is wrong.  The equation is correct, as is.
\end{multipleChoice}

\end{problem}}%}

%%%%%%%%%%%%%%%%%%%%%%


%%%%%%%%%%%%%%%%%%%%%%


\latexProblemContent{
\begin{problem}

What is wrong with the following equation:

\[\int_{\frac{1}{2} \, \pi}^{\frac{7}{4} \, \pi} {-9 \, \cot\left(x\right) \csc\left(x\right)}\;dx = {\frac{9}{\sin\left(x\right)}}\Bigg\vert_{\frac{1}{2} \, \pi}^{\frac{7}{4} \, \pi} = {-9 \, \sqrt{2} - 9}\]

\input{2311_Concept_Integral_0003.HELP.tex}

\begin{multipleChoice}
\choice The antiderivative is incorrect.
\choice[correct] The integrand is not defined over the entire interval.
\choice The bounds are evaluated in the wrong order.
\choice Nothing is wrong.  The equation is correct, as is.
\end{multipleChoice}

\end{problem}}%}

%%%%%%%%%%%%%%%%%%%%%%


%%%%%%%%%%%%%%%%%%%%%%


%%%%%%%%%%%%%%%%%%%%%%


\latexProblemContent{
\begin{problem}

What is wrong with the following equation:

\[\int_{\frac{2}{3} \, \pi}^{\frac{5}{3} \, \pi} {-5 \, \csc\left(x\right)^{2}}\;dx = {\frac{5}{\tan\left(x\right)}}\Bigg\vert_{\frac{2}{3} \, \pi}^{\frac{5}{3} \, \pi} = {0}\]

\input{2311_Concept_Integral_0003.HELP.tex}

\begin{multipleChoice}
\choice The antiderivative is incorrect.
\choice[correct] The integrand is not defined over the entire interval.
\choice The bounds are evaluated in the wrong order.
\choice Nothing is wrong.  The equation is correct, as is.
\end{multipleChoice}

\end{problem}}%}

%%%%%%%%%%%%%%%%%%%%%%


\latexProblemContent{
\begin{problem}

What is wrong with the following equation:

\[\int_{\frac{1}{2} \, \pi}^{\frac{4}{3} \, \pi} {-3 \, \cot\left(x\right) \csc\left(x\right)}\;dx = {\frac{3}{\sin\left(x\right)}}\Bigg\vert_{\frac{1}{2} \, \pi}^{\frac{4}{3} \, \pi} = {-2 \, \sqrt{3} - 3}\]

\input{2311_Concept_Integral_0003.HELP.tex}

\begin{multipleChoice}
\choice The antiderivative is incorrect.
\choice[correct] The integrand is not defined over the entire interval.
\choice The bounds are evaluated in the wrong order.
\choice Nothing is wrong.  The equation is correct, as is.
\end{multipleChoice}

\end{problem}}%}

%%%%%%%%%%%%%%%%%%%%%%


\latexProblemContent{
\begin{problem}

What is wrong with the following equation:

\[\int_{\frac{1}{2} \, \pi}^{\frac{5}{3} \, \pi} {-11 \, \csc\left(x\right)^{2}}\;dx = {\frac{11}{\tan\left(x\right)}}\Bigg\vert_{\frac{1}{2} \, \pi}^{\frac{5}{3} \, \pi} = {-\frac{11}{3} \, \sqrt{3}}\]

\input{2311_Concept_Integral_0003.HELP.tex}

\begin{multipleChoice}
\choice The antiderivative is incorrect.
\choice[correct] The integrand is not defined over the entire interval.
\choice The bounds are evaluated in the wrong order.
\choice Nothing is wrong.  The equation is correct, as is.
\end{multipleChoice}

\end{problem}}%}

%%%%%%%%%%%%%%%%%%%%%%


\latexProblemContent{
\begin{problem}

What is wrong with the following equation:

\[\int_{\frac{2}{3} \, \pi}^{\frac{3}{2} \, \pi} {9 \, \cot\left(x\right) \csc\left(x\right)}\;dx = {-\frac{9}{\sin\left(x\right)}}\Bigg\vert_{\frac{2}{3} \, \pi}^{\frac{3}{2} \, \pi} = {6 \, \sqrt{3} + 9}\]

\input{2311_Concept_Integral_0003.HELP.tex}

\begin{multipleChoice}
\choice The antiderivative is incorrect.
\choice[correct] The integrand is not defined over the entire interval.
\choice The bounds are evaluated in the wrong order.
\choice Nothing is wrong.  The equation is correct, as is.
\end{multipleChoice}

\end{problem}}%}

%%%%%%%%%%%%%%%%%%%%%%


\latexProblemContent{
\begin{problem}

What is wrong with the following equation:

\[\int_{\frac{1}{2} \, \pi}^{\frac{4}{3} \, \pi} {-15 \, \cot\left(x\right) \csc\left(x\right)}\;dx = {\frac{15}{\sin\left(x\right)}}\Bigg\vert_{\frac{1}{2} \, \pi}^{\frac{4}{3} \, \pi} = {-10 \, \sqrt{3} - 15}\]

\input{2311_Concept_Integral_0003.HELP.tex}

\begin{multipleChoice}
\choice The antiderivative is incorrect.
\choice[correct] The integrand is not defined over the entire interval.
\choice The bounds are evaluated in the wrong order.
\choice Nothing is wrong.  The equation is correct, as is.
\end{multipleChoice}

\end{problem}}%}

%%%%%%%%%%%%%%%%%%%%%%


\latexProblemContent{
\begin{problem}

What is wrong with the following equation:

\[\int_{\frac{1}{2} \, \pi}^{\frac{7}{4} \, \pi} {-10 \, \csc\left(x\right)^{2}}\;dx = {\frac{10}{\tan\left(x\right)}}\Bigg\vert_{\frac{1}{2} \, \pi}^{\frac{7}{4} \, \pi} = {-10}\]

\input{2311_Concept_Integral_0003.HELP.tex}

\begin{multipleChoice}
\choice The antiderivative is incorrect.
\choice[correct] The integrand is not defined over the entire interval.
\choice The bounds are evaluated in the wrong order.
\choice Nothing is wrong.  The equation is correct, as is.
\end{multipleChoice}

\end{problem}}%}

%%%%%%%%%%%%%%%%%%%%%%


%%%%%%%%%%%%%%%%%%%%%%


%%%%%%%%%%%%%%%%%%%%%%


\latexProblemContent{
\begin{problem}

What is wrong with the following equation:

\[\int_{\frac{1}{3} \, \pi}^{\frac{4}{3} \, \pi} {12 \, \cot\left(x\right) \csc\left(x\right)}\;dx = {-\frac{12}{\sin\left(x\right)}}\Bigg\vert_{\frac{1}{3} \, \pi}^{\frac{4}{3} \, \pi} = {16 \, \sqrt{3}}\]

\input{2311_Concept_Integral_0003.HELP.tex}

\begin{multipleChoice}
\choice The antiderivative is incorrect.
\choice[correct] The integrand is not defined over the entire interval.
\choice The bounds are evaluated in the wrong order.
\choice Nothing is wrong.  The equation is correct, as is.
\end{multipleChoice}

\end{problem}}%}

%%%%%%%%%%%%%%%%%%%%%%


\latexProblemContent{
\begin{problem}

What is wrong with the following equation:

\[\int_{\frac{1}{2} \, \pi}^{\frac{3}{2} \, \pi} {-13 \, \cot\left(x\right) \csc\left(x\right)}\;dx = {\frac{13}{\sin\left(x\right)}}\Bigg\vert_{\frac{1}{2} \, \pi}^{\frac{3}{2} \, \pi} = {-26}\]

\input{2311_Concept_Integral_0003.HELP.tex}

\begin{multipleChoice}
\choice The antiderivative is incorrect.
\choice[correct] The integrand is not defined over the entire interval.
\choice The bounds are evaluated in the wrong order.
\choice Nothing is wrong.  The equation is correct, as is.
\end{multipleChoice}

\end{problem}}%}

%%%%%%%%%%%%%%%%%%%%%%


\latexProblemContent{
\begin{problem}

What is wrong with the following equation:

\[\int_{\frac{2}{3} \, \pi}^{\frac{5}{4} \, \pi} {-5 \, \csc\left(x\right)^{2}}\;dx = {\frac{5}{\tan\left(x\right)}}\Bigg\vert_{\frac{2}{3} \, \pi}^{\frac{5}{4} \, \pi} = {\frac{5}{3} \, \sqrt{3} + 5}\]

\input{2311_Concept_Integral_0003.HELP.tex}

\begin{multipleChoice}
\choice The antiderivative is incorrect.
\choice[correct] The integrand is not defined over the entire interval.
\choice The bounds are evaluated in the wrong order.
\choice Nothing is wrong.  The equation is correct, as is.
\end{multipleChoice}

\end{problem}}%}

%%%%%%%%%%%%%%%%%%%%%%


\latexProblemContent{
\begin{problem}

What is wrong with the following equation:

\[\int_{\frac{3}{4} \, \pi}^{\frac{4}{3} \, \pi} {-15 \, \cot\left(x\right) \csc\left(x\right)}\;dx = {\frac{15}{\sin\left(x\right)}}\Bigg\vert_{\frac{3}{4} \, \pi}^{\frac{4}{3} \, \pi} = {-10 \, \sqrt{3} - 15 \, \sqrt{2}}\]

\input{2311_Concept_Integral_0003.HELP.tex}

\begin{multipleChoice}
\choice The antiderivative is incorrect.
\choice[correct] The integrand is not defined over the entire interval.
\choice The bounds are evaluated in the wrong order.
\choice Nothing is wrong.  The equation is correct, as is.
\end{multipleChoice}

\end{problem}}%}

%%%%%%%%%%%%%%%%%%%%%%


\latexProblemContent{
\begin{problem}

What is wrong with the following equation:

\[\int_{\frac{2}{3} \, \pi}^{\frac{3}{2} \, \pi} {-5 \, \cot\left(x\right) \csc\left(x\right)}\;dx = {\frac{5}{\sin\left(x\right)}}\Bigg\vert_{\frac{2}{3} \, \pi}^{\frac{3}{2} \, \pi} = {-\frac{10}{3} \, \sqrt{3} - 5}\]

\input{2311_Concept_Integral_0003.HELP.tex}

\begin{multipleChoice}
\choice The antiderivative is incorrect.
\choice[correct] The integrand is not defined over the entire interval.
\choice The bounds are evaluated in the wrong order.
\choice Nothing is wrong.  The equation is correct, as is.
\end{multipleChoice}

\end{problem}}%}

%%%%%%%%%%%%%%%%%%%%%%


\latexProblemContent{
\begin{problem}

What is wrong with the following equation:

\[\int_{\frac{2}{3} \, \pi}^{\frac{4}{3} \, \pi} {13 \, \csc\left(x\right)^{2}}\;dx = {-\frac{13}{\tan\left(x\right)}}\Bigg\vert_{\frac{2}{3} \, \pi}^{\frac{4}{3} \, \pi} = {-\frac{26}{3} \, \sqrt{3}}\]

\input{2311_Concept_Integral_0003.HELP.tex}

\begin{multipleChoice}
\choice The antiderivative is incorrect.
\choice[correct] The integrand is not defined over the entire interval.
\choice The bounds are evaluated in the wrong order.
\choice Nothing is wrong.  The equation is correct, as is.
\end{multipleChoice}

\end{problem}}%}

%%%%%%%%%%%%%%%%%%%%%%


%%%%%%%%%%%%%%%%%%%%%%


\latexProblemContent{
\begin{problem}

What is wrong with the following equation:

\[\int_{\frac{3}{4} \, \pi}^{\frac{5}{4} \, \pi} {-15 \, \csc\left(x\right)^{2}}\;dx = {\frac{15}{\tan\left(x\right)}}\Bigg\vert_{\frac{3}{4} \, \pi}^{\frac{5}{4} \, \pi} = {30}\]

\input{2311_Concept_Integral_0003.HELP.tex}

\begin{multipleChoice}
\choice The antiderivative is incorrect.
\choice[correct] The integrand is not defined over the entire interval.
\choice The bounds are evaluated in the wrong order.
\choice Nothing is wrong.  The equation is correct, as is.
\end{multipleChoice}

\end{problem}}%}

%%%%%%%%%%%%%%%%%%%%%%


\latexProblemContent{
\begin{problem}

What is wrong with the following equation:

\[\int_{\frac{3}{4} \, \pi}^{\frac{3}{2} \, \pi} {14 \, \csc\left(x\right)^{2}}\;dx = {-\frac{14}{\tan\left(x\right)}}\Bigg\vert_{\frac{3}{4} \, \pi}^{\frac{3}{2} \, \pi} = {-14}\]

\input{2311_Concept_Integral_0003.HELP.tex}

\begin{multipleChoice}
\choice The antiderivative is incorrect.
\choice[correct] The integrand is not defined over the entire interval.
\choice The bounds are evaluated in the wrong order.
\choice Nothing is wrong.  The equation is correct, as is.
\end{multipleChoice}

\end{problem}}%}

%%%%%%%%%%%%%%%%%%%%%%


\latexProblemContent{
\begin{problem}

What is wrong with the following equation:

\[\int_{\frac{1}{4} \, \pi}^{\frac{3}{2} \, \pi} {14 \, \csc\left(x\right)^{2}}\;dx = {-\frac{14}{\tan\left(x\right)}}\Bigg\vert_{\frac{1}{4} \, \pi}^{\frac{3}{2} \, \pi} = {14}\]

\input{2311_Concept_Integral_0003.HELP.tex}

\begin{multipleChoice}
\choice The antiderivative is incorrect.
\choice[correct] The integrand is not defined over the entire interval.
\choice The bounds are evaluated in the wrong order.
\choice Nothing is wrong.  The equation is correct, as is.
\end{multipleChoice}

\end{problem}}%}

%%%%%%%%%%%%%%%%%%%%%%


\latexProblemContent{
\begin{problem}

What is wrong with the following equation:

\[\int_{\frac{3}{4} \, \pi}^{\frac{7}{4} \, \pi} {9 \, \csc\left(x\right)^{2}}\;dx = {-\frac{9}{\tan\left(x\right)}}\Bigg\vert_{\frac{3}{4} \, \pi}^{\frac{7}{4} \, \pi} = {0}\]

\input{2311_Concept_Integral_0003.HELP.tex}

\begin{multipleChoice}
\choice The antiderivative is incorrect.
\choice[correct] The integrand is not defined over the entire interval.
\choice The bounds are evaluated in the wrong order.
\choice Nothing is wrong.  The equation is correct, as is.
\end{multipleChoice}

\end{problem}}%}

%%%%%%%%%%%%%%%%%%%%%%


\latexProblemContent{
\begin{problem}

What is wrong with the following equation:

\[\int_{\frac{1}{3} \, \pi}^{\frac{4}{3} \, \pi} {2 \, \csc\left(x\right)^{2}}\;dx = {-\frac{2}{\tan\left(x\right)}}\Bigg\vert_{\frac{1}{3} \, \pi}^{\frac{4}{3} \, \pi} = {0}\]

\input{2311_Concept_Integral_0003.HELP.tex}

\begin{multipleChoice}
\choice The antiderivative is incorrect.
\choice[correct] The integrand is not defined over the entire interval.
\choice The bounds are evaluated in the wrong order.
\choice Nothing is wrong.  The equation is correct, as is.
\end{multipleChoice}

\end{problem}}%}

%%%%%%%%%%%%%%%%%%%%%%


\latexProblemContent{
\begin{problem}

What is wrong with the following equation:

\[\int_{\frac{1}{3} \, \pi}^{\frac{3}{2} \, \pi} {14 \, \csc\left(x\right)^{2}}\;dx = {-\frac{14}{\tan\left(x\right)}}\Bigg\vert_{\frac{1}{3} \, \pi}^{\frac{3}{2} \, \pi} = {\frac{14}{3} \, \sqrt{3}}\]

\input{2311_Concept_Integral_0003.HELP.tex}

\begin{multipleChoice}
\choice The antiderivative is incorrect.
\choice[correct] The integrand is not defined over the entire interval.
\choice The bounds are evaluated in the wrong order.
\choice Nothing is wrong.  The equation is correct, as is.
\end{multipleChoice}

\end{problem}}%}

%%%%%%%%%%%%%%%%%%%%%%


\latexProblemContent{
\begin{problem}

What is wrong with the following equation:

\[\int_{\frac{1}{3} \, \pi}^{\frac{3}{2} \, \pi} {-9 \, \csc\left(x\right)^{2}}\;dx = {\frac{9}{\tan\left(x\right)}}\Bigg\vert_{\frac{1}{3} \, \pi}^{\frac{3}{2} \, \pi} = {-3 \, \sqrt{3}}\]

\input{2311_Concept_Integral_0003.HELP.tex}

\begin{multipleChoice}
\choice The antiderivative is incorrect.
\choice[correct] The integrand is not defined over the entire interval.
\choice The bounds are evaluated in the wrong order.
\choice Nothing is wrong.  The equation is correct, as is.
\end{multipleChoice}

\end{problem}}%}

%%%%%%%%%%%%%%%%%%%%%%


\latexProblemContent{
\begin{problem}

What is wrong with the following equation:

\[\int_{\frac{1}{2} \, \pi}^{\frac{3}{2} \, \pi} {-2 \, \cot\left(x\right) \csc\left(x\right)}\;dx = {\frac{2}{\sin\left(x\right)}}\Bigg\vert_{\frac{1}{2} \, \pi}^{\frac{3}{2} \, \pi} = {-4}\]

\input{2311_Concept_Integral_0003.HELP.tex}

\begin{multipleChoice}
\choice The antiderivative is incorrect.
\choice[correct] The integrand is not defined over the entire interval.
\choice The bounds are evaluated in the wrong order.
\choice Nothing is wrong.  The equation is correct, as is.
\end{multipleChoice}

\end{problem}}%}

%%%%%%%%%%%%%%%%%%%%%%


\latexProblemContent{
\begin{problem}

What is wrong with the following equation:

\[\int_{\frac{1}{2} \, \pi}^{\frac{5}{4} \, \pi} {-4 \, \cot\left(x\right) \csc\left(x\right)}\;dx = {\frac{4}{\sin\left(x\right)}}\Bigg\vert_{\frac{1}{2} \, \pi}^{\frac{5}{4} \, \pi} = {-4 \, \sqrt{2} - 4}\]

\input{2311_Concept_Integral_0003.HELP.tex}

\begin{multipleChoice}
\choice The antiderivative is incorrect.
\choice[correct] The integrand is not defined over the entire interval.
\choice The bounds are evaluated in the wrong order.
\choice Nothing is wrong.  The equation is correct, as is.
\end{multipleChoice}

\end{problem}}%}

%%%%%%%%%%%%%%%%%%%%%%


\latexProblemContent{
\begin{problem}

What is wrong with the following equation:

\[\int_{\frac{1}{4} \, \pi}^{\frac{4}{3} \, \pi} {-9 \, \csc\left(x\right)^{2}}\;dx = {\frac{9}{\tan\left(x\right)}}\Bigg\vert_{\frac{1}{4} \, \pi}^{\frac{4}{3} \, \pi} = {3 \, \sqrt{3} - 9}\]

\input{2311_Concept_Integral_0003.HELP.tex}

\begin{multipleChoice}
\choice The antiderivative is incorrect.
\choice[correct] The integrand is not defined over the entire interval.
\choice The bounds are evaluated in the wrong order.
\choice Nothing is wrong.  The equation is correct, as is.
\end{multipleChoice}

\end{problem}}%}

%%%%%%%%%%%%%%%%%%%%%%


%%%%%%%%%%%%%%%%%%%%%%


\latexProblemContent{
\begin{problem}

What is wrong with the following equation:

\[\int_{\frac{2}{3} \, \pi}^{\frac{7}{4} \, \pi} {-9 \, \csc\left(x\right)^{2}}\;dx = {\frac{9}{\tan\left(x\right)}}\Bigg\vert_{\frac{2}{3} \, \pi}^{\frac{7}{4} \, \pi} = {3 \, \sqrt{3} - 9}\]

\input{2311_Concept_Integral_0003.HELP.tex}

\begin{multipleChoice}
\choice The antiderivative is incorrect.
\choice[correct] The integrand is not defined over the entire interval.
\choice The bounds are evaluated in the wrong order.
\choice Nothing is wrong.  The equation is correct, as is.
\end{multipleChoice}

\end{problem}}%}

%%%%%%%%%%%%%%%%%%%%%%


%%%%%%%%%%%%%%%%%%%%%%


\latexProblemContent{
\begin{problem}

What is wrong with the following equation:

\[\int_{\frac{1}{2} \, \pi}^{\frac{7}{4} \, \pi} {-7 \, \cot\left(x\right) \csc\left(x\right)}\;dx = {\frac{7}{\sin\left(x\right)}}\Bigg\vert_{\frac{1}{2} \, \pi}^{\frac{7}{4} \, \pi} = {-7 \, \sqrt{2} - 7}\]

\input{2311_Concept_Integral_0003.HELP.tex}

\begin{multipleChoice}
\choice The antiderivative is incorrect.
\choice[correct] The integrand is not defined over the entire interval.
\choice The bounds are evaluated in the wrong order.
\choice Nothing is wrong.  The equation is correct, as is.
\end{multipleChoice}

\end{problem}}%}

%%%%%%%%%%%%%%%%%%%%%%


%%%%%%%%%%%%%%%%%%%%%%


%%%%%%%%%%%%%%%%%%%%%%


\latexProblemContent{
\begin{problem}

What is wrong with the following equation:

\[\int_{\frac{2}{3} \, \pi}^{\frac{5}{3} \, \pi} {2 \, \cot\left(x\right) \csc\left(x\right)}\;dx = {-\frac{2}{\sin\left(x\right)}}\Bigg\vert_{\frac{2}{3} \, \pi}^{\frac{5}{3} \, \pi} = {\frac{8}{3} \, \sqrt{3}}\]

\input{2311_Concept_Integral_0003.HELP.tex}

\begin{multipleChoice}
\choice The antiderivative is incorrect.
\choice[correct] The integrand is not defined over the entire interval.
\choice The bounds are evaluated in the wrong order.
\choice Nothing is wrong.  The equation is correct, as is.
\end{multipleChoice}

\end{problem}}%}

%%%%%%%%%%%%%%%%%%%%%%


%%%%%%%%%%%%%%%%%%%%%%


\latexProblemContent{
\begin{problem}

What is wrong with the following equation:

\[\int_{\frac{1}{4} \, \pi}^{\frac{3}{2} \, \pi} {-13 \, \cot\left(x\right) \csc\left(x\right)}\;dx = {\frac{13}{\sin\left(x\right)}}\Bigg\vert_{\frac{1}{4} \, \pi}^{\frac{3}{2} \, \pi} = {-13 \, \sqrt{2} - 13}\]

\input{2311_Concept_Integral_0003.HELP.tex}

\begin{multipleChoice}
\choice The antiderivative is incorrect.
\choice[correct] The integrand is not defined over the entire interval.
\choice The bounds are evaluated in the wrong order.
\choice Nothing is wrong.  The equation is correct, as is.
\end{multipleChoice}

\end{problem}}%}

%%%%%%%%%%%%%%%%%%%%%%


\latexProblemContent{
\begin{problem}

What is wrong with the following equation:

\[\int_{\frac{1}{3} \, \pi}^{\frac{3}{2} \, \pi} {-4 \, \csc\left(x\right)^{2}}\;dx = {\frac{4}{\tan\left(x\right)}}\Bigg\vert_{\frac{1}{3} \, \pi}^{\frac{3}{2} \, \pi} = {-\frac{4}{3} \, \sqrt{3}}\]

\input{2311_Concept_Integral_0003.HELP.tex}

\begin{multipleChoice}
\choice The antiderivative is incorrect.
\choice[correct] The integrand is not defined over the entire interval.
\choice The bounds are evaluated in the wrong order.
\choice Nothing is wrong.  The equation is correct, as is.
\end{multipleChoice}

\end{problem}}%}

%%%%%%%%%%%%%%%%%%%%%%


\latexProblemContent{
\begin{problem}

What is wrong with the following equation:

\[\int_{\frac{3}{4} \, \pi}^{\frac{3}{2} \, \pi} {-4 \, \csc\left(x\right)^{2}}\;dx = {\frac{4}{\tan\left(x\right)}}\Bigg\vert_{\frac{3}{4} \, \pi}^{\frac{3}{2} \, \pi} = {4}\]

\input{2311_Concept_Integral_0003.HELP.tex}

\begin{multipleChoice}
\choice The antiderivative is incorrect.
\choice[correct] The integrand is not defined over the entire interval.
\choice The bounds are evaluated in the wrong order.
\choice Nothing is wrong.  The equation is correct, as is.
\end{multipleChoice}

\end{problem}}%}

%%%%%%%%%%%%%%%%%%%%%%


%%%%%%%%%%%%%%%%%%%%%%


\latexProblemContent{
\begin{problem}

What is wrong with the following equation:

\[\int_{\frac{3}{4} \, \pi}^{\frac{5}{4} \, \pi} {-5 \, \cot\left(x\right) \csc\left(x\right)}\;dx = {\frac{5}{\sin\left(x\right)}}\Bigg\vert_{\frac{3}{4} \, \pi}^{\frac{5}{4} \, \pi} = {-10 \, \sqrt{2}}\]

\input{2311_Concept_Integral_0003.HELP.tex}

\begin{multipleChoice}
\choice The antiderivative is incorrect.
\choice[correct] The integrand is not defined over the entire interval.
\choice The bounds are evaluated in the wrong order.
\choice Nothing is wrong.  The equation is correct, as is.
\end{multipleChoice}

\end{problem}}%}

%%%%%%%%%%%%%%%%%%%%%%


\latexProblemContent{
\begin{problem}

What is wrong with the following equation:

\[\int_{\frac{1}{2} \, \pi}^{\frac{7}{4} \, \pi} {9 \, \cot\left(x\right) \csc\left(x\right)}\;dx = {-\frac{9}{\sin\left(x\right)}}\Bigg\vert_{\frac{1}{2} \, \pi}^{\frac{7}{4} \, \pi} = {9 \, \sqrt{2} + 9}\]

\input{2311_Concept_Integral_0003.HELP.tex}

\begin{multipleChoice}
\choice The antiderivative is incorrect.
\choice[correct] The integrand is not defined over the entire interval.
\choice The bounds are evaluated in the wrong order.
\choice Nothing is wrong.  The equation is correct, as is.
\end{multipleChoice}

\end{problem}}%}

%%%%%%%%%%%%%%%%%%%%%%


