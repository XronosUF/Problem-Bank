
%%%%%%%%%%%%%%%%%%%%%
%\tagged{Cat@One, Cat@Two, Cat@Three, Cat@Four, Cat@Five, Ans@MultiAns, Type@Concept, Topic@Limit, Sub@Trig, Sub@Squeeze}{

\latexProblemContent{
\begin{problem}

The limit as $x\to{-6}$ of $f(x)={{\left(x + 6\right)}^{3} \cos\left(-\frac{11}{x + 6}\right)}$ is $0$.  What is the reason why this is true?

\input{2311-Concept-Limit-0008.HELP.tex}

\begin{multipleChoice}
\choice{The statement is in fact false: $\lim\limits_{x\to{-6}}{{\left(x + 6\right)}^{3} \cos\left(-\frac{11}{x + 6}\right)}\neq0$.}
\choice{The cosine factor decreases to $0$ faster than the polynomial.}
\choice[correct]{The cosine factor is bounded between $-1$ and $1$, so the polynomial forces the function to $0$.}
\choice{The cosine factor directly cancels out the polynomial factor.}
\end{multipleChoice}


What is the name of the theorem that applies to this problem? \qquad \\
The \underline{\answer{Squeeze}} Theorem
\end{problem}}%}

%%%%%%%%%%%%%%%%%%%%%




\latexProblemContent{
\begin{problem}

The limit as $x\to{-10}$ of $f(x)={{\left(x + 10\right)} \cos\left(-\frac{9}{{\left(x + 10\right)}^{2}}\right)}$ is $0$.  What is the reason why this is true?

\input{2311-Concept-Limit-0008.HELP.tex}

\begin{multipleChoice}
\choice{The statement is in fact false: $\lim\limits_{x\to{-10}}{{\left(x + 10\right)} \cos\left(-\frac{9}{{\left(x + 10\right)}^{2}}\right)}\neq0$.}
\choice{The cosine factor decreases to $0$ faster than the polynomial.}
\choice[correct]{The cosine factor is bounded between $-1$ and $1$, so the polynomial forces the function to $0$.}
\choice{The cosine factor directly cancels out the polynomial factor.}
\end{multipleChoice}


What is the name of the theorem that applies to this problem? \qquad \\
The \underline{\answer{Squeeze}} Theorem
\end{problem}}%}

%%%%%%%%%%%%%%%%%%%%%




\latexProblemContent{
\begin{problem}

The limit as $x\to{-14}$ of $f(x)={{\left(x + 14\right)}^{3} \cos\left(-\frac{18}{{\left(x + 14\right)}^{2}}\right)}$ is $0$.  What is the reason why this is true?

\input{2311-Concept-Limit-0008.HELP.tex}

\begin{multipleChoice}
\choice{The statement is in fact false: $\lim\limits_{x\to{-14}}{{\left(x + 14\right)}^{3} \cos\left(-\frac{18}{{\left(x + 14\right)}^{2}}\right)}\neq0$.}
\choice{The cosine factor decreases to $0$ faster than the polynomial.}
\choice[correct]{The cosine factor is bounded between $-1$ and $1$, so the polynomial forces the function to $0$.}
\choice{The cosine factor directly cancels out the polynomial factor.}
\end{multipleChoice}


What is the name of the theorem that applies to this problem? \qquad \\
The \underline{\answer{Squeeze}} Theorem
\end{problem}}%}

%%%%%%%%%%%%%%%%%%%%%




\latexProblemContent{
\begin{problem}

The limit as $x\to{-1}$ of $f(x)={{\left(x + 1\right)}^{2} \cos\left(\frac{21}{{\left(x + 1\right)}^{2}}\right)}$ is $0$.  What is the reason why this is true?

\input{2311-Concept-Limit-0008.HELP.tex}

\begin{multipleChoice}
\choice{The statement is in fact false: $\lim\limits_{x\to{-1}}{{\left(x + 1\right)}^{2} \cos\left(\frac{21}{{\left(x + 1\right)}^{2}}\right)}\neq0$.}
\choice{The cosine factor decreases to $0$ faster than the polynomial.}
\choice[correct]{The cosine factor is bounded between $-1$ and $1$, so the polynomial forces the function to $0$.}
\choice{The cosine factor directly cancels out the polynomial factor.}
\end{multipleChoice}


What is the name of the theorem that applies to this problem? \qquad \\
The \underline{\answer{Squeeze}} Theorem
\end{problem}}%}

%%%%%%%%%%%%%%%%%%%%%




\latexProblemContent{
\begin{problem}

The limit as $x\to{-6}$ of $f(x)={{\left(x + 6\right)}^{2} \cos\left(-\frac{11}{x + 6}\right)}$ is $0$.  What is the reason why this is true?

\input{2311-Concept-Limit-0008.HELP.tex}

\begin{multipleChoice}
\choice{The statement is in fact false: $\lim\limits_{x\to{-6}}{{\left(x + 6\right)}^{2} \cos\left(-\frac{11}{x + 6}\right)}\neq0$.}
\choice{The cosine factor decreases to $0$ faster than the polynomial.}
\choice[correct]{The cosine factor is bounded between $-1$ and $1$, so the polynomial forces the function to $0$.}
\choice{The cosine factor directly cancels out the polynomial factor.}
\end{multipleChoice}


What is the name of the theorem that applies to this problem? \qquad \\
The \underline{\answer{Squeeze}} Theorem
\end{problem}}%}

%%%%%%%%%%%%%%%%%%%%%




\latexProblemContent{
\begin{problem}

The limit as $x\to{-1}$ of $f(x)={{\left(x + 1\right)} \cos\left(\frac{19}{{\left(x + 1\right)}^{2}}\right)}$ is $0$.  What is the reason why this is true?

\input{2311-Concept-Limit-0008.HELP.tex}

\begin{multipleChoice}
\choice{The statement is in fact false: $\lim\limits_{x\to{-1}}{{\left(x + 1\right)} \cos\left(\frac{19}{{\left(x + 1\right)}^{2}}\right)}\neq0$.}
\choice{The cosine factor decreases to $0$ faster than the polynomial.}
\choice[correct]{The cosine factor is bounded between $-1$ and $1$, so the polynomial forces the function to $0$.}
\choice{The cosine factor directly cancels out the polynomial factor.}
\end{multipleChoice}


What is the name of the theorem that applies to this problem? \qquad \\
The \underline{\answer{Squeeze}} Theorem
\end{problem}}%}

%%%%%%%%%%%%%%%%%%%%%




\latexProblemContent{
\begin{problem}

The limit as $x\to{15}$ of $f(x)={{\left(x - 15\right)}^{3} \cos\left(-\frac{3}{x - 15}\right)}$ is $0$.  What is the reason why this is true?

\input{2311-Concept-Limit-0008.HELP.tex}

\begin{multipleChoice}
\choice{The statement is in fact false: $\lim\limits_{x\to{15}}{{\left(x - 15\right)}^{3} \cos\left(-\frac{3}{x - 15}\right)}\neq0$.}
\choice{The cosine factor decreases to $0$ faster than the polynomial.}
\choice[correct]{The cosine factor is bounded between $-1$ and $1$, so the polynomial forces the function to $0$.}
\choice{The cosine factor directly cancels out the polynomial factor.}
\end{multipleChoice}


What is the name of the theorem that applies to this problem? \qquad \\
The \underline{\answer{Squeeze}} Theorem
\end{problem}}%}

%%%%%%%%%%%%%%%%%%%%%




\latexProblemContent{
\begin{problem}

The limit as $x\to{-5}$ of $f(x)={{\left(x + 5\right)}^{2} \cos\left(-\frac{18}{{\left(x + 5\right)}^{2}}\right)}$ is $0$.  What is the reason why this is true?

\input{2311-Concept-Limit-0008.HELP.tex}

\begin{multipleChoice}
\choice{The statement is in fact false: $\lim\limits_{x\to{-5}}{{\left(x + 5\right)}^{2} \cos\left(-\frac{18}{{\left(x + 5\right)}^{2}}\right)}\neq0$.}
\choice{The cosine factor decreases to $0$ faster than the polynomial.}
\choice[correct]{The cosine factor is bounded between $-1$ and $1$, so the polynomial forces the function to $0$.}
\choice{The cosine factor directly cancels out the polynomial factor.}
\end{multipleChoice}


What is the name of the theorem that applies to this problem? \qquad \\
The \underline{\answer{Squeeze}} Theorem
\end{problem}}%}

%%%%%%%%%%%%%%%%%%%%%




\latexProblemContent{
\begin{problem}

The limit as $x\to{0}$ of $f(x)={x^{2} \cos\left(-\frac{22}{x}\right)}$ is $0$.  What is the reason why this is true?

\input{2311-Concept-Limit-0008.HELP.tex}

\begin{multipleChoice}
\choice{The statement is in fact false: $\lim\limits_{x\to{0}}{x^{2} \cos\left(-\frac{22}{x}\right)}\neq0$.}
\choice{The cosine factor decreases to $0$ faster than the polynomial.}
\choice[correct]{The cosine factor is bounded between $-1$ and $1$, so the polynomial forces the function to $0$.}
\choice{The cosine factor directly cancels out the polynomial factor.}
\end{multipleChoice}


What is the name of the theorem that applies to this problem? \qquad \\
The \underline{\answer{Squeeze}} Theorem
\end{problem}}%}

%%%%%%%%%%%%%%%%%%%%%




\latexProblemContent{
\begin{problem}

The limit as $x\to{14}$ of $f(x)={{\left(x - 14\right)} \cos\left(\frac{8}{{\left(x - 14\right)}^{2}}\right)}$ is $0$.  What is the reason why this is true?

\input{2311-Concept-Limit-0008.HELP.tex}

\begin{multipleChoice}
\choice{The statement is in fact false: $\lim\limits_{x\to{14}}{{\left(x - 14\right)} \cos\left(\frac{8}{{\left(x - 14\right)}^{2}}\right)}\neq0$.}
\choice{The cosine factor decreases to $0$ faster than the polynomial.}
\choice[correct]{The cosine factor is bounded between $-1$ and $1$, so the polynomial forces the function to $0$.}
\choice{The cosine factor directly cancels out the polynomial factor.}
\end{multipleChoice}


What is the name of the theorem that applies to this problem? \qquad \\
The \underline{\answer{Squeeze}} Theorem
\end{problem}}%}

%%%%%%%%%%%%%%%%%%%%%




\latexProblemContent{
\begin{problem}

The limit as $x\to{14}$ of $f(x)={{\left(x - 14\right)} \cos\left(\frac{19}{x - 14}\right)}$ is $0$.  What is the reason why this is true?

\input{2311-Concept-Limit-0008.HELP.tex}

\begin{multipleChoice}
\choice{The statement is in fact false: $\lim\limits_{x\to{14}}{{\left(x - 14\right)} \cos\left(\frac{19}{x - 14}\right)}\neq0$.}
\choice{The cosine factor decreases to $0$ faster than the polynomial.}
\choice[correct]{The cosine factor is bounded between $-1$ and $1$, so the polynomial forces the function to $0$.}
\choice{The cosine factor directly cancels out the polynomial factor.}
\end{multipleChoice}


What is the name of the theorem that applies to this problem? \qquad \\
The \underline{\answer{Squeeze}} Theorem
\end{problem}}%}

%%%%%%%%%%%%%%%%%%%%%




\latexProblemContent{
\begin{problem}

The limit as $x\to{-14}$ of $f(x)={{\left(x + 14\right)}^{3} \cos\left(-\frac{4}{x + 14}\right)}$ is $0$.  What is the reason why this is true?

\input{2311-Concept-Limit-0008.HELP.tex}

\begin{multipleChoice}
\choice{The statement is in fact false: $\lim\limits_{x\to{-14}}{{\left(x + 14\right)}^{3} \cos\left(-\frac{4}{x + 14}\right)}\neq0$.}
\choice{The cosine factor decreases to $0$ faster than the polynomial.}
\choice[correct]{The cosine factor is bounded between $-1$ and $1$, so the polynomial forces the function to $0$.}
\choice{The cosine factor directly cancels out the polynomial factor.}
\end{multipleChoice}


What is the name of the theorem that applies to this problem? \qquad \\
The \underline{\answer{Squeeze}} Theorem
\end{problem}}%}

%%%%%%%%%%%%%%%%%%%%%




\latexProblemContent{
\begin{problem}

The limit as $x\to{13}$ of $f(x)={{\left(x - 13\right)}^{2} \cos\left(-\frac{19}{{\left(x - 13\right)}^{2}}\right)}$ is $0$.  What is the reason why this is true?

\input{2311-Concept-Limit-0008.HELP.tex}

\begin{multipleChoice}
\choice{The statement is in fact false: $\lim\limits_{x\to{13}}{{\left(x - 13\right)}^{2} \cos\left(-\frac{19}{{\left(x - 13\right)}^{2}}\right)}\neq0$.}
\choice{The cosine factor decreases to $0$ faster than the polynomial.}
\choice[correct]{The cosine factor is bounded between $-1$ and $1$, so the polynomial forces the function to $0$.}
\choice{The cosine factor directly cancels out the polynomial factor.}
\end{multipleChoice}


What is the name of the theorem that applies to this problem? \qquad \\
The \underline{\answer{Squeeze}} Theorem
\end{problem}}%}

%%%%%%%%%%%%%%%%%%%%%




\latexProblemContent{
\begin{problem}

The limit as $x\to{-15}$ of $f(x)={{\left(x + 15\right)}^{2} \cos\left(\frac{1}{{\left(x + 15\right)}^{2}}\right)}$ is $0$.  What is the reason why this is true?

\input{2311-Concept-Limit-0008.HELP.tex}

\begin{multipleChoice}
\choice{The statement is in fact false: $\lim\limits_{x\to{-15}}{{\left(x + 15\right)}^{2} \cos\left(\frac{1}{{\left(x + 15\right)}^{2}}\right)}\neq0$.}
\choice{The cosine factor decreases to $0$ faster than the polynomial.}
\choice[correct]{The cosine factor is bounded between $-1$ and $1$, so the polynomial forces the function to $0$.}
\choice{The cosine factor directly cancels out the polynomial factor.}
\end{multipleChoice}


What is the name of the theorem that applies to this problem? \qquad \\
The \underline{\answer{Squeeze}} Theorem
\end{problem}}%}

%%%%%%%%%%%%%%%%%%%%%




\latexProblemContent{
\begin{problem}

The limit as $x\to{13}$ of $f(x)={{\left(x - 13\right)} \cos\left(\frac{10}{x - 13}\right)}$ is $0$.  What is the reason why this is true?

\input{2311-Concept-Limit-0008.HELP.tex}

\begin{multipleChoice}
\choice{The statement is in fact false: $\lim\limits_{x\to{13}}{{\left(x - 13\right)} \cos\left(\frac{10}{x - 13}\right)}\neq0$.}
\choice{The cosine factor decreases to $0$ faster than the polynomial.}
\choice[correct]{The cosine factor is bounded between $-1$ and $1$, so the polynomial forces the function to $0$.}
\choice{The cosine factor directly cancels out the polynomial factor.}
\end{multipleChoice}


What is the name of the theorem that applies to this problem? \qquad \\
The \underline{\answer{Squeeze}} Theorem
\end{problem}}%}

%%%%%%%%%%%%%%%%%%%%%




\latexProblemContent{
\begin{problem}

The limit as $x\to{11}$ of $f(x)={{\left(x - 11\right)}^{3} \cos\left(\frac{3}{x - 11}\right)}$ is $0$.  What is the reason why this is true?

\input{2311-Concept-Limit-0008.HELP.tex}

\begin{multipleChoice}
\choice{The statement is in fact false: $\lim\limits_{x\to{11}}{{\left(x - 11\right)}^{3} \cos\left(\frac{3}{x - 11}\right)}\neq0$.}
\choice{The cosine factor decreases to $0$ faster than the polynomial.}
\choice[correct]{The cosine factor is bounded between $-1$ and $1$, so the polynomial forces the function to $0$.}
\choice{The cosine factor directly cancels out the polynomial factor.}
\end{multipleChoice}


What is the name of the theorem that applies to this problem? \qquad \\
The \underline{\answer{Squeeze}} Theorem
\end{problem}}%}

%%%%%%%%%%%%%%%%%%%%%




\latexProblemContent{
\begin{problem}

The limit as $x\to{14}$ of $f(x)={{\left(x - 14\right)}^{2} \cos\left(-\frac{10}{{\left(x - 14\right)}^{2}}\right)}$ is $0$.  What is the reason why this is true?

\input{2311-Concept-Limit-0008.HELP.tex}

\begin{multipleChoice}
\choice{The statement is in fact false: $\lim\limits_{x\to{14}}{{\left(x - 14\right)}^{2} \cos\left(-\frac{10}{{\left(x - 14\right)}^{2}}\right)}\neq0$.}
\choice{The cosine factor decreases to $0$ faster than the polynomial.}
\choice[correct]{The cosine factor is bounded between $-1$ and $1$, so the polynomial forces the function to $0$.}
\choice{The cosine factor directly cancels out the polynomial factor.}
\end{multipleChoice}


What is the name of the theorem that applies to this problem? \qquad \\
The \underline{\answer{Squeeze}} Theorem
\end{problem}}%}

%%%%%%%%%%%%%%%%%%%%%




\latexProblemContent{
\begin{problem}

The limit as $x\to{11}$ of $f(x)={{\left(x - 11\right)}^{2} \cos\left(\frac{21}{x - 11}\right)}$ is $0$.  What is the reason why this is true?

\input{2311-Concept-Limit-0008.HELP.tex}

\begin{multipleChoice}
\choice{The statement is in fact false: $\lim\limits_{x\to{11}}{{\left(x - 11\right)}^{2} \cos\left(\frac{21}{x - 11}\right)}\neq0$.}
\choice{The cosine factor decreases to $0$ faster than the polynomial.}
\choice[correct]{The cosine factor is bounded between $-1$ and $1$, so the polynomial forces the function to $0$.}
\choice{The cosine factor directly cancels out the polynomial factor.}
\end{multipleChoice}


What is the name of the theorem that applies to this problem? \qquad \\
The \underline{\answer{Squeeze}} Theorem
\end{problem}}%}

%%%%%%%%%%%%%%%%%%%%%




\latexProblemContent{
\begin{problem}

The limit as $x\to{0}$ of $f(x)={x \cos\left(\frac{21}{x^{2}}\right)}$ is $0$.  What is the reason why this is true?

\input{2311-Concept-Limit-0008.HELP.tex}

\begin{multipleChoice}
\choice{The statement is in fact false: $\lim\limits_{x\to{0}}{x \cos\left(\frac{21}{x^{2}}\right)}\neq0$.}
\choice{The cosine factor decreases to $0$ faster than the polynomial.}
\choice[correct]{The cosine factor is bounded between $-1$ and $1$, so the polynomial forces the function to $0$.}
\choice{The cosine factor directly cancels out the polynomial factor.}
\end{multipleChoice}


What is the name of the theorem that applies to this problem? \qquad \\
The \underline{\answer{Squeeze}} Theorem
\end{problem}}%}

%%%%%%%%%%%%%%%%%%%%%




\latexProblemContent{
\begin{problem}

The limit as $x\to{-4}$ of $f(x)={{\left(x + 4\right)} \cos\left(-\frac{6}{{\left(x + 4\right)}^{2}}\right)}$ is $0$.  What is the reason why this is true?

\input{2311-Concept-Limit-0008.HELP.tex}

\begin{multipleChoice}
\choice{The statement is in fact false: $\lim\limits_{x\to{-4}}{{\left(x + 4\right)} \cos\left(-\frac{6}{{\left(x + 4\right)}^{2}}\right)}\neq0$.}
\choice{The cosine factor decreases to $0$ faster than the polynomial.}
\choice[correct]{The cosine factor is bounded between $-1$ and $1$, so the polynomial forces the function to $0$.}
\choice{The cosine factor directly cancels out the polynomial factor.}
\end{multipleChoice}


What is the name of the theorem that applies to this problem? \qquad \\
The \underline{\answer{Squeeze}} Theorem
\end{problem}}%}

%%%%%%%%%%%%%%%%%%%%%




\latexProblemContent{
\begin{problem}

The limit as $x\to{1}$ of $f(x)={{\left(x - 1\right)}^{2} \cos\left(\frac{21}{x - 1}\right)}$ is $0$.  What is the reason why this is true?

\input{2311-Concept-Limit-0008.HELP.tex}

\begin{multipleChoice}
\choice{The statement is in fact false: $\lim\limits_{x\to{1}}{{\left(x - 1\right)}^{2} \cos\left(\frac{21}{x - 1}\right)}\neq0$.}
\choice{The cosine factor decreases to $0$ faster than the polynomial.}
\choice[correct]{The cosine factor is bounded between $-1$ and $1$, so the polynomial forces the function to $0$.}
\choice{The cosine factor directly cancels out the polynomial factor.}
\end{multipleChoice}


What is the name of the theorem that applies to this problem? \qquad \\
The \underline{\answer{Squeeze}} Theorem
\end{problem}}%}

%%%%%%%%%%%%%%%%%%%%%




\latexProblemContent{
\begin{problem}

The limit as $x\to{5}$ of $f(x)={{\left(x - 5\right)}^{3} \cos\left(\frac{14}{x - 5}\right)}$ is $0$.  What is the reason why this is true?

\input{2311-Concept-Limit-0008.HELP.tex}

\begin{multipleChoice}
\choice{The statement is in fact false: $\lim\limits_{x\to{5}}{{\left(x - 5\right)}^{3} \cos\left(\frac{14}{x - 5}\right)}\neq0$.}
\choice{The cosine factor decreases to $0$ faster than the polynomial.}
\choice[correct]{The cosine factor is bounded between $-1$ and $1$, so the polynomial forces the function to $0$.}
\choice{The cosine factor directly cancels out the polynomial factor.}
\end{multipleChoice}


What is the name of the theorem that applies to this problem? \qquad \\
The \underline{\answer{Squeeze}} Theorem
\end{problem}}%}

%%%%%%%%%%%%%%%%%%%%%




\latexProblemContent{
\begin{problem}

The limit as $x\to{-9}$ of $f(x)={{\left(x + 9\right)}^{2} \cos\left(\frac{13}{x + 9}\right)}$ is $0$.  What is the reason why this is true?

\input{2311-Concept-Limit-0008.HELP.tex}

\begin{multipleChoice}
\choice{The statement is in fact false: $\lim\limits_{x\to{-9}}{{\left(x + 9\right)}^{2} \cos\left(\frac{13}{x + 9}\right)}\neq0$.}
\choice{The cosine factor decreases to $0$ faster than the polynomial.}
\choice[correct]{The cosine factor is bounded between $-1$ and $1$, so the polynomial forces the function to $0$.}
\choice{The cosine factor directly cancels out the polynomial factor.}
\end{multipleChoice}


What is the name of the theorem that applies to this problem? \qquad \\
The \underline{\answer{Squeeze}} Theorem
\end{problem}}%}

%%%%%%%%%%%%%%%%%%%%%




\latexProblemContent{
\begin{problem}

The limit as $x\to{9}$ of $f(x)={{\left(x - 9\right)} \cos\left(\frac{3}{{\left(x - 9\right)}^{2}}\right)}$ is $0$.  What is the reason why this is true?

\input{2311-Concept-Limit-0008.HELP.tex}

\begin{multipleChoice}
\choice{The statement is in fact false: $\lim\limits_{x\to{9}}{{\left(x - 9\right)} \cos\left(\frac{3}{{\left(x - 9\right)}^{2}}\right)}\neq0$.}
\choice{The cosine factor decreases to $0$ faster than the polynomial.}
\choice[correct]{The cosine factor is bounded between $-1$ and $1$, so the polynomial forces the function to $0$.}
\choice{The cosine factor directly cancels out the polynomial factor.}
\end{multipleChoice}


What is the name of the theorem that applies to this problem? \qquad \\
The \underline{\answer{Squeeze}} Theorem
\end{problem}}%}

%%%%%%%%%%%%%%%%%%%%%




\latexProblemContent{
\begin{problem}

The limit as $x\to{-14}$ of $f(x)={{\left(x + 14\right)}^{2} \cos\left(\frac{14}{{\left(x + 14\right)}^{2}}\right)}$ is $0$.  What is the reason why this is true?

\input{2311-Concept-Limit-0008.HELP.tex}

\begin{multipleChoice}
\choice{The statement is in fact false: $\lim\limits_{x\to{-14}}{{\left(x + 14\right)}^{2} \cos\left(\frac{14}{{\left(x + 14\right)}^{2}}\right)}\neq0$.}
\choice{The cosine factor decreases to $0$ faster than the polynomial.}
\choice[correct]{The cosine factor is bounded between $-1$ and $1$, so the polynomial forces the function to $0$.}
\choice{The cosine factor directly cancels out the polynomial factor.}
\end{multipleChoice}


What is the name of the theorem that applies to this problem? \qquad \\
The \underline{\answer{Squeeze}} Theorem
\end{problem}}%}

%%%%%%%%%%%%%%%%%%%%%




\latexProblemContent{
\begin{problem}

The limit as $x\to{-4}$ of $f(x)={{\left(x + 4\right)}^{3} \cos\left(-\frac{16}{x + 4}\right)}$ is $0$.  What is the reason why this is true?

\input{2311-Concept-Limit-0008.HELP.tex}

\begin{multipleChoice}
\choice{The statement is in fact false: $\lim\limits_{x\to{-4}}{{\left(x + 4\right)}^{3} \cos\left(-\frac{16}{x + 4}\right)}\neq0$.}
\choice{The cosine factor decreases to $0$ faster than the polynomial.}
\choice[correct]{The cosine factor is bounded between $-1$ and $1$, so the polynomial forces the function to $0$.}
\choice{The cosine factor directly cancels out the polynomial factor.}
\end{multipleChoice}


What is the name of the theorem that applies to this problem? \qquad \\
The \underline{\answer{Squeeze}} Theorem
\end{problem}}%}

%%%%%%%%%%%%%%%%%%%%%




\latexProblemContent{
\begin{problem}

The limit as $x\to{-5}$ of $f(x)={{\left(x + 5\right)}^{3} \cos\left(-\frac{19}{{\left(x + 5\right)}^{2}}\right)}$ is $0$.  What is the reason why this is true?

\input{2311-Concept-Limit-0008.HELP.tex}

\begin{multipleChoice}
\choice{The statement is in fact false: $\lim\limits_{x\to{-5}}{{\left(x + 5\right)}^{3} \cos\left(-\frac{19}{{\left(x + 5\right)}^{2}}\right)}\neq0$.}
\choice{The cosine factor decreases to $0$ faster than the polynomial.}
\choice[correct]{The cosine factor is bounded between $-1$ and $1$, so the polynomial forces the function to $0$.}
\choice{The cosine factor directly cancels out the polynomial factor.}
\end{multipleChoice}


What is the name of the theorem that applies to this problem? \qquad \\
The \underline{\answer{Squeeze}} Theorem
\end{problem}}%}

%%%%%%%%%%%%%%%%%%%%%




\latexProblemContent{
\begin{problem}

The limit as $x\to{14}$ of $f(x)={{\left(x - 14\right)} \cos\left(\frac{9}{{\left(x - 14\right)}^{2}}\right)}$ is $0$.  What is the reason why this is true?

\input{2311-Concept-Limit-0008.HELP.tex}

\begin{multipleChoice}
\choice{The statement is in fact false: $\lim\limits_{x\to{14}}{{\left(x - 14\right)} \cos\left(\frac{9}{{\left(x - 14\right)}^{2}}\right)}\neq0$.}
\choice{The cosine factor decreases to $0$ faster than the polynomial.}
\choice[correct]{The cosine factor is bounded between $-1$ and $1$, so the polynomial forces the function to $0$.}
\choice{The cosine factor directly cancels out the polynomial factor.}
\end{multipleChoice}


What is the name of the theorem that applies to this problem? \qquad \\
The \underline{\answer{Squeeze}} Theorem
\end{problem}}%}

%%%%%%%%%%%%%%%%%%%%%




\latexProblemContent{
\begin{problem}

The limit as $x\to{-15}$ of $f(x)={{\left(x + 15\right)}^{2} \cos\left(\frac{17}{x + 15}\right)}$ is $0$.  What is the reason why this is true?

\input{2311-Concept-Limit-0008.HELP.tex}

\begin{multipleChoice}
\choice{The statement is in fact false: $\lim\limits_{x\to{-15}}{{\left(x + 15\right)}^{2} \cos\left(\frac{17}{x + 15}\right)}\neq0$.}
\choice{The cosine factor decreases to $0$ faster than the polynomial.}
\choice[correct]{The cosine factor is bounded between $-1$ and $1$, so the polynomial forces the function to $0$.}
\choice{The cosine factor directly cancels out the polynomial factor.}
\end{multipleChoice}


What is the name of the theorem that applies to this problem? \qquad \\
The \underline{\answer{Squeeze}} Theorem
\end{problem}}%}

%%%%%%%%%%%%%%%%%%%%%




\latexProblemContent{
\begin{problem}

The limit as $x\to{-8}$ of $f(x)={{\left(x + 8\right)} \cos\left(\frac{8}{x + 8}\right)}$ is $0$.  What is the reason why this is true?

\input{2311-Concept-Limit-0008.HELP.tex}

\begin{multipleChoice}
\choice{The statement is in fact false: $\lim\limits_{x\to{-8}}{{\left(x + 8\right)} \cos\left(\frac{8}{x + 8}\right)}\neq0$.}
\choice{The cosine factor decreases to $0$ faster than the polynomial.}
\choice[correct]{The cosine factor is bounded between $-1$ and $1$, so the polynomial forces the function to $0$.}
\choice{The cosine factor directly cancels out the polynomial factor.}
\end{multipleChoice}


What is the name of the theorem that applies to this problem? \qquad \\
The \underline{\answer{Squeeze}} Theorem
\end{problem}}%}

%%%%%%%%%%%%%%%%%%%%%




\latexProblemContent{
\begin{problem}

The limit as $x\to{4}$ of $f(x)={{\left(x - 4\right)}^{3} \cos\left(\frac{4}{{\left(x - 4\right)}^{2}}\right)}$ is $0$.  What is the reason why this is true?

\input{2311-Concept-Limit-0008.HELP.tex}

\begin{multipleChoice}
\choice{The statement is in fact false: $\lim\limits_{x\to{4}}{{\left(x - 4\right)}^{3} \cos\left(\frac{4}{{\left(x - 4\right)}^{2}}\right)}\neq0$.}
\choice{The cosine factor decreases to $0$ faster than the polynomial.}
\choice[correct]{The cosine factor is bounded between $-1$ and $1$, so the polynomial forces the function to $0$.}
\choice{The cosine factor directly cancels out the polynomial factor.}
\end{multipleChoice}


What is the name of the theorem that applies to this problem? \qquad \\
The \underline{\answer{Squeeze}} Theorem
\end{problem}}%}

%%%%%%%%%%%%%%%%%%%%%




\latexProblemContent{
\begin{problem}

The limit as $x\to{15}$ of $f(x)={{\left(x - 15\right)} \cos\left(-\frac{12}{{\left(x - 15\right)}^{2}}\right)}$ is $0$.  What is the reason why this is true?

\input{2311-Concept-Limit-0008.HELP.tex}

\begin{multipleChoice}
\choice{The statement is in fact false: $\lim\limits_{x\to{15}}{{\left(x - 15\right)} \cos\left(-\frac{12}{{\left(x - 15\right)}^{2}}\right)}\neq0$.}
\choice{The cosine factor decreases to $0$ faster than the polynomial.}
\choice[correct]{The cosine factor is bounded between $-1$ and $1$, so the polynomial forces the function to $0$.}
\choice{The cosine factor directly cancels out the polynomial factor.}
\end{multipleChoice}


What is the name of the theorem that applies to this problem? \qquad \\
The \underline{\answer{Squeeze}} Theorem
\end{problem}}%}

%%%%%%%%%%%%%%%%%%%%%




\latexProblemContent{
\begin{problem}

The limit as $x\to{-4}$ of $f(x)={{\left(x + 4\right)} \cos\left(-\frac{9}{x + 4}\right)}$ is $0$.  What is the reason why this is true?

\input{2311-Concept-Limit-0008.HELP.tex}

\begin{multipleChoice}
\choice{The statement is in fact false: $\lim\limits_{x\to{-4}}{{\left(x + 4\right)} \cos\left(-\frac{9}{x + 4}\right)}\neq0$.}
\choice{The cosine factor decreases to $0$ faster than the polynomial.}
\choice[correct]{The cosine factor is bounded between $-1$ and $1$, so the polynomial forces the function to $0$.}
\choice{The cosine factor directly cancels out the polynomial factor.}
\end{multipleChoice}


What is the name of the theorem that applies to this problem? \qquad \\
The \underline{\answer{Squeeze}} Theorem
\end{problem}}%}

%%%%%%%%%%%%%%%%%%%%%




\latexProblemContent{
\begin{problem}

The limit as $x\to{2}$ of $f(x)={{\left(x - 2\right)} \cos\left(-\frac{12}{x - 2}\right)}$ is $0$.  What is the reason why this is true?

\input{2311-Concept-Limit-0008.HELP.tex}

\begin{multipleChoice}
\choice{The statement is in fact false: $\lim\limits_{x\to{2}}{{\left(x - 2\right)} \cos\left(-\frac{12}{x - 2}\right)}\neq0$.}
\choice{The cosine factor decreases to $0$ faster than the polynomial.}
\choice[correct]{The cosine factor is bounded between $-1$ and $1$, so the polynomial forces the function to $0$.}
\choice{The cosine factor directly cancels out the polynomial factor.}
\end{multipleChoice}


What is the name of the theorem that applies to this problem? \qquad \\
The \underline{\answer{Squeeze}} Theorem
\end{problem}}%}

%%%%%%%%%%%%%%%%%%%%%




\latexProblemContent{
\begin{problem}

The limit as $x\to{-11}$ of $f(x)={{\left(x + 11\right)}^{2} \cos\left(\frac{11}{x + 11}\right)}$ is $0$.  What is the reason why this is true?

\input{2311-Concept-Limit-0008.HELP.tex}

\begin{multipleChoice}
\choice{The statement is in fact false: $\lim\limits_{x\to{-11}}{{\left(x + 11\right)}^{2} \cos\left(\frac{11}{x + 11}\right)}\neq0$.}
\choice{The cosine factor decreases to $0$ faster than the polynomial.}
\choice[correct]{The cosine factor is bounded between $-1$ and $1$, so the polynomial forces the function to $0$.}
\choice{The cosine factor directly cancels out the polynomial factor.}
\end{multipleChoice}


What is the name of the theorem that applies to this problem? \qquad \\
The \underline{\answer{Squeeze}} Theorem
\end{problem}}%}

%%%%%%%%%%%%%%%%%%%%%




\latexProblemContent{
\begin{problem}

The limit as $x\to{3}$ of $f(x)={{\left(x - 3\right)} \cos\left(\frac{19}{{\left(x - 3\right)}^{2}}\right)}$ is $0$.  What is the reason why this is true?

\input{2311-Concept-Limit-0008.HELP.tex}

\begin{multipleChoice}
\choice{The statement is in fact false: $\lim\limits_{x\to{3}}{{\left(x - 3\right)} \cos\left(\frac{19}{{\left(x - 3\right)}^{2}}\right)}\neq0$.}
\choice{The cosine factor decreases to $0$ faster than the polynomial.}
\choice[correct]{The cosine factor is bounded between $-1$ and $1$, so the polynomial forces the function to $0$.}
\choice{The cosine factor directly cancels out the polynomial factor.}
\end{multipleChoice}


What is the name of the theorem that applies to this problem? \qquad \\
The \underline{\answer{Squeeze}} Theorem
\end{problem}}%}

%%%%%%%%%%%%%%%%%%%%%




\latexProblemContent{
\begin{problem}

The limit as $x\to{11}$ of $f(x)={{\left(x - 11\right)} \cos\left(-\frac{12}{{\left(x - 11\right)}^{2}}\right)}$ is $0$.  What is the reason why this is true?

\input{2311-Concept-Limit-0008.HELP.tex}

\begin{multipleChoice}
\choice{The statement is in fact false: $\lim\limits_{x\to{11}}{{\left(x - 11\right)} \cos\left(-\frac{12}{{\left(x - 11\right)}^{2}}\right)}\neq0$.}
\choice{The cosine factor decreases to $0$ faster than the polynomial.}
\choice[correct]{The cosine factor is bounded between $-1$ and $1$, so the polynomial forces the function to $0$.}
\choice{The cosine factor directly cancels out the polynomial factor.}
\end{multipleChoice}


What is the name of the theorem that applies to this problem? \qquad \\
The \underline{\answer{Squeeze}} Theorem
\end{problem}}%}

%%%%%%%%%%%%%%%%%%%%%




\latexProblemContent{
\begin{problem}

The limit as $x\to{-14}$ of $f(x)={{\left(x + 14\right)} \cos\left(-\frac{25}{{\left(x + 14\right)}^{2}}\right)}$ is $0$.  What is the reason why this is true?

\input{2311-Concept-Limit-0008.HELP.tex}

\begin{multipleChoice}
\choice{The statement is in fact false: $\lim\limits_{x\to{-14}}{{\left(x + 14\right)} \cos\left(-\frac{25}{{\left(x + 14\right)}^{2}}\right)}\neq0$.}
\choice{The cosine factor decreases to $0$ faster than the polynomial.}
\choice[correct]{The cosine factor is bounded between $-1$ and $1$, so the polynomial forces the function to $0$.}
\choice{The cosine factor directly cancels out the polynomial factor.}
\end{multipleChoice}


What is the name of the theorem that applies to this problem? \qquad \\
The \underline{\answer{Squeeze}} Theorem
\end{problem}}%}

%%%%%%%%%%%%%%%%%%%%%




\latexProblemContent{
\begin{problem}

The limit as $x\to{-9}$ of $f(x)={{\left(x + 9\right)} \cos\left(-\frac{6}{{\left(x + 9\right)}^{2}}\right)}$ is $0$.  What is the reason why this is true?

\input{2311-Concept-Limit-0008.HELP.tex}

\begin{multipleChoice}
\choice{The statement is in fact false: $\lim\limits_{x\to{-9}}{{\left(x + 9\right)} \cos\left(-\frac{6}{{\left(x + 9\right)}^{2}}\right)}\neq0$.}
\choice{The cosine factor decreases to $0$ faster than the polynomial.}
\choice[correct]{The cosine factor is bounded between $-1$ and $1$, so the polynomial forces the function to $0$.}
\choice{The cosine factor directly cancels out the polynomial factor.}
\end{multipleChoice}


What is the name of the theorem that applies to this problem? \qquad \\
The \underline{\answer{Squeeze}} Theorem
\end{problem}}%}

%%%%%%%%%%%%%%%%%%%%%




\latexProblemContent{
\begin{problem}

The limit as $x\to{-10}$ of $f(x)={{\left(x + 10\right)}^{3} \cos\left(-\frac{19}{x + 10}\right)}$ is $0$.  What is the reason why this is true?

\input{2311-Concept-Limit-0008.HELP.tex}

\begin{multipleChoice}
\choice{The statement is in fact false: $\lim\limits_{x\to{-10}}{{\left(x + 10\right)}^{3} \cos\left(-\frac{19}{x + 10}\right)}\neq0$.}
\choice{The cosine factor decreases to $0$ faster than the polynomial.}
\choice[correct]{The cosine factor is bounded between $-1$ and $1$, so the polynomial forces the function to $0$.}
\choice{The cosine factor directly cancels out the polynomial factor.}
\end{multipleChoice}


What is the name of the theorem that applies to this problem? \qquad \\
The \underline{\answer{Squeeze}} Theorem
\end{problem}}%}

%%%%%%%%%%%%%%%%%%%%%




\latexProblemContent{
\begin{problem}

The limit as $x\to{-11}$ of $f(x)={{\left(x + 11\right)}^{2} \cos\left(-\frac{18}{{\left(x + 11\right)}^{2}}\right)}$ is $0$.  What is the reason why this is true?

\input{2311-Concept-Limit-0008.HELP.tex}

\begin{multipleChoice}
\choice{The statement is in fact false: $\lim\limits_{x\to{-11}}{{\left(x + 11\right)}^{2} \cos\left(-\frac{18}{{\left(x + 11\right)}^{2}}\right)}\neq0$.}
\choice{The cosine factor decreases to $0$ faster than the polynomial.}
\choice[correct]{The cosine factor is bounded between $-1$ and $1$, so the polynomial forces the function to $0$.}
\choice{The cosine factor directly cancels out the polynomial factor.}
\end{multipleChoice}


What is the name of the theorem that applies to this problem? \qquad \\
The \underline{\answer{Squeeze}} Theorem
\end{problem}}%}

%%%%%%%%%%%%%%%%%%%%%




\latexProblemContent{
\begin{problem}

The limit as $x\to{12}$ of $f(x)={{\left(x - 12\right)}^{3} \cos\left(\frac{24}{{\left(x - 12\right)}^{2}}\right)}$ is $0$.  What is the reason why this is true?

\input{2311-Concept-Limit-0008.HELP.tex}

\begin{multipleChoice}
\choice{The statement is in fact false: $\lim\limits_{x\to{12}}{{\left(x - 12\right)}^{3} \cos\left(\frac{24}{{\left(x - 12\right)}^{2}}\right)}\neq0$.}
\choice{The cosine factor decreases to $0$ faster than the polynomial.}
\choice[correct]{The cosine factor is bounded between $-1$ and $1$, so the polynomial forces the function to $0$.}
\choice{The cosine factor directly cancels out the polynomial factor.}
\end{multipleChoice}


What is the name of the theorem that applies to this problem? \qquad \\
The \underline{\answer{Squeeze}} Theorem
\end{problem}}%}

%%%%%%%%%%%%%%%%%%%%%




\latexProblemContent{
\begin{problem}

The limit as $x\to{-9}$ of $f(x)={{\left(x + 9\right)} \cos\left(\frac{12}{x + 9}\right)}$ is $0$.  What is the reason why this is true?

\input{2311-Concept-Limit-0008.HELP.tex}

\begin{multipleChoice}
\choice{The statement is in fact false: $\lim\limits_{x\to{-9}}{{\left(x + 9\right)} \cos\left(\frac{12}{x + 9}\right)}\neq0$.}
\choice{The cosine factor decreases to $0$ faster than the polynomial.}
\choice[correct]{The cosine factor is bounded between $-1$ and $1$, so the polynomial forces the function to $0$.}
\choice{The cosine factor directly cancels out the polynomial factor.}
\end{multipleChoice}


What is the name of the theorem that applies to this problem? \qquad \\
The \underline{\answer{Squeeze}} Theorem
\end{problem}}%}

%%%%%%%%%%%%%%%%%%%%%




\latexProblemContent{
\begin{problem}

The limit as $x\to{10}$ of $f(x)={{\left(x - 10\right)}^{3} \cos\left(-\frac{15}{x - 10}\right)}$ is $0$.  What is the reason why this is true?

\input{2311-Concept-Limit-0008.HELP.tex}

\begin{multipleChoice}
\choice{The statement is in fact false: $\lim\limits_{x\to{10}}{{\left(x - 10\right)}^{3} \cos\left(-\frac{15}{x - 10}\right)}\neq0$.}
\choice{The cosine factor decreases to $0$ faster than the polynomial.}
\choice[correct]{The cosine factor is bounded between $-1$ and $1$, so the polynomial forces the function to $0$.}
\choice{The cosine factor directly cancels out the polynomial factor.}
\end{multipleChoice}


What is the name of the theorem that applies to this problem? \qquad \\
The \underline{\answer{Squeeze}} Theorem
\end{problem}}%}

%%%%%%%%%%%%%%%%%%%%%




\latexProblemContent{
\begin{problem}

The limit as $x\to{-9}$ of $f(x)={{\left(x + 9\right)}^{3} \cos\left(\frac{9}{{\left(x + 9\right)}^{2}}\right)}$ is $0$.  What is the reason why this is true?

\input{2311-Concept-Limit-0008.HELP.tex}

\begin{multipleChoice}
\choice{The statement is in fact false: $\lim\limits_{x\to{-9}}{{\left(x + 9\right)}^{3} \cos\left(\frac{9}{{\left(x + 9\right)}^{2}}\right)}\neq0$.}
\choice{The cosine factor decreases to $0$ faster than the polynomial.}
\choice[correct]{The cosine factor is bounded between $-1$ and $1$, so the polynomial forces the function to $0$.}
\choice{The cosine factor directly cancels out the polynomial factor.}
\end{multipleChoice}


What is the name of the theorem that applies to this problem? \qquad \\
The \underline{\answer{Squeeze}} Theorem
\end{problem}}%}

%%%%%%%%%%%%%%%%%%%%%




\latexProblemContent{
\begin{problem}

The limit as $x\to{-10}$ of $f(x)={{\left(x + 10\right)} \cos\left(\frac{11}{{\left(x + 10\right)}^{2}}\right)}$ is $0$.  What is the reason why this is true?

\input{2311-Concept-Limit-0008.HELP.tex}

\begin{multipleChoice}
\choice{The statement is in fact false: $\lim\limits_{x\to{-10}}{{\left(x + 10\right)} \cos\left(\frac{11}{{\left(x + 10\right)}^{2}}\right)}\neq0$.}
\choice{The cosine factor decreases to $0$ faster than the polynomial.}
\choice[correct]{The cosine factor is bounded between $-1$ and $1$, so the polynomial forces the function to $0$.}
\choice{The cosine factor directly cancels out the polynomial factor.}
\end{multipleChoice}


What is the name of the theorem that applies to this problem? \qquad \\
The \underline{\answer{Squeeze}} Theorem
\end{problem}}%}

%%%%%%%%%%%%%%%%%%%%%




\latexProblemContent{
\begin{problem}

The limit as $x\to{-4}$ of $f(x)={{\left(x + 4\right)} \cos\left(-\frac{13}{x + 4}\right)}$ is $0$.  What is the reason why this is true?

\input{2311-Concept-Limit-0008.HELP.tex}

\begin{multipleChoice}
\choice{The statement is in fact false: $\lim\limits_{x\to{-4}}{{\left(x + 4\right)} \cos\left(-\frac{13}{x + 4}\right)}\neq0$.}
\choice{The cosine factor decreases to $0$ faster than the polynomial.}
\choice[correct]{The cosine factor is bounded between $-1$ and $1$, so the polynomial forces the function to $0$.}
\choice{The cosine factor directly cancels out the polynomial factor.}
\end{multipleChoice}


What is the name of the theorem that applies to this problem? \qquad \\
The \underline{\answer{Squeeze}} Theorem
\end{problem}}%}

%%%%%%%%%%%%%%%%%%%%%




\latexProblemContent{
\begin{problem}

The limit as $x\to{9}$ of $f(x)={{\left(x - 9\right)} \cos\left(\frac{18}{{\left(x - 9\right)}^{2}}\right)}$ is $0$.  What is the reason why this is true?

\input{2311-Concept-Limit-0008.HELP.tex}

\begin{multipleChoice}
\choice{The statement is in fact false: $\lim\limits_{x\to{9}}{{\left(x - 9\right)} \cos\left(\frac{18}{{\left(x - 9\right)}^{2}}\right)}\neq0$.}
\choice{The cosine factor decreases to $0$ faster than the polynomial.}
\choice[correct]{The cosine factor is bounded between $-1$ and $1$, so the polynomial forces the function to $0$.}
\choice{The cosine factor directly cancels out the polynomial factor.}
\end{multipleChoice}


What is the name of the theorem that applies to this problem? \qquad \\
The \underline{\answer{Squeeze}} Theorem
\end{problem}}%}

%%%%%%%%%%%%%%%%%%%%%




\latexProblemContent{
\begin{problem}

The limit as $x\to{-7}$ of $f(x)={{\left(x + 7\right)}^{3} \cos\left(-\frac{24}{x + 7}\right)}$ is $0$.  What is the reason why this is true?

\input{2311-Concept-Limit-0008.HELP.tex}

\begin{multipleChoice}
\choice{The statement is in fact false: $\lim\limits_{x\to{-7}}{{\left(x + 7\right)}^{3} \cos\left(-\frac{24}{x + 7}\right)}\neq0$.}
\choice{The cosine factor decreases to $0$ faster than the polynomial.}
\choice[correct]{The cosine factor is bounded between $-1$ and $1$, so the polynomial forces the function to $0$.}
\choice{The cosine factor directly cancels out the polynomial factor.}
\end{multipleChoice}


What is the name of the theorem that applies to this problem? \qquad \\
The \underline{\answer{Squeeze}} Theorem
\end{problem}}%}

%%%%%%%%%%%%%%%%%%%%%




\latexProblemContent{
\begin{problem}

The limit as $x\to{4}$ of $f(x)={{\left(x - 4\right)} \cos\left(\frac{14}{{\left(x - 4\right)}^{2}}\right)}$ is $0$.  What is the reason why this is true?

\input{2311-Concept-Limit-0008.HELP.tex}

\begin{multipleChoice}
\choice{The statement is in fact false: $\lim\limits_{x\to{4}}{{\left(x - 4\right)} \cos\left(\frac{14}{{\left(x - 4\right)}^{2}}\right)}\neq0$.}
\choice{The cosine factor decreases to $0$ faster than the polynomial.}
\choice[correct]{The cosine factor is bounded between $-1$ and $1$, so the polynomial forces the function to $0$.}
\choice{The cosine factor directly cancels out the polynomial factor.}
\end{multipleChoice}


What is the name of the theorem that applies to this problem? \qquad \\
The \underline{\answer{Squeeze}} Theorem
\end{problem}}%}

%%%%%%%%%%%%%%%%%%%%%




\latexProblemContent{
\begin{problem}

The limit as $x\to{-3}$ of $f(x)={{\left(x + 3\right)}^{2} \cos\left(\frac{18}{{\left(x + 3\right)}^{2}}\right)}$ is $0$.  What is the reason why this is true?

\input{2311-Concept-Limit-0008.HELP.tex}

\begin{multipleChoice}
\choice{The statement is in fact false: $\lim\limits_{x\to{-3}}{{\left(x + 3\right)}^{2} \cos\left(\frac{18}{{\left(x + 3\right)}^{2}}\right)}\neq0$.}
\choice{The cosine factor decreases to $0$ faster than the polynomial.}
\choice[correct]{The cosine factor is bounded between $-1$ and $1$, so the polynomial forces the function to $0$.}
\choice{The cosine factor directly cancels out the polynomial factor.}
\end{multipleChoice}


What is the name of the theorem that applies to this problem? \qquad \\
The \underline{\answer{Squeeze}} Theorem
\end{problem}}%}

%%%%%%%%%%%%%%%%%%%%%




\latexProblemContent{
\begin{problem}

The limit as $x\to{8}$ of $f(x)={{\left(x - 8\right)} \cos\left(\frac{14}{x - 8}\right)}$ is $0$.  What is the reason why this is true?

\input{2311-Concept-Limit-0008.HELP.tex}

\begin{multipleChoice}
\choice{The statement is in fact false: $\lim\limits_{x\to{8}}{{\left(x - 8\right)} \cos\left(\frac{14}{x - 8}\right)}\neq0$.}
\choice{The cosine factor decreases to $0$ faster than the polynomial.}
\choice[correct]{The cosine factor is bounded between $-1$ and $1$, so the polynomial forces the function to $0$.}
\choice{The cosine factor directly cancels out the polynomial factor.}
\end{multipleChoice}


What is the name of the theorem that applies to this problem? \qquad \\
The \underline{\answer{Squeeze}} Theorem
\end{problem}}%}

%%%%%%%%%%%%%%%%%%%%%




\latexProblemContent{
\begin{problem}

The limit as $x\to{11}$ of $f(x)={{\left(x - 11\right)}^{2} \cos\left(-\frac{8}{x - 11}\right)}$ is $0$.  What is the reason why this is true?

\input{2311-Concept-Limit-0008.HELP.tex}

\begin{multipleChoice}
\choice{The statement is in fact false: $\lim\limits_{x\to{11}}{{\left(x - 11\right)}^{2} \cos\left(-\frac{8}{x - 11}\right)}\neq0$.}
\choice{The cosine factor decreases to $0$ faster than the polynomial.}
\choice[correct]{The cosine factor is bounded between $-1$ and $1$, so the polynomial forces the function to $0$.}
\choice{The cosine factor directly cancels out the polynomial factor.}
\end{multipleChoice}


What is the name of the theorem that applies to this problem? \qquad \\
The \underline{\answer{Squeeze}} Theorem
\end{problem}}%}

%%%%%%%%%%%%%%%%%%%%%




\latexProblemContent{
\begin{problem}

The limit as $x\to{11}$ of $f(x)={{\left(x - 11\right)} \cos\left(-\frac{21}{x - 11}\right)}$ is $0$.  What is the reason why this is true?

\input{2311-Concept-Limit-0008.HELP.tex}

\begin{multipleChoice}
\choice{The statement is in fact false: $\lim\limits_{x\to{11}}{{\left(x - 11\right)} \cos\left(-\frac{21}{x - 11}\right)}\neq0$.}
\choice{The cosine factor decreases to $0$ faster than the polynomial.}
\choice[correct]{The cosine factor is bounded between $-1$ and $1$, so the polynomial forces the function to $0$.}
\choice{The cosine factor directly cancels out the polynomial factor.}
\end{multipleChoice}


What is the name of the theorem that applies to this problem? \qquad \\
The \underline{\answer{Squeeze}} Theorem
\end{problem}}%}

%%%%%%%%%%%%%%%%%%%%%




\latexProblemContent{
\begin{problem}

The limit as $x\to{-11}$ of $f(x)={{\left(x + 11\right)}^{3} \cos\left(-\frac{3}{x + 11}\right)}$ is $0$.  What is the reason why this is true?

\input{2311-Concept-Limit-0008.HELP.tex}

\begin{multipleChoice}
\choice{The statement is in fact false: $\lim\limits_{x\to{-11}}{{\left(x + 11\right)}^{3} \cos\left(-\frac{3}{x + 11}\right)}\neq0$.}
\choice{The cosine factor decreases to $0$ faster than the polynomial.}
\choice[correct]{The cosine factor is bounded between $-1$ and $1$, so the polynomial forces the function to $0$.}
\choice{The cosine factor directly cancels out the polynomial factor.}
\end{multipleChoice}


What is the name of the theorem that applies to this problem? \qquad \\
The \underline{\answer{Squeeze}} Theorem
\end{problem}}%}

%%%%%%%%%%%%%%%%%%%%%




\latexProblemContent{
\begin{problem}

The limit as $x\to{9}$ of $f(x)={{\left(x - 9\right)}^{3} \cos\left(-\frac{6}{x - 9}\right)}$ is $0$.  What is the reason why this is true?

\input{2311-Concept-Limit-0008.HELP.tex}

\begin{multipleChoice}
\choice{The statement is in fact false: $\lim\limits_{x\to{9}}{{\left(x - 9\right)}^{3} \cos\left(-\frac{6}{x - 9}\right)}\neq0$.}
\choice{The cosine factor decreases to $0$ faster than the polynomial.}
\choice[correct]{The cosine factor is bounded between $-1$ and $1$, so the polynomial forces the function to $0$.}
\choice{The cosine factor directly cancels out the polynomial factor.}
\end{multipleChoice}


What is the name of the theorem that applies to this problem? \qquad \\
The \underline{\answer{Squeeze}} Theorem
\end{problem}}%}

%%%%%%%%%%%%%%%%%%%%%




\latexProblemContent{
\begin{problem}

The limit as $x\to{10}$ of $f(x)={{\left(x - 10\right)}^{3} \cos\left(\frac{8}{x - 10}\right)}$ is $0$.  What is the reason why this is true?

\input{2311-Concept-Limit-0008.HELP.tex}

\begin{multipleChoice}
\choice{The statement is in fact false: $\lim\limits_{x\to{10}}{{\left(x - 10\right)}^{3} \cos\left(\frac{8}{x - 10}\right)}\neq0$.}
\choice{The cosine factor decreases to $0$ faster than the polynomial.}
\choice[correct]{The cosine factor is bounded between $-1$ and $1$, so the polynomial forces the function to $0$.}
\choice{The cosine factor directly cancels out the polynomial factor.}
\end{multipleChoice}


What is the name of the theorem that applies to this problem? \qquad \\
The \underline{\answer{Squeeze}} Theorem
\end{problem}}%}

%%%%%%%%%%%%%%%%%%%%%




\latexProblemContent{
\begin{problem}

The limit as $x\to{10}$ of $f(x)={{\left(x - 10\right)} \cos\left(\frac{10}{{\left(x - 10\right)}^{2}}\right)}$ is $0$.  What is the reason why this is true?

\input{2311-Concept-Limit-0008.HELP.tex}

\begin{multipleChoice}
\choice{The statement is in fact false: $\lim\limits_{x\to{10}}{{\left(x - 10\right)} \cos\left(\frac{10}{{\left(x - 10\right)}^{2}}\right)}\neq0$.}
\choice{The cosine factor decreases to $0$ faster than the polynomial.}
\choice[correct]{The cosine factor is bounded between $-1$ and $1$, so the polynomial forces the function to $0$.}
\choice{The cosine factor directly cancels out the polynomial factor.}
\end{multipleChoice}


What is the name of the theorem that applies to this problem? \qquad \\
The \underline{\answer{Squeeze}} Theorem
\end{problem}}%}

%%%%%%%%%%%%%%%%%%%%%




\latexProblemContent{
\begin{problem}

The limit as $x\to{9}$ of $f(x)={{\left(x - 9\right)} \cos\left(-\frac{6}{{\left(x - 9\right)}^{2}}\right)}$ is $0$.  What is the reason why this is true?

\input{2311-Concept-Limit-0008.HELP.tex}

\begin{multipleChoice}
\choice{The statement is in fact false: $\lim\limits_{x\to{9}}{{\left(x - 9\right)} \cos\left(-\frac{6}{{\left(x - 9\right)}^{2}}\right)}\neq0$.}
\choice{The cosine factor decreases to $0$ faster than the polynomial.}
\choice[correct]{The cosine factor is bounded between $-1$ and $1$, so the polynomial forces the function to $0$.}
\choice{The cosine factor directly cancels out the polynomial factor.}
\end{multipleChoice}


What is the name of the theorem that applies to this problem? \qquad \\
The \underline{\answer{Squeeze}} Theorem
\end{problem}}%}

%%%%%%%%%%%%%%%%%%%%%




\latexProblemContent{
\begin{problem}

The limit as $x\to{-4}$ of $f(x)={{\left(x + 4\right)}^{2} \cos\left(-\frac{8}{x + 4}\right)}$ is $0$.  What is the reason why this is true?

\input{2311-Concept-Limit-0008.HELP.tex}

\begin{multipleChoice}
\choice{The statement is in fact false: $\lim\limits_{x\to{-4}}{{\left(x + 4\right)}^{2} \cos\left(-\frac{8}{x + 4}\right)}\neq0$.}
\choice{The cosine factor decreases to $0$ faster than the polynomial.}
\choice[correct]{The cosine factor is bounded between $-1$ and $1$, so the polynomial forces the function to $0$.}
\choice{The cosine factor directly cancels out the polynomial factor.}
\end{multipleChoice}


What is the name of the theorem that applies to this problem? \qquad \\
The \underline{\answer{Squeeze}} Theorem
\end{problem}}%}

%%%%%%%%%%%%%%%%%%%%%




\latexProblemContent{
\begin{problem}

The limit as $x\to{-8}$ of $f(x)={{\left(x + 8\right)}^{2} \cos\left(-\frac{15}{{\left(x + 8\right)}^{2}}\right)}$ is $0$.  What is the reason why this is true?

\input{2311-Concept-Limit-0008.HELP.tex}

\begin{multipleChoice}
\choice{The statement is in fact false: $\lim\limits_{x\to{-8}}{{\left(x + 8\right)}^{2} \cos\left(-\frac{15}{{\left(x + 8\right)}^{2}}\right)}\neq0$.}
\choice{The cosine factor decreases to $0$ faster than the polynomial.}
\choice[correct]{The cosine factor is bounded between $-1$ and $1$, so the polynomial forces the function to $0$.}
\choice{The cosine factor directly cancels out the polynomial factor.}
\end{multipleChoice}


What is the name of the theorem that applies to this problem? \qquad \\
The \underline{\answer{Squeeze}} Theorem
\end{problem}}%}

%%%%%%%%%%%%%%%%%%%%%




\latexProblemContent{
\begin{problem}

The limit as $x\to{8}$ of $f(x)={{\left(x - 8\right)}^{3} \cos\left(-\frac{17}{{\left(x - 8\right)}^{2}}\right)}$ is $0$.  What is the reason why this is true?

\input{2311-Concept-Limit-0008.HELP.tex}

\begin{multipleChoice}
\choice{The statement is in fact false: $\lim\limits_{x\to{8}}{{\left(x - 8\right)}^{3} \cos\left(-\frac{17}{{\left(x - 8\right)}^{2}}\right)}\neq0$.}
\choice{The cosine factor decreases to $0$ faster than the polynomial.}
\choice[correct]{The cosine factor is bounded between $-1$ and $1$, so the polynomial forces the function to $0$.}
\choice{The cosine factor directly cancels out the polynomial factor.}
\end{multipleChoice}


What is the name of the theorem that applies to this problem? \qquad \\
The \underline{\answer{Squeeze}} Theorem
\end{problem}}%}

%%%%%%%%%%%%%%%%%%%%%




\latexProblemContent{
\begin{problem}

The limit as $x\to{5}$ of $f(x)={{\left(x - 5\right)} \cos\left(\frac{21}{{\left(x - 5\right)}^{2}}\right)}$ is $0$.  What is the reason why this is true?

\input{2311-Concept-Limit-0008.HELP.tex}

\begin{multipleChoice}
\choice{The statement is in fact false: $\lim\limits_{x\to{5}}{{\left(x - 5\right)} \cos\left(\frac{21}{{\left(x - 5\right)}^{2}}\right)}\neq0$.}
\choice{The cosine factor decreases to $0$ faster than the polynomial.}
\choice[correct]{The cosine factor is bounded between $-1$ and $1$, so the polynomial forces the function to $0$.}
\choice{The cosine factor directly cancels out the polynomial factor.}
\end{multipleChoice}


What is the name of the theorem that applies to this problem? \qquad \\
The \underline{\answer{Squeeze}} Theorem
\end{problem}}%}

%%%%%%%%%%%%%%%%%%%%%




\latexProblemContent{
\begin{problem}

The limit as $x\to{-15}$ of $f(x)={{\left(x + 15\right)}^{2} \cos\left(-\frac{1}{x + 15}\right)}$ is $0$.  What is the reason why this is true?

\input{2311-Concept-Limit-0008.HELP.tex}

\begin{multipleChoice}
\choice{The statement is in fact false: $\lim\limits_{x\to{-15}}{{\left(x + 15\right)}^{2} \cos\left(-\frac{1}{x + 15}\right)}\neq0$.}
\choice{The cosine factor decreases to $0$ faster than the polynomial.}
\choice[correct]{The cosine factor is bounded between $-1$ and $1$, so the polynomial forces the function to $0$.}
\choice{The cosine factor directly cancels out the polynomial factor.}
\end{multipleChoice}


What is the name of the theorem that applies to this problem? \qquad \\
The \underline{\answer{Squeeze}} Theorem
\end{problem}}%}

%%%%%%%%%%%%%%%%%%%%%




\latexProblemContent{
\begin{problem}

The limit as $x\to{0}$ of $f(x)={x^{2} \cos\left(-\frac{20}{x^{2}}\right)}$ is $0$.  What is the reason why this is true?

\input{2311-Concept-Limit-0008.HELP.tex}

\begin{multipleChoice}
\choice{The statement is in fact false: $\lim\limits_{x\to{0}}{x^{2} \cos\left(-\frac{20}{x^{2}}\right)}\neq0$.}
\choice{The cosine factor decreases to $0$ faster than the polynomial.}
\choice[correct]{The cosine factor is bounded between $-1$ and $1$, so the polynomial forces the function to $0$.}
\choice{The cosine factor directly cancels out the polynomial factor.}
\end{multipleChoice}


What is the name of the theorem that applies to this problem? \qquad \\
The \underline{\answer{Squeeze}} Theorem
\end{problem}}%}

%%%%%%%%%%%%%%%%%%%%%




\latexProblemContent{
\begin{problem}

The limit as $x\to{6}$ of $f(x)={{\left(x - 6\right)}^{2} \cos\left(-\frac{11}{{\left(x - 6\right)}^{2}}\right)}$ is $0$.  What is the reason why this is true?

\input{2311-Concept-Limit-0008.HELP.tex}

\begin{multipleChoice}
\choice{The statement is in fact false: $\lim\limits_{x\to{6}}{{\left(x - 6\right)}^{2} \cos\left(-\frac{11}{{\left(x - 6\right)}^{2}}\right)}\neq0$.}
\choice{The cosine factor decreases to $0$ faster than the polynomial.}
\choice[correct]{The cosine factor is bounded between $-1$ and $1$, so the polynomial forces the function to $0$.}
\choice{The cosine factor directly cancels out the polynomial factor.}
\end{multipleChoice}


What is the name of the theorem that applies to this problem? \qquad \\
The \underline{\answer{Squeeze}} Theorem
\end{problem}}%}

%%%%%%%%%%%%%%%%%%%%%




\latexProblemContent{
\begin{problem}

The limit as $x\to{-2}$ of $f(x)={{\left(x + 2\right)}^{2} \cos\left(\frac{5}{x + 2}\right)}$ is $0$.  What is the reason why this is true?

\input{2311-Concept-Limit-0008.HELP.tex}

\begin{multipleChoice}
\choice{The statement is in fact false: $\lim\limits_{x\to{-2}}{{\left(x + 2\right)}^{2} \cos\left(\frac{5}{x + 2}\right)}\neq0$.}
\choice{The cosine factor decreases to $0$ faster than the polynomial.}
\choice[correct]{The cosine factor is bounded between $-1$ and $1$, so the polynomial forces the function to $0$.}
\choice{The cosine factor directly cancels out the polynomial factor.}
\end{multipleChoice}


What is the name of the theorem that applies to this problem? \qquad \\
The \underline{\answer{Squeeze}} Theorem
\end{problem}}%}

%%%%%%%%%%%%%%%%%%%%%




\latexProblemContent{
\begin{problem}

The limit as $x\to{9}$ of $f(x)={{\left(x - 9\right)} \cos\left(-\frac{23}{x - 9}\right)}$ is $0$.  What is the reason why this is true?

\input{2311-Concept-Limit-0008.HELP.tex}

\begin{multipleChoice}
\choice{The statement is in fact false: $\lim\limits_{x\to{9}}{{\left(x - 9\right)} \cos\left(-\frac{23}{x - 9}\right)}\neq0$.}
\choice{The cosine factor decreases to $0$ faster than the polynomial.}
\choice[correct]{The cosine factor is bounded between $-1$ and $1$, so the polynomial forces the function to $0$.}
\choice{The cosine factor directly cancels out the polynomial factor.}
\end{multipleChoice}


What is the name of the theorem that applies to this problem? \qquad \\
The \underline{\answer{Squeeze}} Theorem
\end{problem}}%}

%%%%%%%%%%%%%%%%%%%%%




\latexProblemContent{
\begin{problem}

The limit as $x\to{-4}$ of $f(x)={{\left(x + 4\right)}^{3} \cos\left(-\frac{2}{{\left(x + 4\right)}^{2}}\right)}$ is $0$.  What is the reason why this is true?

\input{2311-Concept-Limit-0008.HELP.tex}

\begin{multipleChoice}
\choice{The statement is in fact false: $\lim\limits_{x\to{-4}}{{\left(x + 4\right)}^{3} \cos\left(-\frac{2}{{\left(x + 4\right)}^{2}}\right)}\neq0$.}
\choice{The cosine factor decreases to $0$ faster than the polynomial.}
\choice[correct]{The cosine factor is bounded between $-1$ and $1$, so the polynomial forces the function to $0$.}
\choice{The cosine factor directly cancels out the polynomial factor.}
\end{multipleChoice}


What is the name of the theorem that applies to this problem? \qquad \\
The \underline{\answer{Squeeze}} Theorem
\end{problem}}%}

%%%%%%%%%%%%%%%%%%%%%




\latexProblemContent{
\begin{problem}

The limit as $x\to{0}$ of $f(x)={x^{3} \cos\left(-\frac{12}{x}\right)}$ is $0$.  What is the reason why this is true?

\input{2311-Concept-Limit-0008.HELP.tex}

\begin{multipleChoice}
\choice{The statement is in fact false: $\lim\limits_{x\to{0}}{x^{3} \cos\left(-\frac{12}{x}\right)}\neq0$.}
\choice{The cosine factor decreases to $0$ faster than the polynomial.}
\choice[correct]{The cosine factor is bounded between $-1$ and $1$, so the polynomial forces the function to $0$.}
\choice{The cosine factor directly cancels out the polynomial factor.}
\end{multipleChoice}


What is the name of the theorem that applies to this problem? \qquad \\
The \underline{\answer{Squeeze}} Theorem
\end{problem}}%}

%%%%%%%%%%%%%%%%%%%%%




\latexProblemContent{
\begin{problem}

The limit as $x\to{9}$ of $f(x)={{\left(x - 9\right)}^{3} \cos\left(-\frac{21}{x - 9}\right)}$ is $0$.  What is the reason why this is true?

\input{2311-Concept-Limit-0008.HELP.tex}

\begin{multipleChoice}
\choice{The statement is in fact false: $\lim\limits_{x\to{9}}{{\left(x - 9\right)}^{3} \cos\left(-\frac{21}{x - 9}\right)}\neq0$.}
\choice{The cosine factor decreases to $0$ faster than the polynomial.}
\choice[correct]{The cosine factor is bounded between $-1$ and $1$, so the polynomial forces the function to $0$.}
\choice{The cosine factor directly cancels out the polynomial factor.}
\end{multipleChoice}


What is the name of the theorem that applies to this problem? \qquad \\
The \underline{\answer{Squeeze}} Theorem
\end{problem}}%}

%%%%%%%%%%%%%%%%%%%%%




\latexProblemContent{
\begin{problem}

The limit as $x\to{-9}$ of $f(x)={{\left(x + 9\right)}^{2} \cos\left(-\frac{5}{x + 9}\right)}$ is $0$.  What is the reason why this is true?

\input{2311-Concept-Limit-0008.HELP.tex}

\begin{multipleChoice}
\choice{The statement is in fact false: $\lim\limits_{x\to{-9}}{{\left(x + 9\right)}^{2} \cos\left(-\frac{5}{x + 9}\right)}\neq0$.}
\choice{The cosine factor decreases to $0$ faster than the polynomial.}
\choice[correct]{The cosine factor is bounded between $-1$ and $1$, so the polynomial forces the function to $0$.}
\choice{The cosine factor directly cancels out the polynomial factor.}
\end{multipleChoice}


What is the name of the theorem that applies to this problem? \qquad \\
The \underline{\answer{Squeeze}} Theorem
\end{problem}}%}

%%%%%%%%%%%%%%%%%%%%%




\latexProblemContent{
\begin{problem}

The limit as $x\to{1}$ of $f(x)={{\left(x - 1\right)}^{3} \cos\left(\frac{13}{{\left(x - 1\right)}^{2}}\right)}$ is $0$.  What is the reason why this is true?

\input{2311-Concept-Limit-0008.HELP.tex}

\begin{multipleChoice}
\choice{The statement is in fact false: $\lim\limits_{x\to{1}}{{\left(x - 1\right)}^{3} \cos\left(\frac{13}{{\left(x - 1\right)}^{2}}\right)}\neq0$.}
\choice{The cosine factor decreases to $0$ faster than the polynomial.}
\choice[correct]{The cosine factor is bounded between $-1$ and $1$, so the polynomial forces the function to $0$.}
\choice{The cosine factor directly cancels out the polynomial factor.}
\end{multipleChoice}


What is the name of the theorem that applies to this problem? \qquad \\
The \underline{\answer{Squeeze}} Theorem
\end{problem}}%}

%%%%%%%%%%%%%%%%%%%%%




\latexProblemContent{
\begin{problem}

The limit as $x\to{-10}$ of $f(x)={{\left(x + 10\right)}^{3} \cos\left(\frac{16}{{\left(x + 10\right)}^{2}}\right)}$ is $0$.  What is the reason why this is true?

\input{2311-Concept-Limit-0008.HELP.tex}

\begin{multipleChoice}
\choice{The statement is in fact false: $\lim\limits_{x\to{-10}}{{\left(x + 10\right)}^{3} \cos\left(\frac{16}{{\left(x + 10\right)}^{2}}\right)}\neq0$.}
\choice{The cosine factor decreases to $0$ faster than the polynomial.}
\choice[correct]{The cosine factor is bounded between $-1$ and $1$, so the polynomial forces the function to $0$.}
\choice{The cosine factor directly cancels out the polynomial factor.}
\end{multipleChoice}


What is the name of the theorem that applies to this problem? \qquad \\
The \underline{\answer{Squeeze}} Theorem
\end{problem}}%}

%%%%%%%%%%%%%%%%%%%%%




\latexProblemContent{
\begin{problem}

The limit as $x\to{10}$ of $f(x)={{\left(x - 10\right)} \cos\left(\frac{25}{{\left(x - 10\right)}^{2}}\right)}$ is $0$.  What is the reason why this is true?

\input{2311-Concept-Limit-0008.HELP.tex}

\begin{multipleChoice}
\choice{The statement is in fact false: $\lim\limits_{x\to{10}}{{\left(x - 10\right)} \cos\left(\frac{25}{{\left(x - 10\right)}^{2}}\right)}\neq0$.}
\choice{The cosine factor decreases to $0$ faster than the polynomial.}
\choice[correct]{The cosine factor is bounded between $-1$ and $1$, so the polynomial forces the function to $0$.}
\choice{The cosine factor directly cancels out the polynomial factor.}
\end{multipleChoice}


What is the name of the theorem that applies to this problem? \qquad \\
The \underline{\answer{Squeeze}} Theorem
\end{problem}}%}

%%%%%%%%%%%%%%%%%%%%%




\latexProblemContent{
\begin{problem}

The limit as $x\to{-5}$ of $f(x)={{\left(x + 5\right)}^{2} \cos\left(\frac{3}{{\left(x + 5\right)}^{2}}\right)}$ is $0$.  What is the reason why this is true?

\input{2311-Concept-Limit-0008.HELP.tex}

\begin{multipleChoice}
\choice{The statement is in fact false: $\lim\limits_{x\to{-5}}{{\left(x + 5\right)}^{2} \cos\left(\frac{3}{{\left(x + 5\right)}^{2}}\right)}\neq0$.}
\choice{The cosine factor decreases to $0$ faster than the polynomial.}
\choice[correct]{The cosine factor is bounded between $-1$ and $1$, so the polynomial forces the function to $0$.}
\choice{The cosine factor directly cancels out the polynomial factor.}
\end{multipleChoice}


What is the name of the theorem that applies to this problem? \qquad \\
The \underline{\answer{Squeeze}} Theorem
\end{problem}}%}

%%%%%%%%%%%%%%%%%%%%%




\latexProblemContent{
\begin{problem}

The limit as $x\to{-9}$ of $f(x)={{\left(x + 9\right)}^{2} \cos\left(-\frac{3}{x + 9}\right)}$ is $0$.  What is the reason why this is true?

\input{2311-Concept-Limit-0008.HELP.tex}

\begin{multipleChoice}
\choice{The statement is in fact false: $\lim\limits_{x\to{-9}}{{\left(x + 9\right)}^{2} \cos\left(-\frac{3}{x + 9}\right)}\neq0$.}
\choice{The cosine factor decreases to $0$ faster than the polynomial.}
\choice[correct]{The cosine factor is bounded between $-1$ and $1$, so the polynomial forces the function to $0$.}
\choice{The cosine factor directly cancels out the polynomial factor.}
\end{multipleChoice}


What is the name of the theorem that applies to this problem? \qquad \\
The \underline{\answer{Squeeze}} Theorem
\end{problem}}%}

%%%%%%%%%%%%%%%%%%%%%




\latexProblemContent{
\begin{problem}

The limit as $x\to{-10}$ of $f(x)={{\left(x + 10\right)}^{3} \cos\left(\frac{17}{{\left(x + 10\right)}^{2}}\right)}$ is $0$.  What is the reason why this is true?

\input{2311-Concept-Limit-0008.HELP.tex}

\begin{multipleChoice}
\choice{The statement is in fact false: $\lim\limits_{x\to{-10}}{{\left(x + 10\right)}^{3} \cos\left(\frac{17}{{\left(x + 10\right)}^{2}}\right)}\neq0$.}
\choice{The cosine factor decreases to $0$ faster than the polynomial.}
\choice[correct]{The cosine factor is bounded between $-1$ and $1$, so the polynomial forces the function to $0$.}
\choice{The cosine factor directly cancels out the polynomial factor.}
\end{multipleChoice}


What is the name of the theorem that applies to this problem? \qquad \\
The \underline{\answer{Squeeze}} Theorem
\end{problem}}%}

%%%%%%%%%%%%%%%%%%%%%




\latexProblemContent{
\begin{problem}

The limit as $x\to{-13}$ of $f(x)={{\left(x + 13\right)}^{2} \cos\left(-\frac{23}{{\left(x + 13\right)}^{2}}\right)}$ is $0$.  What is the reason why this is true?

\input{2311-Concept-Limit-0008.HELP.tex}

\begin{multipleChoice}
\choice{The statement is in fact false: $\lim\limits_{x\to{-13}}{{\left(x + 13\right)}^{2} \cos\left(-\frac{23}{{\left(x + 13\right)}^{2}}\right)}\neq0$.}
\choice{The cosine factor decreases to $0$ faster than the polynomial.}
\choice[correct]{The cosine factor is bounded between $-1$ and $1$, so the polynomial forces the function to $0$.}
\choice{The cosine factor directly cancels out the polynomial factor.}
\end{multipleChoice}


What is the name of the theorem that applies to this problem? \qquad \\
The \underline{\answer{Squeeze}} Theorem
\end{problem}}%}

%%%%%%%%%%%%%%%%%%%%%




\latexProblemContent{
\begin{problem}

The limit as $x\to{7}$ of $f(x)={{\left(x - 7\right)}^{2} \cos\left(-\frac{12}{x - 7}\right)}$ is $0$.  What is the reason why this is true?

\input{2311-Concept-Limit-0008.HELP.tex}

\begin{multipleChoice}
\choice{The statement is in fact false: $\lim\limits_{x\to{7}}{{\left(x - 7\right)}^{2} \cos\left(-\frac{12}{x - 7}\right)}\neq0$.}
\choice{The cosine factor decreases to $0$ faster than the polynomial.}
\choice[correct]{The cosine factor is bounded between $-1$ and $1$, so the polynomial forces the function to $0$.}
\choice{The cosine factor directly cancels out the polynomial factor.}
\end{multipleChoice}


What is the name of the theorem that applies to this problem? \qquad \\
The \underline{\answer{Squeeze}} Theorem
\end{problem}}%}

%%%%%%%%%%%%%%%%%%%%%




\latexProblemContent{
\begin{problem}

The limit as $x\to{1}$ of $f(x)={{\left(x - 1\right)}^{3} \cos\left(-\frac{11}{{\left(x - 1\right)}^{2}}\right)}$ is $0$.  What is the reason why this is true?

\input{2311-Concept-Limit-0008.HELP.tex}

\begin{multipleChoice}
\choice{The statement is in fact false: $\lim\limits_{x\to{1}}{{\left(x - 1\right)}^{3} \cos\left(-\frac{11}{{\left(x - 1\right)}^{2}}\right)}\neq0$.}
\choice{The cosine factor decreases to $0$ faster than the polynomial.}
\choice[correct]{The cosine factor is bounded between $-1$ and $1$, so the polynomial forces the function to $0$.}
\choice{The cosine factor directly cancels out the polynomial factor.}
\end{multipleChoice}


What is the name of the theorem that applies to this problem? \qquad \\
The \underline{\answer{Squeeze}} Theorem
\end{problem}}%}

%%%%%%%%%%%%%%%%%%%%%




\latexProblemContent{
\begin{problem}

The limit as $x\to{-13}$ of $f(x)={{\left(x + 13\right)}^{2} \cos\left(-\frac{13}{{\left(x + 13\right)}^{2}}\right)}$ is $0$.  What is the reason why this is true?

\input{2311-Concept-Limit-0008.HELP.tex}

\begin{multipleChoice}
\choice{The statement is in fact false: $\lim\limits_{x\to{-13}}{{\left(x + 13\right)}^{2} \cos\left(-\frac{13}{{\left(x + 13\right)}^{2}}\right)}\neq0$.}
\choice{The cosine factor decreases to $0$ faster than the polynomial.}
\choice[correct]{The cosine factor is bounded between $-1$ and $1$, so the polynomial forces the function to $0$.}
\choice{The cosine factor directly cancels out the polynomial factor.}
\end{multipleChoice}


What is the name of the theorem that applies to this problem? \qquad \\
The \underline{\answer{Squeeze}} Theorem
\end{problem}}%}

%%%%%%%%%%%%%%%%%%%%%




\latexProblemContent{
\begin{problem}

The limit as $x\to{11}$ of $f(x)={{\left(x - 11\right)}^{2} \cos\left(-\frac{25}{x - 11}\right)}$ is $0$.  What is the reason why this is true?

\input{2311-Concept-Limit-0008.HELP.tex}

\begin{multipleChoice}
\choice{The statement is in fact false: $\lim\limits_{x\to{11}}{{\left(x - 11\right)}^{2} \cos\left(-\frac{25}{x - 11}\right)}\neq0$.}
\choice{The cosine factor decreases to $0$ faster than the polynomial.}
\choice[correct]{The cosine factor is bounded between $-1$ and $1$, so the polynomial forces the function to $0$.}
\choice{The cosine factor directly cancels out the polynomial factor.}
\end{multipleChoice}


What is the name of the theorem that applies to this problem? \qquad \\
The \underline{\answer{Squeeze}} Theorem
\end{problem}}%}

%%%%%%%%%%%%%%%%%%%%%




\latexProblemContent{
\begin{problem}

The limit as $x\to{-13}$ of $f(x)={{\left(x + 13\right)}^{3} \cos\left(\frac{6}{x + 13}\right)}$ is $0$.  What is the reason why this is true?

\input{2311-Concept-Limit-0008.HELP.tex}

\begin{multipleChoice}
\choice{The statement is in fact false: $\lim\limits_{x\to{-13}}{{\left(x + 13\right)}^{3} \cos\left(\frac{6}{x + 13}\right)}\neq0$.}
\choice{The cosine factor decreases to $0$ faster than the polynomial.}
\choice[correct]{The cosine factor is bounded between $-1$ and $1$, so the polynomial forces the function to $0$.}
\choice{The cosine factor directly cancels out the polynomial factor.}
\end{multipleChoice}


What is the name of the theorem that applies to this problem? \qquad \\
The \underline{\answer{Squeeze}} Theorem
\end{problem}}%}

%%%%%%%%%%%%%%%%%%%%%




\latexProblemContent{
\begin{problem}

The limit as $x\to{7}$ of $f(x)={{\left(x - 7\right)}^{2} \cos\left(\frac{16}{x - 7}\right)}$ is $0$.  What is the reason why this is true?

\input{2311-Concept-Limit-0008.HELP.tex}

\begin{multipleChoice}
\choice{The statement is in fact false: $\lim\limits_{x\to{7}}{{\left(x - 7\right)}^{2} \cos\left(\frac{16}{x - 7}\right)}\neq0$.}
\choice{The cosine factor decreases to $0$ faster than the polynomial.}
\choice[correct]{The cosine factor is bounded between $-1$ and $1$, so the polynomial forces the function to $0$.}
\choice{The cosine factor directly cancels out the polynomial factor.}
\end{multipleChoice}


What is the name of the theorem that applies to this problem? \qquad \\
The \underline{\answer{Squeeze}} Theorem
\end{problem}}%}

%%%%%%%%%%%%%%%%%%%%%




\latexProblemContent{
\begin{problem}

The limit as $x\to{-3}$ of $f(x)={{\left(x + 3\right)}^{2} \cos\left(\frac{5}{{\left(x + 3\right)}^{2}}\right)}$ is $0$.  What is the reason why this is true?

\input{2311-Concept-Limit-0008.HELP.tex}

\begin{multipleChoice}
\choice{The statement is in fact false: $\lim\limits_{x\to{-3}}{{\left(x + 3\right)}^{2} \cos\left(\frac{5}{{\left(x + 3\right)}^{2}}\right)}\neq0$.}
\choice{The cosine factor decreases to $0$ faster than the polynomial.}
\choice[correct]{The cosine factor is bounded between $-1$ and $1$, so the polynomial forces the function to $0$.}
\choice{The cosine factor directly cancels out the polynomial factor.}
\end{multipleChoice}


What is the name of the theorem that applies to this problem? \qquad \\
The \underline{\answer{Squeeze}} Theorem
\end{problem}}%}

%%%%%%%%%%%%%%%%%%%%%




\latexProblemContent{
\begin{problem}

The limit as $x\to{3}$ of $f(x)={{\left(x - 3\right)}^{3} \cos\left(-\frac{16}{{\left(x - 3\right)}^{2}}\right)}$ is $0$.  What is the reason why this is true?

\input{2311-Concept-Limit-0008.HELP.tex}

\begin{multipleChoice}
\choice{The statement is in fact false: $\lim\limits_{x\to{3}}{{\left(x - 3\right)}^{3} \cos\left(-\frac{16}{{\left(x - 3\right)}^{2}}\right)}\neq0$.}
\choice{The cosine factor decreases to $0$ faster than the polynomial.}
\choice[correct]{The cosine factor is bounded between $-1$ and $1$, so the polynomial forces the function to $0$.}
\choice{The cosine factor directly cancels out the polynomial factor.}
\end{multipleChoice}


What is the name of the theorem that applies to this problem? \qquad \\
The \underline{\answer{Squeeze}} Theorem
\end{problem}}%}

%%%%%%%%%%%%%%%%%%%%%




\latexProblemContent{
\begin{problem}

The limit as $x\to{8}$ of $f(x)={{\left(x - 8\right)}^{2} \cos\left(-\frac{24}{x - 8}\right)}$ is $0$.  What is the reason why this is true?

\input{2311-Concept-Limit-0008.HELP.tex}

\begin{multipleChoice}
\choice{The statement is in fact false: $\lim\limits_{x\to{8}}{{\left(x - 8\right)}^{2} \cos\left(-\frac{24}{x - 8}\right)}\neq0$.}
\choice{The cosine factor decreases to $0$ faster than the polynomial.}
\choice[correct]{The cosine factor is bounded between $-1$ and $1$, so the polynomial forces the function to $0$.}
\choice{The cosine factor directly cancels out the polynomial factor.}
\end{multipleChoice}


What is the name of the theorem that applies to this problem? \qquad \\
The \underline{\answer{Squeeze}} Theorem
\end{problem}}%}

%%%%%%%%%%%%%%%%%%%%%




\latexProblemContent{
\begin{problem}

The limit as $x\to{7}$ of $f(x)={{\left(x - 7\right)} \cos\left(-\frac{15}{x - 7}\right)}$ is $0$.  What is the reason why this is true?

\input{2311-Concept-Limit-0008.HELP.tex}

\begin{multipleChoice}
\choice{The statement is in fact false: $\lim\limits_{x\to{7}}{{\left(x - 7\right)} \cos\left(-\frac{15}{x - 7}\right)}\neq0$.}
\choice{The cosine factor decreases to $0$ faster than the polynomial.}
\choice[correct]{The cosine factor is bounded between $-1$ and $1$, so the polynomial forces the function to $0$.}
\choice{The cosine factor directly cancels out the polynomial factor.}
\end{multipleChoice}


What is the name of the theorem that applies to this problem? \qquad \\
The \underline{\answer{Squeeze}} Theorem
\end{problem}}%}

%%%%%%%%%%%%%%%%%%%%%




\latexProblemContent{
\begin{problem}

The limit as $x\to{-4}$ of $f(x)={{\left(x + 4\right)} \cos\left(\frac{13}{{\left(x + 4\right)}^{2}}\right)}$ is $0$.  What is the reason why this is true?

\input{2311-Concept-Limit-0008.HELP.tex}

\begin{multipleChoice}
\choice{The statement is in fact false: $\lim\limits_{x\to{-4}}{{\left(x + 4\right)} \cos\left(\frac{13}{{\left(x + 4\right)}^{2}}\right)}\neq0$.}
\choice{The cosine factor decreases to $0$ faster than the polynomial.}
\choice[correct]{The cosine factor is bounded between $-1$ and $1$, so the polynomial forces the function to $0$.}
\choice{The cosine factor directly cancels out the polynomial factor.}
\end{multipleChoice}


What is the name of the theorem that applies to this problem? \qquad \\
The \underline{\answer{Squeeze}} Theorem
\end{problem}}%}

%%%%%%%%%%%%%%%%%%%%%




\latexProblemContent{
\begin{problem}

The limit as $x\to{-12}$ of $f(x)={{\left(x + 12\right)}^{3} \cos\left(-\frac{22}{x + 12}\right)}$ is $0$.  What is the reason why this is true?

\input{2311-Concept-Limit-0008.HELP.tex}

\begin{multipleChoice}
\choice{The statement is in fact false: $\lim\limits_{x\to{-12}}{{\left(x + 12\right)}^{3} \cos\left(-\frac{22}{x + 12}\right)}\neq0$.}
\choice{The cosine factor decreases to $0$ faster than the polynomial.}
\choice[correct]{The cosine factor is bounded between $-1$ and $1$, so the polynomial forces the function to $0$.}
\choice{The cosine factor directly cancels out the polynomial factor.}
\end{multipleChoice}


What is the name of the theorem that applies to this problem? \qquad \\
The \underline{\answer{Squeeze}} Theorem
\end{problem}}%}

%%%%%%%%%%%%%%%%%%%%%




\latexProblemContent{
\begin{problem}

The limit as $x\to{-5}$ of $f(x)={{\left(x + 5\right)} \cos\left(\frac{12}{x + 5}\right)}$ is $0$.  What is the reason why this is true?

\input{2311-Concept-Limit-0008.HELP.tex}

\begin{multipleChoice}
\choice{The statement is in fact false: $\lim\limits_{x\to{-5}}{{\left(x + 5\right)} \cos\left(\frac{12}{x + 5}\right)}\neq0$.}
\choice{The cosine factor decreases to $0$ faster than the polynomial.}
\choice[correct]{The cosine factor is bounded between $-1$ and $1$, so the polynomial forces the function to $0$.}
\choice{The cosine factor directly cancels out the polynomial factor.}
\end{multipleChoice}


What is the name of the theorem that applies to this problem? \qquad \\
The \underline{\answer{Squeeze}} Theorem
\end{problem}}%}

%%%%%%%%%%%%%%%%%%%%%




\latexProblemContent{
\begin{problem}

The limit as $x\to{8}$ of $f(x)={{\left(x - 8\right)}^{3} \cos\left(\frac{11}{{\left(x - 8\right)}^{2}}\right)}$ is $0$.  What is the reason why this is true?

\input{2311-Concept-Limit-0008.HELP.tex}

\begin{multipleChoice}
\choice{The statement is in fact false: $\lim\limits_{x\to{8}}{{\left(x - 8\right)}^{3} \cos\left(\frac{11}{{\left(x - 8\right)}^{2}}\right)}\neq0$.}
\choice{The cosine factor decreases to $0$ faster than the polynomial.}
\choice[correct]{The cosine factor is bounded between $-1$ and $1$, so the polynomial forces the function to $0$.}
\choice{The cosine factor directly cancels out the polynomial factor.}
\end{multipleChoice}


What is the name of the theorem that applies to this problem? \qquad \\
The \underline{\answer{Squeeze}} Theorem
\end{problem}}%}

%%%%%%%%%%%%%%%%%%%%%




\latexProblemContent{
\begin{problem}

The limit as $x\to{-10}$ of $f(x)={{\left(x + 10\right)} \cos\left(\frac{2}{x + 10}\right)}$ is $0$.  What is the reason why this is true?

\input{2311-Concept-Limit-0008.HELP.tex}

\begin{multipleChoice}
\choice{The statement is in fact false: $\lim\limits_{x\to{-10}}{{\left(x + 10\right)} \cos\left(\frac{2}{x + 10}\right)}\neq0$.}
\choice{The cosine factor decreases to $0$ faster than the polynomial.}
\choice[correct]{The cosine factor is bounded between $-1$ and $1$, so the polynomial forces the function to $0$.}
\choice{The cosine factor directly cancels out the polynomial factor.}
\end{multipleChoice}


What is the name of the theorem that applies to this problem? \qquad \\
The \underline{\answer{Squeeze}} Theorem
\end{problem}}%}

%%%%%%%%%%%%%%%%%%%%%




\latexProblemContent{
\begin{problem}

The limit as $x\to{-11}$ of $f(x)={{\left(x + 11\right)}^{3} \cos\left(-\frac{15}{x + 11}\right)}$ is $0$.  What is the reason why this is true?

\input{2311-Concept-Limit-0008.HELP.tex}

\begin{multipleChoice}
\choice{The statement is in fact false: $\lim\limits_{x\to{-11}}{{\left(x + 11\right)}^{3} \cos\left(-\frac{15}{x + 11}\right)}\neq0$.}
\choice{The cosine factor decreases to $0$ faster than the polynomial.}
\choice[correct]{The cosine factor is bounded between $-1$ and $1$, so the polynomial forces the function to $0$.}
\choice{The cosine factor directly cancels out the polynomial factor.}
\end{multipleChoice}


What is the name of the theorem that applies to this problem? \qquad \\
The \underline{\answer{Squeeze}} Theorem
\end{problem}}%}

%%%%%%%%%%%%%%%%%%%%%




\latexProblemContent{
\begin{problem}

The limit as $x\to{2}$ of $f(x)={{\left(x - 2\right)}^{2} \cos\left(-\frac{6}{{\left(x - 2\right)}^{2}}\right)}$ is $0$.  What is the reason why this is true?

\input{2311-Concept-Limit-0008.HELP.tex}

\begin{multipleChoice}
\choice{The statement is in fact false: $\lim\limits_{x\to{2}}{{\left(x - 2\right)}^{2} \cos\left(-\frac{6}{{\left(x - 2\right)}^{2}}\right)}\neq0$.}
\choice{The cosine factor decreases to $0$ faster than the polynomial.}
\choice[correct]{The cosine factor is bounded between $-1$ and $1$, so the polynomial forces the function to $0$.}
\choice{The cosine factor directly cancels out the polynomial factor.}
\end{multipleChoice}


What is the name of the theorem that applies to this problem? \qquad \\
The \underline{\answer{Squeeze}} Theorem
\end{problem}}%}

%%%%%%%%%%%%%%%%%%%%%




\latexProblemContent{
\begin{problem}

The limit as $x\to{15}$ of $f(x)={{\left(x - 15\right)}^{3} \cos\left(-\frac{13}{x - 15}\right)}$ is $0$.  What is the reason why this is true?

\input{2311-Concept-Limit-0008.HELP.tex}

\begin{multipleChoice}
\choice{The statement is in fact false: $\lim\limits_{x\to{15}}{{\left(x - 15\right)}^{3} \cos\left(-\frac{13}{x - 15}\right)}\neq0$.}
\choice{The cosine factor decreases to $0$ faster than the polynomial.}
\choice[correct]{The cosine factor is bounded between $-1$ and $1$, so the polynomial forces the function to $0$.}
\choice{The cosine factor directly cancels out the polynomial factor.}
\end{multipleChoice}


What is the name of the theorem that applies to this problem? \qquad \\
The \underline{\answer{Squeeze}} Theorem
\end{problem}}%}

%%%%%%%%%%%%%%%%%%%%%




\latexProblemContent{
\begin{problem}

The limit as $x\to{7}$ of $f(x)={{\left(x - 7\right)}^{2} \cos\left(-\frac{24}{x - 7}\right)}$ is $0$.  What is the reason why this is true?

\input{2311-Concept-Limit-0008.HELP.tex}

\begin{multipleChoice}
\choice{The statement is in fact false: $\lim\limits_{x\to{7}}{{\left(x - 7\right)}^{2} \cos\left(-\frac{24}{x - 7}\right)}\neq0$.}
\choice{The cosine factor decreases to $0$ faster than the polynomial.}
\choice[correct]{The cosine factor is bounded between $-1$ and $1$, so the polynomial forces the function to $0$.}
\choice{The cosine factor directly cancels out the polynomial factor.}
\end{multipleChoice}


What is the name of the theorem that applies to this problem? \qquad \\
The \underline{\answer{Squeeze}} Theorem
\end{problem}}%}

%%%%%%%%%%%%%%%%%%%%%




\latexProblemContent{
\begin{problem}

The limit as $x\to{-6}$ of $f(x)={{\left(x + 6\right)}^{2} \cos\left(\frac{7}{x + 6}\right)}$ is $0$.  What is the reason why this is true?

\input{2311-Concept-Limit-0008.HELP.tex}

\begin{multipleChoice}
\choice{The statement is in fact false: $\lim\limits_{x\to{-6}}{{\left(x + 6\right)}^{2} \cos\left(\frac{7}{x + 6}\right)}\neq0$.}
\choice{The cosine factor decreases to $0$ faster than the polynomial.}
\choice[correct]{The cosine factor is bounded between $-1$ and $1$, so the polynomial forces the function to $0$.}
\choice{The cosine factor directly cancels out the polynomial factor.}
\end{multipleChoice}


What is the name of the theorem that applies to this problem? \qquad \\
The \underline{\answer{Squeeze}} Theorem
\end{problem}}%}

%%%%%%%%%%%%%%%%%%%%%




\latexProblemContent{
\begin{problem}

The limit as $x\to{10}$ of $f(x)={{\left(x - 10\right)}^{3} \cos\left(\frac{10}{{\left(x - 10\right)}^{2}}\right)}$ is $0$.  What is the reason why this is true?

\input{2311-Concept-Limit-0008.HELP.tex}

\begin{multipleChoice}
\choice{The statement is in fact false: $\lim\limits_{x\to{10}}{{\left(x - 10\right)}^{3} \cos\left(\frac{10}{{\left(x - 10\right)}^{2}}\right)}\neq0$.}
\choice{The cosine factor decreases to $0$ faster than the polynomial.}
\choice[correct]{The cosine factor is bounded between $-1$ and $1$, so the polynomial forces the function to $0$.}
\choice{The cosine factor directly cancels out the polynomial factor.}
\end{multipleChoice}


What is the name of the theorem that applies to this problem? \qquad \\
The \underline{\answer{Squeeze}} Theorem
\end{problem}}%}

%%%%%%%%%%%%%%%%%%%%%




\latexProblemContent{
\begin{problem}

The limit as $x\to{9}$ of $f(x)={{\left(x - 9\right)}^{3} \cos\left(-\frac{1}{{\left(x - 9\right)}^{2}}\right)}$ is $0$.  What is the reason why this is true?

\input{2311-Concept-Limit-0008.HELP.tex}

\begin{multipleChoice}
\choice{The statement is in fact false: $\lim\limits_{x\to{9}}{{\left(x - 9\right)}^{3} \cos\left(-\frac{1}{{\left(x - 9\right)}^{2}}\right)}\neq0$.}
\choice{The cosine factor decreases to $0$ faster than the polynomial.}
\choice[correct]{The cosine factor is bounded between $-1$ and $1$, so the polynomial forces the function to $0$.}
\choice{The cosine factor directly cancels out the polynomial factor.}
\end{multipleChoice}


What is the name of the theorem that applies to this problem? \qquad \\
The \underline{\answer{Squeeze}} Theorem
\end{problem}}%}

%%%%%%%%%%%%%%%%%%%%%




\latexProblemContent{
\begin{problem}

The limit as $x\to{-11}$ of $f(x)={{\left(x + 11\right)} \cos\left(\frac{10}{{\left(x + 11\right)}^{2}}\right)}$ is $0$.  What is the reason why this is true?

\input{2311-Concept-Limit-0008.HELP.tex}

\begin{multipleChoice}
\choice{The statement is in fact false: $\lim\limits_{x\to{-11}}{{\left(x + 11\right)} \cos\left(\frac{10}{{\left(x + 11\right)}^{2}}\right)}\neq0$.}
\choice{The cosine factor decreases to $0$ faster than the polynomial.}
\choice[correct]{The cosine factor is bounded between $-1$ and $1$, so the polynomial forces the function to $0$.}
\choice{The cosine factor directly cancels out the polynomial factor.}
\end{multipleChoice}


What is the name of the theorem that applies to this problem? \qquad \\
The \underline{\answer{Squeeze}} Theorem
\end{problem}}%}

%%%%%%%%%%%%%%%%%%%%%




\latexProblemContent{
\begin{problem}

The limit as $x\to{-11}$ of $f(x)={{\left(x + 11\right)}^{3} \cos\left(\frac{11}{{\left(x + 11\right)}^{2}}\right)}$ is $0$.  What is the reason why this is true?

\input{2311-Concept-Limit-0008.HELP.tex}

\begin{multipleChoice}
\choice{The statement is in fact false: $\lim\limits_{x\to{-11}}{{\left(x + 11\right)}^{3} \cos\left(\frac{11}{{\left(x + 11\right)}^{2}}\right)}\neq0$.}
\choice{The cosine factor decreases to $0$ faster than the polynomial.}
\choice[correct]{The cosine factor is bounded between $-1$ and $1$, so the polynomial forces the function to $0$.}
\choice{The cosine factor directly cancels out the polynomial factor.}
\end{multipleChoice}


What is the name of the theorem that applies to this problem? \qquad \\
The \underline{\answer{Squeeze}} Theorem
\end{problem}}%}

%%%%%%%%%%%%%%%%%%%%%




\latexProblemContent{
\begin{problem}

The limit as $x\to{0}$ of $f(x)={x^{2} \cos\left(\frac{14}{x}\right)}$ is $0$.  What is the reason why this is true?

\input{2311-Concept-Limit-0008.HELP.tex}

\begin{multipleChoice}
\choice{The statement is in fact false: $\lim\limits_{x\to{0}}{x^{2} \cos\left(\frac{14}{x}\right)}\neq0$.}
\choice{The cosine factor decreases to $0$ faster than the polynomial.}
\choice[correct]{The cosine factor is bounded between $-1$ and $1$, so the polynomial forces the function to $0$.}
\choice{The cosine factor directly cancels out the polynomial factor.}
\end{multipleChoice}


What is the name of the theorem that applies to this problem? \qquad \\
The \underline{\answer{Squeeze}} Theorem
\end{problem}}%}

%%%%%%%%%%%%%%%%%%%%%




\latexProblemContent{
\begin{problem}

The limit as $x\to{-13}$ of $f(x)={{\left(x + 13\right)} \cos\left(-\frac{23}{x + 13}\right)}$ is $0$.  What is the reason why this is true?

\input{2311-Concept-Limit-0008.HELP.tex}

\begin{multipleChoice}
\choice{The statement is in fact false: $\lim\limits_{x\to{-13}}{{\left(x + 13\right)} \cos\left(-\frac{23}{x + 13}\right)}\neq0$.}
\choice{The cosine factor decreases to $0$ faster than the polynomial.}
\choice[correct]{The cosine factor is bounded between $-1$ and $1$, so the polynomial forces the function to $0$.}
\choice{The cosine factor directly cancels out the polynomial factor.}
\end{multipleChoice}


What is the name of the theorem that applies to this problem? \qquad \\
The \underline{\answer{Squeeze}} Theorem
\end{problem}}%}

%%%%%%%%%%%%%%%%%%%%%




\latexProblemContent{
\begin{problem}

The limit as $x\to{15}$ of $f(x)={{\left(x - 15\right)}^{2} \cos\left(\frac{4}{x - 15}\right)}$ is $0$.  What is the reason why this is true?

\input{2311-Concept-Limit-0008.HELP.tex}

\begin{multipleChoice}
\choice{The statement is in fact false: $\lim\limits_{x\to{15}}{{\left(x - 15\right)}^{2} \cos\left(\frac{4}{x - 15}\right)}\neq0$.}
\choice{The cosine factor decreases to $0$ faster than the polynomial.}
\choice[correct]{The cosine factor is bounded between $-1$ and $1$, so the polynomial forces the function to $0$.}
\choice{The cosine factor directly cancels out the polynomial factor.}
\end{multipleChoice}


What is the name of the theorem that applies to this problem? \qquad \\
The \underline{\answer{Squeeze}} Theorem
\end{problem}}%}

%%%%%%%%%%%%%%%%%%%%%




\latexProblemContent{
\begin{problem}

The limit as $x\to{-1}$ of $f(x)={{\left(x + 1\right)} \cos\left(-\frac{19}{{\left(x + 1\right)}^{2}}\right)}$ is $0$.  What is the reason why this is true?

\input{2311-Concept-Limit-0008.HELP.tex}

\begin{multipleChoice}
\choice{The statement is in fact false: $\lim\limits_{x\to{-1}}{{\left(x + 1\right)} \cos\left(-\frac{19}{{\left(x + 1\right)}^{2}}\right)}\neq0$.}
\choice{The cosine factor decreases to $0$ faster than the polynomial.}
\choice[correct]{The cosine factor is bounded between $-1$ and $1$, so the polynomial forces the function to $0$.}
\choice{The cosine factor directly cancels out the polynomial factor.}
\end{multipleChoice}


What is the name of the theorem that applies to this problem? \qquad \\
The \underline{\answer{Squeeze}} Theorem
\end{problem}}%}

%%%%%%%%%%%%%%%%%%%%%




\latexProblemContent{
\begin{problem}

The limit as $x\to{14}$ of $f(x)={{\left(x - 14\right)}^{3} \cos\left(-\frac{24}{x - 14}\right)}$ is $0$.  What is the reason why this is true?

\input{2311-Concept-Limit-0008.HELP.tex}

\begin{multipleChoice}
\choice{The statement is in fact false: $\lim\limits_{x\to{14}}{{\left(x - 14\right)}^{3} \cos\left(-\frac{24}{x - 14}\right)}\neq0$.}
\choice{The cosine factor decreases to $0$ faster than the polynomial.}
\choice[correct]{The cosine factor is bounded between $-1$ and $1$, so the polynomial forces the function to $0$.}
\choice{The cosine factor directly cancels out the polynomial factor.}
\end{multipleChoice}


What is the name of the theorem that applies to this problem? \qquad \\
The \underline{\answer{Squeeze}} Theorem
\end{problem}}%}

%%%%%%%%%%%%%%%%%%%%%




\latexProblemContent{
\begin{problem}

The limit as $x\to{-14}$ of $f(x)={{\left(x + 14\right)}^{3} \cos\left(-\frac{15}{x + 14}\right)}$ is $0$.  What is the reason why this is true?

\input{2311-Concept-Limit-0008.HELP.tex}

\begin{multipleChoice}
\choice{The statement is in fact false: $\lim\limits_{x\to{-14}}{{\left(x + 14\right)}^{3} \cos\left(-\frac{15}{x + 14}\right)}\neq0$.}
\choice{The cosine factor decreases to $0$ faster than the polynomial.}
\choice[correct]{The cosine factor is bounded between $-1$ and $1$, so the polynomial forces the function to $0$.}
\choice{The cosine factor directly cancels out the polynomial factor.}
\end{multipleChoice}


What is the name of the theorem that applies to this problem? \qquad \\
The \underline{\answer{Squeeze}} Theorem
\end{problem}}%}

%%%%%%%%%%%%%%%%%%%%%




\latexProblemContent{
\begin{problem}

The limit as $x\to{-8}$ of $f(x)={{\left(x + 8\right)}^{3} \cos\left(\frac{25}{{\left(x + 8\right)}^{2}}\right)}$ is $0$.  What is the reason why this is true?

\input{2311-Concept-Limit-0008.HELP.tex}

\begin{multipleChoice}
\choice{The statement is in fact false: $\lim\limits_{x\to{-8}}{{\left(x + 8\right)}^{3} \cos\left(\frac{25}{{\left(x + 8\right)}^{2}}\right)}\neq0$.}
\choice{The cosine factor decreases to $0$ faster than the polynomial.}
\choice[correct]{The cosine factor is bounded between $-1$ and $1$, so the polynomial forces the function to $0$.}
\choice{The cosine factor directly cancels out the polynomial factor.}
\end{multipleChoice}


What is the name of the theorem that applies to this problem? \qquad \\
The \underline{\answer{Squeeze}} Theorem
\end{problem}}%}

%%%%%%%%%%%%%%%%%%%%%




\latexProblemContent{
\begin{problem}

The limit as $x\to{-1}$ of $f(x)={{\left(x + 1\right)} \cos\left(-\frac{3}{{\left(x + 1\right)}^{2}}\right)}$ is $0$.  What is the reason why this is true?

\input{2311-Concept-Limit-0008.HELP.tex}

\begin{multipleChoice}
\choice{The statement is in fact false: $\lim\limits_{x\to{-1}}{{\left(x + 1\right)} \cos\left(-\frac{3}{{\left(x + 1\right)}^{2}}\right)}\neq0$.}
\choice{The cosine factor decreases to $0$ faster than the polynomial.}
\choice[correct]{The cosine factor is bounded between $-1$ and $1$, so the polynomial forces the function to $0$.}
\choice{The cosine factor directly cancels out the polynomial factor.}
\end{multipleChoice}


What is the name of the theorem that applies to this problem? \qquad \\
The \underline{\answer{Squeeze}} Theorem
\end{problem}}%}

%%%%%%%%%%%%%%%%%%%%%




\latexProblemContent{
\begin{problem}

The limit as $x\to{-2}$ of $f(x)={{\left(x + 2\right)} \cos\left(-\frac{16}{x + 2}\right)}$ is $0$.  What is the reason why this is true?

\input{2311-Concept-Limit-0008.HELP.tex}

\begin{multipleChoice}
\choice{The statement is in fact false: $\lim\limits_{x\to{-2}}{{\left(x + 2\right)} \cos\left(-\frac{16}{x + 2}\right)}\neq0$.}
\choice{The cosine factor decreases to $0$ faster than the polynomial.}
\choice[correct]{The cosine factor is bounded between $-1$ and $1$, so the polynomial forces the function to $0$.}
\choice{The cosine factor directly cancels out the polynomial factor.}
\end{multipleChoice}


What is the name of the theorem that applies to this problem? \qquad \\
The \underline{\answer{Squeeze}} Theorem
\end{problem}}%}

%%%%%%%%%%%%%%%%%%%%%




\latexProblemContent{
\begin{problem}

The limit as $x\to{14}$ of $f(x)={{\left(x - 14\right)}^{2} \cos\left(-\frac{18}{{\left(x - 14\right)}^{2}}\right)}$ is $0$.  What is the reason why this is true?

\input{2311-Concept-Limit-0008.HELP.tex}

\begin{multipleChoice}
\choice{The statement is in fact false: $\lim\limits_{x\to{14}}{{\left(x - 14\right)}^{2} \cos\left(-\frac{18}{{\left(x - 14\right)}^{2}}\right)}\neq0$.}
\choice{The cosine factor decreases to $0$ faster than the polynomial.}
\choice[correct]{The cosine factor is bounded between $-1$ and $1$, so the polynomial forces the function to $0$.}
\choice{The cosine factor directly cancels out the polynomial factor.}
\end{multipleChoice}


What is the name of the theorem that applies to this problem? \qquad \\
The \underline{\answer{Squeeze}} Theorem
\end{problem}}%}

%%%%%%%%%%%%%%%%%%%%%




\latexProblemContent{
\begin{problem}

The limit as $x\to{4}$ of $f(x)={{\left(x - 4\right)}^{2} \cos\left(-\frac{22}{{\left(x - 4\right)}^{2}}\right)}$ is $0$.  What is the reason why this is true?

\input{2311-Concept-Limit-0008.HELP.tex}

\begin{multipleChoice}
\choice{The statement is in fact false: $\lim\limits_{x\to{4}}{{\left(x - 4\right)}^{2} \cos\left(-\frac{22}{{\left(x - 4\right)}^{2}}\right)}\neq0$.}
\choice{The cosine factor decreases to $0$ faster than the polynomial.}
\choice[correct]{The cosine factor is bounded between $-1$ and $1$, so the polynomial forces the function to $0$.}
\choice{The cosine factor directly cancels out the polynomial factor.}
\end{multipleChoice}


What is the name of the theorem that applies to this problem? \qquad \\
The \underline{\answer{Squeeze}} Theorem
\end{problem}}%}

%%%%%%%%%%%%%%%%%%%%%




\latexProblemContent{
\begin{problem}

The limit as $x\to{11}$ of $f(x)={{\left(x - 11\right)}^{3} \cos\left(-\frac{24}{x - 11}\right)}$ is $0$.  What is the reason why this is true?

\input{2311-Concept-Limit-0008.HELP.tex}

\begin{multipleChoice}
\choice{The statement is in fact false: $\lim\limits_{x\to{11}}{{\left(x - 11\right)}^{3} \cos\left(-\frac{24}{x - 11}\right)}\neq0$.}
\choice{The cosine factor decreases to $0$ faster than the polynomial.}
\choice[correct]{The cosine factor is bounded between $-1$ and $1$, so the polynomial forces the function to $0$.}
\choice{The cosine factor directly cancels out the polynomial factor.}
\end{multipleChoice}


What is the name of the theorem that applies to this problem? \qquad \\
The \underline{\answer{Squeeze}} Theorem
\end{problem}}%}

%%%%%%%%%%%%%%%%%%%%%




\latexProblemContent{
\begin{problem}

The limit as $x\to{5}$ of $f(x)={{\left(x - 5\right)}^{3} \cos\left(-\frac{15}{{\left(x - 5\right)}^{2}}\right)}$ is $0$.  What is the reason why this is true?

\input{2311-Concept-Limit-0008.HELP.tex}

\begin{multipleChoice}
\choice{The statement is in fact false: $\lim\limits_{x\to{5}}{{\left(x - 5\right)}^{3} \cos\left(-\frac{15}{{\left(x - 5\right)}^{2}}\right)}\neq0$.}
\choice{The cosine factor decreases to $0$ faster than the polynomial.}
\choice[correct]{The cosine factor is bounded between $-1$ and $1$, so the polynomial forces the function to $0$.}
\choice{The cosine factor directly cancels out the polynomial factor.}
\end{multipleChoice}


What is the name of the theorem that applies to this problem? \qquad \\
The \underline{\answer{Squeeze}} Theorem
\end{problem}}%}

%%%%%%%%%%%%%%%%%%%%%




\latexProblemContent{
\begin{problem}

The limit as $x\to{-12}$ of $f(x)={{\left(x + 12\right)}^{3} \cos\left(-\frac{1}{x + 12}\right)}$ is $0$.  What is the reason why this is true?

\input{2311-Concept-Limit-0008.HELP.tex}

\begin{multipleChoice}
\choice{The statement is in fact false: $\lim\limits_{x\to{-12}}{{\left(x + 12\right)}^{3} \cos\left(-\frac{1}{x + 12}\right)}\neq0$.}
\choice{The cosine factor decreases to $0$ faster than the polynomial.}
\choice[correct]{The cosine factor is bounded between $-1$ and $1$, so the polynomial forces the function to $0$.}
\choice{The cosine factor directly cancels out the polynomial factor.}
\end{multipleChoice}


What is the name of the theorem that applies to this problem? \qquad \\
The \underline{\answer{Squeeze}} Theorem
\end{problem}}%}

%%%%%%%%%%%%%%%%%%%%%




\latexProblemContent{
\begin{problem}

The limit as $x\to{9}$ of $f(x)={{\left(x - 9\right)}^{2} \cos\left(\frac{17}{{\left(x - 9\right)}^{2}}\right)}$ is $0$.  What is the reason why this is true?

\input{2311-Concept-Limit-0008.HELP.tex}

\begin{multipleChoice}
\choice{The statement is in fact false: $\lim\limits_{x\to{9}}{{\left(x - 9\right)}^{2} \cos\left(\frac{17}{{\left(x - 9\right)}^{2}}\right)}\neq0$.}
\choice{The cosine factor decreases to $0$ faster than the polynomial.}
\choice[correct]{The cosine factor is bounded between $-1$ and $1$, so the polynomial forces the function to $0$.}
\choice{The cosine factor directly cancels out the polynomial factor.}
\end{multipleChoice}


What is the name of the theorem that applies to this problem? \qquad \\
The \underline{\answer{Squeeze}} Theorem
\end{problem}}%}

%%%%%%%%%%%%%%%%%%%%%




\latexProblemContent{
\begin{problem}

The limit as $x\to{-12}$ of $f(x)={{\left(x + 12\right)} \cos\left(\frac{2}{x + 12}\right)}$ is $0$.  What is the reason why this is true?

\input{2311-Concept-Limit-0008.HELP.tex}

\begin{multipleChoice}
\choice{The statement is in fact false: $\lim\limits_{x\to{-12}}{{\left(x + 12\right)} \cos\left(\frac{2}{x + 12}\right)}\neq0$.}
\choice{The cosine factor decreases to $0$ faster than the polynomial.}
\choice[correct]{The cosine factor is bounded between $-1$ and $1$, so the polynomial forces the function to $0$.}
\choice{The cosine factor directly cancels out the polynomial factor.}
\end{multipleChoice}


What is the name of the theorem that applies to this problem? \qquad \\
The \underline{\answer{Squeeze}} Theorem
\end{problem}}%}

%%%%%%%%%%%%%%%%%%%%%




\latexProblemContent{
\begin{problem}

The limit as $x\to{3}$ of $f(x)={{\left(x - 3\right)}^{2} \cos\left(\frac{8}{x - 3}\right)}$ is $0$.  What is the reason why this is true?

\input{2311-Concept-Limit-0008.HELP.tex}

\begin{multipleChoice}
\choice{The statement is in fact false: $\lim\limits_{x\to{3}}{{\left(x - 3\right)}^{2} \cos\left(\frac{8}{x - 3}\right)}\neq0$.}
\choice{The cosine factor decreases to $0$ faster than the polynomial.}
\choice[correct]{The cosine factor is bounded between $-1$ and $1$, so the polynomial forces the function to $0$.}
\choice{The cosine factor directly cancels out the polynomial factor.}
\end{multipleChoice}


What is the name of the theorem that applies to this problem? \qquad \\
The \underline{\answer{Squeeze}} Theorem
\end{problem}}%}

%%%%%%%%%%%%%%%%%%%%%




\latexProblemContent{
\begin{problem}

The limit as $x\to{-11}$ of $f(x)={{\left(x + 11\right)} \cos\left(\frac{24}{{\left(x + 11\right)}^{2}}\right)}$ is $0$.  What is the reason why this is true?

\input{2311-Concept-Limit-0008.HELP.tex}

\begin{multipleChoice}
\choice{The statement is in fact false: $\lim\limits_{x\to{-11}}{{\left(x + 11\right)} \cos\left(\frac{24}{{\left(x + 11\right)}^{2}}\right)}\neq0$.}
\choice{The cosine factor decreases to $0$ faster than the polynomial.}
\choice[correct]{The cosine factor is bounded between $-1$ and $1$, so the polynomial forces the function to $0$.}
\choice{The cosine factor directly cancels out the polynomial factor.}
\end{multipleChoice}


What is the name of the theorem that applies to this problem? \qquad \\
The \underline{\answer{Squeeze}} Theorem
\end{problem}}%}

%%%%%%%%%%%%%%%%%%%%%




\latexProblemContent{
\begin{problem}

The limit as $x\to{8}$ of $f(x)={{\left(x - 8\right)}^{3} \cos\left(\frac{20}{{\left(x - 8\right)}^{2}}\right)}$ is $0$.  What is the reason why this is true?

\input{2311-Concept-Limit-0008.HELP.tex}

\begin{multipleChoice}
\choice{The statement is in fact false: $\lim\limits_{x\to{8}}{{\left(x - 8\right)}^{3} \cos\left(\frac{20}{{\left(x - 8\right)}^{2}}\right)}\neq0$.}
\choice{The cosine factor decreases to $0$ faster than the polynomial.}
\choice[correct]{The cosine factor is bounded between $-1$ and $1$, so the polynomial forces the function to $0$.}
\choice{The cosine factor directly cancels out the polynomial factor.}
\end{multipleChoice}


What is the name of the theorem that applies to this problem? \qquad \\
The \underline{\answer{Squeeze}} Theorem
\end{problem}}%}

%%%%%%%%%%%%%%%%%%%%%




\latexProblemContent{
\begin{problem}

The limit as $x\to{14}$ of $f(x)={{\left(x - 14\right)}^{2} \cos\left(-\frac{20}{x - 14}\right)}$ is $0$.  What is the reason why this is true?

\input{2311-Concept-Limit-0008.HELP.tex}

\begin{multipleChoice}
\choice{The statement is in fact false: $\lim\limits_{x\to{14}}{{\left(x - 14\right)}^{2} \cos\left(-\frac{20}{x - 14}\right)}\neq0$.}
\choice{The cosine factor decreases to $0$ faster than the polynomial.}
\choice[correct]{The cosine factor is bounded between $-1$ and $1$, so the polynomial forces the function to $0$.}
\choice{The cosine factor directly cancels out the polynomial factor.}
\end{multipleChoice}


What is the name of the theorem that applies to this problem? \qquad \\
The \underline{\answer{Squeeze}} Theorem
\end{problem}}%}

%%%%%%%%%%%%%%%%%%%%%




\latexProblemContent{
\begin{problem}

The limit as $x\to{5}$ of $f(x)={{\left(x - 5\right)}^{2} \cos\left(\frac{25}{{\left(x - 5\right)}^{2}}\right)}$ is $0$.  What is the reason why this is true?

\input{2311-Concept-Limit-0008.HELP.tex}

\begin{multipleChoice}
\choice{The statement is in fact false: $\lim\limits_{x\to{5}}{{\left(x - 5\right)}^{2} \cos\left(\frac{25}{{\left(x - 5\right)}^{2}}\right)}\neq0$.}
\choice{The cosine factor decreases to $0$ faster than the polynomial.}
\choice[correct]{The cosine factor is bounded between $-1$ and $1$, so the polynomial forces the function to $0$.}
\choice{The cosine factor directly cancels out the polynomial factor.}
\end{multipleChoice}


What is the name of the theorem that applies to this problem? \qquad \\
The \underline{\answer{Squeeze}} Theorem
\end{problem}}%}

%%%%%%%%%%%%%%%%%%%%%




\latexProblemContent{
\begin{problem}

The limit as $x\to{-9}$ of $f(x)={{\left(x + 9\right)}^{3} \cos\left(-\frac{7}{{\left(x + 9\right)}^{2}}\right)}$ is $0$.  What is the reason why this is true?

\input{2311-Concept-Limit-0008.HELP.tex}

\begin{multipleChoice}
\choice{The statement is in fact false: $\lim\limits_{x\to{-9}}{{\left(x + 9\right)}^{3} \cos\left(-\frac{7}{{\left(x + 9\right)}^{2}}\right)}\neq0$.}
\choice{The cosine factor decreases to $0$ faster than the polynomial.}
\choice[correct]{The cosine factor is bounded between $-1$ and $1$, so the polynomial forces the function to $0$.}
\choice{The cosine factor directly cancels out the polynomial factor.}
\end{multipleChoice}


What is the name of the theorem that applies to this problem? \qquad \\
The \underline{\answer{Squeeze}} Theorem
\end{problem}}%}

%%%%%%%%%%%%%%%%%%%%%




\latexProblemContent{
\begin{problem}

The limit as $x\to{4}$ of $f(x)={{\left(x - 4\right)} \cos\left(-\frac{8}{{\left(x - 4\right)}^{2}}\right)}$ is $0$.  What is the reason why this is true?

\input{2311-Concept-Limit-0008.HELP.tex}

\begin{multipleChoice}
\choice{The statement is in fact false: $\lim\limits_{x\to{4}}{{\left(x - 4\right)} \cos\left(-\frac{8}{{\left(x - 4\right)}^{2}}\right)}\neq0$.}
\choice{The cosine factor decreases to $0$ faster than the polynomial.}
\choice[correct]{The cosine factor is bounded between $-1$ and $1$, so the polynomial forces the function to $0$.}
\choice{The cosine factor directly cancels out the polynomial factor.}
\end{multipleChoice}


What is the name of the theorem that applies to this problem? \qquad \\
The \underline{\answer{Squeeze}} Theorem
\end{problem}}%}

%%%%%%%%%%%%%%%%%%%%%




\latexProblemContent{
\begin{problem}

The limit as $x\to{3}$ of $f(x)={{\left(x - 3\right)} \cos\left(-\frac{19}{{\left(x - 3\right)}^{2}}\right)}$ is $0$.  What is the reason why this is true?

\input{2311-Concept-Limit-0008.HELP.tex}

\begin{multipleChoice}
\choice{The statement is in fact false: $\lim\limits_{x\to{3}}{{\left(x - 3\right)} \cos\left(-\frac{19}{{\left(x - 3\right)}^{2}}\right)}\neq0$.}
\choice{The cosine factor decreases to $0$ faster than the polynomial.}
\choice[correct]{The cosine factor is bounded between $-1$ and $1$, so the polynomial forces the function to $0$.}
\choice{The cosine factor directly cancels out the polynomial factor.}
\end{multipleChoice}


What is the name of the theorem that applies to this problem? \qquad \\
The \underline{\answer{Squeeze}} Theorem
\end{problem}}%}

%%%%%%%%%%%%%%%%%%%%%




\latexProblemContent{
\begin{problem}

The limit as $x\to{-10}$ of $f(x)={{\left(x + 10\right)}^{2} \cos\left(\frac{22}{{\left(x + 10\right)}^{2}}\right)}$ is $0$.  What is the reason why this is true?

\input{2311-Concept-Limit-0008.HELP.tex}

\begin{multipleChoice}
\choice{The statement is in fact false: $\lim\limits_{x\to{-10}}{{\left(x + 10\right)}^{2} \cos\left(\frac{22}{{\left(x + 10\right)}^{2}}\right)}\neq0$.}
\choice{The cosine factor decreases to $0$ faster than the polynomial.}
\choice[correct]{The cosine factor is bounded between $-1$ and $1$, so the polynomial forces the function to $0$.}
\choice{The cosine factor directly cancels out the polynomial factor.}
\end{multipleChoice}


What is the name of the theorem that applies to this problem? \qquad \\
The \underline{\answer{Squeeze}} Theorem
\end{problem}}%}

%%%%%%%%%%%%%%%%%%%%%




\latexProblemContent{
\begin{problem}

The limit as $x\to{-6}$ of $f(x)={{\left(x + 6\right)} \cos\left(\frac{17}{x + 6}\right)}$ is $0$.  What is the reason why this is true?

\input{2311-Concept-Limit-0008.HELP.tex}

\begin{multipleChoice}
\choice{The statement is in fact false: $\lim\limits_{x\to{-6}}{{\left(x + 6\right)} \cos\left(\frac{17}{x + 6}\right)}\neq0$.}
\choice{The cosine factor decreases to $0$ faster than the polynomial.}
\choice[correct]{The cosine factor is bounded between $-1$ and $1$, so the polynomial forces the function to $0$.}
\choice{The cosine factor directly cancels out the polynomial factor.}
\end{multipleChoice}


What is the name of the theorem that applies to this problem? \qquad \\
The \underline{\answer{Squeeze}} Theorem
\end{problem}}%}

%%%%%%%%%%%%%%%%%%%%%




\latexProblemContent{
\begin{problem}

The limit as $x\to{7}$ of $f(x)={{\left(x - 7\right)}^{3} \cos\left(-\frac{4}{x - 7}\right)}$ is $0$.  What is the reason why this is true?

\input{2311-Concept-Limit-0008.HELP.tex}

\begin{multipleChoice}
\choice{The statement is in fact false: $\lim\limits_{x\to{7}}{{\left(x - 7\right)}^{3} \cos\left(-\frac{4}{x - 7}\right)}\neq0$.}
\choice{The cosine factor decreases to $0$ faster than the polynomial.}
\choice[correct]{The cosine factor is bounded between $-1$ and $1$, so the polynomial forces the function to $0$.}
\choice{The cosine factor directly cancels out the polynomial factor.}
\end{multipleChoice}


What is the name of the theorem that applies to this problem? \qquad \\
The \underline{\answer{Squeeze}} Theorem
\end{problem}}%}

%%%%%%%%%%%%%%%%%%%%%




\latexProblemContent{
\begin{problem}

The limit as $x\to{8}$ of $f(x)={{\left(x - 8\right)} \cos\left(\frac{16}{x - 8}\right)}$ is $0$.  What is the reason why this is true?

\input{2311-Concept-Limit-0008.HELP.tex}

\begin{multipleChoice}
\choice{The statement is in fact false: $\lim\limits_{x\to{8}}{{\left(x - 8\right)} \cos\left(\frac{16}{x - 8}\right)}\neq0$.}
\choice{The cosine factor decreases to $0$ faster than the polynomial.}
\choice[correct]{The cosine factor is bounded between $-1$ and $1$, so the polynomial forces the function to $0$.}
\choice{The cosine factor directly cancels out the polynomial factor.}
\end{multipleChoice}


What is the name of the theorem that applies to this problem? \qquad \\
The \underline{\answer{Squeeze}} Theorem
\end{problem}}%}

%%%%%%%%%%%%%%%%%%%%%




\latexProblemContent{
\begin{problem}

The limit as $x\to{5}$ of $f(x)={{\left(x - 5\right)} \cos\left(\frac{23}{{\left(x - 5\right)}^{2}}\right)}$ is $0$.  What is the reason why this is true?

\input{2311-Concept-Limit-0008.HELP.tex}

\begin{multipleChoice}
\choice{The statement is in fact false: $\lim\limits_{x\to{5}}{{\left(x - 5\right)} \cos\left(\frac{23}{{\left(x - 5\right)}^{2}}\right)}\neq0$.}
\choice{The cosine factor decreases to $0$ faster than the polynomial.}
\choice[correct]{The cosine factor is bounded between $-1$ and $1$, so the polynomial forces the function to $0$.}
\choice{The cosine factor directly cancels out the polynomial factor.}
\end{multipleChoice}


What is the name of the theorem that applies to this problem? \qquad \\
The \underline{\answer{Squeeze}} Theorem
\end{problem}}%}

%%%%%%%%%%%%%%%%%%%%%




\latexProblemContent{
\begin{problem}

The limit as $x\to{-8}$ of $f(x)={{\left(x + 8\right)} \cos\left(-\frac{16}{{\left(x + 8\right)}^{2}}\right)}$ is $0$.  What is the reason why this is true?

\input{2311-Concept-Limit-0008.HELP.tex}

\begin{multipleChoice}
\choice{The statement is in fact false: $\lim\limits_{x\to{-8}}{{\left(x + 8\right)} \cos\left(-\frac{16}{{\left(x + 8\right)}^{2}}\right)}\neq0$.}
\choice{The cosine factor decreases to $0$ faster than the polynomial.}
\choice[correct]{The cosine factor is bounded between $-1$ and $1$, so the polynomial forces the function to $0$.}
\choice{The cosine factor directly cancels out the polynomial factor.}
\end{multipleChoice}


What is the name of the theorem that applies to this problem? \qquad \\
The \underline{\answer{Squeeze}} Theorem
\end{problem}}%}

%%%%%%%%%%%%%%%%%%%%%




\latexProblemContent{
\begin{problem}

The limit as $x\to{11}$ of $f(x)={{\left(x - 11\right)}^{2} \cos\left(-\frac{1}{x - 11}\right)}$ is $0$.  What is the reason why this is true?

\input{2311-Concept-Limit-0008.HELP.tex}

\begin{multipleChoice}
\choice{The statement is in fact false: $\lim\limits_{x\to{11}}{{\left(x - 11\right)}^{2} \cos\left(-\frac{1}{x - 11}\right)}\neq0$.}
\choice{The cosine factor decreases to $0$ faster than the polynomial.}
\choice[correct]{The cosine factor is bounded between $-1$ and $1$, so the polynomial forces the function to $0$.}
\choice{The cosine factor directly cancels out the polynomial factor.}
\end{multipleChoice}


What is the name of the theorem that applies to this problem? \qquad \\
The \underline{\answer{Squeeze}} Theorem
\end{problem}}%}

%%%%%%%%%%%%%%%%%%%%%




\latexProblemContent{
\begin{problem}

The limit as $x\to{5}$ of $f(x)={{\left(x - 5\right)} \cos\left(\frac{23}{x - 5}\right)}$ is $0$.  What is the reason why this is true?

\input{2311-Concept-Limit-0008.HELP.tex}

\begin{multipleChoice}
\choice{The statement is in fact false: $\lim\limits_{x\to{5}}{{\left(x - 5\right)} \cos\left(\frac{23}{x - 5}\right)}\neq0$.}
\choice{The cosine factor decreases to $0$ faster than the polynomial.}
\choice[correct]{The cosine factor is bounded between $-1$ and $1$, so the polynomial forces the function to $0$.}
\choice{The cosine factor directly cancels out the polynomial factor.}
\end{multipleChoice}


What is the name of the theorem that applies to this problem? \qquad \\
The \underline{\answer{Squeeze}} Theorem
\end{problem}}%}

%%%%%%%%%%%%%%%%%%%%%




\latexProblemContent{
\begin{problem}

The limit as $x\to{5}$ of $f(x)={{\left(x - 5\right)}^{3} \cos\left(\frac{15}{x - 5}\right)}$ is $0$.  What is the reason why this is true?

\input{2311-Concept-Limit-0008.HELP.tex}

\begin{multipleChoice}
\choice{The statement is in fact false: $\lim\limits_{x\to{5}}{{\left(x - 5\right)}^{3} \cos\left(\frac{15}{x - 5}\right)}\neq0$.}
\choice{The cosine factor decreases to $0$ faster than the polynomial.}
\choice[correct]{The cosine factor is bounded between $-1$ and $1$, so the polynomial forces the function to $0$.}
\choice{The cosine factor directly cancels out the polynomial factor.}
\end{multipleChoice}


What is the name of the theorem that applies to this problem? \qquad \\
The \underline{\answer{Squeeze}} Theorem
\end{problem}}%}

%%%%%%%%%%%%%%%%%%%%%




\latexProblemContent{
\begin{problem}

The limit as $x\to{1}$ of $f(x)={{\left(x - 1\right)} \cos\left(-\frac{1}{x - 1}\right)}$ is $0$.  What is the reason why this is true?

\input{2311-Concept-Limit-0008.HELP.tex}

\begin{multipleChoice}
\choice{The statement is in fact false: $\lim\limits_{x\to{1}}{{\left(x - 1\right)} \cos\left(-\frac{1}{x - 1}\right)}\neq0$.}
\choice{The cosine factor decreases to $0$ faster than the polynomial.}
\choice[correct]{The cosine factor is bounded between $-1$ and $1$, so the polynomial forces the function to $0$.}
\choice{The cosine factor directly cancels out the polynomial factor.}
\end{multipleChoice}


What is the name of the theorem that applies to this problem? \qquad \\
The \underline{\answer{Squeeze}} Theorem
\end{problem}}%}

%%%%%%%%%%%%%%%%%%%%%




\latexProblemContent{
\begin{problem}

The limit as $x\to{7}$ of $f(x)={{\left(x - 7\right)}^{2} \cos\left(\frac{12}{x - 7}\right)}$ is $0$.  What is the reason why this is true?

\input{2311-Concept-Limit-0008.HELP.tex}

\begin{multipleChoice}
\choice{The statement is in fact false: $\lim\limits_{x\to{7}}{{\left(x - 7\right)}^{2} \cos\left(\frac{12}{x - 7}\right)}\neq0$.}
\choice{The cosine factor decreases to $0$ faster than the polynomial.}
\choice[correct]{The cosine factor is bounded between $-1$ and $1$, so the polynomial forces the function to $0$.}
\choice{The cosine factor directly cancels out the polynomial factor.}
\end{multipleChoice}


What is the name of the theorem that applies to this problem? \qquad \\
The \underline{\answer{Squeeze}} Theorem
\end{problem}}%}

%%%%%%%%%%%%%%%%%%%%%




\latexProblemContent{
\begin{problem}

The limit as $x\to{4}$ of $f(x)={{\left(x - 4\right)} \cos\left(\frac{10}{{\left(x - 4\right)}^{2}}\right)}$ is $0$.  What is the reason why this is true?

\input{2311-Concept-Limit-0008.HELP.tex}

\begin{multipleChoice}
\choice{The statement is in fact false: $\lim\limits_{x\to{4}}{{\left(x - 4\right)} \cos\left(\frac{10}{{\left(x - 4\right)}^{2}}\right)}\neq0$.}
\choice{The cosine factor decreases to $0$ faster than the polynomial.}
\choice[correct]{The cosine factor is bounded between $-1$ and $1$, so the polynomial forces the function to $0$.}
\choice{The cosine factor directly cancels out the polynomial factor.}
\end{multipleChoice}


What is the name of the theorem that applies to this problem? \qquad \\
The \underline{\answer{Squeeze}} Theorem
\end{problem}}%}

%%%%%%%%%%%%%%%%%%%%%




\latexProblemContent{
\begin{problem}

The limit as $x\to{-13}$ of $f(x)={{\left(x + 13\right)} \cos\left(\frac{11}{x + 13}\right)}$ is $0$.  What is the reason why this is true?

\input{2311-Concept-Limit-0008.HELP.tex}

\begin{multipleChoice}
\choice{The statement is in fact false: $\lim\limits_{x\to{-13}}{{\left(x + 13\right)} \cos\left(\frac{11}{x + 13}\right)}\neq0$.}
\choice{The cosine factor decreases to $0$ faster than the polynomial.}
\choice[correct]{The cosine factor is bounded between $-1$ and $1$, so the polynomial forces the function to $0$.}
\choice{The cosine factor directly cancels out the polynomial factor.}
\end{multipleChoice}


What is the name of the theorem that applies to this problem? \qquad \\
The \underline{\answer{Squeeze}} Theorem
\end{problem}}%}

%%%%%%%%%%%%%%%%%%%%%




\latexProblemContent{
\begin{problem}

The limit as $x\to{6}$ of $f(x)={{\left(x - 6\right)}^{2} \cos\left(-\frac{9}{{\left(x - 6\right)}^{2}}\right)}$ is $0$.  What is the reason why this is true?

\input{2311-Concept-Limit-0008.HELP.tex}

\begin{multipleChoice}
\choice{The statement is in fact false: $\lim\limits_{x\to{6}}{{\left(x - 6\right)}^{2} \cos\left(-\frac{9}{{\left(x - 6\right)}^{2}}\right)}\neq0$.}
\choice{The cosine factor decreases to $0$ faster than the polynomial.}
\choice[correct]{The cosine factor is bounded between $-1$ and $1$, so the polynomial forces the function to $0$.}
\choice{The cosine factor directly cancels out the polynomial factor.}
\end{multipleChoice}


What is the name of the theorem that applies to this problem? \qquad \\
The \underline{\answer{Squeeze}} Theorem
\end{problem}}%}

%%%%%%%%%%%%%%%%%%%%%




\latexProblemContent{
\begin{problem}

The limit as $x\to{1}$ of $f(x)={{\left(x - 1\right)}^{3} \cos\left(-\frac{3}{{\left(x - 1\right)}^{2}}\right)}$ is $0$.  What is the reason why this is true?

\input{2311-Concept-Limit-0008.HELP.tex}

\begin{multipleChoice}
\choice{The statement is in fact false: $\lim\limits_{x\to{1}}{{\left(x - 1\right)}^{3} \cos\left(-\frac{3}{{\left(x - 1\right)}^{2}}\right)}\neq0$.}
\choice{The cosine factor decreases to $0$ faster than the polynomial.}
\choice[correct]{The cosine factor is bounded between $-1$ and $1$, so the polynomial forces the function to $0$.}
\choice{The cosine factor directly cancels out the polynomial factor.}
\end{multipleChoice}


What is the name of the theorem that applies to this problem? \qquad \\
The \underline{\answer{Squeeze}} Theorem
\end{problem}}%}

%%%%%%%%%%%%%%%%%%%%%




\latexProblemContent{
\begin{problem}

The limit as $x\to{-13}$ of $f(x)={{\left(x + 13\right)}^{2} \cos\left(\frac{24}{x + 13}\right)}$ is $0$.  What is the reason why this is true?

\input{2311-Concept-Limit-0008.HELP.tex}

\begin{multipleChoice}
\choice{The statement is in fact false: $\lim\limits_{x\to{-13}}{{\left(x + 13\right)}^{2} \cos\left(\frac{24}{x + 13}\right)}\neq0$.}
\choice{The cosine factor decreases to $0$ faster than the polynomial.}
\choice[correct]{The cosine factor is bounded between $-1$ and $1$, so the polynomial forces the function to $0$.}
\choice{The cosine factor directly cancels out the polynomial factor.}
\end{multipleChoice}


What is the name of the theorem that applies to this problem? \qquad \\
The \underline{\answer{Squeeze}} Theorem
\end{problem}}%}

%%%%%%%%%%%%%%%%%%%%%




\latexProblemContent{
\begin{problem}

The limit as $x\to{-8}$ of $f(x)={{\left(x + 8\right)} \cos\left(-\frac{9}{x + 8}\right)}$ is $0$.  What is the reason why this is true?

\input{2311-Concept-Limit-0008.HELP.tex}

\begin{multipleChoice}
\choice{The statement is in fact false: $\lim\limits_{x\to{-8}}{{\left(x + 8\right)} \cos\left(-\frac{9}{x + 8}\right)}\neq0$.}
\choice{The cosine factor decreases to $0$ faster than the polynomial.}
\choice[correct]{The cosine factor is bounded between $-1$ and $1$, so the polynomial forces the function to $0$.}
\choice{The cosine factor directly cancels out the polynomial factor.}
\end{multipleChoice}


What is the name of the theorem that applies to this problem? \qquad \\
The \underline{\answer{Squeeze}} Theorem
\end{problem}}%}

%%%%%%%%%%%%%%%%%%%%%




\latexProblemContent{
\begin{problem}

The limit as $x\to{-3}$ of $f(x)={{\left(x + 3\right)}^{2} \cos\left(-\frac{19}{{\left(x + 3\right)}^{2}}\right)}$ is $0$.  What is the reason why this is true?

\input{2311-Concept-Limit-0008.HELP.tex}

\begin{multipleChoice}
\choice{The statement is in fact false: $\lim\limits_{x\to{-3}}{{\left(x + 3\right)}^{2} \cos\left(-\frac{19}{{\left(x + 3\right)}^{2}}\right)}\neq0$.}
\choice{The cosine factor decreases to $0$ faster than the polynomial.}
\choice[correct]{The cosine factor is bounded between $-1$ and $1$, so the polynomial forces the function to $0$.}
\choice{The cosine factor directly cancels out the polynomial factor.}
\end{multipleChoice}


What is the name of the theorem that applies to this problem? \qquad \\
The \underline{\answer{Squeeze}} Theorem
\end{problem}}%}

%%%%%%%%%%%%%%%%%%%%%




%%%%%%%%%%%%%%%%%%%%%




\latexProblemContent{
\begin{problem}

The limit as $x\to{15}$ of $f(x)={{\left(x - 15\right)}^{2} \cos\left(\frac{15}{{\left(x - 15\right)}^{2}}\right)}$ is $0$.  What is the reason why this is true?

\input{2311-Concept-Limit-0008.HELP.tex}

\begin{multipleChoice}
\choice{The statement is in fact false: $\lim\limits_{x\to{15}}{{\left(x - 15\right)}^{2} \cos\left(\frac{15}{{\left(x - 15\right)}^{2}}\right)}\neq0$.}
\choice{The cosine factor decreases to $0$ faster than the polynomial.}
\choice[correct]{The cosine factor is bounded between $-1$ and $1$, so the polynomial forces the function to $0$.}
\choice{The cosine factor directly cancels out the polynomial factor.}
\end{multipleChoice}


What is the name of the theorem that applies to this problem? \qquad \\
The \underline{\answer{Squeeze}} Theorem
\end{problem}}%}

%%%%%%%%%%%%%%%%%%%%%




\latexProblemContent{
\begin{problem}

The limit as $x\to{10}$ of $f(x)={{\left(x - 10\right)} \cos\left(\frac{10}{x - 10}\right)}$ is $0$.  What is the reason why this is true?

\input{2311-Concept-Limit-0008.HELP.tex}

\begin{multipleChoice}
\choice{The statement is in fact false: $\lim\limits_{x\to{10}}{{\left(x - 10\right)} \cos\left(\frac{10}{x - 10}\right)}\neq0$.}
\choice{The cosine factor decreases to $0$ faster than the polynomial.}
\choice[correct]{The cosine factor is bounded between $-1$ and $1$, so the polynomial forces the function to $0$.}
\choice{The cosine factor directly cancels out the polynomial factor.}
\end{multipleChoice}


What is the name of the theorem that applies to this problem? \qquad \\
The \underline{\answer{Squeeze}} Theorem
\end{problem}}%}

%%%%%%%%%%%%%%%%%%%%%




\latexProblemContent{
\begin{problem}

The limit as $x\to{-1}$ of $f(x)={{\left(x + 1\right)} \cos\left(\frac{1}{{\left(x + 1\right)}^{2}}\right)}$ is $0$.  What is the reason why this is true?

\input{2311-Concept-Limit-0008.HELP.tex}

\begin{multipleChoice}
\choice{The statement is in fact false: $\lim\limits_{x\to{-1}}{{\left(x + 1\right)} \cos\left(\frac{1}{{\left(x + 1\right)}^{2}}\right)}\neq0$.}
\choice{The cosine factor decreases to $0$ faster than the polynomial.}
\choice[correct]{The cosine factor is bounded between $-1$ and $1$, so the polynomial forces the function to $0$.}
\choice{The cosine factor directly cancels out the polynomial factor.}
\end{multipleChoice}


What is the name of the theorem that applies to this problem? \qquad \\
The \underline{\answer{Squeeze}} Theorem
\end{problem}}%}

%%%%%%%%%%%%%%%%%%%%%




\latexProblemContent{
\begin{problem}

The limit as $x\to{5}$ of $f(x)={{\left(x - 5\right)} \cos\left(-\frac{2}{x - 5}\right)}$ is $0$.  What is the reason why this is true?

\input{2311-Concept-Limit-0008.HELP.tex}

\begin{multipleChoice}
\choice{The statement is in fact false: $\lim\limits_{x\to{5}}{{\left(x - 5\right)} \cos\left(-\frac{2}{x - 5}\right)}\neq0$.}
\choice{The cosine factor decreases to $0$ faster than the polynomial.}
\choice[correct]{The cosine factor is bounded between $-1$ and $1$, so the polynomial forces the function to $0$.}
\choice{The cosine factor directly cancels out the polynomial factor.}
\end{multipleChoice}


What is the name of the theorem that applies to this problem? \qquad \\
The \underline{\answer{Squeeze}} Theorem
\end{problem}}%}

%%%%%%%%%%%%%%%%%%%%%




\latexProblemContent{
\begin{problem}

The limit as $x\to{-11}$ of $f(x)={{\left(x + 11\right)} \cos\left(\frac{24}{x + 11}\right)}$ is $0$.  What is the reason why this is true?

\input{2311-Concept-Limit-0008.HELP.tex}

\begin{multipleChoice}
\choice{The statement is in fact false: $\lim\limits_{x\to{-11}}{{\left(x + 11\right)} \cos\left(\frac{24}{x + 11}\right)}\neq0$.}
\choice{The cosine factor decreases to $0$ faster than the polynomial.}
\choice[correct]{The cosine factor is bounded between $-1$ and $1$, so the polynomial forces the function to $0$.}
\choice{The cosine factor directly cancels out the polynomial factor.}
\end{multipleChoice}


What is the name of the theorem that applies to this problem? \qquad \\
The \underline{\answer{Squeeze}} Theorem
\end{problem}}%}

%%%%%%%%%%%%%%%%%%%%%




\latexProblemContent{
\begin{problem}

The limit as $x\to{-15}$ of $f(x)={{\left(x + 15\right)} \cos\left(-\frac{22}{x + 15}\right)}$ is $0$.  What is the reason why this is true?

\input{2311-Concept-Limit-0008.HELP.tex}

\begin{multipleChoice}
\choice{The statement is in fact false: $\lim\limits_{x\to{-15}}{{\left(x + 15\right)} \cos\left(-\frac{22}{x + 15}\right)}\neq0$.}
\choice{The cosine factor decreases to $0$ faster than the polynomial.}
\choice[correct]{The cosine factor is bounded between $-1$ and $1$, so the polynomial forces the function to $0$.}
\choice{The cosine factor directly cancels out the polynomial factor.}
\end{multipleChoice}


What is the name of the theorem that applies to this problem? \qquad \\
The \underline{\answer{Squeeze}} Theorem
\end{problem}}%}

%%%%%%%%%%%%%%%%%%%%%




\latexProblemContent{
\begin{problem}

The limit as $x\to{1}$ of $f(x)={{\left(x - 1\right)} \cos\left(-\frac{25}{x - 1}\right)}$ is $0$.  What is the reason why this is true?

\input{2311-Concept-Limit-0008.HELP.tex}

\begin{multipleChoice}
\choice{The statement is in fact false: $\lim\limits_{x\to{1}}{{\left(x - 1\right)} \cos\left(-\frac{25}{x - 1}\right)}\neq0$.}
\choice{The cosine factor decreases to $0$ faster than the polynomial.}
\choice[correct]{The cosine factor is bounded between $-1$ and $1$, so the polynomial forces the function to $0$.}
\choice{The cosine factor directly cancels out the polynomial factor.}
\end{multipleChoice}


What is the name of the theorem that applies to this problem? \qquad \\
The \underline{\answer{Squeeze}} Theorem
\end{problem}}%}

%%%%%%%%%%%%%%%%%%%%%




\latexProblemContent{
\begin{problem}

The limit as $x\to{-13}$ of $f(x)={{\left(x + 13\right)}^{2} \cos\left(-\frac{1}{{\left(x + 13\right)}^{2}}\right)}$ is $0$.  What is the reason why this is true?

\input{2311-Concept-Limit-0008.HELP.tex}

\begin{multipleChoice}
\choice{The statement is in fact false: $\lim\limits_{x\to{-13}}{{\left(x + 13\right)}^{2} \cos\left(-\frac{1}{{\left(x + 13\right)}^{2}}\right)}\neq0$.}
\choice{The cosine factor decreases to $0$ faster than the polynomial.}
\choice[correct]{The cosine factor is bounded between $-1$ and $1$, so the polynomial forces the function to $0$.}
\choice{The cosine factor directly cancels out the polynomial factor.}
\end{multipleChoice}


What is the name of the theorem that applies to this problem? \qquad \\
The \underline{\answer{Squeeze}} Theorem
\end{problem}}%}

%%%%%%%%%%%%%%%%%%%%%




\latexProblemContent{
\begin{problem}

The limit as $x\to{-11}$ of $f(x)={{\left(x + 11\right)} \cos\left(-\frac{14}{{\left(x + 11\right)}^{2}}\right)}$ is $0$.  What is the reason why this is true?

\input{2311-Concept-Limit-0008.HELP.tex}

\begin{multipleChoice}
\choice{The statement is in fact false: $\lim\limits_{x\to{-11}}{{\left(x + 11\right)} \cos\left(-\frac{14}{{\left(x + 11\right)}^{2}}\right)}\neq0$.}
\choice{The cosine factor decreases to $0$ faster than the polynomial.}
\choice[correct]{The cosine factor is bounded between $-1$ and $1$, so the polynomial forces the function to $0$.}
\choice{The cosine factor directly cancels out the polynomial factor.}
\end{multipleChoice}


What is the name of the theorem that applies to this problem? \qquad \\
The \underline{\answer{Squeeze}} Theorem
\end{problem}}%}

%%%%%%%%%%%%%%%%%%%%%




\latexProblemContent{
\begin{problem}

The limit as $x\to{-14}$ of $f(x)={{\left(x + 14\right)} \cos\left(-\frac{12}{x + 14}\right)}$ is $0$.  What is the reason why this is true?

\input{2311-Concept-Limit-0008.HELP.tex}

\begin{multipleChoice}
\choice{The statement is in fact false: $\lim\limits_{x\to{-14}}{{\left(x + 14\right)} \cos\left(-\frac{12}{x + 14}\right)}\neq0$.}
\choice{The cosine factor decreases to $0$ faster than the polynomial.}
\choice[correct]{The cosine factor is bounded between $-1$ and $1$, so the polynomial forces the function to $0$.}
\choice{The cosine factor directly cancels out the polynomial factor.}
\end{multipleChoice}


What is the name of the theorem that applies to this problem? \qquad \\
The \underline{\answer{Squeeze}} Theorem
\end{problem}}%}

%%%%%%%%%%%%%%%%%%%%%




\latexProblemContent{
\begin{problem}

The limit as $x\to{-14}$ of $f(x)={{\left(x + 14\right)}^{3} \cos\left(\frac{1}{x + 14}\right)}$ is $0$.  What is the reason why this is true?

\input{2311-Concept-Limit-0008.HELP.tex}

\begin{multipleChoice}
\choice{The statement is in fact false: $\lim\limits_{x\to{-14}}{{\left(x + 14\right)}^{3} \cos\left(\frac{1}{x + 14}\right)}\neq0$.}
\choice{The cosine factor decreases to $0$ faster than the polynomial.}
\choice[correct]{The cosine factor is bounded between $-1$ and $1$, so the polynomial forces the function to $0$.}
\choice{The cosine factor directly cancels out the polynomial factor.}
\end{multipleChoice}


What is the name of the theorem that applies to this problem? \qquad \\
The \underline{\answer{Squeeze}} Theorem
\end{problem}}%}

%%%%%%%%%%%%%%%%%%%%%




\latexProblemContent{
\begin{problem}

The limit as $x\to{-10}$ of $f(x)={{\left(x + 10\right)}^{3} \cos\left(\frac{11}{x + 10}\right)}$ is $0$.  What is the reason why this is true?

\input{2311-Concept-Limit-0008.HELP.tex}

\begin{multipleChoice}
\choice{The statement is in fact false: $\lim\limits_{x\to{-10}}{{\left(x + 10\right)}^{3} \cos\left(\frac{11}{x + 10}\right)}\neq0$.}
\choice{The cosine factor decreases to $0$ faster than the polynomial.}
\choice[correct]{The cosine factor is bounded between $-1$ and $1$, so the polynomial forces the function to $0$.}
\choice{The cosine factor directly cancels out the polynomial factor.}
\end{multipleChoice}


What is the name of the theorem that applies to this problem? \qquad \\
The \underline{\answer{Squeeze}} Theorem
\end{problem}}%}

%%%%%%%%%%%%%%%%%%%%%




\latexProblemContent{
\begin{problem}

The limit as $x\to{15}$ of $f(x)={{\left(x - 15\right)} \cos\left(-\frac{2}{{\left(x - 15\right)}^{2}}\right)}$ is $0$.  What is the reason why this is true?

\input{2311-Concept-Limit-0008.HELP.tex}

\begin{multipleChoice}
\choice{The statement is in fact false: $\lim\limits_{x\to{15}}{{\left(x - 15\right)} \cos\left(-\frac{2}{{\left(x - 15\right)}^{2}}\right)}\neq0$.}
\choice{The cosine factor decreases to $0$ faster than the polynomial.}
\choice[correct]{The cosine factor is bounded between $-1$ and $1$, so the polynomial forces the function to $0$.}
\choice{The cosine factor directly cancels out the polynomial factor.}
\end{multipleChoice}


What is the name of the theorem that applies to this problem? \qquad \\
The \underline{\answer{Squeeze}} Theorem
\end{problem}}%}

%%%%%%%%%%%%%%%%%%%%%




\latexProblemContent{
\begin{problem}

The limit as $x\to{13}$ of $f(x)={{\left(x - 13\right)}^{3} \cos\left(-\frac{25}{{\left(x - 13\right)}^{2}}\right)}$ is $0$.  What is the reason why this is true?

\input{2311-Concept-Limit-0008.HELP.tex}

\begin{multipleChoice}
\choice{The statement is in fact false: $\lim\limits_{x\to{13}}{{\left(x - 13\right)}^{3} \cos\left(-\frac{25}{{\left(x - 13\right)}^{2}}\right)}\neq0$.}
\choice{The cosine factor decreases to $0$ faster than the polynomial.}
\choice[correct]{The cosine factor is bounded between $-1$ and $1$, so the polynomial forces the function to $0$.}
\choice{The cosine factor directly cancels out the polynomial factor.}
\end{multipleChoice}


What is the name of the theorem that applies to this problem? \qquad \\
The \underline{\answer{Squeeze}} Theorem
\end{problem}}%}

%%%%%%%%%%%%%%%%%%%%%




%%%%%%%%%%%%%%%%%%%%%




\latexProblemContent{
\begin{problem}

The limit as $x\to{-10}$ of $f(x)={{\left(x + 10\right)}^{3} \cos\left(-\frac{15}{{\left(x + 10\right)}^{2}}\right)}$ is $0$.  What is the reason why this is true?

\input{2311-Concept-Limit-0008.HELP.tex}

\begin{multipleChoice}
\choice{The statement is in fact false: $\lim\limits_{x\to{-10}}{{\left(x + 10\right)}^{3} \cos\left(-\frac{15}{{\left(x + 10\right)}^{2}}\right)}\neq0$.}
\choice{The cosine factor decreases to $0$ faster than the polynomial.}
\choice[correct]{The cosine factor is bounded between $-1$ and $1$, so the polynomial forces the function to $0$.}
\choice{The cosine factor directly cancels out the polynomial factor.}
\end{multipleChoice}


What is the name of the theorem that applies to this problem? \qquad \\
The \underline{\answer{Squeeze}} Theorem
\end{problem}}%}

%%%%%%%%%%%%%%%%%%%%%




\latexProblemContent{
\begin{problem}

The limit as $x\to{-12}$ of $f(x)={{\left(x + 12\right)}^{3} \cos\left(-\frac{20}{x + 12}\right)}$ is $0$.  What is the reason why this is true?

\input{2311-Concept-Limit-0008.HELP.tex}

\begin{multipleChoice}
\choice{The statement is in fact false: $\lim\limits_{x\to{-12}}{{\left(x + 12\right)}^{3} \cos\left(-\frac{20}{x + 12}\right)}\neq0$.}
\choice{The cosine factor decreases to $0$ faster than the polynomial.}
\choice[correct]{The cosine factor is bounded between $-1$ and $1$, so the polynomial forces the function to $0$.}
\choice{The cosine factor directly cancels out the polynomial factor.}
\end{multipleChoice}


What is the name of the theorem that applies to this problem? \qquad \\
The \underline{\answer{Squeeze}} Theorem
\end{problem}}%}

%%%%%%%%%%%%%%%%%%%%%




\latexProblemContent{
\begin{problem}

The limit as $x\to{5}$ of $f(x)={{\left(x - 5\right)} \cos\left(-\frac{6}{x - 5}\right)}$ is $0$.  What is the reason why this is true?

\input{2311-Concept-Limit-0008.HELP.tex}

\begin{multipleChoice}
\choice{The statement is in fact false: $\lim\limits_{x\to{5}}{{\left(x - 5\right)} \cos\left(-\frac{6}{x - 5}\right)}\neq0$.}
\choice{The cosine factor decreases to $0$ faster than the polynomial.}
\choice[correct]{The cosine factor is bounded between $-1$ and $1$, so the polynomial forces the function to $0$.}
\choice{The cosine factor directly cancels out the polynomial factor.}
\end{multipleChoice}


What is the name of the theorem that applies to this problem? \qquad \\
The \underline{\answer{Squeeze}} Theorem
\end{problem}}%}

%%%%%%%%%%%%%%%%%%%%%




\latexProblemContent{
\begin{problem}

The limit as $x\to{5}$ of $f(x)={{\left(x - 5\right)} \cos\left(-\frac{17}{{\left(x - 5\right)}^{2}}\right)}$ is $0$.  What is the reason why this is true?

\input{2311-Concept-Limit-0008.HELP.tex}

\begin{multipleChoice}
\choice{The statement is in fact false: $\lim\limits_{x\to{5}}{{\left(x - 5\right)} \cos\left(-\frac{17}{{\left(x - 5\right)}^{2}}\right)}\neq0$.}
\choice{The cosine factor decreases to $0$ faster than the polynomial.}
\choice[correct]{The cosine factor is bounded between $-1$ and $1$, so the polynomial forces the function to $0$.}
\choice{The cosine factor directly cancels out the polynomial factor.}
\end{multipleChoice}


What is the name of the theorem that applies to this problem? \qquad \\
The \underline{\answer{Squeeze}} Theorem
\end{problem}}%}

%%%%%%%%%%%%%%%%%%%%%




\latexProblemContent{
\begin{problem}

The limit as $x\to{-3}$ of $f(x)={{\left(x + 3\right)}^{3} \cos\left(\frac{16}{x + 3}\right)}$ is $0$.  What is the reason why this is true?

\input{2311-Concept-Limit-0008.HELP.tex}

\begin{multipleChoice}
\choice{The statement is in fact false: $\lim\limits_{x\to{-3}}{{\left(x + 3\right)}^{3} \cos\left(\frac{16}{x + 3}\right)}\neq0$.}
\choice{The cosine factor decreases to $0$ faster than the polynomial.}
\choice[correct]{The cosine factor is bounded between $-1$ and $1$, so the polynomial forces the function to $0$.}
\choice{The cosine factor directly cancels out the polynomial factor.}
\end{multipleChoice}


What is the name of the theorem that applies to this problem? \qquad \\
The \underline{\answer{Squeeze}} Theorem
\end{problem}}%}

%%%%%%%%%%%%%%%%%%%%%




\latexProblemContent{
\begin{problem}

The limit as $x\to{-14}$ of $f(x)={{\left(x + 14\right)}^{2} \cos\left(\frac{5}{{\left(x + 14\right)}^{2}}\right)}$ is $0$.  What is the reason why this is true?

\input{2311-Concept-Limit-0008.HELP.tex}

\begin{multipleChoice}
\choice{The statement is in fact false: $\lim\limits_{x\to{-14}}{{\left(x + 14\right)}^{2} \cos\left(\frac{5}{{\left(x + 14\right)}^{2}}\right)}\neq0$.}
\choice{The cosine factor decreases to $0$ faster than the polynomial.}
\choice[correct]{The cosine factor is bounded between $-1$ and $1$, so the polynomial forces the function to $0$.}
\choice{The cosine factor directly cancels out the polynomial factor.}
\end{multipleChoice}


What is the name of the theorem that applies to this problem? \qquad \\
The \underline{\answer{Squeeze}} Theorem
\end{problem}}%}

%%%%%%%%%%%%%%%%%%%%%




\latexProblemContent{
\begin{problem}

The limit as $x\to{4}$ of $f(x)={{\left(x - 4\right)}^{3} \cos\left(\frac{6}{{\left(x - 4\right)}^{2}}\right)}$ is $0$.  What is the reason why this is true?

\input{2311-Concept-Limit-0008.HELP.tex}

\begin{multipleChoice}
\choice{The statement is in fact false: $\lim\limits_{x\to{4}}{{\left(x - 4\right)}^{3} \cos\left(\frac{6}{{\left(x - 4\right)}^{2}}\right)}\neq0$.}
\choice{The cosine factor decreases to $0$ faster than the polynomial.}
\choice[correct]{The cosine factor is bounded between $-1$ and $1$, so the polynomial forces the function to $0$.}
\choice{The cosine factor directly cancels out the polynomial factor.}
\end{multipleChoice}


What is the name of the theorem that applies to this problem? \qquad \\
The \underline{\answer{Squeeze}} Theorem
\end{problem}}%}

%%%%%%%%%%%%%%%%%%%%%




\latexProblemContent{
\begin{problem}

The limit as $x\to{-9}$ of $f(x)={{\left(x + 9\right)}^{3} \cos\left(-\frac{11}{x + 9}\right)}$ is $0$.  What is the reason why this is true?

\input{2311-Concept-Limit-0008.HELP.tex}

\begin{multipleChoice}
\choice{The statement is in fact false: $\lim\limits_{x\to{-9}}{{\left(x + 9\right)}^{3} \cos\left(-\frac{11}{x + 9}\right)}\neq0$.}
\choice{The cosine factor decreases to $0$ faster than the polynomial.}
\choice[correct]{The cosine factor is bounded between $-1$ and $1$, so the polynomial forces the function to $0$.}
\choice{The cosine factor directly cancels out the polynomial factor.}
\end{multipleChoice}


What is the name of the theorem that applies to this problem? \qquad \\
The \underline{\answer{Squeeze}} Theorem
\end{problem}}%}

%%%%%%%%%%%%%%%%%%%%%




\latexProblemContent{
\begin{problem}

The limit as $x\to{-14}$ of $f(x)={{\left(x + 14\right)} \cos\left(\frac{9}{x + 14}\right)}$ is $0$.  What is the reason why this is true?

\input{2311-Concept-Limit-0008.HELP.tex}

\begin{multipleChoice}
\choice{The statement is in fact false: $\lim\limits_{x\to{-14}}{{\left(x + 14\right)} \cos\left(\frac{9}{x + 14}\right)}\neq0$.}
\choice{The cosine factor decreases to $0$ faster than the polynomial.}
\choice[correct]{The cosine factor is bounded between $-1$ and $1$, so the polynomial forces the function to $0$.}
\choice{The cosine factor directly cancels out the polynomial factor.}
\end{multipleChoice}


What is the name of the theorem that applies to this problem? \qquad \\
The \underline{\answer{Squeeze}} Theorem
\end{problem}}%}

%%%%%%%%%%%%%%%%%%%%%




\latexProblemContent{
\begin{problem}

The limit as $x\to{1}$ of $f(x)={{\left(x - 1\right)} \cos\left(\frac{18}{x - 1}\right)}$ is $0$.  What is the reason why this is true?

\input{2311-Concept-Limit-0008.HELP.tex}

\begin{multipleChoice}
\choice{The statement is in fact false: $\lim\limits_{x\to{1}}{{\left(x - 1\right)} \cos\left(\frac{18}{x - 1}\right)}\neq0$.}
\choice{The cosine factor decreases to $0$ faster than the polynomial.}
\choice[correct]{The cosine factor is bounded between $-1$ and $1$, so the polynomial forces the function to $0$.}
\choice{The cosine factor directly cancels out the polynomial factor.}
\end{multipleChoice}


What is the name of the theorem that applies to this problem? \qquad \\
The \underline{\answer{Squeeze}} Theorem
\end{problem}}%}

%%%%%%%%%%%%%%%%%%%%%




\latexProblemContent{
\begin{problem}

The limit as $x\to{13}$ of $f(x)={{\left(x - 13\right)}^{3} \cos\left(\frac{16}{{\left(x - 13\right)}^{2}}\right)}$ is $0$.  What is the reason why this is true?

\input{2311-Concept-Limit-0008.HELP.tex}

\begin{multipleChoice}
\choice{The statement is in fact false: $\lim\limits_{x\to{13}}{{\left(x - 13\right)}^{3} \cos\left(\frac{16}{{\left(x - 13\right)}^{2}}\right)}\neq0$.}
\choice{The cosine factor decreases to $0$ faster than the polynomial.}
\choice[correct]{The cosine factor is bounded between $-1$ and $1$, so the polynomial forces the function to $0$.}
\choice{The cosine factor directly cancels out the polynomial factor.}
\end{multipleChoice}


What is the name of the theorem that applies to this problem? \qquad \\
The \underline{\answer{Squeeze}} Theorem
\end{problem}}%}

%%%%%%%%%%%%%%%%%%%%%




\latexProblemContent{
\begin{problem}

The limit as $x\to{13}$ of $f(x)={{\left(x - 13\right)}^{3} \cos\left(\frac{17}{x - 13}\right)}$ is $0$.  What is the reason why this is true?

\input{2311-Concept-Limit-0008.HELP.tex}

\begin{multipleChoice}
\choice{The statement is in fact false: $\lim\limits_{x\to{13}}{{\left(x - 13\right)}^{3} \cos\left(\frac{17}{x - 13}\right)}\neq0$.}
\choice{The cosine factor decreases to $0$ faster than the polynomial.}
\choice[correct]{The cosine factor is bounded between $-1$ and $1$, so the polynomial forces the function to $0$.}
\choice{The cosine factor directly cancels out the polynomial factor.}
\end{multipleChoice}


What is the name of the theorem that applies to this problem? \qquad \\
The \underline{\answer{Squeeze}} Theorem
\end{problem}}%}

%%%%%%%%%%%%%%%%%%%%%




\latexProblemContent{
\begin{problem}

The limit as $x\to{-10}$ of $f(x)={{\left(x + 10\right)}^{3} \cos\left(-\frac{11}{{\left(x + 10\right)}^{2}}\right)}$ is $0$.  What is the reason why this is true?

\input{2311-Concept-Limit-0008.HELP.tex}

\begin{multipleChoice}
\choice{The statement is in fact false: $\lim\limits_{x\to{-10}}{{\left(x + 10\right)}^{3} \cos\left(-\frac{11}{{\left(x + 10\right)}^{2}}\right)}\neq0$.}
\choice{The cosine factor decreases to $0$ faster than the polynomial.}
\choice[correct]{The cosine factor is bounded between $-1$ and $1$, so the polynomial forces the function to $0$.}
\choice{The cosine factor directly cancels out the polynomial factor.}
\end{multipleChoice}


What is the name of the theorem that applies to this problem? \qquad \\
The \underline{\answer{Squeeze}} Theorem
\end{problem}}%}

%%%%%%%%%%%%%%%%%%%%%




\latexProblemContent{
\begin{problem}

The limit as $x\to{12}$ of $f(x)={{\left(x - 12\right)}^{3} \cos\left(\frac{12}{{\left(x - 12\right)}^{2}}\right)}$ is $0$.  What is the reason why this is true?

\input{2311-Concept-Limit-0008.HELP.tex}

\begin{multipleChoice}
\choice{The statement is in fact false: $\lim\limits_{x\to{12}}{{\left(x - 12\right)}^{3} \cos\left(\frac{12}{{\left(x - 12\right)}^{2}}\right)}\neq0$.}
\choice{The cosine factor decreases to $0$ faster than the polynomial.}
\choice[correct]{The cosine factor is bounded between $-1$ and $1$, so the polynomial forces the function to $0$.}
\choice{The cosine factor directly cancels out the polynomial factor.}
\end{multipleChoice}


What is the name of the theorem that applies to this problem? \qquad \\
The \underline{\answer{Squeeze}} Theorem
\end{problem}}%}

%%%%%%%%%%%%%%%%%%%%%




\latexProblemContent{
\begin{problem}

The limit as $x\to{-9}$ of $f(x)={{\left(x + 9\right)}^{2} \cos\left(\frac{3}{{\left(x + 9\right)}^{2}}\right)}$ is $0$.  What is the reason why this is true?

\input{2311-Concept-Limit-0008.HELP.tex}

\begin{multipleChoice}
\choice{The statement is in fact false: $\lim\limits_{x\to{-9}}{{\left(x + 9\right)}^{2} \cos\left(\frac{3}{{\left(x + 9\right)}^{2}}\right)}\neq0$.}
\choice{The cosine factor decreases to $0$ faster than the polynomial.}
\choice[correct]{The cosine factor is bounded between $-1$ and $1$, so the polynomial forces the function to $0$.}
\choice{The cosine factor directly cancels out the polynomial factor.}
\end{multipleChoice}


What is the name of the theorem that applies to this problem? \qquad \\
The \underline{\answer{Squeeze}} Theorem
\end{problem}}%}

%%%%%%%%%%%%%%%%%%%%%




\latexProblemContent{
\begin{problem}

The limit as $x\to{3}$ of $f(x)={{\left(x - 3\right)}^{3} \cos\left(-\frac{9}{{\left(x - 3\right)}^{2}}\right)}$ is $0$.  What is the reason why this is true?

\input{2311-Concept-Limit-0008.HELP.tex}

\begin{multipleChoice}
\choice{The statement is in fact false: $\lim\limits_{x\to{3}}{{\left(x - 3\right)}^{3} \cos\left(-\frac{9}{{\left(x - 3\right)}^{2}}\right)}\neq0$.}
\choice{The cosine factor decreases to $0$ faster than the polynomial.}
\choice[correct]{The cosine factor is bounded between $-1$ and $1$, so the polynomial forces the function to $0$.}
\choice{The cosine factor directly cancels out the polynomial factor.}
\end{multipleChoice}


What is the name of the theorem that applies to this problem? \qquad \\
The \underline{\answer{Squeeze}} Theorem
\end{problem}}%}

%%%%%%%%%%%%%%%%%%%%%




\latexProblemContent{
\begin{problem}

The limit as $x\to{1}$ of $f(x)={{\left(x - 1\right)}^{2} \cos\left(-\frac{17}{x - 1}\right)}$ is $0$.  What is the reason why this is true?

\input{2311-Concept-Limit-0008.HELP.tex}

\begin{multipleChoice}
\choice{The statement is in fact false: $\lim\limits_{x\to{1}}{{\left(x - 1\right)}^{2} \cos\left(-\frac{17}{x - 1}\right)}\neq0$.}
\choice{The cosine factor decreases to $0$ faster than the polynomial.}
\choice[correct]{The cosine factor is bounded between $-1$ and $1$, so the polynomial forces the function to $0$.}
\choice{The cosine factor directly cancels out the polynomial factor.}
\end{multipleChoice}


What is the name of the theorem that applies to this problem? \qquad \\
The \underline{\answer{Squeeze}} Theorem
\end{problem}}%}

%%%%%%%%%%%%%%%%%%%%%




\latexProblemContent{
\begin{problem}

The limit as $x\to{-15}$ of $f(x)={{\left(x + 15\right)} \cos\left(-\frac{15}{{\left(x + 15\right)}^{2}}\right)}$ is $0$.  What is the reason why this is true?

\input{2311-Concept-Limit-0008.HELP.tex}

\begin{multipleChoice}
\choice{The statement is in fact false: $\lim\limits_{x\to{-15}}{{\left(x + 15\right)} \cos\left(-\frac{15}{{\left(x + 15\right)}^{2}}\right)}\neq0$.}
\choice{The cosine factor decreases to $0$ faster than the polynomial.}
\choice[correct]{The cosine factor is bounded between $-1$ and $1$, so the polynomial forces the function to $0$.}
\choice{The cosine factor directly cancels out the polynomial factor.}
\end{multipleChoice}


What is the name of the theorem that applies to this problem? \qquad \\
The \underline{\answer{Squeeze}} Theorem
\end{problem}}%}

%%%%%%%%%%%%%%%%%%%%%




\latexProblemContent{
\begin{problem}

The limit as $x\to{-3}$ of $f(x)={{\left(x + 3\right)}^{3} \cos\left(-\frac{15}{{\left(x + 3\right)}^{2}}\right)}$ is $0$.  What is the reason why this is true?

\input{2311-Concept-Limit-0008.HELP.tex}

\begin{multipleChoice}
\choice{The statement is in fact false: $\lim\limits_{x\to{-3}}{{\left(x + 3\right)}^{3} \cos\left(-\frac{15}{{\left(x + 3\right)}^{2}}\right)}\neq0$.}
\choice{The cosine factor decreases to $0$ faster than the polynomial.}
\choice[correct]{The cosine factor is bounded between $-1$ and $1$, so the polynomial forces the function to $0$.}
\choice{The cosine factor directly cancels out the polynomial factor.}
\end{multipleChoice}


What is the name of the theorem that applies to this problem? \qquad \\
The \underline{\answer{Squeeze}} Theorem
\end{problem}}%}

%%%%%%%%%%%%%%%%%%%%%




\latexProblemContent{
\begin{problem}

The limit as $x\to{-7}$ of $f(x)={{\left(x + 7\right)} \cos\left(-\frac{15}{x + 7}\right)}$ is $0$.  What is the reason why this is true?

\input{2311-Concept-Limit-0008.HELP.tex}

\begin{multipleChoice}
\choice{The statement is in fact false: $\lim\limits_{x\to{-7}}{{\left(x + 7\right)} \cos\left(-\frac{15}{x + 7}\right)}\neq0$.}
\choice{The cosine factor decreases to $0$ faster than the polynomial.}
\choice[correct]{The cosine factor is bounded between $-1$ and $1$, so the polynomial forces the function to $0$.}
\choice{The cosine factor directly cancels out the polynomial factor.}
\end{multipleChoice}


What is the name of the theorem that applies to this problem? \qquad \\
The \underline{\answer{Squeeze}} Theorem
\end{problem}}%}

%%%%%%%%%%%%%%%%%%%%%




%%%%%%%%%%%%%%%%%%%%%




\latexProblemContent{
\begin{problem}

The limit as $x\to{-13}$ of $f(x)={{\left(x + 13\right)} \cos\left(-\frac{14}{x + 13}\right)}$ is $0$.  What is the reason why this is true?

\input{2311-Concept-Limit-0008.HELP.tex}

\begin{multipleChoice}
\choice{The statement is in fact false: $\lim\limits_{x\to{-13}}{{\left(x + 13\right)} \cos\left(-\frac{14}{x + 13}\right)}\neq0$.}
\choice{The cosine factor decreases to $0$ faster than the polynomial.}
\choice[correct]{The cosine factor is bounded between $-1$ and $1$, so the polynomial forces the function to $0$.}
\choice{The cosine factor directly cancels out the polynomial factor.}
\end{multipleChoice}


What is the name of the theorem that applies to this problem? \qquad \\
The \underline{\answer{Squeeze}} Theorem
\end{problem}}%}

%%%%%%%%%%%%%%%%%%%%%




\latexProblemContent{
\begin{problem}

The limit as $x\to{-8}$ of $f(x)={{\left(x + 8\right)} \cos\left(-\frac{20}{x + 8}\right)}$ is $0$.  What is the reason why this is true?

\input{2311-Concept-Limit-0008.HELP.tex}

\begin{multipleChoice}
\choice{The statement is in fact false: $\lim\limits_{x\to{-8}}{{\left(x + 8\right)} \cos\left(-\frac{20}{x + 8}\right)}\neq0$.}
\choice{The cosine factor decreases to $0$ faster than the polynomial.}
\choice[correct]{The cosine factor is bounded between $-1$ and $1$, so the polynomial forces the function to $0$.}
\choice{The cosine factor directly cancels out the polynomial factor.}
\end{multipleChoice}


What is the name of the theorem that applies to this problem? \qquad \\
The \underline{\answer{Squeeze}} Theorem
\end{problem}}%}

%%%%%%%%%%%%%%%%%%%%%




\latexProblemContent{
\begin{problem}

The limit as $x\to{8}$ of $f(x)={{\left(x - 8\right)}^{3} \cos\left(-\frac{6}{{\left(x - 8\right)}^{2}}\right)}$ is $0$.  What is the reason why this is true?

\input{2311-Concept-Limit-0008.HELP.tex}

\begin{multipleChoice}
\choice{The statement is in fact false: $\lim\limits_{x\to{8}}{{\left(x - 8\right)}^{3} \cos\left(-\frac{6}{{\left(x - 8\right)}^{2}}\right)}\neq0$.}
\choice{The cosine factor decreases to $0$ faster than the polynomial.}
\choice[correct]{The cosine factor is bounded between $-1$ and $1$, so the polynomial forces the function to $0$.}
\choice{The cosine factor directly cancels out the polynomial factor.}
\end{multipleChoice}


What is the name of the theorem that applies to this problem? \qquad \\
The \underline{\answer{Squeeze}} Theorem
\end{problem}}%}

%%%%%%%%%%%%%%%%%%%%%




\latexProblemContent{
\begin{problem}

The limit as $x\to{-10}$ of $f(x)={{\left(x + 10\right)}^{3} \cos\left(-\frac{25}{x + 10}\right)}$ is $0$.  What is the reason why this is true?

\input{2311-Concept-Limit-0008.HELP.tex}

\begin{multipleChoice}
\choice{The statement is in fact false: $\lim\limits_{x\to{-10}}{{\left(x + 10\right)}^{3} \cos\left(-\frac{25}{x + 10}\right)}\neq0$.}
\choice{The cosine factor decreases to $0$ faster than the polynomial.}
\choice[correct]{The cosine factor is bounded between $-1$ and $1$, so the polynomial forces the function to $0$.}
\choice{The cosine factor directly cancels out the polynomial factor.}
\end{multipleChoice}


What is the name of the theorem that applies to this problem? \qquad \\
The \underline{\answer{Squeeze}} Theorem
\end{problem}}%}

%%%%%%%%%%%%%%%%%%%%%




\latexProblemContent{
\begin{problem}

The limit as $x\to{1}$ of $f(x)={{\left(x - 1\right)}^{3} \cos\left(\frac{4}{{\left(x - 1\right)}^{2}}\right)}$ is $0$.  What is the reason why this is true?

\input{2311-Concept-Limit-0008.HELP.tex}

\begin{multipleChoice}
\choice{The statement is in fact false: $\lim\limits_{x\to{1}}{{\left(x - 1\right)}^{3} \cos\left(\frac{4}{{\left(x - 1\right)}^{2}}\right)}\neq0$.}
\choice{The cosine factor decreases to $0$ faster than the polynomial.}
\choice[correct]{The cosine factor is bounded between $-1$ and $1$, so the polynomial forces the function to $0$.}
\choice{The cosine factor directly cancels out the polynomial factor.}
\end{multipleChoice}


What is the name of the theorem that applies to this problem? \qquad \\
The \underline{\answer{Squeeze}} Theorem
\end{problem}}%}

%%%%%%%%%%%%%%%%%%%%%




\latexProblemContent{
\begin{problem}

The limit as $x\to{-4}$ of $f(x)={{\left(x + 4\right)}^{2} \cos\left(-\frac{21}{x + 4}\right)}$ is $0$.  What is the reason why this is true?

\input{2311-Concept-Limit-0008.HELP.tex}

\begin{multipleChoice}
\choice{The statement is in fact false: $\lim\limits_{x\to{-4}}{{\left(x + 4\right)}^{2} \cos\left(-\frac{21}{x + 4}\right)}\neq0$.}
\choice{The cosine factor decreases to $0$ faster than the polynomial.}
\choice[correct]{The cosine factor is bounded between $-1$ and $1$, so the polynomial forces the function to $0$.}
\choice{The cosine factor directly cancels out the polynomial factor.}
\end{multipleChoice}


What is the name of the theorem that applies to this problem? \qquad \\
The \underline{\answer{Squeeze}} Theorem
\end{problem}}%}

%%%%%%%%%%%%%%%%%%%%%




\latexProblemContent{
\begin{problem}

The limit as $x\to{1}$ of $f(x)={{\left(x - 1\right)} \cos\left(-\frac{14}{{\left(x - 1\right)}^{2}}\right)}$ is $0$.  What is the reason why this is true?

\input{2311-Concept-Limit-0008.HELP.tex}

\begin{multipleChoice}
\choice{The statement is in fact false: $\lim\limits_{x\to{1}}{{\left(x - 1\right)} \cos\left(-\frac{14}{{\left(x - 1\right)}^{2}}\right)}\neq0$.}
\choice{The cosine factor decreases to $0$ faster than the polynomial.}
\choice[correct]{The cosine factor is bounded between $-1$ and $1$, so the polynomial forces the function to $0$.}
\choice{The cosine factor directly cancels out the polynomial factor.}
\end{multipleChoice}


What is the name of the theorem that applies to this problem? \qquad \\
The \underline{\answer{Squeeze}} Theorem
\end{problem}}%}

%%%%%%%%%%%%%%%%%%%%%




\latexProblemContent{
\begin{problem}

The limit as $x\to{4}$ of $f(x)={{\left(x - 4\right)}^{3} \cos\left(\frac{8}{x - 4}\right)}$ is $0$.  What is the reason why this is true?

\input{2311-Concept-Limit-0008.HELP.tex}

\begin{multipleChoice}
\choice{The statement is in fact false: $\lim\limits_{x\to{4}}{{\left(x - 4\right)}^{3} \cos\left(\frac{8}{x - 4}\right)}\neq0$.}
\choice{The cosine factor decreases to $0$ faster than the polynomial.}
\choice[correct]{The cosine factor is bounded between $-1$ and $1$, so the polynomial forces the function to $0$.}
\choice{The cosine factor directly cancels out the polynomial factor.}
\end{multipleChoice}


What is the name of the theorem that applies to this problem? \qquad \\
The \underline{\answer{Squeeze}} Theorem
\end{problem}}%}

%%%%%%%%%%%%%%%%%%%%%




\latexProblemContent{
\begin{problem}

The limit as $x\to{5}$ of $f(x)={{\left(x - 5\right)} \cos\left(-\frac{7}{{\left(x - 5\right)}^{2}}\right)}$ is $0$.  What is the reason why this is true?

\input{2311-Concept-Limit-0008.HELP.tex}

\begin{multipleChoice}
\choice{The statement is in fact false: $\lim\limits_{x\to{5}}{{\left(x - 5\right)} \cos\left(-\frac{7}{{\left(x - 5\right)}^{2}}\right)}\neq0$.}
\choice{The cosine factor decreases to $0$ faster than the polynomial.}
\choice[correct]{The cosine factor is bounded between $-1$ and $1$, so the polynomial forces the function to $0$.}
\choice{The cosine factor directly cancels out the polynomial factor.}
\end{multipleChoice}


What is the name of the theorem that applies to this problem? \qquad \\
The \underline{\answer{Squeeze}} Theorem
\end{problem}}%}

%%%%%%%%%%%%%%%%%%%%%




\latexProblemContent{
\begin{problem}

The limit as $x\to{-13}$ of $f(x)={{\left(x + 13\right)}^{3} \cos\left(\frac{18}{{\left(x + 13\right)}^{2}}\right)}$ is $0$.  What is the reason why this is true?

\input{2311-Concept-Limit-0008.HELP.tex}

\begin{multipleChoice}
\choice{The statement is in fact false: $\lim\limits_{x\to{-13}}{{\left(x + 13\right)}^{3} \cos\left(\frac{18}{{\left(x + 13\right)}^{2}}\right)}\neq0$.}
\choice{The cosine factor decreases to $0$ faster than the polynomial.}
\choice[correct]{The cosine factor is bounded between $-1$ and $1$, so the polynomial forces the function to $0$.}
\choice{The cosine factor directly cancels out the polynomial factor.}
\end{multipleChoice}


What is the name of the theorem that applies to this problem? \qquad \\
The \underline{\answer{Squeeze}} Theorem
\end{problem}}%}

%%%%%%%%%%%%%%%%%%%%%




\latexProblemContent{
\begin{problem}

The limit as $x\to{10}$ of $f(x)={{\left(x - 10\right)}^{2} \cos\left(-\frac{25}{x - 10}\right)}$ is $0$.  What is the reason why this is true?

\input{2311-Concept-Limit-0008.HELP.tex}

\begin{multipleChoice}
\choice{The statement is in fact false: $\lim\limits_{x\to{10}}{{\left(x - 10\right)}^{2} \cos\left(-\frac{25}{x - 10}\right)}\neq0$.}
\choice{The cosine factor decreases to $0$ faster than the polynomial.}
\choice[correct]{The cosine factor is bounded between $-1$ and $1$, so the polynomial forces the function to $0$.}
\choice{The cosine factor directly cancels out the polynomial factor.}
\end{multipleChoice}


What is the name of the theorem that applies to this problem? \qquad \\
The \underline{\answer{Squeeze}} Theorem
\end{problem}}%}

%%%%%%%%%%%%%%%%%%%%%




\latexProblemContent{
\begin{problem}

The limit as $x\to{13}$ of $f(x)={{\left(x - 13\right)} \cos\left(-\frac{13}{x - 13}\right)}$ is $0$.  What is the reason why this is true?

\input{2311-Concept-Limit-0008.HELP.tex}

\begin{multipleChoice}
\choice{The statement is in fact false: $\lim\limits_{x\to{13}}{{\left(x - 13\right)} \cos\left(-\frac{13}{x - 13}\right)}\neq0$.}
\choice{The cosine factor decreases to $0$ faster than the polynomial.}
\choice[correct]{The cosine factor is bounded between $-1$ and $1$, so the polynomial forces the function to $0$.}
\choice{The cosine factor directly cancels out the polynomial factor.}
\end{multipleChoice}


What is the name of the theorem that applies to this problem? \qquad \\
The \underline{\answer{Squeeze}} Theorem
\end{problem}}%}

%%%%%%%%%%%%%%%%%%%%%




\latexProblemContent{
\begin{problem}

The limit as $x\to{-5}$ of $f(x)={{\left(x + 5\right)} \cos\left(-\frac{12}{{\left(x + 5\right)}^{2}}\right)}$ is $0$.  What is the reason why this is true?

\input{2311-Concept-Limit-0008.HELP.tex}

\begin{multipleChoice}
\choice{The statement is in fact false: $\lim\limits_{x\to{-5}}{{\left(x + 5\right)} \cos\left(-\frac{12}{{\left(x + 5\right)}^{2}}\right)}\neq0$.}
\choice{The cosine factor decreases to $0$ faster than the polynomial.}
\choice[correct]{The cosine factor is bounded between $-1$ and $1$, so the polynomial forces the function to $0$.}
\choice{The cosine factor directly cancels out the polynomial factor.}
\end{multipleChoice}


What is the name of the theorem that applies to this problem? \qquad \\
The \underline{\answer{Squeeze}} Theorem
\end{problem}}%}

%%%%%%%%%%%%%%%%%%%%%




\latexProblemContent{
\begin{problem}

The limit as $x\to{13}$ of $f(x)={{\left(x - 13\right)}^{3} \cos\left(-\frac{15}{{\left(x - 13\right)}^{2}}\right)}$ is $0$.  What is the reason why this is true?

\input{2311-Concept-Limit-0008.HELP.tex}

\begin{multipleChoice}
\choice{The statement is in fact false: $\lim\limits_{x\to{13}}{{\left(x - 13\right)}^{3} \cos\left(-\frac{15}{{\left(x - 13\right)}^{2}}\right)}\neq0$.}
\choice{The cosine factor decreases to $0$ faster than the polynomial.}
\choice[correct]{The cosine factor is bounded between $-1$ and $1$, so the polynomial forces the function to $0$.}
\choice{The cosine factor directly cancels out the polynomial factor.}
\end{multipleChoice}


What is the name of the theorem that applies to this problem? \qquad \\
The \underline{\answer{Squeeze}} Theorem
\end{problem}}%}

%%%%%%%%%%%%%%%%%%%%%




\latexProblemContent{
\begin{problem}

The limit as $x\to{9}$ of $f(x)={{\left(x - 9\right)}^{3} \cos\left(\frac{3}{{\left(x - 9\right)}^{2}}\right)}$ is $0$.  What is the reason why this is true?

\input{2311-Concept-Limit-0008.HELP.tex}

\begin{multipleChoice}
\choice{The statement is in fact false: $\lim\limits_{x\to{9}}{{\left(x - 9\right)}^{3} \cos\left(\frac{3}{{\left(x - 9\right)}^{2}}\right)}\neq0$.}
\choice{The cosine factor decreases to $0$ faster than the polynomial.}
\choice[correct]{The cosine factor is bounded between $-1$ and $1$, so the polynomial forces the function to $0$.}
\choice{The cosine factor directly cancels out the polynomial factor.}
\end{multipleChoice}


What is the name of the theorem that applies to this problem? \qquad \\
The \underline{\answer{Squeeze}} Theorem
\end{problem}}%}

%%%%%%%%%%%%%%%%%%%%%




\latexProblemContent{
\begin{problem}

The limit as $x\to{14}$ of $f(x)={{\left(x - 14\right)}^{3} \cos\left(-\frac{15}{{\left(x - 14\right)}^{2}}\right)}$ is $0$.  What is the reason why this is true?

\input{2311-Concept-Limit-0008.HELP.tex}

\begin{multipleChoice}
\choice{The statement is in fact false: $\lim\limits_{x\to{14}}{{\left(x - 14\right)}^{3} \cos\left(-\frac{15}{{\left(x - 14\right)}^{2}}\right)}\neq0$.}
\choice{The cosine factor decreases to $0$ faster than the polynomial.}
\choice[correct]{The cosine factor is bounded between $-1$ and $1$, so the polynomial forces the function to $0$.}
\choice{The cosine factor directly cancels out the polynomial factor.}
\end{multipleChoice}


What is the name of the theorem that applies to this problem? \qquad \\
The \underline{\answer{Squeeze}} Theorem
\end{problem}}%}

%%%%%%%%%%%%%%%%%%%%%




\latexProblemContent{
\begin{problem}

The limit as $x\to{3}$ of $f(x)={{\left(x - 3\right)}^{2} \cos\left(-\frac{9}{{\left(x - 3\right)}^{2}}\right)}$ is $0$.  What is the reason why this is true?

\input{2311-Concept-Limit-0008.HELP.tex}

\begin{multipleChoice}
\choice{The statement is in fact false: $\lim\limits_{x\to{3}}{{\left(x - 3\right)}^{2} \cos\left(-\frac{9}{{\left(x - 3\right)}^{2}}\right)}\neq0$.}
\choice{The cosine factor decreases to $0$ faster than the polynomial.}
\choice[correct]{The cosine factor is bounded between $-1$ and $1$, so the polynomial forces the function to $0$.}
\choice{The cosine factor directly cancels out the polynomial factor.}
\end{multipleChoice}


What is the name of the theorem that applies to this problem? \qquad \\
The \underline{\answer{Squeeze}} Theorem
\end{problem}}%}

%%%%%%%%%%%%%%%%%%%%%




\latexProblemContent{
\begin{problem}

The limit as $x\to{10}$ of $f(x)={{\left(x - 10\right)}^{2} \cos\left(\frac{19}{{\left(x - 10\right)}^{2}}\right)}$ is $0$.  What is the reason why this is true?

\input{2311-Concept-Limit-0008.HELP.tex}

\begin{multipleChoice}
\choice{The statement is in fact false: $\lim\limits_{x\to{10}}{{\left(x - 10\right)}^{2} \cos\left(\frac{19}{{\left(x - 10\right)}^{2}}\right)}\neq0$.}
\choice{The cosine factor decreases to $0$ faster than the polynomial.}
\choice[correct]{The cosine factor is bounded between $-1$ and $1$, so the polynomial forces the function to $0$.}
\choice{The cosine factor directly cancels out the polynomial factor.}
\end{multipleChoice}


What is the name of the theorem that applies to this problem? \qquad \\
The \underline{\answer{Squeeze}} Theorem
\end{problem}}%}

%%%%%%%%%%%%%%%%%%%%%




\latexProblemContent{
\begin{problem}

The limit as $x\to{4}$ of $f(x)={{\left(x - 4\right)}^{3} \cos\left(\frac{20}{{\left(x - 4\right)}^{2}}\right)}$ is $0$.  What is the reason why this is true?

\input{2311-Concept-Limit-0008.HELP.tex}

\begin{multipleChoice}
\choice{The statement is in fact false: $\lim\limits_{x\to{4}}{{\left(x - 4\right)}^{3} \cos\left(\frac{20}{{\left(x - 4\right)}^{2}}\right)}\neq0$.}
\choice{The cosine factor decreases to $0$ faster than the polynomial.}
\choice[correct]{The cosine factor is bounded between $-1$ and $1$, so the polynomial forces the function to $0$.}
\choice{The cosine factor directly cancels out the polynomial factor.}
\end{multipleChoice}


What is the name of the theorem that applies to this problem? \qquad \\
The \underline{\answer{Squeeze}} Theorem
\end{problem}}%}

%%%%%%%%%%%%%%%%%%%%%




\latexProblemContent{
\begin{problem}

The limit as $x\to{12}$ of $f(x)={{\left(x - 12\right)} \cos\left(\frac{13}{{\left(x - 12\right)}^{2}}\right)}$ is $0$.  What is the reason why this is true?

\input{2311-Concept-Limit-0008.HELP.tex}

\begin{multipleChoice}
\choice{The statement is in fact false: $\lim\limits_{x\to{12}}{{\left(x - 12\right)} \cos\left(\frac{13}{{\left(x - 12\right)}^{2}}\right)}\neq0$.}
\choice{The cosine factor decreases to $0$ faster than the polynomial.}
\choice[correct]{The cosine factor is bounded between $-1$ and $1$, so the polynomial forces the function to $0$.}
\choice{The cosine factor directly cancels out the polynomial factor.}
\end{multipleChoice}


What is the name of the theorem that applies to this problem? \qquad \\
The \underline{\answer{Squeeze}} Theorem
\end{problem}}%}

%%%%%%%%%%%%%%%%%%%%%




\latexProblemContent{
\begin{problem}

The limit as $x\to{-12}$ of $f(x)={{\left(x + 12\right)}^{3} \cos\left(\frac{23}{x + 12}\right)}$ is $0$.  What is the reason why this is true?

\input{2311-Concept-Limit-0008.HELP.tex}

\begin{multipleChoice}
\choice{The statement is in fact false: $\lim\limits_{x\to{-12}}{{\left(x + 12\right)}^{3} \cos\left(\frac{23}{x + 12}\right)}\neq0$.}
\choice{The cosine factor decreases to $0$ faster than the polynomial.}
\choice[correct]{The cosine factor is bounded between $-1$ and $1$, so the polynomial forces the function to $0$.}
\choice{The cosine factor directly cancels out the polynomial factor.}
\end{multipleChoice}


What is the name of the theorem that applies to this problem? \qquad \\
The \underline{\answer{Squeeze}} Theorem
\end{problem}}%}

%%%%%%%%%%%%%%%%%%%%%




\latexProblemContent{
\begin{problem}

The limit as $x\to{9}$ of $f(x)={{\left(x - 9\right)} \cos\left(-\frac{23}{{\left(x - 9\right)}^{2}}\right)}$ is $0$.  What is the reason why this is true?

\input{2311-Concept-Limit-0008.HELP.tex}

\begin{multipleChoice}
\choice{The statement is in fact false: $\lim\limits_{x\to{9}}{{\left(x - 9\right)} \cos\left(-\frac{23}{{\left(x - 9\right)}^{2}}\right)}\neq0$.}
\choice{The cosine factor decreases to $0$ faster than the polynomial.}
\choice[correct]{The cosine factor is bounded between $-1$ and $1$, so the polynomial forces the function to $0$.}
\choice{The cosine factor directly cancels out the polynomial factor.}
\end{multipleChoice}


What is the name of the theorem that applies to this problem? \qquad \\
The \underline{\answer{Squeeze}} Theorem
\end{problem}}%}

%%%%%%%%%%%%%%%%%%%%%




\latexProblemContent{
\begin{problem}

The limit as $x\to{14}$ of $f(x)={{\left(x - 14\right)}^{3} \cos\left(-\frac{8}{x - 14}\right)}$ is $0$.  What is the reason why this is true?

\input{2311-Concept-Limit-0008.HELP.tex}

\begin{multipleChoice}
\choice{The statement is in fact false: $\lim\limits_{x\to{14}}{{\left(x - 14\right)}^{3} \cos\left(-\frac{8}{x - 14}\right)}\neq0$.}
\choice{The cosine factor decreases to $0$ faster than the polynomial.}
\choice[correct]{The cosine factor is bounded between $-1$ and $1$, so the polynomial forces the function to $0$.}
\choice{The cosine factor directly cancels out the polynomial factor.}
\end{multipleChoice}


What is the name of the theorem that applies to this problem? \qquad \\
The \underline{\answer{Squeeze}} Theorem
\end{problem}}%}

%%%%%%%%%%%%%%%%%%%%%




\latexProblemContent{
\begin{problem}

The limit as $x\to{-12}$ of $f(x)={{\left(x + 12\right)} \cos\left(\frac{10}{x + 12}\right)}$ is $0$.  What is the reason why this is true?

\input{2311-Concept-Limit-0008.HELP.tex}

\begin{multipleChoice}
\choice{The statement is in fact false: $\lim\limits_{x\to{-12}}{{\left(x + 12\right)} \cos\left(\frac{10}{x + 12}\right)}\neq0$.}
\choice{The cosine factor decreases to $0$ faster than the polynomial.}
\choice[correct]{The cosine factor is bounded between $-1$ and $1$, so the polynomial forces the function to $0$.}
\choice{The cosine factor directly cancels out the polynomial factor.}
\end{multipleChoice}


What is the name of the theorem that applies to this problem? \qquad \\
The \underline{\answer{Squeeze}} Theorem
\end{problem}}%}

%%%%%%%%%%%%%%%%%%%%%




\latexProblemContent{
\begin{problem}

The limit as $x\to{-11}$ of $f(x)={{\left(x + 11\right)}^{3} \cos\left(-\frac{7}{x + 11}\right)}$ is $0$.  What is the reason why this is true?

\input{2311-Concept-Limit-0008.HELP.tex}

\begin{multipleChoice}
\choice{The statement is in fact false: $\lim\limits_{x\to{-11}}{{\left(x + 11\right)}^{3} \cos\left(-\frac{7}{x + 11}\right)}\neq0$.}
\choice{The cosine factor decreases to $0$ faster than the polynomial.}
\choice[correct]{The cosine factor is bounded between $-1$ and $1$, so the polynomial forces the function to $0$.}
\choice{The cosine factor directly cancels out the polynomial factor.}
\end{multipleChoice}


What is the name of the theorem that applies to this problem? \qquad \\
The \underline{\answer{Squeeze}} Theorem
\end{problem}}%}

%%%%%%%%%%%%%%%%%%%%%




\latexProblemContent{
\begin{problem}

The limit as $x\to{-14}$ of $f(x)={{\left(x + 14\right)}^{3} \cos\left(\frac{7}{{\left(x + 14\right)}^{2}}\right)}$ is $0$.  What is the reason why this is true?

\input{2311-Concept-Limit-0008.HELP.tex}

\begin{multipleChoice}
\choice{The statement is in fact false: $\lim\limits_{x\to{-14}}{{\left(x + 14\right)}^{3} \cos\left(\frac{7}{{\left(x + 14\right)}^{2}}\right)}\neq0$.}
\choice{The cosine factor decreases to $0$ faster than the polynomial.}
\choice[correct]{The cosine factor is bounded between $-1$ and $1$, so the polynomial forces the function to $0$.}
\choice{The cosine factor directly cancels out the polynomial factor.}
\end{multipleChoice}


What is the name of the theorem that applies to this problem? \qquad \\
The \underline{\answer{Squeeze}} Theorem
\end{problem}}%}

%%%%%%%%%%%%%%%%%%%%%




\latexProblemContent{
\begin{problem}

The limit as $x\to{-2}$ of $f(x)={{\left(x + 2\right)}^{3} \cos\left(\frac{2}{x + 2}\right)}$ is $0$.  What is the reason why this is true?

\input{2311-Concept-Limit-0008.HELP.tex}

\begin{multipleChoice}
\choice{The statement is in fact false: $\lim\limits_{x\to{-2}}{{\left(x + 2\right)}^{3} \cos\left(\frac{2}{x + 2}\right)}\neq0$.}
\choice{The cosine factor decreases to $0$ faster than the polynomial.}
\choice[correct]{The cosine factor is bounded between $-1$ and $1$, so the polynomial forces the function to $0$.}
\choice{The cosine factor directly cancels out the polynomial factor.}
\end{multipleChoice}


What is the name of the theorem that applies to this problem? \qquad \\
The \underline{\answer{Squeeze}} Theorem
\end{problem}}%}

%%%%%%%%%%%%%%%%%%%%%




\latexProblemContent{
\begin{problem}

The limit as $x\to{14}$ of $f(x)={{\left(x - 14\right)}^{3} \cos\left(-\frac{20}{{\left(x - 14\right)}^{2}}\right)}$ is $0$.  What is the reason why this is true?

\input{2311-Concept-Limit-0008.HELP.tex}

\begin{multipleChoice}
\choice{The statement is in fact false: $\lim\limits_{x\to{14}}{{\left(x - 14\right)}^{3} \cos\left(-\frac{20}{{\left(x - 14\right)}^{2}}\right)}\neq0$.}
\choice{The cosine factor decreases to $0$ faster than the polynomial.}
\choice[correct]{The cosine factor is bounded between $-1$ and $1$, so the polynomial forces the function to $0$.}
\choice{The cosine factor directly cancels out the polynomial factor.}
\end{multipleChoice}


What is the name of the theorem that applies to this problem? \qquad \\
The \underline{\answer{Squeeze}} Theorem
\end{problem}}%}

%%%%%%%%%%%%%%%%%%%%%




\latexProblemContent{
\begin{problem}

The limit as $x\to{-14}$ of $f(x)={{\left(x + 14\right)}^{2} \cos\left(\frac{10}{x + 14}\right)}$ is $0$.  What is the reason why this is true?

\input{2311-Concept-Limit-0008.HELP.tex}

\begin{multipleChoice}
\choice{The statement is in fact false: $\lim\limits_{x\to{-14}}{{\left(x + 14\right)}^{2} \cos\left(\frac{10}{x + 14}\right)}\neq0$.}
\choice{The cosine factor decreases to $0$ faster than the polynomial.}
\choice[correct]{The cosine factor is bounded between $-1$ and $1$, so the polynomial forces the function to $0$.}
\choice{The cosine factor directly cancels out the polynomial factor.}
\end{multipleChoice}


What is the name of the theorem that applies to this problem? \qquad \\
The \underline{\answer{Squeeze}} Theorem
\end{problem}}%}

%%%%%%%%%%%%%%%%%%%%%




\latexProblemContent{
\begin{problem}

The limit as $x\to{7}$ of $f(x)={{\left(x - 7\right)}^{3} \cos\left(-\frac{11}{x - 7}\right)}$ is $0$.  What is the reason why this is true?

\input{2311-Concept-Limit-0008.HELP.tex}

\begin{multipleChoice}
\choice{The statement is in fact false: $\lim\limits_{x\to{7}}{{\left(x - 7\right)}^{3} \cos\left(-\frac{11}{x - 7}\right)}\neq0$.}
\choice{The cosine factor decreases to $0$ faster than the polynomial.}
\choice[correct]{The cosine factor is bounded between $-1$ and $1$, so the polynomial forces the function to $0$.}
\choice{The cosine factor directly cancels out the polynomial factor.}
\end{multipleChoice}


What is the name of the theorem that applies to this problem? \qquad \\
The \underline{\answer{Squeeze}} Theorem
\end{problem}}%}

%%%%%%%%%%%%%%%%%%%%%




\latexProblemContent{
\begin{problem}

The limit as $x\to{-14}$ of $f(x)={{\left(x + 14\right)}^{3} \cos\left(\frac{24}{{\left(x + 14\right)}^{2}}\right)}$ is $0$.  What is the reason why this is true?

\input{2311-Concept-Limit-0008.HELP.tex}

\begin{multipleChoice}
\choice{The statement is in fact false: $\lim\limits_{x\to{-14}}{{\left(x + 14\right)}^{3} \cos\left(\frac{24}{{\left(x + 14\right)}^{2}}\right)}\neq0$.}
\choice{The cosine factor decreases to $0$ faster than the polynomial.}
\choice[correct]{The cosine factor is bounded between $-1$ and $1$, so the polynomial forces the function to $0$.}
\choice{The cosine factor directly cancels out the polynomial factor.}
\end{multipleChoice}


What is the name of the theorem that applies to this problem? \qquad \\
The \underline{\answer{Squeeze}} Theorem
\end{problem}}%}

%%%%%%%%%%%%%%%%%%%%%




\latexProblemContent{
\begin{problem}

The limit as $x\to{-5}$ of $f(x)={{\left(x + 5\right)}^{2} \cos\left(\frac{10}{x + 5}\right)}$ is $0$.  What is the reason why this is true?

\input{2311-Concept-Limit-0008.HELP.tex}

\begin{multipleChoice}
\choice{The statement is in fact false: $\lim\limits_{x\to{-5}}{{\left(x + 5\right)}^{2} \cos\left(\frac{10}{x + 5}\right)}\neq0$.}
\choice{The cosine factor decreases to $0$ faster than the polynomial.}
\choice[correct]{The cosine factor is bounded between $-1$ and $1$, so the polynomial forces the function to $0$.}
\choice{The cosine factor directly cancels out the polynomial factor.}
\end{multipleChoice}


What is the name of the theorem that applies to this problem? \qquad \\
The \underline{\answer{Squeeze}} Theorem
\end{problem}}%}

%%%%%%%%%%%%%%%%%%%%%




\latexProblemContent{
\begin{problem}

The limit as $x\to{0}$ of $f(x)={x^{3} \cos\left(\frac{22}{x}\right)}$ is $0$.  What is the reason why this is true?

\input{2311-Concept-Limit-0008.HELP.tex}

\begin{multipleChoice}
\choice{The statement is in fact false: $\lim\limits_{x\to{0}}{x^{3} \cos\left(\frac{22}{x}\right)}\neq0$.}
\choice{The cosine factor decreases to $0$ faster than the polynomial.}
\choice[correct]{The cosine factor is bounded between $-1$ and $1$, so the polynomial forces the function to $0$.}
\choice{The cosine factor directly cancels out the polynomial factor.}
\end{multipleChoice}


What is the name of the theorem that applies to this problem? \qquad \\
The \underline{\answer{Squeeze}} Theorem
\end{problem}}%}

%%%%%%%%%%%%%%%%%%%%%




\latexProblemContent{
\begin{problem}

The limit as $x\to{1}$ of $f(x)={{\left(x - 1\right)}^{2} \cos\left(\frac{22}{{\left(x - 1\right)}^{2}}\right)}$ is $0$.  What is the reason why this is true?

\input{2311-Concept-Limit-0008.HELP.tex}

\begin{multipleChoice}
\choice{The statement is in fact false: $\lim\limits_{x\to{1}}{{\left(x - 1\right)}^{2} \cos\left(\frac{22}{{\left(x - 1\right)}^{2}}\right)}\neq0$.}
\choice{The cosine factor decreases to $0$ faster than the polynomial.}
\choice[correct]{The cosine factor is bounded between $-1$ and $1$, so the polynomial forces the function to $0$.}
\choice{The cosine factor directly cancels out the polynomial factor.}
\end{multipleChoice}


What is the name of the theorem that applies to this problem? \qquad \\
The \underline{\answer{Squeeze}} Theorem
\end{problem}}%}

%%%%%%%%%%%%%%%%%%%%%




\latexProblemContent{
\begin{problem}

The limit as $x\to{4}$ of $f(x)={{\left(x - 4\right)}^{3} \cos\left(\frac{23}{{\left(x - 4\right)}^{2}}\right)}$ is $0$.  What is the reason why this is true?

\input{2311-Concept-Limit-0008.HELP.tex}

\begin{multipleChoice}
\choice{The statement is in fact false: $\lim\limits_{x\to{4}}{{\left(x - 4\right)}^{3} \cos\left(\frac{23}{{\left(x - 4\right)}^{2}}\right)}\neq0$.}
\choice{The cosine factor decreases to $0$ faster than the polynomial.}
\choice[correct]{The cosine factor is bounded between $-1$ and $1$, so the polynomial forces the function to $0$.}
\choice{The cosine factor directly cancels out the polynomial factor.}
\end{multipleChoice}


What is the name of the theorem that applies to this problem? \qquad \\
The \underline{\answer{Squeeze}} Theorem
\end{problem}}%}

%%%%%%%%%%%%%%%%%%%%%




\latexProblemContent{
\begin{problem}

The limit as $x\to{-9}$ of $f(x)={{\left(x + 9\right)}^{2} \cos\left(-\frac{18}{{\left(x + 9\right)}^{2}}\right)}$ is $0$.  What is the reason why this is true?

\input{2311-Concept-Limit-0008.HELP.tex}

\begin{multipleChoice}
\choice{The statement is in fact false: $\lim\limits_{x\to{-9}}{{\left(x + 9\right)}^{2} \cos\left(-\frac{18}{{\left(x + 9\right)}^{2}}\right)}\neq0$.}
\choice{The cosine factor decreases to $0$ faster than the polynomial.}
\choice[correct]{The cosine factor is bounded between $-1$ and $1$, so the polynomial forces the function to $0$.}
\choice{The cosine factor directly cancels out the polynomial factor.}
\end{multipleChoice}


What is the name of the theorem that applies to this problem? \qquad \\
The \underline{\answer{Squeeze}} Theorem
\end{problem}}%}

%%%%%%%%%%%%%%%%%%%%%




\latexProblemContent{
\begin{problem}

The limit as $x\to{-10}$ of $f(x)={{\left(x + 10\right)}^{3} \cos\left(\frac{1}{{\left(x + 10\right)}^{2}}\right)}$ is $0$.  What is the reason why this is true?

\input{2311-Concept-Limit-0008.HELP.tex}

\begin{multipleChoice}
\choice{The statement is in fact false: $\lim\limits_{x\to{-10}}{{\left(x + 10\right)}^{3} \cos\left(\frac{1}{{\left(x + 10\right)}^{2}}\right)}\neq0$.}
\choice{The cosine factor decreases to $0$ faster than the polynomial.}
\choice[correct]{The cosine factor is bounded between $-1$ and $1$, so the polynomial forces the function to $0$.}
\choice{The cosine factor directly cancels out the polynomial factor.}
\end{multipleChoice}


What is the name of the theorem that applies to this problem? \qquad \\
The \underline{\answer{Squeeze}} Theorem
\end{problem}}%}

%%%%%%%%%%%%%%%%%%%%%




\latexProblemContent{
\begin{problem}

The limit as $x\to{15}$ of $f(x)={{\left(x - 15\right)}^{3} \cos\left(\frac{5}{x - 15}\right)}$ is $0$.  What is the reason why this is true?

\input{2311-Concept-Limit-0008.HELP.tex}

\begin{multipleChoice}
\choice{The statement is in fact false: $\lim\limits_{x\to{15}}{{\left(x - 15\right)}^{3} \cos\left(\frac{5}{x - 15}\right)}\neq0$.}
\choice{The cosine factor decreases to $0$ faster than the polynomial.}
\choice[correct]{The cosine factor is bounded between $-1$ and $1$, so the polynomial forces the function to $0$.}
\choice{The cosine factor directly cancels out the polynomial factor.}
\end{multipleChoice}


What is the name of the theorem that applies to this problem? \qquad \\
The \underline{\answer{Squeeze}} Theorem
\end{problem}}%}

%%%%%%%%%%%%%%%%%%%%%




\latexProblemContent{
\begin{problem}

The limit as $x\to{12}$ of $f(x)={{\left(x - 12\right)}^{3} \cos\left(-\frac{6}{{\left(x - 12\right)}^{2}}\right)}$ is $0$.  What is the reason why this is true?

\input{2311-Concept-Limit-0008.HELP.tex}

\begin{multipleChoice}
\choice{The statement is in fact false: $\lim\limits_{x\to{12}}{{\left(x - 12\right)}^{3} \cos\left(-\frac{6}{{\left(x - 12\right)}^{2}}\right)}\neq0$.}
\choice{The cosine factor decreases to $0$ faster than the polynomial.}
\choice[correct]{The cosine factor is bounded between $-1$ and $1$, so the polynomial forces the function to $0$.}
\choice{The cosine factor directly cancels out the polynomial factor.}
\end{multipleChoice}


What is the name of the theorem that applies to this problem? \qquad \\
The \underline{\answer{Squeeze}} Theorem
\end{problem}}%}

%%%%%%%%%%%%%%%%%%%%%




\latexProblemContent{
\begin{problem}

The limit as $x\to{7}$ of $f(x)={{\left(x - 7\right)} \cos\left(\frac{13}{x - 7}\right)}$ is $0$.  What is the reason why this is true?

\input{2311-Concept-Limit-0008.HELP.tex}

\begin{multipleChoice}
\choice{The statement is in fact false: $\lim\limits_{x\to{7}}{{\left(x - 7\right)} \cos\left(\frac{13}{x - 7}\right)}\neq0$.}
\choice{The cosine factor decreases to $0$ faster than the polynomial.}
\choice[correct]{The cosine factor is bounded between $-1$ and $1$, so the polynomial forces the function to $0$.}
\choice{The cosine factor directly cancels out the polynomial factor.}
\end{multipleChoice}


What is the name of the theorem that applies to this problem? \qquad \\
The \underline{\answer{Squeeze}} Theorem
\end{problem}}%}

%%%%%%%%%%%%%%%%%%%%%




\latexProblemContent{
\begin{problem}

The limit as $x\to{-8}$ of $f(x)={{\left(x + 8\right)}^{2} \cos\left(\frac{10}{{\left(x + 8\right)}^{2}}\right)}$ is $0$.  What is the reason why this is true?

\input{2311-Concept-Limit-0008.HELP.tex}

\begin{multipleChoice}
\choice{The statement is in fact false: $\lim\limits_{x\to{-8}}{{\left(x + 8\right)}^{2} \cos\left(\frac{10}{{\left(x + 8\right)}^{2}}\right)}\neq0$.}
\choice{The cosine factor decreases to $0$ faster than the polynomial.}
\choice[correct]{The cosine factor is bounded between $-1$ and $1$, so the polynomial forces the function to $0$.}
\choice{The cosine factor directly cancels out the polynomial factor.}
\end{multipleChoice}


What is the name of the theorem that applies to this problem? \qquad \\
The \underline{\answer{Squeeze}} Theorem
\end{problem}}%}

%%%%%%%%%%%%%%%%%%%%%




\latexProblemContent{
\begin{problem}

The limit as $x\to{15}$ of $f(x)={{\left(x - 15\right)} \cos\left(-\frac{9}{x - 15}\right)}$ is $0$.  What is the reason why this is true?

\input{2311-Concept-Limit-0008.HELP.tex}

\begin{multipleChoice}
\choice{The statement is in fact false: $\lim\limits_{x\to{15}}{{\left(x - 15\right)} \cos\left(-\frac{9}{x - 15}\right)}\neq0$.}
\choice{The cosine factor decreases to $0$ faster than the polynomial.}
\choice[correct]{The cosine factor is bounded between $-1$ and $1$, so the polynomial forces the function to $0$.}
\choice{The cosine factor directly cancels out the polynomial factor.}
\end{multipleChoice}


What is the name of the theorem that applies to this problem? \qquad \\
The \underline{\answer{Squeeze}} Theorem
\end{problem}}%}

%%%%%%%%%%%%%%%%%%%%%




\latexProblemContent{
\begin{problem}

The limit as $x\to{-1}$ of $f(x)={{\left(x + 1\right)}^{3} \cos\left(\frac{9}{x + 1}\right)}$ is $0$.  What is the reason why this is true?

\input{2311-Concept-Limit-0008.HELP.tex}

\begin{multipleChoice}
\choice{The statement is in fact false: $\lim\limits_{x\to{-1}}{{\left(x + 1\right)}^{3} \cos\left(\frac{9}{x + 1}\right)}\neq0$.}
\choice{The cosine factor decreases to $0$ faster than the polynomial.}
\choice[correct]{The cosine factor is bounded between $-1$ and $1$, so the polynomial forces the function to $0$.}
\choice{The cosine factor directly cancels out the polynomial factor.}
\end{multipleChoice}


What is the name of the theorem that applies to this problem? \qquad \\
The \underline{\answer{Squeeze}} Theorem
\end{problem}}%}

%%%%%%%%%%%%%%%%%%%%%




\latexProblemContent{
\begin{problem}

The limit as $x\to{7}$ of $f(x)={{\left(x - 7\right)} \cos\left(\frac{14}{x - 7}\right)}$ is $0$.  What is the reason why this is true?

\input{2311-Concept-Limit-0008.HELP.tex}

\begin{multipleChoice}
\choice{The statement is in fact false: $\lim\limits_{x\to{7}}{{\left(x - 7\right)} \cos\left(\frac{14}{x - 7}\right)}\neq0$.}
\choice{The cosine factor decreases to $0$ faster than the polynomial.}
\choice[correct]{The cosine factor is bounded between $-1$ and $1$, so the polynomial forces the function to $0$.}
\choice{The cosine factor directly cancels out the polynomial factor.}
\end{multipleChoice}


What is the name of the theorem that applies to this problem? \qquad \\
The \underline{\answer{Squeeze}} Theorem
\end{problem}}%}

%%%%%%%%%%%%%%%%%%%%%




\latexProblemContent{
\begin{problem}

The limit as $x\to{2}$ of $f(x)={{\left(x - 2\right)}^{2} \cos\left(\frac{12}{x - 2}\right)}$ is $0$.  What is the reason why this is true?

\input{2311-Concept-Limit-0008.HELP.tex}

\begin{multipleChoice}
\choice{The statement is in fact false: $\lim\limits_{x\to{2}}{{\left(x - 2\right)}^{2} \cos\left(\frac{12}{x - 2}\right)}\neq0$.}
\choice{The cosine factor decreases to $0$ faster than the polynomial.}
\choice[correct]{The cosine factor is bounded between $-1$ and $1$, so the polynomial forces the function to $0$.}
\choice{The cosine factor directly cancels out the polynomial factor.}
\end{multipleChoice}


What is the name of the theorem that applies to this problem? \qquad \\
The \underline{\answer{Squeeze}} Theorem
\end{problem}}%}

%%%%%%%%%%%%%%%%%%%%%




\latexProblemContent{
\begin{problem}

The limit as $x\to{-11}$ of $f(x)={{\left(x + 11\right)}^{3} \cos\left(\frac{8}{{\left(x + 11\right)}^{2}}\right)}$ is $0$.  What is the reason why this is true?

\input{2311-Concept-Limit-0008.HELP.tex}

\begin{multipleChoice}
\choice{The statement is in fact false: $\lim\limits_{x\to{-11}}{{\left(x + 11\right)}^{3} \cos\left(\frac{8}{{\left(x + 11\right)}^{2}}\right)}\neq0$.}
\choice{The cosine factor decreases to $0$ faster than the polynomial.}
\choice[correct]{The cosine factor is bounded between $-1$ and $1$, so the polynomial forces the function to $0$.}
\choice{The cosine factor directly cancels out the polynomial factor.}
\end{multipleChoice}


What is the name of the theorem that applies to this problem? \qquad \\
The \underline{\answer{Squeeze}} Theorem
\end{problem}}%}

%%%%%%%%%%%%%%%%%%%%%




\latexProblemContent{
\begin{problem}

The limit as $x\to{8}$ of $f(x)={{\left(x - 8\right)}^{2} \cos\left(-\frac{25}{x - 8}\right)}$ is $0$.  What is the reason why this is true?

\input{2311-Concept-Limit-0008.HELP.tex}

\begin{multipleChoice}
\choice{The statement is in fact false: $\lim\limits_{x\to{8}}{{\left(x - 8\right)}^{2} \cos\left(-\frac{25}{x - 8}\right)}\neq0$.}
\choice{The cosine factor decreases to $0$ faster than the polynomial.}
\choice[correct]{The cosine factor is bounded between $-1$ and $1$, so the polynomial forces the function to $0$.}
\choice{The cosine factor directly cancels out the polynomial factor.}
\end{multipleChoice}


What is the name of the theorem that applies to this problem? \qquad \\
The \underline{\answer{Squeeze}} Theorem
\end{problem}}%}

%%%%%%%%%%%%%%%%%%%%%




\latexProblemContent{
\begin{problem}

The limit as $x\to{-11}$ of $f(x)={{\left(x + 11\right)}^{2} \cos\left(\frac{13}{{\left(x + 11\right)}^{2}}\right)}$ is $0$.  What is the reason why this is true?

\input{2311-Concept-Limit-0008.HELP.tex}

\begin{multipleChoice}
\choice{The statement is in fact false: $\lim\limits_{x\to{-11}}{{\left(x + 11\right)}^{2} \cos\left(\frac{13}{{\left(x + 11\right)}^{2}}\right)}\neq0$.}
\choice{The cosine factor decreases to $0$ faster than the polynomial.}
\choice[correct]{The cosine factor is bounded between $-1$ and $1$, so the polynomial forces the function to $0$.}
\choice{The cosine factor directly cancels out the polynomial factor.}
\end{multipleChoice}


What is the name of the theorem that applies to this problem? \qquad \\
The \underline{\answer{Squeeze}} Theorem
\end{problem}}%}

%%%%%%%%%%%%%%%%%%%%%




\latexProblemContent{
\begin{problem}

The limit as $x\to{-2}$ of $f(x)={{\left(x + 2\right)} \cos\left(-\frac{24}{{\left(x + 2\right)}^{2}}\right)}$ is $0$.  What is the reason why this is true?

\input{2311-Concept-Limit-0008.HELP.tex}

\begin{multipleChoice}
\choice{The statement is in fact false: $\lim\limits_{x\to{-2}}{{\left(x + 2\right)} \cos\left(-\frac{24}{{\left(x + 2\right)}^{2}}\right)}\neq0$.}
\choice{The cosine factor decreases to $0$ faster than the polynomial.}
\choice[correct]{The cosine factor is bounded between $-1$ and $1$, so the polynomial forces the function to $0$.}
\choice{The cosine factor directly cancels out the polynomial factor.}
\end{multipleChoice}


What is the name of the theorem that applies to this problem? \qquad \\
The \underline{\answer{Squeeze}} Theorem
\end{problem}}%}

%%%%%%%%%%%%%%%%%%%%%




\latexProblemContent{
\begin{problem}

The limit as $x\to{11}$ of $f(x)={{\left(x - 11\right)}^{2} \cos\left(\frac{22}{x - 11}\right)}$ is $0$.  What is the reason why this is true?

\input{2311-Concept-Limit-0008.HELP.tex}

\begin{multipleChoice}
\choice{The statement is in fact false: $\lim\limits_{x\to{11}}{{\left(x - 11\right)}^{2} \cos\left(\frac{22}{x - 11}\right)}\neq0$.}
\choice{The cosine factor decreases to $0$ faster than the polynomial.}
\choice[correct]{The cosine factor is bounded between $-1$ and $1$, so the polynomial forces the function to $0$.}
\choice{The cosine factor directly cancels out the polynomial factor.}
\end{multipleChoice}


What is the name of the theorem that applies to this problem? \qquad \\
The \underline{\answer{Squeeze}} Theorem
\end{problem}}%}

%%%%%%%%%%%%%%%%%%%%%




\latexProblemContent{
\begin{problem}

The limit as $x\to{1}$ of $f(x)={{\left(x - 1\right)} \cos\left(-\frac{19}{x - 1}\right)}$ is $0$.  What is the reason why this is true?

\input{2311-Concept-Limit-0008.HELP.tex}

\begin{multipleChoice}
\choice{The statement is in fact false: $\lim\limits_{x\to{1}}{{\left(x - 1\right)} \cos\left(-\frac{19}{x - 1}\right)}\neq0$.}
\choice{The cosine factor decreases to $0$ faster than the polynomial.}
\choice[correct]{The cosine factor is bounded between $-1$ and $1$, so the polynomial forces the function to $0$.}
\choice{The cosine factor directly cancels out the polynomial factor.}
\end{multipleChoice}


What is the name of the theorem that applies to this problem? \qquad \\
The \underline{\answer{Squeeze}} Theorem
\end{problem}}%}

%%%%%%%%%%%%%%%%%%%%%




\latexProblemContent{
\begin{problem}

The limit as $x\to{-15}$ of $f(x)={{\left(x + 15\right)}^{2} \cos\left(\frac{13}{{\left(x + 15\right)}^{2}}\right)}$ is $0$.  What is the reason why this is true?

\input{2311-Concept-Limit-0008.HELP.tex}

\begin{multipleChoice}
\choice{The statement is in fact false: $\lim\limits_{x\to{-15}}{{\left(x + 15\right)}^{2} \cos\left(\frac{13}{{\left(x + 15\right)}^{2}}\right)}\neq0$.}
\choice{The cosine factor decreases to $0$ faster than the polynomial.}
\choice[correct]{The cosine factor is bounded between $-1$ and $1$, so the polynomial forces the function to $0$.}
\choice{The cosine factor directly cancels out the polynomial factor.}
\end{multipleChoice}


What is the name of the theorem that applies to this problem? \qquad \\
The \underline{\answer{Squeeze}} Theorem
\end{problem}}%}

%%%%%%%%%%%%%%%%%%%%%




\latexProblemContent{
\begin{problem}

The limit as $x\to{6}$ of $f(x)={{\left(x - 6\right)}^{3} \cos\left(\frac{9}{{\left(x - 6\right)}^{2}}\right)}$ is $0$.  What is the reason why this is true?

\input{2311-Concept-Limit-0008.HELP.tex}

\begin{multipleChoice}
\choice{The statement is in fact false: $\lim\limits_{x\to{6}}{{\left(x - 6\right)}^{3} \cos\left(\frac{9}{{\left(x - 6\right)}^{2}}\right)}\neq0$.}
\choice{The cosine factor decreases to $0$ faster than the polynomial.}
\choice[correct]{The cosine factor is bounded between $-1$ and $1$, so the polynomial forces the function to $0$.}
\choice{The cosine factor directly cancels out the polynomial factor.}
\end{multipleChoice}


What is the name of the theorem that applies to this problem? \qquad \\
The \underline{\answer{Squeeze}} Theorem
\end{problem}}%}

%%%%%%%%%%%%%%%%%%%%%




\latexProblemContent{
\begin{problem}

The limit as $x\to{13}$ of $f(x)={{\left(x - 13\right)} \cos\left(-\frac{4}{x - 13}\right)}$ is $0$.  What is the reason why this is true?

\input{2311-Concept-Limit-0008.HELP.tex}

\begin{multipleChoice}
\choice{The statement is in fact false: $\lim\limits_{x\to{13}}{{\left(x - 13\right)} \cos\left(-\frac{4}{x - 13}\right)}\neq0$.}
\choice{The cosine factor decreases to $0$ faster than the polynomial.}
\choice[correct]{The cosine factor is bounded between $-1$ and $1$, so the polynomial forces the function to $0$.}
\choice{The cosine factor directly cancels out the polynomial factor.}
\end{multipleChoice}


What is the name of the theorem that applies to this problem? \qquad \\
The \underline{\answer{Squeeze}} Theorem
\end{problem}}%}

%%%%%%%%%%%%%%%%%%%%%




\latexProblemContent{
\begin{problem}

The limit as $x\to{-4}$ of $f(x)={{\left(x + 4\right)}^{3} \cos\left(\frac{17}{{\left(x + 4\right)}^{2}}\right)}$ is $0$.  What is the reason why this is true?

\input{2311-Concept-Limit-0008.HELP.tex}

\begin{multipleChoice}
\choice{The statement is in fact false: $\lim\limits_{x\to{-4}}{{\left(x + 4\right)}^{3} \cos\left(\frac{17}{{\left(x + 4\right)}^{2}}\right)}\neq0$.}
\choice{The cosine factor decreases to $0$ faster than the polynomial.}
\choice[correct]{The cosine factor is bounded between $-1$ and $1$, so the polynomial forces the function to $0$.}
\choice{The cosine factor directly cancels out the polynomial factor.}
\end{multipleChoice}


What is the name of the theorem that applies to this problem? \qquad \\
The \underline{\answer{Squeeze}} Theorem
\end{problem}}%}

%%%%%%%%%%%%%%%%%%%%%




\latexProblemContent{
\begin{problem}

The limit as $x\to{-10}$ of $f(x)={{\left(x + 10\right)}^{3} \cos\left(\frac{15}{{\left(x + 10\right)}^{2}}\right)}$ is $0$.  What is the reason why this is true?

\input{2311-Concept-Limit-0008.HELP.tex}

\begin{multipleChoice}
\choice{The statement is in fact false: $\lim\limits_{x\to{-10}}{{\left(x + 10\right)}^{3} \cos\left(\frac{15}{{\left(x + 10\right)}^{2}}\right)}\neq0$.}
\choice{The cosine factor decreases to $0$ faster than the polynomial.}
\choice[correct]{The cosine factor is bounded between $-1$ and $1$, so the polynomial forces the function to $0$.}
\choice{The cosine factor directly cancels out the polynomial factor.}
\end{multipleChoice}


What is the name of the theorem that applies to this problem? \qquad \\
The \underline{\answer{Squeeze}} Theorem
\end{problem}}%}

%%%%%%%%%%%%%%%%%%%%%




\latexProblemContent{
\begin{problem}

The limit as $x\to{6}$ of $f(x)={{\left(x - 6\right)}^{3} \cos\left(-\frac{17}{{\left(x - 6\right)}^{2}}\right)}$ is $0$.  What is the reason why this is true?

\input{2311-Concept-Limit-0008.HELP.tex}

\begin{multipleChoice}
\choice{The statement is in fact false: $\lim\limits_{x\to{6}}{{\left(x - 6\right)}^{3} \cos\left(-\frac{17}{{\left(x - 6\right)}^{2}}\right)}\neq0$.}
\choice{The cosine factor decreases to $0$ faster than the polynomial.}
\choice[correct]{The cosine factor is bounded between $-1$ and $1$, so the polynomial forces the function to $0$.}
\choice{The cosine factor directly cancels out the polynomial factor.}
\end{multipleChoice}


What is the name of the theorem that applies to this problem? \qquad \\
The \underline{\answer{Squeeze}} Theorem
\end{problem}}%}

%%%%%%%%%%%%%%%%%%%%%




\latexProblemContent{
\begin{problem}

The limit as $x\to{-3}$ of $f(x)={{\left(x + 3\right)} \cos\left(\frac{1}{x + 3}\right)}$ is $0$.  What is the reason why this is true?

\input{2311-Concept-Limit-0008.HELP.tex}

\begin{multipleChoice}
\choice{The statement is in fact false: $\lim\limits_{x\to{-3}}{{\left(x + 3\right)} \cos\left(\frac{1}{x + 3}\right)}\neq0$.}
\choice{The cosine factor decreases to $0$ faster than the polynomial.}
\choice[correct]{The cosine factor is bounded between $-1$ and $1$, so the polynomial forces the function to $0$.}
\choice{The cosine factor directly cancels out the polynomial factor.}
\end{multipleChoice}


What is the name of the theorem that applies to this problem? \qquad \\
The \underline{\answer{Squeeze}} Theorem
\end{problem}}%}

%%%%%%%%%%%%%%%%%%%%%




\latexProblemContent{
\begin{problem}

The limit as $x\to{8}$ of $f(x)={{\left(x - 8\right)}^{3} \cos\left(\frac{11}{x - 8}\right)}$ is $0$.  What is the reason why this is true?

\input{2311-Concept-Limit-0008.HELP.tex}

\begin{multipleChoice}
\choice{The statement is in fact false: $\lim\limits_{x\to{8}}{{\left(x - 8\right)}^{3} \cos\left(\frac{11}{x - 8}\right)}\neq0$.}
\choice{The cosine factor decreases to $0$ faster than the polynomial.}
\choice[correct]{The cosine factor is bounded between $-1$ and $1$, so the polynomial forces the function to $0$.}
\choice{The cosine factor directly cancels out the polynomial factor.}
\end{multipleChoice}


What is the name of the theorem that applies to this problem? \qquad \\
The \underline{\answer{Squeeze}} Theorem
\end{problem}}%}

%%%%%%%%%%%%%%%%%%%%%




\latexProblemContent{
\begin{problem}

The limit as $x\to{-5}$ of $f(x)={{\left(x + 5\right)}^{2} \cos\left(-\frac{11}{x + 5}\right)}$ is $0$.  What is the reason why this is true?

\input{2311-Concept-Limit-0008.HELP.tex}

\begin{multipleChoice}
\choice{The statement is in fact false: $\lim\limits_{x\to{-5}}{{\left(x + 5\right)}^{2} \cos\left(-\frac{11}{x + 5}\right)}\neq0$.}
\choice{The cosine factor decreases to $0$ faster than the polynomial.}
\choice[correct]{The cosine factor is bounded between $-1$ and $1$, so the polynomial forces the function to $0$.}
\choice{The cosine factor directly cancels out the polynomial factor.}
\end{multipleChoice}


What is the name of the theorem that applies to this problem? \qquad \\
The \underline{\answer{Squeeze}} Theorem
\end{problem}}%}

%%%%%%%%%%%%%%%%%%%%%




\latexProblemContent{
\begin{problem}

The limit as $x\to{-7}$ of $f(x)={{\left(x + 7\right)}^{3} \cos\left(-\frac{22}{x + 7}\right)}$ is $0$.  What is the reason why this is true?

\input{2311-Concept-Limit-0008.HELP.tex}

\begin{multipleChoice}
\choice{The statement is in fact false: $\lim\limits_{x\to{-7}}{{\left(x + 7\right)}^{3} \cos\left(-\frac{22}{x + 7}\right)}\neq0$.}
\choice{The cosine factor decreases to $0$ faster than the polynomial.}
\choice[correct]{The cosine factor is bounded between $-1$ and $1$, so the polynomial forces the function to $0$.}
\choice{The cosine factor directly cancels out the polynomial factor.}
\end{multipleChoice}


What is the name of the theorem that applies to this problem? \qquad \\
The \underline{\answer{Squeeze}} Theorem
\end{problem}}%}

%%%%%%%%%%%%%%%%%%%%%




\latexProblemContent{
\begin{problem}

The limit as $x\to{7}$ of $f(x)={{\left(x - 7\right)}^{3} \cos\left(-\frac{19}{x - 7}\right)}$ is $0$.  What is the reason why this is true?

\input{2311-Concept-Limit-0008.HELP.tex}

\begin{multipleChoice}
\choice{The statement is in fact false: $\lim\limits_{x\to{7}}{{\left(x - 7\right)}^{3} \cos\left(-\frac{19}{x - 7}\right)}\neq0$.}
\choice{The cosine factor decreases to $0$ faster than the polynomial.}
\choice[correct]{The cosine factor is bounded between $-1$ and $1$, so the polynomial forces the function to $0$.}
\choice{The cosine factor directly cancels out the polynomial factor.}
\end{multipleChoice}


What is the name of the theorem that applies to this problem? \qquad \\
The \underline{\answer{Squeeze}} Theorem
\end{problem}}%}

%%%%%%%%%%%%%%%%%%%%%




\latexProblemContent{
\begin{problem}

The limit as $x\to{14}$ of $f(x)={{\left(x - 14\right)} \cos\left(-\frac{19}{x - 14}\right)}$ is $0$.  What is the reason why this is true?

\input{2311-Concept-Limit-0008.HELP.tex}

\begin{multipleChoice}
\choice{The statement is in fact false: $\lim\limits_{x\to{14}}{{\left(x - 14\right)} \cos\left(-\frac{19}{x - 14}\right)}\neq0$.}
\choice{The cosine factor decreases to $0$ faster than the polynomial.}
\choice[correct]{The cosine factor is bounded between $-1$ and $1$, so the polynomial forces the function to $0$.}
\choice{The cosine factor directly cancels out the polynomial factor.}
\end{multipleChoice}


What is the name of the theorem that applies to this problem? \qquad \\
The \underline{\answer{Squeeze}} Theorem
\end{problem}}%}

%%%%%%%%%%%%%%%%%%%%%




\latexProblemContent{
\begin{problem}

The limit as $x\to{11}$ of $f(x)={{\left(x - 11\right)}^{2} \cos\left(\frac{3}{{\left(x - 11\right)}^{2}}\right)}$ is $0$.  What is the reason why this is true?

\input{2311-Concept-Limit-0008.HELP.tex}

\begin{multipleChoice}
\choice{The statement is in fact false: $\lim\limits_{x\to{11}}{{\left(x - 11\right)}^{2} \cos\left(\frac{3}{{\left(x - 11\right)}^{2}}\right)}\neq0$.}
\choice{The cosine factor decreases to $0$ faster than the polynomial.}
\choice[correct]{The cosine factor is bounded between $-1$ and $1$, so the polynomial forces the function to $0$.}
\choice{The cosine factor directly cancels out the polynomial factor.}
\end{multipleChoice}


What is the name of the theorem that applies to this problem? \qquad \\
The \underline{\answer{Squeeze}} Theorem
\end{problem}}%}

%%%%%%%%%%%%%%%%%%%%%




\latexProblemContent{
\begin{problem}

The limit as $x\to{7}$ of $f(x)={{\left(x - 7\right)} \cos\left(\frac{7}{{\left(x - 7\right)}^{2}}\right)}$ is $0$.  What is the reason why this is true?

\input{2311-Concept-Limit-0008.HELP.tex}

\begin{multipleChoice}
\choice{The statement is in fact false: $\lim\limits_{x\to{7}}{{\left(x - 7\right)} \cos\left(\frac{7}{{\left(x - 7\right)}^{2}}\right)}\neq0$.}
\choice{The cosine factor decreases to $0$ faster than the polynomial.}
\choice[correct]{The cosine factor is bounded between $-1$ and $1$, so the polynomial forces the function to $0$.}
\choice{The cosine factor directly cancels out the polynomial factor.}
\end{multipleChoice}


What is the name of the theorem that applies to this problem? \qquad \\
The \underline{\answer{Squeeze}} Theorem
\end{problem}}%}

%%%%%%%%%%%%%%%%%%%%%




\latexProblemContent{
\begin{problem}

The limit as $x\to{0}$ of $f(x)={x \cos\left(-\frac{19}{x^{2}}\right)}$ is $0$.  What is the reason why this is true?

\input{2311-Concept-Limit-0008.HELP.tex}

\begin{multipleChoice}
\choice{The statement is in fact false: $\lim\limits_{x\to{0}}{x \cos\left(-\frac{19}{x^{2}}\right)}\neq0$.}
\choice{The cosine factor decreases to $0$ faster than the polynomial.}
\choice[correct]{The cosine factor is bounded between $-1$ and $1$, so the polynomial forces the function to $0$.}
\choice{The cosine factor directly cancels out the polynomial factor.}
\end{multipleChoice}


What is the name of the theorem that applies to this problem? \qquad \\
The \underline{\answer{Squeeze}} Theorem
\end{problem}}%}

%%%%%%%%%%%%%%%%%%%%%




\latexProblemContent{
\begin{problem}

The limit as $x\to{4}$ of $f(x)={{\left(x - 4\right)} \cos\left(\frac{1}{x - 4}\right)}$ is $0$.  What is the reason why this is true?

\input{2311-Concept-Limit-0008.HELP.tex}

\begin{multipleChoice}
\choice{The statement is in fact false: $\lim\limits_{x\to{4}}{{\left(x - 4\right)} \cos\left(\frac{1}{x - 4}\right)}\neq0$.}
\choice{The cosine factor decreases to $0$ faster than the polynomial.}
\choice[correct]{The cosine factor is bounded between $-1$ and $1$, so the polynomial forces the function to $0$.}
\choice{The cosine factor directly cancels out the polynomial factor.}
\end{multipleChoice}


What is the name of the theorem that applies to this problem? \qquad \\
The \underline{\answer{Squeeze}} Theorem
\end{problem}}%}

%%%%%%%%%%%%%%%%%%%%%




\latexProblemContent{
\begin{problem}

The limit as $x\to{5}$ of $f(x)={{\left(x - 5\right)}^{2} \cos\left(\frac{3}{x - 5}\right)}$ is $0$.  What is the reason why this is true?

\input{2311-Concept-Limit-0008.HELP.tex}

\begin{multipleChoice}
\choice{The statement is in fact false: $\lim\limits_{x\to{5}}{{\left(x - 5\right)}^{2} \cos\left(\frac{3}{x - 5}\right)}\neq0$.}
\choice{The cosine factor decreases to $0$ faster than the polynomial.}
\choice[correct]{The cosine factor is bounded between $-1$ and $1$, so the polynomial forces the function to $0$.}
\choice{The cosine factor directly cancels out the polynomial factor.}
\end{multipleChoice}


What is the name of the theorem that applies to this problem? \qquad \\
The \underline{\answer{Squeeze}} Theorem
\end{problem}}%}

%%%%%%%%%%%%%%%%%%%%%




\latexProblemContent{
\begin{problem}

The limit as $x\to{-15}$ of $f(x)={{\left(x + 15\right)}^{2} \cos\left(-\frac{10}{x + 15}\right)}$ is $0$.  What is the reason why this is true?

\input{2311-Concept-Limit-0008.HELP.tex}

\begin{multipleChoice}
\choice{The statement is in fact false: $\lim\limits_{x\to{-15}}{{\left(x + 15\right)}^{2} \cos\left(-\frac{10}{x + 15}\right)}\neq0$.}
\choice{The cosine factor decreases to $0$ faster than the polynomial.}
\choice[correct]{The cosine factor is bounded between $-1$ and $1$, so the polynomial forces the function to $0$.}
\choice{The cosine factor directly cancels out the polynomial factor.}
\end{multipleChoice}


What is the name of the theorem that applies to this problem? \qquad \\
The \underline{\answer{Squeeze}} Theorem
\end{problem}}%}

%%%%%%%%%%%%%%%%%%%%%




\latexProblemContent{
\begin{problem}

The limit as $x\to{4}$ of $f(x)={{\left(x - 4\right)}^{2} \cos\left(-\frac{24}{x - 4}\right)}$ is $0$.  What is the reason why this is true?

\input{2311-Concept-Limit-0008.HELP.tex}

\begin{multipleChoice}
\choice{The statement is in fact false: $\lim\limits_{x\to{4}}{{\left(x - 4\right)}^{2} \cos\left(-\frac{24}{x - 4}\right)}\neq0$.}
\choice{The cosine factor decreases to $0$ faster than the polynomial.}
\choice[correct]{The cosine factor is bounded between $-1$ and $1$, so the polynomial forces the function to $0$.}
\choice{The cosine factor directly cancels out the polynomial factor.}
\end{multipleChoice}


What is the name of the theorem that applies to this problem? \qquad \\
The \underline{\answer{Squeeze}} Theorem
\end{problem}}%}

%%%%%%%%%%%%%%%%%%%%%




\latexProblemContent{
\begin{problem}

The limit as $x\to{-12}$ of $f(x)={{\left(x + 12\right)} \cos\left(-\frac{19}{{\left(x + 12\right)}^{2}}\right)}$ is $0$.  What is the reason why this is true?

\input{2311-Concept-Limit-0008.HELP.tex}

\begin{multipleChoice}
\choice{The statement is in fact false: $\lim\limits_{x\to{-12}}{{\left(x + 12\right)} \cos\left(-\frac{19}{{\left(x + 12\right)}^{2}}\right)}\neq0$.}
\choice{The cosine factor decreases to $0$ faster than the polynomial.}
\choice[correct]{The cosine factor is bounded between $-1$ and $1$, so the polynomial forces the function to $0$.}
\choice{The cosine factor directly cancels out the polynomial factor.}
\end{multipleChoice}


What is the name of the theorem that applies to this problem? \qquad \\
The \underline{\answer{Squeeze}} Theorem
\end{problem}}%}

%%%%%%%%%%%%%%%%%%%%%




\latexProblemContent{
\begin{problem}

The limit as $x\to{5}$ of $f(x)={{\left(x - 5\right)} \cos\left(\frac{15}{{\left(x - 5\right)}^{2}}\right)}$ is $0$.  What is the reason why this is true?

\input{2311-Concept-Limit-0008.HELP.tex}

\begin{multipleChoice}
\choice{The statement is in fact false: $\lim\limits_{x\to{5}}{{\left(x - 5\right)} \cos\left(\frac{15}{{\left(x - 5\right)}^{2}}\right)}\neq0$.}
\choice{The cosine factor decreases to $0$ faster than the polynomial.}
\choice[correct]{The cosine factor is bounded between $-1$ and $1$, so the polynomial forces the function to $0$.}
\choice{The cosine factor directly cancels out the polynomial factor.}
\end{multipleChoice}


What is the name of the theorem that applies to this problem? \qquad \\
The \underline{\answer{Squeeze}} Theorem
\end{problem}}%}

%%%%%%%%%%%%%%%%%%%%%




\latexProblemContent{
\begin{problem}

The limit as $x\to{8}$ of $f(x)={{\left(x - 8\right)}^{3} \cos\left(\frac{1}{{\left(x - 8\right)}^{2}}\right)}$ is $0$.  What is the reason why this is true?

\input{2311-Concept-Limit-0008.HELP.tex}

\begin{multipleChoice}
\choice{The statement is in fact false: $\lim\limits_{x\to{8}}{{\left(x - 8\right)}^{3} \cos\left(\frac{1}{{\left(x - 8\right)}^{2}}\right)}\neq0$.}
\choice{The cosine factor decreases to $0$ faster than the polynomial.}
\choice[correct]{The cosine factor is bounded between $-1$ and $1$, so the polynomial forces the function to $0$.}
\choice{The cosine factor directly cancels out the polynomial factor.}
\end{multipleChoice}


What is the name of the theorem that applies to this problem? \qquad \\
The \underline{\answer{Squeeze}} Theorem
\end{problem}}%}

%%%%%%%%%%%%%%%%%%%%%




\latexProblemContent{
\begin{problem}

The limit as $x\to{-15}$ of $f(x)={{\left(x + 15\right)} \cos\left(\frac{18}{{\left(x + 15\right)}^{2}}\right)}$ is $0$.  What is the reason why this is true?

\input{2311-Concept-Limit-0008.HELP.tex}

\begin{multipleChoice}
\choice{The statement is in fact false: $\lim\limits_{x\to{-15}}{{\left(x + 15\right)} \cos\left(\frac{18}{{\left(x + 15\right)}^{2}}\right)}\neq0$.}
\choice{The cosine factor decreases to $0$ faster than the polynomial.}
\choice[correct]{The cosine factor is bounded between $-1$ and $1$, so the polynomial forces the function to $0$.}
\choice{The cosine factor directly cancels out the polynomial factor.}
\end{multipleChoice}


What is the name of the theorem that applies to this problem? \qquad \\
The \underline{\answer{Squeeze}} Theorem
\end{problem}}%}

%%%%%%%%%%%%%%%%%%%%%




\latexProblemContent{
\begin{problem}

The limit as $x\to{-9}$ of $f(x)={{\left(x + 9\right)} \cos\left(\frac{3}{{\left(x + 9\right)}^{2}}\right)}$ is $0$.  What is the reason why this is true?

\input{2311-Concept-Limit-0008.HELP.tex}

\begin{multipleChoice}
\choice{The statement is in fact false: $\lim\limits_{x\to{-9}}{{\left(x + 9\right)} \cos\left(\frac{3}{{\left(x + 9\right)}^{2}}\right)}\neq0$.}
\choice{The cosine factor decreases to $0$ faster than the polynomial.}
\choice[correct]{The cosine factor is bounded between $-1$ and $1$, so the polynomial forces the function to $0$.}
\choice{The cosine factor directly cancels out the polynomial factor.}
\end{multipleChoice}


What is the name of the theorem that applies to this problem? \qquad \\
The \underline{\answer{Squeeze}} Theorem
\end{problem}}%}

%%%%%%%%%%%%%%%%%%%%%




\latexProblemContent{
\begin{problem}

The limit as $x\to{9}$ of $f(x)={{\left(x - 9\right)} \cos\left(\frac{7}{{\left(x - 9\right)}^{2}}\right)}$ is $0$.  What is the reason why this is true?

\input{2311-Concept-Limit-0008.HELP.tex}

\begin{multipleChoice}
\choice{The statement is in fact false: $\lim\limits_{x\to{9}}{{\left(x - 9\right)} \cos\left(\frac{7}{{\left(x - 9\right)}^{2}}\right)}\neq0$.}
\choice{The cosine factor decreases to $0$ faster than the polynomial.}
\choice[correct]{The cosine factor is bounded between $-1$ and $1$, so the polynomial forces the function to $0$.}
\choice{The cosine factor directly cancels out the polynomial factor.}
\end{multipleChoice}


What is the name of the theorem that applies to this problem? \qquad \\
The \underline{\answer{Squeeze}} Theorem
\end{problem}}%}

%%%%%%%%%%%%%%%%%%%%%




\latexProblemContent{
\begin{problem}

The limit as $x\to{0}$ of $f(x)={x^{3} \cos\left(\frac{4}{x^{2}}\right)}$ is $0$.  What is the reason why this is true?

\input{2311-Concept-Limit-0008.HELP.tex}

\begin{multipleChoice}
\choice{The statement is in fact false: $\lim\limits_{x\to{0}}{x^{3} \cos\left(\frac{4}{x^{2}}\right)}\neq0$.}
\choice{The cosine factor decreases to $0$ faster than the polynomial.}
\choice[correct]{The cosine factor is bounded between $-1$ and $1$, so the polynomial forces the function to $0$.}
\choice{The cosine factor directly cancels out the polynomial factor.}
\end{multipleChoice}


What is the name of the theorem that applies to this problem? \qquad \\
The \underline{\answer{Squeeze}} Theorem
\end{problem}}%}

%%%%%%%%%%%%%%%%%%%%%




\latexProblemContent{
\begin{problem}

The limit as $x\to{2}$ of $f(x)={{\left(x - 2\right)} \cos\left(\frac{1}{{\left(x - 2\right)}^{2}}\right)}$ is $0$.  What is the reason why this is true?

\input{2311-Concept-Limit-0008.HELP.tex}

\begin{multipleChoice}
\choice{The statement is in fact false: $\lim\limits_{x\to{2}}{{\left(x - 2\right)} \cos\left(\frac{1}{{\left(x - 2\right)}^{2}}\right)}\neq0$.}
\choice{The cosine factor decreases to $0$ faster than the polynomial.}
\choice[correct]{The cosine factor is bounded between $-1$ and $1$, so the polynomial forces the function to $0$.}
\choice{The cosine factor directly cancels out the polynomial factor.}
\end{multipleChoice}


What is the name of the theorem that applies to this problem? \qquad \\
The \underline{\answer{Squeeze}} Theorem
\end{problem}}%}

%%%%%%%%%%%%%%%%%%%%%




\latexProblemContent{
\begin{problem}

The limit as $x\to{-6}$ of $f(x)={{\left(x + 6\right)} \cos\left(-\frac{21}{{\left(x + 6\right)}^{2}}\right)}$ is $0$.  What is the reason why this is true?

\input{2311-Concept-Limit-0008.HELP.tex}

\begin{multipleChoice}
\choice{The statement is in fact false: $\lim\limits_{x\to{-6}}{{\left(x + 6\right)} \cos\left(-\frac{21}{{\left(x + 6\right)}^{2}}\right)}\neq0$.}
\choice{The cosine factor decreases to $0$ faster than the polynomial.}
\choice[correct]{The cosine factor is bounded between $-1$ and $1$, so the polynomial forces the function to $0$.}
\choice{The cosine factor directly cancels out the polynomial factor.}
\end{multipleChoice}


What is the name of the theorem that applies to this problem? \qquad \\
The \underline{\answer{Squeeze}} Theorem
\end{problem}}%}

%%%%%%%%%%%%%%%%%%%%%




\latexProblemContent{
\begin{problem}

The limit as $x\to{-3}$ of $f(x)={{\left(x + 3\right)} \cos\left(-\frac{1}{x + 3}\right)}$ is $0$.  What is the reason why this is true?

\input{2311-Concept-Limit-0008.HELP.tex}

\begin{multipleChoice}
\choice{The statement is in fact false: $\lim\limits_{x\to{-3}}{{\left(x + 3\right)} \cos\left(-\frac{1}{x + 3}\right)}\neq0$.}
\choice{The cosine factor decreases to $0$ faster than the polynomial.}
\choice[correct]{The cosine factor is bounded between $-1$ and $1$, so the polynomial forces the function to $0$.}
\choice{The cosine factor directly cancels out the polynomial factor.}
\end{multipleChoice}


What is the name of the theorem that applies to this problem? \qquad \\
The \underline{\answer{Squeeze}} Theorem
\end{problem}}%}

%%%%%%%%%%%%%%%%%%%%%




\latexProblemContent{
\begin{problem}

The limit as $x\to{15}$ of $f(x)={{\left(x - 15\right)} \cos\left(\frac{9}{{\left(x - 15\right)}^{2}}\right)}$ is $0$.  What is the reason why this is true?

\input{2311-Concept-Limit-0008.HELP.tex}

\begin{multipleChoice}
\choice{The statement is in fact false: $\lim\limits_{x\to{15}}{{\left(x - 15\right)} \cos\left(\frac{9}{{\left(x - 15\right)}^{2}}\right)}\neq0$.}
\choice{The cosine factor decreases to $0$ faster than the polynomial.}
\choice[correct]{The cosine factor is bounded between $-1$ and $1$, so the polynomial forces the function to $0$.}
\choice{The cosine factor directly cancels out the polynomial factor.}
\end{multipleChoice}


What is the name of the theorem that applies to this problem? \qquad \\
The \underline{\answer{Squeeze}} Theorem
\end{problem}}%}

%%%%%%%%%%%%%%%%%%%%%




\latexProblemContent{
\begin{problem}

The limit as $x\to{2}$ of $f(x)={{\left(x - 2\right)}^{3} \cos\left(-\frac{22}{{\left(x - 2\right)}^{2}}\right)}$ is $0$.  What is the reason why this is true?

\input{2311-Concept-Limit-0008.HELP.tex}

\begin{multipleChoice}
\choice{The statement is in fact false: $\lim\limits_{x\to{2}}{{\left(x - 2\right)}^{3} \cos\left(-\frac{22}{{\left(x - 2\right)}^{2}}\right)}\neq0$.}
\choice{The cosine factor decreases to $0$ faster than the polynomial.}
\choice[correct]{The cosine factor is bounded between $-1$ and $1$, so the polynomial forces the function to $0$.}
\choice{The cosine factor directly cancels out the polynomial factor.}
\end{multipleChoice}


What is the name of the theorem that applies to this problem? \qquad \\
The \underline{\answer{Squeeze}} Theorem
\end{problem}}%}

%%%%%%%%%%%%%%%%%%%%%




\latexProblemContent{
\begin{problem}

The limit as $x\to{-12}$ of $f(x)={{\left(x + 12\right)} \cos\left(\frac{19}{{\left(x + 12\right)}^{2}}\right)}$ is $0$.  What is the reason why this is true?

\input{2311-Concept-Limit-0008.HELP.tex}

\begin{multipleChoice}
\choice{The statement is in fact false: $\lim\limits_{x\to{-12}}{{\left(x + 12\right)} \cos\left(\frac{19}{{\left(x + 12\right)}^{2}}\right)}\neq0$.}
\choice{The cosine factor decreases to $0$ faster than the polynomial.}
\choice[correct]{The cosine factor is bounded between $-1$ and $1$, so the polynomial forces the function to $0$.}
\choice{The cosine factor directly cancels out the polynomial factor.}
\end{multipleChoice}


What is the name of the theorem that applies to this problem? \qquad \\
The \underline{\answer{Squeeze}} Theorem
\end{problem}}%}

%%%%%%%%%%%%%%%%%%%%%




\latexProblemContent{
\begin{problem}

The limit as $x\to{8}$ of $f(x)={{\left(x - 8\right)}^{2} \cos\left(-\frac{11}{x - 8}\right)}$ is $0$.  What is the reason why this is true?

\input{2311-Concept-Limit-0008.HELP.tex}

\begin{multipleChoice}
\choice{The statement is in fact false: $\lim\limits_{x\to{8}}{{\left(x - 8\right)}^{2} \cos\left(-\frac{11}{x - 8}\right)}\neq0$.}
\choice{The cosine factor decreases to $0$ faster than the polynomial.}
\choice[correct]{The cosine factor is bounded between $-1$ and $1$, so the polynomial forces the function to $0$.}
\choice{The cosine factor directly cancels out the polynomial factor.}
\end{multipleChoice}


What is the name of the theorem that applies to this problem? \qquad \\
The \underline{\answer{Squeeze}} Theorem
\end{problem}}%}

%%%%%%%%%%%%%%%%%%%%%




\latexProblemContent{
\begin{problem}

The limit as $x\to{12}$ of $f(x)={{\left(x - 12\right)} \cos\left(-\frac{14}{{\left(x - 12\right)}^{2}}\right)}$ is $0$.  What is the reason why this is true?

\input{2311-Concept-Limit-0008.HELP.tex}

\begin{multipleChoice}
\choice{The statement is in fact false: $\lim\limits_{x\to{12}}{{\left(x - 12\right)} \cos\left(-\frac{14}{{\left(x - 12\right)}^{2}}\right)}\neq0$.}
\choice{The cosine factor decreases to $0$ faster than the polynomial.}
\choice[correct]{The cosine factor is bounded between $-1$ and $1$, so the polynomial forces the function to $0$.}
\choice{The cosine factor directly cancels out the polynomial factor.}
\end{multipleChoice}


What is the name of the theorem that applies to this problem? \qquad \\
The \underline{\answer{Squeeze}} Theorem
\end{problem}}%}

%%%%%%%%%%%%%%%%%%%%%




\latexProblemContent{
\begin{problem}

The limit as $x\to{-14}$ of $f(x)={{\left(x + 14\right)}^{2} \cos\left(-\frac{5}{{\left(x + 14\right)}^{2}}\right)}$ is $0$.  What is the reason why this is true?

\input{2311-Concept-Limit-0008.HELP.tex}

\begin{multipleChoice}
\choice{The statement is in fact false: $\lim\limits_{x\to{-14}}{{\left(x + 14\right)}^{2} \cos\left(-\frac{5}{{\left(x + 14\right)}^{2}}\right)}\neq0$.}
\choice{The cosine factor decreases to $0$ faster than the polynomial.}
\choice[correct]{The cosine factor is bounded between $-1$ and $1$, so the polynomial forces the function to $0$.}
\choice{The cosine factor directly cancels out the polynomial factor.}
\end{multipleChoice}


What is the name of the theorem that applies to this problem? \qquad \\
The \underline{\answer{Squeeze}} Theorem
\end{problem}}%}

%%%%%%%%%%%%%%%%%%%%%




\latexProblemContent{
\begin{problem}

The limit as $x\to{3}$ of $f(x)={{\left(x - 3\right)}^{2} \cos\left(-\frac{13}{{\left(x - 3\right)}^{2}}\right)}$ is $0$.  What is the reason why this is true?

\input{2311-Concept-Limit-0008.HELP.tex}

\begin{multipleChoice}
\choice{The statement is in fact false: $\lim\limits_{x\to{3}}{{\left(x - 3\right)}^{2} \cos\left(-\frac{13}{{\left(x - 3\right)}^{2}}\right)}\neq0$.}
\choice{The cosine factor decreases to $0$ faster than the polynomial.}
\choice[correct]{The cosine factor is bounded between $-1$ and $1$, so the polynomial forces the function to $0$.}
\choice{The cosine factor directly cancels out the polynomial factor.}
\end{multipleChoice}


What is the name of the theorem that applies to this problem? \qquad \\
The \underline{\answer{Squeeze}} Theorem
\end{problem}}%}

%%%%%%%%%%%%%%%%%%%%%




\latexProblemContent{
\begin{problem}

The limit as $x\to{5}$ of $f(x)={{\left(x - 5\right)}^{2} \cos\left(\frac{17}{x - 5}\right)}$ is $0$.  What is the reason why this is true?

\input{2311-Concept-Limit-0008.HELP.tex}

\begin{multipleChoice}
\choice{The statement is in fact false: $\lim\limits_{x\to{5}}{{\left(x - 5\right)}^{2} \cos\left(\frac{17}{x - 5}\right)}\neq0$.}
\choice{The cosine factor decreases to $0$ faster than the polynomial.}
\choice[correct]{The cosine factor is bounded between $-1$ and $1$, so the polynomial forces the function to $0$.}
\choice{The cosine factor directly cancels out the polynomial factor.}
\end{multipleChoice}


What is the name of the theorem that applies to this problem? \qquad \\
The \underline{\answer{Squeeze}} Theorem
\end{problem}}%}

%%%%%%%%%%%%%%%%%%%%%




\latexProblemContent{
\begin{problem}

The limit as $x\to{-6}$ of $f(x)={{\left(x + 6\right)}^{2} \cos\left(-\frac{22}{{\left(x + 6\right)}^{2}}\right)}$ is $0$.  What is the reason why this is true?

\input{2311-Concept-Limit-0008.HELP.tex}

\begin{multipleChoice}
\choice{The statement is in fact false: $\lim\limits_{x\to{-6}}{{\left(x + 6\right)}^{2} \cos\left(-\frac{22}{{\left(x + 6\right)}^{2}}\right)}\neq0$.}
\choice{The cosine factor decreases to $0$ faster than the polynomial.}
\choice[correct]{The cosine factor is bounded between $-1$ and $1$, so the polynomial forces the function to $0$.}
\choice{The cosine factor directly cancels out the polynomial factor.}
\end{multipleChoice}


What is the name of the theorem that applies to this problem? \qquad \\
The \underline{\answer{Squeeze}} Theorem
\end{problem}}%}

%%%%%%%%%%%%%%%%%%%%%




\latexProblemContent{
\begin{problem}

The limit as $x\to{-15}$ of $f(x)={{\left(x + 15\right)}^{2} \cos\left(-\frac{2}{x + 15}\right)}$ is $0$.  What is the reason why this is true?

\input{2311-Concept-Limit-0008.HELP.tex}

\begin{multipleChoice}
\choice{The statement is in fact false: $\lim\limits_{x\to{-15}}{{\left(x + 15\right)}^{2} \cos\left(-\frac{2}{x + 15}\right)}\neq0$.}
\choice{The cosine factor decreases to $0$ faster than the polynomial.}
\choice[correct]{The cosine factor is bounded between $-1$ and $1$, so the polynomial forces the function to $0$.}
\choice{The cosine factor directly cancels out the polynomial factor.}
\end{multipleChoice}


What is the name of the theorem that applies to this problem? \qquad \\
The \underline{\answer{Squeeze}} Theorem
\end{problem}}%}

%%%%%%%%%%%%%%%%%%%%%




\latexProblemContent{
\begin{problem}

The limit as $x\to{-2}$ of $f(x)={{\left(x + 2\right)}^{3} \cos\left(-\frac{21}{x + 2}\right)}$ is $0$.  What is the reason why this is true?

\input{2311-Concept-Limit-0008.HELP.tex}

\begin{multipleChoice}
\choice{The statement is in fact false: $\lim\limits_{x\to{-2}}{{\left(x + 2\right)}^{3} \cos\left(-\frac{21}{x + 2}\right)}\neq0$.}
\choice{The cosine factor decreases to $0$ faster than the polynomial.}
\choice[correct]{The cosine factor is bounded between $-1$ and $1$, so the polynomial forces the function to $0$.}
\choice{The cosine factor directly cancels out the polynomial factor.}
\end{multipleChoice}


What is the name of the theorem that applies to this problem? \qquad \\
The \underline{\answer{Squeeze}} Theorem
\end{problem}}%}

%%%%%%%%%%%%%%%%%%%%%




\latexProblemContent{
\begin{problem}

The limit as $x\to{10}$ of $f(x)={{\left(x - 10\right)} \cos\left(\frac{3}{{\left(x - 10\right)}^{2}}\right)}$ is $0$.  What is the reason why this is true?

\input{2311-Concept-Limit-0008.HELP.tex}

\begin{multipleChoice}
\choice{The statement is in fact false: $\lim\limits_{x\to{10}}{{\left(x - 10\right)} \cos\left(\frac{3}{{\left(x - 10\right)}^{2}}\right)}\neq0$.}
\choice{The cosine factor decreases to $0$ faster than the polynomial.}
\choice[correct]{The cosine factor is bounded between $-1$ and $1$, so the polynomial forces the function to $0$.}
\choice{The cosine factor directly cancels out the polynomial factor.}
\end{multipleChoice}


What is the name of the theorem that applies to this problem? \qquad \\
The \underline{\answer{Squeeze}} Theorem
\end{problem}}%}

%%%%%%%%%%%%%%%%%%%%%




\latexProblemContent{
\begin{problem}

The limit as $x\to{-6}$ of $f(x)={{\left(x + 6\right)}^{3} \cos\left(\frac{15}{x + 6}\right)}$ is $0$.  What is the reason why this is true?

\input{2311-Concept-Limit-0008.HELP.tex}

\begin{multipleChoice}
\choice{The statement is in fact false: $\lim\limits_{x\to{-6}}{{\left(x + 6\right)}^{3} \cos\left(\frac{15}{x + 6}\right)}\neq0$.}
\choice{The cosine factor decreases to $0$ faster than the polynomial.}
\choice[correct]{The cosine factor is bounded between $-1$ and $1$, so the polynomial forces the function to $0$.}
\choice{The cosine factor directly cancels out the polynomial factor.}
\end{multipleChoice}


What is the name of the theorem that applies to this problem? \qquad \\
The \underline{\answer{Squeeze}} Theorem
\end{problem}}%}

%%%%%%%%%%%%%%%%%%%%%




\latexProblemContent{
\begin{problem}

The limit as $x\to{-13}$ of $f(x)={{\left(x + 13\right)}^{2} \cos\left(\frac{8}{{\left(x + 13\right)}^{2}}\right)}$ is $0$.  What is the reason why this is true?

\input{2311-Concept-Limit-0008.HELP.tex}

\begin{multipleChoice}
\choice{The statement is in fact false: $\lim\limits_{x\to{-13}}{{\left(x + 13\right)}^{2} \cos\left(\frac{8}{{\left(x + 13\right)}^{2}}\right)}\neq0$.}
\choice{The cosine factor decreases to $0$ faster than the polynomial.}
\choice[correct]{The cosine factor is bounded between $-1$ and $1$, so the polynomial forces the function to $0$.}
\choice{The cosine factor directly cancels out the polynomial factor.}
\end{multipleChoice}


What is the name of the theorem that applies to this problem? \qquad \\
The \underline{\answer{Squeeze}} Theorem
\end{problem}}%}

%%%%%%%%%%%%%%%%%%%%%




\latexProblemContent{
\begin{problem}

The limit as $x\to{-12}$ of $f(x)={{\left(x + 12\right)} \cos\left(-\frac{6}{x + 12}\right)}$ is $0$.  What is the reason why this is true?

\input{2311-Concept-Limit-0008.HELP.tex}

\begin{multipleChoice}
\choice{The statement is in fact false: $\lim\limits_{x\to{-12}}{{\left(x + 12\right)} \cos\left(-\frac{6}{x + 12}\right)}\neq0$.}
\choice{The cosine factor decreases to $0$ faster than the polynomial.}
\choice[correct]{The cosine factor is bounded between $-1$ and $1$, so the polynomial forces the function to $0$.}
\choice{The cosine factor directly cancels out the polynomial factor.}
\end{multipleChoice}


What is the name of the theorem that applies to this problem? \qquad \\
The \underline{\answer{Squeeze}} Theorem
\end{problem}}%}

%%%%%%%%%%%%%%%%%%%%%




\latexProblemContent{
\begin{problem}

The limit as $x\to{1}$ of $f(x)={{\left(x - 1\right)}^{3} \cos\left(\frac{24}{x - 1}\right)}$ is $0$.  What is the reason why this is true?

\input{2311-Concept-Limit-0008.HELP.tex}

\begin{multipleChoice}
\choice{The statement is in fact false: $\lim\limits_{x\to{1}}{{\left(x - 1\right)}^{3} \cos\left(\frac{24}{x - 1}\right)}\neq0$.}
\choice{The cosine factor decreases to $0$ faster than the polynomial.}
\choice[correct]{The cosine factor is bounded between $-1$ and $1$, so the polynomial forces the function to $0$.}
\choice{The cosine factor directly cancels out the polynomial factor.}
\end{multipleChoice}


What is the name of the theorem that applies to this problem? \qquad \\
The \underline{\answer{Squeeze}} Theorem
\end{problem}}%}

%%%%%%%%%%%%%%%%%%%%%




\latexProblemContent{
\begin{problem}

The limit as $x\to{12}$ of $f(x)={{\left(x - 12\right)} \cos\left(-\frac{5}{x - 12}\right)}$ is $0$.  What is the reason why this is true?

\input{2311-Concept-Limit-0008.HELP.tex}

\begin{multipleChoice}
\choice{The statement is in fact false: $\lim\limits_{x\to{12}}{{\left(x - 12\right)} \cos\left(-\frac{5}{x - 12}\right)}\neq0$.}
\choice{The cosine factor decreases to $0$ faster than the polynomial.}
\choice[correct]{The cosine factor is bounded between $-1$ and $1$, so the polynomial forces the function to $0$.}
\choice{The cosine factor directly cancels out the polynomial factor.}
\end{multipleChoice}


What is the name of the theorem that applies to this problem? \qquad \\
The \underline{\answer{Squeeze}} Theorem
\end{problem}}%}

%%%%%%%%%%%%%%%%%%%%%




\latexProblemContent{
\begin{problem}

The limit as $x\to{6}$ of $f(x)={{\left(x - 6\right)}^{3} \cos\left(-\frac{25}{x - 6}\right)}$ is $0$.  What is the reason why this is true?

\input{2311-Concept-Limit-0008.HELP.tex}

\begin{multipleChoice}
\choice{The statement is in fact false: $\lim\limits_{x\to{6}}{{\left(x - 6\right)}^{3} \cos\left(-\frac{25}{x - 6}\right)}\neq0$.}
\choice{The cosine factor decreases to $0$ faster than the polynomial.}
\choice[correct]{The cosine factor is bounded between $-1$ and $1$, so the polynomial forces the function to $0$.}
\choice{The cosine factor directly cancels out the polynomial factor.}
\end{multipleChoice}


What is the name of the theorem that applies to this problem? \qquad \\
The \underline{\answer{Squeeze}} Theorem
\end{problem}}%}

%%%%%%%%%%%%%%%%%%%%%




\latexProblemContent{
\begin{problem}

The limit as $x\to{-4}$ of $f(x)={{\left(x + 4\right)}^{3} \cos\left(-\frac{8}{{\left(x + 4\right)}^{2}}\right)}$ is $0$.  What is the reason why this is true?

\input{2311-Concept-Limit-0008.HELP.tex}

\begin{multipleChoice}
\choice{The statement is in fact false: $\lim\limits_{x\to{-4}}{{\left(x + 4\right)}^{3} \cos\left(-\frac{8}{{\left(x + 4\right)}^{2}}\right)}\neq0$.}
\choice{The cosine factor decreases to $0$ faster than the polynomial.}
\choice[correct]{The cosine factor is bounded between $-1$ and $1$, so the polynomial forces the function to $0$.}
\choice{The cosine factor directly cancels out the polynomial factor.}
\end{multipleChoice}


What is the name of the theorem that applies to this problem? \qquad \\
The \underline{\answer{Squeeze}} Theorem
\end{problem}}%}

%%%%%%%%%%%%%%%%%%%%%




\latexProblemContent{
\begin{problem}

The limit as $x\to{-2}$ of $f(x)={{\left(x + 2\right)}^{3} \cos\left(-\frac{2}{{\left(x + 2\right)}^{2}}\right)}$ is $0$.  What is the reason why this is true?

\input{2311-Concept-Limit-0008.HELP.tex}

\begin{multipleChoice}
\choice{The statement is in fact false: $\lim\limits_{x\to{-2}}{{\left(x + 2\right)}^{3} \cos\left(-\frac{2}{{\left(x + 2\right)}^{2}}\right)}\neq0$.}
\choice{The cosine factor decreases to $0$ faster than the polynomial.}
\choice[correct]{The cosine factor is bounded between $-1$ and $1$, so the polynomial forces the function to $0$.}
\choice{The cosine factor directly cancels out the polynomial factor.}
\end{multipleChoice}


What is the name of the theorem that applies to this problem? \qquad \\
The \underline{\answer{Squeeze}} Theorem
\end{problem}}%}

%%%%%%%%%%%%%%%%%%%%%




\latexProblemContent{
\begin{problem}

The limit as $x\to{-1}$ of $f(x)={{\left(x + 1\right)} \cos\left(-\frac{17}{{\left(x + 1\right)}^{2}}\right)}$ is $0$.  What is the reason why this is true?

\input{2311-Concept-Limit-0008.HELP.tex}

\begin{multipleChoice}
\choice{The statement is in fact false: $\lim\limits_{x\to{-1}}{{\left(x + 1\right)} \cos\left(-\frac{17}{{\left(x + 1\right)}^{2}}\right)}\neq0$.}
\choice{The cosine factor decreases to $0$ faster than the polynomial.}
\choice[correct]{The cosine factor is bounded between $-1$ and $1$, so the polynomial forces the function to $0$.}
\choice{The cosine factor directly cancels out the polynomial factor.}
\end{multipleChoice}


What is the name of the theorem that applies to this problem? \qquad \\
The \underline{\answer{Squeeze}} Theorem
\end{problem}}%}

%%%%%%%%%%%%%%%%%%%%%




\latexProblemContent{
\begin{problem}

The limit as $x\to{8}$ of $f(x)={{\left(x - 8\right)} \cos\left(\frac{15}{{\left(x - 8\right)}^{2}}\right)}$ is $0$.  What is the reason why this is true?

\input{2311-Concept-Limit-0008.HELP.tex}

\begin{multipleChoice}
\choice{The statement is in fact false: $\lim\limits_{x\to{8}}{{\left(x - 8\right)} \cos\left(\frac{15}{{\left(x - 8\right)}^{2}}\right)}\neq0$.}
\choice{The cosine factor decreases to $0$ faster than the polynomial.}
\choice[correct]{The cosine factor is bounded between $-1$ and $1$, so the polynomial forces the function to $0$.}
\choice{The cosine factor directly cancels out the polynomial factor.}
\end{multipleChoice}


What is the name of the theorem that applies to this problem? \qquad \\
The \underline{\answer{Squeeze}} Theorem
\end{problem}}%}

%%%%%%%%%%%%%%%%%%%%%




\latexProblemContent{
\begin{problem}

The limit as $x\to{1}$ of $f(x)={{\left(x - 1\right)}^{3} \cos\left(-\frac{7}{{\left(x - 1\right)}^{2}}\right)}$ is $0$.  What is the reason why this is true?

\input{2311-Concept-Limit-0008.HELP.tex}

\begin{multipleChoice}
\choice{The statement is in fact false: $\lim\limits_{x\to{1}}{{\left(x - 1\right)}^{3} \cos\left(-\frac{7}{{\left(x - 1\right)}^{2}}\right)}\neq0$.}
\choice{The cosine factor decreases to $0$ faster than the polynomial.}
\choice[correct]{The cosine factor is bounded between $-1$ and $1$, so the polynomial forces the function to $0$.}
\choice{The cosine factor directly cancels out the polynomial factor.}
\end{multipleChoice}


What is the name of the theorem that applies to this problem? \qquad \\
The \underline{\answer{Squeeze}} Theorem
\end{problem}}%}

%%%%%%%%%%%%%%%%%%%%%




\latexProblemContent{
\begin{problem}

The limit as $x\to{-5}$ of $f(x)={{\left(x + 5\right)} \cos\left(\frac{3}{{\left(x + 5\right)}^{2}}\right)}$ is $0$.  What is the reason why this is true?

\input{2311-Concept-Limit-0008.HELP.tex}

\begin{multipleChoice}
\choice{The statement is in fact false: $\lim\limits_{x\to{-5}}{{\left(x + 5\right)} \cos\left(\frac{3}{{\left(x + 5\right)}^{2}}\right)}\neq0$.}
\choice{The cosine factor decreases to $0$ faster than the polynomial.}
\choice[correct]{The cosine factor is bounded between $-1$ and $1$, so the polynomial forces the function to $0$.}
\choice{The cosine factor directly cancels out the polynomial factor.}
\end{multipleChoice}


What is the name of the theorem that applies to this problem? \qquad \\
The \underline{\answer{Squeeze}} Theorem
\end{problem}}%}

%%%%%%%%%%%%%%%%%%%%%




\latexProblemContent{
\begin{problem}

The limit as $x\to{5}$ of $f(x)={{\left(x - 5\right)}^{2} \cos\left(\frac{7}{x - 5}\right)}$ is $0$.  What is the reason why this is true?

\input{2311-Concept-Limit-0008.HELP.tex}

\begin{multipleChoice}
\choice{The statement is in fact false: $\lim\limits_{x\to{5}}{{\left(x - 5\right)}^{2} \cos\left(\frac{7}{x - 5}\right)}\neq0$.}
\choice{The cosine factor decreases to $0$ faster than the polynomial.}
\choice[correct]{The cosine factor is bounded between $-1$ and $1$, so the polynomial forces the function to $0$.}
\choice{The cosine factor directly cancels out the polynomial factor.}
\end{multipleChoice}


What is the name of the theorem that applies to this problem? \qquad \\
The \underline{\answer{Squeeze}} Theorem
\end{problem}}%}

%%%%%%%%%%%%%%%%%%%%%




\latexProblemContent{
\begin{problem}

The limit as $x\to{-9}$ of $f(x)={{\left(x + 9\right)}^{3} \cos\left(-\frac{25}{x + 9}\right)}$ is $0$.  What is the reason why this is true?

\input{2311-Concept-Limit-0008.HELP.tex}

\begin{multipleChoice}
\choice{The statement is in fact false: $\lim\limits_{x\to{-9}}{{\left(x + 9\right)}^{3} \cos\left(-\frac{25}{x + 9}\right)}\neq0$.}
\choice{The cosine factor decreases to $0$ faster than the polynomial.}
\choice[correct]{The cosine factor is bounded between $-1$ and $1$, so the polynomial forces the function to $0$.}
\choice{The cosine factor directly cancels out the polynomial factor.}
\end{multipleChoice}


What is the name of the theorem that applies to this problem? \qquad \\
The \underline{\answer{Squeeze}} Theorem
\end{problem}}%}

%%%%%%%%%%%%%%%%%%%%%




\latexProblemContent{
\begin{problem}

The limit as $x\to{6}$ of $f(x)={{\left(x - 6\right)} \cos\left(\frac{1}{x - 6}\right)}$ is $0$.  What is the reason why this is true?

\input{2311-Concept-Limit-0008.HELP.tex}

\begin{multipleChoice}
\choice{The statement is in fact false: $\lim\limits_{x\to{6}}{{\left(x - 6\right)} \cos\left(\frac{1}{x - 6}\right)}\neq0$.}
\choice{The cosine factor decreases to $0$ faster than the polynomial.}
\choice[correct]{The cosine factor is bounded between $-1$ and $1$, so the polynomial forces the function to $0$.}
\choice{The cosine factor directly cancels out the polynomial factor.}
\end{multipleChoice}


What is the name of the theorem that applies to this problem? \qquad \\
The \underline{\answer{Squeeze}} Theorem
\end{problem}}%}

%%%%%%%%%%%%%%%%%%%%%




\latexProblemContent{
\begin{problem}

The limit as $x\to{8}$ of $f(x)={{\left(x - 8\right)}^{2} \cos\left(-\frac{15}{{\left(x - 8\right)}^{2}}\right)}$ is $0$.  What is the reason why this is true?

\input{2311-Concept-Limit-0008.HELP.tex}

\begin{multipleChoice}
\choice{The statement is in fact false: $\lim\limits_{x\to{8}}{{\left(x - 8\right)}^{2} \cos\left(-\frac{15}{{\left(x - 8\right)}^{2}}\right)}\neq0$.}
\choice{The cosine factor decreases to $0$ faster than the polynomial.}
\choice[correct]{The cosine factor is bounded between $-1$ and $1$, so the polynomial forces the function to $0$.}
\choice{The cosine factor directly cancels out the polynomial factor.}
\end{multipleChoice}


What is the name of the theorem that applies to this problem? \qquad \\
The \underline{\answer{Squeeze}} Theorem
\end{problem}}%}

%%%%%%%%%%%%%%%%%%%%%




\latexProblemContent{
\begin{problem}

The limit as $x\to{-15}$ of $f(x)={{\left(x + 15\right)}^{3} \cos\left(-\frac{18}{{\left(x + 15\right)}^{2}}\right)}$ is $0$.  What is the reason why this is true?

\input{2311-Concept-Limit-0008.HELP.tex}

\begin{multipleChoice}
\choice{The statement is in fact false: $\lim\limits_{x\to{-15}}{{\left(x + 15\right)}^{3} \cos\left(-\frac{18}{{\left(x + 15\right)}^{2}}\right)}\neq0$.}
\choice{The cosine factor decreases to $0$ faster than the polynomial.}
\choice[correct]{The cosine factor is bounded between $-1$ and $1$, so the polynomial forces the function to $0$.}
\choice{The cosine factor directly cancels out the polynomial factor.}
\end{multipleChoice}


What is the name of the theorem that applies to this problem? \qquad \\
The \underline{\answer{Squeeze}} Theorem
\end{problem}}%}

%%%%%%%%%%%%%%%%%%%%%




\latexProblemContent{
\begin{problem}

The limit as $x\to{-6}$ of $f(x)={{\left(x + 6\right)}^{2} \cos\left(\frac{20}{{\left(x + 6\right)}^{2}}\right)}$ is $0$.  What is the reason why this is true?

\input{2311-Concept-Limit-0008.HELP.tex}

\begin{multipleChoice}
\choice{The statement is in fact false: $\lim\limits_{x\to{-6}}{{\left(x + 6\right)}^{2} \cos\left(\frac{20}{{\left(x + 6\right)}^{2}}\right)}\neq0$.}
\choice{The cosine factor decreases to $0$ faster than the polynomial.}
\choice[correct]{The cosine factor is bounded between $-1$ and $1$, so the polynomial forces the function to $0$.}
\choice{The cosine factor directly cancels out the polynomial factor.}
\end{multipleChoice}


What is the name of the theorem that applies to this problem? \qquad \\
The \underline{\answer{Squeeze}} Theorem
\end{problem}}%}

%%%%%%%%%%%%%%%%%%%%%




\latexProblemContent{
\begin{problem}

The limit as $x\to{-5}$ of $f(x)={{\left(x + 5\right)}^{2} \cos\left(\frac{23}{x + 5}\right)}$ is $0$.  What is the reason why this is true?

\input{2311-Concept-Limit-0008.HELP.tex}

\begin{multipleChoice}
\choice{The statement is in fact false: $\lim\limits_{x\to{-5}}{{\left(x + 5\right)}^{2} \cos\left(\frac{23}{x + 5}\right)}\neq0$.}
\choice{The cosine factor decreases to $0$ faster than the polynomial.}
\choice[correct]{The cosine factor is bounded between $-1$ and $1$, so the polynomial forces the function to $0$.}
\choice{The cosine factor directly cancels out the polynomial factor.}
\end{multipleChoice}


What is the name of the theorem that applies to this problem? \qquad \\
The \underline{\answer{Squeeze}} Theorem
\end{problem}}%}

%%%%%%%%%%%%%%%%%%%%%




\latexProblemContent{
\begin{problem}

The limit as $x\to{7}$ of $f(x)={{\left(x - 7\right)} \cos\left(-\frac{3}{{\left(x - 7\right)}^{2}}\right)}$ is $0$.  What is the reason why this is true?

\input{2311-Concept-Limit-0008.HELP.tex}

\begin{multipleChoice}
\choice{The statement is in fact false: $\lim\limits_{x\to{7}}{{\left(x - 7\right)} \cos\left(-\frac{3}{{\left(x - 7\right)}^{2}}\right)}\neq0$.}
\choice{The cosine factor decreases to $0$ faster than the polynomial.}
\choice[correct]{The cosine factor is bounded between $-1$ and $1$, so the polynomial forces the function to $0$.}
\choice{The cosine factor directly cancels out the polynomial factor.}
\end{multipleChoice}


What is the name of the theorem that applies to this problem? \qquad \\
The \underline{\answer{Squeeze}} Theorem
\end{problem}}%}

%%%%%%%%%%%%%%%%%%%%%




\latexProblemContent{
\begin{problem}

The limit as $x\to{-15}$ of $f(x)={{\left(x + 15\right)} \cos\left(\frac{13}{{\left(x + 15\right)}^{2}}\right)}$ is $0$.  What is the reason why this is true?

\input{2311-Concept-Limit-0008.HELP.tex}

\begin{multipleChoice}
\choice{The statement is in fact false: $\lim\limits_{x\to{-15}}{{\left(x + 15\right)} \cos\left(\frac{13}{{\left(x + 15\right)}^{2}}\right)}\neq0$.}
\choice{The cosine factor decreases to $0$ faster than the polynomial.}
\choice[correct]{The cosine factor is bounded between $-1$ and $1$, so the polynomial forces the function to $0$.}
\choice{The cosine factor directly cancels out the polynomial factor.}
\end{multipleChoice}


What is the name of the theorem that applies to this problem? \qquad \\
The \underline{\answer{Squeeze}} Theorem
\end{problem}}%}

%%%%%%%%%%%%%%%%%%%%%




\latexProblemContent{
\begin{problem}

The limit as $x\to{-4}$ of $f(x)={{\left(x + 4\right)}^{2} \cos\left(\frac{15}{{\left(x + 4\right)}^{2}}\right)}$ is $0$.  What is the reason why this is true?

\input{2311-Concept-Limit-0008.HELP.tex}

\begin{multipleChoice}
\choice{The statement is in fact false: $\lim\limits_{x\to{-4}}{{\left(x + 4\right)}^{2} \cos\left(\frac{15}{{\left(x + 4\right)}^{2}}\right)}\neq0$.}
\choice{The cosine factor decreases to $0$ faster than the polynomial.}
\choice[correct]{The cosine factor is bounded between $-1$ and $1$, so the polynomial forces the function to $0$.}
\choice{The cosine factor directly cancels out the polynomial factor.}
\end{multipleChoice}


What is the name of the theorem that applies to this problem? \qquad \\
The \underline{\answer{Squeeze}} Theorem
\end{problem}}%}

%%%%%%%%%%%%%%%%%%%%%




\latexProblemContent{
\begin{problem}

The limit as $x\to{4}$ of $f(x)={{\left(x - 4\right)}^{2} \cos\left(\frac{11}{x - 4}\right)}$ is $0$.  What is the reason why this is true?

\input{2311-Concept-Limit-0008.HELP.tex}

\begin{multipleChoice}
\choice{The statement is in fact false: $\lim\limits_{x\to{4}}{{\left(x - 4\right)}^{2} \cos\left(\frac{11}{x - 4}\right)}\neq0$.}
\choice{The cosine factor decreases to $0$ faster than the polynomial.}
\choice[correct]{The cosine factor is bounded between $-1$ and $1$, so the polynomial forces the function to $0$.}
\choice{The cosine factor directly cancels out the polynomial factor.}
\end{multipleChoice}


What is the name of the theorem that applies to this problem? \qquad \\
The \underline{\answer{Squeeze}} Theorem
\end{problem}}%}

%%%%%%%%%%%%%%%%%%%%%




\latexProblemContent{
\begin{problem}

The limit as $x\to{7}$ of $f(x)={{\left(x - 7\right)} \cos\left(\frac{13}{{\left(x - 7\right)}^{2}}\right)}$ is $0$.  What is the reason why this is true?

\input{2311-Concept-Limit-0008.HELP.tex}

\begin{multipleChoice}
\choice{The statement is in fact false: $\lim\limits_{x\to{7}}{{\left(x - 7\right)} \cos\left(\frac{13}{{\left(x - 7\right)}^{2}}\right)}\neq0$.}
\choice{The cosine factor decreases to $0$ faster than the polynomial.}
\choice[correct]{The cosine factor is bounded between $-1$ and $1$, so the polynomial forces the function to $0$.}
\choice{The cosine factor directly cancels out the polynomial factor.}
\end{multipleChoice}


What is the name of the theorem that applies to this problem? \qquad \\
The \underline{\answer{Squeeze}} Theorem
\end{problem}}%}

%%%%%%%%%%%%%%%%%%%%%




\latexProblemContent{
\begin{problem}

The limit as $x\to{-2}$ of $f(x)={{\left(x + 2\right)} \cos\left(\frac{5}{x + 2}\right)}$ is $0$.  What is the reason why this is true?

\input{2311-Concept-Limit-0008.HELP.tex}

\begin{multipleChoice}
\choice{The statement is in fact false: $\lim\limits_{x\to{-2}}{{\left(x + 2\right)} \cos\left(\frac{5}{x + 2}\right)}\neq0$.}
\choice{The cosine factor decreases to $0$ faster than the polynomial.}
\choice[correct]{The cosine factor is bounded between $-1$ and $1$, so the polynomial forces the function to $0$.}
\choice{The cosine factor directly cancels out the polynomial factor.}
\end{multipleChoice}


What is the name of the theorem that applies to this problem? \qquad \\
The \underline{\answer{Squeeze}} Theorem
\end{problem}}%}

%%%%%%%%%%%%%%%%%%%%%




\latexProblemContent{
\begin{problem}

The limit as $x\to{6}$ of $f(x)={{\left(x - 6\right)}^{3} \cos\left(-\frac{13}{{\left(x - 6\right)}^{2}}\right)}$ is $0$.  What is the reason why this is true?

\input{2311-Concept-Limit-0008.HELP.tex}

\begin{multipleChoice}
\choice{The statement is in fact false: $\lim\limits_{x\to{6}}{{\left(x - 6\right)}^{3} \cos\left(-\frac{13}{{\left(x - 6\right)}^{2}}\right)}\neq0$.}
\choice{The cosine factor decreases to $0$ faster than the polynomial.}
\choice[correct]{The cosine factor is bounded between $-1$ and $1$, so the polynomial forces the function to $0$.}
\choice{The cosine factor directly cancels out the polynomial factor.}
\end{multipleChoice}


What is the name of the theorem that applies to this problem? \qquad \\
The \underline{\answer{Squeeze}} Theorem
\end{problem}}%}

%%%%%%%%%%%%%%%%%%%%%




\latexProblemContent{
\begin{problem}

The limit as $x\to{3}$ of $f(x)={{\left(x - 3\right)} \cos\left(-\frac{8}{x - 3}\right)}$ is $0$.  What is the reason why this is true?

\input{2311-Concept-Limit-0008.HELP.tex}

\begin{multipleChoice}
\choice{The statement is in fact false: $\lim\limits_{x\to{3}}{{\left(x - 3\right)} \cos\left(-\frac{8}{x - 3}\right)}\neq0$.}
\choice{The cosine factor decreases to $0$ faster than the polynomial.}
\choice[correct]{The cosine factor is bounded between $-1$ and $1$, so the polynomial forces the function to $0$.}
\choice{The cosine factor directly cancels out the polynomial factor.}
\end{multipleChoice}


What is the name of the theorem that applies to this problem? \qquad \\
The \underline{\answer{Squeeze}} Theorem
\end{problem}}%}

%%%%%%%%%%%%%%%%%%%%%




\latexProblemContent{
\begin{problem}

The limit as $x\to{14}$ of $f(x)={{\left(x - 14\right)} \cos\left(-\frac{7}{x - 14}\right)}$ is $0$.  What is the reason why this is true?

\input{2311-Concept-Limit-0008.HELP.tex}

\begin{multipleChoice}
\choice{The statement is in fact false: $\lim\limits_{x\to{14}}{{\left(x - 14\right)} \cos\left(-\frac{7}{x - 14}\right)}\neq0$.}
\choice{The cosine factor decreases to $0$ faster than the polynomial.}
\choice[correct]{The cosine factor is bounded between $-1$ and $1$, so the polynomial forces the function to $0$.}
\choice{The cosine factor directly cancels out the polynomial factor.}
\end{multipleChoice}


What is the name of the theorem that applies to this problem? \qquad \\
The \underline{\answer{Squeeze}} Theorem
\end{problem}}%}

%%%%%%%%%%%%%%%%%%%%%




\latexProblemContent{
\begin{problem}

The limit as $x\to{2}$ of $f(x)={{\left(x - 2\right)}^{3} \cos\left(-\frac{21}{x - 2}\right)}$ is $0$.  What is the reason why this is true?

\input{2311-Concept-Limit-0008.HELP.tex}

\begin{multipleChoice}
\choice{The statement is in fact false: $\lim\limits_{x\to{2}}{{\left(x - 2\right)}^{3} \cos\left(-\frac{21}{x - 2}\right)}\neq0$.}
\choice{The cosine factor decreases to $0$ faster than the polynomial.}
\choice[correct]{The cosine factor is bounded between $-1$ and $1$, so the polynomial forces the function to $0$.}
\choice{The cosine factor directly cancels out the polynomial factor.}
\end{multipleChoice}


What is the name of the theorem that applies to this problem? \qquad \\
The \underline{\answer{Squeeze}} Theorem
\end{problem}}%}

%%%%%%%%%%%%%%%%%%%%%




\latexProblemContent{
\begin{problem}

The limit as $x\to{-15}$ of $f(x)={{\left(x + 15\right)} \cos\left(-\frac{8}{{\left(x + 15\right)}^{2}}\right)}$ is $0$.  What is the reason why this is true?

\input{2311-Concept-Limit-0008.HELP.tex}

\begin{multipleChoice}
\choice{The statement is in fact false: $\lim\limits_{x\to{-15}}{{\left(x + 15\right)} \cos\left(-\frac{8}{{\left(x + 15\right)}^{2}}\right)}\neq0$.}
\choice{The cosine factor decreases to $0$ faster than the polynomial.}
\choice[correct]{The cosine factor is bounded between $-1$ and $1$, so the polynomial forces the function to $0$.}
\choice{The cosine factor directly cancels out the polynomial factor.}
\end{multipleChoice}


What is the name of the theorem that applies to this problem? \qquad \\
The \underline{\answer{Squeeze}} Theorem
\end{problem}}%}

%%%%%%%%%%%%%%%%%%%%%




\latexProblemContent{
\begin{problem}

The limit as $x\to{4}$ of $f(x)={{\left(x - 4\right)} \cos\left(-\frac{13}{x - 4}\right)}$ is $0$.  What is the reason why this is true?

\input{2311-Concept-Limit-0008.HELP.tex}

\begin{multipleChoice}
\choice{The statement is in fact false: $\lim\limits_{x\to{4}}{{\left(x - 4\right)} \cos\left(-\frac{13}{x - 4}\right)}\neq0$.}
\choice{The cosine factor decreases to $0$ faster than the polynomial.}
\choice[correct]{The cosine factor is bounded between $-1$ and $1$, so the polynomial forces the function to $0$.}
\choice{The cosine factor directly cancels out the polynomial factor.}
\end{multipleChoice}


What is the name of the theorem that applies to this problem? \qquad \\
The \underline{\answer{Squeeze}} Theorem
\end{problem}}%}

%%%%%%%%%%%%%%%%%%%%%




\latexProblemContent{
\begin{problem}

The limit as $x\to{-15}$ of $f(x)={{\left(x + 15\right)}^{2} \cos\left(\frac{9}{{\left(x + 15\right)}^{2}}\right)}$ is $0$.  What is the reason why this is true?

\input{2311-Concept-Limit-0008.HELP.tex}

\begin{multipleChoice}
\choice{The statement is in fact false: $\lim\limits_{x\to{-15}}{{\left(x + 15\right)}^{2} \cos\left(\frac{9}{{\left(x + 15\right)}^{2}}\right)}\neq0$.}
\choice{The cosine factor decreases to $0$ faster than the polynomial.}
\choice[correct]{The cosine factor is bounded between $-1$ and $1$, so the polynomial forces the function to $0$.}
\choice{The cosine factor directly cancels out the polynomial factor.}
\end{multipleChoice}


What is the name of the theorem that applies to this problem? \qquad \\
The \underline{\answer{Squeeze}} Theorem
\end{problem}}%}

%%%%%%%%%%%%%%%%%%%%%




%%%%%%%%%%%%%%%%%%%%%




\latexProblemContent{
\begin{problem}

The limit as $x\to{-13}$ of $f(x)={{\left(x + 13\right)} \cos\left(-\frac{16}{{\left(x + 13\right)}^{2}}\right)}$ is $0$.  What is the reason why this is true?

\input{2311-Concept-Limit-0008.HELP.tex}

\begin{multipleChoice}
\choice{The statement is in fact false: $\lim\limits_{x\to{-13}}{{\left(x + 13\right)} \cos\left(-\frac{16}{{\left(x + 13\right)}^{2}}\right)}\neq0$.}
\choice{The cosine factor decreases to $0$ faster than the polynomial.}
\choice[correct]{The cosine factor is bounded between $-1$ and $1$, so the polynomial forces the function to $0$.}
\choice{The cosine factor directly cancels out the polynomial factor.}
\end{multipleChoice}


What is the name of the theorem that applies to this problem? \qquad \\
The \underline{\answer{Squeeze}} Theorem
\end{problem}}%}

%%%%%%%%%%%%%%%%%%%%%




\latexProblemContent{
\begin{problem}

The limit as $x\to{-4}$ of $f(x)={{\left(x + 4\right)} \cos\left(\frac{16}{x + 4}\right)}$ is $0$.  What is the reason why this is true?

\input{2311-Concept-Limit-0008.HELP.tex}

\begin{multipleChoice}
\choice{The statement is in fact false: $\lim\limits_{x\to{-4}}{{\left(x + 4\right)} \cos\left(\frac{16}{x + 4}\right)}\neq0$.}
\choice{The cosine factor decreases to $0$ faster than the polynomial.}
\choice[correct]{The cosine factor is bounded between $-1$ and $1$, so the polynomial forces the function to $0$.}
\choice{The cosine factor directly cancels out the polynomial factor.}
\end{multipleChoice}


What is the name of the theorem that applies to this problem? \qquad \\
The \underline{\answer{Squeeze}} Theorem
\end{problem}}%}

%%%%%%%%%%%%%%%%%%%%%




\latexProblemContent{
\begin{problem}

The limit as $x\to{8}$ of $f(x)={{\left(x - 8\right)} \cos\left(-\frac{12}{{\left(x - 8\right)}^{2}}\right)}$ is $0$.  What is the reason why this is true?

\input{2311-Concept-Limit-0008.HELP.tex}

\begin{multipleChoice}
\choice{The statement is in fact false: $\lim\limits_{x\to{8}}{{\left(x - 8\right)} \cos\left(-\frac{12}{{\left(x - 8\right)}^{2}}\right)}\neq0$.}
\choice{The cosine factor decreases to $0$ faster than the polynomial.}
\choice[correct]{The cosine factor is bounded between $-1$ and $1$, so the polynomial forces the function to $0$.}
\choice{The cosine factor directly cancels out the polynomial factor.}
\end{multipleChoice}


What is the name of the theorem that applies to this problem? \qquad \\
The \underline{\answer{Squeeze}} Theorem
\end{problem}}%}

%%%%%%%%%%%%%%%%%%%%%




\latexProblemContent{
\begin{problem}

The limit as $x\to{-15}$ of $f(x)={{\left(x + 15\right)}^{2} \cos\left(\frac{5}{x + 15}\right)}$ is $0$.  What is the reason why this is true?

\input{2311-Concept-Limit-0008.HELP.tex}

\begin{multipleChoice}
\choice{The statement is in fact false: $\lim\limits_{x\to{-15}}{{\left(x + 15\right)}^{2} \cos\left(\frac{5}{x + 15}\right)}\neq0$.}
\choice{The cosine factor decreases to $0$ faster than the polynomial.}
\choice[correct]{The cosine factor is bounded between $-1$ and $1$, so the polynomial forces the function to $0$.}
\choice{The cosine factor directly cancels out the polynomial factor.}
\end{multipleChoice}


What is the name of the theorem that applies to this problem? \qquad \\
The \underline{\answer{Squeeze}} Theorem
\end{problem}}%}

%%%%%%%%%%%%%%%%%%%%%




\latexProblemContent{
\begin{problem}

The limit as $x\to{2}$ of $f(x)={{\left(x - 2\right)}^{3} \cos\left(\frac{8}{x - 2}\right)}$ is $0$.  What is the reason why this is true?

\input{2311-Concept-Limit-0008.HELP.tex}

\begin{multipleChoice}
\choice{The statement is in fact false: $\lim\limits_{x\to{2}}{{\left(x - 2\right)}^{3} \cos\left(\frac{8}{x - 2}\right)}\neq0$.}
\choice{The cosine factor decreases to $0$ faster than the polynomial.}
\choice[correct]{The cosine factor is bounded between $-1$ and $1$, so the polynomial forces the function to $0$.}
\choice{The cosine factor directly cancels out the polynomial factor.}
\end{multipleChoice}


What is the name of the theorem that applies to this problem? \qquad \\
The \underline{\answer{Squeeze}} Theorem
\end{problem}}%}

%%%%%%%%%%%%%%%%%%%%%




\latexProblemContent{
\begin{problem}

The limit as $x\to{-7}$ of $f(x)={{\left(x + 7\right)}^{3} \cos\left(-\frac{19}{x + 7}\right)}$ is $0$.  What is the reason why this is true?

\input{2311-Concept-Limit-0008.HELP.tex}

\begin{multipleChoice}
\choice{The statement is in fact false: $\lim\limits_{x\to{-7}}{{\left(x + 7\right)}^{3} \cos\left(-\frac{19}{x + 7}\right)}\neq0$.}
\choice{The cosine factor decreases to $0$ faster than the polynomial.}
\choice[correct]{The cosine factor is bounded between $-1$ and $1$, so the polynomial forces the function to $0$.}
\choice{The cosine factor directly cancels out the polynomial factor.}
\end{multipleChoice}


What is the name of the theorem that applies to this problem? \qquad \\
The \underline{\answer{Squeeze}} Theorem
\end{problem}}%}

%%%%%%%%%%%%%%%%%%%%%




\latexProblemContent{
\begin{problem}

The limit as $x\to{-12}$ of $f(x)={{\left(x + 12\right)} \cos\left(-\frac{6}{{\left(x + 12\right)}^{2}}\right)}$ is $0$.  What is the reason why this is true?

\input{2311-Concept-Limit-0008.HELP.tex}

\begin{multipleChoice}
\choice{The statement is in fact false: $\lim\limits_{x\to{-12}}{{\left(x + 12\right)} \cos\left(-\frac{6}{{\left(x + 12\right)}^{2}}\right)}\neq0$.}
\choice{The cosine factor decreases to $0$ faster than the polynomial.}
\choice[correct]{The cosine factor is bounded between $-1$ and $1$, so the polynomial forces the function to $0$.}
\choice{The cosine factor directly cancels out the polynomial factor.}
\end{multipleChoice}


What is the name of the theorem that applies to this problem? \qquad \\
The \underline{\answer{Squeeze}} Theorem
\end{problem}}%}

%%%%%%%%%%%%%%%%%%%%%




\latexProblemContent{
\begin{problem}

The limit as $x\to{6}$ of $f(x)={{\left(x - 6\right)}^{3} \cos\left(\frac{12}{x - 6}\right)}$ is $0$.  What is the reason why this is true?

\input{2311-Concept-Limit-0008.HELP.tex}

\begin{multipleChoice}
\choice{The statement is in fact false: $\lim\limits_{x\to{6}}{{\left(x - 6\right)}^{3} \cos\left(\frac{12}{x - 6}\right)}\neq0$.}
\choice{The cosine factor decreases to $0$ faster than the polynomial.}
\choice[correct]{The cosine factor is bounded between $-1$ and $1$, so the polynomial forces the function to $0$.}
\choice{The cosine factor directly cancels out the polynomial factor.}
\end{multipleChoice}


What is the name of the theorem that applies to this problem? \qquad \\
The \underline{\answer{Squeeze}} Theorem
\end{problem}}%}

%%%%%%%%%%%%%%%%%%%%%




\latexProblemContent{
\begin{problem}

The limit as $x\to{14}$ of $f(x)={{\left(x - 14\right)} \cos\left(\frac{13}{x - 14}\right)}$ is $0$.  What is the reason why this is true?

\input{2311-Concept-Limit-0008.HELP.tex}

\begin{multipleChoice}
\choice{The statement is in fact false: $\lim\limits_{x\to{14}}{{\left(x - 14\right)} \cos\left(\frac{13}{x - 14}\right)}\neq0$.}
\choice{The cosine factor decreases to $0$ faster than the polynomial.}
\choice[correct]{The cosine factor is bounded between $-1$ and $1$, so the polynomial forces the function to $0$.}
\choice{The cosine factor directly cancels out the polynomial factor.}
\end{multipleChoice}


What is the name of the theorem that applies to this problem? \qquad \\
The \underline{\answer{Squeeze}} Theorem
\end{problem}}%}

%%%%%%%%%%%%%%%%%%%%%




\latexProblemContent{
\begin{problem}

The limit as $x\to{10}$ of $f(x)={{\left(x - 10\right)} \cos\left(\frac{7}{x - 10}\right)}$ is $0$.  What is the reason why this is true?

\input{2311-Concept-Limit-0008.HELP.tex}

\begin{multipleChoice}
\choice{The statement is in fact false: $\lim\limits_{x\to{10}}{{\left(x - 10\right)} \cos\left(\frac{7}{x - 10}\right)}\neq0$.}
\choice{The cosine factor decreases to $0$ faster than the polynomial.}
\choice[correct]{The cosine factor is bounded between $-1$ and $1$, so the polynomial forces the function to $0$.}
\choice{The cosine factor directly cancels out the polynomial factor.}
\end{multipleChoice}


What is the name of the theorem that applies to this problem? \qquad \\
The \underline{\answer{Squeeze}} Theorem
\end{problem}}%}

%%%%%%%%%%%%%%%%%%%%%




\latexProblemContent{
\begin{problem}

The limit as $x\to{5}$ of $f(x)={{\left(x - 5\right)} \cos\left(\frac{1}{{\left(x - 5\right)}^{2}}\right)}$ is $0$.  What is the reason why this is true?

\input{2311-Concept-Limit-0008.HELP.tex}

\begin{multipleChoice}
\choice{The statement is in fact false: $\lim\limits_{x\to{5}}{{\left(x - 5\right)} \cos\left(\frac{1}{{\left(x - 5\right)}^{2}}\right)}\neq0$.}
\choice{The cosine factor decreases to $0$ faster than the polynomial.}
\choice[correct]{The cosine factor is bounded between $-1$ and $1$, so the polynomial forces the function to $0$.}
\choice{The cosine factor directly cancels out the polynomial factor.}
\end{multipleChoice}


What is the name of the theorem that applies to this problem? \qquad \\
The \underline{\answer{Squeeze}} Theorem
\end{problem}}%}

%%%%%%%%%%%%%%%%%%%%%




\latexProblemContent{
\begin{problem}

The limit as $x\to{5}$ of $f(x)={{\left(x - 5\right)}^{3} \cos\left(-\frac{5}{x - 5}\right)}$ is $0$.  What is the reason why this is true?

\input{2311-Concept-Limit-0008.HELP.tex}

\begin{multipleChoice}
\choice{The statement is in fact false: $\lim\limits_{x\to{5}}{{\left(x - 5\right)}^{3} \cos\left(-\frac{5}{x - 5}\right)}\neq0$.}
\choice{The cosine factor decreases to $0$ faster than the polynomial.}
\choice[correct]{The cosine factor is bounded between $-1$ and $1$, so the polynomial forces the function to $0$.}
\choice{The cosine factor directly cancels out the polynomial factor.}
\end{multipleChoice}


What is the name of the theorem that applies to this problem? \qquad \\
The \underline{\answer{Squeeze}} Theorem
\end{problem}}%}

%%%%%%%%%%%%%%%%%%%%%




\latexProblemContent{
\begin{problem}

The limit as $x\to{0}$ of $f(x)={x \cos\left(-\frac{1}{x^{2}}\right)}$ is $0$.  What is the reason why this is true?

\input{2311-Concept-Limit-0008.HELP.tex}

\begin{multipleChoice}
\choice{The statement is in fact false: $\lim\limits_{x\to{0}}{x \cos\left(-\frac{1}{x^{2}}\right)}\neq0$.}
\choice{The cosine factor decreases to $0$ faster than the polynomial.}
\choice[correct]{The cosine factor is bounded between $-1$ and $1$, so the polynomial forces the function to $0$.}
\choice{The cosine factor directly cancels out the polynomial factor.}
\end{multipleChoice}


What is the name of the theorem that applies to this problem? \qquad \\
The \underline{\answer{Squeeze}} Theorem
\end{problem}}%}

%%%%%%%%%%%%%%%%%%%%%




\latexProblemContent{
\begin{problem}

The limit as $x\to{7}$ of $f(x)={{\left(x - 7\right)}^{2} \cos\left(\frac{15}{{\left(x - 7\right)}^{2}}\right)}$ is $0$.  What is the reason why this is true?

\input{2311-Concept-Limit-0008.HELP.tex}

\begin{multipleChoice}
\choice{The statement is in fact false: $\lim\limits_{x\to{7}}{{\left(x - 7\right)}^{2} \cos\left(\frac{15}{{\left(x - 7\right)}^{2}}\right)}\neq0$.}
\choice{The cosine factor decreases to $0$ faster than the polynomial.}
\choice[correct]{The cosine factor is bounded between $-1$ and $1$, so the polynomial forces the function to $0$.}
\choice{The cosine factor directly cancels out the polynomial factor.}
\end{multipleChoice}


What is the name of the theorem that applies to this problem? \qquad \\
The \underline{\answer{Squeeze}} Theorem
\end{problem}}%}

%%%%%%%%%%%%%%%%%%%%%




\latexProblemContent{
\begin{problem}

The limit as $x\to{1}$ of $f(x)={{\left(x - 1\right)}^{2} \cos\left(\frac{6}{{\left(x - 1\right)}^{2}}\right)}$ is $0$.  What is the reason why this is true?

\input{2311-Concept-Limit-0008.HELP.tex}

\begin{multipleChoice}
\choice{The statement is in fact false: $\lim\limits_{x\to{1}}{{\left(x - 1\right)}^{2} \cos\left(\frac{6}{{\left(x - 1\right)}^{2}}\right)}\neq0$.}
\choice{The cosine factor decreases to $0$ faster than the polynomial.}
\choice[correct]{The cosine factor is bounded between $-1$ and $1$, so the polynomial forces the function to $0$.}
\choice{The cosine factor directly cancels out the polynomial factor.}
\end{multipleChoice}


What is the name of the theorem that applies to this problem? \qquad \\
The \underline{\answer{Squeeze}} Theorem
\end{problem}}%}

%%%%%%%%%%%%%%%%%%%%%




\latexProblemContent{
\begin{problem}

The limit as $x\to{-3}$ of $f(x)={{\left(x + 3\right)} \cos\left(-\frac{9}{x + 3}\right)}$ is $0$.  What is the reason why this is true?

\input{2311-Concept-Limit-0008.HELP.tex}

\begin{multipleChoice}
\choice{The statement is in fact false: $\lim\limits_{x\to{-3}}{{\left(x + 3\right)} \cos\left(-\frac{9}{x + 3}\right)}\neq0$.}
\choice{The cosine factor decreases to $0$ faster than the polynomial.}
\choice[correct]{The cosine factor is bounded between $-1$ and $1$, so the polynomial forces the function to $0$.}
\choice{The cosine factor directly cancels out the polynomial factor.}
\end{multipleChoice}


What is the name of the theorem that applies to this problem? \qquad \\
The \underline{\answer{Squeeze}} Theorem
\end{problem}}%}

%%%%%%%%%%%%%%%%%%%%%




\latexProblemContent{
\begin{problem}

The limit as $x\to{-13}$ of $f(x)={{\left(x + 13\right)} \cos\left(\frac{19}{x + 13}\right)}$ is $0$.  What is the reason why this is true?

\input{2311-Concept-Limit-0008.HELP.tex}

\begin{multipleChoice}
\choice{The statement is in fact false: $\lim\limits_{x\to{-13}}{{\left(x + 13\right)} \cos\left(\frac{19}{x + 13}\right)}\neq0$.}
\choice{The cosine factor decreases to $0$ faster than the polynomial.}
\choice[correct]{The cosine factor is bounded between $-1$ and $1$, so the polynomial forces the function to $0$.}
\choice{The cosine factor directly cancels out the polynomial factor.}
\end{multipleChoice}


What is the name of the theorem that applies to this problem? \qquad \\
The \underline{\answer{Squeeze}} Theorem
\end{problem}}%}

%%%%%%%%%%%%%%%%%%%%%




\latexProblemContent{
\begin{problem}

The limit as $x\to{10}$ of $f(x)={{\left(x - 10\right)}^{2} \cos\left(-\frac{10}{{\left(x - 10\right)}^{2}}\right)}$ is $0$.  What is the reason why this is true?

\input{2311-Concept-Limit-0008.HELP.tex}

\begin{multipleChoice}
\choice{The statement is in fact false: $\lim\limits_{x\to{10}}{{\left(x - 10\right)}^{2} \cos\left(-\frac{10}{{\left(x - 10\right)}^{2}}\right)}\neq0$.}
\choice{The cosine factor decreases to $0$ faster than the polynomial.}
\choice[correct]{The cosine factor is bounded between $-1$ and $1$, so the polynomial forces the function to $0$.}
\choice{The cosine factor directly cancels out the polynomial factor.}
\end{multipleChoice}


What is the name of the theorem that applies to this problem? \qquad \\
The \underline{\answer{Squeeze}} Theorem
\end{problem}}%}

%%%%%%%%%%%%%%%%%%%%%




\latexProblemContent{
\begin{problem}

The limit as $x\to{-11}$ of $f(x)={{\left(x + 11\right)}^{3} \cos\left(-\frac{13}{x + 11}\right)}$ is $0$.  What is the reason why this is true?

\input{2311-Concept-Limit-0008.HELP.tex}

\begin{multipleChoice}
\choice{The statement is in fact false: $\lim\limits_{x\to{-11}}{{\left(x + 11\right)}^{3} \cos\left(-\frac{13}{x + 11}\right)}\neq0$.}
\choice{The cosine factor decreases to $0$ faster than the polynomial.}
\choice[correct]{The cosine factor is bounded between $-1$ and $1$, so the polynomial forces the function to $0$.}
\choice{The cosine factor directly cancels out the polynomial factor.}
\end{multipleChoice}


What is the name of the theorem that applies to this problem? \qquad \\
The \underline{\answer{Squeeze}} Theorem
\end{problem}}%}

%%%%%%%%%%%%%%%%%%%%%




\latexProblemContent{
\begin{problem}

The limit as $x\to{12}$ of $f(x)={{\left(x - 12\right)} \cos\left(-\frac{16}{{\left(x - 12\right)}^{2}}\right)}$ is $0$.  What is the reason why this is true?

\input{2311-Concept-Limit-0008.HELP.tex}

\begin{multipleChoice}
\choice{The statement is in fact false: $\lim\limits_{x\to{12}}{{\left(x - 12\right)} \cos\left(-\frac{16}{{\left(x - 12\right)}^{2}}\right)}\neq0$.}
\choice{The cosine factor decreases to $0$ faster than the polynomial.}
\choice[correct]{The cosine factor is bounded between $-1$ and $1$, so the polynomial forces the function to $0$.}
\choice{The cosine factor directly cancels out the polynomial factor.}
\end{multipleChoice}


What is the name of the theorem that applies to this problem? \qquad \\
The \underline{\answer{Squeeze}} Theorem
\end{problem}}%}

%%%%%%%%%%%%%%%%%%%%%




\latexProblemContent{
\begin{problem}

The limit as $x\to{13}$ of $f(x)={{\left(x - 13\right)}^{2} \cos\left(\frac{25}{{\left(x - 13\right)}^{2}}\right)}$ is $0$.  What is the reason why this is true?

\input{2311-Concept-Limit-0008.HELP.tex}

\begin{multipleChoice}
\choice{The statement is in fact false: $\lim\limits_{x\to{13}}{{\left(x - 13\right)}^{2} \cos\left(\frac{25}{{\left(x - 13\right)}^{2}}\right)}\neq0$.}
\choice{The cosine factor decreases to $0$ faster than the polynomial.}
\choice[correct]{The cosine factor is bounded between $-1$ and $1$, so the polynomial forces the function to $0$.}
\choice{The cosine factor directly cancels out the polynomial factor.}
\end{multipleChoice}


What is the name of the theorem that applies to this problem? \qquad \\
The \underline{\answer{Squeeze}} Theorem
\end{problem}}%}

%%%%%%%%%%%%%%%%%%%%%




\latexProblemContent{
\begin{problem}

The limit as $x\to{12}$ of $f(x)={{\left(x - 12\right)} \cos\left(\frac{3}{x - 12}\right)}$ is $0$.  What is the reason why this is true?

\input{2311-Concept-Limit-0008.HELP.tex}

\begin{multipleChoice}
\choice{The statement is in fact false: $\lim\limits_{x\to{12}}{{\left(x - 12\right)} \cos\left(\frac{3}{x - 12}\right)}\neq0$.}
\choice{The cosine factor decreases to $0$ faster than the polynomial.}
\choice[correct]{The cosine factor is bounded between $-1$ and $1$, so the polynomial forces the function to $0$.}
\choice{The cosine factor directly cancels out the polynomial factor.}
\end{multipleChoice}


What is the name of the theorem that applies to this problem? \qquad \\
The \underline{\answer{Squeeze}} Theorem
\end{problem}}%}

%%%%%%%%%%%%%%%%%%%%%




\latexProblemContent{
\begin{problem}

The limit as $x\to{-12}$ of $f(x)={{\left(x + 12\right)}^{2} \cos\left(-\frac{13}{x + 12}\right)}$ is $0$.  What is the reason why this is true?

\input{2311-Concept-Limit-0008.HELP.tex}

\begin{multipleChoice}
\choice{The statement is in fact false: $\lim\limits_{x\to{-12}}{{\left(x + 12\right)}^{2} \cos\left(-\frac{13}{x + 12}\right)}\neq0$.}
\choice{The cosine factor decreases to $0$ faster than the polynomial.}
\choice[correct]{The cosine factor is bounded between $-1$ and $1$, so the polynomial forces the function to $0$.}
\choice{The cosine factor directly cancels out the polynomial factor.}
\end{multipleChoice}


What is the name of the theorem that applies to this problem? \qquad \\
The \underline{\answer{Squeeze}} Theorem
\end{problem}}%}

%%%%%%%%%%%%%%%%%%%%%




\latexProblemContent{
\begin{problem}

The limit as $x\to{3}$ of $f(x)={{\left(x - 3\right)}^{3} \cos\left(\frac{7}{x - 3}\right)}$ is $0$.  What is the reason why this is true?

\input{2311-Concept-Limit-0008.HELP.tex}

\begin{multipleChoice}
\choice{The statement is in fact false: $\lim\limits_{x\to{3}}{{\left(x - 3\right)}^{3} \cos\left(\frac{7}{x - 3}\right)}\neq0$.}
\choice{The cosine factor decreases to $0$ faster than the polynomial.}
\choice[correct]{The cosine factor is bounded between $-1$ and $1$, so the polynomial forces the function to $0$.}
\choice{The cosine factor directly cancels out the polynomial factor.}
\end{multipleChoice}


What is the name of the theorem that applies to this problem? \qquad \\
The \underline{\answer{Squeeze}} Theorem
\end{problem}}%}

%%%%%%%%%%%%%%%%%%%%%




\latexProblemContent{
\begin{problem}

The limit as $x\to{-5}$ of $f(x)={{\left(x + 5\right)} \cos\left(\frac{21}{{\left(x + 5\right)}^{2}}\right)}$ is $0$.  What is the reason why this is true?

\input{2311-Concept-Limit-0008.HELP.tex}

\begin{multipleChoice}
\choice{The statement is in fact false: $\lim\limits_{x\to{-5}}{{\left(x + 5\right)} \cos\left(\frac{21}{{\left(x + 5\right)}^{2}}\right)}\neq0$.}
\choice{The cosine factor decreases to $0$ faster than the polynomial.}
\choice[correct]{The cosine factor is bounded between $-1$ and $1$, so the polynomial forces the function to $0$.}
\choice{The cosine factor directly cancels out the polynomial factor.}
\end{multipleChoice}


What is the name of the theorem that applies to this problem? \qquad \\
The \underline{\answer{Squeeze}} Theorem
\end{problem}}%}

%%%%%%%%%%%%%%%%%%%%%




\latexProblemContent{
\begin{problem}

The limit as $x\to{-14}$ of $f(x)={{\left(x + 14\right)}^{2} \cos\left(\frac{19}{{\left(x + 14\right)}^{2}}\right)}$ is $0$.  What is the reason why this is true?

\input{2311-Concept-Limit-0008.HELP.tex}

\begin{multipleChoice}
\choice{The statement is in fact false: $\lim\limits_{x\to{-14}}{{\left(x + 14\right)}^{2} \cos\left(\frac{19}{{\left(x + 14\right)}^{2}}\right)}\neq0$.}
\choice{The cosine factor decreases to $0$ faster than the polynomial.}
\choice[correct]{The cosine factor is bounded between $-1$ and $1$, so the polynomial forces the function to $0$.}
\choice{The cosine factor directly cancels out the polynomial factor.}
\end{multipleChoice}


What is the name of the theorem that applies to this problem? \qquad \\
The \underline{\answer{Squeeze}} Theorem
\end{problem}}%}

%%%%%%%%%%%%%%%%%%%%%




\latexProblemContent{
\begin{problem}

The limit as $x\to{9}$ of $f(x)={{\left(x - 9\right)} \cos\left(\frac{8}{x - 9}\right)}$ is $0$.  What is the reason why this is true?

\input{2311-Concept-Limit-0008.HELP.tex}

\begin{multipleChoice}
\choice{The statement is in fact false: $\lim\limits_{x\to{9}}{{\left(x - 9\right)} \cos\left(\frac{8}{x - 9}\right)}\neq0$.}
\choice{The cosine factor decreases to $0$ faster than the polynomial.}
\choice[correct]{The cosine factor is bounded between $-1$ and $1$, so the polynomial forces the function to $0$.}
\choice{The cosine factor directly cancels out the polynomial factor.}
\end{multipleChoice}


What is the name of the theorem that applies to this problem? \qquad \\
The \underline{\answer{Squeeze}} Theorem
\end{problem}}%}

%%%%%%%%%%%%%%%%%%%%%




\latexProblemContent{
\begin{problem}

The limit as $x\to{2}$ of $f(x)={{\left(x - 2\right)} \cos\left(-\frac{3}{x - 2}\right)}$ is $0$.  What is the reason why this is true?

\input{2311-Concept-Limit-0008.HELP.tex}

\begin{multipleChoice}
\choice{The statement is in fact false: $\lim\limits_{x\to{2}}{{\left(x - 2\right)} \cos\left(-\frac{3}{x - 2}\right)}\neq0$.}
\choice{The cosine factor decreases to $0$ faster than the polynomial.}
\choice[correct]{The cosine factor is bounded between $-1$ and $1$, so the polynomial forces the function to $0$.}
\choice{The cosine factor directly cancels out the polynomial factor.}
\end{multipleChoice}


What is the name of the theorem that applies to this problem? \qquad \\
The \underline{\answer{Squeeze}} Theorem
\end{problem}}%}

%%%%%%%%%%%%%%%%%%%%%




\latexProblemContent{
\begin{problem}

The limit as $x\to{-11}$ of $f(x)={{\left(x + 11\right)}^{2} \cos\left(-\frac{8}{{\left(x + 11\right)}^{2}}\right)}$ is $0$.  What is the reason why this is true?

\input{2311-Concept-Limit-0008.HELP.tex}

\begin{multipleChoice}
\choice{The statement is in fact false: $\lim\limits_{x\to{-11}}{{\left(x + 11\right)}^{2} \cos\left(-\frac{8}{{\left(x + 11\right)}^{2}}\right)}\neq0$.}
\choice{The cosine factor decreases to $0$ faster than the polynomial.}
\choice[correct]{The cosine factor is bounded between $-1$ and $1$, so the polynomial forces the function to $0$.}
\choice{The cosine factor directly cancels out the polynomial factor.}
\end{multipleChoice}


What is the name of the theorem that applies to this problem? \qquad \\
The \underline{\answer{Squeeze}} Theorem
\end{problem}}%}

%%%%%%%%%%%%%%%%%%%%%




\latexProblemContent{
\begin{problem}

The limit as $x\to{3}$ of $f(x)={{\left(x - 3\right)}^{3} \cos\left(-\frac{22}{{\left(x - 3\right)}^{2}}\right)}$ is $0$.  What is the reason why this is true?

\input{2311-Concept-Limit-0008.HELP.tex}

\begin{multipleChoice}
\choice{The statement is in fact false: $\lim\limits_{x\to{3}}{{\left(x - 3\right)}^{3} \cos\left(-\frac{22}{{\left(x - 3\right)}^{2}}\right)}\neq0$.}
\choice{The cosine factor decreases to $0$ faster than the polynomial.}
\choice[correct]{The cosine factor is bounded between $-1$ and $1$, so the polynomial forces the function to $0$.}
\choice{The cosine factor directly cancels out the polynomial factor.}
\end{multipleChoice}


What is the name of the theorem that applies to this problem? \qquad \\
The \underline{\answer{Squeeze}} Theorem
\end{problem}}%}

%%%%%%%%%%%%%%%%%%%%%




\latexProblemContent{
\begin{problem}

The limit as $x\to{-9}$ of $f(x)={{\left(x + 9\right)} \cos\left(\frac{13}{x + 9}\right)}$ is $0$.  What is the reason why this is true?

\input{2311-Concept-Limit-0008.HELP.tex}

\begin{multipleChoice}
\choice{The statement is in fact false: $\lim\limits_{x\to{-9}}{{\left(x + 9\right)} \cos\left(\frac{13}{x + 9}\right)}\neq0$.}
\choice{The cosine factor decreases to $0$ faster than the polynomial.}
\choice[correct]{The cosine factor is bounded between $-1$ and $1$, so the polynomial forces the function to $0$.}
\choice{The cosine factor directly cancels out the polynomial factor.}
\end{multipleChoice}


What is the name of the theorem that applies to this problem? \qquad \\
The \underline{\answer{Squeeze}} Theorem
\end{problem}}%}

%%%%%%%%%%%%%%%%%%%%%




\latexProblemContent{
\begin{problem}

The limit as $x\to{14}$ of $f(x)={{\left(x - 14\right)} \cos\left(-\frac{12}{{\left(x - 14\right)}^{2}}\right)}$ is $0$.  What is the reason why this is true?

\input{2311-Concept-Limit-0008.HELP.tex}

\begin{multipleChoice}
\choice{The statement is in fact false: $\lim\limits_{x\to{14}}{{\left(x - 14\right)} \cos\left(-\frac{12}{{\left(x - 14\right)}^{2}}\right)}\neq0$.}
\choice{The cosine factor decreases to $0$ faster than the polynomial.}
\choice[correct]{The cosine factor is bounded between $-1$ and $1$, so the polynomial forces the function to $0$.}
\choice{The cosine factor directly cancels out the polynomial factor.}
\end{multipleChoice}


What is the name of the theorem that applies to this problem? \qquad \\
The \underline{\answer{Squeeze}} Theorem
\end{problem}}%}

%%%%%%%%%%%%%%%%%%%%%




\latexProblemContent{
\begin{problem}

The limit as $x\to{-6}$ of $f(x)={{\left(x + 6\right)}^{2} \cos\left(\frac{15}{{\left(x + 6\right)}^{2}}\right)}$ is $0$.  What is the reason why this is true?

\input{2311-Concept-Limit-0008.HELP.tex}

\begin{multipleChoice}
\choice{The statement is in fact false: $\lim\limits_{x\to{-6}}{{\left(x + 6\right)}^{2} \cos\left(\frac{15}{{\left(x + 6\right)}^{2}}\right)}\neq0$.}
\choice{The cosine factor decreases to $0$ faster than the polynomial.}
\choice[correct]{The cosine factor is bounded between $-1$ and $1$, so the polynomial forces the function to $0$.}
\choice{The cosine factor directly cancels out the polynomial factor.}
\end{multipleChoice}


What is the name of the theorem that applies to this problem? \qquad \\
The \underline{\answer{Squeeze}} Theorem
\end{problem}}%}

%%%%%%%%%%%%%%%%%%%%%




\latexProblemContent{
\begin{problem}

The limit as $x\to{-3}$ of $f(x)={{\left(x + 3\right)} \cos\left(-\frac{23}{x + 3}\right)}$ is $0$.  What is the reason why this is true?

\input{2311-Concept-Limit-0008.HELP.tex}

\begin{multipleChoice}
\choice{The statement is in fact false: $\lim\limits_{x\to{-3}}{{\left(x + 3\right)} \cos\left(-\frac{23}{x + 3}\right)}\neq0$.}
\choice{The cosine factor decreases to $0$ faster than the polynomial.}
\choice[correct]{The cosine factor is bounded between $-1$ and $1$, so the polynomial forces the function to $0$.}
\choice{The cosine factor directly cancels out the polynomial factor.}
\end{multipleChoice}


What is the name of the theorem that applies to this problem? \qquad \\
The \underline{\answer{Squeeze}} Theorem
\end{problem}}%}

%%%%%%%%%%%%%%%%%%%%%




\latexProblemContent{
\begin{problem}

The limit as $x\to{-13}$ of $f(x)={{\left(x + 13\right)}^{2} \cos\left(-\frac{6}{{\left(x + 13\right)}^{2}}\right)}$ is $0$.  What is the reason why this is true?

\input{2311-Concept-Limit-0008.HELP.tex}

\begin{multipleChoice}
\choice{The statement is in fact false: $\lim\limits_{x\to{-13}}{{\left(x + 13\right)}^{2} \cos\left(-\frac{6}{{\left(x + 13\right)}^{2}}\right)}\neq0$.}
\choice{The cosine factor decreases to $0$ faster than the polynomial.}
\choice[correct]{The cosine factor is bounded between $-1$ and $1$, so the polynomial forces the function to $0$.}
\choice{The cosine factor directly cancels out the polynomial factor.}
\end{multipleChoice}


What is the name of the theorem that applies to this problem? \qquad \\
The \underline{\answer{Squeeze}} Theorem
\end{problem}}%}

%%%%%%%%%%%%%%%%%%%%%




\latexProblemContent{
\begin{problem}

The limit as $x\to{3}$ of $f(x)={{\left(x - 3\right)} \cos\left(\frac{7}{x - 3}\right)}$ is $0$.  What is the reason why this is true?

\input{2311-Concept-Limit-0008.HELP.tex}

\begin{multipleChoice}
\choice{The statement is in fact false: $\lim\limits_{x\to{3}}{{\left(x - 3\right)} \cos\left(\frac{7}{x - 3}\right)}\neq0$.}
\choice{The cosine factor decreases to $0$ faster than the polynomial.}
\choice[correct]{The cosine factor is bounded between $-1$ and $1$, so the polynomial forces the function to $0$.}
\choice{The cosine factor directly cancels out the polynomial factor.}
\end{multipleChoice}


What is the name of the theorem that applies to this problem? \qquad \\
The \underline{\answer{Squeeze}} Theorem
\end{problem}}%}

%%%%%%%%%%%%%%%%%%%%%




%%%%%%%%%%%%%%%%%%%%%




\latexProblemContent{
\begin{problem}

The limit as $x\to{8}$ of $f(x)={{\left(x - 8\right)} \cos\left(-\frac{22}{{\left(x - 8\right)}^{2}}\right)}$ is $0$.  What is the reason why this is true?

\input{2311-Concept-Limit-0008.HELP.tex}

\begin{multipleChoice}
\choice{The statement is in fact false: $\lim\limits_{x\to{8}}{{\left(x - 8\right)} \cos\left(-\frac{22}{{\left(x - 8\right)}^{2}}\right)}\neq0$.}
\choice{The cosine factor decreases to $0$ faster than the polynomial.}
\choice[correct]{The cosine factor is bounded between $-1$ and $1$, so the polynomial forces the function to $0$.}
\choice{The cosine factor directly cancels out the polynomial factor.}
\end{multipleChoice}


What is the name of the theorem that applies to this problem? \qquad \\
The \underline{\answer{Squeeze}} Theorem
\end{problem}}%}

%%%%%%%%%%%%%%%%%%%%%




\latexProblemContent{
\begin{problem}

The limit as $x\to{-2}$ of $f(x)={{\left(x + 2\right)}^{2} \cos\left(\frac{4}{x + 2}\right)}$ is $0$.  What is the reason why this is true?

\input{2311-Concept-Limit-0008.HELP.tex}

\begin{multipleChoice}
\choice{The statement is in fact false: $\lim\limits_{x\to{-2}}{{\left(x + 2\right)}^{2} \cos\left(\frac{4}{x + 2}\right)}\neq0$.}
\choice{The cosine factor decreases to $0$ faster than the polynomial.}
\choice[correct]{The cosine factor is bounded between $-1$ and $1$, so the polynomial forces the function to $0$.}
\choice{The cosine factor directly cancels out the polynomial factor.}
\end{multipleChoice}


What is the name of the theorem that applies to this problem? \qquad \\
The \underline{\answer{Squeeze}} Theorem
\end{problem}}%}

%%%%%%%%%%%%%%%%%%%%%




\latexProblemContent{
\begin{problem}

The limit as $x\to{-4}$ of $f(x)={{\left(x + 4\right)} \cos\left(-\frac{15}{x + 4}\right)}$ is $0$.  What is the reason why this is true?

\input{2311-Concept-Limit-0008.HELP.tex}

\begin{multipleChoice}
\choice{The statement is in fact false: $\lim\limits_{x\to{-4}}{{\left(x + 4\right)} \cos\left(-\frac{15}{x + 4}\right)}\neq0$.}
\choice{The cosine factor decreases to $0$ faster than the polynomial.}
\choice[correct]{The cosine factor is bounded between $-1$ and $1$, so the polynomial forces the function to $0$.}
\choice{The cosine factor directly cancels out the polynomial factor.}
\end{multipleChoice}


What is the name of the theorem that applies to this problem? \qquad \\
The \underline{\answer{Squeeze}} Theorem
\end{problem}}%}

%%%%%%%%%%%%%%%%%%%%%




\latexProblemContent{
\begin{problem}

The limit as $x\to{3}$ of $f(x)={{\left(x - 3\right)} \cos\left(\frac{1}{{\left(x - 3\right)}^{2}}\right)}$ is $0$.  What is the reason why this is true?

\input{2311-Concept-Limit-0008.HELP.tex}

\begin{multipleChoice}
\choice{The statement is in fact false: $\lim\limits_{x\to{3}}{{\left(x - 3\right)} \cos\left(\frac{1}{{\left(x - 3\right)}^{2}}\right)}\neq0$.}
\choice{The cosine factor decreases to $0$ faster than the polynomial.}
\choice[correct]{The cosine factor is bounded between $-1$ and $1$, so the polynomial forces the function to $0$.}
\choice{The cosine factor directly cancels out the polynomial factor.}
\end{multipleChoice}


What is the name of the theorem that applies to this problem? \qquad \\
The \underline{\answer{Squeeze}} Theorem
\end{problem}}%}

%%%%%%%%%%%%%%%%%%%%%




\latexProblemContent{
\begin{problem}

The limit as $x\to{-3}$ of $f(x)={{\left(x + 3\right)}^{2} \cos\left(-\frac{15}{x + 3}\right)}$ is $0$.  What is the reason why this is true?

\input{2311-Concept-Limit-0008.HELP.tex}

\begin{multipleChoice}
\choice{The statement is in fact false: $\lim\limits_{x\to{-3}}{{\left(x + 3\right)}^{2} \cos\left(-\frac{15}{x + 3}\right)}\neq0$.}
\choice{The cosine factor decreases to $0$ faster than the polynomial.}
\choice[correct]{The cosine factor is bounded between $-1$ and $1$, so the polynomial forces the function to $0$.}
\choice{The cosine factor directly cancels out the polynomial factor.}
\end{multipleChoice}


What is the name of the theorem that applies to this problem? \qquad \\
The \underline{\answer{Squeeze}} Theorem
\end{problem}}%}

%%%%%%%%%%%%%%%%%%%%%




\latexProblemContent{
\begin{problem}

The limit as $x\to{15}$ of $f(x)={{\left(x - 15\right)}^{2} \cos\left(-\frac{1}{{\left(x - 15\right)}^{2}}\right)}$ is $0$.  What is the reason why this is true?

\input{2311-Concept-Limit-0008.HELP.tex}

\begin{multipleChoice}
\choice{The statement is in fact false: $\lim\limits_{x\to{15}}{{\left(x - 15\right)}^{2} \cos\left(-\frac{1}{{\left(x - 15\right)}^{2}}\right)}\neq0$.}
\choice{The cosine factor decreases to $0$ faster than the polynomial.}
\choice[correct]{The cosine factor is bounded between $-1$ and $1$, so the polynomial forces the function to $0$.}
\choice{The cosine factor directly cancels out the polynomial factor.}
\end{multipleChoice}


What is the name of the theorem that applies to this problem? \qquad \\
The \underline{\answer{Squeeze}} Theorem
\end{problem}}%}

%%%%%%%%%%%%%%%%%%%%%




\latexProblemContent{
\begin{problem}

The limit as $x\to{-8}$ of $f(x)={{\left(x + 8\right)}^{2} \cos\left(-\frac{11}{x + 8}\right)}$ is $0$.  What is the reason why this is true?

\input{2311-Concept-Limit-0008.HELP.tex}

\begin{multipleChoice}
\choice{The statement is in fact false: $\lim\limits_{x\to{-8}}{{\left(x + 8\right)}^{2} \cos\left(-\frac{11}{x + 8}\right)}\neq0$.}
\choice{The cosine factor decreases to $0$ faster than the polynomial.}
\choice[correct]{The cosine factor is bounded between $-1$ and $1$, so the polynomial forces the function to $0$.}
\choice{The cosine factor directly cancels out the polynomial factor.}
\end{multipleChoice}


What is the name of the theorem that applies to this problem? \qquad \\
The \underline{\answer{Squeeze}} Theorem
\end{problem}}%}

%%%%%%%%%%%%%%%%%%%%%




\latexProblemContent{
\begin{problem}

The limit as $x\to{5}$ of $f(x)={{\left(x - 5\right)}^{3} \cos\left(-\frac{2}{x - 5}\right)}$ is $0$.  What is the reason why this is true?

\input{2311-Concept-Limit-0008.HELP.tex}

\begin{multipleChoice}
\choice{The statement is in fact false: $\lim\limits_{x\to{5}}{{\left(x - 5\right)}^{3} \cos\left(-\frac{2}{x - 5}\right)}\neq0$.}
\choice{The cosine factor decreases to $0$ faster than the polynomial.}
\choice[correct]{The cosine factor is bounded between $-1$ and $1$, so the polynomial forces the function to $0$.}
\choice{The cosine factor directly cancels out the polynomial factor.}
\end{multipleChoice}


What is the name of the theorem that applies to this problem? \qquad \\
The \underline{\answer{Squeeze}} Theorem
\end{problem}}%}

%%%%%%%%%%%%%%%%%%%%%




\latexProblemContent{
\begin{problem}

The limit as $x\to{-11}$ of $f(x)={{\left(x + 11\right)}^{3} \cos\left(-\frac{6}{{\left(x + 11\right)}^{2}}\right)}$ is $0$.  What is the reason why this is true?

\input{2311-Concept-Limit-0008.HELP.tex}

\begin{multipleChoice}
\choice{The statement is in fact false: $\lim\limits_{x\to{-11}}{{\left(x + 11\right)}^{3} \cos\left(-\frac{6}{{\left(x + 11\right)}^{2}}\right)}\neq0$.}
\choice{The cosine factor decreases to $0$ faster than the polynomial.}
\choice[correct]{The cosine factor is bounded between $-1$ and $1$, so the polynomial forces the function to $0$.}
\choice{The cosine factor directly cancels out the polynomial factor.}
\end{multipleChoice}


What is the name of the theorem that applies to this problem? \qquad \\
The \underline{\answer{Squeeze}} Theorem
\end{problem}}%}

%%%%%%%%%%%%%%%%%%%%%




\latexProblemContent{
\begin{problem}

The limit as $x\to{5}$ of $f(x)={{\left(x - 5\right)}^{2} \cos\left(\frac{12}{{\left(x - 5\right)}^{2}}\right)}$ is $0$.  What is the reason why this is true?

\input{2311-Concept-Limit-0008.HELP.tex}

\begin{multipleChoice}
\choice{The statement is in fact false: $\lim\limits_{x\to{5}}{{\left(x - 5\right)}^{2} \cos\left(\frac{12}{{\left(x - 5\right)}^{2}}\right)}\neq0$.}
\choice{The cosine factor decreases to $0$ faster than the polynomial.}
\choice[correct]{The cosine factor is bounded between $-1$ and $1$, so the polynomial forces the function to $0$.}
\choice{The cosine factor directly cancels out the polynomial factor.}
\end{multipleChoice}


What is the name of the theorem that applies to this problem? \qquad \\
The \underline{\answer{Squeeze}} Theorem
\end{problem}}%}

%%%%%%%%%%%%%%%%%%%%%




\latexProblemContent{
\begin{problem}

The limit as $x\to{1}$ of $f(x)={{\left(x - 1\right)} \cos\left(\frac{23}{x - 1}\right)}$ is $0$.  What is the reason why this is true?

\input{2311-Concept-Limit-0008.HELP.tex}

\begin{multipleChoice}
\choice{The statement is in fact false: $\lim\limits_{x\to{1}}{{\left(x - 1\right)} \cos\left(\frac{23}{x - 1}\right)}\neq0$.}
\choice{The cosine factor decreases to $0$ faster than the polynomial.}
\choice[correct]{The cosine factor is bounded between $-1$ and $1$, so the polynomial forces the function to $0$.}
\choice{The cosine factor directly cancels out the polynomial factor.}
\end{multipleChoice}


What is the name of the theorem that applies to this problem? \qquad \\
The \underline{\answer{Squeeze}} Theorem
\end{problem}}%}

%%%%%%%%%%%%%%%%%%%%%




\latexProblemContent{
\begin{problem}

The limit as $x\to{-15}$ of $f(x)={{\left(x + 15\right)}^{3} \cos\left(-\frac{16}{{\left(x + 15\right)}^{2}}\right)}$ is $0$.  What is the reason why this is true?

\input{2311-Concept-Limit-0008.HELP.tex}

\begin{multipleChoice}
\choice{The statement is in fact false: $\lim\limits_{x\to{-15}}{{\left(x + 15\right)}^{3} \cos\left(-\frac{16}{{\left(x + 15\right)}^{2}}\right)}\neq0$.}
\choice{The cosine factor decreases to $0$ faster than the polynomial.}
\choice[correct]{The cosine factor is bounded between $-1$ and $1$, so the polynomial forces the function to $0$.}
\choice{The cosine factor directly cancels out the polynomial factor.}
\end{multipleChoice}


What is the name of the theorem that applies to this problem? \qquad \\
The \underline{\answer{Squeeze}} Theorem
\end{problem}}%}

%%%%%%%%%%%%%%%%%%%%%




\latexProblemContent{
\begin{problem}

The limit as $x\to{-3}$ of $f(x)={{\left(x + 3\right)}^{3} \cos\left(\frac{13}{{\left(x + 3\right)}^{2}}\right)}$ is $0$.  What is the reason why this is true?

\input{2311-Concept-Limit-0008.HELP.tex}

\begin{multipleChoice}
\choice{The statement is in fact false: $\lim\limits_{x\to{-3}}{{\left(x + 3\right)}^{3} \cos\left(\frac{13}{{\left(x + 3\right)}^{2}}\right)}\neq0$.}
\choice{The cosine factor decreases to $0$ faster than the polynomial.}
\choice[correct]{The cosine factor is bounded between $-1$ and $1$, so the polynomial forces the function to $0$.}
\choice{The cosine factor directly cancels out the polynomial factor.}
\end{multipleChoice}


What is the name of the theorem that applies to this problem? \qquad \\
The \underline{\answer{Squeeze}} Theorem
\end{problem}}%}

%%%%%%%%%%%%%%%%%%%%%




\latexProblemContent{
\begin{problem}

The limit as $x\to{-15}$ of $f(x)={{\left(x + 15\right)}^{3} \cos\left(-\frac{11}{{\left(x + 15\right)}^{2}}\right)}$ is $0$.  What is the reason why this is true?

\input{2311-Concept-Limit-0008.HELP.tex}

\begin{multipleChoice}
\choice{The statement is in fact false: $\lim\limits_{x\to{-15}}{{\left(x + 15\right)}^{3} \cos\left(-\frac{11}{{\left(x + 15\right)}^{2}}\right)}\neq0$.}
\choice{The cosine factor decreases to $0$ faster than the polynomial.}
\choice[correct]{The cosine factor is bounded between $-1$ and $1$, so the polynomial forces the function to $0$.}
\choice{The cosine factor directly cancels out the polynomial factor.}
\end{multipleChoice}


What is the name of the theorem that applies to this problem? \qquad \\
The \underline{\answer{Squeeze}} Theorem
\end{problem}}%}

%%%%%%%%%%%%%%%%%%%%%




\latexProblemContent{
\begin{problem}

The limit as $x\to{-12}$ of $f(x)={{\left(x + 12\right)} \cos\left(-\frac{17}{{\left(x + 12\right)}^{2}}\right)}$ is $0$.  What is the reason why this is true?

\input{2311-Concept-Limit-0008.HELP.tex}

\begin{multipleChoice}
\choice{The statement is in fact false: $\lim\limits_{x\to{-12}}{{\left(x + 12\right)} \cos\left(-\frac{17}{{\left(x + 12\right)}^{2}}\right)}\neq0$.}
\choice{The cosine factor decreases to $0$ faster than the polynomial.}
\choice[correct]{The cosine factor is bounded between $-1$ and $1$, so the polynomial forces the function to $0$.}
\choice{The cosine factor directly cancels out the polynomial factor.}
\end{multipleChoice}


What is the name of the theorem that applies to this problem? \qquad \\
The \underline{\answer{Squeeze}} Theorem
\end{problem}}%}

%%%%%%%%%%%%%%%%%%%%%




\latexProblemContent{
\begin{problem}

The limit as $x\to{10}$ of $f(x)={{\left(x - 10\right)}^{3} \cos\left(-\frac{7}{x - 10}\right)}$ is $0$.  What is the reason why this is true?

\input{2311-Concept-Limit-0008.HELP.tex}

\begin{multipleChoice}
\choice{The statement is in fact false: $\lim\limits_{x\to{10}}{{\left(x - 10\right)}^{3} \cos\left(-\frac{7}{x - 10}\right)}\neq0$.}
\choice{The cosine factor decreases to $0$ faster than the polynomial.}
\choice[correct]{The cosine factor is bounded between $-1$ and $1$, so the polynomial forces the function to $0$.}
\choice{The cosine factor directly cancels out the polynomial factor.}
\end{multipleChoice}


What is the name of the theorem that applies to this problem? \qquad \\
The \underline{\answer{Squeeze}} Theorem
\end{problem}}%}

%%%%%%%%%%%%%%%%%%%%%




\latexProblemContent{
\begin{problem}

The limit as $x\to{0}$ of $f(x)={x^{3} \cos\left(-\frac{11}{x}\right)}$ is $0$.  What is the reason why this is true?

\input{2311-Concept-Limit-0008.HELP.tex}

\begin{multipleChoice}
\choice{The statement is in fact false: $\lim\limits_{x\to{0}}{x^{3} \cos\left(-\frac{11}{x}\right)}\neq0$.}
\choice{The cosine factor decreases to $0$ faster than the polynomial.}
\choice[correct]{The cosine factor is bounded between $-1$ and $1$, so the polynomial forces the function to $0$.}
\choice{The cosine factor directly cancels out the polynomial factor.}
\end{multipleChoice}


What is the name of the theorem that applies to this problem? \qquad \\
The \underline{\answer{Squeeze}} Theorem
\end{problem}}%}

%%%%%%%%%%%%%%%%%%%%%




\latexProblemContent{
\begin{problem}

The limit as $x\to{3}$ of $f(x)={{\left(x - 3\right)}^{2} \cos\left(-\frac{20}{{\left(x - 3\right)}^{2}}\right)}$ is $0$.  What is the reason why this is true?

\input{2311-Concept-Limit-0008.HELP.tex}

\begin{multipleChoice}
\choice{The statement is in fact false: $\lim\limits_{x\to{3}}{{\left(x - 3\right)}^{2} \cos\left(-\frac{20}{{\left(x - 3\right)}^{2}}\right)}\neq0$.}
\choice{The cosine factor decreases to $0$ faster than the polynomial.}
\choice[correct]{The cosine factor is bounded between $-1$ and $1$, so the polynomial forces the function to $0$.}
\choice{The cosine factor directly cancels out the polynomial factor.}
\end{multipleChoice}


What is the name of the theorem that applies to this problem? \qquad \\
The \underline{\answer{Squeeze}} Theorem
\end{problem}}%}

%%%%%%%%%%%%%%%%%%%%%




\latexProblemContent{
\begin{problem}

The limit as $x\to{9}$ of $f(x)={{\left(x - 9\right)}^{3} \cos\left(\frac{20}{x - 9}\right)}$ is $0$.  What is the reason why this is true?

\input{2311-Concept-Limit-0008.HELP.tex}

\begin{multipleChoice}
\choice{The statement is in fact false: $\lim\limits_{x\to{9}}{{\left(x - 9\right)}^{3} \cos\left(\frac{20}{x - 9}\right)}\neq0$.}
\choice{The cosine factor decreases to $0$ faster than the polynomial.}
\choice[correct]{The cosine factor is bounded between $-1$ and $1$, so the polynomial forces the function to $0$.}
\choice{The cosine factor directly cancels out the polynomial factor.}
\end{multipleChoice}


What is the name of the theorem that applies to this problem? \qquad \\
The \underline{\answer{Squeeze}} Theorem
\end{problem}}%}

%%%%%%%%%%%%%%%%%%%%%




\latexProblemContent{
\begin{problem}

The limit as $x\to{-8}$ of $f(x)={{\left(x + 8\right)}^{3} \cos\left(\frac{24}{x + 8}\right)}$ is $0$.  What is the reason why this is true?

\input{2311-Concept-Limit-0008.HELP.tex}

\begin{multipleChoice}
\choice{The statement is in fact false: $\lim\limits_{x\to{-8}}{{\left(x + 8\right)}^{3} \cos\left(\frac{24}{x + 8}\right)}\neq0$.}
\choice{The cosine factor decreases to $0$ faster than the polynomial.}
\choice[correct]{The cosine factor is bounded between $-1$ and $1$, so the polynomial forces the function to $0$.}
\choice{The cosine factor directly cancels out the polynomial factor.}
\end{multipleChoice}


What is the name of the theorem that applies to this problem? \qquad \\
The \underline{\answer{Squeeze}} Theorem
\end{problem}}%}

%%%%%%%%%%%%%%%%%%%%%




\latexProblemContent{
\begin{problem}

The limit as $x\to{8}$ of $f(x)={{\left(x - 8\right)}^{2} \cos\left(-\frac{19}{{\left(x - 8\right)}^{2}}\right)}$ is $0$.  What is the reason why this is true?

\input{2311-Concept-Limit-0008.HELP.tex}

\begin{multipleChoice}
\choice{The statement is in fact false: $\lim\limits_{x\to{8}}{{\left(x - 8\right)}^{2} \cos\left(-\frac{19}{{\left(x - 8\right)}^{2}}\right)}\neq0$.}
\choice{The cosine factor decreases to $0$ faster than the polynomial.}
\choice[correct]{The cosine factor is bounded between $-1$ and $1$, so the polynomial forces the function to $0$.}
\choice{The cosine factor directly cancels out the polynomial factor.}
\end{multipleChoice}


What is the name of the theorem that applies to this problem? \qquad \\
The \underline{\answer{Squeeze}} Theorem
\end{problem}}%}

%%%%%%%%%%%%%%%%%%%%%




\latexProblemContent{
\begin{problem}

The limit as $x\to{-5}$ of $f(x)={{\left(x + 5\right)}^{2} \cos\left(\frac{14}{{\left(x + 5\right)}^{2}}\right)}$ is $0$.  What is the reason why this is true?

\input{2311-Concept-Limit-0008.HELP.tex}

\begin{multipleChoice}
\choice{The statement is in fact false: $\lim\limits_{x\to{-5}}{{\left(x + 5\right)}^{2} \cos\left(\frac{14}{{\left(x + 5\right)}^{2}}\right)}\neq0$.}
\choice{The cosine factor decreases to $0$ faster than the polynomial.}
\choice[correct]{The cosine factor is bounded between $-1$ and $1$, so the polynomial forces the function to $0$.}
\choice{The cosine factor directly cancels out the polynomial factor.}
\end{multipleChoice}


What is the name of the theorem that applies to this problem? \qquad \\
The \underline{\answer{Squeeze}} Theorem
\end{problem}}%}

%%%%%%%%%%%%%%%%%%%%%




\latexProblemContent{
\begin{problem}

The limit as $x\to{2}$ of $f(x)={{\left(x - 2\right)} \cos\left(-\frac{17}{{\left(x - 2\right)}^{2}}\right)}$ is $0$.  What is the reason why this is true?

\input{2311-Concept-Limit-0008.HELP.tex}

\begin{multipleChoice}
\choice{The statement is in fact false: $\lim\limits_{x\to{2}}{{\left(x - 2\right)} \cos\left(-\frac{17}{{\left(x - 2\right)}^{2}}\right)}\neq0$.}
\choice{The cosine factor decreases to $0$ faster than the polynomial.}
\choice[correct]{The cosine factor is bounded between $-1$ and $1$, so the polynomial forces the function to $0$.}
\choice{The cosine factor directly cancels out the polynomial factor.}
\end{multipleChoice}


What is the name of the theorem that applies to this problem? \qquad \\
The \underline{\answer{Squeeze}} Theorem
\end{problem}}%}

%%%%%%%%%%%%%%%%%%%%%




\latexProblemContent{
\begin{problem}

The limit as $x\to{-1}$ of $f(x)={{\left(x + 1\right)}^{3} \cos\left(\frac{20}{{\left(x + 1\right)}^{2}}\right)}$ is $0$.  What is the reason why this is true?

\input{2311-Concept-Limit-0008.HELP.tex}

\begin{multipleChoice}
\choice{The statement is in fact false: $\lim\limits_{x\to{-1}}{{\left(x + 1\right)}^{3} \cos\left(\frac{20}{{\left(x + 1\right)}^{2}}\right)}\neq0$.}
\choice{The cosine factor decreases to $0$ faster than the polynomial.}
\choice[correct]{The cosine factor is bounded between $-1$ and $1$, so the polynomial forces the function to $0$.}
\choice{The cosine factor directly cancels out the polynomial factor.}
\end{multipleChoice}


What is the name of the theorem that applies to this problem? \qquad \\
The \underline{\answer{Squeeze}} Theorem
\end{problem}}%}

%%%%%%%%%%%%%%%%%%%%%




\latexProblemContent{
\begin{problem}

The limit as $x\to{-1}$ of $f(x)={{\left(x + 1\right)}^{3} \cos\left(\frac{4}{x + 1}\right)}$ is $0$.  What is the reason why this is true?

\input{2311-Concept-Limit-0008.HELP.tex}

\begin{multipleChoice}
\choice{The statement is in fact false: $\lim\limits_{x\to{-1}}{{\left(x + 1\right)}^{3} \cos\left(\frac{4}{x + 1}\right)}\neq0$.}
\choice{The cosine factor decreases to $0$ faster than the polynomial.}
\choice[correct]{The cosine factor is bounded between $-1$ and $1$, so the polynomial forces the function to $0$.}
\choice{The cosine factor directly cancels out the polynomial factor.}
\end{multipleChoice}


What is the name of the theorem that applies to this problem? \qquad \\
The \underline{\answer{Squeeze}} Theorem
\end{problem}}%}

%%%%%%%%%%%%%%%%%%%%%




\latexProblemContent{
\begin{problem}

The limit as $x\to{-6}$ of $f(x)={{\left(x + 6\right)}^{3} \cos\left(\frac{9}{{\left(x + 6\right)}^{2}}\right)}$ is $0$.  What is the reason why this is true?

\input{2311-Concept-Limit-0008.HELP.tex}

\begin{multipleChoice}
\choice{The statement is in fact false: $\lim\limits_{x\to{-6}}{{\left(x + 6\right)}^{3} \cos\left(\frac{9}{{\left(x + 6\right)}^{2}}\right)}\neq0$.}
\choice{The cosine factor decreases to $0$ faster than the polynomial.}
\choice[correct]{The cosine factor is bounded between $-1$ and $1$, so the polynomial forces the function to $0$.}
\choice{The cosine factor directly cancels out the polynomial factor.}
\end{multipleChoice}


What is the name of the theorem that applies to this problem? \qquad \\
The \underline{\answer{Squeeze}} Theorem
\end{problem}}%}

%%%%%%%%%%%%%%%%%%%%%




\latexProblemContent{
\begin{problem}

The limit as $x\to{-5}$ of $f(x)={{\left(x + 5\right)}^{2} \cos\left(\frac{2}{x + 5}\right)}$ is $0$.  What is the reason why this is true?

\input{2311-Concept-Limit-0008.HELP.tex}

\begin{multipleChoice}
\choice{The statement is in fact false: $\lim\limits_{x\to{-5}}{{\left(x + 5\right)}^{2} \cos\left(\frac{2}{x + 5}\right)}\neq0$.}
\choice{The cosine factor decreases to $0$ faster than the polynomial.}
\choice[correct]{The cosine factor is bounded between $-1$ and $1$, so the polynomial forces the function to $0$.}
\choice{The cosine factor directly cancels out the polynomial factor.}
\end{multipleChoice}


What is the name of the theorem that applies to this problem? \qquad \\
The \underline{\answer{Squeeze}} Theorem
\end{problem}}%}

%%%%%%%%%%%%%%%%%%%%%




\latexProblemContent{
\begin{problem}

The limit as $x\to{13}$ of $f(x)={{\left(x - 13\right)}^{2} \cos\left(\frac{4}{{\left(x - 13\right)}^{2}}\right)}$ is $0$.  What is the reason why this is true?

\input{2311-Concept-Limit-0008.HELP.tex}

\begin{multipleChoice}
\choice{The statement is in fact false: $\lim\limits_{x\to{13}}{{\left(x - 13\right)}^{2} \cos\left(\frac{4}{{\left(x - 13\right)}^{2}}\right)}\neq0$.}
\choice{The cosine factor decreases to $0$ faster than the polynomial.}
\choice[correct]{The cosine factor is bounded between $-1$ and $1$, so the polynomial forces the function to $0$.}
\choice{The cosine factor directly cancels out the polynomial factor.}
\end{multipleChoice}


What is the name of the theorem that applies to this problem? \qquad \\
The \underline{\answer{Squeeze}} Theorem
\end{problem}}%}

%%%%%%%%%%%%%%%%%%%%%




\latexProblemContent{
\begin{problem}

The limit as $x\to{6}$ of $f(x)={{\left(x - 6\right)}^{2} \cos\left(\frac{3}{x - 6}\right)}$ is $0$.  What is the reason why this is true?

\input{2311-Concept-Limit-0008.HELP.tex}

\begin{multipleChoice}
\choice{The statement is in fact false: $\lim\limits_{x\to{6}}{{\left(x - 6\right)}^{2} \cos\left(\frac{3}{x - 6}\right)}\neq0$.}
\choice{The cosine factor decreases to $0$ faster than the polynomial.}
\choice[correct]{The cosine factor is bounded between $-1$ and $1$, so the polynomial forces the function to $0$.}
\choice{The cosine factor directly cancels out the polynomial factor.}
\end{multipleChoice}


What is the name of the theorem that applies to this problem? \qquad \\
The \underline{\answer{Squeeze}} Theorem
\end{problem}}%}

%%%%%%%%%%%%%%%%%%%%%




\latexProblemContent{
\begin{problem}

The limit as $x\to{13}$ of $f(x)={{\left(x - 13\right)} \cos\left(\frac{22}{x - 13}\right)}$ is $0$.  What is the reason why this is true?

\input{2311-Concept-Limit-0008.HELP.tex}

\begin{multipleChoice}
\choice{The statement is in fact false: $\lim\limits_{x\to{13}}{{\left(x - 13\right)} \cos\left(\frac{22}{x - 13}\right)}\neq0$.}
\choice{The cosine factor decreases to $0$ faster than the polynomial.}
\choice[correct]{The cosine factor is bounded between $-1$ and $1$, so the polynomial forces the function to $0$.}
\choice{The cosine factor directly cancels out the polynomial factor.}
\end{multipleChoice}


What is the name of the theorem that applies to this problem? \qquad \\
The \underline{\answer{Squeeze}} Theorem
\end{problem}}%}

%%%%%%%%%%%%%%%%%%%%%




%%%%%%%%%%%%%%%%%%%%%




\latexProblemContent{
\begin{problem}

The limit as $x\to{4}$ of $f(x)={{\left(x - 4\right)}^{3} \cos\left(-\frac{22}{{\left(x - 4\right)}^{2}}\right)}$ is $0$.  What is the reason why this is true?

\input{2311-Concept-Limit-0008.HELP.tex}

\begin{multipleChoice}
\choice{The statement is in fact false: $\lim\limits_{x\to{4}}{{\left(x - 4\right)}^{3} \cos\left(-\frac{22}{{\left(x - 4\right)}^{2}}\right)}\neq0$.}
\choice{The cosine factor decreases to $0$ faster than the polynomial.}
\choice[correct]{The cosine factor is bounded between $-1$ and $1$, so the polynomial forces the function to $0$.}
\choice{The cosine factor directly cancels out the polynomial factor.}
\end{multipleChoice}


What is the name of the theorem that applies to this problem? \qquad \\
The \underline{\answer{Squeeze}} Theorem
\end{problem}}%}

%%%%%%%%%%%%%%%%%%%%%




\latexProblemContent{
\begin{problem}

The limit as $x\to{-15}$ of $f(x)={{\left(x + 15\right)}^{2} \cos\left(-\frac{7}{x + 15}\right)}$ is $0$.  What is the reason why this is true?

\input{2311-Concept-Limit-0008.HELP.tex}

\begin{multipleChoice}
\choice{The statement is in fact false: $\lim\limits_{x\to{-15}}{{\left(x + 15\right)}^{2} \cos\left(-\frac{7}{x + 15}\right)}\neq0$.}
\choice{The cosine factor decreases to $0$ faster than the polynomial.}
\choice[correct]{The cosine factor is bounded between $-1$ and $1$, so the polynomial forces the function to $0$.}
\choice{The cosine factor directly cancels out the polynomial factor.}
\end{multipleChoice}


What is the name of the theorem that applies to this problem? \qquad \\
The \underline{\answer{Squeeze}} Theorem
\end{problem}}%}

%%%%%%%%%%%%%%%%%%%%%




\latexProblemContent{
\begin{problem}

The limit as $x\to{-4}$ of $f(x)={{\left(x + 4\right)}^{2} \cos\left(\frac{16}{{\left(x + 4\right)}^{2}}\right)}$ is $0$.  What is the reason why this is true?

\input{2311-Concept-Limit-0008.HELP.tex}

\begin{multipleChoice}
\choice{The statement is in fact false: $\lim\limits_{x\to{-4}}{{\left(x + 4\right)}^{2} \cos\left(\frac{16}{{\left(x + 4\right)}^{2}}\right)}\neq0$.}
\choice{The cosine factor decreases to $0$ faster than the polynomial.}
\choice[correct]{The cosine factor is bounded between $-1$ and $1$, so the polynomial forces the function to $0$.}
\choice{The cosine factor directly cancels out the polynomial factor.}
\end{multipleChoice}


What is the name of the theorem that applies to this problem? \qquad \\
The \underline{\answer{Squeeze}} Theorem
\end{problem}}%}

%%%%%%%%%%%%%%%%%%%%%




\latexProblemContent{
\begin{problem}

The limit as $x\to{2}$ of $f(x)={{\left(x - 2\right)} \cos\left(-\frac{25}{{\left(x - 2\right)}^{2}}\right)}$ is $0$.  What is the reason why this is true?

\input{2311-Concept-Limit-0008.HELP.tex}

\begin{multipleChoice}
\choice{The statement is in fact false: $\lim\limits_{x\to{2}}{{\left(x - 2\right)} \cos\left(-\frac{25}{{\left(x - 2\right)}^{2}}\right)}\neq0$.}
\choice{The cosine factor decreases to $0$ faster than the polynomial.}
\choice[correct]{The cosine factor is bounded between $-1$ and $1$, so the polynomial forces the function to $0$.}
\choice{The cosine factor directly cancels out the polynomial factor.}
\end{multipleChoice}


What is the name of the theorem that applies to this problem? \qquad \\
The \underline{\answer{Squeeze}} Theorem
\end{problem}}%}

%%%%%%%%%%%%%%%%%%%%%




\latexProblemContent{
\begin{problem}

The limit as $x\to{3}$ of $f(x)={{\left(x - 3\right)}^{3} \cos\left(\frac{24}{x - 3}\right)}$ is $0$.  What is the reason why this is true?

\input{2311-Concept-Limit-0008.HELP.tex}

\begin{multipleChoice}
\choice{The statement is in fact false: $\lim\limits_{x\to{3}}{{\left(x - 3\right)}^{3} \cos\left(\frac{24}{x - 3}\right)}\neq0$.}
\choice{The cosine factor decreases to $0$ faster than the polynomial.}
\choice[correct]{The cosine factor is bounded between $-1$ and $1$, so the polynomial forces the function to $0$.}
\choice{The cosine factor directly cancels out the polynomial factor.}
\end{multipleChoice}


What is the name of the theorem that applies to this problem? \qquad \\
The \underline{\answer{Squeeze}} Theorem
\end{problem}}%}

%%%%%%%%%%%%%%%%%%%%%




\latexProblemContent{
\begin{problem}

The limit as $x\to{11}$ of $f(x)={{\left(x - 11\right)}^{3} \cos\left(\frac{4}{x - 11}\right)}$ is $0$.  What is the reason why this is true?

\input{2311-Concept-Limit-0008.HELP.tex}

\begin{multipleChoice}
\choice{The statement is in fact false: $\lim\limits_{x\to{11}}{{\left(x - 11\right)}^{3} \cos\left(\frac{4}{x - 11}\right)}\neq0$.}
\choice{The cosine factor decreases to $0$ faster than the polynomial.}
\choice[correct]{The cosine factor is bounded between $-1$ and $1$, so the polynomial forces the function to $0$.}
\choice{The cosine factor directly cancels out the polynomial factor.}
\end{multipleChoice}


What is the name of the theorem that applies to this problem? \qquad \\
The \underline{\answer{Squeeze}} Theorem
\end{problem}}%}

%%%%%%%%%%%%%%%%%%%%%




\latexProblemContent{
\begin{problem}

The limit as $x\to{13}$ of $f(x)={{\left(x - 13\right)}^{2} \cos\left(\frac{11}{{\left(x - 13\right)}^{2}}\right)}$ is $0$.  What is the reason why this is true?

\input{2311-Concept-Limit-0008.HELP.tex}

\begin{multipleChoice}
\choice{The statement is in fact false: $\lim\limits_{x\to{13}}{{\left(x - 13\right)}^{2} \cos\left(\frac{11}{{\left(x - 13\right)}^{2}}\right)}\neq0$.}
\choice{The cosine factor decreases to $0$ faster than the polynomial.}
\choice[correct]{The cosine factor is bounded between $-1$ and $1$, so the polynomial forces the function to $0$.}
\choice{The cosine factor directly cancels out the polynomial factor.}
\end{multipleChoice}


What is the name of the theorem that applies to this problem? \qquad \\
The \underline{\answer{Squeeze}} Theorem
\end{problem}}%}

%%%%%%%%%%%%%%%%%%%%%




\latexProblemContent{
\begin{problem}

The limit as $x\to{5}$ of $f(x)={{\left(x - 5\right)} \cos\left(-\frac{3}{{\left(x - 5\right)}^{2}}\right)}$ is $0$.  What is the reason why this is true?

\input{2311-Concept-Limit-0008.HELP.tex}

\begin{multipleChoice}
\choice{The statement is in fact false: $\lim\limits_{x\to{5}}{{\left(x - 5\right)} \cos\left(-\frac{3}{{\left(x - 5\right)}^{2}}\right)}\neq0$.}
\choice{The cosine factor decreases to $0$ faster than the polynomial.}
\choice[correct]{The cosine factor is bounded between $-1$ and $1$, so the polynomial forces the function to $0$.}
\choice{The cosine factor directly cancels out the polynomial factor.}
\end{multipleChoice}


What is the name of the theorem that applies to this problem? \qquad \\
The \underline{\answer{Squeeze}} Theorem
\end{problem}}%}

%%%%%%%%%%%%%%%%%%%%%




\latexProblemContent{
\begin{problem}

The limit as $x\to{-15}$ of $f(x)={{\left(x + 15\right)}^{2} \cos\left(\frac{25}{{\left(x + 15\right)}^{2}}\right)}$ is $0$.  What is the reason why this is true?

\input{2311-Concept-Limit-0008.HELP.tex}

\begin{multipleChoice}
\choice{The statement is in fact false: $\lim\limits_{x\to{-15}}{{\left(x + 15\right)}^{2} \cos\left(\frac{25}{{\left(x + 15\right)}^{2}}\right)}\neq0$.}
\choice{The cosine factor decreases to $0$ faster than the polynomial.}
\choice[correct]{The cosine factor is bounded between $-1$ and $1$, so the polynomial forces the function to $0$.}
\choice{The cosine factor directly cancels out the polynomial factor.}
\end{multipleChoice}


What is the name of the theorem that applies to this problem? \qquad \\
The \underline{\answer{Squeeze}} Theorem
\end{problem}}%}

%%%%%%%%%%%%%%%%%%%%%




\latexProblemContent{
\begin{problem}

The limit as $x\to{-6}$ of $f(x)={{\left(x + 6\right)}^{2} \cos\left(\frac{25}{{\left(x + 6\right)}^{2}}\right)}$ is $0$.  What is the reason why this is true?

\input{2311-Concept-Limit-0008.HELP.tex}

\begin{multipleChoice}
\choice{The statement is in fact false: $\lim\limits_{x\to{-6}}{{\left(x + 6\right)}^{2} \cos\left(\frac{25}{{\left(x + 6\right)}^{2}}\right)}\neq0$.}
\choice{The cosine factor decreases to $0$ faster than the polynomial.}
\choice[correct]{The cosine factor is bounded between $-1$ and $1$, so the polynomial forces the function to $0$.}
\choice{The cosine factor directly cancels out the polynomial factor.}
\end{multipleChoice}


What is the name of the theorem that applies to this problem? \qquad \\
The \underline{\answer{Squeeze}} Theorem
\end{problem}}%}

%%%%%%%%%%%%%%%%%%%%%




\latexProblemContent{
\begin{problem}

The limit as $x\to{-5}$ of $f(x)={{\left(x + 5\right)}^{2} \cos\left(-\frac{18}{x + 5}\right)}$ is $0$.  What is the reason why this is true?

\input{2311-Concept-Limit-0008.HELP.tex}

\begin{multipleChoice}
\choice{The statement is in fact false: $\lim\limits_{x\to{-5}}{{\left(x + 5\right)}^{2} \cos\left(-\frac{18}{x + 5}\right)}\neq0$.}
\choice{The cosine factor decreases to $0$ faster than the polynomial.}
\choice[correct]{The cosine factor is bounded between $-1$ and $1$, so the polynomial forces the function to $0$.}
\choice{The cosine factor directly cancels out the polynomial factor.}
\end{multipleChoice}


What is the name of the theorem that applies to this problem? \qquad \\
The \underline{\answer{Squeeze}} Theorem
\end{problem}}%}

%%%%%%%%%%%%%%%%%%%%%




\latexProblemContent{
\begin{problem}

The limit as $x\to{-15}$ of $f(x)={{\left(x + 15\right)} \cos\left(\frac{23}{{\left(x + 15\right)}^{2}}\right)}$ is $0$.  What is the reason why this is true?

\input{2311-Concept-Limit-0008.HELP.tex}

\begin{multipleChoice}
\choice{The statement is in fact false: $\lim\limits_{x\to{-15}}{{\left(x + 15\right)} \cos\left(\frac{23}{{\left(x + 15\right)}^{2}}\right)}\neq0$.}
\choice{The cosine factor decreases to $0$ faster than the polynomial.}
\choice[correct]{The cosine factor is bounded between $-1$ and $1$, so the polynomial forces the function to $0$.}
\choice{The cosine factor directly cancels out the polynomial factor.}
\end{multipleChoice}


What is the name of the theorem that applies to this problem? \qquad \\
The \underline{\answer{Squeeze}} Theorem
\end{problem}}%}

%%%%%%%%%%%%%%%%%%%%%




\latexProblemContent{
\begin{problem}

The limit as $x\to{-13}$ of $f(x)={{\left(x + 13\right)}^{3} \cos\left(-\frac{14}{x + 13}\right)}$ is $0$.  What is the reason why this is true?

\input{2311-Concept-Limit-0008.HELP.tex}

\begin{multipleChoice}
\choice{The statement is in fact false: $\lim\limits_{x\to{-13}}{{\left(x + 13\right)}^{3} \cos\left(-\frac{14}{x + 13}\right)}\neq0$.}
\choice{The cosine factor decreases to $0$ faster than the polynomial.}
\choice[correct]{The cosine factor is bounded between $-1$ and $1$, so the polynomial forces the function to $0$.}
\choice{The cosine factor directly cancels out the polynomial factor.}
\end{multipleChoice}


What is the name of the theorem that applies to this problem? \qquad \\
The \underline{\answer{Squeeze}} Theorem
\end{problem}}%}

%%%%%%%%%%%%%%%%%%%%%




%%%%%%%%%%%%%%%%%%%%%




\latexProblemContent{
\begin{problem}

The limit as $x\to{-5}$ of $f(x)={{\left(x + 5\right)} \cos\left(\frac{18}{x + 5}\right)}$ is $0$.  What is the reason why this is true?

\input{2311-Concept-Limit-0008.HELP.tex}

\begin{multipleChoice}
\choice{The statement is in fact false: $\lim\limits_{x\to{-5}}{{\left(x + 5\right)} \cos\left(\frac{18}{x + 5}\right)}\neq0$.}
\choice{The cosine factor decreases to $0$ faster than the polynomial.}
\choice[correct]{The cosine factor is bounded between $-1$ and $1$, so the polynomial forces the function to $0$.}
\choice{The cosine factor directly cancels out the polynomial factor.}
\end{multipleChoice}


What is the name of the theorem that applies to this problem? \qquad \\
The \underline{\answer{Squeeze}} Theorem
\end{problem}}%}

%%%%%%%%%%%%%%%%%%%%%




\latexProblemContent{
\begin{problem}

The limit as $x\to{-5}$ of $f(x)={{\left(x + 5\right)}^{3} \cos\left(\frac{22}{x + 5}\right)}$ is $0$.  What is the reason why this is true?

\input{2311-Concept-Limit-0008.HELP.tex}

\begin{multipleChoice}
\choice{The statement is in fact false: $\lim\limits_{x\to{-5}}{{\left(x + 5\right)}^{3} \cos\left(\frac{22}{x + 5}\right)}\neq0$.}
\choice{The cosine factor decreases to $0$ faster than the polynomial.}
\choice[correct]{The cosine factor is bounded between $-1$ and $1$, so the polynomial forces the function to $0$.}
\choice{The cosine factor directly cancels out the polynomial factor.}
\end{multipleChoice}


What is the name of the theorem that applies to this problem? \qquad \\
The \underline{\answer{Squeeze}} Theorem
\end{problem}}%}

%%%%%%%%%%%%%%%%%%%%%




\latexProblemContent{
\begin{problem}

The limit as $x\to{9}$ of $f(x)={{\left(x - 9\right)} \cos\left(-\frac{21}{{\left(x - 9\right)}^{2}}\right)}$ is $0$.  What is the reason why this is true?

\input{2311-Concept-Limit-0008.HELP.tex}

\begin{multipleChoice}
\choice{The statement is in fact false: $\lim\limits_{x\to{9}}{{\left(x - 9\right)} \cos\left(-\frac{21}{{\left(x - 9\right)}^{2}}\right)}\neq0$.}
\choice{The cosine factor decreases to $0$ faster than the polynomial.}
\choice[correct]{The cosine factor is bounded between $-1$ and $1$, so the polynomial forces the function to $0$.}
\choice{The cosine factor directly cancels out the polynomial factor.}
\end{multipleChoice}


What is the name of the theorem that applies to this problem? \qquad \\
The \underline{\answer{Squeeze}} Theorem
\end{problem}}%}

%%%%%%%%%%%%%%%%%%%%%




\latexProblemContent{
\begin{problem}

The limit as $x\to{3}$ of $f(x)={{\left(x - 3\right)}^{2} \cos\left(-\frac{11}{{\left(x - 3\right)}^{2}}\right)}$ is $0$.  What is the reason why this is true?

\input{2311-Concept-Limit-0008.HELP.tex}

\begin{multipleChoice}
\choice{The statement is in fact false: $\lim\limits_{x\to{3}}{{\left(x - 3\right)}^{2} \cos\left(-\frac{11}{{\left(x - 3\right)}^{2}}\right)}\neq0$.}
\choice{The cosine factor decreases to $0$ faster than the polynomial.}
\choice[correct]{The cosine factor is bounded between $-1$ and $1$, so the polynomial forces the function to $0$.}
\choice{The cosine factor directly cancels out the polynomial factor.}
\end{multipleChoice}


What is the name of the theorem that applies to this problem? \qquad \\
The \underline{\answer{Squeeze}} Theorem
\end{problem}}%}

%%%%%%%%%%%%%%%%%%%%%




\latexProblemContent{
\begin{problem}

The limit as $x\to{4}$ of $f(x)={{\left(x - 4\right)}^{3} \cos\left(\frac{24}{{\left(x - 4\right)}^{2}}\right)}$ is $0$.  What is the reason why this is true?

\input{2311-Concept-Limit-0008.HELP.tex}

\begin{multipleChoice}
\choice{The statement is in fact false: $\lim\limits_{x\to{4}}{{\left(x - 4\right)}^{3} \cos\left(\frac{24}{{\left(x - 4\right)}^{2}}\right)}\neq0$.}
\choice{The cosine factor decreases to $0$ faster than the polynomial.}
\choice[correct]{The cosine factor is bounded between $-1$ and $1$, so the polynomial forces the function to $0$.}
\choice{The cosine factor directly cancels out the polynomial factor.}
\end{multipleChoice}


What is the name of the theorem that applies to this problem? \qquad \\
The \underline{\answer{Squeeze}} Theorem
\end{problem}}%}

%%%%%%%%%%%%%%%%%%%%%




\latexProblemContent{
\begin{problem}

The limit as $x\to{-6}$ of $f(x)={{\left(x + 6\right)}^{3} \cos\left(-\frac{13}{x + 6}\right)}$ is $0$.  What is the reason why this is true?

\input{2311-Concept-Limit-0008.HELP.tex}

\begin{multipleChoice}
\choice{The statement is in fact false: $\lim\limits_{x\to{-6}}{{\left(x + 6\right)}^{3} \cos\left(-\frac{13}{x + 6}\right)}\neq0$.}
\choice{The cosine factor decreases to $0$ faster than the polynomial.}
\choice[correct]{The cosine factor is bounded between $-1$ and $1$, so the polynomial forces the function to $0$.}
\choice{The cosine factor directly cancels out the polynomial factor.}
\end{multipleChoice}


What is the name of the theorem that applies to this problem? \qquad \\
The \underline{\answer{Squeeze}} Theorem
\end{problem}}%}

%%%%%%%%%%%%%%%%%%%%%




\latexProblemContent{
\begin{problem}

The limit as $x\to{-9}$ of $f(x)={{\left(x + 9\right)}^{2} \cos\left(\frac{14}{x + 9}\right)}$ is $0$.  What is the reason why this is true?

\input{2311-Concept-Limit-0008.HELP.tex}

\begin{multipleChoice}
\choice{The statement is in fact false: $\lim\limits_{x\to{-9}}{{\left(x + 9\right)}^{2} \cos\left(\frac{14}{x + 9}\right)}\neq0$.}
\choice{The cosine factor decreases to $0$ faster than the polynomial.}
\choice[correct]{The cosine factor is bounded between $-1$ and $1$, so the polynomial forces the function to $0$.}
\choice{The cosine factor directly cancels out the polynomial factor.}
\end{multipleChoice}


What is the name of the theorem that applies to this problem? \qquad \\
The \underline{\answer{Squeeze}} Theorem
\end{problem}}%}

%%%%%%%%%%%%%%%%%%%%%




\latexProblemContent{
\begin{problem}

The limit as $x\to{-13}$ of $f(x)={{\left(x + 13\right)}^{3} \cos\left(-\frac{25}{{\left(x + 13\right)}^{2}}\right)}$ is $0$.  What is the reason why this is true?

\input{2311-Concept-Limit-0008.HELP.tex}

\begin{multipleChoice}
\choice{The statement is in fact false: $\lim\limits_{x\to{-13}}{{\left(x + 13\right)}^{3} \cos\left(-\frac{25}{{\left(x + 13\right)}^{2}}\right)}\neq0$.}
\choice{The cosine factor decreases to $0$ faster than the polynomial.}
\choice[correct]{The cosine factor is bounded between $-1$ and $1$, so the polynomial forces the function to $0$.}
\choice{The cosine factor directly cancels out the polynomial factor.}
\end{multipleChoice}


What is the name of the theorem that applies to this problem? \qquad \\
The \underline{\answer{Squeeze}} Theorem
\end{problem}}%}

%%%%%%%%%%%%%%%%%%%%%




\latexProblemContent{
\begin{problem}

The limit as $x\to{-9}$ of $f(x)={{\left(x + 9\right)}^{2} \cos\left(\frac{8}{x + 9}\right)}$ is $0$.  What is the reason why this is true?

\input{2311-Concept-Limit-0008.HELP.tex}

\begin{multipleChoice}
\choice{The statement is in fact false: $\lim\limits_{x\to{-9}}{{\left(x + 9\right)}^{2} \cos\left(\frac{8}{x + 9}\right)}\neq0$.}
\choice{The cosine factor decreases to $0$ faster than the polynomial.}
\choice[correct]{The cosine factor is bounded between $-1$ and $1$, so the polynomial forces the function to $0$.}
\choice{The cosine factor directly cancels out the polynomial factor.}
\end{multipleChoice}


What is the name of the theorem that applies to this problem? \qquad \\
The \underline{\answer{Squeeze}} Theorem
\end{problem}}%}

%%%%%%%%%%%%%%%%%%%%%




\latexProblemContent{
\begin{problem}

The limit as $x\to{8}$ of $f(x)={{\left(x - 8\right)} \cos\left(-\frac{1}{{\left(x - 8\right)}^{2}}\right)}$ is $0$.  What is the reason why this is true?

\input{2311-Concept-Limit-0008.HELP.tex}

\begin{multipleChoice}
\choice{The statement is in fact false: $\lim\limits_{x\to{8}}{{\left(x - 8\right)} \cos\left(-\frac{1}{{\left(x - 8\right)}^{2}}\right)}\neq0$.}
\choice{The cosine factor decreases to $0$ faster than the polynomial.}
\choice[correct]{The cosine factor is bounded between $-1$ and $1$, so the polynomial forces the function to $0$.}
\choice{The cosine factor directly cancels out the polynomial factor.}
\end{multipleChoice}


What is the name of the theorem that applies to this problem? \qquad \\
The \underline{\answer{Squeeze}} Theorem
\end{problem}}%}

%%%%%%%%%%%%%%%%%%%%%




\latexProblemContent{
\begin{problem}

The limit as $x\to{-13}$ of $f(x)={{\left(x + 13\right)} \cos\left(\frac{21}{x + 13}\right)}$ is $0$.  What is the reason why this is true?

\input{2311-Concept-Limit-0008.HELP.tex}

\begin{multipleChoice}
\choice{The statement is in fact false: $\lim\limits_{x\to{-13}}{{\left(x + 13\right)} \cos\left(\frac{21}{x + 13}\right)}\neq0$.}
\choice{The cosine factor decreases to $0$ faster than the polynomial.}
\choice[correct]{The cosine factor is bounded between $-1$ and $1$, so the polynomial forces the function to $0$.}
\choice{The cosine factor directly cancels out the polynomial factor.}
\end{multipleChoice}


What is the name of the theorem that applies to this problem? \qquad \\
The \underline{\answer{Squeeze}} Theorem
\end{problem}}%}

%%%%%%%%%%%%%%%%%%%%%




\latexProblemContent{
\begin{problem}

The limit as $x\to{8}$ of $f(x)={{\left(x - 8\right)}^{3} \cos\left(-\frac{8}{{\left(x - 8\right)}^{2}}\right)}$ is $0$.  What is the reason why this is true?

\input{2311-Concept-Limit-0008.HELP.tex}

\begin{multipleChoice}
\choice{The statement is in fact false: $\lim\limits_{x\to{8}}{{\left(x - 8\right)}^{3} \cos\left(-\frac{8}{{\left(x - 8\right)}^{2}}\right)}\neq0$.}
\choice{The cosine factor decreases to $0$ faster than the polynomial.}
\choice[correct]{The cosine factor is bounded between $-1$ and $1$, so the polynomial forces the function to $0$.}
\choice{The cosine factor directly cancels out the polynomial factor.}
\end{multipleChoice}


What is the name of the theorem that applies to this problem? \qquad \\
The \underline{\answer{Squeeze}} Theorem
\end{problem}}%}

%%%%%%%%%%%%%%%%%%%%%




\latexProblemContent{
\begin{problem}

The limit as $x\to{-13}$ of $f(x)={{\left(x + 13\right)} \cos\left(-\frac{2}{{\left(x + 13\right)}^{2}}\right)}$ is $0$.  What is the reason why this is true?

\input{2311-Concept-Limit-0008.HELP.tex}

\begin{multipleChoice}
\choice{The statement is in fact false: $\lim\limits_{x\to{-13}}{{\left(x + 13\right)} \cos\left(-\frac{2}{{\left(x + 13\right)}^{2}}\right)}\neq0$.}
\choice{The cosine factor decreases to $0$ faster than the polynomial.}
\choice[correct]{The cosine factor is bounded between $-1$ and $1$, so the polynomial forces the function to $0$.}
\choice{The cosine factor directly cancels out the polynomial factor.}
\end{multipleChoice}


What is the name of the theorem that applies to this problem? \qquad \\
The \underline{\answer{Squeeze}} Theorem
\end{problem}}%}

%%%%%%%%%%%%%%%%%%%%%




\latexProblemContent{
\begin{problem}

The limit as $x\to{10}$ of $f(x)={{\left(x - 10\right)}^{3} \cos\left(-\frac{3}{x - 10}\right)}$ is $0$.  What is the reason why this is true?

\input{2311-Concept-Limit-0008.HELP.tex}

\begin{multipleChoice}
\choice{The statement is in fact false: $\lim\limits_{x\to{10}}{{\left(x - 10\right)}^{3} \cos\left(-\frac{3}{x - 10}\right)}\neq0$.}
\choice{The cosine factor decreases to $0$ faster than the polynomial.}
\choice[correct]{The cosine factor is bounded between $-1$ and $1$, so the polynomial forces the function to $0$.}
\choice{The cosine factor directly cancels out the polynomial factor.}
\end{multipleChoice}


What is the name of the theorem that applies to this problem? \qquad \\
The \underline{\answer{Squeeze}} Theorem
\end{problem}}%}

%%%%%%%%%%%%%%%%%%%%%




\latexProblemContent{
\begin{problem}

The limit as $x\to{-6}$ of $f(x)={{\left(x + 6\right)} \cos\left(\frac{9}{x + 6}\right)}$ is $0$.  What is the reason why this is true?

\input{2311-Concept-Limit-0008.HELP.tex}

\begin{multipleChoice}
\choice{The statement is in fact false: $\lim\limits_{x\to{-6}}{{\left(x + 6\right)} \cos\left(\frac{9}{x + 6}\right)}\neq0$.}
\choice{The cosine factor decreases to $0$ faster than the polynomial.}
\choice[correct]{The cosine factor is bounded between $-1$ and $1$, so the polynomial forces the function to $0$.}
\choice{The cosine factor directly cancels out the polynomial factor.}
\end{multipleChoice}


What is the name of the theorem that applies to this problem? \qquad \\
The \underline{\answer{Squeeze}} Theorem
\end{problem}}%}

%%%%%%%%%%%%%%%%%%%%%




\latexProblemContent{
\begin{problem}

The limit as $x\to{-10}$ of $f(x)={{\left(x + 10\right)}^{3} \cos\left(\frac{24}{{\left(x + 10\right)}^{2}}\right)}$ is $0$.  What is the reason why this is true?

\input{2311-Concept-Limit-0008.HELP.tex}

\begin{multipleChoice}
\choice{The statement is in fact false: $\lim\limits_{x\to{-10}}{{\left(x + 10\right)}^{3} \cos\left(\frac{24}{{\left(x + 10\right)}^{2}}\right)}\neq0$.}
\choice{The cosine factor decreases to $0$ faster than the polynomial.}
\choice[correct]{The cosine factor is bounded between $-1$ and $1$, so the polynomial forces the function to $0$.}
\choice{The cosine factor directly cancels out the polynomial factor.}
\end{multipleChoice}


What is the name of the theorem that applies to this problem? \qquad \\
The \underline{\answer{Squeeze}} Theorem
\end{problem}}%}

%%%%%%%%%%%%%%%%%%%%%




\latexProblemContent{
\begin{problem}

The limit as $x\to{1}$ of $f(x)={{\left(x - 1\right)}^{3} \cos\left(\frac{13}{x - 1}\right)}$ is $0$.  What is the reason why this is true?

\input{2311-Concept-Limit-0008.HELP.tex}

\begin{multipleChoice}
\choice{The statement is in fact false: $\lim\limits_{x\to{1}}{{\left(x - 1\right)}^{3} \cos\left(\frac{13}{x - 1}\right)}\neq0$.}
\choice{The cosine factor decreases to $0$ faster than the polynomial.}
\choice[correct]{The cosine factor is bounded between $-1$ and $1$, so the polynomial forces the function to $0$.}
\choice{The cosine factor directly cancels out the polynomial factor.}
\end{multipleChoice}


What is the name of the theorem that applies to this problem? \qquad \\
The \underline{\answer{Squeeze}} Theorem
\end{problem}}%}

%%%%%%%%%%%%%%%%%%%%%




\latexProblemContent{
\begin{problem}

The limit as $x\to{10}$ of $f(x)={{\left(x - 10\right)}^{2} \cos\left(-\frac{2}{{\left(x - 10\right)}^{2}}\right)}$ is $0$.  What is the reason why this is true?

\input{2311-Concept-Limit-0008.HELP.tex}

\begin{multipleChoice}
\choice{The statement is in fact false: $\lim\limits_{x\to{10}}{{\left(x - 10\right)}^{2} \cos\left(-\frac{2}{{\left(x - 10\right)}^{2}}\right)}\neq0$.}
\choice{The cosine factor decreases to $0$ faster than the polynomial.}
\choice[correct]{The cosine factor is bounded between $-1$ and $1$, so the polynomial forces the function to $0$.}
\choice{The cosine factor directly cancels out the polynomial factor.}
\end{multipleChoice}


What is the name of the theorem that applies to this problem? \qquad \\
The \underline{\answer{Squeeze}} Theorem
\end{problem}}%}

%%%%%%%%%%%%%%%%%%%%%




\latexProblemContent{
\begin{problem}

The limit as $x\to{5}$ of $f(x)={{\left(x - 5\right)}^{3} \cos\left(\frac{6}{{\left(x - 5\right)}^{2}}\right)}$ is $0$.  What is the reason why this is true?

\input{2311-Concept-Limit-0008.HELP.tex}

\begin{multipleChoice}
\choice{The statement is in fact false: $\lim\limits_{x\to{5}}{{\left(x - 5\right)}^{3} \cos\left(\frac{6}{{\left(x - 5\right)}^{2}}\right)}\neq0$.}
\choice{The cosine factor decreases to $0$ faster than the polynomial.}
\choice[correct]{The cosine factor is bounded between $-1$ and $1$, so the polynomial forces the function to $0$.}
\choice{The cosine factor directly cancels out the polynomial factor.}
\end{multipleChoice}


What is the name of the theorem that applies to this problem? \qquad \\
The \underline{\answer{Squeeze}} Theorem
\end{problem}}%}

%%%%%%%%%%%%%%%%%%%%%




\latexProblemContent{
\begin{problem}

The limit as $x\to{6}$ of $f(x)={{\left(x - 6\right)} \cos\left(-\frac{1}{x - 6}\right)}$ is $0$.  What is the reason why this is true?

\input{2311-Concept-Limit-0008.HELP.tex}

\begin{multipleChoice}
\choice{The statement is in fact false: $\lim\limits_{x\to{6}}{{\left(x - 6\right)} \cos\left(-\frac{1}{x - 6}\right)}\neq0$.}
\choice{The cosine factor decreases to $0$ faster than the polynomial.}
\choice[correct]{The cosine factor is bounded between $-1$ and $1$, so the polynomial forces the function to $0$.}
\choice{The cosine factor directly cancels out the polynomial factor.}
\end{multipleChoice}


What is the name of the theorem that applies to this problem? \qquad \\
The \underline{\answer{Squeeze}} Theorem
\end{problem}}%}

%%%%%%%%%%%%%%%%%%%%%




\latexProblemContent{
\begin{problem}

The limit as $x\to{14}$ of $f(x)={{\left(x - 14\right)}^{2} \cos\left(-\frac{23}{x - 14}\right)}$ is $0$.  What is the reason why this is true?

\input{2311-Concept-Limit-0008.HELP.tex}

\begin{multipleChoice}
\choice{The statement is in fact false: $\lim\limits_{x\to{14}}{{\left(x - 14\right)}^{2} \cos\left(-\frac{23}{x - 14}\right)}\neq0$.}
\choice{The cosine factor decreases to $0$ faster than the polynomial.}
\choice[correct]{The cosine factor is bounded between $-1$ and $1$, so the polynomial forces the function to $0$.}
\choice{The cosine factor directly cancels out the polynomial factor.}
\end{multipleChoice}


What is the name of the theorem that applies to this problem? \qquad \\
The \underline{\answer{Squeeze}} Theorem
\end{problem}}%}

%%%%%%%%%%%%%%%%%%%%%




\latexProblemContent{
\begin{problem}

The limit as $x\to{-10}$ of $f(x)={{\left(x + 10\right)}^{2} \cos\left(\frac{4}{x + 10}\right)}$ is $0$.  What is the reason why this is true?

\input{2311-Concept-Limit-0008.HELP.tex}

\begin{multipleChoice}
\choice{The statement is in fact false: $\lim\limits_{x\to{-10}}{{\left(x + 10\right)}^{2} \cos\left(\frac{4}{x + 10}\right)}\neq0$.}
\choice{The cosine factor decreases to $0$ faster than the polynomial.}
\choice[correct]{The cosine factor is bounded between $-1$ and $1$, so the polynomial forces the function to $0$.}
\choice{The cosine factor directly cancels out the polynomial factor.}
\end{multipleChoice}


What is the name of the theorem that applies to this problem? \qquad \\
The \underline{\answer{Squeeze}} Theorem
\end{problem}}%}

%%%%%%%%%%%%%%%%%%%%%




\latexProblemContent{
\begin{problem}

The limit as $x\to{14}$ of $f(x)={{\left(x - 14\right)}^{3} \cos\left(-\frac{12}{{\left(x - 14\right)}^{2}}\right)}$ is $0$.  What is the reason why this is true?

\input{2311-Concept-Limit-0008.HELP.tex}

\begin{multipleChoice}
\choice{The statement is in fact false: $\lim\limits_{x\to{14}}{{\left(x - 14\right)}^{3} \cos\left(-\frac{12}{{\left(x - 14\right)}^{2}}\right)}\neq0$.}
\choice{The cosine factor decreases to $0$ faster than the polynomial.}
\choice[correct]{The cosine factor is bounded between $-1$ and $1$, so the polynomial forces the function to $0$.}
\choice{The cosine factor directly cancels out the polynomial factor.}
\end{multipleChoice}


What is the name of the theorem that applies to this problem? \qquad \\
The \underline{\answer{Squeeze}} Theorem
\end{problem}}%}

%%%%%%%%%%%%%%%%%%%%%




\latexProblemContent{
\begin{problem}

The limit as $x\to{-7}$ of $f(x)={{\left(x + 7\right)}^{2} \cos\left(-\frac{2}{{\left(x + 7\right)}^{2}}\right)}$ is $0$.  What is the reason why this is true?

\input{2311-Concept-Limit-0008.HELP.tex}

\begin{multipleChoice}
\choice{The statement is in fact false: $\lim\limits_{x\to{-7}}{{\left(x + 7\right)}^{2} \cos\left(-\frac{2}{{\left(x + 7\right)}^{2}}\right)}\neq0$.}
\choice{The cosine factor decreases to $0$ faster than the polynomial.}
\choice[correct]{The cosine factor is bounded between $-1$ and $1$, so the polynomial forces the function to $0$.}
\choice{The cosine factor directly cancels out the polynomial factor.}
\end{multipleChoice}


What is the name of the theorem that applies to this problem? \qquad \\
The \underline{\answer{Squeeze}} Theorem
\end{problem}}%}

%%%%%%%%%%%%%%%%%%%%%




\latexProblemContent{
\begin{problem}

The limit as $x\to{-11}$ of $f(x)={{\left(x + 11\right)} \cos\left(-\frac{23}{x + 11}\right)}$ is $0$.  What is the reason why this is true?

\input{2311-Concept-Limit-0008.HELP.tex}

\begin{multipleChoice}
\choice{The statement is in fact false: $\lim\limits_{x\to{-11}}{{\left(x + 11\right)} \cos\left(-\frac{23}{x + 11}\right)}\neq0$.}
\choice{The cosine factor decreases to $0$ faster than the polynomial.}
\choice[correct]{The cosine factor is bounded between $-1$ and $1$, so the polynomial forces the function to $0$.}
\choice{The cosine factor directly cancels out the polynomial factor.}
\end{multipleChoice}


What is the name of the theorem that applies to this problem? \qquad \\
The \underline{\answer{Squeeze}} Theorem
\end{problem}}%}

%%%%%%%%%%%%%%%%%%%%%




\latexProblemContent{
\begin{problem}

The limit as $x\to{4}$ of $f(x)={{\left(x - 4\right)}^{3} \cos\left(\frac{23}{x - 4}\right)}$ is $0$.  What is the reason why this is true?

\input{2311-Concept-Limit-0008.HELP.tex}

\begin{multipleChoice}
\choice{The statement is in fact false: $\lim\limits_{x\to{4}}{{\left(x - 4\right)}^{3} \cos\left(\frac{23}{x - 4}\right)}\neq0$.}
\choice{The cosine factor decreases to $0$ faster than the polynomial.}
\choice[correct]{The cosine factor is bounded between $-1$ and $1$, so the polynomial forces the function to $0$.}
\choice{The cosine factor directly cancels out the polynomial factor.}
\end{multipleChoice}


What is the name of the theorem that applies to this problem? \qquad \\
The \underline{\answer{Squeeze}} Theorem
\end{problem}}%}

%%%%%%%%%%%%%%%%%%%%%




\latexProblemContent{
\begin{problem}

The limit as $x\to{10}$ of $f(x)={{\left(x - 10\right)}^{2} \cos\left(\frac{16}{x - 10}\right)}$ is $0$.  What is the reason why this is true?

\input{2311-Concept-Limit-0008.HELP.tex}

\begin{multipleChoice}
\choice{The statement is in fact false: $\lim\limits_{x\to{10}}{{\left(x - 10\right)}^{2} \cos\left(\frac{16}{x - 10}\right)}\neq0$.}
\choice{The cosine factor decreases to $0$ faster than the polynomial.}
\choice[correct]{The cosine factor is bounded between $-1$ and $1$, so the polynomial forces the function to $0$.}
\choice{The cosine factor directly cancels out the polynomial factor.}
\end{multipleChoice}


What is the name of the theorem that applies to this problem? \qquad \\
The \underline{\answer{Squeeze}} Theorem
\end{problem}}%}

%%%%%%%%%%%%%%%%%%%%%




\latexProblemContent{
\begin{problem}

The limit as $x\to{1}$ of $f(x)={{\left(x - 1\right)} \cos\left(-\frac{17}{x - 1}\right)}$ is $0$.  What is the reason why this is true?

\input{2311-Concept-Limit-0008.HELP.tex}

\begin{multipleChoice}
\choice{The statement is in fact false: $\lim\limits_{x\to{1}}{{\left(x - 1\right)} \cos\left(-\frac{17}{x - 1}\right)}\neq0$.}
\choice{The cosine factor decreases to $0$ faster than the polynomial.}
\choice[correct]{The cosine factor is bounded between $-1$ and $1$, so the polynomial forces the function to $0$.}
\choice{The cosine factor directly cancels out the polynomial factor.}
\end{multipleChoice}


What is the name of the theorem that applies to this problem? \qquad \\
The \underline{\answer{Squeeze}} Theorem
\end{problem}}%}

%%%%%%%%%%%%%%%%%%%%%




\latexProblemContent{
\begin{problem}

The limit as $x\to{-3}$ of $f(x)={{\left(x + 3\right)} \cos\left(\frac{25}{{\left(x + 3\right)}^{2}}\right)}$ is $0$.  What is the reason why this is true?

\input{2311-Concept-Limit-0008.HELP.tex}

\begin{multipleChoice}
\choice{The statement is in fact false: $\lim\limits_{x\to{-3}}{{\left(x + 3\right)} \cos\left(\frac{25}{{\left(x + 3\right)}^{2}}\right)}\neq0$.}
\choice{The cosine factor decreases to $0$ faster than the polynomial.}
\choice[correct]{The cosine factor is bounded between $-1$ and $1$, so the polynomial forces the function to $0$.}
\choice{The cosine factor directly cancels out the polynomial factor.}
\end{multipleChoice}


What is the name of the theorem that applies to this problem? \qquad \\
The \underline{\answer{Squeeze}} Theorem
\end{problem}}%}

%%%%%%%%%%%%%%%%%%%%%




\latexProblemContent{
\begin{problem}

The limit as $x\to{-11}$ of $f(x)={{\left(x + 11\right)}^{2} \cos\left(\frac{16}{x + 11}\right)}$ is $0$.  What is the reason why this is true?

\input{2311-Concept-Limit-0008.HELP.tex}

\begin{multipleChoice}
\choice{The statement is in fact false: $\lim\limits_{x\to{-11}}{{\left(x + 11\right)}^{2} \cos\left(\frac{16}{x + 11}\right)}\neq0$.}
\choice{The cosine factor decreases to $0$ faster than the polynomial.}
\choice[correct]{The cosine factor is bounded between $-1$ and $1$, so the polynomial forces the function to $0$.}
\choice{The cosine factor directly cancels out the polynomial factor.}
\end{multipleChoice}


What is the name of the theorem that applies to this problem? \qquad \\
The \underline{\answer{Squeeze}} Theorem
\end{problem}}%}

%%%%%%%%%%%%%%%%%%%%%




\latexProblemContent{
\begin{problem}

The limit as $x\to{2}$ of $f(x)={{\left(x - 2\right)}^{2} \cos\left(\frac{18}{x - 2}\right)}$ is $0$.  What is the reason why this is true?

\input{2311-Concept-Limit-0008.HELP.tex}

\begin{multipleChoice}
\choice{The statement is in fact false: $\lim\limits_{x\to{2}}{{\left(x - 2\right)}^{2} \cos\left(\frac{18}{x - 2}\right)}\neq0$.}
\choice{The cosine factor decreases to $0$ faster than the polynomial.}
\choice[correct]{The cosine factor is bounded between $-1$ and $1$, so the polynomial forces the function to $0$.}
\choice{The cosine factor directly cancels out the polynomial factor.}
\end{multipleChoice}


What is the name of the theorem that applies to this problem? \qquad \\
The \underline{\answer{Squeeze}} Theorem
\end{problem}}%}

%%%%%%%%%%%%%%%%%%%%%




\latexProblemContent{
\begin{problem}

The limit as $x\to{-14}$ of $f(x)={{\left(x + 14\right)} \cos\left(-\frac{5}{{\left(x + 14\right)}^{2}}\right)}$ is $0$.  What is the reason why this is true?

\input{2311-Concept-Limit-0008.HELP.tex}

\begin{multipleChoice}
\choice{The statement is in fact false: $\lim\limits_{x\to{-14}}{{\left(x + 14\right)} \cos\left(-\frac{5}{{\left(x + 14\right)}^{2}}\right)}\neq0$.}
\choice{The cosine factor decreases to $0$ faster than the polynomial.}
\choice[correct]{The cosine factor is bounded between $-1$ and $1$, so the polynomial forces the function to $0$.}
\choice{The cosine factor directly cancels out the polynomial factor.}
\end{multipleChoice}


What is the name of the theorem that applies to this problem? \qquad \\
The \underline{\answer{Squeeze}} Theorem
\end{problem}}%}

%%%%%%%%%%%%%%%%%%%%%




\latexProblemContent{
\begin{problem}

The limit as $x\to{-5}$ of $f(x)={{\left(x + 5\right)} \cos\left(\frac{12}{{\left(x + 5\right)}^{2}}\right)}$ is $0$.  What is the reason why this is true?

\input{2311-Concept-Limit-0008.HELP.tex}

\begin{multipleChoice}
\choice{The statement is in fact false: $\lim\limits_{x\to{-5}}{{\left(x + 5\right)} \cos\left(\frac{12}{{\left(x + 5\right)}^{2}}\right)}\neq0$.}
\choice{The cosine factor decreases to $0$ faster than the polynomial.}
\choice[correct]{The cosine factor is bounded between $-1$ and $1$, so the polynomial forces the function to $0$.}
\choice{The cosine factor directly cancels out the polynomial factor.}
\end{multipleChoice}


What is the name of the theorem that applies to this problem? \qquad \\
The \underline{\answer{Squeeze}} Theorem
\end{problem}}%}

%%%%%%%%%%%%%%%%%%%%%




\latexProblemContent{
\begin{problem}

The limit as $x\to{-12}$ of $f(x)={{\left(x + 12\right)}^{2} \cos\left(-\frac{24}{{\left(x + 12\right)}^{2}}\right)}$ is $0$.  What is the reason why this is true?

\input{2311-Concept-Limit-0008.HELP.tex}

\begin{multipleChoice}
\choice{The statement is in fact false: $\lim\limits_{x\to{-12}}{{\left(x + 12\right)}^{2} \cos\left(-\frac{24}{{\left(x + 12\right)}^{2}}\right)}\neq0$.}
\choice{The cosine factor decreases to $0$ faster than the polynomial.}
\choice[correct]{The cosine factor is bounded between $-1$ and $1$, so the polynomial forces the function to $0$.}
\choice{The cosine factor directly cancels out the polynomial factor.}
\end{multipleChoice}


What is the name of the theorem that applies to this problem? \qquad \\
The \underline{\answer{Squeeze}} Theorem
\end{problem}}%}

%%%%%%%%%%%%%%%%%%%%%




\latexProblemContent{
\begin{problem}

The limit as $x\to{-9}$ of $f(x)={{\left(x + 9\right)} \cos\left(\frac{25}{x + 9}\right)}$ is $0$.  What is the reason why this is true?

\input{2311-Concept-Limit-0008.HELP.tex}

\begin{multipleChoice}
\choice{The statement is in fact false: $\lim\limits_{x\to{-9}}{{\left(x + 9\right)} \cos\left(\frac{25}{x + 9}\right)}\neq0$.}
\choice{The cosine factor decreases to $0$ faster than the polynomial.}
\choice[correct]{The cosine factor is bounded between $-1$ and $1$, so the polynomial forces the function to $0$.}
\choice{The cosine factor directly cancels out the polynomial factor.}
\end{multipleChoice}


What is the name of the theorem that applies to this problem? \qquad \\
The \underline{\answer{Squeeze}} Theorem
\end{problem}}%}

%%%%%%%%%%%%%%%%%%%%%




\latexProblemContent{
\begin{problem}

The limit as $x\to{-15}$ of $f(x)={{\left(x + 15\right)}^{3} \cos\left(\frac{21}{x + 15}\right)}$ is $0$.  What is the reason why this is true?

\input{2311-Concept-Limit-0008.HELP.tex}

\begin{multipleChoice}
\choice{The statement is in fact false: $\lim\limits_{x\to{-15}}{{\left(x + 15\right)}^{3} \cos\left(\frac{21}{x + 15}\right)}\neq0$.}
\choice{The cosine factor decreases to $0$ faster than the polynomial.}
\choice[correct]{The cosine factor is bounded between $-1$ and $1$, so the polynomial forces the function to $0$.}
\choice{The cosine factor directly cancels out the polynomial factor.}
\end{multipleChoice}


What is the name of the theorem that applies to this problem? \qquad \\
The \underline{\answer{Squeeze}} Theorem
\end{problem}}%}

%%%%%%%%%%%%%%%%%%%%%




\latexProblemContent{
\begin{problem}

The limit as $x\to{1}$ of $f(x)={{\left(x - 1\right)} \cos\left(-\frac{20}{x - 1}\right)}$ is $0$.  What is the reason why this is true?

\input{2311-Concept-Limit-0008.HELP.tex}

\begin{multipleChoice}
\choice{The statement is in fact false: $\lim\limits_{x\to{1}}{{\left(x - 1\right)} \cos\left(-\frac{20}{x - 1}\right)}\neq0$.}
\choice{The cosine factor decreases to $0$ faster than the polynomial.}
\choice[correct]{The cosine factor is bounded between $-1$ and $1$, so the polynomial forces the function to $0$.}
\choice{The cosine factor directly cancels out the polynomial factor.}
\end{multipleChoice}


What is the name of the theorem that applies to this problem? \qquad \\
The \underline{\answer{Squeeze}} Theorem
\end{problem}}%}

%%%%%%%%%%%%%%%%%%%%%




\latexProblemContent{
\begin{problem}

The limit as $x\to{1}$ of $f(x)={{\left(x - 1\right)}^{2} \cos\left(-\frac{11}{x - 1}\right)}$ is $0$.  What is the reason why this is true?

\input{2311-Concept-Limit-0008.HELP.tex}

\begin{multipleChoice}
\choice{The statement is in fact false: $\lim\limits_{x\to{1}}{{\left(x - 1\right)}^{2} \cos\left(-\frac{11}{x - 1}\right)}\neq0$.}
\choice{The cosine factor decreases to $0$ faster than the polynomial.}
\choice[correct]{The cosine factor is bounded between $-1$ and $1$, so the polynomial forces the function to $0$.}
\choice{The cosine factor directly cancels out the polynomial factor.}
\end{multipleChoice}


What is the name of the theorem that applies to this problem? \qquad \\
The \underline{\answer{Squeeze}} Theorem
\end{problem}}%}

%%%%%%%%%%%%%%%%%%%%%




\latexProblemContent{
\begin{problem}

The limit as $x\to{9}$ of $f(x)={{\left(x - 9\right)}^{3} \cos\left(-\frac{5}{{\left(x - 9\right)}^{2}}\right)}$ is $0$.  What is the reason why this is true?

\input{2311-Concept-Limit-0008.HELP.tex}

\begin{multipleChoice}
\choice{The statement is in fact false: $\lim\limits_{x\to{9}}{{\left(x - 9\right)}^{3} \cos\left(-\frac{5}{{\left(x - 9\right)}^{2}}\right)}\neq0$.}
\choice{The cosine factor decreases to $0$ faster than the polynomial.}
\choice[correct]{The cosine factor is bounded between $-1$ and $1$, so the polynomial forces the function to $0$.}
\choice{The cosine factor directly cancels out the polynomial factor.}
\end{multipleChoice}


What is the name of the theorem that applies to this problem? \qquad \\
The \underline{\answer{Squeeze}} Theorem
\end{problem}}%}

%%%%%%%%%%%%%%%%%%%%%




\latexProblemContent{
\begin{problem}

The limit as $x\to{-13}$ of $f(x)={{\left(x + 13\right)} \cos\left(-\frac{7}{x + 13}\right)}$ is $0$.  What is the reason why this is true?

\input{2311-Concept-Limit-0008.HELP.tex}

\begin{multipleChoice}
\choice{The statement is in fact false: $\lim\limits_{x\to{-13}}{{\left(x + 13\right)} \cos\left(-\frac{7}{x + 13}\right)}\neq0$.}
\choice{The cosine factor decreases to $0$ faster than the polynomial.}
\choice[correct]{The cosine factor is bounded between $-1$ and $1$, so the polynomial forces the function to $0$.}
\choice{The cosine factor directly cancels out the polynomial factor.}
\end{multipleChoice}


What is the name of the theorem that applies to this problem? \qquad \\
The \underline{\answer{Squeeze}} Theorem
\end{problem}}%}

%%%%%%%%%%%%%%%%%%%%%




\latexProblemContent{
\begin{problem}

The limit as $x\to{0}$ of $f(x)={x \cos\left(\frac{16}{x^{2}}\right)}$ is $0$.  What is the reason why this is true?

\input{2311-Concept-Limit-0008.HELP.tex}

\begin{multipleChoice}
\choice{The statement is in fact false: $\lim\limits_{x\to{0}}{x \cos\left(\frac{16}{x^{2}}\right)}\neq0$.}
\choice{The cosine factor decreases to $0$ faster than the polynomial.}
\choice[correct]{The cosine factor is bounded between $-1$ and $1$, so the polynomial forces the function to $0$.}
\choice{The cosine factor directly cancels out the polynomial factor.}
\end{multipleChoice}


What is the name of the theorem that applies to this problem? \qquad \\
The \underline{\answer{Squeeze}} Theorem
\end{problem}}%}

%%%%%%%%%%%%%%%%%%%%%




\latexProblemContent{
\begin{problem}

The limit as $x\to{12}$ of $f(x)={{\left(x - 12\right)}^{3} \cos\left(\frac{6}{x - 12}\right)}$ is $0$.  What is the reason why this is true?

\input{2311-Concept-Limit-0008.HELP.tex}

\begin{multipleChoice}
\choice{The statement is in fact false: $\lim\limits_{x\to{12}}{{\left(x - 12\right)}^{3} \cos\left(\frac{6}{x - 12}\right)}\neq0$.}
\choice{The cosine factor decreases to $0$ faster than the polynomial.}
\choice[correct]{The cosine factor is bounded between $-1$ and $1$, so the polynomial forces the function to $0$.}
\choice{The cosine factor directly cancels out the polynomial factor.}
\end{multipleChoice}


What is the name of the theorem that applies to this problem? \qquad \\
The \underline{\answer{Squeeze}} Theorem
\end{problem}}%}

%%%%%%%%%%%%%%%%%%%%%




\latexProblemContent{
\begin{problem}

The limit as $x\to{12}$ of $f(x)={{\left(x - 12\right)}^{3} \cos\left(-\frac{10}{x - 12}\right)}$ is $0$.  What is the reason why this is true?

\input{2311-Concept-Limit-0008.HELP.tex}

\begin{multipleChoice}
\choice{The statement is in fact false: $\lim\limits_{x\to{12}}{{\left(x - 12\right)}^{3} \cos\left(-\frac{10}{x - 12}\right)}\neq0$.}
\choice{The cosine factor decreases to $0$ faster than the polynomial.}
\choice[correct]{The cosine factor is bounded between $-1$ and $1$, so the polynomial forces the function to $0$.}
\choice{The cosine factor directly cancels out the polynomial factor.}
\end{multipleChoice}


What is the name of the theorem that applies to this problem? \qquad \\
The \underline{\answer{Squeeze}} Theorem
\end{problem}}%}

%%%%%%%%%%%%%%%%%%%%%




\latexProblemContent{
\begin{problem}

The limit as $x\to{10}$ of $f(x)={{\left(x - 10\right)} \cos\left(-\frac{4}{x - 10}\right)}$ is $0$.  What is the reason why this is true?

\input{2311-Concept-Limit-0008.HELP.tex}

\begin{multipleChoice}
\choice{The statement is in fact false: $\lim\limits_{x\to{10}}{{\left(x - 10\right)} \cos\left(-\frac{4}{x - 10}\right)}\neq0$.}
\choice{The cosine factor decreases to $0$ faster than the polynomial.}
\choice[correct]{The cosine factor is bounded between $-1$ and $1$, so the polynomial forces the function to $0$.}
\choice{The cosine factor directly cancels out the polynomial factor.}
\end{multipleChoice}


What is the name of the theorem that applies to this problem? \qquad \\
The \underline{\answer{Squeeze}} Theorem
\end{problem}}%}

%%%%%%%%%%%%%%%%%%%%%




\latexProblemContent{
\begin{problem}

The limit as $x\to{5}$ of $f(x)={{\left(x - 5\right)} \cos\left(-\frac{3}{x - 5}\right)}$ is $0$.  What is the reason why this is true?

\input{2311-Concept-Limit-0008.HELP.tex}

\begin{multipleChoice}
\choice{The statement is in fact false: $\lim\limits_{x\to{5}}{{\left(x - 5\right)} \cos\left(-\frac{3}{x - 5}\right)}\neq0$.}
\choice{The cosine factor decreases to $0$ faster than the polynomial.}
\choice[correct]{The cosine factor is bounded between $-1$ and $1$, so the polynomial forces the function to $0$.}
\choice{The cosine factor directly cancels out the polynomial factor.}
\end{multipleChoice}


What is the name of the theorem that applies to this problem? \qquad \\
The \underline{\answer{Squeeze}} Theorem
\end{problem}}%}

%%%%%%%%%%%%%%%%%%%%%




\latexProblemContent{
\begin{problem}

The limit as $x\to{4}$ of $f(x)={{\left(x - 4\right)}^{3} \cos\left(\frac{21}{x - 4}\right)}$ is $0$.  What is the reason why this is true?

\input{2311-Concept-Limit-0008.HELP.tex}

\begin{multipleChoice}
\choice{The statement is in fact false: $\lim\limits_{x\to{4}}{{\left(x - 4\right)}^{3} \cos\left(\frac{21}{x - 4}\right)}\neq0$.}
\choice{The cosine factor decreases to $0$ faster than the polynomial.}
\choice[correct]{The cosine factor is bounded between $-1$ and $1$, so the polynomial forces the function to $0$.}
\choice{The cosine factor directly cancels out the polynomial factor.}
\end{multipleChoice}


What is the name of the theorem that applies to this problem? \qquad \\
The \underline{\answer{Squeeze}} Theorem
\end{problem}}%}

%%%%%%%%%%%%%%%%%%%%%




\latexProblemContent{
\begin{problem}

The limit as $x\to{-1}$ of $f(x)={{\left(x + 1\right)}^{3} \cos\left(\frac{8}{x + 1}\right)}$ is $0$.  What is the reason why this is true?

\input{2311-Concept-Limit-0008.HELP.tex}

\begin{multipleChoice}
\choice{The statement is in fact false: $\lim\limits_{x\to{-1}}{{\left(x + 1\right)}^{3} \cos\left(\frac{8}{x + 1}\right)}\neq0$.}
\choice{The cosine factor decreases to $0$ faster than the polynomial.}
\choice[correct]{The cosine factor is bounded between $-1$ and $1$, so the polynomial forces the function to $0$.}
\choice{The cosine factor directly cancels out the polynomial factor.}
\end{multipleChoice}


What is the name of the theorem that applies to this problem? \qquad \\
The \underline{\answer{Squeeze}} Theorem
\end{problem}}%}

%%%%%%%%%%%%%%%%%%%%%




\latexProblemContent{
\begin{problem}

The limit as $x\to{12}$ of $f(x)={{\left(x - 12\right)}^{3} \cos\left(\frac{5}{x - 12}\right)}$ is $0$.  What is the reason why this is true?

\input{2311-Concept-Limit-0008.HELP.tex}

\begin{multipleChoice}
\choice{The statement is in fact false: $\lim\limits_{x\to{12}}{{\left(x - 12\right)}^{3} \cos\left(\frac{5}{x - 12}\right)}\neq0$.}
\choice{The cosine factor decreases to $0$ faster than the polynomial.}
\choice[correct]{The cosine factor is bounded between $-1$ and $1$, so the polynomial forces the function to $0$.}
\choice{The cosine factor directly cancels out the polynomial factor.}
\end{multipleChoice}


What is the name of the theorem that applies to this problem? \qquad \\
The \underline{\answer{Squeeze}} Theorem
\end{problem}}%}

%%%%%%%%%%%%%%%%%%%%%




\latexProblemContent{
\begin{problem}

The limit as $x\to{-2}$ of $f(x)={{\left(x + 2\right)}^{3} \cos\left(-\frac{17}{{\left(x + 2\right)}^{2}}\right)}$ is $0$.  What is the reason why this is true?

\input{2311-Concept-Limit-0008.HELP.tex}

\begin{multipleChoice}
\choice{The statement is in fact false: $\lim\limits_{x\to{-2}}{{\left(x + 2\right)}^{3} \cos\left(-\frac{17}{{\left(x + 2\right)}^{2}}\right)}\neq0$.}
\choice{The cosine factor decreases to $0$ faster than the polynomial.}
\choice[correct]{The cosine factor is bounded between $-1$ and $1$, so the polynomial forces the function to $0$.}
\choice{The cosine factor directly cancels out the polynomial factor.}
\end{multipleChoice}


What is the name of the theorem that applies to this problem? \qquad \\
The \underline{\answer{Squeeze}} Theorem
\end{problem}}%}

%%%%%%%%%%%%%%%%%%%%%




\latexProblemContent{
\begin{problem}

The limit as $x\to{7}$ of $f(x)={{\left(x - 7\right)}^{2} \cos\left(\frac{6}{{\left(x - 7\right)}^{2}}\right)}$ is $0$.  What is the reason why this is true?

\input{2311-Concept-Limit-0008.HELP.tex}

\begin{multipleChoice}
\choice{The statement is in fact false: $\lim\limits_{x\to{7}}{{\left(x - 7\right)}^{2} \cos\left(\frac{6}{{\left(x - 7\right)}^{2}}\right)}\neq0$.}
\choice{The cosine factor decreases to $0$ faster than the polynomial.}
\choice[correct]{The cosine factor is bounded between $-1$ and $1$, so the polynomial forces the function to $0$.}
\choice{The cosine factor directly cancels out the polynomial factor.}
\end{multipleChoice}


What is the name of the theorem that applies to this problem? \qquad \\
The \underline{\answer{Squeeze}} Theorem
\end{problem}}%}

%%%%%%%%%%%%%%%%%%%%%




\latexProblemContent{
\begin{problem}

The limit as $x\to{5}$ of $f(x)={{\left(x - 5\right)}^{3} \cos\left(-\frac{12}{x - 5}\right)}$ is $0$.  What is the reason why this is true?

\input{2311-Concept-Limit-0008.HELP.tex}

\begin{multipleChoice}
\choice{The statement is in fact false: $\lim\limits_{x\to{5}}{{\left(x - 5\right)}^{3} \cos\left(-\frac{12}{x - 5}\right)}\neq0$.}
\choice{The cosine factor decreases to $0$ faster than the polynomial.}
\choice[correct]{The cosine factor is bounded between $-1$ and $1$, so the polynomial forces the function to $0$.}
\choice{The cosine factor directly cancels out the polynomial factor.}
\end{multipleChoice}


What is the name of the theorem that applies to this problem? \qquad \\
The \underline{\answer{Squeeze}} Theorem
\end{problem}}%}

%%%%%%%%%%%%%%%%%%%%%




\latexProblemContent{
\begin{problem}

The limit as $x\to{-5}$ of $f(x)={{\left(x + 5\right)}^{3} \cos\left(\frac{2}{x + 5}\right)}$ is $0$.  What is the reason why this is true?

\input{2311-Concept-Limit-0008.HELP.tex}

\begin{multipleChoice}
\choice{The statement is in fact false: $\lim\limits_{x\to{-5}}{{\left(x + 5\right)}^{3} \cos\left(\frac{2}{x + 5}\right)}\neq0$.}
\choice{The cosine factor decreases to $0$ faster than the polynomial.}
\choice[correct]{The cosine factor is bounded between $-1$ and $1$, so the polynomial forces the function to $0$.}
\choice{The cosine factor directly cancels out the polynomial factor.}
\end{multipleChoice}


What is the name of the theorem that applies to this problem? \qquad \\
The \underline{\answer{Squeeze}} Theorem
\end{problem}}%}

%%%%%%%%%%%%%%%%%%%%%




\latexProblemContent{
\begin{problem}

The limit as $x\to{0}$ of $f(x)={x \cos\left(\frac{18}{x^{2}}\right)}$ is $0$.  What is the reason why this is true?

\input{2311-Concept-Limit-0008.HELP.tex}

\begin{multipleChoice}
\choice{The statement is in fact false: $\lim\limits_{x\to{0}}{x \cos\left(\frac{18}{x^{2}}\right)}\neq0$.}
\choice{The cosine factor decreases to $0$ faster than the polynomial.}
\choice[correct]{The cosine factor is bounded between $-1$ and $1$, so the polynomial forces the function to $0$.}
\choice{The cosine factor directly cancels out the polynomial factor.}
\end{multipleChoice}


What is the name of the theorem that applies to this problem? \qquad \\
The \underline{\answer{Squeeze}} Theorem
\end{problem}}%}

%%%%%%%%%%%%%%%%%%%%%




\latexProblemContent{
\begin{problem}

The limit as $x\to{-10}$ of $f(x)={{\left(x + 10\right)}^{3} \cos\left(-\frac{18}{{\left(x + 10\right)}^{2}}\right)}$ is $0$.  What is the reason why this is true?

\input{2311-Concept-Limit-0008.HELP.tex}

\begin{multipleChoice}
\choice{The statement is in fact false: $\lim\limits_{x\to{-10}}{{\left(x + 10\right)}^{3} \cos\left(-\frac{18}{{\left(x + 10\right)}^{2}}\right)}\neq0$.}
\choice{The cosine factor decreases to $0$ faster than the polynomial.}
\choice[correct]{The cosine factor is bounded between $-1$ and $1$, so the polynomial forces the function to $0$.}
\choice{The cosine factor directly cancels out the polynomial factor.}
\end{multipleChoice}


What is the name of the theorem that applies to this problem? \qquad \\
The \underline{\answer{Squeeze}} Theorem
\end{problem}}%}

%%%%%%%%%%%%%%%%%%%%%




\latexProblemContent{
\begin{problem}

The limit as $x\to{14}$ of $f(x)={{\left(x - 14\right)}^{2} \cos\left(\frac{18}{{\left(x - 14\right)}^{2}}\right)}$ is $0$.  What is the reason why this is true?

\input{2311-Concept-Limit-0008.HELP.tex}

\begin{multipleChoice}
\choice{The statement is in fact false: $\lim\limits_{x\to{14}}{{\left(x - 14\right)}^{2} \cos\left(\frac{18}{{\left(x - 14\right)}^{2}}\right)}\neq0$.}
\choice{The cosine factor decreases to $0$ faster than the polynomial.}
\choice[correct]{The cosine factor is bounded between $-1$ and $1$, so the polynomial forces the function to $0$.}
\choice{The cosine factor directly cancels out the polynomial factor.}
\end{multipleChoice}


What is the name of the theorem that applies to this problem? \qquad \\
The \underline{\answer{Squeeze}} Theorem
\end{problem}}%}

%%%%%%%%%%%%%%%%%%%%%




\latexProblemContent{
\begin{problem}

The limit as $x\to{11}$ of $f(x)={{\left(x - 11\right)}^{2} \cos\left(\frac{19}{x - 11}\right)}$ is $0$.  What is the reason why this is true?

\input{2311-Concept-Limit-0008.HELP.tex}

\begin{multipleChoice}
\choice{The statement is in fact false: $\lim\limits_{x\to{11}}{{\left(x - 11\right)}^{2} \cos\left(\frac{19}{x - 11}\right)}\neq0$.}
\choice{The cosine factor decreases to $0$ faster than the polynomial.}
\choice[correct]{The cosine factor is bounded between $-1$ and $1$, so the polynomial forces the function to $0$.}
\choice{The cosine factor directly cancels out the polynomial factor.}
\end{multipleChoice}


What is the name of the theorem that applies to this problem? \qquad \\
The \underline{\answer{Squeeze}} Theorem
\end{problem}}%}

%%%%%%%%%%%%%%%%%%%%%




\latexProblemContent{
\begin{problem}

The limit as $x\to{-10}$ of $f(x)={{\left(x + 10\right)}^{3} \cos\left(\frac{23}{x + 10}\right)}$ is $0$.  What is the reason why this is true?

\input{2311-Concept-Limit-0008.HELP.tex}

\begin{multipleChoice}
\choice{The statement is in fact false: $\lim\limits_{x\to{-10}}{{\left(x + 10\right)}^{3} \cos\left(\frac{23}{x + 10}\right)}\neq0$.}
\choice{The cosine factor decreases to $0$ faster than the polynomial.}
\choice[correct]{The cosine factor is bounded between $-1$ and $1$, so the polynomial forces the function to $0$.}
\choice{The cosine factor directly cancels out the polynomial factor.}
\end{multipleChoice}


What is the name of the theorem that applies to this problem? \qquad \\
The \underline{\answer{Squeeze}} Theorem
\end{problem}}%}

%%%%%%%%%%%%%%%%%%%%%




\latexProblemContent{
\begin{problem}

The limit as $x\to{-1}$ of $f(x)={{\left(x + 1\right)} \cos\left(-\frac{8}{x + 1}\right)}$ is $0$.  What is the reason why this is true?

\input{2311-Concept-Limit-0008.HELP.tex}

\begin{multipleChoice}
\choice{The statement is in fact false: $\lim\limits_{x\to{-1}}{{\left(x + 1\right)} \cos\left(-\frac{8}{x + 1}\right)}\neq0$.}
\choice{The cosine factor decreases to $0$ faster than the polynomial.}
\choice[correct]{The cosine factor is bounded between $-1$ and $1$, so the polynomial forces the function to $0$.}
\choice{The cosine factor directly cancels out the polynomial factor.}
\end{multipleChoice}


What is the name of the theorem that applies to this problem? \qquad \\
The \underline{\answer{Squeeze}} Theorem
\end{problem}}%}

%%%%%%%%%%%%%%%%%%%%%




\latexProblemContent{
\begin{problem}

The limit as $x\to{8}$ of $f(x)={{\left(x - 8\right)}^{3} \cos\left(-\frac{2}{x - 8}\right)}$ is $0$.  What is the reason why this is true?

\input{2311-Concept-Limit-0008.HELP.tex}

\begin{multipleChoice}
\choice{The statement is in fact false: $\lim\limits_{x\to{8}}{{\left(x - 8\right)}^{3} \cos\left(-\frac{2}{x - 8}\right)}\neq0$.}
\choice{The cosine factor decreases to $0$ faster than the polynomial.}
\choice[correct]{The cosine factor is bounded between $-1$ and $1$, so the polynomial forces the function to $0$.}
\choice{The cosine factor directly cancels out the polynomial factor.}
\end{multipleChoice}


What is the name of the theorem that applies to this problem? \qquad \\
The \underline{\answer{Squeeze}} Theorem
\end{problem}}%}

%%%%%%%%%%%%%%%%%%%%%




\latexProblemContent{
\begin{problem}

The limit as $x\to{-1}$ of $f(x)={{\left(x + 1\right)}^{2} \cos\left(\frac{6}{x + 1}\right)}$ is $0$.  What is the reason why this is true?

\input{2311-Concept-Limit-0008.HELP.tex}

\begin{multipleChoice}
\choice{The statement is in fact false: $\lim\limits_{x\to{-1}}{{\left(x + 1\right)}^{2} \cos\left(\frac{6}{x + 1}\right)}\neq0$.}
\choice{The cosine factor decreases to $0$ faster than the polynomial.}
\choice[correct]{The cosine factor is bounded between $-1$ and $1$, so the polynomial forces the function to $0$.}
\choice{The cosine factor directly cancels out the polynomial factor.}
\end{multipleChoice}


What is the name of the theorem that applies to this problem? \qquad \\
The \underline{\answer{Squeeze}} Theorem
\end{problem}}%}

%%%%%%%%%%%%%%%%%%%%%




\latexProblemContent{
\begin{problem}

The limit as $x\to{-4}$ of $f(x)={{\left(x + 4\right)}^{3} \cos\left(\frac{4}{x + 4}\right)}$ is $0$.  What is the reason why this is true?

\input{2311-Concept-Limit-0008.HELP.tex}

\begin{multipleChoice}
\choice{The statement is in fact false: $\lim\limits_{x\to{-4}}{{\left(x + 4\right)}^{3} \cos\left(\frac{4}{x + 4}\right)}\neq0$.}
\choice{The cosine factor decreases to $0$ faster than the polynomial.}
\choice[correct]{The cosine factor is bounded between $-1$ and $1$, so the polynomial forces the function to $0$.}
\choice{The cosine factor directly cancels out the polynomial factor.}
\end{multipleChoice}


What is the name of the theorem that applies to this problem? \qquad \\
The \underline{\answer{Squeeze}} Theorem
\end{problem}}%}

%%%%%%%%%%%%%%%%%%%%%




\latexProblemContent{
\begin{problem}

The limit as $x\to{-10}$ of $f(x)={{\left(x + 10\right)}^{2} \cos\left(-\frac{22}{x + 10}\right)}$ is $0$.  What is the reason why this is true?

\input{2311-Concept-Limit-0008.HELP.tex}

\begin{multipleChoice}
\choice{The statement is in fact false: $\lim\limits_{x\to{-10}}{{\left(x + 10\right)}^{2} \cos\left(-\frac{22}{x + 10}\right)}\neq0$.}
\choice{The cosine factor decreases to $0$ faster than the polynomial.}
\choice[correct]{The cosine factor is bounded between $-1$ and $1$, so the polynomial forces the function to $0$.}
\choice{The cosine factor directly cancels out the polynomial factor.}
\end{multipleChoice}


What is the name of the theorem that applies to this problem? \qquad \\
The \underline{\answer{Squeeze}} Theorem
\end{problem}}%}

%%%%%%%%%%%%%%%%%%%%%




\latexProblemContent{
\begin{problem}

The limit as $x\to{5}$ of $f(x)={{\left(x - 5\right)} \cos\left(-\frac{5}{x - 5}\right)}$ is $0$.  What is the reason why this is true?

\input{2311-Concept-Limit-0008.HELP.tex}

\begin{multipleChoice}
\choice{The statement is in fact false: $\lim\limits_{x\to{5}}{{\left(x - 5\right)} \cos\left(-\frac{5}{x - 5}\right)}\neq0$.}
\choice{The cosine factor decreases to $0$ faster than the polynomial.}
\choice[correct]{The cosine factor is bounded between $-1$ and $1$, so the polynomial forces the function to $0$.}
\choice{The cosine factor directly cancels out the polynomial factor.}
\end{multipleChoice}


What is the name of the theorem that applies to this problem? \qquad \\
The \underline{\answer{Squeeze}} Theorem
\end{problem}}%}

%%%%%%%%%%%%%%%%%%%%%




\latexProblemContent{
\begin{problem}

The limit as $x\to{-10}$ of $f(x)={{\left(x + 10\right)} \cos\left(-\frac{3}{x + 10}\right)}$ is $0$.  What is the reason why this is true?

\input{2311-Concept-Limit-0008.HELP.tex}

\begin{multipleChoice}
\choice{The statement is in fact false: $\lim\limits_{x\to{-10}}{{\left(x + 10\right)} \cos\left(-\frac{3}{x + 10}\right)}\neq0$.}
\choice{The cosine factor decreases to $0$ faster than the polynomial.}
\choice[correct]{The cosine factor is bounded between $-1$ and $1$, so the polynomial forces the function to $0$.}
\choice{The cosine factor directly cancels out the polynomial factor.}
\end{multipleChoice}


What is the name of the theorem that applies to this problem? \qquad \\
The \underline{\answer{Squeeze}} Theorem
\end{problem}}%}

%%%%%%%%%%%%%%%%%%%%%




\latexProblemContent{
\begin{problem}

The limit as $x\to{-15}$ of $f(x)={{\left(x + 15\right)} \cos\left(-\frac{6}{x + 15}\right)}$ is $0$.  What is the reason why this is true?

\input{2311-Concept-Limit-0008.HELP.tex}

\begin{multipleChoice}
\choice{The statement is in fact false: $\lim\limits_{x\to{-15}}{{\left(x + 15\right)} \cos\left(-\frac{6}{x + 15}\right)}\neq0$.}
\choice{The cosine factor decreases to $0$ faster than the polynomial.}
\choice[correct]{The cosine factor is bounded between $-1$ and $1$, so the polynomial forces the function to $0$.}
\choice{The cosine factor directly cancels out the polynomial factor.}
\end{multipleChoice}


What is the name of the theorem that applies to this problem? \qquad \\
The \underline{\answer{Squeeze}} Theorem
\end{problem}}%}

%%%%%%%%%%%%%%%%%%%%%




\latexProblemContent{
\begin{problem}

The limit as $x\to{12}$ of $f(x)={{\left(x - 12\right)} \cos\left(-\frac{18}{{\left(x - 12\right)}^{2}}\right)}$ is $0$.  What is the reason why this is true?

\input{2311-Concept-Limit-0008.HELP.tex}

\begin{multipleChoice}
\choice{The statement is in fact false: $\lim\limits_{x\to{12}}{{\left(x - 12\right)} \cos\left(-\frac{18}{{\left(x - 12\right)}^{2}}\right)}\neq0$.}
\choice{The cosine factor decreases to $0$ faster than the polynomial.}
\choice[correct]{The cosine factor is bounded between $-1$ and $1$, so the polynomial forces the function to $0$.}
\choice{The cosine factor directly cancels out the polynomial factor.}
\end{multipleChoice}


What is the name of the theorem that applies to this problem? \qquad \\
The \underline{\answer{Squeeze}} Theorem
\end{problem}}%}

%%%%%%%%%%%%%%%%%%%%%




\latexProblemContent{
\begin{problem}

The limit as $x\to{-3}$ of $f(x)={{\left(x + 3\right)}^{2} \cos\left(-\frac{24}{{\left(x + 3\right)}^{2}}\right)}$ is $0$.  What is the reason why this is true?

\input{2311-Concept-Limit-0008.HELP.tex}

\begin{multipleChoice}
\choice{The statement is in fact false: $\lim\limits_{x\to{-3}}{{\left(x + 3\right)}^{2} \cos\left(-\frac{24}{{\left(x + 3\right)}^{2}}\right)}\neq0$.}
\choice{The cosine factor decreases to $0$ faster than the polynomial.}
\choice[correct]{The cosine factor is bounded between $-1$ and $1$, so the polynomial forces the function to $0$.}
\choice{The cosine factor directly cancels out the polynomial factor.}
\end{multipleChoice}


What is the name of the theorem that applies to this problem? \qquad \\
The \underline{\answer{Squeeze}} Theorem
\end{problem}}%}

%%%%%%%%%%%%%%%%%%%%%




\latexProblemContent{
\begin{problem}

The limit as $x\to{2}$ of $f(x)={{\left(x - 2\right)} \cos\left(-\frac{8}{{\left(x - 2\right)}^{2}}\right)}$ is $0$.  What is the reason why this is true?

\input{2311-Concept-Limit-0008.HELP.tex}

\begin{multipleChoice}
\choice{The statement is in fact false: $\lim\limits_{x\to{2}}{{\left(x - 2\right)} \cos\left(-\frac{8}{{\left(x - 2\right)}^{2}}\right)}\neq0$.}
\choice{The cosine factor decreases to $0$ faster than the polynomial.}
\choice[correct]{The cosine factor is bounded between $-1$ and $1$, so the polynomial forces the function to $0$.}
\choice{The cosine factor directly cancels out the polynomial factor.}
\end{multipleChoice}


What is the name of the theorem that applies to this problem? \qquad \\
The \underline{\answer{Squeeze}} Theorem
\end{problem}}%}

%%%%%%%%%%%%%%%%%%%%%




\latexProblemContent{
\begin{problem}

The limit as $x\to{2}$ of $f(x)={{\left(x - 2\right)}^{3} \cos\left(\frac{13}{{\left(x - 2\right)}^{2}}\right)}$ is $0$.  What is the reason why this is true?

\input{2311-Concept-Limit-0008.HELP.tex}

\begin{multipleChoice}
\choice{The statement is in fact false: $\lim\limits_{x\to{2}}{{\left(x - 2\right)}^{3} \cos\left(\frac{13}{{\left(x - 2\right)}^{2}}\right)}\neq0$.}
\choice{The cosine factor decreases to $0$ faster than the polynomial.}
\choice[correct]{The cosine factor is bounded between $-1$ and $1$, so the polynomial forces the function to $0$.}
\choice{The cosine factor directly cancels out the polynomial factor.}
\end{multipleChoice}


What is the name of the theorem that applies to this problem? \qquad \\
The \underline{\answer{Squeeze}} Theorem
\end{problem}}%}

%%%%%%%%%%%%%%%%%%%%%




\latexProblemContent{
\begin{problem}

The limit as $x\to{9}$ of $f(x)={{\left(x - 9\right)} \cos\left(-\frac{8}{x - 9}\right)}$ is $0$.  What is the reason why this is true?

\input{2311-Concept-Limit-0008.HELP.tex}

\begin{multipleChoice}
\choice{The statement is in fact false: $\lim\limits_{x\to{9}}{{\left(x - 9\right)} \cos\left(-\frac{8}{x - 9}\right)}\neq0$.}
\choice{The cosine factor decreases to $0$ faster than the polynomial.}
\choice[correct]{The cosine factor is bounded between $-1$ and $1$, so the polynomial forces the function to $0$.}
\choice{The cosine factor directly cancels out the polynomial factor.}
\end{multipleChoice}


What is the name of the theorem that applies to this problem? \qquad \\
The \underline{\answer{Squeeze}} Theorem
\end{problem}}%}

%%%%%%%%%%%%%%%%%%%%%




\latexProblemContent{
\begin{problem}

The limit as $x\to{-6}$ of $f(x)={{\left(x + 6\right)}^{2} \cos\left(\frac{7}{{\left(x + 6\right)}^{2}}\right)}$ is $0$.  What is the reason why this is true?

\input{2311-Concept-Limit-0008.HELP.tex}

\begin{multipleChoice}
\choice{The statement is in fact false: $\lim\limits_{x\to{-6}}{{\left(x + 6\right)}^{2} \cos\left(\frac{7}{{\left(x + 6\right)}^{2}}\right)}\neq0$.}
\choice{The cosine factor decreases to $0$ faster than the polynomial.}
\choice[correct]{The cosine factor is bounded between $-1$ and $1$, so the polynomial forces the function to $0$.}
\choice{The cosine factor directly cancels out the polynomial factor.}
\end{multipleChoice}


What is the name of the theorem that applies to this problem? \qquad \\
The \underline{\answer{Squeeze}} Theorem
\end{problem}}%}

%%%%%%%%%%%%%%%%%%%%%




\latexProblemContent{
\begin{problem}

The limit as $x\to{6}$ of $f(x)={{\left(x - 6\right)}^{3} \cos\left(-\frac{12}{x - 6}\right)}$ is $0$.  What is the reason why this is true?

\input{2311-Concept-Limit-0008.HELP.tex}

\begin{multipleChoice}
\choice{The statement is in fact false: $\lim\limits_{x\to{6}}{{\left(x - 6\right)}^{3} \cos\left(-\frac{12}{x - 6}\right)}\neq0$.}
\choice{The cosine factor decreases to $0$ faster than the polynomial.}
\choice[correct]{The cosine factor is bounded between $-1$ and $1$, so the polynomial forces the function to $0$.}
\choice{The cosine factor directly cancels out the polynomial factor.}
\end{multipleChoice}


What is the name of the theorem that applies to this problem? \qquad \\
The \underline{\answer{Squeeze}} Theorem
\end{problem}}%}

%%%%%%%%%%%%%%%%%%%%%




\latexProblemContent{
\begin{problem}

The limit as $x\to{11}$ of $f(x)={{\left(x - 11\right)} \cos\left(-\frac{1}{x - 11}\right)}$ is $0$.  What is the reason why this is true?

\input{2311-Concept-Limit-0008.HELP.tex}

\begin{multipleChoice}
\choice{The statement is in fact false: $\lim\limits_{x\to{11}}{{\left(x - 11\right)} \cos\left(-\frac{1}{x - 11}\right)}\neq0$.}
\choice{The cosine factor decreases to $0$ faster than the polynomial.}
\choice[correct]{The cosine factor is bounded between $-1$ and $1$, so the polynomial forces the function to $0$.}
\choice{The cosine factor directly cancels out the polynomial factor.}
\end{multipleChoice}


What is the name of the theorem that applies to this problem? \qquad \\
The \underline{\answer{Squeeze}} Theorem
\end{problem}}%}

%%%%%%%%%%%%%%%%%%%%%




\latexProblemContent{
\begin{problem}

The limit as $x\to{-1}$ of $f(x)={{\left(x + 1\right)} \cos\left(-\frac{9}{{\left(x + 1\right)}^{2}}\right)}$ is $0$.  What is the reason why this is true?

\input{2311-Concept-Limit-0008.HELP.tex}

\begin{multipleChoice}
\choice{The statement is in fact false: $\lim\limits_{x\to{-1}}{{\left(x + 1\right)} \cos\left(-\frac{9}{{\left(x + 1\right)}^{2}}\right)}\neq0$.}
\choice{The cosine factor decreases to $0$ faster than the polynomial.}
\choice[correct]{The cosine factor is bounded between $-1$ and $1$, so the polynomial forces the function to $0$.}
\choice{The cosine factor directly cancels out the polynomial factor.}
\end{multipleChoice}


What is the name of the theorem that applies to this problem? \qquad \\
The \underline{\answer{Squeeze}} Theorem
\end{problem}}%}

%%%%%%%%%%%%%%%%%%%%%




\latexProblemContent{
\begin{problem}

The limit as $x\to{0}$ of $f(x)={x^{2} \cos\left(\frac{1}{x^{2}}\right)}$ is $0$.  What is the reason why this is true?

\input{2311-Concept-Limit-0008.HELP.tex}

\begin{multipleChoice}
\choice{The statement is in fact false: $\lim\limits_{x\to{0}}{x^{2} \cos\left(\frac{1}{x^{2}}\right)}\neq0$.}
\choice{The cosine factor decreases to $0$ faster than the polynomial.}
\choice[correct]{The cosine factor is bounded between $-1$ and $1$, so the polynomial forces the function to $0$.}
\choice{The cosine factor directly cancels out the polynomial factor.}
\end{multipleChoice}


What is the name of the theorem that applies to this problem? \qquad \\
The \underline{\answer{Squeeze}} Theorem
\end{problem}}%}

%%%%%%%%%%%%%%%%%%%%%




\latexProblemContent{
\begin{problem}

The limit as $x\to{5}$ of $f(x)={{\left(x - 5\right)}^{3} \cos\left(-\frac{22}{{\left(x - 5\right)}^{2}}\right)}$ is $0$.  What is the reason why this is true?

\input{2311-Concept-Limit-0008.HELP.tex}

\begin{multipleChoice}
\choice{The statement is in fact false: $\lim\limits_{x\to{5}}{{\left(x - 5\right)}^{3} \cos\left(-\frac{22}{{\left(x - 5\right)}^{2}}\right)}\neq0$.}
\choice{The cosine factor decreases to $0$ faster than the polynomial.}
\choice[correct]{The cosine factor is bounded between $-1$ and $1$, so the polynomial forces the function to $0$.}
\choice{The cosine factor directly cancels out the polynomial factor.}
\end{multipleChoice}


What is the name of the theorem that applies to this problem? \qquad \\
The \underline{\answer{Squeeze}} Theorem
\end{problem}}%}

%%%%%%%%%%%%%%%%%%%%%




\latexProblemContent{
\begin{problem}

The limit as $x\to{6}$ of $f(x)={{\left(x - 6\right)}^{3} \cos\left(\frac{7}{x - 6}\right)}$ is $0$.  What is the reason why this is true?

\input{2311-Concept-Limit-0008.HELP.tex}

\begin{multipleChoice}
\choice{The statement is in fact false: $\lim\limits_{x\to{6}}{{\left(x - 6\right)}^{3} \cos\left(\frac{7}{x - 6}\right)}\neq0$.}
\choice{The cosine factor decreases to $0$ faster than the polynomial.}
\choice[correct]{The cosine factor is bounded between $-1$ and $1$, so the polynomial forces the function to $0$.}
\choice{The cosine factor directly cancels out the polynomial factor.}
\end{multipleChoice}


What is the name of the theorem that applies to this problem? \qquad \\
The \underline{\answer{Squeeze}} Theorem
\end{problem}}%}

%%%%%%%%%%%%%%%%%%%%%




%%%%%%%%%%%%%%%%%%%%%




\latexProblemContent{
\begin{problem}

The limit as $x\to{-12}$ of $f(x)={{\left(x + 12\right)}^{2} \cos\left(\frac{24}{{\left(x + 12\right)}^{2}}\right)}$ is $0$.  What is the reason why this is true?

\input{2311-Concept-Limit-0008.HELP.tex}

\begin{multipleChoice}
\choice{The statement is in fact false: $\lim\limits_{x\to{-12}}{{\left(x + 12\right)}^{2} \cos\left(\frac{24}{{\left(x + 12\right)}^{2}}\right)}\neq0$.}
\choice{The cosine factor decreases to $0$ faster than the polynomial.}
\choice[correct]{The cosine factor is bounded between $-1$ and $1$, so the polynomial forces the function to $0$.}
\choice{The cosine factor directly cancels out the polynomial factor.}
\end{multipleChoice}


What is the name of the theorem that applies to this problem? \qquad \\
The \underline{\answer{Squeeze}} Theorem
\end{problem}}%}

%%%%%%%%%%%%%%%%%%%%%




\latexProblemContent{
\begin{problem}

The limit as $x\to{11}$ of $f(x)={{\left(x - 11\right)}^{3} \cos\left(-\frac{9}{x - 11}\right)}$ is $0$.  What is the reason why this is true?

\input{2311-Concept-Limit-0008.HELP.tex}

\begin{multipleChoice}
\choice{The statement is in fact false: $\lim\limits_{x\to{11}}{{\left(x - 11\right)}^{3} \cos\left(-\frac{9}{x - 11}\right)}\neq0$.}
\choice{The cosine factor decreases to $0$ faster than the polynomial.}
\choice[correct]{The cosine factor is bounded between $-1$ and $1$, so the polynomial forces the function to $0$.}
\choice{The cosine factor directly cancels out the polynomial factor.}
\end{multipleChoice}


What is the name of the theorem that applies to this problem? \qquad \\
The \underline{\answer{Squeeze}} Theorem
\end{problem}}%}

%%%%%%%%%%%%%%%%%%%%%




\latexProblemContent{
\begin{problem}

The limit as $x\to{-10}$ of $f(x)={{\left(x + 10\right)}^{2} \cos\left(-\frac{5}{x + 10}\right)}$ is $0$.  What is the reason why this is true?

\input{2311-Concept-Limit-0008.HELP.tex}

\begin{multipleChoice}
\choice{The statement is in fact false: $\lim\limits_{x\to{-10}}{{\left(x + 10\right)}^{2} \cos\left(-\frac{5}{x + 10}\right)}\neq0$.}
\choice{The cosine factor decreases to $0$ faster than the polynomial.}
\choice[correct]{The cosine factor is bounded between $-1$ and $1$, so the polynomial forces the function to $0$.}
\choice{The cosine factor directly cancels out the polynomial factor.}
\end{multipleChoice}


What is the name of the theorem that applies to this problem? \qquad \\
The \underline{\answer{Squeeze}} Theorem
\end{problem}}%}

%%%%%%%%%%%%%%%%%%%%%




\latexProblemContent{
\begin{problem}

The limit as $x\to{-1}$ of $f(x)={{\left(x + 1\right)} \cos\left(\frac{20}{{\left(x + 1\right)}^{2}}\right)}$ is $0$.  What is the reason why this is true?

\input{2311-Concept-Limit-0008.HELP.tex}

\begin{multipleChoice}
\choice{The statement is in fact false: $\lim\limits_{x\to{-1}}{{\left(x + 1\right)} \cos\left(\frac{20}{{\left(x + 1\right)}^{2}}\right)}\neq0$.}
\choice{The cosine factor decreases to $0$ faster than the polynomial.}
\choice[correct]{The cosine factor is bounded between $-1$ and $1$, so the polynomial forces the function to $0$.}
\choice{The cosine factor directly cancels out the polynomial factor.}
\end{multipleChoice}


What is the name of the theorem that applies to this problem? \qquad \\
The \underline{\answer{Squeeze}} Theorem
\end{problem}}%}

%%%%%%%%%%%%%%%%%%%%%




\latexProblemContent{
\begin{problem}

The limit as $x\to{2}$ of $f(x)={{\left(x - 2\right)}^{3} \cos\left(\frac{11}{x - 2}\right)}$ is $0$.  What is the reason why this is true?

\input{2311-Concept-Limit-0008.HELP.tex}

\begin{multipleChoice}
\choice{The statement is in fact false: $\lim\limits_{x\to{2}}{{\left(x - 2\right)}^{3} \cos\left(\frac{11}{x - 2}\right)}\neq0$.}
\choice{The cosine factor decreases to $0$ faster than the polynomial.}
\choice[correct]{The cosine factor is bounded between $-1$ and $1$, so the polynomial forces the function to $0$.}
\choice{The cosine factor directly cancels out the polynomial factor.}
\end{multipleChoice}


What is the name of the theorem that applies to this problem? \qquad \\
The \underline{\answer{Squeeze}} Theorem
\end{problem}}%}

%%%%%%%%%%%%%%%%%%%%%




\latexProblemContent{
\begin{problem}

The limit as $x\to{4}$ of $f(x)={{\left(x - 4\right)}^{3} \cos\left(\frac{8}{{\left(x - 4\right)}^{2}}\right)}$ is $0$.  What is the reason why this is true?

\input{2311-Concept-Limit-0008.HELP.tex}

\begin{multipleChoice}
\choice{The statement is in fact false: $\lim\limits_{x\to{4}}{{\left(x - 4\right)}^{3} \cos\left(\frac{8}{{\left(x - 4\right)}^{2}}\right)}\neq0$.}
\choice{The cosine factor decreases to $0$ faster than the polynomial.}
\choice[correct]{The cosine factor is bounded between $-1$ and $1$, so the polynomial forces the function to $0$.}
\choice{The cosine factor directly cancels out the polynomial factor.}
\end{multipleChoice}


What is the name of the theorem that applies to this problem? \qquad \\
The \underline{\answer{Squeeze}} Theorem
\end{problem}}%}

%%%%%%%%%%%%%%%%%%%%%




\latexProblemContent{
\begin{problem}

The limit as $x\to{14}$ of $f(x)={{\left(x - 14\right)}^{2} \cos\left(-\frac{4}{x - 14}\right)}$ is $0$.  What is the reason why this is true?

\input{2311-Concept-Limit-0008.HELP.tex}

\begin{multipleChoice}
\choice{The statement is in fact false: $\lim\limits_{x\to{14}}{{\left(x - 14\right)}^{2} \cos\left(-\frac{4}{x - 14}\right)}\neq0$.}
\choice{The cosine factor decreases to $0$ faster than the polynomial.}
\choice[correct]{The cosine factor is bounded between $-1$ and $1$, so the polynomial forces the function to $0$.}
\choice{The cosine factor directly cancels out the polynomial factor.}
\end{multipleChoice}


What is the name of the theorem that applies to this problem? \qquad \\
The \underline{\answer{Squeeze}} Theorem
\end{problem}}%}

%%%%%%%%%%%%%%%%%%%%%




\latexProblemContent{
\begin{problem}

The limit as $x\to{5}$ of $f(x)={{\left(x - 5\right)}^{2} \cos\left(\frac{12}{x - 5}\right)}$ is $0$.  What is the reason why this is true?

\input{2311-Concept-Limit-0008.HELP.tex}

\begin{multipleChoice}
\choice{The statement is in fact false: $\lim\limits_{x\to{5}}{{\left(x - 5\right)}^{2} \cos\left(\frac{12}{x - 5}\right)}\neq0$.}
\choice{The cosine factor decreases to $0$ faster than the polynomial.}
\choice[correct]{The cosine factor is bounded between $-1$ and $1$, so the polynomial forces the function to $0$.}
\choice{The cosine factor directly cancels out the polynomial factor.}
\end{multipleChoice}


What is the name of the theorem that applies to this problem? \qquad \\
The \underline{\answer{Squeeze}} Theorem
\end{problem}}%}

%%%%%%%%%%%%%%%%%%%%%




%%%%%%%%%%%%%%%%%%%%%




\latexProblemContent{
\begin{problem}

The limit as $x\to{-9}$ of $f(x)={{\left(x + 9\right)} \cos\left(\frac{5}{x + 9}\right)}$ is $0$.  What is the reason why this is true?

\input{2311-Concept-Limit-0008.HELP.tex}

\begin{multipleChoice}
\choice{The statement is in fact false: $\lim\limits_{x\to{-9}}{{\left(x + 9\right)} \cos\left(\frac{5}{x + 9}\right)}\neq0$.}
\choice{The cosine factor decreases to $0$ faster than the polynomial.}
\choice[correct]{The cosine factor is bounded between $-1$ and $1$, so the polynomial forces the function to $0$.}
\choice{The cosine factor directly cancels out the polynomial factor.}
\end{multipleChoice}


What is the name of the theorem that applies to this problem? \qquad \\
The \underline{\answer{Squeeze}} Theorem
\end{problem}}%}

%%%%%%%%%%%%%%%%%%%%%




\latexProblemContent{
\begin{problem}

The limit as $x\to{11}$ of $f(x)={{\left(x - 11\right)} \cos\left(\frac{8}{x - 11}\right)}$ is $0$.  What is the reason why this is true?

\input{2311-Concept-Limit-0008.HELP.tex}

\begin{multipleChoice}
\choice{The statement is in fact false: $\lim\limits_{x\to{11}}{{\left(x - 11\right)} \cos\left(\frac{8}{x - 11}\right)}\neq0$.}
\choice{The cosine factor decreases to $0$ faster than the polynomial.}
\choice[correct]{The cosine factor is bounded between $-1$ and $1$, so the polynomial forces the function to $0$.}
\choice{The cosine factor directly cancels out the polynomial factor.}
\end{multipleChoice}


What is the name of the theorem that applies to this problem? \qquad \\
The \underline{\answer{Squeeze}} Theorem
\end{problem}}%}

%%%%%%%%%%%%%%%%%%%%%




\latexProblemContent{
\begin{problem}

The limit as $x\to{-10}$ of $f(x)={{\left(x + 10\right)}^{2} \cos\left(\frac{5}{x + 10}\right)}$ is $0$.  What is the reason why this is true?

\input{2311-Concept-Limit-0008.HELP.tex}

\begin{multipleChoice}
\choice{The statement is in fact false: $\lim\limits_{x\to{-10}}{{\left(x + 10\right)}^{2} \cos\left(\frac{5}{x + 10}\right)}\neq0$.}
\choice{The cosine factor decreases to $0$ faster than the polynomial.}
\choice[correct]{The cosine factor is bounded between $-1$ and $1$, so the polynomial forces the function to $0$.}
\choice{The cosine factor directly cancels out the polynomial factor.}
\end{multipleChoice}


What is the name of the theorem that applies to this problem? \qquad \\
The \underline{\answer{Squeeze}} Theorem
\end{problem}}%}

%%%%%%%%%%%%%%%%%%%%%




\latexProblemContent{
\begin{problem}

The limit as $x\to{-9}$ of $f(x)={{\left(x + 9\right)}^{3} \cos\left(-\frac{17}{x + 9}\right)}$ is $0$.  What is the reason why this is true?

\input{2311-Concept-Limit-0008.HELP.tex}

\begin{multipleChoice}
\choice{The statement is in fact false: $\lim\limits_{x\to{-9}}{{\left(x + 9\right)}^{3} \cos\left(-\frac{17}{x + 9}\right)}\neq0$.}
\choice{The cosine factor decreases to $0$ faster than the polynomial.}
\choice[correct]{The cosine factor is bounded between $-1$ and $1$, so the polynomial forces the function to $0$.}
\choice{The cosine factor directly cancels out the polynomial factor.}
\end{multipleChoice}


What is the name of the theorem that applies to this problem? \qquad \\
The \underline{\answer{Squeeze}} Theorem
\end{problem}}%}

%%%%%%%%%%%%%%%%%%%%%




\latexProblemContent{
\begin{problem}

The limit as $x\to{11}$ of $f(x)={{\left(x - 11\right)}^{2} \cos\left(\frac{14}{x - 11}\right)}$ is $0$.  What is the reason why this is true?

\input{2311-Concept-Limit-0008.HELP.tex}

\begin{multipleChoice}
\choice{The statement is in fact false: $\lim\limits_{x\to{11}}{{\left(x - 11\right)}^{2} \cos\left(\frac{14}{x - 11}\right)}\neq0$.}
\choice{The cosine factor decreases to $0$ faster than the polynomial.}
\choice[correct]{The cosine factor is bounded between $-1$ and $1$, so the polynomial forces the function to $0$.}
\choice{The cosine factor directly cancels out the polynomial factor.}
\end{multipleChoice}


What is the name of the theorem that applies to this problem? \qquad \\
The \underline{\answer{Squeeze}} Theorem
\end{problem}}%}

%%%%%%%%%%%%%%%%%%%%%




\latexProblemContent{
\begin{problem}

The limit as $x\to{0}$ of $f(x)={x \cos\left(\frac{20}{x}\right)}$ is $0$.  What is the reason why this is true?

\input{2311-Concept-Limit-0008.HELP.tex}

\begin{multipleChoice}
\choice{The statement is in fact false: $\lim\limits_{x\to{0}}{x \cos\left(\frac{20}{x}\right)}\neq0$.}
\choice{The cosine factor decreases to $0$ faster than the polynomial.}
\choice[correct]{The cosine factor is bounded between $-1$ and $1$, so the polynomial forces the function to $0$.}
\choice{The cosine factor directly cancels out the polynomial factor.}
\end{multipleChoice}


What is the name of the theorem that applies to this problem? \qquad \\
The \underline{\answer{Squeeze}} Theorem
\end{problem}}%}

%%%%%%%%%%%%%%%%%%%%%




\latexProblemContent{
\begin{problem}

The limit as $x\to{-4}$ of $f(x)={{\left(x + 4\right)} \cos\left(\frac{20}{x + 4}\right)}$ is $0$.  What is the reason why this is true?

\input{2311-Concept-Limit-0008.HELP.tex}

\begin{multipleChoice}
\choice{The statement is in fact false: $\lim\limits_{x\to{-4}}{{\left(x + 4\right)} \cos\left(\frac{20}{x + 4}\right)}\neq0$.}
\choice{The cosine factor decreases to $0$ faster than the polynomial.}
\choice[correct]{The cosine factor is bounded between $-1$ and $1$, so the polynomial forces the function to $0$.}
\choice{The cosine factor directly cancels out the polynomial factor.}
\end{multipleChoice}


What is the name of the theorem that applies to this problem? \qquad \\
The \underline{\answer{Squeeze}} Theorem
\end{problem}}%}

%%%%%%%%%%%%%%%%%%%%%




\latexProblemContent{
\begin{problem}

The limit as $x\to{-14}$ of $f(x)={{\left(x + 14\right)}^{3} \cos\left(\frac{19}{x + 14}\right)}$ is $0$.  What is the reason why this is true?

\input{2311-Concept-Limit-0008.HELP.tex}

\begin{multipleChoice}
\choice{The statement is in fact false: $\lim\limits_{x\to{-14}}{{\left(x + 14\right)}^{3} \cos\left(\frac{19}{x + 14}\right)}\neq0$.}
\choice{The cosine factor decreases to $0$ faster than the polynomial.}
\choice[correct]{The cosine factor is bounded between $-1$ and $1$, so the polynomial forces the function to $0$.}
\choice{The cosine factor directly cancels out the polynomial factor.}
\end{multipleChoice}


What is the name of the theorem that applies to this problem? \qquad \\
The \underline{\answer{Squeeze}} Theorem
\end{problem}}%}

%%%%%%%%%%%%%%%%%%%%%




\latexProblemContent{
\begin{problem}

The limit as $x\to{-4}$ of $f(x)={{\left(x + 4\right)}^{3} \cos\left(-\frac{24}{x + 4}\right)}$ is $0$.  What is the reason why this is true?

\input{2311-Concept-Limit-0008.HELP.tex}

\begin{multipleChoice}
\choice{The statement is in fact false: $\lim\limits_{x\to{-4}}{{\left(x + 4\right)}^{3} \cos\left(-\frac{24}{x + 4}\right)}\neq0$.}
\choice{The cosine factor decreases to $0$ faster than the polynomial.}
\choice[correct]{The cosine factor is bounded between $-1$ and $1$, so the polynomial forces the function to $0$.}
\choice{The cosine factor directly cancels out the polynomial factor.}
\end{multipleChoice}


What is the name of the theorem that applies to this problem? \qquad \\
The \underline{\answer{Squeeze}} Theorem
\end{problem}}%}

%%%%%%%%%%%%%%%%%%%%%




\latexProblemContent{
\begin{problem}

The limit as $x\to{-5}$ of $f(x)={{\left(x + 5\right)} \cos\left(\frac{11}{x + 5}\right)}$ is $0$.  What is the reason why this is true?

\input{2311-Concept-Limit-0008.HELP.tex}

\begin{multipleChoice}
\choice{The statement is in fact false: $\lim\limits_{x\to{-5}}{{\left(x + 5\right)} \cos\left(\frac{11}{x + 5}\right)}\neq0$.}
\choice{The cosine factor decreases to $0$ faster than the polynomial.}
\choice[correct]{The cosine factor is bounded between $-1$ and $1$, so the polynomial forces the function to $0$.}
\choice{The cosine factor directly cancels out the polynomial factor.}
\end{multipleChoice}


What is the name of the theorem that applies to this problem? \qquad \\
The \underline{\answer{Squeeze}} Theorem
\end{problem}}%}

%%%%%%%%%%%%%%%%%%%%%




%%%%%%%%%%%%%%%%%%%%%




\latexProblemContent{
\begin{problem}

The limit as $x\to{-12}$ of $f(x)={{\left(x + 12\right)}^{2} \cos\left(-\frac{17}{x + 12}\right)}$ is $0$.  What is the reason why this is true?

\input{2311-Concept-Limit-0008.HELP.tex}

\begin{multipleChoice}
\choice{The statement is in fact false: $\lim\limits_{x\to{-12}}{{\left(x + 12\right)}^{2} \cos\left(-\frac{17}{x + 12}\right)}\neq0$.}
\choice{The cosine factor decreases to $0$ faster than the polynomial.}
\choice[correct]{The cosine factor is bounded between $-1$ and $1$, so the polynomial forces the function to $0$.}
\choice{The cosine factor directly cancels out the polynomial factor.}
\end{multipleChoice}


What is the name of the theorem that applies to this problem? \qquad \\
The \underline{\answer{Squeeze}} Theorem
\end{problem}}%}

%%%%%%%%%%%%%%%%%%%%%




\latexProblemContent{
\begin{problem}

The limit as $x\to{9}$ of $f(x)={{\left(x - 9\right)} \cos\left(\frac{19}{{\left(x - 9\right)}^{2}}\right)}$ is $0$.  What is the reason why this is true?

\input{2311-Concept-Limit-0008.HELP.tex}

\begin{multipleChoice}
\choice{The statement is in fact false: $\lim\limits_{x\to{9}}{{\left(x - 9\right)} \cos\left(\frac{19}{{\left(x - 9\right)}^{2}}\right)}\neq0$.}
\choice{The cosine factor decreases to $0$ faster than the polynomial.}
\choice[correct]{The cosine factor is bounded between $-1$ and $1$, so the polynomial forces the function to $0$.}
\choice{The cosine factor directly cancels out the polynomial factor.}
\end{multipleChoice}


What is the name of the theorem that applies to this problem? \qquad \\
The \underline{\answer{Squeeze}} Theorem
\end{problem}}%}

%%%%%%%%%%%%%%%%%%%%%




\latexProblemContent{
\begin{problem}

The limit as $x\to{3}$ of $f(x)={{\left(x - 3\right)} \cos\left(-\frac{21}{{\left(x - 3\right)}^{2}}\right)}$ is $0$.  What is the reason why this is true?

\input{2311-Concept-Limit-0008.HELP.tex}

\begin{multipleChoice}
\choice{The statement is in fact false: $\lim\limits_{x\to{3}}{{\left(x - 3\right)} \cos\left(-\frac{21}{{\left(x - 3\right)}^{2}}\right)}\neq0$.}
\choice{The cosine factor decreases to $0$ faster than the polynomial.}
\choice[correct]{The cosine factor is bounded between $-1$ and $1$, so the polynomial forces the function to $0$.}
\choice{The cosine factor directly cancels out the polynomial factor.}
\end{multipleChoice}


What is the name of the theorem that applies to this problem? \qquad \\
The \underline{\answer{Squeeze}} Theorem
\end{problem}}%}

%%%%%%%%%%%%%%%%%%%%%




\latexProblemContent{
\begin{problem}

The limit as $x\to{7}$ of $f(x)={{\left(x - 7\right)}^{2} \cos\left(\frac{13}{{\left(x - 7\right)}^{2}}\right)}$ is $0$.  What is the reason why this is true?

\input{2311-Concept-Limit-0008.HELP.tex}

\begin{multipleChoice}
\choice{The statement is in fact false: $\lim\limits_{x\to{7}}{{\left(x - 7\right)}^{2} \cos\left(\frac{13}{{\left(x - 7\right)}^{2}}\right)}\neq0$.}
\choice{The cosine factor decreases to $0$ faster than the polynomial.}
\choice[correct]{The cosine factor is bounded between $-1$ and $1$, so the polynomial forces the function to $0$.}
\choice{The cosine factor directly cancels out the polynomial factor.}
\end{multipleChoice}


What is the name of the theorem that applies to this problem? \qquad \\
The \underline{\answer{Squeeze}} Theorem
\end{problem}}%}

%%%%%%%%%%%%%%%%%%%%%




\latexProblemContent{
\begin{problem}

The limit as $x\to{-14}$ of $f(x)={{\left(x + 14\right)} \cos\left(\frac{3}{x + 14}\right)}$ is $0$.  What is the reason why this is true?

\input{2311-Concept-Limit-0008.HELP.tex}

\begin{multipleChoice}
\choice{The statement is in fact false: $\lim\limits_{x\to{-14}}{{\left(x + 14\right)} \cos\left(\frac{3}{x + 14}\right)}\neq0$.}
\choice{The cosine factor decreases to $0$ faster than the polynomial.}
\choice[correct]{The cosine factor is bounded between $-1$ and $1$, so the polynomial forces the function to $0$.}
\choice{The cosine factor directly cancels out the polynomial factor.}
\end{multipleChoice}


What is the name of the theorem that applies to this problem? \qquad \\
The \underline{\answer{Squeeze}} Theorem
\end{problem}}%}

%%%%%%%%%%%%%%%%%%%%%




\latexProblemContent{
\begin{problem}

The limit as $x\to{14}$ of $f(x)={{\left(x - 14\right)} \cos\left(\frac{24}{{\left(x - 14\right)}^{2}}\right)}$ is $0$.  What is the reason why this is true?

\input{2311-Concept-Limit-0008.HELP.tex}

\begin{multipleChoice}
\choice{The statement is in fact false: $\lim\limits_{x\to{14}}{{\left(x - 14\right)} \cos\left(\frac{24}{{\left(x - 14\right)}^{2}}\right)}\neq0$.}
\choice{The cosine factor decreases to $0$ faster than the polynomial.}
\choice[correct]{The cosine factor is bounded between $-1$ and $1$, so the polynomial forces the function to $0$.}
\choice{The cosine factor directly cancels out the polynomial factor.}
\end{multipleChoice}


What is the name of the theorem that applies to this problem? \qquad \\
The \underline{\answer{Squeeze}} Theorem
\end{problem}}%}

%%%%%%%%%%%%%%%%%%%%%




\latexProblemContent{
\begin{problem}

The limit as $x\to{-7}$ of $f(x)={{\left(x + 7\right)} \cos\left(\frac{1}{x + 7}\right)}$ is $0$.  What is the reason why this is true?

\input{2311-Concept-Limit-0008.HELP.tex}

\begin{multipleChoice}
\choice{The statement is in fact false: $\lim\limits_{x\to{-7}}{{\left(x + 7\right)} \cos\left(\frac{1}{x + 7}\right)}\neq0$.}
\choice{The cosine factor decreases to $0$ faster than the polynomial.}
\choice[correct]{The cosine factor is bounded between $-1$ and $1$, so the polynomial forces the function to $0$.}
\choice{The cosine factor directly cancels out the polynomial factor.}
\end{multipleChoice}


What is the name of the theorem that applies to this problem? \qquad \\
The \underline{\answer{Squeeze}} Theorem
\end{problem}}%}

%%%%%%%%%%%%%%%%%%%%%




\latexProblemContent{
\begin{problem}

The limit as $x\to{12}$ of $f(x)={{\left(x - 12\right)} \cos\left(\frac{10}{x - 12}\right)}$ is $0$.  What is the reason why this is true?

\input{2311-Concept-Limit-0008.HELP.tex}

\begin{multipleChoice}
\choice{The statement is in fact false: $\lim\limits_{x\to{12}}{{\left(x - 12\right)} \cos\left(\frac{10}{x - 12}\right)}\neq0$.}
\choice{The cosine factor decreases to $0$ faster than the polynomial.}
\choice[correct]{The cosine factor is bounded between $-1$ and $1$, so the polynomial forces the function to $0$.}
\choice{The cosine factor directly cancels out the polynomial factor.}
\end{multipleChoice}


What is the name of the theorem that applies to this problem? \qquad \\
The \underline{\answer{Squeeze}} Theorem
\end{problem}}%}

%%%%%%%%%%%%%%%%%%%%%

