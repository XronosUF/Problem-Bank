%%%%%%%%%%%%%%%%%%%%%
%\tagged{Ans@ShortAns, Type@Compute, Topic@Integral, Sub@TrigSub, Func@Arctrig, File@0080}{
\begin{sagesilent}
# Define coefficients, powers, etc
A = NonZeroInt(-5,5)
B = RandInt(-5,5)
C = NonZeroInt(-10,10)


# Vector of possible arctrig functions
funcvec = [arcsin(x), arccos(x), arctan(x), arcsec(x)]
p = RandInt(0,3)

# Select our arctrig function, then take the derivative to get the integrand.
Func1 = funcvec[p]
F(x) = C*Func1(A*x + B)

Integrand(x) = derivative(F(x),x)

\end{sagesilent}

\latexProblemContent{
\ifVerboseLocation This is Integration Compute Question 0080. \\ \fi
\begin{problem}
Evaluate the integral
\input{Integral-Compute-0080.HELP.tex}
\[
\int \sage{Integrand(x)} dx = \answer[validator=sameDerivative]{\sage{F(x)} + C}
\]


\end{problem}}%}
%%%%%%%%%%%%%%%%%%%%


%%%%%%%%%%%%%%%%%%%%%
%\tagged{Ans@ShortAns, Type@Compute, Topic@Integral, Sub@ByParts, Sub@Guided, Func@Exp, Func@Trig, File@0101}{

\begin{sagesilent}
a = NonZeroInt(-5,5)
b = NonZeroInt(-5,5)
c = NonZeroInt(-5,5)
pwr = RandInt(1,3)

IntFuncVec = [e^x,sin(x),cos(x)]

p0 = RandInt(0,2)

df(x) = a*IntFuncVec[p0](b*x+c)
f(x) = integral(df(x),x)
G(x) = x^pwr 
g(x) = derivative(G(x),x)
h(x) = df(x)*G(x)

Ans(x) = HyperSimp(integral(h(x),x))

\end{sagesilent}


\latexProblemContent{
\ifVerboseLocation This is Integration Compute Question 0101. \\ \fi
\begin{problem}
We aim to calculate the following improper integral:
\[
\int \sage{h(x)} dx. 
\]
\input{Integral-Compute-0101.HELP.tex}

First, need to determine which integration technique would be appropriate here. 

\begin{multipleChoice}
\choice{We should just integrate directly, integrating the monomial term $\sage{G(x)}$ and the extra term $\sage{df(x)}$ separately to get our integral.}
\choice{We should integrate using (only) a $u$-substitution on the inner function: $u = \sage{b*x+c}$, and rewriting the $\sage{G(x)}$ in terms of $u$ }
\choice{We should integrate using (only) a $u$-substitution of: $u = \sage{G(x)}$, and rewriting the exponent of $\sage{df(x)}$ in terms of $u$}
\choice[correct]{We should use Integration by parts, using $u = \sage{G(x)}$ and $dv = \sage{df(x)}$}
\choice{We will should Integration by parts, using $u = \sage{df(x)}$ and $dv = \sage{G(x)}$}
\end{multipleChoice}

\begin{problem}
Now that we have determined we need to use integration by parts (and chosen $u$ and $dv$), we need to compute $du$ and $v$;

\[
u = \sage{G(x)} \hspace{1in} dv = \sage{df(x)}
\]
\[
du = \answer{\sage{g(x)}} \hspace{1in} v = \answer{\sage{f(x)}} 
\]

\begin{problem}

Next we can determine the result of the integration by parts step.
\[
\int \sage{h(x)} = \answer{\sage{G(x)*f(x)}} - \int \answer{\sage{g(x)*f(x)}}dx
\]

\begin{problem}

Now, using the previous steps, we can calculate our answer!
\[
\int \sage{h(x)} dx = \answer[validator=sameDerivative]{\sage{Ans(x)} + C}
\]

\end{problem}
\end{problem}
\end{problem}
\end{problem}}%}

%%%%%%%%%%%%%%%%%%%%%%



%%%%%%%%%%%%%%%%%%%%%%
%%\tagged{Ans@Long, Type@Compute, Topic@Integral, Sub@Guided, Sub@TrigSub, File@0102}{

\begin{sagesilent}
var('a,b,c,theta')
a1 = RandInt(-7,7)
b1 = RandInt(1,7)
flip = RandInt(0,1)
s1 = (-1)^flip
flip2 = RandInt(0,1-flip)
s2 = (-1)^flip2

xpwr = RandInt(-3,3)

rad1 = s2*(x-a1)
rad2 = s1*b1^2

radicand(x) = expand(s2*(x-a1)^2 + s1*b1^2)

if s1 < 0:
    pythForm = (x-b)^2 - a^2
    fake1 = (x-b)^2 + a^2
    fake2 = a^2 - (x-b)^2
    sub(z) = b1*sec(z)-a1
    radsimp(x) = b1*tan(x)
elif s2 < 0:
    pythForm = a^2 - (x-b)^2
    fake1 = (x-b)^2 - a^2
    fake2 = (x-b)^2 + a^2
    sub(z)=b1*sin(z)-a1
    radsimp(x) = b1*cos(x)
else:
    pythForm = a^2 + (x-b)^2
    fake1 = (x-b)^2 - a^2
    fake2 = a^2 - (x-b)^2
    sub(z) = b1*tan(z)-a1
    radsimp(x) = b1*sec(x)

integrandTheta(x) = (sub(x)^xpwr)/radsimp(x)
integrandTheta(x) = factor(integrandTheta(x))

intTheta(x) = integral(integrandTheta(x),x)

dtheta(z) = derivative(sub(z),z)
Integrand = x^xpwr*(radicand(x)^(1/2))
Ans = integral(Integrand(x),x)
Ans = HyperSimp(factor(Ans))

\end{sagesilent}

\latexProblemContent{
\ifVerboseLocation This is Integration Compute Question 0102. \\ \fi
\begin{problem}

We aim to calculate the following integral:

\[
\int \sage{Integrand(x)} dx. 
\]
\input{Integral-Compute-0102.HELP.tex}

First, we must complete the square in order to choose the Pythagorean expression the function inside the square root resembles. Complete the square and leave answer in terms of $x$.

\[
\begin{array}{lr}
\sage{radicand(x)}= (\answer{\sage{rad1}})^2 + \answer{\sage{rad2}} 
\end{array}
\]

\begin{problem}
First, determine the Pythagorean expression the function inside the square root resembles. 

\begin{multipleChoice}
\choice{$\sage{fake1}$}
\choice[correct]{$\sage{pythForm}$}
\choice{$\sage{fake2}$}
\end{multipleChoice}
\begin{problem}
In this case, 

\[
\begin{array}{lr}
a = \answer{\sage{b1}} 
\end{array}
\]

\begin{problem}
This suggests the substitution should be

\[
\begin{array}{lr}
x = \answer{\sage{sub(theta)}}  \\
dx = \answer{\sage{dtheta(theta)}} d\theta
\end{array}
\]

\begin{problem}
Therefore, we rid the square root in the integrand by using the following substitution

\[
\begin{array}{lr}
 \sqrt{\sage{radicand(x)}} = \answer{\sage{radsimp(theta)}}
\end{array}
\]

\begin{problem}
Using our trig substitution, we can calculate the integral in terms of $\theta$

 \[
\int \sage{Integrand(x)} dx  = \answer[validator=sameDerivative]{\sage{intTheta(theta)}+ C}
\]

\begin{problem}
Now, using the previous steps, we can calculate our final answer in terms of $x$!
\[
\int  \sage{Integrand(x)} dx  = \answer[validator=sameDerivative]{\sage{Ans} + C}
\]

\end{problem}
\end{problem}
\end{problem}
\end{problem}
\end{problem}
\end{problem}
\end{problem}}%}
                     
%%%%%%%%%%%%%%%%%%%%%%



%%%%%%%%%%%%%%%%%%%%%%
%%\tagged{Ans@Long, Type@Compute, Topic@Integral, Sub@Guided, Sub@TrigSub, File@0103}{

\begin{sagesilent}
var('a,b,c,theta')
a1 = RandInt(1,10)
b1 = RandInt(1,7)
flip = RandInt(0,1)
s1 = (-1)^flip
flip2 = RandInt(0,1-flip)
s2 = (-1)^flip2

xpwr = RandInt(-3,3)

radicand(x) = s2*(a1*x)^2 + s1*b1^2

if s1 < 0:
    pythForm = (b*x)^2 - a^2
    fake1 = (b*x)^2 + a^2
    fake2 = a^2 - (b*x)^2
    sub(z) = b1*1/a1*sec(z)
    radsimp(x) = b1*tan(x)
elif s2 < 0:
    pythForm = a^2 - (b*x)^2
    fake1 = (b*x)^2 - a^2
    fake2 = (b*x)^2 + a^2
    sub(z)=b1*1/a1*sin(z)
    radsimp(x) = b1*cos(x)
else:
    pythForm = a^2 + (b*x)^2
    fake1 = (b*x)^2 - a^2
    fake2 = a^2 - (b*x)^2
    sub(z) = b1*1/a1*tan(z)
    radsimp(x) = b1*sec(x)

integrandTheta(x) = (sub(x)^xpwr)/radsimp(x)
intTheta(x) = integral(integrandTheta(x),x)

dtheta(z) = derivative(sub(z),z)
Integrand = x^xpwr/(radicand(x)^(1/2))
Ans = HyperSimp(integral(Integrand(x),x))

\end{sagesilent}

\latexProblemContent{
\ifVerboseLocation This is Integration Compute Question 0103. \\ \fi

\begin{problem}
We wish to calculate the following integral:
\[
\int \sage{Integrand(x)} dx. 
\]
\input{Integral-Compute-0103.HELP.tex}

First, determine the Pythagorean expression the function inside the square root resembles. 
\begin{multipleChoice}
\choice{$\sage{fake1}$}
\choice[correct]{$\sage{pythForm}$}
\choice{$\sage{fake2}$}
\end{multipleChoice}
\begin{problem}
In this case, 
\[
\begin{array}{lr}
a = \answer{\sage{b1}} 
\end{array}
\]
\begin{problem}
This suggests the substitution should be
\[
\begin{array}{lr}
x = \answer{\sage{sub(theta)}}  \\
dx = \answer{\sage{dtheta(theta)}} d\theta
\end{array}
\]
\begin{problem}
Therefore, we rid the square root in the integrand by using the following substitution
\[
\begin{array}{lr}
 \sqrt{\sage{radicand(x)}} = \answer{\sage{radsimp(theta)}}
\end{array}
\]
\begin{problem}
Using our trig substitution, we can calculate the integral in terms of $\theta$
 \[
\int \sage{Integrand(x)} dx  = \answer[validator=sameDerivative]{\sage{intTheta(theta)}+ C}
\]
\begin{problem}
Now, using the previous steps, we can calculate our final answer in terms of $x$!
\[
\int  \sage{Integrand(x)} dx  = \answer[validator=sameDerivative]{\sage{Ans} + C}
\]
\end{problem}
\end{problem}
\end{problem}
\end{problem}
\end{problem}
\end{problem}}%}

%%%%%%%%%%%%%%%%%%%%%%


%%%%%%%%%%%%%%%%%%%%%%
%%\tagged{Ans@Long, Type@Compute, Topic@Integral, Sub@TrigInt, Sub@Guided, Func@Trig, File@0104}{


\begin{sagesilent}

funcVec = [tan(x), sec(x)]

p = RandInt(0,1)

evenfunc = funcVec[p]
oddfunc = funcVec[1-p]

if p == 0:
    a1 = RandInt(0,1)
    pwr1 = 2*a1
    a2 = RandInt(0,2-a1)
    pwr2 = 2*a2+1
    usub(x) = (oddfunc^(2*a2+1)*(oddfunc^2-1)^(a1))
    identL(x) = (tan(x))^2 
    identR(x) = (sec(x))^2 - 1
else:
    a1 = RandInt(1,4)
    pwr1 = 2*a1
    a2 = RandInt(0,3)
    pwr2 = 2*a2+1
    usub(x) = (oddfunc^(2*a2+1)*(oddfunc^2+1)^(a1-1)*evenfunc^2)
    identL(x) = (sec(x))^2
    identR(x) = (tan(x))^2 + 1


pwrVec = [pwr1, pwr2]



Integrand(x) = evenfunc(x)^pwr1*oddfunc(x)^pwr2

Ans = HyperSimp(integral(Integrand(x),x))

\end{sagesilent}


\latexProblemContent{
\ifVerboseLocation This is Integration Compute Question 0104. \\ \fi

\begin{problem}
We wish to calculate the following Trigonometric Integral:
\[
\int \sage{Integrand(x)} \mathop{dx}.
\]

\input{Integral-Compute-0104.HELP.tex}
The powers of $\sage{evenfunc}$ and $\sage{oddfunc}$ in the integrand are $\answer{\sage{pwr1}}$ and $\answer{\sage{pwr2}}$ respectively. 
\begin{problem}
Since the power for $\sage{evenfunc}$ is even and the power for $\sage{oddfunc}$ is odd, the best course of action is:

\begin{multipleChoice}
    \choice{We should use Integration by Parts with $u = \sage{evenfunc^pwr1}$ and $dv = \sage{oddfunc^pwr2}$.}
    \choice{We should use Integration by Parts with $u = \sage{oddfunc^pwr2}$ and $dv = \sage{evenfunc^pwr1}$.}
    \choice[correct]{We should use a Pythagorean identity to convert some (or all) of the $\sage{evenfunc}$ into some $\sage{oddfunc}$}
    \choice{We should convert the entire expression to sines and cosines before attempting the integrate directly.}
\end{multipleChoice}

\begin{problem}

What variant of the Pythagorean identity do we need to utilize?

\[
\sage{identL(x)}=\answer{\sage{identR(x)}}
\]

\begin{problem}
Utilizing the above identity and converting our integrand into something integrable we have:

\[
\int   \answer{\sage{usub(x)}} \mathop{dx}.
\]
\begin{problem}
Finally, we have that
\[
	\int \sage{Integrand(x)} \mathop{dx} = \answer[validator=sameDerivative]{\sage{Ans(x)}+C}
\]
\end{problem}
\end{problem}
\end{problem}
\end{problem}
\end{problem}}%}

%%%%%%%%%%%%%%%%%%%%%%




%%%%%%%%%%%%%%%%%%%%%%
%%\tagged{Ans@Long, Type@Compute, Topic@Integral, Sub@TrigInt, Sub@Guided, Func@Trig, File@0105}{

\begin{sagesilent}
a1 = RandInt(1,3)
a2 = RandInt(0,3)
pwr1 = 2*a1
pwr2 = 2*a2+1

usub(u) = expand(u^(2*a1)*(1-u^2)^(a2))


pwrVec = [pwr1, pwr2]
funcVec = [sin(x), cos(x)]

p = RandInt(0,1)

evenfunc = funcVec[p]
oddfunc = funcVec[1-p]

Integrand(x) = evenfunc(x)^pwr1*oddfunc(x)^pwr2

Ans = HyperSimp(integral(Integrand(x),x))

\end{sagesilent}


\latexProblemContent{
\ifVerboseLocation This is Integration Compute Question 0105. \\ \fi

\begin{problem}
We wish to calculate the following Trigonometric Integral:
\[
\int \sage{Integrand(x)} \mathop{dx}.
\]
\input{Integral-Compute-0105.HELP.tex}

The powers of $\sage{evenfunc}$ and $\sage{oddfunc}$ in the integrand are $\answer{\sage{pwr1}}$ and $\answer{\sage{pwr2}}$ respectively. 

\begin{problem}

Since the power for $\sage{evenfunc}$ is even and the power for $\sage{oddfunc}$ is odd, the best course of action is:

\begin{multipleChoice}
\choice[correct]{We should save a $\sage{oddfunc}$ and convert the remaining $\sage{oddfunc}$ to $\sage{evenfunc}$ via the Pythagorean identity, then follow up with a $u$-substitution}
\choice{We should save a $\sage{evenfunc}$ and convert the remaining $\sage{evenfunc}$ to $\sage{oddfunc}$ via the Pythagorean identity, then follow up with a $u$-substitution }
\choice{We should use Integration by parts, using $u = \sage{oddfunc(x)^pwr2}$ and $dv = \sage{evenfunc(x)^pwr1}$}
\choice{We should use Integration by parts, using $u = \sage{evenfunc(x)^pwr1}$ and $dv = \sage{oddfunc(x)^pwr2}$}
\end{multipleChoice}

\begin{problem}
Now that we have saved a $\sage{oddfunc}$ and converted the remaining $\sage{oddfunc}$ to $\sage{evenfunc}$ via the Pythagorean identity, the appropriate $u$-substitution will turn our integral into the following one

\[
    \int \answer{\sage{usub(u)}} \mathop{du}
\]
\begin{problem}

After applying the power rule and back-substituting to the variable $x$, our final answer is: 
\[
\int \sage{Integrand(x)} \mathop{dx}= \answer[validator=sameDerivative]{\sage{Ans}}
\]

\end{problem}
\end{problem}
\end{problem}
\end{problem}}%}

%%%%%%%%%%%%%%%%%%%%%%


%%%%%%%%%%%%%%%%%%%%%%
%%\tagged{Ans@Long, Type@Compute, Topic@Integral, Sub@PartialFractions, Sub@Guided, Func@Rational, Func@Poly, File@0106}{
\begin{sagesilent}
a = NonZeroInt(-5,5)
b = RandInt(1,10)
c = RandInt(-3,3)
a1 = NonZeroInt(-5,5)
b1 = NonZeroInt(-5,5)

f1denom = x-a
f2denom = x^2+b

f1 = a1/(x-a)
f2 = (c*x+b1)/(x^2+b)

Integrand = factor((f1+f2).simplify_rational())

Ans = HyperSimp(integral(f2,x))

\end{sagesilent}


\latexProblemContent{
\ifVerboseLocation This is Integration Compute Question 0106. \\ \fi

\begin{problem}
We wish to calculate the following integral:
\[
\int \sage{Integrand} dx. 
\]
\input{Integral-Compute-0106.HELP.tex}

First, determine the Practial Fraction Decomposition of  $\sage{Integrand}$% DEBUG: $\sage{f1}=f1$ and $\sage{f2}=f2$

\[
\begin{array}{lr}
\sage{Integrand} = \dfrac{A}{ \answer{\sage{f1denom}} } + \dfrac{Bx+C}{\answer{\sage{f2denom}} }
\end{array}
\]
\begin{problem}
Now, determine the constant coefficients, $A$, $B$, and $C$. 

 \[
\begin{array}{lr}
A = \answer{\sage{a1}}  \\
B = \answer{\sage{c}}\\
C = \answer{\sage{b1}}
\end{array}
\]
\begin{problem}
Now we can rewrite the integral

\[
\begin{array}{lr}
 \int \sage{Integrand} dx = \int \answer{ \sage{f1 + f2} } dx 
\end{array}
\]

\begin{problem}

Now, using the previous steps, we can calculate the integral:
\[
\int \sage{Integrand} dx = \answer[validator=sameDerivative]{  \sage{a1*log(abs(x-a))} + \sage{Ans} +C }
\]
\end{problem}
\end{problem}
\end{problem}
\end{problem}}%}
%%%%%%%%%%%%%%%%%%%%%%



%%%%%%%%%%%%%%%%%%%%%%
%%\tagged{Ans@MC, Type@Compute, Topic@Integral, Sub@Improper, File@0081}{
%\begin{sagesilent}
%# Define coefficients, powers, etc
%A11 = NonZeroInt(1,5)
%B11 = NonZeroInt(1,5)
%A12 = NonZeroInt(1,5)
%B12 = NonZeroInt(1,5)
%A21 = NonZeroInt(1,5)
%B21 = NonZeroInt(1,5)
%A22 = NonZeroInt(1,5)
%B22 = NonZeroInt(1,5)
%A31 = NonZeroInt(1,5)
%B31 = NonZeroInt(1,5)
%A32 = NonZeroInt(1,5)
%B32 = NonZeroInt(1,5)
%A41 = NonZeroInt(1,5)
%B41 = NonZeroInt(1,5)
%A42 = NonZeroInt(1,5)
%B42 = NonZeroInt(1,5)
%
%pwr = RandInt(2,6)
%pwr2 = RandInt(pwr+2,10)
%
%slowvec = [ln(x), x^pwr, x^(1/(2*pwr+1))]
%fastvec = [e^(x), x^pwr2] 
%funcvec = [ln(x), x^(1/(2*pwr+1)), x^pwr, x^pwr2, e^x]
%
%selectslow = RandInt(0,2)
%selectfast = RandInt(0,1)
%select11 = RandInt(0,4)
%select21 = RandInt(0,4)
%select31 = RandInt(0,4)
%select41 = RandInt(0,4)
%select12 = RandInt(0,4)
%select22 = RandInt(0,4)
%select32 = RandInt(0,4)
%select42 = RandInt(0,4)
%
%Fslow = slowvec[selectslow]
%Ffast = fastvec[selectfast]
%
%F1 = Fslow/Ffast
%F2 = funcvec[select11](A11*x + B11)/funcvec[select12](A12*x + B12)
%F3 = funcvec[select21](A21*x + B21)/funcvec[select22](A22*x + B22)
%F4 = funcvec[select31](A31*x + B31)/funcvec[select32](A32*x + B32)
%F5 = funcvec[select41](A41*x + B41)/funcvec[select42](A42*x + B42)
%
%
%if select11 < select12:
%    ans2T = "correct"
%else:
%    ans2T = ""
%
%if select21 < select22:
%    ans3T = "correct"
%else:
%    ans3T = ""
%
%if select31 < select32:
%    ans4T = "correct"
%else:
%    ans4T = ""
%
%if select41 < select42:
%    ans5T = "correct"
%else:
%    ans5T = ""
%
%# This will look bad on the pdf, but I think once the merging programs run it will come out correct after the \sage command gets removed. Will probably need to do a replace-all to remove the "\texttt" part of the \texttt{correct} NEEDS TO BE TESTED.
%
%\end{sagesilent}
%
%\latexProblemContent{
%\begin{problem}
%Which of the following is a correct statement? (p \textit{all} that apply)
%\input{Integral-Compute-0081.HELP.tex}
%\begin{multipleChoice}
%\choice[correct]{$\int\limits_{0}^\infty \sage{F1(x)} dx$}
%\choice[$\sage{ans2T}$]{$\int\limits_{0}^\infty \sage{F2(x)} dx$}
%\choice[$\sage{ans3T}$]{$\int\limits_{0}^\infty \sage{F3(x)} dx$}
%\choice[$\sage{ans4T}$]{$\int\limits_{0}^\infty \sage{F4(x)} dx$}
%\choice[$\sage{ans5T}$]{$\int\limits_{0}^\infty \sage{F5(x)} dx$}
%\end{multipleChoice}
%
%
%\end{problem}}%}
%%%%%%%%%%%%%%%%%%%%%%
%
%
%
%
%
%
%
%
%%%%%%%%%%%%%%%%%%%%%%%%%%%%%%%%%%%%%%%%%%%%%%%%%%%%%%%%%%%%%%%%%%%%%%%%%%%%%%%%
%%
%%
%%
%%
%%
%%
%%
%%
%%%%%%%%%%%%%%%%%%%%%%%%%%%%%%%%%%%%%%%%%%%%%%%%%%%%%%%%%%%%%%%%%%%%%%%%%%%%%%%%
%%%%%%%%%%%%%%%%%%%%%%%%%%%%%%%%%%%%%%%%%%%%%%%%%%%%%%%%%%%%%%%%%%%%%%%%%%%%%%%%
%%%%%%%%%%%%%%%%%%%%										%%%%%%%%%%%%%%%%%%%%%%%
%%%%%%%%%%%%%%%%%%%%				Concept					%%%%%%%%%%%%%%%%%%%%%%%
%%%%%%%%%%%%%%%%%%%%										%%%%%%%%%%%%%%%%%%%%%%%
%%%%%%%%%%%%%%%%%%%%%%%%%%%%%%%%%%%%%%%%%%%%%%%%%%%%%%%%%%%%%%%%%%%%%%%%%%%%%%%%
%%%%%%%%%%%%%%%%%%%%%%%%%%%%%%%%%%%%%%%%%%%%%%%%%%%%%%%%%%%%%%%%%%%%%%%%%%%%%%%%
%%
%%
%%
%%
%%
%%
%%
%%
%%
%%%%%%%%%%%%%%%%%%%%%%%%%%%%%%%%%%%%%%%%%%%%%%%%%%%%%%%%%%%%%%%%%%%%%%%%%%%%%%%
%
%
%
%
%%%%%%%%%%%%%%%%%%%%%%%%%%%%%%%%%%%%%%%%%%%%%%%%%%%%%%%%%%%%%%%%%%%%%%%%%%%%%%%
%%%%%%%%%%%%%%%%%%%				Calc 1 Concept			%%%%%%%%%%%%%%%%%%%%%%%
%%%%%%%%%%%%%%%%%%%%%%%%%%%%%%%%%%%%%%%%%%%%%%%%%%%%%%%%%%%%%%%%%%%%%%%%%%%%%%%


%%%%%%%%%%%%%%%%%%%%%%%
%%\tagged{Ans@MC, Type@Concept, Topic@Integral, Sub@Definite, Sub@Theorems-FTC, Func@Rational, File@0001}{
\begin{sagesilent}
a = NonZeroInt(-10,10)
l = NonZeroInt(-6,-1)  
u = NonZeroInt(1,6)

p = RandInt(0,3)
v = [a/x, a/x^2, a/x^3, a/x^4]
F = v[p]
f = integrate(F,x)
Ans = f(u)-f(l)
\end{sagesilent}

\latexProblemContent{
\ifVerboseLocation This is Integration Concept Question 0001. \\ \fi
\begin{problem}

What is wrong with the following equation:

\[
\int_{\sage{l}}^{\sage{u}} \sage{F}\;dx = \sage{f}\Bigg\vert_{\sage{l}}^{\sage{u}} = \sage{Ans}
\]

\input{Integral-Concept-0001.HELP.tex}

\begin{multipleChoice}
\choice{The bounds are evaluated in the wrong order.}
\choice{The antiderivative is incorrect.}
\choice[correct]{The integrand is not defined over the entire interval.}
\choice{Nothing is wrong.  The equation is correct, as is.}
\end{multipleChoice}

\end{problem}}%}
%%%%%%%%%%%%%%%%%%%%%%


%%%%%%%%%%%%%%%%%%%%%%%
%%\tagged{Ans@MC, Type@Concept, Topic@Integral, Sub@Definite, Sub@Theorems-FTC, Func@Trig, File@0002}{
\begin{sagesilent}
b = NonZeroInt(-10,10)
l = RandAng(0,pi/3)
u = RandAng(2*pi/3,pi)

p = RandInt(0,1)
vt = [b*sec(x)*tan(x), b*sec(x)^2]
F = vt[p]
f = integrate(F,x)
Ans = f(u)-f(l)
\end{sagesilent}

\latexProblemContent{
\ifVerboseLocation This is Integration Concept Question 0002. \\ \fi
\begin{problem}

What is wrong with the following equation:

\[
\int_{\sage{l}}^{\sage{u}} \sage{F}\;dx = \sage{f}\Bigg\vert_{\sage{l}}^{\sage{u}} = \sage{Ans}
\]

\input{Integral-Concept-0002.HELP.tex}

\begin{multipleChoice}
\choice{The antiderivative is incorrect.}
\choice[correct]{The integrand is not defined over the entire interval.}
\choice{The bounds are evaluated in the wrong order.}
\choice{Nothing is wrong.  The equation is correct, as is.}
\end{multipleChoice}

\end{problem}}%}
%%%%%%%%%%%%%%%%%%%%%%



%%%%%%%%%%%%%%%%%%%%%%%
%%\tagged{Ans@MC, Type@Concept, Topic@Integral, Sub@Definite, Sub@Theorems-FTC, Func@Trig, File@0003}{
\begin{sagesilent}
b = NonZeroInt(-10,10)
l = RandAng(pi/6,5*pi/6)
u = RandAng(7*pi/6,11*pi/6)

p = RandInt(0,1)
vt = [b*csc(x)^2, b*csc(x)*cot(x)]
F = vt[p]
f = integrate(F,x)
Ans = f(u)-f(l)
\end{sagesilent}

\latexProblemContent{
\ifVerboseLocation This is Integration Concept Question 0003. \\ \fi
\begin{problem}

What is wrong with the following equation:

\[
\int_{\sage{l}}^{\sage{u}} \sage{F}\;dx = \sage{f}\Bigg\vert_{\sage{l}}^{\sage{u}} = \sage{Ans}
\]

\input{Integral-Concept-0003.HELP.tex}

\begin{multipleChoice}
\choice{The antiderivative is incorrect.}
\choice[correct]{The integrand is not defined over the entire interval.}
\choice{The bounds are evaluated in the wrong order.}
\choice{Nothing is wrong.  The equation is correct, as is.}
\end{multipleChoice}

\end{problem}}%}
%%%%%%%%%%%%%%%%%%%%%%
