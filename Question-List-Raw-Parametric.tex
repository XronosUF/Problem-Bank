%%%%%%%%%%%%%%%%%%%%%%%%%%
%%%%%%%%%%%%%%%%%%%%%%%%%%



%%%%%%%%%%%%%%%%%%%%%%%
%%\tagged{Ans@ShortAns, Type@Compute, Topic@Parametric, Sub@ParametricEq, File@0001}{
\begin{sagesilent}
# Define variables and constants/exponents
var('x,t')
a=NonZeroInt(-10,10)
b=RandInt(-10,10)
p=RandInt(0,6)

#Define the general vector for parameterization
v1=[a*t+b, a*t^2+b, a*t^3+b, a*exp(t)+b, a*log(t)+b, a*sin(t)+b, a*cos(t)+b]
xT=t
yT=v1[p]

#Define the Answer function
Ans=yT(x)

\end{sagesilent}

\latexProblemContent{
\ifVerboseLocation This is Parametric Compute Question 0001. \\ \fi
\begin{problem}
Given the following parameterization, determine the associated curve being traced.

\[\left\{\begin{array}{lll}
x(t) &=& \sage{xT}\\
y(t) &=& \sage{yT}
\end{array}\right.,\quad t>0\]

\input{Parametric-Compute-0001.HELP.tex}

\[y=\answer{\sage{Ans}}\]

\end{problem}}%}
%%%%%%%%%%%%%%%%%%%%%%

%%%%%%%%%%%%%%%%%%%%%%%
%%\tagged{Ans@ShortAns, Type@Compute, Topic@Parametric, Sub@ParametricEq, File@0002}{
\begin{sagesilent}
# Define variables and constants/exponents
var('x,t')
a=NonZeroInt(-10,10)
p=RandInt(0,6)

#Define the general vector for parameterization
v2=[t^2, t^3, 1/t, 1/t^2, exp(t), log(t), sin(t), cos(t)]
xT=a*t
yT=v2[p]

#Define the Answer function
Ans=yT(x/a)

\end{sagesilent}

\latexProblemContent{
\ifVerboseLocation This is Parametric Compute Question 0002. \\ \fi
\begin{problem}
Given the following parameterization, determine the associated curve being traced.

\[\left\{\begin{array}{lll}
x(t) &=& \sage{xT}\\
y(t) &=& \sage{yT}
\end{array}\right.,\quad t>0\]

\input{Parametric-Compute-0002.HELP.tex}

\[y=\answer{\sage{Ans}}\]

\end{problem}}%}
%%%%%%%%%%%%%%%%%%%%%%

%%%%%%%%%%%%%%%%%%%%%%%
%%\tagged{Ans@ShortAns, Type@Compute, Topic@Parametric, Sub@ParametricEq, File@0003}{
\begin{sagesilent}
# Define variables and constants/exponents
var('x,t')
a=NonZeroInt(-10,10)
b=RandInt(-10,10)
p=RandInt(0,6)

#Define the general vector for parameterization
v2=[t^2+b, t^3+b, 1/t+b, 1/t^2+b, exp(t)+b, log(t)+b, sin(t)+b, cos(t)+b]
xT=a*t^2
yT=v2[p]

#Define the Answer function
Ans=yT(sqrt(x/a))

\end{sagesilent}

\latexProblemContent{
\ifVerboseLocation This is Parametric Compute Question 0003. \\ \fi
\begin{problem}
Given the following parameterization, determine the associated curve being traced.

\[\left\{\begin{array}{lll}
x(t) &=& \sage{xT}\\
y(t) &=& \sage{yT}
\end{array}\right.,\quad t>0\]

\input{Parametric-Compute-0003.HELP.tex}

\[y=\answer{\sage{Ans}}\]

\end{problem}}%}
%%%%%%%%%%%%%%%%%%%%%%

%%%%%%%%%%%%%%%%%%%%%%%
%%\tagged{Ans@ShortAns, Type@Compute, Topic@Parametric, Sub@ParametricEq, File@0004}{
\begin{sagesilent}
# Define variables and constants/exponents
var('x,t')
a=NonZeroInt(-10,10)
b=RandInt(-10,10)
p=RandInt(0,6)

#Define the general vector for parameterization
v2=[t^2+b, t^3+b, 1/t+b, 1/t^2+b, exp(t)+b, log(t)+b, sin(t)+b, cos(t)+b]
xT=a*sqrt(t)
yT=v2[p]

#Define the Answer function
Ans=yT((x/a)^2)

\end{sagesilent}

\latexProblemContent{
\ifVerboseLocation This is Parametric Compute Question 0004. \\ \fi
\begin{problem}
Given the following parameterization, determine the associated curve being traced.

\[\left\{\begin{array}{lll}
x(t) &=& \sage{xT}\\
y(t) &=& \sage{yT}
\end{array}\right.,\quad t>0\]

\input{Parametric-Compute-0004.HELP.tex}

\[y=\answer{\sage{Ans}}\]

\end{problem}}%}
%%%%%%%%%%%%%%%%%%%%%%

%%%%%%%%%%%%%%%%%%%%%%%
%%\tagged{Ans@ShortAns, Type@Compute, Topic@Parametric, Sub@DerivPara, File@0005}{
\begin{sagesilent}
# Define variables and constants/exponents
var('x,t')
a=NonZeroInt(-10,10)
b=RandInt(-10,10)
p=RandInt(0,6)

#Define the general vector for parameterization
v1=[a*t+b, a*t^2+b, a*t^3+b, a*exp(t)+b, a*log(t)+b, a*sin(t)+b, a*cos(t)+b]
xT=t
yT=v1[p]
Dx=diff(xT,t)
Dy=diff(yT,t)
DDy=diff(Dy/Dx,t)

#Define the Answers
Ans1=Dy/Dx
Ans2=DDy/Dx


\end{sagesilent}

\latexProblemContent{
\ifVerboseLocation This is Parametric Compute Question 0005. \\ \fi
\begin{problem}
Given the following parameterization, find $\dfrac{dy}{dx}$ and $\dfrac{d^2y}{dx^2}$.

\[\left\{\begin{array}{lll}
x(t) &=& \sage{xT}\\
y(t) &=& \sage{yT}
\end{array}\right.,\quad t>0\]

\input{Parametric-Compute-0005.HELP.tex}

\[\frac{dy}{dx}=\answer{\sage{Ans1}}\qquad \frac{d^2y}{dx^2}=\answer{\sage{Ans2}}\]

\end{problem}}%}
%%%%%%%%%%%%%%%%%%%%%%

%%%%%%%%%%%%%%%%%%%%%%%
%%\tagged{Ans@ShortAns, Type@Compute, Topic@Parametric, Sub@DerivPara, File@0006}{
\begin{sagesilent}
# Define variables and constants/exponents
var('x,t')
a=NonZeroInt(-10,10)
p=RandInt(0,6)

#Define the general vector for parameterization
v2=[t^2, t^3, 1/t, 1/t^2, exp(t), log(t), sin(t), cos(t)]
xT=a*t
yT=v2[p]
Dx=diff(xT,t)
Dy=diff(yT,t)
DDy=diff(Dy/Dx,t)

#Define the Answers
Ans1=Dy/Dx
Ans2=DDy/Dx

\end{sagesilent}

\latexProblemContent{
\ifVerboseLocation This is Parametric Compute Question 0006. \\ \fi
\begin{problem}
Given the following parameterization, find $\dfrac{dy}{dx}$ and $\dfrac{d^2y}{dx^2}$.

\[\left\{\begin{array}{lll}
x(t) &=& \sage{xT}\\
y(t) &=& \sage{yT}
\end{array}\right.,\quad t>0\]

\input{Parametric-Compute-0006.HELP.tex}

\[\frac{dy}{dx}=\answer{\sage{Ans1}}\qquad \frac{d^2y}{dx^2}=\answer{\sage{Ans2}}\]

\end{problem}}%}
%%%%%%%%%%%%%%%%%%%%%%

%%%%%%%%%%%%%%%%%%%%%%%
%%\tagged{Ans@ShortAns, Type@Compute, Topic@Parametric, Sub@DerivPara, File@0007}{
\begin{sagesilent}
# Define variables and constants/exponents
var('x,t')
a=NonZeroInt(-10,10)
b=RandInt(-10,10)
p=RandInt(0,6)

#Define the general vector for parameterization
v2=[t^2+b, t^3+b, 1/t+b, 1/t^2+b, exp(t)+b, log(t)+b, sin(t)+b, cos(t)+b]
xT=a*t^2
yT=v2[p]
Dx=diff(xT,t)
Dy=diff(yT,t)
DDy=diff(Dy/Dx,t)

#Define the Answers
Ans1=Dy/Dx
Ans2=DDy/Dx

\end{sagesilent}

\latexProblemContent{
\ifVerboseLocation This is Parametric Compute Question 0007. \\ \fi
\begin{problem}
Given the following parameterization, find $\dfrac{dy}{dx}$ and $\dfrac{d^2y}{dx^2}$.

\[\left\{\begin{array}{lll}
x(t) &=& \sage{xT}\\
y(t) &=& \sage{yT}
\end{array}\right.,\quad t>0\]

\input{Parametric-Compute-0007.HELP.tex}

\[\frac{dy}{dx}=\answer{\sage{Ans1}}\qquad \frac{d^2y}{dx^2}=\answer{\sage{Ans2}}\]

\end{problem}}%}
%%%%%%%%%%%%%%%%%%%%%%

%%%%%%%%%%%%%%%%%%%%%%%
%%\tagged{Ans@ShortAns, Type@Compute, Topic@Parametric, Sub@DerivPara, File@0008}{
\begin{sagesilent}
# Define variables and constants/exponents
var('x,t')
a=NonZeroInt(-10,10)
b=RandInt(-10,10)
p=RandInt(0,6)

#Define the general vector for parameterization
v2=[t^2+b, t^3+b, 1/t+b, 1/t^2+b, exp(t)+b, log(t)+b, sin(t)+b, cos(t)+b]
xT=a*sqrt(t)
yT=v2[p]
Dx=diff(xT,t)
Dy=diff(yT,t)
DDy=diff(Dy/Dx,t)

#Define the Answers
Ans1=Dy/Dx
Ans2=DDy/Dx

\end{sagesilent}

\latexProblemContent{
\ifVerboseLocation This is Parametric Compute Question 0008. \\ \fi
\begin{problem}
Given the following parameterization, find $\dfrac{dy}{dx}$ and $\dfrac{d^2y}{dx^2}$.

\[\left\{\begin{array}{lll}
x(t) &=& \sage{xT}\\
y(t) &=& \sage{yT}
\end{array}\right.,\quad t>0\]

\input{Parametric-Compute-0008.HELP.tex}

\[\frac{dy}{dx}=\answer{\sage{Ans1}}\qquad \frac{d^2y}{dx^2}=\answer{\sage{Ans2}}\]

\end{problem}}%}
%%%%%%%%%%%%%%%%%%%%%%

%%%%%%%%%%%%%%%%%%%%%%%
%%\tagged{Ans@ShortAns, Type@Compute, Topic@Parametric, Sub@DerivPara, Sub@Tan-Line, File@0009}{
\begin{sagesilent}
# Define variables and constants/exponents
var('x,t')
a=NonZeroInt(-10,10)
b=RandInt(-10,10)
p=RandInt(0,5)
d=RandInt(1,8)

#Define the general vector for parameterization
v1=[a*t^2+b, a*t^3+b, a*exp(t)+b, a*log(t)+b, a*sin(t)+b, a*cos(t)+b]
xT=t
yT=v1[p]
Dx=diff(xT,t)
Dy=diff(yT,t)
xVal=xT(d)
yVal=yT(d)
Deriv=Dy/Dx

#Define the Answer
Ans=Deriv(d)*(x-xVal)+yVal
\end{sagesilent}

\latexProblemContent{
\ifVerboseLocation This is Parametric Compute Question 0009. \\ \fi
\begin{problem}
Given the following parameterization, find the equation of the tangent line to the curve at $t=\sage{d}$.

\[\left\{\begin{array}{lll}
x(t) &=& \sage{xT}\\
y(t) &=& \sage{yT}
\end{array}\right.,\quad t>0\]

\input{Parametric-Compute-0009.HELP.tex}

\[y=\answer{\sage{Ans}}\]

\end{problem}}%}
%%%%%%%%%%%%%%%%%%%%%%

%%%%%%%%%%%%%%%%%%%%%%%
%%\tagged{Ans@ShortAns, Type@Compute, Topic@Parametric, Sub@DerivPara, Sub@Tan-Line, File@0010}{
\begin{sagesilent}
# Define variables and constants/exponents
var('x,t')
a=NonZeroInt(-10,10)
p=RandInt(0,6)
d=RandInt(1,8)

#Define the general vector for parameterization
v2=[t^2, t^3, 1/t, 1/t^2, exp(t), log(t), sin(t), cos(t)]
xT=a*t
yT=v2[p]
Dx=diff(xT,t)
Dy=diff(yT,t)
xVal=xT(d)
yVal=yT(d)
Deriv=Dy/Dx

#Define the Answer
Ans=Deriv(d)*(x-xVal)+yVal
\end{sagesilent}

\latexProblemContent{
\ifVerboseLocation This is Parametric Compute Question 0010. \\ \fi
\begin{problem}
Given the following parameterization, find the equation of the tangent line to the curve at $t=\sage{d}$.

\[\left\{\begin{array}{lll}
x(t) &=& \sage{xT}\\
y(t) &=& \sage{yT}
\end{array}\right.,\quad t>0\]

\input{Parametric-Compute-0010.HELP.tex}

\[y=\answer{\sage{Ans}}\]

\end{problem}}%}
%%%%%%%%%%%%%%%%%%%%%%

%%%%%%%%%%%%%%%%%%%%%%%
%%\tagged{Ans@ShortAns, Type@Compute, Topic@Parametric, Sub@DerivPara, Sub@Tan-Line, File@0011}{
\begin{sagesilent}
# Define variables and constants/exponents
var('x,t')
a=NonZeroInt(-10,10)
b=RandInt(-10,10)
p=RandInt(0,6)
d=RandInt(1,8)

#Define the general vector for parameterization
v2=[t^2+a*t, t^3+b, 1/t+b, 1/t^2+b, exp(t)+b, log(t)+b, sin(t)+b, cos(t)+b]
xT=a*t^2
yT=v2[p]
Dx=diff(xT,t)
Dy=diff(yT,t)
xVal=xT(d)
yVal=yT(d)
Deriv=Dy/Dx

#Define the Answer
Ans=Deriv(d)*(x-xVal)+yVal
\end{sagesilent}

\latexProblemContent{
\ifVerboseLocation This is Parametric Compute Question 0011. \\ \fi
\begin{problem}
Given the following parameterization, find the equation of the tangent line to the curve at $t=\sage{d}$.

\[\left\{\begin{array}{lll}
x(t) &=& \sage{xT}\\
y(t) &=& \sage{yT}
\end{array}\right.,\quad t>0\]

\input{Parametric-Compute-0011.HELP.tex}

\[y=\answer{\sage{Ans}}\]

\end{problem}}%}
%%%%%%%%%%%%%%%%%%%%%%

%%%%%%%%%%%%%%%%%%%%%%%
%%\tagged{Ans@ShortAns, Type@Compute, Topic@Parametric, Sub@DerivPara, Sub@Tan-Line, File@0012}{
\begin{sagesilent}
# Define variables and constants/exponents
var('x,t')
a=NonZeroInt(-10,10)
b=RandInt(-10,10)
p=RandInt(0,6)
d=RandInt(1,8)

#Define the general vector for parameterization
v2=[t^2+b, t^3+b, 1/t+b, 1/t^2+b, exp(t)+b, log(t)+b, sin(t)+b, cos(t)+b]
xT=a*sqrt(t)
yT=v2[p]
Dx=diff(xT,t)
Dy=diff(yT,t)
xVal=xT(d)
yVal=yT(d)
Deriv=Dy/Dx

#Define the Answer
Ans=Deriv(d)*(x-xVal)+yVal
\end{sagesilent}

\latexProblemContent{
\ifVerboseLocation This is Parametric Compute Question 0012. \\ \fi
\begin{problem}
Given the following parameterization, find the equation of the tangent line to the curve at $t=\sage{d}$.

\[\left\{\begin{array}{lll}
x(t) &=& \sage{xT}\\
y(t) &=& \sage{yT}
\end{array}\right.,\quad t>0\]

\input{Parametric-Compute-0012.HELP.tex}

\[y=\answer{\sage{Ans}}\]

\end{problem}}%}
%%%%%%%%%%%%%%%%%%%%%%

%%%%%%%%%%%%%%%%%%%%%%%
%%\tagged{Ans@ShortAns, Type@Compute, Topic@Parametric, Sub@DerivPara, Sub@ArcLength, File@0013}{
\begin{sagesilent}
# Define variables and constants/exponents
var('x,t')
a=RandInt(-10,10)
b=NonZeroInt(-5,5)
c=RandInt(-10,10)
d=NonZeroInt(-5,5)
d1=RandInt(0,2)
d2=RandInt(d1+1,10)
# NEED p/q to not have 1/3 or 3/1
p=RandInt(1,3)
q=RandInt(1,3)
if p==1:
   if q==3:
      q=RandInt(1,2)
if p==3:
   if q==1:
      q=RandInt(2,3)

#Define the general parameterization
xT=a+b*t^p
yT=c+d*t^q
Dx=diff(xT,t)
Dy=diff(yT,t)
ARC=sqrt((Dx)^2+(Dy)^2)

#Define the Answer
Ans=integral(ARC,t,d1,d2)
\end{sagesilent}

\latexProblemContent{
\ifVerboseLocation This is Parametric Compute Question 0013. \\ \fi
\begin{problem}
Determine the arc length of the following parametric curve from $t=\sage{d1}$ to $t=\sage{d2}$.

\[\left\lbrace\begin{array}{lll}
x(t) &=& \sage{xT}\\
y(t) &=& \sage{yT}
\end{array}\right.,\quad t>0\]

\input{Parametric-Compute-0013.HELP.tex}

\[y=\answer{\sage{Ans}}\]

\end{problem}}%}
%%%%%%%%%%%%%%%%%%%%%%


























