%%%%%%%%%%%%%%%%%%%%%%%%%%
%%%%%%%%%%%%%%%%%%%%%%%%%%



%%%%%%%%%%%%%%%%%%%%%%%
%%\tagged{Ans@ShortAns, Type@Compute, Topic@Polar, Sub@PolarCoordinates, File@0001}{
\begin{sagesilent}
# Define variables and constants/exponents
a=NonZeroInt(-10,10)
b=RandInt(-10,10)
p=RandInt(0,1)
v=[b,sgn(b)*sqrt(abs(b))]
B=v[p]

#Define the coordinates
arr=sqrt(a^2+B^2)
if sgn(a)==1:
   theta=arctan(B/a)
if sgn(a)==-1:
   theta=arctan(B/a)+pi
\end{sagesilent}

\latexProblemContent{
\ifVerboseLocation This is Polar Compute Question 0001. \\ \fi
\begin{problem}
Convert the Cartesian coordinates into polar coordinates with $0\leq\theta\leq2\pi$ and $r>0$:

\input{Polar-Compute-0001.HELP.tex}

\[(\sage{a},\sage{B}) \longrightarrow \left(\answer{\sage{arr}},\answer{\sage{theta}}\right)\]

\end{problem}}%}
%%%%%%%%%%%%%%%%%%%%%%


%%%%%%%%%%%%%%%%%%%%%%%
%%\tagged{Ans@ShortAns, Type@Compute, Topic@Polar, Sub@PolarCoordinates, File@0002}{
\begin{sagesilent}
# Define variables and constants/exponents
a=NonZeroInt(-10,10)
b=RandInt(-10,10)
p=RandInt(0,1)
v=[b,sgn(b)*sqrt(abs(b))]
B=v[p]

#Define the coordinates
arr=-sqrt(a^2+B^2)
if sgn(a)==1:
   theta=arctan(B/a)+pi
if sgn(a)==-1:
   theta=arctan(B/a)
\end{sagesilent}

\latexProblemContent{
\ifVerboseLocation This is Polar Compute Question 0002. \\ \fi
\begin{problem}
Convert the Cartesian coordinates into polar coordinates with $0\leq\theta\leq2\pi$ and $r<0$:

\input{Polar-Compute-0002.HELP.tex}

\[(\sage{a},\sage{B}) \longrightarrow \left(\answer{\sage{arr}},\answer{\sage{theta}}\right)\]

\end{problem}}%}
%%%%%%%%%%%%%%%%%%%%%%

%%%%%%%%%%%%%%%%%%%%%%%
%%\tagged{Ans@ShortAns, Type@Compute, Topic@Polar, Sub@PolarCoordinates, File@0003}{
\begin{sagesilent}
# Define variables and constants/exponents
b=NonZeroInt(-10,10)
p=RandInt(0,1)
v=[b,sgn(b)*sqrt(abs(b))]
B=v[p]

#Define the coordinates
arr=sqrt(B^2)
if b>0:
   theta=pi/2
if b<0:
   theta=3*pi/2
\end{sagesilent}

\latexProblemContent{
\ifVerboseLocation This is Polar Compute Question 0003. \\ \fi
\begin{problem}
Convert the Cartesian coordinates into polar coordinates with $0\leq\theta\leq2\pi$ and $r>0$:

\input{Polar-Compute-0003.HELP.tex}

\[(0,\sage{b}) \longrightarrow \left(\answer{\sage{arr}},\answer{\sage{theta}}\right)\]

\end{problem}}%}
%%%%%%%%%%%%%%%%%%%%%%

%%%%%%%%%%%%%%%%%%%%%%%
%%\tagged{Ans@ShortAns, Type@Compute, Topic@Polar, Sub@PolarCoordinates, File@0004}{
\begin{sagesilent}
# Define variables and constants/exponents
r=NonZeroInt(-10,10)
angle=RandAng(0,2*pi)

#Define the coordinates
xVal=r*cos(angle)
yVal=r*sin(angle)
\end{sagesilent}

\latexProblemContent{
\ifVerboseLocation This is Polar Compute Question 0004. \\ \fi
\begin{problem}
Convert the polar coordinates into Cartesian coordinates:

\input{Polar-Compute-0004.HELP.tex}

\[\left(\sage{r},\sage{angle}\right) \longrightarrow \left(\answer{\sage{xVal}},\answer{\sage{yVal}}\right)\]

\end{problem}}%}
%%%%%%%%%%%%%%%%%%%%%%

%%%%%%%%%%%%%%%%%%%%%%%
%%\tagged{Ans@ShortAns, Type@Compute, Topic@Polar, Sub@PolarEq, File@0005}{
\begin{sagesilent}
# Define variables and constants/exponents
var('XTRtheta')
r=NonZeroInt(1,10)
p=RandInt(0,2)
q=RandInt(0,2)
v1=[0,r,-r]

#Define the coordinates
xVal=v1[p]
if p==0:
   if q==0:
      q=RandInt(1,2)
      yVal=v1[q]
   else:
      yVal=v1[q]
if p>0:
   yVal=0

#Define the Answer using sage char replacement
if p==0:
   Ans=2*yVal*sin(XTRtheta)
if p>0:
   Ans=2*xVal*cos(XTRtheta)
\end{sagesilent}

\latexProblemContent{
\ifVerboseLocation This is Polar Compute Question 0005. \\ \fi
\begin{problem}
Write the polar equation for a circle of radius $\sage{r}$ and center $(\sage{xVal},\sage{yVal})$.

\input{Polar-Compute-0005.HELP.tex}

\[r=\answer{\sage{Ans}}\]

\end{problem}}%}
%%%%%%%%%%%%%%%%%%%%%%

%%%%%%%%%%%%%%%%%%%%%%%
%%\tagged{Ans@ShortAns, Type@Compute, Topic@Polar, Sub@Rose, Sub@AreaPolar, File@0006}{
\begin{sagesilent}
# Define variables and constants/exponents
var('XTRtheta')
a=RandInt(1,10)
b=RandInt(2,10)
p=RandInt(0,1)
v1=[a*cos(b*XTRtheta), a*sin(b*XTRtheta)]

#Define the number of petals and the equation
if mod(b,2)==0:
   Petals=2*b
else:
   Petals=b

R=v1[p]

#Define the Answer using sage char replacement
Ans=integral(R^2,XTRtheta,0,pi/(2*b))
\end{sagesilent}

\latexProblemContent{
\ifVerboseLocation This is Polar Compute Question 0006. \\ \fi
\begin{problem}
How many petals does $r=\sage{R}$ have?

\input{Polar-Compute-0006.HELP.tex}

\[\answer{\sage{Petals}}\]
\begin{problem}
Find the area enclosed by one petal.

\[\answer{\sage{Ans}}\]

\end{problem}
\end{problem}}%}
%%%%%%%%%%%%%%%%%%%%%%

%%%%%%%%%%%%%%%%%%%%%%%
%%\tagged{Ans@ShortAns, Type@Compute, Topic@Polar, Sub@Rose, Sub@AreaPolar, File@0007}{
\begin{sagesilent}
# Define variables and constants/exponents
var('XTRtheta')
a=RandInt(1,10)
b=RandInt(2,10)
p=RandInt(0,1)
v1=[a*cos(b*XTRtheta), a*sin(b*XTRtheta)]

#Define the number of petals and the equation
if mod(b,2)==0:
   Petals=2*b
else:
   Petals=b
   
d=NonZeroInt(1,Petals-1)
R=v1[p]

#Define the Answer using sage char replacement
Ans=integral(R^2,XTRtheta,0,pi/(2*b))*d
\end{sagesilent}

\latexProblemContent{
\ifVerboseLocation This is Polar Compute Question 0007. \\ \fi
\begin{problem}
How many petals does $r=\sage{R}$ have?

\input{Polar-Compute-0007.HELP.tex}

\[\answer{\sage{Petals}}\]
\begin{problem}
Find the area enclosed by $\sage{d}$ petal(s).

\[\answer{\sage{Ans}}\]

\end{problem}
\end{problem}}%}
%%%%%%%%%%%%%%%%%%%%%%

%%%%%%%%%%%%%%%%%%%%%%%
%%\tagged{Ans@ShortAns, Type@Compute, Topic@Polar, Sub@Cardioid, Sub@AreaPolar, File@0008}{
\begin{sagesilent}
# Define variables and constants/exponents
var('XTRtheta')
a=RandInt(1,10)
p=RandInt(0,1)
q=RandInt(0,1)
v1=[a*(1+(-1)^q*cos(XTRtheta)), a*(1+(-1)^q*sin(XTRtheta))]

#Define the equation
R=v1[p]

#Define the Answer using sage char replacement
if p==0:
   Ans=integral(R^2,XTRtheta,0,pi)
if p==1:
   Ans=integral(R^2,XTRtheta,-pi/2,pi/2)
\end{sagesilent}

\latexProblemContent{
\ifVerboseLocation This is Polar Compute Question 0008. \\ \fi
\begin{problem}
Find the area enclosed by the cardioid $r=\sage{R}$.

\input{Polar-Compute-0008.HELP.tex}

\[\answer{\sage{Ans}}\]
\end{problem}}%}
%%%%%%%%%%%%%%%%%%%%%%

%%%%%%%%%%%%%%%%%%%%%%%
%%\tagged{Ans@ShortAns, Type@Compute, Topic@Polar, Sub@Cardioid, Sub@AreaPolar, File@0009}{
\begin{sagesilent}
# Define variables and constants/exponents
var('XTRtheta')
a=RandInt(1,10)
b=RandInt(2,5)
p=RandInt(0,1)
q=RandInt(0,1)
v1=[a*(1+(-1)^q*(1/b)*cos(XTRtheta)), a*(1+(-1)^q*(1/b)*sin(XTRtheta))]

#Define the equation
R=v1[p]

#Define the Answer using sage char replacement
if p==0: # Cosine version
   Ans=integral(R^2,XTRtheta,0,pi)
if p==1: # Sine version
   Ans=integral(R^2,XTRtheta,-pi/2,pi/2)
\end{sagesilent}

\latexProblemContent{
\ifVerboseLocation This is Polar Compute Question 0009. \\ \fi
\begin{problem}
Find the area enclosed by the cardioid $r=\sage{R}$.

\input{Polar-Compute-0009.HELP.tex}

\[\answer{\sage{Ans}}\]
\end{problem}}%}
%%%%%%%%%%%%%%%%%%%%%%

%%%%%%%%%%%%%%%%%%%%%%%
%%\tagged{Ans@ShortAns, Type@Compute, Topic@Polar, Sub@Limacon, Sub@AreaPolar, File@0010}{
\begin{sagesilent}
# Define variables and constants/exponents
var('XTRtheta')
a=RandInt(1,5)
b=RandInt(2,5)
p=RandInt(0,1)
q=RandInt(0,1)
v1=[a*(1+(-1)^q*b*cos(XTRtheta)), a*(1+(-1)^q*b*sin(XTRtheta))]

#Define the equation
R=v1[p]

#Define the Answer using sage char replacement
if p==0: ## If Cosine
   if q==0:  #If Positve b
      Ans=integral(R^2,XTRtheta,arccos(-1/b),pi)
   else: #If Negative b
      Ans=integral(R^2,XTRtheta,0,arccos(1/b))
if p==1: ## If Sine
   if q==0:  #If Positive b
      Ans=integral(R^2,XTRtheta,arcsin(-1/b),3*pi/2)
   else:  ##If Negative b
      Ans=integral(R^2,XTRtheta,arcsin(1/b),pi/2)
\end{sagesilent}

\latexProblemContent{
\ifVerboseLocation This is Polar Compute Question 0010. \\ \fi
\begin{problem}
Find the area enclosed by the inner loop of the Lima{\c}on $r=\sage{R}$.

\input{Polar-Compute-0010.HELP.tex}

\[\answer{\sage{Ans}}\]
\end{problem}}%}
%%%%%%%%%%%%%%%%%%%%%%

%%%%%%%%%%%%%%%%%%%%%%%
%%\tagged{Ans@ShortAns, Type@Compute, Topic@Polar, Sub@Cardioid, Sub@PolarGraph, Sub@AreaPolar, File@0011}{
\begin{sagesilent}
# Define variables and constants/exponents
var('XTRtheta')
a=RandInt(1,10)
p=RandInt(0,1)
q=RandInt(0,1)
v1=[a*(1+(-1)^q*cos(XTRtheta)), a*(1+(-1)^q*sin(XTRtheta))]

#Define the equation
R1=v1[p]
R2=a
RSquared=R1^2-R2^2

#Define the Answer using sage char replacement
if p==0: # COSINE
   if q==0: #POSITIVE
      Ans=integral(RSquared,XTRtheta,0,pi/2)
   else: #NEGATIVE
      Ans=integral(RSquared,XTRtheta,pi/2,pi)
if p==1: #SINE
   if q==0: #POSITIVE
      Ans=integral(RSquared,XTRtheta,0,pi/2)
   else: #NEGATIVE
      Ans=integral(RSquared,XTRtheta,pi,3*pi/2)
\end{sagesilent}

\latexProblemContent{
\ifVerboseLocation This is Polar Compute Question 0011. \\ \fi
\begin{problem}
Find the are of the region that lies inside $r_1=\sage{R1}$ and outside $r_2=\sage{R2}$.

\input{Polar-Compute-0011.HELP.tex}

\[\answer{\sage{Ans}}\]
\end{problem}}%}
%%%%%%%%%%%%%%%%%%%%%%

%%%%%%%%%%%%%%%%%%%%%%%
%%\tagged{Ans@ShortAns, Type@Compute, Topic@Polar, Sub@Cardioid, Sub@PolarGraph, Sub@AreaPolar, File@0012}{
\begin{sagesilent}
# Define variables and constants/exponents
var('XTRtheta')
a=RandInt(1,10)
p=RandInt(0,1)
v1=[a*(1+(1/a)*cos(XTRtheta)), a*(1+(1/a)*sin(XTRtheta))]
v2=[(a+1)*cos(XTRtheta), (a+1)*sin(XTRtheta)]

#Define the equation
R1=v1[p]
R2=v2[p]

#Define the Answer using sage char replacement
if p==0: # COSINE
   Ans=integral(R1^2,XTRtheta,0,pi)-integral(R2^2,XTRtheta,0,pi/2)  
if p==1: #SINE
   Ans=integral(R1^2,XTRtheta,-pi/2,pi/2)-integral(R2^2,XTRtheta,0,pi/2)
\end{sagesilent}

\latexProblemContent{
\ifVerboseLocation This is Polar Compute Question 0012. \\ \fi
\begin{problem}
Find the are of the region that lies inside $r_1=\sage{R1}$ and outside $r_2=\sage{R2}$.

\input{Polar-Compute-0012.HELP.tex}

\[\answer{\sage{Ans}}\]
\end{problem}}%}
%%%%%%%%%%%%%%%%%%%%%%























