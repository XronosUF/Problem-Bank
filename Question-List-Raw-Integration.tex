%%%%%%  This is the raw list of questions before processing %%%%%%


%%%%%%%%%%%%%%%%%%%%%%%%%%%%%%%%%%%%%%%%%%%%%%%%%%%%%%%%%%%%%%%%%%%%%%%%%%%%%%%
%%%%%%%%%%%%%%%%%%%%%%%%%%%%%%%%%%%%%%%%%%%%%%%%%%%%%%%%%%%%%%%%%%%%%%%%%%%%%%%
%%%%%%%%%%%%%%%%%%%										%%%%%%%%%%%%%%%%%%%%%%%
%%%%%%%%%%%%%%%%%%%				Computation				%%%%%%%%%%%%%%%%%%%%%%%
%%%%%%%%%%%%%%%%%%%										%%%%%%%%%%%%%%%%%%%%%%%
%%%%%%%%%%%%%%%%%%%%%%%%%%%%%%%%%%%%%%%%%%%%%%%%%%%%%%%%%%%%%%%%%%%%%%%%%%%%%%%
%%%%%%%%%%%%%%%%%%%%%%%%%%%%%%%%%%%%%%%%%%%%%%%%%%%%%%%%%%%%%%%%%%%%%%%%%%%%%%%



%%%%%%%%%%%%%%%%%%%%%%%%%%%%%%%%%%%%%%%%%%%%%%%%%%%%%%%%%%%%%%%%%%%%%%%%%%%%%%%
%%%%%%%%%%%%%%%%%%%			MAC2311: Calculus 1			%%%%%%%%%%%%%%%%%%%%%%%
%%%%%%%%%%%%%%%%%%%%%%%%%%%%%%%%%%%%%%%%%%%%%%%%%%%%%%%%%%%%%%%%%%%%%%%%%%%%%%%


%%%%%%%%%%%%%%%%%%%%%%%
%%\tagged{Ans@ShortAns, Type@Compute, Topic@Integral, Sub@Poly, Sub@Riemann, File@0001}{
\begin{sagesilent}
# Build the Interval
a = NonZeroInt(-8,8) 				# Left end point
n = NonZeroInt(3,7)					# Number of rectangles
b = a + n				# Fix right endpoint based on left endpoint and number of rec.
M = max( abs(a), abs(b)) 			# Max val useful for some functions
m = min( abs(a), abs(b))			# Min val useful for some functions

# Build the function
v = [ x - a, x^2 - m^2, -x^2 + M^2, b - x]# Make a vector of functions
p = RandInt(0,3)					# Select a function
f = v[p]							# Assign function

v2 = ['right', 'left']				# Vector of endpoints. (Note, we can add more!)
p2 = RandInt(0,1)					# Pick Endpoint
Endpoint = v2[p2]					#Initialize Endpoint

# Calculate the area
Area = 0							# Initialize Area
for j in range(n):					# Note range(n) = 0, 1, 2, ..., n-1.
    if p2 == 0:						# Right Endpoint case
        Xpt = j + 1					# Shift to right x anchor point by adding width
    if p2 == 1:						# Left Endpoint case
        Xpt = j						# Use left x anchor point.
# Case not implemented    if p2 == 2:						# Midpoint case
# Case not implemented        Xpt = j + 0.5				# Shift half a width
    Area = Area + f(a+Xpt)			# Exploit that rectangles are width 1
   
Ans=Area
\end{sagesilent}

\latexProblemContent{
\begin{problem}

Estimate the area under the graph of $f(x)=\sage{f}$ from $x=\sage{a}$ to $x=\sage{b}$ using $\sage{n}$ rectangles and $\sage{Endpoint}$ endpoints.

\input{Integral-Compute-0001.HELP.tex}

\[\mbox{Area}\approx\answer{\sage{Ans}}\]
\end{problem}}%}
%%%%%%%%%%%%%%%%%%%%%%


%%%%%%%%%%%%%%%%%%%%%%%
%%\tagged{Ans@ShortAns, Type@Compute, Topic@Integral, Sub@Poly, Sub@Riemann, File@0002}{
\begin{sagesilent}
# Build the Interval
Parity = (-1)^RandInt(0,1)			# Determine if you are on negative or pos side.
LeftEnd = NonZeroInt(1,3)			# Left end point
n = NonZeroInt(3,5)					# Number of rectangles
RightEnd = LeftEnd + n				# Determine Right Endpoint
Delx = (RightEnd - LeftEnd)/n		# Width of each rectangle.

# Build the function
v = [ 1/(x), 1/(x^2), 1/(-x^2), 1/(x)]# Make a vector of functions
p = RandInt(0,3)					# Select a function
f = v[p]							# Assign function

v2 = ['right', 'left']				# Vector of endpoints. (Note, we can add more!)
p2 = RandInt(0,1)					# Pick Endpoint
Endpoint = v2[p2]					# Initialize Endpoint

# Calculate the area
Area = 0							# Initialize Area
for j in range(n):					# Note range(n) = 0, 1, 2, ..., n-1.
    if p2 == 0:						# Right Endpoint case
        Xpt = j + 1					# Shift to right x anchor point by adding width
    if p2 == 1:						# Left Endpoint case
        Xpt = j						# Use left x anchor point.
# Case not implemented    if p2 == 2:					# Midpoint case
# Case not implemented        Xpt = j + 0.5				# Shift half a width
    Area = Area + f(LeftEnd+(Delx*Xpt))	# Exploit that rectangles are width 1
   
Ans=abs(Area)
\end{sagesilent}

\latexProblemContent{
\begin{problem}

Estimate the area under the graph of $f(x)=\sage{f}$ from $x=\sage{LeftEnd}$ to $x=\sage{RightEnd}$ using $\sage{n}$ rectangles and $\sage{Endpoint}$ endpoints.

\input{Integral-Compute-0002.HELP.tex}

\[\mbox{Area}\approx\answer{\sage{Ans}}\]
\end{problem}}%}
%%%%%%%%%%%%%%%%%%%%%%




%%%%%%%%%%%%%%%%%%%%%%%
%%\tagged{Ans@ShortAns, Type@Compute, Topic@Integral, Sub@Rational, Sub@Riemann, File@0003}{
\begin{sagesilent}
# Build the Interval
a = NonZeroInt(0,8) 				# Left end point
n = NonZeroInt(3,7)					# Number of rectangles
b = a + n				# Fix right endpoint based on left endpoint and number of rec.
M = max( abs(a), abs(b)) 			# Max val useful for some functions
m = min( abs(a), abs(b))			# Min val useful for some functions

# Build the function
v = [ sqrt(x - a), sqrt(x^2 - m^2), sqrt(-x^2 + M^2), sqrt(b - x)]# Make a vector of functions
p = RandInt(0,3)					# Select a function
f = v[p]							# Assign function

v2 = ['right', 'left']				# Vector of endpoints. (Note, we can add more!)
p2 = RandInt(0,1)					# Pick Endpoint
Endpoint = v2[p2]					# Initialize Endpoint

# Calculate the area
Area = 0							# Initialize Area
for j in range(n):					# Note range(n) = 0, 1, 2, ..., n-1.
    if p2 == 0:						# Right Endpoint case
        Xpt = j + 1					# Shift to right x anchor point by adding width
    if p2 == 1:						# Left Endpoint case
        Xpt = j						# Use left x anchor point.
# Case not implemented    if p2 == 2:					# Midpoint case
# Case not implemented        Xpt = j + 0.5				# Shift half a width
    Area = Area + f(a+(Xpt))	# Exploit that rectangles are width 1
   
Ans=abs(Area)
\end{sagesilent}

\latexProblemContent{
\begin{problem}

Estimate the area under the graph of $f(x)=\sage{f}$ from $x=\sage{a}$ to $x=\sage{b}$ using $\sage{n}$ rectangles and $\sage{Endpoint}$ endpoints.

\input{Integral-Compute-0003.HELP.tex}

\[\mbox{Area}\approx\answer{\sage{Ans}}\]
\end{problem}}%}
%%%%%%%%%%%%%%%%%%%%%%

%%%%%%%% 10 Compute (Riemann Sums problems ^) %%%%%%%%%%%%%



%%%%%%%%%%%%%%%%%%%%%%%
%%\tagged{Ans@ShortAns, Type@Compute, Topic@Integral, Sub@Definite, File@0004}{
\begin{sagesilent}
# Build Interval
a = NonZeroInt(-10,10)
Width = RandInt(3,10)
b = a + Width
r = RandInt(1, 5)

# Build Function
v = [abs(x), x, sqrt(r^2 - x^2)]
p = RandInt(0,2)
f = v[p]

if p==2:
   Area = (1/2)*pi*r^2
   a = -r
   b = r
else:
   c1 = NonZeroInt(-5,5)
   c2 = RandInt(-5,5)
   f(x) = f(c1*x + c2)
   g(x) = abs(f(x))
   Area = integral(g(x), x, a, b)

\end{sagesilent}

\latexProblemContent{
\begin{problem}

Evaluate the definite integral by interpreting it in terms of areas.  

\input{Integral-Compute-0004.HELP.tex}

\[\int_{\sage{a}}^{\sage{b}} \sage{f(x)}\;dx= \answer{\sage{Area}}\]
\end{problem}}%}
%%%%%%%%%%%%%%%%%%%%%%




%%%%%%%%%%%%%%%%%%%%%%%
%%\tagged{Ans@ShortAns, Type@Compute, Topic@Integral, Sub@Definite, \Sub@Piecewise, File@0005}{
\begin{sagesilent}
a = NonZeroInt(-5,5)
b=Integer(randint(a+3,a+8))
c = NonZeroInt(1,5)
m1=NonZeroInt(1,4)
m2=NonZeroInt(1,4)
d=(-2*m1*a-2*c)/(b-a)
l=Integer(randint(a-5,a))
u=Integer(randint(a+1,b+5))

f1=m1*x+c
f2=d*(x-a)+(m1*a+c)
f3=m2*(x-b)-(m1*a+c)
f = piecewise([([-20,a],f1), ((a,b), f2), ([b,20], f3)])

Ans=integrate(f,x,l,u)
\end{sagesilent}

\latexProblemContent{
\begin{problem}

Given the piecewise function 
\[f(x)=\left\{\begin{array}{ll}
\sage{f1}\; , & x\leq\sage{a}\\[3pt]
\sage{f2}\; , & \sage{a}< x< \sage{b}\\[3pt]
\sage{f3}\; , & x\geq\sage{b}
\end{array} \right.\]
evaluate the following definite integral by interpreting it in terms of areas.  

\input{Integral-Compute-0005.HELP.tex}

\[\int_{\sage{l}}^{\sage{u}} f(x)\;dx= \answer{\sage{Ans}}\]  
\end{problem}}%}
%%%%%%%%%%%%%%%%%%%%%%


%%%%%%%%%%%%%%%%%%%%%%%
%%\tagged{Ans@ShortAns, Type@Compute, Topic@Integral, Sub@Definite, Sub@Theorems-FTC, File@0006}{
\begin{sagesilent}
var('x,t')
a = NonZeroInt(1,5)
b = NonZeroInt(-10,10)
c = NonZeroInt(1,5)

p=Integer(randint(0,11))
v=[sqrt(x-b), log(x-c), exp(x-b), (x-b)^2, (x-b)^3, (x-b)^4, (x-b), sin(x-b), cos(x-b), 1/(x-b), 1/(x-b)^2, 1/(x-b)^3]
f=v[p]
Ans=f(t)
\end{sagesilent}

\latexProblemContent{
\begin{problem}

Use the Fundamental Theorem of Calculus to find the derivative of the function.
\[g(t)=\int_{\sage{a}}^{t} \sage{f}\;dx\]

\input{Integral-Compute-0006.HELP.tex}

\[\dfrac{d}{dt}(g(t))=\answer{\sage{Ans}}\]
\end{problem}}%}
%%%%%%%%%%%%%%%%%%%%%%

%%%%%%%%%%%%%%%%%%%%%%%
%%\tagged{Ans@ShortAns, Type@Compute, Topic@Integral, Sub@Definite, Sub@Theorems-FTC, File@0007}{
\begin{sagesilent}
var('x,t')
a = NonZeroInt(1,5)
b = NonZeroInt(-10,10)
c = NonZeroInt(1,5)

p=Integer(randint(0,11))
q=Integer(randint(0,11))
v1=[sqrt(x-b), log(x-c), exp(x-b), (x-b)^2, (x-b)^3, (x-b)^4, (x-b), sin(x-b), cos(x-b), 1/(x-b), 1/(x-b)^2, 1/(x-b)^3]
v2=[sqrt(x), log(x), exp(x), (x)^2, (x)^3, (x)^4, (x-b), sin(x), cos(x), 1/(x), 1/(x)^2, 1/(x)^3]
f=v1[p]
g=v2[p]
h=f*g
Ans=f(t)*g(t)
\end{sagesilent}

\latexProblemContent{
\begin{problem}

Use the Fundamental Theorem of Calculus to find the derivative of the function.
\[g(t)=\int_{\sage{a}}^{t} \sage{h}\;dx\]

\input{Integral-Compute-0007.HELP.tex}

\[\dfrac{d}{dt}(g(t))=\answer{\sage{Ans}}\]
\end{problem}}%}
%%%%%%%%%%%%%%%%%%%%%%


%%%%%%%%%%%%%%%%%%%%%%%
%%\tagged{Ans@ShortAns, Type@Compute, Topic@Integral, Sub@Definite, Sub@Theorems-FTC, File@0008}{
\begin{sagesilent}
var('x,t')
a = NonZeroInt(1,5)
b = NonZeroInt(-10,10)
c = NonZeroInt(1,5)

p=Integer(randint(0,11))
q=Integer(randint(0,11))
v1=[sqrt(x-b), log(x-c), exp(x-b), (x-b)^2, (x-b)^3, (x-b)^4, (x-b), sin(x-b), cos(x-b), 1/(x-b), 1/(x-b)^2, 1/(x-b)^3]
v2=[sqrt(x), log(x), exp(x), (x)^2, (x)^3, (x)^4, (x-b), sin(x), cos(x), 1/(x), 1/(x)^2, 1/(x)^3]
f=v1[p]
g=v2[p]
h=g(f)
Ans=g(f(t))
\end{sagesilent}

\latexProblemContent{
\begin{problem}

Use the Fundamental Theorem of Calculus to find the derivative of the function.
\[g(t)=\int_{\sage{a}}^{t} \sage{h}\;dx\]

\input{Integral-Compute-0008.HELP.tex}

\[\dfrac{d}{dt}(g(t))=\answer{\sage{Ans}}\]
\end{problem}}%}
%%%%%%%%%%%%%%%%%%%%%%


%%%%%%%%%%%%%%%%%%%%%%%
%%\tagged{Ans@ShortAns, Type@Compute, Topic@Integral, Sub@Definite, Sub@Theorems-FTC, Sub@Chain-Rule, File@0009}{
\begin{sagesilent}
var('x,t')
a = NonZeroInt(1,5)
b = NonZeroInt(-10,10)
c = NonZeroInt(1,5)

p=Integer(randint(0,11))
q=Integer(randint(0,10))
v1=[sqrt(x-b), log(x-c), exp(x-b), (x-b)^2, (x-b)^3, (x-b)^4, (x-b), sin(x-b), cos(x-b), 1/(x-b), 1/(x-b)^2, 1/(x-b)^3]
v2=[sqrt(x), log(x), exp(x), (x)^2, (x)^3, (x)^4, sin(x), cos(x), 1/(x), 1/(x)^2, 1/(x)^3]
f=v1[p]
g=v2[q]

Ans=(diff(g,x))(t)*f(t)

\end{sagesilent}

\latexProblemContent{
\begin{problem}

Use the Fundamental Theorem of Calculus to find the derivative of the function.
\[g(t)=\int_{\sage{a}}^{\sage{g(t)}} \sage{f}\;dx\]

\input{Integral-Compute-0009.HELP.tex}

\[\dfrac{d}{dt}(g(t))=\answer{\sage{Ans}}\]
\end{problem}}%}
%%%%%%%%%%%%%%%%%%%%%%

%%%%%%%%%%%% 18 Compute %%%%%%%%%%%%

%%%%%%%%%%%%%%%%%%%%%%%
%%\tagged{Ans@ShortAns, Type@Compute, Topic@Integral, Sub@Definite, Sub@Theorems-FTC, Sub@Poly, File@0010}{
\begin{sagesilent}
a = NonZeroInt(-5,5)
b = Integer(randint(-8,8))
c = NonZeroInt(1,5)
l=Integer(randint(-10,5))
u=Integer(randint(l,12))

p=Integer(randint(0,4))
vpoly=[(x-a), expand((x-a)^2), expand((x-a)^3), expand((x-a)*(x-b)), expand((x-a)^2*(x-b))]
F=vpoly[p]
Ans=integrate(F,x,l,u)
\end{sagesilent}

\latexProblemContent{
\begin{problem}

Use the Fundamental Theorem of Calculus to evaluate the integral.

\input{Integral-Compute-0010.HELP.tex}

\[\int_{\sage{l}}^{\sage{u}} \sage{F}\;dx=\answer{\sage{Ans}}\]
\end{problem}}%}
%%%%%%%%%%%%%%%%%%%%%%

%%%%%%%%%%%%%%%%%%%%%%%
%%\tagged{Ans@ShortAns, Type@Compute, Topic@Integral, Sub@Definite, Sub@Theorems-FTC, Sub@Poly, File@0011}{
\begin{sagesilent}
var('x')
a = NonZeroInt(-5,5)
b = NonZeroInt(-8,8)
l=Integer(randint(1,20))
u=Integer(randint(l,21))

F=(x-a)/(b*sqrt(x))
Ans=integrate(F,x,l,u)
\end{sagesilent}

\latexProblemContent{
\begin{problem}

Use the Fundamental Theorem of Calculus to evaluate the integral.

\input{Integral-Compute-0011.HELP.tex}

\[\int_{\sage{l}}^{\sage{u}} \sage{F}\;dx=\answer{\sage{Ans}}\]
\end{problem}}%}
%%%%%%%%%%%%%%%%%%%%%%

%%%%%%%%%%%%%%%%%%%%%%%
%%\tagged{Ans@ShortAns, Type@Compute, Topic@Integral, Sub@Definite, Sub@Theorems-FTC, Sub@Trig, File@0012}{
\begin{sagesilent} 
b = NonZeroInt(-10,10)
l = RandAng(0,pi)
u = RandAng(l,2*pi)
p=Integer(randint(0,1))
vtrig=[b*sin(x), b*cos(x)]

F=vtrig[p]
Ans=integrate(F,x,l,u)
\end{sagesilent}

\latexProblemContent{
\begin{problem}

Use the Fundamental Theorem of Calculus to evaluate the integral.

\input{Integral-Compute-0012.HELP.tex}

\[\int_{\sage{l}}^{\sage{u}} \sage{F}\;dx=\answer{\sage{Ans}}\]
\end{problem}}%}
%%%%%%%%%%%%%%%%%%%%%%

%%%%%%%%%%%%%%%%%%%%%%%
%%\tagged{Ans@ShortAns, Type@Compute, Topic@Integral, Sub@Definite, Sub@Theorems-FTC, Sub@Rational, File@0013}{
\begin{sagesilent}
a = NonZeroInt(-10,10)

l=NonZeroInt(-10,5)  
if l<0:
   while l==-1:
      l=NonZeroInt(-10,-3)
   while l==-2:
      l=NonZeroInt(-10,-3)
   u=NonZeroInt(l+1,-2)
elif l>0:
   u=NonZeroInt(l+1,12)
p=Integer(randint(0,3))
v=[a/x,a/x^2, a/x^3, a/x^4]
F=v[p]
Ans=integrate(F,x,l,u)
\end{sagesilent}

\latexProblemContent{
\begin{problem}

Use the Fundamental Theorem of Calculus to evaluate the integral.

\input{Integral-Compute-0013.HELP.tex}

\[\int_{\sage{l}}^{\sage{u}} \sage{F}\;dx=\answer{\sage{Ans}}\]
\end{problem}}%}
%%%%%%%%%%%%%%%%%%%%%%

%%%%%%%%%%%%%%%%%%%%%%%
%%\tagged{Ans@ShortAns, Type@Compute, Topic@Integral, Sub@Indefinite, File@0014}{
\begin{sagesilent}
a = NonZeroInt(-10,10)
b = NonZeroInt(-10,10)
p=Integer(randint(0,10))
v2=[sqrt(x), exp(x), (x)^2, (x)^3, (x)^4, (x-b), sin(x), cos(x), 1/(x), 1/(x)^2, 1/(x)^3]
F=v2[p]
G=a*F
Ans=integrate(G,x)
\end{sagesilent}

\latexProblemContent{
\begin{problem}

Compute the indefinite integral:

\input{Integral-Compute-0014.HELP.tex}

\[\int\;\sage{G}\;dx = \answer{\sage{Ans}+C}\]
\end{problem}}%}
%%%%%%%%%%%%%%%%%%%%%%

%%%%%%%%%%%%%%%%%%%%%%%
%%\tagged{Ans@ShortAns, Type@Compute, Topic@Integral, Sub@Indefinite, File@0015}{
\begin{sagesilent}
a = NonZeroInt(-10,10)
b = NonZeroInt(-10,10)
p=Integer(randint(0,10))
q=Integer(randint(0,10))
v2=[sqrt(x), exp(x), (x)^2, (x)^3, (x)^4, (x-b), sin(x), cos(x), 1/(x), 1/(x)^2, 1/(x)^3]
F=v2[p]
G=v2[q]
Ans=integrate(a*(F+G),x)
\end{sagesilent}

\latexProblemContent{
\begin{problem}

Compute the indefinite integral:

\input{Integral-Compute-0015.HELP.tex}

\[\int\;\sage{a*(F+G)}\;dx = \answer{\sage{Ans}+C}\]
\end{problem}}%}
%%%%%%%%%%%%%%%%%%%%%%

%%%%%%%%%%%%%%%%%%%%%%%
%%\tagged{Ans@ShortAns, Type@Compute, Topic@Integral, Sub@Indefinite, Sub@Sub-u, File@0016}{
\begin{sagesilent}
a = NonZeroInt(-10,10)
b = NonZeroInt(-10,10)
p=Integer(randint(0,11))
v2=[sqrt(x), exp(x), log(x), (x)^2, (x)^3, (x)^4, (x-b), sin(x), cos(x), 1/(x), 1/(x)^2, 1/(x)^3]
F=v2[p]
G=diff(F,x)
Ans=integrate(a*G*F,x)
\end{sagesilent}

\latexProblemContent{
\begin{problem}

Compute the indefinite integral:

\input{Integral-Compute-0016.HELP.tex}

\[\int\;\sage{a*G*F}\;dx = \answer{\sage{Ans}+C}\]
\end{problem}}%}
%%%%%%%%%%%%%%%%%%%%%%

%%%%%%%%%%%%%%%%%%%%%%%
%%\tagged{Ans@ShortAns, Type@Compute, Topic@Integral, Sub@Indefinite, Sub@Sub-u, File@0017}{
\begin{sagesilent}
a = NonZeroInt(-10,10)
b = NonZeroInt(-10,10)
p=Integer(randint(0,11))
q=Integer(randint(0,11))
v2=[sqrt(x), exp(x), log(x), (x)^2, (x)^3, (x)^4, (x-b), sin(x), cos(x), 1/(x), 1/(x)^2, 1/(x)^3]
F=v2[p]
G=v2[q]
f=diff(F,x)
Ans=integrate(a*G(F)*f,x)
\end{sagesilent}

\latexProblemContent{
\begin{problem}

Compute the indefinite integral:

\input{Integral-Compute-0017.HELP.tex}

\[\int\;\sage{a*G(F)*f}\;dx = \answer{\sage{Ans}+C}\]
\end{problem}}%}
%%%%%%%%%%%%%%%%%%%%%%


%%%%%%%%%%%%%%%%%%%%%%%
%%\tagged{Ans@ShortAns, Type@Compute, Topic@Integral, Sub@Indefinite, Sub@Sub-u, File@0018}{
\begin{sagesilent}
a = NonZeroInt(-10,10)
b = NonZeroInt(-10,10)
c = NonZeroInt(1,5)
p=Integer(randint(0,11))
q=Integer(randint(0,3))
v1=[sqrt(x-b), log(x-c), exp(x-b), (x-b)^2, (x-b)^3, (x-b)^4, (x-b), sin(x-b), cos(x-b), 1/(x-b), 1/(x-b)^2, 1/(x-b)^3]
v=[a/x,a/x^2, a/x^3, a/x^4]
F=v1[p]
G=v[q]
f=diff(F,x)
Ans=integrate(G(F)*f,x)
\end{sagesilent}

\latexProblemContent{
\begin{problem}

Compute the indefinite integral:

\input{Integral-Compute-0018.HELP.tex}

\[\int\;\sage{G(F)*f}\;dx = \answer{\sage{Ans}+C}\]
\end{problem}}%}
%%%%%%%%%%%%%%%%%%%%%%


%%%%%%%%%%%%%%%%%%%%%%%
%%\tagged{Ans@ShortAns, Type@Compute, Topic@Integral, Sub@Indefinite, Sub@Sub-u, Sub@Arctrig, File@0019}{
\begin{sagesilent}
a = NonZeroInt(-10,10)
b = NonZeroInt(-10,10)
p=Integer(randint(0,1))
v=[a*arcsin(b*x), a*arccos(b*x), a*arctan(b*x)]
F=v[p]
f=diff(F,x)
Ans=integrate(F*f,x)
\end{sagesilent}

\latexProblemContent{
\begin{problem}

Compute the indefinite integral:

\input{Integral-Compute-0019.HELP.tex}

\[\int\;\sage{F*f}\;dx = \answer{\sage{Ans}+C}\]
\end{problem}}%}
%%%%%%%%%%%%%%%%%%%%%%

%%%%%%%%%%%%%%%%%%%%%%%%%%%%%%%%%%%%%%%%%%%%%%%%%%%%%%%%%%%%%%%%%%%%%%%%%%%%%%%








%%%%%%%%%%%%%%%%%%%%%%%%%%%%%%%%%%%%%%%%%%%%%%%%%%%%%%%%%%%%%%%%%%%%%%%%%%%%%%%
%%%%%%%%%%%%%%%%%%%			MAC2312 Calculus 2			%%%%%%%%%%%%%%%%%%%%%%%
%%%%%%%%%%%%%%%%%%%%%%%%%%%%%%%%%%%%%%%%%%%%%%%%%%%%%%%%%%%%%%%%%%%%%%%%%%%%%%%






%%%%%%%%%%%%%%%%%%%%%%%%%%%%%%%%%%%%%%%%%%%%%%%%%%%%%%%%%%%%%%%%%%%%%%%%%%%%%%%


%%%%%%%%%%%%%%%%%%%%%%%
%%\tagged{Ans@ShortAns, Type@Compute, Topic@Integral, Sub@Poly, Sub@Exp, Sub@ByParts, File@0020}{
\begin{sagesilent}
# Define variables and constants/exponents
var('x')
p=NonZeroInt(2,5)
s=Integer(randint(-5,5))
q=NonZeroInt(-9,9)

#Define the two functions to be multiplied
f=(x-s)^p
g=exp(q*x)

#Compute their integral and factor the answer
A=integral(f*g,x)
Ans=A.factor()
\end{sagesilent}

\latexProblemContent{
\begin{problem}

Compute the following integral:

\input{Integral-Compute-0020.HELP.tex}

\[\int{\sage{f}\cdot\sage{g}\;dx} = \answer{\sage{Ans}+C}\]
\end{problem}}%}
%%%%%%%%%%%%%%%%%%%%%%


%%%%%%%%%%%%%%%%%%%%%%%
%%\tagged{Ans@ShortAns, Type@Compute, Topic@Integral, Sub@Poly, Sub@Exp, Sub@ByParts, File@0021}{
\begin{sagesilent}
# This is a simpler version of the previous problem
# Define variables and constants/exponents
var('x')
a=NonZeroInt(-9,9)
p=NonZeroInt(2,6)
q=NonZeroInt(-9,9)

#Define the two functions to be multiplied
f=(x)^p
g=exp(q*x)

#Compute their integral and factor the answer
A=integral(f*g,x)
Ans=A.factor()
\end{sagesilent}

\latexProblemContent{
\begin{problem}

Compute the following integral:

\input{Integral-Compute-0021.HELP.tex}

\[\int{\sage{f}\cdot\sage{g}\;dx} = \answer{\sage{Ans}+C}\]
\end{problem}}%}
%%%%%%%%%%%%%%%%%%%%%%


%%%%%%%%%%%%%%%%%%%%%%%
%%\tagged{Ans@ShortAns, Type@Compute, Topic@Integral, Sub@Log, Sub@Arctrig, Sub@ByParts, File@0022}{
\begin{sagesilent}
# Define variables and constants/exponents
var('x')
a=NonZeroInt(-9,9)
b=NonZeroInt(-9,9)
c=NonZeroInt(-9,9)

#Define and choose potential function archtypes
v=[a*log(x), b*arctan(x), c*x*arctan(x)]
p=Integer(randint(0,2))
f=v[p]

#Compute Integral
Ans=integral(f,x)
\end{sagesilent}

\latexProblemContent{
\begin{problem}

Compute the following integral:

\input{Integral-Compute-0022.HELP.tex}

\[\int{\sage{f}\;dx} = \answer{\sage{Ans}+C}\]
\end{problem}}%}
%%%%%%%%%%%%%%%%%%%%%%

%%%%%%%%%%%%%%%%%%%%%%%
%%\tagged{Ans@ShortAns, Type@Compute, Topic@Integral, Sub@Exp, Sub@Trig, Sub@ByParts, File@0023}{
\begin{sagesilent}
# Define variables and constants/exponents
var('x')
a=NonZeroInt(-9,9)
b=NonZeroInt(-9,9)
c=NonZeroInt(-9,9)

# Define and choose the function
v=[a*exp(b*x)*sin(c*x), a*exp(b*x)*cos(c*x)]
p=Integer(randint(0,1))
f=v[p]

#Compute the Integral
Ans=integral(f,x)
\end{sagesilent}

\latexProblemContent{
\begin{problem}

Compute the following integral:

\input{Integral-Compute-0023.HELP.tex}

\[\int{\sage{f}\;dx} = \answer{\sage{Ans}+C}\]
\end{problem}}%}
%%%%%%%%%%%%%%%%%%%%%%


%%%%%%%%%%%%%%%%%%%%%%%
%%\tagged{Ans@ShortAns, Type@Compute, Topic@Integral, Sub@Poly, Sub@Log, Sub@ByParts, File@0024}{
\begin{sagesilent}
# Define variables and constants/exponents
var('x')
a=NonZeroInt(-9,9)
b=NonZeroInt(1,9)
c=NonZeroInt(-9,9)

#Define the function
f=a*x^b*log(c*x)

#Compute the answer
Ans=integral(f,x)
\end{sagesilent}

\latexProblemContent{
\begin{problem}

Compute the following integral:

\input{Integral-Compute-0024.HELP.tex}

\[\int{\sage{f}\;dx} = \answer{\sage{Ans}+C}\]
\end{problem}}%}
%%%%%%%%%%%%%%%%%%%%%%



%%%%%%%%%%%%%%%%%%%%%%%
%%\tagged{Ans@ShortAns, Type@Compute, Topic@Integral, Sub@Poly, Sub@Exp, Sub@Trig, Sub@ByParts, File@0025}{
\begin{sagesilent}
# Define variable
var('x')

# Define coefficients, powers, etc
A = NonZeroInt(-5,5)
B = Integer(randint(-5,5))
C = NonZeroInt(-10,10)
pwr1 = Integer(randint(1,3))

func1 = (x-B)^pwr1

funcvec = [e^x, sin(x), cos(x)]
select = Integer(randint(0,2))

func2 = funcvec[select](x-A)

F(x) = C*func1(x)*func2(x)

Ans = (integral(F(x),x)).factor()

\end{sagesilent}

\latexProblemContent{
\begin{problem}
Evaluate:
\input{Integral-Compute-0025.HELP.tex}
\[\int \sage{F(x)} dx = \answer{\sage{Ans}+C}\]

\end{problem}}%}
%%%%%%%%%%%%%%%%%%%%%%


%%%%%%%%%%%%%%%%%%%%%%%
%%\tagged{Ans@ShortAns, Type@Compute, Topic@Integral, Sub@Poly, Sub@Exp, Sub@Trig, Sub@Definite, Sub@ByParts, File@0026}{
\begin{sagesilent}
# Define variable
var('x')
# Define coefficients, powers, etc
A = NonZeroInt(-5,5)
B = Integer(randint(-5,5))
C = NonZeroInt(-10,10)
Start = Integer(randint(-5,5))
End = Integer(randint(Start, 10))
pwr1 = Integer(randint(1,3))

# Outside function designed to be the differentiation term in by-parts
func1 = (x-B)^pwr1

# Inside function designed to be the integration term in by-parts

funcvec = [e^x, sin(x), cos(x)]
select = Integer(randint(0,2))

func2 = funcvec[select](x-A)

# Final integrand
F(x) = C*func1(x)*func2(x)

# Compute answer of integral from the Start point to the End point.
Ans = integral(F(x),x,Start,End)

\end{sagesilent}

\latexProblemContent{
\begin{problem}
Evaluate:
\input{Integral-Compute-0026.HELP.tex}
\[\int_{\sage{Start}}^{\sage{End}} \sage{F(x)} dx = \answer{\sage{Ans}+C}\]

\end{problem}}%}
%%%%%%%%%%%%%%%%%%%%%%


%%%%%%%%%%%%%%%%%%%%%%%
%%\tagged{Ans@ShortAns, Type@Compute, Topic@Integral, Sub@TrigInt, Sub@Trig, File@0027}{
\begin{sagesilent}
# Define variables and constants/exponents
var('x')
a=NonZeroInt(-9,9)
b=Integer(randint(1,4))
pwr=2*b+1

# Define function and latex output
v=[a*sin(x)^(pwr), a*cos(x)^(pwr)]
u=[latex(sin), latex(cos)]
p=Integer(randint(0,1))
f=v[p]

# Compute Answer
Ans=integral(f,x)
\end{sagesilent}

\latexProblemContent{
\begin{problem}

Compute the following integral:

\input{Integral-Compute-0027.HELP.tex}

\[\int{\sage{f}\;dx} = \answer{\sage{Ans}+C}\]
\end{problem}}%}
%%%%%%%%%%%%%%%%%%%%%%


%%%%%%%%%%%%%%%%%%%%%%%
%%\tagged{Ans@ShortAns, Type@Compute, Topic@Integral, Sub@TrigInt, Sub@Trig, File@0028}{
\begin{sagesilent}
# Define variables and constants/exponents
var('x')
a=NonZeroInt(-9,9)
b=Integer(randint(1,4))
pwr=2*b

# Define function and latex output
v=[a*sin(x)^(pwr), a*cos(x)^(pwr)]
u=[latex(sin), latex(cos)]
p=Integer(randint(0,1))
f=v[p]

# Compute Answer
Ans=integral(f,x)
\end{sagesilent}

\latexProblemContent{
\begin{problem}

Compute the following integral:

\input{Integral-Compute-0028.HELP.tex}

\[\int{\sage{f}\;dx} = \answer{\sage{Ans}+C}\]
\end{problem}}%}
%%%%%%%%%%%%%%%%%%%%%%


%%%%%%%%%%%%%%%%%%%%%%%
%%\tagged{Ans@ShortAns, Type@Compute, Topic@Integral, Sub@TrigInt, Sub@Trig, Sub@EvenOdd, File@0029}{
\begin{sagesilent}
# Define variables and constants/exponents
var('x')
a=NonZeroInt(-9,9)
b=Integer(randint(1,4))
c=Integer(randint(1,4))
pwrOdd=2*b+1
pwrEven=2*c

# Define function and latex output
v=[a*sin(x)^(pwrOdd)*cos(x)^(pwrEven), a*cos(x)^(pwrOdd)*sin(x)^(pwrEven)]
u=[latex(sin), latex(cos)]
w=[latex(cos), latex(sin)]
p=Integer(randint(0,1))
f=v[p]

# Compute Answer
Ans=integral(f,x)
\end{sagesilent}

\latexProblemContent{
\begin{problem}

Compute the following integral:

\input{Integral-Compute-0029.HELP.tex}

\[\int{\sage{a}\sage{u[p]}^{\sage{pwrOdd}}(x)\sage{w[p]}^{\sage{pwrEven}}(x)\;dx} = \answer{\sage{Ans}+C}\]
\end{problem}}%}
%%%%%%%%%%%%%%%%%%%%%%

%%%%%%%%%%%%%%%%%%%%%%%
%%\tagged{Ans@ShortAns, Type@Compute, Topic@Integral, Sub@TrigInt, Sub@Trig, Sub@EvenEven, File@0030}{
\begin{sagesilent}
# Define variables and constants/exponents
var('x')
a=NonZeroInt(-9,9)
b=Integer(randint(1,4))
c=Integer(randint(1,4))
pwrEven1=2*b
pwrEven2=2*c

# Define function and latex output
v=[a*sin(x)^(pwrEven1)*cos(x)^(pwrEven2), a*cos(x)^(pwrEven1)*sin(x)^(pwrEven2)]
u=[latex(sin), latex(cos)]
w=[latex(cos), latex(sin)]
p=Integer(randint(0,1))
f=v[p]

# Compute Answer
Ans=integral(f,x)
\end{sagesilent}

\latexProblemContent{
\begin{problem}

Compute the following integral:

\input{Integral-Compute-0030.HELP.tex}

\[\int{\sage{a}\sage{u[p]}^{\sage{pwrEven1}}(x)\sage{w[p]}^{\sage{pwrEven2}}(x)\;dx} = \answer{\sage{Ans}+C}\]
\end{problem}}%}
%%%%%%%%%%%%%%%%%%%%%%


%%%%%%%%%%%%%%%%%%%%%%%
%%\tagged{Ans@ShortAns, Type@Compute, Topic@Integral, Sub@TrigInt, Sub@Trig, File@0031}{
\begin{sagesilent}
# Define variables and constants/exponents
var('x')
a=NonZeroInt(-9,9)
b=NonZeroInt(-9,9)
c=NonZeroInt(-9,9)

# Define and choose function array
v=[a*sin(b*x)*cos(c*x), a*cos(b*x)*cos(c*x), a*sin(b*x)*sin(c*x)]
p=Integer(randint(0,2))
f=v[p]

# Compute Answer
Ans=integral(f,x)
\end{sagesilent}

\latexProblemContent{
\begin{problem}

Compute the following integral:

\input{Integral-Compute-0031.HELP.tex}

\[\int{\sage{f}\;dx} = \answer{\sage{Ans}+C}\]
\end{problem}}%}
%%%%%%%%%%%%%%%%%%%%%%


%%%%%%%%%%%%%%%%%%%%%%%
%%\tagged{Ans@ShortAns, Type@Compute, Topic@Integral, Sub@TrigInt, Sub@Trig, File@0032}{
\begin{sagesilent}
# Define variables and constants/exponents
var('x')
a=NonZeroInt(-9,9)
b=Integer(randint(1,9))
c=Integer(randint(1,4))
pwrEven=2*c

#Define fucntion and latex output
f=a*tan(x)^b*sec(x)^(pwrEven)
u=latex(tan)
w=latex(sec)

#Compute answer
Ans=integral(f,x)
\end{sagesilent}

\latexProblemContent{
\begin{problem}

Compute the following integral:

\input{Integral-Compute-0032.HELP.tex}

\[\int{\sage{a}\sage{u}^{\sage{b}}(x)\sage{w}^{\sage{pwrEven}}(x)\;dx} = \answer{\sage{Ans}+C}\]
\end{problem}}%}
%%%%%%%%%%%%%%%%%%%%%%


%%%%%%%%%%%%%%%%%%%%%%%
%%\tagged{Ans@ShortAns, Type@Compute, Topic@Integral, Sub@TrigInt, Sub@Trig, Sub@OddOdd, File@0033}{
\begin{sagesilent}
# Define variables and constants/exponents
var('x')
a=NonZeroInt(-9,9)
b=Integer(randint(0,4))
c=Integer(randint(0,4))
pwrOdd1=2*b+1
pwrOdd2=2*c+1

# Define function and latex output
f=a*tan(x)^(pwrOdd1)*sec(x)^(pwrOdd2)
u=latex(tan)
w=latex(sec)

#Compute answer
Ans=integral(f,x)
\end{sagesilent}

\latexProblemContent{
\begin{problem}

Compute the following integral:

\input{Integral-Compute-0033.HELP.tex}

\[\int{\sage{a}\sage{u}^{\sage{pwrOdd1}}(x)\sage{w}^{\sage{pwrOdd2}}(x)\;dx} = \answer{\sage{Ans}+C}\]
\end{problem}}%}
%%%%%%%%%%%%%%%%%%%%%%


%%%%%%%%%%%%%%%%%%%%%%%
%%\tagged{Ans@ShortAns, Type@Compute, Topic@Integral, Sub@TrigInt, Sub@Trig, Sub@EvenOdd, File@0034}{
\begin{sagesilent}
# Define variables and constants/exponents
var('x')
a=NonZeroInt(-9,9)
b=Integer(randint(1,4))
c=Integer(randint(0,4))
pwrEven=2*b
pwrOdd=2*c+1

# Define function and latex output
f=a*tan(x)^(pwrEven)*sec(x)^(pwrOdd)
u=latex(tan)
w=latex(sec)

#Compute answer
Ans=integral(f,x)
\end{sagesilent}

\latexProblemContent{
\begin{problem}

Compute the following integral:

\input{Integral-Compute-0034.HELP.tex}

\[\int{\sage{a}\sage{u}^{\sage{pwrEven}}(x)\sage{w}^{\sage{pwrOdd}}(x)\;dx} = \answer{\sage{Ans}+C}\]
\end{problem}}%}
%%%%%%%%%%%%%%%%%%%%%%

%%%%%%%%%%%%%%%%%%%%%%%
%%\tagged{Ans@ShortAns, Type@Compute, Topic@Integral, Sub@TrigInt, Sub@Trig, File@0035}{
\begin{sagesilent}
# First get all the random rolls for the constants, powers, and whether or not it's going to have a negative exponent.

n21 = RandInt(0,2)
pwr = (2 * n21) + 3
flip = RandInt(0,1)
freq = NonZeroInt(-5,5)
Coef = NonZeroInt(-5,5)

# Set up the vectors to grab trig functions.

trigvec = [sin(x), cos(x), sec(x)*tan(x), tan(x), csc(x)*cot(x), cot(x)]

select = RandInt(0,5)

# The function that sage uses to calculate the answer
ToCalcInt = (Coef*trigvec[select](freq*x))^(pwr)

# Now we need to create the function the student sees to integrate (the actual integrand). If flip=1 we have a negative exponent, so we convert each trig function to it's co-function in the trigvec1 and trigvec2 to pull appropriate functions. Otherwise, we can just use the same function that sage is using to calculate. If we didn't use cofunctions, sage would helpfully negate the negatives and kill it.

if flip == 1:
    Intvec = [csc(x), sec(x), cos(x)*cot(x), cot(x), sin(x)*tan(x), tan(x)]
else:
    Intvec = [sin(x), cos(x), sec(x)*tan(x), tan(x), csc(x)*cot(x), cot(x)]

# Actually create the integrand with the above vectors
Integrand = (Coef*Intvec[select](freq*x))

# Calculate the actual power separately to prevent sage from trying to cancel negatives.
Integrandpower = (pwr)*((-1)^flip)

CalcInt = integral(ToCalcInt(x),x)
Ans = CalcInt#.simplify()
\end{sagesilent}

\latexProblemContent{
\begin{problem}

Evaluate the integral. \\

\input{Integral-Compute-0035.HELP.tex}

\[
\int \left(\sage{Integrand}\right)^{\sage{Integrandpower}} dx=\answer{\sage{Ans}+C}
\]
\end{problem}}%}
%%%%%%%%%%%%%%%%%%%%%%


%%%%%%%%%%%%%%%%%%%%%%%
%%\tagged{Ans@ShortAns, Type@Compute, Topic@Integral, Sub@TrigInt, Sub@Trig, Sub@EvenOdd, File@0036}{
\begin{sagesilent}
# First get all the random rolls for the constants, powers, and whether or not it's going to have a negative exponent.

n1 = RandInt(1,3)
power1 = 2*n1
n2 = RandInt(0, 4-n1) # Curb the size of the power combination to avoid giant ans
power2 = 2*n2 + 1
flip = RandInt(0,1)
freq = NonZeroInt(-5,5)
Coef = NonZeroInt(-5,5)

#Set up the vectors to grab matching trig functions, first one is even power, second is odd power.

trigvec1 = [sin(x), cos(x), sec(x), csc(x)]
trigvec2 = [cos(x), sin(x), tan(x), cot(x)]

select = RandInt(0,3)

#Grab base trig functions for writing sanity
F1 = trigvec1[select]
F2 = trigvec2[select]

#This is what sage uses to calculate the answer. For readability it helps to make sure it's integrating a positive power, so we assume the power is positive, and adjust the negative part in the Integrand part.

ToCalcInt = Coef*(F1(freq*x)^power1)*(F2(freq*x)^power2)


#Now we need to create the function the student sees to integrate (the actual integrand). If flip=1 we have a negative exponent, so we convert each trig function to it's co-function in the trigvec1 and trigvec2 to pull appropriate functions. Otherwise, we can just use the same function that sage is using to calculate. If we didn't use cofunctions, sage would helpfully negate the negatives and kill it.

if flip == 1:
    Intvec1 = [csc(x), sec(x), cos(x), sin(x)]
    Intvec2 = [sec(x), csc(x), cot(x), tan(x)]
    Integrand = (Coef) * (Intvec1[select](freq*x))^(-power1) * (Intvec2[select](freq*x))^(-power2)
else:
    Integrand = ToCalcInt


# Now have sage calculate the answer
CalcInt = integral(ToCalcInt(x),x)

# Write the answer for ximera. In this case, the 'simplify' just causes it to be worse.
Ans = CalcInt
\end{sagesilent}

\latexProblemContent{
\begin{problem}

Evaluate the integral. \\

\input{Integral-Compute-0036.HELP.tex}

\[
\int \sage{Integrand} dx=\answer{\sage{Ans}+C}
\]

\end{problem}}%}
%%%%%%%%%%%%%%%%%%%%%%

%%%%%%%%%%%%%%%%  17 Problems so far... finished up to trig integrals %%%%%%%


%%%%%%%%%%%%%%%%%%%%%%%
%%\tagged{Ans@ShortAns, Type@Compute, Topic@Integral, Sub@TrigSub, Sub@Trig, File@0037}{
\begin{sagesilent}
# Define variables and constants/exponents
var('x')
a=NonZeroInt(-9,9)
b=NonZeroInt(-9,9)

# Choose denominator for trig sub
v=[sqrt(b^2-x^2), sqrt(b^2+x^2), sqrt(x^2-b^2)]
p=Integer(randint(0,2))
denom=v[p]
f=a/denom

# Compute answer
Ans=integral(f,x)
\end{sagesilent}

\latexProblemContent{
\begin{problem}

Compute the following integral:

\input{Integral-Compute-0037.HELP.tex}

\[\int{\sage{f}\;dx} = \answer{\sage{Ans}+C}\]
\end{problem}}%}
%%%%%%%%%%%%%%%%%%%%%%

%%%%%%%%%%%%%%%%%%%%%%%
%%\tagged{Ans@ShortAns, Type@Compute, Topic@Integral, Sub@TrigSub, Sub@Trig, File@0038}{
\begin{sagesilent}
# Define variables and constants/exponents
var('x')
a=NonZeroInt(-9,9)
b=NonZeroInt(-9,9)

# Choose denominator for trig sub
v=[sqrt(b^2-x^2), sqrt(b^2+x^2), sqrt(x^2-b^2)]
p=Integer(randint(0,2))
denom=v[p]
f=a/(x^2*denom)

# Compute answer
Ans=integral(f,x)
\end{sagesilent}

\latexProblemContent{
\begin{problem}

Compute the following integral:

\input{Integral-Compute-0038.HELP.tex}

\[\int{\dfrac{\sage{a}}{x^2\sage{denom}}\;dx} = \answer{\sage{Ans}+C}\]
\end{problem}}%}
%%%%%%%%%%%%%%%%%%%%%%


%%%%%%%%%%%%%%%%%%%%%%%
%%\tagged{Ans@ShortAns, Type@Compute, Topic@Integral, Sub@TrigSub, Sub@Trig, File@0039}{
\begin{sagesilent}
# Define variables and constants/exponents
var('x')
a=NonZeroInt(-9,9)
b=NonZeroInt(-9,9)

# Choose denominator for trig sub
v=[sqrt(b^2-x^2), sqrt(b^2+x^2), sqrt(x^2-b^2)]
p=Integer(randint(0,2))
denom=v[p]
f=a*x^3/denom

# Compute answer
Ans=integral(f,x)
\end{sagesilent}

\latexProblemContent{
\begin{problem}

Compute the following integral:

\input{Integral-Compute-0039.HELP.tex}

\[\int{\sage{f}\;dx} = \answer{\sage{Ans}+C}\]
\end{problem}}%}
%%%%%%%%%%%%%%%%%%%%%%

%%%%%%%%%%%%%%%%%%%%%%%
%%\tagged{Ans@ShortAns, Type@Compute, Topic@Integral, Sub@TrigSub, Sub@Trig, File@0040}{
\begin{sagesilent}
# Define variables and constants/exponents
var('x')
a=NonZeroInt(-9,9)
b=NonZeroInt(-9,9)

# Choose denominator for trig sub
v=[sqrt(b^2-x^2), sqrt(x^2-b^2)]
p=Integer(randint(0,1))
denom=v[p]
f=a*x^2/denom

# Compute answer
Ans=integral(f,x)
\end{sagesilent}

\latexProblemContent{
\begin{problem}

Compute the following integral:

\input{Integral-Compute-0040.HELP.tex}

\[\int{\sage{f}\;dx} = \answer{\sage{Ans}+C}\]
\end{problem}}%}
%%%%%%%%%%%%%%%%%%%%%%

%%%%%%%%%%%%%%%%%%%%%%%
%%\tagged{Ans@ShortAns, Type@Compute, Topic@Integral, Sub@TrigSub, Sub@Trig, File@0041}{
\begin{sagesilent}
# Define variables and constants/exponents
var('x')
a=NonZeroInt(-9,9)
b=NonZeroInt(-9,9)

# Choose denominator for trig sub
v=[sqrt(b^2-x^2), sqrt(x^2-b^2)]
p=Integer(randint(0,1))
denom=v[p]
f=a*x/denom

# Compute answer
Ans=integral(f,x)
\end{sagesilent}

\latexProblemContent{
\begin{problem}

Compute the following integral:

\input{Integral-Compute-0041.HELP.tex}

\[\int{\sage{f}\;dx} = \answer{\sage{Ans}+C}\]
\end{problem}}%}
%%%%%%%%%%%%%%%%%%%%%%


%%%%%%%%%%%%%%%%%%%%%%%
%%\tagged{Ans@ShortAns, Type@Compute, Topic@Integral, Sub@TrigSub, Sub@Trig, File@0042}{
\begin{sagesilent}
# Define variables and constants/exponents
var('x')
a=NonZeroInt(-9,9)
b=NonZeroInt(-9,9)
c=NonZeroInt(-9,9)
p=Integer(randint(0,2))

# Choose denominator and function
v=[sqrt(c^2-(x-b)^2), sqrt(c^2+(x-b)^2), sqrt((x-b)^2-c^2)]
denom=expand(v[p])
f=a/denom

# Compute answer
Ans=integral(f,x)
\end{sagesilent}

\latexProblemContent{
\begin{problem}

Compute the following integral:

\input{Integral-Compute-0042.HELP.tex}

\[\int{\sage{f}\;dx} = \answer{\sage{Ans}+C}\]
\end{problem}}%}
%%%%%%%%%%%%%%%%%%%%%%


%%%%%%%%%%%%%%%%%%%%%%%
%%\tagged{Ans@ShortAns, Type@Compute, Topic@Integral, Sub@TrigSub, Sub@Trig, File@0043}{
\begin{sagesilent}
# Define variables and constants/exponents
var('x')
a=NonZeroInt(-9,9)
b=NonZeroInt(-9,9)
c=NonZeroInt(-9,9)
p=Integer(randint(0,2))

# Choose denominator and function
v=[sqrt(c^2-(x-b)^2), sqrt(c^2+(x-b)^2), sqrt((x-b)^2-c^2)]
denom=expand(v[p])
f=a*x/denom

# Compute answer
Ans=integral(f,x)
\end{sagesilent}

\latexProblemContent{
\begin{problem}

Compute the following integral:

\input{Integral-Compute-0043.HELP.tex}

\[\int{\sage{f}\;dx} = \answer{\sage{Ans}+C}\]
\end{problem}}%}
%%%%%%%%%%%%%%%%%%%%%%


%%%%%%%%%%%%%%%%%%%%%%%
%%\tagged{Ans@ShortAns, Type@Compute, Topic@Integral, Sub@TrigSub, Sub@Trig, File@0044}{
\begin{sagesilent}
# Define variables and constants/exponents
var('x')
a=NonZeroInt(-9,9)
b=NonZeroInt(-9,9)
c=NonZeroInt(-9,9)
p=Integer(randint(0,2))

# Choose denominator and function
v=[sqrt(c^2-(x-b)^2), sqrt(c^2+(x-b)^2), sqrt((x-b)^2-c^2)]
denom=expand(v[p])
f=a*x^2/denom

# Compute answer
Ans=integral(f,x)
\end{sagesilent}

\latexProblemContent{
\begin{problem}

Compute the following integral:

\input{Integral-Compute-0044.HELP.tex}

\[\int{\sage{f}\;dx} = \answer{\sage{Ans}+C}\]
\end{problem}}%}
%%%%%%%%%%%%%%%%%%%%%%

%%%%%%%%%%%%%%%%%%%%%%%
%%\tagged{Ans@ShortAns, Type@Compute, Topic@Integral, Sub@TrigSub, Sub@Trig, File@0045}{
\begin{sagesilent}
# Define variables and constants/exponents
var('x')
a=NonZeroInt(-9,9)
b=NonZeroInt(-9,9)
c=NonZeroInt(-9,9)
p=Integer(randint(0,2))

# Choose denominator and function
v=[sqrt(c^2-(x-b)^2), sqrt(c^2+(x-b)^2), sqrt((x-b)^2-c^2)]
denom=expand(v[p])
f=denom

# Compute answer
Ans=integral(f,x)
\end{sagesilent}

\latexProblemContent{
\begin{problem}

Compute the following integral:

\input{Integral-Compute-0045.HELP.tex}

\[\int{\sage{f}\;dx} = \answer{\sage{Ans}+C}\]
\end{problem}}%}
%%%%%%%%%%%%%%%%%%%%%%


%%%%%%%%%%%%%%%%%%%%%%%
%%\tagged{Ans@ShortAns, Type@Compute, Topic@Integral, Sub@TrigSub, Sub@Trig, File@0046}{
\begin{sagesilent}
# Define variables and constants/exponents
var('x')
a=NonZeroInt(-9,9)
b=NonZeroInt(-9,9)
c=NonZeroInt(-9,9)
p=Integer(randint(0,2))

# Choose denominator and function
v=[sqrt(c^2-(x-b)^2), sqrt(c^2+(x-b)^2), sqrt((x-b)^2-c^2)]
denom=expand(v[p])
f=a/(denom)^3

# Compute answer
Ans=integral(f,x)
\end{sagesilent}

\latexProblemContent{
\begin{problem}

Compute the following integral:

\input{Integral-Compute-0046.HELP.tex}

\[\int{\sage{f}\;dx} = \answer{\sage{Ans}+C}\]
\end{problem}}%}
%%%%%%%%%%%%%%%%%%%%%%

%%%%%%%%%%%%%%%%%%%%%%%
%%\tagged{Ans@ShortAns, Type@Compute, Topic@Integral, Sub@TrigSub, Sub@Trig, File@0047}{
\begin{sagesilent}
# Define variables and constants/exponents
var('x')
a=NonZeroInt(-9,9)
b=NonZeroInt(-9,9)
c=NonZeroInt(-9,9)
p=Integer(randint(0,2))

# Choose denominator and function
v=[sqrt(c^2-(x-b)^2), sqrt(c^2+(x-b)^2), sqrt((x-b)^2-c^2)]
denom=expand(v[p])
f=a/(denom)^5

# Compute answer
Ans=integral(f,x)
\end{sagesilent}

\latexProblemContent{
\begin{problem}

Compute the following integral:

\input{Integral-Compute-0047.HELP.tex}

\[\int{\sage{f}\;dx} = \answer{\sage{Ans}+C}\]
\end{problem}}%}
%%%%%%%%%%%%%%%%%%%%%%


%%%%%%%%%%%%%%%%%%%%%%%
%%\tagged{Ans@ShortAns, Type@Compute, Topic@Integral, Sub@TrigSub, Sub@Trig, File@0048}{
\begin{sagesilent}
# Define variables and constants/exponents
var('x')
a=NonZeroInt(-9,9)
b=NonZeroInt(-9,9)

# Choose denominator for trig sub
v=[sqrt(b^2-x^2), sqrt(b^2+x^2), sqrt(x^2-b^2)]
u=[x,x^2]
p=Integer(randint(0,2))
q=Integer(randint(0,1))
denom=v[p]
pwr=u[q]
f=a*denom/pwr

# Compute answer
Ans=integral(f,x)
\end{sagesilent}

\latexProblemContent{
\begin{problem}

Compute the following integral:

\input{Integral-Compute-0048.HELP.tex}

\[\int{\sage{f}\;dx} = \answer{\sage{Ans}+C}\]
\end{problem}}%}
%%%%%%%%%%%%%%%%%%%%%%


%%%%%%%%%%%%%%%%%%%%%%%
%%\tagged{Ans@ShortAns, Type@Compute, Topic@Integral, Sub@TrigSub, Sub@Trig, File@0049}{
\begin{sagesilent}
# Define coefficients, powers, etc
A = NonZeroInt(-5,5)
B = RandInt(-5,5)
C = NonZeroInt(-10,10)


# Vector of possible arctrig functions
funcvec = [arcsin(x), arccos(x), arctan(x), arcsec(x)]
select = RandInt(0,3)

# Select our arctrig function, then take the derivative to get the integrand.
Func1 = funcvec[select]
F(x) = C*Func1(A*x + B)

Integrand(x) = derivative(F(x),x)
\end{sagesilent}

\latexProblemContent{
\begin{problem}

Compute the following integral:

\input{Integral-Compute-0049.HELP.tex}

\[
\int \sage{Integrand(x)} dx = \answer{\sage{F(x)}+C}
\]

\end{problem}}%}
%%%%%%%%%%%%%%%%%%%%%%

%%%%%%%%%%%%%%%%%%%%%%%
%%\tagged{Ans@ShortAns, Type@Compute, Topic@Integral, Sub@PartialFractions, Sub@LinearFactors, File@0050}{
\begin{sagesilent}
# Define variables and constants/exponents
var('x')
a=NonZeroInt(-9,9)
b=NonZeroInt(-9,9)
c=NonZeroInt(-9,9)
d=NonZeroInt(-9,9)


# Define Rational Expression
f=(a*x+b)/((x-c)*(x-d))

# Compute Answer
Ans=integral(f,x)
\end{sagesilent}

\latexProblemContent{
\begin{problem}

Compute the following integral:

\input{Integral-Compute-0050.HELP.tex}

\[\int{\sage{f}\;dx} = \answer{\sage{Ans}+C}\]
\end{problem}}%}
%%%%%%%%%%%%%%%%%%%%%%


%%%%%%%%%%%%%%%%%%%%%%%
%%\tagged{Ans@ShortAns, Type@Compute, Topic@Integral, Sub@PartialFractions, Sub@LinearFactors, File@0051}{
\begin{sagesilent}
# Define variables and constants/exponents
var('x')
a=NonZeroInt(-9,9)
b=NonZeroInt(-9,9)
c=NonZeroInt(-9,9)
d=NonZeroInt(-9,9)
e=NonZeroInt(-9,9)


# Define Rational Expression
f=(a*x+b)/((x-c)*(x-d)*(x-e))

# Compute Answer
Ans=integral(f,x)
\end{sagesilent}

\latexProblemContent{
\begin{problem}

Compute the following integral:

\input{Integral-Compute-0051.HELP.tex}

\[\int{\sage{f}\;dx} = \answer{\sage{Ans}+C}\]
\end{problem}}%}
%%%%%%%%%%%%%%%%%%%%%%

%%%%%%%%%%%%%%%%%%%%%%%
%%\tagged{Ans@ShortAns, Type@Compute, Topic@Integral, Sub@PartialFractions, Sub@LinearFactors, File@0052}{
\begin{sagesilent}
# Define variables and constants/exponents
var('x')
a=NonZeroInt(-9,9)
b=NonZeroInt(-9,9)
c=NonZeroInt(-9,9)
d=NonZeroInt(-9,9)
e=NonZeroInt(-9,9)


# Define Rational Expression
f=(a*x+b)/((c*x+d)*(x-e))

# Compute Answer
Ans=integral(f,x)
\end{sagesilent}

\latexProblemContent{
\begin{problem}

Compute the following integral:

\input{Integral-Compute-0052.HELP.tex}

\[\int{\sage{f}\;dx} = \answer{\sage{Ans}+C}\]
\end{problem}}%}
%%%%%%%%%%%%%%%%%%%%%%


%%%%%%%%%%%%%%%%%%%%%%%
%%\tagged{Ans@ShortAns, Type@Compute, Topic@Integral, Sub@PartialFractions, Sub@LinearFactors, File@0053}{
\begin{sagesilent}
# Define variables and constants/exponents
var('x')
a=NonZeroInt(-9,9)
b=NonZeroInt(-9,9)
c=NonZeroInt(-9,9)
d=NonZeroInt(-9,9)


# Define Rational Expression
Numer=expand((x-a)*(x-b))
denom=expand((x-c)*(x-d))
f=Numer/denom

# Compute Answer
Ans=integral(f,x)
\end{sagesilent}

\latexProblemContent{
\begin{problem}

Compute the following integral:

\input{Integral-Compute-0053.HELP.tex}

\[\int{\sage{f}\;dx} = \answer{\sage{Ans}+C}\]
\end{problem}}%}
%%%%%%%%%%%%%%%%%%%%%%

%%%%%%%%%%%%%%%%%%%%%%%
%%\tagged{Ans@ShortAns, Type@Compute, Topic@Integral, Sub@PartialFractions, Sub@LinearFactors, File@0054}{
\begin{sagesilent}
# Define variables and constants/exponents
var('x')
a=NonZeroInt(-9,9)
b=NonZeroInt(-9,9)
c=NonZeroInt(-9,9)
d=NonZeroInt(-9,9)
e=NonZeroInt(-9,9)
g=NonZeroInt(-9,9)


# Define Rational Expression
Numer=expand((a*x+b)*(x-g))
denom=(x-c)*(x-d)*(x-e)
f=Numer/denom

# Compute Answer
Ans=integral(f,x)
\end{sagesilent}

\latexProblemContent{
\begin{problem}

Compute the following integral:

\input{Integral-Compute-0054.HELP.tex}

\[\int{\sage{f}\;dx} = \answer{\sage{Ans}+C}\]
\end{problem}}%}
%%%%%%%%%%%%%%%%%%%%%%


%%%%%%%%%%%%%%%%%%%%%%%
%%\tagged{Ans@ShortAns, Type@Compute, Topic@Integral, Sub@PartialFractions, Sub@LinearFactors, File@0055}{
\begin{sagesilent}
# Define variables and constants/exponents
var('x')
a=NonZeroInt(-9,9)
b=NonZeroInt(-9,9)
c=NonZeroInt(-9,9)

# Define Rational Expression
Numer=expand((a*x+b))
denom=(x-c)^2
f=Numer/denom

# Compute Answer
Ans=integral(f,x)
\end{sagesilent}

\latexProblemContent{
\begin{problem}

Compute the following integral:

\input{Integral-Compute-0055.HELP.tex}

\[\int{\sage{f}\;dx} = \answer{\sage{Ans}+C}\]
\end{problem}}%}
%%%%%%%%%%%%%%%%%%%%%%


%%%%%%%%%%%%%%%%%%%%%%%
%%\tagged{Ans@ShortAns, Type@Compute, Topic@Integral, Sub@PartialFractions, Sub@LinearFactors, File@0056}{
\begin{sagesilent}
# Define variables and constants/exponents
var('x')
a=NonZeroInt(-9,9)
b=NonZeroInt(-9,9)
c=NonZeroInt(-9,9)
d=NonZeroInt(-9,9)

# Define Rational Expression
Numer=expand((a*x+b)*(x-d))
denom=(x-c)^2
f=Numer/denom

# Compute Answer
Ans=integral(f,x)
\end{sagesilent}

\latexProblemContent{
\begin{problem}

Compute the following integral:

\input{Integral-Compute-0056.HELP.tex}

\[\int{\sage{f}\;dx} = \answer{\sage{Ans}+C}\]
\end{problem}}%}
%%%%%%%%%%%%%%%%%%%%%%



%%%%%%%%%%%%%%%%%%%%%%%
%%\tagged{Ans@ShortAns, Type@Compute, Topic@Integral, Sub@PartialFractions, Sub@LinearFactors, File@0057}{
\begin{sagesilent}
# Define variables and constants/exponents
var('x')
a=NonZeroInt(-9,9)
b=NonZeroInt(-9,9)
c=NonZeroInt(-9,9)
d=NonZeroInt(-9,9)

# Define Rational Expression
Numer=expand((a*x+b))
denom=(x-c)^2*(x-d)
f=Numer/denom

# Compute Answer
Ans=integral(f,x)
\end{sagesilent}

\latexProblemContent{
\begin{problem}

Compute the following integral:

\input{Integral-Compute-0057.HELP.tex}

\[\int{\sage{f}\;dx} = \answer{\sage{Ans}+C}\]
\end{problem}}%}
%%%%%%%%%%%%%%%%%%%%%%

%%%%%%%%%%%%%%%%%%%%%%%
%%\tagged{Ans@ShortAns, Type@Compute, Topic@Integral, Sub@PartialFractions, Sub@LinearFactors, File@0058}{
\begin{sagesilent}
# Define variables and constants/exponents
var('x')
a=NonZeroInt(-9,9)
b=NonZeroInt(-9,9)
c=NonZeroInt(-9,9)
d=NonZeroInt(-9,9)
e=NonZeroInt(-9,9)

# Define Rational Expression
Numer=expand((a*x+b)*(x-e))
denom=(x-c)^2*(x-d)
f=Numer/denom

# Compute Answer
Ans=integral(f,x)
\end{sagesilent}

\latexProblemContent{
\begin{problem}

Compute the following integral:

\input{Integral-Compute-0058.HELP.tex}

\[\int{\sage{f}\;dx} = \answer{\sage{Ans}+C}\]
\end{problem}}%}
%%%%%%%%%%%%%%%%%%%%%%


%%%%%%%%%%%%%%%%%%%%%%%
%%\tagged{Ans@ShortAns, Type@Compute, Topic@Integral, Sub@PartialFractions, Sub@LinearFactors, File@0059}{
\begin{sagesilent}
# Define variables and constants/exponents
var('x')
a=NonZeroInt(-9,9)
b=NonZeroInt(-9,9)
c=NonZeroInt(-9,9)
d=NonZeroInt(-9,9)

# Define Rational Expression
Numer=expand((a*x+b))
denom=(x-c)^2*(x-d)^2
f=Numer/denom

# Compute Answer
Ans=integral(f,x)
\end{sagesilent}

\latexProblemContent{
\begin{problem}

Compute the following integral:

\input{Integral-Compute-0059.HELP.tex}

\[\int{\sage{f}\;dx} = \answer{\sage{Ans}+C}\]
\end{problem}}%}
%%%%%%%%%%%%%%%%%%%%%%

%%%%%%%%%%%%%%%%%%%%%%%
%%\tagged{Ans@ShortAns, Type@Compute, Topic@Integral, Sub@PartialFractions, Sub@LinearFactors, File@0060}{
\begin{sagesilent}
# Define variables and constants/exponents
var('x')
a=NonZeroInt(-9,9)
b=NonZeroInt(-9,9)
c=NonZeroInt(-9,9)
d=NonZeroInt(-9,9)

# Define Rational Expression
Numer=expand((a*x+b)*(x-d))
denom=(x-c)^3
f=Numer/denom

# Compute Answer
Ans=integral(f,x)
\end{sagesilent}

\latexProblemContent{
\begin{problem}

Compute the following integral:

\input{Integral-Compute-0060.HELP.tex}

\[\int{\sage{f}\;dx} = \answer{\sage{Ans}+C}\]
\end{problem}}%}
%%%%%%%%%%%%%%%%%%%%%%

%%%%%%%%%%%%%%%%%%%%%%%
%%\tagged{Ans@ShortAns, Type@Compute, Topic@Integral, Sub@PartialFractions, Sub@IrreducibleQuad, Sub@LinearFactors, File@0061}{
\begin{sagesilent}
# Define variables and constants/exponents
var('x')
a=NonZeroInt(-9,9)
b=NonZeroInt(-9,9)
c=NonZeroInt(-9,9)
d=NonZeroInt(-9,9)
e=NonZeroInt(-9,9)
g=NonZeroInt(-9,9)

# Compute the discriminant and make sure we have an irreducible quadratic
Disc=b^2-4*a*c

while Disc>=0:
   a=NonZeroInt(-9,9)
   b=NonZeroInt(-9,9)
   Disc=b^2-4*a*c


# Define Rational Expression
Numer=expand(e*x-g)
denom=(a*x^2+b*x+c)*(x-d)
f=Numer/denom

# Compute Answer
Ans=integral(f,x)
\end{sagesilent}

\latexProblemContent{
\begin{problem}

Compute the following integral:

\input{Integral-Compute-0061.HELP.tex}

\[\int{\sage{f}\;dx} = \answer{\sage{Ans}+C}\]
\end{problem}}%}
%%%%%%%%%%%%%%%%%%%%%%

%%%%%%%%%%%%%%%%%%%%%%%
%%\tagged{Ans@ShortAns, Type@Compute, Topic@Integral, Sub@PartialFractions, Sub@IrreducibleQuad, Sub@LinearFactors, File@0062}{
\begin{sagesilent}
# Define variables and constants/exponents
var('x')
a=NonZeroInt(-9,9)
b=NonZeroInt(-9,9)
c=NonZeroInt(-9,9)
d=NonZeroInt(-9,9)
e=NonZeroInt(-9,9)
g=NonZeroInt(-9,9)
h=NonZeroInt(-9,9)

# Compute the discriminant and make sure we have an irreducible quadratic
Disc=b^2-4*a*c

while Disc>=0:
   a=NonZeroInt(-9,9)
   b=NonZeroInt(-9,9)
   Disc=b^2-4*a*c


# Define Rational Expression
Numer=expand((e*x-g)*(x-h))
denom=(a*x^2+b*x+c)*(x-d)
f=Numer/denom

# Compute Answer
Ans=integral(f,x)
\end{sagesilent}

\latexProblemContent{
\begin{problem}

Compute the following integral:

\input{Integral-Compute-0062.HELP.tex}

\[\int{\sage{f}\;dx} = \answer{\sage{Ans}+C}\]
\end{problem}}%}
%%%%%%%%%%%%%%%%%%%%%%

%%%%%%%%%%%%%%%%%%%%%%%
%%\tagged{Ans@ShortAns, Type@Compute, Topic@Integral, Sub@PartialFractions, Sub@IrreducibleQuad, Sub@LinearFactors, File@0063}{
\begin{sagesilent}
# Define variables and constants/exponents
var('x')
a=NonZeroInt(-9,9)
b=NonZeroInt(-9,9)
c=NonZeroInt(-9,9)
d=NonZeroInt(-9,9)
e=NonZeroInt(-9,9)
g=NonZeroInt(-9,9)

# Compute the discriminant and make sure we have an irreducible quadratic
Disc=b^2-4*a*c

while Disc>=0:
   a=NonZeroInt(-9,9)
   b=NonZeroInt(-9,9)
   Disc=b^2-4*a*c


# Define Rational Expression
Numer=expand((e*x-g))
denom=(a*x^2+b*x+c)*(x-d)^2
f=Numer/denom

# Compute Answer
Ans=integral(f,x)
\end{sagesilent}

\latexProblemContent{
\begin{problem}

Compute the following integral:

\input{Integral-Compute-0063.HELP.tex}

\[\int{\sage{f}\;dx} = \answer{\sage{Ans}+C}\]
\end{problem}}%}
%%%%%%%%%%%%%%%%%%%%%%

%%%%%%%%%%%%%%%%%%%%%%%
%%\tagged{Ans@ShortAns, Type@Compute, Topic@Integral, Sub@PartialFractions, Sub@IrreducibleQuad, File@0064}{
\begin{sagesilent}
# Define variables and constants/exponents
var('x')
a=NonZeroInt(-9,9)
b=NonZeroInt(-9,9)
c=NonZeroInt(-9,9)
e=NonZeroInt(-9,9)
g=NonZeroInt(-9,9)

# Compute the discriminant and make sure we have an irreducible quadratic
Disc=b^2-4*a*c

while Disc>=0:
   a=NonZeroInt(-9,9)
   b=NonZeroInt(-9,9)
   Disc=b^2-4*a*c


# Define Rational Expression
Numer=expand((e*x-g))
denom=(a*x^2+b*x+c)^2
f=Numer/denom

# Compute Answer
Ans=integral(f,x)
\end{sagesilent}

\latexProblemContent{
\begin{problem}

Compute the following integral:

\input{Integral-Compute-0064.HELP.tex}

\[\int{\sage{f}\;dx} = \answer{\sage{Ans}+C}\]
\end{problem}}%}
%%%%%%%%%%%%%%%%%%%%%%


%%%%%%%%%%%%%%%%%%%%%%%
%%\tagged{Ans@ShortAns, Type@Compute, Topic@Integral, Sub@TrigSub, File@0065}{
\begin{sagesilent}
# Define variables and constants/exponents
var('x')
a=NonZeroInt(-9,9)
b=NonZeroInt(1,4)
c=NonZeroInt(-9,9)


# Define Rational Expression
Numer=a
denom=sqrt(b^2*x^2+c)

f=Numer/(x*denom)

# Compute Answer
Ans=integral(f,x)
\end{sagesilent}

\latexProblemContent{
\begin{problem}

Compute the following integral:

\input{Integral-Compute-0065.HELP.tex}

\[\int{\dfrac{\sage{a}}{x\cdot\sage{denom}}\;dx} = \answer{\sage{Ans}+C}\]
\end{problem}}%}
%%%%%%%%%%%%%%%%%%%%%%

%%%%%%%%%%%%%%%%%%%%%%%
%%\tagged{Ans@ShortAns, Type@Compute, Topic@Integral, Sub@TrigSub, Sub@Sub-u, File@0066}{
\begin{sagesilent}
# Define variables and constants/exponents
var('x')
a=NonZeroInt(-9,9)
b=NonZeroInt(-9,9)

# Define Rational Expression
Numer=a*sqrt(x)
denom=(b+x^3)

f=Numer/denom

# Compute Answer
Ans=integral(f,x)
\end{sagesilent}

\latexProblemContent{
\begin{problem}

Compute the following integral:

\input{Integral-Compute-0066.HELP.tex}

\[\int{\sage{f}\;dx} = \answer{\sage{Ans}+C}\]
\end{problem}}%}
%%%%%%%%%%%%%%%%%%%%%%

%%%%%%%%%%%%%%%%%%%%%%%
%%\tagged{Ans@ShortAns, Type@Compute, Topic@Integral, Sub@ByParts, Sub@Sub-u, File@0067}{
\begin{sagesilent}
# Define variables and constants/exponents
var('x')
a=NonZeroInt(-9,9)

# Define Rational Expression
vNumer=[sin(x)^3, arctan(x), sec(x)^3]
vDenom=[cos(x), x^2, tan(x)^2]
p=Integer(randint(0,2))
Numer=a*vNumer[p]
denom=vDenom[p]

f=Numer/denom

# Compute Answer
Ans=integral(f,x)
\end{sagesilent}

\latexProblemContent{
\begin{problem}

Compute the following integral:

\input{Integral-Compute-0067.HELP.tex}

\[\int{\sage{f}\;dx} = \answer{\sage{Ans}+C}\]
\end{problem}}%}
%%%%%%%%%%%%%%%%%%%%%%


%%%%%%%%%%%%%%%%%%%%%%%
%%\tagged{Ans@ShortAns, Type@Compute, Topic@Integral, Sub@Improper, File@0068}{
%%%  This question might not work because of the nested "problem" environment.
%%%  If it fails to nest properly, simply remove the outer question that has
%%%  them choose convergent/divergent. 

\begin{sagesilent}
# Define variables and constants/exponents
var('x')
a=NonZeroInt(-9,9)
b=NonZeroInt(-9,-1)
c=Integer(randint(-9,9))

# Define Function and parts for latex display
f=a*x*exp((b*x^2))
fone=a*x
fTwo=b*x^2

# Compute the indefinite integral and then evaluate at the bounds
INT=integral(f,x)
Ans=limit(INT,x=infinity)-limit(INT,x=c)
\end{sagesilent}

\latexProblemContent{
\begin{problem}
\begin{problem}
Determine if the integral converges or diverges.  If it converges, determine what it converges to.
\[\int_{\sage{c}}^\infty{\sage{fone}e^{\sage{fTwo}}\;dx}\]

\input{Integral-Compute-0068.HELP.tex}

\begin{multipleChoice}
\choice{Diverges}
\choice[correct]{Converges}
\end{multipleChoice}
\end{problem}

\[\int_{\sage{c}}^\infty{\sage{fone}e^{\sage{fTwo}}\;dx} = \answer{\sage{Ans}}\]

\end{problem}}%}
%%%%%%%%%%%%%%%%%%%%%%

%%%%%%%%%%%%%%%%%%%%%%%
%%\tagged{Ans@ShortAns, Type@Compute, Topic@Integral, Sub@Improper, Sub@Rational, File@0069}{
%%%  This question might not work because of the nested "problem" environment.
%%%  If it fails to nest properly, simply remove the outer question that has
%%%  them choose convergent/divergent. 

\begin{sagesilent}
# Define variables and constants/exponents
var('x')
a=NonZeroInt(-9,9)
b=NonZeroInt(1,9)
c=NonZeroInt(1,9)

# Define Function
f=a/x^b

# Compute the indefinite integral and then evaluate at the bounds
INT=integral(f,x)
Ans=limit(INT,x=infinity)-limit(INT,x=c)
\end{sagesilent}

\latexProblemContent{
\begin{problem}
\begin{problem}
Determine if the integral converges or diverges.  If it converges, determine what it converges to.
\[\int_{\sage{c}}^\infty{\sage{f}\;dx}\]

\input{Integral-Compute-0069.HELP.tex}

\begin{multipleChoice}
\choice{Diverges}
\choice[correct]{Converges}
\end{multipleChoice}
\end{problem}

\[\int_{\sage{c}}^\infty{\sage{f}\;dx} = \answer{\sage{Ans}}\]

\end{problem}}%}
%%%%%%%%%%%%%%%%%%%%%%

%%%%%%%%%%%%%%%%%%%%%%%
%%\tagged{Ans@ShortAns, Type@Compute, Topic@Integral, Sub@Improper, File@0070}{
%%%  This question might not work because of the nested "problem" environment.
%%%  If it fails to nest properly, simply remove the outer question that has
%%%  them choose convergent/divergent. 

\begin{sagesilent}
# Define variables and constants/exponents
var('x')
a=NonZeroInt(-9,9)
b=NonZeroInt(-9,9)

# Define Function
f=a/sqrt(b^2-x)

# Compute the indefinite integral and then evaluate at the bounds
INT=integral(f,x)
Ans=limit(INT,x=b^2)-limit(INT,x=0)
\end{sagesilent}

\latexProblemContent{
\begin{problem}
\begin{problem}
Determine if the integral converges or diverges.  If it converges, determine what it converges to.
\[\int_{0}^{\sage{b^2}}{\sage{f}\;dx}\]

\input{Integral-Compute-0070.HELP.tex}

\begin{multipleChoice}
\choice{Diverges}
\choice[correct]{Converges}
\end{multipleChoice}
\end{problem}

\[\int_{0}^{\sage{b^2}}{\sage{f}\;dx} = \answer{\sage{Ans}}\]

\end{problem}}%}
%%%%%%%%%%%%%%%%%%%%%%

%%%%%%%%%%%%%%%%%%%%%%%
%%\tagged{Ans@ShortAns, Type@Compute, Topic@Integral, Sub@Improper, File@0071}{
%%%  This question might not work because of the nested "problem" environment.
%%%  If it fails to nest properly, simply remove the outer question that has
%%%  them choose convergent/divergent. 

\begin{sagesilent}
# Define variables and constants/exponents
var('x')
a=NonZeroInt(-9,9)
b=NonZeroInt(-9,9)
c=NonZeroInt(-9,9)

# Define Function
f=a/sqrt(b^2-x)

# Compute the indefinite integral and then evaluate at the bounds
INT=integral(f,x)
Ans=limit(INT,x=b^2)-limit(INT,x=c)
\end{sagesilent}

\latexProblemContent{
\begin{problem}
\begin{problem}
Determine if the integral converges or diverges.  If it converges, determine what it converges to.
\[\int_{\sage{c}}^{\sage{b^2}}{\sage{f}\;dx}\]

\input{Integral-Compute-0071.HELP.tex}

\begin{multipleChoice}
\choice{Diverges}
\choice[correct]{Converges}
\end{multipleChoice}
\end{problem}

\[\int_{\sage{c}}^{\sage{b^2}}{\sage{f}\;dx} = \answer{\sage{Ans}}\]

\end{problem}}%}
%%%%%%%%%%%%%%%%%%%%%%

%%%%%%%%%%%%%%%%%%%%%%%
%%\tagged{Ans@ShortAns, Type@Compute, Topic@Integral, Sub@Improper, File@0072}{
%%%  This question might not work because of the nested "problem" environment.
%%%  If it fails to nest properly, simply remove the outer question that has
%%%  them choose convergent/divergent. 

\begin{sagesilent}
# Define variables and constants/exponents
var('x')
a=NonZeroInt(-9,9)
b=NonZeroInt(-9,9)
c=NonZeroInt(-9,9)

# Define Function
f=a*exp(x)/(exp(2*x)+b)

# Compute the indefinite integral and then evaluate at the bounds
INT=integral(f,x)
Ans=limit(INT,x=infinity)-limit(INT,x=c)
\end{sagesilent}

\latexProblemContent{
\begin{problem}
\begin{problem}
Determine if the integral converges or diverges.  If it converges, determine what it converges to.
\[\int_{\sage{c}}^{\infty}{\sage{f}\;dx}\]

\input{Integral-Compute-0072.HELP.tex}

\begin{multipleChoice}
\choice{Diverges}
\choice[correct]{Converges}
\end{multipleChoice}
\end{problem}

\[\int_{\sage{c}}^{\infty}{\sage{f}\;dx} = \answer{\sage{Ans}}\]

\end{problem}}%}
%%%%%%%%%%%%%%%%%%%%%%


%%%%%%%%%%%%%%%%%%%%%%%
%%\tagged{Ans@MC, Type@Compute, Topic@Integral, Sub@Improper, File@0073}{

\begin{sagesilent}
# Define variables and constants/exponents
var('x')
a=NonZeroInt(-9,9)
b=Integer(randint(-9,9))
c=NonZeroInt(1,9)

# Define Function
f=a/(x-b)
\end{sagesilent}

\latexProblemContent{
\begin{problem}
Determine if the integral converges or diverges.  If it converges, determine what it converges to.
\[\int_{\sage{c}}^{\infty}{\sage{f}\;dx}\]

\input{Integral-Compute-0073.HELP.tex}

\begin{multipleChoice}
\choice[correct]{Diverges}
\choice{Converges}
\end{multipleChoice}

\end{problem}}%}
%%%%%%%%%%%%%%%%%%%%%%

%%%%%%%%%%%%%%%%%%%%%%%
%%\tagged{Ans@MC, Type@Compute, Topic@Integral, Sub@Improper, File@0074}{

\begin{sagesilent}
# Define variables and constants/exponents
var('x')
a=NonZeroInt(-9,9)
b=Integer(randint(-9,9))

# Define Function
f=a*log(x-b)/(x-b)
\end{sagesilent}

\latexProblemContent{
\begin{problem}
Determine if the integral converges or diverges.  If it converges, determine what it converges to.
\[\int_{\sage{b+1}}^{\infty}{\sage{f}\;dx}\]

\input{Integral-Compute-0074.HELP.tex}

\begin{multipleChoice}
\choice[correct]{Diverges}
\choice{Converges}
\end{multipleChoice}

\end{problem}}%}
%%%%%%%%%%%%%%%%%%%%%%

%%%%%%%%%%%%%%%%%%%%%
%\tagged{Ans@ShortAns, Type@Compute, Topic@Integral, Sub@PartialFractions, File@0075}{
\begin{sagesilent}
# First grab all our coefficients
A = NonZeroInt(-10,10)
B = NonZeroInt(-10,10)
C = RandInt(-10,10)

# Form the functions that are being added together, then let sage add them together into a single fraction for us to make sure everything works nicely.
f1bot = x + RandInt(-5,5) # Linear factor
f2bot = x^2 + RandInt(1,10) # irreducable quadratic factor

# Make the rational functions
f1total = (A)/f1bot
f2total = (B*x+C)/f2bot

# Now we need to make the single rational function that needs to be decomposed. First we add them, then simplify (to get 1 fraction). This has a completely distributed function though, so we then need to apply .factor() to factor the bottom again.
FtotalT1 = f1total + f2total
FtotalT2 = FtotalT1.simplify_rational()
Ftotal = FtotalT2.factor()

# Now we choose which letters we want them to find. We use the cOrd1/cOrd2 to ensure that we write the letters in the question in alphabetical order.
letterChoice = ['A', 'B', 'C']
choice1 = RandInt(0,2)
choice2 = NonZeroInt(0,2,[choice1])
cOrd1 = min(choice1, choice2)
cOrd2 = max(choice1, choice2)
letC1 = letterChoice[cOrd1]
letC2 = letterChoice[cOrd2]

# Let sage get the correct answers based on which letters are being asked for.
ansvec = [A, B, C]
ans = ansvec[choice1] + ansvec[choice2]

\end{sagesilent}

\latexProblemContent{
\begin{problem}
\input{Integral-Compute-0075.HELP.tex}
After performing the partial fraction decomposition  
\[
\sage{Ftotal(x)}=\dfrac{A}{\sage{f1bot(x)}}+\dfrac{Bx+C}{\sage{f2bot(x)}}
\]

Then $\sage{letC1}+\sage{letC2} =\answer{\sage{ans}}$.\\

\end{problem}}%}
%%%%%%%%%%%%%%%%%%%%%



%%%%%%%%%%%%%%%%%%%%%
%\tagged{Ans@ShortAns, Type@Compute, Topic@Integral, Sub@TrigInt, File@0076}{
\begin{sagesilent}
# First get all the random rolls for the constants, powers, and whether or not it's going to have a negative exponent.

n = RandInt(0,2)
power = 2*n+3
flip = RandInt(0,1)
freq = NonZeroInt(-5,5)
Coef = NonZeroInt(-5,5)

# Set up the vectors to grab trig functions.

trigvec = [sin(x), cos(x), sec(x)*tan(x), tan(x), csc(x)*cot(x), cot(x)]

select = RandInt(0,5)

# The function that sage uses to calculate the answer
ToCalcInt = (Coef*trigvec[select](freq*x))^(power)

# Now we need to create the function the student sees to integrate (the actual integrand). If flip=1 we have a negative exponent, so we convert each trig function to it's co-function in the trigvec1 and trigvec2 to pull appropriate functions. Otherwise, we can just use the same function that sage is using to calculate. If we didn't use cofunctions, sage would helpfully negate the negatives and kill it.

if flip == 1:
    Intvec = [csc(x), sec(x), cos(x)*cot(x), cot(x), sin(x)*tan(x), tan(x)]
else:
    Intvec = [sin(x), cos(x), sec(x)*tan(x), tan(x), csc(x)*cot(x), cot(x)]

# Actually create the integrand with the above vectors
Integrand = (Coef*Intvec[select](freq*x))

# Calculate the actual power separately to prevent sage from trying to cancel negatives.
Integrandpower = (power*(-1)^flip)



CalcInt = integral(ToCalcInt(x),x)
ans = CalcInt#.simplify()

\end{sagesilent}

\latexProblemContent{
\begin{problem}
\input{Integral-Compute-0076.HELP.tex}
Evaluate the integral. \\
\[
\int \left(\sage{Integrand}\right)^{\sage{Integrandpower}} dx=\sage{ans}
\]

\end{problem}}%}
%%%%%%%%%%%%%%%%%%%%%



%%%%%%%%%%%%%%%%%%%%%
%\tagged{Ans@ShortAns, Type@Compute, Topic@Integral, Sub@TrigInt, Sub@EvenOdd, File@0077}{
\begin{sagesilent}
# First get all the random rolls for the constants, powers, and whether or not it's going to have a negative exponent.

n1 = RandInt(1,3)
power1 = 2*n1
n2 = RandInt(0,4-n1) # Curb the size of the power combination to avoid giant ans
power2 = 2*n2+1
flip = 1#RandInt(0,1)
freq = NonZeroInt(-5,5)
Coef = NonZeroInt(-5,5)

#Set up the vectors to grab matching trig functions, first one is even power, second is odd power.

trigvec1 = [sin(x), cos(x), sec(x), csc(x)]
trigvec2 = [cos(x), sin(x), tan(x), cot(x)]

select = RandInt(0,3)

#Grab base trig functions for writing sanity
F1 = trigvec1[select]
F2 = trigvec2[select]

#This is what sage uses to calculate the answer. For readability it helps to make sure it's integrating a positive power, so we assume the power is positive, and adjust the negative part in the Integrand part.

ToCalcInt = Coef*(F1(freq*x)^power1)*(F2(freq*x)^power2)


#Now we need to create the function the student sees to integrate (the actual integrand). If flip=1 we have a negative exponent, so we convert each trig function to it's co-function in the trigvec1 and trigvec2 to pull appropriate functions. Otherwise, we can just use the same function that sage is using to calculate. If we didn't use cofunctions, sage would helpfully negate the negatives and kill it.

if flip == 1:
    Intvec1 = [csc(x), sec(x), cos(x), sin(x)]
    Intvec2 = [sec(x), csc(x), cot(x), tan(x)]
    Integrand = (Coef) * (Intvec1[select](freq*x))^(-power1) * (Intvec2[select](freq*x))^(-power2)
else:
    Integrand = ToCalcInt


# Now have sage calculate the answer
CalcInt = integral(ToCalcInt(x),x)

# Write the answer for ximera. In this case, the 'simplify' just causes it to be worse.
ans = CalcInt

\end{sagesilent}

\latexProblemContent{
\begin{problem}
\input{Integral-Compute-0077.HELP.tex}
Evaluate the integral. \\

\[
\int \sage{Integrand} dx=\sage{ans}
\]

\end{problem}}%}
%%%%%%%%%%%%%%%%%%%%%

%%%%%%%%%%%%%%%%%%%%%
%\tagged{Ans@ShortAns, Type@Compute, Topic@Integral, Sub@TrigInt, File@0078}{
\begin{sagesilent}
A = NonZeroInt(-20,20)
B = NonZeroInt(-10,10)
n1 = RandInt(0,1)
n2 = RandInt(0,1-n1)

pwr1 = 2*n1+1
pwr2 = 2*n2

func = A * sec(B*x)^(pwr1)*tan(B*x)^(pwr2)

ans = integral(func(x),x)

\end{sagesilent}

\latexProblemContent{
\begin{problem}
Evaluate:
\input{Integral-Compute-0078.HELP.tex}
\[
\int \sage{func(x)} dx = \answer{\sage{ans}}
\]

\end{problem}}%}
%%%%%%%%%%%%%%%%%%%%%


%%%%%%%%%%%%%%%%%%%%%%%%%%%%%%%%%%%%%%%%%%%%%%%%%%%%%%%%%%%%%%

%%%%%%%%%%%%%%%%%%%%%
%\tagged{Ans@ShortAns, Type@Compute, Topic@Integral, Sub@ByParts, File@0079}{
\begin{sagesilent}
# Define coefficients, powers, etc
A = NonZeroInt(-5,5)
B = RandInt(-5,5)
C = NonZeroInt(-10,10)
Start = RandInt(-5,5)
End = RandInt(Start, 10)
pwr1 = RandInt(1,3)

# Outside function designed to be the differentiation term in by-parts
func1 = (x-B)^pwr1

# Inside function designed to be the integration term in by-parts

funcvec = [e^x, sin(x), cos(x)]
select = RandInt(0,2)

func2 = funcvec[select](x-A)

# Final integrand
F(x) = C*func1(x)*func2(x)

# Compute answer of integral from the Start point to the End point.
ans2 = integral(F(x),x,Start,End)

\end{sagesilent}

\latexProblemContent{
\begin{problem}
Evaluate:
\input{Integral-Compute-0079.HELP.tex}
$\int\limits_{\sage{Start}}^{\sage{End}} \sage{F(x)} dx = \answer{\sage{ans2}}$



\end{problem}}%}
%%%%%%%%%%%%%%%%%%%%%





%%%%%%%%%%%%%%%%%%%%%
%\tagged{Ans@ShortAns, Type@Compute, Topic@Integral, Sub@TrigSub, File@0080}{
\begin{sagesilent}
# Define coefficients, powers, etc
A = NonZeroInt(-5,5)
B = RandInt(-5,5)
C = NonZeroInt(-10,10)


# Vector of possible arctrig functions
funcvec = [arcsin(x), arccos(x), arctan(x), arcsec(x)]
select = RandInt(0,3)

# Select our arctrig function, then take the derivative to get the integrand.
Func1 = funcvec[select]
F(x) = C*Func1(A*x + B)

Integrand(x) = derivative(F(x),x)

\end{sagesilent}

\latexProblemContent{
\begin{problem}
Evaluate the integral
\input{Integral-Compute-0080.HELP.tex}
\[
\int \sage{Integrand(x)} dx = \answer{\sage{F(x)}}
\]


\end{problem}}%}
%%%%%%%%%%%%%%%%%%%%%



%%%%%%%%%%%%%%%%%%%%%%
%%\tagged{Ans@MC, Type@Compute, Topic@Integral, Sub@Improper, File@0081}{
%\begin{sagesilent}
%# Define coefficients, powers, etc
%A11 = NonZeroInt(1,5)
%B11 = NonZeroInt(1,5)
%A12 = NonZeroInt(1,5)
%B12 = NonZeroInt(1,5)
%A21 = NonZeroInt(1,5)
%B21 = NonZeroInt(1,5)
%A22 = NonZeroInt(1,5)
%B22 = NonZeroInt(1,5)
%A31 = NonZeroInt(1,5)
%B31 = NonZeroInt(1,5)
%A32 = NonZeroInt(1,5)
%B32 = NonZeroInt(1,5)
%A41 = NonZeroInt(1,5)
%B41 = NonZeroInt(1,5)
%A42 = NonZeroInt(1,5)
%B42 = NonZeroInt(1,5)
%
%pwr = RandInt(2,6)
%pwr2 = RandInt(pwr+2,10)
%
%slowvec = [ln(x), x^pwr, x^(1/(2*pwr+1))]
%fastvec = [e^(x), x^pwr2] 
%funcvec = [ln(x), x^(1/(2*pwr+1)), x^pwr, x^pwr2, e^x]
%
%selectslow = RandInt(0,2)
%selectfast = RandInt(0,1)
%select11 = RandInt(0,4)
%select21 = RandInt(0,4)
%select31 = RandInt(0,4)
%select41 = RandInt(0,4)
%select12 = RandInt(0,4)
%select22 = RandInt(0,4)
%select32 = RandInt(0,4)
%select42 = RandInt(0,4)
%
%Fslow = slowvec[selectslow]
%Ffast = fastvec[selectfast]
%
%F1 = Fslow/Ffast
%F2 = funcvec[select11](A11*x + B11)/funcvec[select12](A12*x + B12)
%F3 = funcvec[select21](A21*x + B21)/funcvec[select22](A22*x + B22)
%F4 = funcvec[select31](A31*x + B31)/funcvec[select32](A32*x + B32)
%F5 = funcvec[select41](A41*x + B41)/funcvec[select42](A42*x + B42)
%
%
%if select11 < select12:
%    ans2T = "correct"
%else:
%    ans2T = ""
%
%if select21 < select22:
%    ans3T = "correct"
%else:
%    ans3T = ""
%
%if select31 < select32:
%    ans4T = "correct"
%else:
%    ans4T = ""
%
%if select41 < select42:
%    ans5T = "correct"
%else:
%    ans5T = ""
%
%# This will look bad on the pdf, but I think once the merging programs run it will come out correct after the \sage command gets removed. Will probably need to do a replace-all to remove the "\texttt" part of the \texttt{correct} NEEDS TO BE TESTED.
%
%\end{sagesilent}
%
%\latexProblemContent{
%\begin{problem}
%Which of the following is a correct statement? (select \textit{all} that apply)
\input{Integral-Compute-0081.HELP.tex}
%\begin{multipleChoice}
%\choice[correct]{$\int\limits_{0}^\infty \sage{F1(x)} dx$}
%\choice[$\sage{ans2T}$]{$\int\limits_{0}^\infty \sage{F2(x)} dx$}
%\choice[$\sage{ans3T}$]{$\int\limits_{0}^\infty \sage{F3(x)} dx$}
%\choice[$\sage{ans4T}$]{$\int\limits_{0}^\infty \sage{F4(x)} dx$}
%\choice[$\sage{ans5T}$]{$\int\limits_{0}^\infty \sage{F5(x)} dx$}
%\end{multipleChoice}
%
%
%\end{problem}}%}
%%%%%%%%%%%%%%%%%%%%%%








%%%%%%%%%%%%%%%%%%%%%%%%%%%%%%%%%%%%%%%%%%%%%%%%%%%%%%%%%%%%%%%%%%%%%%%%%%%%%%%
%
%
%
%
%
%
%
%
%%%%%%%%%%%%%%%%%%%%%%%%%%%%%%%%%%%%%%%%%%%%%%%%%%%%%%%%%%%%%%%%%%%%%%%%%%%%%%%
%%%%%%%%%%%%%%%%%%%%%%%%%%%%%%%%%%%%%%%%%%%%%%%%%%%%%%%%%%%%%%%%%%%%%%%%%%%%%%%
%%%%%%%%%%%%%%%%%%%										%%%%%%%%%%%%%%%%%%%%%%%
%%%%%%%%%%%%%%%%%%%				Concept					%%%%%%%%%%%%%%%%%%%%%%%
%%%%%%%%%%%%%%%%%%%										%%%%%%%%%%%%%%%%%%%%%%%
%%%%%%%%%%%%%%%%%%%%%%%%%%%%%%%%%%%%%%%%%%%%%%%%%%%%%%%%%%%%%%%%%%%%%%%%%%%%%%%
%%%%%%%%%%%%%%%%%%%%%%%%%%%%%%%%%%%%%%%%%%%%%%%%%%%%%%%%%%%%%%%%%%%%%%%%%%%%%%%
%
%
%
%
%
%
%
%
%
%%%%%%%%%%%%%%%%%%%%%%%%%%%%%%%%%%%%%%%%%%%%%%%%%%%%%%%%%%%%%%%%%%%%%%%%%%%%%%%
%
%
%
%
%%%%%%%%%%%%%%%%%%%%%%%%%%%%%%%%%%%%%%%%%%%%%%%%%%%%%%%%%%%%%%%%%%%%%%%%%%%%%%%
%%%%%%%%%%%%%%%%%%%				Calc 1 Concept			%%%%%%%%%%%%%%%%%%%%%%%
%%%%%%%%%%%%%%%%%%%%%%%%%%%%%%%%%%%%%%%%%%%%%%%%%%%%%%%%%%%%%%%%%%%%%%%%%%%%%%%


%%%%%%%%%%%%%%%%%%%%%%%
%%\tagged{Ans@MC, Type@Concept, Topic@Integral, Sub@Definite, Sub@Theorems-FTC, Sub@Rational, File@0001}{
\begin{sagesilent}
a = NonZeroInt(-10,10)
l=NonZeroInt(-6,-1)  
u=NonZeroInt(1,6)

p=Integer(randint(0,3))
v=[a/x,a/x^2, a/x^3, a/x^4]
F=v[p]
f=integrate(F,x)
Ans=f(u)-f(l)
\end{sagesilent}

\latexProblemContent{
\begin{problem}

What is wrong with the following equation:

\[\int_{\sage{l}}^{\sage{u}} \sage{F}\;dx = \sage{f}\Bigg\vert_{\sage{l}}^{\sage{u}} = \sage{Ans}\]

\input{Integral-Concept-0001.HELP.tex}

\begin{multipleChoice}
\choice The bounds are evaluated in the wrong order.
\choice The antiderivative is incorrect.
\choice[correct] The integrand is not defined over the entire interval.
\choice Nothing is wrong.  The equation is correct, as is.
\end{multipleChoice}

\end{problem}}%}
%%%%%%%%%%%%%%%%%%%%%%


%%%%%%%%%%%%%%%%%%%%%%%
%%\tagged{Ans@MC, Type@Concept, Topic@Integral, Sub@Definite, Sub@Theorems-FTC, Sub@Trig, File@0002}{
\begin{sagesilent}
b = NonZeroInt(-10,10)
l = RandAng(0,pi/3)
u = RandAng(2*pi/3,pi)

p=Integer(randint(0,1))
vt=[b*sec(x)*tan(x), b*sec(x)^2]
F=vt[p]
f=integrate(F,x)
Ans=f(u)-f(l)
\end{sagesilent}

\latexProblemContent{
\begin{problem}

What is wrong with the following equation:

\[\int_{\sage{l}}^{\sage{u}} \sage{F}\;dx = \sage{f}\Bigg\vert_{\sage{l}}^{\sage{u}} = \sage{Ans}\]

\input{Integral-Concept-0002.HELP.tex}

\begin{multipleChoice}
\choice The antiderivative is incorrect.
\choice[correct] The integrand is not defined over the entire interval.
\choice The bounds are evaluated in the wrong order.
\choice Nothing is wrong.  The equation is correct, as is.
\end{multipleChoice}

\end{problem}}%}
%%%%%%%%%%%%%%%%%%%%%%



%%%%%%%%%%%%%%%%%%%%%%%
%%\tagged{Ans@MC, Type@Concept, Topic@Integral, Sub@Definite, Sub@Theorems-FTC, Sub@Trig, File@0003}{
\begin{sagesilent}
b = NonZeroInt(-10,10)
l = RandAng(pi/6,5*pi/6)
u = RandAng(7*pi/6,11*pi/6)

p=Integer(randint(0,1))
vt=[b*csc(x)^2, b*csc(x)*cot(x)]
F=vt[p]
f=integrate(F,x)
Ans=f(u)-f(l)
\end{sagesilent}

\latexProblemContent{
\begin{problem}

What is wrong with the following equation:

\[\int_{\sage{l}}^{\sage{u}} \sage{F}\;dx = \sage{f}\Bigg\vert_{\sage{l}}^{\sage{u}} = \sage{Ans}\]

\input{Integral-Concept-0003.HELP.tex}

\begin{multipleChoice}
\choice The antiderivative is incorrect.
\choice[correct] The integrand is not defined over the entire interval.
\choice The bounds are evaluated in the wrong order.
\choice Nothing is wrong.  The equation is correct, as is.
\end{multipleChoice}

\end{problem}}%}
%%%%%%%%%%%%%%%%%%%%%%
