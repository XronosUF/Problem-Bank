
%%%%%%%%%%%%%%%%%%%%%%%%%%
%%%%%%%%%%%%%%%%%%%%%%%%%%


%%%%%%%%%%%%%%%%%%%%%%%
%%\tagged{Ans@ShortAns, Type@Compute, Topic@Series, Sub@Geometric, Sub@Convergence, File@0001}{
\begin{sagesilent}
# Define variables and constants/exponents
var('x,n')
a=NonZeroInt(-9,9)
b=NonZeroInt(1,9)
c=NonZeroInt(1,9)
while c^2<=b^2:
   b=NonZeroInt(1,9)
   c=NonZeroInt(1,9)

#Define the general term of the series
r=b/c
f=a*(r)^x

#Compute the sum of the series
Ans=sum(f,x,0,infinity)

\end{sagesilent}

\latexProblemContent{
\ifVerboseLocation This is Series Compute Question 0001. \\ \fi
\begin{problem}
Determine if the series converges or diverges.  If it converges, find its sum. 

\[\sum_{n=0}^\infty \sage{f(n)}\]

\input{Series-Compute-0001.HELP.tex}

\begin{multipleChoice}
\choice{Diverges}
\choice[correct]{Converges}
\end{multipleChoice}

\begin{problem}
\[\sum_{n=0}^\infty \sage{f(n)} = \answer{\sage{Ans}}\]

\end{problem}

\end{problem}}%}
%%%%%%%%%%%%%%%%%%%%%%

%%%%%%%%%%%%%%%%%%%%%%%
%%\tagged{Ans@ShortAns, Type@Compute, Topic@Series, Sub@Geometric, Sub@Divergence, File@0002}{
\begin{sagesilent}
# Define variables and constants/exponents
var('x,n')
a=NonZeroInt(-9,9)
b=RandInt(2,9)
c=RandInt(1,b-1)

#Define the general term of the series
r=b/c
f=a*(r)^x


\end{sagesilent}

\latexProblemContent{
\ifVerboseLocation This is Series Compute Question 0002. \\ \fi
\begin{problem}
Determine if the series converges or diverges.  If it converges, find its sum. 

\[\sum_{n=0}^\infty \sage{f(n)}\]

\input{Series-Compute-0002.HELP.tex}

\begin{multipleChoice}
\choice[correct]{Diverges}
\choice{Converges}
\end{multipleChoice}

\end{problem}}%}
%%%%%%%%%%%%%%%%%%%%%%

%%%%%%%%%%%%%%%%%%%%%%%
%%\tagged{Ans@ShortAns, Type@Compute, Topic@Series, Sub@Geometric, Sub@Convergence, File@0003}{
\begin{sagesilent}
# Define variables and constants/exponents
var('x,n')
a=NonZeroInt(-9,9)
b=NonZeroInt(1,9)
c=NonZeroInt(1,9)
while c^2<=b^2:
   b=NonZeroInt(1,9)
   c=NonZeroInt(1,9)
d=RandInt(0,9)

#Define the general term of the series
r=b/c
f=a*(r)^x

#Compute the sum of the series
Ans=sum(f,x,d,infinity)

\end{sagesilent}

\latexProblemContent{
\ifVerboseLocation This is Series Compute Question 0003. \\ \fi
\begin{problem}
Determine if the series converges or diverges.  If it converges, find its sum. 

\[\sum_{n=\sage{d}}^\infty \sage{f(n)}\]

\input{Series-Compute-0003.HELP.tex}

\begin{multipleChoice}
\choice{Diverges}
\choice[correct]{Converges}
\end{multipleChoice}

\begin{problem}
\[\sum_{n=\sage{d}}^\infty \sage{f(n)} = \answer{\sage{Ans}}\]

\end{problem}

\end{problem}}%}
%%%%%%%%%%%%%%%%%%%%%%


%%%%%%%%%%%%%%%%%%%%%%%
%%\tagged{Ans@ShortAns, Type@Compute, Topic@Series, Sub@Geometric, Sub@Divergence, File@0004}{
\begin{sagesilent}
# Define variables and constants/exponents
var('x,n')
a=NonZeroInt(-9,9)
b=NonZeroInt(-9,9)
c=NonZeroInt(-9,9, [0, b])
while c^2>b^2:
   b=NonZeroInt(-9,9)
   c=NonZeroInt(-9,9,[0, b])
d=RandInt(0,9)

#Define the general term of the series
r=b/c

f(x)=a*(r)^x

\end{sagesilent}

\latexProblemContent{
\ifVerboseLocation This is Series Compute Question 0004. \\ \fi
\begin{problem}
Determine if the series converges or diverges.  If it converges, find its sum. 

\[\sum_{n=\sage{d}}^\infty \sage{f(n)}\]

\input{Series-Compute-0004.HELP.tex}

\begin{multipleChoice}
\choice[correct]{Diverges}
\choice{Converges}
\end{multipleChoice}


\end{problem}}%}
%%%%%%%%%%%%%%%%%%%%%%



%%%%%%%%%%%%%%%%%%%%%%%
%%\tagged{Ans@ShortAns, Type@Compute, Topic@Series, Sub@Geometric, Sub@Convergence, File@0005}{
\begin{sagesilent}
# Define variables and constants/exponents
var('x,n')
a=NonZeroInt(-9,9)
b=NonZeroInt(2,9)
d=RandInt(0,9)

#Define the general term of the series

f=a/b^x

#Compute the sum of the series
Ans=sum(f,x,d,infinity)

\end{sagesilent}

\latexProblemContent{
\ifVerboseLocation This is Series Compute Question 0005. \\ \fi
\begin{problem}
Determine if the series converges or diverges.  If it converges, find its sum. 

\[\sum_{n=\sage{d}}^\infty \sage{f(n)}\]

\input{Series-Compute-0005.HELP.tex}

\begin{multipleChoice}
\choice{Diverges}
\choice[correct]{Converges}
\end{multipleChoice}

\begin{problem}
\[\sum_{n=\sage{d}}^\infty \sage{f(n)} = \answer{\sage{Ans}}\]

\end{problem}

\end{problem}}%}
%%%%%%%%%%%%%%%%%%%%%%


%%%%%%%%%%%%%%%%%%%%%%%
%%\tagged{Ans@ShortAns, Type@Compute, Topic@Series, Sub@Telescopic, Sub@Convergence, File@0007}{
\begin{sagesilent}
# Define variables and constants/exponents
var('x,n')
a=NonZeroInt(-9,9)
b=RandInt(1,5)
c=RandInt(1,9)

#Define the general term of the series
f=(a*b)/(n^2+(2*c+b)*n+(c^2+c*b))
g=a/(x+c)

#Compute the sum of the series
Ans=sum(g,x,0,b-1)

\end{sagesilent}

\latexProblemContent{
\ifVerboseLocation This is Series Compute Question 0007. \\ \fi
\begin{problem}
Determine if the series converges or diverges.  If it converges, find its sum. 

\[\sum_{n=0}^\infty \sage{f}\]

\input{Series-Compute-0007.HELP.tex}

\begin{multipleChoice}
\choice{Diverges}
\choice[correct]{Converges}
\end{multipleChoice}

\begin{problem}
\[\sum_{n=0}^\infty \sage{f(n)} = \answer{\sage{Ans}}\]

\end{problem}

\end{problem}}%}
%%%%%%%%%%%%%%%%%%%%%%

%%%%%%%%%%%%%%%%%%%%%%%
%%\tagged{Ans@ShortAns, Type@Compute, Topic@Series, Sub@Telescopic, Sub@Divergence, File@0008}{
\begin{sagesilent}
# Define variables and constants/exponents
var('x,n')
a=NonZeroInt(-9,9)
b=NonZeroInt(1,5)
c=RandInt(0,9)

#Define the general term of the series
R=(x+c)/(x+(c+b))
g=log(R)
f=a*g

\end{sagesilent}

\latexProblemContent{
\ifVerboseLocation This is Series Compute Question 0008. \\ \fi
\begin{problem}
Determine if the series converges or diverges.  If it converges, find its sum. 

\[\sum_{n=1}^\infty \sage{f(n)}\]

\input{Series-Compute-0008.HELP.tex}

\begin{multipleChoice}
\choice[correct]{Diverges}
\choice{Converges}
\end{multipleChoice}

\end{problem}}%}
%%%%%%%%%%%%%%%%%%%%%%

%%%%%%%%%%%%%%%%%%%%%%%
%%\tagged{Ans@ShortAns, Type@Compute, Topic@Series, Sub@Telescopic, Sub@Convergence, File@0009}{
\begin{sagesilent}
# Define variables and constants/exponents
var('x,n')
a=NonZeroInt(-9,9)
c=RandInt(1,9)

#Define the general term of the series
g=1/(n^3-c^2*n)
f=a*c^2*g
F=1/n


#Compute the sum of the series
Ans=a/2*(sum(F,n,1,c)-sum(F,n,c+1,2*c))

\end{sagesilent}

\latexProblemContent{
\ifVerboseLocation This is Series Compute Question 0009. \\ \fi
\begin{problem}
Determine if the series converges or diverges.  If it converges, find its sum. 

\[\sum_{n=\sage{c+1}}^\infty \sage{f}\]

\input{Series-Compute-0009.HELP.tex}

\begin{multipleChoice}
\choice{Diverges}
\choice[correct]{Converges}
\end{multipleChoice}

\begin{problem}
\[\sum_{n=\sage{c+1}}^\infty \sage{f} = \answer{\sage{Ans}}\]

\end{problem}

\end{problem}}%}
%%%%%%%%%%%%%%%%%%%%%%

%%%%%%%%%%%%%%%%%%%%%%%
%%\tagged{Ans@ShortAns, Type@Compute, Topic@Series, Sub@Telescopic, Sub@Convergence, File@0010}{
\begin{sagesilent}
# Define variables and constants/exponents
var('x,n')
a=NonZeroInt(-9,9)
c=NonZeroInt(1,5)

#Define the general term of the series
f=a*log(1-c^2/x^2)

#Compute the sum of the series
Ans=sum(f,x,c+1,infinity)

\end{sagesilent}

\latexProblemContent{
\ifVerboseLocation This is Series Compute Question 0010. \\ \fi
\begin{problem}
Determine if the series converges or diverges.  If it converges, find its sum. 

\[\sum_{n=\sage{c+1}}^\infty \sage{f(n)}\]

\input{Series-Compute-0010.HELP.tex}

\begin{multipleChoice}
\choice{Diverges}
\choice[correct]{Converges}
\end{multipleChoice}

\begin{problem}
\[\sum_{n=\sage{c+1}}^\infty \sage{f(n)} = \answer{\sage{Ans}}\]

\end{problem}

\end{problem}}%}
%%%%%%%%%%%%%%%%%%%%%%

%%%%%%%%%%%%%%%%%%%%%%%
%%\tagged{Ans@ShortAns, Type@Compute, Topic@Series, Sub@IntegralTest, Sub@Divergence, File@0011}{
\begin{sagesilent}
# Define variables and constants/exponents
var('x,n')
a=NonZeroInt(-9,9)
d=NonZeroInt(2,20)

#Define the general term of the series
f=a*log(x)/x

\end{sagesilent}

\latexProblemContent{
\ifVerboseLocation This is Series Compute Question 0011. \\ \fi
\begin{problem}
Determine if the series converges or diverges.  

\[\sum_{n=\sage{d}}^\infty \sage{f(n)}\]

\input{Series-Compute-0011.HELP.tex}

\begin{multipleChoice}
\choice[correct]{Diverges}
\choice{Converges}
\end{multipleChoice}

\end{problem}}%}
%%%%%%%%%%%%%%%%%%%%%%

%%%%%%%%%%%%%%%%%%%%%%%
%%\tagged{Ans@ShortAns, Type@Compute, Topic@Series, Sub@IntegralTest, Sub@Convergence, File@0012}{
\begin{sagesilent}
# Define variables and constants/exponents
var('x,n')
a=NonZeroInt(-9,9)
b=NonZeroInt(1,9)
d=NonZeroInt(1,20)

#Define the general term of the series
f=a*x*exp(-b*x^2)

\end{sagesilent}

\latexProblemContent{
\ifVerboseLocation This is Series Compute Question 0012. \\ \fi
\begin{problem}
Determine if the series converges or diverges.  

\[\sum_{n=\sage{d}}^\infty \sage{f(n)}\]

\input{Series-Compute-0012.HELP.tex}

\begin{multipleChoice}
\choice{Diverges}
\choice[correct]{Converges}
\end{multipleChoice}

\end{problem}}%}
%%%%%%%%%%%%%%%%%%%%%%


%%%%%%%%%%%%%%%%%%%%%%%
%%\tagged{Ans@ShortAns, Type@Compute, Topic@Series, Sub@PTest, Sub@Convergence, File@0013}{
\begin{sagesilent}
# Define variables and constants/exponents
var('x,n')
a=NonZeroInt(-9,9)
b=RandInt(2,20)
d=NonZeroInt(1,20)

#Define the general term of the series
f=a*1/x^b

\end{sagesilent}

\latexProblemContent{
\ifVerboseLocation This is Series Compute Question 0013. \\ \fi
\begin{problem}
Determine if the series converges or diverges.  

\[\sum_{n=\sage{d}}^\infty \sage{f(n)}\]

\input{Series-Compute-0013.HELP.tex}

\begin{multipleChoice}
\choice{Diverges}
\choice[correct]{Converges}
\end{multipleChoice}

\end{problem}}%}
%%%%%%%%%%%%%%%%%%%%%%

%%%%%%%%%%%%%%%%%%%%%%%
%%\tagged{Ans@ShortAns, Type@Compute, Topic@Series, Sub@PTest, Sub@Divergence, File@0014}{
\begin{sagesilent}
# Define variables and constants/exponents
var('x,n')
a=NonZeroInt(-9,9)
b=NonZeroInt(-20,1)
d=NonZeroInt(1,20)

#Define the general term of the series
f=a*1/x^b

\end{sagesilent}

\latexProblemContent{
\ifVerboseLocation This is Series Compute Question 0014. \\ \fi
\begin{problem}
Determine if the series converges or diverges.  

\[\sum_{n=\sage{d}}^\infty \sage{f(n)}\]

\input{Series-Compute-0014.HELP.tex}

\begin{multipleChoice}
\choice[correct]{Diverges}
\choice{Converges}
\end{multipleChoice}

\end{problem}}%}
%%%%%%%%%%%%%%%%%%%%%%

%%%%%%%%%%%%%%%%%%%%%%%
%%\tagged{Ans@ShortAns, Type@Compute, Topic@Series, Sub@DirectComparison, Sub@Convergence, File@0015}{
\begin{sagesilent}
# Define variables and constants/exponents
var('x,n')
a=NonZeroInt(1,20)
b=RandInt(0,20)
c=NonZeroInt(2,9)
d=NonZeroInt(1,20)

#Define the general term of the series
f=a/(x^c+b)

\end{sagesilent}

\latexProblemContent{
\ifVerboseLocation This is Series Compute Question 0015. \\ \fi
\begin{problem}
Determine if the series converges or diverges.  

\[\sum_{n=\sage{d}}^\infty \sage{f(n)}\]

\input{Series-Compute-0015.HELP.tex}

\begin{multipleChoice}
\choice{Diverges}
\choice[correct]{Converges}
\end{multipleChoice}

\end{problem}}%}
%%%%%%%%%%%%%%%%%%%%%%

%%%%%%%%%%%%%%%%%%%%%%%
%%\tagged{Ans@ShortAns, Type@Compute, Topic@Series, Sub@DirectComparison, Sub@Divergence, File@0016}{
\begin{sagesilent}
# Define variables and constants/exponents
var('x,n')
a=NonZeroInt(1,20)
b=RandInt(0,20)
d=NonZeroInt(1,20)

#Define the general term of the series
f=a/(sqrt(x)-b)

\end{sagesilent}

\latexProblemContent{
\ifVerboseLocation This is Series Compute Question 0016. \\ \fi
\begin{problem}
Determine if the series converges or diverges.  

\[\sum_{n=\sage{d}}^\infty \sage{f(n)}\]

\input{Series-Compute-0016.HELP.tex}

\begin{multipleChoice}
\choice[correct]{Diverges}
\choice{Converges}
\end{multipleChoice}

\end{problem}}%}
%%%%%%%%%%%%%%%%%%%%%%

%%%%%%%%%%%%%%%%%%%%%%%
%%\tagged{Ans@ShortAns, Type@Compute, Topic@Series, Sub@LimitComparison, Sub@Divergence, File@0017}{
\begin{sagesilent}
# Define variables and constants/exponents
var('x,n')
a=NonZeroInt(1,20)
b=RandInt(0,20)
d=NonZeroInt(1,20)

#Define the general term of the series
f=a/(sqrt(x)+b)

\end{sagesilent}

\latexProblemContent{
\ifVerboseLocation This is Series Compute Question 0017. \\ \fi
\begin{problem}
Determine if the series converges or diverges.  

\[\sum_{n=\sage{d}}^\infty \sage{f(n)}\]

\input{Series-Compute-0017.HELP.tex}

\begin{multipleChoice}
\choice[correct]{Diverges}
\choice{Converges}
\end{multipleChoice}

\end{problem}}%}
%%%%%%%%%%%%%%%%%%%%%%

%%%%%%%%%%%%%%%%%%%%%%%
%%\tagged{Ans@ShortAns, Type@Compute, Topic@Series, Sub@LimitComparison, Sub@Divergence, File@0018}{
\begin{sagesilent}
# Define variables and constants/exponents
var('x,n')
a=NonZeroInt(1,20)
b=RandInt(1,20)
d=NonZeroInt(1,20)
p=RandInt(0,1)

#Define the general term of the series
v=[a*sin(b/x), a*tan(b/x)]
f=v[p]

\end{sagesilent}

\latexProblemContent{
\ifVerboseLocation This is Series Compute Question 0018. \\ \fi
\begin{problem}
Determine if the series converges or diverges.  

\[\sum_{n=\sage{d}}^\infty \sage{f(n)}\]

\input{Series-Compute-0018.HELP.tex}

\begin{multipleChoice}
\choice[correct]{Diverges}
\choice{Converges}
\end{multipleChoice}

\end{problem}}%}
%%%%%%%%%%%%%%%%%%%%%%

%%%%%%%%%%%%%%%%%%%%%%%
%%\tagged{Ans@ShortAns, Type@Compute, Topic@Series, Sub@DirectComparison, Sub@Convergence, File@0019}{
\begin{sagesilent}
# Define variables and constants/exponents
var('x,n')
a=NonZeroInt(1,20)
b=RandInt(1,20)
d=NonZeroInt(1,20)

#Define the general term of the series
f=(a+(-1)^x)/(b*x*sqrt(x))

\end{sagesilent}

\latexProblemContent{
\ifVerboseLocation This is Series Compute Question 0019. \\ \fi
\begin{problem}
Determine if the series converges or diverges.  

\[\sum_{n=\sage{d}}^\infty \sage{f(n)}\]

\input{Series-Compute-0019.HELP.tex}

\begin{multipleChoice}
\choice{Diverges}
\choice[correct]{Converges}
\end{multipleChoice}

\end{problem}}%}
%%%%%%%%%%%%%%%%%%%%%%

%%%%%%%%%%%%%%%%%%%%%%%
%%\tagged{Ans@ShortAns, Type@Compute, Topic@Series, Sub@LimitComparison, Sub@Convergence, File@0020}{
\begin{sagesilent}
# Define variables and constants/exponents
var('x,n')
a=NonZeroInt(1,10)
b=RandInt(0,10)
c=NonZeroInt(1,10)
d=NonZeroInt(1,20)
g=NonZeroInt(1,9)
p=RandInt(0,1)
s=NonZeroInt(1,9)
t=NonZeroInt(-9,9)
r=NonZeroInt(-9,9)

#Define the general term of the series
v=[a*x^2+b*x, a*x^3+b*x^2+c*x]
w=[s*x^2+t*x+r, s*x^3+t*x^2+r]
Top=v[p]
Bot=w[p]
f=Top/(g^x*(Bot))

\end{sagesilent}

\latexProblemContent{
\ifVerboseLocation This is Series Compute Question 0020. \\ \fi
\begin{problem}
Determine if the series converges or diverges.  

\[\sum_{n=\sage{d}}^\infty \sage{f(n)}\]

\input{Series-Compute-0020.HELP.tex}

\begin{multipleChoice}
\choice{Diverges}
\choice[correct]{Converges}
\end{multipleChoice}

\end{problem}}%}
%%%%%%%%%%%%%%%%%%%%%%

%%%%%%%%%%%%%%%%%%%%%%%
%%\tagged{Ans@ShortAns, Type@Compute, Topic@Series, Sub@DirectComparison, Sub@Convergence, File@0021}{
\begin{sagesilent}
# Define variables and constants/exponents
var('x,n')
a=RandInt(1,20)
b=RandInt(0,20)
t=RandInt(1,9)
s=2*t+1
d=RandInt(1,20)

#Define the general term of the series
f=a/sqrt(n^s+b^2)

\end{sagesilent}

\latexProblemContent{
\ifVerboseLocation This is Series Compute Question 0021. \\ \fi
\begin{problem}
Determine if the series converges or diverges.  

\[\sum_{n=\sage{d}}^\infty \sage{f}\]

\input{Series-Compute-0021.HELP.tex}

\begin{multipleChoice}
\choice{Diverges}
\choice[correct]{Converges}
\end{multipleChoice}

\end{problem}}%}
%%%%%%%%%%%%%%%%%%%%%%

%%%%%%%%%%%%%%%%%%%%%%%
%%\tagged{Ans@ShortAns, Type@Compute, Topic@Series, Sub@LimitComparison, Sub@Convergence, File@0022}{
\begin{sagesilent}
# Define variables and constants/exponents
var('x,n')
a=NonZeroInt(1,20)
b=NonZeroInt(1,9)
c=NonZeroInt(-9,9)
d=NonZeroInt(1,20)
e=NonZeroInt(-9,9)

#Define the general term of the series
f=a*sqrt(x)/(b*x^2+c*x+e)

\end{sagesilent}

\latexProblemContent{
\ifVerboseLocation This is Series Compute Question 0022. \\ \fi
\begin{problem}
Determine if the series converges or diverges.  

\[\sum_{n=\sage{d}}^\infty \sage{f(n)}\]

\input{Series-Compute-0022.HELP.tex}

\begin{multipleChoice}
\choice{Diverges}
\choice[correct]{Converges}
\end{multipleChoice}

\end{problem}}%}
%%%%%%%%%%%%%%%%%%%%%%

%%%%%%%%%%%%%%%%%%%%%%%
%%\tagged{Ans@ShortAns, Type@Compute, Topic@Series, Sub@DirectComparison, Sub@LimitComparison, Sub@Convergence, File@0023}{
\begin{sagesilent}
# Define variables and constants/exponents
var('x,n')
a=RandInt(2,20)
b=RandInt(1,a-1)
c=RandInt(1,20)
e=RandInt(1,a)
d=RandInt(1,20)

#Define the general term of the series
f=(e+b^n)/(c+a^n)

\end{sagesilent}

\latexProblemContent{
\ifVerboseLocation This is Series Compute Question 0023. \\ \fi
\begin{problem}
Determine if the series converges or diverges.  

\[\sum_{n=\sage{d}}^\infty \sage{f}\]

\input{Series-Compute-0023.HELP.tex}

\begin{multipleChoice}
\choice{Diverges}
\choice[correct]{Converges}
\end{multipleChoice}

\end{problem}}%}
%%%%%%%%%%%%%%%%%%%%%%

%%%%%%%%%%%%%%%%%%%%%%%
%%\tagged{Ans@ShortAns, Type@Compute, Topic@Series, Sub@DirectComparison, Sub@Convergence, File@0024}{
\begin{sagesilent}
# Define variables and constants/exponents
var('x,n')
a=NonZeroInt(1,20)
b=NonZeroInt(2,20)
d=NonZeroInt(1,20)

#Define the general term of the series
p=RandInt(0,1)
v=[a+sin(x), a+cos(x)]
f=v[p]/b^x

\end{sagesilent}

\latexProblemContent{
\ifVerboseLocation This is Series Compute Question 0024. \\ \fi
\begin{problem}
Determine if the series converges or diverges.  

\[\sum_{n=\sage{d}}^\infty \sage{f(n)}\]

\input{Series-Compute-0024.HELP.tex}

\begin{multipleChoice}
\choice{Diverges}
\choice[correct]{Converges}
\end{multipleChoice}

\end{problem}}%}
%%%%%%%%%%%%%%%%%%%%%%

%%%%%%%%%%%%%%%%%%%%%%%
%%\tagged{Ans@ShortAns, Type@Compute, Topic@Series, Sub@Alternating, Sub@Convergence, File@0025}{
\begin{sagesilent}
# Define variables and constants/exponents
var('x,n')
a=NonZeroInt(1,20)
s=NonZeroInt(1,9)
d=NonZeroInt(1,20)

#Define the general term of the series
f=a*(-1)^(x+1)/x^s

\end{sagesilent}

\latexProblemContent{
\ifVerboseLocation This is Series Compute Question 0025. \\ \fi
\begin{problem}
Determine if the series converges or diverges.  

\[\sum_{n=\sage{d}}^\infty \sage{f(n)}\]

\input{Series-Compute-0025.HELP.tex}

\begin{multipleChoice}
\choice{Diverges}
\choice[correct]{Converges}
\end{multipleChoice}

\end{problem}}%}
%%%%%%%%%%%%%%%%%%%%%%

%%%%%%%%%%%%%%%%%%%%%%%
%%\tagged{Ans@ShortAns, Type@Compute, Topic@Series, Sub@Alternating, Sub@Convergence, File@0026}{
\begin{sagesilent}
# Define variables and constants/exponents
var('x,n')
a=NonZeroInt(1,20)
b=NonZeroInt(2,20)
d=NonZeroInt(1,20)

#Define the general term of the series
f=a*(-1)^(x+1)*b^(1/x)/x

\end{sagesilent}

\latexProblemContent{
\ifVerboseLocation This is Series Compute Question 0026. \\ \fi
\begin{problem}
Determine if the series converges or diverges.  

\[\sum_{n=\sage{d}}^\infty \sage{f(n)}\]

\input{Series-Compute-0026.HELP.tex}

\begin{multipleChoice}
\choice{Diverges}
\choice[correct]{Converges}
\end{multipleChoice}

\end{problem}}%}
%%%%%%%%%%%%%%%%%%%%%%

%%%%%%%%%%%%%%%%%%%%%%%
%%\tagged{Ans@ShortAns, Type@Compute, Topic@Series, Sub@Alternating, Sub@Divergence, File@0027}{
\begin{sagesilent}
# Define variables and constants/exponents
var('x,n')
a=NonZeroInt(1,20)
b=NonZeroInt(1,20)
c=NonZeroInt(-9,9)
d=NonZeroInt(1,20)

#Define the general term of the series
f=(-1)^x*a*x/(b*x+c)

\end{sagesilent}

\latexProblemContent{
\ifVerboseLocation This is Series Compute Question 0027. \\ \fi
\begin{problem}
Determine if the series converges or diverges.  

\[\sum_{n=\sage{d}}^\infty \sage{f(n)}\]

\input{Series-Compute-0027.HELP.tex}

\begin{multipleChoice}
\choice[correct]{Diverges}
\choice{Converges}
\end{multipleChoice}

\end{problem}}%}
%%%%%%%%%%%%%%%%%%%%%%

%%%%%%%%%%%%%%%%%%%%%%%
%%\tagged{Ans@ShortAns, Type@Compute, Topic@Series, Sub@Alternating, Sub@Convergence, File@0028}{
\begin{sagesilent}
# Define variables and constants/exponents
var('x,n')
a=NonZeroInt(1,20)
p=RandInt(0,1)
q=RandInt(0,1)
d=NonZeroInt(1,20)

#Define the general term of the series
v=[a*sin(x*pi/2), a*cos(x*pi)]
w=[factorial(x), x^x]
f=v[p]/w[q]

\end{sagesilent}

\latexProblemContent{
\ifVerboseLocation This is Series Compute Question 0028. \\ \fi
\begin{problem}
Determine if the series converges or diverges.  

\[\sum_{n=\sage{d}}^\infty \sage{f(n)}\]

\input{Series-Compute-0028.HELP.tex}

\begin{multipleChoice}
\choice{Diverges}
\choice[correct]{Converges}
\end{multipleChoice}

\end{problem}}%}
%%%%%%%%%%%%%%%%%%%%%%

%%%%%%%%%%%%%%%%%%%%%%%
%%\tagged{Ans@ShortAns, Type@Compute, Topic@Series, Sub@Alternating, Sub@Convergence, File@0029}{
\begin{sagesilent}
# Define variables and constants/exponents
var('x,n')
a=NonZeroInt(1,20)
b=NonZeroInt(1,20)
s=NonZeroInt(1,9)
d=NonZeroInt(1,20)

#Define the general term of the series
f=a*(-1)^x*x^s/b^x

\end{sagesilent}

\latexProblemContent{
\ifVerboseLocation This is Series Compute Question 0029. \\ \fi
\begin{problem}
Determine if the series converges or diverges.  

\[\sum_{n=\sage{d}}^\infty \sage{f(n)}\]

\input{Series-Compute-0029.HELP.tex}

\begin{multipleChoice}
\choice{Diverges}
\choice[correct]{Converges}
\end{multipleChoice}

\end{problem}}%}
%%%%%%%%%%%%%%%%%%%%%%

%%%%%%%%%%%%%%%%%%%%%%%
%%\tagged{Ans@ShortAns, Type@Compute, Topic@Series, Sub@RatioTest, Sub@Convergence, File@0030}{
\begin{sagesilent}
# Define variables and constants/exponents
var('x,n')
a=NonZeroInt(2,20)
d=NonZeroInt(1,20)

#Define the general term of the series
f=a^x/factorial(x)

\end{sagesilent}

\latexProblemContent{
\ifVerboseLocation This is Series Compute Question 0030. \\ \fi
\begin{problem}
Determine if the series converges or diverges.  

\[\sum_{n=\sage{d}}^\infty \sage{f(n)}\]

\input{Series-Compute-0030.HELP.tex}

\begin{multipleChoice}
\choice{Diverges}
\choice[correct]{Converges}
\end{multipleChoice}

\end{problem}}%}
%%%%%%%%%%%%%%%%%%%%%%

%%%%%%%%%%%%%%%%%%%%%%%
%%\tagged{Ans@ShortAns, Type@Compute, Topic@Series, Sub@RatioTest, Sub@Divergence, File@0031}{
\begin{sagesilent}
# Define variables and constants/exponents
var('x,n')
a=NonZeroInt(2,20)
d=NonZeroInt(1,20)
s=NonZeroInt(1,9)

#Define the general term of the series
f=a^x/x^s

\end{sagesilent}

\latexProblemContent{
\ifVerboseLocation This is Series Compute Question 0031. \\ \fi
\begin{problem}
Determine if the series converges or diverges.  

\[\sum_{n=\sage{d}}^\infty \sage{f(n)}\]

\input{Series-Compute-0031.HELP.tex}

\begin{multipleChoice}
\choice[correct]{Diverges}
\choice{Converges}
\end{multipleChoice}

\end{problem}}%}
%%%%%%%%%%%%%%%%%%%%%%

%%%%%%%%%%%%%%%%%%%%%%%
%%\tagged{Ans@ShortAns, Type@Compute, Topic@Series, Sub@RatioTest, Sub@Convergence, File@0032}{
\begin{sagesilent}
# Define variables and constants/exponents
var('x,n')
a=NonZeroInt(2,20)
d=NonZeroInt(1,20)
s=NonZeroInt(1,9)

#Define the general term of the series
f=x^s/a^x

\end{sagesilent}

\latexProblemContent{
\ifVerboseLocation This is Series Compute Question 0032. \\ \fi
\begin{problem}
Determine if the series converges or diverges.  

\[\sum_{n=\sage{d}}^\infty \sage{f(n)}\]

\input{Series-Compute-0032.HELP.tex}

\begin{multipleChoice}
\choice{Diverges}
\choice[correct]{Converges}
\end{multipleChoice}

\end{problem}}%}
%%%%%%%%%%%%%%%%%%%%%%

%%%%%%%%%%%%%%%%%%%%%%%
%%\tagged{Ans@ShortAns, Type@Compute, Topic@Series, Sub@RatioTest, Sub@Convergence, File@0033}{
\begin{sagesilent}
# Define variables and constants/exponents
var('n')
a=NonZeroInt(2,20)
d=NonZeroInt(1,20)
s=NonZeroInt(1,9)

#Define the general term of the series
f=(-1)^(n+1)*n^s*a^n/factorial(n)

\end{sagesilent}

\latexProblemContent{
\ifVerboseLocation This is Series Compute Question 0033. \\ \fi
\begin{problem}
Determine if the series converges or diverges.  

\[\sum_{n=\sage{d}}^\infty \sage{f(n)}\]

\input{Series-Compute-0033.HELP.tex}

\begin{multipleChoice}
\choice{Diverges}
\choice[correct]{Converges}
\end{multipleChoice}

\end{problem}}%}
%%%%%%%%%%%%%%%%%%%%%%


%%%%%%%%%%%%%%%%%%%%%%%
%%\tagged{Ans@ShortAns, Type@Compute, Topic@Series, Sub@RatioTest, Sub@Convergence, File@0034}{
\begin{sagesilent}
# Define variables and constants/exponents
var('n')
a=RandInt(1,9)
b=RandInt(1,9)
c=RandInt(-2,2)
d=RandInt(0,2)
if d==0:
   c=RandInt(1,2)
e=RandInt(1,9)
nonNeg=b*d+c # To check that the prodcut doesn't have negative or zero terms
while nonNeg<=0:
   b=RandInt(1,9)
   nonNeg=b*d+c
LIM=b/a # To check that the series actually converges
while LIM>1:
   a=RandInt(b+1,10)
   LIM=b/a

\end{sagesilent}

\latexProblemContent{
\ifVerboseLocation This is Series Compute Question 0034. \\ \fi
\begin{problem}
Determine if the series converges or diverges.  
\[\sum_{n=\sage{d}}^\infty
\dfrac{(-1)^n(\sage{b*d+c}\cdot\sage{b*(d+1)+c}\cdot\sage{b*(d+2)+c}\cdots(\sage{b*n+c}))}
{\sage{a^n}\; (\sage{e*n})!}\]

\input{Series-Compute-0034.HELP.tex}

\begin{multipleChoice}
\choice{Diverges}
\choice[correct]{Converges}
\end{multipleChoice}

\end{problem}}%}
%%%%%%%%%%%%%%%%%%%%%%

%%%%%%%%%%%%%%%%%%%%%%%
%%\tagged{Ans@ShortAns, Type@Compute, Topic@Series, Sub@RootTest, Sub@Convergence, File@0035}{
\begin{sagesilent}
# Define variables and constants/exponents
var('x,n')
a=NonZeroInt(1,20)
b=NonZeroInt(a,21)
c=NonZeroInt(1,9)
d=NonZeroInt(1,20)

#Define the general term of the series
f=(a*x/(b*x+c))^x

\end{sagesilent}

\latexProblemContent{
\ifVerboseLocation This is Series Compute Question 0035. \\ \fi
\begin{problem}
Determine if the series converges or diverges.  

\[\sum_{n=\sage{d}}^\infty \sage{f(n)}\]

\input{Series-Compute-0035.HELP.tex}

\begin{multipleChoice}
\choice{Diverges}
\choice[correct]{Converges}
\end{multipleChoice}

\end{problem}}%}
%%%%%%%%%%%%%%%%%%%%%%

%%%%%%%%%%%%%%%%%%%%%%%
%%\tagged{Ans@ShortAns, Type@Compute, Topic@Series, Sub@RootTest, Sub@Convergence, File@0036}{
\begin{sagesilent}
# Define variables and constants/exponents
var('x,n')
a=NonZeroInt(-20,-1)
d=NonZeroInt(1,20)

#Define the general term of the series
f=(1+a/x)^(x^2)

\end{sagesilent}

\latexProblemContent{
\ifVerboseLocation This is Series Compute Question 0036. \\ \fi
\begin{problem}
Determine if the series converges or diverges.  

\[\sum_{n=\sage{d}}^\infty \sage{f(n)}\]

\input{Series-Compute-0036.HELP.tex}

\begin{multipleChoice}
\choice{Diverges}
\choice[correct]{Converges}
\end{multipleChoice}

\end{problem}}%}
%%%%%%%%%%%%%%%%%%%%%%

%%%%%%%%%%%%%%%%%%%%%%%
%%\tagged{Ans@ShortAns, Type@Compute, Topic@Series, Sub@RootTest, Sub@Divergence, File@0037}{
\begin{sagesilent}
# Define variables and constants/exponents
var('x,n')
a=NonZeroInt(1,20)
d=NonZeroInt(1,20)

#Define the general term of the series
f=(1+a/x)^(x^2)

\end{sagesilent}

\latexProblemContent{
\ifVerboseLocation This is Series Compute Question 0037. \\ \fi
\begin{problem}
Determine if the series converges or diverges.  

\[\sum_{n=\sage{d}}^\infty \sage{f(n)}\]

\input{Series-Compute-0037.HELP.tex}

\begin{multipleChoice}
\choice[correct]{Diverges}
\choice{Converges}
\end{multipleChoice}

\end{problem}}%}
%%%%%%%%%%%%%%%%%%%%%%

%%%%%%%%%%%%%%%%%%%%%%%
%%\tagged{Ans@ShortAns, Type@Compute, Topic@Series, Sub@RootTest, Sub@Convergence, File@0038}{
\begin{sagesilent}
# Define variables and constants/exponents
var('x,n')
a=NonZeroInt(1,20)
d=NonZeroInt(1,20)

#Define the general term of the series
f=a*x/(log(x))^x

\end{sagesilent}

\latexProblemContent{
\ifVerboseLocation This is Series Compute Question 0038. \\ \fi
\begin{problem}
Determine if the series converges or diverges.  

\[\sum_{n=\sage{d}}^\infty \sage{f(n)}\]

\input{Series-Compute-0038.HELP.tex}

\begin{multipleChoice}
\choice{Diverges}
\choice[correct]{Converges}
\end{multipleChoice}

\end{problem}}%}
%%%%%%%%%%%%%%%%%%%%%%

%%%%%%%%%%%%%%%%%%%%%%%
%%\tagged{Ans@ShortAns, Type@Compute, Topic@Series, Sub@RootTest, Sub@Convergence, File@0039}{
\begin{sagesilent}
# Define variables and constants/exponents
var('x,n')
a=NonZeroInt(1,20)
b=NonZeroInt(1,20)
d=NonZeroInt(1,20)

#Define the general term of the series
f=a^(b*x)/x^x

\end{sagesilent}

\latexProblemContent{
\ifVerboseLocation This is Series Compute Question 0039. \\ \fi
\begin{problem}
Determine if the series converges or diverges.  

\[\sum_{n=\sage{d}}^\infty \sage{f(n)}\]

\input{Series-Compute-0039.HELP.tex}

\begin{multipleChoice}
\choice{Diverges}
\choice[correct]{Converges}
\end{multipleChoice}

\end{problem}}%}
%%%%%%%%%%%%%%%%%%%%%%

%%%%%%%%%%%%%%%%%%%%%%%
%%\tagged{Ans@ShortAns, Type@Compute, Topic@Series, Sub@RootTest, Sub@Divergence, File@0040}{
\begin{sagesilent}
# Define variables and constants/exponents
var('x,n')
a=RandInt(2,20)
b=RandInt(1,a-1)
c=NonZeroInt(-20,20)
d=RandInt(1,20)
e=NonZeroInt(-20,20)

#Define the general term of the series
f=((a*n^2+c)/(b*n^2+e))^n

\end{sagesilent}

\latexProblemContent{
\ifVerboseLocation This is Series Compute Question 0040. \\ \fi
\begin{problem}
Determine if the series converges or diverges.  

\[\sum_{n=\sage{d}}^\infty \sage{f}\]

\input{Series-Compute-0040.HELP.tex}

\begin{multipleChoice}
\choice[correct]{Diverges}
\choice{Converges}
\end{multipleChoice}

\end{problem}}%}
%%%%%%%%%%%%%%%%%%%%%%

%%%%%%%%%%%%%%%%%%%%%%%
%%\tagged{Ans@ShortAns, Type@Compute, Topic@Series, Sub@RootTest, Sub@Convergence, File@0041}{
\begin{sagesilent}
# Define variables and constants/exponents
var('x,n')
a=NonZeroInt(1,19)
b=NonZeroInt(a+1,21)
c=NonZeroInt(-20,20)
d=NonZeroInt(1,20)
e=NonZeroInt(-20,20)

#Define the general term of the series
f=((a*n^2+c)/(b*n^2+e))^n

\end{sagesilent}

\latexProblemContent{
\ifVerboseLocation This is Series Compute Question 0041. \\ \fi
\begin{problem}
Determine if the series converges or diverges.  

\[\sum_{n=\sage{d}}^\infty \sage{f}\]

\input{Series-Compute-0041.HELP.tex}

\begin{multipleChoice}
\choice{Diverges}
\choice[correct]{Converges}
\end{multipleChoice}

\end{problem}}%}
%%%%%%%%%%%%%%%%%%%%%%

%%%%%%%%%%%%%%%%%%%%%%%
%%\tagged{Ans@ShortAns, Type@Compute, Topic@Series, Sub@RootTest, Sub@Divergence, File@0042}{
\begin{sagesilent}
# Define variables and constants/exponents
var('x,n')
a=RandInt(1,20)
b=RandInt(1,20)
d=RandInt(1,20)

#Define the general term of the series
f=(a*factorial(n))^n/n^(b*n)

\end{sagesilent}

\latexProblemContent{
\ifVerboseLocation This is Series Compute Question 0042. \\ \fi
\begin{problem}
Determine if the series converges or diverges.  

\[\sum_{n=\sage{d}}^\infty \sage{f}\]

\input{Series-Compute-0042.HELP.tex}

\begin{multipleChoice}
\choice[correct]{Diverges}
\choice{Converges}
\end{multipleChoice}

\end{problem}}%}
%%%%%%%%%%%%%%%%%%%%%%

%%%%%%%%%%%%%%%%% Finished with series tests %%%%%%%%%%%%%%%


%%%%%%%%%%%%%%%%%%%%%%%
%%\tagged{Ans@ShortAns, Type@Compute, Topic@Series, Sub@PowerSeries, Sub@RadiusOfConvergence, Sub@IntervalOfConvergence, File@0043}{
\begin{sagesilent}
# Define variables and constants/exponents
var('x,n')
a=NonZeroInt(1,20)
b=NonZeroInt(1,20)
d=NonZeroInt(1,20)

#Define the general term of the series
f=a*x^(b*n)/factorial(n)

\end{sagesilent}

\latexProblemContent{
\ifVerboseLocation This is Series Compute Question 0043. \\ \fi
\begin{problem}
Determine the radius of convergence and the interval of convergence. 

\[\sum_{n=\sage{d}}^\infty \sage{f}\]

\input{Series-Compute-0043.HELP.tex}

\[\mbox{Radius of Convergence}:\; \answer{\infty}\qquad \mbox{Interval of Convergence}:\; (\answer{-\infty},\answer{\infty})\]

\end{problem}}%}
%%%%%%%%%%%%%%%%%%%%%%

%%%%%%%%%%%%%%%%%%%%%%%
%%\tagged{Ans@ShortAns, Type@Compute, Topic@Series, Sub@PowerSeries, Sub@RadiusOfConvergence, Sub@IntervalOfConvergence, File@0044}{
\begin{sagesilent}
# Define variables and constants/exponents
var('x,n')
a=RandInt(-20,20)
d=RandInt(0,20)

#Define the general term of the series
f=(x-a)^n

\end{sagesilent}

\latexProblemContent{
\ifVerboseLocation This is Series Compute Question 0044. \\ \fi
\begin{problem}
Determine the radius of convergence and the interval of convergence. 

\[\sum_{n=\sage{d}}^\infty \sage{f}\]

\input{Series-Compute-0044.HELP.tex}

\[\mbox{Radius of Convergence}:\; \answer{1}\qquad \mbox{Interval of Convergence}:\; = \answer{\sage{a-1}}<x<\answer{\sage{a+1}}\]

\end{problem}}%}
%%%%%%%%%%%%%%%%%%%%%%

%%%%%%%%%%%%%%%%%%%%%%%
%%\tagged{Ans@ShortAns, Type@Compute, Topic@Series, Sub@PowerSeries, Sub@RadiusOfConvergence, Sub@IntervalOfConvergence, File@0045}{
\begin{sagesilent}
# Define variables and constants/exponents
var('x,n')
a=RandInt(-20,20)
b=NonZeroInt(1,20)
d=RandInt(0,20)

#Define the general term of the series
f=(x-a)^n/(n+b)

\end{sagesilent}

\latexProblemContent{
\ifVerboseLocation This is Series Compute Question 0045. \\ \fi
\begin{problem}
Determine the radius of convergence and the interval of convergence. 

\[\sum_{n=\sage{d}}^\infty \sage{f}\]

\input{Series-Compute-0045.HELP.tex}

\[\mbox{Radius of Convergence}:\; \answer{1}\qquad \mbox{Interval of Convergence}:\; = \answer{\sage{a-1}}<x<\answer{\sage{a+1}}\]

\end{problem}}%}
%%%%%%%%%%%%%%%%%%%%%%

%%%%%%%%%%%%%%%%%%%%%%%
%%\tagged{Ans@ShortAns, Type@Compute, Topic@Series, Sub@PowerSeries, Sub@RadiusOfConvergence, Sub@IntervalOfConvergence, File@0046}{
\begin{sagesilent}
# Define variables and constants/exponents
var('x,n')
a=RandInt(-20,20)
b=NonZeroInt(1,20)
c=NonZeroInt(1,9)
d=RandInt(0,20)

#Define the general term of the series
f=(-1)^n*(x-a)^n/((n+b)*c^n)

\end{sagesilent}

\latexProblemContent{
\ifVerboseLocation This is Series Compute Question 0046. \\ \fi
\begin{problem}
Determine the radius of convergence and the interval of convergence. 

\[\sum_{n=\sage{d}}^\infty \sage{f}\]

\input{Series-Compute-0046.HELP.tex}

\[\mbox{Radius of Convergence}:\; \answer{\sage{c}}\qquad \mbox{Interval of Convergence}:\; = \answer{\sage{a-c}}<x<\answer{\sage{a+c}}\]

\end{problem}}%}
%%%%%%%%%%%%%%%%%%%%%%

%%%%%%%%%%%%%%%%%%%%%%%
%%\tagged{Ans@MC, Type@Compute, Topic@Series, Sub@IntegralTest, Sub@Divergence, File@0047}{
\begin{sagesilent}
# Define variables and constants/exponents
var('x,n')
a=RandInt(1,20)
b=RandInt(1,20)
c=RandInt(1,20)
d=RandInt(1,20)
p=RandInt(1,10)
q=p+1

#Define the general term of the series
f=a*n^p/(b*n^q+c)


\end{sagesilent}

\latexProblemContent{
\ifVerboseLocation This is Series Compute Question 0047. \\ \fi
\begin{problem}
Determine if the series converges or diverges.

\[\sum_{n=\sage{d}}^\infty \sage{f}\]

\input{Series-Compute-0047.HELP.tex}

\begin{multipleChoice}
\choice[correct]{Diverges}
\choice{Converges}
\end{multipleChoice}

\end{problem}}%}
%%%%%%%%%%%%%%%%%%%%%%


%%%%%%%%%%%%%%%%%%%%%%%
%%\tagged{Ans@MC, Type@Compute, Topic@Series, Sub@IntegralTest, Sub@Convergence, File@0048}{
\begin{sagesilent}
# Define variables and constants/exponents
var('x,n')
a=RandInt(1,20)
d=RandInt(2,20)
p=RandInt(2,10)


#Define the general term of the series
f=a/(n*(log(n))^p)


\end{sagesilent}

\latexProblemContent{
\ifVerboseLocation This is Series Compute Question 0048. \\ \fi
\begin{problem}
Determine if the series converges or diverges.

\[\sum_{n=\sage{d}}^\infty \sage{f}\]

\input{Series-Compute-0048.HELP.tex}

\begin{multipleChoice}
\choice{Diverges}
\choice[correct]{Converges}
\end{multipleChoice}

\end{problem}}%}
%%%%%%%%%%%%%%%%%%%%%%


%%%%%%%%%%%%%%%%%%%%%%%
%%\tagged{Ans@MC, Type@Compute, Topic@Series, Sub@PTest, Sub@Convergence, File@0049}{
\begin{sagesilent}
# Define variables and constants/exponents
var('x,n')
a=RandInt(1,20)
d=RandInt(2,20)
p=RandInt(1,10)
q=RandInt(2,10)

\end{sagesilent}

\latexProblemContent{
\ifVerboseLocation This is Series Compute Question 0049. \\ \fi
\begin{problem}
Determine if the series converges or diverges.

\[\sum_{n=\sage{d}}^\infty \frac{\sage{a}}{n^{\sage{p}}\sqrt[\sage{q}]{n}}\]

\input{Series-Compute-0049.HELP.tex}

\begin{multipleChoice}
\choice{Diverges}
\choice[correct]{Converges}
\end{multipleChoice}

\end{problem}}%}
%%%%%%%%%%%%%%%%%%%%%%

%%%%%%%%%%%%%%%%%%%%%%%
%%\tagged{Ans@ShortAns, Type@Compute, Topic@Series, Sub@Alternating, Sub@ErrorEstimation, Sub@Convergence, File@0050}{
\begin{sagesilent}
# Define variables and constants/exponents
var('x,n')
a=RandInt(1,20)
b=RandInt(3,20)
d=RandInt(1,5)
p=RandInt(1,10)

#Define the general term of the series
f=(-1)^n*a/(n^p)
g=a/n^p

\end{sagesilent}

\latexProblemContent{
\ifVerboseLocation This is Series Compute Question 0050. \\ \fi
\begin{problem}
What is the error in estimating the sum of the series with the first $\sage{b}$ terms?

\[\sum_{n=\sage{d}}^\infty \sage{f}\]

\input{Series-Compute-0050.HELP.tex}

\[|S-S_{\sage{b}}|< \answer{\sage{g(b+d)}}\]

\end{problem}}%}
%%%%%%%%%%%%%%%%%%%%%%

%%%%%%%%%%%%%%%%%%%%%%%
%%\tagged{Ans@ShortAns, Type@Compute, Topic@Series, Sub@Alternating, Sub@ErrorEstimation, Sub@Convergence, File@0051}{
\begin{sagesilent}
# Define variables and constants/exponents
var('x,n')
a=RandInt(1,20)
b=RandInt(3,7)
d=RandInt(1,3)

#Define the general term of the series
f=(-1)^n*a/(n.factorial())
g=a/(n.factorial())

\end{sagesilent}

\latexProblemContent{
\ifVerboseLocation This is Series Compute Question 0051. \\ \fi
\begin{problem}
What is the error in estimating the sum of the series with the first $\sage{b}$ terms?

\[\sum_{n=\sage{d}}^\infty \sage{f}\]

\input{Series-Compute-0051.HELP.tex}

\[|S-S_{\sage{b}}|< \answer{\sage{g(b+d)}}\]

\end{problem}}%}
%%%%%%%%%%%%%%%%%%%%%%

%%%%%%%%%%%%%%%%%%%%%%%
%%\tagged{Ans@ShortAns, Type@Compute, Topic@Series, Sub@PowerSeries, Sub@Representation, Sub@Rational, File@0052}{
\begin{sagesilent}
# Define variables and constants/exponents
var('x,n')
a=NonZeroInt(-20,20)
b=NonZeroInt(-20,20)

#Define the general term of the series
f=b*x
g=1-f

#Series general term and radius
Ans=a*(f)^n
Rad=abs(1/b)

\end{sagesilent}

\latexProblemContent{
\ifVerboseLocation This is Series Compute Question 0052. \\ \fi
\begin{problem}
Find a power series representation for $\dfrac{\sage{a}}{\sage{g}}$.
	
\input{Series-Compute-0052.HELP.tex}

\[\sum_{n=\answer{0}}^\infty \answer{\sage{Ans}}\]

\begin{problem}
What is the radius of convergence? \[|x|<\answer{\sage{Rad}}\]

\end{problem}
\end{problem}}%}
%%%%%%%%%%%%%%%%%%%%%%

%%%%%%%%%%%%%%%%%%%%%%%
%%\tagged{Ans@ShortAns, Type@Compute, Topic@Series, Sub@PowerSeries, Sub@Representation, Sub@Rational, File@0053}{
\begin{sagesilent}
# Define variables and constants/exponents
var('x,n')
a=NonZeroInt(-20,20)
b=NonZeroInt(-20,20)
c=RandInt(1,20)
p=RandInt(1,4)

#Define the general term of the series
f=b*x^p
g=c-f

#Series general term and radius
Ans=a/c*(1/c*f)^n
Rad=abs(c/b)^(1/p)

\end{sagesilent}

\latexProblemContent{
\ifVerboseLocation This is Series Compute Question 0053. \\ \fi
\begin{problem}
Find a power series representation for $\dfrac{\sage{a}}{\sage{g}}$.

\input{Series-Compute-0053.HELP.tex}

\[\sum_{n=\answer{0}}^\infty \answer{\sage{Ans}}\]

\begin{problem}
What is the radius of convergence? \[|x|<\answer{\sage{Rad}}\]

\end{problem}
\end{problem}}%}
%%%%%%%%%%%%%%%%%%%%%%

%%%%%%%%%%%%%%%%%%%%%%%
%%\tagged{Ans@ShortAns, Type@Compute, Topic@Series, Sub@PowerSeries, Sub@Representation, File@0054}{
\begin{sagesilent}
# Define variables and constants/exponents
var('x,n')
a=NonZeroInt(-20,20)
b=NonZeroInt(-20,20)
c=RandInt(1,20)
p=RandInt(1,4)
q=RandInt(1,5)

#Define the general term of the series
f=b*x^p
g=c-f
#h=x^q
# This might need some work.........

#Series general term and radius
Ans=(a/c)*(b/c)^n*(x)^(n*p)
Rad=abs(c/b)^(1/p)

\end{sagesilent}

\latexProblemContent{
\ifVerboseLocation This is Series Compute Question 0054. \\ \fi
\begin{problem}
Find a power series representation for $\dfrac{\sage{a}}{\sage{g}}$.

\input{Series-Compute-0054.HELP.tex}

\[\sum_{n=\answer{0}}^\infty \answer{\sage{Ans}}\]

\begin{problem}
What is the radius of convergence? \[|x|<\answer{\sage{Rad}}\]

\end{problem}
\end{problem}}%}
%%%%%%%%%%%%%%%%%%%%%%


%%%%%%%%%%%%%%%%%%%%%%%
%%\tagged{Ans@ShortAns, Type@Compute, Topic@Series, Sub@PowerSeries, Sub@Representation, Sub@Log, File@0055}{
\begin{sagesilent}
# Define variables and constants/exponents
var('x,n')
a=NonZeroInt(1,10)
b=NonZeroInt(-10,10)
p=RandInt(1,10)


#Define the general original function and the general term with ROC
originf=a-b*x^p

#Series general term and radius
Ans=(b/a)^(n+1)*(1/(n+1))*x^(p*(n+1))
Rad=abs(a/b)^(1/p)

\end{sagesilent}

\latexProblemContent{
\ifVerboseLocation This is Series Compute Question 0055. \\ \fi
\begin{problem}
Find a power series representation for $\log(\sage{originf})$.

\input{Series-Compute-0055.HELP.tex}

\[\sum_{n=\answer{0}}^\infty \answer{\sage{Ans}}+\log(\sage{a})\]

\begin{problem}
What is the radius of convergence? \[|x|<\answer{\sage{Rad}}\]

\end{problem}
\end{problem}}%}
%%%%%%%%%%%%%%%%%%%%%%


%%%%%%%%%%%%%%%%%%%%%%%
%%\tagged{Ans@ShortAns, Type@Compute, Topic@Series, Sub@PowerSeries, Sub@Representation, Sub@Arctrig, File@0056}{
\begin{sagesilent}
# Define variables and constants/exponents
var('x,n')
a=NonZeroInt(-10,10)
b=NonZeroInt(1,10)
p=RandInt(1,10)


#Define the general original function and the general term with ROC
originf=b*x^p

#Series general term and radius
Ans=a*(-1)^n/(2*n+1)*b^(2*n+1)*x^(p*(2*n+1))
Rad=abs(1/b)^(1/(2*p))


\end{sagesilent}

\latexProblemContent{
\ifVerboseLocation This is Series Compute Question 0056. \\ \fi
\begin{problem}
Find a power series representation for $\sage{a}\arctan(\sage{originf})$.

\input{Series-Compute-0056.HELP.tex}

\[\sum_{n=\answer{0}}^\infty \answer{\sage{Ans}}\]

\begin{problem}
What is the radius of convergence? \[|x|<\answer{\sage{Rad}}\]

\end{problem}
\end{problem}}%}
%%%%%%%%%%%%%%%%%%%%%%


%%%%%%%%%%%%%%%%%%%%%%%
%%\tagged{Ans@ShortAns, Type@Compute, Topic@Series, Sub@PowerSeries, Sub@Representation, Sub@Rational, File@0057}{
\begin{sagesilent}
# Define variables and constants/exponents
var('x,n')
a=NonZeroInt(-10,10)
b=NonZeroInt(1,10)
p=RandInt(1,10)


#Define the general original function and the general term with ROC
originTop=a*x^p
originBot=1+b*x

#Series general term and radius
Ans=a*(-1)^n*b^n*x^(n+p)
Rad=abs(1/b)

\end{sagesilent}

\latexProblemContent{
\ifVerboseLocation This is Series Compute Question 0057. \\ \fi
\begin{problem}
Find a power series representation for $\dfrac{\sage{originTop}}{\sage{originBot}}$.

\input{Series-Compute-0057.HELP.tex}

\[\sum_{n=\answer{0}}^\infty \answer{\sage{Ans}}\]

\begin{problem}
What is the radius of convergence? \[|x|<\answer{\sage{Rad}}\]

\end{problem}
\end{problem}}%}
%%%%%%%%%%%%%%%%%%%%%%


%%%%%%%%%%%%%%%%%%%%%%%
%%\tagged{Ans@ShortAns, Type@Compute, Topic@Series, Sub@PowerSeries, Sub@Representation, Sub@Rational, File@0058}{
\begin{sagesilent}
# Define variables and constants/exponents
var('x,n')
a=NonZeroInt(-10,10)
b=NonZeroInt(1,10)
p=RandInt(1,10)


#Define the general original function and the general term with ROC
originTop=a*x^p
originBot=(1+b*x)^2

#Series general term and radius
Ans=a*n*(-1)^(n+1)*b^(n-1)*x^(n+p-1)
Rad=abs(1/b)

\end{sagesilent}

\latexProblemContent{
\ifVerboseLocation This is Series Compute Question 0058. \\ \fi
\begin{problem}
Find a power series representation for $\dfrac{\sage{originTop}}{\sage{originBot}}$.

\input{Series-Compute-0058.HELP.tex}

\[\sum_{n=\answer{1}}^\infty \answer{\sage{Ans}}\]

\begin{problem}
What is the radius of convergence? \[|x|<\answer{\sage{Rad}}\]

\end{problem}
\end{problem}}%}
%%%%%%%%%%%%%%%%%%%%%%


%%%%%%%%%%%%%%%%%%%%%%%
%%\tagged{Ans@ShortAns, Type@Compute, Topic@Series, Sub@PowerSeries, Sub@Representation, Sub@Rational, File@0059}{
\begin{sagesilent}
# Define variables and constants/exponents
var('x,n')
a=NonZeroInt(-10,10)
b=NonZeroInt(1,10)
p=RandInt(1,10)


#Define the general original function and the general term with ROC
originTop=a*x^p
originBot=(1+b*x)^3

#Series general term and radius
Ans=a/2*n*(n-1)*(-1)^n*b^(n-2)*x^(n+p-2)
Rad=abs(1/b)

\end{sagesilent}

\latexProblemContent{
\ifVerboseLocation This is Series Compute Question 0059. \\ \fi
\begin{problem}
Find a power series representation for $\dfrac{\sage{originTop}}{\sage{originBot}}$.

\input{Series-Compute-0059.HELP.tex}

\[\sum_{n=\answer{2}}^\infty \answer{\sage{Ans}}\]

\begin{problem}
What is the radius of convergence? \[|x|<\answer{\sage{Rad}}\]

\end{problem}
\end{problem}}%}
%%%%%%%%%%%%%%%%%%%%%%

%%%%%%%%%%%%%%%%%%%%%%%
%%\tagged{Ans@ShortAns, Type@Compute, Topic@Series, Sub@PowerSeries, Sub@Taylor, Sub@Exp, File@0060}{
\begin{sagesilent}
# Define variables and constants/exponents
var('x,n')
a=NonZeroInt(-10,10)
c=RandInt(-5,5)
d=RandInt(3,6)

#Define the general original function and the first few taylor terms
f=exp(a*x)
ans=taylor(f,x,c,d-1)

#Define general taylor term
fCoeff=a^n*exp(a*c)/n.factorial()
TAY=fCoeff*(x-c)^n

\end{sagesilent}

\latexProblemContent{
\ifVerboseLocation This is Series Compute Question 0060. \\ \fi
\begin{problem}
Determine the first $\sage{d}$ nonzero terms of the Taylor series for $f(x)=\sage{f}$ centered at $\sage{c}$.

\input{Series-Compute-0060.HELP.tex}

\[\answer{\sage{ans}}\]

\begin{problem}
What is full Taylor series?
\[\sum_{n=\answer{0}}^\infty{\answer{\sage{TAY}}}\]

\end{problem}
\end{problem}}%}
%%%%%%%%%%%%%%%%%%%%%%

%%%%%%%%%%%%%%%%%%%%%%%
%%\tagged{Ans@ShortAns, Type@Compute, Topic@Series, Sub@PowerSeries, Sub@Taylor, Sub@Rational, File@0061}{
\begin{sagesilent}
# Define variables and constants/exponents
var('x,n')
a=NonZeroInt(-10,10)
c=NonZeroInt(-5,5)
d=RandInt(3,6)
p=RandInt(1,10)

#Define the general original function and the first few taylor terms
f=a/x^p
ans=taylor(f,x,c,d-1)

#Define general taylor term
fCoeff=((-1)^n*a*(p+n-1).factorial()*c^(-p-n))/(n.factorial()*(p-1).factorial())
TAY=fCoeff*(x-c)^n

\end{sagesilent}

\latexProblemContent{
\ifVerboseLocation This is Series Compute Question 0061. \\ \fi
\begin{problem}
Determine the first $\sage{d}$ nonzero terms of the Taylor series for $f(x)=\sage{f}$ centered at $\sage{c}$.

\input{Series-Compute-0061.HELP.tex}

\[\answer{\sage{ans}}\]

\begin{problem}
What is full Taylor series?
\[\sum_{n=\answer{0}}^\infty{\answer{\sage{TAY}}}\]

\end{problem}
\end{problem}}%}
%%%%%%%%%%%%%%%%%%%%%%


%%%%%%%%%%%%%%%%%%%%%%%
%%\tagged{Ans@ShortAns, Type@Compute, Topic@Series, Sub@PowerSeries, Sub@Taylor, Sub@Radical, File@0062}{
\begin{sagesilent}
# Define variables and constants/exponents
var('x,n')
a=NonZeroInt(-10,10)
c=NonZeroInt(1,7)
d=NonZeroInt(3,6)
p=RandInt(1,10)

#Define the general original function and the first few taylor terms
f=a/x^(1/p)
ans=taylor(f,x,c,d-1)
\end{sagesilent}

\latexProblemContent{
\ifVerboseLocation This is Series Compute Question 0062. \\ \fi
\begin{problem}
Determine the first $\sage{d}$ nonzero terms of the Taylor series for $f(x)=\sage{f}$ centered at $\sage{c}$.

\input{Series-Compute-0062.HELP.tex}

\[\answer{\sage{ans}}\]

\end{problem}}%}
%%%%%%%%%%%%%%%%%%%%%%

%%%%%%%%%%%%%%%%%%%%%%%
%%\tagged{Ans@ShortAns, Type@Compute, Topic@Series, Sub@PowerSeries, Sub@Taylor, Sub@Radical, File@0063}{
\begin{sagesilent}
# Define variables and constants/exponents
var('x,n')
a=NonZeroInt(-10,10)
c=NonZeroInt(1,7)
d=NonZeroInt(3,6)
p=RandInt(1,10)

#Define the general original function and the first few taylor terms
f=a*x^(1/p)
ans=taylor(f,x,c,d-1)
\end{sagesilent}

\latexProblemContent{
\ifVerboseLocation This is Series Compute Question 0063. \\ \fi
\begin{problem}
Determine the first $\sage{d}$ nonzero terms of the Taylor series for $f(x)=\sage{f}$ centered at $\sage{c}$.

\input{Series-Compute-0063.HELP.tex}

\[\answer{\sage{ans}}\]

\end{problem}}%}
%%%%%%%%%%%%%%%%%%%%%%


%%%%%%%%%%%%%%%%%%%%%%%
%%\tagged{Ans@ShortAns, Type@Compute, Topic@Series, Sub@PowerSeries, Sub@Taylor, Sub@Exp, File@0064}{
\begin{sagesilent}
# Define variables and constants/exponents
var('x,n')
b=RandInt(1,10)
d=RandInt(1,3)
p=RandInt(1,10)

#Define the general original function
f=(-1)^n*b^n*x^(p*n)/n.factorial()

#Define the sum of the series
ans1=exp(-b*x^p)
ans2=sum(f,n,0,d-1)
ANS=ans1-ans2
\end{sagesilent}

\latexProblemContent{
\ifVerboseLocation This is Series Compute Question 0064. \\ \fi
\begin{problem}
Determine the sum of the series \[\sum_{n=\sage{d}}^\infty \sage{f}\]

\input{Series-Compute-0064.HELP.tex}

\[\answer{\sage{ANS}}\]

\end{problem}}%}
%%%%%%%%%%%%%%%%%%%%%%

%%%%%%%%%%%%%%%%%%%%%%%
%%\tagged{Ans@ShortAns, Type@Compute, Topic@Series, Sub@PowerSeries, Sub@Taylor, Sub@Trig, File@0065}{
\begin{sagesilent}
# Define variables and constants/exponents
var('x,n')
b=RandInt(1,10)
d=RandInt(1,3)
p=RandInt(1,10)

#Define the general original function
f=(-1)^n*b^(2*n+1)*x^(p*(2*n+1))/(2*n+1).factorial()

#Define the sum of the series
ans1=sin(b*x^p)
ans2=sum(f,n,0,d-1)
ANS=ans1-ans2
\end{sagesilent}

\latexProblemContent{
\ifVerboseLocation This is Series Compute Question 0065. \\ \fi
\begin{problem}
Determine the sum of the series \[\sum_{n=\sage{d}}^\infty \sage{f}\]

\input{Series-Compute-0065.HELP.tex}

\[\answer{\sage{ANS}}\]

\end{problem}}%}
%%%%%%%%%%%%%%%%%%%%%%

%%%%%%%%%%%%%%%%%%%%%%%
%%\tagged{Ans@ShortAns, Type@Compute, Topic@Series, Sub@PowerSeries, Sub@Taylor, Sub@Trig, File@0066}{
\begin{sagesilent}
# Define variables and constants/exponents
var('x,n')
b=RandInt(1,10)
d=RandInt(1,3)
p=RandInt(1,10)

#Define the general original function
f=(-1)^n*b^(2*n)*x^(p*(2*n))/(2*n).factorial()

#Define the sum of the series
ans1=cos(b*x^p)
ans2=sum(f,n,0,d-1)
ANS=ans1-ans2
\end{sagesilent}

\latexProblemContent{
\ifVerboseLocation This is Series Compute Question 0066. \\ \fi
\begin{problem}
Determine the sum of the series \[\sum_{n=\sage{d}}^\infty \sage{f}\]

\input{Series-Compute-0066.HELP.tex}

\[\answer{\sage{ANS}}\]

\end{problem}}%}
%%%%%%%%%%%%%%%%%%%%%%

%%%%%%%%%%%%%%%%%%%%%%%
%%\tagged{Ans@ShortAns, Type@Compute, Topic@Series, Sub@PowerSeries, Sub@Taylor, Sub@Arctrig, File@0067}{
\begin{sagesilent}
# Define variables and constants/exponents
var('x,n')
b=RandInt(1,10)
d=RandInt(1,3)
p=RandInt(1,10)

#Define the general original function
f=(-1)^n*b^(2*n+1)*x^(p*(2*n+1))/(2*n+1)

#Define the sum of the series
ans1=arctan(b*x^p)
ans2=sum(f,n,0,d-1)
ANS=ans1-ans2
\end{sagesilent}

\latexProblemContent{
\ifVerboseLocation This is Series Compute Question 0067. \\ \fi
\begin{problem}
Determine the sum of the series \[\sum_{n=\sage{d}}^\infty \sage{f}\]

\input{Series-Compute-0067.HELP.tex}

\[\answer{\sage{ANS}}\]

\end{problem}}%}
%%%%%%%%%%%%%%%%%%%%%%

%%%%%%%%%%%%%%%%%%%%%%%
%%\tagged{Ans@ShortAns, Type@Compute, Topic@Series, Sub@PowerSeries, Sub@Taylor, Sub@Log, File@0068}{
\begin{sagesilent}
# Define variables and constants/exponents
var('x,n')
b=RandInt(1,10)
d=RandInt(2,4)
p=RandInt(1,10)

#Define the general original function
f=b^(n)*x^(p*(n))/(n)

#Define the sum of the series
ans1=log(1-b*x^p)
ans2=sum(f,n,1,d-1)
ANS=ans1+ans2
\end{sagesilent}

\latexProblemContent{
\ifVerboseLocation This is Series Compute Question 0068. \\ \fi
\begin{problem}
Determine the sum of the series \[-\sum_{n=\sage{d}}^\infty \sage{f}\]

\input{Series-Compute-0068.HELP.tex}

\[\answer{\sage{ANS}}\]

\end{problem}}%}
%%%%%%%%%%%%%%%%%%%%%%

%%%%%%%%%%%%%%%%%%%%%%%
%%\tagged{Ans@ShortAns, Type@Compute, Topic@Series, Sub@PowerSeries, Sub@Taylor, Sub@Log, File@0069}{
\begin{sagesilent}
# Define variables and constants/exponents
var('x,n')
b=RandInt(1,10)
d=RandInt(2,4)
p=RandInt(1,10)

#Define the general original function
f=(-1)^(n+1)*b^(n)*x^(p*(n))/(n)

#Define the sum of the series
ans1=log(1+b*x^p)
ans2=sum(f,n,1,d-1)
ANS=ans1-ans2
\end{sagesilent}

\latexProblemContent{
\ifVerboseLocation This is Series Compute Question 0069. \\ \fi
\begin{problem}
Determine the sum of the series \[\sum_{n=\sage{d}}^\infty \sage{f}\]

\input{Series-Compute-0069.HELP.tex}

\[\answer{\sage{ANS}}\]

\end{problem}}%}
%%%%%%%%%%%%%%%%%%%%%%









