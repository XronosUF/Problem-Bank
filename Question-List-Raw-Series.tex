\documentclass{ximera}
\usepackage{PackageLoader}
\usepackage{sagetex}
%%%%%%  This is the raw list of questions before processing %%%%%%
\renewcommand{\latexProblemContent}[1]{#1}
\renewcommand{\choice}[2][No]{\item Is this correct? #1 \\ What is the choice? #2}

%%%%%%%%%%%%%%%%%%%%%%%%%%
%%%%%%%%%%%%%%%%%%%%%%%%%%

\begin{document}
\input{JasonSageMacrosNEW}

%%%% ADD IN Sub@PTest - Done %%%%


%%%%%%%%%%%%%%%%%%%%%%%%%%
%%%%%%%%%%%%%%%%%%%%%%%%%%


%%%%%%%%%%%%%%%%%%%%%%%
%%\tagged{Ans@ShortAns, Type@Compute, Topic@Series, Sub@Geometric, Sub@Convergence, File@0001}{
\begin{sagesilent}
# Define variables and constants/exponents
var('x,n')
a=NonZeroInt(-9,9)
b=NonZeroInt(1,9)
c=NonZeroInt(1,9)
while c^2<=b^2:
   b=NonZeroInt(1,9)
   c=NonZeroInt(1,9)

#Define the general term of the series
r=b/c
f=a*(r)^x

#Compute the sum of the series
Ans=sum(f,x,0,infinity)

\end{sagesilent}

\latexProblemContent{
\ifVerboseLocation This is Series Compute Question 0001. \\ \fi
\begin{problem}
Determine if the series converges or diverges.  If it converges, find its sum. 

\[\sum_{n=0}^\infty \sage{f(n)}\]

\input{Series-Compute-0001.HELP.tex}

\begin{multipleChoice}
\choice{Diverges}
\choice[correct]{Converges}
\end{multipleChoice}

\begin{problem}
\[\sum_{n=0}^\infty \sage{f(n)} = \answer{\sage{Ans}}\]

\end{problem}

\end{problem}}%}
%%%%%%%%%%%%%%%%%%%%%%

%%%%%%%%%%%%%%%%%%%%%%%
%%\tagged{Ans@ShortAns, Type@Compute, Topic@Series, Sub@Geometric, Sub@Divergence, File@0002}{
\begin{sagesilent}
# Define variables and constants/exponents
var('x,n')
a=NonZeroInt(-9,9)
b=RandInt(2,9)
c=RandInt(1,b-1)

#Define the general term of the series
r=b/c
f=a*(r)^x


\end{sagesilent}

\latexProblemContent{
\ifVerboseLocation This is Series Compute Question 0002. \\ \fi
\begin{problem}
Determine if the series converges or diverges.  If it converges, find its sum. 

\[\sum_{n=0}^\infty \sage{f(n)}\]

\input{Series-Compute-0002.HELP.tex}

\begin{multipleChoice}
\choice[correct]{Diverges}
\choice{Converges}
\end{multipleChoice}

\end{problem}}%}
%%%%%%%%%%%%%%%%%%%%%%

%%%%%%%%%%%%%%%%%%%%%%%
%%\tagged{Ans@ShortAns, Type@Compute, Topic@Series, Sub@Geometric, Sub@Convergence, File@0003}{
\begin{sagesilent}
# Define variables and constants/exponents
var('x,n')
a=NonZeroInt(-9,9)
b=NonZeroInt(1,9)
c=NonZeroInt(1,9)
while c^2<=b^2:
   b=NonZeroInt(1,9)
   c=NonZeroInt(1,9)
d=RandInt(0,9)

#Define the general term of the series
r=b/c
f=a*(r)^x

#Compute the sum of the series
Ans=sum(f,x,d,infinity)

\end{sagesilent}

\latexProblemContent{
\ifVerboseLocation This is Series Compute Question 0003. \\ \fi
\begin{problem}
Determine if the series converges or diverges.  If it converges, find its sum. 

\[\sum_{n=\sage{d}}^\infty \sage{f(n)}\]

\input{Series-Compute-0003.HELP.tex}

\begin{multipleChoice}
\choice{Diverges}
\choice[correct]{Converges}
\end{multipleChoice}

\begin{problem}
\[\sum_{n=\sage{d}}^\infty \sage{f(n)} = \answer{\sage{Ans}}\]

\end{problem}

\end{problem}}%}
%%%%%%%%%%%%%%%%%%%%%%

%%%%%%%%%%%%%%%%%%%%%%%
%%\tagged{Ans@ShortAns, Type@Compute, Topic@Series, Sub@Geometric, Sub@Divergence, File@0004}{
\begin{sagesilent}
# Define variables and constants/exponents
var('x,n')
a=NonZeroInt(-9,9)
b=NonZeroInt(-9,9)
c=NonZeroInt(-9,9)
while c^2>b^2:
   b=NonZeroInt(-9,9)
   c=NonZeroInt(-9,9)
d=RandInt(0,9)

#Define the general term of the series
r=b/c
f=a*(r)^x


\end{sagesilent}

\latexProblemContent{
\ifVerboseLocation This is Series Compute Question 0004. \\ \fi
\begin{problem}
Determine if the series converges or diverges.  If it converges, find its sum. 

\[\sum_{n=\sage{d}}^\infty \sage{f(n)}\]

\input{Series-Compute-0004.HELP.tex}

\begin{multipleChoice}
\choice[correct]{Diverges}
\choice{Converges}
\end{multipleChoice}

\end{problem}}%}
%%%%%%%%%%%%%%%%%%%%%%

%%%%%%%%%%%%%%%%%%%%%%%
%%\tagged{Ans@ShortAns, Type@Compute, Topic@Series, Sub@Geometric, Sub@Convergence, File@0005}{
\begin{sagesilent}
# Define variables and constants/exponents
var('x,n')
a=NonZeroInt(-9,9)
b=NonZeroInt(2,9)
d=RandInt(0,9)

#Define the general term of the series

f=a/b^x

#Compute the sum of the series
Ans=sum(f,x,d,infinity)

\end{sagesilent}

\latexProblemContent{
\ifVerboseLocation This is Series Compute Question 0005. \\ \fi
\begin{problem}
Determine if the series converges or diverges.  If it converges, find its sum. 

\[\sum_{n=\sage{d}}^\infty \sage{f(n)}\]

\input{Series-Compute-0005.HELP.tex}

\begin{multipleChoice}
\choice{Diverges}
\choice[correct]{Converges}
\end{multipleChoice}

\begin{problem}
\[\sum_{n=\sage{d}}^\infty \sage{f(n)} = \answer{\sage{Ans}}\]

\end{problem}

\end{problem}}%}
%%%%%%%%%%%%%%%%%%%%%%


%%%%%%%%%%%%%%%%%%%%%%%
%%\tagged{Ans@ShortAns, Type@Compute, Topic@Series, Sub@Telescopic, Sub@Convergence, File@0007}{
\begin{sagesilent}
# Define variables and constants/exponents
var('x,n')
a=NonZeroInt(-9,9)
b=RandInt(1,5)
c=RandInt(1,9)

#Define the general term of the series
f=(a*b)/(n^2+(2*c+b)*n+(c^2+c*b))
g=a/(x+c)

#Compute the sum of the series
Ans=sum(g,x,0,b-1)

\end{sagesilent}

\latexProblemContent{
\ifVerboseLocation This is Series Compute Question 0007. \\ \fi
\begin{problem}
Determine if the series converges or diverges.  If it converges, find its sum. 

\[\sum_{n=0}^\infty \sage{f}\]

\input{Series-Compute-0007.HELP.tex}

\begin{multipleChoice}
\choice{Diverges}
\choice[correct]{Converges}
\end{multipleChoice}

\begin{problem}
\[\sum_{n=0}^\infty \sage{f(n)} = \answer{\sage{Ans}}\]

\end{problem}

\end{problem}}%}
%%%%%%%%%%%%%%%%%%%%%%

%%%%%%%%%%%%%%%%%%%%%%%
%%\tagged{Ans@ShortAns, Type@Compute, Topic@Series, Sub@Telescopic, Sub@Divergence, File@0008}{
\begin{sagesilent}
# Define variables and constants/exponents
var('x,n')
a=NonZeroInt(-9,9)
b=NonZeroInt(1,5)
c=RandInt(0,9)

#Define the general term of the series
R=(x+c)/(x+(c+b))
g=log(R)
f=a*g

\end{sagesilent}

\latexProblemContent{
\ifVerboseLocation This is Series Compute Question 0008. \\ \fi
\begin{problem}
Determine if the series converges or diverges.  If it converges, find its sum. 

\[\sum_{n=1}^\infty \sage{f(n)}\]

\input{Series-Compute-0008.HELP.tex}

\begin{multipleChoice}
\choice[correct]{Diverges}
\choice{Converges}
\end{multipleChoice}

\end{problem}}%}
%%%%%%%%%%%%%%%%%%%%%%

%%%%%%%%%%%%%%%%%%%%%%%
%%\tagged{Ans@ShortAns, Type@Compute, Topic@Series, Sub@Telescopic, Sub@Convergence, File@0009}{
\begin{sagesilent}
# Define variables and constants/exponents
var('x,n')
a=NonZeroInt(-9,9)
c=RandInt(1,9)

#Define the general term of the series
g=1/(n^3-c^2*n)
f=a*c^2*g
F=1/n


#Compute the sum of the series
Ans=a/2*(sum(F,n,1,c)-sum(F,n,c+1,2*c))

\end{sagesilent}

\latexProblemContent{
\ifVerboseLocation This is Series Compute Question 0009. \\ \fi
\begin{problem}
Determine if the series converges or diverges.  If it converges, find its sum. 

\[\sum_{n=\sage{c+1}}^\infty \sage{f}\]

\input{Series-Compute-0009.HELP.tex}

\begin{multipleChoice}
\choice{Diverges}
\choice[correct]{Converges}
\end{multipleChoice}

\begin{problem}
\[\sum_{n=\sage{c+1}}^\infty \sage{f} = \answer{\sage{Ans}}\]

\end{problem}

\end{problem}}%}
%%%%%%%%%%%%%%%%%%%%%%

%%%%%%%%%%%%%%%%%%%%%%%
%%\tagged{Ans@ShortAns, Type@Compute, Topic@Series, Sub@Telescopic, Sub@Convergence, File@0010}{
\begin{sagesilent}
# Define variables and constants/exponents
var('x,n')
a=NonZeroInt(-9,9)
c=NonZeroInt(1,5)

#Define the general term of the series
f=a*log(1-c^2/x^2)

#Compute the sum of the series
Ans=sum(f,x,c+1,infinity)

\end{sagesilent}

\latexProblemContent{
\ifVerboseLocation This is Series Compute Question 0010. \\ \fi
\begin{problem}
Determine if the series converges or diverges.  If it converges, find its sum. 

\[\sum_{n=\sage{c+1}}^\infty \sage{f(n)}\]

\input{Series-Compute-0010.HELP.tex}

\begin{multipleChoice}
\choice{Diverges}
\choice[correct]{Converges}
\end{multipleChoice}

\begin{problem}
\[\sum_{n=\sage{c+1}}^\infty \sage{f(n)} = \answer{\sage{Ans}}\]

\end{problem}

\end{problem}}%}
%%%%%%%%%%%%%%%%%%%%%%

%%%%%%%%%%%%%%%%%%%%%%%
%%\tagged{Ans@ShortAns, Type@Compute, Topic@Series, Sub@IntegralTest, Sub@Divergence, File@0011}{
\begin{sagesilent}
# Define variables and constants/exponents
var('x,n')
a=NonZeroInt(-9,9)
d=NonZeroInt(2,20)

#Define the general term of the series
f=a*log(x)/x

\end{sagesilent}

\latexProblemContent{
\ifVerboseLocation This is Series Compute Question 0011. \\ \fi
\begin{problem}
Determine if the series converges or diverges.  

\[\sum_{n=\sage{d}}^\infty \sage{f(n)}\]

\input{Series-Compute-0011.HELP.tex}

\begin{multipleChoice}
\choice[correct]{Diverges}
\choice{Converges}
\end{multipleChoice}

\end{problem}}%}
%%%%%%%%%%%%%%%%%%%%%%

%%%%%%%%%%%%%%%%%%%%%%%
%%\tagged{Ans@ShortAns, Type@Compute, Topic@Series, Sub@IntegralTest, Sub@Convergence, File@0012}{
\begin{sagesilent}
# Define variables and constants/exponents
var('x,n')
a=NonZeroInt(-9,9)
b=NonZeroInt(1,9)
d=NonZeroInt(1,20)

#Define the general term of the series
f=a*x*exp(-b*x^2)

\end{sagesilent}

\latexProblemContent{
\ifVerboseLocation This is Series Compute Question 0012. \\ \fi
\begin{problem}
Determine if the series converges or diverges.  

\[\sum_{n=\sage{d}}^\infty \sage{f(n)}\]

\input{Series-Compute-0012.HELP.tex}

\begin{multipleChoice}
\choice{Diverges}
\choice[correct]{Converges}
\end{multipleChoice}

\end{problem}}%}
%%%%%%%%%%%%%%%%%%%%%%


%%%%%%%%%%%%%%%%%%%%%%%
%%\tagged{Ans@ShortAns, Type@Compute, Topic@Series, Sub@PTest, Sub@Convergence, File@0013}{
\begin{sagesilent}
# Define variables and constants/exponents
var('x,n')
a=NonZeroInt(-9,9)
b=RandInt(2,20)
d=NonZeroInt(1,20)

#Define the general term of the series
f=a*1/x^b

\end{sagesilent}

\latexProblemContent{
\ifVerboseLocation This is Series Compute Question 0013. \\ \fi
\begin{problem}
Determine if the series converges or diverges.  

\[\sum_{n=\sage{d}}^\infty \sage{f(n)}\]

\input{Series-Compute-0013.HELP.tex}

\begin{multipleChoice}
\choice{Diverges}
\choice[correct]{Converges}
\end{multipleChoice}

\end{problem}}%}
%%%%%%%%%%%%%%%%%%%%%%

%%%%%%%%%%%%%%%%%%%%%%%
%%\tagged{Ans@ShortAns, Type@Compute, Topic@Series, Sub@PTest, Sub@Divergence, File@0014}{
\begin{sagesilent}
# Define variables and constants/exponents
var('x,n')
a=NonZeroInt(-9,9)
b=NonZeroInt(-20,1)
d=NonZeroInt(1,20)

#Define the general term of the series
f=a*1/x^b

\end{sagesilent}

\latexProblemContent{
\ifVerboseLocation This is Series Compute Question 0014. \\ \fi
\begin{problem}
Determine if the series converges or diverges.  

\[\sum_{n=\sage{d}}^\infty \sage{f(n)}\]

\input{Series-Compute-0014.HELP.tex}

\begin{multipleChoice}
\choice[correct]{Diverges}
\choice{Converges}
\end{multipleChoice}

\end{problem}}%}
%%%%%%%%%%%%%%%%%%%%%%

%%%%%%%%%%%%%%%%%%%%%%%
%%\tagged{Ans@ShortAns, Type@Compute, Topic@Series, Sub@DirectComparison, Sub@Convergence, File@0015}{
\begin{sagesilent}
# Define variables and constants/exponents
var('x,n')
a=NonZeroInt(1,20)
b=RandInt(0,20)
c=NonZeroInt(1,9)
d=NonZeroInt(1,20)

#Define the general term of the series
f=a/(x^c+b)

\end{sagesilent}

\latexProblemContent{
\ifVerboseLocation This is Series Compute Question 0015. \\ \fi
\begin{problem}
Determine if the series converges or diverges.  

\[\sum_{n=\sage{d}}^\infty \sage{f(n)}\]

\input{Series-Compute-0015.HELP.tex}

\begin{multipleChoice}
\choice{Diverges}
\choice[correct]{Converges}
\end{multipleChoice}

\end{problem}}%}
%%%%%%%%%%%%%%%%%%%%%%

%%%%%%%%%%%%%%%%%%%%%%%
%%\tagged{Ans@ShortAns, Type@Compute, Topic@Series, Sub@DirectComparison, Sub@Divergence, File@0016}{
\begin{sagesilent}
# Define variables and constants/exponents
var('x,n')
a=NonZeroInt(1,20)
b=RandInt(0,20)
d=NonZeroInt(1,20)

#Define the general term of the series
f=a/(sqrt(x)-b)

\end{sagesilent}

\latexProblemContent{
\ifVerboseLocation This is Series Compute Question 0016. \\ \fi
\begin{problem}
Determine if the series converges or diverges.  

\[\sum_{n=\sage{d}}^\infty \sage{f(n)}\]

\input{Series-Compute-0016.HELP.tex}

\begin{multipleChoice}
\choice[correct]{Diverges}
\choice{Converges}
\end{multipleChoice}

\end{problem}}%}
%%%%%%%%%%%%%%%%%%%%%%

%%%%%%%%%%%%%%%%%%%%%%%
%%\tagged{Ans@ShortAns, Type@Compute, Topic@Series, Sub@LimitComparison, Sub@Divergence, File@0017}{
\begin{sagesilent}
# Define variables and constants/exponents
var('x,n')
a=NonZeroInt(1,20)
b=RandInt(0,20)
d=NonZeroInt(1,20)

#Define the general term of the series
f=a/(sqrt(x)+b)

\end{sagesilent}

\latexProblemContent{
\ifVerboseLocation This is Series Compute Question 0017. \\ \fi
\begin{problem}
Determine if the series converges or diverges.  

\[\sum_{n=\sage{d}}^\infty \sage{f(n)}\]

\input{Series-Compute-0017.HELP.tex}

\begin{multipleChoice}
\choice[correct]{Diverges}
\choice{Converges}
\end{multipleChoice}

\end{problem}}%}
%%%%%%%%%%%%%%%%%%%%%%

%%%%%%%%%%%%%%%%%%%%%%%
%%\tagged{Ans@ShortAns, Type@Compute, Topic@Series, Sub@LimitComparison, Sub@Divergence, File@0018}{
\begin{sagesilent}
# Define variables and constants/exponents
var('x,n')
a=NonZeroInt(1,20)
b=RandInt(1,20)
d=NonZeroInt(1,20)
p=RandInt(0,1)

#Define the general term of the series
v=[a*sin(b/x), a*tan(b/x)]
f=v[p]

\end{sagesilent}

\latexProblemContent{
\ifVerboseLocation This is Series Compute Question 0018. \\ \fi
\begin{problem}
Determine if the series converges or diverges.  

\[\sum_{n=\sage{d}}^\infty \sage{f(n)}\]

\input{Series-Compute-0018.HELP.tex}

\begin{multipleChoice}
\choice[correct]{Diverges}
\choice{Converges}
\end{multipleChoice}

\end{problem}}%}
%%%%%%%%%%%%%%%%%%%%%%

%%%%%%%%%%%%%%%%%%%%%%%
%%\tagged{Ans@ShortAns, Type@Compute, Topic@Series, Sub@DirectComparison, Sub@Convergence, File@0019}{
\begin{sagesilent}
# Define variables and constants/exponents
var('x,n')
a=NonZeroInt(1,20)
b=RandInt(1,20)
d=NonZeroInt(1,20)

#Define the general term of the series
f=(a+(-1)^x)/(b*x*sqrt(x))

\end{sagesilent}

\latexProblemContent{
\ifVerboseLocation This is Series Compute Question 0019. \\ \fi
\begin{problem}
Determine if the series converges or diverges.  

\[\sum_{n=\sage{d}}^\infty \sage{f(n)}\]

\input{Series-Compute-0019.HELP.tex}

\begin{multipleChoice}
\choice{Diverges}
\choice[correct]{Converges}
\end{multipleChoice}

\end{problem}}%}
%%%%%%%%%%%%%%%%%%%%%%

%%%%%%%%%%%%%%%%%%%%%%%
%%\tagged{Ans@ShortAns, Type@Compute, Topic@Series, Sub@LimitComparison, Sub@Convergence, File@0020}{
\begin{sagesilent}
# Define variables and constants/exponents
var('x,n')
a=NonZeroInt(1,10)
b=RandInt(0,10)
c=NonZeroInt(1,10)
d=NonZeroInt(1,20)
g=NonZeroInt(1,9)
p=RandInt(0,1)
s=NonZeroInt(1,9)
t=NonZeroInt(-9,9)
r=NonZeroInt(-9,9)

#Define the general term of the series
v=[a*x^2+b*x, a*x^3+b*x^2+c*x]
w=[s*x^2+t*x+r, s*x^3+t*x^2+r]
Top=v[p]
Bot=w[p]
f=Top/(g^x*(Bot))

\end{sagesilent}

\latexProblemContent{
\ifVerboseLocation This is Series Compute Question 0020. \\ \fi
\begin{problem}
Determine if the series converges or diverges.  

\[\sum_{n=\sage{d}}^\infty \sage{f(n)}\]

\input{Series-Compute-0020.HELP.tex}

\begin{multipleChoice}
\choice{Diverges}
\choice[correct]{Converges}
\end{multipleChoice}

\end{problem}}%}
%%%%%%%%%%%%%%%%%%%%%%

%%%%%%%%%%%%%%%%%%%%%%%
%%\tagged{Ans@ShortAns, Type@Compute, Topic@Series, Sub@DirectComparison, Sub@Convergence, File@0021}{
\begin{sagesilent}
# Define variables and constants/exponents
var('x,n')
a=RandInt(1,20)
b=RandInt(0,20)
t=RandInt(1,9)
s=2*t+1
d=RandInt(1,20)

#Define the general term of the series
f=a/sqrt(n^s+b^2)

\end{sagesilent}

\latexProblemContent{
\ifVerboseLocation This is Series Compute Question 0021. \\ \fi
\begin{problem}
Determine if the series converges or diverges.  

\[\sum_{n=\sage{d}}^\infty \sage{f}\]

\input{Series-Compute-0021.HELP.tex}

\begin{multipleChoice}
\choice{Diverges}
\choice[correct]{Converges}
\end{multipleChoice}

\end{problem}}%}
%%%%%%%%%%%%%%%%%%%%%%

%%%%%%%%%%%%%%%%%%%%%%%
%%\tagged{Ans@ShortAns, Type@Compute, Topic@Series, Sub@LimitComparison, Sub@Convergence, File@0022}{
\begin{sagesilent}
# Define variables and constants/exponents
var('x,n')
a=NonZeroInt(1,20)
b=NonZeroInt(1,9)
c=NonZeroInt(-9,9)
d=NonZeroInt(1,20)
e=NonZeroInt(-9,9)

#Define the general term of the series
f=a*sqrt(x)/(b*x^2+c*x+e)

\end{sagesilent}

\latexProblemContent{
\ifVerboseLocation This is Series Compute Question 0022. \\ \fi
\begin{problem}
Determine if the series converges or diverges.  

\[\sum_{n=\sage{d}}^\infty \sage{f(n)}\]

\input{Series-Compute-0022.HELP.tex}

\begin{multipleChoice}
\choice{Diverges}
\choice[correct]{Converges}
\end{multipleChoice}

\end{problem}}%}
%%%%%%%%%%%%%%%%%%%%%%

%%%%%%%%%%%%%%%%%%%%%%%
%%\tagged{Ans@ShortAns, Type@Compute, Topic@Series, Sub@DirectComparison, Sub@LimitComparison, Sub@Convergence, File@0023}{
\begin{sagesilent}
# Define variables and constants/exponents
var('x,n')
a=RandInt(2,20)
b=RandInt(1,a-1)
c=RandInt(1,20)
e=RandInt(1,a)
d=RandInt(1,20)

#Define the general term of the series
f=(e+b^n)/(c+a^n)

\end{sagesilent}

\latexProblemContent{
\ifVerboseLocation This is Series Compute Question 0023. \\ \fi
\begin{problem}
Determine if the series converges or diverges.  

\[\sum_{n=\sage{d}}^\infty \sage{f}\]

\input{Series-Compute-0023.HELP.tex}

\begin{multipleChoice}
\choice{Diverges}
\choice[correct]{Converges}
\end{multipleChoice}

\end{problem}}%}
%%%%%%%%%%%%%%%%%%%%%%

%%%%%%%%%%%%%%%%%%%%%%%
%%\tagged{Ans@ShortAns, Type@Compute, Topic@Series, Sub@DirectComparison, Sub@Convergence, File@0024}{
\begin{sagesilent}
# Define variables and constants/exponents
var('x,n')
a=NonZeroInt(1,20)
b=NonZeroInt(2,20)
d=NonZeroInt(1,20)

#Define the general term of the series
p=RandInt(0,1)
v=[a+sin(x), a+cos(x)]
f=v[p]/b^x

\end{sagesilent}

\latexProblemContent{
\ifVerboseLocation This is Series Compute Question 0024. \\ \fi
\begin{problem}
Determine if the series converges or diverges.  

\[\sum_{n=\sage{d}}^\infty \sage{f(n)}\]

\input{Series-Compute-0024.HELP.tex}

\begin{multipleChoice}
\choice{Diverges}
\choice[correct]{Converges}
\end{multipleChoice}

\end{problem}}%}
%%%%%%%%%%%%%%%%%%%%%%

%%%%%%%%%%%%%%%%%%%%%%%
%%\tagged{Ans@ShortAns, Type@Compute, Topic@Series, Sub@Alternating, Sub@Convergence, File@0025}{
\begin{sagesilent}
# Define variables and constants/exponents
var('x,n')
a=NonZeroInt(1,20)
s=NonZeroInt(1,9)
d=NonZeroInt(1,20)

#Define the general term of the series
f=a*(-1)^(x+1)/x^s

\end{sagesilent}

\latexProblemContent{
\ifVerboseLocation This is Series Compute Question 0025. \\ \fi
\begin{problem}
Determine if the series converges or diverges.  

\[\sum_{n=\sage{d}}^\infty \sage{f(n)}\]

\input{Series-Compute-0025.HELP.tex}

\begin{multipleChoice}
\choice{Diverges}
\choice[correct]{Converges}
\end{multipleChoice}

\end{problem}}%}
%%%%%%%%%%%%%%%%%%%%%%

%%%%%%%%%%%%%%%%%%%%%%%
%%\tagged{Ans@ShortAns, Type@Compute, Topic@Series, Sub@Alternating, Sub@Convergence, File@0026}{
\begin{sagesilent}
# Define variables and constants/exponents
var('x,n')
a=NonZeroInt(1,20)
b=NonZeroInt(2,20)
d=NonZeroInt(1,20)

#Define the general term of the series
f=a*(-1)^(x+1)*b^(1/x)/x

\end{sagesilent}

\latexProblemContent{
\ifVerboseLocation This is Series Compute Question 0026. \\ \fi
\begin{problem}
Determine if the series converges or diverges.  

\[\sum_{n=\sage{d}}^\infty \sage{f(n)}\]

\input{Series-Compute-0026.HELP.tex}

\begin{multipleChoice}
\choice{Diverges}
\choice[correct]{Converges}
\end{multipleChoice}

\end{problem}}%}
%%%%%%%%%%%%%%%%%%%%%%

%%%%%%%%%%%%%%%%%%%%%%%
%%\tagged{Ans@ShortAns, Type@Compute, Topic@Series, Sub@Alternating, Sub@Divergence, File@0027}{
\begin{sagesilent}
# Define variables and constants/exponents
var('x,n')
a=NonZeroInt(1,20)
b=NonZeroInt(1,20)
c=NonZeroInt(-9,9)
d=NonZeroInt(1,20)

#Define the general term of the series
f=(-1)^x*a*x/(b*x+c)

\end{sagesilent}

\latexProblemContent{
\ifVerboseLocation This is Series Compute Question 0027. \\ \fi
\begin{problem}
Determine if the series converges or diverges.  

\[\sum_{n=\sage{d}}^\infty \sage{f(n)}\]

\input{Series-Compute-0027.HELP.tex}

\begin{multipleChoice}
\choice[correct]{Diverges}
\choice{Converges}
\end{multipleChoice}

\end{problem}}%}
%%%%%%%%%%%%%%%%%%%%%%

%%%%%%%%%%%%%%%%%%%%%%%
%%\tagged{Ans@ShortAns, Type@Compute, Topic@Series, Sub@Alternating, Sub@Convergence, File@0028}{
\begin{sagesilent}
# Define variables and constants/exponents
var('x,n')
a=NonZeroInt(1,20)
p=RandInt(0,1)
q=RandInt(0,1)
d=NonZeroInt(1,20)

#Define the general term of the series
v=[a*sin(x*pi/2), a*cos(x*pi)]
w=[factorial(x), x^x]
f=v[p]/w[q]

\end{sagesilent}

\latexProblemContent{
\ifVerboseLocation This is Series Compute Question 0028. \\ \fi
\begin{problem}
Determine if the series converges or diverges.  

\[\sum_{n=\sage{d}}^\infty \sage{f(n)}\]

\input{Series-Compute-0028.HELP.tex}

\begin{multipleChoice}
\choice{Diverges}
\choice[correct]{Converges}
\end{multipleChoice}

\end{problem}}%}
%%%%%%%%%%%%%%%%%%%%%%

%%%%%%%%%%%%%%%%%%%%%%%
%%\tagged{Ans@ShortAns, Type@Compute, Topic@Series, Sub@Alternating, Sub@Convergence, File@0029}{
\begin{sagesilent}
# Define variables and constants/exponents
var('x,n')
a=NonZeroInt(1,20)
b=NonZeroInt(1,20)
s=NonZeroInt(1,9)
d=NonZeroInt(1,20)

#Define the general term of the series
f=a*(-1)^x*x^s/b^x

\end{sagesilent}

\latexProblemContent{
\ifVerboseLocation This is Series Compute Question 0029. \\ \fi
\begin{problem}
Determine if the series converges or diverges.  

\[\sum_{n=\sage{d}}^\infty \sage{f(n)}\]

\input{Series-Compute-0029.HELP.tex}

\begin{multipleChoice}
\choice{Diverges}
\choice[correct]{Converges}
\end{multipleChoice}

\end{problem}}%}
%%%%%%%%%%%%%%%%%%%%%%

%%%%%%%%%%%%%%%%%%%%%%%
%%\tagged{Ans@ShortAns, Type@Compute, Topic@Series, Sub@RatioTest, Sub@Convergence, File@0030}{
\begin{sagesilent}
# Define variables and constants/exponents
var('x,n')
a=NonZeroInt(2,20)
d=NonZeroInt(1,20)

#Define the general term of the series
f=a^x/factorial(x)

\end{sagesilent}

\latexProblemContent{
\ifVerboseLocation This is Series Compute Question 0030. \\ \fi
\begin{problem}
Determine if the series converges or diverges.  

\[\sum_{n=\sage{d}}^\infty \sage{f(n)}\]

\input{Series-Compute-0030.HELP.tex}

\begin{multipleChoice}
\choice{Diverges}
\choice[correct]{Converges}
\end{multipleChoice}

\end{problem}}%}
%%%%%%%%%%%%%%%%%%%%%%

%%%%%%%%%%%%%%%%%%%%%%%
%%\tagged{Ans@ShortAns, Type@Compute, Topic@Series, Sub@RatioTest, Sub@Divergence, File@0031}{
\begin{sagesilent}
# Define variables and constants/exponents
var('x,n')
a=NonZeroInt(2,20)
d=NonZeroInt(1,20)
s=NonZeroInt(1,9)

#Define the general term of the series
f=a^x/x^s

\end{sagesilent}

\latexProblemContent{
\ifVerboseLocation This is Series Compute Question 0031. \\ \fi
\begin{problem}
Determine if the series converges or diverges.  

\[\sum_{n=\sage{d}}^\infty \sage{f(n)}\]

\input{Series-Compute-0031.HELP.tex}

\begin{multipleChoice}
\choice[correct]{Diverges}
\choice{Converges}
\end{multipleChoice}

\end{problem}}%}
%%%%%%%%%%%%%%%%%%%%%%

%%%%%%%%%%%%%%%%%%%%%%%
%%\tagged{Ans@ShortAns, Type@Compute, Topic@Series, Sub@RatioTest, Sub@Convergence, File@0032}{
\begin{sagesilent}
# Define variables and constants/exponents
var('x,n')
a=NonZeroInt(2,20)
d=NonZeroInt(1,20)
s=NonZeroInt(1,9)

#Define the general term of the series
f=x^s/a^x

\end{sagesilent}

\latexProblemContent{
\ifVerboseLocation This is Series Compute Question 0032. \\ \fi
\begin{problem}
Determine if the series converges or diverges.  

\[\sum_{n=\sage{d}}^\infty \sage{f(n)}\]

\input{Series-Compute-0032.HELP.tex}

\begin{multipleChoice}
\choice{Diverges}
\choice[correct]{Converges}
\end{multipleChoice}

\end{problem}}%}
%%%%%%%%%%%%%%%%%%%%%%

%%%%%%%%%%%%%%%%%%%%%%%
%%\tagged{Ans@ShortAns, Type@Compute, Topic@Series, Sub@RatioTest, Sub@Convergence, File@0033}{
\begin{sagesilent}
# Define variables and constants/exponents
var('x,n')
a=NonZeroInt(2,20)
d=NonZeroInt(1,20)
s=NonZeroInt(1,9)

#Define the general term of the series
f=(-1)^(x+1)*n^s*a^x/factorial(x)

\end{sagesilent}

\latexProblemContent{
\ifVerboseLocation This is Series Compute Question 0033. \\ \fi
\begin{problem}
Determine if the series converges or diverges.  

\[\sum_{n=\sage{d}}^\infty \sage{f(n)}\]

\input{Series-Compute-0033.HELP.tex}

\begin{multipleChoice}
\choice{Diverges}
\choice[correct]{Converges}
\end{multipleChoice}

\end{problem}}%}
%%%%%%%%%%%%%%%%%%%%%%


%%%%%%%%%%%%%%%%%%%%%%%
%%\tagged{Ans@ShortAns, Type@Compute, Topic@Series, Sub@RatioTest, Sub@Convergence, File@0034}{
\begin{sagesilent}
# Define variables and constants/exponents
var('n')
a=RandInt(1,9)
b=RandInt(1,9)
c=RandInt(-2,2)
d=RandInt(0,2)
e=RandInt(1,9)

\end{sagesilent}

\latexProblemContent{
\ifVerboseLocation This is Series Compute Question 0034. \\ \fi
\begin{problem}
Determine if the series converges or diverges.  
\[\sum_{n=\sage{d}}^\infty
\dfrac{(-1)^n(\sage{b*d+c}\cdot\sage{b*(d+1)+c}\cdot\sage{b*(d+2)+c}\cdots(\sage{b*n+c}))}
{\sage{a^n}\; (\sage{e*n})!}\]

\input{Series-Compute-0034.HELP.tex}

\begin{multipleChoice}
\choice{Diverges}
\choice[correct]{Converges}
\end{multipleChoice}

\end{problem}}%}
%%%%%%%%%%%%%%%%%%%%%%

%%%%%%%%%%%%%%%%%%%%%%%
%%\tagged{Ans@ShortAns, Type@Compute, Topic@Series, Sub@RootTest, Sub@Convergence, File@0035}{
\begin{sagesilent}
# Define variables and constants/exponents
var('x,n')
a=NonZeroInt(1,20)
b=NonZeroInt(a,21)
c=NonZeroInt(1,9)
d=NonZeroInt(1,20)

#Define the general term of the series
f=(a*x/(b*x+c))^x

\end{sagesilent}

\latexProblemContent{
\ifVerboseLocation This is Series Compute Question 0035. \\ \fi
\begin{problem}
Determine if the series converges or diverges.  

\[\sum_{n=\sage{d}}^\infty \sage{f(n)}\]

\input{Series-Compute-0035.HELP.tex}

\begin{multipleChoice}
\choice{Diverges}
\choice[correct]{Converges}
\end{multipleChoice}

\end{problem}}%}
%%%%%%%%%%%%%%%%%%%%%%

%%%%%%%%%%%%%%%%%%%%%%%
%%\tagged{Ans@ShortAns, Type@Compute, Topic@Series, Sub@RootTest, Sub@Convergence, File@0036}{
\begin{sagesilent}
# Define variables and constants/exponents
var('x,n')
a=NonZeroInt(-20,-1)
d=NonZeroInt(1,20)

#Define the general term of the series
f=(1+a/x)^(x^2)

\end{sagesilent}

\latexProblemContent{
\ifVerboseLocation This is Series Compute Question 0036. \\ \fi
\begin{problem}
Determine if the series converges or diverges.  

\[\sum_{n=\sage{d}}^\infty \sage{f(n)}\]

\input{Series-Compute-0036.HELP.tex}

\begin{multipleChoice}
\choice{Diverges}
\choice[correct]{Converges}
\end{multipleChoice}

\end{problem}}%}
%%%%%%%%%%%%%%%%%%%%%%

%%%%%%%%%%%%%%%%%%%%%%%
%%\tagged{Ans@ShortAns, Type@Compute, Topic@Series, Sub@RootTest, Sub@Divergence, File@0037}{
\begin{sagesilent}
# Define variables and constants/exponents
var('x,n')
a=NonZeroInt(1,20)
d=NonZeroInt(1,20)

#Define the general term of the series
f=(1+a/x)^(x^2)

\end{sagesilent}

\latexProblemContent{
\ifVerboseLocation This is Series Compute Question 0037. \\ \fi
\begin{problem}
Determine if the series converges or diverges.  

\[\sum_{n=\sage{d}}^\infty \sage{f(n)}\]

\input{Series-Compute-0037.HELP.tex}

\begin{multipleChoice}
\choice[correct]{Diverges}
\choice{Converges}
\end{multipleChoice}

\end{problem}}%}
%%%%%%%%%%%%%%%%%%%%%%

%%%%%%%%%%%%%%%%%%%%%%%
%%\tagged{Ans@ShortAns, Type@Compute, Topic@Series, Sub@RootTest, Sub@Convergence, File@0038}{
\begin{sagesilent}
# Define variables and constants/exponents
var('x,n')
a=NonZeroInt(1,20)
d=NonZeroInt(1,20)

#Define the general term of the series
f=a*x/(log(x))^x

\end{sagesilent}

\latexProblemContent{
\ifVerboseLocation This is Series Compute Question 0038. \\ \fi
\begin{problem}
Determine if the series converges or diverges.  

\[\sum_{n=\sage{d}}^\infty \sage{f(n)}\]

\input{Series-Compute-0038.HELP.tex}

\begin{multipleChoice}
\choice{Diverges}
\choice[correct]{Converges}
\end{multipleChoice}

\end{problem}}%}
%%%%%%%%%%%%%%%%%%%%%%

%%%%%%%%%%%%%%%%%%%%%%%
%%\tagged{Ans@ShortAns, Type@Compute, Topic@Series, Sub@RootTest, Sub@Convergence, File@0039}{
\begin{sagesilent}
# Define variables and constants/exponents
var('x,n')
a=NonZeroInt(1,20)
b=NonZeroInt(1,20)
d=NonZeroInt(1,20)

#Define the general term of the series
f=a^(b*x)/x^x

\end{sagesilent}

\latexProblemContent{
\ifVerboseLocation This is Series Compute Question 0039. \\ \fi
\begin{problem}
Determine if the series converges or diverges.  

\[\sum_{n=\sage{d}}^\infty \sage{f(n)}\]

\input{Series-Compute-0039.HELP.tex}

\begin{multipleChoice}
\choice{Diverges}
\choice[correct]{Converges}
\end{multipleChoice}

\end{problem}}%}
%%%%%%%%%%%%%%%%%%%%%%

%%%%%%%%%%%%%%%%%%%%%%%
%%\tagged{Ans@ShortAns, Type@Compute, Topic@Series, Sub@RootTest, Sub@Divergence, File@0040}{
\begin{sagesilent}
# Define variables and constants/exponents
var('x,n')
a=RandInt(2,20)
b=RandInt(1,a-1)
c=NonZeroInt(-20,20)
d=RandInt(1,20)
e=NonZeroInt(-20,20)

#Define the general term of the series
f=((a*n^2+c)/(b*n^2+e))^n

\end{sagesilent}

\latexProblemContent{
\ifVerboseLocation This is Series Compute Question 0040. \\ \fi
\begin{problem}
Determine if the series converges or diverges.  

\[\sum_{n=\sage{d}}^\infty \sage{f}\]

\input{Series-Compute-0040.HELP.tex}

\begin{multipleChoice}
\choice[correct]{Diverges}
\choice{Converges}
\end{multipleChoice}

\end{problem}}%}
%%%%%%%%%%%%%%%%%%%%%%

%%%%%%%%%%%%%%%%%%%%%%%
%%\tagged{Ans@ShortAns, Type@Compute, Topic@Series, Sub@RootTest, Sub@Convergence, File@0041}{
\begin{sagesilent}
# Define variables and constants/exponents
var('x,n')
a=NonZeroInt(1,19)
b=NonZeroInt(a+1,21)
c=NonZeroInt(-20,20)
d=NonZeroInt(1,20)
e=NonZeroInt(-20,20)

#Define the general term of the series
f=((a*n^2+c)/(b*n^2+e))^n

\end{sagesilent}

\latexProblemContent{
\ifVerboseLocation This is Series Compute Question 0041. \\ \fi
\begin{problem}
Determine if the series converges or diverges.  

\[\sum_{n=\sage{d}}^\infty \sage{f}\]

\input{Series-Compute-0041.HELP.tex}

\begin{multipleChoice}
\choice{Diverges}
\choice[correct]{Converges}
\end{multipleChoice}

\end{problem}}%}
%%%%%%%%%%%%%%%%%%%%%%

%%%%%%%%%%%%%%%%%%%%%%%
%%\tagged{Ans@ShortAns, Type@Compute, Topic@Series, Sub@RootTest, Sub@Divergence, File@0042}{
\begin{sagesilent}
# Define variables and constants/exponents
var('x,n')
a=RandInt(1,20)
b=RandInt(1,20)
d=RandInt(1,20)

#Define the general term of the series
f=(a*factorial(n))^n/n^(b*n)

\end{sagesilent}

\latexProblemContent{
\ifVerboseLocation This is Series Compute Question 0042. \\ \fi
\begin{problem}
Determine if the series converges or diverges.  

\[\sum_{n=\sage{d}}^\infty \sage{f}\]

\input{Series-Compute-0042.HELP.tex}

\begin{multipleChoice}
\choice[correct]{Diverges}
\choice{Converges}
\end{multipleChoice}

\end{problem}}%}
%%%%%%%%%%%%%%%%%%%%%%

%%%%%%%%%%%%%%%%% Finished with series tests %%%%%%%%%%%%%%%


%%%%%%%%%%%%%%%%%%%%%%%
%%\tagged{Ans@ShortAns, Type@Compute, Topic@Series, Sub@PowerSeries, Sub@RadiusOfConvergence, Sub@IntervalOfConvergence, File@0043}{
\begin{sagesilent}
# Define variables and constants/exponents
var('x,n')
a=NonZeroInt(1,20)
b=NonZeroInt(1,20)
d=NonZeroInt(1,20)

#Define the general term of the series
f=a*x^(b*n)/factorial(n)

\end{sagesilent}

\latexProblemContent{
\ifVerboseLocation This is Series Compute Question 0043. \\ \fi
\begin{problem}
Determine the radius of convergence and the interval of convergence. 

\[\sum_{n=\sage{d}}^\infty \sage{f}\]

\input{Series-Compute-0043.HELP.tex}

\[\mbox{Radius of Convergence}:\; \answer{\infty}\qquad \mbox{Interval of Convergence}:\; (\answer{-\infty},\answer{\infty})\]

\end{problem}}%}
%%%%%%%%%%%%%%%%%%%%%%

%%%%%%%%%%%%%%%%%%%%%%%
%%\tagged{Ans@ShortAns, Type@Compute, Topic@Series, Sub@PowerSeries, Sub@RadiusOfConvergence, Sub@IntervalOfConvergence, File@0044}{
\begin{sagesilent}
# Define variables and constants/exponents
var('x,n')
a=RandInt(-20,20)
d=RandInt(0,20)

#Define the general term of the series
f=(x-a)^n

\end{sagesilent}

\latexProblemContent{
\ifVerboseLocation This is Series Compute Question 0044. \\ \fi
\begin{problem}
Determine the radius of convergence and the interval of convergence. 

\[\sum_{n=\sage{d}}^\infty \sage{f}\]

\input{Series-Compute-0044.HELP.tex}

\[\mbox{Radius of Convergence}:\; \answer{1}\qquad \mbox{Interval of Convergence}:\; = \answer{\sage{a-1}}<x<\answer{\sage{a+1}}\]

\end{problem}}%}
%%%%%%%%%%%%%%%%%%%%%%

%%%%%%%%%%%%%%%%%%%%%%%
%%\tagged{Ans@ShortAns, Type@Compute, Topic@Series, Sub@PowerSeries, Sub@RadiusOfConvergence, Sub@IntervalOfConvergence, File@0045}{
\begin{sagesilent}
# Define variables and constants/exponents
var('x,n')
a=RandInt(-20,20)
b=NonZeroInt(1,20)
d=RandInt(0,20)

#Define the general term of the series
f=(x-a)^n/(n+b)

\end{sagesilent}

\latexProblemContent{
\ifVerboseLocation This is Series Compute Question 0045. \\ \fi
\begin{problem}
Determine the radius of convergence and the interval of convergence. 

\[\sum_{n=\sage{d}}^\infty \sage{f}\]

\input{Series-Compute-0045.HELP.tex}

\[\mbox{Radius of Convergence}:\; \answer{1}\qquad \mbox{Interval of Convergence}:\; = \answer{\sage{a-1}}<x<\answer{\sage{a+1}}\]

\end{problem}}%}
%%%%%%%%%%%%%%%%%%%%%%

%%%%%%%%%%%%%%%%%%%%%%%%
%%%\tagged{Ans@ShortAns, Type@Compute, Topic@Series, Sub@PowerSeries, Sub@RadiusOfConvergence, Sub@IntervalOfConvergence, File@0046}{
%\begin{sagesilent}
%# Define variables and constants/exponents
%var('x,n')
%a=RandInt(-20,20)
%b=NonZeroInt(1,20)
%c=NonZeroInt(1,9)
%d=RandInt(0,20)
%
%#Define the general term of the series
%f=(-1)^n*(x-a)^n/((n+b)*c^n)
%
%\end{sagesilent}
%
%\latexProblemContent{
%\ifVerboseLocation This is Series Compute Question 0046. \\ \fi
%\begin{problem}
%Determine the radius of convergence and the interval of convergence. 
%
%\[\sum_{n=\sage{d}}^\infty \sage{f}\]
%
%\input{Series-Compute-0046.HELP.tex}
%
%\[\mbox{Radius of Convergence}:\; \answer{c}\qquad \mbox{Interval of Convergence}:\; = \answer{\sage{a-c}}<x<\answer{\sage{a+c}}\]
%
%\end{problem}}%}
%%%%%%%%%%%%%%%%%%%%%%%


\end{document}
