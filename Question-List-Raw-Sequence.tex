%%%%%%%%%%%%%%%%%%%%%%%%%%
%%%%%%%%%%%%%%%%%%%%%%%%%%


%%%%%%%%%%%%%%%%%%%%%%%
%%\tagged{Ans@ShortAns, Type@Compute, Topic@Sequence, Sub@Poly, File@0001}{
\begin{sagesilent}
# Define variables and constants/exponents
var('x,n')
a=NonZeroInt(-5,5)
b = NonZeroInt(-20,20)

#Define the general term of the sequence
f(x)=b/x^a

\end{sagesilent}

\latexProblemContent{
\ifVerboseLocation This is Sequence Compute Question 0001. \\ \fi
\begin{problem}

Find the general term of the sequence 

\input{Sequence-Compute-0001.HELP.tex}

\[\{a_n\} = \left\{\sage{f(1)}, \sage{f(2)}, \sage{f(3)}, \sage{f(4)}, \sage{f(5)}, \ldots \right\} \qquad a_n=\answer{\sage{f(n)}}\]
\end{problem}}%}
%%%%%%%%%%%%%%%%%%%%%%

%%%%%%%%%%%%%%%%%%%%%%%
%%\tagged{Ans@ShortAns, Type@Compute, Topic@Sequence, Sub@Alternating, Sub@Poly, File@0002}{
\begin{sagesilent}
# Define variables and constants/exponents
var('x,n')
a=NonZeroInt(-5,5)
b = NonZeroInt(-20,20)

#Define the general term of the sequence
f(x)=b*(-1)^x/x^a

\end{sagesilent}

\latexProblemContent{
\ifVerboseLocation This is Sequence Compute Question 0002. \\ \fi
\begin{problem}

Find the general term of the sequence 

\input{Sequence-Compute-0002.HELP.tex}

\[\{a_n\} = \left\{\sage{f(1)}, \sage{f(2)}, \sage{f(3)}, \sage{f(4)}, \sage{f(5)}, \ldots \right\} \qquad a_n=\answer{\sage{f(n)}}\]
\end{problem}}%}
%%%%%%%%%%%%%%%%%%%%%%

%%%%%%%%%%%%%%%%%%%%%%%
%%\tagged{Ans@ShortAns, Type@Compute, Topic@Sequence, Sub@Convergence, Sub@Alternating, Sub@Poly, File@0003}{
\begin{sagesilent}
# Define variables and constants/exponents
var('x,n')
a=NonZeroInt(1,9)

coef1 = NonZeroInt(-15,15)
coef2 = NonZeroInt(-15,15)

coef = coef1/coef2


#Define the general term of the sequence
f=coef*(-1)^x/x^a

#Compute limit of sequence
# Ans=limit(f,x=infinity)# Ans is always 0 by question design.

\end{sagesilent}

\latexProblemContent{
\ifVerboseLocation This is Sequence Compute Question 0003. \\ \fi
\begin{problem}

Determine whether the sequence converges or diverges.  If it converges, find its limit.

\[\{a_n\} = \left\{\sage{f(n)}\right\}\]

\input{Sequence-Compute-0003.HELP.tex}


\begin{multipleChoice}
\choice{Diverges}
\choice[correct]{Converges}
\end{multipleChoice}

\begin{problem}
\[\lim_{n\to\infty} a_n = \answer{0}\]

\end{problem}

\end{problem}}%}
%%%%%%%%%%%%%%%%%%%%%%


%%%%%%%%%%%%%%%%%%%%%%%
%%\tagged{Ans@ShortAns, Type@Compute, Topic@Sequence, Sub@Convergence, Sub@Factorial, File@0004}{
\begin{sagesilent}
# Define variables and constants/exponents
var('x,n')
a=NonZeroInt(1,5)
coef1 = NonZeroInt(-15,15)
coef2 = NonZeroInt(-15,15)

temp = coef1/coef2
ctop = numerator(temp)
cbot = denominator(temp)

#Define the general term of the sequence
ftop=(x-a)
fbot=(x+a)

#Compute limit of sequence
# Ans=limit(f,x=infinity) # This will always be 0 as the problem is designed.

\end{sagesilent}

\latexProblemContent{
\ifVerboseLocation This is Sequence Compute Question 0004. \\ \fi
\begin{problem}

Determine whether the sequence converges or diverges.  If it converges, find its limit.

\[\{a_n\} = \left\lbrace \frac{\sage{ctop} \left( \sage{ftop(n)} \right) !}{\sage{cbot} \left( \sage{fbot(n)}\right)!} \right\rbrace\]

\input{Sequence-Compute-0004.HELP.tex}


\begin{multipleChoice}
\choice{Diverges}
\choice[correct]{Converges}
\end{multipleChoice}

\begin{problem}
\[\lim_{n\to\infty} a_n = \answer{0}\]

\end{problem}
\end{problem}}%}
%%%%%%%%%%%%%%%%%%%%%%


%%%%%%%%%%%%%%%%%%%%%%%
%%\tagged{Ans@ShortAns, Type@Compute, Topic@Sequence, Sub@Divergence, Sub@Poly, File@0005}{
\begin{sagesilent}
# Define variables and constants/exponents
var('x,n')
c1=RandInt(3,20)
c2 = RandInt(1,c1)
a=c1/c2 # >=1 by design.
p=Integer(randint(0,2))
s=Integer(randint(1,9))


#Define the general term of the sequence
v=[a*x^s, (-a)^x, a*cos(pi*x)]
f=v[p]

#Compute limit of sequence
Ans=limit(f,x=infinity)

\end{sagesilent}

\latexProblemContent{
\ifVerboseLocation This is Sequence Compute Question 0005. \\ \fi
\begin{problem}
Determine whether the sequence converges or diverges.  If it converges, find its limit.

\[\{a_n\} = \left\{\sage{f(n)}\right\}\]

\input{Sequence-Compute-0005.HELP.tex}


\begin{multipleChoice}
\choice[correct]{Diverges}
\choice{Converges}
\end{multipleChoice}

\end{problem}}%}
%%%%%%%%%%%%%%%%%%%%%%

%%%%%%%%%%%%%%%%%%%%%%%
%%\tagged{Ans@ShortAns, Type@Compute, Topic@Sequence, Sub@Trig, Sub@Convergence, File@0006}{
\begin{sagesilent}
# Define variables and constants/exponents
var('x,n')

p=Integer(randint(0,1))

coef1 = NonZeroInt(-15,15)
coef2 = NonZeroInt(-15,15)

coef = coef1/coef2


#Define the general term of the sequence
v=[coef*cos(2*pi*x), coef*sin(2*pi*x)]
f=v[p]

#Compute limit of sequence
if p==0:
   Ans=coef
else:
   Ans=0

\end{sagesilent}

\latexProblemContent{
\ifVerboseLocation This is Sequence Compute Question 0006. \\ \fi
\begin{problem}
Determine whether the sequence converges or diverges.  If it converges, find its limit.

\[\{a_n\} = \left\{\sage{f(n)}\right\}\]

\input{Sequence-Compute-0006.HELP.tex}


\begin{multipleChoice}
\choice{Diverges}
\choice[correct]{Converges}
\end{multipleChoice}

\begin{problem}
\[\lim_{n\to\infty} a_n = \answer{\sage{Ans}}\]

\end{problem}

\end{problem}}%}
%%%%%%%%%%%%%%%%%%%%%%

%%%%%%%%%%%%%%%%%%%%%%%
%%\tagged{Ans@ShortAns, Type@Compute, Topic@Sequence, Sub@Trig, Sub@Exp, Sub@Convergence, File@0007}{
\begin{sagesilent}
# Define variables and constants/exponents
var('x,n')
a=NonZeroInt(-9,9)
b=NonZeroInt(1,9)
c=NonZeroInt(1,9)


#Define the general term of the sequence
f=a*cos(x)^b/c^x

#Compute limit of sequence
Ans=limit(f,x=infinity)

\end{sagesilent}

\latexProblemContent{
\ifVerboseLocation This is Sequence Compute Question 0007. \\ \fi
\begin{problem}
Determine whether the sequence converges or diverges.  If it converges, find its limit.

\[\{a_n\} = \left\{\sage{f(n)}\right\}\]

\input{Sequence-Compute-0007.HELP.tex}


\begin{multipleChoice}
\choice{Diverges}
\choice[correct]{Converges}
\end{multipleChoice}

\begin{problem}
\[\lim_{n\to\infty} a_n = \answer{\sage{Ans}}\]

\end{problem}

\end{problem}}%}
%%%%%%%%%%%%%%%%%%%%%%

%%%%%%%%%%%%%%%%%%%%%%%
%%\tagged{Ans@ShortAns, Type@Compute, Topic@Sequence, Sub@Exp, Sub@Convergence, File@0008}{
\begin{sagesilent}
# Define variables and constants/exponents
var('x,n')
a=NonZeroInt(-9,9)
b=NonZeroInt(-9,9)


#Define the general term of the sequence
f=a*(1+b/x)^x

#Compute limit of sequence
Ans=limit(f,x=infinity)

\end{sagesilent}

\latexProblemContent{
\ifVerboseLocation This is Sequence Compute Question 0008. \\ \fi
\begin{problem}

Determine whether the sequence converges or diverges.  If it converges, find its limit.

\[\{a_n\} = \left\{\sage{f(n)}\right\}\]

\input{Sequence-Compute-0008.HELP.tex}


\begin{multipleChoice}
\choice{Diverges}
\choice[correct]{Converges}
\end{multipleChoice}

\begin{problem}
\[\lim_{n\to\infty} a_n = \answer{\sage{Ans}}\]


\end{problem}

\end{problem}}%}
%%%%%%%%%%%%%%%%%%%%%%

%%%%%%%%%%%%%%%%%%%%%%%
%%\tagged{Ans@ShortAns, Type@Compute, Topic@Sequence, Sub@Trig, Sub@Convergence, File@0009}{
\begin{sagesilent}
# Define variables and constants/exponents
var('x,n')
a=NonZeroInt(1,9)

coef1 = NonZeroInt(-15,15)
coef2 = NonZeroInt(-15,15)

coef = coef1/coef2


#Define the general term of the sequence
f=coef*x/a*sin(a/x)

#Compute limit of sequence
# Ans=limit(f,x=infinity) # Answer is always coef by design.

\end{sagesilent}

\latexProblemContent{
\ifVerboseLocation This is Sequence Compute Question 0009. \\ \fi
\begin{problem}
Determine whether the sequence converges or diverges.  If it converges, find its limit.

\[\{a_n\} = \left\{\sage{f(n)}\right\}\]

\input{Sequence-Compute-0009.HELP.tex}


\begin{multipleChoice}
\choice{Diverges}
\choice[correct]{Converges}
\end{multipleChoice}

\begin{problem}
\[\lim_{n\to\infty} a_n = \answer{\sage{coef}}\]

\end{problem}

\end{problem}}%}
%%%%%%%%%%%%%%%%%%%%%%

%%%%%%%%%%%%%%%%%%%%%%%
%%\tagged{Ans@ShortAns, Type@Compute, Topic@Sequence, Sub@Squeeze, Sub@Rational, Sub@Convergence, File@0010}{
\begin{sagesilent}
# Define variables and constants/exponents
var('x,n')
a=NonZeroInt(-9,9)
b=NonZeroInt(-9,9)


#Define the general term of the sequence
f=(-1)^(x+a)*x/(x^2+b)

#Compute limit of sequence
Ans=limit(f,x=infinity)

\end{sagesilent}

\latexProblemContent{
\ifVerboseLocation This is Sequence Compute Question 0010. \\ \fi
\begin{problem}
Determine whether the sequence converges or diverges.  If it converges, find its limit.

\[\{a_n\} = \left\{\sage{f(n)}\right\}\]

\input{Sequence-Compute-0010.HELP.tex}


\begin{multipleChoice}
\choice{Diverges}
\choice[correct]{Converges}
\end{multipleChoice}

\begin{problem}
\[\lim_{n\to\infty} a_n = \answer{\sage{Ans}}\]

\end{problem}

\end{problem}}%}
%%%%%%%%%%%%%%%%%%%%%%

%%%%%%%%%%%%%%%%%%%%%%%
%%\tagged{Ans@ShortAns, Type@Compute, Topic@Sequence, Sub@Squeeze, Sub@Trig, Sub@Convergence, File@0011}{
\begin{sagesilent}
# Define variables and constants/exponents
var('x,n')
a=NonZeroInt(-9,9)
b=NonZeroInt(-9,9)
c=NonZeroInt(1,9)
p=Integer(randint(0,1))

#Define the general term of the sequence
v=[a*sin(b*x), a*cos(b*x)]
f=v[p]/x^c

#Compute limit of sequence
Ans=limit(f,x=infinity)

\end{sagesilent}

\latexProblemContent{
\ifVerboseLocation This is Sequence Compute Question 0011. \\ \fi
\begin{problem}
Determine whether the sequence converges or diverges.  If it converges, find its limit.

\[\{a_n\} = \left\{\sage{f(n)}\right\}\]

\input{Sequence-Compute-0011.HELP.tex}


\begin{multipleChoice}
\choice{Diverges}
\choice[correct]{Converges}
\end{multipleChoice}

\begin{problem}
\[\lim_{n\to\infty} a_n = \answer{\sage{Ans}}\]

\end{problem}

\end{problem}}%}
%%%%%%%%%%%%%%%%%%%%%%

%%%%%%%%%%%%%%%%%%%%%%%
%%\tagged{Ans@ShortAns, Type@Compute, Topic@Sequence, Sub@Geometric, Sub@Convergence, File@0012}{
\begin{sagesilent}
# Define variables and constants/exponents
var('x,n')
bT=RandInt(2,9)
aT=RandInt(1,bT-1)
flip1 = RandInt(0,1)
flip2 = RandInt(0,1)

a = (-1)^flip1*aT
b = (-1)^flip2*bT

c=NonZeroInt(1,9)

#Define the general term of the sequence
f=(a^x)/b^(x+c)

#Compute limit of sequence
Ans=limit(f,x=infinity)

\end{sagesilent}

\latexProblemContent{
\ifVerboseLocation This is Sequence Compute Question 0012. \\ \fi
\begin{problem}
Determine whether the sequence converges or diverges.  If it converges, find its limit.

\[\{a_n\} = \left\{\sage{f(n)}\right\}\]

\input{Sequence-Compute-0012.HELP.tex}


\begin{multipleChoice}
\choice{Diverges}
\choice[correct]{Converges}
\end{multipleChoice}

\begin{problem}
\[\lim_{n\to\infty} a_n = \answer{\sage{Ans}}\]

\end{problem}

\end{problem}}%}
%%%%%%%%%%%%%%%%%%%%%%

%%%%%%%%%%%%%%%%%%%%%%%
%%\tagged{Ans@ShortAns, Type@Compute, Topic@Sequence, Sub@Geometric, Sub@Divergence, File@0013}{
\begin{sagesilent}
# Define variables and constants/exponents
var('x,n')
a=NonZeroInt(-9,9)
b=NonZeroInt(-9,9)
c=NonZeroInt(1,9)

while a^2<=b^2:
   a=NonZeroInt(-9,9)
   b=NonZeroInt(-9,9)

#Define the general term of the sequence
f=(a^x)/b^(x+c)

\end{sagesilent}

\latexProblemContent{
\ifVerboseLocation This is Sequence Compute Question 0013. \\ \fi
\begin{problem}
Determine whether the sequence converges or diverges.  If it converges, find its limit.

\[\{a_n\} = \left\{\sage{f(n)}\right\}\]

\input{Sequence-Compute-0013.HELP.tex}

\begin{multipleChoice}
\choice[correct]{Diverges}
\choice{Converges}
\end{multipleChoice}

\end{problem}}%}
%%%%%%%%%%%%%%%%%%%%%%

%%%%%%%%%%%%%%%%%%%%%%%
%%\tagged{Ans@ShortAns, Type@Compute, Topic@Sequence, Sub@Rational, Sub@Convergence, File@0014}{
\begin{sagesilent}
# Define variables and constants/exponents
var('x,n')
a=NonZeroInt(-9,9)
b=RandInt(-9,9)
c=NonZeroInt(-9,9)
d=RandInt(-9,9)

s=RandInt(1,8)
t=RandInt(s,9)

if t == s:
   while b/a == d/c:
      d = RandInt(-9,9)

#Define the general term of the sequence
f=(a*x^s+b)/(c*x^t+d)

#Compute limit of sequence

if t>s:
   Ans = 0
else:
   Ans = a/c

\end{sagesilent}

\latexProblemContent{
\ifVerboseLocation This is Sequence Compute Question 0014. \\ \fi
\begin{problem}
Determine whether the sequence converges or diverges.  If it converges, find its limit.

\[\{a_n\} = \left\{\sage{f(n)}\right\}\]

\input{Sequence-Compute-0014.HELP.tex}

\begin{multipleChoice}
\choice{Diverges}
\choice[correct]{Converges}
\end{multipleChoice}

\begin{problem}
\[\lim_{n\to\infty} a_n = \answer{\sage{Ans}}\]

\end{problem}

\end{problem}}%}
%%%%%%%%%%%%%%%%%%%%%%




%%%%%%%%%%%%%%%%%%%%%%%
%%\tagged{Ans@ShortAns, Type@Compute, Topic@Sequence, Sub@Rational, Sub@Divergence, File@0015}{
\begin{sagesilent}
# Define variables and constants/exponents
var('x,n')
a=NonZeroInt(-9,9)
b=NonZeroInt(-9,9)
s=RandInt(1,9)

#Define the general term of the sequence
f=exp(x)/(a*x^s+b)

\end{sagesilent}

\latexProblemContent{
\ifVerboseLocation This is Sequence Compute Question 0015. \\ \fi
\begin{problem}
Determine whether the sequence converges or diverges.  If it converges, find its limit.

\[\{a_n\} = \left\{\sage{f(n)}\right\}\]

\input{Sequence-Compute-0015.HELP.tex}

\begin{multipleChoice}
\choice[correct]{Diverges}
\choice{Converges}
\end{multipleChoice}

\end{problem}}%}
%%%%%%%%%%%%%%%%%%%%%%

%%%%%%%%%%%%%%%%%%%%%%%
%%\tagged{Ans@ShortAns, Type@Compute, Topic@Sequence, Sub@Recursive, Sub@Convergence, File@0016}{
\begin{sagesilent}
# Define variables and constants/exponents
var('x,n')
a=NonZeroInt(2,9)

coef1 = RandInt(3,20)
coef2 = RandInt(1,coef1)
coef = coef1/coef2 # >=1 by design for monotonicity

ans = coef^2 * a
\end{sagesilent}

\latexProblemContent{
\ifVerboseLocation This is Sequence Compute Question 0016. \\ \fi
\begin{problem}
The sequence $\left\{\sage{coef}\sqrt{\sage{a}},\sage{coef}\sqrt{\sage{a}\sqrt{\sage{a}}}, \sage{coef}\sqrt{\sage{a}\sqrt{\sage{a}\sqrt{\sage{a}}}},\ldots\right\}$ is monotonically increasing and bounded above.  Find the limit.

\input{Sequence-Compute-0016.HELP.tex}

\[\answer{\sage{ans}}\]

\end{problem}}%}
%%%%%%%%%%%%%%%%%%%%%%






%%%%%%%%%%%%%%%%%%%%%%%%%%%%%%%%%%%%%%%%%%%%%%%%%%%%%%%%%%%%%%%%%%%%%%%%%%%%%%%
%
%
%
%
%
%
%
%
%%%%%%%%%%%%%%%%%%%%%%%%%%%%%%%%%%%%%%%%%%%%%%%%%%%%%%%%%%%%%%%%%%%%%%%%%%%%%%%
%%%%%%%%%%%%%%%%%%%%%%%%%%%%%%%%%%%%%%%%%%%%%%%%%%%%%%%%%%%%%%%%%%%%%%%%%%%%%%%
%%%%%%%%%%%%%%%%%%%										%%%%%%%%%%%%%%%%%%%%%%%
%%%%%%%%%%%%%%%%%%%				Concept					%%%%%%%%%%%%%%%%%%%%%%%
%%%%%%%%%%%%%%%%%%%										%%%%%%%%%%%%%%%%%%%%%%%
%%%%%%%%%%%%%%%%%%%%%%%%%%%%%%%%%%%%%%%%%%%%%%%%%%%%%%%%%%%%%%%%%%%%%%%%%%%%%%%
%%%%%%%%%%%%%%%%%%%%%%%%%%%%%%%%%%%%%%%%%%%%%%%%%%%%%%%%%%%%%%%%%%%%%%%%%%%%%%%
%
%
%
%
%
%
%
%
%
%%%%%%%%%%%%%%%%%%%%%%%%%%%%%%%%%%%%%%%%%%%%%%%%%%%%%%%%%%%%%%%%%%%%%%%%%%%%%%%



%%%%%%%%%%%%%%%%%%%%%%%
%%\tagged{Ans@ShortAns, Type@Concept, Topic@Sequence, Sub@Alternating, Sub@Convergence, File@0001}{
\begin{sagesilent}
# Define variables and constants/exponents
var('x,n')


a=NonZeroInt(1,30)
pwr = RandInt(1,4)
shift = NonZeroInt(1,30)

an = shift + (a/(n^pwr))



\end{sagesilent}

\latexProblemContent{
\ifVerboseLocation This is Sequence Concept Question 0001. \\ \fi
\begin{problem}
Suppose a sequence $\{a_n\}$ is alternating, and the absolute value of the sequence, $\{ |a_n| \}$, is strictly decreasing. Must the sequence converge?

\input{Sequence-Concept-0001.HELP.tex}
\begin{multipleChoice}
\choice{ Yes }
\choice[correct]{ No }
\end{multipleChoice}

\begin{problem}
Consider the sequence $a_n = \left\{ (-1)^n \left(\sage{an} \right) \right\}$. What is the limit of this sequence? (Put DNE if the limit does not exist).


\[\lim\limits_{n\rightarrow\infty} a_n =  \answer{DNE}.\]

\end{problem}


\end{problem}}%}
%%%%%%%%%%%%%%%%%%%%%%




%%%%%%%%%%%%%%%%%%%%%%%
%%\tagged{Ans@ShortAns, Type@Concept, Topic@Sequence, Sub@Convergence, File@0002}{
\begin{sagesilent}
# Define variables and constants/exponents
var('x,n')


a=NonZeroInt(-10,10)
pwr = RandInt(1,4)
shift = RandInt(-10,10)

an = shift + (a*n^pwr)



if pwr == 1:
   ans1 = 'Yes'
   ans1F = 'No'
   ans2 = 'Yes'
   ans2F = 'No'
   ans3 = 'Yes'
   ans3F = 'No'

if pwr == 2:
   ans1 = 'No'
   ans1F = 'Yes'
   ans2 = 'Yes'
   ans2F = 'No'
   ans3 = 'Yes'
   ans3F = 'No'

if pwr == 3:
   ans1 = 'No'
   ans1F = 'Yes'
   ans2 = 'No'
   ans2F = 'Yes'
   ans3 = 'Yes'
   ans3F = 'No'

if pwr == 4:
   ans1 = 'No'
   ans1F = 'Yes'
   ans2 = 'No'
   ans2F = 'Yes'
   ans3 = 'No'
   ans3F = 'Yes'

\end{sagesilent}

\latexProblemContent{
\ifVerboseLocation This is Sequence Concept Question 0002. \\ \fi
\begin{problem}
Consider the sequence $a_n = \{\sage{an}\}$. 


1) Does the sequence $\{\frac{a_n}{n}\}$ converge?
\begin{multipleChoice}
\choice[correct]{$\sage{ans1}$}
\choice{$\sage{ans1F}$}
\end{multipleChoice}

2) Does the sequence $\{\frac{a_n}{n^2}\}$ converge?
\begin{multipleChoice}
\choice[correct]{$\sage{ans2}$}
\choice{$\sage{ans2F}$}
\end{multipleChoice}

3) Does the sequence $\{\frac{a_n}{n^3}\}$ converge?
\begin{multipleChoice}
\choice[correct]{$\sage{ans3}$}
\choice{$\sage{ans3F}$}
\end{multipleChoice}

4) Does the sequence $\{\frac{a_n}{n^4}\}$ converge?
\begin{multipleChoice}
\choice[correct]{Yes}
\choice{No}
\end{multipleChoice}



\input{Sequence-Concept-0002.HELP.tex}



\end{problem}}%}
%%%%%%%%%%%%%%%%%%%%%%






