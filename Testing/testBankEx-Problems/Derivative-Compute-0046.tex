% Ans        : ShortAns
% File       : 0046
% Sub        : Optimization
% Topic      : Derivative
% Type       : Compute

\ProblemFileHeader{30}
\ifquestionPull
\ifproblemToFind\latexProblemContent{
\ifVerboseLocation This is Derivative Compute Question 0046. \\ \fi
\begin{problem}

A Norman window has the shape of a rectangle surmounted by a semicircle (i.e. the diameter of the semicircle is equal to the width of the rectangle).  If the perimeter of the window is $64$ ft, what area will allow in the greatest amount of light?

\input{Derivative-Compute-0046.HELP.tex}

\[\mbox{The largest possible area is}\; \answer{\frac{2048}{3 \, \pi + 4}}\]
\end{problem}}

%%%%%%%%%%%%%%%%%%%%%%

\latexProblemContent{
\ifVerboseLocation This is Derivative Compute Question 0046. \\ \fi
\begin{problem}

A Norman window has the shape of a rectangle surmounted by a semicircle (i.e. the diameter of the semicircle is equal to the width of the rectangle).  If the perimeter of the window is $12$ ft, what area will allow in the greatest amount of light?

\input{Derivative-Compute-0046.HELP.tex}

\[\mbox{The largest possible area is}\; \answer{\frac{72}{3 \, \pi + 4}}\]
\end{problem}}

%%%%%%%%%%%%%%%%%%%%%%

\latexProblemContent{
\ifVerboseLocation This is Derivative Compute Question 0046. \\ \fi
\begin{problem}

A Norman window has the shape of a rectangle surmounted by a semicircle (i.e. the diameter of the semicircle is equal to the width of the rectangle).  If the perimeter of the window is $2$ ft, what area will allow in the greatest amount of light?

\input{Derivative-Compute-0046.HELP.tex}

\[\mbox{The largest possible area is}\; \answer{\frac{2}{3 \, \pi + 4}}\]
\end{problem}}

%%%%%%%%%%%%%%%%%%%%%%

\latexProblemContent{
\ifVerboseLocation This is Derivative Compute Question 0046. \\ \fi
\begin{problem}

A Norman window has the shape of a rectangle surmounted by a semicircle (i.e. the diameter of the semicircle is equal to the width of the rectangle).  If the perimeter of the window is $30$ ft, what area will allow in the greatest amount of light?

\input{Derivative-Compute-0046.HELP.tex}

\[\mbox{The largest possible area is}\; \answer{\frac{450}{3 \, \pi + 4}}\]
\end{problem}}

%%%%%%%%%%%%%%%%%%%%%%

\latexProblemContent{
\ifVerboseLocation This is Derivative Compute Question 0046. \\ \fi
\begin{problem}

A Norman window has the shape of a rectangle surmounted by a semicircle (i.e. the diameter of the semicircle is equal to the width of the rectangle).  If the perimeter of the window is $7$ ft, what area will allow in the greatest amount of light?

\input{Derivative-Compute-0046.HELP.tex}

\[\mbox{The largest possible area is}\; \answer{\frac{49}{2 \, {\left(3 \, \pi + 4\right)}}}\]
\end{problem}}

%%%%%%%%%%%%%%%%%%%%%%

\latexProblemContent{
\ifVerboseLocation This is Derivative Compute Question 0046. \\ \fi
\begin{problem}

A Norman window has the shape of a rectangle surmounted by a semicircle (i.e. the diameter of the semicircle is equal to the width of the rectangle).  If the perimeter of the window is $9$ ft, what area will allow in the greatest amount of light?

\input{Derivative-Compute-0046.HELP.tex}

\[\mbox{The largest possible area is}\; \answer{\frac{81}{2 \, {\left(3 \, \pi + 4\right)}}}\]
\end{problem}}

%%%%%%%%%%%%%%%%%%%%%%

\latexProblemContent{
\ifVerboseLocation This is Derivative Compute Question 0046. \\ \fi
\begin{problem}

A Norman window has the shape of a rectangle surmounted by a semicircle (i.e. the diameter of the semicircle is equal to the width of the rectangle).  If the perimeter of the window is $24$ ft, what area will allow in the greatest amount of light?

\input{Derivative-Compute-0046.HELP.tex}

\[\mbox{The largest possible area is}\; \answer{\frac{288}{3 \, \pi + 4}}\]
\end{problem}}

%%%%%%%%%%%%%%%%%%%%%%

\latexProblemContent{
\ifVerboseLocation This is Derivative Compute Question 0046. \\ \fi
\begin{problem}

A Norman window has the shape of a rectangle surmounted by a semicircle (i.e. the diameter of the semicircle is equal to the width of the rectangle).  If the perimeter of the window is $25$ ft, what area will allow in the greatest amount of light?

\input{Derivative-Compute-0046.HELP.tex}

\[\mbox{The largest possible area is}\; \answer{\frac{625}{2 \, {\left(3 \, \pi + 4\right)}}}\]
\end{problem}}

%%%%%%%%%%%%%%%%%%%%%%

\latexProblemContent{
\ifVerboseLocation This is Derivative Compute Question 0046. \\ \fi
\begin{problem}

A Norman window has the shape of a rectangle surmounted by a semicircle (i.e. the diameter of the semicircle is equal to the width of the rectangle).  If the perimeter of the window is $14$ ft, what area will allow in the greatest amount of light?

\input{Derivative-Compute-0046.HELP.tex}

\[\mbox{The largest possible area is}\; \answer{\frac{98}{3 \, \pi + 4}}\]
\end{problem}}

%%%%%%%%%%%%%%%%%%%%%%

\latexProblemContent{
\ifVerboseLocation This is Derivative Compute Question 0046. \\ \fi
\begin{problem}

A Norman window has the shape of a rectangle surmounted by a semicircle (i.e. the diameter of the semicircle is equal to the width of the rectangle).  If the perimeter of the window is $10$ ft, what area will allow in the greatest amount of light?

\input{Derivative-Compute-0046.HELP.tex}

\[\mbox{The largest possible area is}\; \answer{\frac{50}{3 \, \pi + 4}}\]
\end{problem}}

%%%%%%%%%%%%%%%%%%%%%%

\latexProblemContent{
\ifVerboseLocation This is Derivative Compute Question 0046. \\ \fi
\begin{problem}

A Norman window has the shape of a rectangle surmounted by a semicircle (i.e. the diameter of the semicircle is equal to the width of the rectangle).  If the perimeter of the window is $36$ ft, what area will allow in the greatest amount of light?

\input{Derivative-Compute-0046.HELP.tex}

\[\mbox{The largest possible area is}\; \answer{\frac{648}{3 \, \pi + 4}}\]
\end{problem}}

%%%%%%%%%%%%%%%%%%%%%%

\latexProblemContent{
\ifVerboseLocation This is Derivative Compute Question 0046. \\ \fi
\begin{problem}

A Norman window has the shape of a rectangle surmounted by a semicircle (i.e. the diameter of the semicircle is equal to the width of the rectangle).  If the perimeter of the window is $1$ ft, what area will allow in the greatest amount of light?

\input{Derivative-Compute-0046.HELP.tex}

\[\mbox{The largest possible area is}\; \answer{\frac{1}{2 \, {\left(3 \, \pi + 4\right)}}}\]
\end{problem}}

%%%%%%%%%%%%%%%%%%%%%%

\latexProblemContent{
\ifVerboseLocation This is Derivative Compute Question 0046. \\ \fi
\begin{problem}

A Norman window has the shape of a rectangle surmounted by a semicircle (i.e. the diameter of the semicircle is equal to the width of the rectangle).  If the perimeter of the window is $40$ ft, what area will allow in the greatest amount of light?

\input{Derivative-Compute-0046.HELP.tex}

\[\mbox{The largest possible area is}\; \answer{\frac{800}{3 \, \pi + 4}}\]
\end{problem}}

%%%%%%%%%%%%%%%%%%%%%%

\latexProblemContent{
\ifVerboseLocation This is Derivative Compute Question 0046. \\ \fi
\begin{problem}

A Norman window has the shape of a rectangle surmounted by a semicircle (i.e. the diameter of the semicircle is equal to the width of the rectangle).  If the perimeter of the window is $42$ ft, what area will allow in the greatest amount of light?

\input{Derivative-Compute-0046.HELP.tex}

\[\mbox{The largest possible area is}\; \answer{\frac{882}{3 \, \pi + 4}}\]
\end{problem}}

%%%%%%%%%%%%%%%%%%%%%%

\latexProblemContent{
\ifVerboseLocation This is Derivative Compute Question 0046. \\ \fi
\begin{problem}

A Norman window has the shape of a rectangle surmounted by a semicircle (i.e. the diameter of the semicircle is equal to the width of the rectangle).  If the perimeter of the window is $4$ ft, what area will allow in the greatest amount of light?

\input{Derivative-Compute-0046.HELP.tex}

\[\mbox{The largest possible area is}\; \answer{\frac{8}{3 \, \pi + 4}}\]
\end{problem}}

%%%%%%%%%%%%%%%%%%%%%%

\latexProblemContent{
\ifVerboseLocation This is Derivative Compute Question 0046. \\ \fi
\begin{problem}

A Norman window has the shape of a rectangle surmounted by a semicircle (i.e. the diameter of the semicircle is equal to the width of the rectangle).  If the perimeter of the window is $49$ ft, what area will allow in the greatest amount of light?

\input{Derivative-Compute-0046.HELP.tex}

\[\mbox{The largest possible area is}\; \answer{\frac{2401}{2 \, {\left(3 \, \pi + 4\right)}}}\]
\end{problem}}

%%%%%%%%%%%%%%%%%%%%%%

\latexProblemContent{
\ifVerboseLocation This is Derivative Compute Question 0046. \\ \fi
\begin{problem}

A Norman window has the shape of a rectangle surmounted by a semicircle (i.e. the diameter of the semicircle is equal to the width of the rectangle).  If the perimeter of the window is $21$ ft, what area will allow in the greatest amount of light?

\input{Derivative-Compute-0046.HELP.tex}

\[\mbox{The largest possible area is}\; \answer{\frac{441}{2 \, {\left(3 \, \pi + 4\right)}}}\]
\end{problem}}

%%%%%%%%%%%%%%%%%%%%%%

\latexProblemContent{
\ifVerboseLocation This is Derivative Compute Question 0046. \\ \fi
\begin{problem}

A Norman window has the shape of a rectangle surmounted by a semicircle (i.e. the diameter of the semicircle is equal to the width of the rectangle).  If the perimeter of the window is $18$ ft, what area will allow in the greatest amount of light?

\input{Derivative-Compute-0046.HELP.tex}

\[\mbox{The largest possible area is}\; \answer{\frac{162}{3 \, \pi + 4}}\]
\end{problem}}

%%%%%%%%%%%%%%%%%%%%%%

\latexProblemContent{
\ifVerboseLocation This is Derivative Compute Question 0046. \\ \fi
\begin{problem}

A Norman window has the shape of a rectangle surmounted by a semicircle (i.e. the diameter of the semicircle is equal to the width of the rectangle).  If the perimeter of the window is $8$ ft, what area will allow in the greatest amount of light?

\input{Derivative-Compute-0046.HELP.tex}

\[\mbox{The largest possible area is}\; \answer{\frac{32}{3 \, \pi + 4}}\]
\end{problem}}

%%%%%%%%%%%%%%%%%%%%%%

\latexProblemContent{
\ifVerboseLocation This is Derivative Compute Question 0046. \\ \fi
\begin{problem}

A Norman window has the shape of a rectangle surmounted by a semicircle (i.e. the diameter of the semicircle is equal to the width of the rectangle).  If the perimeter of the window is $35$ ft, what area will allow in the greatest amount of light?

\input{Derivative-Compute-0046.HELP.tex}

\[\mbox{The largest possible area is}\; \answer{\frac{1225}{2 \, {\left(3 \, \pi + 4\right)}}}\]
\end{problem}}

%%%%%%%%%%%%%%%%%%%%%%

\latexProblemContent{
\ifVerboseLocation This is Derivative Compute Question 0046. \\ \fi
\begin{problem}

A Norman window has the shape of a rectangle surmounted by a semicircle (i.e. the diameter of the semicircle is equal to the width of the rectangle).  If the perimeter of the window is $48$ ft, what area will allow in the greatest amount of light?

\input{Derivative-Compute-0046.HELP.tex}

\[\mbox{The largest possible area is}\; \answer{\frac{1152}{3 \, \pi + 4}}\]
\end{problem}}

%%%%%%%%%%%%%%%%%%%%%%

\latexProblemContent{
\ifVerboseLocation This is Derivative Compute Question 0046. \\ \fi
\begin{problem}

A Norman window has the shape of a rectangle surmounted by a semicircle (i.e. the diameter of the semicircle is equal to the width of the rectangle).  If the perimeter of the window is $15$ ft, what area will allow in the greatest amount of light?

\input{Derivative-Compute-0046.HELP.tex}

\[\mbox{The largest possible area is}\; \answer{\frac{225}{2 \, {\left(3 \, \pi + 4\right)}}}\]
\end{problem}}

%%%%%%%%%%%%%%%%%%%%%%

\latexProblemContent{
\ifVerboseLocation This is Derivative Compute Question 0046. \\ \fi
\begin{problem}

A Norman window has the shape of a rectangle surmounted by a semicircle (i.e. the diameter of the semicircle is equal to the width of the rectangle).  If the perimeter of the window is $28$ ft, what area will allow in the greatest amount of light?

\input{Derivative-Compute-0046.HELP.tex}

\[\mbox{The largest possible area is}\; \answer{\frac{392}{3 \, \pi + 4}}\]
\end{problem}}

%%%%%%%%%%%%%%%%%%%%%%

\latexProblemContent{
\ifVerboseLocation This is Derivative Compute Question 0046. \\ \fi
\begin{problem}

A Norman window has the shape of a rectangle surmounted by a semicircle (i.e. the diameter of the semicircle is equal to the width of the rectangle).  If the perimeter of the window is $56$ ft, what area will allow in the greatest amount of light?

\input{Derivative-Compute-0046.HELP.tex}

\[\mbox{The largest possible area is}\; \answer{\frac{1568}{3 \, \pi + 4}}\]
\end{problem}}

%%%%%%%%%%%%%%%%%%%%%%

\latexProblemContent{
\ifVerboseLocation This is Derivative Compute Question 0046. \\ \fi
\begin{problem}

A Norman window has the shape of a rectangle surmounted by a semicircle (i.e. the diameter of the semicircle is equal to the width of the rectangle).  If the perimeter of the window is $3$ ft, what area will allow in the greatest amount of light?

\input{Derivative-Compute-0046.HELP.tex}

\[\mbox{The largest possible area is}\; \answer{\frac{9}{2 \, {\left(3 \, \pi + 4\right)}}}\]
\end{problem}}

%%%%%%%%%%%%%%%%%%%%%%

\latexProblemContent{
\ifVerboseLocation This is Derivative Compute Question 0046. \\ \fi
\begin{problem}

A Norman window has the shape of a rectangle surmounted by a semicircle (i.e. the diameter of the semicircle is equal to the width of the rectangle).  If the perimeter of the window is $20$ ft, what area will allow in the greatest amount of light?

\input{Derivative-Compute-0046.HELP.tex}

\[\mbox{The largest possible area is}\; \answer{\frac{200}{3 \, \pi + 4}}\]
\end{problem}}

%%%%%%%%%%%%%%%%%%%%%%

\latexProblemContent{
\ifVerboseLocation This is Derivative Compute Question 0046. \\ \fi
\begin{problem}

A Norman window has the shape of a rectangle surmounted by a semicircle (i.e. the diameter of the semicircle is equal to the width of the rectangle).  If the perimeter of the window is $5$ ft, what area will allow in the greatest amount of light?

\input{Derivative-Compute-0046.HELP.tex}

\[\mbox{The largest possible area is}\; \answer{\frac{25}{2 \, {\left(3 \, \pi + 4\right)}}}\]
\end{problem}}

%%%%%%%%%%%%%%%%%%%%%%

\latexProblemContent{
\ifVerboseLocation This is Derivative Compute Question 0046. \\ \fi
\begin{problem}

A Norman window has the shape of a rectangle surmounted by a semicircle (i.e. the diameter of the semicircle is equal to the width of the rectangle).  If the perimeter of the window is $6$ ft, what area will allow in the greatest amount of light?

\input{Derivative-Compute-0046.HELP.tex}

\[\mbox{The largest possible area is}\; \answer{\frac{18}{3 \, \pi + 4}}\]
\end{problem}}

%%%%%%%%%%%%%%%%%%%%%%

\latexProblemContent{
\ifVerboseLocation This is Derivative Compute Question 0046. \\ \fi
\begin{problem}

A Norman window has the shape of a rectangle surmounted by a semicircle (i.e. the diameter of the semicircle is equal to the width of the rectangle).  If the perimeter of the window is $16$ ft, what area will allow in the greatest amount of light?

\input{Derivative-Compute-0046.HELP.tex}

\[\mbox{The largest possible area is}\; \answer{\frac{128}{3 \, \pi + 4}}\]
\end{problem}}

%%%%%%%%%%%%%%%%%%%%%%

\latexProblemContent{
\ifVerboseLocation This is Derivative Compute Question 0046. \\ \fi
\begin{problem}

A Norman window has the shape of a rectangle surmounted by a semicircle (i.e. the diameter of the semicircle is equal to the width of the rectangle).  If the perimeter of the window is $32$ ft, what area will allow in the greatest amount of light?

\input{Derivative-Compute-0046.HELP.tex}

\[\mbox{The largest possible area is}\; \answer{\frac{512}{3 \, \pi + 4}}\]
\end{problem}}\fi             %% end of \ifproblemToFind near top of file
\fi             %% end of \ifquestionCount near top of file
\ProblemFileFooter