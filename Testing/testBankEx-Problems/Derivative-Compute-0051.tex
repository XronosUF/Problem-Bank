% Ans        : ShortAns
% File       : 0051
% Sub        : Exp, LHopital
% Topic      : Derivative
% Type       : Compute

\ProblemFileHeader{500}
\ifquestionPull
\ifproblemToFind\latexProblemContent{
\ifVerboseLocation This is Derivative Compute Question 0051. \\ \fi
\begin{problem}

Find the limit.  Use L'H$\hat{o}$pital's rule where appropriate.

\input{Derivative-Compute-0051.HELP.tex}

\[\lim\limits_{x\to\infty} 8 \, {\left(x - 1\right)}^{2} e^{\left(-x - 1\right)}=\answer{0}\]
\end{problem}}

%%%%%%%%%%%%%%%%%%%%%%

\latexProblemContent{
\ifVerboseLocation This is Derivative Compute Question 0051. \\ \fi
\begin{problem}

Find the limit.  Use L'H$\hat{o}$pital's rule where appropriate.

\input{Derivative-Compute-0051.HELP.tex}

\[\lim\limits_{x\to\infty} 7 \, {\left(x + 6\right)}^{3} e^{\left(-x + 1\right)}=\answer{0}\]
\end{problem}}

%%%%%%%%%%%%%%%%%%%%%%

\latexProblemContent{
\ifVerboseLocation This is Derivative Compute Question 0051. \\ \fi
\begin{problem}

Find the limit.  Use L'H$\hat{o}$pital's rule where appropriate.

\input{Derivative-Compute-0051.HELP.tex}

\[\lim\limits_{x\to\infty} 8 \, {\left(x + 3\right)} e^{\left(-x - 5\right)}=\answer{0}\]
\end{problem}}

%%%%%%%%%%%%%%%%%%%%%%

\latexProblemContent{
\ifVerboseLocation This is Derivative Compute Question 0051. \\ \fi
\begin{problem}

Find the limit.  Use L'H$\hat{o}$pital's rule where appropriate.

\input{Derivative-Compute-0051.HELP.tex}

\[\lim\limits_{x\to\infty} 7 \, {\left(x - 8\right)}^{2} e^{\left(-x - 3\right)}=\answer{0}\]
\end{problem}}

%%%%%%%%%%%%%%%%%%%%%%

\latexProblemContent{
\ifVerboseLocation This is Derivative Compute Question 0051. \\ \fi
\begin{problem}

Find the limit.  Use L'H$\hat{o}$pital's rule where appropriate.

\input{Derivative-Compute-0051.HELP.tex}

\[\lim\limits_{x\to\infty} 8 \, {\left(x + 8\right)} e^{\left(-x + 5\right)}=\answer{0}\]
\end{problem}}

%%%%%%%%%%%%%%%%%%%%%%

\latexProblemContent{
\ifVerboseLocation This is Derivative Compute Question 0051. \\ \fi
\begin{problem}

Find the limit.  Use L'H$\hat{o}$pital's rule where appropriate.

\input{Derivative-Compute-0051.HELP.tex}

\[\lim\limits_{x\to\infty} -2 \, {\left(x + 8\right)} e^{\left(-x + 5\right)}=\answer{0}\]
\end{problem}}

%%%%%%%%%%%%%%%%%%%%%%

\latexProblemContent{
\ifVerboseLocation This is Derivative Compute Question 0051. \\ \fi
\begin{problem}

Find the limit.  Use L'H$\hat{o}$pital's rule where appropriate.

\input{Derivative-Compute-0051.HELP.tex}

\[\lim\limits_{x\to\infty} -{\left(x - 7\right)}^{3} e^{\left(-x - 1\right)}=\answer{0}\]
\end{problem}}

%%%%%%%%%%%%%%%%%%%%%%

\latexProblemContent{
\ifVerboseLocation This is Derivative Compute Question 0051. \\ \fi
\begin{problem}

Find the limit.  Use L'H$\hat{o}$pital's rule where appropriate.

\input{Derivative-Compute-0051.HELP.tex}

\[\lim\limits_{x\to\infty} -4 \, {\left(x + 3\right)}^{3} e^{\left(-x - 6\right)}=\answer{0}\]
\end{problem}}

%%%%%%%%%%%%%%%%%%%%%%

\latexProblemContent{
\ifVerboseLocation This is Derivative Compute Question 0051. \\ \fi
\begin{problem}

Find the limit.  Use L'H$\hat{o}$pital's rule where appropriate.

\input{Derivative-Compute-0051.HELP.tex}

\[\lim\limits_{x\to\infty} 3 \, {\left(x - 3\right)} e^{\left(-x - 1\right)}=\answer{0}\]
\end{problem}}

%%%%%%%%%%%%%%%%%%%%%%

\latexProblemContent{
\ifVerboseLocation This is Derivative Compute Question 0051. \\ \fi
\begin{problem}

Find the limit.  Use L'H$\hat{o}$pital's rule where appropriate.

\input{Derivative-Compute-0051.HELP.tex}

\[\lim\limits_{x\to\infty} -{\left(x + 2\right)} e^{\left(-x - 8\right)}=\answer{0}\]
\end{problem}}

%%%%%%%%%%%%%%%%%%%%%%

\latexProblemContent{
\ifVerboseLocation This is Derivative Compute Question 0051. \\ \fi
\begin{problem}

Find the limit.  Use L'H$\hat{o}$pital's rule where appropriate.

\input{Derivative-Compute-0051.HELP.tex}

\[\lim\limits_{x\to\infty} 5 \, {\left(x - 4\right)} e^{\left(-x + 7\right)}=\answer{0}\]
\end{problem}}

%%%%%%%%%%%%%%%%%%%%%%

\latexProblemContent{
\ifVerboseLocation This is Derivative Compute Question 0051. \\ \fi
\begin{problem}

Find the limit.  Use L'H$\hat{o}$pital's rule where appropriate.

\input{Derivative-Compute-0051.HELP.tex}

\[\lim\limits_{x\to\infty} 5 \, {\left(x + 2\right)} e^{\left(-x - 3\right)}=\answer{0}\]
\end{problem}}

%%%%%%%%%%%%%%%%%%%%%%

\latexProblemContent{
\ifVerboseLocation This is Derivative Compute Question 0051. \\ \fi
\begin{problem}

Find the limit.  Use L'H$\hat{o}$pital's rule where appropriate.

\input{Derivative-Compute-0051.HELP.tex}

\[\lim\limits_{x\to\infty} -4 \, {\left(x + 8\right)}^{3} e^{\left(-x + 7\right)}=\answer{0}\]
\end{problem}}

%%%%%%%%%%%%%%%%%%%%%%

\latexProblemContent{
\ifVerboseLocation This is Derivative Compute Question 0051. \\ \fi
\begin{problem}

Find the limit.  Use L'H$\hat{o}$pital's rule where appropriate.

\input{Derivative-Compute-0051.HELP.tex}

\[\lim\limits_{x\to\infty} -7 \, {\left(x + 4\right)}^{2} e^{\left(-x - 2\right)}=\answer{0}\]
\end{problem}}

%%%%%%%%%%%%%%%%%%%%%%

\latexProblemContent{
\ifVerboseLocation This is Derivative Compute Question 0051. \\ \fi
\begin{problem}

Find the limit.  Use L'H$\hat{o}$pital's rule where appropriate.

\input{Derivative-Compute-0051.HELP.tex}

\[\lim\limits_{x\to\infty} -5 \, {\left(x + 1\right)} e^{\left(-x + 2\right)}=\answer{0}\]
\end{problem}}

%%%%%%%%%%%%%%%%%%%%%%

\latexProblemContent{
\ifVerboseLocation This is Derivative Compute Question 0051. \\ \fi
\begin{problem}

Find the limit.  Use L'H$\hat{o}$pital's rule where appropriate.

\input{Derivative-Compute-0051.HELP.tex}

\[\lim\limits_{x\to\infty} -4 \, {\left(x - 8\right)}^{2} e^{\left(-x + 7\right)}=\answer{0}\]
\end{problem}}

%%%%%%%%%%%%%%%%%%%%%%

\latexProblemContent{
\ifVerboseLocation This is Derivative Compute Question 0051. \\ \fi
\begin{problem}

Find the limit.  Use L'H$\hat{o}$pital's rule where appropriate.

\input{Derivative-Compute-0051.HELP.tex}

\[\lim\limits_{x\to\infty} -8 \, {\left(x - 7\right)}^{3} e^{\left(-x - 1\right)}=\answer{0}\]
\end{problem}}

%%%%%%%%%%%%%%%%%%%%%%

\latexProblemContent{
\ifVerboseLocation This is Derivative Compute Question 0051. \\ \fi
\begin{problem}

Find the limit.  Use L'H$\hat{o}$pital's rule where appropriate.

\input{Derivative-Compute-0051.HELP.tex}

\[\lim\limits_{x\to\infty} -6 \, {\left(x - 6\right)} e^{\left(-x - 8\right)}=\answer{0}\]
\end{problem}}

%%%%%%%%%%%%%%%%%%%%%%

\latexProblemContent{
\ifVerboseLocation This is Derivative Compute Question 0051. \\ \fi
\begin{problem}

Find the limit.  Use L'H$\hat{o}$pital's rule where appropriate.

\input{Derivative-Compute-0051.HELP.tex}

\[\lim\limits_{x\to\infty} -{\left(x + 2\right)}^{2} e^{\left(-x - 4\right)}=\answer{0}\]
\end{problem}}

%%%%%%%%%%%%%%%%%%%%%%

\latexProblemContent{
\ifVerboseLocation This is Derivative Compute Question 0051. \\ \fi
\begin{problem}

Find the limit.  Use L'H$\hat{o}$pital's rule where appropriate.

\input{Derivative-Compute-0051.HELP.tex}

\[\lim\limits_{x\to\infty} 5 \, {\left(x + 6\right)}^{2} e^{\left(-x - 4\right)}=\answer{0}\]
\end{problem}}

%%%%%%%%%%%%%%%%%%%%%%

\latexProblemContent{
\ifVerboseLocation This is Derivative Compute Question 0051. \\ \fi
\begin{problem}

Find the limit.  Use L'H$\hat{o}$pital's rule where appropriate.

\input{Derivative-Compute-0051.HELP.tex}

\[\lim\limits_{x\to\infty} -6 \, {\left(x - 2\right)} e^{\left(-x - 6\right)}=\answer{0}\]
\end{problem}}

%%%%%%%%%%%%%%%%%%%%%%

\latexProblemContent{
\ifVerboseLocation This is Derivative Compute Question 0051. \\ \fi
\begin{problem}

Find the limit.  Use L'H$\hat{o}$pital's rule where appropriate.

\input{Derivative-Compute-0051.HELP.tex}

\[\lim\limits_{x\to\infty} 8 \, {\left(x - 7\right)} e^{\left(-x - 3\right)}=\answer{0}\]
\end{problem}}

%%%%%%%%%%%%%%%%%%%%%%

\latexProblemContent{
\ifVerboseLocation This is Derivative Compute Question 0051. \\ \fi
\begin{problem}

Find the limit.  Use L'H$\hat{o}$pital's rule where appropriate.

\input{Derivative-Compute-0051.HELP.tex}

\[\lim\limits_{x\to\infty} 6 \, {\left(x - 6\right)} e^{\left(-x + 6\right)}=\answer{0}\]
\end{problem}}

%%%%%%%%%%%%%%%%%%%%%%

\latexProblemContent{
\ifVerboseLocation This is Derivative Compute Question 0051. \\ \fi
\begin{problem}

Find the limit.  Use L'H$\hat{o}$pital's rule where appropriate.

\input{Derivative-Compute-0051.HELP.tex}

\[\lim\limits_{x\to\infty} -2 \, {\left(x + 1\right)}^{3} e^{\left(-x + 4\right)}=\answer{0}\]
\end{problem}}

%%%%%%%%%%%%%%%%%%%%%%

\latexProblemContent{
\ifVerboseLocation This is Derivative Compute Question 0051. \\ \fi
\begin{problem}

Find the limit.  Use L'H$\hat{o}$pital's rule where appropriate.

\input{Derivative-Compute-0051.HELP.tex}

\[\lim\limits_{x\to\infty} 2 \, {\left(x - 3\right)}^{2} e^{\left(-x + 2\right)}=\answer{0}\]
\end{problem}}

%%%%%%%%%%%%%%%%%%%%%%

\latexProblemContent{
\ifVerboseLocation This is Derivative Compute Question 0051. \\ \fi
\begin{problem}

Find the limit.  Use L'H$\hat{o}$pital's rule where appropriate.

\input{Derivative-Compute-0051.HELP.tex}

\[\lim\limits_{x\to\infty} -3 \, {\left(x + 4\right)}^{3} e^{\left(-x - 3\right)}=\answer{0}\]
\end{problem}}

%%%%%%%%%%%%%%%%%%%%%%

\latexProblemContent{
\ifVerboseLocation This is Derivative Compute Question 0051. \\ \fi
\begin{problem}

Find the limit.  Use L'H$\hat{o}$pital's rule where appropriate.

\input{Derivative-Compute-0051.HELP.tex}

\[\lim\limits_{x\to\infty} 5 \, {\left(x + 4\right)} e^{\left(-x - 6\right)}=\answer{0}\]
\end{problem}}

%%%%%%%%%%%%%%%%%%%%%%

\latexProblemContent{
\ifVerboseLocation This is Derivative Compute Question 0051. \\ \fi
\begin{problem}

Find the limit.  Use L'H$\hat{o}$pital's rule where appropriate.

\input{Derivative-Compute-0051.HELP.tex}

\[\lim\limits_{x\to\infty} {\left(x + 8\right)}^{2} e^{\left(-x + 7\right)}=\answer{0}\]
\end{problem}}

%%%%%%%%%%%%%%%%%%%%%%

\latexProblemContent{
\ifVerboseLocation This is Derivative Compute Question 0051. \\ \fi
\begin{problem}

Find the limit.  Use L'H$\hat{o}$pital's rule where appropriate.

\input{Derivative-Compute-0051.HELP.tex}

\[\lim\limits_{x\to\infty} 5 \, {\left(x + 4\right)} e^{\left(-x - 5\right)}=\answer{0}\]
\end{problem}}

%%%%%%%%%%%%%%%%%%%%%%

\latexProblemContent{
\ifVerboseLocation This is Derivative Compute Question 0051. \\ \fi
\begin{problem}

Find the limit.  Use L'H$\hat{o}$pital's rule where appropriate.

\input{Derivative-Compute-0051.HELP.tex}

\[\lim\limits_{x\to\infty} 5 \, {\left(x - 2\right)} e^{\left(-x + 5\right)}=\answer{0}\]
\end{problem}}

%%%%%%%%%%%%%%%%%%%%%%

\latexProblemContent{
\ifVerboseLocation This is Derivative Compute Question 0051. \\ \fi
\begin{problem}

Find the limit.  Use L'H$\hat{o}$pital's rule where appropriate.

\input{Derivative-Compute-0051.HELP.tex}

\[\lim\limits_{x\to\infty} -5 \, {\left(x + 7\right)}^{3} e^{\left(-x - 4\right)}=\answer{0}\]
\end{problem}}

%%%%%%%%%%%%%%%%%%%%%%

\latexProblemContent{
\ifVerboseLocation This is Derivative Compute Question 0051. \\ \fi
\begin{problem}

Find the limit.  Use L'H$\hat{o}$pital's rule where appropriate.

\input{Derivative-Compute-0051.HELP.tex}

\[\lim\limits_{x\to\infty} 2 \, {\left(x - 8\right)}^{3} e^{\left(-x + 7\right)}=\answer{0}\]
\end{problem}}

%%%%%%%%%%%%%%%%%%%%%%

\latexProblemContent{
\ifVerboseLocation This is Derivative Compute Question 0051. \\ \fi
\begin{problem}

Find the limit.  Use L'H$\hat{o}$pital's rule where appropriate.

\input{Derivative-Compute-0051.HELP.tex}

\[\lim\limits_{x\to\infty} 7 \, {\left(x - 6\right)}^{3} e^{\left(-x - 2\right)}=\answer{0}\]
\end{problem}}

%%%%%%%%%%%%%%%%%%%%%%

\latexProblemContent{
\ifVerboseLocation This is Derivative Compute Question 0051. \\ \fi
\begin{problem}

Find the limit.  Use L'H$\hat{o}$pital's rule where appropriate.

\input{Derivative-Compute-0051.HELP.tex}

\[\lim\limits_{x\to\infty} -8 \, {\left(x + 2\right)}^{3} e^{\left(-x + 6\right)}=\answer{0}\]
\end{problem}}

%%%%%%%%%%%%%%%%%%%%%%

\latexProblemContent{
\ifVerboseLocation This is Derivative Compute Question 0051. \\ \fi
\begin{problem}

Find the limit.  Use L'H$\hat{o}$pital's rule where appropriate.

\input{Derivative-Compute-0051.HELP.tex}

\[\lim\limits_{x\to\infty} -{\left(x + 8\right)} e^{\left(-x - 6\right)}=\answer{0}\]
\end{problem}}

%%%%%%%%%%%%%%%%%%%%%%

\latexProblemContent{
\ifVerboseLocation This is Derivative Compute Question 0051. \\ \fi
\begin{problem}

Find the limit.  Use L'H$\hat{o}$pital's rule where appropriate.

\input{Derivative-Compute-0051.HELP.tex}

\[\lim\limits_{x\to\infty} 6 \, {\left(x + 8\right)}^{2} e^{\left(-x - 4\right)}=\answer{0}\]
\end{problem}}

%%%%%%%%%%%%%%%%%%%%%%

\latexProblemContent{
\ifVerboseLocation This is Derivative Compute Question 0051. \\ \fi
\begin{problem}

Find the limit.  Use L'H$\hat{o}$pital's rule where appropriate.

\input{Derivative-Compute-0051.HELP.tex}

\[\lim\limits_{x\to\infty} -3 \, {\left(x + 4\right)} e^{\left(-x + 2\right)}=\answer{0}\]
\end{problem}}

%%%%%%%%%%%%%%%%%%%%%%

\latexProblemContent{
\ifVerboseLocation This is Derivative Compute Question 0051. \\ \fi
\begin{problem}

Find the limit.  Use L'H$\hat{o}$pital's rule where appropriate.

\input{Derivative-Compute-0051.HELP.tex}

\[\lim\limits_{x\to\infty} -5 \, {\left(x - 8\right)}^{3} e^{\left(-x - 2\right)}=\answer{0}\]
\end{problem}}

%%%%%%%%%%%%%%%%%%%%%%

\latexProblemContent{
\ifVerboseLocation This is Derivative Compute Question 0051. \\ \fi
\begin{problem}

Find the limit.  Use L'H$\hat{o}$pital's rule where appropriate.

\input{Derivative-Compute-0051.HELP.tex}

\[\lim\limits_{x\to\infty} 3 \, {\left(x - 8\right)}^{3} e^{\left(-x + 3\right)}=\answer{0}\]
\end{problem}}

%%%%%%%%%%%%%%%%%%%%%%

\latexProblemContent{
\ifVerboseLocation This is Derivative Compute Question 0051. \\ \fi
\begin{problem}

Find the limit.  Use L'H$\hat{o}$pital's rule where appropriate.

\input{Derivative-Compute-0051.HELP.tex}

\[\lim\limits_{x\to\infty} -{\left(x - 2\right)}^{3} e^{\left(-x - 7\right)}=\answer{0}\]
\end{problem}}

%%%%%%%%%%%%%%%%%%%%%%

\latexProblemContent{
\ifVerboseLocation This is Derivative Compute Question 0051. \\ \fi
\begin{problem}

Find the limit.  Use L'H$\hat{o}$pital's rule where appropriate.

\input{Derivative-Compute-0051.HELP.tex}

\[\lim\limits_{x\to\infty} 6 \, {\left(x - 4\right)}^{2} e^{\left(-x - 8\right)}=\answer{0}\]
\end{problem}}

%%%%%%%%%%%%%%%%%%%%%%

\latexProblemContent{
\ifVerboseLocation This is Derivative Compute Question 0051. \\ \fi
\begin{problem}

Find the limit.  Use L'H$\hat{o}$pital's rule where appropriate.

\input{Derivative-Compute-0051.HELP.tex}

\[\lim\limits_{x\to\infty} 7 \, {\left(x - 7\right)}^{3} e^{\left(-x - 1\right)}=\answer{0}\]
\end{problem}}

%%%%%%%%%%%%%%%%%%%%%%

\latexProblemContent{
\ifVerboseLocation This is Derivative Compute Question 0051. \\ \fi
\begin{problem}

Find the limit.  Use L'H$\hat{o}$pital's rule where appropriate.

\input{Derivative-Compute-0051.HELP.tex}

\[\lim\limits_{x\to\infty} 4 \, {\left(x + 3\right)}^{3} e^{\left(-x + 3\right)}=\answer{0}\]
\end{problem}}

%%%%%%%%%%%%%%%%%%%%%%

\latexProblemContent{
\ifVerboseLocation This is Derivative Compute Question 0051. \\ \fi
\begin{problem}

Find the limit.  Use L'H$\hat{o}$pital's rule where appropriate.

\input{Derivative-Compute-0051.HELP.tex}

\[\lim\limits_{x\to\infty} -6 \, {\left(x - 5\right)}^{3} e^{\left(-x - 8\right)}=\answer{0}\]
\end{problem}}

%%%%%%%%%%%%%%%%%%%%%%

\latexProblemContent{
\ifVerboseLocation This is Derivative Compute Question 0051. \\ \fi
\begin{problem}

Find the limit.  Use L'H$\hat{o}$pital's rule where appropriate.

\input{Derivative-Compute-0051.HELP.tex}

\[\lim\limits_{x\to\infty} -5 \, {\left(x + 6\right)}^{3} e^{\left(-x + 3\right)}=\answer{0}\]
\end{problem}}

%%%%%%%%%%%%%%%%%%%%%%

\latexProblemContent{
\ifVerboseLocation This is Derivative Compute Question 0051. \\ \fi
\begin{problem}

Find the limit.  Use L'H$\hat{o}$pital's rule where appropriate.

\input{Derivative-Compute-0051.HELP.tex}

\[\lim\limits_{x\to\infty} -{\left(x + 6\right)}^{3} e^{\left(-x + 1\right)}=\answer{0}\]
\end{problem}}

%%%%%%%%%%%%%%%%%%%%%%

\latexProblemContent{
\ifVerboseLocation This is Derivative Compute Question 0051. \\ \fi
\begin{problem}

Find the limit.  Use L'H$\hat{o}$pital's rule where appropriate.

\input{Derivative-Compute-0051.HELP.tex}

\[\lim\limits_{x\to\infty} -7 \, {\left(x - 8\right)}^{3} e^{\left(-x + 1\right)}=\answer{0}\]
\end{problem}}

%%%%%%%%%%%%%%%%%%%%%%

\latexProblemContent{
\ifVerboseLocation This is Derivative Compute Question 0051. \\ \fi
\begin{problem}

Find the limit.  Use L'H$\hat{o}$pital's rule where appropriate.

\input{Derivative-Compute-0051.HELP.tex}

\[\lim\limits_{x\to\infty} -6 \, {\left(x + 3\right)}^{3} e^{\left(-x + 8\right)}=\answer{0}\]
\end{problem}}

%%%%%%%%%%%%%%%%%%%%%%

\latexProblemContent{
\ifVerboseLocation This is Derivative Compute Question 0051. \\ \fi
\begin{problem}

Find the limit.  Use L'H$\hat{o}$pital's rule where appropriate.

\input{Derivative-Compute-0051.HELP.tex}

\[\lim\limits_{x\to\infty} 7 \, {\left(x + 1\right)}^{2} e^{\left(-x - 8\right)}=\answer{0}\]
\end{problem}}

%%%%%%%%%%%%%%%%%%%%%%

\latexProblemContent{
\ifVerboseLocation This is Derivative Compute Question 0051. \\ \fi
\begin{problem}

Find the limit.  Use L'H$\hat{o}$pital's rule where appropriate.

\input{Derivative-Compute-0051.HELP.tex}

\[\lim\limits_{x\to\infty} 7 \, {\left(x - 2\right)}^{3} e^{\left(-x + 2\right)}=\answer{0}\]
\end{problem}}

%%%%%%%%%%%%%%%%%%%%%%

\latexProblemContent{
\ifVerboseLocation This is Derivative Compute Question 0051. \\ \fi
\begin{problem}

Find the limit.  Use L'H$\hat{o}$pital's rule where appropriate.

\input{Derivative-Compute-0051.HELP.tex}

\[\lim\limits_{x\to\infty} 2 \, {\left(x + 1\right)} e^{\left(-x - 2\right)}=\answer{0}\]
\end{problem}}

%%%%%%%%%%%%%%%%%%%%%%

\latexProblemContent{
\ifVerboseLocation This is Derivative Compute Question 0051. \\ \fi
\begin{problem}

Find the limit.  Use L'H$\hat{o}$pital's rule where appropriate.

\input{Derivative-Compute-0051.HELP.tex}

\[\lim\limits_{x\to\infty} 6 \, {\left(x - 2\right)}^{2} e^{\left(-x - 3\right)}=\answer{0}\]
\end{problem}}

%%%%%%%%%%%%%%%%%%%%%%

\latexProblemContent{
\ifVerboseLocation This is Derivative Compute Question 0051. \\ \fi
\begin{problem}

Find the limit.  Use L'H$\hat{o}$pital's rule where appropriate.

\input{Derivative-Compute-0051.HELP.tex}

\[\lim\limits_{x\to\infty} 7 \, {\left(x - 1\right)} e^{\left(-x - 8\right)}=\answer{0}\]
\end{problem}}

%%%%%%%%%%%%%%%%%%%%%%

\latexProblemContent{
\ifVerboseLocation This is Derivative Compute Question 0051. \\ \fi
\begin{problem}

Find the limit.  Use L'H$\hat{o}$pital's rule where appropriate.

\input{Derivative-Compute-0051.HELP.tex}

\[\lim\limits_{x\to\infty} 5 \, {\left(x + 6\right)}^{3} e^{\left(-x + 3\right)}=\answer{0}\]
\end{problem}}

%%%%%%%%%%%%%%%%%%%%%%

\latexProblemContent{
\ifVerboseLocation This is Derivative Compute Question 0051. \\ \fi
\begin{problem}

Find the limit.  Use L'H$\hat{o}$pital's rule where appropriate.

\input{Derivative-Compute-0051.HELP.tex}

\[\lim\limits_{x\to\infty} -5 \, {\left(x + 8\right)}^{3} e^{\left(-x + 5\right)}=\answer{0}\]
\end{problem}}

%%%%%%%%%%%%%%%%%%%%%%

\latexProblemContent{
\ifVerboseLocation This is Derivative Compute Question 0051. \\ \fi
\begin{problem}

Find the limit.  Use L'H$\hat{o}$pital's rule where appropriate.

\input{Derivative-Compute-0051.HELP.tex}

\[\lim\limits_{x\to\infty} -2 \, {\left(x + 3\right)} e^{\left(-x + 3\right)}=\answer{0}\]
\end{problem}}

%%%%%%%%%%%%%%%%%%%%%%

\latexProblemContent{
\ifVerboseLocation This is Derivative Compute Question 0051. \\ \fi
\begin{problem}

Find the limit.  Use L'H$\hat{o}$pital's rule where appropriate.

\input{Derivative-Compute-0051.HELP.tex}

\[\lim\limits_{x\to\infty} -4 \, {\left(x + 4\right)}^{2} e^{\left(-x + 4\right)}=\answer{0}\]
\end{problem}}

%%%%%%%%%%%%%%%%%%%%%%

\latexProblemContent{
\ifVerboseLocation This is Derivative Compute Question 0051. \\ \fi
\begin{problem}

Find the limit.  Use L'H$\hat{o}$pital's rule where appropriate.

\input{Derivative-Compute-0051.HELP.tex}

\[\lim\limits_{x\to\infty} 5 \, {\left(x - 3\right)}^{3} e^{\left(-x - 4\right)}=\answer{0}\]
\end{problem}}

%%%%%%%%%%%%%%%%%%%%%%

\latexProblemContent{
\ifVerboseLocation This is Derivative Compute Question 0051. \\ \fi
\begin{problem}

Find the limit.  Use L'H$\hat{o}$pital's rule where appropriate.

\input{Derivative-Compute-0051.HELP.tex}

\[\lim\limits_{x\to\infty} -5 \, {\left(x - 2\right)}^{2} e^{\left(-x - 3\right)}=\answer{0}\]
\end{problem}}

%%%%%%%%%%%%%%%%%%%%%%

\latexProblemContent{
\ifVerboseLocation This is Derivative Compute Question 0051. \\ \fi
\begin{problem}

Find the limit.  Use L'H$\hat{o}$pital's rule where appropriate.

\input{Derivative-Compute-0051.HELP.tex}

\[\lim\limits_{x\to\infty} 5 \, {\left(x - 2\right)}^{2} e^{\left(-x - 5\right)}=\answer{0}\]
\end{problem}}

%%%%%%%%%%%%%%%%%%%%%%

\latexProblemContent{
\ifVerboseLocation This is Derivative Compute Question 0051. \\ \fi
\begin{problem}

Find the limit.  Use L'H$\hat{o}$pital's rule where appropriate.

\input{Derivative-Compute-0051.HELP.tex}

\[\lim\limits_{x\to\infty} 6 \, {\left(x + 8\right)}^{2} e^{\left(-x - 2\right)}=\answer{0}\]
\end{problem}}

%%%%%%%%%%%%%%%%%%%%%%

\latexProblemContent{
\ifVerboseLocation This is Derivative Compute Question 0051. \\ \fi
\begin{problem}

Find the limit.  Use L'H$\hat{o}$pital's rule where appropriate.

\input{Derivative-Compute-0051.HELP.tex}

\[\lim\limits_{x\to\infty} -3 \, {\left(x + 5\right)}^{2} e^{\left(-x - 5\right)}=\answer{0}\]
\end{problem}}

%%%%%%%%%%%%%%%%%%%%%%

\latexProblemContent{
\ifVerboseLocation This is Derivative Compute Question 0051. \\ \fi
\begin{problem}

Find the limit.  Use L'H$\hat{o}$pital's rule where appropriate.

\input{Derivative-Compute-0051.HELP.tex}

\[\lim\limits_{x\to\infty} 3 \, {\left(x - 4\right)} e^{\left(-x - 1\right)}=\answer{0}\]
\end{problem}}

%%%%%%%%%%%%%%%%%%%%%%

\latexProblemContent{
\ifVerboseLocation This is Derivative Compute Question 0051. \\ \fi
\begin{problem}

Find the limit.  Use L'H$\hat{o}$pital's rule where appropriate.

\input{Derivative-Compute-0051.HELP.tex}

\[\lim\limits_{x\to\infty} 6 \, {\left(x - 4\right)}^{2} e^{\left(-x - 3\right)}=\answer{0}\]
\end{problem}}

%%%%%%%%%%%%%%%%%%%%%%

\latexProblemContent{
\ifVerboseLocation This is Derivative Compute Question 0051. \\ \fi
\begin{problem}

Find the limit.  Use L'H$\hat{o}$pital's rule where appropriate.

\input{Derivative-Compute-0051.HELP.tex}

\[\lim\limits_{x\to\infty} 7 \, {\left(x + 6\right)}^{3} e^{\left(-x - 6\right)}=\answer{0}\]
\end{problem}}

%%%%%%%%%%%%%%%%%%%%%%

\latexProblemContent{
\ifVerboseLocation This is Derivative Compute Question 0051. \\ \fi
\begin{problem}

Find the limit.  Use L'H$\hat{o}$pital's rule where appropriate.

\input{Derivative-Compute-0051.HELP.tex}

\[\lim\limits_{x\to\infty} -8 \, {\left(x - 5\right)} e^{\left(-x - 2\right)}=\answer{0}\]
\end{problem}}

%%%%%%%%%%%%%%%%%%%%%%

\latexProblemContent{
\ifVerboseLocation This is Derivative Compute Question 0051. \\ \fi
\begin{problem}

Find the limit.  Use L'H$\hat{o}$pital's rule where appropriate.

\input{Derivative-Compute-0051.HELP.tex}

\[\lim\limits_{x\to\infty} 4 \, {\left(x + 2\right)}^{3} e^{\left(-x + 3\right)}=\answer{0}\]
\end{problem}}

%%%%%%%%%%%%%%%%%%%%%%

\latexProblemContent{
\ifVerboseLocation This is Derivative Compute Question 0051. \\ \fi
\begin{problem}

Find the limit.  Use L'H$\hat{o}$pital's rule where appropriate.

\input{Derivative-Compute-0051.HELP.tex}

\[\lim\limits_{x\to\infty} 7 \, {\left(x + 5\right)}^{2} e^{\left(-x + 1\right)}=\answer{0}\]
\end{problem}}

%%%%%%%%%%%%%%%%%%%%%%

\latexProblemContent{
\ifVerboseLocation This is Derivative Compute Question 0051. \\ \fi
\begin{problem}

Find the limit.  Use L'H$\hat{o}$pital's rule where appropriate.

\input{Derivative-Compute-0051.HELP.tex}

\[\lim\limits_{x\to\infty} -3 \, {\left(x + 6\right)}^{3} e^{\left(-x + 1\right)}=\answer{0}\]
\end{problem}}

%%%%%%%%%%%%%%%%%%%%%%

\latexProblemContent{
\ifVerboseLocation This is Derivative Compute Question 0051. \\ \fi
\begin{problem}

Find the limit.  Use L'H$\hat{o}$pital's rule where appropriate.

\input{Derivative-Compute-0051.HELP.tex}

\[\lim\limits_{x\to\infty} 2 \, {\left(x + 5\right)} e^{\left(-x + 1\right)}=\answer{0}\]
\end{problem}}

%%%%%%%%%%%%%%%%%%%%%%

\latexProblemContent{
\ifVerboseLocation This is Derivative Compute Question 0051. \\ \fi
\begin{problem}

Find the limit.  Use L'H$\hat{o}$pital's rule where appropriate.

\input{Derivative-Compute-0051.HELP.tex}

\[\lim\limits_{x\to\infty} 6 \, {\left(x - 2\right)}^{3} e^{\left(-x - 5\right)}=\answer{0}\]
\end{problem}}

%%%%%%%%%%%%%%%%%%%%%%

\latexProblemContent{
\ifVerboseLocation This is Derivative Compute Question 0051. \\ \fi
\begin{problem}

Find the limit.  Use L'H$\hat{o}$pital's rule where appropriate.

\input{Derivative-Compute-0051.HELP.tex}

\[\lim\limits_{x\to\infty} 5 \, {\left(x + 4\right)}^{3} e^{\left(-x - 8\right)}=\answer{0}\]
\end{problem}}

%%%%%%%%%%%%%%%%%%%%%%

\latexProblemContent{
\ifVerboseLocation This is Derivative Compute Question 0051. \\ \fi
\begin{problem}

Find the limit.  Use L'H$\hat{o}$pital's rule where appropriate.

\input{Derivative-Compute-0051.HELP.tex}

\[\lim\limits_{x\to\infty} 8 \, {\left(x + 2\right)} e^{\left(-x + 1\right)}=\answer{0}\]
\end{problem}}

%%%%%%%%%%%%%%%%%%%%%%

\latexProblemContent{
\ifVerboseLocation This is Derivative Compute Question 0051. \\ \fi
\begin{problem}

Find the limit.  Use L'H$\hat{o}$pital's rule where appropriate.

\input{Derivative-Compute-0051.HELP.tex}

\[\lim\limits_{x\to\infty} 4 \, {\left(x - 8\right)} e^{\left(-x - 4\right)}=\answer{0}\]
\end{problem}}

%%%%%%%%%%%%%%%%%%%%%%

\latexProblemContent{
\ifVerboseLocation This is Derivative Compute Question 0051. \\ \fi
\begin{problem}

Find the limit.  Use L'H$\hat{o}$pital's rule where appropriate.

\input{Derivative-Compute-0051.HELP.tex}

\[\lim\limits_{x\to\infty} 6 \, {\left(x + 6\right)}^{3} e^{\left(-x - 1\right)}=\answer{0}\]
\end{problem}}

%%%%%%%%%%%%%%%%%%%%%%

\latexProblemContent{
\ifVerboseLocation This is Derivative Compute Question 0051. \\ \fi
\begin{problem}

Find the limit.  Use L'H$\hat{o}$pital's rule where appropriate.

\input{Derivative-Compute-0051.HELP.tex}

\[\lim\limits_{x\to\infty} -7 \, {\left(x - 4\right)}^{3} e^{\left(-x + 6\right)}=\answer{0}\]
\end{problem}}

%%%%%%%%%%%%%%%%%%%%%%

\latexProblemContent{
\ifVerboseLocation This is Derivative Compute Question 0051. \\ \fi
\begin{problem}

Find the limit.  Use L'H$\hat{o}$pital's rule where appropriate.

\input{Derivative-Compute-0051.HELP.tex}

\[\lim\limits_{x\to\infty} -8 \, {\left(x + 4\right)} e^{\left(-x - 4\right)}=\answer{0}\]
\end{problem}}

%%%%%%%%%%%%%%%%%%%%%%

\latexProblemContent{
\ifVerboseLocation This is Derivative Compute Question 0051. \\ \fi
\begin{problem}

Find the limit.  Use L'H$\hat{o}$pital's rule where appropriate.

\input{Derivative-Compute-0051.HELP.tex}

\[\lim\limits_{x\to\infty} 3 \, {\left(x - 4\right)} e^{\left(-x - 8\right)}=\answer{0}\]
\end{problem}}

%%%%%%%%%%%%%%%%%%%%%%

\latexProblemContent{
\ifVerboseLocation This is Derivative Compute Question 0051. \\ \fi
\begin{problem}

Find the limit.  Use L'H$\hat{o}$pital's rule where appropriate.

\input{Derivative-Compute-0051.HELP.tex}

\[\lim\limits_{x\to\infty} -2 \, {\left(x - 2\right)}^{2} e^{\left(-x + 6\right)}=\answer{0}\]
\end{problem}}

%%%%%%%%%%%%%%%%%%%%%%

\latexProblemContent{
\ifVerboseLocation This is Derivative Compute Question 0051. \\ \fi
\begin{problem}

Find the limit.  Use L'H$\hat{o}$pital's rule where appropriate.

\input{Derivative-Compute-0051.HELP.tex}

\[\lim\limits_{x\to\infty} 8 \, {\left(x - 6\right)}^{3} e^{\left(-x + 7\right)}=\answer{0}\]
\end{problem}}

%%%%%%%%%%%%%%%%%%%%%%

\latexProblemContent{
\ifVerboseLocation This is Derivative Compute Question 0051. \\ \fi
\begin{problem}

Find the limit.  Use L'H$\hat{o}$pital's rule where appropriate.

\input{Derivative-Compute-0051.HELP.tex}

\[\lim\limits_{x\to\infty} -6 \, {\left(x - 5\right)}^{3} e^{\left(-x - 1\right)}=\answer{0}\]
\end{problem}}

%%%%%%%%%%%%%%%%%%%%%%

\latexProblemContent{
\ifVerboseLocation This is Derivative Compute Question 0051. \\ \fi
\begin{problem}

Find the limit.  Use L'H$\hat{o}$pital's rule where appropriate.

\input{Derivative-Compute-0051.HELP.tex}

\[\lim\limits_{x\to\infty} -8 \, {\left(x - 8\right)}^{3} e^{\left(-x - 7\right)}=\answer{0}\]
\end{problem}}

%%%%%%%%%%%%%%%%%%%%%%

\latexProblemContent{
\ifVerboseLocation This is Derivative Compute Question 0051. \\ \fi
\begin{problem}

Find the limit.  Use L'H$\hat{o}$pital's rule where appropriate.

\input{Derivative-Compute-0051.HELP.tex}

\[\lim\limits_{x\to\infty} 6 \, {\left(x + 5\right)}^{3} e^{\left(-x + 5\right)}=\answer{0}\]
\end{problem}}

%%%%%%%%%%%%%%%%%%%%%%

\latexProblemContent{
\ifVerboseLocation This is Derivative Compute Question 0051. \\ \fi
\begin{problem}

Find the limit.  Use L'H$\hat{o}$pital's rule where appropriate.

\input{Derivative-Compute-0051.HELP.tex}

\[\lim\limits_{x\to\infty} 8 \, {\left(x - 2\right)} e^{\left(-x - 3\right)}=\answer{0}\]
\end{problem}}

%%%%%%%%%%%%%%%%%%%%%%

\latexProblemContent{
\ifVerboseLocation This is Derivative Compute Question 0051. \\ \fi
\begin{problem}

Find the limit.  Use L'H$\hat{o}$pital's rule where appropriate.

\input{Derivative-Compute-0051.HELP.tex}

\[\lim\limits_{x\to\infty} -8 \, {\left(x - 8\right)} e^{\left(-x - 4\right)}=\answer{0}\]
\end{problem}}

%%%%%%%%%%%%%%%%%%%%%%

\latexProblemContent{
\ifVerboseLocation This is Derivative Compute Question 0051. \\ \fi
\begin{problem}

Find the limit.  Use L'H$\hat{o}$pital's rule where appropriate.

\input{Derivative-Compute-0051.HELP.tex}

\[\lim\limits_{x\to\infty} 8 \, {\left(x - 1\right)} e^{\left(-x - 5\right)}=\answer{0}\]
\end{problem}}

%%%%%%%%%%%%%%%%%%%%%%

\latexProblemContent{
\ifVerboseLocation This is Derivative Compute Question 0051. \\ \fi
\begin{problem}

Find the limit.  Use L'H$\hat{o}$pital's rule where appropriate.

\input{Derivative-Compute-0051.HELP.tex}

\[\lim\limits_{x\to\infty} 8 \, {\left(x + 5\right)} e^{\left(-x - 7\right)}=\answer{0}\]
\end{problem}}

%%%%%%%%%%%%%%%%%%%%%%

\latexProblemContent{
\ifVerboseLocation This is Derivative Compute Question 0051. \\ \fi
\begin{problem}

Find the limit.  Use L'H$\hat{o}$pital's rule where appropriate.

\input{Derivative-Compute-0051.HELP.tex}

\[\lim\limits_{x\to\infty} 6 \, {\left(x - 8\right)}^{3} e^{\left(-x + 1\right)}=\answer{0}\]
\end{problem}}

%%%%%%%%%%%%%%%%%%%%%%

\latexProblemContent{
\ifVerboseLocation This is Derivative Compute Question 0051. \\ \fi
\begin{problem}

Find the limit.  Use L'H$\hat{o}$pital's rule where appropriate.

\input{Derivative-Compute-0051.HELP.tex}

\[\lim\limits_{x\to\infty} 2 \, {\left(x - 4\right)}^{3} e^{\left(-x - 1\right)}=\answer{0}\]
\end{problem}}

%%%%%%%%%%%%%%%%%%%%%%

\latexProblemContent{
\ifVerboseLocation This is Derivative Compute Question 0051. \\ \fi
\begin{problem}

Find the limit.  Use L'H$\hat{o}$pital's rule where appropriate.

\input{Derivative-Compute-0051.HELP.tex}

\[\lim\limits_{x\to\infty} 3 \, {\left(x + 1\right)} e^{\left(-x + 4\right)}=\answer{0}\]
\end{problem}}

%%%%%%%%%%%%%%%%%%%%%%

\latexProblemContent{
\ifVerboseLocation This is Derivative Compute Question 0051. \\ \fi
\begin{problem}

Find the limit.  Use L'H$\hat{o}$pital's rule where appropriate.

\input{Derivative-Compute-0051.HELP.tex}

\[\lim\limits_{x\to\infty} -7 \, {\left(x - 2\right)}^{2} e^{\left(-x + 7\right)}=\answer{0}\]
\end{problem}}

%%%%%%%%%%%%%%%%%%%%%%

\latexProblemContent{
\ifVerboseLocation This is Derivative Compute Question 0051. \\ \fi
\begin{problem}

Find the limit.  Use L'H$\hat{o}$pital's rule where appropriate.

\input{Derivative-Compute-0051.HELP.tex}

\[\lim\limits_{x\to\infty} 4 \, {\left(x - 8\right)}^{2} e^{\left(-x + 7\right)}=\answer{0}\]
\end{problem}}

%%%%%%%%%%%%%%%%%%%%%%

\latexProblemContent{
\ifVerboseLocation This is Derivative Compute Question 0051. \\ \fi
\begin{problem}

Find the limit.  Use L'H$\hat{o}$pital's rule where appropriate.

\input{Derivative-Compute-0051.HELP.tex}

\[\lim\limits_{x\to\infty} 5 \, {\left(x + 7\right)}^{3} e^{\left(-x - 2\right)}=\answer{0}\]
\end{problem}}

%%%%%%%%%%%%%%%%%%%%%%

\latexProblemContent{
\ifVerboseLocation This is Derivative Compute Question 0051. \\ \fi
\begin{problem}

Find the limit.  Use L'H$\hat{o}$pital's rule where appropriate.

\input{Derivative-Compute-0051.HELP.tex}

\[\lim\limits_{x\to\infty} 7 \, {\left(x + 1\right)}^{3} e^{\left(-x - 6\right)}=\answer{0}\]
\end{problem}}

%%%%%%%%%%%%%%%%%%%%%%

\latexProblemContent{
\ifVerboseLocation This is Derivative Compute Question 0051. \\ \fi
\begin{problem}

Find the limit.  Use L'H$\hat{o}$pital's rule where appropriate.

\input{Derivative-Compute-0051.HELP.tex}

\[\lim\limits_{x\to\infty} {\left(x - 5\right)}^{3} e^{\left(-x - 6\right)}=\answer{0}\]
\end{problem}}

%%%%%%%%%%%%%%%%%%%%%%

\latexProblemContent{
\ifVerboseLocation This is Derivative Compute Question 0051. \\ \fi
\begin{problem}

Find the limit.  Use L'H$\hat{o}$pital's rule where appropriate.

\input{Derivative-Compute-0051.HELP.tex}

\[\lim\limits_{x\to\infty} 8 \, {\left(x + 7\right)}^{3} e^{\left(-x + 7\right)}=\answer{0}\]
\end{problem}}

%%%%%%%%%%%%%%%%%%%%%%

\latexProblemContent{
\ifVerboseLocation This is Derivative Compute Question 0051. \\ \fi
\begin{problem}

Find the limit.  Use L'H$\hat{o}$pital's rule where appropriate.

\input{Derivative-Compute-0051.HELP.tex}

\[\lim\limits_{x\to\infty} -{\left(x - 6\right)}^{2} e^{\left(-x - 5\right)}=\answer{0}\]
\end{problem}}

%%%%%%%%%%%%%%%%%%%%%%

\latexProblemContent{
\ifVerboseLocation This is Derivative Compute Question 0051. \\ \fi
\begin{problem}

Find the limit.  Use L'H$\hat{o}$pital's rule where appropriate.

\input{Derivative-Compute-0051.HELP.tex}

\[\lim\limits_{x\to\infty} 5 \, {\left(x - 3\right)} e^{\left(-x + 3\right)}=\answer{0}\]
\end{problem}}

%%%%%%%%%%%%%%%%%%%%%%

\latexProblemContent{
\ifVerboseLocation This is Derivative Compute Question 0051. \\ \fi
\begin{problem}

Find the limit.  Use L'H$\hat{o}$pital's rule where appropriate.

\input{Derivative-Compute-0051.HELP.tex}

\[\lim\limits_{x\to\infty} {\left(x + 5\right)}^{3} e^{\left(-x - 5\right)}=\answer{0}\]
\end{problem}}

%%%%%%%%%%%%%%%%%%%%%%

\latexProblemContent{
\ifVerboseLocation This is Derivative Compute Question 0051. \\ \fi
\begin{problem}

Find the limit.  Use L'H$\hat{o}$pital's rule where appropriate.

\input{Derivative-Compute-0051.HELP.tex}

\[\lim\limits_{x\to\infty} -2 \, {\left(x + 1\right)} e^{\left(-x + 5\right)}=\answer{0}\]
\end{problem}}

%%%%%%%%%%%%%%%%%%%%%%

\latexProblemContent{
\ifVerboseLocation This is Derivative Compute Question 0051. \\ \fi
\begin{problem}

Find the limit.  Use L'H$\hat{o}$pital's rule where appropriate.

\input{Derivative-Compute-0051.HELP.tex}

\[\lim\limits_{x\to\infty} {\left(x - 3\right)}^{2} e^{\left(-x + 2\right)}=\answer{0}\]
\end{problem}}

%%%%%%%%%%%%%%%%%%%%%%

\latexProblemContent{
\ifVerboseLocation This is Derivative Compute Question 0051. \\ \fi
\begin{problem}

Find the limit.  Use L'H$\hat{o}$pital's rule where appropriate.

\input{Derivative-Compute-0051.HELP.tex}

\[\lim\limits_{x\to\infty} 4 \, {\left(x + 7\right)}^{3} e^{\left(-x + 4\right)}=\answer{0}\]
\end{problem}}

%%%%%%%%%%%%%%%%%%%%%%

\latexProblemContent{
\ifVerboseLocation This is Derivative Compute Question 0051. \\ \fi
\begin{problem}

Find the limit.  Use L'H$\hat{o}$pital's rule where appropriate.

\input{Derivative-Compute-0051.HELP.tex}

\[\lim\limits_{x\to\infty} {\left(x + 6\right)}^{3} e^{\left(-x - 3\right)}=\answer{0}\]
\end{problem}}

%%%%%%%%%%%%%%%%%%%%%%

\latexProblemContent{
\ifVerboseLocation This is Derivative Compute Question 0051. \\ \fi
\begin{problem}

Find the limit.  Use L'H$\hat{o}$pital's rule where appropriate.

\input{Derivative-Compute-0051.HELP.tex}

\[\lim\limits_{x\to\infty} {\left(x + 2\right)}^{3} e^{\left(-x + 3\right)}=\answer{0}\]
\end{problem}}

%%%%%%%%%%%%%%%%%%%%%%

\latexProblemContent{
\ifVerboseLocation This is Derivative Compute Question 0051. \\ \fi
\begin{problem}

Find the limit.  Use L'H$\hat{o}$pital's rule where appropriate.

\input{Derivative-Compute-0051.HELP.tex}

\[\lim\limits_{x\to\infty} 5 \, {\left(x - 4\right)}^{3} e^{\left(-x + 5\right)}=\answer{0}\]
\end{problem}}

%%%%%%%%%%%%%%%%%%%%%%

\latexProblemContent{
\ifVerboseLocation This is Derivative Compute Question 0051. \\ \fi
\begin{problem}

Find the limit.  Use L'H$\hat{o}$pital's rule where appropriate.

\input{Derivative-Compute-0051.HELP.tex}

\[\lim\limits_{x\to\infty} -8 \, {\left(x + 3\right)} e^{\left(-x - 4\right)}=\answer{0}\]
\end{problem}}

%%%%%%%%%%%%%%%%%%%%%%

\latexProblemContent{
\ifVerboseLocation This is Derivative Compute Question 0051. \\ \fi
\begin{problem}

Find the limit.  Use L'H$\hat{o}$pital's rule where appropriate.

\input{Derivative-Compute-0051.HELP.tex}

\[\lim\limits_{x\to\infty} -5 \, {\left(x - 3\right)} e^{\left(-x + 4\right)}=\answer{0}\]
\end{problem}}

%%%%%%%%%%%%%%%%%%%%%%

\latexProblemContent{
\ifVerboseLocation This is Derivative Compute Question 0051. \\ \fi
\begin{problem}

Find the limit.  Use L'H$\hat{o}$pital's rule where appropriate.

\input{Derivative-Compute-0051.HELP.tex}

\[\lim\limits_{x\to\infty} 2 \, {\left(x + 5\right)}^{3} e^{\left(-x + 3\right)}=\answer{0}\]
\end{problem}}

%%%%%%%%%%%%%%%%%%%%%%

\latexProblemContent{
\ifVerboseLocation This is Derivative Compute Question 0051. \\ \fi
\begin{problem}

Find the limit.  Use L'H$\hat{o}$pital's rule where appropriate.

\input{Derivative-Compute-0051.HELP.tex}

\[\lim\limits_{x\to\infty} {\left(x + 1\right)} e^{\left(-x + 4\right)}=\answer{0}\]
\end{problem}}

%%%%%%%%%%%%%%%%%%%%%%

\latexProblemContent{
\ifVerboseLocation This is Derivative Compute Question 0051. \\ \fi
\begin{problem}

Find the limit.  Use L'H$\hat{o}$pital's rule where appropriate.

\input{Derivative-Compute-0051.HELP.tex}

\[\lim\limits_{x\to\infty} -8 \, {\left(x + 8\right)}^{3} e^{\left(-x - 2\right)}=\answer{0}\]
\end{problem}}

%%%%%%%%%%%%%%%%%%%%%%

\latexProblemContent{
\ifVerboseLocation This is Derivative Compute Question 0051. \\ \fi
\begin{problem}

Find the limit.  Use L'H$\hat{o}$pital's rule where appropriate.

\input{Derivative-Compute-0051.HELP.tex}

\[\lim\limits_{x\to\infty} 3 \, {\left(x - 1\right)} e^{\left(-x + 2\right)}=\answer{0}\]
\end{problem}}

%%%%%%%%%%%%%%%%%%%%%%

\latexProblemContent{
\ifVerboseLocation This is Derivative Compute Question 0051. \\ \fi
\begin{problem}

Find the limit.  Use L'H$\hat{o}$pital's rule where appropriate.

\input{Derivative-Compute-0051.HELP.tex}

\[\lim\limits_{x\to\infty} 2 \, {\left(x - 7\right)}^{3} e^{\left(-x + 7\right)}=\answer{0}\]
\end{problem}}

%%%%%%%%%%%%%%%%%%%%%%

\latexProblemContent{
\ifVerboseLocation This is Derivative Compute Question 0051. \\ \fi
\begin{problem}

Find the limit.  Use L'H$\hat{o}$pital's rule where appropriate.

\input{Derivative-Compute-0051.HELP.tex}

\[\lim\limits_{x\to\infty} 6 \, {\left(x - 6\right)}^{3} e^{\left(-x - 1\right)}=\answer{0}\]
\end{problem}}

%%%%%%%%%%%%%%%%%%%%%%

\latexProblemContent{
\ifVerboseLocation This is Derivative Compute Question 0051. \\ \fi
\begin{problem}

Find the limit.  Use L'H$\hat{o}$pital's rule where appropriate.

\input{Derivative-Compute-0051.HELP.tex}

\[\lim\limits_{x\to\infty} -{\left(x + 5\right)}^{2} e^{\left(-x + 1\right)}=\answer{0}\]
\end{problem}}

%%%%%%%%%%%%%%%%%%%%%%

\latexProblemContent{
\ifVerboseLocation This is Derivative Compute Question 0051. \\ \fi
\begin{problem}

Find the limit.  Use L'H$\hat{o}$pital's rule where appropriate.

\input{Derivative-Compute-0051.HELP.tex}

\[\lim\limits_{x\to\infty} -8 \, {\left(x + 8\right)} e^{\left(-x + 1\right)}=\answer{0}\]
\end{problem}}

%%%%%%%%%%%%%%%%%%%%%%

\latexProblemContent{
\ifVerboseLocation This is Derivative Compute Question 0051. \\ \fi
\begin{problem}

Find the limit.  Use L'H$\hat{o}$pital's rule where appropriate.

\input{Derivative-Compute-0051.HELP.tex}

\[\lim\limits_{x\to\infty} 6 \, {\left(x - 4\right)} e^{\left(-x - 5\right)}=\answer{0}\]
\end{problem}}

%%%%%%%%%%%%%%%%%%%%%%

\latexProblemContent{
\ifVerboseLocation This is Derivative Compute Question 0051. \\ \fi
\begin{problem}

Find the limit.  Use L'H$\hat{o}$pital's rule where appropriate.

\input{Derivative-Compute-0051.HELP.tex}

\[\lim\limits_{x\to\infty} {\left(x - 4\right)} e^{\left(-x + 3\right)}=\answer{0}\]
\end{problem}}

%%%%%%%%%%%%%%%%%%%%%%

\latexProblemContent{
\ifVerboseLocation This is Derivative Compute Question 0051. \\ \fi
\begin{problem}

Find the limit.  Use L'H$\hat{o}$pital's rule where appropriate.

\input{Derivative-Compute-0051.HELP.tex}

\[\lim\limits_{x\to\infty} 4 \, {\left(x - 6\right)} e^{\left(-x - 4\right)}=\answer{0}\]
\end{problem}}

%%%%%%%%%%%%%%%%%%%%%%

\latexProblemContent{
\ifVerboseLocation This is Derivative Compute Question 0051. \\ \fi
\begin{problem}

Find the limit.  Use L'H$\hat{o}$pital's rule where appropriate.

\input{Derivative-Compute-0051.HELP.tex}

\[\lim\limits_{x\to\infty} -5 \, {\left(x - 6\right)}^{3} e^{\left(-x - 2\right)}=\answer{0}\]
\end{problem}}

%%%%%%%%%%%%%%%%%%%%%%

\latexProblemContent{
\ifVerboseLocation This is Derivative Compute Question 0051. \\ \fi
\begin{problem}

Find the limit.  Use L'H$\hat{o}$pital's rule where appropriate.

\input{Derivative-Compute-0051.HELP.tex}

\[\lim\limits_{x\to\infty} 3 \, {\left(x + 6\right)}^{3} e^{\left(-x + 3\right)}=\answer{0}\]
\end{problem}}

%%%%%%%%%%%%%%%%%%%%%%

\latexProblemContent{
\ifVerboseLocation This is Derivative Compute Question 0051. \\ \fi
\begin{problem}

Find the limit.  Use L'H$\hat{o}$pital's rule where appropriate.

\input{Derivative-Compute-0051.HELP.tex}

\[\lim\limits_{x\to\infty} 3 \, {\left(x - 5\right)}^{2} e^{\left(-x - 1\right)}=\answer{0}\]
\end{problem}}

%%%%%%%%%%%%%%%%%%%%%%

\latexProblemContent{
\ifVerboseLocation This is Derivative Compute Question 0051. \\ \fi
\begin{problem}

Find the limit.  Use L'H$\hat{o}$pital's rule where appropriate.

\input{Derivative-Compute-0051.HELP.tex}

\[\lim\limits_{x\to\infty} -5 \, {\left(x - 4\right)}^{2} e^{\left(-x + 2\right)}=\answer{0}\]
\end{problem}}

%%%%%%%%%%%%%%%%%%%%%%

\latexProblemContent{
\ifVerboseLocation This is Derivative Compute Question 0051. \\ \fi
\begin{problem}

Find the limit.  Use L'H$\hat{o}$pital's rule where appropriate.

\input{Derivative-Compute-0051.HELP.tex}

\[\lim\limits_{x\to\infty} 4 \, {\left(x - 2\right)}^{3} e^{\left(-x + 8\right)}=\answer{0}\]
\end{problem}}

%%%%%%%%%%%%%%%%%%%%%%

\latexProblemContent{
\ifVerboseLocation This is Derivative Compute Question 0051. \\ \fi
\begin{problem}

Find the limit.  Use L'H$\hat{o}$pital's rule where appropriate.

\input{Derivative-Compute-0051.HELP.tex}

\[\lim\limits_{x\to\infty} -3 \, {\left(x + 4\right)}^{2} e^{\left(-x - 1\right)}=\answer{0}\]
\end{problem}}

%%%%%%%%%%%%%%%%%%%%%%

\latexProblemContent{
\ifVerboseLocation This is Derivative Compute Question 0051. \\ \fi
\begin{problem}

Find the limit.  Use L'H$\hat{o}$pital's rule where appropriate.

\input{Derivative-Compute-0051.HELP.tex}

\[\lim\limits_{x\to\infty} -7 \, {\left(x - 4\right)}^{2} e^{\left(-x + 3\right)}=\answer{0}\]
\end{problem}}

%%%%%%%%%%%%%%%%%%%%%%

\latexProblemContent{
\ifVerboseLocation This is Derivative Compute Question 0051. \\ \fi
\begin{problem}

Find the limit.  Use L'H$\hat{o}$pital's rule where appropriate.

\input{Derivative-Compute-0051.HELP.tex}

\[\lim\limits_{x\to\infty} {\left(x + 7\right)}^{2} e^{\left(-x - 7\right)}=\answer{0}\]
\end{problem}}

%%%%%%%%%%%%%%%%%%%%%%

\latexProblemContent{
\ifVerboseLocation This is Derivative Compute Question 0051. \\ \fi
\begin{problem}

Find the limit.  Use L'H$\hat{o}$pital's rule where appropriate.

\input{Derivative-Compute-0051.HELP.tex}

\[\lim\limits_{x\to\infty} -{\left(x - 6\right)}^{3} e^{\left(-x - 2\right)}=\answer{0}\]
\end{problem}}

%%%%%%%%%%%%%%%%%%%%%%

\latexProblemContent{
\ifVerboseLocation This is Derivative Compute Question 0051. \\ \fi
\begin{problem}

Find the limit.  Use L'H$\hat{o}$pital's rule where appropriate.

\input{Derivative-Compute-0051.HELP.tex}

\[\lim\limits_{x\to\infty} -7 \, {\left(x + 2\right)}^{2} e^{\left(-x - 3\right)}=\answer{0}\]
\end{problem}}

%%%%%%%%%%%%%%%%%%%%%%

\latexProblemContent{
\ifVerboseLocation This is Derivative Compute Question 0051. \\ \fi
\begin{problem}

Find the limit.  Use L'H$\hat{o}$pital's rule where appropriate.

\input{Derivative-Compute-0051.HELP.tex}

\[\lim\limits_{x\to\infty} -4 \, {\left(x - 7\right)} e^{\left(-x - 8\right)}=\answer{0}\]
\end{problem}}

%%%%%%%%%%%%%%%%%%%%%%

\latexProblemContent{
\ifVerboseLocation This is Derivative Compute Question 0051. \\ \fi
\begin{problem}

Find the limit.  Use L'H$\hat{o}$pital's rule where appropriate.

\input{Derivative-Compute-0051.HELP.tex}

\[\lim\limits_{x\to\infty} 5 \, {\left(x + 6\right)}^{3} e^{\left(-x + 7\right)}=\answer{0}\]
\end{problem}}

%%%%%%%%%%%%%%%%%%%%%%

\latexProblemContent{
\ifVerboseLocation This is Derivative Compute Question 0051. \\ \fi
\begin{problem}

Find the limit.  Use L'H$\hat{o}$pital's rule where appropriate.

\input{Derivative-Compute-0051.HELP.tex}

\[\lim\limits_{x\to\infty} -2 \, {\left(x - 1\right)} e^{\left(-x - 3\right)}=\answer{0}\]
\end{problem}}

%%%%%%%%%%%%%%%%%%%%%%

\latexProblemContent{
\ifVerboseLocation This is Derivative Compute Question 0051. \\ \fi
\begin{problem}

Find the limit.  Use L'H$\hat{o}$pital's rule where appropriate.

\input{Derivative-Compute-0051.HELP.tex}

\[\lim\limits_{x\to\infty} 2 \, {\left(x - 8\right)}^{3} e^{\left(-x - 7\right)}=\answer{0}\]
\end{problem}}

%%%%%%%%%%%%%%%%%%%%%%

\latexProblemContent{
\ifVerboseLocation This is Derivative Compute Question 0051. \\ \fi
\begin{problem}

Find the limit.  Use L'H$\hat{o}$pital's rule where appropriate.

\input{Derivative-Compute-0051.HELP.tex}

\[\lim\limits_{x\to\infty} 3 \, {\left(x + 3\right)} e^{\left(-x - 8\right)}=\answer{0}\]
\end{problem}}

%%%%%%%%%%%%%%%%%%%%%%

\latexProblemContent{
\ifVerboseLocation This is Derivative Compute Question 0051. \\ \fi
\begin{problem}

Find the limit.  Use L'H$\hat{o}$pital's rule where appropriate.

\input{Derivative-Compute-0051.HELP.tex}

\[\lim\limits_{x\to\infty} 7 \, {\left(x + 2\right)}^{2} e^{\left(-x - 2\right)}=\answer{0}\]
\end{problem}}

%%%%%%%%%%%%%%%%%%%%%%

\latexProblemContent{
\ifVerboseLocation This is Derivative Compute Question 0051. \\ \fi
\begin{problem}

Find the limit.  Use L'H$\hat{o}$pital's rule where appropriate.

\input{Derivative-Compute-0051.HELP.tex}

\[\lim\limits_{x\to\infty} {\left(x + 4\right)} e^{\left(-x + 7\right)}=\answer{0}\]
\end{problem}}

%%%%%%%%%%%%%%%%%%%%%%

\latexProblemContent{
\ifVerboseLocation This is Derivative Compute Question 0051. \\ \fi
\begin{problem}

Find the limit.  Use L'H$\hat{o}$pital's rule where appropriate.

\input{Derivative-Compute-0051.HELP.tex}

\[\lim\limits_{x\to\infty} 8 \, {\left(x + 6\right)} e^{\left(-x + 4\right)}=\answer{0}\]
\end{problem}}

%%%%%%%%%%%%%%%%%%%%%%

\latexProblemContent{
\ifVerboseLocation This is Derivative Compute Question 0051. \\ \fi
\begin{problem}

Find the limit.  Use L'H$\hat{o}$pital's rule where appropriate.

\input{Derivative-Compute-0051.HELP.tex}

\[\lim\limits_{x\to\infty} 2 \, {\left(x - 3\right)}^{3} e^{\left(-x - 2\right)}=\answer{0}\]
\end{problem}}

%%%%%%%%%%%%%%%%%%%%%%

\latexProblemContent{
\ifVerboseLocation This is Derivative Compute Question 0051. \\ \fi
\begin{problem}

Find the limit.  Use L'H$\hat{o}$pital's rule where appropriate.

\input{Derivative-Compute-0051.HELP.tex}

\[\lim\limits_{x\to\infty} 5 \, {\left(x + 5\right)} e^{\left(-x + 4\right)}=\answer{0}\]
\end{problem}}

%%%%%%%%%%%%%%%%%%%%%%

\latexProblemContent{
\ifVerboseLocation This is Derivative Compute Question 0051. \\ \fi
\begin{problem}

Find the limit.  Use L'H$\hat{o}$pital's rule where appropriate.

\input{Derivative-Compute-0051.HELP.tex}

\[\lim\limits_{x\to\infty} 4 \, {\left(x - 8\right)}^{3} e^{\left(-x + 8\right)}=\answer{0}\]
\end{problem}}

%%%%%%%%%%%%%%%%%%%%%%

\latexProblemContent{
\ifVerboseLocation This is Derivative Compute Question 0051. \\ \fi
\begin{problem}

Find the limit.  Use L'H$\hat{o}$pital's rule where appropriate.

\input{Derivative-Compute-0051.HELP.tex}

\[\lim\limits_{x\to\infty} 4 \, {\left(x - 3\right)}^{3} e^{\left(-x - 4\right)}=\answer{0}\]
\end{problem}}

%%%%%%%%%%%%%%%%%%%%%%

\latexProblemContent{
\ifVerboseLocation This is Derivative Compute Question 0051. \\ \fi
\begin{problem}

Find the limit.  Use L'H$\hat{o}$pital's rule where appropriate.

\input{Derivative-Compute-0051.HELP.tex}

\[\lim\limits_{x\to\infty} -2 \, {\left(x - 5\right)} e^{\left(-x - 1\right)}=\answer{0}\]
\end{problem}}

%%%%%%%%%%%%%%%%%%%%%%

\latexProblemContent{
\ifVerboseLocation This is Derivative Compute Question 0051. \\ \fi
\begin{problem}

Find the limit.  Use L'H$\hat{o}$pital's rule where appropriate.

\input{Derivative-Compute-0051.HELP.tex}

\[\lim\limits_{x\to\infty} 7 \, {\left(x + 4\right)}^{3} e^{\left(-x + 3\right)}=\answer{0}\]
\end{problem}}

%%%%%%%%%%%%%%%%%%%%%%

\latexProblemContent{
\ifVerboseLocation This is Derivative Compute Question 0051. \\ \fi
\begin{problem}

Find the limit.  Use L'H$\hat{o}$pital's rule where appropriate.

\input{Derivative-Compute-0051.HELP.tex}

\[\lim\limits_{x\to\infty} -{\left(x - 8\right)}^{3} e^{\left(-x - 1\right)}=\answer{0}\]
\end{problem}}

%%%%%%%%%%%%%%%%%%%%%%

\latexProblemContent{
\ifVerboseLocation This is Derivative Compute Question 0051. \\ \fi
\begin{problem}

Find the limit.  Use L'H$\hat{o}$pital's rule where appropriate.

\input{Derivative-Compute-0051.HELP.tex}

\[\lim\limits_{x\to\infty} -5 \, {\left(x + 8\right)}^{2} e^{\left(-x + 8\right)}=\answer{0}\]
\end{problem}}

%%%%%%%%%%%%%%%%%%%%%%

\latexProblemContent{
\ifVerboseLocation This is Derivative Compute Question 0051. \\ \fi
\begin{problem}

Find the limit.  Use L'H$\hat{o}$pital's rule where appropriate.

\input{Derivative-Compute-0051.HELP.tex}

\[\lim\limits_{x\to\infty} -8 \, {\left(x + 8\right)}^{2} e^{\left(-x + 4\right)}=\answer{0}\]
\end{problem}}

%%%%%%%%%%%%%%%%%%%%%%

\latexProblemContent{
\ifVerboseLocation This is Derivative Compute Question 0051. \\ \fi
\begin{problem}

Find the limit.  Use L'H$\hat{o}$pital's rule where appropriate.

\input{Derivative-Compute-0051.HELP.tex}

\[\lim\limits_{x\to\infty} -3 \, {\left(x - 4\right)}^{3} e^{\left(-x - 3\right)}=\answer{0}\]
\end{problem}}

%%%%%%%%%%%%%%%%%%%%%%

\latexProblemContent{
\ifVerboseLocation This is Derivative Compute Question 0051. \\ \fi
\begin{problem}

Find the limit.  Use L'H$\hat{o}$pital's rule where appropriate.

\input{Derivative-Compute-0051.HELP.tex}

\[\lim\limits_{x\to\infty} -5 \, {\left(x + 3\right)}^{3} e^{\left(-x - 8\right)}=\answer{0}\]
\end{problem}}

%%%%%%%%%%%%%%%%%%%%%%

\latexProblemContent{
\ifVerboseLocation This is Derivative Compute Question 0051. \\ \fi
\begin{problem}

Find the limit.  Use L'H$\hat{o}$pital's rule where appropriate.

\input{Derivative-Compute-0051.HELP.tex}

\[\lim\limits_{x\to\infty} -3 \, {\left(x + 3\right)}^{3} e^{\left(-x + 3\right)}=\answer{0}\]
\end{problem}}

%%%%%%%%%%%%%%%%%%%%%%

\latexProblemContent{
\ifVerboseLocation This is Derivative Compute Question 0051. \\ \fi
\begin{problem}

Find the limit.  Use L'H$\hat{o}$pital's rule where appropriate.

\input{Derivative-Compute-0051.HELP.tex}

\[\lim\limits_{x\to\infty} 5 \, {\left(x + 5\right)}^{3} e^{\left(-x + 5\right)}=\answer{0}\]
\end{problem}}

%%%%%%%%%%%%%%%%%%%%%%

\latexProblemContent{
\ifVerboseLocation This is Derivative Compute Question 0051. \\ \fi
\begin{problem}

Find the limit.  Use L'H$\hat{o}$pital's rule where appropriate.

\input{Derivative-Compute-0051.HELP.tex}

\[\lim\limits_{x\to\infty} -5 \, {\left(x + 5\right)}^{2} e^{\left(-x - 4\right)}=\answer{0}\]
\end{problem}}

%%%%%%%%%%%%%%%%%%%%%%

\latexProblemContent{
\ifVerboseLocation This is Derivative Compute Question 0051. \\ \fi
\begin{problem}

Find the limit.  Use L'H$\hat{o}$pital's rule where appropriate.

\input{Derivative-Compute-0051.HELP.tex}

\[\lim\limits_{x\to\infty} 2 \, {\left(x - 1\right)}^{3} e^{\left(-x + 2\right)}=\answer{0}\]
\end{problem}}

%%%%%%%%%%%%%%%%%%%%%%

\latexProblemContent{
\ifVerboseLocation This is Derivative Compute Question 0051. \\ \fi
\begin{problem}

Find the limit.  Use L'H$\hat{o}$pital's rule where appropriate.

\input{Derivative-Compute-0051.HELP.tex}

\[\lim\limits_{x\to\infty} -2 \, {\left(x - 1\right)}^{2} e^{\left(-x + 7\right)}=\answer{0}\]
\end{problem}}

%%%%%%%%%%%%%%%%%%%%%%

\latexProblemContent{
\ifVerboseLocation This is Derivative Compute Question 0051. \\ \fi
\begin{problem}

Find the limit.  Use L'H$\hat{o}$pital's rule where appropriate.

\input{Derivative-Compute-0051.HELP.tex}

\[\lim\limits_{x\to\infty} 4 \, {\left(x + 5\right)}^{3} e^{\left(-x + 2\right)}=\answer{0}\]
\end{problem}}

%%%%%%%%%%%%%%%%%%%%%%

\latexProblemContent{
\ifVerboseLocation This is Derivative Compute Question 0051. \\ \fi
\begin{problem}

Find the limit.  Use L'H$\hat{o}$pital's rule where appropriate.

\input{Derivative-Compute-0051.HELP.tex}

\[\lim\limits_{x\to\infty} 8 \, {\left(x + 3\right)} e^{\left(-x - 6\right)}=\answer{0}\]
\end{problem}}

%%%%%%%%%%%%%%%%%%%%%%

\latexProblemContent{
\ifVerboseLocation This is Derivative Compute Question 0051. \\ \fi
\begin{problem}

Find the limit.  Use L'H$\hat{o}$pital's rule where appropriate.

\input{Derivative-Compute-0051.HELP.tex}

\[\lim\limits_{x\to\infty} 2 \, {\left(x + 3\right)}^{3} e^{\left(-x + 8\right)}=\answer{0}\]
\end{problem}}

%%%%%%%%%%%%%%%%%%%%%%

\latexProblemContent{
\ifVerboseLocation This is Derivative Compute Question 0051. \\ \fi
\begin{problem}

Find the limit.  Use L'H$\hat{o}$pital's rule where appropriate.

\input{Derivative-Compute-0051.HELP.tex}

\[\lim\limits_{x\to\infty} 2 \, {\left(x + 2\right)}^{3} e^{\left(-x + 3\right)}=\answer{0}\]
\end{problem}}

%%%%%%%%%%%%%%%%%%%%%%

\latexProblemContent{
\ifVerboseLocation This is Derivative Compute Question 0051. \\ \fi
\begin{problem}

Find the limit.  Use L'H$\hat{o}$pital's rule where appropriate.

\input{Derivative-Compute-0051.HELP.tex}

\[\lim\limits_{x\to\infty} 2 \, {\left(x + 5\right)}^{3} e^{\left(-x - 3\right)}=\answer{0}\]
\end{problem}}

%%%%%%%%%%%%%%%%%%%%%%

\latexProblemContent{
\ifVerboseLocation This is Derivative Compute Question 0051. \\ \fi
\begin{problem}

Find the limit.  Use L'H$\hat{o}$pital's rule where appropriate.

\input{Derivative-Compute-0051.HELP.tex}

\[\lim\limits_{x\to\infty} 4 \, {\left(x - 5\right)}^{2} e^{\left(-x + 3\right)}=\answer{0}\]
\end{problem}}

%%%%%%%%%%%%%%%%%%%%%%

\latexProblemContent{
\ifVerboseLocation This is Derivative Compute Question 0051. \\ \fi
\begin{problem}

Find the limit.  Use L'H$\hat{o}$pital's rule where appropriate.

\input{Derivative-Compute-0051.HELP.tex}

\[\lim\limits_{x\to\infty} {\left(x - 7\right)}^{2} e^{\left(-x - 4\right)}=\answer{0}\]
\end{problem}}

%%%%%%%%%%%%%%%%%%%%%%

\latexProblemContent{
\ifVerboseLocation This is Derivative Compute Question 0051. \\ \fi
\begin{problem}

Find the limit.  Use L'H$\hat{o}$pital's rule where appropriate.

\input{Derivative-Compute-0051.HELP.tex}

\[\lim\limits_{x\to\infty} -7 \, {\left(x - 1\right)}^{3} e^{\left(-x + 8\right)}=\answer{0}\]
\end{problem}}

%%%%%%%%%%%%%%%%%%%%%%

\latexProblemContent{
\ifVerboseLocation This is Derivative Compute Question 0051. \\ \fi
\begin{problem}

Find the limit.  Use L'H$\hat{o}$pital's rule where appropriate.

\input{Derivative-Compute-0051.HELP.tex}

\[\lim\limits_{x\to\infty} -6 \, {\left(x + 1\right)}^{2} e^{\left(-x + 4\right)}=\answer{0}\]
\end{problem}}

%%%%%%%%%%%%%%%%%%%%%%

\latexProblemContent{
\ifVerboseLocation This is Derivative Compute Question 0051. \\ \fi
\begin{problem}

Find the limit.  Use L'H$\hat{o}$pital's rule where appropriate.

\input{Derivative-Compute-0051.HELP.tex}

\[\lim\limits_{x\to\infty} 2 \, {\left(x - 7\right)}^{2} e^{\left(-x + 7\right)}=\answer{0}\]
\end{problem}}

%%%%%%%%%%%%%%%%%%%%%%

\latexProblemContent{
\ifVerboseLocation This is Derivative Compute Question 0051. \\ \fi
\begin{problem}

Find the limit.  Use L'H$\hat{o}$pital's rule where appropriate.

\input{Derivative-Compute-0051.HELP.tex}

\[\lim\limits_{x\to\infty} 5 \, {\left(x - 2\right)}^{2} e^{\left(-x - 8\right)}=\answer{0}\]
\end{problem}}

%%%%%%%%%%%%%%%%%%%%%%

\latexProblemContent{
\ifVerboseLocation This is Derivative Compute Question 0051. \\ \fi
\begin{problem}

Find the limit.  Use L'H$\hat{o}$pital's rule where appropriate.

\input{Derivative-Compute-0051.HELP.tex}

\[\lim\limits_{x\to\infty} 4 \, {\left(x - 6\right)}^{3} e^{\left(-x + 8\right)}=\answer{0}\]
\end{problem}}

%%%%%%%%%%%%%%%%%%%%%%

\latexProblemContent{
\ifVerboseLocation This is Derivative Compute Question 0051. \\ \fi
\begin{problem}

Find the limit.  Use L'H$\hat{o}$pital's rule where appropriate.

\input{Derivative-Compute-0051.HELP.tex}

\[\lim\limits_{x\to\infty} 4 \, {\left(x - 7\right)}^{2} e^{\left(-x - 6\right)}=\answer{0}\]
\end{problem}}

%%%%%%%%%%%%%%%%%%%%%%

\latexProblemContent{
\ifVerboseLocation This is Derivative Compute Question 0051. \\ \fi
\begin{problem}

Find the limit.  Use L'H$\hat{o}$pital's rule where appropriate.

\input{Derivative-Compute-0051.HELP.tex}

\[\lim\limits_{x\to\infty} -{\left(x + 8\right)}^{2} e^{\left(-x - 2\right)}=\answer{0}\]
\end{problem}}

%%%%%%%%%%%%%%%%%%%%%%

\latexProblemContent{
\ifVerboseLocation This is Derivative Compute Question 0051. \\ \fi
\begin{problem}

Find the limit.  Use L'H$\hat{o}$pital's rule where appropriate.

\input{Derivative-Compute-0051.HELP.tex}

\[\lim\limits_{x\to\infty} -{\left(x + 5\right)} e^{\left(-x - 2\right)}=\answer{0}\]
\end{problem}}

%%%%%%%%%%%%%%%%%%%%%%

\latexProblemContent{
\ifVerboseLocation This is Derivative Compute Question 0051. \\ \fi
\begin{problem}

Find the limit.  Use L'H$\hat{o}$pital's rule where appropriate.

\input{Derivative-Compute-0051.HELP.tex}

\[\lim\limits_{x\to\infty} 5 \, {\left(x + 3\right)}^{2} e^{\left(-x - 7\right)}=\answer{0}\]
\end{problem}}

%%%%%%%%%%%%%%%%%%%%%%

\latexProblemContent{
\ifVerboseLocation This is Derivative Compute Question 0051. \\ \fi
\begin{problem}

Find the limit.  Use L'H$\hat{o}$pital's rule where appropriate.

\input{Derivative-Compute-0051.HELP.tex}

\[\lim\limits_{x\to\infty} -6 \, {\left(x + 5\right)}^{3} e^{\left(-x - 7\right)}=\answer{0}\]
\end{problem}}

%%%%%%%%%%%%%%%%%%%%%%

\latexProblemContent{
\ifVerboseLocation This is Derivative Compute Question 0051. \\ \fi
\begin{problem}

Find the limit.  Use L'H$\hat{o}$pital's rule where appropriate.

\input{Derivative-Compute-0051.HELP.tex}

\[\lim\limits_{x\to\infty} -8 \, {\left(x + 8\right)} e^{\left(-x + 8\right)}=\answer{0}\]
\end{problem}}

%%%%%%%%%%%%%%%%%%%%%%

\latexProblemContent{
\ifVerboseLocation This is Derivative Compute Question 0051. \\ \fi
\begin{problem}

Find the limit.  Use L'H$\hat{o}$pital's rule where appropriate.

\input{Derivative-Compute-0051.HELP.tex}

\[\lim\limits_{x\to\infty} 5 \, {\left(x - 7\right)}^{3} e^{\left(-x + 5\right)}=\answer{0}\]
\end{problem}}

%%%%%%%%%%%%%%%%%%%%%%

\latexProblemContent{
\ifVerboseLocation This is Derivative Compute Question 0051. \\ \fi
\begin{problem}

Find the limit.  Use L'H$\hat{o}$pital's rule where appropriate.

\input{Derivative-Compute-0051.HELP.tex}

\[\lim\limits_{x\to\infty} 4 \, {\left(x + 6\right)} e^{\left(-x - 7\right)}=\answer{0}\]
\end{problem}}

%%%%%%%%%%%%%%%%%%%%%%

\latexProblemContent{
\ifVerboseLocation This is Derivative Compute Question 0051. \\ \fi
\begin{problem}

Find the limit.  Use L'H$\hat{o}$pital's rule where appropriate.

\input{Derivative-Compute-0051.HELP.tex}

\[\lim\limits_{x\to\infty} 3 \, {\left(x + 4\right)}^{3} e^{\left(-x + 8\right)}=\answer{0}\]
\end{problem}}

%%%%%%%%%%%%%%%%%%%%%%

\latexProblemContent{
\ifVerboseLocation This is Derivative Compute Question 0051. \\ \fi
\begin{problem}

Find the limit.  Use L'H$\hat{o}$pital's rule where appropriate.

\input{Derivative-Compute-0051.HELP.tex}

\[\lim\limits_{x\to\infty} 3 \, {\left(x - 2\right)}^{2} e^{\left(-x + 1\right)}=\answer{0}\]
\end{problem}}

%%%%%%%%%%%%%%%%%%%%%%

\latexProblemContent{
\ifVerboseLocation This is Derivative Compute Question 0051. \\ \fi
\begin{problem}

Find the limit.  Use L'H$\hat{o}$pital's rule where appropriate.

\input{Derivative-Compute-0051.HELP.tex}

\[\lim\limits_{x\to\infty} -4 \, {\left(x - 1\right)} e^{\left(-x - 3\right)}=\answer{0}\]
\end{problem}}

%%%%%%%%%%%%%%%%%%%%%%

\latexProblemContent{
\ifVerboseLocation This is Derivative Compute Question 0051. \\ \fi
\begin{problem}

Find the limit.  Use L'H$\hat{o}$pital's rule where appropriate.

\input{Derivative-Compute-0051.HELP.tex}

\[\lim\limits_{x\to\infty} 4 \, {\left(x + 1\right)}^{3} e^{\left(-x + 7\right)}=\answer{0}\]
\end{problem}}

%%%%%%%%%%%%%%%%%%%%%%

\latexProblemContent{
\ifVerboseLocation This is Derivative Compute Question 0051. \\ \fi
\begin{problem}

Find the limit.  Use L'H$\hat{o}$pital's rule where appropriate.

\input{Derivative-Compute-0051.HELP.tex}

\[\lim\limits_{x\to\infty} 5 \, {\left(x + 8\right)} e^{\left(-x + 3\right)}=\answer{0}\]
\end{problem}}

%%%%%%%%%%%%%%%%%%%%%%

\latexProblemContent{
\ifVerboseLocation This is Derivative Compute Question 0051. \\ \fi
\begin{problem}

Find the limit.  Use L'H$\hat{o}$pital's rule where appropriate.

\input{Derivative-Compute-0051.HELP.tex}

\[\lim\limits_{x\to\infty} 8 \, {\left(x + 6\right)}^{2} e^{\left(-x + 5\right)}=\answer{0}\]
\end{problem}}

%%%%%%%%%%%%%%%%%%%%%%

\latexProblemContent{
\ifVerboseLocation This is Derivative Compute Question 0051. \\ \fi
\begin{problem}

Find the limit.  Use L'H$\hat{o}$pital's rule where appropriate.

\input{Derivative-Compute-0051.HELP.tex}

\[\lim\limits_{x\to\infty} -3 \, {\left(x + 1\right)}^{3} e^{\left(-x - 5\right)}=\answer{0}\]
\end{problem}}

%%%%%%%%%%%%%%%%%%%%%%

\latexProblemContent{
\ifVerboseLocation This is Derivative Compute Question 0051. \\ \fi
\begin{problem}

Find the limit.  Use L'H$\hat{o}$pital's rule where appropriate.

\input{Derivative-Compute-0051.HELP.tex}

\[\lim\limits_{x\to\infty} 3 \, {\left(x - 8\right)} e^{\left(-x - 5\right)}=\answer{0}\]
\end{problem}}

%%%%%%%%%%%%%%%%%%%%%%

\latexProblemContent{
\ifVerboseLocation This is Derivative Compute Question 0051. \\ \fi
\begin{problem}

Find the limit.  Use L'H$\hat{o}$pital's rule where appropriate.

\input{Derivative-Compute-0051.HELP.tex}

\[\lim\limits_{x\to\infty} 6 \, {\left(x - 8\right)}^{3} e^{\left(-x - 6\right)}=\answer{0}\]
\end{problem}}

%%%%%%%%%%%%%%%%%%%%%%

\latexProblemContent{
\ifVerboseLocation This is Derivative Compute Question 0051. \\ \fi
\begin{problem}

Find the limit.  Use L'H$\hat{o}$pital's rule where appropriate.

\input{Derivative-Compute-0051.HELP.tex}

\[\lim\limits_{x\to\infty} 6 \, {\left(x - 6\right)}^{2} e^{\left(-x + 7\right)}=\answer{0}\]
\end{problem}}

%%%%%%%%%%%%%%%%%%%%%%

\latexProblemContent{
\ifVerboseLocation This is Derivative Compute Question 0051. \\ \fi
\begin{problem}

Find the limit.  Use L'H$\hat{o}$pital's rule where appropriate.

\input{Derivative-Compute-0051.HELP.tex}

\[\lim\limits_{x\to\infty} -6 \, {\left(x - 4\right)}^{3} e^{\left(-x - 7\right)}=\answer{0}\]
\end{problem}}

%%%%%%%%%%%%%%%%%%%%%%

\latexProblemContent{
\ifVerboseLocation This is Derivative Compute Question 0051. \\ \fi
\begin{problem}

Find the limit.  Use L'H$\hat{o}$pital's rule where appropriate.

\input{Derivative-Compute-0051.HELP.tex}

\[\lim\limits_{x\to\infty} 8 \, {\left(x + 2\right)}^{2} e^{\left(-x - 8\right)}=\answer{0}\]
\end{problem}}

%%%%%%%%%%%%%%%%%%%%%%

\latexProblemContent{
\ifVerboseLocation This is Derivative Compute Question 0051. \\ \fi
\begin{problem}

Find the limit.  Use L'H$\hat{o}$pital's rule where appropriate.

\input{Derivative-Compute-0051.HELP.tex}

\[\lim\limits_{x\to\infty} 3 \, {\left(x - 6\right)}^{2} e^{\left(-x + 8\right)}=\answer{0}\]
\end{problem}}

%%%%%%%%%%%%%%%%%%%%%%

\latexProblemContent{
\ifVerboseLocation This is Derivative Compute Question 0051. \\ \fi
\begin{problem}

Find the limit.  Use L'H$\hat{o}$pital's rule where appropriate.

\input{Derivative-Compute-0051.HELP.tex}

\[\lim\limits_{x\to\infty} -7 \, {\left(x - 5\right)}^{2} e^{\left(-x + 3\right)}=\answer{0}\]
\end{problem}}

%%%%%%%%%%%%%%%%%%%%%%

\latexProblemContent{
\ifVerboseLocation This is Derivative Compute Question 0051. \\ \fi
\begin{problem}

Find the limit.  Use L'H$\hat{o}$pital's rule where appropriate.

\input{Derivative-Compute-0051.HELP.tex}

\[\lim\limits_{x\to\infty} 3 \, {\left(x + 5\right)} e^{\left(-x - 3\right)}=\answer{0}\]
\end{problem}}

%%%%%%%%%%%%%%%%%%%%%%

\latexProblemContent{
\ifVerboseLocation This is Derivative Compute Question 0051. \\ \fi
\begin{problem}

Find the limit.  Use L'H$\hat{o}$pital's rule where appropriate.

\input{Derivative-Compute-0051.HELP.tex}

\[\lim\limits_{x\to\infty} -2 \, {\left(x + 6\right)}^{3} e^{\left(-x - 5\right)}=\answer{0}\]
\end{problem}}

%%%%%%%%%%%%%%%%%%%%%%

\latexProblemContent{
\ifVerboseLocation This is Derivative Compute Question 0051. \\ \fi
\begin{problem}

Find the limit.  Use L'H$\hat{o}$pital's rule where appropriate.

\input{Derivative-Compute-0051.HELP.tex}

\[\lim\limits_{x\to\infty} 7 \, {\left(x - 1\right)}^{3} e^{\left(-x - 3\right)}=\answer{0}\]
\end{problem}}

%%%%%%%%%%%%%%%%%%%%%%

\latexProblemContent{
\ifVerboseLocation This is Derivative Compute Question 0051. \\ \fi
\begin{problem}

Find the limit.  Use L'H$\hat{o}$pital's rule where appropriate.

\input{Derivative-Compute-0051.HELP.tex}

\[\lim\limits_{x\to\infty} -{\left(x - 7\right)}^{3} e^{\left(-x - 4\right)}=\answer{0}\]
\end{problem}}

%%%%%%%%%%%%%%%%%%%%%%

\latexProblemContent{
\ifVerboseLocation This is Derivative Compute Question 0051. \\ \fi
\begin{problem}

Find the limit.  Use L'H$\hat{o}$pital's rule where appropriate.

\input{Derivative-Compute-0051.HELP.tex}

\[\lim\limits_{x\to\infty} -5 \, {\left(x + 1\right)}^{2} e^{\left(-x + 5\right)}=\answer{0}\]
\end{problem}}

%%%%%%%%%%%%%%%%%%%%%%

\latexProblemContent{
\ifVerboseLocation This is Derivative Compute Question 0051. \\ \fi
\begin{problem}

Find the limit.  Use L'H$\hat{o}$pital's rule where appropriate.

\input{Derivative-Compute-0051.HELP.tex}

\[\lim\limits_{x\to\infty} 7 \, {\left(x + 1\right)} e^{\left(-x + 7\right)}=\answer{0}\]
\end{problem}}

%%%%%%%%%%%%%%%%%%%%%%

\latexProblemContent{
\ifVerboseLocation This is Derivative Compute Question 0051. \\ \fi
\begin{problem}

Find the limit.  Use L'H$\hat{o}$pital's rule where appropriate.

\input{Derivative-Compute-0051.HELP.tex}

\[\lim\limits_{x\to\infty} 4 \, {\left(x + 4\right)}^{3} e^{\left(-x - 2\right)}=\answer{0}\]
\end{problem}}

%%%%%%%%%%%%%%%%%%%%%%

\latexProblemContent{
\ifVerboseLocation This is Derivative Compute Question 0051. \\ \fi
\begin{problem}

Find the limit.  Use L'H$\hat{o}$pital's rule where appropriate.

\input{Derivative-Compute-0051.HELP.tex}

\[\lim\limits_{x\to\infty} -6 \, {\left(x - 7\right)}^{3} e^{\left(-x + 2\right)}=\answer{0}\]
\end{problem}}

%%%%%%%%%%%%%%%%%%%%%%

\latexProblemContent{
\ifVerboseLocation This is Derivative Compute Question 0051. \\ \fi
\begin{problem}

Find the limit.  Use L'H$\hat{o}$pital's rule where appropriate.

\input{Derivative-Compute-0051.HELP.tex}

\[\lim\limits_{x\to\infty} 3 \, {\left(x + 2\right)}^{2} e^{\left(-x - 2\right)}=\answer{0}\]
\end{problem}}

%%%%%%%%%%%%%%%%%%%%%%

\latexProblemContent{
\ifVerboseLocation This is Derivative Compute Question 0051. \\ \fi
\begin{problem}

Find the limit.  Use L'H$\hat{o}$pital's rule where appropriate.

\input{Derivative-Compute-0051.HELP.tex}

\[\lim\limits_{x\to\infty} -2 \, {\left(x + 6\right)}^{3} e^{\left(-x - 7\right)}=\answer{0}\]
\end{problem}}

%%%%%%%%%%%%%%%%%%%%%%

\latexProblemContent{
\ifVerboseLocation This is Derivative Compute Question 0051. \\ \fi
\begin{problem}

Find the limit.  Use L'H$\hat{o}$pital's rule where appropriate.

\input{Derivative-Compute-0051.HELP.tex}

\[\lim\limits_{x\to\infty} -5 \, {\left(x - 2\right)}^{2} e^{\left(-x + 7\right)}=\answer{0}\]
\end{problem}}

%%%%%%%%%%%%%%%%%%%%%%

\latexProblemContent{
\ifVerboseLocation This is Derivative Compute Question 0051. \\ \fi
\begin{problem}

Find the limit.  Use L'H$\hat{o}$pital's rule where appropriate.

\input{Derivative-Compute-0051.HELP.tex}

\[\lim\limits_{x\to\infty} 4 \, {\left(x + 3\right)}^{2} e^{\left(-x + 8\right)}=\answer{0}\]
\end{problem}}

%%%%%%%%%%%%%%%%%%%%%%

\latexProblemContent{
\ifVerboseLocation This is Derivative Compute Question 0051. \\ \fi
\begin{problem}

Find the limit.  Use L'H$\hat{o}$pital's rule where appropriate.

\input{Derivative-Compute-0051.HELP.tex}

\[\lim\limits_{x\to\infty} 2 \, {\left(x + 7\right)}^{3} e^{\left(-x + 3\right)}=\answer{0}\]
\end{problem}}

%%%%%%%%%%%%%%%%%%%%%%

\latexProblemContent{
\ifVerboseLocation This is Derivative Compute Question 0051. \\ \fi
\begin{problem}

Find the limit.  Use L'H$\hat{o}$pital's rule where appropriate.

\input{Derivative-Compute-0051.HELP.tex}

\[\lim\limits_{x\to\infty} 4 \, {\left(x - 2\right)}^{2} e^{\left(-x - 1\right)}=\answer{0}\]
\end{problem}}

%%%%%%%%%%%%%%%%%%%%%%

\latexProblemContent{
\ifVerboseLocation This is Derivative Compute Question 0051. \\ \fi
\begin{problem}

Find the limit.  Use L'H$\hat{o}$pital's rule where appropriate.

\input{Derivative-Compute-0051.HELP.tex}

\[\lim\limits_{x\to\infty} -8 \, {\left(x + 3\right)} e^{\left(-x + 7\right)}=\answer{0}\]
\end{problem}}

%%%%%%%%%%%%%%%%%%%%%%

\latexProblemContent{
\ifVerboseLocation This is Derivative Compute Question 0051. \\ \fi
\begin{problem}

Find the limit.  Use L'H$\hat{o}$pital's rule where appropriate.

\input{Derivative-Compute-0051.HELP.tex}

\[\lim\limits_{x\to\infty} -{\left(x + 7\right)}^{2} e^{\left(-x + 4\right)}=\answer{0}\]
\end{problem}}

%%%%%%%%%%%%%%%%%%%%%%

\latexProblemContent{
\ifVerboseLocation This is Derivative Compute Question 0051. \\ \fi
\begin{problem}

Find the limit.  Use L'H$\hat{o}$pital's rule where appropriate.

\input{Derivative-Compute-0051.HELP.tex}

\[\lim\limits_{x\to\infty} -7 \, {\left(x - 1\right)}^{2} e^{\left(-x + 8\right)}=\answer{0}\]
\end{problem}}

%%%%%%%%%%%%%%%%%%%%%%

\latexProblemContent{
\ifVerboseLocation This is Derivative Compute Question 0051. \\ \fi
\begin{problem}

Find the limit.  Use L'H$\hat{o}$pital's rule where appropriate.

\input{Derivative-Compute-0051.HELP.tex}

\[\lim\limits_{x\to\infty} 3 \, {\left(x - 8\right)} e^{\left(-x - 7\right)}=\answer{0}\]
\end{problem}}

%%%%%%%%%%%%%%%%%%%%%%

\latexProblemContent{
\ifVerboseLocation This is Derivative Compute Question 0051. \\ \fi
\begin{problem}

Find the limit.  Use L'H$\hat{o}$pital's rule where appropriate.

\input{Derivative-Compute-0051.HELP.tex}

\[\lim\limits_{x\to\infty} -7 \, {\left(x - 7\right)} e^{\left(-x - 3\right)}=\answer{0}\]
\end{problem}}

%%%%%%%%%%%%%%%%%%%%%%

\latexProblemContent{
\ifVerboseLocation This is Derivative Compute Question 0051. \\ \fi
\begin{problem}

Find the limit.  Use L'H$\hat{o}$pital's rule where appropriate.

\input{Derivative-Compute-0051.HELP.tex}

\[\lim\limits_{x\to\infty} -2 \, {\left(x + 7\right)} e^{\left(-x + 6\right)}=\answer{0}\]
\end{problem}}

%%%%%%%%%%%%%%%%%%%%%%

\latexProblemContent{
\ifVerboseLocation This is Derivative Compute Question 0051. \\ \fi
\begin{problem}

Find the limit.  Use L'H$\hat{o}$pital's rule where appropriate.

\input{Derivative-Compute-0051.HELP.tex}

\[\lim\limits_{x\to\infty} -4 \, {\left(x - 2\right)}^{3} e^{\left(-x - 1\right)}=\answer{0}\]
\end{problem}}

%%%%%%%%%%%%%%%%%%%%%%

\latexProblemContent{
\ifVerboseLocation This is Derivative Compute Question 0051. \\ \fi
\begin{problem}

Find the limit.  Use L'H$\hat{o}$pital's rule where appropriate.

\input{Derivative-Compute-0051.HELP.tex}

\[\lim\limits_{x\to\infty} 6 \, {\left(x + 1\right)}^{3} e^{\left(-x - 3\right)}=\answer{0}\]
\end{problem}}

%%%%%%%%%%%%%%%%%%%%%%

\latexProblemContent{
\ifVerboseLocation This is Derivative Compute Question 0051. \\ \fi
\begin{problem}

Find the limit.  Use L'H$\hat{o}$pital's rule where appropriate.

\input{Derivative-Compute-0051.HELP.tex}

\[\lim\limits_{x\to\infty} 4 \, {\left(x + 4\right)}^{2} e^{\left(-x + 7\right)}=\answer{0}\]
\end{problem}}

%%%%%%%%%%%%%%%%%%%%%%

\latexProblemContent{
\ifVerboseLocation This is Derivative Compute Question 0051. \\ \fi
\begin{problem}

Find the limit.  Use L'H$\hat{o}$pital's rule where appropriate.

\input{Derivative-Compute-0051.HELP.tex}

\[\lim\limits_{x\to\infty} -{\left(x + 8\right)}^{2} e^{\left(-x - 4\right)}=\answer{0}\]
\end{problem}}

%%%%%%%%%%%%%%%%%%%%%%

\latexProblemContent{
\ifVerboseLocation This is Derivative Compute Question 0051. \\ \fi
\begin{problem}

Find the limit.  Use L'H$\hat{o}$pital's rule where appropriate.

\input{Derivative-Compute-0051.HELP.tex}

\[\lim\limits_{x\to\infty} 6 \, {\left(x + 3\right)}^{3} e^{\left(-x - 4\right)}=\answer{0}\]
\end{problem}}

%%%%%%%%%%%%%%%%%%%%%%

\latexProblemContent{
\ifVerboseLocation This is Derivative Compute Question 0051. \\ \fi
\begin{problem}

Find the limit.  Use L'H$\hat{o}$pital's rule where appropriate.

\input{Derivative-Compute-0051.HELP.tex}

\[\lim\limits_{x\to\infty} 7 \, {\left(x + 5\right)}^{3} e^{\left(-x - 5\right)}=\answer{0}\]
\end{problem}}

%%%%%%%%%%%%%%%%%%%%%%

\latexProblemContent{
\ifVerboseLocation This is Derivative Compute Question 0051. \\ \fi
\begin{problem}

Find the limit.  Use L'H$\hat{o}$pital's rule where appropriate.

\input{Derivative-Compute-0051.HELP.tex}

\[\lim\limits_{x\to\infty} -{\left(x + 5\right)}^{2} e^{\left(-x - 2\right)}=\answer{0}\]
\end{problem}}

%%%%%%%%%%%%%%%%%%%%%%

\latexProblemContent{
\ifVerboseLocation This is Derivative Compute Question 0051. \\ \fi
\begin{problem}

Find the limit.  Use L'H$\hat{o}$pital's rule where appropriate.

\input{Derivative-Compute-0051.HELP.tex}

\[\lim\limits_{x\to\infty} 5 \, {\left(x + 1\right)}^{2} e^{\left(-x - 7\right)}=\answer{0}\]
\end{problem}}

%%%%%%%%%%%%%%%%%%%%%%

\latexProblemContent{
\ifVerboseLocation This is Derivative Compute Question 0051. \\ \fi
\begin{problem}

Find the limit.  Use L'H$\hat{o}$pital's rule where appropriate.

\input{Derivative-Compute-0051.HELP.tex}

\[\lim\limits_{x\to\infty} 8 \, {\left(x + 3\right)}^{2} e^{\left(-x + 1\right)}=\answer{0}\]
\end{problem}}

%%%%%%%%%%%%%%%%%%%%%%

\latexProblemContent{
\ifVerboseLocation This is Derivative Compute Question 0051. \\ \fi
\begin{problem}

Find the limit.  Use L'H$\hat{o}$pital's rule where appropriate.

\input{Derivative-Compute-0051.HELP.tex}

\[\lim\limits_{x\to\infty} 3 \, {\left(x + 1\right)}^{2} e^{\left(-x - 4\right)}=\answer{0}\]
\end{problem}}

%%%%%%%%%%%%%%%%%%%%%%

\latexProblemContent{
\ifVerboseLocation This is Derivative Compute Question 0051. \\ \fi
\begin{problem}

Find the limit.  Use L'H$\hat{o}$pital's rule where appropriate.

\input{Derivative-Compute-0051.HELP.tex}

\[\lim\limits_{x\to\infty} -{\left(x - 4\right)}^{2} e^{\left(-x - 6\right)}=\answer{0}\]
\end{problem}}

%%%%%%%%%%%%%%%%%%%%%%

\latexProblemContent{
\ifVerboseLocation This is Derivative Compute Question 0051. \\ \fi
\begin{problem}

Find the limit.  Use L'H$\hat{o}$pital's rule where appropriate.

\input{Derivative-Compute-0051.HELP.tex}

\[\lim\limits_{x\to\infty} -5 \, {\left(x - 3\right)}^{2} e^{\left(-x - 2\right)}=\answer{0}\]
\end{problem}}

%%%%%%%%%%%%%%%%%%%%%%

\latexProblemContent{
\ifVerboseLocation This is Derivative Compute Question 0051. \\ \fi
\begin{problem}

Find the limit.  Use L'H$\hat{o}$pital's rule where appropriate.

\input{Derivative-Compute-0051.HELP.tex}

\[\lim\limits_{x\to\infty} -7 \, {\left(x - 5\right)}^{3} e^{\left(-x + 4\right)}=\answer{0}\]
\end{problem}}

%%%%%%%%%%%%%%%%%%%%%%

\latexProblemContent{
\ifVerboseLocation This is Derivative Compute Question 0051. \\ \fi
\begin{problem}

Find the limit.  Use L'H$\hat{o}$pital's rule where appropriate.

\input{Derivative-Compute-0051.HELP.tex}

\[\lim\limits_{x\to\infty} 4 \, {\left(x + 1\right)}^{3} e^{\left(-x - 3\right)}=\answer{0}\]
\end{problem}}

%%%%%%%%%%%%%%%%%%%%%%

\latexProblemContent{
\ifVerboseLocation This is Derivative Compute Question 0051. \\ \fi
\begin{problem}

Find the limit.  Use L'H$\hat{o}$pital's rule where appropriate.

\input{Derivative-Compute-0051.HELP.tex}

\[\lim\limits_{x\to\infty} 4 \, {\left(x + 4\right)} e^{\left(-x - 2\right)}=\answer{0}\]
\end{problem}}

%%%%%%%%%%%%%%%%%%%%%%

\latexProblemContent{
\ifVerboseLocation This is Derivative Compute Question 0051. \\ \fi
\begin{problem}

Find the limit.  Use L'H$\hat{o}$pital's rule where appropriate.

\input{Derivative-Compute-0051.HELP.tex}

\[\lim\limits_{x\to\infty} -5 \, {\left(x + 4\right)}^{3} e^{\left(-x + 1\right)}=\answer{0}\]
\end{problem}}

%%%%%%%%%%%%%%%%%%%%%%

\latexProblemContent{
\ifVerboseLocation This is Derivative Compute Question 0051. \\ \fi
\begin{problem}

Find the limit.  Use L'H$\hat{o}$pital's rule where appropriate.

\input{Derivative-Compute-0051.HELP.tex}

\[\lim\limits_{x\to\infty} -4 \, {\left(x - 3\right)}^{2} e^{\left(-x - 5\right)}=\answer{0}\]
\end{problem}}

%%%%%%%%%%%%%%%%%%%%%%

\latexProblemContent{
\ifVerboseLocation This is Derivative Compute Question 0051. \\ \fi
\begin{problem}

Find the limit.  Use L'H$\hat{o}$pital's rule where appropriate.

\input{Derivative-Compute-0051.HELP.tex}

\[\lim\limits_{x\to\infty} {\left(x - 6\right)} e^{\left(-x + 8\right)}=\answer{0}\]
\end{problem}}

%%%%%%%%%%%%%%%%%%%%%%

\latexProblemContent{
\ifVerboseLocation This is Derivative Compute Question 0051. \\ \fi
\begin{problem}

Find the limit.  Use L'H$\hat{o}$pital's rule where appropriate.

\input{Derivative-Compute-0051.HELP.tex}

\[\lim\limits_{x\to\infty} -6 \, {\left(x + 1\right)}^{2} e^{\left(-x - 2\right)}=\answer{0}\]
\end{problem}}

%%%%%%%%%%%%%%%%%%%%%%

\latexProblemContent{
\ifVerboseLocation This is Derivative Compute Question 0051. \\ \fi
\begin{problem}

Find the limit.  Use L'H$\hat{o}$pital's rule where appropriate.

\input{Derivative-Compute-0051.HELP.tex}

\[\lim\limits_{x\to\infty} 2 \, {\left(x + 1\right)} e^{\left(-x - 5\right)}=\answer{0}\]
\end{problem}}

%%%%%%%%%%%%%%%%%%%%%%

\latexProblemContent{
\ifVerboseLocation This is Derivative Compute Question 0051. \\ \fi
\begin{problem}

Find the limit.  Use L'H$\hat{o}$pital's rule where appropriate.

\input{Derivative-Compute-0051.HELP.tex}

\[\lim\limits_{x\to\infty} 2 \, {\left(x + 8\right)}^{2} e^{\left(-x + 6\right)}=\answer{0}\]
\end{problem}}

%%%%%%%%%%%%%%%%%%%%%%

\latexProblemContent{
\ifVerboseLocation This is Derivative Compute Question 0051. \\ \fi
\begin{problem}

Find the limit.  Use L'H$\hat{o}$pital's rule where appropriate.

\input{Derivative-Compute-0051.HELP.tex}

\[\lim\limits_{x\to\infty} -{\left(x - 6\right)} e^{\left(-x + 6\right)}=\answer{0}\]
\end{problem}}

%%%%%%%%%%%%%%%%%%%%%%

\latexProblemContent{
\ifVerboseLocation This is Derivative Compute Question 0051. \\ \fi
\begin{problem}

Find the limit.  Use L'H$\hat{o}$pital's rule where appropriate.

\input{Derivative-Compute-0051.HELP.tex}

\[\lim\limits_{x\to\infty} 3 \, {\left(x + 8\right)}^{2} e^{\left(-x - 5\right)}=\answer{0}\]
\end{problem}}

%%%%%%%%%%%%%%%%%%%%%%

\latexProblemContent{
\ifVerboseLocation This is Derivative Compute Question 0051. \\ \fi
\begin{problem}

Find the limit.  Use L'H$\hat{o}$pital's rule where appropriate.

\input{Derivative-Compute-0051.HELP.tex}

\[\lim\limits_{x\to\infty} {\left(x + 6\right)} e^{\left(-x - 3\right)}=\answer{0}\]
\end{problem}}

%%%%%%%%%%%%%%%%%%%%%%

\latexProblemContent{
\ifVerboseLocation This is Derivative Compute Question 0051. \\ \fi
\begin{problem}

Find the limit.  Use L'H$\hat{o}$pital's rule where appropriate.

\input{Derivative-Compute-0051.HELP.tex}

\[\lim\limits_{x\to\infty} -3 \, {\left(x + 1\right)} e^{\left(-x - 2\right)}=\answer{0}\]
\end{problem}}

%%%%%%%%%%%%%%%%%%%%%%

\latexProblemContent{
\ifVerboseLocation This is Derivative Compute Question 0051. \\ \fi
\begin{problem}

Find the limit.  Use L'H$\hat{o}$pital's rule where appropriate.

\input{Derivative-Compute-0051.HELP.tex}

\[\lim\limits_{x\to\infty} -5 \, {\left(x - 8\right)}^{3} e^{\left(-x + 5\right)}=\answer{0}\]
\end{problem}}

%%%%%%%%%%%%%%%%%%%%%%

\latexProblemContent{
\ifVerboseLocation This is Derivative Compute Question 0051. \\ \fi
\begin{problem}

Find the limit.  Use L'H$\hat{o}$pital's rule where appropriate.

\input{Derivative-Compute-0051.HELP.tex}

\[\lim\limits_{x\to\infty} -4 \, {\left(x + 3\right)} e^{\left(-x - 8\right)}=\answer{0}\]
\end{problem}}

%%%%%%%%%%%%%%%%%%%%%%

\latexProblemContent{
\ifVerboseLocation This is Derivative Compute Question 0051. \\ \fi
\begin{problem}

Find the limit.  Use L'H$\hat{o}$pital's rule where appropriate.

\input{Derivative-Compute-0051.HELP.tex}

\[\lim\limits_{x\to\infty} {\left(x - 2\right)}^{2} e^{\left(-x - 5\right)}=\answer{0}\]
\end{problem}}

%%%%%%%%%%%%%%%%%%%%%%

\latexProblemContent{
\ifVerboseLocation This is Derivative Compute Question 0051. \\ \fi
\begin{problem}

Find the limit.  Use L'H$\hat{o}$pital's rule where appropriate.

\input{Derivative-Compute-0051.HELP.tex}

\[\lim\limits_{x\to\infty} 3 \, {\left(x + 6\right)}^{2} e^{\left(-x + 1\right)}=\answer{0}\]
\end{problem}}

%%%%%%%%%%%%%%%%%%%%%%

\latexProblemContent{
\ifVerboseLocation This is Derivative Compute Question 0051. \\ \fi
\begin{problem}

Find the limit.  Use L'H$\hat{o}$pital's rule where appropriate.

\input{Derivative-Compute-0051.HELP.tex}

\[\lim\limits_{x\to\infty} -2 \, {\left(x - 7\right)}^{3} e^{\left(-x - 6\right)}=\answer{0}\]
\end{problem}}

%%%%%%%%%%%%%%%%%%%%%%

\latexProblemContent{
\ifVerboseLocation This is Derivative Compute Question 0051. \\ \fi
\begin{problem}

Find the limit.  Use L'H$\hat{o}$pital's rule where appropriate.

\input{Derivative-Compute-0051.HELP.tex}

\[\lim\limits_{x\to\infty} 3 \, {\left(x + 5\right)}^{2} e^{\left(-x - 6\right)}=\answer{0}\]
\end{problem}}

%%%%%%%%%%%%%%%%%%%%%%

\latexProblemContent{
\ifVerboseLocation This is Derivative Compute Question 0051. \\ \fi
\begin{problem}

Find the limit.  Use L'H$\hat{o}$pital's rule where appropriate.

\input{Derivative-Compute-0051.HELP.tex}

\[\lim\limits_{x\to\infty} -4 \, {\left(x - 8\right)}^{2} e^{\left(-x - 6\right)}=\answer{0}\]
\end{problem}}

%%%%%%%%%%%%%%%%%%%%%%

\latexProblemContent{
\ifVerboseLocation This is Derivative Compute Question 0051. \\ \fi
\begin{problem}

Find the limit.  Use L'H$\hat{o}$pital's rule where appropriate.

\input{Derivative-Compute-0051.HELP.tex}

\[\lim\limits_{x\to\infty} -7 \, {\left(x + 3\right)}^{3} e^{\left(-x + 7\right)}=\answer{0}\]
\end{problem}}

%%%%%%%%%%%%%%%%%%%%%%

\latexProblemContent{
\ifVerboseLocation This is Derivative Compute Question 0051. \\ \fi
\begin{problem}

Find the limit.  Use L'H$\hat{o}$pital's rule where appropriate.

\input{Derivative-Compute-0051.HELP.tex}

\[\lim\limits_{x\to\infty} -4 \, {\left(x + 1\right)}^{3} e^{\left(-x - 7\right)}=\answer{0}\]
\end{problem}}

%%%%%%%%%%%%%%%%%%%%%%

\latexProblemContent{
\ifVerboseLocation This is Derivative Compute Question 0051. \\ \fi
\begin{problem}

Find the limit.  Use L'H$\hat{o}$pital's rule where appropriate.

\input{Derivative-Compute-0051.HELP.tex}

\[\lim\limits_{x\to\infty} -8 \, {\left(x - 4\right)}^{3} e^{\left(-x + 5\right)}=\answer{0}\]
\end{problem}}

%%%%%%%%%%%%%%%%%%%%%%

\latexProblemContent{
\ifVerboseLocation This is Derivative Compute Question 0051. \\ \fi
\begin{problem}

Find the limit.  Use L'H$\hat{o}$pital's rule where appropriate.

\input{Derivative-Compute-0051.HELP.tex}

\[\lim\limits_{x\to\infty} 4 \, {\left(x - 3\right)}^{2} e^{\left(-x + 3\right)}=\answer{0}\]
\end{problem}}

%%%%%%%%%%%%%%%%%%%%%%

\latexProblemContent{
\ifVerboseLocation This is Derivative Compute Question 0051. \\ \fi
\begin{problem}

Find the limit.  Use L'H$\hat{o}$pital's rule where appropriate.

\input{Derivative-Compute-0051.HELP.tex}

\[\lim\limits_{x\to\infty} {\left(x + 6\right)}^{2} e^{\left(-x + 8\right)}=\answer{0}\]
\end{problem}}

%%%%%%%%%%%%%%%%%%%%%%

\latexProblemContent{
\ifVerboseLocation This is Derivative Compute Question 0051. \\ \fi
\begin{problem}

Find the limit.  Use L'H$\hat{o}$pital's rule where appropriate.

\input{Derivative-Compute-0051.HELP.tex}

\[\lim\limits_{x\to\infty} -{\left(x + 4\right)}^{2} e^{\left(-x - 1\right)}=\answer{0}\]
\end{problem}}

%%%%%%%%%%%%%%%%%%%%%%

\latexProblemContent{
\ifVerboseLocation This is Derivative Compute Question 0051. \\ \fi
\begin{problem}

Find the limit.  Use L'H$\hat{o}$pital's rule where appropriate.

\input{Derivative-Compute-0051.HELP.tex}

\[\lim\limits_{x\to\infty} {\left(x + 4\right)} e^{\left(-x + 8\right)}=\answer{0}\]
\end{problem}}

%%%%%%%%%%%%%%%%%%%%%%

\latexProblemContent{
\ifVerboseLocation This is Derivative Compute Question 0051. \\ \fi
\begin{problem}

Find the limit.  Use L'H$\hat{o}$pital's rule where appropriate.

\input{Derivative-Compute-0051.HELP.tex}

\[\lim\limits_{x\to\infty} -8 \, {\left(x + 7\right)}^{3} e^{\left(-x + 4\right)}=\answer{0}\]
\end{problem}}

%%%%%%%%%%%%%%%%%%%%%%

\latexProblemContent{
\ifVerboseLocation This is Derivative Compute Question 0051. \\ \fi
\begin{problem}

Find the limit.  Use L'H$\hat{o}$pital's rule where appropriate.

\input{Derivative-Compute-0051.HELP.tex}

\[\lim\limits_{x\to\infty} -3 \, {\left(x - 6\right)}^{3} e^{\left(-x - 2\right)}=\answer{0}\]
\end{problem}}

%%%%%%%%%%%%%%%%%%%%%%

\latexProblemContent{
\ifVerboseLocation This is Derivative Compute Question 0051. \\ \fi
\begin{problem}

Find the limit.  Use L'H$\hat{o}$pital's rule where appropriate.

\input{Derivative-Compute-0051.HELP.tex}

\[\lim\limits_{x\to\infty} -{\left(x + 8\right)}^{2} e^{\left(-x + 5\right)}=\answer{0}\]
\end{problem}}

%%%%%%%%%%%%%%%%%%%%%%

\latexProblemContent{
\ifVerboseLocation This is Derivative Compute Question 0051. \\ \fi
\begin{problem}

Find the limit.  Use L'H$\hat{o}$pital's rule where appropriate.

\input{Derivative-Compute-0051.HELP.tex}

\[\lim\limits_{x\to\infty} 3 \, {\left(x - 1\right)}^{2} e^{\left(-x + 3\right)}=\answer{0}\]
\end{problem}}

%%%%%%%%%%%%%%%%%%%%%%

\latexProblemContent{
\ifVerboseLocation This is Derivative Compute Question 0051. \\ \fi
\begin{problem}

Find the limit.  Use L'H$\hat{o}$pital's rule where appropriate.

\input{Derivative-Compute-0051.HELP.tex}

\[\lim\limits_{x\to\infty} 4 \, {\left(x + 5\right)}^{3} e^{\left(-x - 2\right)}=\answer{0}\]
\end{problem}}

%%%%%%%%%%%%%%%%%%%%%%

\latexProblemContent{
\ifVerboseLocation This is Derivative Compute Question 0051. \\ \fi
\begin{problem}

Find the limit.  Use L'H$\hat{o}$pital's rule where appropriate.

\input{Derivative-Compute-0051.HELP.tex}

\[\lim\limits_{x\to\infty} -6 \, {\left(x - 8\right)} e^{\left(-x - 7\right)}=\answer{0}\]
\end{problem}}

%%%%%%%%%%%%%%%%%%%%%%

\latexProblemContent{
\ifVerboseLocation This is Derivative Compute Question 0051. \\ \fi
\begin{problem}

Find the limit.  Use L'H$\hat{o}$pital's rule where appropriate.

\input{Derivative-Compute-0051.HELP.tex}

\[\lim\limits_{x\to\infty} -{\left(x - 8\right)}^{2} e^{\left(-x + 2\right)}=\answer{0}\]
\end{problem}}

%%%%%%%%%%%%%%%%%%%%%%

\latexProblemContent{
\ifVerboseLocation This is Derivative Compute Question 0051. \\ \fi
\begin{problem}

Find the limit.  Use L'H$\hat{o}$pital's rule where appropriate.

\input{Derivative-Compute-0051.HELP.tex}

\[\lim\limits_{x\to\infty} -3 \, {\left(x + 1\right)} e^{\left(-x + 3\right)}=\answer{0}\]
\end{problem}}

%%%%%%%%%%%%%%%%%%%%%%

\latexProblemContent{
\ifVerboseLocation This is Derivative Compute Question 0051. \\ \fi
\begin{problem}

Find the limit.  Use L'H$\hat{o}$pital's rule where appropriate.

\input{Derivative-Compute-0051.HELP.tex}

\[\lim\limits_{x\to\infty} 6 \, {\left(x + 8\right)} e^{\left(-x - 8\right)}=\answer{0}\]
\end{problem}}

%%%%%%%%%%%%%%%%%%%%%%

\latexProblemContent{
\ifVerboseLocation This is Derivative Compute Question 0051. \\ \fi
\begin{problem}

Find the limit.  Use L'H$\hat{o}$pital's rule where appropriate.

\input{Derivative-Compute-0051.HELP.tex}

\[\lim\limits_{x\to\infty} -3 \, {\left(x - 3\right)}^{2} e^{\left(-x + 6\right)}=\answer{0}\]
\end{problem}}

%%%%%%%%%%%%%%%%%%%%%%

\latexProblemContent{
\ifVerboseLocation This is Derivative Compute Question 0051. \\ \fi
\begin{problem}

Find the limit.  Use L'H$\hat{o}$pital's rule where appropriate.

\input{Derivative-Compute-0051.HELP.tex}

\[\lim\limits_{x\to\infty} -4 \, {\left(x - 3\right)} e^{\left(-x + 6\right)}=\answer{0}\]
\end{problem}}

%%%%%%%%%%%%%%%%%%%%%%

\latexProblemContent{
\ifVerboseLocation This is Derivative Compute Question 0051. \\ \fi
\begin{problem}

Find the limit.  Use L'H$\hat{o}$pital's rule where appropriate.

\input{Derivative-Compute-0051.HELP.tex}

\[\lim\limits_{x\to\infty} -4 \, {\left(x - 5\right)} e^{\left(-x - 4\right)}=\answer{0}\]
\end{problem}}

%%%%%%%%%%%%%%%%%%%%%%

\latexProblemContent{
\ifVerboseLocation This is Derivative Compute Question 0051. \\ \fi
\begin{problem}

Find the limit.  Use L'H$\hat{o}$pital's rule where appropriate.

\input{Derivative-Compute-0051.HELP.tex}

\[\lim\limits_{x\to\infty} 5 \, {\left(x - 8\right)}^{3} e^{\left(-x + 6\right)}=\answer{0}\]
\end{problem}}

%%%%%%%%%%%%%%%%%%%%%%

\latexProblemContent{
\ifVerboseLocation This is Derivative Compute Question 0051. \\ \fi
\begin{problem}

Find the limit.  Use L'H$\hat{o}$pital's rule where appropriate.

\input{Derivative-Compute-0051.HELP.tex}

\[\lim\limits_{x\to\infty} 4 \, {\left(x - 8\right)} e^{\left(-x - 7\right)}=\answer{0}\]
\end{problem}}

%%%%%%%%%%%%%%%%%%%%%%

\latexProblemContent{
\ifVerboseLocation This is Derivative Compute Question 0051. \\ \fi
\begin{problem}

Find the limit.  Use L'H$\hat{o}$pital's rule where appropriate.

\input{Derivative-Compute-0051.HELP.tex}

\[\lim\limits_{x\to\infty} 6 \, {\left(x - 3\right)} e^{\left(-x + 3\right)}=\answer{0}\]
\end{problem}}

%%%%%%%%%%%%%%%%%%%%%%

\latexProblemContent{
\ifVerboseLocation This is Derivative Compute Question 0051. \\ \fi
\begin{problem}

Find the limit.  Use L'H$\hat{o}$pital's rule where appropriate.

\input{Derivative-Compute-0051.HELP.tex}

\[\lim\limits_{x\to\infty} -6 \, {\left(x - 5\right)} e^{\left(-x - 2\right)}=\answer{0}\]
\end{problem}}

%%%%%%%%%%%%%%%%%%%%%%

\latexProblemContent{
\ifVerboseLocation This is Derivative Compute Question 0051. \\ \fi
\begin{problem}

Find the limit.  Use L'H$\hat{o}$pital's rule where appropriate.

\input{Derivative-Compute-0051.HELP.tex}

\[\lim\limits_{x\to\infty} {\left(x + 3\right)}^{3} e^{\left(-x + 3\right)}=\answer{0}\]
\end{problem}}

%%%%%%%%%%%%%%%%%%%%%%

\latexProblemContent{
\ifVerboseLocation This is Derivative Compute Question 0051. \\ \fi
\begin{problem}

Find the limit.  Use L'H$\hat{o}$pital's rule where appropriate.

\input{Derivative-Compute-0051.HELP.tex}

\[\lim\limits_{x\to\infty} 5 \, {\left(x + 6\right)}^{2} e^{\left(-x + 7\right)}=\answer{0}\]
\end{problem}}

%%%%%%%%%%%%%%%%%%%%%%

\latexProblemContent{
\ifVerboseLocation This is Derivative Compute Question 0051. \\ \fi
\begin{problem}

Find the limit.  Use L'H$\hat{o}$pital's rule where appropriate.

\input{Derivative-Compute-0051.HELP.tex}

\[\lim\limits_{x\to\infty} -8 \, {\left(x - 6\right)}^{2} e^{\left(-x - 8\right)}=\answer{0}\]
\end{problem}}

%%%%%%%%%%%%%%%%%%%%%%

\latexProblemContent{
\ifVerboseLocation This is Derivative Compute Question 0051. \\ \fi
\begin{problem}

Find the limit.  Use L'H$\hat{o}$pital's rule where appropriate.

\input{Derivative-Compute-0051.HELP.tex}

\[\lim\limits_{x\to\infty} -3 \, {\left(x - 3\right)} e^{\left(-x + 7\right)}=\answer{0}\]
\end{problem}}

%%%%%%%%%%%%%%%%%%%%%%

\latexProblemContent{
\ifVerboseLocation This is Derivative Compute Question 0051. \\ \fi
\begin{problem}

Find the limit.  Use L'H$\hat{o}$pital's rule where appropriate.

\input{Derivative-Compute-0051.HELP.tex}

\[\lim\limits_{x\to\infty} 6 \, {\left(x + 1\right)}^{2} e^{\left(-x - 3\right)}=\answer{0}\]
\end{problem}}

%%%%%%%%%%%%%%%%%%%%%%

\latexProblemContent{
\ifVerboseLocation This is Derivative Compute Question 0051. \\ \fi
\begin{problem}

Find the limit.  Use L'H$\hat{o}$pital's rule where appropriate.

\input{Derivative-Compute-0051.HELP.tex}

\[\lim\limits_{x\to\infty} 3 \, {\left(x - 5\right)} e^{\left(-x + 5\right)}=\answer{0}\]
\end{problem}}

%%%%%%%%%%%%%%%%%%%%%%

\latexProblemContent{
\ifVerboseLocation This is Derivative Compute Question 0051. \\ \fi
\begin{problem}

Find the limit.  Use L'H$\hat{o}$pital's rule where appropriate.

\input{Derivative-Compute-0051.HELP.tex}

\[\lim\limits_{x\to\infty} 3 \, {\left(x + 1\right)} e^{\left(-x - 6\right)}=\answer{0}\]
\end{problem}}

%%%%%%%%%%%%%%%%%%%%%%

\latexProblemContent{
\ifVerboseLocation This is Derivative Compute Question 0051. \\ \fi
\begin{problem}

Find the limit.  Use L'H$\hat{o}$pital's rule where appropriate.

\input{Derivative-Compute-0051.HELP.tex}

\[\lim\limits_{x\to\infty} 4 \, {\left(x - 1\right)}^{2} e^{\left(-x - 8\right)}=\answer{0}\]
\end{problem}}

%%%%%%%%%%%%%%%%%%%%%%

\latexProblemContent{
\ifVerboseLocation This is Derivative Compute Question 0051. \\ \fi
\begin{problem}

Find the limit.  Use L'H$\hat{o}$pital's rule where appropriate.

\input{Derivative-Compute-0051.HELP.tex}

\[\lim\limits_{x\to\infty} -3 \, {\left(x - 1\right)}^{3} e^{\left(-x - 3\right)}=\answer{0}\]
\end{problem}}

%%%%%%%%%%%%%%%%%%%%%%

\latexProblemContent{
\ifVerboseLocation This is Derivative Compute Question 0051. \\ \fi
\begin{problem}

Find the limit.  Use L'H$\hat{o}$pital's rule where appropriate.

\input{Derivative-Compute-0051.HELP.tex}

\[\lim\limits_{x\to\infty} -2 \, {\left(x + 4\right)} e^{\left(-x + 6\right)}=\answer{0}\]
\end{problem}}

%%%%%%%%%%%%%%%%%%%%%%

\latexProblemContent{
\ifVerboseLocation This is Derivative Compute Question 0051. \\ \fi
\begin{problem}

Find the limit.  Use L'H$\hat{o}$pital's rule where appropriate.

\input{Derivative-Compute-0051.HELP.tex}

\[\lim\limits_{x\to\infty} -3 \, {\left(x + 5\right)}^{2} e^{\left(-x - 1\right)}=\answer{0}\]
\end{problem}}

%%%%%%%%%%%%%%%%%%%%%%

\latexProblemContent{
\ifVerboseLocation This is Derivative Compute Question 0051. \\ \fi
\begin{problem}

Find the limit.  Use L'H$\hat{o}$pital's rule where appropriate.

\input{Derivative-Compute-0051.HELP.tex}

\[\lim\limits_{x\to\infty} -3 \, {\left(x - 5\right)} e^{\left(-x + 4\right)}=\answer{0}\]
\end{problem}}

%%%%%%%%%%%%%%%%%%%%%%

\latexProblemContent{
\ifVerboseLocation This is Derivative Compute Question 0051. \\ \fi
\begin{problem}

Find the limit.  Use L'H$\hat{o}$pital's rule where appropriate.

\input{Derivative-Compute-0051.HELP.tex}

\[\lim\limits_{x\to\infty} 2 \, {\left(x - 6\right)} e^{\left(-x - 4\right)}=\answer{0}\]
\end{problem}}

%%%%%%%%%%%%%%%%%%%%%%

\latexProblemContent{
\ifVerboseLocation This is Derivative Compute Question 0051. \\ \fi
\begin{problem}

Find the limit.  Use L'H$\hat{o}$pital's rule where appropriate.

\input{Derivative-Compute-0051.HELP.tex}

\[\lim\limits_{x\to\infty} 4 \, {\left(x + 5\right)}^{2} e^{\left(-x + 4\right)}=\answer{0}\]
\end{problem}}

%%%%%%%%%%%%%%%%%%%%%%

\latexProblemContent{
\ifVerboseLocation This is Derivative Compute Question 0051. \\ \fi
\begin{problem}

Find the limit.  Use L'H$\hat{o}$pital's rule where appropriate.

\input{Derivative-Compute-0051.HELP.tex}

\[\lim\limits_{x\to\infty} -5 \, {\left(x + 3\right)}^{2} e^{\left(-x - 6\right)}=\answer{0}\]
\end{problem}}

%%%%%%%%%%%%%%%%%%%%%%

\latexProblemContent{
\ifVerboseLocation This is Derivative Compute Question 0051. \\ \fi
\begin{problem}

Find the limit.  Use L'H$\hat{o}$pital's rule where appropriate.

\input{Derivative-Compute-0051.HELP.tex}

\[\lim\limits_{x\to\infty} 4 \, {\left(x - 7\right)} e^{\left(-x - 2\right)}=\answer{0}\]
\end{problem}}

%%%%%%%%%%%%%%%%%%%%%%

\latexProblemContent{
\ifVerboseLocation This is Derivative Compute Question 0051. \\ \fi
\begin{problem}

Find the limit.  Use L'H$\hat{o}$pital's rule where appropriate.

\input{Derivative-Compute-0051.HELP.tex}

\[\lim\limits_{x\to\infty} -5 \, {\left(x + 2\right)}^{2} e^{\left(-x + 8\right)}=\answer{0}\]
\end{problem}}

%%%%%%%%%%%%%%%%%%%%%%

\latexProblemContent{
\ifVerboseLocation This is Derivative Compute Question 0051. \\ \fi
\begin{problem}

Find the limit.  Use L'H$\hat{o}$pital's rule where appropriate.

\input{Derivative-Compute-0051.HELP.tex}

\[\lim\limits_{x\to\infty} -4 \, {\left(x - 6\right)}^{3} e^{\left(-x + 2\right)}=\answer{0}\]
\end{problem}}

%%%%%%%%%%%%%%%%%%%%%%

\latexProblemContent{
\ifVerboseLocation This is Derivative Compute Question 0051. \\ \fi
\begin{problem}

Find the limit.  Use L'H$\hat{o}$pital's rule where appropriate.

\input{Derivative-Compute-0051.HELP.tex}

\[\lim\limits_{x\to\infty} -{\left(x + 3\right)} e^{\left(-x + 5\right)}=\answer{0}\]
\end{problem}}

%%%%%%%%%%%%%%%%%%%%%%

\latexProblemContent{
\ifVerboseLocation This is Derivative Compute Question 0051. \\ \fi
\begin{problem}

Find the limit.  Use L'H$\hat{o}$pital's rule where appropriate.

\input{Derivative-Compute-0051.HELP.tex}

\[\lim\limits_{x\to\infty} -6 \, {\left(x + 2\right)} e^{\left(-x + 1\right)}=\answer{0}\]
\end{problem}}

%%%%%%%%%%%%%%%%%%%%%%

\latexProblemContent{
\ifVerboseLocation This is Derivative Compute Question 0051. \\ \fi
\begin{problem}

Find the limit.  Use L'H$\hat{o}$pital's rule where appropriate.

\input{Derivative-Compute-0051.HELP.tex}

\[\lim\limits_{x\to\infty} 5 \, {\left(x + 1\right)}^{2} e^{\left(-x + 6\right)}=\answer{0}\]
\end{problem}}

%%%%%%%%%%%%%%%%%%%%%%

\latexProblemContent{
\ifVerboseLocation This is Derivative Compute Question 0051. \\ \fi
\begin{problem}

Find the limit.  Use L'H$\hat{o}$pital's rule where appropriate.

\input{Derivative-Compute-0051.HELP.tex}

\[\lim\limits_{x\to\infty} {\left(x + 5\right)} e^{\left(-x - 1\right)}=\answer{0}\]
\end{problem}}

%%%%%%%%%%%%%%%%%%%%%%

\latexProblemContent{
\ifVerboseLocation This is Derivative Compute Question 0051. \\ \fi
\begin{problem}

Find the limit.  Use L'H$\hat{o}$pital's rule where appropriate.

\input{Derivative-Compute-0051.HELP.tex}

\[\lim\limits_{x\to\infty} 3 \, {\left(x - 3\right)}^{2} e^{\left(-x + 7\right)}=\answer{0}\]
\end{problem}}

%%%%%%%%%%%%%%%%%%%%%%

\latexProblemContent{
\ifVerboseLocation This is Derivative Compute Question 0051. \\ \fi
\begin{problem}

Find the limit.  Use L'H$\hat{o}$pital's rule where appropriate.

\input{Derivative-Compute-0051.HELP.tex}

\[\lim\limits_{x\to\infty} 7 \, {\left(x - 8\right)} e^{\left(-x + 8\right)}=\answer{0}\]
\end{problem}}

%%%%%%%%%%%%%%%%%%%%%%

\latexProblemContent{
\ifVerboseLocation This is Derivative Compute Question 0051. \\ \fi
\begin{problem}

Find the limit.  Use L'H$\hat{o}$pital's rule where appropriate.

\input{Derivative-Compute-0051.HELP.tex}

\[\lim\limits_{x\to\infty} 3 \, {\left(x - 8\right)}^{2} e^{\left(-x + 3\right)}=\answer{0}\]
\end{problem}}

%%%%%%%%%%%%%%%%%%%%%%

\latexProblemContent{
\ifVerboseLocation This is Derivative Compute Question 0051. \\ \fi
\begin{problem}

Find the limit.  Use L'H$\hat{o}$pital's rule where appropriate.

\input{Derivative-Compute-0051.HELP.tex}

\[\lim\limits_{x\to\infty} -4 \, {\left(x + 5\right)}^{2} e^{\left(-x + 6\right)}=\answer{0}\]
\end{problem}}

%%%%%%%%%%%%%%%%%%%%%%

\latexProblemContent{
\ifVerboseLocation This is Derivative Compute Question 0051. \\ \fi
\begin{problem}

Find the limit.  Use L'H$\hat{o}$pital's rule where appropriate.

\input{Derivative-Compute-0051.HELP.tex}

\[\lim\limits_{x\to\infty} 4 \, {\left(x + 7\right)}^{2} e^{\left(-x + 8\right)}=\answer{0}\]
\end{problem}}

%%%%%%%%%%%%%%%%%%%%%%

\latexProblemContent{
\ifVerboseLocation This is Derivative Compute Question 0051. \\ \fi
\begin{problem}

Find the limit.  Use L'H$\hat{o}$pital's rule where appropriate.

\input{Derivative-Compute-0051.HELP.tex}

\[\lim\limits_{x\to\infty} -{\left(x - 6\right)} e^{\left(-x - 2\right)}=\answer{0}\]
\end{problem}}

%%%%%%%%%%%%%%%%%%%%%%

\latexProblemContent{
\ifVerboseLocation This is Derivative Compute Question 0051. \\ \fi
\begin{problem}

Find the limit.  Use L'H$\hat{o}$pital's rule where appropriate.

\input{Derivative-Compute-0051.HELP.tex}

\[\lim\limits_{x\to\infty} -4 \, {\left(x + 1\right)} e^{\left(-x + 1\right)}=\answer{0}\]
\end{problem}}

%%%%%%%%%%%%%%%%%%%%%%

\latexProblemContent{
\ifVerboseLocation This is Derivative Compute Question 0051. \\ \fi
\begin{problem}

Find the limit.  Use L'H$\hat{o}$pital's rule where appropriate.

\input{Derivative-Compute-0051.HELP.tex}

\[\lim\limits_{x\to\infty} 5 \, {\left(x - 5\right)} e^{\left(-x + 5\right)}=\answer{0}\]
\end{problem}}

%%%%%%%%%%%%%%%%%%%%%%

\latexProblemContent{
\ifVerboseLocation This is Derivative Compute Question 0051. \\ \fi
\begin{problem}

Find the limit.  Use L'H$\hat{o}$pital's rule where appropriate.

\input{Derivative-Compute-0051.HELP.tex}

\[\lim\limits_{x\to\infty} -{\left(x - 2\right)} e^{\left(-x - 7\right)}=\answer{0}\]
\end{problem}}

%%%%%%%%%%%%%%%%%%%%%%

\latexProblemContent{
\ifVerboseLocation This is Derivative Compute Question 0051. \\ \fi
\begin{problem}

Find the limit.  Use L'H$\hat{o}$pital's rule where appropriate.

\input{Derivative-Compute-0051.HELP.tex}

\[\lim\limits_{x\to\infty} 4 \, {\left(x + 4\right)} e^{\left(-x + 5\right)}=\answer{0}\]
\end{problem}}

%%%%%%%%%%%%%%%%%%%%%%

\latexProblemContent{
\ifVerboseLocation This is Derivative Compute Question 0051. \\ \fi
\begin{problem}

Find the limit.  Use L'H$\hat{o}$pital's rule where appropriate.

\input{Derivative-Compute-0051.HELP.tex}

\[\lim\limits_{x\to\infty} -8 \, {\left(x + 5\right)}^{3} e^{\left(-x - 2\right)}=\answer{0}\]
\end{problem}}

%%%%%%%%%%%%%%%%%%%%%%

\latexProblemContent{
\ifVerboseLocation This is Derivative Compute Question 0051. \\ \fi
\begin{problem}

Find the limit.  Use L'H$\hat{o}$pital's rule where appropriate.

\input{Derivative-Compute-0051.HELP.tex}

\[\lim\limits_{x\to\infty} -4 \, {\left(x - 7\right)}^{2} e^{\left(-x - 6\right)}=\answer{0}\]
\end{problem}}

%%%%%%%%%%%%%%%%%%%%%%

\latexProblemContent{
\ifVerboseLocation This is Derivative Compute Question 0051. \\ \fi
\begin{problem}

Find the limit.  Use L'H$\hat{o}$pital's rule where appropriate.

\input{Derivative-Compute-0051.HELP.tex}

\[\lim\limits_{x\to\infty} 5 \, {\left(x + 1\right)} e^{\left(-x + 5\right)}=\answer{0}\]
\end{problem}}

%%%%%%%%%%%%%%%%%%%%%%

\latexProblemContent{
\ifVerboseLocation This is Derivative Compute Question 0051. \\ \fi
\begin{problem}

Find the limit.  Use L'H$\hat{o}$pital's rule where appropriate.

\input{Derivative-Compute-0051.HELP.tex}

\[\lim\limits_{x\to\infty} -5 \, {\left(x - 3\right)} e^{\left(-x + 1\right)}=\answer{0}\]
\end{problem}}

%%%%%%%%%%%%%%%%%%%%%%

\latexProblemContent{
\ifVerboseLocation This is Derivative Compute Question 0051. \\ \fi
\begin{problem}

Find the limit.  Use L'H$\hat{o}$pital's rule where appropriate.

\input{Derivative-Compute-0051.HELP.tex}

\[\lim\limits_{x\to\infty} -6 \, {\left(x + 5\right)}^{2} e^{\left(-x + 5\right)}=\answer{0}\]
\end{problem}}

%%%%%%%%%%%%%%%%%%%%%%

\latexProblemContent{
\ifVerboseLocation This is Derivative Compute Question 0051. \\ \fi
\begin{problem}

Find the limit.  Use L'H$\hat{o}$pital's rule where appropriate.

\input{Derivative-Compute-0051.HELP.tex}

\[\lim\limits_{x\to\infty} -3 \, {\left(x + 6\right)} e^{\left(-x + 8\right)}=\answer{0}\]
\end{problem}}

%%%%%%%%%%%%%%%%%%%%%%

\latexProblemContent{
\ifVerboseLocation This is Derivative Compute Question 0051. \\ \fi
\begin{problem}

Find the limit.  Use L'H$\hat{o}$pital's rule where appropriate.

\input{Derivative-Compute-0051.HELP.tex}

\[\lim\limits_{x\to\infty} -4 \, {\left(x + 4\right)}^{2} e^{\left(-x + 2\right)}=\answer{0}\]
\end{problem}}

%%%%%%%%%%%%%%%%%%%%%%

\latexProblemContent{
\ifVerboseLocation This is Derivative Compute Question 0051. \\ \fi
\begin{problem}

Find the limit.  Use L'H$\hat{o}$pital's rule where appropriate.

\input{Derivative-Compute-0051.HELP.tex}

\[\lim\limits_{x\to\infty} 3 \, {\left(x + 2\right)}^{3} e^{\left(-x - 5\right)}=\answer{0}\]
\end{problem}}

%%%%%%%%%%%%%%%%%%%%%%

\latexProblemContent{
\ifVerboseLocation This is Derivative Compute Question 0051. \\ \fi
\begin{problem}

Find the limit.  Use L'H$\hat{o}$pital's rule where appropriate.

\input{Derivative-Compute-0051.HELP.tex}

\[\lim\limits_{x\to\infty} 7 \, {\left(x - 4\right)}^{3} e^{\left(-x - 6\right)}=\answer{0}\]
\end{problem}}

%%%%%%%%%%%%%%%%%%%%%%

\latexProblemContent{
\ifVerboseLocation This is Derivative Compute Question 0051. \\ \fi
\begin{problem}

Find the limit.  Use L'H$\hat{o}$pital's rule where appropriate.

\input{Derivative-Compute-0051.HELP.tex}

\[\lim\limits_{x\to\infty} -2 \, {\left(x - 2\right)}^{2} e^{\left(-x - 7\right)}=\answer{0}\]
\end{problem}}

%%%%%%%%%%%%%%%%%%%%%%

\latexProblemContent{
\ifVerboseLocation This is Derivative Compute Question 0051. \\ \fi
\begin{problem}

Find the limit.  Use L'H$\hat{o}$pital's rule where appropriate.

\input{Derivative-Compute-0051.HELP.tex}

\[\lim\limits_{x\to\infty} -8 \, {\left(x + 1\right)}^{2} e^{\left(-x - 4\right)}=\answer{0}\]
\end{problem}}

%%%%%%%%%%%%%%%%%%%%%%

\latexProblemContent{
\ifVerboseLocation This is Derivative Compute Question 0051. \\ \fi
\begin{problem}

Find the limit.  Use L'H$\hat{o}$pital's rule where appropriate.

\input{Derivative-Compute-0051.HELP.tex}

\[\lim\limits_{x\to\infty} -5 \, {\left(x - 7\right)}^{3} e^{\left(-x - 8\right)}=\answer{0}\]
\end{problem}}

%%%%%%%%%%%%%%%%%%%%%%

\latexProblemContent{
\ifVerboseLocation This is Derivative Compute Question 0051. \\ \fi
\begin{problem}

Find the limit.  Use L'H$\hat{o}$pital's rule where appropriate.

\input{Derivative-Compute-0051.HELP.tex}

\[\lim\limits_{x\to\infty} 6 \, {\left(x - 7\right)}^{2} e^{\left(-x + 6\right)}=\answer{0}\]
\end{problem}}

%%%%%%%%%%%%%%%%%%%%%%

\latexProblemContent{
\ifVerboseLocation This is Derivative Compute Question 0051. \\ \fi
\begin{problem}

Find the limit.  Use L'H$\hat{o}$pital's rule where appropriate.

\input{Derivative-Compute-0051.HELP.tex}

\[\lim\limits_{x\to\infty} -2 \, {\left(x - 5\right)}^{2} e^{\left(-x + 6\right)}=\answer{0}\]
\end{problem}}

%%%%%%%%%%%%%%%%%%%%%%

\latexProblemContent{
\ifVerboseLocation This is Derivative Compute Question 0051. \\ \fi
\begin{problem}

Find the limit.  Use L'H$\hat{o}$pital's rule where appropriate.

\input{Derivative-Compute-0051.HELP.tex}

\[\lim\limits_{x\to\infty} 4 \, {\left(x - 4\right)}^{2} e^{\left(-x + 8\right)}=\answer{0}\]
\end{problem}}

%%%%%%%%%%%%%%%%%%%%%%

\latexProblemContent{
\ifVerboseLocation This is Derivative Compute Question 0051. \\ \fi
\begin{problem}

Find the limit.  Use L'H$\hat{o}$pital's rule where appropriate.

\input{Derivative-Compute-0051.HELP.tex}

\[\lim\limits_{x\to\infty} {\left(x + 7\right)}^{2} e^{\left(-x - 3\right)}=\answer{0}\]
\end{problem}}

%%%%%%%%%%%%%%%%%%%%%%

\latexProblemContent{
\ifVerboseLocation This is Derivative Compute Question 0051. \\ \fi
\begin{problem}

Find the limit.  Use L'H$\hat{o}$pital's rule where appropriate.

\input{Derivative-Compute-0051.HELP.tex}

\[\lim\limits_{x\to\infty} -4 \, {\left(x + 5\right)}^{2} e^{\left(-x + 8\right)}=\answer{0}\]
\end{problem}}

%%%%%%%%%%%%%%%%%%%%%%

\latexProblemContent{
\ifVerboseLocation This is Derivative Compute Question 0051. \\ \fi
\begin{problem}

Find the limit.  Use L'H$\hat{o}$pital's rule where appropriate.

\input{Derivative-Compute-0051.HELP.tex}

\[\lim\limits_{x\to\infty} -6 \, {\left(x - 5\right)}^{2} e^{\left(-x - 4\right)}=\answer{0}\]
\end{problem}}

%%%%%%%%%%%%%%%%%%%%%%

\latexProblemContent{
\ifVerboseLocation This is Derivative Compute Question 0051. \\ \fi
\begin{problem}

Find the limit.  Use L'H$\hat{o}$pital's rule where appropriate.

\input{Derivative-Compute-0051.HELP.tex}

\[\lim\limits_{x\to\infty} 6 \, {\left(x + 2\right)}^{3} e^{\left(-x + 4\right)}=\answer{0}\]
\end{problem}}

%%%%%%%%%%%%%%%%%%%%%%

\latexProblemContent{
\ifVerboseLocation This is Derivative Compute Question 0051. \\ \fi
\begin{problem}

Find the limit.  Use L'H$\hat{o}$pital's rule where appropriate.

\input{Derivative-Compute-0051.HELP.tex}

\[\lim\limits_{x\to\infty} 8 \, {\left(x + 3\right)} e^{\left(-x + 4\right)}=\answer{0}\]
\end{problem}}

%%%%%%%%%%%%%%%%%%%%%%

\latexProblemContent{
\ifVerboseLocation This is Derivative Compute Question 0051. \\ \fi
\begin{problem}

Find the limit.  Use L'H$\hat{o}$pital's rule where appropriate.

\input{Derivative-Compute-0051.HELP.tex}

\[\lim\limits_{x\to\infty} -8 \, {\left(x + 7\right)} e^{\left(-x - 6\right)}=\answer{0}\]
\end{problem}}

%%%%%%%%%%%%%%%%%%%%%%

\latexProblemContent{
\ifVerboseLocation This is Derivative Compute Question 0051. \\ \fi
\begin{problem}

Find the limit.  Use L'H$\hat{o}$pital's rule where appropriate.

\input{Derivative-Compute-0051.HELP.tex}

\[\lim\limits_{x\to\infty} -5 \, {\left(x + 4\right)} e^{\left(-x + 4\right)}=\answer{0}\]
\end{problem}}

%%%%%%%%%%%%%%%%%%%%%%

\latexProblemContent{
\ifVerboseLocation This is Derivative Compute Question 0051. \\ \fi
\begin{problem}

Find the limit.  Use L'H$\hat{o}$pital's rule where appropriate.

\input{Derivative-Compute-0051.HELP.tex}

\[\lim\limits_{x\to\infty} {\left(x - 7\right)}^{3} e^{\left(-x - 8\right)}=\answer{0}\]
\end{problem}}

%%%%%%%%%%%%%%%%%%%%%%

\latexProblemContent{
\ifVerboseLocation This is Derivative Compute Question 0051. \\ \fi
\begin{problem}

Find the limit.  Use L'H$\hat{o}$pital's rule where appropriate.

\input{Derivative-Compute-0051.HELP.tex}

\[\lim\limits_{x\to\infty} 8 \, {\left(x + 7\right)}^{3} e^{\left(-x - 5\right)}=\answer{0}\]
\end{problem}}

%%%%%%%%%%%%%%%%%%%%%%

\latexProblemContent{
\ifVerboseLocation This is Derivative Compute Question 0051. \\ \fi
\begin{problem}

Find the limit.  Use L'H$\hat{o}$pital's rule where appropriate.

\input{Derivative-Compute-0051.HELP.tex}

\[\lim\limits_{x\to\infty} -5 \, {\left(x + 2\right)} e^{\left(-x + 3\right)}=\answer{0}\]
\end{problem}}

%%%%%%%%%%%%%%%%%%%%%%

\latexProblemContent{
\ifVerboseLocation This is Derivative Compute Question 0051. \\ \fi
\begin{problem}

Find the limit.  Use L'H$\hat{o}$pital's rule where appropriate.

\input{Derivative-Compute-0051.HELP.tex}

\[\lim\limits_{x\to\infty} -5 \, {\left(x - 4\right)}^{3} e^{\left(-x - 4\right)}=\answer{0}\]
\end{problem}}

%%%%%%%%%%%%%%%%%%%%%%

\latexProblemContent{
\ifVerboseLocation This is Derivative Compute Question 0051. \\ \fi
\begin{problem}

Find the limit.  Use L'H$\hat{o}$pital's rule where appropriate.

\input{Derivative-Compute-0051.HELP.tex}

\[\lim\limits_{x\to\infty} -3 \, {\left(x - 2\right)} e^{\left(-x - 2\right)}=\answer{0}\]
\end{problem}}

%%%%%%%%%%%%%%%%%%%%%%

\latexProblemContent{
\ifVerboseLocation This is Derivative Compute Question 0051. \\ \fi
\begin{problem}

Find the limit.  Use L'H$\hat{o}$pital's rule where appropriate.

\input{Derivative-Compute-0051.HELP.tex}

\[\lim\limits_{x\to\infty} -7 \, {\left(x - 6\right)} e^{\left(-x - 4\right)}=\answer{0}\]
\end{problem}}

%%%%%%%%%%%%%%%%%%%%%%

\latexProblemContent{
\ifVerboseLocation This is Derivative Compute Question 0051. \\ \fi
\begin{problem}

Find the limit.  Use L'H$\hat{o}$pital's rule where appropriate.

\input{Derivative-Compute-0051.HELP.tex}

\[\lim\limits_{x\to\infty} {\left(x + 4\right)}^{2} e^{\left(-x - 5\right)}=\answer{0}\]
\end{problem}}

%%%%%%%%%%%%%%%%%%%%%%

\latexProblemContent{
\ifVerboseLocation This is Derivative Compute Question 0051. \\ \fi
\begin{problem}

Find the limit.  Use L'H$\hat{o}$pital's rule where appropriate.

\input{Derivative-Compute-0051.HELP.tex}

\[\lim\limits_{x\to\infty} 4 \, {\left(x - 4\right)}^{2} e^{\left(-x - 3\right)}=\answer{0}\]
\end{problem}}

%%%%%%%%%%%%%%%%%%%%%%

\latexProblemContent{
\ifVerboseLocation This is Derivative Compute Question 0051. \\ \fi
\begin{problem}

Find the limit.  Use L'H$\hat{o}$pital's rule where appropriate.

\input{Derivative-Compute-0051.HELP.tex}

\[\lim\limits_{x\to\infty} 6 \, {\left(x - 4\right)} e^{\left(-x - 4\right)}=\answer{0}\]
\end{problem}}

%%%%%%%%%%%%%%%%%%%%%%

\latexProblemContent{
\ifVerboseLocation This is Derivative Compute Question 0051. \\ \fi
\begin{problem}

Find the limit.  Use L'H$\hat{o}$pital's rule where appropriate.

\input{Derivative-Compute-0051.HELP.tex}

\[\lim\limits_{x\to\infty} 7 \, {\left(x + 1\right)}^{2} e^{\left(-x - 6\right)}=\answer{0}\]
\end{problem}}

%%%%%%%%%%%%%%%%%%%%%%

\latexProblemContent{
\ifVerboseLocation This is Derivative Compute Question 0051. \\ \fi
\begin{problem}

Find the limit.  Use L'H$\hat{o}$pital's rule where appropriate.

\input{Derivative-Compute-0051.HELP.tex}

\[\lim\limits_{x\to\infty} 5 \, {\left(x - 2\right)} e^{\left(-x + 3\right)}=\answer{0}\]
\end{problem}}

%%%%%%%%%%%%%%%%%%%%%%

\latexProblemContent{
\ifVerboseLocation This is Derivative Compute Question 0051. \\ \fi
\begin{problem}

Find the limit.  Use L'H$\hat{o}$pital's rule where appropriate.

\input{Derivative-Compute-0051.HELP.tex}

\[\lim\limits_{x\to\infty} -4 \, {\left(x + 5\right)} e^{\left(-x - 4\right)}=\answer{0}\]
\end{problem}}

%%%%%%%%%%%%%%%%%%%%%%

\latexProblemContent{
\ifVerboseLocation This is Derivative Compute Question 0051. \\ \fi
\begin{problem}

Find the limit.  Use L'H$\hat{o}$pital's rule where appropriate.

\input{Derivative-Compute-0051.HELP.tex}

\[\lim\limits_{x\to\infty} 6 \, {\left(x + 3\right)}^{3} e^{\left(-x - 7\right)}=\answer{0}\]
\end{problem}}

%%%%%%%%%%%%%%%%%%%%%%

\latexProblemContent{
\ifVerboseLocation This is Derivative Compute Question 0051. \\ \fi
\begin{problem}

Find the limit.  Use L'H$\hat{o}$pital's rule where appropriate.

\input{Derivative-Compute-0051.HELP.tex}

\[\lim\limits_{x\to\infty} -3 \, {\left(x - 8\right)}^{3} e^{\left(-x + 1\right)}=\answer{0}\]
\end{problem}}

%%%%%%%%%%%%%%%%%%%%%%

\latexProblemContent{
\ifVerboseLocation This is Derivative Compute Question 0051. \\ \fi
\begin{problem}

Find the limit.  Use L'H$\hat{o}$pital's rule where appropriate.

\input{Derivative-Compute-0051.HELP.tex}

\[\lim\limits_{x\to\infty} 2 \, {\left(x - 7\right)}^{3} e^{\left(-x - 8\right)}=\answer{0}\]
\end{problem}}

%%%%%%%%%%%%%%%%%%%%%%

\latexProblemContent{
\ifVerboseLocation This is Derivative Compute Question 0051. \\ \fi
\begin{problem}

Find the limit.  Use L'H$\hat{o}$pital's rule where appropriate.

\input{Derivative-Compute-0051.HELP.tex}

\[\lim\limits_{x\to\infty} 3 \, {\left(x - 2\right)} e^{\left(-x - 1\right)}=\answer{0}\]
\end{problem}}

%%%%%%%%%%%%%%%%%%%%%%

\latexProblemContent{
\ifVerboseLocation This is Derivative Compute Question 0051. \\ \fi
\begin{problem}

Find the limit.  Use L'H$\hat{o}$pital's rule where appropriate.

\input{Derivative-Compute-0051.HELP.tex}

\[\lim\limits_{x\to\infty} -2 \, {\left(x - 3\right)}^{3} e^{\left(-x - 7\right)}=\answer{0}\]
\end{problem}}

%%%%%%%%%%%%%%%%%%%%%%

\latexProblemContent{
\ifVerboseLocation This is Derivative Compute Question 0051. \\ \fi
\begin{problem}

Find the limit.  Use L'H$\hat{o}$pital's rule where appropriate.

\input{Derivative-Compute-0051.HELP.tex}

\[\lim\limits_{x\to\infty} 3 \, {\left(x - 4\right)}^{2} e^{\left(-x + 8\right)}=\answer{0}\]
\end{problem}}

%%%%%%%%%%%%%%%%%%%%%%

\latexProblemContent{
\ifVerboseLocation This is Derivative Compute Question 0051. \\ \fi
\begin{problem}

Find the limit.  Use L'H$\hat{o}$pital's rule where appropriate.

\input{Derivative-Compute-0051.HELP.tex}

\[\lim\limits_{x\to\infty} -7 \, {\left(x + 5\right)}^{2} e^{\left(-x - 3\right)}=\answer{0}\]
\end{problem}}

%%%%%%%%%%%%%%%%%%%%%%

\latexProblemContent{
\ifVerboseLocation This is Derivative Compute Question 0051. \\ \fi
\begin{problem}

Find the limit.  Use L'H$\hat{o}$pital's rule where appropriate.

\input{Derivative-Compute-0051.HELP.tex}

\[\lim\limits_{x\to\infty} -8 \, {\left(x - 4\right)}^{2} e^{\left(-x - 3\right)}=\answer{0}\]
\end{problem}}

%%%%%%%%%%%%%%%%%%%%%%

\latexProblemContent{
\ifVerboseLocation This is Derivative Compute Question 0051. \\ \fi
\begin{problem}

Find the limit.  Use L'H$\hat{o}$pital's rule where appropriate.

\input{Derivative-Compute-0051.HELP.tex}

\[\lim\limits_{x\to\infty} -6 \, {\left(x + 7\right)}^{3} e^{\left(-x - 2\right)}=\answer{0}\]
\end{problem}}

%%%%%%%%%%%%%%%%%%%%%%

\latexProblemContent{
\ifVerboseLocation This is Derivative Compute Question 0051. \\ \fi
\begin{problem}

Find the limit.  Use L'H$\hat{o}$pital's rule where appropriate.

\input{Derivative-Compute-0051.HELP.tex}

\[\lim\limits_{x\to\infty} -2 \, {\left(x - 6\right)}^{3} e^{\left(-x - 7\right)}=\answer{0}\]
\end{problem}}

%%%%%%%%%%%%%%%%%%%%%%

\latexProblemContent{
\ifVerboseLocation This is Derivative Compute Question 0051. \\ \fi
\begin{problem}

Find the limit.  Use L'H$\hat{o}$pital's rule where appropriate.

\input{Derivative-Compute-0051.HELP.tex}

\[\lim\limits_{x\to\infty} -8 \, {\left(x - 1\right)}^{3} e^{\left(-x + 6\right)}=\answer{0}\]
\end{problem}}

%%%%%%%%%%%%%%%%%%%%%%

\latexProblemContent{
\ifVerboseLocation This is Derivative Compute Question 0051. \\ \fi
\begin{problem}

Find the limit.  Use L'H$\hat{o}$pital's rule where appropriate.

\input{Derivative-Compute-0051.HELP.tex}

\[\lim\limits_{x\to\infty} 5 \, {\left(x - 5\right)} e^{\left(-x + 3\right)}=\answer{0}\]
\end{problem}}

%%%%%%%%%%%%%%%%%%%%%%

\latexProblemContent{
\ifVerboseLocation This is Derivative Compute Question 0051. \\ \fi
\begin{problem}

Find the limit.  Use L'H$\hat{o}$pital's rule where appropriate.

\input{Derivative-Compute-0051.HELP.tex}

\[\lim\limits_{x\to\infty} 7 \, {\left(x - 8\right)}^{3} e^{\left(-x - 1\right)}=\answer{0}\]
\end{problem}}

%%%%%%%%%%%%%%%%%%%%%%

\latexProblemContent{
\ifVerboseLocation This is Derivative Compute Question 0051. \\ \fi
\begin{problem}

Find the limit.  Use L'H$\hat{o}$pital's rule where appropriate.

\input{Derivative-Compute-0051.HELP.tex}

\[\lim\limits_{x\to\infty} -2 \, {\left(x - 3\right)} e^{\left(-x + 7\right)}=\answer{0}\]
\end{problem}}

%%%%%%%%%%%%%%%%%%%%%%

\latexProblemContent{
\ifVerboseLocation This is Derivative Compute Question 0051. \\ \fi
\begin{problem}

Find the limit.  Use L'H$\hat{o}$pital's rule where appropriate.

\input{Derivative-Compute-0051.HELP.tex}

\[\lim\limits_{x\to\infty} -4 \, {\left(x - 5\right)}^{2} e^{\left(-x - 6\right)}=\answer{0}\]
\end{problem}}

%%%%%%%%%%%%%%%%%%%%%%

\latexProblemContent{
\ifVerboseLocation This is Derivative Compute Question 0051. \\ \fi
\begin{problem}

Find the limit.  Use L'H$\hat{o}$pital's rule where appropriate.

\input{Derivative-Compute-0051.HELP.tex}

\[\lim\limits_{x\to\infty} -4 \, {\left(x - 3\right)} e^{\left(-x - 5\right)}=\answer{0}\]
\end{problem}}

%%%%%%%%%%%%%%%%%%%%%%

\latexProblemContent{
\ifVerboseLocation This is Derivative Compute Question 0051. \\ \fi
\begin{problem}

Find the limit.  Use L'H$\hat{o}$pital's rule where appropriate.

\input{Derivative-Compute-0051.HELP.tex}

\[\lim\limits_{x\to\infty} 3 \, {\left(x - 1\right)}^{3} e^{\left(-x + 5\right)}=\answer{0}\]
\end{problem}}

%%%%%%%%%%%%%%%%%%%%%%

\latexProblemContent{
\ifVerboseLocation This is Derivative Compute Question 0051. \\ \fi
\begin{problem}

Find the limit.  Use L'H$\hat{o}$pital's rule where appropriate.

\input{Derivative-Compute-0051.HELP.tex}

\[\lim\limits_{x\to\infty} 2 \, {\left(x - 5\right)}^{3} e^{\left(-x - 1\right)}=\answer{0}\]
\end{problem}}

%%%%%%%%%%%%%%%%%%%%%%

\latexProblemContent{
\ifVerboseLocation This is Derivative Compute Question 0051. \\ \fi
\begin{problem}

Find the limit.  Use L'H$\hat{o}$pital's rule where appropriate.

\input{Derivative-Compute-0051.HELP.tex}

\[\lim\limits_{x\to\infty} -7 \, {\left(x - 4\right)}^{2} e^{\left(-x - 4\right)}=\answer{0}\]
\end{problem}}

%%%%%%%%%%%%%%%%%%%%%%

\latexProblemContent{
\ifVerboseLocation This is Derivative Compute Question 0051. \\ \fi
\begin{problem}

Find the limit.  Use L'H$\hat{o}$pital's rule where appropriate.

\input{Derivative-Compute-0051.HELP.tex}

\[\lim\limits_{x\to\infty} -2 \, {\left(x - 3\right)}^{3} e^{\left(-x - 1\right)}=\answer{0}\]
\end{problem}}

%%%%%%%%%%%%%%%%%%%%%%

\latexProblemContent{
\ifVerboseLocation This is Derivative Compute Question 0051. \\ \fi
\begin{problem}

Find the limit.  Use L'H$\hat{o}$pital's rule where appropriate.

\input{Derivative-Compute-0051.HELP.tex}

\[\lim\limits_{x\to\infty} -2 \, {\left(x - 3\right)} e^{\left(-x + 4\right)}=\answer{0}\]
\end{problem}}

%%%%%%%%%%%%%%%%%%%%%%

\latexProblemContent{
\ifVerboseLocation This is Derivative Compute Question 0051. \\ \fi
\begin{problem}

Find the limit.  Use L'H$\hat{o}$pital's rule where appropriate.

\input{Derivative-Compute-0051.HELP.tex}

\[\lim\limits_{x\to\infty} 6 \, {\left(x - 4\right)} e^{\left(-x - 6\right)}=\answer{0}\]
\end{problem}}

%%%%%%%%%%%%%%%%%%%%%%

\latexProblemContent{
\ifVerboseLocation This is Derivative Compute Question 0051. \\ \fi
\begin{problem}

Find the limit.  Use L'H$\hat{o}$pital's rule where appropriate.

\input{Derivative-Compute-0051.HELP.tex}

\[\lim\limits_{x\to\infty} 7 \, {\left(x - 6\right)}^{3} e^{\left(-x + 2\right)}=\answer{0}\]
\end{problem}}

%%%%%%%%%%%%%%%%%%%%%%

\latexProblemContent{
\ifVerboseLocation This is Derivative Compute Question 0051. \\ \fi
\begin{problem}

Find the limit.  Use L'H$\hat{o}$pital's rule where appropriate.

\input{Derivative-Compute-0051.HELP.tex}

\[\lim\limits_{x\to\infty} -8 \, {\left(x - 2\right)} e^{\left(-x + 7\right)}=\answer{0}\]
\end{problem}}

%%%%%%%%%%%%%%%%%%%%%%

\latexProblemContent{
\ifVerboseLocation This is Derivative Compute Question 0051. \\ \fi
\begin{problem}

Find the limit.  Use L'H$\hat{o}$pital's rule where appropriate.

\input{Derivative-Compute-0051.HELP.tex}

\[\lim\limits_{x\to\infty} -6 \, {\left(x + 5\right)}^{3} e^{\left(-x + 6\right)}=\answer{0}\]
\end{problem}}

%%%%%%%%%%%%%%%%%%%%%%

\latexProblemContent{
\ifVerboseLocation This is Derivative Compute Question 0051. \\ \fi
\begin{problem}

Find the limit.  Use L'H$\hat{o}$pital's rule where appropriate.

\input{Derivative-Compute-0051.HELP.tex}

\[\lim\limits_{x\to\infty} -4 \, {\left(x + 7\right)}^{3} e^{\left(-x - 7\right)}=\answer{0}\]
\end{problem}}

%%%%%%%%%%%%%%%%%%%%%%

\latexProblemContent{
\ifVerboseLocation This is Derivative Compute Question 0051. \\ \fi
\begin{problem}

Find the limit.  Use L'H$\hat{o}$pital's rule where appropriate.

\input{Derivative-Compute-0051.HELP.tex}

\[\lim\limits_{x\to\infty} -2 \, {\left(x + 6\right)}^{3} e^{\left(-x - 8\right)}=\answer{0}\]
\end{problem}}

%%%%%%%%%%%%%%%%%%%%%%

\latexProblemContent{
\ifVerboseLocation This is Derivative Compute Question 0051. \\ \fi
\begin{problem}

Find the limit.  Use L'H$\hat{o}$pital's rule where appropriate.

\input{Derivative-Compute-0051.HELP.tex}

\[\lim\limits_{x\to\infty} 6 \, {\left(x - 2\right)} e^{\left(-x - 7\right)}=\answer{0}\]
\end{problem}}

%%%%%%%%%%%%%%%%%%%%%%

\latexProblemContent{
\ifVerboseLocation This is Derivative Compute Question 0051. \\ \fi
\begin{problem}

Find the limit.  Use L'H$\hat{o}$pital's rule where appropriate.

\input{Derivative-Compute-0051.HELP.tex}

\[\lim\limits_{x\to\infty} 8 \, {\left(x - 5\right)} e^{\left(-x - 3\right)}=\answer{0}\]
\end{problem}}

%%%%%%%%%%%%%%%%%%%%%%

\latexProblemContent{
\ifVerboseLocation This is Derivative Compute Question 0051. \\ \fi
\begin{problem}

Find the limit.  Use L'H$\hat{o}$pital's rule where appropriate.

\input{Derivative-Compute-0051.HELP.tex}

\[\lim\limits_{x\to\infty} 3 \, {\left(x - 3\right)}^{3} e^{\left(-x + 1\right)}=\answer{0}\]
\end{problem}}

%%%%%%%%%%%%%%%%%%%%%%

\latexProblemContent{
\ifVerboseLocation This is Derivative Compute Question 0051. \\ \fi
\begin{problem}

Find the limit.  Use L'H$\hat{o}$pital's rule where appropriate.

\input{Derivative-Compute-0051.HELP.tex}

\[\lim\limits_{x\to\infty} -7 \, {\left(x + 3\right)}^{2} e^{\left(-x - 2\right)}=\answer{0}\]
\end{problem}}

%%%%%%%%%%%%%%%%%%%%%%

\latexProblemContent{
\ifVerboseLocation This is Derivative Compute Question 0051. \\ \fi
\begin{problem}

Find the limit.  Use L'H$\hat{o}$pital's rule where appropriate.

\input{Derivative-Compute-0051.HELP.tex}

\[\lim\limits_{x\to\infty} -2 \, {\left(x - 5\right)}^{3} e^{\left(-x - 4\right)}=\answer{0}\]
\end{problem}}

%%%%%%%%%%%%%%%%%%%%%%

\latexProblemContent{
\ifVerboseLocation This is Derivative Compute Question 0051. \\ \fi
\begin{problem}

Find the limit.  Use L'H$\hat{o}$pital's rule where appropriate.

\input{Derivative-Compute-0051.HELP.tex}

\[\lim\limits_{x\to\infty} -5 \, {\left(x - 2\right)}^{3} e^{\left(-x - 8\right)}=\answer{0}\]
\end{problem}}

%%%%%%%%%%%%%%%%%%%%%%

\latexProblemContent{
\ifVerboseLocation This is Derivative Compute Question 0051. \\ \fi
\begin{problem}

Find the limit.  Use L'H$\hat{o}$pital's rule where appropriate.

\input{Derivative-Compute-0051.HELP.tex}

\[\lim\limits_{x\to\infty} 4 \, {\left(x + 6\right)}^{3} e^{\left(-x - 8\right)}=\answer{0}\]
\end{problem}}

%%%%%%%%%%%%%%%%%%%%%%

\latexProblemContent{
\ifVerboseLocation This is Derivative Compute Question 0051. \\ \fi
\begin{problem}

Find the limit.  Use L'H$\hat{o}$pital's rule where appropriate.

\input{Derivative-Compute-0051.HELP.tex}

\[\lim\limits_{x\to\infty} 5 \, {\left(x + 3\right)} e^{\left(-x + 5\right)}=\answer{0}\]
\end{problem}}

%%%%%%%%%%%%%%%%%%%%%%

\latexProblemContent{
\ifVerboseLocation This is Derivative Compute Question 0051. \\ \fi
\begin{problem}

Find the limit.  Use L'H$\hat{o}$pital's rule where appropriate.

\input{Derivative-Compute-0051.HELP.tex}

\[\lim\limits_{x\to\infty} -4 \, {\left(x - 4\right)}^{3} e^{\left(-x - 1\right)}=\answer{0}\]
\end{problem}}

%%%%%%%%%%%%%%%%%%%%%%

\latexProblemContent{
\ifVerboseLocation This is Derivative Compute Question 0051. \\ \fi
\begin{problem}

Find the limit.  Use L'H$\hat{o}$pital's rule where appropriate.

\input{Derivative-Compute-0051.HELP.tex}

\[\lim\limits_{x\to\infty} 7 \, {\left(x - 8\right)} e^{\left(-x - 3\right)}=\answer{0}\]
\end{problem}}

%%%%%%%%%%%%%%%%%%%%%%

\latexProblemContent{
\ifVerboseLocation This is Derivative Compute Question 0051. \\ \fi
\begin{problem}

Find the limit.  Use L'H$\hat{o}$pital's rule where appropriate.

\input{Derivative-Compute-0051.HELP.tex}

\[\lim\limits_{x\to\infty} -{\left(x - 5\right)}^{2} e^{\left(-x + 3\right)}=\answer{0}\]
\end{problem}}

%%%%%%%%%%%%%%%%%%%%%%

\latexProblemContent{
\ifVerboseLocation This is Derivative Compute Question 0051. \\ \fi
\begin{problem}

Find the limit.  Use L'H$\hat{o}$pital's rule where appropriate.

\input{Derivative-Compute-0051.HELP.tex}

\[\lim\limits_{x\to\infty} 4 \, {\left(x - 1\right)} e^{\left(-x + 8\right)}=\answer{0}\]
\end{problem}}

%%%%%%%%%%%%%%%%%%%%%%

\latexProblemContent{
\ifVerboseLocation This is Derivative Compute Question 0051. \\ \fi
\begin{problem}

Find the limit.  Use L'H$\hat{o}$pital's rule where appropriate.

\input{Derivative-Compute-0051.HELP.tex}

\[\lim\limits_{x\to\infty} -4 \, {\left(x - 5\right)}^{2} e^{\left(-x - 1\right)}=\answer{0}\]
\end{problem}}

%%%%%%%%%%%%%%%%%%%%%%

\latexProblemContent{
\ifVerboseLocation This is Derivative Compute Question 0051. \\ \fi
\begin{problem}

Find the limit.  Use L'H$\hat{o}$pital's rule where appropriate.

\input{Derivative-Compute-0051.HELP.tex}

\[\lim\limits_{x\to\infty} -6 \, {\left(x + 7\right)}^{3} e^{\left(-x - 3\right)}=\answer{0}\]
\end{problem}}

%%%%%%%%%%%%%%%%%%%%%%

\latexProblemContent{
\ifVerboseLocation This is Derivative Compute Question 0051. \\ \fi
\begin{problem}

Find the limit.  Use L'H$\hat{o}$pital's rule where appropriate.

\input{Derivative-Compute-0051.HELP.tex}

\[\lim\limits_{x\to\infty} -5 \, {\left(x + 4\right)} e^{\left(-x + 3\right)}=\answer{0}\]
\end{problem}}

%%%%%%%%%%%%%%%%%%%%%%

\latexProblemContent{
\ifVerboseLocation This is Derivative Compute Question 0051. \\ \fi
\begin{problem}

Find the limit.  Use L'H$\hat{o}$pital's rule where appropriate.

\input{Derivative-Compute-0051.HELP.tex}

\[\lim\limits_{x\to\infty} 8 \, {\left(x + 6\right)}^{2} e^{\left(-x - 8\right)}=\answer{0}\]
\end{problem}}

%%%%%%%%%%%%%%%%%%%%%%

\latexProblemContent{
\ifVerboseLocation This is Derivative Compute Question 0051. \\ \fi
\begin{problem}

Find the limit.  Use L'H$\hat{o}$pital's rule where appropriate.

\input{Derivative-Compute-0051.HELP.tex}

\[\lim\limits_{x\to\infty} -8 \, {\left(x + 4\right)}^{2} e^{\left(-x - 3\right)}=\answer{0}\]
\end{problem}}

%%%%%%%%%%%%%%%%%%%%%%

\latexProblemContent{
\ifVerboseLocation This is Derivative Compute Question 0051. \\ \fi
\begin{problem}

Find the limit.  Use L'H$\hat{o}$pital's rule where appropriate.

\input{Derivative-Compute-0051.HELP.tex}

\[\lim\limits_{x\to\infty} -8 \, {\left(x - 8\right)}^{2} e^{\left(-x - 7\right)}=\answer{0}\]
\end{problem}}

%%%%%%%%%%%%%%%%%%%%%%

\latexProblemContent{
\ifVerboseLocation This is Derivative Compute Question 0051. \\ \fi
\begin{problem}

Find the limit.  Use L'H$\hat{o}$pital's rule where appropriate.

\input{Derivative-Compute-0051.HELP.tex}

\[\lim\limits_{x\to\infty} -8 \, {\left(x + 1\right)}^{3} e^{\left(-x + 8\right)}=\answer{0}\]
\end{problem}}

%%%%%%%%%%%%%%%%%%%%%%

\latexProblemContent{
\ifVerboseLocation This is Derivative Compute Question 0051. \\ \fi
\begin{problem}

Find the limit.  Use L'H$\hat{o}$pital's rule where appropriate.

\input{Derivative-Compute-0051.HELP.tex}

\[\lim\limits_{x\to\infty} 2 \, {\left(x + 6\right)} e^{\left(-x + 2\right)}=\answer{0}\]
\end{problem}}

%%%%%%%%%%%%%%%%%%%%%%

\latexProblemContent{
\ifVerboseLocation This is Derivative Compute Question 0051. \\ \fi
\begin{problem}

Find the limit.  Use L'H$\hat{o}$pital's rule where appropriate.

\input{Derivative-Compute-0051.HELP.tex}

\[\lim\limits_{x\to\infty} 2 \, {\left(x - 8\right)} e^{\left(-x - 7\right)}=\answer{0}\]
\end{problem}}

%%%%%%%%%%%%%%%%%%%%%%

\latexProblemContent{
\ifVerboseLocation This is Derivative Compute Question 0051. \\ \fi
\begin{problem}

Find the limit.  Use L'H$\hat{o}$pital's rule where appropriate.

\input{Derivative-Compute-0051.HELP.tex}

\[\lim\limits_{x\to\infty} 8 \, {\left(x - 6\right)}^{3} e^{\left(-x - 5\right)}=\answer{0}\]
\end{problem}}

%%%%%%%%%%%%%%%%%%%%%%

\latexProblemContent{
\ifVerboseLocation This is Derivative Compute Question 0051. \\ \fi
\begin{problem}

Find the limit.  Use L'H$\hat{o}$pital's rule where appropriate.

\input{Derivative-Compute-0051.HELP.tex}

\[\lim\limits_{x\to\infty} -2 \, {\left(x - 6\right)} e^{\left(-x + 3\right)}=\answer{0}\]
\end{problem}}

%%%%%%%%%%%%%%%%%%%%%%

\latexProblemContent{
\ifVerboseLocation This is Derivative Compute Question 0051. \\ \fi
\begin{problem}

Find the limit.  Use L'H$\hat{o}$pital's rule where appropriate.

\input{Derivative-Compute-0051.HELP.tex}

\[\lim\limits_{x\to\infty} -2 \, {\left(x - 2\right)}^{3} e^{\left(-x - 4\right)}=\answer{0}\]
\end{problem}}

%%%%%%%%%%%%%%%%%%%%%%

\latexProblemContent{
\ifVerboseLocation This is Derivative Compute Question 0051. \\ \fi
\begin{problem}

Find the limit.  Use L'H$\hat{o}$pital's rule where appropriate.

\input{Derivative-Compute-0051.HELP.tex}

\[\lim\limits_{x\to\infty} -5 \, {\left(x + 1\right)}^{3} e^{\left(-x - 5\right)}=\answer{0}\]
\end{problem}}

%%%%%%%%%%%%%%%%%%%%%%

\latexProblemContent{
\ifVerboseLocation This is Derivative Compute Question 0051. \\ \fi
\begin{problem}

Find the limit.  Use L'H$\hat{o}$pital's rule where appropriate.

\input{Derivative-Compute-0051.HELP.tex}

\[\lim\limits_{x\to\infty} -3 \, {\left(x + 6\right)}^{3} e^{\left(-x + 5\right)}=\answer{0}\]
\end{problem}}

%%%%%%%%%%%%%%%%%%%%%%

\latexProblemContent{
\ifVerboseLocation This is Derivative Compute Question 0051. \\ \fi
\begin{problem}

Find the limit.  Use L'H$\hat{o}$pital's rule where appropriate.

\input{Derivative-Compute-0051.HELP.tex}

\[\lim\limits_{x\to\infty} -8 \, {\left(x - 2\right)}^{3} e^{\left(-x + 8\right)}=\answer{0}\]
\end{problem}}

%%%%%%%%%%%%%%%%%%%%%%

\latexProblemContent{
\ifVerboseLocation This is Derivative Compute Question 0051. \\ \fi
\begin{problem}

Find the limit.  Use L'H$\hat{o}$pital's rule where appropriate.

\input{Derivative-Compute-0051.HELP.tex}

\[\lim\limits_{x\to\infty} 5 \, {\left(x + 4\right)}^{2} e^{\left(-x - 1\right)}=\answer{0}\]
\end{problem}}

%%%%%%%%%%%%%%%%%%%%%%

\latexProblemContent{
\ifVerboseLocation This is Derivative Compute Question 0051. \\ \fi
\begin{problem}

Find the limit.  Use L'H$\hat{o}$pital's rule where appropriate.

\input{Derivative-Compute-0051.HELP.tex}

\[\lim\limits_{x\to\infty} -5 \, {\left(x - 8\right)}^{2} e^{\left(-x + 6\right)}=\answer{0}\]
\end{problem}}

%%%%%%%%%%%%%%%%%%%%%%

\latexProblemContent{
\ifVerboseLocation This is Derivative Compute Question 0051. \\ \fi
\begin{problem}

Find the limit.  Use L'H$\hat{o}$pital's rule where appropriate.

\input{Derivative-Compute-0051.HELP.tex}

\[\lim\limits_{x\to\infty} -3 \, {\left(x - 3\right)}^{2} e^{\left(-x - 6\right)}=\answer{0}\]
\end{problem}}

%%%%%%%%%%%%%%%%%%%%%%

\latexProblemContent{
\ifVerboseLocation This is Derivative Compute Question 0051. \\ \fi
\begin{problem}

Find the limit.  Use L'H$\hat{o}$pital's rule where appropriate.

\input{Derivative-Compute-0051.HELP.tex}

\[\lim\limits_{x\to\infty} -4 \, {\left(x + 3\right)}^{2} e^{\left(-x - 3\right)}=\answer{0}\]
\end{problem}}

%%%%%%%%%%%%%%%%%%%%%%

\latexProblemContent{
\ifVerboseLocation This is Derivative Compute Question 0051. \\ \fi
\begin{problem}

Find the limit.  Use L'H$\hat{o}$pital's rule where appropriate.

\input{Derivative-Compute-0051.HELP.tex}

\[\lim\limits_{x\to\infty} -3 \, {\left(x - 1\right)}^{2} e^{\left(-x - 3\right)}=\answer{0}\]
\end{problem}}

%%%%%%%%%%%%%%%%%%%%%%

\latexProblemContent{
\ifVerboseLocation This is Derivative Compute Question 0051. \\ \fi
\begin{problem}

Find the limit.  Use L'H$\hat{o}$pital's rule where appropriate.

\input{Derivative-Compute-0051.HELP.tex}

\[\lim\limits_{x\to\infty} 4 \, {\left(x - 7\right)}^{2} e^{\left(-x - 3\right)}=\answer{0}\]
\end{problem}}

%%%%%%%%%%%%%%%%%%%%%%

\latexProblemContent{
\ifVerboseLocation This is Derivative Compute Question 0051. \\ \fi
\begin{problem}

Find the limit.  Use L'H$\hat{o}$pital's rule where appropriate.

\input{Derivative-Compute-0051.HELP.tex}

\[\lim\limits_{x\to\infty} 8 \, {\left(x - 8\right)}^{2} e^{\left(-x + 2\right)}=\answer{0}\]
\end{problem}}

%%%%%%%%%%%%%%%%%%%%%%

\latexProblemContent{
\ifVerboseLocation This is Derivative Compute Question 0051. \\ \fi
\begin{problem}

Find the limit.  Use L'H$\hat{o}$pital's rule where appropriate.

\input{Derivative-Compute-0051.HELP.tex}

\[\lim\limits_{x\to\infty} -5 \, {\left(x - 5\right)}^{2} e^{\left(-x + 2\right)}=\answer{0}\]
\end{problem}}

%%%%%%%%%%%%%%%%%%%%%%

\latexProblemContent{
\ifVerboseLocation This is Derivative Compute Question 0051. \\ \fi
\begin{problem}

Find the limit.  Use L'H$\hat{o}$pital's rule where appropriate.

\input{Derivative-Compute-0051.HELP.tex}

\[\lim\limits_{x\to\infty} 4 \, {\left(x + 1\right)}^{3} e^{\left(-x + 5\right)}=\answer{0}\]
\end{problem}}

%%%%%%%%%%%%%%%%%%%%%%

\latexProblemContent{
\ifVerboseLocation This is Derivative Compute Question 0051. \\ \fi
\begin{problem}

Find the limit.  Use L'H$\hat{o}$pital's rule where appropriate.

\input{Derivative-Compute-0051.HELP.tex}

\[\lim\limits_{x\to\infty} -3 \, {\left(x - 3\right)}^{3} e^{\left(-x + 6\right)}=\answer{0}\]
\end{problem}}

%%%%%%%%%%%%%%%%%%%%%%

\latexProblemContent{
\ifVerboseLocation This is Derivative Compute Question 0051. \\ \fi
\begin{problem}

Find the limit.  Use L'H$\hat{o}$pital's rule where appropriate.

\input{Derivative-Compute-0051.HELP.tex}

\[\lim\limits_{x\to\infty} 6 \, {\left(x + 4\right)}^{2} e^{\left(-x + 1\right)}=\answer{0}\]
\end{problem}}

%%%%%%%%%%%%%%%%%%%%%%

\latexProblemContent{
\ifVerboseLocation This is Derivative Compute Question 0051. \\ \fi
\begin{problem}

Find the limit.  Use L'H$\hat{o}$pital's rule where appropriate.

\input{Derivative-Compute-0051.HELP.tex}

\[\lim\limits_{x\to\infty} 4 \, {\left(x - 3\right)}^{3} e^{\left(-x + 6\right)}=\answer{0}\]
\end{problem}}

%%%%%%%%%%%%%%%%%%%%%%

\latexProblemContent{
\ifVerboseLocation This is Derivative Compute Question 0051. \\ \fi
\begin{problem}

Find the limit.  Use L'H$\hat{o}$pital's rule where appropriate.

\input{Derivative-Compute-0051.HELP.tex}

\[\lim\limits_{x\to\infty} 4 \, {\left(x + 4\right)} e^{\left(-x + 8\right)}=\answer{0}\]
\end{problem}}

%%%%%%%%%%%%%%%%%%%%%%

\latexProblemContent{
\ifVerboseLocation This is Derivative Compute Question 0051. \\ \fi
\begin{problem}

Find the limit.  Use L'H$\hat{o}$pital's rule where appropriate.

\input{Derivative-Compute-0051.HELP.tex}

\[\lim\limits_{x\to\infty} {\left(x - 8\right)} e^{\left(-x + 7\right)}=\answer{0}\]
\end{problem}}

%%%%%%%%%%%%%%%%%%%%%%

\latexProblemContent{
\ifVerboseLocation This is Derivative Compute Question 0051. \\ \fi
\begin{problem}

Find the limit.  Use L'H$\hat{o}$pital's rule where appropriate.

\input{Derivative-Compute-0051.HELP.tex}

\[\lim\limits_{x\to\infty} -8 \, {\left(x + 3\right)} e^{\left(-x + 8\right)}=\answer{0}\]
\end{problem}}

%%%%%%%%%%%%%%%%%%%%%%

\latexProblemContent{
\ifVerboseLocation This is Derivative Compute Question 0051. \\ \fi
\begin{problem}

Find the limit.  Use L'H$\hat{o}$pital's rule where appropriate.

\input{Derivative-Compute-0051.HELP.tex}

\[\lim\limits_{x\to\infty} 8 \, {\left(x - 8\right)}^{2} e^{\left(-x + 5\right)}=\answer{0}\]
\end{problem}}

%%%%%%%%%%%%%%%%%%%%%%

\latexProblemContent{
\ifVerboseLocation This is Derivative Compute Question 0051. \\ \fi
\begin{problem}

Find the limit.  Use L'H$\hat{o}$pital's rule where appropriate.

\input{Derivative-Compute-0051.HELP.tex}

\[\lim\limits_{x\to\infty} 6 \, {\left(x - 4\right)}^{3} e^{\left(-x + 4\right)}=\answer{0}\]
\end{problem}}

%%%%%%%%%%%%%%%%%%%%%%

\latexProblemContent{
\ifVerboseLocation This is Derivative Compute Question 0051. \\ \fi
\begin{problem}

Find the limit.  Use L'H$\hat{o}$pital's rule where appropriate.

\input{Derivative-Compute-0051.HELP.tex}

\[\lim\limits_{x\to\infty} -{\left(x + 6\right)}^{3} e^{\left(-x + 3\right)}=\answer{0}\]
\end{problem}}

%%%%%%%%%%%%%%%%%%%%%%

\latexProblemContent{
\ifVerboseLocation This is Derivative Compute Question 0051. \\ \fi
\begin{problem}

Find the limit.  Use L'H$\hat{o}$pital's rule where appropriate.

\input{Derivative-Compute-0051.HELP.tex}

\[\lim\limits_{x\to\infty} -6 \, {\left(x - 8\right)}^{3} e^{\left(-x - 2\right)}=\answer{0}\]
\end{problem}}

%%%%%%%%%%%%%%%%%%%%%%

\latexProblemContent{
\ifVerboseLocation This is Derivative Compute Question 0051. \\ \fi
\begin{problem}

Find the limit.  Use L'H$\hat{o}$pital's rule where appropriate.

\input{Derivative-Compute-0051.HELP.tex}

\[\lim\limits_{x\to\infty} -2 \, {\left(x - 1\right)} e^{\left(-x + 3\right)}=\answer{0}\]
\end{problem}}

%%%%%%%%%%%%%%%%%%%%%%

\latexProblemContent{
\ifVerboseLocation This is Derivative Compute Question 0051. \\ \fi
\begin{problem}

Find the limit.  Use L'H$\hat{o}$pital's rule where appropriate.

\input{Derivative-Compute-0051.HELP.tex}

\[\lim\limits_{x\to\infty} 7 \, {\left(x - 3\right)}^{3} e^{\left(-x - 8\right)}=\answer{0}\]
\end{problem}}

%%%%%%%%%%%%%%%%%%%%%%

\latexProblemContent{
\ifVerboseLocation This is Derivative Compute Question 0051. \\ \fi
\begin{problem}

Find the limit.  Use L'H$\hat{o}$pital's rule where appropriate.

\input{Derivative-Compute-0051.HELP.tex}

\[\lim\limits_{x\to\infty} -7 \, {\left(x + 6\right)} e^{\left(-x - 6\right)}=\answer{0}\]
\end{problem}}

%%%%%%%%%%%%%%%%%%%%%%

\latexProblemContent{
\ifVerboseLocation This is Derivative Compute Question 0051. \\ \fi
\begin{problem}

Find the limit.  Use L'H$\hat{o}$pital's rule where appropriate.

\input{Derivative-Compute-0051.HELP.tex}

\[\lim\limits_{x\to\infty} -6 \, {\left(x - 8\right)} e^{\left(-x - 4\right)}=\answer{0}\]
\end{problem}}

%%%%%%%%%%%%%%%%%%%%%%

\latexProblemContent{
\ifVerboseLocation This is Derivative Compute Question 0051. \\ \fi
\begin{problem}

Find the limit.  Use L'H$\hat{o}$pital's rule where appropriate.

\input{Derivative-Compute-0051.HELP.tex}

\[\lim\limits_{x\to\infty} -8 \, {\left(x + 6\right)}^{2} e^{\left(-x + 1\right)}=\answer{0}\]
\end{problem}}

%%%%%%%%%%%%%%%%%%%%%%

\latexProblemContent{
\ifVerboseLocation This is Derivative Compute Question 0051. \\ \fi
\begin{problem}

Find the limit.  Use L'H$\hat{o}$pital's rule where appropriate.

\input{Derivative-Compute-0051.HELP.tex}

\[\lim\limits_{x\to\infty} 7 \, {\left(x - 2\right)}^{3} e^{\left(-x + 4\right)}=\answer{0}\]
\end{problem}}

%%%%%%%%%%%%%%%%%%%%%%

\latexProblemContent{
\ifVerboseLocation This is Derivative Compute Question 0051. \\ \fi
\begin{problem}

Find the limit.  Use L'H$\hat{o}$pital's rule where appropriate.

\input{Derivative-Compute-0051.HELP.tex}

\[\lim\limits_{x\to\infty} -2 \, {\left(x - 8\right)}^{3} e^{\left(-x + 6\right)}=\answer{0}\]
\end{problem}}

%%%%%%%%%%%%%%%%%%%%%%

\latexProblemContent{
\ifVerboseLocation This is Derivative Compute Question 0051. \\ \fi
\begin{problem}

Find the limit.  Use L'H$\hat{o}$pital's rule where appropriate.

\input{Derivative-Compute-0051.HELP.tex}

\[\lim\limits_{x\to\infty} -5 \, {\left(x - 6\right)}^{3} e^{\left(-x + 3\right)}=\answer{0}\]
\end{problem}}

%%%%%%%%%%%%%%%%%%%%%%

\latexProblemContent{
\ifVerboseLocation This is Derivative Compute Question 0051. \\ \fi
\begin{problem}

Find the limit.  Use L'H$\hat{o}$pital's rule where appropriate.

\input{Derivative-Compute-0051.HELP.tex}

\[\lim\limits_{x\to\infty} 5 \, {\left(x + 4\right)}^{3} e^{\left(-x + 5\right)}=\answer{0}\]
\end{problem}}

%%%%%%%%%%%%%%%%%%%%%%

\latexProblemContent{
\ifVerboseLocation This is Derivative Compute Question 0051. \\ \fi
\begin{problem}

Find the limit.  Use L'H$\hat{o}$pital's rule where appropriate.

\input{Derivative-Compute-0051.HELP.tex}

\[\lim\limits_{x\to\infty} 7 \, {\left(x - 7\right)}^{3} e^{\left(-x + 3\right)}=\answer{0}\]
\end{problem}}

%%%%%%%%%%%%%%%%%%%%%%

\latexProblemContent{
\ifVerboseLocation This is Derivative Compute Question 0051. \\ \fi
\begin{problem}

Find the limit.  Use L'H$\hat{o}$pital's rule where appropriate.

\input{Derivative-Compute-0051.HELP.tex}

\[\lim\limits_{x\to\infty} -3 \, {\left(x + 2\right)} e^{\left(-x - 8\right)}=\answer{0}\]
\end{problem}}

%%%%%%%%%%%%%%%%%%%%%%

\latexProblemContent{
\ifVerboseLocation This is Derivative Compute Question 0051. \\ \fi
\begin{problem}

Find the limit.  Use L'H$\hat{o}$pital's rule where appropriate.

\input{Derivative-Compute-0051.HELP.tex}

\[\lim\limits_{x\to\infty} -6 \, {\left(x + 6\right)}^{2} e^{\left(-x + 8\right)}=\answer{0}\]
\end{problem}}

%%%%%%%%%%%%%%%%%%%%%%

\latexProblemContent{
\ifVerboseLocation This is Derivative Compute Question 0051. \\ \fi
\begin{problem}

Find the limit.  Use L'H$\hat{o}$pital's rule where appropriate.

\input{Derivative-Compute-0051.HELP.tex}

\[\lim\limits_{x\to\infty} -2 \, {\left(x + 6\right)}^{2} e^{\left(-x - 8\right)}=\answer{0}\]
\end{problem}}

%%%%%%%%%%%%%%%%%%%%%%

\latexProblemContent{
\ifVerboseLocation This is Derivative Compute Question 0051. \\ \fi
\begin{problem}

Find the limit.  Use L'H$\hat{o}$pital's rule where appropriate.

\input{Derivative-Compute-0051.HELP.tex}

\[\lim\limits_{x\to\infty} 6 \, {\left(x + 1\right)}^{3} e^{\left(-x + 7\right)}=\answer{0}\]
\end{problem}}

%%%%%%%%%%%%%%%%%%%%%%

\latexProblemContent{
\ifVerboseLocation This is Derivative Compute Question 0051. \\ \fi
\begin{problem}

Find the limit.  Use L'H$\hat{o}$pital's rule where appropriate.

\input{Derivative-Compute-0051.HELP.tex}

\[\lim\limits_{x\to\infty} 3 \, {\left(x + 8\right)} e^{\left(-x - 5\right)}=\answer{0}\]
\end{problem}}

%%%%%%%%%%%%%%%%%%%%%%

\latexProblemContent{
\ifVerboseLocation This is Derivative Compute Question 0051. \\ \fi
\begin{problem}

Find the limit.  Use L'H$\hat{o}$pital's rule where appropriate.

\input{Derivative-Compute-0051.HELP.tex}

\[\lim\limits_{x\to\infty} -8 \, {\left(x + 5\right)} e^{\left(-x - 2\right)}=\answer{0}\]
\end{problem}}

%%%%%%%%%%%%%%%%%%%%%%

\latexProblemContent{
\ifVerboseLocation This is Derivative Compute Question 0051. \\ \fi
\begin{problem}

Find the limit.  Use L'H$\hat{o}$pital's rule where appropriate.

\input{Derivative-Compute-0051.HELP.tex}

\[\lim\limits_{x\to\infty} 4 \, {\left(x - 6\right)}^{2} e^{\left(-x + 3\right)}=\answer{0}\]
\end{problem}}

%%%%%%%%%%%%%%%%%%%%%%

\latexProblemContent{
\ifVerboseLocation This is Derivative Compute Question 0051. \\ \fi
\begin{problem}

Find the limit.  Use L'H$\hat{o}$pital's rule where appropriate.

\input{Derivative-Compute-0051.HELP.tex}

\[\lim\limits_{x\to\infty} 2 \, {\left(x - 1\right)}^{3} e^{\left(-x + 1\right)}=\answer{0}\]
\end{problem}}

%%%%%%%%%%%%%%%%%%%%%%

\latexProblemContent{
\ifVerboseLocation This is Derivative Compute Question 0051. \\ \fi
\begin{problem}

Find the limit.  Use L'H$\hat{o}$pital's rule where appropriate.

\input{Derivative-Compute-0051.HELP.tex}

\[\lim\limits_{x\to\infty} -3 \, {\left(x + 7\right)}^{3} e^{\left(-x + 2\right)}=\answer{0}\]
\end{problem}}

%%%%%%%%%%%%%%%%%%%%%%

\latexProblemContent{
\ifVerboseLocation This is Derivative Compute Question 0051. \\ \fi
\begin{problem}

Find the limit.  Use L'H$\hat{o}$pital's rule where appropriate.

\input{Derivative-Compute-0051.HELP.tex}

\[\lim\limits_{x\to\infty} 3 \, {\left(x + 3\right)}^{2} e^{\left(-x + 3\right)}=\answer{0}\]
\end{problem}}

%%%%%%%%%%%%%%%%%%%%%%

\latexProblemContent{
\ifVerboseLocation This is Derivative Compute Question 0051. \\ \fi
\begin{problem}

Find the limit.  Use L'H$\hat{o}$pital's rule where appropriate.

\input{Derivative-Compute-0051.HELP.tex}

\[\lim\limits_{x\to\infty} {\left(x + 2\right)}^{3} e^{\left(-x - 5\right)}=\answer{0}\]
\end{problem}}

%%%%%%%%%%%%%%%%%%%%%%

\latexProblemContent{
\ifVerboseLocation This is Derivative Compute Question 0051. \\ \fi
\begin{problem}

Find the limit.  Use L'H$\hat{o}$pital's rule where appropriate.

\input{Derivative-Compute-0051.HELP.tex}

\[\lim\limits_{x\to\infty} 3 \, {\left(x + 4\right)}^{3} e^{\left(-x + 3\right)}=\answer{0}\]
\end{problem}}

%%%%%%%%%%%%%%%%%%%%%%

\latexProblemContent{
\ifVerboseLocation This is Derivative Compute Question 0051. \\ \fi
\begin{problem}

Find the limit.  Use L'H$\hat{o}$pital's rule where appropriate.

\input{Derivative-Compute-0051.HELP.tex}

\[\lim\limits_{x\to\infty} -7 \, {\left(x - 6\right)}^{3} e^{\left(-x + 5\right)}=\answer{0}\]
\end{problem}}

%%%%%%%%%%%%%%%%%%%%%%

\latexProblemContent{
\ifVerboseLocation This is Derivative Compute Question 0051. \\ \fi
\begin{problem}

Find the limit.  Use L'H$\hat{o}$pital's rule where appropriate.

\input{Derivative-Compute-0051.HELP.tex}

\[\lim\limits_{x\to\infty} 5 \, {\left(x - 8\right)}^{2} e^{\left(-x + 2\right)}=\answer{0}\]
\end{problem}}

%%%%%%%%%%%%%%%%%%%%%%

\latexProblemContent{
\ifVerboseLocation This is Derivative Compute Question 0051. \\ \fi
\begin{problem}

Find the limit.  Use L'H$\hat{o}$pital's rule where appropriate.

\input{Derivative-Compute-0051.HELP.tex}

\[\lim\limits_{x\to\infty} 4 \, {\left(x - 1\right)} e^{\left(-x - 4\right)}=\answer{0}\]
\end{problem}}

%%%%%%%%%%%%%%%%%%%%%%

\latexProblemContent{
\ifVerboseLocation This is Derivative Compute Question 0051. \\ \fi
\begin{problem}

Find the limit.  Use L'H$\hat{o}$pital's rule where appropriate.

\input{Derivative-Compute-0051.HELP.tex}

\[\lim\limits_{x\to\infty} 8 \, {\left(x + 5\right)}^{3} e^{\left(-x + 1\right)}=\answer{0}\]
\end{problem}}

%%%%%%%%%%%%%%%%%%%%%%

\latexProblemContent{
\ifVerboseLocation This is Derivative Compute Question 0051. \\ \fi
\begin{problem}

Find the limit.  Use L'H$\hat{o}$pital's rule where appropriate.

\input{Derivative-Compute-0051.HELP.tex}

\[\lim\limits_{x\to\infty} 8 \, {\left(x - 5\right)}^{2} e^{\left(-x + 3\right)}=\answer{0}\]
\end{problem}}

%%%%%%%%%%%%%%%%%%%%%%

\latexProblemContent{
\ifVerboseLocation This is Derivative Compute Question 0051. \\ \fi
\begin{problem}

Find the limit.  Use L'H$\hat{o}$pital's rule where appropriate.

\input{Derivative-Compute-0051.HELP.tex}

\[\lim\limits_{x\to\infty} -6 \, {\left(x - 7\right)}^{2} e^{\left(-x + 3\right)}=\answer{0}\]
\end{problem}}

%%%%%%%%%%%%%%%%%%%%%%

\latexProblemContent{
\ifVerboseLocation This is Derivative Compute Question 0051. \\ \fi
\begin{problem}

Find the limit.  Use L'H$\hat{o}$pital's rule where appropriate.

\input{Derivative-Compute-0051.HELP.tex}

\[\lim\limits_{x\to\infty} 2 \, {\left(x + 6\right)}^{3} e^{\left(-x + 1\right)}=\answer{0}\]
\end{problem}}

%%%%%%%%%%%%%%%%%%%%%%

\latexProblemContent{
\ifVerboseLocation This is Derivative Compute Question 0051. \\ \fi
\begin{problem}

Find the limit.  Use L'H$\hat{o}$pital's rule where appropriate.

\input{Derivative-Compute-0051.HELP.tex}

\[\lim\limits_{x\to\infty} 3 \, {\left(x + 5\right)}^{3} e^{\left(-x + 7\right)}=\answer{0}\]
\end{problem}}

%%%%%%%%%%%%%%%%%%%%%%

\latexProblemContent{
\ifVerboseLocation This is Derivative Compute Question 0051. \\ \fi
\begin{problem}

Find the limit.  Use L'H$\hat{o}$pital's rule where appropriate.

\input{Derivative-Compute-0051.HELP.tex}

\[\lim\limits_{x\to\infty} -2 \, {\left(x - 6\right)}^{3} e^{\left(-x - 1\right)}=\answer{0}\]
\end{problem}}

%%%%%%%%%%%%%%%%%%%%%%

\latexProblemContent{
\ifVerboseLocation This is Derivative Compute Question 0051. \\ \fi
\begin{problem}

Find the limit.  Use L'H$\hat{o}$pital's rule where appropriate.

\input{Derivative-Compute-0051.HELP.tex}

\[\lim\limits_{x\to\infty} 3 \, {\left(x - 1\right)}^{2} e^{\left(-x + 1\right)}=\answer{0}\]
\end{problem}}

%%%%%%%%%%%%%%%%%%%%%%

\latexProblemContent{
\ifVerboseLocation This is Derivative Compute Question 0051. \\ \fi
\begin{problem}

Find the limit.  Use L'H$\hat{o}$pital's rule where appropriate.

\input{Derivative-Compute-0051.HELP.tex}

\[\lim\limits_{x\to\infty} -{\left(x + 7\right)}^{3} e^{\left(-x + 8\right)}=\answer{0}\]
\end{problem}}

%%%%%%%%%%%%%%%%%%%%%%

\latexProblemContent{
\ifVerboseLocation This is Derivative Compute Question 0051. \\ \fi
\begin{problem}

Find the limit.  Use L'H$\hat{o}$pital's rule where appropriate.

\input{Derivative-Compute-0051.HELP.tex}

\[\lim\limits_{x\to\infty} 6 \, {\left(x + 5\right)}^{3} e^{\left(-x - 6\right)}=\answer{0}\]
\end{problem}}

%%%%%%%%%%%%%%%%%%%%%%

\latexProblemContent{
\ifVerboseLocation This is Derivative Compute Question 0051. \\ \fi
\begin{problem}

Find the limit.  Use L'H$\hat{o}$pital's rule where appropriate.

\input{Derivative-Compute-0051.HELP.tex}

\[\lim\limits_{x\to\infty} -2 \, {\left(x + 1\right)} e^{\left(-x - 2\right)}=\answer{0}\]
\end{problem}}

%%%%%%%%%%%%%%%%%%%%%%

\latexProblemContent{
\ifVerboseLocation This is Derivative Compute Question 0051. \\ \fi
\begin{problem}

Find the limit.  Use L'H$\hat{o}$pital's rule where appropriate.

\input{Derivative-Compute-0051.HELP.tex}

\[\lim\limits_{x\to\infty} -7 \, {\left(x - 3\right)}^{3} e^{\left(-x - 4\right)}=\answer{0}\]
\end{problem}}

%%%%%%%%%%%%%%%%%%%%%%

\latexProblemContent{
\ifVerboseLocation This is Derivative Compute Question 0051. \\ \fi
\begin{problem}

Find the limit.  Use L'H$\hat{o}$pital's rule where appropriate.

\input{Derivative-Compute-0051.HELP.tex}

\[\lim\limits_{x\to\infty} 8 \, {\left(x + 5\right)} e^{\left(-x + 6\right)}=\answer{0}\]
\end{problem}}

%%%%%%%%%%%%%%%%%%%%%%

\latexProblemContent{
\ifVerboseLocation This is Derivative Compute Question 0051. \\ \fi
\begin{problem}

Find the limit.  Use L'H$\hat{o}$pital's rule where appropriate.

\input{Derivative-Compute-0051.HELP.tex}

\[\lim\limits_{x\to\infty} 8 \, {\left(x - 2\right)}^{3} e^{\left(-x + 4\right)}=\answer{0}\]
\end{problem}}

%%%%%%%%%%%%%%%%%%%%%%

\latexProblemContent{
\ifVerboseLocation This is Derivative Compute Question 0051. \\ \fi
\begin{problem}

Find the limit.  Use L'H$\hat{o}$pital's rule where appropriate.

\input{Derivative-Compute-0051.HELP.tex}

\[\lim\limits_{x\to\infty} -4 \, {\left(x - 2\right)}^{3} e^{\left(-x - 2\right)}=\answer{0}\]
\end{problem}}

%%%%%%%%%%%%%%%%%%%%%%

\latexProblemContent{
\ifVerboseLocation This is Derivative Compute Question 0051. \\ \fi
\begin{problem}

Find the limit.  Use L'H$\hat{o}$pital's rule where appropriate.

\input{Derivative-Compute-0051.HELP.tex}

\[\lim\limits_{x\to\infty} 2 \, {\left(x + 6\right)}^{2} e^{\left(-x - 3\right)}=\answer{0}\]
\end{problem}}

%%%%%%%%%%%%%%%%%%%%%%

\latexProblemContent{
\ifVerboseLocation This is Derivative Compute Question 0051. \\ \fi
\begin{problem}

Find the limit.  Use L'H$\hat{o}$pital's rule where appropriate.

\input{Derivative-Compute-0051.HELP.tex}

\[\lim\limits_{x\to\infty} -2 \, {\left(x + 7\right)}^{3} e^{\left(-x + 4\right)}=\answer{0}\]
\end{problem}}

%%%%%%%%%%%%%%%%%%%%%%

\latexProblemContent{
\ifVerboseLocation This is Derivative Compute Question 0051. \\ \fi
\begin{problem}

Find the limit.  Use L'H$\hat{o}$pital's rule where appropriate.

\input{Derivative-Compute-0051.HELP.tex}

\[\lim\limits_{x\to\infty} -5 \, {\left(x - 6\right)} e^{\left(-x + 7\right)}=\answer{0}\]
\end{problem}}

%%%%%%%%%%%%%%%%%%%%%%

\latexProblemContent{
\ifVerboseLocation This is Derivative Compute Question 0051. \\ \fi
\begin{problem}

Find the limit.  Use L'H$\hat{o}$pital's rule where appropriate.

\input{Derivative-Compute-0051.HELP.tex}

\[\lim\limits_{x\to\infty} 6 \, {\left(x + 6\right)}^{2} e^{\left(-x - 5\right)}=\answer{0}\]
\end{problem}}

%%%%%%%%%%%%%%%%%%%%%%

\latexProblemContent{
\ifVerboseLocation This is Derivative Compute Question 0051. \\ \fi
\begin{problem}

Find the limit.  Use L'H$\hat{o}$pital's rule where appropriate.

\input{Derivative-Compute-0051.HELP.tex}

\[\lim\limits_{x\to\infty} -8 \, {\left(x - 5\right)} e^{\left(-x + 7\right)}=\answer{0}\]
\end{problem}}

%%%%%%%%%%%%%%%%%%%%%%

\latexProblemContent{
\ifVerboseLocation This is Derivative Compute Question 0051. \\ \fi
\begin{problem}

Find the limit.  Use L'H$\hat{o}$pital's rule where appropriate.

\input{Derivative-Compute-0051.HELP.tex}

\[\lim\limits_{x\to\infty} 7 \, {\left(x - 6\right)}^{3} e^{\left(-x + 6\right)}=\answer{0}\]
\end{problem}}

%%%%%%%%%%%%%%%%%%%%%%

\latexProblemContent{
\ifVerboseLocation This is Derivative Compute Question 0051. \\ \fi
\begin{problem}

Find the limit.  Use L'H$\hat{o}$pital's rule where appropriate.

\input{Derivative-Compute-0051.HELP.tex}

\[\lim\limits_{x\to\infty} {\left(x + 4\right)}^{3} e^{\left(-x + 2\right)}=\answer{0}\]
\end{problem}}

%%%%%%%%%%%%%%%%%%%%%%

\latexProblemContent{
\ifVerboseLocation This is Derivative Compute Question 0051. \\ \fi
\begin{problem}

Find the limit.  Use L'H$\hat{o}$pital's rule where appropriate.

\input{Derivative-Compute-0051.HELP.tex}

\[\lim\limits_{x\to\infty} 8 \, {\left(x + 7\right)}^{3} e^{\left(-x + 6\right)}=\answer{0}\]
\end{problem}}

%%%%%%%%%%%%%%%%%%%%%%

\latexProblemContent{
\ifVerboseLocation This is Derivative Compute Question 0051. \\ \fi
\begin{problem}

Find the limit.  Use L'H$\hat{o}$pital's rule where appropriate.

\input{Derivative-Compute-0051.HELP.tex}

\[\lim\limits_{x\to\infty} 3 \, {\left(x + 5\right)}^{2} e^{\left(-x - 8\right)}=\answer{0}\]
\end{problem}}

%%%%%%%%%%%%%%%%%%%%%%

\latexProblemContent{
\ifVerboseLocation This is Derivative Compute Question 0051. \\ \fi
\begin{problem}

Find the limit.  Use L'H$\hat{o}$pital's rule where appropriate.

\input{Derivative-Compute-0051.HELP.tex}

\[\lim\limits_{x\to\infty} -2 \, {\left(x + 2\right)}^{3} e^{\left(-x + 2\right)}=\answer{0}\]
\end{problem}}

%%%%%%%%%%%%%%%%%%%%%%

\latexProblemContent{
\ifVerboseLocation This is Derivative Compute Question 0051. \\ \fi
\begin{problem}

Find the limit.  Use L'H$\hat{o}$pital's rule where appropriate.

\input{Derivative-Compute-0051.HELP.tex}

\[\lim\limits_{x\to\infty} 8 \, {\left(x + 8\right)} e^{\left(-x + 2\right)}=\answer{0}\]
\end{problem}}

%%%%%%%%%%%%%%%%%%%%%%

\latexProblemContent{
\ifVerboseLocation This is Derivative Compute Question 0051. \\ \fi
\begin{problem}

Find the limit.  Use L'H$\hat{o}$pital's rule where appropriate.

\input{Derivative-Compute-0051.HELP.tex}

\[\lim\limits_{x\to\infty} -2 \, {\left(x - 8\right)} e^{\left(-x - 5\right)}=\answer{0}\]
\end{problem}}

%%%%%%%%%%%%%%%%%%%%%%

\latexProblemContent{
\ifVerboseLocation This is Derivative Compute Question 0051. \\ \fi
\begin{problem}

Find the limit.  Use L'H$\hat{o}$pital's rule where appropriate.

\input{Derivative-Compute-0051.HELP.tex}

\[\lim\limits_{x\to\infty} -6 \, {\left(x - 4\right)}^{3} e^{\left(-x - 1\right)}=\answer{0}\]
\end{problem}}

%%%%%%%%%%%%%%%%%%%%%%

\latexProblemContent{
\ifVerboseLocation This is Derivative Compute Question 0051. \\ \fi
\begin{problem}

Find the limit.  Use L'H$\hat{o}$pital's rule where appropriate.

\input{Derivative-Compute-0051.HELP.tex}

\[\lim\limits_{x\to\infty} 2 \, {\left(x + 4\right)}^{3} e^{\left(-x - 2\right)}=\answer{0}\]
\end{problem}}

%%%%%%%%%%%%%%%%%%%%%%

\latexProblemContent{
\ifVerboseLocation This is Derivative Compute Question 0051. \\ \fi
\begin{problem}

Find the limit.  Use L'H$\hat{o}$pital's rule where appropriate.

\input{Derivative-Compute-0051.HELP.tex}

\[\lim\limits_{x\to\infty} 4 \, {\left(x + 8\right)}^{2} e^{\left(-x - 4\right)}=\answer{0}\]
\end{problem}}

%%%%%%%%%%%%%%%%%%%%%%

\latexProblemContent{
\ifVerboseLocation This is Derivative Compute Question 0051. \\ \fi
\begin{problem}

Find the limit.  Use L'H$\hat{o}$pital's rule where appropriate.

\input{Derivative-Compute-0051.HELP.tex}

\[\lim\limits_{x\to\infty} -{\left(x + 5\right)}^{2} e^{\left(-x + 8\right)}=\answer{0}\]
\end{problem}}

%%%%%%%%%%%%%%%%%%%%%%

\latexProblemContent{
\ifVerboseLocation This is Derivative Compute Question 0051. \\ \fi
\begin{problem}

Find the limit.  Use L'H$\hat{o}$pital's rule where appropriate.

\input{Derivative-Compute-0051.HELP.tex}

\[\lim\limits_{x\to\infty} -8 \, {\left(x - 1\right)}^{2} e^{\left(-x + 4\right)}=\answer{0}\]
\end{problem}}

%%%%%%%%%%%%%%%%%%%%%%

\latexProblemContent{
\ifVerboseLocation This is Derivative Compute Question 0051. \\ \fi
\begin{problem}

Find the limit.  Use L'H$\hat{o}$pital's rule where appropriate.

\input{Derivative-Compute-0051.HELP.tex}

\[\lim\limits_{x\to\infty} {\left(x - 1\right)}^{2} e^{\left(-x - 5\right)}=\answer{0}\]
\end{problem}}

%%%%%%%%%%%%%%%%%%%%%%

\latexProblemContent{
\ifVerboseLocation This is Derivative Compute Question 0051. \\ \fi
\begin{problem}

Find the limit.  Use L'H$\hat{o}$pital's rule where appropriate.

\input{Derivative-Compute-0051.HELP.tex}

\[\lim\limits_{x\to\infty} 7 \, {\left(x - 2\right)} e^{\left(-x + 1\right)}=\answer{0}\]
\end{problem}}

%%%%%%%%%%%%%%%%%%%%%%

\latexProblemContent{
\ifVerboseLocation This is Derivative Compute Question 0051. \\ \fi
\begin{problem}

Find the limit.  Use L'H$\hat{o}$pital's rule where appropriate.

\input{Derivative-Compute-0051.HELP.tex}

\[\lim\limits_{x\to\infty} 8 \, {\left(x + 5\right)}^{2} e^{\left(-x + 6\right)}=\answer{0}\]
\end{problem}}

%%%%%%%%%%%%%%%%%%%%%%

\latexProblemContent{
\ifVerboseLocation This is Derivative Compute Question 0051. \\ \fi
\begin{problem}

Find the limit.  Use L'H$\hat{o}$pital's rule where appropriate.

\input{Derivative-Compute-0051.HELP.tex}

\[\lim\limits_{x\to\infty} 2 \, {\left(x + 2\right)} e^{\left(-x - 5\right)}=\answer{0}\]
\end{problem}}

%%%%%%%%%%%%%%%%%%%%%%

\latexProblemContent{
\ifVerboseLocation This is Derivative Compute Question 0051. \\ \fi
\begin{problem}

Find the limit.  Use L'H$\hat{o}$pital's rule where appropriate.

\input{Derivative-Compute-0051.HELP.tex}

\[\lim\limits_{x\to\infty} -2 \, {\left(x + 4\right)}^{2} e^{\left(-x - 4\right)}=\answer{0}\]
\end{problem}}

%%%%%%%%%%%%%%%%%%%%%%

\latexProblemContent{
\ifVerboseLocation This is Derivative Compute Question 0051. \\ \fi
\begin{problem}

Find the limit.  Use L'H$\hat{o}$pital's rule where appropriate.

\input{Derivative-Compute-0051.HELP.tex}

\[\lim\limits_{x\to\infty} 8 \, {\left(x - 5\right)}^{3} e^{\left(-x - 8\right)}=\answer{0}\]
\end{problem}}

%%%%%%%%%%%%%%%%%%%%%%

\latexProblemContent{
\ifVerboseLocation This is Derivative Compute Question 0051. \\ \fi
\begin{problem}

Find the limit.  Use L'H$\hat{o}$pital's rule where appropriate.

\input{Derivative-Compute-0051.HELP.tex}

\[\lim\limits_{x\to\infty} 2 \, {\left(x + 6\right)}^{3} e^{\left(-x - 7\right)}=\answer{0}\]
\end{problem}}

%%%%%%%%%%%%%%%%%%%%%%

\latexProblemContent{
\ifVerboseLocation This is Derivative Compute Question 0051. \\ \fi
\begin{problem}

Find the limit.  Use L'H$\hat{o}$pital's rule where appropriate.

\input{Derivative-Compute-0051.HELP.tex}

\[\lim\limits_{x\to\infty} -{\left(x - 3\right)} e^{\left(-x - 4\right)}=\answer{0}\]
\end{problem}}

%%%%%%%%%%%%%%%%%%%%%%

\latexProblemContent{
\ifVerboseLocation This is Derivative Compute Question 0051. \\ \fi
\begin{problem}

Find the limit.  Use L'H$\hat{o}$pital's rule where appropriate.

\input{Derivative-Compute-0051.HELP.tex}

\[\lim\limits_{x\to\infty} 5 \, {\left(x - 5\right)}^{2} e^{\left(-x + 8\right)}=\answer{0}\]
\end{problem}}

%%%%%%%%%%%%%%%%%%%%%%

\latexProblemContent{
\ifVerboseLocation This is Derivative Compute Question 0051. \\ \fi
\begin{problem}

Find the limit.  Use L'H$\hat{o}$pital's rule where appropriate.

\input{Derivative-Compute-0051.HELP.tex}

\[\lim\limits_{x\to\infty} {\left(x - 3\right)} e^{\left(-x - 6\right)}=\answer{0}\]
\end{problem}}

%%%%%%%%%%%%%%%%%%%%%%

\latexProblemContent{
\ifVerboseLocation This is Derivative Compute Question 0051. \\ \fi
\begin{problem}

Find the limit.  Use L'H$\hat{o}$pital's rule where appropriate.

\input{Derivative-Compute-0051.HELP.tex}

\[\lim\limits_{x\to\infty} -6 \, {\left(x - 6\right)}^{3} e^{\left(-x - 4\right)}=\answer{0}\]
\end{problem}}

%%%%%%%%%%%%%%%%%%%%%%

\latexProblemContent{
\ifVerboseLocation This is Derivative Compute Question 0051. \\ \fi
\begin{problem}

Find the limit.  Use L'H$\hat{o}$pital's rule where appropriate.

\input{Derivative-Compute-0051.HELP.tex}

\[\lim\limits_{x\to\infty} -7 \, {\left(x - 2\right)} e^{\left(-x + 6\right)}=\answer{0}\]
\end{problem}}

%%%%%%%%%%%%%%%%%%%%%%

\latexProblemContent{
\ifVerboseLocation This is Derivative Compute Question 0051. \\ \fi
\begin{problem}

Find the limit.  Use L'H$\hat{o}$pital's rule where appropriate.

\input{Derivative-Compute-0051.HELP.tex}

\[\lim\limits_{x\to\infty} -6 \, {\left(x - 2\right)}^{3} e^{\left(-x + 5\right)}=\answer{0}\]
\end{problem}}

%%%%%%%%%%%%%%%%%%%%%%

\latexProblemContent{
\ifVerboseLocation This is Derivative Compute Question 0051. \\ \fi
\begin{problem}

Find the limit.  Use L'H$\hat{o}$pital's rule where appropriate.

\input{Derivative-Compute-0051.HELP.tex}

\[\lim\limits_{x\to\infty} 4 \, {\left(x + 5\right)}^{3} e^{\left(-x + 8\right)}=\answer{0}\]
\end{problem}}

%%%%%%%%%%%%%%%%%%%%%%

\latexProblemContent{
\ifVerboseLocation This is Derivative Compute Question 0051. \\ \fi
\begin{problem}

Find the limit.  Use L'H$\hat{o}$pital's rule where appropriate.

\input{Derivative-Compute-0051.HELP.tex}

\[\lim\limits_{x\to\infty} -2 \, {\left(x - 5\right)} e^{\left(-x + 1\right)}=\answer{0}\]
\end{problem}}

%%%%%%%%%%%%%%%%%%%%%%

\latexProblemContent{
\ifVerboseLocation This is Derivative Compute Question 0051. \\ \fi
\begin{problem}

Find the limit.  Use L'H$\hat{o}$pital's rule where appropriate.

\input{Derivative-Compute-0051.HELP.tex}

\[\lim\limits_{x\to\infty} 7 \, {\left(x - 8\right)} e^{\left(-x - 1\right)}=\answer{0}\]
\end{problem}}

%%%%%%%%%%%%%%%%%%%%%%

\latexProblemContent{
\ifVerboseLocation This is Derivative Compute Question 0051. \\ \fi
\begin{problem}

Find the limit.  Use L'H$\hat{o}$pital's rule where appropriate.

\input{Derivative-Compute-0051.HELP.tex}

\[\lim\limits_{x\to\infty} 8 \, {\left(x + 2\right)}^{2} e^{\left(-x - 3\right)}=\answer{0}\]
\end{problem}}

%%%%%%%%%%%%%%%%%%%%%%

\latexProblemContent{
\ifVerboseLocation This is Derivative Compute Question 0051. \\ \fi
\begin{problem}

Find the limit.  Use L'H$\hat{o}$pital's rule where appropriate.

\input{Derivative-Compute-0051.HELP.tex}

\[\lim\limits_{x\to\infty} 3 \, {\left(x + 7\right)}^{2} e^{\left(-x - 1\right)}=\answer{0}\]
\end{problem}}

%%%%%%%%%%%%%%%%%%%%%%

\latexProblemContent{
\ifVerboseLocation This is Derivative Compute Question 0051. \\ \fi
\begin{problem}

Find the limit.  Use L'H$\hat{o}$pital's rule where appropriate.

\input{Derivative-Compute-0051.HELP.tex}

\[\lim\limits_{x\to\infty} 4 \, {\left(x - 5\right)}^{2} e^{\left(-x + 6\right)}=\answer{0}\]
\end{problem}}

%%%%%%%%%%%%%%%%%%%%%%

\latexProblemContent{
\ifVerboseLocation This is Derivative Compute Question 0051. \\ \fi
\begin{problem}

Find the limit.  Use L'H$\hat{o}$pital's rule where appropriate.

\input{Derivative-Compute-0051.HELP.tex}

\[\lim\limits_{x\to\infty} -5 \, {\left(x + 5\right)}^{3} e^{\left(-x + 1\right)}=\answer{0}\]
\end{problem}}

%%%%%%%%%%%%%%%%%%%%%%

\latexProblemContent{
\ifVerboseLocation This is Derivative Compute Question 0051. \\ \fi
\begin{problem}

Find the limit.  Use L'H$\hat{o}$pital's rule where appropriate.

\input{Derivative-Compute-0051.HELP.tex}

\[\lim\limits_{x\to\infty} -2 \, {\left(x - 5\right)} e^{\left(-x + 2\right)}=\answer{0}\]
\end{problem}}

%%%%%%%%%%%%%%%%%%%%%%

\latexProblemContent{
\ifVerboseLocation This is Derivative Compute Question 0051. \\ \fi
\begin{problem}

Find the limit.  Use L'H$\hat{o}$pital's rule where appropriate.

\input{Derivative-Compute-0051.HELP.tex}

\[\lim\limits_{x\to\infty} 4 \, {\left(x - 7\right)}^{3} e^{\left(-x + 8\right)}=\answer{0}\]
\end{problem}}

%%%%%%%%%%%%%%%%%%%%%%

\latexProblemContent{
\ifVerboseLocation This is Derivative Compute Question 0051. \\ \fi
\begin{problem}

Find the limit.  Use L'H$\hat{o}$pital's rule where appropriate.

\input{Derivative-Compute-0051.HELP.tex}

\[\lim\limits_{x\to\infty} 4 \, {\left(x - 1\right)}^{2} e^{\left(-x - 4\right)}=\answer{0}\]
\end{problem}}

%%%%%%%%%%%%%%%%%%%%%%

\latexProblemContent{
\ifVerboseLocation This is Derivative Compute Question 0051. \\ \fi
\begin{problem}

Find the limit.  Use L'H$\hat{o}$pital's rule where appropriate.

\input{Derivative-Compute-0051.HELP.tex}

\[\lim\limits_{x\to\infty} -4 \, {\left(x + 3\right)} e^{\left(-x - 1\right)}=\answer{0}\]
\end{problem}}

%%%%%%%%%%%%%%%%%%%%%%

\latexProblemContent{
\ifVerboseLocation This is Derivative Compute Question 0051. \\ \fi
\begin{problem}

Find the limit.  Use L'H$\hat{o}$pital's rule where appropriate.

\input{Derivative-Compute-0051.HELP.tex}

\[\lim\limits_{x\to\infty} -3 \, {\left(x - 5\right)} e^{\left(-x - 6\right)}=\answer{0}\]
\end{problem}}

%%%%%%%%%%%%%%%%%%%%%%

\latexProblemContent{
\ifVerboseLocation This is Derivative Compute Question 0051. \\ \fi
\begin{problem}

Find the limit.  Use L'H$\hat{o}$pital's rule where appropriate.

\input{Derivative-Compute-0051.HELP.tex}

\[\lim\limits_{x\to\infty} -{\left(x + 2\right)} e^{\left(-x - 7\right)}=\answer{0}\]
\end{problem}}

%%%%%%%%%%%%%%%%%%%%%%

\latexProblemContent{
\ifVerboseLocation This is Derivative Compute Question 0051. \\ \fi
\begin{problem}

Find the limit.  Use L'H$\hat{o}$pital's rule where appropriate.

\input{Derivative-Compute-0051.HELP.tex}

\[\lim\limits_{x\to\infty} 8 \, {\left(x + 2\right)}^{2} e^{\left(-x + 5\right)}=\answer{0}\]
\end{problem}}

%%%%%%%%%%%%%%%%%%%%%%

\latexProblemContent{
\ifVerboseLocation This is Derivative Compute Question 0051. \\ \fi
\begin{problem}

Find the limit.  Use L'H$\hat{o}$pital's rule where appropriate.

\input{Derivative-Compute-0051.HELP.tex}

\[\lim\limits_{x\to\infty} 2 \, {\left(x + 7\right)}^{2} e^{\left(-x - 4\right)}=\answer{0}\]
\end{problem}}

%%%%%%%%%%%%%%%%%%%%%%

\latexProblemContent{
\ifVerboseLocation This is Derivative Compute Question 0051. \\ \fi
\begin{problem}

Find the limit.  Use L'H$\hat{o}$pital's rule where appropriate.

\input{Derivative-Compute-0051.HELP.tex}

\[\lim\limits_{x\to\infty} -8 \, {\left(x - 8\right)} e^{\left(-x + 3\right)}=\answer{0}\]
\end{problem}}

%%%%%%%%%%%%%%%%%%%%%%

\latexProblemContent{
\ifVerboseLocation This is Derivative Compute Question 0051. \\ \fi
\begin{problem}

Find the limit.  Use L'H$\hat{o}$pital's rule where appropriate.

\input{Derivative-Compute-0051.HELP.tex}

\[\lim\limits_{x\to\infty} -3 \, {\left(x - 6\right)}^{3} e^{\left(-x - 5\right)}=\answer{0}\]
\end{problem}}

%%%%%%%%%%%%%%%%%%%%%%

\latexProblemContent{
\ifVerboseLocation This is Derivative Compute Question 0051. \\ \fi
\begin{problem}

Find the limit.  Use L'H$\hat{o}$pital's rule where appropriate.

\input{Derivative-Compute-0051.HELP.tex}

\[\lim\limits_{x\to\infty} -8 \, {\left(x + 7\right)} e^{\left(-x + 4\right)}=\answer{0}\]
\end{problem}}

%%%%%%%%%%%%%%%%%%%%%%

\latexProblemContent{
\ifVerboseLocation This is Derivative Compute Question 0051. \\ \fi
\begin{problem}

Find the limit.  Use L'H$\hat{o}$pital's rule where appropriate.

\input{Derivative-Compute-0051.HELP.tex}

\[\lim\limits_{x\to\infty} -{\left(x + 7\right)}^{2} e^{\left(-x - 4\right)}=\answer{0}\]
\end{problem}}

%%%%%%%%%%%%%%%%%%%%%%

\latexProblemContent{
\ifVerboseLocation This is Derivative Compute Question 0051. \\ \fi
\begin{problem}

Find the limit.  Use L'H$\hat{o}$pital's rule where appropriate.

\input{Derivative-Compute-0051.HELP.tex}

\[\lim\limits_{x\to\infty} -5 \, {\left(x - 7\right)} e^{\left(-x + 3\right)}=\answer{0}\]
\end{problem}}

%%%%%%%%%%%%%%%%%%%%%%

\latexProblemContent{
\ifVerboseLocation This is Derivative Compute Question 0051. \\ \fi
\begin{problem}

Find the limit.  Use L'H$\hat{o}$pital's rule where appropriate.

\input{Derivative-Compute-0051.HELP.tex}

\[\lim\limits_{x\to\infty} 3 \, {\left(x + 5\right)} e^{\left(-x + 5\right)}=\answer{0}\]
\end{problem}}

%%%%%%%%%%%%%%%%%%%%%%

\latexProblemContent{
\ifVerboseLocation This is Derivative Compute Question 0051. \\ \fi
\begin{problem}

Find the limit.  Use L'H$\hat{o}$pital's rule where appropriate.

\input{Derivative-Compute-0051.HELP.tex}

\[\lim\limits_{x\to\infty} -2 \, {\left(x - 5\right)}^{2} e^{\left(-x - 5\right)}=\answer{0}\]
\end{problem}}

%%%%%%%%%%%%%%%%%%%%%%

\latexProblemContent{
\ifVerboseLocation This is Derivative Compute Question 0051. \\ \fi
\begin{problem}

Find the limit.  Use L'H$\hat{o}$pital's rule where appropriate.

\input{Derivative-Compute-0051.HELP.tex}

\[\lim\limits_{x\to\infty} -2 \, {\left(x + 7\right)} e^{\left(-x + 8\right)}=\answer{0}\]
\end{problem}}

%%%%%%%%%%%%%%%%%%%%%%

\latexProblemContent{
\ifVerboseLocation This is Derivative Compute Question 0051. \\ \fi
\begin{problem}

Find the limit.  Use L'H$\hat{o}$pital's rule where appropriate.

\input{Derivative-Compute-0051.HELP.tex}

\[\lim\limits_{x\to\infty} -{\left(x + 2\right)} e^{\left(-x + 6\right)}=\answer{0}\]
\end{problem}}

%%%%%%%%%%%%%%%%%%%%%%

\latexProblemContent{
\ifVerboseLocation This is Derivative Compute Question 0051. \\ \fi
\begin{problem}

Find the limit.  Use L'H$\hat{o}$pital's rule where appropriate.

\input{Derivative-Compute-0051.HELP.tex}

\[\lim\limits_{x\to\infty} -6 \, {\left(x - 2\right)}^{2} e^{\left(-x + 7\right)}=\answer{0}\]
\end{problem}}

%%%%%%%%%%%%%%%%%%%%%%

\latexProblemContent{
\ifVerboseLocation This is Derivative Compute Question 0051. \\ \fi
\begin{problem}

Find the limit.  Use L'H$\hat{o}$pital's rule where appropriate.

\input{Derivative-Compute-0051.HELP.tex}

\[\lim\limits_{x\to\infty} -5 \, {\left(x + 5\right)} e^{\left(-x + 5\right)}=\answer{0}\]
\end{problem}}

%%%%%%%%%%%%%%%%%%%%%%

\latexProblemContent{
\ifVerboseLocation This is Derivative Compute Question 0051. \\ \fi
\begin{problem}

Find the limit.  Use L'H$\hat{o}$pital's rule where appropriate.

\input{Derivative-Compute-0051.HELP.tex}

\[\lim\limits_{x\to\infty} -5 \, {\left(x + 6\right)} e^{\left(-x + 6\right)}=\answer{0}\]
\end{problem}}

%%%%%%%%%%%%%%%%%%%%%%

\latexProblemContent{
\ifVerboseLocation This is Derivative Compute Question 0051. \\ \fi
\begin{problem}

Find the limit.  Use L'H$\hat{o}$pital's rule where appropriate.

\input{Derivative-Compute-0051.HELP.tex}

\[\lim\limits_{x\to\infty} -3 \, {\left(x - 3\right)}^{2} e^{\left(-x - 8\right)}=\answer{0}\]
\end{problem}}

%%%%%%%%%%%%%%%%%%%%%%

\latexProblemContent{
\ifVerboseLocation This is Derivative Compute Question 0051. \\ \fi
\begin{problem}

Find the limit.  Use L'H$\hat{o}$pital's rule where appropriate.

\input{Derivative-Compute-0051.HELP.tex}

\[\lim\limits_{x\to\infty} 7 \, {\left(x - 3\right)}^{2} e^{\left(-x - 5\right)}=\answer{0}\]
\end{problem}}

%%%%%%%%%%%%%%%%%%%%%%

\latexProblemContent{
\ifVerboseLocation This is Derivative Compute Question 0051. \\ \fi
\begin{problem}

Find the limit.  Use L'H$\hat{o}$pital's rule where appropriate.

\input{Derivative-Compute-0051.HELP.tex}

\[\lim\limits_{x\to\infty} 7 \, {\left(x - 7\right)}^{3} e^{\left(-x - 4\right)}=\answer{0}\]
\end{problem}}

%%%%%%%%%%%%%%%%%%%%%%

\latexProblemContent{
\ifVerboseLocation This is Derivative Compute Question 0051. \\ \fi
\begin{problem}

Find the limit.  Use L'H$\hat{o}$pital's rule where appropriate.

\input{Derivative-Compute-0051.HELP.tex}

\[\lim\limits_{x\to\infty} 3 \, {\left(x - 1\right)}^{2} e^{\left(-x + 4\right)}=\answer{0}\]
\end{problem}}\fi             %% end of \ifproblemToFind near top of file
\fi             %% end of \ifquestionCount near top of file
\ProblemFileFooter