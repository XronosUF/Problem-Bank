% Ans        : ShortAns
% File       : 0027
% Sub        : Poly, Power-Rule
% Topic      : Derivative
% Type       : Compute

\ProblemFileHeader{119}
\ifquestionPull
\ifproblemToFind\latexProblemContent{
\ifVerboseLocation This is Derivative Compute Question 0027. \\ \fi
\begin{problem}

Compute the following derivative:

\input{Derivative-Compute-0027.HELP.tex}

\[\dfrac{d}{dx}\left(x^{2} - 9 \, x + 20\right)=\answer{2 \, x - 9}\]
\end{problem}}

%%%%%%%%%%%%%%%%%%%%%%

\latexProblemContent{
\ifVerboseLocation This is Derivative Compute Question 0027. \\ \fi
\begin{problem}

Compute the following derivative:

\input{Derivative-Compute-0027.HELP.tex}

\[\dfrac{d}{dx}\left(x^{3} + 2 \, x^{2} - 4 \, x - 8\right)=\answer{3 \, x^{2} + 4 \, x - 4}\]
\end{problem}}

%%%%%%%%%%%%%%%%%%%%%%

\latexProblemContent{
\ifVerboseLocation This is Derivative Compute Question 0027. \\ \fi
\begin{problem}

Compute the following derivative:

\input{Derivative-Compute-0027.HELP.tex}

\[\dfrac{d}{dx}\left(x^{3} - 11 \, x^{2} + 39 \, x - 45\right)=\answer{3 \, x^{2} - 22 \, x + 39}\]
\end{problem}}

%%%%%%%%%%%%%%%%%%%%%%

\latexProblemContent{
\ifVerboseLocation This is Derivative Compute Question 0027. \\ \fi
\begin{problem}

Compute the following derivative:

\input{Derivative-Compute-0027.HELP.tex}

\[\dfrac{d}{dx}\left(x^{2} + 4 \, x + 3\right)=\answer{2 \, x + 4}\]
\end{problem}}

%%%%%%%%%%%%%%%%%%%%%%

\latexProblemContent{
\ifVerboseLocation This is Derivative Compute Question 0027. \\ \fi
\begin{problem}

Compute the following derivative:

\input{Derivative-Compute-0027.HELP.tex}

\[\dfrac{d}{dx}\left(x^{3} + 6 \, x^{2} - 32\right)=\answer{3 \, x^{2} + 12 \, x}\]
\end{problem}}

%%%%%%%%%%%%%%%%%%%%%%

\latexProblemContent{
\ifVerboseLocation This is Derivative Compute Question 0027. \\ \fi
\begin{problem}

Compute the following derivative:

\input{Derivative-Compute-0027.HELP.tex}

\[\dfrac{d}{dx}\left(x^{3} - 3 \, x^{2} - 9 \, x - 5\right)=\answer{3 \, x^{2} - 6 \, x - 9}\]
\end{problem}}

%%%%%%%%%%%%%%%%%%%%%%

\latexProblemContent{
\ifVerboseLocation This is Derivative Compute Question 0027. \\ \fi
\begin{problem}

Compute the following derivative:

\input{Derivative-Compute-0027.HELP.tex}

\[\dfrac{d}{dx}\left(x^{3} - 14 \, x^{2} + 65 \, x - 100\right)=\answer{3 \, x^{2} - 28 \, x + 65}\]
\end{problem}}

%%%%%%%%%%%%%%%%%%%%%%

\latexProblemContent{
\ifVerboseLocation This is Derivative Compute Question 0027. \\ \fi
\begin{problem}

Compute the following derivative:

\input{Derivative-Compute-0027.HELP.tex}

\[\dfrac{d}{dx}\left(x^{2} - 8 \, x + 15\right)=\answer{2 \, x - 8}\]
\end{problem}}

%%%%%%%%%%%%%%%%%%%%%%

\latexProblemContent{
\ifVerboseLocation This is Derivative Compute Question 0027. \\ \fi
\begin{problem}

Compute the following derivative:

\input{Derivative-Compute-0027.HELP.tex}

\[\dfrac{d}{dx}\left(x^{2} + 5 \, x + 6\right)=\answer{2 \, x + 5}\]
\end{problem}}

%%%%%%%%%%%%%%%%%%%%%%

\latexProblemContent{
\ifVerboseLocation This is Derivative Compute Question 0027. \\ \fi
\begin{problem}

Compute the following derivative:

\input{Derivative-Compute-0027.HELP.tex}

\[\dfrac{d}{dx}\left(x^{3} + 5 \, x^{2} + 7 \, x + 3\right)=\answer{3 \, x^{2} + 10 \, x + 7}\]
\end{problem}}

%%%%%%%%%%%%%%%%%%%%%%

\latexProblemContent{
\ifVerboseLocation This is Derivative Compute Question 0027. \\ \fi
\begin{problem}

Compute the following derivative:

\input{Derivative-Compute-0027.HELP.tex}

\[\dfrac{d}{dx}\left(x^{3} + 8 \, x^{2} + 20 \, x + 16\right)=\answer{3 \, x^{2} + 16 \, x + 20}\]
\end{problem}}

%%%%%%%%%%%%%%%%%%%%%%

\latexProblemContent{
\ifVerboseLocation This is Derivative Compute Question 0027. \\ \fi
\begin{problem}

Compute the following derivative:

\input{Derivative-Compute-0027.HELP.tex}

\[\dfrac{d}{dx}\left(x^{3} - 7 \, x^{2} + 8 \, x + 16\right)=\answer{3 \, x^{2} - 14 \, x + 8}\]
\end{problem}}

%%%%%%%%%%%%%%%%%%%%%%

\latexProblemContent{
\ifVerboseLocation This is Derivative Compute Question 0027. \\ \fi
\begin{problem}

Compute the following derivative:

\input{Derivative-Compute-0027.HELP.tex}

\[\dfrac{d}{dx}\left(x^{3} + x^{2} - 21 \, x - 45\right)=\answer{3 \, x^{2} + 2 \, x - 21}\]
\end{problem}}

%%%%%%%%%%%%%%%%%%%%%%

\latexProblemContent{
\ifVerboseLocation This is Derivative Compute Question 0027. \\ \fi
\begin{problem}

Compute the following derivative:

\input{Derivative-Compute-0027.HELP.tex}

\[\dfrac{d}{dx}\left(x^{3} - 9 \, x^{2} + 15 \, x + 25\right)=\answer{3 \, x^{2} - 18 \, x + 15}\]
\end{problem}}

%%%%%%%%%%%%%%%%%%%%%%

\latexProblemContent{
\ifVerboseLocation This is Derivative Compute Question 0027. \\ \fi
\begin{problem}

Compute the following derivative:

\input{Derivative-Compute-0027.HELP.tex}

\[\dfrac{d}{dx}\left(x^{3} - 5 \, x^{2} + 8 \, x - 4\right)=\answer{3 \, x^{2} - 10 \, x + 8}\]
\end{problem}}

%%%%%%%%%%%%%%%%%%%%%%

\latexProblemContent{
\ifVerboseLocation This is Derivative Compute Question 0027. \\ \fi
\begin{problem}

Compute the following derivative:

\input{Derivative-Compute-0027.HELP.tex}

\[\dfrac{d}{dx}\left(x^{3} - 3 \, x^{2} - 24 \, x + 80\right)=\answer{3 \, x^{2} - 6 \, x - 24}\]
\end{problem}}

%%%%%%%%%%%%%%%%%%%%%%

\latexProblemContent{
\ifVerboseLocation This is Derivative Compute Question 0027. \\ \fi
\begin{problem}

Compute the following derivative:

\input{Derivative-Compute-0027.HELP.tex}

\[\dfrac{d}{dx}\left(x^{2} + 3 \, x - 10\right)=\answer{2 \, x + 3}\]
\end{problem}}

%%%%%%%%%%%%%%%%%%%%%%

\latexProblemContent{
\ifVerboseLocation This is Derivative Compute Question 0027. \\ \fi
\begin{problem}

Compute the following derivative:

\input{Derivative-Compute-0027.HELP.tex}

\[\dfrac{d}{dx}\left(x^{2} - 5 \, x + 4\right)=\answer{2 \, x - 5}\]
\end{problem}}

%%%%%%%%%%%%%%%%%%%%%%

\latexProblemContent{
\ifVerboseLocation This is Derivative Compute Question 0027. \\ \fi
\begin{problem}

Compute the following derivative:

\input{Derivative-Compute-0027.HELP.tex}

\[\dfrac{d}{dx}\left(x^{3} - 10 \, x^{2} + 33 \, x - 36\right)=\answer{3 \, x^{2} - 20 \, x + 33}\]
\end{problem}}

%%%%%%%%%%%%%%%%%%%%%%

\latexProblemContent{
\ifVerboseLocation This is Derivative Compute Question 0027. \\ \fi
\begin{problem}

Compute the following derivative:

\input{Derivative-Compute-0027.HELP.tex}

\[\dfrac{d}{dx}\left(x^{3} + 5 \, x^{2} - 8 \, x - 48\right)=\answer{3 \, x^{2} + 10 \, x - 8}\]
\end{problem}}

%%%%%%%%%%%%%%%%%%%%%%

\latexProblemContent{
\ifVerboseLocation This is Derivative Compute Question 0027. \\ \fi
\begin{problem}

Compute the following derivative:

\input{Derivative-Compute-0027.HELP.tex}

\[\dfrac{d}{dx}\left(x^{3} - 3 \, x^{2} + 4\right)=\answer{3 \, x^{2} - 6 \, x}\]
\end{problem}}

%%%%%%%%%%%%%%%%%%%%%%

\latexProblemContent{
\ifVerboseLocation This is Derivative Compute Question 0027. \\ \fi
\begin{problem}

Compute the following derivative:

\input{Derivative-Compute-0027.HELP.tex}

\[\dfrac{d}{dx}\left(x^{2} + 6 \, x + 5\right)=\answer{2 \, x + 6}\]
\end{problem}}

%%%%%%%%%%%%%%%%%%%%%%

\latexProblemContent{
\ifVerboseLocation This is Derivative Compute Question 0027. \\ \fi
\begin{problem}

Compute the following derivative:

\input{Derivative-Compute-0027.HELP.tex}

\[\dfrac{d}{dx}\left(x^{3} + 11 \, x^{2} + 35 \, x + 25\right)=\answer{3 \, x^{2} + 22 \, x + 35}\]
\end{problem}}

%%%%%%%%%%%%%%%%%%%%%%

\latexProblemContent{
\ifVerboseLocation This is Derivative Compute Question 0027. \\ \fi
\begin{problem}

Compute the following derivative:

\input{Derivative-Compute-0027.HELP.tex}

\[\dfrac{d}{dx}\left(x^{3} - 9 \, x^{2} + 24 \, x - 16\right)=\answer{3 \, x^{2} - 18 \, x + 24}\]
\end{problem}}

%%%%%%%%%%%%%%%%%%%%%%

\latexProblemContent{
\ifVerboseLocation This is Derivative Compute Question 0027. \\ \fi
\begin{problem}

Compute the following derivative:

\input{Derivative-Compute-0027.HELP.tex}

\[\dfrac{d}{dx}\left(x^{3} + 7 \, x^{2} + 15 \, x + 9\right)=\answer{3 \, x^{2} + 14 \, x + 15}\]
\end{problem}}

%%%%%%%%%%%%%%%%%%%%%%

\latexProblemContent{
\ifVerboseLocation This is Derivative Compute Question 0027. \\ \fi
\begin{problem}

Compute the following derivative:

\input{Derivative-Compute-0027.HELP.tex}

\[\dfrac{d}{dx}\left(x^{3} + 4 \, x^{2} - 16 \, x - 64\right)=\answer{3 \, x^{2} + 8 \, x - 16}\]
\end{problem}}

%%%%%%%%%%%%%%%%%%%%%%

\latexProblemContent{
\ifVerboseLocation This is Derivative Compute Question 0027. \\ \fi
\begin{problem}

Compute the following derivative:

\input{Derivative-Compute-0027.HELP.tex}

\[\dfrac{d}{dx}\left(x^{3} - 3 \, x - 2\right)=\answer{3 \, x^{2} - 3}\]
\end{problem}}

%%%%%%%%%%%%%%%%%%%%%%

\latexProblemContent{
\ifVerboseLocation This is Derivative Compute Question 0027. \\ \fi
\begin{problem}

Compute the following derivative:

\input{Derivative-Compute-0027.HELP.tex}

\[\dfrac{d}{dx}\left(x^{2} - 6 \, x + 8\right)=\answer{2 \, x - 6}\]
\end{problem}}

%%%%%%%%%%%%%%%%%%%%%%

\latexProblemContent{
\ifVerboseLocation This is Derivative Compute Question 0027. \\ \fi
\begin{problem}

Compute the following derivative:

\input{Derivative-Compute-0027.HELP.tex}

\[\dfrac{d}{dx}\left(x^{3} - 11 \, x^{2} + 35 \, x - 25\right)=\answer{3 \, x^{2} - 22 \, x + 35}\]
\end{problem}}

%%%%%%%%%%%%%%%%%%%%%%

\latexProblemContent{
\ifVerboseLocation This is Derivative Compute Question 0027. \\ \fi
\begin{problem}

Compute the following derivative:

\input{Derivative-Compute-0027.HELP.tex}

\[\dfrac{d}{dx}\left(x^{2} + 4 \, x - 5\right)=\answer{2 \, x + 4}\]
\end{problem}}

%%%%%%%%%%%%%%%%%%%%%%

\latexProblemContent{
\ifVerboseLocation This is Derivative Compute Question 0027. \\ \fi
\begin{problem}

Compute the following derivative:

\input{Derivative-Compute-0027.HELP.tex}

\[\dfrac{d}{dx}\left(x^{3} - 2 \, x^{2} - 7 \, x - 4\right)=\answer{3 \, x^{2} - 4 \, x - 7}\]
\end{problem}}

%%%%%%%%%%%%%%%%%%%%%%

\latexProblemContent{
\ifVerboseLocation This is Derivative Compute Question 0027. \\ \fi
\begin{problem}

Compute the following derivative:

\input{Derivative-Compute-0027.HELP.tex}

\[\dfrac{d}{dx}\left(x^{3} + 8 \, x^{2} + 5 \, x - 50\right)=\answer{3 \, x^{2} + 16 \, x + 5}\]
\end{problem}}

%%%%%%%%%%%%%%%%%%%%%%

\latexProblemContent{
\ifVerboseLocation This is Derivative Compute Question 0027. \\ \fi
\begin{problem}

Compute the following derivative:

\input{Derivative-Compute-0027.HELP.tex}

\[\dfrac{d}{dx}\left(x^{3} - x^{2} - x + 1\right)=\answer{3 \, x^{2} - 2 \, x - 1}\]
\end{problem}}

%%%%%%%%%%%%%%%%%%%%%%

\latexProblemContent{
\ifVerboseLocation This is Derivative Compute Question 0027. \\ \fi
\begin{problem}

Compute the following derivative:

\input{Derivative-Compute-0027.HELP.tex}

\[\dfrac{d}{dx}\left(x^{2} + 8 \, x + 15\right)=\answer{2 \, x + 8}\]
\end{problem}}

%%%%%%%%%%%%%%%%%%%%%%

\latexProblemContent{
\ifVerboseLocation This is Derivative Compute Question 0027. \\ \fi
\begin{problem}

Compute the following derivative:

\input{Derivative-Compute-0027.HELP.tex}

\[\dfrac{d}{dx}\left(x^{2} + 2 \, x - 8\right)=\answer{2 \, x + 2}\]
\end{problem}}

%%%%%%%%%%%%%%%%%%%%%%

\latexProblemContent{
\ifVerboseLocation This is Derivative Compute Question 0027. \\ \fi
\begin{problem}

Compute the following derivative:

\input{Derivative-Compute-0027.HELP.tex}

\[\dfrac{d}{dx}\left(x^{2} + 2 \, x - 15\right)=\answer{2 \, x + 2}\]
\end{problem}}

%%%%%%%%%%%%%%%%%%%%%%

\latexProblemContent{
\ifVerboseLocation This is Derivative Compute Question 0027. \\ \fi
\begin{problem}

Compute the following derivative:

\input{Derivative-Compute-0027.HELP.tex}

\[\dfrac{d}{dx}\left(x^{2}\right)=\answer{2 \, x}\]
\end{problem}}

%%%%%%%%%%%%%%%%%%%%%%

\latexProblemContent{
\ifVerboseLocation This is Derivative Compute Question 0027. \\ \fi
\begin{problem}

Compute the following derivative:

\input{Derivative-Compute-0027.HELP.tex}

\[\dfrac{d}{dx}\left(x^{3} + 3 \, x^{2} - 9 \, x - 27\right)=\answer{3 \, x^{2} + 6 \, x - 9}\]
\end{problem}}

%%%%%%%%%%%%%%%%%%%%%%

\latexProblemContent{
\ifVerboseLocation This is Derivative Compute Question 0027. \\ \fi
\begin{problem}

Compute the following derivative:

\input{Derivative-Compute-0027.HELP.tex}

\[\dfrac{d}{dx}\left(x^{3} + 3 \, x^{2} - 4\right)=\answer{3 \, x^{2} + 6 \, x}\]
\end{problem}}

%%%%%%%%%%%%%%%%%%%%%%

\latexProblemContent{
\ifVerboseLocation This is Derivative Compute Question 0027. \\ \fi
\begin{problem}

Compute the following derivative:

\input{Derivative-Compute-0027.HELP.tex}

\[\dfrac{d}{dx}\left(x^{2} - 3 \, x - 4\right)=\answer{2 \, x - 3}\]
\end{problem}}

%%%%%%%%%%%%%%%%%%%%%%

\latexProblemContent{
\ifVerboseLocation This is Derivative Compute Question 0027. \\ \fi
\begin{problem}

Compute the following derivative:

\input{Derivative-Compute-0027.HELP.tex}

\[\dfrac{d}{dx}\left(x^{2} - x - 20\right)=\answer{2 \, x - 1}\]
\end{problem}}

%%%%%%%%%%%%%%%%%%%%%%

\latexProblemContent{
\ifVerboseLocation This is Derivative Compute Question 0027. \\ \fi
\begin{problem}

Compute the following derivative:

\input{Derivative-Compute-0027.HELP.tex}

\[\dfrac{d}{dx}\left(x^{2} + 2 \, x - 3\right)=\answer{2 \, x + 2}\]
\end{problem}}

%%%%%%%%%%%%%%%%%%%%%%

\latexProblemContent{
\ifVerboseLocation This is Derivative Compute Question 0027. \\ \fi
\begin{problem}

Compute the following derivative:

\input{Derivative-Compute-0027.HELP.tex}

\[\dfrac{d}{dx}\left(x^{3} + 9 \, x^{2} + 24 \, x + 20\right)=\answer{3 \, x^{2} + 18 \, x + 24}\]
\end{problem}}

%%%%%%%%%%%%%%%%%%%%%%

\latexProblemContent{
\ifVerboseLocation This is Derivative Compute Question 0027. \\ \fi
\begin{problem}

Compute the following derivative:

\input{Derivative-Compute-0027.HELP.tex}

\[\dfrac{d}{dx}\left(x^{3} - 8 \, x^{2} + 5 \, x + 50\right)=\answer{3 \, x^{2} - 16 \, x + 5}\]
\end{problem}}

%%%%%%%%%%%%%%%%%%%%%%

\latexProblemContent{
\ifVerboseLocation This is Derivative Compute Question 0027. \\ \fi
\begin{problem}

Compute the following derivative:

\input{Derivative-Compute-0027.HELP.tex}

\[\dfrac{d}{dx}\left(x^{3} + 12 \, x^{2} + 45 \, x + 50\right)=\answer{3 \, x^{2} + 24 \, x + 45}\]
\end{problem}}

%%%%%%%%%%%%%%%%%%%%%%

\latexProblemContent{
\ifVerboseLocation This is Derivative Compute Question 0027. \\ \fi
\begin{problem}

Compute the following derivative:

\input{Derivative-Compute-0027.HELP.tex}

\[\dfrac{d}{dx}\left(x^{2} - 7 \, x + 12\right)=\answer{2 \, x - 7}\]
\end{problem}}

%%%%%%%%%%%%%%%%%%%%%%

\latexProblemContent{
\ifVerboseLocation This is Derivative Compute Question 0027. \\ \fi
\begin{problem}

Compute the following derivative:

\input{Derivative-Compute-0027.HELP.tex}

\[\dfrac{d}{dx}\left(x^{3} + 4 \, x^{2} - 3 \, x - 18\right)=\answer{3 \, x^{2} + 8 \, x - 3}\]
\end{problem}}

%%%%%%%%%%%%%%%%%%%%%%

\latexProblemContent{
\ifVerboseLocation This is Derivative Compute Question 0027. \\ \fi
\begin{problem}

Compute the following derivative:

\input{Derivative-Compute-0027.HELP.tex}

\[\dfrac{d}{dx}\left(x^{3} + 14 \, x^{2} + 65 \, x + 100\right)=\answer{3 \, x^{2} + 28 \, x + 65}\]
\end{problem}}

%%%%%%%%%%%%%%%%%%%%%%

\latexProblemContent{
\ifVerboseLocation This is Derivative Compute Question 0027. \\ \fi
\begin{problem}

Compute the following derivative:

\input{Derivative-Compute-0027.HELP.tex}

\[\dfrac{d}{dx}\left(x^{3} + 10 \, x^{2} + 33 \, x + 36\right)=\answer{3 \, x^{2} + 20 \, x + 33}\]
\end{problem}}

%%%%%%%%%%%%%%%%%%%%%%

\latexProblemContent{
\ifVerboseLocation This is Derivative Compute Question 0027. \\ \fi
\begin{problem}

Compute the following derivative:

\input{Derivative-Compute-0027.HELP.tex}

\[\dfrac{d}{dx}\left(x^{2} + x - 12\right)=\answer{2 \, x + 1}\]
\end{problem}}

%%%%%%%%%%%%%%%%%%%%%%

\latexProblemContent{
\ifVerboseLocation This is Derivative Compute Question 0027. \\ \fi
\begin{problem}

Compute the following derivative:

\input{Derivative-Compute-0027.HELP.tex}

\[\dfrac{d}{dx}\left(x^{3} - 6 \, x^{2} - 15 \, x + 100\right)=\answer{3 \, x^{2} - 12 \, x - 15}\]
\end{problem}}

%%%%%%%%%%%%%%%%%%%%%%

\latexProblemContent{
\ifVerboseLocation This is Derivative Compute Question 0027. \\ \fi
\begin{problem}

Compute the following derivative:

\input{Derivative-Compute-0027.HELP.tex}

\[\dfrac{d}{dx}\left(x^{3} + 9 \, x^{2} + 15 \, x - 25\right)=\answer{3 \, x^{2} + 18 \, x + 15}\]
\end{problem}}

%%%%%%%%%%%%%%%%%%%%%%

\latexProblemContent{
\ifVerboseLocation This is Derivative Compute Question 0027. \\ \fi
\begin{problem}

Compute the following derivative:

\input{Derivative-Compute-0027.HELP.tex}

\[\dfrac{d}{dx}\left(x^{3} + 10 \, x^{2} + 32 \, x + 32\right)=\answer{3 \, x^{2} + 20 \, x + 32}\]
\end{problem}}

%%%%%%%%%%%%%%%%%%%%%%

\latexProblemContent{
\ifVerboseLocation This is Derivative Compute Question 0027. \\ \fi
\begin{problem}

Compute the following derivative:

\input{Derivative-Compute-0027.HELP.tex}

\[\dfrac{d}{dx}\left(x^{2} - x - 12\right)=\answer{2 \, x - 1}\]
\end{problem}}

%%%%%%%%%%%%%%%%%%%%%%

\latexProblemContent{
\ifVerboseLocation This is Derivative Compute Question 0027. \\ \fi
\begin{problem}

Compute the following derivative:

\input{Derivative-Compute-0027.HELP.tex}

\[\dfrac{d}{dx}\left(x^{3} - 6 \, x^{2} + 9 \, x - 4\right)=\answer{3 \, x^{2} - 12 \, x + 9}\]
\end{problem}}

%%%%%%%%%%%%%%%%%%%%%%

\latexProblemContent{
\ifVerboseLocation This is Derivative Compute Question 0027. \\ \fi
\begin{problem}

Compute the following derivative:

\input{Derivative-Compute-0027.HELP.tex}

\[\dfrac{d}{dx}\left(x^{2} + 7 \, x + 12\right)=\answer{2 \, x + 7}\]
\end{problem}}

%%%%%%%%%%%%%%%%%%%%%%

\latexProblemContent{
\ifVerboseLocation This is Derivative Compute Question 0027. \\ \fi
\begin{problem}

Compute the following derivative:

\input{Derivative-Compute-0027.HELP.tex}

\[\dfrac{d}{dx}\left(x^{3} - 9 \, x^{2} + 24 \, x - 20\right)=\answer{3 \, x^{2} - 18 \, x + 24}\]
\end{problem}}

%%%%%%%%%%%%%%%%%%%%%%

\latexProblemContent{
\ifVerboseLocation This is Derivative Compute Question 0027. \\ \fi
\begin{problem}

Compute the following derivative:

\input{Derivative-Compute-0027.HELP.tex}

\[\dfrac{d}{dx}\left(x^{3} + 13 \, x^{2} + 55 \, x + 75\right)=\answer{3 \, x^{2} + 26 \, x + 55}\]
\end{problem}}

%%%%%%%%%%%%%%%%%%%%%%

\latexProblemContent{
\ifVerboseLocation This is Derivative Compute Question 0027. \\ \fi
\begin{problem}

Compute the following derivative:

\input{Derivative-Compute-0027.HELP.tex}

\[\dfrac{d}{dx}\left(x^{2} - 1\right)=\answer{2 \, x}\]
\end{problem}}

%%%%%%%%%%%%%%%%%%%%%%

\latexProblemContent{
\ifVerboseLocation This is Derivative Compute Question 0027. \\ \fi
\begin{problem}

Compute the following derivative:

\input{Derivative-Compute-0027.HELP.tex}

\[\dfrac{d}{dx}\left(x^{3} + 5 \, x^{2} + 3 \, x - 9\right)=\answer{3 \, x^{2} + 10 \, x + 3}\]
\end{problem}}

%%%%%%%%%%%%%%%%%%%%%%

\latexProblemContent{
\ifVerboseLocation This is Derivative Compute Question 0027. \\ \fi
\begin{problem}

Compute the following derivative:

\input{Derivative-Compute-0027.HELP.tex}

\[\dfrac{d}{dx}\left(x^{3} + 7 \, x^{2} + 8 \, x - 16\right)=\answer{3 \, x^{2} + 14 \, x + 8}\]
\end{problem}}

%%%%%%%%%%%%%%%%%%%%%%

\latexProblemContent{
\ifVerboseLocation This is Derivative Compute Question 0027. \\ \fi
\begin{problem}

Compute the following derivative:

\input{Derivative-Compute-0027.HELP.tex}

\[\dfrac{d}{dx}\left(x^{3} - 2 \, x^{2} - 4 \, x + 8\right)=\answer{3 \, x^{2} - 4 \, x - 4}\]
\end{problem}}

%%%%%%%%%%%%%%%%%%%%%%

\latexProblemContent{
\ifVerboseLocation This is Derivative Compute Question 0027. \\ \fi
\begin{problem}

Compute the following derivative:

\input{Derivative-Compute-0027.HELP.tex}

\[\dfrac{d}{dx}\left(x^{3} - 8 \, x^{2} + 21 \, x - 18\right)=\answer{3 \, x^{2} - 16 \, x + 21}\]
\end{problem}}

%%%%%%%%%%%%%%%%%%%%%%

\latexProblemContent{
\ifVerboseLocation This is Derivative Compute Question 0027. \\ \fi
\begin{problem}

Compute the following derivative:

\input{Derivative-Compute-0027.HELP.tex}

\[\dfrac{d}{dx}\left(x^{3} - 12 \, x^{2} + 45 \, x - 50\right)=\answer{3 \, x^{2} - 24 \, x + 45}\]
\end{problem}}

%%%%%%%%%%%%%%%%%%%%%%

\latexProblemContent{
\ifVerboseLocation This is Derivative Compute Question 0027. \\ \fi
\begin{problem}

Compute the following derivative:

\input{Derivative-Compute-0027.HELP.tex}

\[\dfrac{d}{dx}\left(x^{3} - 12 \, x + 16\right)=\answer{3 \, x^{2} - 12}\]
\end{problem}}

%%%%%%%%%%%%%%%%%%%%%%

\latexProblemContent{
\ifVerboseLocation This is Derivative Compute Question 0027. \\ \fi
\begin{problem}

Compute the following derivative:

\input{Derivative-Compute-0027.HELP.tex}

\[\dfrac{d}{dx}\left(x^{3} - 7 \, x^{2} + 16 \, x - 12\right)=\answer{3 \, x^{2} - 14 \, x + 16}\]
\end{problem}}

%%%%%%%%%%%%%%%%%%%%%%

\latexProblemContent{
\ifVerboseLocation This is Derivative Compute Question 0027. \\ \fi
\begin{problem}

Compute the following derivative:

\input{Derivative-Compute-0027.HELP.tex}

\[\dfrac{d}{dx}\left(x^{2} - 2 \, x - 8\right)=\answer{2 \, x - 2}\]
\end{problem}}

%%%%%%%%%%%%%%%%%%%%%%

\latexProblemContent{
\ifVerboseLocation This is Derivative Compute Question 0027. \\ \fi
\begin{problem}

Compute the following derivative:

\input{Derivative-Compute-0027.HELP.tex}

\[\dfrac{d}{dx}\left(x^{3} - 2 \, x^{2} - 15 \, x + 36\right)=\answer{3 \, x^{2} - 4 \, x - 15}\]
\end{problem}}

%%%%%%%%%%%%%%%%%%%%%%

\latexProblemContent{
\ifVerboseLocation This is Derivative Compute Question 0027. \\ \fi
\begin{problem}

Compute the following derivative:

\input{Derivative-Compute-0027.HELP.tex}

\[\dfrac{d}{dx}\left(x^{3} - 10 \, x^{2} + 32 \, x - 32\right)=\answer{3 \, x^{2} - 20 \, x + 32}\]
\end{problem}}

%%%%%%%%%%%%%%%%%%%%%%

\latexProblemContent{
\ifVerboseLocation This is Derivative Compute Question 0027. \\ \fi
\begin{problem}

Compute the following derivative:

\input{Derivative-Compute-0027.HELP.tex}

\[\dfrac{d}{dx}\left(x^{3} + 7 \, x^{2} - 5 \, x - 75\right)=\answer{3 \, x^{2} + 14 \, x - 5}\]
\end{problem}}

%%%%%%%%%%%%%%%%%%%%%%

\latexProblemContent{
\ifVerboseLocation This is Derivative Compute Question 0027. \\ \fi
\begin{problem}

Compute the following derivative:

\input{Derivative-Compute-0027.HELP.tex}

\[\dfrac{d}{dx}\left(x^{3} + 13 \, x^{2} + 56 \, x + 80\right)=\answer{3 \, x^{2} + 26 \, x + 56}\]
\end{problem}}

%%%%%%%%%%%%%%%%%%%%%%

\latexProblemContent{
\ifVerboseLocation This is Derivative Compute Question 0027. \\ \fi
\begin{problem}

Compute the following derivative:

\input{Derivative-Compute-0027.HELP.tex}

\[\dfrac{d}{dx}\left(x^{3} - 4 \, x^{2} - 16 \, x + 64\right)=\answer{3 \, x^{2} - 8 \, x - 16}\]
\end{problem}}

%%%%%%%%%%%%%%%%%%%%%%

\latexProblemContent{
\ifVerboseLocation This is Derivative Compute Question 0027. \\ \fi
\begin{problem}

Compute the following derivative:

\input{Derivative-Compute-0027.HELP.tex}

\[\dfrac{d}{dx}\left(x^{2} + x - 6\right)=\answer{2 \, x + 1}\]
\end{problem}}

%%%%%%%%%%%%%%%%%%%%%%

\latexProblemContent{
\ifVerboseLocation This is Derivative Compute Question 0027. \\ \fi
\begin{problem}

Compute the following derivative:

\input{Derivative-Compute-0027.HELP.tex}

\[\dfrac{d}{dx}\left(x^{2} - 7 \, x + 10\right)=\answer{2 \, x - 7}\]
\end{problem}}

%%%%%%%%%%%%%%%%%%%%%%

\latexProblemContent{
\ifVerboseLocation This is Derivative Compute Question 0027. \\ \fi
\begin{problem}

Compute the following derivative:

\input{Derivative-Compute-0027.HELP.tex}

\[\dfrac{d}{dx}\left(x^{2} - 9\right)=\answer{2 \, x}\]
\end{problem}}

%%%%%%%%%%%%%%%%%%%%%%

\latexProblemContent{
\ifVerboseLocation This is Derivative Compute Question 0027. \\ \fi
\begin{problem}

Compute the following derivative:

\input{Derivative-Compute-0027.HELP.tex}

\[\dfrac{d}{dx}\left(x^{3} + 2 \, x^{2} - 15 \, x - 36\right)=\answer{3 \, x^{2} + 4 \, x - 15}\]
\end{problem}}

%%%%%%%%%%%%%%%%%%%%%%

\latexProblemContent{
\ifVerboseLocation This is Derivative Compute Question 0027. \\ \fi
\begin{problem}

Compute the following derivative:

\input{Derivative-Compute-0027.HELP.tex}

\[\dfrac{d}{dx}\left(x^{3} - 5 \, x^{2} - 25 \, x + 125\right)=\answer{3 \, x^{2} - 10 \, x - 25}\]
\end{problem}}

%%%%%%%%%%%%%%%%%%%%%%

\latexProblemContent{
\ifVerboseLocation This is Derivative Compute Question 0027. \\ \fi
\begin{problem}

Compute the following derivative:

\input{Derivative-Compute-0027.HELP.tex}

\[\dfrac{d}{dx}\left(x^{3} + 11 \, x^{2} + 39 \, x + 45\right)=\answer{3 \, x^{2} + 22 \, x + 39}\]
\end{problem}}

%%%%%%%%%%%%%%%%%%%%%%

\latexProblemContent{
\ifVerboseLocation This is Derivative Compute Question 0027. \\ \fi
\begin{problem}

Compute the following derivative:

\input{Derivative-Compute-0027.HELP.tex}

\[\dfrac{d}{dx}\left(x^{2} - 4 \, x - 5\right)=\answer{2 \, x - 4}\]
\end{problem}}

%%%%%%%%%%%%%%%%%%%%%%

\latexProblemContent{
\ifVerboseLocation This is Derivative Compute Question 0027. \\ \fi
\begin{problem}

Compute the following derivative:

\input{Derivative-Compute-0027.HELP.tex}

\[\dfrac{d}{dx}\left(x^{3}\right)=\answer{3 \, x^{2}}\]
\end{problem}}

%%%%%%%%%%%%%%%%%%%%%%

\latexProblemContent{
\ifVerboseLocation This is Derivative Compute Question 0027. \\ \fi
\begin{problem}

Compute the following derivative:

\input{Derivative-Compute-0027.HELP.tex}

\[\dfrac{d}{dx}\left(x^{4}\right)=\answer{4 \, x^{3}}\]
\end{problem}}

%%%%%%%%%%%%%%%%%%%%%%

\latexProblemContent{
\ifVerboseLocation This is Derivative Compute Question 0027. \\ \fi
\begin{problem}

Compute the following derivative:

\input{Derivative-Compute-0027.HELP.tex}

\[\dfrac{d}{dx}\left(x^{2} - 5 \, x + 6\right)=\answer{2 \, x - 5}\]
\end{problem}}

%%%%%%%%%%%%%%%%%%%%%%

\latexProblemContent{
\ifVerboseLocation This is Derivative Compute Question 0027. \\ \fi
\begin{problem}

Compute the following derivative:

\input{Derivative-Compute-0027.HELP.tex}

\[\dfrac{d}{dx}\left(x^{3} + 11 \, x^{2} + 40 \, x + 48\right)=\answer{3 \, x^{2} + 22 \, x + 40}\]
\end{problem}}

%%%%%%%%%%%%%%%%%%%%%%

\latexProblemContent{
\ifVerboseLocation This is Derivative Compute Question 0027. \\ \fi
\begin{problem}

Compute the following derivative:

\input{Derivative-Compute-0027.HELP.tex}

\[\dfrac{d}{dx}\left(x^{3} - 13 \, x^{2} + 55 \, x - 75\right)=\answer{3 \, x^{2} - 26 \, x + 55}\]
\end{problem}}

%%%%%%%%%%%%%%%%%%%%%%

\latexProblemContent{
\ifVerboseLocation This is Derivative Compute Question 0027. \\ \fi
\begin{problem}

Compute the following derivative:

\input{Derivative-Compute-0027.HELP.tex}

\[\dfrac{d}{dx}\left(x^{2} - 4 \, x + 3\right)=\answer{2 \, x - 4}\]
\end{problem}}

%%%%%%%%%%%%%%%%%%%%%%

\latexProblemContent{
\ifVerboseLocation This is Derivative Compute Question 0027. \\ \fi
\begin{problem}

Compute the following derivative:

\input{Derivative-Compute-0027.HELP.tex}

\[\dfrac{d}{dx}\left(x^{3} - 13 \, x^{2} + 56 \, x - 80\right)=\answer{3 \, x^{2} - 26 \, x + 56}\]
\end{problem}}

%%%%%%%%%%%%%%%%%%%%%%

\latexProblemContent{
\ifVerboseLocation This is Derivative Compute Question 0027. \\ \fi
\begin{problem}

Compute the following derivative:

\input{Derivative-Compute-0027.HELP.tex}

\[\dfrac{d}{dx}\left(x^{3} - 5 \, x^{2} - 8 \, x + 48\right)=\answer{3 \, x^{2} - 10 \, x - 8}\]
\end{problem}}

%%%%%%%%%%%%%%%%%%%%%%

\latexProblemContent{
\ifVerboseLocation This is Derivative Compute Question 0027. \\ \fi
\begin{problem}

Compute the following derivative:

\input{Derivative-Compute-0027.HELP.tex}

\[\dfrac{d}{dx}\left(x^{3} - 5 \, x^{2} + 7 \, x - 3\right)=\answer{3 \, x^{2} - 10 \, x + 7}\]
\end{problem}}

%%%%%%%%%%%%%%%%%%%%%%

\latexProblemContent{
\ifVerboseLocation This is Derivative Compute Question 0027. \\ \fi
\begin{problem}

Compute the following derivative:

\input{Derivative-Compute-0027.HELP.tex}

\[\dfrac{d}{dx}\left(x^{3} - 6 \, x^{2} + 32\right)=\answer{3 \, x^{2} - 12 \, x}\]
\end{problem}}

%%%%%%%%%%%%%%%%%%%%%%

\latexProblemContent{
\ifVerboseLocation This is Derivative Compute Question 0027. \\ \fi
\begin{problem}

Compute the following derivative:

\input{Derivative-Compute-0027.HELP.tex}

\[\dfrac{d}{dx}\left(x^{3} + x^{2} - 16 \, x + 20\right)=\answer{3 \, x^{2} + 2 \, x - 16}\]
\end{problem}}

%%%%%%%%%%%%%%%%%%%%%%

\latexProblemContent{
\ifVerboseLocation This is Derivative Compute Question 0027. \\ \fi
\begin{problem}

Compute the following derivative:

\input{Derivative-Compute-0027.HELP.tex}

\[\dfrac{d}{dx}\left(x^{2} + x - 2\right)=\answer{2 \, x + 1}\]
\end{problem}}

%%%%%%%%%%%%%%%%%%%%%%

\latexProblemContent{
\ifVerboseLocation This is Derivative Compute Question 0027. \\ \fi
\begin{problem}

Compute the following derivative:

\input{Derivative-Compute-0027.HELP.tex}

\[\dfrac{d}{dx}\left(x^{2} - x - 2\right)=\answer{2 \, x - 1}\]
\end{problem}}

%%%%%%%%%%%%%%%%%%%%%%

\latexProblemContent{
\ifVerboseLocation This is Derivative Compute Question 0027. \\ \fi
\begin{problem}

Compute the following derivative:

\input{Derivative-Compute-0027.HELP.tex}

\[\dfrac{d}{dx}\left(x^{3} + 6 \, x^{2} - 15 \, x - 100\right)=\answer{3 \, x^{2} + 12 \, x - 15}\]
\end{problem}}

%%%%%%%%%%%%%%%%%%%%%%

\latexProblemContent{
\ifVerboseLocation This is Derivative Compute Question 0027. \\ \fi
\begin{problem}

Compute the following derivative:

\input{Derivative-Compute-0027.HELP.tex}

\[\dfrac{d}{dx}\left(x^{3} - 7 \, x^{2} + 15 \, x - 9\right)=\answer{3 \, x^{2} - 14 \, x + 15}\]
\end{problem}}

%%%%%%%%%%%%%%%%%%%%%%

\latexProblemContent{
\ifVerboseLocation This is Derivative Compute Question 0027. \\ \fi
\begin{problem}

Compute the following derivative:

\input{Derivative-Compute-0027.HELP.tex}

\[\dfrac{d}{dx}\left(x^{2} - 3 \, x + 2\right)=\answer{2 \, x - 3}\]
\end{problem}}

%%%%%%%%%%%%%%%%%%%%%%

\latexProblemContent{
\ifVerboseLocation This is Derivative Compute Question 0027. \\ \fi
\begin{problem}

Compute the following derivative:

\input{Derivative-Compute-0027.HELP.tex}

\[\dfrac{d}{dx}\left(x^{3} - x^{2} - 16 \, x - 20\right)=\answer{3 \, x^{2} - 2 \, x - 16}\]
\end{problem}}

%%%%%%%%%%%%%%%%%%%%%%

\latexProblemContent{
\ifVerboseLocation This is Derivative Compute Question 0027. \\ \fi
\begin{problem}

Compute the following derivative:

\input{Derivative-Compute-0027.HELP.tex}

\[\dfrac{d}{dx}\left(x^{3} + 6 \, x^{2} + 9 \, x + 4\right)=\answer{3 \, x^{2} + 12 \, x + 9}\]
\end{problem}}

%%%%%%%%%%%%%%%%%%%%%%

\latexProblemContent{
\ifVerboseLocation This is Derivative Compute Question 0027. \\ \fi
\begin{problem}

Compute the following derivative:

\input{Derivative-Compute-0027.HELP.tex}

\[\dfrac{d}{dx}\left(x^{2} + 6 \, x + 8\right)=\answer{2 \, x + 6}\]
\end{problem}}

%%%%%%%%%%%%%%%%%%%%%%

\latexProblemContent{
\ifVerboseLocation This is Derivative Compute Question 0027. \\ \fi
\begin{problem}

Compute the following derivative:

\input{Derivative-Compute-0027.HELP.tex}

\[\dfrac{d}{dx}\left(x^{2} - 16\right)=\answer{2 \, x}\]
\end{problem}}

%%%%%%%%%%%%%%%%%%%%%%

\latexProblemContent{
\ifVerboseLocation This is Derivative Compute Question 0027. \\ \fi
\begin{problem}

Compute the following derivative:

\input{Derivative-Compute-0027.HELP.tex}

\[\dfrac{d}{dx}\left(x^{2} + x - 20\right)=\answer{2 \, x + 1}\]
\end{problem}}

%%%%%%%%%%%%%%%%%%%%%%

\latexProblemContent{
\ifVerboseLocation This is Derivative Compute Question 0027. \\ \fi
\begin{problem}

Compute the following derivative:

\input{Derivative-Compute-0027.HELP.tex}

\[\dfrac{d}{dx}\left(x^{3} - x^{2} - 5 \, x - 3\right)=\answer{3 \, x^{2} - 2 \, x - 5}\]
\end{problem}}

%%%%%%%%%%%%%%%%%%%%%%

\latexProblemContent{
\ifVerboseLocation This is Derivative Compute Question 0027. \\ \fi
\begin{problem}

Compute the following derivative:

\input{Derivative-Compute-0027.HELP.tex}

\[\dfrac{d}{dx}\left(x^{2} - 2 \, x - 15\right)=\answer{2 \, x - 2}\]
\end{problem}}

%%%%%%%%%%%%%%%%%%%%%%

\latexProblemContent{
\ifVerboseLocation This is Derivative Compute Question 0027. \\ \fi
\begin{problem}

Compute the following derivative:

\input{Derivative-Compute-0027.HELP.tex}

\[\dfrac{d}{dx}\left(x^{3} + 3 \, x^{2} - 24 \, x - 80\right)=\answer{3 \, x^{2} + 6 \, x - 24}\]
\end{problem}}

%%%%%%%%%%%%%%%%%%%%%%

\latexProblemContent{
\ifVerboseLocation This is Derivative Compute Question 0027. \\ \fi
\begin{problem}

Compute the following derivative:

\input{Derivative-Compute-0027.HELP.tex}

\[\dfrac{d}{dx}\left(x^{3} - 3 \, x + 2\right)=\answer{3 \, x^{2} - 3}\]
\end{problem}}

%%%%%%%%%%%%%%%%%%%%%%

\latexProblemContent{
\ifVerboseLocation This is Derivative Compute Question 0027. \\ \fi
\begin{problem}

Compute the following derivative:

\input{Derivative-Compute-0027.HELP.tex}

\[\dfrac{d}{dx}\left(x^{3} + 7 \, x^{2} + 11 \, x + 5\right)=\answer{3 \, x^{2} + 14 \, x + 11}\]
\end{problem}}

%%%%%%%%%%%%%%%%%%%%%%

\latexProblemContent{
\ifVerboseLocation This is Derivative Compute Question 0027. \\ \fi
\begin{problem}

Compute the following derivative:

\input{Derivative-Compute-0027.HELP.tex}

\[\dfrac{d}{dx}\left(x^{2} + 3 \, x - 4\right)=\answer{2 \, x + 3}\]
\end{problem}}

%%%%%%%%%%%%%%%%%%%%%%

\latexProblemContent{
\ifVerboseLocation This is Derivative Compute Question 0027. \\ \fi
\begin{problem}

Compute the following derivative:

\input{Derivative-Compute-0027.HELP.tex}

\[\dfrac{d}{dx}\left(x^{3} + x^{2} - x - 1\right)=\answer{3 \, x^{2} + 2 \, x - 1}\]
\end{problem}}

%%%%%%%%%%%%%%%%%%%%%%

\latexProblemContent{
\ifVerboseLocation This is Derivative Compute Question 0027. \\ \fi
\begin{problem}

Compute the following derivative:

\input{Derivative-Compute-0027.HELP.tex}

\[\dfrac{d}{dx}\left(x^{3} + 5 \, x^{2} + 8 \, x + 4\right)=\answer{3 \, x^{2} + 10 \, x + 8}\]
\end{problem}}

%%%%%%%%%%%%%%%%%%%%%%

\latexProblemContent{
\ifVerboseLocation This is Derivative Compute Question 0027. \\ \fi
\begin{problem}

Compute the following derivative:

\input{Derivative-Compute-0027.HELP.tex}

\[\dfrac{d}{dx}\left(x^{2} + 7 \, x + 10\right)=\answer{2 \, x + 7}\]
\end{problem}}

%%%%%%%%%%%%%%%%%%%%%%

\latexProblemContent{
\ifVerboseLocation This is Derivative Compute Question 0027. \\ \fi
\begin{problem}

Compute the following derivative:

\input{Derivative-Compute-0027.HELP.tex}

\[\dfrac{d}{dx}\left(x^{3} + 8 \, x^{2} + 21 \, x + 18\right)=\answer{3 \, x^{2} + 16 \, x + 21}\]
\end{problem}}

%%%%%%%%%%%%%%%%%%%%%%

\latexProblemContent{
\ifVerboseLocation This is Derivative Compute Question 0027. \\ \fi
\begin{problem}

Compute the following derivative:

\input{Derivative-Compute-0027.HELP.tex}

\[\dfrac{d}{dx}\left(x^{2} - 6 \, x + 5\right)=\answer{2 \, x - 6}\]
\end{problem}}

%%%%%%%%%%%%%%%%%%%%%%

\latexProblemContent{
\ifVerboseLocation This is Derivative Compute Question 0027. \\ \fi
\begin{problem}

Compute the following derivative:

\input{Derivative-Compute-0027.HELP.tex}

\[\dfrac{d}{dx}\left(x^{3} - 4 \, x^{2} - 3 \, x + 18\right)=\answer{3 \, x^{2} - 8 \, x - 3}\]
\end{problem}}

%%%%%%%%%%%%%%%%%%%%%%

\latexProblemContent{
\ifVerboseLocation This is Derivative Compute Question 0027. \\ \fi
\begin{problem}

Compute the following derivative:

\input{Derivative-Compute-0027.HELP.tex}

\[\dfrac{d}{dx}\left(x^{3} + 3 \, x^{2} - 9 \, x + 5\right)=\answer{3 \, x^{2} + 6 \, x - 9}\]
\end{problem}}

%%%%%%%%%%%%%%%%%%%%%%

\latexProblemContent{
\ifVerboseLocation This is Derivative Compute Question 0027. \\ \fi
\begin{problem}

Compute the following derivative:

\input{Derivative-Compute-0027.HELP.tex}

\[\dfrac{d}{dx}\left(x^{3} - 5 \, x^{2} + 3 \, x + 9\right)=\answer{3 \, x^{2} - 10 \, x + 3}\]
\end{problem}}

%%%%%%%%%%%%%%%%%%%%%%

\latexProblemContent{
\ifVerboseLocation This is Derivative Compute Question 0027. \\ \fi
\begin{problem}

Compute the following derivative:

\input{Derivative-Compute-0027.HELP.tex}

\[\dfrac{d}{dx}\left(x^{3} - 11 \, x^{2} + 40 \, x - 48\right)=\answer{3 \, x^{2} - 22 \, x + 40}\]
\end{problem}}

%%%%%%%%%%%%%%%%%%%%%%

\latexProblemContent{
\ifVerboseLocation This is Derivative Compute Question 0027. \\ \fi
\begin{problem}

Compute the following derivative:

\input{Derivative-Compute-0027.HELP.tex}

\[\dfrac{d}{dx}\left(x^{3} + x^{2} - 5 \, x + 3\right)=\answer{3 \, x^{2} + 2 \, x - 5}\]
\end{problem}}

%%%%%%%%%%%%%%%%%%%%%%

\latexProblemContent{
\ifVerboseLocation This is Derivative Compute Question 0027. \\ \fi
\begin{problem}

Compute the following derivative:

\input{Derivative-Compute-0027.HELP.tex}

\[\dfrac{d}{dx}\left(x^{3} + 5 \, x^{2} - 25 \, x - 125\right)=\answer{3 \, x^{2} + 10 \, x - 25}\]
\end{problem}}

%%%%%%%%%%%%%%%%%%%%%%

\latexProblemContent{
\ifVerboseLocation This is Derivative Compute Question 0027. \\ \fi
\begin{problem}

Compute the following derivative:

\input{Derivative-Compute-0027.HELP.tex}

\[\dfrac{d}{dx}\left(x^{2} - x - 6\right)=\answer{2 \, x - 1}\]
\end{problem}}

%%%%%%%%%%%%%%%%%%%%%%

\latexProblemContent{
\ifVerboseLocation This is Derivative Compute Question 0027. \\ \fi
\begin{problem}

Compute the following derivative:

\input{Derivative-Compute-0027.HELP.tex}

\[\dfrac{d}{dx}\left(x^{3} - x^{2} - 8 \, x + 12\right)=\answer{3 \, x^{2} - 2 \, x - 8}\]
\end{problem}}\fi             %% end of \ifproblemToFind near top of file
\fi             %% end of \ifquestionCount near top of file
\ProblemFileFooter