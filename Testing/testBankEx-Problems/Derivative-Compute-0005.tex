% Ans        : MultiAns
% File       : 0005
% Sub        : FormalDef, DifferenceQuotient, Poly
% Topic      : Derivative
% Type       : Compute

\ProblemFileHeader{236}
\ifquestionPull
\ifproblemToFind\latexProblemContent{
\ifVerboseLocation This is Derivative Compute Question 0005. \\ \fi
\begin{problem}

Compute the following limit definition of derivative.
\[\lim_{h\to0}\frac{2 \, {\left(h + 2\right)}^{3} - 16}{h}=\answer{24}\]
What function is being differentiated? (Assume there is no horizontal translation)
\[f(x)=\answer{2 \, x^{3}}\]
At what $x-$value are you computing the derivative (given your previous answer)?
\[x=\answer{2}\]

\input{Derivative-Compute-0005.HELP.tex}


\end{problem}}

%%%%%%%%%%%%%%%%%%%%%%

\latexProblemContent{
\ifVerboseLocation This is Derivative Compute Question 0005. \\ \fi
\begin{problem}

Compute the following limit definition of derivative.
\[\lim_{h\to0}\frac{5 \, {\left(h + 1\right)}^{3} - 5}{h}=\answer{15}\]
What function is being differentiated? (Assume there is no horizontal translation)
\[f(x)=\answer{5 \, x^{3}}\]
At what $x-$value are you computing the derivative (given your previous answer)?
\[x=\answer{1}\]

\input{Derivative-Compute-0005.HELP.tex}


\end{problem}}

%%%%%%%%%%%%%%%%%%%%%%

\latexProblemContent{
\ifVerboseLocation This is Derivative Compute Question 0005. \\ \fi
\begin{problem}

Compute the following limit definition of derivative.
\[\lim_{h\to0}\frac{3 \, \sqrt{3} - 3 \, \sqrt{h + 3}}{h}=\answer{-\frac{1}{2} \, \sqrt{3}}\]
What function is being differentiated? (Assume there is no horizontal translation)
\[f(x)=\answer{-3 \, \sqrt{x}}\]
At what $x-$value are you computing the derivative (given your previous answer)?
\[x=\answer{3}\]

\input{Derivative-Compute-0005.HELP.tex}


\end{problem}}

%%%%%%%%%%%%%%%%%%%%%%

\latexProblemContent{
\ifVerboseLocation This is Derivative Compute Question 0005. \\ \fi
\begin{problem}

Compute the following limit definition of derivative.
\[\lim_{h\to0}\frac{\frac{4}{h - 2} + 2}{h}=\answer{-1}\]
What function is being differentiated? (Assume there is no horizontal translation)
\[f(x)=\answer{\frac{4}{x}}\]
At what $x-$value are you computing the derivative (given your previous answer)?
\[x=\answer{-2}\]

\input{Derivative-Compute-0005.HELP.tex}


\end{problem}}

%%%%%%%%%%%%%%%%%%%%%%

\latexProblemContent{
\ifVerboseLocation This is Derivative Compute Question 0005. \\ \fi
\begin{problem}

Compute the following limit definition of derivative.
\[\lim_{h\to0}\frac{3 \, {\left(h - 4\right)}^{2} - 48}{h}=\answer{-24}\]
What function is being differentiated? (Assume there is no horizontal translation)
\[f(x)=\answer{3 \, x^{2}}\]
At what $x-$value are you computing the derivative (given your previous answer)?
\[x=\answer{-4}\]

\input{Derivative-Compute-0005.HELP.tex}


\end{problem}}

%%%%%%%%%%%%%%%%%%%%%%

\latexProblemContent{
\ifVerboseLocation This is Derivative Compute Question 0005. \\ \fi
\begin{problem}

Compute the following limit definition of derivative.
\[\lim_{h\to0}\frac{\frac{4}{h + 2} - 2}{h}=\answer{-1}\]
What function is being differentiated? (Assume there is no horizontal translation)
\[f(x)=\answer{\frac{4}{x}}\]
At what $x-$value are you computing the derivative (given your previous answer)?
\[x=\answer{2}\]

\input{Derivative-Compute-0005.HELP.tex}


\end{problem}}

%%%%%%%%%%%%%%%%%%%%%%

\latexProblemContent{
\ifVerboseLocation This is Derivative Compute Question 0005. \\ \fi
\begin{problem}

Compute the following limit definition of derivative.
\[\lim_{h\to0}\frac{2 \, \sqrt{h + 1} - 2}{h}=\answer{1}\]
What function is being differentiated? (Assume there is no horizontal translation)
\[f(x)=\answer{2 \, \sqrt{x}}\]
At what $x-$value are you computing the derivative (given your previous answer)?
\[x=\answer{1}\]

\input{Derivative-Compute-0005.HELP.tex}


\end{problem}}

%%%%%%%%%%%%%%%%%%%%%%

\latexProblemContent{
\ifVerboseLocation This is Derivative Compute Question 0005. \\ \fi
\begin{problem}

Compute the following limit definition of derivative.
\[\lim_{h\to0}\frac{-4 \, \sqrt{h + 1} + 4}{h}=\answer{-2}\]
What function is being differentiated? (Assume there is no horizontal translation)
\[f(x)=\answer{-4 \, \sqrt{x}}\]
At what $x-$value are you computing the derivative (given your previous answer)?
\[x=\answer{1}\]

\input{Derivative-Compute-0005.HELP.tex}


\end{problem}}

%%%%%%%%%%%%%%%%%%%%%%

\latexProblemContent{
\ifVerboseLocation This is Derivative Compute Question 0005. \\ \fi
\begin{problem}

Compute the following limit definition of derivative.
\[\lim_{h\to0}\frac{-\frac{2}{h + 2} + 1}{h}=\answer{\frac{1}{2}}\]
What function is being differentiated? (Assume there is no horizontal translation)
\[f(x)=\answer{-\frac{2}{x}}\]
At what $x-$value are you computing the derivative (given your previous answer)?
\[x=\answer{2}\]

\input{Derivative-Compute-0005.HELP.tex}


\end{problem}}

%%%%%%%%%%%%%%%%%%%%%%

\latexProblemContent{
\ifVerboseLocation This is Derivative Compute Question 0005. \\ \fi
\begin{problem}

Compute the following limit definition of derivative.
\[\lim_{h\to0}\frac{-2 \, \sqrt{2} + 2 \, \sqrt{h + 2}}{h}=\answer{\frac{1}{2} \, \sqrt{2}}\]
What function is being differentiated? (Assume there is no horizontal translation)
\[f(x)=\answer{2 \, \sqrt{x}}\]
At what $x-$value are you computing the derivative (given your previous answer)?
\[x=\answer{2}\]

\input{Derivative-Compute-0005.HELP.tex}


\end{problem}}

%%%%%%%%%%%%%%%%%%%%%%

\latexProblemContent{
\ifVerboseLocation This is Derivative Compute Question 0005. \\ \fi
\begin{problem}

Compute the following limit definition of derivative.
\[\lim_{h\to0}\frac{-5 \, {\left(h - 4\right)}^{3} - 320}{h}=\answer{-240}\]
What function is being differentiated? (Assume there is no horizontal translation)
\[f(x)=\answer{-5 \, x^{3}}\]
At what $x-$value are you computing the derivative (given your previous answer)?
\[x=\answer{-4}\]

\input{Derivative-Compute-0005.HELP.tex}


\end{problem}}

%%%%%%%%%%%%%%%%%%%%%%

\latexProblemContent{
\ifVerboseLocation This is Derivative Compute Question 0005. \\ \fi
\begin{problem}

Compute the following limit definition of derivative.
\[\lim_{h\to0}\frac{-{\left(h - 2\right)}^{2} + 4}{h}=\answer{4}\]
What function is being differentiated? (Assume there is no horizontal translation)
\[f(x)=\answer{-x^{2}}\]
At what $x-$value are you computing the derivative (given your previous answer)?
\[x=\answer{-2}\]

\input{Derivative-Compute-0005.HELP.tex}


\end{problem}}

%%%%%%%%%%%%%%%%%%%%%%

\latexProblemContent{
\ifVerboseLocation This is Derivative Compute Question 0005. \\ \fi
\begin{problem}

Compute the following limit definition of derivative.
\[\lim_{h\to0}\frac{3 \, {\left(h - 2\right)}^{3} + 24}{h}=\answer{36}\]
What function is being differentiated? (Assume there is no horizontal translation)
\[f(x)=\answer{3 \, x^{3}}\]
At what $x-$value are you computing the derivative (given your previous answer)?
\[x=\answer{-2}\]

\input{Derivative-Compute-0005.HELP.tex}


\end{problem}}

%%%%%%%%%%%%%%%%%%%%%%

\latexProblemContent{
\ifVerboseLocation This is Derivative Compute Question 0005. \\ \fi
\begin{problem}

Compute the following limit definition of derivative.
\[\lim_{h\to0}\frac{-3 \, \sqrt{2} + 3 \, \sqrt{h + 2}}{h}=\answer{\frac{3}{4} \, \sqrt{2}}\]
What function is being differentiated? (Assume there is no horizontal translation)
\[f(x)=\answer{3 \, \sqrt{x}}\]
At what $x-$value are you computing the derivative (given your previous answer)?
\[x=\answer{2}\]

\input{Derivative-Compute-0005.HELP.tex}


\end{problem}}

%%%%%%%%%%%%%%%%%%%%%%

\latexProblemContent{
\ifVerboseLocation This is Derivative Compute Question 0005. \\ \fi
\begin{problem}

Compute the following limit definition of derivative.
\[\lim_{h\to0}\frac{3 \, {\left(h + 3\right)}^{2} - 27}{h}=\answer{18}\]
What function is being differentiated? (Assume there is no horizontal translation)
\[f(x)=\answer{3 \, x^{2}}\]
At what $x-$value are you computing the derivative (given your previous answer)?
\[x=\answer{3}\]

\input{Derivative-Compute-0005.HELP.tex}


\end{problem}}

%%%%%%%%%%%%%%%%%%%%%%

\latexProblemContent{
\ifVerboseLocation This is Derivative Compute Question 0005. \\ \fi
\begin{problem}

Compute the following limit definition of derivative.
\[\lim_{h\to0}\frac{\frac{3}{h + 4} - \frac{3}{4}}{h}=\answer{-\frac{3}{16}}\]
What function is being differentiated? (Assume there is no horizontal translation)
\[f(x)=\answer{\frac{3}{x}}\]
At what $x-$value are you computing the derivative (given your previous answer)?
\[x=\answer{4}\]

\input{Derivative-Compute-0005.HELP.tex}


\end{problem}}

%%%%%%%%%%%%%%%%%%%%%%

\latexProblemContent{
\ifVerboseLocation This is Derivative Compute Question 0005. \\ \fi
\begin{problem}

Compute the following limit definition of derivative.
\[\lim_{h\to0}\frac{-2 \, \sqrt{h + 1} + 2}{h}=\answer{-1}\]
What function is being differentiated? (Assume there is no horizontal translation)
\[f(x)=\answer{-2 \, \sqrt{x}}\]
At what $x-$value are you computing the derivative (given your previous answer)?
\[x=\answer{1}\]

\input{Derivative-Compute-0005.HELP.tex}


\end{problem}}

%%%%%%%%%%%%%%%%%%%%%%

\latexProblemContent{
\ifVerboseLocation This is Derivative Compute Question 0005. \\ \fi
\begin{problem}

Compute the following limit definition of derivative.
\[\lim_{h\to0}\frac{-{\left(h + 4\right)}^{2} + 16}{h}=\answer{-8}\]
What function is being differentiated? (Assume there is no horizontal translation)
\[f(x)=\answer{-x^{2}}\]
At what $x-$value are you computing the derivative (given your previous answer)?
\[x=\answer{4}\]

\input{Derivative-Compute-0005.HELP.tex}


\end{problem}}

%%%%%%%%%%%%%%%%%%%%%%

\latexProblemContent{
\ifVerboseLocation This is Derivative Compute Question 0005. \\ \fi
\begin{problem}

Compute the following limit definition of derivative.
\[\lim_{h\to0}\frac{-{\left(h + 4\right)}^{3} + 64}{h}=\answer{-48}\]
What function is being differentiated? (Assume there is no horizontal translation)
\[f(x)=\answer{-x^{3}}\]
At what $x-$value are you computing the derivative (given your previous answer)?
\[x=\answer{4}\]

\input{Derivative-Compute-0005.HELP.tex}


\end{problem}}

%%%%%%%%%%%%%%%%%%%%%%

\latexProblemContent{
\ifVerboseLocation This is Derivative Compute Question 0005. \\ \fi
\begin{problem}

Compute the following limit definition of derivative.
\[\lim_{h\to0}\frac{3 \, \sqrt{2} - 3 \, \sqrt{h + 2}}{h}=\answer{-\frac{3}{4} \, \sqrt{2}}\]
What function is being differentiated? (Assume there is no horizontal translation)
\[f(x)=\answer{-3 \, \sqrt{x}}\]
At what $x-$value are you computing the derivative (given your previous answer)?
\[x=\answer{2}\]

\input{Derivative-Compute-0005.HELP.tex}


\end{problem}}

%%%%%%%%%%%%%%%%%%%%%%

\latexProblemContent{
\ifVerboseLocation This is Derivative Compute Question 0005. \\ \fi
\begin{problem}

Compute the following limit definition of derivative.
\[\lim_{h\to0}\frac{-5 \, {\left(h - 2\right)}^{2} + 20}{h}=\answer{20}\]
What function is being differentiated? (Assume there is no horizontal translation)
\[f(x)=\answer{-5 \, x^{2}}\]
At what $x-$value are you computing the derivative (given your previous answer)?
\[x=\answer{-2}\]

\input{Derivative-Compute-0005.HELP.tex}


\end{problem}}

%%%%%%%%%%%%%%%%%%%%%%

\latexProblemContent{
\ifVerboseLocation This is Derivative Compute Question 0005. \\ \fi
\begin{problem}

Compute the following limit definition of derivative.
\[\lim_{h\to0}\frac{-4 \, {\left(h - 2\right)}^{2} + 16}{h}=\answer{16}\]
What function is being differentiated? (Assume there is no horizontal translation)
\[f(x)=\answer{-4 \, x^{2}}\]
At what $x-$value are you computing the derivative (given your previous answer)?
\[x=\answer{-2}\]

\input{Derivative-Compute-0005.HELP.tex}


\end{problem}}

%%%%%%%%%%%%%%%%%%%%%%

\latexProblemContent{
\ifVerboseLocation This is Derivative Compute Question 0005. \\ \fi
\begin{problem}

Compute the following limit definition of derivative.
\[\lim_{h\to0}\frac{-{\left(h - 3\right)}^{3} - 27}{h}=\answer{-27}\]
What function is being differentiated? (Assume there is no horizontal translation)
\[f(x)=\answer{-x^{3}}\]
At what $x-$value are you computing the derivative (given your previous answer)?
\[x=\answer{-3}\]

\input{Derivative-Compute-0005.HELP.tex}


\end{problem}}

%%%%%%%%%%%%%%%%%%%%%%

\latexProblemContent{
\ifVerboseLocation This is Derivative Compute Question 0005. \\ \fi
\begin{problem}

Compute the following limit definition of derivative.
\[\lim_{h\to0}\frac{-{\left(h - 3\right)}^{2} + 9}{h}=\answer{6}\]
What function is being differentiated? (Assume there is no horizontal translation)
\[f(x)=\answer{-x^{2}}\]
At what $x-$value are you computing the derivative (given your previous answer)?
\[x=\answer{-3}\]

\input{Derivative-Compute-0005.HELP.tex}


\end{problem}}

%%%%%%%%%%%%%%%%%%%%%%

\latexProblemContent{
\ifVerboseLocation This is Derivative Compute Question 0005. \\ \fi
\begin{problem}

Compute the following limit definition of derivative.
\[\lim_{h\to0}\frac{-4 \, \sqrt{3} + 4 \, \sqrt{h + 3}}{h}=\answer{\frac{2}{3} \, \sqrt{3}}\]
What function is being differentiated? (Assume there is no horizontal translation)
\[f(x)=\answer{4 \, \sqrt{x}}\]
At what $x-$value are you computing the derivative (given your previous answer)?
\[x=\answer{3}\]

\input{Derivative-Compute-0005.HELP.tex}


\end{problem}}

%%%%%%%%%%%%%%%%%%%%%%

\latexProblemContent{
\ifVerboseLocation This is Derivative Compute Question 0005. \\ \fi
\begin{problem}

Compute the following limit definition of derivative.
\[\lim_{h\to0}\frac{2 \, \sqrt{h + 4} - 4}{h}=\answer{\frac{1}{2}}\]
What function is being differentiated? (Assume there is no horizontal translation)
\[f(x)=\answer{2 \, \sqrt{x}}\]
At what $x-$value are you computing the derivative (given your previous answer)?
\[x=\answer{4}\]

\input{Derivative-Compute-0005.HELP.tex}


\end{problem}}

%%%%%%%%%%%%%%%%%%%%%%

\latexProblemContent{
\ifVerboseLocation This is Derivative Compute Question 0005. \\ \fi
\begin{problem}

Compute the following limit definition of derivative.
\[\lim_{h\to0}\frac{-2 \, {\left(h - 1\right)}^{3} - 2}{h}=\answer{-6}\]
What function is being differentiated? (Assume there is no horizontal translation)
\[f(x)=\answer{-2 \, x^{3}}\]
At what $x-$value are you computing the derivative (given your previous answer)?
\[x=\answer{-1}\]

\input{Derivative-Compute-0005.HELP.tex}


\end{problem}}

%%%%%%%%%%%%%%%%%%%%%%

\latexProblemContent{
\ifVerboseLocation This is Derivative Compute Question 0005. \\ \fi
\begin{problem}

Compute the following limit definition of derivative.
\[\lim_{h\to0}\frac{5 \, {\left(h - 2\right)}^{2} - 20}{h}=\answer{-20}\]
What function is being differentiated? (Assume there is no horizontal translation)
\[f(x)=\answer{5 \, x^{2}}\]
At what $x-$value are you computing the derivative (given your previous answer)?
\[x=\answer{-2}\]

\input{Derivative-Compute-0005.HELP.tex}


\end{problem}}

%%%%%%%%%%%%%%%%%%%%%%

\latexProblemContent{
\ifVerboseLocation This is Derivative Compute Question 0005. \\ \fi
\begin{problem}

Compute the following limit definition of derivative.
\[\lim_{h\to0}\frac{-5 \, {\left(h - 4\right)}^{2} + 80}{h}=\answer{40}\]
What function is being differentiated? (Assume there is no horizontal translation)
\[f(x)=\answer{-5 \, x^{2}}\]
At what $x-$value are you computing the derivative (given your previous answer)?
\[x=\answer{-4}\]

\input{Derivative-Compute-0005.HELP.tex}


\end{problem}}

%%%%%%%%%%%%%%%%%%%%%%

\latexProblemContent{
\ifVerboseLocation This is Derivative Compute Question 0005. \\ \fi
\begin{problem}

Compute the following limit definition of derivative.
\[\lim_{h\to0}\frac{3 \, {\left(h - 4\right)}^{3} + 192}{h}=\answer{144}\]
What function is being differentiated? (Assume there is no horizontal translation)
\[f(x)=\answer{3 \, x^{3}}\]
At what $x-$value are you computing the derivative (given your previous answer)?
\[x=\answer{-4}\]

\input{Derivative-Compute-0005.HELP.tex}


\end{problem}}

%%%%%%%%%%%%%%%%%%%%%%

\latexProblemContent{
\ifVerboseLocation This is Derivative Compute Question 0005. \\ \fi
\begin{problem}

Compute the following limit definition of derivative.
\[\lim_{h\to0}\frac{-2 \, \sqrt{h + 4} + 4}{h}=\answer{-\frac{1}{2}}\]
What function is being differentiated? (Assume there is no horizontal translation)
\[f(x)=\answer{-2 \, \sqrt{x}}\]
At what $x-$value are you computing the derivative (given your previous answer)?
\[x=\answer{4}\]

\input{Derivative-Compute-0005.HELP.tex}


\end{problem}}

%%%%%%%%%%%%%%%%%%%%%%

\latexProblemContent{
\ifVerboseLocation This is Derivative Compute Question 0005. \\ \fi
\begin{problem}

Compute the following limit definition of derivative.
\[\lim_{h\to0}\frac{\sqrt{h + 1} - 1}{h}=\answer{\frac{1}{2}}\]
What function is being differentiated? (Assume there is no horizontal translation)
\[f(x)=\answer{\sqrt{x}}\]
At what $x-$value are you computing the derivative (given your previous answer)?
\[x=\answer{1}\]

\input{Derivative-Compute-0005.HELP.tex}


\end{problem}}

%%%%%%%%%%%%%%%%%%%%%%

\latexProblemContent{
\ifVerboseLocation This is Derivative Compute Question 0005. \\ \fi
\begin{problem}

Compute the following limit definition of derivative.
\[\lim_{h\to0}\frac{4 \, \sqrt{h + 1} - 4}{h}=\answer{2}\]
What function is being differentiated? (Assume there is no horizontal translation)
\[f(x)=\answer{4 \, \sqrt{x}}\]
At what $x-$value are you computing the derivative (given your previous answer)?
\[x=\answer{1}\]

\input{Derivative-Compute-0005.HELP.tex}


\end{problem}}

%%%%%%%%%%%%%%%%%%%%%%

\latexProblemContent{
\ifVerboseLocation This is Derivative Compute Question 0005. \\ \fi
\begin{problem}

Compute the following limit definition of derivative.
\[\lim_{h\to0}\frac{\frac{5}{h + 4} - \frac{5}{4}}{h}=\answer{-\frac{5}{16}}\]
What function is being differentiated? (Assume there is no horizontal translation)
\[f(x)=\answer{\frac{5}{x}}\]
At what $x-$value are you computing the derivative (given your previous answer)?
\[x=\answer{4}\]

\input{Derivative-Compute-0005.HELP.tex}


\end{problem}}

%%%%%%%%%%%%%%%%%%%%%%

\latexProblemContent{
\ifVerboseLocation This is Derivative Compute Question 0005. \\ \fi
\begin{problem}

Compute the following limit definition of derivative.
\[\lim_{h\to0}\frac{-{\left(h - 1\right)}^{2} + 1}{h}=\answer{2}\]
What function is being differentiated? (Assume there is no horizontal translation)
\[f(x)=\answer{-x^{2}}\]
At what $x-$value are you computing the derivative (given your previous answer)?
\[x=\answer{-1}\]

\input{Derivative-Compute-0005.HELP.tex}


\end{problem}}

%%%%%%%%%%%%%%%%%%%%%%

\latexProblemContent{
\ifVerboseLocation This is Derivative Compute Question 0005. \\ \fi
\begin{problem}

Compute the following limit definition of derivative.
\[\lim_{h\to0}\frac{2 \, \sqrt{2} - 2 \, \sqrt{h + 2}}{h}=\answer{-\frac{1}{2} \, \sqrt{2}}\]
What function is being differentiated? (Assume there is no horizontal translation)
\[f(x)=\answer{-2 \, \sqrt{x}}\]
At what $x-$value are you computing the derivative (given your previous answer)?
\[x=\answer{2}\]

\input{Derivative-Compute-0005.HELP.tex}


\end{problem}}

%%%%%%%%%%%%%%%%%%%%%%

\latexProblemContent{
\ifVerboseLocation This is Derivative Compute Question 0005. \\ \fi
\begin{problem}

Compute the following limit definition of derivative.
\[\lim_{h\to0}\frac{-3 \, \sqrt{3} + 3 \, \sqrt{h + 3}}{h}=\answer{\frac{1}{2} \, \sqrt{3}}\]
What function is being differentiated? (Assume there is no horizontal translation)
\[f(x)=\answer{3 \, \sqrt{x}}\]
At what $x-$value are you computing the derivative (given your previous answer)?
\[x=\answer{3}\]

\input{Derivative-Compute-0005.HELP.tex}


\end{problem}}

%%%%%%%%%%%%%%%%%%%%%%

\latexProblemContent{
\ifVerboseLocation This is Derivative Compute Question 0005. \\ \fi
\begin{problem}

Compute the following limit definition of derivative.
\[\lim_{h\to0}\frac{-\frac{4}{h + 1} + 4}{h}=\answer{4}\]
What function is being differentiated? (Assume there is no horizontal translation)
\[f(x)=\answer{-\frac{4}{x}}\]
At what $x-$value are you computing the derivative (given your previous answer)?
\[x=\answer{1}\]

\input{Derivative-Compute-0005.HELP.tex}


\end{problem}}

%%%%%%%%%%%%%%%%%%%%%%

\latexProblemContent{
\ifVerboseLocation This is Derivative Compute Question 0005. \\ \fi
\begin{problem}

Compute the following limit definition of derivative.
\[\lim_{h\to0}\frac{4 \, \sqrt{h + 4} - 8}{h}=\answer{1}\]
What function is being differentiated? (Assume there is no horizontal translation)
\[f(x)=\answer{4 \, \sqrt{x}}\]
At what $x-$value are you computing the derivative (given your previous answer)?
\[x=\answer{4}\]

\input{Derivative-Compute-0005.HELP.tex}


\end{problem}}

%%%%%%%%%%%%%%%%%%%%%%

\latexProblemContent{
\ifVerboseLocation This is Derivative Compute Question 0005. \\ \fi
\begin{problem}

Compute the following limit definition of derivative.
\[\lim_{h\to0}\frac{-\sqrt{2} + \sqrt{h + 2}}{h}=\answer{\frac{1}{4} \, \sqrt{2}}\]
What function is being differentiated? (Assume there is no horizontal translation)
\[f(x)=\answer{\sqrt{x}}\]
At what $x-$value are you computing the derivative (given your previous answer)?
\[x=\answer{2}\]

\input{Derivative-Compute-0005.HELP.tex}


\end{problem}}

%%%%%%%%%%%%%%%%%%%%%%

\latexProblemContent{
\ifVerboseLocation This is Derivative Compute Question 0005. \\ \fi
\begin{problem}

Compute the following limit definition of derivative.
\[\lim_{h\to0}\frac{\frac{3}{h - 1} + 3}{h}=\answer{-3}\]
What function is being differentiated? (Assume there is no horizontal translation)
\[f(x)=\answer{\frac{3}{x}}\]
At what $x-$value are you computing the derivative (given your previous answer)?
\[x=\answer{-1}\]

\input{Derivative-Compute-0005.HELP.tex}


\end{problem}}

%%%%%%%%%%%%%%%%%%%%%%

\latexProblemContent{
\ifVerboseLocation This is Derivative Compute Question 0005. \\ \fi
\begin{problem}

Compute the following limit definition of derivative.
\[\lim_{h\to0}\frac{-5 \, {\left(h + 2\right)}^{3} + 40}{h}=\answer{-60}\]
What function is being differentiated? (Assume there is no horizontal translation)
\[f(x)=\answer{-5 \, x^{3}}\]
At what $x-$value are you computing the derivative (given your previous answer)?
\[x=\answer{2}\]

\input{Derivative-Compute-0005.HELP.tex}


\end{problem}}

%%%%%%%%%%%%%%%%%%%%%%

\latexProblemContent{
\ifVerboseLocation This is Derivative Compute Question 0005. \\ \fi
\begin{problem}

Compute the following limit definition of derivative.
\[\lim_{h\to0}\frac{-\frac{3}{h - 3} - 1}{h}=\answer{\frac{1}{3}}\]
What function is being differentiated? (Assume there is no horizontal translation)
\[f(x)=\answer{-\frac{3}{x}}\]
At what $x-$value are you computing the derivative (given your previous answer)?
\[x=\answer{-3}\]

\input{Derivative-Compute-0005.HELP.tex}


\end{problem}}

%%%%%%%%%%%%%%%%%%%%%%

\latexProblemContent{
\ifVerboseLocation This is Derivative Compute Question 0005. \\ \fi
\begin{problem}

Compute the following limit definition of derivative.
\[\lim_{h\to0}\frac{\frac{4}{h - 4} + 1}{h}=\answer{-\frac{1}{4}}\]
What function is being differentiated? (Assume there is no horizontal translation)
\[f(x)=\answer{\frac{4}{x}}\]
At what $x-$value are you computing the derivative (given your previous answer)?
\[x=\answer{-4}\]

\input{Derivative-Compute-0005.HELP.tex}


\end{problem}}

%%%%%%%%%%%%%%%%%%%%%%

\latexProblemContent{
\ifVerboseLocation This is Derivative Compute Question 0005. \\ \fi
\begin{problem}

Compute the following limit definition of derivative.
\[\lim_{h\to0}\frac{-\frac{1}{h + 2} + \frac{1}{2}}{h}=\answer{\frac{1}{4}}\]
What function is being differentiated? (Assume there is no horizontal translation)
\[f(x)=\answer{-\frac{1}{x}}\]
At what $x-$value are you computing the derivative (given your previous answer)?
\[x=\answer{2}\]

\input{Derivative-Compute-0005.HELP.tex}


\end{problem}}

%%%%%%%%%%%%%%%%%%%%%%

\latexProblemContent{
\ifVerboseLocation This is Derivative Compute Question 0005. \\ \fi
\begin{problem}

Compute the following limit definition of derivative.
\[\lim_{h\to0}\frac{5 \, {\left(h - 4\right)}^{3} + 320}{h}=\answer{240}\]
What function is being differentiated? (Assume there is no horizontal translation)
\[f(x)=\answer{5 \, x^{3}}\]
At what $x-$value are you computing the derivative (given your previous answer)?
\[x=\answer{-4}\]

\input{Derivative-Compute-0005.HELP.tex}


\end{problem}}

%%%%%%%%%%%%%%%%%%%%%%

\latexProblemContent{
\ifVerboseLocation This is Derivative Compute Question 0005. \\ \fi
\begin{problem}

Compute the following limit definition of derivative.
\[\lim_{h\to0}\frac{3 \, {\left(h + 4\right)}^{2} - 48}{h}=\answer{24}\]
What function is being differentiated? (Assume there is no horizontal translation)
\[f(x)=\answer{3 \, x^{2}}\]
At what $x-$value are you computing the derivative (given your previous answer)?
\[x=\answer{4}\]

\input{Derivative-Compute-0005.HELP.tex}


\end{problem}}

%%%%%%%%%%%%%%%%%%%%%%

\latexProblemContent{
\ifVerboseLocation This is Derivative Compute Question 0005. \\ \fi
\begin{problem}

Compute the following limit definition of derivative.
\[\lim_{h\to0}\frac{-3 \, {\left(h + 3\right)}^{3} + 81}{h}=\answer{-81}\]
What function is being differentiated? (Assume there is no horizontal translation)
\[f(x)=\answer{-3 \, x^{3}}\]
At what $x-$value are you computing the derivative (given your previous answer)?
\[x=\answer{3}\]

\input{Derivative-Compute-0005.HELP.tex}


\end{problem}}

%%%%%%%%%%%%%%%%%%%%%%

\latexProblemContent{
\ifVerboseLocation This is Derivative Compute Question 0005. \\ \fi
\begin{problem}

Compute the following limit definition of derivative.
\[\lim_{h\to0}\frac{\frac{2}{h - 4} + \frac{1}{2}}{h}=\answer{-\frac{1}{8}}\]
What function is being differentiated? (Assume there is no horizontal translation)
\[f(x)=\answer{\frac{2}{x}}\]
At what $x-$value are you computing the derivative (given your previous answer)?
\[x=\answer{-4}\]

\input{Derivative-Compute-0005.HELP.tex}


\end{problem}}

%%%%%%%%%%%%%%%%%%%%%%

\latexProblemContent{
\ifVerboseLocation This is Derivative Compute Question 0005. \\ \fi
\begin{problem}

Compute the following limit definition of derivative.
\[\lim_{h\to0}\frac{4 \, {\left(h - 3\right)}^{3} + 108}{h}=\answer{108}\]
What function is being differentiated? (Assume there is no horizontal translation)
\[f(x)=\answer{4 \, x^{3}}\]
At what $x-$value are you computing the derivative (given your previous answer)?
\[x=\answer{-3}\]

\input{Derivative-Compute-0005.HELP.tex}


\end{problem}}

%%%%%%%%%%%%%%%%%%%%%%

\latexProblemContent{
\ifVerboseLocation This is Derivative Compute Question 0005. \\ \fi
\begin{problem}

Compute the following limit definition of derivative.
\[\lim_{h\to0}\frac{-5 \, {\left(h + 1\right)}^{2} + 5}{h}=\answer{-10}\]
What function is being differentiated? (Assume there is no horizontal translation)
\[f(x)=\answer{-5 \, x^{2}}\]
At what $x-$value are you computing the derivative (given your previous answer)?
\[x=\answer{1}\]

\input{Derivative-Compute-0005.HELP.tex}


\end{problem}}

%%%%%%%%%%%%%%%%%%%%%%

\latexProblemContent{
\ifVerboseLocation This is Derivative Compute Question 0005. \\ \fi
\begin{problem}

Compute the following limit definition of derivative.
\[\lim_{h\to0}\frac{-4 \, {\left(h - 3\right)}^{2} + 36}{h}=\answer{24}\]
What function is being differentiated? (Assume there is no horizontal translation)
\[f(x)=\answer{-4 \, x^{2}}\]
At what $x-$value are you computing the derivative (given your previous answer)?
\[x=\answer{-3}\]

\input{Derivative-Compute-0005.HELP.tex}


\end{problem}}

%%%%%%%%%%%%%%%%%%%%%%

\latexProblemContent{
\ifVerboseLocation This is Derivative Compute Question 0005. \\ \fi
\begin{problem}

Compute the following limit definition of derivative.
\[\lim_{h\to0}\frac{2 \, {\left(h + 4\right)}^{3} - 128}{h}=\answer{96}\]
What function is being differentiated? (Assume there is no horizontal translation)
\[f(x)=\answer{2 \, x^{3}}\]
At what $x-$value are you computing the derivative (given your previous answer)?
\[x=\answer{4}\]

\input{Derivative-Compute-0005.HELP.tex}


\end{problem}}

%%%%%%%%%%%%%%%%%%%%%%

\latexProblemContent{
\ifVerboseLocation This is Derivative Compute Question 0005. \\ \fi
\begin{problem}

Compute the following limit definition of derivative.
\[\lim_{h\to0}\frac{-\frac{5}{h - 1} - 5}{h}=\answer{5}\]
What function is being differentiated? (Assume there is no horizontal translation)
\[f(x)=\answer{-\frac{5}{x}}\]
At what $x-$value are you computing the derivative (given your previous answer)?
\[x=\answer{-1}\]

\input{Derivative-Compute-0005.HELP.tex}


\end{problem}}

%%%%%%%%%%%%%%%%%%%%%%

\latexProblemContent{
\ifVerboseLocation This is Derivative Compute Question 0005. \\ \fi
\begin{problem}

Compute the following limit definition of derivative.
\[\lim_{h\to0}\frac{-3 \, {\left(h + 1\right)}^{3} + 3}{h}=\answer{-9}\]
What function is being differentiated? (Assume there is no horizontal translation)
\[f(x)=\answer{-3 \, x^{3}}\]
At what $x-$value are you computing the derivative (given your previous answer)?
\[x=\answer{1}\]

\input{Derivative-Compute-0005.HELP.tex}


\end{problem}}

%%%%%%%%%%%%%%%%%%%%%%

\latexProblemContent{
\ifVerboseLocation This is Derivative Compute Question 0005. \\ \fi
\begin{problem}

Compute the following limit definition of derivative.
\[\lim_{h\to0}\frac{-2 \, \sqrt{3} + 2 \, \sqrt{h + 3}}{h}=\answer{\frac{1}{3} \, \sqrt{3}}\]
What function is being differentiated? (Assume there is no horizontal translation)
\[f(x)=\answer{2 \, \sqrt{x}}\]
At what $x-$value are you computing the derivative (given your previous answer)?
\[x=\answer{3}\]

\input{Derivative-Compute-0005.HELP.tex}


\end{problem}}

%%%%%%%%%%%%%%%%%%%%%%

\latexProblemContent{
\ifVerboseLocation This is Derivative Compute Question 0005. \\ \fi
\begin{problem}

Compute the following limit definition of derivative.
\[\lim_{h\to0}\frac{\frac{1}{h - 1} + 1}{h}=\answer{-1}\]
What function is being differentiated? (Assume there is no horizontal translation)
\[f(x)=\answer{\frac{1}{x}}\]
At what $x-$value are you computing the derivative (given your previous answer)?
\[x=\answer{-1}\]

\input{Derivative-Compute-0005.HELP.tex}


\end{problem}}

%%%%%%%%%%%%%%%%%%%%%%

\latexProblemContent{
\ifVerboseLocation This is Derivative Compute Question 0005. \\ \fi
\begin{problem}

Compute the following limit definition of derivative.
\[\lim_{h\to0}\frac{\frac{4}{h - 1} + 4}{h}=\answer{-4}\]
What function is being differentiated? (Assume there is no horizontal translation)
\[f(x)=\answer{\frac{4}{x}}\]
At what $x-$value are you computing the derivative (given your previous answer)?
\[x=\answer{-1}\]

\input{Derivative-Compute-0005.HELP.tex}


\end{problem}}

%%%%%%%%%%%%%%%%%%%%%%

\latexProblemContent{
\ifVerboseLocation This is Derivative Compute Question 0005. \\ \fi
\begin{problem}

Compute the following limit definition of derivative.
\[\lim_{h\to0}\frac{-3 \, {\left(h - 2\right)}^{2} + 12}{h}=\answer{12}\]
What function is being differentiated? (Assume there is no horizontal translation)
\[f(x)=\answer{-3 \, x^{2}}\]
At what $x-$value are you computing the derivative (given your previous answer)?
\[x=\answer{-2}\]

\input{Derivative-Compute-0005.HELP.tex}


\end{problem}}

%%%%%%%%%%%%%%%%%%%%%%

\latexProblemContent{
\ifVerboseLocation This is Derivative Compute Question 0005. \\ \fi
\begin{problem}

Compute the following limit definition of derivative.
\[\lim_{h\to0}\frac{-4 \, {\left(h - 4\right)}^{3} - 256}{h}=\answer{-192}\]
What function is being differentiated? (Assume there is no horizontal translation)
\[f(x)=\answer{-4 \, x^{3}}\]
At what $x-$value are you computing the derivative (given your previous answer)?
\[x=\answer{-4}\]

\input{Derivative-Compute-0005.HELP.tex}


\end{problem}}

%%%%%%%%%%%%%%%%%%%%%%

\latexProblemContent{
\ifVerboseLocation This is Derivative Compute Question 0005. \\ \fi
\begin{problem}

Compute the following limit definition of derivative.
\[\lim_{h\to0}\frac{-5 \, {\left(h - 3\right)}^{3} - 135}{h}=\answer{-135}\]
What function is being differentiated? (Assume there is no horizontal translation)
\[f(x)=\answer{-5 \, x^{3}}\]
At what $x-$value are you computing the derivative (given your previous answer)?
\[x=\answer{-3}\]

\input{Derivative-Compute-0005.HELP.tex}


\end{problem}}

%%%%%%%%%%%%%%%%%%%%%%

\latexProblemContent{
\ifVerboseLocation This is Derivative Compute Question 0005. \\ \fi
\begin{problem}

Compute the following limit definition of derivative.
\[\lim_{h\to0}\frac{3 \, \sqrt{h + 4} - 6}{h}=\answer{\frac{3}{4}}\]
What function is being differentiated? (Assume there is no horizontal translation)
\[f(x)=\answer{3 \, \sqrt{x}}\]
At what $x-$value are you computing the derivative (given your previous answer)?
\[x=\answer{4}\]

\input{Derivative-Compute-0005.HELP.tex}


\end{problem}}

%%%%%%%%%%%%%%%%%%%%%%

\latexProblemContent{
\ifVerboseLocation This is Derivative Compute Question 0005. \\ \fi
\begin{problem}

Compute the following limit definition of derivative.
\[\lim_{h\to0}\frac{-3 \, {\left(h - 3\right)}^{2} + 27}{h}=\answer{18}\]
What function is being differentiated? (Assume there is no horizontal translation)
\[f(x)=\answer{-3 \, x^{2}}\]
At what $x-$value are you computing the derivative (given your previous answer)?
\[x=\answer{-3}\]

\input{Derivative-Compute-0005.HELP.tex}


\end{problem}}

%%%%%%%%%%%%%%%%%%%%%%

\latexProblemContent{
\ifVerboseLocation This is Derivative Compute Question 0005. \\ \fi
\begin{problem}

Compute the following limit definition of derivative.
\[\lim_{h\to0}\frac{2 \, {\left(h - 3\right)}^{2} - 18}{h}=\answer{-12}\]
What function is being differentiated? (Assume there is no horizontal translation)
\[f(x)=\answer{2 \, x^{2}}\]
At what $x-$value are you computing the derivative (given your previous answer)?
\[x=\answer{-3}\]

\input{Derivative-Compute-0005.HELP.tex}


\end{problem}}

%%%%%%%%%%%%%%%%%%%%%%

\latexProblemContent{
\ifVerboseLocation This is Derivative Compute Question 0005. \\ \fi
\begin{problem}

Compute the following limit definition of derivative.
\[\lim_{h\to0}\frac{-\frac{5}{h + 3} + \frac{5}{3}}{h}=\answer{\frac{5}{9}}\]
What function is being differentiated? (Assume there is no horizontal translation)
\[f(x)=\answer{-\frac{5}{x}}\]
At what $x-$value are you computing the derivative (given your previous answer)?
\[x=\answer{3}\]

\input{Derivative-Compute-0005.HELP.tex}


\end{problem}}

%%%%%%%%%%%%%%%%%%%%%%

\latexProblemContent{
\ifVerboseLocation This is Derivative Compute Question 0005. \\ \fi
\begin{problem}

Compute the following limit definition of derivative.
\[\lim_{h\to0}\frac{-{\left(h + 3\right)}^{2} + 9}{h}=\answer{-6}\]
What function is being differentiated? (Assume there is no horizontal translation)
\[f(x)=\answer{-x^{2}}\]
At what $x-$value are you computing the derivative (given your previous answer)?
\[x=\answer{3}\]

\input{Derivative-Compute-0005.HELP.tex}


\end{problem}}

%%%%%%%%%%%%%%%%%%%%%%

\latexProblemContent{
\ifVerboseLocation This is Derivative Compute Question 0005. \\ \fi
\begin{problem}

Compute the following limit definition of derivative.
\[\lim_{h\to0}\frac{3 \, {\left(h - 2\right)}^{2} - 12}{h}=\answer{-12}\]
What function is being differentiated? (Assume there is no horizontal translation)
\[f(x)=\answer{3 \, x^{2}}\]
At what $x-$value are you computing the derivative (given your previous answer)?
\[x=\answer{-2}\]

\input{Derivative-Compute-0005.HELP.tex}


\end{problem}}

%%%%%%%%%%%%%%%%%%%%%%

\latexProblemContent{
\ifVerboseLocation This is Derivative Compute Question 0005. \\ \fi
\begin{problem}

Compute the following limit definition of derivative.
\[\lim_{h\to0}\frac{-2 \, {\left(h - 4\right)}^{3} - 128}{h}=\answer{-96}\]
What function is being differentiated? (Assume there is no horizontal translation)
\[f(x)=\answer{-2 \, x^{3}}\]
At what $x-$value are you computing the derivative (given your previous answer)?
\[x=\answer{-4}\]

\input{Derivative-Compute-0005.HELP.tex}


\end{problem}}

%%%%%%%%%%%%%%%%%%%%%%

\latexProblemContent{
\ifVerboseLocation This is Derivative Compute Question 0005. \\ \fi
\begin{problem}

Compute the following limit definition of derivative.
\[\lim_{h\to0}\frac{4 \, {\left(h + 4\right)}^{2} - 64}{h}=\answer{32}\]
What function is being differentiated? (Assume there is no horizontal translation)
\[f(x)=\answer{4 \, x^{2}}\]
At what $x-$value are you computing the derivative (given your previous answer)?
\[x=\answer{4}\]

\input{Derivative-Compute-0005.HELP.tex}


\end{problem}}

%%%%%%%%%%%%%%%%%%%%%%

\latexProblemContent{
\ifVerboseLocation This is Derivative Compute Question 0005. \\ \fi
\begin{problem}

Compute the following limit definition of derivative.
\[\lim_{h\to0}\frac{-\sqrt{3} + \sqrt{h + 3}}{h}=\answer{\frac{1}{6} \, \sqrt{3}}\]
What function is being differentiated? (Assume there is no horizontal translation)
\[f(x)=\answer{\sqrt{x}}\]
At what $x-$value are you computing the derivative (given your previous answer)?
\[x=\answer{3}\]

\input{Derivative-Compute-0005.HELP.tex}


\end{problem}}

%%%%%%%%%%%%%%%%%%%%%%

\latexProblemContent{
\ifVerboseLocation This is Derivative Compute Question 0005. \\ \fi
\begin{problem}

Compute the following limit definition of derivative.
\[\lim_{h\to0}\frac{4 \, {\left(h - 2\right)}^{2} - 16}{h}=\answer{-16}\]
What function is being differentiated? (Assume there is no horizontal translation)
\[f(x)=\answer{4 \, x^{2}}\]
At what $x-$value are you computing the derivative (given your previous answer)?
\[x=\answer{-2}\]

\input{Derivative-Compute-0005.HELP.tex}


\end{problem}}

%%%%%%%%%%%%%%%%%%%%%%

\latexProblemContent{
\ifVerboseLocation This is Derivative Compute Question 0005. \\ \fi
\begin{problem}

Compute the following limit definition of derivative.
\[\lim_{h\to0}\frac{-\frac{2}{h - 3} - \frac{2}{3}}{h}=\answer{\frac{2}{9}}\]
What function is being differentiated? (Assume there is no horizontal translation)
\[f(x)=\answer{-\frac{2}{x}}\]
At what $x-$value are you computing the derivative (given your previous answer)?
\[x=\answer{-3}\]

\input{Derivative-Compute-0005.HELP.tex}


\end{problem}}

%%%%%%%%%%%%%%%%%%%%%%

\latexProblemContent{
\ifVerboseLocation This is Derivative Compute Question 0005. \\ \fi
\begin{problem}

Compute the following limit definition of derivative.
\[\lim_{h\to0}\frac{-\frac{5}{h + 4} + \frac{5}{4}}{h}=\answer{\frac{5}{16}}\]
What function is being differentiated? (Assume there is no horizontal translation)
\[f(x)=\answer{-\frac{5}{x}}\]
At what $x-$value are you computing the derivative (given your previous answer)?
\[x=\answer{4}\]

\input{Derivative-Compute-0005.HELP.tex}


\end{problem}}

%%%%%%%%%%%%%%%%%%%%%%

\latexProblemContent{
\ifVerboseLocation This is Derivative Compute Question 0005. \\ \fi
\begin{problem}

Compute the following limit definition of derivative.
\[\lim_{h\to0}\frac{-5 \, \sqrt{h + 1} + 5}{h}=\answer{-\frac{5}{2}}\]
What function is being differentiated? (Assume there is no horizontal translation)
\[f(x)=\answer{-5 \, \sqrt{x}}\]
At what $x-$value are you computing the derivative (given your previous answer)?
\[x=\answer{1}\]

\input{Derivative-Compute-0005.HELP.tex}


\end{problem}}

%%%%%%%%%%%%%%%%%%%%%%

\latexProblemContent{
\ifVerboseLocation This is Derivative Compute Question 0005. \\ \fi
\begin{problem}

Compute the following limit definition of derivative.
\[\lim_{h\to0}\frac{5 \, \sqrt{h + 1} - 5}{h}=\answer{\frac{5}{2}}\]
What function is being differentiated? (Assume there is no horizontal translation)
\[f(x)=\answer{5 \, \sqrt{x}}\]
At what $x-$value are you computing the derivative (given your previous answer)?
\[x=\answer{1}\]

\input{Derivative-Compute-0005.HELP.tex}


\end{problem}}

%%%%%%%%%%%%%%%%%%%%%%

\latexProblemContent{
\ifVerboseLocation This is Derivative Compute Question 0005. \\ \fi
\begin{problem}

Compute the following limit definition of derivative.
\[\lim_{h\to0}\frac{4 \, \sqrt{2} - 4 \, \sqrt{h + 2}}{h}=\answer{-\sqrt{2}}\]
What function is being differentiated? (Assume there is no horizontal translation)
\[f(x)=\answer{-4 \, \sqrt{x}}\]
At what $x-$value are you computing the derivative (given your previous answer)?
\[x=\answer{2}\]

\input{Derivative-Compute-0005.HELP.tex}


\end{problem}}

%%%%%%%%%%%%%%%%%%%%%%

\latexProblemContent{
\ifVerboseLocation This is Derivative Compute Question 0005. \\ \fi
\begin{problem}

Compute the following limit definition of derivative.
\[\lim_{h\to0}\frac{\frac{2}{h - 3} + \frac{2}{3}}{h}=\answer{-\frac{2}{9}}\]
What function is being differentiated? (Assume there is no horizontal translation)
\[f(x)=\answer{\frac{2}{x}}\]
At what $x-$value are you computing the derivative (given your previous answer)?
\[x=\answer{-3}\]

\input{Derivative-Compute-0005.HELP.tex}


\end{problem}}

%%%%%%%%%%%%%%%%%%%%%%

\latexProblemContent{
\ifVerboseLocation This is Derivative Compute Question 0005. \\ \fi
\begin{problem}

Compute the following limit definition of derivative.
\[\lim_{h\to0}\frac{{\left(h + 4\right)}^{3} - 64}{h}=\answer{48}\]
What function is being differentiated? (Assume there is no horizontal translation)
\[f(x)=\answer{x^{3}}\]
At what $x-$value are you computing the derivative (given your previous answer)?
\[x=\answer{4}\]

\input{Derivative-Compute-0005.HELP.tex}


\end{problem}}

%%%%%%%%%%%%%%%%%%%%%%

\latexProblemContent{
\ifVerboseLocation This is Derivative Compute Question 0005. \\ \fi
\begin{problem}

Compute the following limit definition of derivative.
\[\lim_{h\to0}\frac{3 \, {\left(h + 2\right)}^{2} - 12}{h}=\answer{12}\]
What function is being differentiated? (Assume there is no horizontal translation)
\[f(x)=\answer{3 \, x^{2}}\]
At what $x-$value are you computing the derivative (given your previous answer)?
\[x=\answer{2}\]

\input{Derivative-Compute-0005.HELP.tex}


\end{problem}}

%%%%%%%%%%%%%%%%%%%%%%

\latexProblemContent{
\ifVerboseLocation This is Derivative Compute Question 0005. \\ \fi
\begin{problem}

Compute the following limit definition of derivative.
\[\lim_{h\to0}\frac{-4 \, {\left(h + 3\right)}^{3} + 108}{h}=\answer{-108}\]
What function is being differentiated? (Assume there is no horizontal translation)
\[f(x)=\answer{-4 \, x^{3}}\]
At what $x-$value are you computing the derivative (given your previous answer)?
\[x=\answer{3}\]

\input{Derivative-Compute-0005.HELP.tex}


\end{problem}}

%%%%%%%%%%%%%%%%%%%%%%

\latexProblemContent{
\ifVerboseLocation This is Derivative Compute Question 0005. \\ \fi
\begin{problem}

Compute the following limit definition of derivative.
\[\lim_{h\to0}\frac{-3 \, {\left(h - 1\right)}^{3} - 3}{h}=\answer{-9}\]
What function is being differentiated? (Assume there is no horizontal translation)
\[f(x)=\answer{-3 \, x^{3}}\]
At what $x-$value are you computing the derivative (given your previous answer)?
\[x=\answer{-1}\]

\input{Derivative-Compute-0005.HELP.tex}


\end{problem}}

%%%%%%%%%%%%%%%%%%%%%%

\latexProblemContent{
\ifVerboseLocation This is Derivative Compute Question 0005. \\ \fi
\begin{problem}

Compute the following limit definition of derivative.
\[\lim_{h\to0}\frac{{\left(h - 2\right)}^{2} - 4}{h}=\answer{-4}\]
What function is being differentiated? (Assume there is no horizontal translation)
\[f(x)=\answer{x^{2}}\]
At what $x-$value are you computing the derivative (given your previous answer)?
\[x=\answer{-2}\]

\input{Derivative-Compute-0005.HELP.tex}


\end{problem}}

%%%%%%%%%%%%%%%%%%%%%%

\latexProblemContent{
\ifVerboseLocation This is Derivative Compute Question 0005. \\ \fi
\begin{problem}

Compute the following limit definition of derivative.
\[\lim_{h\to0}\frac{-5 \, {\left(h + 1\right)}^{3} + 5}{h}=\answer{-15}\]
What function is being differentiated? (Assume there is no horizontal translation)
\[f(x)=\answer{-5 \, x^{3}}\]
At what $x-$value are you computing the derivative (given your previous answer)?
\[x=\answer{1}\]

\input{Derivative-Compute-0005.HELP.tex}


\end{problem}}

%%%%%%%%%%%%%%%%%%%%%%

\latexProblemContent{
\ifVerboseLocation This is Derivative Compute Question 0005. \\ \fi
\begin{problem}

Compute the following limit definition of derivative.
\[\lim_{h\to0}\frac{-2 \, {\left(h - 2\right)}^{2} + 8}{h}=\answer{8}\]
What function is being differentiated? (Assume there is no horizontal translation)
\[f(x)=\answer{-2 \, x^{2}}\]
At what $x-$value are you computing the derivative (given your previous answer)?
\[x=\answer{-2}\]

\input{Derivative-Compute-0005.HELP.tex}


\end{problem}}

%%%%%%%%%%%%%%%%%%%%%%

\latexProblemContent{
\ifVerboseLocation This is Derivative Compute Question 0005. \\ \fi
\begin{problem}

Compute the following limit definition of derivative.
\[\lim_{h\to0}\frac{\frac{5}{h + 1} - 5}{h}=\answer{-5}\]
What function is being differentiated? (Assume there is no horizontal translation)
\[f(x)=\answer{\frac{5}{x}}\]
At what $x-$value are you computing the derivative (given your previous answer)?
\[x=\answer{1}\]

\input{Derivative-Compute-0005.HELP.tex}


\end{problem}}

%%%%%%%%%%%%%%%%%%%%%%

\latexProblemContent{
\ifVerboseLocation This is Derivative Compute Question 0005. \\ \fi
\begin{problem}

Compute the following limit definition of derivative.
\[\lim_{h\to0}\frac{\frac{4}{h - 3} + \frac{4}{3}}{h}=\answer{-\frac{4}{9}}\]
What function is being differentiated? (Assume there is no horizontal translation)
\[f(x)=\answer{\frac{4}{x}}\]
At what $x-$value are you computing the derivative (given your previous answer)?
\[x=\answer{-3}\]

\input{Derivative-Compute-0005.HELP.tex}


\end{problem}}

%%%%%%%%%%%%%%%%%%%%%%

\latexProblemContent{
\ifVerboseLocation This is Derivative Compute Question 0005. \\ \fi
\begin{problem}

Compute the following limit definition of derivative.
\[\lim_{h\to0}\frac{5 \, {\left(h + 2\right)}^{3} - 40}{h}=\answer{60}\]
What function is being differentiated? (Assume there is no horizontal translation)
\[f(x)=\answer{5 \, x^{3}}\]
At what $x-$value are you computing the derivative (given your previous answer)?
\[x=\answer{2}\]

\input{Derivative-Compute-0005.HELP.tex}


\end{problem}}

%%%%%%%%%%%%%%%%%%%%%%

\latexProblemContent{
\ifVerboseLocation This is Derivative Compute Question 0005. \\ \fi
\begin{problem}

Compute the following limit definition of derivative.
\[\lim_{h\to0}\frac{5 \, {\left(h + 1\right)}^{2} - 5}{h}=\answer{10}\]
What function is being differentiated? (Assume there is no horizontal translation)
\[f(x)=\answer{5 \, x^{2}}\]
At what $x-$value are you computing the derivative (given your previous answer)?
\[x=\answer{1}\]

\input{Derivative-Compute-0005.HELP.tex}


\end{problem}}

%%%%%%%%%%%%%%%%%%%%%%

\latexProblemContent{
\ifVerboseLocation This is Derivative Compute Question 0005. \\ \fi
\begin{problem}

Compute the following limit definition of derivative.
\[\lim_{h\to0}\frac{\sqrt{2} - \sqrt{h + 2}}{h}=\answer{-\frac{1}{4} \, \sqrt{2}}\]
What function is being differentiated? (Assume there is no horizontal translation)
\[f(x)=\answer{-\sqrt{x}}\]
At what $x-$value are you computing the derivative (given your previous answer)?
\[x=\answer{2}\]

\input{Derivative-Compute-0005.HELP.tex}


\end{problem}}

%%%%%%%%%%%%%%%%%%%%%%

\latexProblemContent{
\ifVerboseLocation This is Derivative Compute Question 0005. \\ \fi
\begin{problem}

Compute the following limit definition of derivative.
\[\lim_{h\to0}\frac{-3 \, {\left(h + 4\right)}^{2} + 48}{h}=\answer{-24}\]
What function is being differentiated? (Assume there is no horizontal translation)
\[f(x)=\answer{-3 \, x^{2}}\]
At what $x-$value are you computing the derivative (given your previous answer)?
\[x=\answer{4}\]

\input{Derivative-Compute-0005.HELP.tex}


\end{problem}}

%%%%%%%%%%%%%%%%%%%%%%

\latexProblemContent{
\ifVerboseLocation This is Derivative Compute Question 0005. \\ \fi
\begin{problem}

Compute the following limit definition of derivative.
\[\lim_{h\to0}\frac{-4 \, {\left(h + 4\right)}^{3} + 256}{h}=\answer{-192}\]
What function is being differentiated? (Assume there is no horizontal translation)
\[f(x)=\answer{-4 \, x^{3}}\]
At what $x-$value are you computing the derivative (given your previous answer)?
\[x=\answer{4}\]

\input{Derivative-Compute-0005.HELP.tex}


\end{problem}}

%%%%%%%%%%%%%%%%%%%%%%

\latexProblemContent{
\ifVerboseLocation This is Derivative Compute Question 0005. \\ \fi
\begin{problem}

Compute the following limit definition of derivative.
\[\lim_{h\to0}\frac{-\frac{4}{h - 1} - 4}{h}=\answer{4}\]
What function is being differentiated? (Assume there is no horizontal translation)
\[f(x)=\answer{-\frac{4}{x}}\]
At what $x-$value are you computing the derivative (given your previous answer)?
\[x=\answer{-1}\]

\input{Derivative-Compute-0005.HELP.tex}


\end{problem}}

%%%%%%%%%%%%%%%%%%%%%%

\latexProblemContent{
\ifVerboseLocation This is Derivative Compute Question 0005. \\ \fi
\begin{problem}

Compute the following limit definition of derivative.
\[\lim_{h\to0}\frac{2 \, {\left(h + 1\right)}^{3} - 2}{h}=\answer{6}\]
What function is being differentiated? (Assume there is no horizontal translation)
\[f(x)=\answer{2 \, x^{3}}\]
At what $x-$value are you computing the derivative (given your previous answer)?
\[x=\answer{1}\]

\input{Derivative-Compute-0005.HELP.tex}


\end{problem}}

%%%%%%%%%%%%%%%%%%%%%%

\latexProblemContent{
\ifVerboseLocation This is Derivative Compute Question 0005. \\ \fi
\begin{problem}

Compute the following limit definition of derivative.
\[\lim_{h\to0}\frac{-\frac{5}{h - 2} - \frac{5}{2}}{h}=\answer{\frac{5}{4}}\]
What function is being differentiated? (Assume there is no horizontal translation)
\[f(x)=\answer{-\frac{5}{x}}\]
At what $x-$value are you computing the derivative (given your previous answer)?
\[x=\answer{-2}\]

\input{Derivative-Compute-0005.HELP.tex}


\end{problem}}

%%%%%%%%%%%%%%%%%%%%%%

\latexProblemContent{
\ifVerboseLocation This is Derivative Compute Question 0005. \\ \fi
\begin{problem}

Compute the following limit definition of derivative.
\[\lim_{h\to0}\frac{-2 \, {\left(h + 1\right)}^{3} + 2}{h}=\answer{-6}\]
What function is being differentiated? (Assume there is no horizontal translation)
\[f(x)=\answer{-2 \, x^{3}}\]
At what $x-$value are you computing the derivative (given your previous answer)?
\[x=\answer{1}\]

\input{Derivative-Compute-0005.HELP.tex}


\end{problem}}

%%%%%%%%%%%%%%%%%%%%%%

\latexProblemContent{
\ifVerboseLocation This is Derivative Compute Question 0005. \\ \fi
\begin{problem}

Compute the following limit definition of derivative.
\[\lim_{h\to0}\frac{-3 \, {\left(h - 3\right)}^{3} - 81}{h}=\answer{-81}\]
What function is being differentiated? (Assume there is no horizontal translation)
\[f(x)=\answer{-3 \, x^{3}}\]
At what $x-$value are you computing the derivative (given your previous answer)?
\[x=\answer{-3}\]

\input{Derivative-Compute-0005.HELP.tex}


\end{problem}}

%%%%%%%%%%%%%%%%%%%%%%

\latexProblemContent{
\ifVerboseLocation This is Derivative Compute Question 0005. \\ \fi
\begin{problem}

Compute the following limit definition of derivative.
\[\lim_{h\to0}\frac{-\frac{5}{h - 4} - \frac{5}{4}}{h}=\answer{\frac{5}{16}}\]
What function is being differentiated? (Assume there is no horizontal translation)
\[f(x)=\answer{-\frac{5}{x}}\]
At what $x-$value are you computing the derivative (given your previous answer)?
\[x=\answer{-4}\]

\input{Derivative-Compute-0005.HELP.tex}


\end{problem}}

%%%%%%%%%%%%%%%%%%%%%%

\latexProblemContent{
\ifVerboseLocation This is Derivative Compute Question 0005. \\ \fi
\begin{problem}

Compute the following limit definition of derivative.
\[\lim_{h\to0}\frac{2 \, {\left(h - 2\right)}^{2} - 8}{h}=\answer{-8}\]
What function is being differentiated? (Assume there is no horizontal translation)
\[f(x)=\answer{2 \, x^{2}}\]
At what $x-$value are you computing the derivative (given your previous answer)?
\[x=\answer{-2}\]

\input{Derivative-Compute-0005.HELP.tex}


\end{problem}}

%%%%%%%%%%%%%%%%%%%%%%

\latexProblemContent{
\ifVerboseLocation This is Derivative Compute Question 0005. \\ \fi
\begin{problem}

Compute the following limit definition of derivative.
\[\lim_{h\to0}\frac{-\sqrt{h + 4} + 2}{h}=\answer{-\frac{1}{4}}\]
What function is being differentiated? (Assume there is no horizontal translation)
\[f(x)=\answer{-\sqrt{x}}\]
At what $x-$value are you computing the derivative (given your previous answer)?
\[x=\answer{4}\]

\input{Derivative-Compute-0005.HELP.tex}


\end{problem}}

%%%%%%%%%%%%%%%%%%%%%%

\latexProblemContent{
\ifVerboseLocation This is Derivative Compute Question 0005. \\ \fi
\begin{problem}

Compute the following limit definition of derivative.
\[\lim_{h\to0}\frac{-\frac{3}{h + 2} + \frac{3}{2}}{h}=\answer{\frac{3}{4}}\]
What function is being differentiated? (Assume there is no horizontal translation)
\[f(x)=\answer{-\frac{3}{x}}\]
At what $x-$value are you computing the derivative (given your previous answer)?
\[x=\answer{2}\]

\input{Derivative-Compute-0005.HELP.tex}


\end{problem}}

%%%%%%%%%%%%%%%%%%%%%%

\latexProblemContent{
\ifVerboseLocation This is Derivative Compute Question 0005. \\ \fi
\begin{problem}

Compute the following limit definition of derivative.
\[\lim_{h\to0}\frac{{\left(h - 4\right)}^{3} + 64}{h}=\answer{48}\]
What function is being differentiated? (Assume there is no horizontal translation)
\[f(x)=\answer{x^{3}}\]
At what $x-$value are you computing the derivative (given your previous answer)?
\[x=\answer{-4}\]

\input{Derivative-Compute-0005.HELP.tex}


\end{problem}}

%%%%%%%%%%%%%%%%%%%%%%

\latexProblemContent{
\ifVerboseLocation This is Derivative Compute Question 0005. \\ \fi
\begin{problem}

Compute the following limit definition of derivative.
\[\lim_{h\to0}\frac{-3 \, {\left(h + 2\right)}^{3} + 24}{h}=\answer{-36}\]
What function is being differentiated? (Assume there is no horizontal translation)
\[f(x)=\answer{-3 \, x^{3}}\]
At what $x-$value are you computing the derivative (given your previous answer)?
\[x=\answer{2}\]

\input{Derivative-Compute-0005.HELP.tex}


\end{problem}}

%%%%%%%%%%%%%%%%%%%%%%

\latexProblemContent{
\ifVerboseLocation This is Derivative Compute Question 0005. \\ \fi
\begin{problem}

Compute the following limit definition of derivative.
\[\lim_{h\to0}\frac{5 \, {\left(h - 3\right)}^{2} - 45}{h}=\answer{-30}\]
What function is being differentiated? (Assume there is no horizontal translation)
\[f(x)=\answer{5 \, x^{2}}\]
At what $x-$value are you computing the derivative (given your previous answer)?
\[x=\answer{-3}\]

\input{Derivative-Compute-0005.HELP.tex}


\end{problem}}

%%%%%%%%%%%%%%%%%%%%%%

\latexProblemContent{
\ifVerboseLocation This is Derivative Compute Question 0005. \\ \fi
\begin{problem}

Compute the following limit definition of derivative.
\[\lim_{h\to0}\frac{-5 \, {\left(h + 4\right)}^{2} + 80}{h}=\answer{-40}\]
What function is being differentiated? (Assume there is no horizontal translation)
\[f(x)=\answer{-5 \, x^{2}}\]
At what $x-$value are you computing the derivative (given your previous answer)?
\[x=\answer{4}\]

\input{Derivative-Compute-0005.HELP.tex}


\end{problem}}

%%%%%%%%%%%%%%%%%%%%%%

\latexProblemContent{
\ifVerboseLocation This is Derivative Compute Question 0005. \\ \fi
\begin{problem}

Compute the following limit definition of derivative.
\[\lim_{h\to0}\frac{2 \, \sqrt{3} - 2 \, \sqrt{h + 3}}{h}=\answer{-\frac{1}{3} \, \sqrt{3}}\]
What function is being differentiated? (Assume there is no horizontal translation)
\[f(x)=\answer{-2 \, \sqrt{x}}\]
At what $x-$value are you computing the derivative (given your previous answer)?
\[x=\answer{3}\]

\input{Derivative-Compute-0005.HELP.tex}


\end{problem}}

%%%%%%%%%%%%%%%%%%%%%%

\latexProblemContent{
\ifVerboseLocation This is Derivative Compute Question 0005. \\ \fi
\begin{problem}

Compute the following limit definition of derivative.
\[\lim_{h\to0}\frac{4 \, {\left(h + 3\right)}^{3} - 108}{h}=\answer{108}\]
What function is being differentiated? (Assume there is no horizontal translation)
\[f(x)=\answer{4 \, x^{3}}\]
At what $x-$value are you computing the derivative (given your previous answer)?
\[x=\answer{3}\]

\input{Derivative-Compute-0005.HELP.tex}


\end{problem}}

%%%%%%%%%%%%%%%%%%%%%%

\latexProblemContent{
\ifVerboseLocation This is Derivative Compute Question 0005. \\ \fi
\begin{problem}

Compute the following limit definition of derivative.
\[\lim_{h\to0}\frac{-{\left(h - 1\right)}^{3} - 1}{h}=\answer{-3}\]
What function is being differentiated? (Assume there is no horizontal translation)
\[f(x)=\answer{-x^{3}}\]
At what $x-$value are you computing the derivative (given your previous answer)?
\[x=\answer{-1}\]

\input{Derivative-Compute-0005.HELP.tex}


\end{problem}}

%%%%%%%%%%%%%%%%%%%%%%

\latexProblemContent{
\ifVerboseLocation This is Derivative Compute Question 0005. \\ \fi
\begin{problem}

Compute the following limit definition of derivative.
\[\lim_{h\to0}\frac{-3 \, {\left(h - 2\right)}^{3} - 24}{h}=\answer{-36}\]
What function is being differentiated? (Assume there is no horizontal translation)
\[f(x)=\answer{-3 \, x^{3}}\]
At what $x-$value are you computing the derivative (given your previous answer)?
\[x=\answer{-2}\]

\input{Derivative-Compute-0005.HELP.tex}


\end{problem}}

%%%%%%%%%%%%%%%%%%%%%%

\latexProblemContent{
\ifVerboseLocation This is Derivative Compute Question 0005. \\ \fi
\begin{problem}

Compute the following limit definition of derivative.
\[\lim_{h\to0}\frac{\frac{5}{h - 2} + \frac{5}{2}}{h}=\answer{-\frac{5}{4}}\]
What function is being differentiated? (Assume there is no horizontal translation)
\[f(x)=\answer{\frac{5}{x}}\]
At what $x-$value are you computing the derivative (given your previous answer)?
\[x=\answer{-2}\]

\input{Derivative-Compute-0005.HELP.tex}


\end{problem}}

%%%%%%%%%%%%%%%%%%%%%%

\latexProblemContent{
\ifVerboseLocation This is Derivative Compute Question 0005. \\ \fi
\begin{problem}

Compute the following limit definition of derivative.
\[\lim_{h\to0}\frac{-5 \, {\left(h + 3\right)}^{2} + 45}{h}=\answer{-30}\]
What function is being differentiated? (Assume there is no horizontal translation)
\[f(x)=\answer{-5 \, x^{2}}\]
At what $x-$value are you computing the derivative (given your previous answer)?
\[x=\answer{3}\]

\input{Derivative-Compute-0005.HELP.tex}


\end{problem}}

%%%%%%%%%%%%%%%%%%%%%%

\latexProblemContent{
\ifVerboseLocation This is Derivative Compute Question 0005. \\ \fi
\begin{problem}

Compute the following limit definition of derivative.
\[\lim_{h\to0}\frac{5 \, \sqrt{h + 4} - 10}{h}=\answer{\frac{5}{4}}\]
What function is being differentiated? (Assume there is no horizontal translation)
\[f(x)=\answer{5 \, \sqrt{x}}\]
At what $x-$value are you computing the derivative (given your previous answer)?
\[x=\answer{4}\]

\input{Derivative-Compute-0005.HELP.tex}


\end{problem}}

%%%%%%%%%%%%%%%%%%%%%%

\latexProblemContent{
\ifVerboseLocation This is Derivative Compute Question 0005. \\ \fi
\begin{problem}

Compute the following limit definition of derivative.
\[\lim_{h\to0}\frac{5 \, {\left(h + 3\right)}^{2} - 45}{h}=\answer{30}\]
What function is being differentiated? (Assume there is no horizontal translation)
\[f(x)=\answer{5 \, x^{2}}\]
At what $x-$value are you computing the derivative (given your previous answer)?
\[x=\answer{3}\]

\input{Derivative-Compute-0005.HELP.tex}


\end{problem}}

%%%%%%%%%%%%%%%%%%%%%%

\latexProblemContent{
\ifVerboseLocation This is Derivative Compute Question 0005. \\ \fi
\begin{problem}

Compute the following limit definition of derivative.
\[\lim_{h\to0}\frac{\frac{2}{h + 1} - 2}{h}=\answer{-2}\]
What function is being differentiated? (Assume there is no horizontal translation)
\[f(x)=\answer{\frac{2}{x}}\]
At what $x-$value are you computing the derivative (given your previous answer)?
\[x=\answer{1}\]

\input{Derivative-Compute-0005.HELP.tex}


\end{problem}}

%%%%%%%%%%%%%%%%%%%%%%

\latexProblemContent{
\ifVerboseLocation This is Derivative Compute Question 0005. \\ \fi
\begin{problem}

Compute the following limit definition of derivative.
\[\lim_{h\to0}\frac{-5 \, {\left(h + 4\right)}^{3} + 320}{h}=\answer{-240}\]
What function is being differentiated? (Assume there is no horizontal translation)
\[f(x)=\answer{-5 \, x^{3}}\]
At what $x-$value are you computing the derivative (given your previous answer)?
\[x=\answer{4}\]

\input{Derivative-Compute-0005.HELP.tex}


\end{problem}}

%%%%%%%%%%%%%%%%%%%%%%

\latexProblemContent{
\ifVerboseLocation This is Derivative Compute Question 0005. \\ \fi
\begin{problem}

Compute the following limit definition of derivative.
\[\lim_{h\to0}\frac{{\left(h + 3\right)}^{3} - 27}{h}=\answer{27}\]
What function is being differentiated? (Assume there is no horizontal translation)
\[f(x)=\answer{x^{3}}\]
At what $x-$value are you computing the derivative (given your previous answer)?
\[x=\answer{3}\]

\input{Derivative-Compute-0005.HELP.tex}


\end{problem}}

%%%%%%%%%%%%%%%%%%%%%%

\latexProblemContent{
\ifVerboseLocation This is Derivative Compute Question 0005. \\ \fi
\begin{problem}

Compute the following limit definition of derivative.
\[\lim_{h\to0}\frac{\sqrt{3} - \sqrt{h + 3}}{h}=\answer{-\frac{1}{6} \, \sqrt{3}}\]
What function is being differentiated? (Assume there is no horizontal translation)
\[f(x)=\answer{-\sqrt{x}}\]
At what $x-$value are you computing the derivative (given your previous answer)?
\[x=\answer{3}\]

\input{Derivative-Compute-0005.HELP.tex}


\end{problem}}

%%%%%%%%%%%%%%%%%%%%%%

\latexProblemContent{
\ifVerboseLocation This is Derivative Compute Question 0005. \\ \fi
\begin{problem}

Compute the following limit definition of derivative.
\[\lim_{h\to0}\frac{\frac{1}{h - 2} + \frac{1}{2}}{h}=\answer{-\frac{1}{4}}\]
What function is being differentiated? (Assume there is no horizontal translation)
\[f(x)=\answer{\frac{1}{x}}\]
At what $x-$value are you computing the derivative (given your previous answer)?
\[x=\answer{-2}\]

\input{Derivative-Compute-0005.HELP.tex}


\end{problem}}

%%%%%%%%%%%%%%%%%%%%%%

\latexProblemContent{
\ifVerboseLocation This is Derivative Compute Question 0005. \\ \fi
\begin{problem}

Compute the following limit definition of derivative.
\[\lim_{h\to0}\frac{\frac{1}{h + 4} - \frac{1}{4}}{h}=\answer{-\frac{1}{16}}\]
What function is being differentiated? (Assume there is no horizontal translation)
\[f(x)=\answer{\frac{1}{x}}\]
At what $x-$value are you computing the derivative (given your previous answer)?
\[x=\answer{4}\]

\input{Derivative-Compute-0005.HELP.tex}


\end{problem}}

%%%%%%%%%%%%%%%%%%%%%%

\latexProblemContent{
\ifVerboseLocation This is Derivative Compute Question 0005. \\ \fi
\begin{problem}

Compute the following limit definition of derivative.
\[\lim_{h\to0}\frac{-2 \, {\left(h - 4\right)}^{2} + 32}{h}=\answer{16}\]
What function is being differentiated? (Assume there is no horizontal translation)
\[f(x)=\answer{-2 \, x^{2}}\]
At what $x-$value are you computing the derivative (given your previous answer)?
\[x=\answer{-4}\]

\input{Derivative-Compute-0005.HELP.tex}


\end{problem}}

%%%%%%%%%%%%%%%%%%%%%%

\latexProblemContent{
\ifVerboseLocation This is Derivative Compute Question 0005. \\ \fi
\begin{problem}

Compute the following limit definition of derivative.
\[\lim_{h\to0}\frac{-4 \, {\left(h - 1\right)}^{3} - 4}{h}=\answer{-12}\]
What function is being differentiated? (Assume there is no horizontal translation)
\[f(x)=\answer{-4 \, x^{3}}\]
At what $x-$value are you computing the derivative (given your previous answer)?
\[x=\answer{-1}\]

\input{Derivative-Compute-0005.HELP.tex}


\end{problem}}

%%%%%%%%%%%%%%%%%%%%%%

\latexProblemContent{
\ifVerboseLocation This is Derivative Compute Question 0005. \\ \fi
\begin{problem}

Compute the following limit definition of derivative.
\[\lim_{h\to0}\frac{3 \, {\left(h + 2\right)}^{3} - 24}{h}=\answer{36}\]
What function is being differentiated? (Assume there is no horizontal translation)
\[f(x)=\answer{3 \, x^{3}}\]
At what $x-$value are you computing the derivative (given your previous answer)?
\[x=\answer{2}\]

\input{Derivative-Compute-0005.HELP.tex}


\end{problem}}

%%%%%%%%%%%%%%%%%%%%%%

\latexProblemContent{
\ifVerboseLocation This is Derivative Compute Question 0005. \\ \fi
\begin{problem}

Compute the following limit definition of derivative.
\[\lim_{h\to0}\frac{5 \, {\left(h - 3\right)}^{3} + 135}{h}=\answer{135}\]
What function is being differentiated? (Assume there is no horizontal translation)
\[f(x)=\answer{5 \, x^{3}}\]
At what $x-$value are you computing the derivative (given your previous answer)?
\[x=\answer{-3}\]

\input{Derivative-Compute-0005.HELP.tex}


\end{problem}}

%%%%%%%%%%%%%%%%%%%%%%

\latexProblemContent{
\ifVerboseLocation This is Derivative Compute Question 0005. \\ \fi
\begin{problem}

Compute the following limit definition of derivative.
\[\lim_{h\to0}\frac{-4 \, \sqrt{h + 4} + 8}{h}=\answer{-1}\]
What function is being differentiated? (Assume there is no horizontal translation)
\[f(x)=\answer{-4 \, \sqrt{x}}\]
At what $x-$value are you computing the derivative (given your previous answer)?
\[x=\answer{4}\]

\input{Derivative-Compute-0005.HELP.tex}


\end{problem}}

%%%%%%%%%%%%%%%%%%%%%%

\latexProblemContent{
\ifVerboseLocation This is Derivative Compute Question 0005. \\ \fi
\begin{problem}

Compute the following limit definition of derivative.
\[\lim_{h\to0}\frac{-2 \, {\left(h + 1\right)}^{2} + 2}{h}=\answer{-4}\]
What function is being differentiated? (Assume there is no horizontal translation)
\[f(x)=\answer{-2 \, x^{2}}\]
At what $x-$value are you computing the derivative (given your previous answer)?
\[x=\answer{1}\]

\input{Derivative-Compute-0005.HELP.tex}


\end{problem}}

%%%%%%%%%%%%%%%%%%%%%%

\latexProblemContent{
\ifVerboseLocation This is Derivative Compute Question 0005. \\ \fi
\begin{problem}

Compute the following limit definition of derivative.
\[\lim_{h\to0}\frac{4 \, \sqrt{3} - 4 \, \sqrt{h + 3}}{h}=\answer{-\frac{2}{3} \, \sqrt{3}}\]
What function is being differentiated? (Assume there is no horizontal translation)
\[f(x)=\answer{-4 \, \sqrt{x}}\]
At what $x-$value are you computing the derivative (given your previous answer)?
\[x=\answer{3}\]

\input{Derivative-Compute-0005.HELP.tex}


\end{problem}}

%%%%%%%%%%%%%%%%%%%%%%

\latexProblemContent{
\ifVerboseLocation This is Derivative Compute Question 0005. \\ \fi
\begin{problem}

Compute the following limit definition of derivative.
\[\lim_{h\to0}\frac{\frac{3}{h + 3} - 1}{h}=\answer{-\frac{1}{3}}\]
What function is being differentiated? (Assume there is no horizontal translation)
\[f(x)=\answer{\frac{3}{x}}\]
At what $x-$value are you computing the derivative (given your previous answer)?
\[x=\answer{3}\]

\input{Derivative-Compute-0005.HELP.tex}


\end{problem}}

%%%%%%%%%%%%%%%%%%%%%%

\latexProblemContent{
\ifVerboseLocation This is Derivative Compute Question 0005. \\ \fi
\begin{problem}

Compute the following limit definition of derivative.
\[\lim_{h\to0}\frac{3 \, {\left(h - 1\right)}^{3} + 3}{h}=\answer{9}\]
What function is being differentiated? (Assume there is no horizontal translation)
\[f(x)=\answer{3 \, x^{3}}\]
At what $x-$value are you computing the derivative (given your previous answer)?
\[x=\answer{-1}\]

\input{Derivative-Compute-0005.HELP.tex}


\end{problem}}

%%%%%%%%%%%%%%%%%%%%%%

\latexProblemContent{
\ifVerboseLocation This is Derivative Compute Question 0005. \\ \fi
\begin{problem}

Compute the following limit definition of derivative.
\[\lim_{h\to0}\frac{-5 \, \sqrt{3} + 5 \, \sqrt{h + 3}}{h}=\answer{\frac{5}{6} \, \sqrt{3}}\]
What function is being differentiated? (Assume there is no horizontal translation)
\[f(x)=\answer{5 \, \sqrt{x}}\]
At what $x-$value are you computing the derivative (given your previous answer)?
\[x=\answer{3}\]

\input{Derivative-Compute-0005.HELP.tex}


\end{problem}}

%%%%%%%%%%%%%%%%%%%%%%

\latexProblemContent{
\ifVerboseLocation This is Derivative Compute Question 0005. \\ \fi
\begin{problem}

Compute the following limit definition of derivative.
\[\lim_{h\to0}\frac{4 \, {\left(h + 1\right)}^{2} - 4}{h}=\answer{8}\]
What function is being differentiated? (Assume there is no horizontal translation)
\[f(x)=\answer{4 \, x^{2}}\]
At what $x-$value are you computing the derivative (given your previous answer)?
\[x=\answer{1}\]

\input{Derivative-Compute-0005.HELP.tex}


\end{problem}}

%%%%%%%%%%%%%%%%%%%%%%

\latexProblemContent{
\ifVerboseLocation This is Derivative Compute Question 0005. \\ \fi
\begin{problem}

Compute the following limit definition of derivative.
\[\lim_{h\to0}\frac{-4 \, {\left(h + 2\right)}^{3} + 32}{h}=\answer{-48}\]
What function is being differentiated? (Assume there is no horizontal translation)
\[f(x)=\answer{-4 \, x^{3}}\]
At what $x-$value are you computing the derivative (given your previous answer)?
\[x=\answer{2}\]

\input{Derivative-Compute-0005.HELP.tex}


\end{problem}}

%%%%%%%%%%%%%%%%%%%%%%

\latexProblemContent{
\ifVerboseLocation This is Derivative Compute Question 0005. \\ \fi
\begin{problem}

Compute the following limit definition of derivative.
\[\lim_{h\to0}\frac{-\frac{4}{h + 3} + \frac{4}{3}}{h}=\answer{\frac{4}{9}}\]
What function is being differentiated? (Assume there is no horizontal translation)
\[f(x)=\answer{-\frac{4}{x}}\]
At what $x-$value are you computing the derivative (given your previous answer)?
\[x=\answer{3}\]

\input{Derivative-Compute-0005.HELP.tex}


\end{problem}}

%%%%%%%%%%%%%%%%%%%%%%

\latexProblemContent{
\ifVerboseLocation This is Derivative Compute Question 0005. \\ \fi
\begin{problem}

Compute the following limit definition of derivative.
\[\lim_{h\to0}\frac{-2 \, {\left(h + 3\right)}^{3} + 54}{h}=\answer{-54}\]
What function is being differentiated? (Assume there is no horizontal translation)
\[f(x)=\answer{-2 \, x^{3}}\]
At what $x-$value are you computing the derivative (given your previous answer)?
\[x=\answer{3}\]

\input{Derivative-Compute-0005.HELP.tex}


\end{problem}}

%%%%%%%%%%%%%%%%%%%%%%

\latexProblemContent{
\ifVerboseLocation This is Derivative Compute Question 0005. \\ \fi
\begin{problem}

Compute the following limit definition of derivative.
\[\lim_{h\to0}\frac{-2 \, {\left(h - 3\right)}^{3} - 54}{h}=\answer{-54}\]
What function is being differentiated? (Assume there is no horizontal translation)
\[f(x)=\answer{-2 \, x^{3}}\]
At what $x-$value are you computing the derivative (given your previous answer)?
\[x=\answer{-3}\]

\input{Derivative-Compute-0005.HELP.tex}


\end{problem}}

%%%%%%%%%%%%%%%%%%%%%%

\latexProblemContent{
\ifVerboseLocation This is Derivative Compute Question 0005. \\ \fi
\begin{problem}

Compute the following limit definition of derivative.
\[\lim_{h\to0}\frac{2 \, {\left(h - 2\right)}^{3} + 16}{h}=\answer{24}\]
What function is being differentiated? (Assume there is no horizontal translation)
\[f(x)=\answer{2 \, x^{3}}\]
At what $x-$value are you computing the derivative (given your previous answer)?
\[x=\answer{-2}\]

\input{Derivative-Compute-0005.HELP.tex}


\end{problem}}

%%%%%%%%%%%%%%%%%%%%%%

\latexProblemContent{
\ifVerboseLocation This is Derivative Compute Question 0005. \\ \fi
\begin{problem}

Compute the following limit definition of derivative.
\[\lim_{h\to0}\frac{\frac{2}{h + 4} - \frac{1}{2}}{h}=\answer{-\frac{1}{8}}\]
What function is being differentiated? (Assume there is no horizontal translation)
\[f(x)=\answer{\frac{2}{x}}\]
At what $x-$value are you computing the derivative (given your previous answer)?
\[x=\answer{4}\]

\input{Derivative-Compute-0005.HELP.tex}


\end{problem}}

%%%%%%%%%%%%%%%%%%%%%%

\latexProblemContent{
\ifVerboseLocation This is Derivative Compute Question 0005. \\ \fi
\begin{problem}

Compute the following limit definition of derivative.
\[\lim_{h\to0}\frac{2 \, {\left(h - 1\right)}^{2} - 2}{h}=\answer{-4}\]
What function is being differentiated? (Assume there is no horizontal translation)
\[f(x)=\answer{2 \, x^{2}}\]
At what $x-$value are you computing the derivative (given your previous answer)?
\[x=\answer{-1}\]

\input{Derivative-Compute-0005.HELP.tex}


\end{problem}}

%%%%%%%%%%%%%%%%%%%%%%

\latexProblemContent{
\ifVerboseLocation This is Derivative Compute Question 0005. \\ \fi
\begin{problem}

Compute the following limit definition of derivative.
\[\lim_{h\to0}\frac{{\left(h - 1\right)}^{2} - 1}{h}=\answer{-2}\]
What function is being differentiated? (Assume there is no horizontal translation)
\[f(x)=\answer{x^{2}}\]
At what $x-$value are you computing the derivative (given your previous answer)?
\[x=\answer{-1}\]

\input{Derivative-Compute-0005.HELP.tex}


\end{problem}}

%%%%%%%%%%%%%%%%%%%%%%

\latexProblemContent{
\ifVerboseLocation This is Derivative Compute Question 0005. \\ \fi
\begin{problem}

Compute the following limit definition of derivative.
\[\lim_{h\to0}\frac{{\left(h + 1\right)}^{2} - 1}{h}=\answer{2}\]
What function is being differentiated? (Assume there is no horizontal translation)
\[f(x)=\answer{x^{2}}\]
At what $x-$value are you computing the derivative (given your previous answer)?
\[x=\answer{1}\]

\input{Derivative-Compute-0005.HELP.tex}


\end{problem}}

%%%%%%%%%%%%%%%%%%%%%%

\latexProblemContent{
\ifVerboseLocation This is Derivative Compute Question 0005. \\ \fi
\begin{problem}

Compute the following limit definition of derivative.
\[\lim_{h\to0}\frac{-2 \, {\left(h + 4\right)}^{3} + 128}{h}=\answer{-96}\]
What function is being differentiated? (Assume there is no horizontal translation)
\[f(x)=\answer{-2 \, x^{3}}\]
At what $x-$value are you computing the derivative (given your previous answer)?
\[x=\answer{4}\]

\input{Derivative-Compute-0005.HELP.tex}


\end{problem}}

%%%%%%%%%%%%%%%%%%%%%%

\latexProblemContent{
\ifVerboseLocation This is Derivative Compute Question 0005. \\ \fi
\begin{problem}

Compute the following limit definition of derivative.
\[\lim_{h\to0}\frac{-\frac{1}{h + 4} + \frac{1}{4}}{h}=\answer{\frac{1}{16}}\]
What function is being differentiated? (Assume there is no horizontal translation)
\[f(x)=\answer{-\frac{1}{x}}\]
At what $x-$value are you computing the derivative (given your previous answer)?
\[x=\answer{4}\]

\input{Derivative-Compute-0005.HELP.tex}


\end{problem}}

%%%%%%%%%%%%%%%%%%%%%%

\latexProblemContent{
\ifVerboseLocation This is Derivative Compute Question 0005. \\ \fi
\begin{problem}

Compute the following limit definition of derivative.
\[\lim_{h\to0}\frac{-4 \, {\left(h + 1\right)}^{3} + 4}{h}=\answer{-12}\]
What function is being differentiated? (Assume there is no horizontal translation)
\[f(x)=\answer{-4 \, x^{3}}\]
At what $x-$value are you computing the derivative (given your previous answer)?
\[x=\answer{1}\]

\input{Derivative-Compute-0005.HELP.tex}


\end{problem}}

%%%%%%%%%%%%%%%%%%%%%%

\latexProblemContent{
\ifVerboseLocation This is Derivative Compute Question 0005. \\ \fi
\begin{problem}

Compute the following limit definition of derivative.
\[\lim_{h\to0}\frac{3 \, \sqrt{h + 1} - 3}{h}=\answer{\frac{3}{2}}\]
What function is being differentiated? (Assume there is no horizontal translation)
\[f(x)=\answer{3 \, \sqrt{x}}\]
At what $x-$value are you computing the derivative (given your previous answer)?
\[x=\answer{1}\]

\input{Derivative-Compute-0005.HELP.tex}


\end{problem}}

%%%%%%%%%%%%%%%%%%%%%%

\latexProblemContent{
\ifVerboseLocation This is Derivative Compute Question 0005. \\ \fi
\begin{problem}

Compute the following limit definition of derivative.
\[\lim_{h\to0}\frac{-2 \, {\left(h + 3\right)}^{2} + 18}{h}=\answer{-12}\]
What function is being differentiated? (Assume there is no horizontal translation)
\[f(x)=\answer{-2 \, x^{2}}\]
At what $x-$value are you computing the derivative (given your previous answer)?
\[x=\answer{3}\]

\input{Derivative-Compute-0005.HELP.tex}


\end{problem}}

%%%%%%%%%%%%%%%%%%%%%%

\latexProblemContent{
\ifVerboseLocation This is Derivative Compute Question 0005. \\ \fi
\begin{problem}

Compute the following limit definition of derivative.
\[\lim_{h\to0}\frac{2 \, {\left(h + 3\right)}^{3} - 54}{h}=\answer{54}\]
What function is being differentiated? (Assume there is no horizontal translation)
\[f(x)=\answer{2 \, x^{3}}\]
At what $x-$value are you computing the derivative (given your previous answer)?
\[x=\answer{3}\]

\input{Derivative-Compute-0005.HELP.tex}


\end{problem}}

%%%%%%%%%%%%%%%%%%%%%%

\latexProblemContent{
\ifVerboseLocation This is Derivative Compute Question 0005. \\ \fi
\begin{problem}

Compute the following limit definition of derivative.
\[\lim_{h\to0}\frac{-3 \, \sqrt{h + 1} + 3}{h}=\answer{-\frac{3}{2}}\]
What function is being differentiated? (Assume there is no horizontal translation)
\[f(x)=\answer{-3 \, \sqrt{x}}\]
At what $x-$value are you computing the derivative (given your previous answer)?
\[x=\answer{1}\]

\input{Derivative-Compute-0005.HELP.tex}


\end{problem}}

%%%%%%%%%%%%%%%%%%%%%%

\latexProblemContent{
\ifVerboseLocation This is Derivative Compute Question 0005. \\ \fi
\begin{problem}

Compute the following limit definition of derivative.
\[\lim_{h\to0}\frac{-3 \, {\left(h + 1\right)}^{2} + 3}{h}=\answer{-6}\]
What function is being differentiated? (Assume there is no horizontal translation)
\[f(x)=\answer{-3 \, x^{2}}\]
At what $x-$value are you computing the derivative (given your previous answer)?
\[x=\answer{1}\]

\input{Derivative-Compute-0005.HELP.tex}


\end{problem}}

%%%%%%%%%%%%%%%%%%%%%%

\latexProblemContent{
\ifVerboseLocation This is Derivative Compute Question 0005. \\ \fi
\begin{problem}

Compute the following limit definition of derivative.
\[\lim_{h\to0}\frac{\sqrt{h + 4} - 2}{h}=\answer{\frac{1}{4}}\]
What function is being differentiated? (Assume there is no horizontal translation)
\[f(x)=\answer{\sqrt{x}}\]
At what $x-$value are you computing the derivative (given your previous answer)?
\[x=\answer{4}\]

\input{Derivative-Compute-0005.HELP.tex}


\end{problem}}

%%%%%%%%%%%%%%%%%%%%%%

\latexProblemContent{
\ifVerboseLocation This is Derivative Compute Question 0005. \\ \fi
\begin{problem}

Compute the following limit definition of derivative.
\[\lim_{h\to0}\frac{-5 \, {\left(h - 1\right)}^{3} - 5}{h}=\answer{-15}\]
What function is being differentiated? (Assume there is no horizontal translation)
\[f(x)=\answer{-5 \, x^{3}}\]
At what $x-$value are you computing the derivative (given your previous answer)?
\[x=\answer{-1}\]

\input{Derivative-Compute-0005.HELP.tex}


\end{problem}}

%%%%%%%%%%%%%%%%%%%%%%

\latexProblemContent{
\ifVerboseLocation This is Derivative Compute Question 0005. \\ \fi
\begin{problem}

Compute the following limit definition of derivative.
\[\lim_{h\to0}\frac{-{\left(h - 4\right)}^{2} + 16}{h}=\answer{8}\]
What function is being differentiated? (Assume there is no horizontal translation)
\[f(x)=\answer{-x^{2}}\]
At what $x-$value are you computing the derivative (given your previous answer)?
\[x=\answer{-4}\]

\input{Derivative-Compute-0005.HELP.tex}


\end{problem}}

%%%%%%%%%%%%%%%%%%%%%%

\latexProblemContent{
\ifVerboseLocation This is Derivative Compute Question 0005. \\ \fi
\begin{problem}

Compute the following limit definition of derivative.
\[\lim_{h\to0}\frac{4 \, {\left(h - 1\right)}^{3} + 4}{h}=\answer{12}\]
What function is being differentiated? (Assume there is no horizontal translation)
\[f(x)=\answer{4 \, x^{3}}\]
At what $x-$value are you computing the derivative (given your previous answer)?
\[x=\answer{-1}\]

\input{Derivative-Compute-0005.HELP.tex}


\end{problem}}

%%%%%%%%%%%%%%%%%%%%%%

\latexProblemContent{
\ifVerboseLocation This is Derivative Compute Question 0005. \\ \fi
\begin{problem}

Compute the following limit definition of derivative.
\[\lim_{h\to0}\frac{-\frac{3}{h - 2} - \frac{3}{2}}{h}=\answer{\frac{3}{4}}\]
What function is being differentiated? (Assume there is no horizontal translation)
\[f(x)=\answer{-\frac{3}{x}}\]
At what $x-$value are you computing the derivative (given your previous answer)?
\[x=\answer{-2}\]

\input{Derivative-Compute-0005.HELP.tex}


\end{problem}}

%%%%%%%%%%%%%%%%%%%%%%

\latexProblemContent{
\ifVerboseLocation This is Derivative Compute Question 0005. \\ \fi
\begin{problem}

Compute the following limit definition of derivative.
\[\lim_{h\to0}\frac{-5 \, {\left(h + 2\right)}^{2} + 20}{h}=\answer{-20}\]
What function is being differentiated? (Assume there is no horizontal translation)
\[f(x)=\answer{-5 \, x^{2}}\]
At what $x-$value are you computing the derivative (given your previous answer)?
\[x=\answer{2}\]

\input{Derivative-Compute-0005.HELP.tex}


\end{problem}}

%%%%%%%%%%%%%%%%%%%%%%

\latexProblemContent{
\ifVerboseLocation This is Derivative Compute Question 0005. \\ \fi
\begin{problem}

Compute the following limit definition of derivative.
\[\lim_{h\to0}\frac{5 \, {\left(h + 4\right)}^{2} - 80}{h}=\answer{40}\]
What function is being differentiated? (Assume there is no horizontal translation)
\[f(x)=\answer{5 \, x^{2}}\]
At what $x-$value are you computing the derivative (given your previous answer)?
\[x=\answer{4}\]

\input{Derivative-Compute-0005.HELP.tex}


\end{problem}}

%%%%%%%%%%%%%%%%%%%%%%

\latexProblemContent{
\ifVerboseLocation This is Derivative Compute Question 0005. \\ \fi
\begin{problem}

Compute the following limit definition of derivative.
\[\lim_{h\to0}\frac{{\left(h + 1\right)}^{3} - 1}{h}=\answer{3}\]
What function is being differentiated? (Assume there is no horizontal translation)
\[f(x)=\answer{x^{3}}\]
At what $x-$value are you computing the derivative (given your previous answer)?
\[x=\answer{1}\]

\input{Derivative-Compute-0005.HELP.tex}


\end{problem}}

%%%%%%%%%%%%%%%%%%%%%%

\latexProblemContent{
\ifVerboseLocation This is Derivative Compute Question 0005. \\ \fi
\begin{problem}

Compute the following limit definition of derivative.
\[\lim_{h\to0}\frac{-{\left(h + 3\right)}^{3} + 27}{h}=\answer{-27}\]
What function is being differentiated? (Assume there is no horizontal translation)
\[f(x)=\answer{-x^{3}}\]
At what $x-$value are you computing the derivative (given your previous answer)?
\[x=\answer{3}\]

\input{Derivative-Compute-0005.HELP.tex}


\end{problem}}

%%%%%%%%%%%%%%%%%%%%%%

\latexProblemContent{
\ifVerboseLocation This is Derivative Compute Question 0005. \\ \fi
\begin{problem}

Compute the following limit definition of derivative.
\[\lim_{h\to0}\frac{{\left(h + 3\right)}^{2} - 9}{h}=\answer{6}\]
What function is being differentiated? (Assume there is no horizontal translation)
\[f(x)=\answer{x^{2}}\]
At what $x-$value are you computing the derivative (given your previous answer)?
\[x=\answer{3}\]

\input{Derivative-Compute-0005.HELP.tex}


\end{problem}}

%%%%%%%%%%%%%%%%%%%%%%

\latexProblemContent{
\ifVerboseLocation This is Derivative Compute Question 0005. \\ \fi
\begin{problem}

Compute the following limit definition of derivative.
\[\lim_{h\to0}\frac{-4 \, {\left(h + 2\right)}^{2} + 16}{h}=\answer{-16}\]
What function is being differentiated? (Assume there is no horizontal translation)
\[f(x)=\answer{-4 \, x^{2}}\]
At what $x-$value are you computing the derivative (given your previous answer)?
\[x=\answer{2}\]

\input{Derivative-Compute-0005.HELP.tex}


\end{problem}}

%%%%%%%%%%%%%%%%%%%%%%

\latexProblemContent{
\ifVerboseLocation This is Derivative Compute Question 0005. \\ \fi
\begin{problem}

Compute the following limit definition of derivative.
\[\lim_{h\to0}\frac{5 \, {\left(h + 3\right)}^{3} - 135}{h}=\answer{135}\]
What function is being differentiated? (Assume there is no horizontal translation)
\[f(x)=\answer{5 \, x^{3}}\]
At what $x-$value are you computing the derivative (given your previous answer)?
\[x=\answer{3}\]

\input{Derivative-Compute-0005.HELP.tex}


\end{problem}}

%%%%%%%%%%%%%%%%%%%%%%

\latexProblemContent{
\ifVerboseLocation This is Derivative Compute Question 0005. \\ \fi
\begin{problem}

Compute the following limit definition of derivative.
\[\lim_{h\to0}\frac{5 \, {\left(h - 2\right)}^{3} + 40}{h}=\answer{60}\]
What function is being differentiated? (Assume there is no horizontal translation)
\[f(x)=\answer{5 \, x^{3}}\]
At what $x-$value are you computing the derivative (given your previous answer)?
\[x=\answer{-2}\]

\input{Derivative-Compute-0005.HELP.tex}


\end{problem}}

%%%%%%%%%%%%%%%%%%%%%%

\latexProblemContent{
\ifVerboseLocation This is Derivative Compute Question 0005. \\ \fi
\begin{problem}

Compute the following limit definition of derivative.
\[\lim_{h\to0}\frac{\frac{1}{h - 4} + \frac{1}{4}}{h}=\answer{-\frac{1}{16}}\]
What function is being differentiated? (Assume there is no horizontal translation)
\[f(x)=\answer{\frac{1}{x}}\]
At what $x-$value are you computing the derivative (given your previous answer)?
\[x=\answer{-4}\]

\input{Derivative-Compute-0005.HELP.tex}


\end{problem}}

%%%%%%%%%%%%%%%%%%%%%%

\latexProblemContent{
\ifVerboseLocation This is Derivative Compute Question 0005. \\ \fi
\begin{problem}

Compute the following limit definition of derivative.
\[\lim_{h\to0}\frac{-\frac{4}{h + 2} + 2}{h}=\answer{1}\]
What function is being differentiated? (Assume there is no horizontal translation)
\[f(x)=\answer{-\frac{4}{x}}\]
At what $x-$value are you computing the derivative (given your previous answer)?
\[x=\answer{2}\]

\input{Derivative-Compute-0005.HELP.tex}


\end{problem}}

%%%%%%%%%%%%%%%%%%%%%%

\latexProblemContent{
\ifVerboseLocation This is Derivative Compute Question 0005. \\ \fi
\begin{problem}

Compute the following limit definition of derivative.
\[\lim_{h\to0}\frac{-\frac{3}{h - 1} - 3}{h}=\answer{3}\]
What function is being differentiated? (Assume there is no horizontal translation)
\[f(x)=\answer{-\frac{3}{x}}\]
At what $x-$value are you computing the derivative (given your previous answer)?
\[x=\answer{-1}\]

\input{Derivative-Compute-0005.HELP.tex}


\end{problem}}

%%%%%%%%%%%%%%%%%%%%%%

\latexProblemContent{
\ifVerboseLocation This is Derivative Compute Question 0005. \\ \fi
\begin{problem}

Compute the following limit definition of derivative.
\[\lim_{h\to0}\frac{-\frac{4}{h - 3} - \frac{4}{3}}{h}=\answer{\frac{4}{9}}\]
What function is being differentiated? (Assume there is no horizontal translation)
\[f(x)=\answer{-\frac{4}{x}}\]
At what $x-$value are you computing the derivative (given your previous answer)?
\[x=\answer{-3}\]

\input{Derivative-Compute-0005.HELP.tex}


\end{problem}}

%%%%%%%%%%%%%%%%%%%%%%

\latexProblemContent{
\ifVerboseLocation This is Derivative Compute Question 0005. \\ \fi
\begin{problem}

Compute the following limit definition of derivative.
\[\lim_{h\to0}\frac{-4 \, {\left(h + 4\right)}^{2} + 64}{h}=\answer{-32}\]
What function is being differentiated? (Assume there is no horizontal translation)
\[f(x)=\answer{-4 \, x^{2}}\]
At what $x-$value are you computing the derivative (given your previous answer)?
\[x=\answer{4}\]

\input{Derivative-Compute-0005.HELP.tex}


\end{problem}}

%%%%%%%%%%%%%%%%%%%%%%

\latexProblemContent{
\ifVerboseLocation This is Derivative Compute Question 0005. \\ \fi
\begin{problem}

Compute the following limit definition of derivative.
\[\lim_{h\to0}\frac{5 \, \sqrt{3} - 5 \, \sqrt{h + 3}}{h}=\answer{-\frac{5}{6} \, \sqrt{3}}\]
What function is being differentiated? (Assume there is no horizontal translation)
\[f(x)=\answer{-5 \, \sqrt{x}}\]
At what $x-$value are you computing the derivative (given your previous answer)?
\[x=\answer{3}\]

\input{Derivative-Compute-0005.HELP.tex}


\end{problem}}

%%%%%%%%%%%%%%%%%%%%%%

\latexProblemContent{
\ifVerboseLocation This is Derivative Compute Question 0005. \\ \fi
\begin{problem}

Compute the following limit definition of derivative.
\[\lim_{h\to0}\frac{-\frac{3}{h + 4} + \frac{3}{4}}{h}=\answer{\frac{3}{16}}\]
What function is being differentiated? (Assume there is no horizontal translation)
\[f(x)=\answer{-\frac{3}{x}}\]
At what $x-$value are you computing the derivative (given your previous answer)?
\[x=\answer{4}\]

\input{Derivative-Compute-0005.HELP.tex}


\end{problem}}

%%%%%%%%%%%%%%%%%%%%%%

\latexProblemContent{
\ifVerboseLocation This is Derivative Compute Question 0005. \\ \fi
\begin{problem}

Compute the following limit definition of derivative.
\[\lim_{h\to0}\frac{2 \, {\left(h + 4\right)}^{2} - 32}{h}=\answer{16}\]
What function is being differentiated? (Assume there is no horizontal translation)
\[f(x)=\answer{2 \, x^{2}}\]
At what $x-$value are you computing the derivative (given your previous answer)?
\[x=\answer{4}\]

\input{Derivative-Compute-0005.HELP.tex}


\end{problem}}

%%%%%%%%%%%%%%%%%%%%%%

\latexProblemContent{
\ifVerboseLocation This is Derivative Compute Question 0005. \\ \fi
\begin{problem}

Compute the following limit definition of derivative.
\[\lim_{h\to0}\frac{4 \, {\left(h + 2\right)}^{2} - 16}{h}=\answer{16}\]
What function is being differentiated? (Assume there is no horizontal translation)
\[f(x)=\answer{4 \, x^{2}}\]
At what $x-$value are you computing the derivative (given your previous answer)?
\[x=\answer{2}\]

\input{Derivative-Compute-0005.HELP.tex}


\end{problem}}

%%%%%%%%%%%%%%%%%%%%%%

\latexProblemContent{
\ifVerboseLocation This is Derivative Compute Question 0005. \\ \fi
\begin{problem}

Compute the following limit definition of derivative.
\[\lim_{h\to0}\frac{-3 \, \sqrt{h + 4} + 6}{h}=\answer{-\frac{3}{4}}\]
What function is being differentiated? (Assume there is no horizontal translation)
\[f(x)=\answer{-3 \, \sqrt{x}}\]
At what $x-$value are you computing the derivative (given your previous answer)?
\[x=\answer{4}\]

\input{Derivative-Compute-0005.HELP.tex}


\end{problem}}

%%%%%%%%%%%%%%%%%%%%%%

\latexProblemContent{
\ifVerboseLocation This is Derivative Compute Question 0005. \\ \fi
\begin{problem}

Compute the following limit definition of derivative.
\[\lim_{h\to0}\frac{-{\left(h + 2\right)}^{2} + 4}{h}=\answer{-4}\]
What function is being differentiated? (Assume there is no horizontal translation)
\[f(x)=\answer{-x^{2}}\]
At what $x-$value are you computing the derivative (given your previous answer)?
\[x=\answer{2}\]

\input{Derivative-Compute-0005.HELP.tex}


\end{problem}}

%%%%%%%%%%%%%%%%%%%%%%

\latexProblemContent{
\ifVerboseLocation This is Derivative Compute Question 0005. \\ \fi
\begin{problem}

Compute the following limit definition of derivative.
\[\lim_{h\to0}\frac{\frac{3}{h - 3} + 1}{h}=\answer{-\frac{1}{3}}\]
What function is being differentiated? (Assume there is no horizontal translation)
\[f(x)=\answer{\frac{3}{x}}\]
At what $x-$value are you computing the derivative (given your previous answer)?
\[x=\answer{-3}\]

\input{Derivative-Compute-0005.HELP.tex}


\end{problem}}

%%%%%%%%%%%%%%%%%%%%%%

\latexProblemContent{
\ifVerboseLocation This is Derivative Compute Question 0005. \\ \fi
\begin{problem}

Compute the following limit definition of derivative.
\[\lim_{h\to0}\frac{\frac{5}{h + 3} - \frac{5}{3}}{h}=\answer{-\frac{5}{9}}\]
What function is being differentiated? (Assume there is no horizontal translation)
\[f(x)=\answer{\frac{5}{x}}\]
At what $x-$value are you computing the derivative (given your previous answer)?
\[x=\answer{3}\]

\input{Derivative-Compute-0005.HELP.tex}


\end{problem}}

%%%%%%%%%%%%%%%%%%%%%%

\latexProblemContent{
\ifVerboseLocation This is Derivative Compute Question 0005. \\ \fi
\begin{problem}

Compute the following limit definition of derivative.
\[\lim_{h\to0}\frac{-2 \, {\left(h + 4\right)}^{2} + 32}{h}=\answer{-16}\]
What function is being differentiated? (Assume there is no horizontal translation)
\[f(x)=\answer{-2 \, x^{2}}\]
At what $x-$value are you computing the derivative (given your previous answer)?
\[x=\answer{4}\]

\input{Derivative-Compute-0005.HELP.tex}


\end{problem}}

%%%%%%%%%%%%%%%%%%%%%%

\latexProblemContent{
\ifVerboseLocation This is Derivative Compute Question 0005. \\ \fi
\begin{problem}

Compute the following limit definition of derivative.
\[\lim_{h\to0}\frac{-\frac{1}{h + 1} + 1}{h}=\answer{1}\]
What function is being differentiated? (Assume there is no horizontal translation)
\[f(x)=\answer{-\frac{1}{x}}\]
At what $x-$value are you computing the derivative (given your previous answer)?
\[x=\answer{1}\]

\input{Derivative-Compute-0005.HELP.tex}


\end{problem}}

%%%%%%%%%%%%%%%%%%%%%%

\latexProblemContent{
\ifVerboseLocation This is Derivative Compute Question 0005. \\ \fi
\begin{problem}

Compute the following limit definition of derivative.
\[\lim_{h\to0}\frac{\frac{2}{h + 2} - 1}{h}=\answer{-\frac{1}{2}}\]
What function is being differentiated? (Assume there is no horizontal translation)
\[f(x)=\answer{\frac{2}{x}}\]
At what $x-$value are you computing the derivative (given your previous answer)?
\[x=\answer{2}\]

\input{Derivative-Compute-0005.HELP.tex}


\end{problem}}

%%%%%%%%%%%%%%%%%%%%%%

\latexProblemContent{
\ifVerboseLocation This is Derivative Compute Question 0005. \\ \fi
\begin{problem}

Compute the following limit definition of derivative.
\[\lim_{h\to0}\frac{-3 \, {\left(h + 4\right)}^{3} + 192}{h}=\answer{-144}\]
What function is being differentiated? (Assume there is no horizontal translation)
\[f(x)=\answer{-3 \, x^{3}}\]
At what $x-$value are you computing the derivative (given your previous answer)?
\[x=\answer{4}\]

\input{Derivative-Compute-0005.HELP.tex}


\end{problem}}

%%%%%%%%%%%%%%%%%%%%%%

\latexProblemContent{
\ifVerboseLocation This is Derivative Compute Question 0005. \\ \fi
\begin{problem}

Compute the following limit definition of derivative.
\[\lim_{h\to0}\frac{4 \, {\left(h - 4\right)}^{2} - 64}{h}=\answer{-32}\]
What function is being differentiated? (Assume there is no horizontal translation)
\[f(x)=\answer{4 \, x^{2}}\]
At what $x-$value are you computing the derivative (given your previous answer)?
\[x=\answer{-4}\]

\input{Derivative-Compute-0005.HELP.tex}


\end{problem}}

%%%%%%%%%%%%%%%%%%%%%%

\latexProblemContent{
\ifVerboseLocation This is Derivative Compute Question 0005. \\ \fi
\begin{problem}

Compute the following limit definition of derivative.
\[\lim_{h\to0}\frac{3 \, {\left(h + 4\right)}^{3} - 192}{h}=\answer{144}\]
What function is being differentiated? (Assume there is no horizontal translation)
\[f(x)=\answer{3 \, x^{3}}\]
At what $x-$value are you computing the derivative (given your previous answer)?
\[x=\answer{4}\]

\input{Derivative-Compute-0005.HELP.tex}


\end{problem}}

%%%%%%%%%%%%%%%%%%%%%%

\latexProblemContent{
\ifVerboseLocation This is Derivative Compute Question 0005. \\ \fi
\begin{problem}

Compute the following limit definition of derivative.
\[\lim_{h\to0}\frac{-3 \, {\left(h - 4\right)}^{3} - 192}{h}=\answer{-144}\]
What function is being differentiated? (Assume there is no horizontal translation)
\[f(x)=\answer{-3 \, x^{3}}\]
At what $x-$value are you computing the derivative (given your previous answer)?
\[x=\answer{-4}\]

\input{Derivative-Compute-0005.HELP.tex}


\end{problem}}

%%%%%%%%%%%%%%%%%%%%%%

\latexProblemContent{
\ifVerboseLocation This is Derivative Compute Question 0005. \\ \fi
\begin{problem}

Compute the following limit definition of derivative.
\[\lim_{h\to0}\frac{-\frac{5}{h - 3} - \frac{5}{3}}{h}=\answer{\frac{5}{9}}\]
What function is being differentiated? (Assume there is no horizontal translation)
\[f(x)=\answer{-\frac{5}{x}}\]
At what $x-$value are you computing the derivative (given your previous answer)?
\[x=\answer{-3}\]

\input{Derivative-Compute-0005.HELP.tex}


\end{problem}}

%%%%%%%%%%%%%%%%%%%%%%

\latexProblemContent{
\ifVerboseLocation This is Derivative Compute Question 0005. \\ \fi
\begin{problem}

Compute the following limit definition of derivative.
\[\lim_{h\to0}\frac{-\frac{2}{h - 2} - 1}{h}=\answer{\frac{1}{2}}\]
What function is being differentiated? (Assume there is no horizontal translation)
\[f(x)=\answer{-\frac{2}{x}}\]
At what $x-$value are you computing the derivative (given your previous answer)?
\[x=\answer{-2}\]

\input{Derivative-Compute-0005.HELP.tex}


\end{problem}}

%%%%%%%%%%%%%%%%%%%%%%

\latexProblemContent{
\ifVerboseLocation This is Derivative Compute Question 0005. \\ \fi
\begin{problem}

Compute the following limit definition of derivative.
\[\lim_{h\to0}\frac{\frac{5}{h - 3} + \frac{5}{3}}{h}=\answer{-\frac{5}{9}}\]
What function is being differentiated? (Assume there is no horizontal translation)
\[f(x)=\answer{\frac{5}{x}}\]
At what $x-$value are you computing the derivative (given your previous answer)?
\[x=\answer{-3}\]

\input{Derivative-Compute-0005.HELP.tex}


\end{problem}}

%%%%%%%%%%%%%%%%%%%%%%

\latexProblemContent{
\ifVerboseLocation This is Derivative Compute Question 0005. \\ \fi
\begin{problem}

Compute the following limit definition of derivative.
\[\lim_{h\to0}\frac{-\frac{2}{h + 4} + \frac{1}{2}}{h}=\answer{\frac{1}{8}}\]
What function is being differentiated? (Assume there is no horizontal translation)
\[f(x)=\answer{-\frac{2}{x}}\]
At what $x-$value are you computing the derivative (given your previous answer)?
\[x=\answer{4}\]

\input{Derivative-Compute-0005.HELP.tex}


\end{problem}}

%%%%%%%%%%%%%%%%%%%%%%

\latexProblemContent{
\ifVerboseLocation This is Derivative Compute Question 0005. \\ \fi
\begin{problem}

Compute the following limit definition of derivative.
\[\lim_{h\to0}\frac{-\frac{3}{h + 3} + 1}{h}=\answer{\frac{1}{3}}\]
What function is being differentiated? (Assume there is no horizontal translation)
\[f(x)=\answer{-\frac{3}{x}}\]
At what $x-$value are you computing the derivative (given your previous answer)?
\[x=\answer{3}\]

\input{Derivative-Compute-0005.HELP.tex}


\end{problem}}

%%%%%%%%%%%%%%%%%%%%%%

\latexProblemContent{
\ifVerboseLocation This is Derivative Compute Question 0005. \\ \fi
\begin{problem}

Compute the following limit definition of derivative.
\[\lim_{h\to0}\frac{-\frac{1}{h + 3} + \frac{1}{3}}{h}=\answer{\frac{1}{9}}\]
What function is being differentiated? (Assume there is no horizontal translation)
\[f(x)=\answer{-\frac{1}{x}}\]
At what $x-$value are you computing the derivative (given your previous answer)?
\[x=\answer{3}\]

\input{Derivative-Compute-0005.HELP.tex}


\end{problem}}

%%%%%%%%%%%%%%%%%%%%%%

\latexProblemContent{
\ifVerboseLocation This is Derivative Compute Question 0005. \\ \fi
\begin{problem}

Compute the following limit definition of derivative.
\[\lim_{h\to0}\frac{\frac{5}{h - 1} + 5}{h}=\answer{-5}\]
What function is being differentiated? (Assume there is no horizontal translation)
\[f(x)=\answer{\frac{5}{x}}\]
At what $x-$value are you computing the derivative (given your previous answer)?
\[x=\answer{-1}\]

\input{Derivative-Compute-0005.HELP.tex}


\end{problem}}

%%%%%%%%%%%%%%%%%%%%%%

\latexProblemContent{
\ifVerboseLocation This is Derivative Compute Question 0005. \\ \fi
\begin{problem}

Compute the following limit definition of derivative.
\[\lim_{h\to0}\frac{{\left(h - 2\right)}^{3} + 8}{h}=\answer{12}\]
What function is being differentiated? (Assume there is no horizontal translation)
\[f(x)=\answer{x^{3}}\]
At what $x-$value are you computing the derivative (given your previous answer)?
\[x=\answer{-2}\]

\input{Derivative-Compute-0005.HELP.tex}


\end{problem}}

%%%%%%%%%%%%%%%%%%%%%%

\latexProblemContent{
\ifVerboseLocation This is Derivative Compute Question 0005. \\ \fi
\begin{problem}

Compute the following limit definition of derivative.
\[\lim_{h\to0}\frac{-{\left(h + 2\right)}^{3} + 8}{h}=\answer{-12}\]
What function is being differentiated? (Assume there is no horizontal translation)
\[f(x)=\answer{-x^{3}}\]
At what $x-$value are you computing the derivative (given your previous answer)?
\[x=\answer{2}\]

\input{Derivative-Compute-0005.HELP.tex}


\end{problem}}

%%%%%%%%%%%%%%%%%%%%%%

\latexProblemContent{
\ifVerboseLocation This is Derivative Compute Question 0005. \\ \fi
\begin{problem}

Compute the following limit definition of derivative.
\[\lim_{h\to0}\frac{-{\left(h - 2\right)}^{3} - 8}{h}=\answer{-12}\]
What function is being differentiated? (Assume there is no horizontal translation)
\[f(x)=\answer{-x^{3}}\]
At what $x-$value are you computing the derivative (given your previous answer)?
\[x=\answer{-2}\]

\input{Derivative-Compute-0005.HELP.tex}


\end{problem}}

%%%%%%%%%%%%%%%%%%%%%%

\latexProblemContent{
\ifVerboseLocation This is Derivative Compute Question 0005. \\ \fi
\begin{problem}

Compute the following limit definition of derivative.
\[\lim_{h\to0}\frac{{\left(h + 2\right)}^{2} - 4}{h}=\answer{4}\]
What function is being differentiated? (Assume there is no horizontal translation)
\[f(x)=\answer{x^{2}}\]
At what $x-$value are you computing the derivative (given your previous answer)?
\[x=\answer{2}\]

\input{Derivative-Compute-0005.HELP.tex}


\end{problem}}

%%%%%%%%%%%%%%%%%%%%%%

\latexProblemContent{
\ifVerboseLocation This is Derivative Compute Question 0005. \\ \fi
\begin{problem}

Compute the following limit definition of derivative.
\[\lim_{h\to0}\frac{-{\left(h + 1\right)}^{2} + 1}{h}=\answer{-2}\]
What function is being differentiated? (Assume there is no horizontal translation)
\[f(x)=\answer{-x^{2}}\]
At what $x-$value are you computing the derivative (given your previous answer)?
\[x=\answer{1}\]

\input{Derivative-Compute-0005.HELP.tex}


\end{problem}}

%%%%%%%%%%%%%%%%%%%%%%

\latexProblemContent{
\ifVerboseLocation This is Derivative Compute Question 0005. \\ \fi
\begin{problem}

Compute the following limit definition of derivative.
\[\lim_{h\to0}\frac{\frac{5}{h - 4} + \frac{5}{4}}{h}=\answer{-\frac{5}{16}}\]
What function is being differentiated? (Assume there is no horizontal translation)
\[f(x)=\answer{\frac{5}{x}}\]
At what $x-$value are you computing the derivative (given your previous answer)?
\[x=\answer{-4}\]

\input{Derivative-Compute-0005.HELP.tex}


\end{problem}}

%%%%%%%%%%%%%%%%%%%%%%

\latexProblemContent{
\ifVerboseLocation This is Derivative Compute Question 0005. \\ \fi
\begin{problem}

Compute the following limit definition of derivative.
\[\lim_{h\to0}\frac{\frac{1}{h - 3} + \frac{1}{3}}{h}=\answer{-\frac{1}{9}}\]
What function is being differentiated? (Assume there is no horizontal translation)
\[f(x)=\answer{\frac{1}{x}}\]
At what $x-$value are you computing the derivative (given your previous answer)?
\[x=\answer{-3}\]

\input{Derivative-Compute-0005.HELP.tex}


\end{problem}}

%%%%%%%%%%%%%%%%%%%%%%

\latexProblemContent{
\ifVerboseLocation This is Derivative Compute Question 0005. \\ \fi
\begin{problem}

Compute the following limit definition of derivative.
\[\lim_{h\to0}\frac{-4 \, {\left(h - 3\right)}^{3} - 108}{h}=\answer{-108}\]
What function is being differentiated? (Assume there is no horizontal translation)
\[f(x)=\answer{-4 \, x^{3}}\]
At what $x-$value are you computing the derivative (given your previous answer)?
\[x=\answer{-3}\]

\input{Derivative-Compute-0005.HELP.tex}


\end{problem}}

%%%%%%%%%%%%%%%%%%%%%%

\latexProblemContent{
\ifVerboseLocation This is Derivative Compute Question 0005. \\ \fi
\begin{problem}

Compute the following limit definition of derivative.
\[\lim_{h\to0}\frac{5 \, \sqrt{2} - 5 \, \sqrt{h + 2}}{h}=\answer{-\frac{5}{4} \, \sqrt{2}}\]
What function is being differentiated? (Assume there is no horizontal translation)
\[f(x)=\answer{-5 \, \sqrt{x}}\]
At what $x-$value are you computing the derivative (given your previous answer)?
\[x=\answer{2}\]

\input{Derivative-Compute-0005.HELP.tex}


\end{problem}}

%%%%%%%%%%%%%%%%%%%%%%

\latexProblemContent{
\ifVerboseLocation This is Derivative Compute Question 0005. \\ \fi
\begin{problem}

Compute the following limit definition of derivative.
\[\lim_{h\to0}\frac{5 \, {\left(h - 1\right)}^{2} - 5}{h}=\answer{-10}\]
What function is being differentiated? (Assume there is no horizontal translation)
\[f(x)=\answer{5 \, x^{2}}\]
At what $x-$value are you computing the derivative (given your previous answer)?
\[x=\answer{-1}\]

\input{Derivative-Compute-0005.HELP.tex}


\end{problem}}

%%%%%%%%%%%%%%%%%%%%%%

\latexProblemContent{
\ifVerboseLocation This is Derivative Compute Question 0005. \\ \fi
\begin{problem}

Compute the following limit definition of derivative.
\[\lim_{h\to0}\frac{3 \, {\left(h + 1\right)}^{3} - 3}{h}=\answer{9}\]
What function is being differentiated? (Assume there is no horizontal translation)
\[f(x)=\answer{3 \, x^{3}}\]
At what $x-$value are you computing the derivative (given your previous answer)?
\[x=\answer{1}\]

\input{Derivative-Compute-0005.HELP.tex}


\end{problem}}

%%%%%%%%%%%%%%%%%%%%%%

\latexProblemContent{
\ifVerboseLocation This is Derivative Compute Question 0005. \\ \fi
\begin{problem}

Compute the following limit definition of derivative.
\[\lim_{h\to0}\frac{{\left(h - 4\right)}^{2} - 16}{h}=\answer{-8}\]
What function is being differentiated? (Assume there is no horizontal translation)
\[f(x)=\answer{x^{2}}\]
At what $x-$value are you computing the derivative (given your previous answer)?
\[x=\answer{-4}\]

\input{Derivative-Compute-0005.HELP.tex}


\end{problem}}

%%%%%%%%%%%%%%%%%%%%%%

\latexProblemContent{
\ifVerboseLocation This is Derivative Compute Question 0005. \\ \fi
\begin{problem}

Compute the following limit definition of derivative.
\[\lim_{h\to0}\frac{-3 \, {\left(h + 2\right)}^{2} + 12}{h}=\answer{-12}\]
What function is being differentiated? (Assume there is no horizontal translation)
\[f(x)=\answer{-3 \, x^{2}}\]
At what $x-$value are you computing the derivative (given your previous answer)?
\[x=\answer{2}\]

\input{Derivative-Compute-0005.HELP.tex}


\end{problem}}

%%%%%%%%%%%%%%%%%%%%%%

\latexProblemContent{
\ifVerboseLocation This is Derivative Compute Question 0005. \\ \fi
\begin{problem}

Compute the following limit definition of derivative.
\[\lim_{h\to0}\frac{-{\left(h + 1\right)}^{3} + 1}{h}=\answer{-3}\]
What function is being differentiated? (Assume there is no horizontal translation)
\[f(x)=\answer{-x^{3}}\]
At what $x-$value are you computing the derivative (given your previous answer)?
\[x=\answer{1}\]

\input{Derivative-Compute-0005.HELP.tex}


\end{problem}}

%%%%%%%%%%%%%%%%%%%%%%

\latexProblemContent{
\ifVerboseLocation This is Derivative Compute Question 0005. \\ \fi
\begin{problem}

Compute the following limit definition of derivative.
\[\lim_{h\to0}\frac{-\frac{1}{h - 4} - \frac{1}{4}}{h}=\answer{\frac{1}{16}}\]
What function is being differentiated? (Assume there is no horizontal translation)
\[f(x)=\answer{-\frac{1}{x}}\]
At what $x-$value are you computing the derivative (given your previous answer)?
\[x=\answer{-4}\]

\input{Derivative-Compute-0005.HELP.tex}


\end{problem}}

%%%%%%%%%%%%%%%%%%%%%%

\latexProblemContent{
\ifVerboseLocation This is Derivative Compute Question 0005. \\ \fi
\begin{problem}

Compute the following limit definition of derivative.
\[\lim_{h\to0}\frac{3 \, {\left(h - 1\right)}^{2} - 3}{h}=\answer{-6}\]
What function is being differentiated? (Assume there is no horizontal translation)
\[f(x)=\answer{3 \, x^{2}}\]
At what $x-$value are you computing the derivative (given your previous answer)?
\[x=\answer{-1}\]

\input{Derivative-Compute-0005.HELP.tex}


\end{problem}}

%%%%%%%%%%%%%%%%%%%%%%

\latexProblemContent{
\ifVerboseLocation This is Derivative Compute Question 0005. \\ \fi
\begin{problem}

Compute the following limit definition of derivative.
\[\lim_{h\to0}\frac{4 \, {\left(h - 1\right)}^{2} - 4}{h}=\answer{-8}\]
What function is being differentiated? (Assume there is no horizontal translation)
\[f(x)=\answer{4 \, x^{2}}\]
At what $x-$value are you computing the derivative (given your previous answer)?
\[x=\answer{-1}\]

\input{Derivative-Compute-0005.HELP.tex}


\end{problem}}

%%%%%%%%%%%%%%%%%%%%%%

\latexProblemContent{
\ifVerboseLocation This is Derivative Compute Question 0005. \\ \fi
\begin{problem}

Compute the following limit definition of derivative.
\[\lim_{h\to0}\frac{-\frac{4}{h - 2} - 2}{h}=\answer{1}\]
What function is being differentiated? (Assume there is no horizontal translation)
\[f(x)=\answer{-\frac{4}{x}}\]
At what $x-$value are you computing the derivative (given your previous answer)?
\[x=\answer{-2}\]

\input{Derivative-Compute-0005.HELP.tex}


\end{problem}}

%%%%%%%%%%%%%%%%%%%%%%

\latexProblemContent{
\ifVerboseLocation This is Derivative Compute Question 0005. \\ \fi
\begin{problem}

Compute the following limit definition of derivative.
\[\lim_{h\to0}\frac{-5 \, \sqrt{2} + 5 \, \sqrt{h + 2}}{h}=\answer{\frac{5}{4} \, \sqrt{2}}\]
What function is being differentiated? (Assume there is no horizontal translation)
\[f(x)=\answer{5 \, \sqrt{x}}\]
At what $x-$value are you computing the derivative (given your previous answer)?
\[x=\answer{2}\]

\input{Derivative-Compute-0005.HELP.tex}


\end{problem}}

%%%%%%%%%%%%%%%%%%%%%%

\latexProblemContent{
\ifVerboseLocation This is Derivative Compute Question 0005. \\ \fi
\begin{problem}

Compute the following limit definition of derivative.
\[\lim_{h\to0}\frac{\frac{2}{h - 2} + 1}{h}=\answer{-\frac{1}{2}}\]
What function is being differentiated? (Assume there is no horizontal translation)
\[f(x)=\answer{\frac{2}{x}}\]
At what $x-$value are you computing the derivative (given your previous answer)?
\[x=\answer{-2}\]

\input{Derivative-Compute-0005.HELP.tex}


\end{problem}}

%%%%%%%%%%%%%%%%%%%%%%

\latexProblemContent{
\ifVerboseLocation This is Derivative Compute Question 0005. \\ \fi
\begin{problem}

Compute the following limit definition of derivative.
\[\lim_{h\to0}\frac{-\frac{5}{h + 1} + 5}{h}=\answer{5}\]
What function is being differentiated? (Assume there is no horizontal translation)
\[f(x)=\answer{-\frac{5}{x}}\]
At what $x-$value are you computing the derivative (given your previous answer)?
\[x=\answer{1}\]

\input{Derivative-Compute-0005.HELP.tex}


\end{problem}}

%%%%%%%%%%%%%%%%%%%%%%

\latexProblemContent{
\ifVerboseLocation This is Derivative Compute Question 0005. \\ \fi
\begin{problem}

Compute the following limit definition of derivative.
\[\lim_{h\to0}\frac{4 \, {\left(h + 1\right)}^{3} - 4}{h}=\answer{12}\]
What function is being differentiated? (Assume there is no horizontal translation)
\[f(x)=\answer{4 \, x^{3}}\]
At what $x-$value are you computing the derivative (given your previous answer)?
\[x=\answer{1}\]

\input{Derivative-Compute-0005.HELP.tex}


\end{problem}}

%%%%%%%%%%%%%%%%%%%%%%

\latexProblemContent{
\ifVerboseLocation This is Derivative Compute Question 0005. \\ \fi
\begin{problem}

Compute the following limit definition of derivative.
\[\lim_{h\to0}\frac{-\sqrt{h + 1} + 1}{h}=\answer{-\frac{1}{2}}\]
What function is being differentiated? (Assume there is no horizontal translation)
\[f(x)=\answer{-\sqrt{x}}\]
At what $x-$value are you computing the derivative (given your previous answer)?
\[x=\answer{1}\]

\input{Derivative-Compute-0005.HELP.tex}


\end{problem}}

%%%%%%%%%%%%%%%%%%%%%%

\latexProblemContent{
\ifVerboseLocation This is Derivative Compute Question 0005. \\ \fi
\begin{problem}

Compute the following limit definition of derivative.
\[\lim_{h\to0}\frac{2 \, {\left(h + 1\right)}^{2} - 2}{h}=\answer{4}\]
What function is being differentiated? (Assume there is no horizontal translation)
\[f(x)=\answer{2 \, x^{2}}\]
At what $x-$value are you computing the derivative (given your previous answer)?
\[x=\answer{1}\]

\input{Derivative-Compute-0005.HELP.tex}


\end{problem}}

%%%%%%%%%%%%%%%%%%%%%%

\latexProblemContent{
\ifVerboseLocation This is Derivative Compute Question 0005. \\ \fi
\begin{problem}

Compute the following limit definition of derivative.
\[\lim_{h\to0}\frac{2 \, {\left(h - 3\right)}^{3} + 54}{h}=\answer{54}\]
What function is being differentiated? (Assume there is no horizontal translation)
\[f(x)=\answer{2 \, x^{3}}\]
At what $x-$value are you computing the derivative (given your previous answer)?
\[x=\answer{-3}\]

\input{Derivative-Compute-0005.HELP.tex}


\end{problem}}

%%%%%%%%%%%%%%%%%%%%%%

\latexProblemContent{
\ifVerboseLocation This is Derivative Compute Question 0005. \\ \fi
\begin{problem}

Compute the following limit definition of derivative.
\[\lim_{h\to0}\frac{\frac{3}{h - 2} + \frac{3}{2}}{h}=\answer{-\frac{3}{4}}\]
What function is being differentiated? (Assume there is no horizontal translation)
\[f(x)=\answer{\frac{3}{x}}\]
At what $x-$value are you computing the derivative (given your previous answer)?
\[x=\answer{-2}\]

\input{Derivative-Compute-0005.HELP.tex}


\end{problem}}

%%%%%%%%%%%%%%%%%%%%%%

\latexProblemContent{
\ifVerboseLocation This is Derivative Compute Question 0005. \\ \fi
\begin{problem}

Compute the following limit definition of derivative.
\[\lim_{h\to0}\frac{-4 \, {\left(h - 4\right)}^{2} + 64}{h}=\answer{32}\]
What function is being differentiated? (Assume there is no horizontal translation)
\[f(x)=\answer{-4 \, x^{2}}\]
At what $x-$value are you computing the derivative (given your previous answer)?
\[x=\answer{-4}\]

\input{Derivative-Compute-0005.HELP.tex}


\end{problem}}

%%%%%%%%%%%%%%%%%%%%%%

\latexProblemContent{
\ifVerboseLocation This is Derivative Compute Question 0005. \\ \fi
\begin{problem}

Compute the following limit definition of derivative.
\[\lim_{h\to0}\frac{\frac{4}{h + 3} - \frac{4}{3}}{h}=\answer{-\frac{4}{9}}\]
What function is being differentiated? (Assume there is no horizontal translation)
\[f(x)=\answer{\frac{4}{x}}\]
At what $x-$value are you computing the derivative (given your previous answer)?
\[x=\answer{3}\]

\input{Derivative-Compute-0005.HELP.tex}


\end{problem}}

%%%%%%%%%%%%%%%%%%%%%%

\latexProblemContent{
\ifVerboseLocation This is Derivative Compute Question 0005. \\ \fi
\begin{problem}

Compute the following limit definition of derivative.
\[\lim_{h\to0}\frac{-\frac{4}{h - 4} - 1}{h}=\answer{\frac{1}{4}}\]
What function is being differentiated? (Assume there is no horizontal translation)
\[f(x)=\answer{-\frac{4}{x}}\]
At what $x-$value are you computing the derivative (given your previous answer)?
\[x=\answer{-4}\]

\input{Derivative-Compute-0005.HELP.tex}


\end{problem}}

%%%%%%%%%%%%%%%%%%%%%%

\latexProblemContent{
\ifVerboseLocation This is Derivative Compute Question 0005. \\ \fi
\begin{problem}

Compute the following limit definition of derivative.
\[\lim_{h\to0}\frac{-2 \, {\left(h - 3\right)}^{2} + 18}{h}=\answer{12}\]
What function is being differentiated? (Assume there is no horizontal translation)
\[f(x)=\answer{-2 \, x^{2}}\]
At what $x-$value are you computing the derivative (given your previous answer)?
\[x=\answer{-3}\]

\input{Derivative-Compute-0005.HELP.tex}


\end{problem}}

%%%%%%%%%%%%%%%%%%%%%%

\latexProblemContent{
\ifVerboseLocation This is Derivative Compute Question 0005. \\ \fi
\begin{problem}

Compute the following limit definition of derivative.
\[\lim_{h\to0}\frac{-\frac{2}{h + 3} + \frac{2}{3}}{h}=\answer{\frac{2}{9}}\]
What function is being differentiated? (Assume there is no horizontal translation)
\[f(x)=\answer{-\frac{2}{x}}\]
At what $x-$value are you computing the derivative (given your previous answer)?
\[x=\answer{3}\]

\input{Derivative-Compute-0005.HELP.tex}


\end{problem}}

%%%%%%%%%%%%%%%%%%%%%%

\latexProblemContent{
\ifVerboseLocation This is Derivative Compute Question 0005. \\ \fi
\begin{problem}

Compute the following limit definition of derivative.
\[\lim_{h\to0}\frac{\frac{1}{h + 3} - \frac{1}{3}}{h}=\answer{-\frac{1}{9}}\]
What function is being differentiated? (Assume there is no horizontal translation)
\[f(x)=\answer{\frac{1}{x}}\]
At what $x-$value are you computing the derivative (given your previous answer)?
\[x=\answer{3}\]

\input{Derivative-Compute-0005.HELP.tex}


\end{problem}}

%%%%%%%%%%%%%%%%%%%%%%

\latexProblemContent{
\ifVerboseLocation This is Derivative Compute Question 0005. \\ \fi
\begin{problem}

Compute the following limit definition of derivative.
\[\lim_{h\to0}\frac{-\frac{2}{h - 1} - 2}{h}=\answer{2}\]
What function is being differentiated? (Assume there is no horizontal translation)
\[f(x)=\answer{-\frac{2}{x}}\]
At what $x-$value are you computing the derivative (given your previous answer)?
\[x=\answer{-1}\]

\input{Derivative-Compute-0005.HELP.tex}


\end{problem}}

%%%%%%%%%%%%%%%%%%%%%%

\latexProblemContent{
\ifVerboseLocation This is Derivative Compute Question 0005. \\ \fi
\begin{problem}

Compute the following limit definition of derivative.
\[\lim_{h\to0}\frac{\frac{2}{h + 3} - \frac{2}{3}}{h}=\answer{-\frac{2}{9}}\]
What function is being differentiated? (Assume there is no horizontal translation)
\[f(x)=\answer{\frac{2}{x}}\]
At what $x-$value are you computing the derivative (given your previous answer)?
\[x=\answer{3}\]

\input{Derivative-Compute-0005.HELP.tex}


\end{problem}}

%%%%%%%%%%%%%%%%%%%%%%

\latexProblemContent{
\ifVerboseLocation This is Derivative Compute Question 0005. \\ \fi
\begin{problem}

Compute the following limit definition of derivative.
\[\lim_{h\to0}\frac{5 \, {\left(h + 4\right)}^{3} - 320}{h}=\answer{240}\]
What function is being differentiated? (Assume there is no horizontal translation)
\[f(x)=\answer{5 \, x^{3}}\]
At what $x-$value are you computing the derivative (given your previous answer)?
\[x=\answer{4}\]

\input{Derivative-Compute-0005.HELP.tex}


\end{problem}}

%%%%%%%%%%%%%%%%%%%%%%

\latexProblemContent{
\ifVerboseLocation This is Derivative Compute Question 0005. \\ \fi
\begin{problem}

Compute the following limit definition of derivative.
\[\lim_{h\to0}\frac{4 \, {\left(h - 2\right)}^{3} + 32}{h}=\answer{48}\]
What function is being differentiated? (Assume there is no horizontal translation)
\[f(x)=\answer{4 \, x^{3}}\]
At what $x-$value are you computing the derivative (given your previous answer)?
\[x=\answer{-2}\]

\input{Derivative-Compute-0005.HELP.tex}


\end{problem}}

%%%%%%%%%%%%%%%%%%%%%%

\latexProblemContent{
\ifVerboseLocation This is Derivative Compute Question 0005. \\ \fi
\begin{problem}

Compute the following limit definition of derivative.
\[\lim_{h\to0}\frac{-4 \, \sqrt{2} + 4 \, \sqrt{h + 2}}{h}=\answer{\sqrt{2}}\]
What function is being differentiated? (Assume there is no horizontal translation)
\[f(x)=\answer{4 \, \sqrt{x}}\]
At what $x-$value are you computing the derivative (given your previous answer)?
\[x=\answer{2}\]

\input{Derivative-Compute-0005.HELP.tex}


\end{problem}}

%%%%%%%%%%%%%%%%%%%%%%

\latexProblemContent{
\ifVerboseLocation This is Derivative Compute Question 0005. \\ \fi
\begin{problem}

Compute the following limit definition of derivative.
\[\lim_{h\to0}\frac{-4 \, {\left(h + 1\right)}^{2} + 4}{h}=\answer{-8}\]
What function is being differentiated? (Assume there is no horizontal translation)
\[f(x)=\answer{-4 \, x^{2}}\]
At what $x-$value are you computing the derivative (given your previous answer)?
\[x=\answer{1}\]

\input{Derivative-Compute-0005.HELP.tex}


\end{problem}}

%%%%%%%%%%%%%%%%%%%%%%

\latexProblemContent{
\ifVerboseLocation This is Derivative Compute Question 0005. \\ \fi
\begin{problem}

Compute the following limit definition of derivative.
\[\lim_{h\to0}\frac{-\frac{1}{h - 1} - 1}{h}=\answer{1}\]
What function is being differentiated? (Assume there is no horizontal translation)
\[f(x)=\answer{-\frac{1}{x}}\]
At what $x-$value are you computing the derivative (given your previous answer)?
\[x=\answer{-1}\]

\input{Derivative-Compute-0005.HELP.tex}


\end{problem}}

%%%%%%%%%%%%%%%%%%%%%%

\latexProblemContent{
\ifVerboseLocation This is Derivative Compute Question 0005. \\ \fi
\begin{problem}

Compute the following limit definition of derivative.
\[\lim_{h\to0}\frac{-\frac{4}{h + 4} + 1}{h}=\answer{\frac{1}{4}}\]
What function is being differentiated? (Assume there is no horizontal translation)
\[f(x)=\answer{-\frac{4}{x}}\]
At what $x-$value are you computing the derivative (given your previous answer)?
\[x=\answer{4}\]

\input{Derivative-Compute-0005.HELP.tex}


\end{problem}}

%%%%%%%%%%%%%%%%%%%%%%

\latexProblemContent{
\ifVerboseLocation This is Derivative Compute Question 0005. \\ \fi
\begin{problem}

Compute the following limit definition of derivative.
\[\lim_{h\to0}\frac{4 \, {\left(h + 2\right)}^{3} - 32}{h}=\answer{48}\]
What function is being differentiated? (Assume there is no horizontal translation)
\[f(x)=\answer{4 \, x^{3}}\]
At what $x-$value are you computing the derivative (given your previous answer)?
\[x=\answer{2}\]

\input{Derivative-Compute-0005.HELP.tex}


\end{problem}}

%%%%%%%%%%%%%%%%%%%%%%

\latexProblemContent{
\ifVerboseLocation This is Derivative Compute Question 0005. \\ \fi
\begin{problem}

Compute the following limit definition of derivative.
\[\lim_{h\to0}\frac{-5 \, {\left(h - 2\right)}^{3} - 40}{h}=\answer{-60}\]
What function is being differentiated? (Assume there is no horizontal translation)
\[f(x)=\answer{-5 \, x^{3}}\]
At what $x-$value are you computing the derivative (given your previous answer)?
\[x=\answer{-2}\]

\input{Derivative-Compute-0005.HELP.tex}


\end{problem}}

%%%%%%%%%%%%%%%%%%%%%%

\latexProblemContent{
\ifVerboseLocation This is Derivative Compute Question 0005. \\ \fi
\begin{problem}

Compute the following limit definition of derivative.
\[\lim_{h\to0}\frac{-3 \, {\left(h - 1\right)}^{2} + 3}{h}=\answer{6}\]
What function is being differentiated? (Assume there is no horizontal translation)
\[f(x)=\answer{-3 \, x^{2}}\]
At what $x-$value are you computing the derivative (given your previous answer)?
\[x=\answer{-1}\]

\input{Derivative-Compute-0005.HELP.tex}


\end{problem}}

%%%%%%%%%%%%%%%%%%%%%%

\latexProblemContent{
\ifVerboseLocation This is Derivative Compute Question 0005. \\ \fi
\begin{problem}

Compute the following limit definition of derivative.
\[\lim_{h\to0}\frac{{\left(h + 4\right)}^{2} - 16}{h}=\answer{8}\]
What function is being differentiated? (Assume there is no horizontal translation)
\[f(x)=\answer{x^{2}}\]
At what $x-$value are you computing the derivative (given your previous answer)?
\[x=\answer{4}\]

\input{Derivative-Compute-0005.HELP.tex}


\end{problem}}

%%%%%%%%%%%%%%%%%%%%%%

\latexProblemContent{
\ifVerboseLocation This is Derivative Compute Question 0005. \\ \fi
\begin{problem}

Compute the following limit definition of derivative.
\[\lim_{h\to0}\frac{4 \, {\left(h - 3\right)}^{2} - 36}{h}=\answer{-24}\]
What function is being differentiated? (Assume there is no horizontal translation)
\[f(x)=\answer{4 \, x^{2}}\]
At what $x-$value are you computing the derivative (given your previous answer)?
\[x=\answer{-3}\]

\input{Derivative-Compute-0005.HELP.tex}


\end{problem}}

%%%%%%%%%%%%%%%%%%%%%%

\latexProblemContent{
\ifVerboseLocation This is Derivative Compute Question 0005. \\ \fi
\begin{problem}

Compute the following limit definition of derivative.
\[\lim_{h\to0}\frac{-4 \, {\left(h - 1\right)}^{2} + 4}{h}=\answer{8}\]
What function is being differentiated? (Assume there is no horizontal translation)
\[f(x)=\answer{-4 \, x^{2}}\]
At what $x-$value are you computing the derivative (given your previous answer)?
\[x=\answer{-1}\]

\input{Derivative-Compute-0005.HELP.tex}


\end{problem}}

%%%%%%%%%%%%%%%%%%%%%%

\latexProblemContent{
\ifVerboseLocation This is Derivative Compute Question 0005. \\ \fi
\begin{problem}

Compute the following limit definition of derivative.
\[\lim_{h\to0}\frac{-\frac{1}{h - 3} - \frac{1}{3}}{h}=\answer{\frac{1}{9}}\]
What function is being differentiated? (Assume there is no horizontal translation)
\[f(x)=\answer{-\frac{1}{x}}\]
At what $x-$value are you computing the derivative (given your previous answer)?
\[x=\answer{-3}\]

\input{Derivative-Compute-0005.HELP.tex}


\end{problem}}

%%%%%%%%%%%%%%%%%%%%%%

\latexProblemContent{
\ifVerboseLocation This is Derivative Compute Question 0005. \\ \fi
\begin{problem}

Compute the following limit definition of derivative.
\[\lim_{h\to0}\frac{-5 \, {\left(h - 1\right)}^{2} + 5}{h}=\answer{10}\]
What function is being differentiated? (Assume there is no horizontal translation)
\[f(x)=\answer{-5 \, x^{2}}\]
At what $x-$value are you computing the derivative (given your previous answer)?
\[x=\answer{-1}\]

\input{Derivative-Compute-0005.HELP.tex}


\end{problem}}

%%%%%%%%%%%%%%%%%%%%%%

\latexProblemContent{
\ifVerboseLocation This is Derivative Compute Question 0005. \\ \fi
\begin{problem}

Compute the following limit definition of derivative.
\[\lim_{h\to0}\frac{-3 \, {\left(h + 3\right)}^{2} + 27}{h}=\answer{-18}\]
What function is being differentiated? (Assume there is no horizontal translation)
\[f(x)=\answer{-3 \, x^{2}}\]
At what $x-$value are you computing the derivative (given your previous answer)?
\[x=\answer{3}\]

\input{Derivative-Compute-0005.HELP.tex}


\end{problem}}

%%%%%%%%%%%%%%%%%%%%%%

\latexProblemContent{
\ifVerboseLocation This is Derivative Compute Question 0005. \\ \fi
\begin{problem}

Compute the following limit definition of derivative.
\[\lim_{h\to0}\frac{2 \, {\left(h - 1\right)}^{3} + 2}{h}=\answer{6}\]
What function is being differentiated? (Assume there is no horizontal translation)
\[f(x)=\answer{2 \, x^{3}}\]
At what $x-$value are you computing the derivative (given your previous answer)?
\[x=\answer{-1}\]

\input{Derivative-Compute-0005.HELP.tex}


\end{problem}}\fi             %% end of \ifproblemToFind near top of file
\fi             %% end of \ifquestionCount near top of file
\ProblemFileFooter