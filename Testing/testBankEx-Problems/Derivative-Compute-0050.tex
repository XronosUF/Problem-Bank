% Ans        : ShortAns
% File       : 0050
% Sub        : Trig, LHopital
% Topic      : Derivative
% Type       : Compute

\ProblemFileHeader{500}
\ifquestionPull
\ifproblemToFind\latexProblemContent{
\ifVerboseLocation This is Derivative Compute Question 0050. \\ \fi
\begin{problem}

Find the limit.  Use L'H$\hat{o}$pital's rule where appropriate.

\input{Derivative-Compute-0050.HELP.tex}

\[\lim\limits_{x\to\infty} {\left(\frac{8}{x} + 1\right)}^{4 \, x} + 17=\answer{e^{32} + 17}\]
\end{problem}}

%%%%%%%%%%%%%%%%%%%%%%

\latexProblemContent{
\ifVerboseLocation This is Derivative Compute Question 0050. \\ \fi
\begin{problem}

Find the limit.  Use L'H$\hat{o}$pital's rule where appropriate.

\input{Derivative-Compute-0050.HELP.tex}

\[\lim\limits_{x\to\infty} {\left(-\frac{6}{x} + 1\right)}^{2 \, x} - 1=\answer{e^{\left(-12\right)} - 1}\]
\end{problem}}

%%%%%%%%%%%%%%%%%%%%%%

\latexProblemContent{
\ifVerboseLocation This is Derivative Compute Question 0050. \\ \fi
\begin{problem}

Find the limit.  Use L'H$\hat{o}$pital's rule where appropriate.

\input{Derivative-Compute-0050.HELP.tex}

\[\lim\limits_{x\to\infty} {\left(\frac{8}{x} + 1\right)}^{6 \, x} + 12=\answer{e^{48} + 12}\]
\end{problem}}

%%%%%%%%%%%%%%%%%%%%%%

\latexProblemContent{
\ifVerboseLocation This is Derivative Compute Question 0050. \\ \fi
\begin{problem}

Find the limit.  Use L'H$\hat{o}$pital's rule where appropriate.

\input{Derivative-Compute-0050.HELP.tex}

\[\lim\limits_{x\to\infty} {\left(\frac{8}{x} + 1\right)}^{-10 \, x} + 2=\answer{e^{\left(-80\right)} + 2}\]
\end{problem}}

%%%%%%%%%%%%%%%%%%%%%%

\latexProblemContent{
\ifVerboseLocation This is Derivative Compute Question 0050. \\ \fi
\begin{problem}

Find the limit.  Use L'H$\hat{o}$pital's rule where appropriate.

\input{Derivative-Compute-0050.HELP.tex}

\[\lim\limits_{x\to\infty} {\left(-\frac{1}{x} + 1\right)}^{-x} + 16=\answer{e + 16}\]
\end{problem}}

%%%%%%%%%%%%%%%%%%%%%%

\latexProblemContent{
\ifVerboseLocation This is Derivative Compute Question 0050. \\ \fi
\begin{problem}

Find the limit.  Use L'H$\hat{o}$pital's rule where appropriate.

\input{Derivative-Compute-0050.HELP.tex}

\[\lim\limits_{x\to\infty} {\left(\frac{9}{x} + 1\right)}^{2 \, x} - 2=\answer{e^{18} - 2}\]
\end{problem}}

%%%%%%%%%%%%%%%%%%%%%%

\latexProblemContent{
\ifVerboseLocation This is Derivative Compute Question 0050. \\ \fi
\begin{problem}

Find the limit.  Use L'H$\hat{o}$pital's rule where appropriate.

\input{Derivative-Compute-0050.HELP.tex}

\[\lim\limits_{x\to\infty} {\left(-\frac{4}{x} + 1\right)}^{-10 \, x} + 6=\answer{e^{40} + 6}\]
\end{problem}}

%%%%%%%%%%%%%%%%%%%%%%

\latexProblemContent{
\ifVerboseLocation This is Derivative Compute Question 0050. \\ \fi
\begin{problem}

Find the limit.  Use L'H$\hat{o}$pital's rule where appropriate.

\input{Derivative-Compute-0050.HELP.tex}

\[\lim\limits_{x\to\infty} {\left(-\frac{7}{x} + 1\right)}^{5 \, x} - 19=\answer{e^{\left(-35\right)} - 19}\]
\end{problem}}

%%%%%%%%%%%%%%%%%%%%%%

\latexProblemContent{
\ifVerboseLocation This is Derivative Compute Question 0050. \\ \fi
\begin{problem}

Find the limit.  Use L'H$\hat{o}$pital's rule where appropriate.

\input{Derivative-Compute-0050.HELP.tex}

\[\lim\limits_{x\to\infty} {\left(-\frac{6}{x} + 1\right)}^{4 \, x} + 10=\answer{e^{\left(-24\right)} + 10}\]
\end{problem}}

%%%%%%%%%%%%%%%%%%%%%%

\latexProblemContent{
\ifVerboseLocation This is Derivative Compute Question 0050. \\ \fi
\begin{problem}

Find the limit.  Use L'H$\hat{o}$pital's rule where appropriate.

\input{Derivative-Compute-0050.HELP.tex}

\[\lim\limits_{x\to\infty} {\left(-\frac{5}{x} + 1\right)}^{-7 \, x} - 7=\answer{e^{35} - 7}\]
\end{problem}}

%%%%%%%%%%%%%%%%%%%%%%

\latexProblemContent{
\ifVerboseLocation This is Derivative Compute Question 0050. \\ \fi
\begin{problem}

Find the limit.  Use L'H$\hat{o}$pital's rule where appropriate.

\input{Derivative-Compute-0050.HELP.tex}

\[\lim\limits_{x\to\infty} {\left(-\frac{2}{x} + 1\right)}^{-3 \, x} - 16=\answer{e^{6} - 16}\]
\end{problem}}

%%%%%%%%%%%%%%%%%%%%%%

\latexProblemContent{
\ifVerboseLocation This is Derivative Compute Question 0050. \\ \fi
\begin{problem}

Find the limit.  Use L'H$\hat{o}$pital's rule where appropriate.

\input{Derivative-Compute-0050.HELP.tex}

\[\lim\limits_{x\to\infty} {\left(-\frac{7}{x} + 1\right)}^{4 \, x} - 9=\answer{e^{\left(-28\right)} - 9}\]
\end{problem}}

%%%%%%%%%%%%%%%%%%%%%%

\latexProblemContent{
\ifVerboseLocation This is Derivative Compute Question 0050. \\ \fi
\begin{problem}

Find the limit.  Use L'H$\hat{o}$pital's rule where appropriate.

\input{Derivative-Compute-0050.HELP.tex}

\[\lim\limits_{x\to\infty} {\left(-\frac{5}{x} + 1\right)}^{6 \, x} - 12=\answer{e^{\left(-30\right)} - 12}\]
\end{problem}}

%%%%%%%%%%%%%%%%%%%%%%

\latexProblemContent{
\ifVerboseLocation This is Derivative Compute Question 0050. \\ \fi
\begin{problem}

Find the limit.  Use L'H$\hat{o}$pital's rule where appropriate.

\input{Derivative-Compute-0050.HELP.tex}

\[\lim\limits_{x\to\infty} {\left(-\frac{3}{x} + 1\right)}^{10 \, x} - 19=\answer{e^{\left(-30\right)} - 19}\]
\end{problem}}

%%%%%%%%%%%%%%%%%%%%%%

\latexProblemContent{
\ifVerboseLocation This is Derivative Compute Question 0050. \\ \fi
\begin{problem}

Find the limit.  Use L'H$\hat{o}$pital's rule where appropriate.

\input{Derivative-Compute-0050.HELP.tex}

\[\lim\limits_{x\to\infty} {\left(-\frac{3}{x} + 1\right)}^{-5 \, x} - 6=\answer{e^{15} - 6}\]
\end{problem}}

%%%%%%%%%%%%%%%%%%%%%%

\latexProblemContent{
\ifVerboseLocation This is Derivative Compute Question 0050. \\ \fi
\begin{problem}

Find the limit.  Use L'H$\hat{o}$pital's rule where appropriate.

\input{Derivative-Compute-0050.HELP.tex}

\[\lim\limits_{x\to\infty} {\left(-\frac{9}{x} + 1\right)}^{4 \, x} + 1=\answer{e^{\left(-36\right)} + 1}\]
\end{problem}}

%%%%%%%%%%%%%%%%%%%%%%

\latexProblemContent{
\ifVerboseLocation This is Derivative Compute Question 0050. \\ \fi
\begin{problem}

Find the limit.  Use L'H$\hat{o}$pital's rule where appropriate.

\input{Derivative-Compute-0050.HELP.tex}

\[\lim\limits_{x\to\infty} {\left(-\frac{5}{x} + 1\right)}^{-6 \, x} + 10=\answer{e^{30} + 10}\]
\end{problem}}

%%%%%%%%%%%%%%%%%%%%%%

\latexProblemContent{
\ifVerboseLocation This is Derivative Compute Question 0050. \\ \fi
\begin{problem}

Find the limit.  Use L'H$\hat{o}$pital's rule where appropriate.

\input{Derivative-Compute-0050.HELP.tex}

\[\lim\limits_{x\to\infty} {\left(-\frac{9}{x} + 1\right)}^{-10 \, x} + 3=\answer{e^{90} + 3}\]
\end{problem}}

%%%%%%%%%%%%%%%%%%%%%%

\latexProblemContent{
\ifVerboseLocation This is Derivative Compute Question 0050. \\ \fi
\begin{problem}

Find the limit.  Use L'H$\hat{o}$pital's rule where appropriate.

\input{Derivative-Compute-0050.HELP.tex}

\[\lim\limits_{x\to\infty} {\left(-\frac{4}{x} + 1\right)}^{-3 \, x} - 10=\answer{e^{12} - 10}\]
\end{problem}}

%%%%%%%%%%%%%%%%%%%%%%

\latexProblemContent{
\ifVerboseLocation This is Derivative Compute Question 0050. \\ \fi
\begin{problem}

Find the limit.  Use L'H$\hat{o}$pital's rule where appropriate.

\input{Derivative-Compute-0050.HELP.tex}

\[\lim\limits_{x\to\infty} {\left(-\frac{5}{x} + 1\right)}^{-7 \, x} + 11=\answer{e^{35} + 11}\]
\end{problem}}

%%%%%%%%%%%%%%%%%%%%%%

\latexProblemContent{
\ifVerboseLocation This is Derivative Compute Question 0050. \\ \fi
\begin{problem}

Find the limit.  Use L'H$\hat{o}$pital's rule where appropriate.

\input{Derivative-Compute-0050.HELP.tex}

\[\lim\limits_{x\to\infty} {\left(\frac{7}{x} + 1\right)}^{-8 \, x} + 16=\answer{e^{\left(-56\right)} + 16}\]
\end{problem}}

%%%%%%%%%%%%%%%%%%%%%%

\latexProblemContent{
\ifVerboseLocation This is Derivative Compute Question 0050. \\ \fi
\begin{problem}

Find the limit.  Use L'H$\hat{o}$pital's rule where appropriate.

\input{Derivative-Compute-0050.HELP.tex}

\[\lim\limits_{x\to\infty} {\left(-\frac{10}{x} + 1\right)}^{7 \, x} + 14=\answer{e^{\left(-70\right)} + 14}\]
\end{problem}}

%%%%%%%%%%%%%%%%%%%%%%

\latexProblemContent{
\ifVerboseLocation This is Derivative Compute Question 0050. \\ \fi
\begin{problem}

Find the limit.  Use L'H$\hat{o}$pital's rule where appropriate.

\input{Derivative-Compute-0050.HELP.tex}

\[\lim\limits_{x\to\infty} {\left(-\frac{7}{x} + 1\right)}^{6 \, x} - 3=\answer{e^{\left(-42\right)} - 3}\]
\end{problem}}

%%%%%%%%%%%%%%%%%%%%%%

\latexProblemContent{
\ifVerboseLocation This is Derivative Compute Question 0050. \\ \fi
\begin{problem}

Find the limit.  Use L'H$\hat{o}$pital's rule where appropriate.

\input{Derivative-Compute-0050.HELP.tex}

\[\lim\limits_{x\to\infty} {\left(\frac{5}{x} + 1\right)}^{-7 \, x} + 4=\answer{e^{\left(-35\right)} + 4}\]
\end{problem}}

%%%%%%%%%%%%%%%%%%%%%%

\latexProblemContent{
\ifVerboseLocation This is Derivative Compute Question 0050. \\ \fi
\begin{problem}

Find the limit.  Use L'H$\hat{o}$pital's rule where appropriate.

\input{Derivative-Compute-0050.HELP.tex}

\[\lim\limits_{x\to\infty} {\left(-\frac{9}{x} + 1\right)}^{-10 \, x} + 5=\answer{e^{90} + 5}\]
\end{problem}}

%%%%%%%%%%%%%%%%%%%%%%

\latexProblemContent{
\ifVerboseLocation This is Derivative Compute Question 0050. \\ \fi
\begin{problem}

Find the limit.  Use L'H$\hat{o}$pital's rule where appropriate.

\input{Derivative-Compute-0050.HELP.tex}

\[\lim\limits_{x\to\infty} {\left(-\frac{7}{x} + 1\right)}^{10 \, x} - 20=\answer{e^{\left(-70\right)} - 20}\]
\end{problem}}

%%%%%%%%%%%%%%%%%%%%%%

\latexProblemContent{
\ifVerboseLocation This is Derivative Compute Question 0050. \\ \fi
\begin{problem}

Find the limit.  Use L'H$\hat{o}$pital's rule where appropriate.

\input{Derivative-Compute-0050.HELP.tex}

\[\lim\limits_{x\to\infty} {\left(\frac{2}{x} + 1\right)}^{7 \, x} + 4=\answer{e^{14} + 4}\]
\end{problem}}

%%%%%%%%%%%%%%%%%%%%%%

\latexProblemContent{
\ifVerboseLocation This is Derivative Compute Question 0050. \\ \fi
\begin{problem}

Find the limit.  Use L'H$\hat{o}$pital's rule where appropriate.

\input{Derivative-Compute-0050.HELP.tex}

\[\lim\limits_{x\to\infty} {\left(\frac{7}{x} + 1\right)}^{-x} + 12=\answer{e^{\left(-7\right)} + 12}\]
\end{problem}}

%%%%%%%%%%%%%%%%%%%%%%

\latexProblemContent{
\ifVerboseLocation This is Derivative Compute Question 0050. \\ \fi
\begin{problem}

Find the limit.  Use L'H$\hat{o}$pital's rule where appropriate.

\input{Derivative-Compute-0050.HELP.tex}

\[\lim\limits_{x\to\infty} {\left(\frac{10}{x} + 1\right)}^{x} + 12=\answer{e^{10} + 12}\]
\end{problem}}

%%%%%%%%%%%%%%%%%%%%%%

\latexProblemContent{
\ifVerboseLocation This is Derivative Compute Question 0050. \\ \fi
\begin{problem}

Find the limit.  Use L'H$\hat{o}$pital's rule where appropriate.

\input{Derivative-Compute-0050.HELP.tex}

\[\lim\limits_{x\to\infty} {\left(\frac{4}{x} + 1\right)}^{-3 \, x} - 6=\answer{e^{\left(-12\right)} - 6}\]
\end{problem}}

%%%%%%%%%%%%%%%%%%%%%%

\latexProblemContent{
\ifVerboseLocation This is Derivative Compute Question 0050. \\ \fi
\begin{problem}

Find the limit.  Use L'H$\hat{o}$pital's rule where appropriate.

\input{Derivative-Compute-0050.HELP.tex}

\[\lim\limits_{x\to\infty} {\left(\frac{6}{x} + 1\right)}^{10 \, x} - 13=\answer{e^{60} - 13}\]
\end{problem}}

%%%%%%%%%%%%%%%%%%%%%%

\latexProblemContent{
\ifVerboseLocation This is Derivative Compute Question 0050. \\ \fi
\begin{problem}

Find the limit.  Use L'H$\hat{o}$pital's rule where appropriate.

\input{Derivative-Compute-0050.HELP.tex}

\[\lim\limits_{x\to\infty} {\left(\frac{7}{x} + 1\right)}^{4 \, x} - 3=\answer{e^{28} - 3}\]
\end{problem}}

%%%%%%%%%%%%%%%%%%%%%%

\latexProblemContent{
\ifVerboseLocation This is Derivative Compute Question 0050. \\ \fi
\begin{problem}

Find the limit.  Use L'H$\hat{o}$pital's rule where appropriate.

\input{Derivative-Compute-0050.HELP.tex}

\[\lim\limits_{x\to\infty} {\left(-\frac{2}{x} + 1\right)}^{-x} - 17=\answer{e^{2} - 17}\]
\end{problem}}

%%%%%%%%%%%%%%%%%%%%%%

\latexProblemContent{
\ifVerboseLocation This is Derivative Compute Question 0050. \\ \fi
\begin{problem}

Find the limit.  Use L'H$\hat{o}$pital's rule where appropriate.

\input{Derivative-Compute-0050.HELP.tex}

\[\lim\limits_{x\to\infty} {\left(\frac{6}{x} + 1\right)}^{-4 \, x} - 19=\answer{e^{\left(-24\right)} - 19}\]
\end{problem}}

%%%%%%%%%%%%%%%%%%%%%%

\latexProblemContent{
\ifVerboseLocation This is Derivative Compute Question 0050. \\ \fi
\begin{problem}

Find the limit.  Use L'H$\hat{o}$pital's rule where appropriate.

\input{Derivative-Compute-0050.HELP.tex}

\[\lim\limits_{x\to\infty} {\left(-\frac{3}{x} + 1\right)}^{-7 \, x} - 7=\answer{e^{21} - 7}\]
\end{problem}}

%%%%%%%%%%%%%%%%%%%%%%

\latexProblemContent{
\ifVerboseLocation This is Derivative Compute Question 0050. \\ \fi
\begin{problem}

Find the limit.  Use L'H$\hat{o}$pital's rule where appropriate.

\input{Derivative-Compute-0050.HELP.tex}

\[\lim\limits_{x\to\infty} {\left(\frac{6}{x} + 1\right)}^{5 \, x} - 20=\answer{e^{30} - 20}\]
\end{problem}}

%%%%%%%%%%%%%%%%%%%%%%

\latexProblemContent{
\ifVerboseLocation This is Derivative Compute Question 0050. \\ \fi
\begin{problem}

Find the limit.  Use L'H$\hat{o}$pital's rule where appropriate.

\input{Derivative-Compute-0050.HELP.tex}

\[\lim\limits_{x\to\infty} {\left(-\frac{6}{x} + 1\right)}^{-7 \, x} - 20=\answer{e^{42} - 20}\]
\end{problem}}

%%%%%%%%%%%%%%%%%%%%%%

\latexProblemContent{
\ifVerboseLocation This is Derivative Compute Question 0050. \\ \fi
\begin{problem}

Find the limit.  Use L'H$\hat{o}$pital's rule where appropriate.

\input{Derivative-Compute-0050.HELP.tex}

\[\lim\limits_{x\to\infty} {\left(\frac{1}{x} + 1\right)}^{x} + 20=\answer{e + 20}\]
\end{problem}}

%%%%%%%%%%%%%%%%%%%%%%

\latexProblemContent{
\ifVerboseLocation This is Derivative Compute Question 0050. \\ \fi
\begin{problem}

Find the limit.  Use L'H$\hat{o}$pital's rule where appropriate.

\input{Derivative-Compute-0050.HELP.tex}

\[\lim\limits_{x\to\infty} {\left(-\frac{2}{x} + 1\right)}^{-9 \, x} - 12=\answer{e^{18} - 12}\]
\end{problem}}

%%%%%%%%%%%%%%%%%%%%%%

\latexProblemContent{
\ifVerboseLocation This is Derivative Compute Question 0050. \\ \fi
\begin{problem}

Find the limit.  Use L'H$\hat{o}$pital's rule where appropriate.

\input{Derivative-Compute-0050.HELP.tex}

\[\lim\limits_{x\to\infty} {\left(\frac{6}{x} + 1\right)}^{x} + 14=\answer{e^{6} + 14}\]
\end{problem}}

%%%%%%%%%%%%%%%%%%%%%%

\latexProblemContent{
\ifVerboseLocation This is Derivative Compute Question 0050. \\ \fi
\begin{problem}

Find the limit.  Use L'H$\hat{o}$pital's rule where appropriate.

\input{Derivative-Compute-0050.HELP.tex}

\[\lim\limits_{x\to\infty} {\left(-\frac{10}{x} + 1\right)}^{-4 \, x} - 18=\answer{e^{40} - 18}\]
\end{problem}}

%%%%%%%%%%%%%%%%%%%%%%

\latexProblemContent{
\ifVerboseLocation This is Derivative Compute Question 0050. \\ \fi
\begin{problem}

Find the limit.  Use L'H$\hat{o}$pital's rule where appropriate.

\input{Derivative-Compute-0050.HELP.tex}

\[\lim\limits_{x\to\infty} {\left(\frac{4}{x} + 1\right)}^{7 \, x} + 8=\answer{e^{28} + 8}\]
\end{problem}}

%%%%%%%%%%%%%%%%%%%%%%

\latexProblemContent{
\ifVerboseLocation This is Derivative Compute Question 0050. \\ \fi
\begin{problem}

Find the limit.  Use L'H$\hat{o}$pital's rule where appropriate.

\input{Derivative-Compute-0050.HELP.tex}

\[\lim\limits_{x\to\infty} {\left(-\frac{2}{x} + 1\right)}^{-5 \, x} - 4=\answer{e^{10} - 4}\]
\end{problem}}

%%%%%%%%%%%%%%%%%%%%%%

\latexProblemContent{
\ifVerboseLocation This is Derivative Compute Question 0050. \\ \fi
\begin{problem}

Find the limit.  Use L'H$\hat{o}$pital's rule where appropriate.

\input{Derivative-Compute-0050.HELP.tex}

\[\lim\limits_{x\to\infty} {\left(-\frac{1}{x} + 1\right)}^{2 \, x} + 17=\answer{e^{\left(-2\right)} + 17}\]
\end{problem}}

%%%%%%%%%%%%%%%%%%%%%%

\latexProblemContent{
\ifVerboseLocation This is Derivative Compute Question 0050. \\ \fi
\begin{problem}

Find the limit.  Use L'H$\hat{o}$pital's rule where appropriate.

\input{Derivative-Compute-0050.HELP.tex}

\[\lim\limits_{x\to\infty} {\left(-\frac{2}{x} + 1\right)}^{-3 \, x} - 14=\answer{e^{6} - 14}\]
\end{problem}}

%%%%%%%%%%%%%%%%%%%%%%

\latexProblemContent{
\ifVerboseLocation This is Derivative Compute Question 0050. \\ \fi
\begin{problem}

Find the limit.  Use L'H$\hat{o}$pital's rule where appropriate.

\input{Derivative-Compute-0050.HELP.tex}

\[\lim\limits_{x\to\infty} {\left(-\frac{8}{x} + 1\right)}^{5 \, x} - 2=\answer{e^{\left(-40\right)} - 2}\]
\end{problem}}

%%%%%%%%%%%%%%%%%%%%%%

\latexProblemContent{
\ifVerboseLocation This is Derivative Compute Question 0050. \\ \fi
\begin{problem}

Find the limit.  Use L'H$\hat{o}$pital's rule where appropriate.

\input{Derivative-Compute-0050.HELP.tex}

\[\lim\limits_{x\to\infty} {\left(-\frac{1}{x} + 1\right)}^{-10 \, x} - 5=\answer{e^{10} - 5}\]
\end{problem}}

%%%%%%%%%%%%%%%%%%%%%%

\latexProblemContent{
\ifVerboseLocation This is Derivative Compute Question 0050. \\ \fi
\begin{problem}

Find the limit.  Use L'H$\hat{o}$pital's rule where appropriate.

\input{Derivative-Compute-0050.HELP.tex}

\[\lim\limits_{x\to\infty} {\left(-\frac{2}{x} + 1\right)}^{-5 \, x} + 17=\answer{e^{10} + 17}\]
\end{problem}}

%%%%%%%%%%%%%%%%%%%%%%

\latexProblemContent{
\ifVerboseLocation This is Derivative Compute Question 0050. \\ \fi
\begin{problem}

Find the limit.  Use L'H$\hat{o}$pital's rule where appropriate.

\input{Derivative-Compute-0050.HELP.tex}

\[\lim\limits_{x\to\infty} {\left(-\frac{1}{x} + 1\right)}^{8 \, x} + 20=\answer{e^{\left(-8\right)} + 20}\]
\end{problem}}

%%%%%%%%%%%%%%%%%%%%%%

\latexProblemContent{
\ifVerboseLocation This is Derivative Compute Question 0050. \\ \fi
\begin{problem}

Find the limit.  Use L'H$\hat{o}$pital's rule where appropriate.

\input{Derivative-Compute-0050.HELP.tex}

\[\lim\limits_{x\to\infty} {\left(\frac{3}{x} + 1\right)}^{-6 \, x} + 19=\answer{e^{\left(-18\right)} + 19}\]
\end{problem}}

%%%%%%%%%%%%%%%%%%%%%%

\latexProblemContent{
\ifVerboseLocation This is Derivative Compute Question 0050. \\ \fi
\begin{problem}

Find the limit.  Use L'H$\hat{o}$pital's rule where appropriate.

\input{Derivative-Compute-0050.HELP.tex}

\[\lim\limits_{x\to\infty} {\left(\frac{7}{x} + 1\right)}^{-2 \, x} - 7=\answer{e^{\left(-14\right)} - 7}\]
\end{problem}}

%%%%%%%%%%%%%%%%%%%%%%

\latexProblemContent{
\ifVerboseLocation This is Derivative Compute Question 0050. \\ \fi
\begin{problem}

Find the limit.  Use L'H$\hat{o}$pital's rule where appropriate.

\input{Derivative-Compute-0050.HELP.tex}

\[\lim\limits_{x\to\infty} {\left(\frac{2}{x} + 1\right)}^{-5 \, x} - 16=\answer{e^{\left(-10\right)} - 16}\]
\end{problem}}

%%%%%%%%%%%%%%%%%%%%%%

\latexProblemContent{
\ifVerboseLocation This is Derivative Compute Question 0050. \\ \fi
\begin{problem}

Find the limit.  Use L'H$\hat{o}$pital's rule where appropriate.

\input{Derivative-Compute-0050.HELP.tex}

\[\lim\limits_{x\to\infty} {\left(\frac{6}{x} + 1\right)}^{5 \, x} - 8=\answer{e^{30} - 8}\]
\end{problem}}

%%%%%%%%%%%%%%%%%%%%%%

\latexProblemContent{
\ifVerboseLocation This is Derivative Compute Question 0050. \\ \fi
\begin{problem}

Find the limit.  Use L'H$\hat{o}$pital's rule where appropriate.

\input{Derivative-Compute-0050.HELP.tex}

\[\lim\limits_{x\to\infty} {\left(-\frac{6}{x} + 1\right)}^{x} - 11=\answer{e^{\left(-6\right)} - 11}\]
\end{problem}}

%%%%%%%%%%%%%%%%%%%%%%

\latexProblemContent{
\ifVerboseLocation This is Derivative Compute Question 0050. \\ \fi
\begin{problem}

Find the limit.  Use L'H$\hat{o}$pital's rule where appropriate.

\input{Derivative-Compute-0050.HELP.tex}

\[\lim\limits_{x\to\infty} {\left(\frac{1}{x} + 1\right)}^{-4 \, x} + 11=\answer{e^{\left(-4\right)} + 11}\]
\end{problem}}

%%%%%%%%%%%%%%%%%%%%%%

\latexProblemContent{
\ifVerboseLocation This is Derivative Compute Question 0050. \\ \fi
\begin{problem}

Find the limit.  Use L'H$\hat{o}$pital's rule where appropriate.

\input{Derivative-Compute-0050.HELP.tex}

\[\lim\limits_{x\to\infty} {\left(\frac{10}{x} + 1\right)}^{6 \, x}=\answer{e^{60}}\]
\end{problem}}

%%%%%%%%%%%%%%%%%%%%%%

\latexProblemContent{
\ifVerboseLocation This is Derivative Compute Question 0050. \\ \fi
\begin{problem}

Find the limit.  Use L'H$\hat{o}$pital's rule where appropriate.

\input{Derivative-Compute-0050.HELP.tex}

\[\lim\limits_{x\to\infty} {\left(\frac{3}{x} + 1\right)}^{-5 \, x} - 5=\answer{e^{\left(-15\right)} - 5}\]
\end{problem}}

%%%%%%%%%%%%%%%%%%%%%%

\latexProblemContent{
\ifVerboseLocation This is Derivative Compute Question 0050. \\ \fi
\begin{problem}

Find the limit.  Use L'H$\hat{o}$pital's rule where appropriate.

\input{Derivative-Compute-0050.HELP.tex}

\[\lim\limits_{x\to\infty} {\left(-\frac{6}{x} + 1\right)}^{-10 \, x} - 5=\answer{e^{60} - 5}\]
\end{problem}}

%%%%%%%%%%%%%%%%%%%%%%

\latexProblemContent{
\ifVerboseLocation This is Derivative Compute Question 0050. \\ \fi
\begin{problem}

Find the limit.  Use L'H$\hat{o}$pital's rule where appropriate.

\input{Derivative-Compute-0050.HELP.tex}

\[\lim\limits_{x\to\infty} {\left(\frac{6}{x} + 1\right)}^{-8 \, x} + 13=\answer{e^{\left(-48\right)} + 13}\]
\end{problem}}

%%%%%%%%%%%%%%%%%%%%%%

\latexProblemContent{
\ifVerboseLocation This is Derivative Compute Question 0050. \\ \fi
\begin{problem}

Find the limit.  Use L'H$\hat{o}$pital's rule where appropriate.

\input{Derivative-Compute-0050.HELP.tex}

\[\lim\limits_{x\to\infty} {\left(-\frac{2}{x} + 1\right)}^{5 \, x} - 10=\answer{e^{\left(-10\right)} - 10}\]
\end{problem}}

%%%%%%%%%%%%%%%%%%%%%%

\latexProblemContent{
\ifVerboseLocation This is Derivative Compute Question 0050. \\ \fi
\begin{problem}

Find the limit.  Use L'H$\hat{o}$pital's rule where appropriate.

\input{Derivative-Compute-0050.HELP.tex}

\[\lim\limits_{x\to\infty} {\left(-\frac{2}{x} + 1\right)}^{-x} + 18=\answer{e^{2} + 18}\]
\end{problem}}

%%%%%%%%%%%%%%%%%%%%%%

\latexProblemContent{
\ifVerboseLocation This is Derivative Compute Question 0050. \\ \fi
\begin{problem}

Find the limit.  Use L'H$\hat{o}$pital's rule where appropriate.

\input{Derivative-Compute-0050.HELP.tex}

\[\lim\limits_{x\to\infty} {\left(\frac{1}{x} + 1\right)}^{3 \, x} - 1=\answer{e^{3} - 1}\]
\end{problem}}

%%%%%%%%%%%%%%%%%%%%%%

\latexProblemContent{
\ifVerboseLocation This is Derivative Compute Question 0050. \\ \fi
\begin{problem}

Find the limit.  Use L'H$\hat{o}$pital's rule where appropriate.

\input{Derivative-Compute-0050.HELP.tex}

\[\lim\limits_{x\to\infty} {\left(\frac{9}{x} + 1\right)}^{-10 \, x} - 4=\answer{e^{\left(-90\right)} - 4}\]
\end{problem}}

%%%%%%%%%%%%%%%%%%%%%%

\latexProblemContent{
\ifVerboseLocation This is Derivative Compute Question 0050. \\ \fi
\begin{problem}

Find the limit.  Use L'H$\hat{o}$pital's rule where appropriate.

\input{Derivative-Compute-0050.HELP.tex}

\[\lim\limits_{x\to\infty} {\left(\frac{3}{x} + 1\right)}^{-2 \, x}=\answer{e^{\left(-6\right)}}\]
\end{problem}}

%%%%%%%%%%%%%%%%%%%%%%

\latexProblemContent{
\ifVerboseLocation This is Derivative Compute Question 0050. \\ \fi
\begin{problem}

Find the limit.  Use L'H$\hat{o}$pital's rule where appropriate.

\input{Derivative-Compute-0050.HELP.tex}

\[\lim\limits_{x\to\infty} {\left(\frac{4}{x} + 1\right)}^{-9 \, x} - 7=\answer{e^{\left(-36\right)} - 7}\]
\end{problem}}

%%%%%%%%%%%%%%%%%%%%%%

\latexProblemContent{
\ifVerboseLocation This is Derivative Compute Question 0050. \\ \fi
\begin{problem}

Find the limit.  Use L'H$\hat{o}$pital's rule where appropriate.

\input{Derivative-Compute-0050.HELP.tex}

\[\lim\limits_{x\to\infty} {\left(\frac{3}{x} + 1\right)}^{3 \, x} + 17=\answer{e^{9} + 17}\]
\end{problem}}

%%%%%%%%%%%%%%%%%%%%%%

\latexProblemContent{
\ifVerboseLocation This is Derivative Compute Question 0050. \\ \fi
\begin{problem}

Find the limit.  Use L'H$\hat{o}$pital's rule where appropriate.

\input{Derivative-Compute-0050.HELP.tex}

\[\lim\limits_{x\to\infty} {\left(-\frac{7}{x} + 1\right)}^{5 \, x} + 14=\answer{e^{\left(-35\right)} + 14}\]
\end{problem}}

%%%%%%%%%%%%%%%%%%%%%%

\latexProblemContent{
\ifVerboseLocation This is Derivative Compute Question 0050. \\ \fi
\begin{problem}

Find the limit.  Use L'H$\hat{o}$pital's rule where appropriate.

\input{Derivative-Compute-0050.HELP.tex}

\[\lim\limits_{x\to\infty} {\left(-\frac{4}{x} + 1\right)}^{-x} + 20=\answer{e^{4} + 20}\]
\end{problem}}

%%%%%%%%%%%%%%%%%%%%%%

\latexProblemContent{
\ifVerboseLocation This is Derivative Compute Question 0050. \\ \fi
\begin{problem}

Find the limit.  Use L'H$\hat{o}$pital's rule where appropriate.

\input{Derivative-Compute-0050.HELP.tex}

\[\lim\limits_{x\to\infty} {\left(\frac{1}{x} + 1\right)}^{6 \, x} + 10=\answer{e^{6} + 10}\]
\end{problem}}

%%%%%%%%%%%%%%%%%%%%%%

\latexProblemContent{
\ifVerboseLocation This is Derivative Compute Question 0050. \\ \fi
\begin{problem}

Find the limit.  Use L'H$\hat{o}$pital's rule where appropriate.

\input{Derivative-Compute-0050.HELP.tex}

\[\lim\limits_{x\to\infty} {\left(\frac{7}{x} + 1\right)}^{9 \, x} - 10=\answer{e^{63} - 10}\]
\end{problem}}

%%%%%%%%%%%%%%%%%%%%%%

\latexProblemContent{
\ifVerboseLocation This is Derivative Compute Question 0050. \\ \fi
\begin{problem}

Find the limit.  Use L'H$\hat{o}$pital's rule where appropriate.

\input{Derivative-Compute-0050.HELP.tex}

\[\lim\limits_{x\to\infty} {\left(-\frac{2}{x} + 1\right)}^{8 \, x} + 12=\answer{e^{\left(-16\right)} + 12}\]
\end{problem}}

%%%%%%%%%%%%%%%%%%%%%%

\latexProblemContent{
\ifVerboseLocation This is Derivative Compute Question 0050. \\ \fi
\begin{problem}

Find the limit.  Use L'H$\hat{o}$pital's rule where appropriate.

\input{Derivative-Compute-0050.HELP.tex}

\[\lim\limits_{x\to\infty} {\left(-\frac{4}{x} + 1\right)}^{-9 \, x} - 2=\answer{e^{36} - 2}\]
\end{problem}}

%%%%%%%%%%%%%%%%%%%%%%

\latexProblemContent{
\ifVerboseLocation This is Derivative Compute Question 0050. \\ \fi
\begin{problem}

Find the limit.  Use L'H$\hat{o}$pital's rule where appropriate.

\input{Derivative-Compute-0050.HELP.tex}

\[\lim\limits_{x\to\infty} {\left(\frac{8}{x} + 1\right)}^{-3 \, x} - 10=\answer{e^{\left(-24\right)} - 10}\]
\end{problem}}

%%%%%%%%%%%%%%%%%%%%%%

\latexProblemContent{
\ifVerboseLocation This is Derivative Compute Question 0050. \\ \fi
\begin{problem}

Find the limit.  Use L'H$\hat{o}$pital's rule where appropriate.

\input{Derivative-Compute-0050.HELP.tex}

\[\lim\limits_{x\to\infty} {\left(\frac{7}{x} + 1\right)}^{9 \, x} + 11=\answer{e^{63} + 11}\]
\end{problem}}

%%%%%%%%%%%%%%%%%%%%%%

\latexProblemContent{
\ifVerboseLocation This is Derivative Compute Question 0050. \\ \fi
\begin{problem}

Find the limit.  Use L'H$\hat{o}$pital's rule where appropriate.

\input{Derivative-Compute-0050.HELP.tex}

\[\lim\limits_{x\to\infty} {\left(\frac{10}{x} + 1\right)}^{-4 \, x} + 20=\answer{e^{\left(-40\right)} + 20}\]
\end{problem}}

%%%%%%%%%%%%%%%%%%%%%%

\latexProblemContent{
\ifVerboseLocation This is Derivative Compute Question 0050. \\ \fi
\begin{problem}

Find the limit.  Use L'H$\hat{o}$pital's rule where appropriate.

\input{Derivative-Compute-0050.HELP.tex}

\[\lim\limits_{x\to\infty} {\left(\frac{6}{x} + 1\right)}^{x} + 15=\answer{e^{6} + 15}\]
\end{problem}}

%%%%%%%%%%%%%%%%%%%%%%

\latexProblemContent{
\ifVerboseLocation This is Derivative Compute Question 0050. \\ \fi
\begin{problem}

Find the limit.  Use L'H$\hat{o}$pital's rule where appropriate.

\input{Derivative-Compute-0050.HELP.tex}

\[\lim\limits_{x\to\infty} {\left(\frac{1}{x} + 1\right)}^{-6 \, x} + 16=\answer{e^{\left(-6\right)} + 16}\]
\end{problem}}

%%%%%%%%%%%%%%%%%%%%%%

\latexProblemContent{
\ifVerboseLocation This is Derivative Compute Question 0050. \\ \fi
\begin{problem}

Find the limit.  Use L'H$\hat{o}$pital's rule where appropriate.

\input{Derivative-Compute-0050.HELP.tex}

\[\lim\limits_{x\to\infty} {\left(-\frac{4}{x} + 1\right)}^{-7 \, x} - 5=\answer{e^{28} - 5}\]
\end{problem}}

%%%%%%%%%%%%%%%%%%%%%%

\latexProblemContent{
\ifVerboseLocation This is Derivative Compute Question 0050. \\ \fi
\begin{problem}

Find the limit.  Use L'H$\hat{o}$pital's rule where appropriate.

\input{Derivative-Compute-0050.HELP.tex}

\[\lim\limits_{x\to\infty} {\left(\frac{6}{x} + 1\right)}^{8 \, x} + 19=\answer{e^{48} + 19}\]
\end{problem}}

%%%%%%%%%%%%%%%%%%%%%%

\latexProblemContent{
\ifVerboseLocation This is Derivative Compute Question 0050. \\ \fi
\begin{problem}

Find the limit.  Use L'H$\hat{o}$pital's rule where appropriate.

\input{Derivative-Compute-0050.HELP.tex}

\[\lim\limits_{x\to\infty} {\left(\frac{8}{x} + 1\right)}^{-3 \, x} - 4=\answer{e^{\left(-24\right)} - 4}\]
\end{problem}}

%%%%%%%%%%%%%%%%%%%%%%

\latexProblemContent{
\ifVerboseLocation This is Derivative Compute Question 0050. \\ \fi
\begin{problem}

Find the limit.  Use L'H$\hat{o}$pital's rule where appropriate.

\input{Derivative-Compute-0050.HELP.tex}

\[\lim\limits_{x\to\infty} {\left(\frac{6}{x} + 1\right)}^{-3 \, x} - 7=\answer{e^{\left(-18\right)} - 7}\]
\end{problem}}

%%%%%%%%%%%%%%%%%%%%%%

\latexProblemContent{
\ifVerboseLocation This is Derivative Compute Question 0050. \\ \fi
\begin{problem}

Find the limit.  Use L'H$\hat{o}$pital's rule where appropriate.

\input{Derivative-Compute-0050.HELP.tex}

\[\lim\limits_{x\to\infty} {\left(-\frac{4}{x} + 1\right)}^{-10 \, x} - 8=\answer{e^{40} - 8}\]
\end{problem}}

%%%%%%%%%%%%%%%%%%%%%%

\latexProblemContent{
\ifVerboseLocation This is Derivative Compute Question 0050. \\ \fi
\begin{problem}

Find the limit.  Use L'H$\hat{o}$pital's rule where appropriate.

\input{Derivative-Compute-0050.HELP.tex}

\[\lim\limits_{x\to\infty} {\left(\frac{9}{x} + 1\right)}^{7 \, x} - 14=\answer{e^{63} - 14}\]
\end{problem}}

%%%%%%%%%%%%%%%%%%%%%%

\latexProblemContent{
\ifVerboseLocation This is Derivative Compute Question 0050. \\ \fi
\begin{problem}

Find the limit.  Use L'H$\hat{o}$pital's rule where appropriate.

\input{Derivative-Compute-0050.HELP.tex}

\[\lim\limits_{x\to\infty} {\left(-\frac{1}{x} + 1\right)}^{-4 \, x} - 9=\answer{e^{4} - 9}\]
\end{problem}}

%%%%%%%%%%%%%%%%%%%%%%

\latexProblemContent{
\ifVerboseLocation This is Derivative Compute Question 0050. \\ \fi
\begin{problem}

Find the limit.  Use L'H$\hat{o}$pital's rule where appropriate.

\input{Derivative-Compute-0050.HELP.tex}

\[\lim\limits_{x\to\infty} {\left(-\frac{8}{x} + 1\right)}^{-6 \, x} + 14=\answer{e^{48} + 14}\]
\end{problem}}

%%%%%%%%%%%%%%%%%%%%%%

\latexProblemContent{
\ifVerboseLocation This is Derivative Compute Question 0050. \\ \fi
\begin{problem}

Find the limit.  Use L'H$\hat{o}$pital's rule where appropriate.

\input{Derivative-Compute-0050.HELP.tex}

\[\lim\limits_{x\to\infty} {\left(\frac{4}{x} + 1\right)}^{4 \, x} - 11=\answer{e^{16} - 11}\]
\end{problem}}

%%%%%%%%%%%%%%%%%%%%%%

\latexProblemContent{
\ifVerboseLocation This is Derivative Compute Question 0050. \\ \fi
\begin{problem}

Find the limit.  Use L'H$\hat{o}$pital's rule where appropriate.

\input{Derivative-Compute-0050.HELP.tex}

\[\lim\limits_{x\to\infty} {\left(-\frac{9}{x} + 1\right)}^{6 \, x} + 7=\answer{e^{\left(-54\right)} + 7}\]
\end{problem}}

%%%%%%%%%%%%%%%%%%%%%%

\latexProblemContent{
\ifVerboseLocation This is Derivative Compute Question 0050. \\ \fi
\begin{problem}

Find the limit.  Use L'H$\hat{o}$pital's rule where appropriate.

\input{Derivative-Compute-0050.HELP.tex}

\[\lim\limits_{x\to\infty} {\left(-\frac{7}{x} + 1\right)}^{4 \, x} - 10=\answer{e^{\left(-28\right)} - 10}\]
\end{problem}}

%%%%%%%%%%%%%%%%%%%%%%

\latexProblemContent{
\ifVerboseLocation This is Derivative Compute Question 0050. \\ \fi
\begin{problem}

Find the limit.  Use L'H$\hat{o}$pital's rule where appropriate.

\input{Derivative-Compute-0050.HELP.tex}

\[\lim\limits_{x\to\infty} {\left(\frac{8}{x} + 1\right)}^{-9 \, x} + 6=\answer{e^{\left(-72\right)} + 6}\]
\end{problem}}

%%%%%%%%%%%%%%%%%%%%%%

\latexProblemContent{
\ifVerboseLocation This is Derivative Compute Question 0050. \\ \fi
\begin{problem}

Find the limit.  Use L'H$\hat{o}$pital's rule where appropriate.

\input{Derivative-Compute-0050.HELP.tex}

\[\lim\limits_{x\to\infty} {\left(\frac{1}{x} + 1\right)}^{5 \, x} - 18=\answer{e^{5} - 18}\]
\end{problem}}

%%%%%%%%%%%%%%%%%%%%%%

\latexProblemContent{
\ifVerboseLocation This is Derivative Compute Question 0050. \\ \fi
\begin{problem}

Find the limit.  Use L'H$\hat{o}$pital's rule where appropriate.

\input{Derivative-Compute-0050.HELP.tex}

\[\lim\limits_{x\to\infty} {\left(-\frac{5}{x} + 1\right)}^{4 \, x} - 6=\answer{e^{\left(-20\right)} - 6}\]
\end{problem}}

%%%%%%%%%%%%%%%%%%%%%%

\latexProblemContent{
\ifVerboseLocation This is Derivative Compute Question 0050. \\ \fi
\begin{problem}

Find the limit.  Use L'H$\hat{o}$pital's rule where appropriate.

\input{Derivative-Compute-0050.HELP.tex}

\[\lim\limits_{x\to\infty} {\left(-\frac{7}{x} + 1\right)}^{6 \, x} + 12=\answer{e^{\left(-42\right)} + 12}\]
\end{problem}}

%%%%%%%%%%%%%%%%%%%%%%

\latexProblemContent{
\ifVerboseLocation This is Derivative Compute Question 0050. \\ \fi
\begin{problem}

Find the limit.  Use L'H$\hat{o}$pital's rule where appropriate.

\input{Derivative-Compute-0050.HELP.tex}

\[\lim\limits_{x\to\infty} {\left(-\frac{10}{x} + 1\right)}^{x} - 10=\answer{e^{\left(-10\right)} - 10}\]
\end{problem}}

%%%%%%%%%%%%%%%%%%%%%%

\latexProblemContent{
\ifVerboseLocation This is Derivative Compute Question 0050. \\ \fi
\begin{problem}

Find the limit.  Use L'H$\hat{o}$pital's rule where appropriate.

\input{Derivative-Compute-0050.HELP.tex}

\[\lim\limits_{x\to\infty} {\left(-\frac{8}{x} + 1\right)}^{-4 \, x} + 18=\answer{e^{32} + 18}\]
\end{problem}}

%%%%%%%%%%%%%%%%%%%%%%

\latexProblemContent{
\ifVerboseLocation This is Derivative Compute Question 0050. \\ \fi
\begin{problem}

Find the limit.  Use L'H$\hat{o}$pital's rule where appropriate.

\input{Derivative-Compute-0050.HELP.tex}

\[\lim\limits_{x\to\infty} {\left(-\frac{7}{x} + 1\right)}^{4 \, x} - 14=\answer{e^{\left(-28\right)} - 14}\]
\end{problem}}

%%%%%%%%%%%%%%%%%%%%%%

\latexProblemContent{
\ifVerboseLocation This is Derivative Compute Question 0050. \\ \fi
\begin{problem}

Find the limit.  Use L'H$\hat{o}$pital's rule where appropriate.

\input{Derivative-Compute-0050.HELP.tex}

\[\lim\limits_{x\to\infty} {\left(\frac{3}{x} + 1\right)}^{-5 \, x} - 8=\answer{e^{\left(-15\right)} - 8}\]
\end{problem}}

%%%%%%%%%%%%%%%%%%%%%%

\latexProblemContent{
\ifVerboseLocation This is Derivative Compute Question 0050. \\ \fi
\begin{problem}

Find the limit.  Use L'H$\hat{o}$pital's rule where appropriate.

\input{Derivative-Compute-0050.HELP.tex}

\[\lim\limits_{x\to\infty} {\left(-\frac{6}{x} + 1\right)}^{7 \, x} - 4=\answer{e^{\left(-42\right)} - 4}\]
\end{problem}}

%%%%%%%%%%%%%%%%%%%%%%

\latexProblemContent{
\ifVerboseLocation This is Derivative Compute Question 0050. \\ \fi
\begin{problem}

Find the limit.  Use L'H$\hat{o}$pital's rule where appropriate.

\input{Derivative-Compute-0050.HELP.tex}

\[\lim\limits_{x\to\infty} {\left(-\frac{6}{x} + 1\right)}^{-9 \, x} + 18=\answer{e^{54} + 18}\]
\end{problem}}

%%%%%%%%%%%%%%%%%%%%%%

\latexProblemContent{
\ifVerboseLocation This is Derivative Compute Question 0050. \\ \fi
\begin{problem}

Find the limit.  Use L'H$\hat{o}$pital's rule where appropriate.

\input{Derivative-Compute-0050.HELP.tex}

\[\lim\limits_{x\to\infty} {\left(\frac{1}{x} + 1\right)}^{5 \, x} - 1=\answer{e^{5} - 1}\]
\end{problem}}

%%%%%%%%%%%%%%%%%%%%%%

\latexProblemContent{
\ifVerboseLocation This is Derivative Compute Question 0050. \\ \fi
\begin{problem}

Find the limit.  Use L'H$\hat{o}$pital's rule where appropriate.

\input{Derivative-Compute-0050.HELP.tex}

\[\lim\limits_{x\to\infty} {\left(-\frac{10}{x} + 1\right)}^{7 \, x} - 1=\answer{e^{\left(-70\right)} - 1}\]
\end{problem}}

%%%%%%%%%%%%%%%%%%%%%%

\latexProblemContent{
\ifVerboseLocation This is Derivative Compute Question 0050. \\ \fi
\begin{problem}

Find the limit.  Use L'H$\hat{o}$pital's rule where appropriate.

\input{Derivative-Compute-0050.HELP.tex}

\[\lim\limits_{x\to\infty} {\left(-\frac{5}{x} + 1\right)}^{7 \, x} - 5=\answer{e^{\left(-35\right)} - 5}\]
\end{problem}}

%%%%%%%%%%%%%%%%%%%%%%

\latexProblemContent{
\ifVerboseLocation This is Derivative Compute Question 0050. \\ \fi
\begin{problem}

Find the limit.  Use L'H$\hat{o}$pital's rule where appropriate.

\input{Derivative-Compute-0050.HELP.tex}

\[\lim\limits_{x\to\infty} {\left(\frac{2}{x} + 1\right)}^{-9 \, x} + 6=\answer{e^{\left(-18\right)} + 6}\]
\end{problem}}

%%%%%%%%%%%%%%%%%%%%%%

\latexProblemContent{
\ifVerboseLocation This is Derivative Compute Question 0050. \\ \fi
\begin{problem}

Find the limit.  Use L'H$\hat{o}$pital's rule where appropriate.

\input{Derivative-Compute-0050.HELP.tex}

\[\lim\limits_{x\to\infty} {\left(\frac{3}{x} + 1\right)}^{-6 \, x} - 16=\answer{e^{\left(-18\right)} - 16}\]
\end{problem}}

%%%%%%%%%%%%%%%%%%%%%%

\latexProblemContent{
\ifVerboseLocation This is Derivative Compute Question 0050. \\ \fi
\begin{problem}

Find the limit.  Use L'H$\hat{o}$pital's rule where appropriate.

\input{Derivative-Compute-0050.HELP.tex}

\[\lim\limits_{x\to\infty} {\left(\frac{9}{x} + 1\right)}^{2 \, x} - 1=\answer{e^{18} - 1}\]
\end{problem}}

%%%%%%%%%%%%%%%%%%%%%%

\latexProblemContent{
\ifVerboseLocation This is Derivative Compute Question 0050. \\ \fi
\begin{problem}

Find the limit.  Use L'H$\hat{o}$pital's rule where appropriate.

\input{Derivative-Compute-0050.HELP.tex}

\[\lim\limits_{x\to\infty} {\left(-\frac{8}{x} + 1\right)}^{-5 \, x} + 14=\answer{e^{40} + 14}\]
\end{problem}}

%%%%%%%%%%%%%%%%%%%%%%

\latexProblemContent{
\ifVerboseLocation This is Derivative Compute Question 0050. \\ \fi
\begin{problem}

Find the limit.  Use L'H$\hat{o}$pital's rule where appropriate.

\input{Derivative-Compute-0050.HELP.tex}

\[\lim\limits_{x\to\infty} {\left(\frac{5}{x} + 1\right)}^{2 \, x} - 9=\answer{e^{10} - 9}\]
\end{problem}}

%%%%%%%%%%%%%%%%%%%%%%

\latexProblemContent{
\ifVerboseLocation This is Derivative Compute Question 0050. \\ \fi
\begin{problem}

Find the limit.  Use L'H$\hat{o}$pital's rule where appropriate.

\input{Derivative-Compute-0050.HELP.tex}

\[\lim\limits_{x\to\infty} {\left(\frac{4}{x} + 1\right)}^{7 \, x} + 14=\answer{e^{28} + 14}\]
\end{problem}}

%%%%%%%%%%%%%%%%%%%%%%

\latexProblemContent{
\ifVerboseLocation This is Derivative Compute Question 0050. \\ \fi
\begin{problem}

Find the limit.  Use L'H$\hat{o}$pital's rule where appropriate.

\input{Derivative-Compute-0050.HELP.tex}

\[\lim\limits_{x\to\infty} {\left(\frac{4}{x} + 1\right)}^{3 \, x} - 16=\answer{e^{12} - 16}\]
\end{problem}}

%%%%%%%%%%%%%%%%%%%%%%

\latexProblemContent{
\ifVerboseLocation This is Derivative Compute Question 0050. \\ \fi
\begin{problem}

Find the limit.  Use L'H$\hat{o}$pital's rule where appropriate.

\input{Derivative-Compute-0050.HELP.tex}

\[\lim\limits_{x\to\infty} {\left(\frac{1}{x} + 1\right)}^{3 \, x} - 4=\answer{e^{3} - 4}\]
\end{problem}}

%%%%%%%%%%%%%%%%%%%%%%

\latexProblemContent{
\ifVerboseLocation This is Derivative Compute Question 0050. \\ \fi
\begin{problem}

Find the limit.  Use L'H$\hat{o}$pital's rule where appropriate.

\input{Derivative-Compute-0050.HELP.tex}

\[\lim\limits_{x\to\infty} {\left(-\frac{6}{x} + 1\right)}^{5 \, x} + 16=\answer{e^{\left(-30\right)} + 16}\]
\end{problem}}

%%%%%%%%%%%%%%%%%%%%%%

\latexProblemContent{
\ifVerboseLocation This is Derivative Compute Question 0050. \\ \fi
\begin{problem}

Find the limit.  Use L'H$\hat{o}$pital's rule where appropriate.

\input{Derivative-Compute-0050.HELP.tex}

\[\lim\limits_{x\to\infty} {\left(-\frac{9}{x} + 1\right)}^{9 \, x}=\answer{e^{\left(-81\right)}}\]
\end{problem}}

%%%%%%%%%%%%%%%%%%%%%%

\latexProblemContent{
\ifVerboseLocation This is Derivative Compute Question 0050. \\ \fi
\begin{problem}

Find the limit.  Use L'H$\hat{o}$pital's rule where appropriate.

\input{Derivative-Compute-0050.HELP.tex}

\[\lim\limits_{x\to\infty} {\left(-\frac{4}{x} + 1\right)}^{9 \, x} - 20=\answer{e^{\left(-36\right)} - 20}\]
\end{problem}}

%%%%%%%%%%%%%%%%%%%%%%

\latexProblemContent{
\ifVerboseLocation This is Derivative Compute Question 0050. \\ \fi
\begin{problem}

Find the limit.  Use L'H$\hat{o}$pital's rule where appropriate.

\input{Derivative-Compute-0050.HELP.tex}

\[\lim\limits_{x\to\infty} {\left(-\frac{2}{x} + 1\right)}^{4 \, x} - 20=\answer{e^{\left(-8\right)} - 20}\]
\end{problem}}

%%%%%%%%%%%%%%%%%%%%%%

\latexProblemContent{
\ifVerboseLocation This is Derivative Compute Question 0050. \\ \fi
\begin{problem}

Find the limit.  Use L'H$\hat{o}$pital's rule where appropriate.

\input{Derivative-Compute-0050.HELP.tex}

\[\lim\limits_{x\to\infty} {\left(-\frac{6}{x} + 1\right)}^{6 \, x} - 6=\answer{e^{\left(-36\right)} - 6}\]
\end{problem}}

%%%%%%%%%%%%%%%%%%%%%%

\latexProblemContent{
\ifVerboseLocation This is Derivative Compute Question 0050. \\ \fi
\begin{problem}

Find the limit.  Use L'H$\hat{o}$pital's rule where appropriate.

\input{Derivative-Compute-0050.HELP.tex}

\[\lim\limits_{x\to\infty} {\left(-\frac{8}{x} + 1\right)}^{9 \, x} - 14=\answer{e^{\left(-72\right)} - 14}\]
\end{problem}}

%%%%%%%%%%%%%%%%%%%%%%

\latexProblemContent{
\ifVerboseLocation This is Derivative Compute Question 0050. \\ \fi
\begin{problem}

Find the limit.  Use L'H$\hat{o}$pital's rule where appropriate.

\input{Derivative-Compute-0050.HELP.tex}

\[\lim\limits_{x\to\infty} {\left(-\frac{6}{x} + 1\right)}^{9 \, x} + 11=\answer{e^{\left(-54\right)} + 11}\]
\end{problem}}

%%%%%%%%%%%%%%%%%%%%%%

\latexProblemContent{
\ifVerboseLocation This is Derivative Compute Question 0050. \\ \fi
\begin{problem}

Find the limit.  Use L'H$\hat{o}$pital's rule where appropriate.

\input{Derivative-Compute-0050.HELP.tex}

\[\lim\limits_{x\to\infty} {\left(\frac{10}{x} + 1\right)}^{3 \, x} - 19=\answer{e^{30} - 19}\]
\end{problem}}

%%%%%%%%%%%%%%%%%%%%%%

\latexProblemContent{
\ifVerboseLocation This is Derivative Compute Question 0050. \\ \fi
\begin{problem}

Find the limit.  Use L'H$\hat{o}$pital's rule where appropriate.

\input{Derivative-Compute-0050.HELP.tex}

\[\lim\limits_{x\to\infty} {\left(-\frac{3}{x} + 1\right)}^{-5 \, x} + 9=\answer{e^{15} + 9}\]
\end{problem}}

%%%%%%%%%%%%%%%%%%%%%%

\latexProblemContent{
\ifVerboseLocation This is Derivative Compute Question 0050. \\ \fi
\begin{problem}

Find the limit.  Use L'H$\hat{o}$pital's rule where appropriate.

\input{Derivative-Compute-0050.HELP.tex}

\[\lim\limits_{x\to\infty} {\left(-\frac{7}{x} + 1\right)}^{3 \, x}=\answer{e^{\left(-21\right)}}\]
\end{problem}}

%%%%%%%%%%%%%%%%%%%%%%

\latexProblemContent{
\ifVerboseLocation This is Derivative Compute Question 0050. \\ \fi
\begin{problem}

Find the limit.  Use L'H$\hat{o}$pital's rule where appropriate.

\input{Derivative-Compute-0050.HELP.tex}

\[\lim\limits_{x\to\infty} {\left(-\frac{8}{x} + 1\right)}^{-5 \, x} - 4=\answer{e^{40} - 4}\]
\end{problem}}

%%%%%%%%%%%%%%%%%%%%%%

\latexProblemContent{
\ifVerboseLocation This is Derivative Compute Question 0050. \\ \fi
\begin{problem}

Find the limit.  Use L'H$\hat{o}$pital's rule where appropriate.

\input{Derivative-Compute-0050.HELP.tex}

\[\lim\limits_{x\to\infty} {\left(-\frac{9}{x} + 1\right)}^{-x} - 4=\answer{e^{9} - 4}\]
\end{problem}}

%%%%%%%%%%%%%%%%%%%%%%

\latexProblemContent{
\ifVerboseLocation This is Derivative Compute Question 0050. \\ \fi
\begin{problem}

Find the limit.  Use L'H$\hat{o}$pital's rule where appropriate.

\input{Derivative-Compute-0050.HELP.tex}

\[\lim\limits_{x\to\infty} {\left(\frac{2}{x} + 1\right)}^{6 \, x} + 12=\answer{e^{12} + 12}\]
\end{problem}}

%%%%%%%%%%%%%%%%%%%%%%

\latexProblemContent{
\ifVerboseLocation This is Derivative Compute Question 0050. \\ \fi
\begin{problem}

Find the limit.  Use L'H$\hat{o}$pital's rule where appropriate.

\input{Derivative-Compute-0050.HELP.tex}

\[\lim\limits_{x\to\infty} {\left(\frac{1}{x} + 1\right)}^{4 \, x} + 20=\answer{e^{4} + 20}\]
\end{problem}}

%%%%%%%%%%%%%%%%%%%%%%

\latexProblemContent{
\ifVerboseLocation This is Derivative Compute Question 0050. \\ \fi
\begin{problem}

Find the limit.  Use L'H$\hat{o}$pital's rule where appropriate.

\input{Derivative-Compute-0050.HELP.tex}

\[\lim\limits_{x\to\infty} {\left(\frac{7}{x} + 1\right)}^{3 \, x} + 3=\answer{e^{21} + 3}\]
\end{problem}}

%%%%%%%%%%%%%%%%%%%%%%

\latexProblemContent{
\ifVerboseLocation This is Derivative Compute Question 0050. \\ \fi
\begin{problem}

Find the limit.  Use L'H$\hat{o}$pital's rule where appropriate.

\input{Derivative-Compute-0050.HELP.tex}

\[\lim\limits_{x\to\infty} {\left(-\frac{8}{x} + 1\right)}^{x} + 1=\answer{e^{\left(-8\right)} + 1}\]
\end{problem}}

%%%%%%%%%%%%%%%%%%%%%%

\latexProblemContent{
\ifVerboseLocation This is Derivative Compute Question 0050. \\ \fi
\begin{problem}

Find the limit.  Use L'H$\hat{o}$pital's rule where appropriate.

\input{Derivative-Compute-0050.HELP.tex}

\[\lim\limits_{x\to\infty} {\left(-\frac{8}{x} + 1\right)}^{5 \, x} + 18=\answer{e^{\left(-40\right)} + 18}\]
\end{problem}}

%%%%%%%%%%%%%%%%%%%%%%

\latexProblemContent{
\ifVerboseLocation This is Derivative Compute Question 0050. \\ \fi
\begin{problem}

Find the limit.  Use L'H$\hat{o}$pital's rule where appropriate.

\input{Derivative-Compute-0050.HELP.tex}

\[\lim\limits_{x\to\infty} {\left(-\frac{7}{x} + 1\right)}^{-5 \, x} + 11=\answer{e^{35} + 11}\]
\end{problem}}

%%%%%%%%%%%%%%%%%%%%%%

\latexProblemContent{
\ifVerboseLocation This is Derivative Compute Question 0050. \\ \fi
\begin{problem}

Find the limit.  Use L'H$\hat{o}$pital's rule where appropriate.

\input{Derivative-Compute-0050.HELP.tex}

\[\lim\limits_{x\to\infty} {\left(\frac{4}{x} + 1\right)}^{3 \, x} - 11=\answer{e^{12} - 11}\]
\end{problem}}

%%%%%%%%%%%%%%%%%%%%%%

\latexProblemContent{
\ifVerboseLocation This is Derivative Compute Question 0050. \\ \fi
\begin{problem}

Find the limit.  Use L'H$\hat{o}$pital's rule where appropriate.

\input{Derivative-Compute-0050.HELP.tex}

\[\lim\limits_{x\to\infty} {\left(\frac{6}{x} + 1\right)}^{-5 \, x} - 16=\answer{e^{\left(-30\right)} - 16}\]
\end{problem}}

%%%%%%%%%%%%%%%%%%%%%%

\latexProblemContent{
\ifVerboseLocation This is Derivative Compute Question 0050. \\ \fi
\begin{problem}

Find the limit.  Use L'H$\hat{o}$pital's rule where appropriate.

\input{Derivative-Compute-0050.HELP.tex}

\[\lim\limits_{x\to\infty} {\left(-\frac{6}{x} + 1\right)}^{5 \, x} - 1=\answer{e^{\left(-30\right)} - 1}\]
\end{problem}}

%%%%%%%%%%%%%%%%%%%%%%

\latexProblemContent{
\ifVerboseLocation This is Derivative Compute Question 0050. \\ \fi
\begin{problem}

Find the limit.  Use L'H$\hat{o}$pital's rule where appropriate.

\input{Derivative-Compute-0050.HELP.tex}

\[\lim\limits_{x\to\infty} {\left(\frac{7}{x} + 1\right)}^{3 \, x} + 9=\answer{e^{21} + 9}\]
\end{problem}}

%%%%%%%%%%%%%%%%%%%%%%

\latexProblemContent{
\ifVerboseLocation This is Derivative Compute Question 0050. \\ \fi
\begin{problem}

Find the limit.  Use L'H$\hat{o}$pital's rule where appropriate.

\input{Derivative-Compute-0050.HELP.tex}

\[\lim\limits_{x\to\infty} {\left(-\frac{6}{x} + 1\right)}^{-x} + 15=\answer{e^{6} + 15}\]
\end{problem}}

%%%%%%%%%%%%%%%%%%%%%%

\latexProblemContent{
\ifVerboseLocation This is Derivative Compute Question 0050. \\ \fi
\begin{problem}

Find the limit.  Use L'H$\hat{o}$pital's rule where appropriate.

\input{Derivative-Compute-0050.HELP.tex}

\[\lim\limits_{x\to\infty} {\left(-\frac{5}{x} + 1\right)}^{10 \, x} - 20=\answer{e^{\left(-50\right)} - 20}\]
\end{problem}}

%%%%%%%%%%%%%%%%%%%%%%

\latexProblemContent{
\ifVerboseLocation This is Derivative Compute Question 0050. \\ \fi
\begin{problem}

Find the limit.  Use L'H$\hat{o}$pital's rule where appropriate.

\input{Derivative-Compute-0050.HELP.tex}

\[\lim\limits_{x\to\infty} {\left(-\frac{10}{x} + 1\right)}^{2 \, x}=\answer{e^{\left(-20\right)}}\]
\end{problem}}

%%%%%%%%%%%%%%%%%%%%%%

\latexProblemContent{
\ifVerboseLocation This is Derivative Compute Question 0050. \\ \fi
\begin{problem}

Find the limit.  Use L'H$\hat{o}$pital's rule where appropriate.

\input{Derivative-Compute-0050.HELP.tex}

\[\lim\limits_{x\to\infty} {\left(-\frac{5}{x} + 1\right)}^{9 \, x} - 13=\answer{e^{\left(-45\right)} - 13}\]
\end{problem}}

%%%%%%%%%%%%%%%%%%%%%%

\latexProblemContent{
\ifVerboseLocation This is Derivative Compute Question 0050. \\ \fi
\begin{problem}

Find the limit.  Use L'H$\hat{o}$pital's rule where appropriate.

\input{Derivative-Compute-0050.HELP.tex}

\[\lim\limits_{x\to\infty} {\left(\frac{6}{x} + 1\right)}^{x} - 19=\answer{e^{6} - 19}\]
\end{problem}}

%%%%%%%%%%%%%%%%%%%%%%

\latexProblemContent{
\ifVerboseLocation This is Derivative Compute Question 0050. \\ \fi
\begin{problem}

Find the limit.  Use L'H$\hat{o}$pital's rule where appropriate.

\input{Derivative-Compute-0050.HELP.tex}

\[\lim\limits_{x\to\infty} {\left(\frac{6}{x} + 1\right)}^{8 \, x} + 17=\answer{e^{48} + 17}\]
\end{problem}}

%%%%%%%%%%%%%%%%%%%%%%

\latexProblemContent{
\ifVerboseLocation This is Derivative Compute Question 0050. \\ \fi
\begin{problem}

Find the limit.  Use L'H$\hat{o}$pital's rule where appropriate.

\input{Derivative-Compute-0050.HELP.tex}

\[\lim\limits_{x\to\infty} {\left(\frac{6}{x} + 1\right)}^{2 \, x} - 14=\answer{e^{12} - 14}\]
\end{problem}}

%%%%%%%%%%%%%%%%%%%%%%

\latexProblemContent{
\ifVerboseLocation This is Derivative Compute Question 0050. \\ \fi
\begin{problem}

Find the limit.  Use L'H$\hat{o}$pital's rule where appropriate.

\input{Derivative-Compute-0050.HELP.tex}

\[\lim\limits_{x\to\infty} {\left(-\frac{10}{x} + 1\right)}^{3 \, x} + 18=\answer{e^{\left(-30\right)} + 18}\]
\end{problem}}

%%%%%%%%%%%%%%%%%%%%%%

\latexProblemContent{
\ifVerboseLocation This is Derivative Compute Question 0050. \\ \fi
\begin{problem}

Find the limit.  Use L'H$\hat{o}$pital's rule where appropriate.

\input{Derivative-Compute-0050.HELP.tex}

\[\lim\limits_{x\to\infty} {\left(-\frac{1}{x} + 1\right)}^{2 \, x} - 11=\answer{e^{\left(-2\right)} - 11}\]
\end{problem}}

%%%%%%%%%%%%%%%%%%%%%%

\latexProblemContent{
\ifVerboseLocation This is Derivative Compute Question 0050. \\ \fi
\begin{problem}

Find the limit.  Use L'H$\hat{o}$pital's rule where appropriate.

\input{Derivative-Compute-0050.HELP.tex}

\[\lim\limits_{x\to\infty} {\left(-\frac{10}{x} + 1\right)}^{-8 \, x} - 8=\answer{e^{80} - 8}\]
\end{problem}}

%%%%%%%%%%%%%%%%%%%%%%

\latexProblemContent{
\ifVerboseLocation This is Derivative Compute Question 0050. \\ \fi
\begin{problem}

Find the limit.  Use L'H$\hat{o}$pital's rule where appropriate.

\input{Derivative-Compute-0050.HELP.tex}

\[\lim\limits_{x\to\infty} {\left(-\frac{4}{x} + 1\right)}^{2 \, x} - 5=\answer{e^{\left(-8\right)} - 5}\]
\end{problem}}

%%%%%%%%%%%%%%%%%%%%%%

\latexProblemContent{
\ifVerboseLocation This is Derivative Compute Question 0050. \\ \fi
\begin{problem}

Find the limit.  Use L'H$\hat{o}$pital's rule where appropriate.

\input{Derivative-Compute-0050.HELP.tex}

\[\lim\limits_{x\to\infty} {\left(-\frac{4}{x} + 1\right)}^{-5 \, x} - 9=\answer{e^{20} - 9}\]
\end{problem}}

%%%%%%%%%%%%%%%%%%%%%%

\latexProblemContent{
\ifVerboseLocation This is Derivative Compute Question 0050. \\ \fi
\begin{problem}

Find the limit.  Use L'H$\hat{o}$pital's rule where appropriate.

\input{Derivative-Compute-0050.HELP.tex}

\[\lim\limits_{x\to\infty} {\left(-\frac{3}{x} + 1\right)}^{x} + 9=\answer{e^{\left(-3\right)} + 9}\]
\end{problem}}

%%%%%%%%%%%%%%%%%%%%%%

\latexProblemContent{
\ifVerboseLocation This is Derivative Compute Question 0050. \\ \fi
\begin{problem}

Find the limit.  Use L'H$\hat{o}$pital's rule where appropriate.

\input{Derivative-Compute-0050.HELP.tex}

\[\lim\limits_{x\to\infty} {\left(-\frac{8}{x} + 1\right)}^{-2 \, x} + 19=\answer{e^{16} + 19}\]
\end{problem}}

%%%%%%%%%%%%%%%%%%%%%%

\latexProblemContent{
\ifVerboseLocation This is Derivative Compute Question 0050. \\ \fi
\begin{problem}

Find the limit.  Use L'H$\hat{o}$pital's rule where appropriate.

\input{Derivative-Compute-0050.HELP.tex}

\[\lim\limits_{x\to\infty} {\left(-\frac{1}{x} + 1\right)}^{-7 \, x} - 12=\answer{e^{7} - 12}\]
\end{problem}}

%%%%%%%%%%%%%%%%%%%%%%

\latexProblemContent{
\ifVerboseLocation This is Derivative Compute Question 0050. \\ \fi
\begin{problem}

Find the limit.  Use L'H$\hat{o}$pital's rule where appropriate.

\input{Derivative-Compute-0050.HELP.tex}

\[\lim\limits_{x\to\infty} {\left(-\frac{2}{x} + 1\right)}^{x} - 2=\answer{e^{\left(-2\right)} - 2}\]
\end{problem}}

%%%%%%%%%%%%%%%%%%%%%%

\latexProblemContent{
\ifVerboseLocation This is Derivative Compute Question 0050. \\ \fi
\begin{problem}

Find the limit.  Use L'H$\hat{o}$pital's rule where appropriate.

\input{Derivative-Compute-0050.HELP.tex}

\[\lim\limits_{x\to\infty} {\left(-\frac{8}{x} + 1\right)}^{3 \, x} - 16=\answer{e^{\left(-24\right)} - 16}\]
\end{problem}}

%%%%%%%%%%%%%%%%%%%%%%

\latexProblemContent{
\ifVerboseLocation This is Derivative Compute Question 0050. \\ \fi
\begin{problem}

Find the limit.  Use L'H$\hat{o}$pital's rule where appropriate.

\input{Derivative-Compute-0050.HELP.tex}

\[\lim\limits_{x\to\infty} {\left(\frac{1}{x} + 1\right)}^{-7 \, x} + 16=\answer{e^{\left(-7\right)} + 16}\]
\end{problem}}

%%%%%%%%%%%%%%%%%%%%%%

\latexProblemContent{
\ifVerboseLocation This is Derivative Compute Question 0050. \\ \fi
\begin{problem}

Find the limit.  Use L'H$\hat{o}$pital's rule where appropriate.

\input{Derivative-Compute-0050.HELP.tex}

\[\lim\limits_{x\to\infty} {\left(\frac{6}{x} + 1\right)}^{10 \, x}=\answer{e^{60}}\]
\end{problem}}

%%%%%%%%%%%%%%%%%%%%%%

\latexProblemContent{
\ifVerboseLocation This is Derivative Compute Question 0050. \\ \fi
\begin{problem}

Find the limit.  Use L'H$\hat{o}$pital's rule where appropriate.

\input{Derivative-Compute-0050.HELP.tex}

\[\lim\limits_{x\to\infty} {\left(-\frac{10}{x} + 1\right)}^{-2 \, x} + 12=\answer{e^{20} + 12}\]
\end{problem}}

%%%%%%%%%%%%%%%%%%%%%%

\latexProblemContent{
\ifVerboseLocation This is Derivative Compute Question 0050. \\ \fi
\begin{problem}

Find the limit.  Use L'H$\hat{o}$pital's rule where appropriate.

\input{Derivative-Compute-0050.HELP.tex}

\[\lim\limits_{x\to\infty} {\left(-\frac{7}{x} + 1\right)}^{-3 \, x} + 7=\answer{e^{21} + 7}\]
\end{problem}}

%%%%%%%%%%%%%%%%%%%%%%

\latexProblemContent{
\ifVerboseLocation This is Derivative Compute Question 0050. \\ \fi
\begin{problem}

Find the limit.  Use L'H$\hat{o}$pital's rule where appropriate.

\input{Derivative-Compute-0050.HELP.tex}

\[\lim\limits_{x\to\infty} {\left(-\frac{6}{x} + 1\right)}^{7 \, x} - 16=\answer{e^{\left(-42\right)} - 16}\]
\end{problem}}

%%%%%%%%%%%%%%%%%%%%%%

\latexProblemContent{
\ifVerboseLocation This is Derivative Compute Question 0050. \\ \fi
\begin{problem}

Find the limit.  Use L'H$\hat{o}$pital's rule where appropriate.

\input{Derivative-Compute-0050.HELP.tex}

\[\lim\limits_{x\to\infty} {\left(\frac{4}{x} + 1\right)}^{5 \, x} + 5=\answer{e^{20} + 5}\]
\end{problem}}

%%%%%%%%%%%%%%%%%%%%%%

\latexProblemContent{
\ifVerboseLocation This is Derivative Compute Question 0050. \\ \fi
\begin{problem}

Find the limit.  Use L'H$\hat{o}$pital's rule where appropriate.

\input{Derivative-Compute-0050.HELP.tex}

\[\lim\limits_{x\to\infty} {\left(-\frac{10}{x} + 1\right)}^{-6 \, x} + 7=\answer{e^{60} + 7}\]
\end{problem}}

%%%%%%%%%%%%%%%%%%%%%%

\latexProblemContent{
\ifVerboseLocation This is Derivative Compute Question 0050. \\ \fi
\begin{problem}

Find the limit.  Use L'H$\hat{o}$pital's rule where appropriate.

\input{Derivative-Compute-0050.HELP.tex}

\[\lim\limits_{x\to\infty} {\left(\frac{4}{x} + 1\right)}^{3 \, x} + 6=\answer{e^{12} + 6}\]
\end{problem}}

%%%%%%%%%%%%%%%%%%%%%%

\latexProblemContent{
\ifVerboseLocation This is Derivative Compute Question 0050. \\ \fi
\begin{problem}

Find the limit.  Use L'H$\hat{o}$pital's rule where appropriate.

\input{Derivative-Compute-0050.HELP.tex}

\[\lim\limits_{x\to\infty} {\left(\frac{5}{x} + 1\right)}^{-6 \, x} + 17=\answer{e^{\left(-30\right)} + 17}\]
\end{problem}}

%%%%%%%%%%%%%%%%%%%%%%

\latexProblemContent{
\ifVerboseLocation This is Derivative Compute Question 0050. \\ \fi
\begin{problem}

Find the limit.  Use L'H$\hat{o}$pital's rule where appropriate.

\input{Derivative-Compute-0050.HELP.tex}

\[\lim\limits_{x\to\infty} {\left(\frac{9}{x} + 1\right)}^{2 \, x} + 12=\answer{e^{18} + 12}\]
\end{problem}}

%%%%%%%%%%%%%%%%%%%%%%

\latexProblemContent{
\ifVerboseLocation This is Derivative Compute Question 0050. \\ \fi
\begin{problem}

Find the limit.  Use L'H$\hat{o}$pital's rule where appropriate.

\input{Derivative-Compute-0050.HELP.tex}

\[\lim\limits_{x\to\infty} {\left(-\frac{8}{x} + 1\right)}^{8 \, x} - 19=\answer{e^{\left(-64\right)} - 19}\]
\end{problem}}

%%%%%%%%%%%%%%%%%%%%%%

\latexProblemContent{
\ifVerboseLocation This is Derivative Compute Question 0050. \\ \fi
\begin{problem}

Find the limit.  Use L'H$\hat{o}$pital's rule where appropriate.

\input{Derivative-Compute-0050.HELP.tex}

\[\lim\limits_{x\to\infty} {\left(-\frac{6}{x} + 1\right)}^{x} - 10=\answer{e^{\left(-6\right)} - 10}\]
\end{problem}}

%%%%%%%%%%%%%%%%%%%%%%

\latexProblemContent{
\ifVerboseLocation This is Derivative Compute Question 0050. \\ \fi
\begin{problem}

Find the limit.  Use L'H$\hat{o}$pital's rule where appropriate.

\input{Derivative-Compute-0050.HELP.tex}

\[\lim\limits_{x\to\infty} {\left(\frac{10}{x} + 1\right)}^{5 \, x} + 8=\answer{e^{50} + 8}\]
\end{problem}}

%%%%%%%%%%%%%%%%%%%%%%

\latexProblemContent{
\ifVerboseLocation This is Derivative Compute Question 0050. \\ \fi
\begin{problem}

Find the limit.  Use L'H$\hat{o}$pital's rule where appropriate.

\input{Derivative-Compute-0050.HELP.tex}

\[\lim\limits_{x\to\infty} {\left(\frac{4}{x} + 1\right)}^{7 \, x} + 6=\answer{e^{28} + 6}\]
\end{problem}}

%%%%%%%%%%%%%%%%%%%%%%

\latexProblemContent{
\ifVerboseLocation This is Derivative Compute Question 0050. \\ \fi
\begin{problem}

Find the limit.  Use L'H$\hat{o}$pital's rule where appropriate.

\input{Derivative-Compute-0050.HELP.tex}

\[\lim\limits_{x\to\infty} {\left(\frac{7}{x} + 1\right)}^{5 \, x} + 6=\answer{e^{35} + 6}\]
\end{problem}}

%%%%%%%%%%%%%%%%%%%%%%

\latexProblemContent{
\ifVerboseLocation This is Derivative Compute Question 0050. \\ \fi
\begin{problem}

Find the limit.  Use L'H$\hat{o}$pital's rule where appropriate.

\input{Derivative-Compute-0050.HELP.tex}

\[\lim\limits_{x\to\infty} {\left(-\frac{10}{x} + 1\right)}^{7 \, x} - 4=\answer{e^{\left(-70\right)} - 4}\]
\end{problem}}

%%%%%%%%%%%%%%%%%%%%%%

\latexProblemContent{
\ifVerboseLocation This is Derivative Compute Question 0050. \\ \fi
\begin{problem}

Find the limit.  Use L'H$\hat{o}$pital's rule where appropriate.

\input{Derivative-Compute-0050.HELP.tex}

\[\lim\limits_{x\to\infty} {\left(-\frac{2}{x} + 1\right)}^{9 \, x} + 2=\answer{e^{\left(-18\right)} + 2}\]
\end{problem}}

%%%%%%%%%%%%%%%%%%%%%%

\latexProblemContent{
\ifVerboseLocation This is Derivative Compute Question 0050. \\ \fi
\begin{problem}

Find the limit.  Use L'H$\hat{o}$pital's rule where appropriate.

\input{Derivative-Compute-0050.HELP.tex}

\[\lim\limits_{x\to\infty} {\left(-\frac{2}{x} + 1\right)}^{7 \, x} - 20=\answer{e^{\left(-14\right)} - 20}\]
\end{problem}}

%%%%%%%%%%%%%%%%%%%%%%

\latexProblemContent{
\ifVerboseLocation This is Derivative Compute Question 0050. \\ \fi
\begin{problem}

Find the limit.  Use L'H$\hat{o}$pital's rule where appropriate.

\input{Derivative-Compute-0050.HELP.tex}

\[\lim\limits_{x\to\infty} {\left(\frac{6}{x} + 1\right)}^{5 \, x} + 10=\answer{e^{30} + 10}\]
\end{problem}}

%%%%%%%%%%%%%%%%%%%%%%

\latexProblemContent{
\ifVerboseLocation This is Derivative Compute Question 0050. \\ \fi
\begin{problem}

Find the limit.  Use L'H$\hat{o}$pital's rule where appropriate.

\input{Derivative-Compute-0050.HELP.tex}

\[\lim\limits_{x\to\infty} {\left(\frac{2}{x} + 1\right)}^{3 \, x} + 18=\answer{e^{6} + 18}\]
\end{problem}}

%%%%%%%%%%%%%%%%%%%%%%

\latexProblemContent{
\ifVerboseLocation This is Derivative Compute Question 0050. \\ \fi
\begin{problem}

Find the limit.  Use L'H$\hat{o}$pital's rule where appropriate.

\input{Derivative-Compute-0050.HELP.tex}

\[\lim\limits_{x\to\infty} {\left(\frac{5}{x} + 1\right)}^{-9 \, x} - 18=\answer{e^{\left(-45\right)} - 18}\]
\end{problem}}

%%%%%%%%%%%%%%%%%%%%%%

\latexProblemContent{
\ifVerboseLocation This is Derivative Compute Question 0050. \\ \fi
\begin{problem}

Find the limit.  Use L'H$\hat{o}$pital's rule where appropriate.

\input{Derivative-Compute-0050.HELP.tex}

\[\lim\limits_{x\to\infty} {\left(-\frac{2}{x} + 1\right)}^{9 \, x} + 19=\answer{e^{\left(-18\right)} + 19}\]
\end{problem}}

%%%%%%%%%%%%%%%%%%%%%%

\latexProblemContent{
\ifVerboseLocation This is Derivative Compute Question 0050. \\ \fi
\begin{problem}

Find the limit.  Use L'H$\hat{o}$pital's rule where appropriate.

\input{Derivative-Compute-0050.HELP.tex}

\[\lim\limits_{x\to\infty} {\left(\frac{5}{x} + 1\right)}^{-9 \, x} + 2=\answer{e^{\left(-45\right)} + 2}\]
\end{problem}}

%%%%%%%%%%%%%%%%%%%%%%

\latexProblemContent{
\ifVerboseLocation This is Derivative Compute Question 0050. \\ \fi
\begin{problem}

Find the limit.  Use L'H$\hat{o}$pital's rule where appropriate.

\input{Derivative-Compute-0050.HELP.tex}

\[\lim\limits_{x\to\infty} {\left(-\frac{6}{x} + 1\right)}^{3 \, x} + 2=\answer{e^{\left(-18\right)} + 2}\]
\end{problem}}

%%%%%%%%%%%%%%%%%%%%%%

\latexProblemContent{
\ifVerboseLocation This is Derivative Compute Question 0050. \\ \fi
\begin{problem}

Find the limit.  Use L'H$\hat{o}$pital's rule where appropriate.

\input{Derivative-Compute-0050.HELP.tex}

\[\lim\limits_{x\to\infty} {\left(-\frac{6}{x} + 1\right)}^{2 \, x} + 10=\answer{e^{\left(-12\right)} + 10}\]
\end{problem}}

%%%%%%%%%%%%%%%%%%%%%%

\latexProblemContent{
\ifVerboseLocation This is Derivative Compute Question 0050. \\ \fi
\begin{problem}

Find the limit.  Use L'H$\hat{o}$pital's rule where appropriate.

\input{Derivative-Compute-0050.HELP.tex}

\[\lim\limits_{x\to\infty} {\left(\frac{2}{x} + 1\right)}^{-4 \, x} - 10=\answer{e^{\left(-8\right)} - 10}\]
\end{problem}}

%%%%%%%%%%%%%%%%%%%%%%

\latexProblemContent{
\ifVerboseLocation This is Derivative Compute Question 0050. \\ \fi
\begin{problem}

Find the limit.  Use L'H$\hat{o}$pital's rule where appropriate.

\input{Derivative-Compute-0050.HELP.tex}

\[\lim\limits_{x\to\infty} {\left(\frac{2}{x} + 1\right)}^{4 \, x} + 15=\answer{e^{8} + 15}\]
\end{problem}}

%%%%%%%%%%%%%%%%%%%%%%

\latexProblemContent{
\ifVerboseLocation This is Derivative Compute Question 0050. \\ \fi
\begin{problem}

Find the limit.  Use L'H$\hat{o}$pital's rule where appropriate.

\input{Derivative-Compute-0050.HELP.tex}

\[\lim\limits_{x\to\infty} {\left(-\frac{9}{x} + 1\right)}^{-8 \, x} + 4=\answer{e^{72} + 4}\]
\end{problem}}

%%%%%%%%%%%%%%%%%%%%%%

\latexProblemContent{
\ifVerboseLocation This is Derivative Compute Question 0050. \\ \fi
\begin{problem}

Find the limit.  Use L'H$\hat{o}$pital's rule where appropriate.

\input{Derivative-Compute-0050.HELP.tex}

\[\lim\limits_{x\to\infty} {\left(-\frac{5}{x} + 1\right)}^{3 \, x} - 11=\answer{e^{\left(-15\right)} - 11}\]
\end{problem}}

%%%%%%%%%%%%%%%%%%%%%%

\latexProblemContent{
\ifVerboseLocation This is Derivative Compute Question 0050. \\ \fi
\begin{problem}

Find the limit.  Use L'H$\hat{o}$pital's rule where appropriate.

\input{Derivative-Compute-0050.HELP.tex}

\[\lim\limits_{x\to\infty} {\left(-\frac{4}{x} + 1\right)}^{8 \, x} + 20=\answer{e^{\left(-32\right)} + 20}\]
\end{problem}}

%%%%%%%%%%%%%%%%%%%%%%

\latexProblemContent{
\ifVerboseLocation This is Derivative Compute Question 0050. \\ \fi
\begin{problem}

Find the limit.  Use L'H$\hat{o}$pital's rule where appropriate.

\input{Derivative-Compute-0050.HELP.tex}

\[\lim\limits_{x\to\infty} {\left(\frac{4}{x} + 1\right)}^{4 \, x} + 8=\answer{e^{16} + 8}\]
\end{problem}}

%%%%%%%%%%%%%%%%%%%%%%

\latexProblemContent{
\ifVerboseLocation This is Derivative Compute Question 0050. \\ \fi
\begin{problem}

Find the limit.  Use L'H$\hat{o}$pital's rule where appropriate.

\input{Derivative-Compute-0050.HELP.tex}

\[\lim\limits_{x\to\infty} {\left(\frac{10}{x} + 1\right)}^{10 \, x} + 9=\answer{e^{100} + 9}\]
\end{problem}}

%%%%%%%%%%%%%%%%%%%%%%

\latexProblemContent{
\ifVerboseLocation This is Derivative Compute Question 0050. \\ \fi
\begin{problem}

Find the limit.  Use L'H$\hat{o}$pital's rule where appropriate.

\input{Derivative-Compute-0050.HELP.tex}

\[\lim\limits_{x\to\infty} {\left(\frac{9}{x} + 1\right)}^{-3 \, x} - 3=\answer{e^{\left(-27\right)} - 3}\]
\end{problem}}

%%%%%%%%%%%%%%%%%%%%%%

\latexProblemContent{
\ifVerboseLocation This is Derivative Compute Question 0050. \\ \fi
\begin{problem}

Find the limit.  Use L'H$\hat{o}$pital's rule where appropriate.

\input{Derivative-Compute-0050.HELP.tex}

\[\lim\limits_{x\to\infty} {\left(\frac{5}{x} + 1\right)}^{10 \, x} + 12=\answer{e^{50} + 12}\]
\end{problem}}

%%%%%%%%%%%%%%%%%%%%%%

\latexProblemContent{
\ifVerboseLocation This is Derivative Compute Question 0050. \\ \fi
\begin{problem}

Find the limit.  Use L'H$\hat{o}$pital's rule where appropriate.

\input{Derivative-Compute-0050.HELP.tex}

\[\lim\limits_{x\to\infty} {\left(\frac{2}{x} + 1\right)}^{-9 \, x} + 19=\answer{e^{\left(-18\right)} + 19}\]
\end{problem}}

%%%%%%%%%%%%%%%%%%%%%%

\latexProblemContent{
\ifVerboseLocation This is Derivative Compute Question 0050. \\ \fi
\begin{problem}

Find the limit.  Use L'H$\hat{o}$pital's rule where appropriate.

\input{Derivative-Compute-0050.HELP.tex}

\[\lim\limits_{x\to\infty} {\left(-\frac{5}{x} + 1\right)}^{2 \, x} - 14=\answer{e^{\left(-10\right)} - 14}\]
\end{problem}}

%%%%%%%%%%%%%%%%%%%%%%

\latexProblemContent{
\ifVerboseLocation This is Derivative Compute Question 0050. \\ \fi
\begin{problem}

Find the limit.  Use L'H$\hat{o}$pital's rule where appropriate.

\input{Derivative-Compute-0050.HELP.tex}

\[\lim\limits_{x\to\infty} {\left(-\frac{10}{x} + 1\right)}^{-10 \, x} - 3=\answer{e^{100} - 3}\]
\end{problem}}

%%%%%%%%%%%%%%%%%%%%%%

\latexProblemContent{
\ifVerboseLocation This is Derivative Compute Question 0050. \\ \fi
\begin{problem}

Find the limit.  Use L'H$\hat{o}$pital's rule where appropriate.

\input{Derivative-Compute-0050.HELP.tex}

\[\lim\limits_{x\to\infty} {\left(\frac{2}{x} + 1\right)}^{-x} - 10=\answer{e^{\left(-2\right)} - 10}\]
\end{problem}}

%%%%%%%%%%%%%%%%%%%%%%

\latexProblemContent{
\ifVerboseLocation This is Derivative Compute Question 0050. \\ \fi
\begin{problem}

Find the limit.  Use L'H$\hat{o}$pital's rule where appropriate.

\input{Derivative-Compute-0050.HELP.tex}

\[\lim\limits_{x\to\infty} {\left(\frac{8}{x} + 1\right)}^{3 \, x} - 5=\answer{e^{24} - 5}\]
\end{problem}}

%%%%%%%%%%%%%%%%%%%%%%

\latexProblemContent{
\ifVerboseLocation This is Derivative Compute Question 0050. \\ \fi
\begin{problem}

Find the limit.  Use L'H$\hat{o}$pital's rule where appropriate.

\input{Derivative-Compute-0050.HELP.tex}

\[\lim\limits_{x\to\infty} {\left(\frac{1}{x} + 1\right)}^{8 \, x} - 13=\answer{e^{8} - 13}\]
\end{problem}}

%%%%%%%%%%%%%%%%%%%%%%

\latexProblemContent{
\ifVerboseLocation This is Derivative Compute Question 0050. \\ \fi
\begin{problem}

Find the limit.  Use L'H$\hat{o}$pital's rule where appropriate.

\input{Derivative-Compute-0050.HELP.tex}

\[\lim\limits_{x\to\infty} {\left(\frac{1}{x} + 1\right)}^{7 \, x} - 1=\answer{e^{7} - 1}\]
\end{problem}}

%%%%%%%%%%%%%%%%%%%%%%

\latexProblemContent{
\ifVerboseLocation This is Derivative Compute Question 0050. \\ \fi
\begin{problem}

Find the limit.  Use L'H$\hat{o}$pital's rule where appropriate.

\input{Derivative-Compute-0050.HELP.tex}

\[\lim\limits_{x\to\infty} {\left(-\frac{8}{x} + 1\right)}^{-10 \, x} + 18=\answer{e^{80} + 18}\]
\end{problem}}

%%%%%%%%%%%%%%%%%%%%%%

\latexProblemContent{
\ifVerboseLocation This is Derivative Compute Question 0050. \\ \fi
\begin{problem}

Find the limit.  Use L'H$\hat{o}$pital's rule where appropriate.

\input{Derivative-Compute-0050.HELP.tex}

\[\lim\limits_{x\to\infty} {\left(-\frac{3}{x} + 1\right)}^{-x} + 15=\answer{e^{3} + 15}\]
\end{problem}}

%%%%%%%%%%%%%%%%%%%%%%

\latexProblemContent{
\ifVerboseLocation This is Derivative Compute Question 0050. \\ \fi
\begin{problem}

Find the limit.  Use L'H$\hat{o}$pital's rule where appropriate.

\input{Derivative-Compute-0050.HELP.tex}

\[\lim\limits_{x\to\infty} {\left(\frac{7}{x} + 1\right)}^{x} - 18=\answer{e^{7} - 18}\]
\end{problem}}

%%%%%%%%%%%%%%%%%%%%%%

\latexProblemContent{
\ifVerboseLocation This is Derivative Compute Question 0050. \\ \fi
\begin{problem}

Find the limit.  Use L'H$\hat{o}$pital's rule where appropriate.

\input{Derivative-Compute-0050.HELP.tex}

\[\lim\limits_{x\to\infty} {\left(-\frac{7}{x} + 1\right)}^{10 \, x} - 11=\answer{e^{\left(-70\right)} - 11}\]
\end{problem}}

%%%%%%%%%%%%%%%%%%%%%%

\latexProblemContent{
\ifVerboseLocation This is Derivative Compute Question 0050. \\ \fi
\begin{problem}

Find the limit.  Use L'H$\hat{o}$pital's rule where appropriate.

\input{Derivative-Compute-0050.HELP.tex}

\[\lim\limits_{x\to\infty} {\left(\frac{2}{x} + 1\right)}^{4 \, x} + 16=\answer{e^{8} + 16}\]
\end{problem}}

%%%%%%%%%%%%%%%%%%%%%%

\latexProblemContent{
\ifVerboseLocation This is Derivative Compute Question 0050. \\ \fi
\begin{problem}

Find the limit.  Use L'H$\hat{o}$pital's rule where appropriate.

\input{Derivative-Compute-0050.HELP.tex}

\[\lim\limits_{x\to\infty} {\left(\frac{9}{x} + 1\right)}^{-4 \, x} + 20=\answer{e^{\left(-36\right)} + 20}\]
\end{problem}}

%%%%%%%%%%%%%%%%%%%%%%

\latexProblemContent{
\ifVerboseLocation This is Derivative Compute Question 0050. \\ \fi
\begin{problem}

Find the limit.  Use L'H$\hat{o}$pital's rule where appropriate.

\input{Derivative-Compute-0050.HELP.tex}

\[\lim\limits_{x\to\infty} {\left(\frac{7}{x} + 1\right)}^{-7 \, x} - 12=\answer{e^{\left(-49\right)} - 12}\]
\end{problem}}

%%%%%%%%%%%%%%%%%%%%%%

\latexProblemContent{
\ifVerboseLocation This is Derivative Compute Question 0050. \\ \fi
\begin{problem}

Find the limit.  Use L'H$\hat{o}$pital's rule where appropriate.

\input{Derivative-Compute-0050.HELP.tex}

\[\lim\limits_{x\to\infty} {\left(\frac{7}{x} + 1\right)}^{9 \, x} + 18=\answer{e^{63} + 18}\]
\end{problem}}

%%%%%%%%%%%%%%%%%%%%%%

\latexProblemContent{
\ifVerboseLocation This is Derivative Compute Question 0050. \\ \fi
\begin{problem}

Find the limit.  Use L'H$\hat{o}$pital's rule where appropriate.

\input{Derivative-Compute-0050.HELP.tex}

\[\lim\limits_{x\to\infty} {\left(\frac{9}{x} + 1\right)}^{-7 \, x} + 17=\answer{e^{\left(-63\right)} + 17}\]
\end{problem}}

%%%%%%%%%%%%%%%%%%%%%%

\latexProblemContent{
\ifVerboseLocation This is Derivative Compute Question 0050. \\ \fi
\begin{problem}

Find the limit.  Use L'H$\hat{o}$pital's rule where appropriate.

\input{Derivative-Compute-0050.HELP.tex}

\[\lim\limits_{x\to\infty} {\left(\frac{6}{x} + 1\right)}^{-5 \, x} - 17=\answer{e^{\left(-30\right)} - 17}\]
\end{problem}}

%%%%%%%%%%%%%%%%%%%%%%

\latexProblemContent{
\ifVerboseLocation This is Derivative Compute Question 0050. \\ \fi
\begin{problem}

Find the limit.  Use L'H$\hat{o}$pital's rule where appropriate.

\input{Derivative-Compute-0050.HELP.tex}

\[\lim\limits_{x\to\infty} {\left(\frac{3}{x} + 1\right)}^{10 \, x} - 1=\answer{e^{30} - 1}\]
\end{problem}}

%%%%%%%%%%%%%%%%%%%%%%

\latexProblemContent{
\ifVerboseLocation This is Derivative Compute Question 0050. \\ \fi
\begin{problem}

Find the limit.  Use L'H$\hat{o}$pital's rule where appropriate.

\input{Derivative-Compute-0050.HELP.tex}

\[\lim\limits_{x\to\infty} {\left(-\frac{7}{x} + 1\right)}^{-9 \, x} + 20=\answer{e^{63} + 20}\]
\end{problem}}

%%%%%%%%%%%%%%%%%%%%%%

\latexProblemContent{
\ifVerboseLocation This is Derivative Compute Question 0050. \\ \fi
\begin{problem}

Find the limit.  Use L'H$\hat{o}$pital's rule where appropriate.

\input{Derivative-Compute-0050.HELP.tex}

\[\lim\limits_{x\to\infty} {\left(-\frac{8}{x} + 1\right)}^{2 \, x} + 12=\answer{e^{\left(-16\right)} + 12}\]
\end{problem}}

%%%%%%%%%%%%%%%%%%%%%%

\latexProblemContent{
\ifVerboseLocation This is Derivative Compute Question 0050. \\ \fi
\begin{problem}

Find the limit.  Use L'H$\hat{o}$pital's rule where appropriate.

\input{Derivative-Compute-0050.HELP.tex}

\[\lim\limits_{x\to\infty} {\left(\frac{10}{x} + 1\right)}^{8 \, x} + 2=\answer{e^{80} + 2}\]
\end{problem}}

%%%%%%%%%%%%%%%%%%%%%%

\latexProblemContent{
\ifVerboseLocation This is Derivative Compute Question 0050. \\ \fi
\begin{problem}

Find the limit.  Use L'H$\hat{o}$pital's rule where appropriate.

\input{Derivative-Compute-0050.HELP.tex}

\[\lim\limits_{x\to\infty} {\left(\frac{1}{x} + 1\right)}^{-3 \, x} - 19=\answer{e^{\left(-3\right)} - 19}\]
\end{problem}}

%%%%%%%%%%%%%%%%%%%%%%

\latexProblemContent{
\ifVerboseLocation This is Derivative Compute Question 0050. \\ \fi
\begin{problem}

Find the limit.  Use L'H$\hat{o}$pital's rule where appropriate.

\input{Derivative-Compute-0050.HELP.tex}

\[\lim\limits_{x\to\infty} {\left(-\frac{7}{x} + 1\right)}^{4 \, x} + 5=\answer{e^{\left(-28\right)} + 5}\]
\end{problem}}

%%%%%%%%%%%%%%%%%%%%%%

\latexProblemContent{
\ifVerboseLocation This is Derivative Compute Question 0050. \\ \fi
\begin{problem}

Find the limit.  Use L'H$\hat{o}$pital's rule where appropriate.

\input{Derivative-Compute-0050.HELP.tex}

\[\lim\limits_{x\to\infty} {\left(-\frac{5}{x} + 1\right)}^{-3 \, x} + 8=\answer{e^{15} + 8}\]
\end{problem}}

%%%%%%%%%%%%%%%%%%%%%%

\latexProblemContent{
\ifVerboseLocation This is Derivative Compute Question 0050. \\ \fi
\begin{problem}

Find the limit.  Use L'H$\hat{o}$pital's rule where appropriate.

\input{Derivative-Compute-0050.HELP.tex}

\[\lim\limits_{x\to\infty} {\left(-\frac{6}{x} + 1\right)}^{6 \, x} + 5=\answer{e^{\left(-36\right)} + 5}\]
\end{problem}}

%%%%%%%%%%%%%%%%%%%%%%

\latexProblemContent{
\ifVerboseLocation This is Derivative Compute Question 0050. \\ \fi
\begin{problem}

Find the limit.  Use L'H$\hat{o}$pital's rule where appropriate.

\input{Derivative-Compute-0050.HELP.tex}

\[\lim\limits_{x\to\infty} {\left(\frac{9}{x} + 1\right)}^{-9 \, x} - 13=\answer{e^{\left(-81\right)} - 13}\]
\end{problem}}

%%%%%%%%%%%%%%%%%%%%%%

\latexProblemContent{
\ifVerboseLocation This is Derivative Compute Question 0050. \\ \fi
\begin{problem}

Find the limit.  Use L'H$\hat{o}$pital's rule where appropriate.

\input{Derivative-Compute-0050.HELP.tex}

\[\lim\limits_{x\to\infty} {\left(\frac{6}{x} + 1\right)}^{-6 \, x} + 18=\answer{e^{\left(-36\right)} + 18}\]
\end{problem}}

%%%%%%%%%%%%%%%%%%%%%%

\latexProblemContent{
\ifVerboseLocation This is Derivative Compute Question 0050. \\ \fi
\begin{problem}

Find the limit.  Use L'H$\hat{o}$pital's rule where appropriate.

\input{Derivative-Compute-0050.HELP.tex}

\[\lim\limits_{x\to\infty} {\left(-\frac{4}{x} + 1\right)}^{x} - 7=\answer{e^{\left(-4\right)} - 7}\]
\end{problem}}

%%%%%%%%%%%%%%%%%%%%%%

\latexProblemContent{
\ifVerboseLocation This is Derivative Compute Question 0050. \\ \fi
\begin{problem}

Find the limit.  Use L'H$\hat{o}$pital's rule where appropriate.

\input{Derivative-Compute-0050.HELP.tex}

\[\lim\limits_{x\to\infty} {\left(\frac{5}{x} + 1\right)}^{2 \, x} + 17=\answer{e^{10} + 17}\]
\end{problem}}

%%%%%%%%%%%%%%%%%%%%%%

\latexProblemContent{
\ifVerboseLocation This is Derivative Compute Question 0050. \\ \fi
\begin{problem}

Find the limit.  Use L'H$\hat{o}$pital's rule where appropriate.

\input{Derivative-Compute-0050.HELP.tex}

\[\lim\limits_{x\to\infty} {\left(\frac{4}{x} + 1\right)}^{10 \, x} - 1=\answer{e^{40} - 1}\]
\end{problem}}

%%%%%%%%%%%%%%%%%%%%%%

\latexProblemContent{
\ifVerboseLocation This is Derivative Compute Question 0050. \\ \fi
\begin{problem}

Find the limit.  Use L'H$\hat{o}$pital's rule where appropriate.

\input{Derivative-Compute-0050.HELP.tex}

\[\lim\limits_{x\to\infty} {\left(-\frac{3}{x} + 1\right)}^{-2 \, x} + 4=\answer{e^{6} + 4}\]
\end{problem}}

%%%%%%%%%%%%%%%%%%%%%%

\latexProblemContent{
\ifVerboseLocation This is Derivative Compute Question 0050. \\ \fi
\begin{problem}

Find the limit.  Use L'H$\hat{o}$pital's rule where appropriate.

\input{Derivative-Compute-0050.HELP.tex}

\[\lim\limits_{x\to\infty} {\left(-\frac{3}{x} + 1\right)}^{6 \, x} - 3=\answer{e^{\left(-18\right)} - 3}\]
\end{problem}}

%%%%%%%%%%%%%%%%%%%%%%

\latexProblemContent{
\ifVerboseLocation This is Derivative Compute Question 0050. \\ \fi
\begin{problem}

Find the limit.  Use L'H$\hat{o}$pital's rule where appropriate.

\input{Derivative-Compute-0050.HELP.tex}

\[\lim\limits_{x\to\infty} {\left(\frac{9}{x} + 1\right)}^{-5 \, x} + 14=\answer{e^{\left(-45\right)} + 14}\]
\end{problem}}

%%%%%%%%%%%%%%%%%%%%%%

\latexProblemContent{
\ifVerboseLocation This is Derivative Compute Question 0050. \\ \fi
\begin{problem}

Find the limit.  Use L'H$\hat{o}$pital's rule where appropriate.

\input{Derivative-Compute-0050.HELP.tex}

\[\lim\limits_{x\to\infty} {\left(\frac{7}{x} + 1\right)}^{-4 \, x} + 7=\answer{e^{\left(-28\right)} + 7}\]
\end{problem}}

%%%%%%%%%%%%%%%%%%%%%%

\latexProblemContent{
\ifVerboseLocation This is Derivative Compute Question 0050. \\ \fi
\begin{problem}

Find the limit.  Use L'H$\hat{o}$pital's rule where appropriate.

\input{Derivative-Compute-0050.HELP.tex}

\[\lim\limits_{x\to\infty} {\left(\frac{6}{x} + 1\right)}^{9 \, x} - 3=\answer{e^{54} - 3}\]
\end{problem}}

%%%%%%%%%%%%%%%%%%%%%%

\latexProblemContent{
\ifVerboseLocation This is Derivative Compute Question 0050. \\ \fi
\begin{problem}

Find the limit.  Use L'H$\hat{o}$pital's rule where appropriate.

\input{Derivative-Compute-0050.HELP.tex}

\[\lim\limits_{x\to\infty} {\left(\frac{3}{x} + 1\right)}^{9 \, x} + 6=\answer{e^{27} + 6}\]
\end{problem}}

%%%%%%%%%%%%%%%%%%%%%%

\latexProblemContent{
\ifVerboseLocation This is Derivative Compute Question 0050. \\ \fi
\begin{problem}

Find the limit.  Use L'H$\hat{o}$pital's rule where appropriate.

\input{Derivative-Compute-0050.HELP.tex}

\[\lim\limits_{x\to\infty} {\left(\frac{6}{x} + 1\right)}^{7 \, x} + 7=\answer{e^{42} + 7}\]
\end{problem}}

%%%%%%%%%%%%%%%%%%%%%%

\latexProblemContent{
\ifVerboseLocation This is Derivative Compute Question 0050. \\ \fi
\begin{problem}

Find the limit.  Use L'H$\hat{o}$pital's rule where appropriate.

\input{Derivative-Compute-0050.HELP.tex}

\[\lim\limits_{x\to\infty} {\left(\frac{2}{x} + 1\right)}^{10 \, x} + 3=\answer{e^{20} + 3}\]
\end{problem}}

%%%%%%%%%%%%%%%%%%%%%%

\latexProblemContent{
\ifVerboseLocation This is Derivative Compute Question 0050. \\ \fi
\begin{problem}

Find the limit.  Use L'H$\hat{o}$pital's rule where appropriate.

\input{Derivative-Compute-0050.HELP.tex}

\[\lim\limits_{x\to\infty} {\left(\frac{10}{x} + 1\right)}^{10 \, x} + 6=\answer{e^{100} + 6}\]
\end{problem}}

%%%%%%%%%%%%%%%%%%%%%%

\latexProblemContent{
\ifVerboseLocation This is Derivative Compute Question 0050. \\ \fi
\begin{problem}

Find the limit.  Use L'H$\hat{o}$pital's rule where appropriate.

\input{Derivative-Compute-0050.HELP.tex}

\[\lim\limits_{x\to\infty} {\left(\frac{6}{x} + 1\right)}^{-x} + 3=\answer{e^{\left(-6\right)} + 3}\]
\end{problem}}

%%%%%%%%%%%%%%%%%%%%%%

\latexProblemContent{
\ifVerboseLocation This is Derivative Compute Question 0050. \\ \fi
\begin{problem}

Find the limit.  Use L'H$\hat{o}$pital's rule where appropriate.

\input{Derivative-Compute-0050.HELP.tex}

\[\lim\limits_{x\to\infty} {\left(-\frac{4}{x} + 1\right)}^{-7 \, x} + 5=\answer{e^{28} + 5}\]
\end{problem}}

%%%%%%%%%%%%%%%%%%%%%%

\latexProblemContent{
\ifVerboseLocation This is Derivative Compute Question 0050. \\ \fi
\begin{problem}

Find the limit.  Use L'H$\hat{o}$pital's rule where appropriate.

\input{Derivative-Compute-0050.HELP.tex}

\[\lim\limits_{x\to\infty} {\left(\frac{6}{x} + 1\right)}^{-8 \, x} + 9=\answer{e^{\left(-48\right)} + 9}\]
\end{problem}}

%%%%%%%%%%%%%%%%%%%%%%

\latexProblemContent{
\ifVerboseLocation This is Derivative Compute Question 0050. \\ \fi
\begin{problem}

Find the limit.  Use L'H$\hat{o}$pital's rule where appropriate.

\input{Derivative-Compute-0050.HELP.tex}

\[\lim\limits_{x\to\infty} {\left(-\frac{9}{x} + 1\right)}^{x} + 20=\answer{e^{\left(-9\right)} + 20}\]
\end{problem}}

%%%%%%%%%%%%%%%%%%%%%%

\latexProblemContent{
\ifVerboseLocation This is Derivative Compute Question 0050. \\ \fi
\begin{problem}

Find the limit.  Use L'H$\hat{o}$pital's rule where appropriate.

\input{Derivative-Compute-0050.HELP.tex}

\[\lim\limits_{x\to\infty} {\left(-\frac{1}{x} + 1\right)}^{4 \, x} - 4=\answer{e^{\left(-4\right)} - 4}\]
\end{problem}}

%%%%%%%%%%%%%%%%%%%%%%

\latexProblemContent{
\ifVerboseLocation This is Derivative Compute Question 0050. \\ \fi
\begin{problem}

Find the limit.  Use L'H$\hat{o}$pital's rule where appropriate.

\input{Derivative-Compute-0050.HELP.tex}

\[\lim\limits_{x\to\infty} {\left(-\frac{2}{x} + 1\right)}^{6 \, x} + 8=\answer{e^{\left(-12\right)} + 8}\]
\end{problem}}

%%%%%%%%%%%%%%%%%%%%%%

\latexProblemContent{
\ifVerboseLocation This is Derivative Compute Question 0050. \\ \fi
\begin{problem}

Find the limit.  Use L'H$\hat{o}$pital's rule where appropriate.

\input{Derivative-Compute-0050.HELP.tex}

\[\lim\limits_{x\to\infty} {\left(\frac{2}{x} + 1\right)}^{x} + 15=\answer{e^{2} + 15}\]
\end{problem}}

%%%%%%%%%%%%%%%%%%%%%%

\latexProblemContent{
\ifVerboseLocation This is Derivative Compute Question 0050. \\ \fi
\begin{problem}

Find the limit.  Use L'H$\hat{o}$pital's rule where appropriate.

\input{Derivative-Compute-0050.HELP.tex}

\[\lim\limits_{x\to\infty} {\left(\frac{1}{x} + 1\right)}^{8 \, x} - 5=\answer{e^{8} - 5}\]
\end{problem}}

%%%%%%%%%%%%%%%%%%%%%%

\latexProblemContent{
\ifVerboseLocation This is Derivative Compute Question 0050. \\ \fi
\begin{problem}

Find the limit.  Use L'H$\hat{o}$pital's rule where appropriate.

\input{Derivative-Compute-0050.HELP.tex}

\[\lim\limits_{x\to\infty} {\left(\frac{5}{x} + 1\right)}^{4 \, x} - 3=\answer{e^{20} - 3}\]
\end{problem}}

%%%%%%%%%%%%%%%%%%%%%%

\latexProblemContent{
\ifVerboseLocation This is Derivative Compute Question 0050. \\ \fi
\begin{problem}

Find the limit.  Use L'H$\hat{o}$pital's rule where appropriate.

\input{Derivative-Compute-0050.HELP.tex}

\[\lim\limits_{x\to\infty} {\left(\frac{1}{x} + 1\right)}^{-x} + 15=\answer{e^{\left(-1\right)} + 15}\]
\end{problem}}

%%%%%%%%%%%%%%%%%%%%%%

\latexProblemContent{
\ifVerboseLocation This is Derivative Compute Question 0050. \\ \fi
\begin{problem}

Find the limit.  Use L'H$\hat{o}$pital's rule where appropriate.

\input{Derivative-Compute-0050.HELP.tex}

\[\lim\limits_{x\to\infty} {\left(-\frac{4}{x} + 1\right)}^{6 \, x} + 16=\answer{e^{\left(-24\right)} + 16}\]
\end{problem}}

%%%%%%%%%%%%%%%%%%%%%%

\latexProblemContent{
\ifVerboseLocation This is Derivative Compute Question 0050. \\ \fi
\begin{problem}

Find the limit.  Use L'H$\hat{o}$pital's rule where appropriate.

\input{Derivative-Compute-0050.HELP.tex}

\[\lim\limits_{x\to\infty} {\left(-\frac{6}{x} + 1\right)}^{-3 \, x} + 14=\answer{e^{18} + 14}\]
\end{problem}}

%%%%%%%%%%%%%%%%%%%%%%

\latexProblemContent{
\ifVerboseLocation This is Derivative Compute Question 0050. \\ \fi
\begin{problem}

Find the limit.  Use L'H$\hat{o}$pital's rule where appropriate.

\input{Derivative-Compute-0050.HELP.tex}

\[\lim\limits_{x\to\infty} {\left(\frac{7}{x} + 1\right)}^{10 \, x} + 11=\answer{e^{70} + 11}\]
\end{problem}}

%%%%%%%%%%%%%%%%%%%%%%

\latexProblemContent{
\ifVerboseLocation This is Derivative Compute Question 0050. \\ \fi
\begin{problem}

Find the limit.  Use L'H$\hat{o}$pital's rule where appropriate.

\input{Derivative-Compute-0050.HELP.tex}

\[\lim\limits_{x\to\infty} {\left(-\frac{1}{x} + 1\right)}^{-9 \, x} - 7=\answer{e^{9} - 7}\]
\end{problem}}

%%%%%%%%%%%%%%%%%%%%%%

\latexProblemContent{
\ifVerboseLocation This is Derivative Compute Question 0050. \\ \fi
\begin{problem}

Find the limit.  Use L'H$\hat{o}$pital's rule where appropriate.

\input{Derivative-Compute-0050.HELP.tex}

\[\lim\limits_{x\to\infty} {\left(-\frac{1}{x} + 1\right)}^{-3 \, x} - 4=\answer{e^{3} - 4}\]
\end{problem}}

%%%%%%%%%%%%%%%%%%%%%%

\latexProblemContent{
\ifVerboseLocation This is Derivative Compute Question 0050. \\ \fi
\begin{problem}

Find the limit.  Use L'H$\hat{o}$pital's rule where appropriate.

\input{Derivative-Compute-0050.HELP.tex}

\[\lim\limits_{x\to\infty} {\left(-\frac{4}{x} + 1\right)}^{-8 \, x} + 15=\answer{e^{32} + 15}\]
\end{problem}}

%%%%%%%%%%%%%%%%%%%%%%

\latexProblemContent{
\ifVerboseLocation This is Derivative Compute Question 0050. \\ \fi
\begin{problem}

Find the limit.  Use L'H$\hat{o}$pital's rule where appropriate.

\input{Derivative-Compute-0050.HELP.tex}

\[\lim\limits_{x\to\infty} {\left(-\frac{8}{x} + 1\right)}^{10 \, x} - 6=\answer{e^{\left(-80\right)} - 6}\]
\end{problem}}

%%%%%%%%%%%%%%%%%%%%%%

\latexProblemContent{
\ifVerboseLocation This is Derivative Compute Question 0050. \\ \fi
\begin{problem}

Find the limit.  Use L'H$\hat{o}$pital's rule where appropriate.

\input{Derivative-Compute-0050.HELP.tex}

\[\lim\limits_{x\to\infty} {\left(\frac{5}{x} + 1\right)}^{-x} + 13=\answer{e^{\left(-5\right)} + 13}\]
\end{problem}}

%%%%%%%%%%%%%%%%%%%%%%

\latexProblemContent{
\ifVerboseLocation This is Derivative Compute Question 0050. \\ \fi
\begin{problem}

Find the limit.  Use L'H$\hat{o}$pital's rule where appropriate.

\input{Derivative-Compute-0050.HELP.tex}

\[\lim\limits_{x\to\infty} {\left(-\frac{5}{x} + 1\right)}^{5 \, x} - 19=\answer{e^{\left(-25\right)} - 19}\]
\end{problem}}

%%%%%%%%%%%%%%%%%%%%%%

\latexProblemContent{
\ifVerboseLocation This is Derivative Compute Question 0050. \\ \fi
\begin{problem}

Find the limit.  Use L'H$\hat{o}$pital's rule where appropriate.

\input{Derivative-Compute-0050.HELP.tex}

\[\lim\limits_{x\to\infty} {\left(\frac{4}{x} + 1\right)}^{-9 \, x} + 1=\answer{e^{\left(-36\right)} + 1}\]
\end{problem}}

%%%%%%%%%%%%%%%%%%%%%%

\latexProblemContent{
\ifVerboseLocation This is Derivative Compute Question 0050. \\ \fi
\begin{problem}

Find the limit.  Use L'H$\hat{o}$pital's rule where appropriate.

\input{Derivative-Compute-0050.HELP.tex}

\[\lim\limits_{x\to\infty} {\left(\frac{7}{x} + 1\right)}^{-3 \, x} + 4=\answer{e^{\left(-21\right)} + 4}\]
\end{problem}}

%%%%%%%%%%%%%%%%%%%%%%

\latexProblemContent{
\ifVerboseLocation This is Derivative Compute Question 0050. \\ \fi
\begin{problem}

Find the limit.  Use L'H$\hat{o}$pital's rule where appropriate.

\input{Derivative-Compute-0050.HELP.tex}

\[\lim\limits_{x\to\infty} {\left(-\frac{5}{x} + 1\right)}^{3 \, x} - 6=\answer{e^{\left(-15\right)} - 6}\]
\end{problem}}

%%%%%%%%%%%%%%%%%%%%%%

\latexProblemContent{
\ifVerboseLocation This is Derivative Compute Question 0050. \\ \fi
\begin{problem}

Find the limit.  Use L'H$\hat{o}$pital's rule where appropriate.

\input{Derivative-Compute-0050.HELP.tex}

\[\lim\limits_{x\to\infty} {\left(-\frac{9}{x} + 1\right)}^{-2 \, x}=\answer{e^{18}}\]
\end{problem}}

%%%%%%%%%%%%%%%%%%%%%%

\latexProblemContent{
\ifVerboseLocation This is Derivative Compute Question 0050. \\ \fi
\begin{problem}

Find the limit.  Use L'H$\hat{o}$pital's rule where appropriate.

\input{Derivative-Compute-0050.HELP.tex}

\[\lim\limits_{x\to\infty} {\left(-\frac{9}{x} + 1\right)}^{-10 \, x} - 7=\answer{e^{90} - 7}\]
\end{problem}}

%%%%%%%%%%%%%%%%%%%%%%

\latexProblemContent{
\ifVerboseLocation This is Derivative Compute Question 0050. \\ \fi
\begin{problem}

Find the limit.  Use L'H$\hat{o}$pital's rule where appropriate.

\input{Derivative-Compute-0050.HELP.tex}

\[\lim\limits_{x\to\infty} {\left(\frac{2}{x} + 1\right)}^{-7 \, x} + 5=\answer{e^{\left(-14\right)} + 5}\]
\end{problem}}

%%%%%%%%%%%%%%%%%%%%%%

\latexProblemContent{
\ifVerboseLocation This is Derivative Compute Question 0050. \\ \fi
\begin{problem}

Find the limit.  Use L'H$\hat{o}$pital's rule where appropriate.

\input{Derivative-Compute-0050.HELP.tex}

\[\lim\limits_{x\to\infty} {\left(\frac{7}{x} + 1\right)}^{5 \, x} - 19=\answer{e^{35} - 19}\]
\end{problem}}

%%%%%%%%%%%%%%%%%%%%%%

\latexProblemContent{
\ifVerboseLocation This is Derivative Compute Question 0050. \\ \fi
\begin{problem}

Find the limit.  Use L'H$\hat{o}$pital's rule where appropriate.

\input{Derivative-Compute-0050.HELP.tex}

\[\lim\limits_{x\to\infty} {\left(-\frac{10}{x} + 1\right)}^{5 \, x} - 4=\answer{e^{\left(-50\right)} - 4}\]
\end{problem}}

%%%%%%%%%%%%%%%%%%%%%%

\latexProblemContent{
\ifVerboseLocation This is Derivative Compute Question 0050. \\ \fi
\begin{problem}

Find the limit.  Use L'H$\hat{o}$pital's rule where appropriate.

\input{Derivative-Compute-0050.HELP.tex}

\[\lim\limits_{x\to\infty} {\left(\frac{9}{x} + 1\right)}^{-7 \, x} - 15=\answer{e^{\left(-63\right)} - 15}\]
\end{problem}}

%%%%%%%%%%%%%%%%%%%%%%

\latexProblemContent{
\ifVerboseLocation This is Derivative Compute Question 0050. \\ \fi
\begin{problem}

Find the limit.  Use L'H$\hat{o}$pital's rule where appropriate.

\input{Derivative-Compute-0050.HELP.tex}

\[\lim\limits_{x\to\infty} {\left(-\frac{6}{x} + 1\right)}^{3 \, x} + 1=\answer{e^{\left(-18\right)} + 1}\]
\end{problem}}

%%%%%%%%%%%%%%%%%%%%%%

\latexProblemContent{
\ifVerboseLocation This is Derivative Compute Question 0050. \\ \fi
\begin{problem}

Find the limit.  Use L'H$\hat{o}$pital's rule where appropriate.

\input{Derivative-Compute-0050.HELP.tex}

\[\lim\limits_{x\to\infty} {\left(-\frac{4}{x} + 1\right)}^{x} + 20=\answer{e^{\left(-4\right)} + 20}\]
\end{problem}}

%%%%%%%%%%%%%%%%%%%%%%

\latexProblemContent{
\ifVerboseLocation This is Derivative Compute Question 0050. \\ \fi
\begin{problem}

Find the limit.  Use L'H$\hat{o}$pital's rule where appropriate.

\input{Derivative-Compute-0050.HELP.tex}

\[\lim\limits_{x\to\infty} {\left(-\frac{4}{x} + 1\right)}^{-6 \, x} - 5=\answer{e^{24} - 5}\]
\end{problem}}

%%%%%%%%%%%%%%%%%%%%%%

\latexProblemContent{
\ifVerboseLocation This is Derivative Compute Question 0050. \\ \fi
\begin{problem}

Find the limit.  Use L'H$\hat{o}$pital's rule where appropriate.

\input{Derivative-Compute-0050.HELP.tex}

\[\lim\limits_{x\to\infty} {\left(-\frac{9}{x} + 1\right)}^{-2 \, x} + 2=\answer{e^{18} + 2}\]
\end{problem}}

%%%%%%%%%%%%%%%%%%%%%%

\latexProblemContent{
\ifVerboseLocation This is Derivative Compute Question 0050. \\ \fi
\begin{problem}

Find the limit.  Use L'H$\hat{o}$pital's rule where appropriate.

\input{Derivative-Compute-0050.HELP.tex}

\[\lim\limits_{x\to\infty} {\left(-\frac{7}{x} + 1\right)}^{-5 \, x} - 16=\answer{e^{35} - 16}\]
\end{problem}}

%%%%%%%%%%%%%%%%%%%%%%

\latexProblemContent{
\ifVerboseLocation This is Derivative Compute Question 0050. \\ \fi
\begin{problem}

Find the limit.  Use L'H$\hat{o}$pital's rule where appropriate.

\input{Derivative-Compute-0050.HELP.tex}

\[\lim\limits_{x\to\infty} {\left(-\frac{9}{x} + 1\right)}^{-3 \, x} - 9=\answer{e^{27} - 9}\]
\end{problem}}

%%%%%%%%%%%%%%%%%%%%%%

\latexProblemContent{
\ifVerboseLocation This is Derivative Compute Question 0050. \\ \fi
\begin{problem}

Find the limit.  Use L'H$\hat{o}$pital's rule where appropriate.

\input{Derivative-Compute-0050.HELP.tex}

\[\lim\limits_{x\to\infty} {\left(-\frac{6}{x} + 1\right)}^{-8 \, x} - 13=\answer{e^{48} - 13}\]
\end{problem}}

%%%%%%%%%%%%%%%%%%%%%%

\latexProblemContent{
\ifVerboseLocation This is Derivative Compute Question 0050. \\ \fi
\begin{problem}

Find the limit.  Use L'H$\hat{o}$pital's rule where appropriate.

\input{Derivative-Compute-0050.HELP.tex}

\[\lim\limits_{x\to\infty} {\left(-\frac{4}{x} + 1\right)}^{3 \, x} + 6=\answer{e^{\left(-12\right)} + 6}\]
\end{problem}}

%%%%%%%%%%%%%%%%%%%%%%

\latexProblemContent{
\ifVerboseLocation This is Derivative Compute Question 0050. \\ \fi
\begin{problem}

Find the limit.  Use L'H$\hat{o}$pital's rule where appropriate.

\input{Derivative-Compute-0050.HELP.tex}

\[\lim\limits_{x\to\infty} {\left(\frac{2}{x} + 1\right)}^{9 \, x} + 3=\answer{e^{18} + 3}\]
\end{problem}}

%%%%%%%%%%%%%%%%%%%%%%

\latexProblemContent{
\ifVerboseLocation This is Derivative Compute Question 0050. \\ \fi
\begin{problem}

Find the limit.  Use L'H$\hat{o}$pital's rule where appropriate.

\input{Derivative-Compute-0050.HELP.tex}

\[\lim\limits_{x\to\infty} {\left(\frac{1}{x} + 1\right)}^{2 \, x} + 15=\answer{e^{2} + 15}\]
\end{problem}}

%%%%%%%%%%%%%%%%%%%%%%

\latexProblemContent{
\ifVerboseLocation This is Derivative Compute Question 0050. \\ \fi
\begin{problem}

Find the limit.  Use L'H$\hat{o}$pital's rule where appropriate.

\input{Derivative-Compute-0050.HELP.tex}

\[\lim\limits_{x\to\infty} {\left(-\frac{3}{x} + 1\right)}^{-7 \, x} + 17=\answer{e^{21} + 17}\]
\end{problem}}

%%%%%%%%%%%%%%%%%%%%%%

\latexProblemContent{
\ifVerboseLocation This is Derivative Compute Question 0050. \\ \fi
\begin{problem}

Find the limit.  Use L'H$\hat{o}$pital's rule where appropriate.

\input{Derivative-Compute-0050.HELP.tex}

\[\lim\limits_{x\to\infty} {\left(-\frac{2}{x} + 1\right)}^{x} - 19=\answer{e^{\left(-2\right)} - 19}\]
\end{problem}}

%%%%%%%%%%%%%%%%%%%%%%

\latexProblemContent{
\ifVerboseLocation This is Derivative Compute Question 0050. \\ \fi
\begin{problem}

Find the limit.  Use L'H$\hat{o}$pital's rule where appropriate.

\input{Derivative-Compute-0050.HELP.tex}

\[\lim\limits_{x\to\infty} {\left(-\frac{4}{x} + 1\right)}^{-5 \, x} - 12=\answer{e^{20} - 12}\]
\end{problem}}

%%%%%%%%%%%%%%%%%%%%%%

\latexProblemContent{
\ifVerboseLocation This is Derivative Compute Question 0050. \\ \fi
\begin{problem}

Find the limit.  Use L'H$\hat{o}$pital's rule where appropriate.

\input{Derivative-Compute-0050.HELP.tex}

\[\lim\limits_{x\to\infty} {\left(-\frac{7}{x} + 1\right)}^{9 \, x} + 11=\answer{e^{\left(-63\right)} + 11}\]
\end{problem}}

%%%%%%%%%%%%%%%%%%%%%%

\latexProblemContent{
\ifVerboseLocation This is Derivative Compute Question 0050. \\ \fi
\begin{problem}

Find the limit.  Use L'H$\hat{o}$pital's rule where appropriate.

\input{Derivative-Compute-0050.HELP.tex}

\[\lim\limits_{x\to\infty} {\left(\frac{4}{x} + 1\right)}^{-4 \, x} - 6=\answer{e^{\left(-16\right)} - 6}\]
\end{problem}}

%%%%%%%%%%%%%%%%%%%%%%

\latexProblemContent{
\ifVerboseLocation This is Derivative Compute Question 0050. \\ \fi
\begin{problem}

Find the limit.  Use L'H$\hat{o}$pital's rule where appropriate.

\input{Derivative-Compute-0050.HELP.tex}

\[\lim\limits_{x\to\infty} {\left(\frac{6}{x} + 1\right)}^{-3 \, x} - 20=\answer{e^{\left(-18\right)} - 20}\]
\end{problem}}

%%%%%%%%%%%%%%%%%%%%%%

\latexProblemContent{
\ifVerboseLocation This is Derivative Compute Question 0050. \\ \fi
\begin{problem}

Find the limit.  Use L'H$\hat{o}$pital's rule where appropriate.

\input{Derivative-Compute-0050.HELP.tex}

\[\lim\limits_{x\to\infty} {\left(-\frac{7}{x} + 1\right)}^{8 \, x} - 5=\answer{e^{\left(-56\right)} - 5}\]
\end{problem}}

%%%%%%%%%%%%%%%%%%%%%%

\latexProblemContent{
\ifVerboseLocation This is Derivative Compute Question 0050. \\ \fi
\begin{problem}

Find the limit.  Use L'H$\hat{o}$pital's rule where appropriate.

\input{Derivative-Compute-0050.HELP.tex}

\[\lim\limits_{x\to\infty} {\left(\frac{6}{x} + 1\right)}^{2 \, x} - 12=\answer{e^{12} - 12}\]
\end{problem}}

%%%%%%%%%%%%%%%%%%%%%%

\latexProblemContent{
\ifVerboseLocation This is Derivative Compute Question 0050. \\ \fi
\begin{problem}

Find the limit.  Use L'H$\hat{o}$pital's rule where appropriate.

\input{Derivative-Compute-0050.HELP.tex}

\[\lim\limits_{x\to\infty} {\left(\frac{5}{x} + 1\right)}^{-7 \, x} + 6=\answer{e^{\left(-35\right)} + 6}\]
\end{problem}}

%%%%%%%%%%%%%%%%%%%%%%

\latexProblemContent{
\ifVerboseLocation This is Derivative Compute Question 0050. \\ \fi
\begin{problem}

Find the limit.  Use L'H$\hat{o}$pital's rule where appropriate.

\input{Derivative-Compute-0050.HELP.tex}

\[\lim\limits_{x\to\infty} {\left(-\frac{1}{x} + 1\right)}^{4 \, x} - 5=\answer{e^{\left(-4\right)} - 5}\]
\end{problem}}

%%%%%%%%%%%%%%%%%%%%%%

\latexProblemContent{
\ifVerboseLocation This is Derivative Compute Question 0050. \\ \fi
\begin{problem}

Find the limit.  Use L'H$\hat{o}$pital's rule where appropriate.

\input{Derivative-Compute-0050.HELP.tex}

\[\lim\limits_{x\to\infty} {\left(\frac{10}{x} + 1\right)}^{x} - 10=\answer{e^{10} - 10}\]
\end{problem}}

%%%%%%%%%%%%%%%%%%%%%%

\latexProblemContent{
\ifVerboseLocation This is Derivative Compute Question 0050. \\ \fi
\begin{problem}

Find the limit.  Use L'H$\hat{o}$pital's rule where appropriate.

\input{Derivative-Compute-0050.HELP.tex}

\[\lim\limits_{x\to\infty} {\left(\frac{4}{x} + 1\right)}^{9 \, x} - 11=\answer{e^{36} - 11}\]
\end{problem}}

%%%%%%%%%%%%%%%%%%%%%%

\latexProblemContent{
\ifVerboseLocation This is Derivative Compute Question 0050. \\ \fi
\begin{problem}

Find the limit.  Use L'H$\hat{o}$pital's rule where appropriate.

\input{Derivative-Compute-0050.HELP.tex}

\[\lim\limits_{x\to\infty} {\left(\frac{9}{x} + 1\right)}^{-2 \, x} + 15=\answer{e^{\left(-18\right)} + 15}\]
\end{problem}}

%%%%%%%%%%%%%%%%%%%%%%

\latexProblemContent{
\ifVerboseLocation This is Derivative Compute Question 0050. \\ \fi
\begin{problem}

Find the limit.  Use L'H$\hat{o}$pital's rule where appropriate.

\input{Derivative-Compute-0050.HELP.tex}

\[\lim\limits_{x\to\infty} {\left(-\frac{5}{x} + 1\right)}^{-3 \, x}=\answer{e^{15}}\]
\end{problem}}

%%%%%%%%%%%%%%%%%%%%%%

\latexProblemContent{
\ifVerboseLocation This is Derivative Compute Question 0050. \\ \fi
\begin{problem}

Find the limit.  Use L'H$\hat{o}$pital's rule where appropriate.

\input{Derivative-Compute-0050.HELP.tex}

\[\lim\limits_{x\to\infty} {\left(\frac{4}{x} + 1\right)}^{2 \, x} - 20=\answer{e^{8} - 20}\]
\end{problem}}

%%%%%%%%%%%%%%%%%%%%%%

\latexProblemContent{
\ifVerboseLocation This is Derivative Compute Question 0050. \\ \fi
\begin{problem}

Find the limit.  Use L'H$\hat{o}$pital's rule where appropriate.

\input{Derivative-Compute-0050.HELP.tex}

\[\lim\limits_{x\to\infty} {\left(\frac{6}{x} + 1\right)}^{10 \, x} + 19=\answer{e^{60} + 19}\]
\end{problem}}

%%%%%%%%%%%%%%%%%%%%%%

\latexProblemContent{
\ifVerboseLocation This is Derivative Compute Question 0050. \\ \fi
\begin{problem}

Find the limit.  Use L'H$\hat{o}$pital's rule where appropriate.

\input{Derivative-Compute-0050.HELP.tex}

\[\lim\limits_{x\to\infty} {\left(-\frac{10}{x} + 1\right)}^{-10 \, x} - 5=\answer{e^{100} - 5}\]
\end{problem}}

%%%%%%%%%%%%%%%%%%%%%%

\latexProblemContent{
\ifVerboseLocation This is Derivative Compute Question 0050. \\ \fi
\begin{problem}

Find the limit.  Use L'H$\hat{o}$pital's rule where appropriate.

\input{Derivative-Compute-0050.HELP.tex}

\[\lim\limits_{x\to\infty} {\left(\frac{1}{x} + 1\right)}^{2 \, x} - 19=\answer{e^{2} - 19}\]
\end{problem}}

%%%%%%%%%%%%%%%%%%%%%%

\latexProblemContent{
\ifVerboseLocation This is Derivative Compute Question 0050. \\ \fi
\begin{problem}

Find the limit.  Use L'H$\hat{o}$pital's rule where appropriate.

\input{Derivative-Compute-0050.HELP.tex}

\[\lim\limits_{x\to\infty} {\left(-\frac{6}{x} + 1\right)}^{6 \, x} + 18=\answer{e^{\left(-36\right)} + 18}\]
\end{problem}}

%%%%%%%%%%%%%%%%%%%%%%

\latexProblemContent{
\ifVerboseLocation This is Derivative Compute Question 0050. \\ \fi
\begin{problem}

Find the limit.  Use L'H$\hat{o}$pital's rule where appropriate.

\input{Derivative-Compute-0050.HELP.tex}

\[\lim\limits_{x\to\infty} {\left(-\frac{1}{x} + 1\right)}^{10 \, x} - 17=\answer{e^{\left(-10\right)} - 17}\]
\end{problem}}

%%%%%%%%%%%%%%%%%%%%%%

\latexProblemContent{
\ifVerboseLocation This is Derivative Compute Question 0050. \\ \fi
\begin{problem}

Find the limit.  Use L'H$\hat{o}$pital's rule where appropriate.

\input{Derivative-Compute-0050.HELP.tex}

\[\lim\limits_{x\to\infty} {\left(-\frac{5}{x} + 1\right)}^{-8 \, x} + 6=\answer{e^{40} + 6}\]
\end{problem}}

%%%%%%%%%%%%%%%%%%%%%%

\latexProblemContent{
\ifVerboseLocation This is Derivative Compute Question 0050. \\ \fi
\begin{problem}

Find the limit.  Use L'H$\hat{o}$pital's rule where appropriate.

\input{Derivative-Compute-0050.HELP.tex}

\[\lim\limits_{x\to\infty} {\left(-\frac{2}{x} + 1\right)}^{8 \, x} + 7=\answer{e^{\left(-16\right)} + 7}\]
\end{problem}}

%%%%%%%%%%%%%%%%%%%%%%

\latexProblemContent{
\ifVerboseLocation This is Derivative Compute Question 0050. \\ \fi
\begin{problem}

Find the limit.  Use L'H$\hat{o}$pital's rule where appropriate.

\input{Derivative-Compute-0050.HELP.tex}

\[\lim\limits_{x\to\infty} {\left(\frac{6}{x} + 1\right)}^{-4 \, x} - 18=\answer{e^{\left(-24\right)} - 18}\]
\end{problem}}

%%%%%%%%%%%%%%%%%%%%%%

\latexProblemContent{
\ifVerboseLocation This is Derivative Compute Question 0050. \\ \fi
\begin{problem}

Find the limit.  Use L'H$\hat{o}$pital's rule where appropriate.

\input{Derivative-Compute-0050.HELP.tex}

\[\lim\limits_{x\to\infty} {\left(-\frac{2}{x} + 1\right)}^{-3 \, x} + 9=\answer{e^{6} + 9}\]
\end{problem}}

%%%%%%%%%%%%%%%%%%%%%%

\latexProblemContent{
\ifVerboseLocation This is Derivative Compute Question 0050. \\ \fi
\begin{problem}

Find the limit.  Use L'H$\hat{o}$pital's rule where appropriate.

\input{Derivative-Compute-0050.HELP.tex}

\[\lim\limits_{x\to\infty} {\left(-\frac{6}{x} + 1\right)}^{-8 \, x} - 17=\answer{e^{48} - 17}\]
\end{problem}}

%%%%%%%%%%%%%%%%%%%%%%

\latexProblemContent{
\ifVerboseLocation This is Derivative Compute Question 0050. \\ \fi
\begin{problem}

Find the limit.  Use L'H$\hat{o}$pital's rule where appropriate.

\input{Derivative-Compute-0050.HELP.tex}

\[\lim\limits_{x\to\infty} {\left(\frac{1}{x} + 1\right)}^{6 \, x} + 2=\answer{e^{6} + 2}\]
\end{problem}}

%%%%%%%%%%%%%%%%%%%%%%

\latexProblemContent{
\ifVerboseLocation This is Derivative Compute Question 0050. \\ \fi
\begin{problem}

Find the limit.  Use L'H$\hat{o}$pital's rule where appropriate.

\input{Derivative-Compute-0050.HELP.tex}

\[\lim\limits_{x\to\infty} {\left(\frac{5}{x} + 1\right)}^{-6 \, x} + 16=\answer{e^{\left(-30\right)} + 16}\]
\end{problem}}

%%%%%%%%%%%%%%%%%%%%%%

\latexProblemContent{
\ifVerboseLocation This is Derivative Compute Question 0050. \\ \fi
\begin{problem}

Find the limit.  Use L'H$\hat{o}$pital's rule where appropriate.

\input{Derivative-Compute-0050.HELP.tex}

\[\lim\limits_{x\to\infty} {\left(-\frac{3}{x} + 1\right)}^{-7 \, x} + 13=\answer{e^{21} + 13}\]
\end{problem}}

%%%%%%%%%%%%%%%%%%%%%%

\latexProblemContent{
\ifVerboseLocation This is Derivative Compute Question 0050. \\ \fi
\begin{problem}

Find the limit.  Use L'H$\hat{o}$pital's rule where appropriate.

\input{Derivative-Compute-0050.HELP.tex}

\[\lim\limits_{x\to\infty} {\left(-\frac{1}{x} + 1\right)}^{5 \, x} + 10=\answer{e^{\left(-5\right)} + 10}\]
\end{problem}}

%%%%%%%%%%%%%%%%%%%%%%

\latexProblemContent{
\ifVerboseLocation This is Derivative Compute Question 0050. \\ \fi
\begin{problem}

Find the limit.  Use L'H$\hat{o}$pital's rule where appropriate.

\input{Derivative-Compute-0050.HELP.tex}

\[\lim\limits_{x\to\infty} {\left(-\frac{9}{x} + 1\right)}^{2 \, x} - 15=\answer{e^{\left(-18\right)} - 15}\]
\end{problem}}

%%%%%%%%%%%%%%%%%%%%%%

\latexProblemContent{
\ifVerboseLocation This is Derivative Compute Question 0050. \\ \fi
\begin{problem}

Find the limit.  Use L'H$\hat{o}$pital's rule where appropriate.

\input{Derivative-Compute-0050.HELP.tex}

\[\lim\limits_{x\to\infty} {\left(\frac{3}{x} + 1\right)}^{8 \, x} - 5=\answer{e^{24} - 5}\]
\end{problem}}

%%%%%%%%%%%%%%%%%%%%%%

\latexProblemContent{
\ifVerboseLocation This is Derivative Compute Question 0050. \\ \fi
\begin{problem}

Find the limit.  Use L'H$\hat{o}$pital's rule where appropriate.

\input{Derivative-Compute-0050.HELP.tex}

\[\lim\limits_{x\to\infty} {\left(\frac{8}{x} + 1\right)}^{x} - 8=\answer{e^{8} - 8}\]
\end{problem}}

%%%%%%%%%%%%%%%%%%%%%%

\latexProblemContent{
\ifVerboseLocation This is Derivative Compute Question 0050. \\ \fi
\begin{problem}

Find the limit.  Use L'H$\hat{o}$pital's rule where appropriate.

\input{Derivative-Compute-0050.HELP.tex}

\[\lim\limits_{x\to\infty} {\left(-\frac{5}{x} + 1\right)}^{-3 \, x} - 2=\answer{e^{15} - 2}\]
\end{problem}}

%%%%%%%%%%%%%%%%%%%%%%

\latexProblemContent{
\ifVerboseLocation This is Derivative Compute Question 0050. \\ \fi
\begin{problem}

Find the limit.  Use L'H$\hat{o}$pital's rule where appropriate.

\input{Derivative-Compute-0050.HELP.tex}

\[\lim\limits_{x\to\infty} {\left(-\frac{1}{x} + 1\right)}^{-5 \, x} + 13=\answer{e^{5} + 13}\]
\end{problem}}

%%%%%%%%%%%%%%%%%%%%%%

\latexProblemContent{
\ifVerboseLocation This is Derivative Compute Question 0050. \\ \fi
\begin{problem}

Find the limit.  Use L'H$\hat{o}$pital's rule where appropriate.

\input{Derivative-Compute-0050.HELP.tex}

\[\lim\limits_{x\to\infty} {\left(-\frac{3}{x} + 1\right)}^{7 \, x}=\answer{e^{\left(-21\right)}}\]
\end{problem}}

%%%%%%%%%%%%%%%%%%%%%%

\latexProblemContent{
\ifVerboseLocation This is Derivative Compute Question 0050. \\ \fi
\begin{problem}

Find the limit.  Use L'H$\hat{o}$pital's rule where appropriate.

\input{Derivative-Compute-0050.HELP.tex}

\[\lim\limits_{x\to\infty} {\left(\frac{10}{x} + 1\right)}^{2 \, x} - 20=\answer{e^{20} - 20}\]
\end{problem}}

%%%%%%%%%%%%%%%%%%%%%%

\latexProblemContent{
\ifVerboseLocation This is Derivative Compute Question 0050. \\ \fi
\begin{problem}

Find the limit.  Use L'H$\hat{o}$pital's rule where appropriate.

\input{Derivative-Compute-0050.HELP.tex}

\[\lim\limits_{x\to\infty} {\left(\frac{7}{x} + 1\right)}^{-9 \, x} - 7=\answer{e^{\left(-63\right)} - 7}\]
\end{problem}}

%%%%%%%%%%%%%%%%%%%%%%

\latexProblemContent{
\ifVerboseLocation This is Derivative Compute Question 0050. \\ \fi
\begin{problem}

Find the limit.  Use L'H$\hat{o}$pital's rule where appropriate.

\input{Derivative-Compute-0050.HELP.tex}

\[\lim\limits_{x\to\infty} {\left(\frac{2}{x} + 1\right)}^{6 \, x} + 5=\answer{e^{12} + 5}\]
\end{problem}}

%%%%%%%%%%%%%%%%%%%%%%

\latexProblemContent{
\ifVerboseLocation This is Derivative Compute Question 0050. \\ \fi
\begin{problem}

Find the limit.  Use L'H$\hat{o}$pital's rule where appropriate.

\input{Derivative-Compute-0050.HELP.tex}

\[\lim\limits_{x\to\infty} {\left(\frac{8}{x} + 1\right)}^{3 \, x} - 12=\answer{e^{24} - 12}\]
\end{problem}}

%%%%%%%%%%%%%%%%%%%%%%

\latexProblemContent{
\ifVerboseLocation This is Derivative Compute Question 0050. \\ \fi
\begin{problem}

Find the limit.  Use L'H$\hat{o}$pital's rule where appropriate.

\input{Derivative-Compute-0050.HELP.tex}

\[\lim\limits_{x\to\infty} {\left(\frac{4}{x} + 1\right)}^{5 \, x} - 2=\answer{e^{20} - 2}\]
\end{problem}}

%%%%%%%%%%%%%%%%%%%%%%

\latexProblemContent{
\ifVerboseLocation This is Derivative Compute Question 0050. \\ \fi
\begin{problem}

Find the limit.  Use L'H$\hat{o}$pital's rule where appropriate.

\input{Derivative-Compute-0050.HELP.tex}

\[\lim\limits_{x\to\infty} {\left(\frac{8}{x} + 1\right)}^{10 \, x} - 7=\answer{e^{80} - 7}\]
\end{problem}}

%%%%%%%%%%%%%%%%%%%%%%

\latexProblemContent{
\ifVerboseLocation This is Derivative Compute Question 0050. \\ \fi
\begin{problem}

Find the limit.  Use L'H$\hat{o}$pital's rule where appropriate.

\input{Derivative-Compute-0050.HELP.tex}

\[\lim\limits_{x\to\infty} {\left(-\frac{9}{x} + 1\right)}^{8 \, x} - 20=\answer{e^{\left(-72\right)} - 20}\]
\end{problem}}

%%%%%%%%%%%%%%%%%%%%%%

\latexProblemContent{
\ifVerboseLocation This is Derivative Compute Question 0050. \\ \fi
\begin{problem}

Find the limit.  Use L'H$\hat{o}$pital's rule where appropriate.

\input{Derivative-Compute-0050.HELP.tex}

\[\lim\limits_{x\to\infty} {\left(-\frac{1}{x} + 1\right)}^{-10 \, x}=\answer{e^{10}}\]
\end{problem}}

%%%%%%%%%%%%%%%%%%%%%%

\latexProblemContent{
\ifVerboseLocation This is Derivative Compute Question 0050. \\ \fi
\begin{problem}

Find the limit.  Use L'H$\hat{o}$pital's rule where appropriate.

\input{Derivative-Compute-0050.HELP.tex}

\[\lim\limits_{x\to\infty} {\left(\frac{9}{x} + 1\right)}^{3 \, x} - 14=\answer{e^{27} - 14}\]
\end{problem}}

%%%%%%%%%%%%%%%%%%%%%%

\latexProblemContent{
\ifVerboseLocation This is Derivative Compute Question 0050. \\ \fi
\begin{problem}

Find the limit.  Use L'H$\hat{o}$pital's rule where appropriate.

\input{Derivative-Compute-0050.HELP.tex}

\[\lim\limits_{x\to\infty} {\left(\frac{2}{x} + 1\right)}^{-10 \, x} - 11=\answer{e^{\left(-20\right)} - 11}\]
\end{problem}}

%%%%%%%%%%%%%%%%%%%%%%

\latexProblemContent{
\ifVerboseLocation This is Derivative Compute Question 0050. \\ \fi
\begin{problem}

Find the limit.  Use L'H$\hat{o}$pital's rule where appropriate.

\input{Derivative-Compute-0050.HELP.tex}

\[\lim\limits_{x\to\infty} {\left(\frac{5}{x} + 1\right)}^{-5 \, x} + 12=\answer{e^{\left(-25\right)} + 12}\]
\end{problem}}

%%%%%%%%%%%%%%%%%%%%%%

\latexProblemContent{
\ifVerboseLocation This is Derivative Compute Question 0050. \\ \fi
\begin{problem}

Find the limit.  Use L'H$\hat{o}$pital's rule where appropriate.

\input{Derivative-Compute-0050.HELP.tex}

\[\lim\limits_{x\to\infty} {\left(\frac{9}{x} + 1\right)}^{-2 \, x} + 3=\answer{e^{\left(-18\right)} + 3}\]
\end{problem}}

%%%%%%%%%%%%%%%%%%%%%%

\latexProblemContent{
\ifVerboseLocation This is Derivative Compute Question 0050. \\ \fi
\begin{problem}

Find the limit.  Use L'H$\hat{o}$pital's rule where appropriate.

\input{Derivative-Compute-0050.HELP.tex}

\[\lim\limits_{x\to\infty} {\left(-\frac{6}{x} + 1\right)}^{2 \, x} + 14=\answer{e^{\left(-12\right)} + 14}\]
\end{problem}}

%%%%%%%%%%%%%%%%%%%%%%

\latexProblemContent{
\ifVerboseLocation This is Derivative Compute Question 0050. \\ \fi
\begin{problem}

Find the limit.  Use L'H$\hat{o}$pital's rule where appropriate.

\input{Derivative-Compute-0050.HELP.tex}

\[\lim\limits_{x\to\infty} {\left(-\frac{3}{x} + 1\right)}^{3 \, x} + 15=\answer{e^{\left(-9\right)} + 15}\]
\end{problem}}

%%%%%%%%%%%%%%%%%%%%%%

\latexProblemContent{
\ifVerboseLocation This is Derivative Compute Question 0050. \\ \fi
\begin{problem}

Find the limit.  Use L'H$\hat{o}$pital's rule where appropriate.

\input{Derivative-Compute-0050.HELP.tex}

\[\lim\limits_{x\to\infty} {\left(\frac{2}{x} + 1\right)}^{2 \, x} + 4=\answer{e^{4} + 4}\]
\end{problem}}

%%%%%%%%%%%%%%%%%%%%%%

\latexProblemContent{
\ifVerboseLocation This is Derivative Compute Question 0050. \\ \fi
\begin{problem}

Find the limit.  Use L'H$\hat{o}$pital's rule where appropriate.

\input{Derivative-Compute-0050.HELP.tex}

\[\lim\limits_{x\to\infty} {\left(\frac{3}{x} + 1\right)}^{-x} - 6=\answer{e^{\left(-3\right)} - 6}\]
\end{problem}}

%%%%%%%%%%%%%%%%%%%%%%

\latexProblemContent{
\ifVerboseLocation This is Derivative Compute Question 0050. \\ \fi
\begin{problem}

Find the limit.  Use L'H$\hat{o}$pital's rule where appropriate.

\input{Derivative-Compute-0050.HELP.tex}

\[\lim\limits_{x\to\infty} {\left(\frac{7}{x} + 1\right)}^{-7 \, x} + 11=\answer{e^{\left(-49\right)} + 11}\]
\end{problem}}

%%%%%%%%%%%%%%%%%%%%%%

\latexProblemContent{
\ifVerboseLocation This is Derivative Compute Question 0050. \\ \fi
\begin{problem}

Find the limit.  Use L'H$\hat{o}$pital's rule where appropriate.

\input{Derivative-Compute-0050.HELP.tex}

\[\lim\limits_{x\to\infty} {\left(-\frac{9}{x} + 1\right)}^{-8 \, x} + 13=\answer{e^{72} + 13}\]
\end{problem}}

%%%%%%%%%%%%%%%%%%%%%%

\latexProblemContent{
\ifVerboseLocation This is Derivative Compute Question 0050. \\ \fi
\begin{problem}

Find the limit.  Use L'H$\hat{o}$pital's rule where appropriate.

\input{Derivative-Compute-0050.HELP.tex}

\[\lim\limits_{x\to\infty} {\left(-\frac{6}{x} + 1\right)}^{-10 \, x} + 10=\answer{e^{60} + 10}\]
\end{problem}}

%%%%%%%%%%%%%%%%%%%%%%

\latexProblemContent{
\ifVerboseLocation This is Derivative Compute Question 0050. \\ \fi
\begin{problem}

Find the limit.  Use L'H$\hat{o}$pital's rule where appropriate.

\input{Derivative-Compute-0050.HELP.tex}

\[\lim\limits_{x\to\infty} {\left(-\frac{3}{x} + 1\right)}^{-9 \, x} - 18=\answer{e^{27} - 18}\]
\end{problem}}

%%%%%%%%%%%%%%%%%%%%%%

\latexProblemContent{
\ifVerboseLocation This is Derivative Compute Question 0050. \\ \fi
\begin{problem}

Find the limit.  Use L'H$\hat{o}$pital's rule where appropriate.

\input{Derivative-Compute-0050.HELP.tex}

\[\lim\limits_{x\to\infty} {\left(-\frac{8}{x} + 1\right)}^{3 \, x} - 8=\answer{e^{\left(-24\right)} - 8}\]
\end{problem}}

%%%%%%%%%%%%%%%%%%%%%%

\latexProblemContent{
\ifVerboseLocation This is Derivative Compute Question 0050. \\ \fi
\begin{problem}

Find the limit.  Use L'H$\hat{o}$pital's rule where appropriate.

\input{Derivative-Compute-0050.HELP.tex}

\[\lim\limits_{x\to\infty} {\left(\frac{6}{x} + 1\right)}^{-4 \, x} + 9=\answer{e^{\left(-24\right)} + 9}\]
\end{problem}}

%%%%%%%%%%%%%%%%%%%%%%

\latexProblemContent{
\ifVerboseLocation This is Derivative Compute Question 0050. \\ \fi
\begin{problem}

Find the limit.  Use L'H$\hat{o}$pital's rule where appropriate.

\input{Derivative-Compute-0050.HELP.tex}

\[\lim\limits_{x\to\infty} {\left(-\frac{2}{x} + 1\right)}^{4 \, x} + 11=\answer{e^{\left(-8\right)} + 11}\]
\end{problem}}

%%%%%%%%%%%%%%%%%%%%%%

\latexProblemContent{
\ifVerboseLocation This is Derivative Compute Question 0050. \\ \fi
\begin{problem}

Find the limit.  Use L'H$\hat{o}$pital's rule where appropriate.

\input{Derivative-Compute-0050.HELP.tex}

\[\lim\limits_{x\to\infty} {\left(\frac{5}{x} + 1\right)}^{4 \, x} - 16=\answer{e^{20} - 16}\]
\end{problem}}

%%%%%%%%%%%%%%%%%%%%%%

\latexProblemContent{
\ifVerboseLocation This is Derivative Compute Question 0050. \\ \fi
\begin{problem}

Find the limit.  Use L'H$\hat{o}$pital's rule where appropriate.

\input{Derivative-Compute-0050.HELP.tex}

\[\lim\limits_{x\to\infty} {\left(\frac{1}{x} + 1\right)}^{-4 \, x} - 19=\answer{e^{\left(-4\right)} - 19}\]
\end{problem}}

%%%%%%%%%%%%%%%%%%%%%%

\latexProblemContent{
\ifVerboseLocation This is Derivative Compute Question 0050. \\ \fi
\begin{problem}

Find the limit.  Use L'H$\hat{o}$pital's rule where appropriate.

\input{Derivative-Compute-0050.HELP.tex}

\[\lim\limits_{x\to\infty} {\left(-\frac{10}{x} + 1\right)}^{8 \, x} + 9=\answer{e^{\left(-80\right)} + 9}\]
\end{problem}}

%%%%%%%%%%%%%%%%%%%%%%

\latexProblemContent{
\ifVerboseLocation This is Derivative Compute Question 0050. \\ \fi
\begin{problem}

Find the limit.  Use L'H$\hat{o}$pital's rule where appropriate.

\input{Derivative-Compute-0050.HELP.tex}

\[\lim\limits_{x\to\infty} {\left(\frac{2}{x} + 1\right)}^{-5 \, x} + 9=\answer{e^{\left(-10\right)} + 9}\]
\end{problem}}

%%%%%%%%%%%%%%%%%%%%%%

\latexProblemContent{
\ifVerboseLocation This is Derivative Compute Question 0050. \\ \fi
\begin{problem}

Find the limit.  Use L'H$\hat{o}$pital's rule where appropriate.

\input{Derivative-Compute-0050.HELP.tex}

\[\lim\limits_{x\to\infty} {\left(\frac{9}{x} + 1\right)}^{6 \, x} + 19=\answer{e^{54} + 19}\]
\end{problem}}

%%%%%%%%%%%%%%%%%%%%%%

\latexProblemContent{
\ifVerboseLocation This is Derivative Compute Question 0050. \\ \fi
\begin{problem}

Find the limit.  Use L'H$\hat{o}$pital's rule where appropriate.

\input{Derivative-Compute-0050.HELP.tex}

\[\lim\limits_{x\to\infty} {\left(\frac{4}{x} + 1\right)}^{-x} - 7=\answer{e^{\left(-4\right)} - 7}\]
\end{problem}}

%%%%%%%%%%%%%%%%%%%%%%

\latexProblemContent{
\ifVerboseLocation This is Derivative Compute Question 0050. \\ \fi
\begin{problem}

Find the limit.  Use L'H$\hat{o}$pital's rule where appropriate.

\input{Derivative-Compute-0050.HELP.tex}

\[\lim\limits_{x\to\infty} {\left(\frac{1}{x} + 1\right)}^{8 \, x} + 8=\answer{e^{8} + 8}\]
\end{problem}}

%%%%%%%%%%%%%%%%%%%%%%

\latexProblemContent{
\ifVerboseLocation This is Derivative Compute Question 0050. \\ \fi
\begin{problem}

Find the limit.  Use L'H$\hat{o}$pital's rule where appropriate.

\input{Derivative-Compute-0050.HELP.tex}

\[\lim\limits_{x\to\infty} {\left(-\frac{8}{x} + 1\right)}^{-2 \, x} + 7=\answer{e^{16} + 7}\]
\end{problem}}

%%%%%%%%%%%%%%%%%%%%%%

\latexProblemContent{
\ifVerboseLocation This is Derivative Compute Question 0050. \\ \fi
\begin{problem}

Find the limit.  Use L'H$\hat{o}$pital's rule where appropriate.

\input{Derivative-Compute-0050.HELP.tex}

\[\lim\limits_{x\to\infty} {\left(\frac{5}{x} + 1\right)}^{2 \, x} + 19=\answer{e^{10} + 19}\]
\end{problem}}

%%%%%%%%%%%%%%%%%%%%%%

\latexProblemContent{
\ifVerboseLocation This is Derivative Compute Question 0050. \\ \fi
\begin{problem}

Find the limit.  Use L'H$\hat{o}$pital's rule where appropriate.

\input{Derivative-Compute-0050.HELP.tex}

\[\lim\limits_{x\to\infty} {\left(-\frac{3}{x} + 1\right)}^{-8 \, x} - 6=\answer{e^{24} - 6}\]
\end{problem}}

%%%%%%%%%%%%%%%%%%%%%%

\latexProblemContent{
\ifVerboseLocation This is Derivative Compute Question 0050. \\ \fi
\begin{problem}

Find the limit.  Use L'H$\hat{o}$pital's rule where appropriate.

\input{Derivative-Compute-0050.HELP.tex}

\[\lim\limits_{x\to\infty} {\left(-\frac{9}{x} + 1\right)}^{-3 \, x} - 6=\answer{e^{27} - 6}\]
\end{problem}}

%%%%%%%%%%%%%%%%%%%%%%

\latexProblemContent{
\ifVerboseLocation This is Derivative Compute Question 0050. \\ \fi
\begin{problem}

Find the limit.  Use L'H$\hat{o}$pital's rule where appropriate.

\input{Derivative-Compute-0050.HELP.tex}

\[\lim\limits_{x\to\infty} {\left(\frac{7}{x} + 1\right)}^{2 \, x} + 20=\answer{e^{14} + 20}\]
\end{problem}}

%%%%%%%%%%%%%%%%%%%%%%

\latexProblemContent{
\ifVerboseLocation This is Derivative Compute Question 0050. \\ \fi
\begin{problem}

Find the limit.  Use L'H$\hat{o}$pital's rule where appropriate.

\input{Derivative-Compute-0050.HELP.tex}

\[\lim\limits_{x\to\infty} {\left(\frac{3}{x} + 1\right)}^{-8 \, x} + 15=\answer{e^{\left(-24\right)} + 15}\]
\end{problem}}

%%%%%%%%%%%%%%%%%%%%%%

\latexProblemContent{
\ifVerboseLocation This is Derivative Compute Question 0050. \\ \fi
\begin{problem}

Find the limit.  Use L'H$\hat{o}$pital's rule where appropriate.

\input{Derivative-Compute-0050.HELP.tex}

\[\lim\limits_{x\to\infty} {\left(\frac{7}{x} + 1\right)}^{-5 \, x} + 5=\answer{e^{\left(-35\right)} + 5}\]
\end{problem}}

%%%%%%%%%%%%%%%%%%%%%%

\latexProblemContent{
\ifVerboseLocation This is Derivative Compute Question 0050. \\ \fi
\begin{problem}

Find the limit.  Use L'H$\hat{o}$pital's rule where appropriate.

\input{Derivative-Compute-0050.HELP.tex}

\[\lim\limits_{x\to\infty} {\left(-\frac{5}{x} + 1\right)}^{-4 \, x} + 4=\answer{e^{20} + 4}\]
\end{problem}}

%%%%%%%%%%%%%%%%%%%%%%

\latexProblemContent{
\ifVerboseLocation This is Derivative Compute Question 0050. \\ \fi
\begin{problem}

Find the limit.  Use L'H$\hat{o}$pital's rule where appropriate.

\input{Derivative-Compute-0050.HELP.tex}

\[\lim\limits_{x\to\infty} {\left(\frac{2}{x} + 1\right)}^{6 \, x} + 9=\answer{e^{12} + 9}\]
\end{problem}}

%%%%%%%%%%%%%%%%%%%%%%

\latexProblemContent{
\ifVerboseLocation This is Derivative Compute Question 0050. \\ \fi
\begin{problem}

Find the limit.  Use L'H$\hat{o}$pital's rule where appropriate.

\input{Derivative-Compute-0050.HELP.tex}

\[\lim\limits_{x\to\infty} {\left(\frac{1}{x} + 1\right)}^{3 \, x} - 5=\answer{e^{3} - 5}\]
\end{problem}}

%%%%%%%%%%%%%%%%%%%%%%

\latexProblemContent{
\ifVerboseLocation This is Derivative Compute Question 0050. \\ \fi
\begin{problem}

Find the limit.  Use L'H$\hat{o}$pital's rule where appropriate.

\input{Derivative-Compute-0050.HELP.tex}

\[\lim\limits_{x\to\infty} {\left(-\frac{10}{x} + 1\right)}^{-8 \, x} - 14=\answer{e^{80} - 14}\]
\end{problem}}

%%%%%%%%%%%%%%%%%%%%%%

\latexProblemContent{
\ifVerboseLocation This is Derivative Compute Question 0050. \\ \fi
\begin{problem}

Find the limit.  Use L'H$\hat{o}$pital's rule where appropriate.

\input{Derivative-Compute-0050.HELP.tex}

\[\lim\limits_{x\to\infty} {\left(-\frac{3}{x} + 1\right)}^{-2 \, x} - 6=\answer{e^{6} - 6}\]
\end{problem}}

%%%%%%%%%%%%%%%%%%%%%%

\latexProblemContent{
\ifVerboseLocation This is Derivative Compute Question 0050. \\ \fi
\begin{problem}

Find the limit.  Use L'H$\hat{o}$pital's rule where appropriate.

\input{Derivative-Compute-0050.HELP.tex}

\[\lim\limits_{x\to\infty} {\left(\frac{7}{x} + 1\right)}^{9 \, x} + 17=\answer{e^{63} + 17}\]
\end{problem}}

%%%%%%%%%%%%%%%%%%%%%%

\latexProblemContent{
\ifVerboseLocation This is Derivative Compute Question 0050. \\ \fi
\begin{problem}

Find the limit.  Use L'H$\hat{o}$pital's rule where appropriate.

\input{Derivative-Compute-0050.HELP.tex}

\[\lim\limits_{x\to\infty} {\left(\frac{3}{x} + 1\right)}^{-9 \, x} - 18=\answer{e^{\left(-27\right)} - 18}\]
\end{problem}}

%%%%%%%%%%%%%%%%%%%%%%

\latexProblemContent{
\ifVerboseLocation This is Derivative Compute Question 0050. \\ \fi
\begin{problem}

Find the limit.  Use L'H$\hat{o}$pital's rule where appropriate.

\input{Derivative-Compute-0050.HELP.tex}

\[\lim\limits_{x\to\infty} {\left(\frac{4}{x} + 1\right)}^{-x} - 9=\answer{e^{\left(-4\right)} - 9}\]
\end{problem}}

%%%%%%%%%%%%%%%%%%%%%%

\latexProblemContent{
\ifVerboseLocation This is Derivative Compute Question 0050. \\ \fi
\begin{problem}

Find the limit.  Use L'H$\hat{o}$pital's rule where appropriate.

\input{Derivative-Compute-0050.HELP.tex}

\[\lim\limits_{x\to\infty} {\left(\frac{8}{x} + 1\right)}^{-x} - 10=\answer{e^{\left(-8\right)} - 10}\]
\end{problem}}

%%%%%%%%%%%%%%%%%%%%%%

\latexProblemContent{
\ifVerboseLocation This is Derivative Compute Question 0050. \\ \fi
\begin{problem}

Find the limit.  Use L'H$\hat{o}$pital's rule where appropriate.

\input{Derivative-Compute-0050.HELP.tex}

\[\lim\limits_{x\to\infty} {\left(-\frac{7}{x} + 1\right)}^{6 \, x} + 8=\answer{e^{\left(-42\right)} + 8}\]
\end{problem}}

%%%%%%%%%%%%%%%%%%%%%%

\latexProblemContent{
\ifVerboseLocation This is Derivative Compute Question 0050. \\ \fi
\begin{problem}

Find the limit.  Use L'H$\hat{o}$pital's rule where appropriate.

\input{Derivative-Compute-0050.HELP.tex}

\[\lim\limits_{x\to\infty} {\left(\frac{8}{x} + 1\right)}^{-2 \, x} - 7=\answer{e^{\left(-16\right)} - 7}\]
\end{problem}}

%%%%%%%%%%%%%%%%%%%%%%

\latexProblemContent{
\ifVerboseLocation This is Derivative Compute Question 0050. \\ \fi
\begin{problem}

Find the limit.  Use L'H$\hat{o}$pital's rule where appropriate.

\input{Derivative-Compute-0050.HELP.tex}

\[\lim\limits_{x\to\infty} {\left(-\frac{1}{x} + 1\right)}^{5 \, x} - 11=\answer{e^{\left(-5\right)} - 11}\]
\end{problem}}

%%%%%%%%%%%%%%%%%%%%%%

\latexProblemContent{
\ifVerboseLocation This is Derivative Compute Question 0050. \\ \fi
\begin{problem}

Find the limit.  Use L'H$\hat{o}$pital's rule where appropriate.

\input{Derivative-Compute-0050.HELP.tex}

\[\lim\limits_{x\to\infty} {\left(-\frac{2}{x} + 1\right)}^{-x} + 7=\answer{e^{2} + 7}\]
\end{problem}}

%%%%%%%%%%%%%%%%%%%%%%

\latexProblemContent{
\ifVerboseLocation This is Derivative Compute Question 0050. \\ \fi
\begin{problem}

Find the limit.  Use L'H$\hat{o}$pital's rule where appropriate.

\input{Derivative-Compute-0050.HELP.tex}

\[\lim\limits_{x\to\infty} {\left(-\frac{3}{x} + 1\right)}^{x} - 10=\answer{e^{\left(-3\right)} - 10}\]
\end{problem}}

%%%%%%%%%%%%%%%%%%%%%%

\latexProblemContent{
\ifVerboseLocation This is Derivative Compute Question 0050. \\ \fi
\begin{problem}

Find the limit.  Use L'H$\hat{o}$pital's rule where appropriate.

\input{Derivative-Compute-0050.HELP.tex}

\[\lim\limits_{x\to\infty} {\left(\frac{10}{x} + 1\right)}^{10 \, x} + 20=\answer{e^{100} + 20}\]
\end{problem}}

%%%%%%%%%%%%%%%%%%%%%%

\latexProblemContent{
\ifVerboseLocation This is Derivative Compute Question 0050. \\ \fi
\begin{problem}

Find the limit.  Use L'H$\hat{o}$pital's rule where appropriate.

\input{Derivative-Compute-0050.HELP.tex}

\[\lim\limits_{x\to\infty} {\left(\frac{6}{x} + 1\right)}^{-10 \, x} + 4=\answer{e^{\left(-60\right)} + 4}\]
\end{problem}}

%%%%%%%%%%%%%%%%%%%%%%

\latexProblemContent{
\ifVerboseLocation This is Derivative Compute Question 0050. \\ \fi
\begin{problem}

Find the limit.  Use L'H$\hat{o}$pital's rule where appropriate.

\input{Derivative-Compute-0050.HELP.tex}

\[\lim\limits_{x\to\infty} {\left(-\frac{10}{x} + 1\right)}^{x} - 15=\answer{e^{\left(-10\right)} - 15}\]
\end{problem}}

%%%%%%%%%%%%%%%%%%%%%%

\latexProblemContent{
\ifVerboseLocation This is Derivative Compute Question 0050. \\ \fi
\begin{problem}

Find the limit.  Use L'H$\hat{o}$pital's rule where appropriate.

\input{Derivative-Compute-0050.HELP.tex}

\[\lim\limits_{x\to\infty} {\left(-\frac{3}{x} + 1\right)}^{7 \, x} - 19=\answer{e^{\left(-21\right)} - 19}\]
\end{problem}}

%%%%%%%%%%%%%%%%%%%%%%

\latexProblemContent{
\ifVerboseLocation This is Derivative Compute Question 0050. \\ \fi
\begin{problem}

Find the limit.  Use L'H$\hat{o}$pital's rule where appropriate.

\input{Derivative-Compute-0050.HELP.tex}

\[\lim\limits_{x\to\infty} {\left(\frac{2}{x} + 1\right)}^{-8 \, x} + 19=\answer{e^{\left(-16\right)} + 19}\]
\end{problem}}

%%%%%%%%%%%%%%%%%%%%%%

\latexProblemContent{
\ifVerboseLocation This is Derivative Compute Question 0050. \\ \fi
\begin{problem}

Find the limit.  Use L'H$\hat{o}$pital's rule where appropriate.

\input{Derivative-Compute-0050.HELP.tex}

\[\lim\limits_{x\to\infty} {\left(\frac{6}{x} + 1\right)}^{3 \, x} - 15=\answer{e^{18} - 15}\]
\end{problem}}

%%%%%%%%%%%%%%%%%%%%%%

\latexProblemContent{
\ifVerboseLocation This is Derivative Compute Question 0050. \\ \fi
\begin{problem}

Find the limit.  Use L'H$\hat{o}$pital's rule where appropriate.

\input{Derivative-Compute-0050.HELP.tex}

\[\lim\limits_{x\to\infty} {\left(-\frac{2}{x} + 1\right)}^{4 \, x} + 15=\answer{e^{\left(-8\right)} + 15}\]
\end{problem}}

%%%%%%%%%%%%%%%%%%%%%%

\latexProblemContent{
\ifVerboseLocation This is Derivative Compute Question 0050. \\ \fi
\begin{problem}

Find the limit.  Use L'H$\hat{o}$pital's rule where appropriate.

\input{Derivative-Compute-0050.HELP.tex}

\[\lim\limits_{x\to\infty} {\left(-\frac{10}{x} + 1\right)}^{-5 \, x} - 14=\answer{e^{50} - 14}\]
\end{problem}}

%%%%%%%%%%%%%%%%%%%%%%

\latexProblemContent{
\ifVerboseLocation This is Derivative Compute Question 0050. \\ \fi
\begin{problem}

Find the limit.  Use L'H$\hat{o}$pital's rule where appropriate.

\input{Derivative-Compute-0050.HELP.tex}

\[\lim\limits_{x\to\infty} {\left(-\frac{8}{x} + 1\right)}^{-5 \, x} - 10=\answer{e^{40} - 10}\]
\end{problem}}

%%%%%%%%%%%%%%%%%%%%%%

\latexProblemContent{
\ifVerboseLocation This is Derivative Compute Question 0050. \\ \fi
\begin{problem}

Find the limit.  Use L'H$\hat{o}$pital's rule where appropriate.

\input{Derivative-Compute-0050.HELP.tex}

\[\lim\limits_{x\to\infty} {\left(-\frac{9}{x} + 1\right)}^{-2 \, x} - 2=\answer{e^{18} - 2}\]
\end{problem}}

%%%%%%%%%%%%%%%%%%%%%%

\latexProblemContent{
\ifVerboseLocation This is Derivative Compute Question 0050. \\ \fi
\begin{problem}

Find the limit.  Use L'H$\hat{o}$pital's rule where appropriate.

\input{Derivative-Compute-0050.HELP.tex}

\[\lim\limits_{x\to\infty} {\left(\frac{10}{x} + 1\right)}^{-8 \, x} - 17=\answer{e^{\left(-80\right)} - 17}\]
\end{problem}}

%%%%%%%%%%%%%%%%%%%%%%

\latexProblemContent{
\ifVerboseLocation This is Derivative Compute Question 0050. \\ \fi
\begin{problem}

Find the limit.  Use L'H$\hat{o}$pital's rule where appropriate.

\input{Derivative-Compute-0050.HELP.tex}

\[\lim\limits_{x\to\infty} {\left(-\frac{3}{x} + 1\right)}^{x} - 17=\answer{e^{\left(-3\right)} - 17}\]
\end{problem}}

%%%%%%%%%%%%%%%%%%%%%%

\latexProblemContent{
\ifVerboseLocation This is Derivative Compute Question 0050. \\ \fi
\begin{problem}

Find the limit.  Use L'H$\hat{o}$pital's rule where appropriate.

\input{Derivative-Compute-0050.HELP.tex}

\[\lim\limits_{x\to\infty} {\left(\frac{3}{x} + 1\right)}^{8 \, x} - 13=\answer{e^{24} - 13}\]
\end{problem}}

%%%%%%%%%%%%%%%%%%%%%%

\latexProblemContent{
\ifVerboseLocation This is Derivative Compute Question 0050. \\ \fi
\begin{problem}

Find the limit.  Use L'H$\hat{o}$pital's rule where appropriate.

\input{Derivative-Compute-0050.HELP.tex}

\[\lim\limits_{x\to\infty} {\left(\frac{3}{x} + 1\right)}^{7 \, x} + 19=\answer{e^{21} + 19}\]
\end{problem}}

%%%%%%%%%%%%%%%%%%%%%%

\latexProblemContent{
\ifVerboseLocation This is Derivative Compute Question 0050. \\ \fi
\begin{problem}

Find the limit.  Use L'H$\hat{o}$pital's rule where appropriate.

\input{Derivative-Compute-0050.HELP.tex}

\[\lim\limits_{x\to\infty} {\left(\frac{2}{x} + 1\right)}^{7 \, x} + 18=\answer{e^{14} + 18}\]
\end{problem}}

%%%%%%%%%%%%%%%%%%%%%%

\latexProblemContent{
\ifVerboseLocation This is Derivative Compute Question 0050. \\ \fi
\begin{problem}

Find the limit.  Use L'H$\hat{o}$pital's rule where appropriate.

\input{Derivative-Compute-0050.HELP.tex}

\[\lim\limits_{x\to\infty} {\left(-\frac{3}{x} + 1\right)}^{-5 \, x} + 13=\answer{e^{15} + 13}\]
\end{problem}}

%%%%%%%%%%%%%%%%%%%%%%

\latexProblemContent{
\ifVerboseLocation This is Derivative Compute Question 0050. \\ \fi
\begin{problem}

Find the limit.  Use L'H$\hat{o}$pital's rule where appropriate.

\input{Derivative-Compute-0050.HELP.tex}

\[\lim\limits_{x\to\infty} {\left(-\frac{1}{x} + 1\right)}^{9 \, x} - 6=\answer{e^{\left(-9\right)} - 6}\]
\end{problem}}

%%%%%%%%%%%%%%%%%%%%%%

\latexProblemContent{
\ifVerboseLocation This is Derivative Compute Question 0050. \\ \fi
\begin{problem}

Find the limit.  Use L'H$\hat{o}$pital's rule where appropriate.

\input{Derivative-Compute-0050.HELP.tex}

\[\lim\limits_{x\to\infty} {\left(-\frac{1}{x} + 1\right)}^{10 \, x} - 2=\answer{e^{\left(-10\right)} - 2}\]
\end{problem}}

%%%%%%%%%%%%%%%%%%%%%%

\latexProblemContent{
\ifVerboseLocation This is Derivative Compute Question 0050. \\ \fi
\begin{problem}

Find the limit.  Use L'H$\hat{o}$pital's rule where appropriate.

\input{Derivative-Compute-0050.HELP.tex}

\[\lim\limits_{x\to\infty} {\left(\frac{6}{x} + 1\right)}^{-3 \, x} + 17=\answer{e^{\left(-18\right)} + 17}\]
\end{problem}}

%%%%%%%%%%%%%%%%%%%%%%

\latexProblemContent{
\ifVerboseLocation This is Derivative Compute Question 0050. \\ \fi
\begin{problem}

Find the limit.  Use L'H$\hat{o}$pital's rule where appropriate.

\input{Derivative-Compute-0050.HELP.tex}

\[\lim\limits_{x\to\infty} {\left(\frac{1}{x} + 1\right)}^{-5 \, x} - 9=\answer{e^{\left(-5\right)} - 9}\]
\end{problem}}

%%%%%%%%%%%%%%%%%%%%%%

\latexProblemContent{
\ifVerboseLocation This is Derivative Compute Question 0050. \\ \fi
\begin{problem}

Find the limit.  Use L'H$\hat{o}$pital's rule where appropriate.

\input{Derivative-Compute-0050.HELP.tex}

\[\lim\limits_{x\to\infty} {\left(\frac{1}{x} + 1\right)}^{-3 \, x} + 8=\answer{e^{\left(-3\right)} + 8}\]
\end{problem}}

%%%%%%%%%%%%%%%%%%%%%%

\latexProblemContent{
\ifVerboseLocation This is Derivative Compute Question 0050. \\ \fi
\begin{problem}

Find the limit.  Use L'H$\hat{o}$pital's rule where appropriate.

\input{Derivative-Compute-0050.HELP.tex}

\[\lim\limits_{x\to\infty} {\left(\frac{9}{x} + 1\right)}^{8 \, x} - 12=\answer{e^{72} - 12}\]
\end{problem}}

%%%%%%%%%%%%%%%%%%%%%%

\latexProblemContent{
\ifVerboseLocation This is Derivative Compute Question 0050. \\ \fi
\begin{problem}

Find the limit.  Use L'H$\hat{o}$pital's rule where appropriate.

\input{Derivative-Compute-0050.HELP.tex}

\[\lim\limits_{x\to\infty} {\left(-\frac{8}{x} + 1\right)}^{9 \, x} + 17=\answer{e^{\left(-72\right)} + 17}\]
\end{problem}}

%%%%%%%%%%%%%%%%%%%%%%

\latexProblemContent{
\ifVerboseLocation This is Derivative Compute Question 0050. \\ \fi
\begin{problem}

Find the limit.  Use L'H$\hat{o}$pital's rule where appropriate.

\input{Derivative-Compute-0050.HELP.tex}

\[\lim\limits_{x\to\infty} {\left(-\frac{6}{x} + 1\right)}^{10 \, x} + 14=\answer{e^{\left(-60\right)} + 14}\]
\end{problem}}

%%%%%%%%%%%%%%%%%%%%%%

\latexProblemContent{
\ifVerboseLocation This is Derivative Compute Question 0050. \\ \fi
\begin{problem}

Find the limit.  Use L'H$\hat{o}$pital's rule where appropriate.

\input{Derivative-Compute-0050.HELP.tex}

\[\lim\limits_{x\to\infty} {\left(\frac{7}{x} + 1\right)}^{5 \, x} - 17=\answer{e^{35} - 17}\]
\end{problem}}

%%%%%%%%%%%%%%%%%%%%%%

\latexProblemContent{
\ifVerboseLocation This is Derivative Compute Question 0050. \\ \fi
\begin{problem}

Find the limit.  Use L'H$\hat{o}$pital's rule where appropriate.

\input{Derivative-Compute-0050.HELP.tex}

\[\lim\limits_{x\to\infty} {\left(-\frac{4}{x} + 1\right)}^{10 \, x} - 4=\answer{e^{\left(-40\right)} - 4}\]
\end{problem}}

%%%%%%%%%%%%%%%%%%%%%%

\latexProblemContent{
\ifVerboseLocation This is Derivative Compute Question 0050. \\ \fi
\begin{problem}

Find the limit.  Use L'H$\hat{o}$pital's rule where appropriate.

\input{Derivative-Compute-0050.HELP.tex}

\[\lim\limits_{x\to\infty} {\left(\frac{3}{x} + 1\right)}^{-x} - 16=\answer{e^{\left(-3\right)} - 16}\]
\end{problem}}

%%%%%%%%%%%%%%%%%%%%%%

\latexProblemContent{
\ifVerboseLocation This is Derivative Compute Question 0050. \\ \fi
\begin{problem}

Find the limit.  Use L'H$\hat{o}$pital's rule where appropriate.

\input{Derivative-Compute-0050.HELP.tex}

\[\lim\limits_{x\to\infty} {\left(-\frac{7}{x} + 1\right)}^{4 \, x} - 15=\answer{e^{\left(-28\right)} - 15}\]
\end{problem}}

%%%%%%%%%%%%%%%%%%%%%%

\latexProblemContent{
\ifVerboseLocation This is Derivative Compute Question 0050. \\ \fi
\begin{problem}

Find the limit.  Use L'H$\hat{o}$pital's rule where appropriate.

\input{Derivative-Compute-0050.HELP.tex}

\[\lim\limits_{x\to\infty} {\left(\frac{2}{x} + 1\right)}^{-4 \, x} - 15=\answer{e^{\left(-8\right)} - 15}\]
\end{problem}}

%%%%%%%%%%%%%%%%%%%%%%

\latexProblemContent{
\ifVerboseLocation This is Derivative Compute Question 0050. \\ \fi
\begin{problem}

Find the limit.  Use L'H$\hat{o}$pital's rule where appropriate.

\input{Derivative-Compute-0050.HELP.tex}

\[\lim\limits_{x\to\infty} {\left(-\frac{1}{x} + 1\right)}^{-9 \, x} + 1=\answer{e^{9} + 1}\]
\end{problem}}

%%%%%%%%%%%%%%%%%%%%%%

\latexProblemContent{
\ifVerboseLocation This is Derivative Compute Question 0050. \\ \fi
\begin{problem}

Find the limit.  Use L'H$\hat{o}$pital's rule where appropriate.

\input{Derivative-Compute-0050.HELP.tex}

\[\lim\limits_{x\to\infty} {\left(\frac{3}{x} + 1\right)}^{-2 \, x} + 13=\answer{e^{\left(-6\right)} + 13}\]
\end{problem}}

%%%%%%%%%%%%%%%%%%%%%%

\latexProblemContent{
\ifVerboseLocation This is Derivative Compute Question 0050. \\ \fi
\begin{problem}

Find the limit.  Use L'H$\hat{o}$pital's rule where appropriate.

\input{Derivative-Compute-0050.HELP.tex}

\[\lim\limits_{x\to\infty} {\left(-\frac{4}{x} + 1\right)}^{2 \, x} - 8=\answer{e^{\left(-8\right)} - 8}\]
\end{problem}}

%%%%%%%%%%%%%%%%%%%%%%

\latexProblemContent{
\ifVerboseLocation This is Derivative Compute Question 0050. \\ \fi
\begin{problem}

Find the limit.  Use L'H$\hat{o}$pital's rule where appropriate.

\input{Derivative-Compute-0050.HELP.tex}

\[\lim\limits_{x\to\infty} {\left(-\frac{4}{x} + 1\right)}^{-6 \, x} - 13=\answer{e^{24} - 13}\]
\end{problem}}

%%%%%%%%%%%%%%%%%%%%%%

\latexProblemContent{
\ifVerboseLocation This is Derivative Compute Question 0050. \\ \fi
\begin{problem}

Find the limit.  Use L'H$\hat{o}$pital's rule where appropriate.

\input{Derivative-Compute-0050.HELP.tex}

\[\lim\limits_{x\to\infty} {\left(-\frac{1}{x} + 1\right)}^{7 \, x} + 16=\answer{e^{\left(-7\right)} + 16}\]
\end{problem}}

%%%%%%%%%%%%%%%%%%%%%%

\latexProblemContent{
\ifVerboseLocation This is Derivative Compute Question 0050. \\ \fi
\begin{problem}

Find the limit.  Use L'H$\hat{o}$pital's rule where appropriate.

\input{Derivative-Compute-0050.HELP.tex}

\[\lim\limits_{x\to\infty} {\left(\frac{6}{x} + 1\right)}^{-7 \, x} - 9=\answer{e^{\left(-42\right)} - 9}\]
\end{problem}}

%%%%%%%%%%%%%%%%%%%%%%

\latexProblemContent{
\ifVerboseLocation This is Derivative Compute Question 0050. \\ \fi
\begin{problem}

Find the limit.  Use L'H$\hat{o}$pital's rule where appropriate.

\input{Derivative-Compute-0050.HELP.tex}

\[\lim\limits_{x\to\infty} {\left(\frac{3}{x} + 1\right)}^{5 \, x} + 17=\answer{e^{15} + 17}\]
\end{problem}}

%%%%%%%%%%%%%%%%%%%%%%

\latexProblemContent{
\ifVerboseLocation This is Derivative Compute Question 0050. \\ \fi
\begin{problem}

Find the limit.  Use L'H$\hat{o}$pital's rule where appropriate.

\input{Derivative-Compute-0050.HELP.tex}

\[\lim\limits_{x\to\infty} {\left(-\frac{3}{x} + 1\right)}^{-3 \, x} + 17=\answer{e^{9} + 17}\]
\end{problem}}

%%%%%%%%%%%%%%%%%%%%%%

\latexProblemContent{
\ifVerboseLocation This is Derivative Compute Question 0050. \\ \fi
\begin{problem}

Find the limit.  Use L'H$\hat{o}$pital's rule where appropriate.

\input{Derivative-Compute-0050.HELP.tex}

\[\lim\limits_{x\to\infty} {\left(\frac{8}{x} + 1\right)}^{-7 \, x} + 11=\answer{e^{\left(-56\right)} + 11}\]
\end{problem}}

%%%%%%%%%%%%%%%%%%%%%%

\latexProblemContent{
\ifVerboseLocation This is Derivative Compute Question 0050. \\ \fi
\begin{problem}

Find the limit.  Use L'H$\hat{o}$pital's rule where appropriate.

\input{Derivative-Compute-0050.HELP.tex}

\[\lim\limits_{x\to\infty} {\left(-\frac{8}{x} + 1\right)}^{5 \, x} + 8=\answer{e^{\left(-40\right)} + 8}\]
\end{problem}}

%%%%%%%%%%%%%%%%%%%%%%

\latexProblemContent{
\ifVerboseLocation This is Derivative Compute Question 0050. \\ \fi
\begin{problem}

Find the limit.  Use L'H$\hat{o}$pital's rule where appropriate.

\input{Derivative-Compute-0050.HELP.tex}

\[\lim\limits_{x\to\infty} {\left(\frac{9}{x} + 1\right)}^{-10 \, x} + 7=\answer{e^{\left(-90\right)} + 7}\]
\end{problem}}

%%%%%%%%%%%%%%%%%%%%%%

\latexProblemContent{
\ifVerboseLocation This is Derivative Compute Question 0050. \\ \fi
\begin{problem}

Find the limit.  Use L'H$\hat{o}$pital's rule where appropriate.

\input{Derivative-Compute-0050.HELP.tex}

\[\lim\limits_{x\to\infty} {\left(-\frac{5}{x} + 1\right)}^{7 \, x} + 3=\answer{e^{\left(-35\right)} + 3}\]
\end{problem}}

%%%%%%%%%%%%%%%%%%%%%%

\latexProblemContent{
\ifVerboseLocation This is Derivative Compute Question 0050. \\ \fi
\begin{problem}

Find the limit.  Use L'H$\hat{o}$pital's rule where appropriate.

\input{Derivative-Compute-0050.HELP.tex}

\[\lim\limits_{x\to\infty} {\left(-\frac{5}{x} + 1\right)}^{-x} + 20=\answer{e^{5} + 20}\]
\end{problem}}

%%%%%%%%%%%%%%%%%%%%%%

\latexProblemContent{
\ifVerboseLocation This is Derivative Compute Question 0050. \\ \fi
\begin{problem}

Find the limit.  Use L'H$\hat{o}$pital's rule where appropriate.

\input{Derivative-Compute-0050.HELP.tex}

\[\lim\limits_{x\to\infty} {\left(-\frac{8}{x} + 1\right)}^{4 \, x} - 8=\answer{e^{\left(-32\right)} - 8}\]
\end{problem}}

%%%%%%%%%%%%%%%%%%%%%%

\latexProblemContent{
\ifVerboseLocation This is Derivative Compute Question 0050. \\ \fi
\begin{problem}

Find the limit.  Use L'H$\hat{o}$pital's rule where appropriate.

\input{Derivative-Compute-0050.HELP.tex}

\[\lim\limits_{x\to\infty} {\left(\frac{1}{x} + 1\right)}^{-4 \, x} + 16=\answer{e^{\left(-4\right)} + 16}\]
\end{problem}}

%%%%%%%%%%%%%%%%%%%%%%

\latexProblemContent{
\ifVerboseLocation This is Derivative Compute Question 0050. \\ \fi
\begin{problem}

Find the limit.  Use L'H$\hat{o}$pital's rule where appropriate.

\input{Derivative-Compute-0050.HELP.tex}

\[\lim\limits_{x\to\infty} {\left(\frac{4}{x} + 1\right)}^{-10 \, x} + 10=\answer{e^{\left(-40\right)} + 10}\]
\end{problem}}

%%%%%%%%%%%%%%%%%%%%%%

\latexProblemContent{
\ifVerboseLocation This is Derivative Compute Question 0050. \\ \fi
\begin{problem}

Find the limit.  Use L'H$\hat{o}$pital's rule where appropriate.

\input{Derivative-Compute-0050.HELP.tex}

\[\lim\limits_{x\to\infty} {\left(\frac{8}{x} + 1\right)}^{7 \, x} - 20=\answer{e^{56} - 20}\]
\end{problem}}

%%%%%%%%%%%%%%%%%%%%%%

\latexProblemContent{
\ifVerboseLocation This is Derivative Compute Question 0050. \\ \fi
\begin{problem}

Find the limit.  Use L'H$\hat{o}$pital's rule where appropriate.

\input{Derivative-Compute-0050.HELP.tex}

\[\lim\limits_{x\to\infty} {\left(\frac{6}{x} + 1\right)}^{-7 \, x} - 5=\answer{e^{\left(-42\right)} - 5}\]
\end{problem}}

%%%%%%%%%%%%%%%%%%%%%%

\latexProblemContent{
\ifVerboseLocation This is Derivative Compute Question 0050. \\ \fi
\begin{problem}

Find the limit.  Use L'H$\hat{o}$pital's rule where appropriate.

\input{Derivative-Compute-0050.HELP.tex}

\[\lim\limits_{x\to\infty} {\left(-\frac{2}{x} + 1\right)}^{5 \, x}=\answer{e^{\left(-10\right)}}\]
\end{problem}}

%%%%%%%%%%%%%%%%%%%%%%

\latexProblemContent{
\ifVerboseLocation This is Derivative Compute Question 0050. \\ \fi
\begin{problem}

Find the limit.  Use L'H$\hat{o}$pital's rule where appropriate.

\input{Derivative-Compute-0050.HELP.tex}

\[\lim\limits_{x\to\infty} {\left(\frac{1}{x} + 1\right)}^{8 \, x} - 14=\answer{e^{8} - 14}\]
\end{problem}}

%%%%%%%%%%%%%%%%%%%%%%

\latexProblemContent{
\ifVerboseLocation This is Derivative Compute Question 0050. \\ \fi
\begin{problem}

Find the limit.  Use L'H$\hat{o}$pital's rule where appropriate.

\input{Derivative-Compute-0050.HELP.tex}

\[\lim\limits_{x\to\infty} {\left(-\frac{1}{x} + 1\right)}^{4 \, x} + 8=\answer{e^{\left(-4\right)} + 8}\]
\end{problem}}

%%%%%%%%%%%%%%%%%%%%%%

\latexProblemContent{
\ifVerboseLocation This is Derivative Compute Question 0050. \\ \fi
\begin{problem}

Find the limit.  Use L'H$\hat{o}$pital's rule where appropriate.

\input{Derivative-Compute-0050.HELP.tex}

\[\lim\limits_{x\to\infty} {\left(-\frac{3}{x} + 1\right)}^{-6 \, x} - 20=\answer{e^{18} - 20}\]
\end{problem}}

%%%%%%%%%%%%%%%%%%%%%%

\latexProblemContent{
\ifVerboseLocation This is Derivative Compute Question 0050. \\ \fi
\begin{problem}

Find the limit.  Use L'H$\hat{o}$pital's rule where appropriate.

\input{Derivative-Compute-0050.HELP.tex}

\[\lim\limits_{x\to\infty} {\left(\frac{9}{x} + 1\right)}^{-2 \, x} - 20=\answer{e^{\left(-18\right)} - 20}\]
\end{problem}}

%%%%%%%%%%%%%%%%%%%%%%

\latexProblemContent{
\ifVerboseLocation This is Derivative Compute Question 0050. \\ \fi
\begin{problem}

Find the limit.  Use L'H$\hat{o}$pital's rule where appropriate.

\input{Derivative-Compute-0050.HELP.tex}

\[\lim\limits_{x\to\infty} {\left(\frac{9}{x} + 1\right)}^{6 \, x} + 12=\answer{e^{54} + 12}\]
\end{problem}}

%%%%%%%%%%%%%%%%%%%%%%

\latexProblemContent{
\ifVerboseLocation This is Derivative Compute Question 0050. \\ \fi
\begin{problem}

Find the limit.  Use L'H$\hat{o}$pital's rule where appropriate.

\input{Derivative-Compute-0050.HELP.tex}

\[\lim\limits_{x\to\infty} {\left(\frac{4}{x} + 1\right)}^{-9 \, x} - 18=\answer{e^{\left(-36\right)} - 18}\]
\end{problem}}

%%%%%%%%%%%%%%%%%%%%%%

\latexProblemContent{
\ifVerboseLocation This is Derivative Compute Question 0050. \\ \fi
\begin{problem}

Find the limit.  Use L'H$\hat{o}$pital's rule where appropriate.

\input{Derivative-Compute-0050.HELP.tex}

\[\lim\limits_{x\to\infty} {\left(\frac{9}{x} + 1\right)}^{8 \, x} - 1=\answer{e^{72} - 1}\]
\end{problem}}

%%%%%%%%%%%%%%%%%%%%%%

\latexProblemContent{
\ifVerboseLocation This is Derivative Compute Question 0050. \\ \fi
\begin{problem}

Find the limit.  Use L'H$\hat{o}$pital's rule where appropriate.

\input{Derivative-Compute-0050.HELP.tex}

\[\lim\limits_{x\to\infty} {\left(\frac{9}{x} + 1\right)}^{-4 \, x} + 15=\answer{e^{\left(-36\right)} + 15}\]
\end{problem}}

%%%%%%%%%%%%%%%%%%%%%%

\latexProblemContent{
\ifVerboseLocation This is Derivative Compute Question 0050. \\ \fi
\begin{problem}

Find the limit.  Use L'H$\hat{o}$pital's rule where appropriate.

\input{Derivative-Compute-0050.HELP.tex}

\[\lim\limits_{x\to\infty} {\left(\frac{8}{x} + 1\right)}^{2 \, x} - 2=\answer{e^{16} - 2}\]
\end{problem}}

%%%%%%%%%%%%%%%%%%%%%%

\latexProblemContent{
\ifVerboseLocation This is Derivative Compute Question 0050. \\ \fi
\begin{problem}

Find the limit.  Use L'H$\hat{o}$pital's rule where appropriate.

\input{Derivative-Compute-0050.HELP.tex}

\[\lim\limits_{x\to\infty} {\left(-\frac{1}{x} + 1\right)}^{-9 \, x} + 12=\answer{e^{9} + 12}\]
\end{problem}}

%%%%%%%%%%%%%%%%%%%%%%

\latexProblemContent{
\ifVerboseLocation This is Derivative Compute Question 0050. \\ \fi
\begin{problem}

Find the limit.  Use L'H$\hat{o}$pital's rule where appropriate.

\input{Derivative-Compute-0050.HELP.tex}

\[\lim\limits_{x\to\infty} {\left(\frac{6}{x} + 1\right)}^{8 \, x} + 1=\answer{e^{48} + 1}\]
\end{problem}}

%%%%%%%%%%%%%%%%%%%%%%

\latexProblemContent{
\ifVerboseLocation This is Derivative Compute Question 0050. \\ \fi
\begin{problem}

Find the limit.  Use L'H$\hat{o}$pital's rule where appropriate.

\input{Derivative-Compute-0050.HELP.tex}

\[\lim\limits_{x\to\infty} {\left(-\frac{1}{x} + 1\right)}^{3 \, x} - 18=\answer{e^{\left(-3\right)} - 18}\]
\end{problem}}

%%%%%%%%%%%%%%%%%%%%%%

\latexProblemContent{
\ifVerboseLocation This is Derivative Compute Question 0050. \\ \fi
\begin{problem}

Find the limit.  Use L'H$\hat{o}$pital's rule where appropriate.

\input{Derivative-Compute-0050.HELP.tex}

\[\lim\limits_{x\to\infty} {\left(\frac{2}{x} + 1\right)}^{x} - 11=\answer{e^{2} - 11}\]
\end{problem}}

%%%%%%%%%%%%%%%%%%%%%%

\latexProblemContent{
\ifVerboseLocation This is Derivative Compute Question 0050. \\ \fi
\begin{problem}

Find the limit.  Use L'H$\hat{o}$pital's rule where appropriate.

\input{Derivative-Compute-0050.HELP.tex}

\[\lim\limits_{x\to\infty} {\left(\frac{9}{x} + 1\right)}^{9 \, x} - 8=\answer{e^{81} - 8}\]
\end{problem}}

%%%%%%%%%%%%%%%%%%%%%%

\latexProblemContent{
\ifVerboseLocation This is Derivative Compute Question 0050. \\ \fi
\begin{problem}

Find the limit.  Use L'H$\hat{o}$pital's rule where appropriate.

\input{Derivative-Compute-0050.HELP.tex}

\[\lim\limits_{x\to\infty} {\left(\frac{7}{x} + 1\right)}^{-9 \, x} - 12=\answer{e^{\left(-63\right)} - 12}\]
\end{problem}}

%%%%%%%%%%%%%%%%%%%%%%

\latexProblemContent{
\ifVerboseLocation This is Derivative Compute Question 0050. \\ \fi
\begin{problem}

Find the limit.  Use L'H$\hat{o}$pital's rule where appropriate.

\input{Derivative-Compute-0050.HELP.tex}

\[\lim\limits_{x\to\infty} {\left(-\frac{10}{x} + 1\right)}^{-3 \, x} + 16=\answer{e^{30} + 16}\]
\end{problem}}

%%%%%%%%%%%%%%%%%%%%%%

\latexProblemContent{
\ifVerboseLocation This is Derivative Compute Question 0050. \\ \fi
\begin{problem}

Find the limit.  Use L'H$\hat{o}$pital's rule where appropriate.

\input{Derivative-Compute-0050.HELP.tex}

\[\lim\limits_{x\to\infty} {\left(\frac{3}{x} + 1\right)}^{-x} + 3=\answer{e^{\left(-3\right)} + 3}\]
\end{problem}}

%%%%%%%%%%%%%%%%%%%%%%

\latexProblemContent{
\ifVerboseLocation This is Derivative Compute Question 0050. \\ \fi
\begin{problem}

Find the limit.  Use L'H$\hat{o}$pital's rule where appropriate.

\input{Derivative-Compute-0050.HELP.tex}

\[\lim\limits_{x\to\infty} {\left(\frac{2}{x} + 1\right)}^{-8 \, x} + 2=\answer{e^{\left(-16\right)} + 2}\]
\end{problem}}

%%%%%%%%%%%%%%%%%%%%%%

\latexProblemContent{
\ifVerboseLocation This is Derivative Compute Question 0050. \\ \fi
\begin{problem}

Find the limit.  Use L'H$\hat{o}$pital's rule where appropriate.

\input{Derivative-Compute-0050.HELP.tex}

\[\lim\limits_{x\to\infty} {\left(\frac{9}{x} + 1\right)}^{-5 \, x} + 16=\answer{e^{\left(-45\right)} + 16}\]
\end{problem}}

%%%%%%%%%%%%%%%%%%%%%%

\latexProblemContent{
\ifVerboseLocation This is Derivative Compute Question 0050. \\ \fi
\begin{problem}

Find the limit.  Use L'H$\hat{o}$pital's rule where appropriate.

\input{Derivative-Compute-0050.HELP.tex}

\[\lim\limits_{x\to\infty} {\left(\frac{7}{x} + 1\right)}^{-7 \, x} - 18=\answer{e^{\left(-49\right)} - 18}\]
\end{problem}}

%%%%%%%%%%%%%%%%%%%%%%

\latexProblemContent{
\ifVerboseLocation This is Derivative Compute Question 0050. \\ \fi
\begin{problem}

Find the limit.  Use L'H$\hat{o}$pital's rule where appropriate.

\input{Derivative-Compute-0050.HELP.tex}

\[\lim\limits_{x\to\infty} {\left(-\frac{3}{x} + 1\right)}^{6 \, x} - 13=\answer{e^{\left(-18\right)} - 13}\]
\end{problem}}

%%%%%%%%%%%%%%%%%%%%%%

\latexProblemContent{
\ifVerboseLocation This is Derivative Compute Question 0050. \\ \fi
\begin{problem}

Find the limit.  Use L'H$\hat{o}$pital's rule where appropriate.

\input{Derivative-Compute-0050.HELP.tex}

\[\lim\limits_{x\to\infty} {\left(-\frac{5}{x} + 1\right)}^{-4 \, x} - 11=\answer{e^{20} - 11}\]
\end{problem}}

%%%%%%%%%%%%%%%%%%%%%%

\latexProblemContent{
\ifVerboseLocation This is Derivative Compute Question 0050. \\ \fi
\begin{problem}

Find the limit.  Use L'H$\hat{o}$pital's rule where appropriate.

\input{Derivative-Compute-0050.HELP.tex}

\[\lim\limits_{x\to\infty} {\left(\frac{1}{x} + 1\right)}^{5 \, x} - 13=\answer{e^{5} - 13}\]
\end{problem}}

%%%%%%%%%%%%%%%%%%%%%%

\latexProblemContent{
\ifVerboseLocation This is Derivative Compute Question 0050. \\ \fi
\begin{problem}

Find the limit.  Use L'H$\hat{o}$pital's rule where appropriate.

\input{Derivative-Compute-0050.HELP.tex}

\[\lim\limits_{x\to\infty} {\left(-\frac{1}{x} + 1\right)}^{-2 \, x} + 6=\answer{e^{2} + 6}\]
\end{problem}}

%%%%%%%%%%%%%%%%%%%%%%

\latexProblemContent{
\ifVerboseLocation This is Derivative Compute Question 0050. \\ \fi
\begin{problem}

Find the limit.  Use L'H$\hat{o}$pital's rule where appropriate.

\input{Derivative-Compute-0050.HELP.tex}

\[\lim\limits_{x\to\infty} {\left(-\frac{2}{x} + 1\right)}^{-5 \, x} - 7=\answer{e^{10} - 7}\]
\end{problem}}

%%%%%%%%%%%%%%%%%%%%%%

\latexProblemContent{
\ifVerboseLocation This is Derivative Compute Question 0050. \\ \fi
\begin{problem}

Find the limit.  Use L'H$\hat{o}$pital's rule where appropriate.

\input{Derivative-Compute-0050.HELP.tex}

\[\lim\limits_{x\to\infty} {\left(\frac{7}{x} + 1\right)}^{-5 \, x} + 17=\answer{e^{\left(-35\right)} + 17}\]
\end{problem}}

%%%%%%%%%%%%%%%%%%%%%%

\latexProblemContent{
\ifVerboseLocation This is Derivative Compute Question 0050. \\ \fi
\begin{problem}

Find the limit.  Use L'H$\hat{o}$pital's rule where appropriate.

\input{Derivative-Compute-0050.HELP.tex}

\[\lim\limits_{x\to\infty} {\left(-\frac{6}{x} + 1\right)}^{7 \, x} + 7=\answer{e^{\left(-42\right)} + 7}\]
\end{problem}}

%%%%%%%%%%%%%%%%%%%%%%

\latexProblemContent{
\ifVerboseLocation This is Derivative Compute Question 0050. \\ \fi
\begin{problem}

Find the limit.  Use L'H$\hat{o}$pital's rule where appropriate.

\input{Derivative-Compute-0050.HELP.tex}

\[\lim\limits_{x\to\infty} {\left(\frac{3}{x} + 1\right)}^{8 \, x} + 4=\answer{e^{24} + 4}\]
\end{problem}}

%%%%%%%%%%%%%%%%%%%%%%

\latexProblemContent{
\ifVerboseLocation This is Derivative Compute Question 0050. \\ \fi
\begin{problem}

Find the limit.  Use L'H$\hat{o}$pital's rule where appropriate.

\input{Derivative-Compute-0050.HELP.tex}

\[\lim\limits_{x\to\infty} {\left(-\frac{3}{x} + 1\right)}^{-x} - 10=\answer{e^{3} - 10}\]
\end{problem}}

%%%%%%%%%%%%%%%%%%%%%%

\latexProblemContent{
\ifVerboseLocation This is Derivative Compute Question 0050. \\ \fi
\begin{problem}

Find the limit.  Use L'H$\hat{o}$pital's rule where appropriate.

\input{Derivative-Compute-0050.HELP.tex}

\[\lim\limits_{x\to\infty} {\left(-\frac{7}{x} + 1\right)}^{3 \, x} - 6=\answer{e^{\left(-21\right)} - 6}\]
\end{problem}}

%%%%%%%%%%%%%%%%%%%%%%

\latexProblemContent{
\ifVerboseLocation This is Derivative Compute Question 0050. \\ \fi
\begin{problem}

Find the limit.  Use L'H$\hat{o}$pital's rule where appropriate.

\input{Derivative-Compute-0050.HELP.tex}

\[\lim\limits_{x\to\infty} {\left(-\frac{6}{x} + 1\right)}^{-9 \, x} + 5=\answer{e^{54} + 5}\]
\end{problem}}

%%%%%%%%%%%%%%%%%%%%%%

\latexProblemContent{
\ifVerboseLocation This is Derivative Compute Question 0050. \\ \fi
\begin{problem}

Find the limit.  Use L'H$\hat{o}$pital's rule where appropriate.

\input{Derivative-Compute-0050.HELP.tex}

\[\lim\limits_{x\to\infty} {\left(-\frac{5}{x} + 1\right)}^{6 \, x} + 9=\answer{e^{\left(-30\right)} + 9}\]
\end{problem}}

%%%%%%%%%%%%%%%%%%%%%%

\latexProblemContent{
\ifVerboseLocation This is Derivative Compute Question 0050. \\ \fi
\begin{problem}

Find the limit.  Use L'H$\hat{o}$pital's rule where appropriate.

\input{Derivative-Compute-0050.HELP.tex}

\[\lim\limits_{x\to\infty} {\left(-\frac{2}{x} + 1\right)}^{6 \, x} + 18=\answer{e^{\left(-12\right)} + 18}\]
\end{problem}}

%%%%%%%%%%%%%%%%%%%%%%

\latexProblemContent{
\ifVerboseLocation This is Derivative Compute Question 0050. \\ \fi
\begin{problem}

Find the limit.  Use L'H$\hat{o}$pital's rule where appropriate.

\input{Derivative-Compute-0050.HELP.tex}

\[\lim\limits_{x\to\infty} {\left(\frac{9}{x} + 1\right)}^{8 \, x} + 7=\answer{e^{72} + 7}\]
\end{problem}}

%%%%%%%%%%%%%%%%%%%%%%

\latexProblemContent{
\ifVerboseLocation This is Derivative Compute Question 0050. \\ \fi
\begin{problem}

Find the limit.  Use L'H$\hat{o}$pital's rule where appropriate.

\input{Derivative-Compute-0050.HELP.tex}

\[\lim\limits_{x\to\infty} {\left(\frac{2}{x} + 1\right)}^{6 \, x} - 20=\answer{e^{12} - 20}\]
\end{problem}}

%%%%%%%%%%%%%%%%%%%%%%

\latexProblemContent{
\ifVerboseLocation This is Derivative Compute Question 0050. \\ \fi
\begin{problem}

Find the limit.  Use L'H$\hat{o}$pital's rule where appropriate.

\input{Derivative-Compute-0050.HELP.tex}

\[\lim\limits_{x\to\infty} {\left(\frac{3}{x} + 1\right)}^{-6 \, x} + 13=\answer{e^{\left(-18\right)} + 13}\]
\end{problem}}

%%%%%%%%%%%%%%%%%%%%%%

\latexProblemContent{
\ifVerboseLocation This is Derivative Compute Question 0050. \\ \fi
\begin{problem}

Find the limit.  Use L'H$\hat{o}$pital's rule where appropriate.

\input{Derivative-Compute-0050.HELP.tex}

\[\lim\limits_{x\to\infty} {\left(-\frac{2}{x} + 1\right)}^{3 \, x} - 12=\answer{e^{\left(-6\right)} - 12}\]
\end{problem}}

%%%%%%%%%%%%%%%%%%%%%%

\latexProblemContent{
\ifVerboseLocation This is Derivative Compute Question 0050. \\ \fi
\begin{problem}

Find the limit.  Use L'H$\hat{o}$pital's rule where appropriate.

\input{Derivative-Compute-0050.HELP.tex}

\[\lim\limits_{x\to\infty} {\left(\frac{10}{x} + 1\right)}^{10 \, x} - 7=\answer{e^{100} - 7}\]
\end{problem}}

%%%%%%%%%%%%%%%%%%%%%%

\latexProblemContent{
\ifVerboseLocation This is Derivative Compute Question 0050. \\ \fi
\begin{problem}

Find the limit.  Use L'H$\hat{o}$pital's rule where appropriate.

\input{Derivative-Compute-0050.HELP.tex}

\[\lim\limits_{x\to\infty} {\left(-\frac{3}{x} + 1\right)}^{-10 \, x} - 17=\answer{e^{30} - 17}\]
\end{problem}}

%%%%%%%%%%%%%%%%%%%%%%

\latexProblemContent{
\ifVerboseLocation This is Derivative Compute Question 0050. \\ \fi
\begin{problem}

Find the limit.  Use L'H$\hat{o}$pital's rule where appropriate.

\input{Derivative-Compute-0050.HELP.tex}

\[\lim\limits_{x\to\infty} {\left(-\frac{8}{x} + 1\right)}^{-x} - 18=\answer{e^{8} - 18}\]
\end{problem}}

%%%%%%%%%%%%%%%%%%%%%%

\latexProblemContent{
\ifVerboseLocation This is Derivative Compute Question 0050. \\ \fi
\begin{problem}

Find the limit.  Use L'H$\hat{o}$pital's rule where appropriate.

\input{Derivative-Compute-0050.HELP.tex}

\[\lim\limits_{x\to\infty} {\left(-\frac{3}{x} + 1\right)}^{9 \, x} + 15=\answer{e^{\left(-27\right)} + 15}\]
\end{problem}}

%%%%%%%%%%%%%%%%%%%%%%

\latexProblemContent{
\ifVerboseLocation This is Derivative Compute Question 0050. \\ \fi
\begin{problem}

Find the limit.  Use L'H$\hat{o}$pital's rule where appropriate.

\input{Derivative-Compute-0050.HELP.tex}

\[\lim\limits_{x\to\infty} {\left(-\frac{2}{x} + 1\right)}^{6 \, x} - 20=\answer{e^{\left(-12\right)} - 20}\]
\end{problem}}

%%%%%%%%%%%%%%%%%%%%%%

\latexProblemContent{
\ifVerboseLocation This is Derivative Compute Question 0050. \\ \fi
\begin{problem}

Find the limit.  Use L'H$\hat{o}$pital's rule where appropriate.

\input{Derivative-Compute-0050.HELP.tex}

\[\lim\limits_{x\to\infty} {\left(-\frac{6}{x} + 1\right)}^{-8 \, x} + 4=\answer{e^{48} + 4}\]
\end{problem}}

%%%%%%%%%%%%%%%%%%%%%%

\latexProblemContent{
\ifVerboseLocation This is Derivative Compute Question 0050. \\ \fi
\begin{problem}

Find the limit.  Use L'H$\hat{o}$pital's rule where appropriate.

\input{Derivative-Compute-0050.HELP.tex}

\[\lim\limits_{x\to\infty} {\left(\frac{10}{x} + 1\right)}^{-x} + 7=\answer{e^{\left(-10\right)} + 7}\]
\end{problem}}

%%%%%%%%%%%%%%%%%%%%%%

\latexProblemContent{
\ifVerboseLocation This is Derivative Compute Question 0050. \\ \fi
\begin{problem}

Find the limit.  Use L'H$\hat{o}$pital's rule where appropriate.

\input{Derivative-Compute-0050.HELP.tex}

\[\lim\limits_{x\to\infty} {\left(-\frac{1}{x} + 1\right)}^{10 \, x} - 9=\answer{e^{\left(-10\right)} - 9}\]
\end{problem}}

%%%%%%%%%%%%%%%%%%%%%%

\latexProblemContent{
\ifVerboseLocation This is Derivative Compute Question 0050. \\ \fi
\begin{problem}

Find the limit.  Use L'H$\hat{o}$pital's rule where appropriate.

\input{Derivative-Compute-0050.HELP.tex}

\[\lim\limits_{x\to\infty} {\left(-\frac{7}{x} + 1\right)}^{-8 \, x} + 8=\answer{e^{56} + 8}\]
\end{problem}}

%%%%%%%%%%%%%%%%%%%%%%

\latexProblemContent{
\ifVerboseLocation This is Derivative Compute Question 0050. \\ \fi
\begin{problem}

Find the limit.  Use L'H$\hat{o}$pital's rule where appropriate.

\input{Derivative-Compute-0050.HELP.tex}

\[\lim\limits_{x\to\infty} {\left(\frac{7}{x} + 1\right)}^{-7 \, x} + 2=\answer{e^{\left(-49\right)} + 2}\]
\end{problem}}

%%%%%%%%%%%%%%%%%%%%%%

\latexProblemContent{
\ifVerboseLocation This is Derivative Compute Question 0050. \\ \fi
\begin{problem}

Find the limit.  Use L'H$\hat{o}$pital's rule where appropriate.

\input{Derivative-Compute-0050.HELP.tex}

\[\lim\limits_{x\to\infty} {\left(-\frac{2}{x} + 1\right)}^{5 \, x} + 13=\answer{e^{\left(-10\right)} + 13}\]
\end{problem}}

%%%%%%%%%%%%%%%%%%%%%%

\latexProblemContent{
\ifVerboseLocation This is Derivative Compute Question 0050. \\ \fi
\begin{problem}

Find the limit.  Use L'H$\hat{o}$pital's rule where appropriate.

\input{Derivative-Compute-0050.HELP.tex}

\[\lim\limits_{x\to\infty} {\left(\frac{6}{x} + 1\right)}^{4 \, x} - 20=\answer{e^{24} - 20}\]
\end{problem}}

%%%%%%%%%%%%%%%%%%%%%%

\latexProblemContent{
\ifVerboseLocation This is Derivative Compute Question 0050. \\ \fi
\begin{problem}

Find the limit.  Use L'H$\hat{o}$pital's rule where appropriate.

\input{Derivative-Compute-0050.HELP.tex}

\[\lim\limits_{x\to\infty} {\left(\frac{7}{x} + 1\right)}^{6 \, x} + 6=\answer{e^{42} + 6}\]
\end{problem}}

%%%%%%%%%%%%%%%%%%%%%%

\latexProblemContent{
\ifVerboseLocation This is Derivative Compute Question 0050. \\ \fi
\begin{problem}

Find the limit.  Use L'H$\hat{o}$pital's rule where appropriate.

\input{Derivative-Compute-0050.HELP.tex}

\[\lim\limits_{x\to\infty} {\left(\frac{8}{x} + 1\right)}^{-9 \, x} + 12=\answer{e^{\left(-72\right)} + 12}\]
\end{problem}}

%%%%%%%%%%%%%%%%%%%%%%

\latexProblemContent{
\ifVerboseLocation This is Derivative Compute Question 0050. \\ \fi
\begin{problem}

Find the limit.  Use L'H$\hat{o}$pital's rule where appropriate.

\input{Derivative-Compute-0050.HELP.tex}

\[\lim\limits_{x\to\infty} {\left(\frac{9}{x} + 1\right)}^{-8 \, x}=\answer{e^{\left(-72\right)}}\]
\end{problem}}

%%%%%%%%%%%%%%%%%%%%%%

\latexProblemContent{
\ifVerboseLocation This is Derivative Compute Question 0050. \\ \fi
\begin{problem}

Find the limit.  Use L'H$\hat{o}$pital's rule where appropriate.

\input{Derivative-Compute-0050.HELP.tex}

\[\lim\limits_{x\to\infty} {\left(-\frac{1}{x} + 1\right)}^{4 \, x} + 20=\answer{e^{\left(-4\right)} + 20}\]
\end{problem}}

%%%%%%%%%%%%%%%%%%%%%%

\latexProblemContent{
\ifVerboseLocation This is Derivative Compute Question 0050. \\ \fi
\begin{problem}

Find the limit.  Use L'H$\hat{o}$pital's rule where appropriate.

\input{Derivative-Compute-0050.HELP.tex}

\[\lim\limits_{x\to\infty} {\left(-\frac{8}{x} + 1\right)}^{-3 \, x} + 18=\answer{e^{24} + 18}\]
\end{problem}}

%%%%%%%%%%%%%%%%%%%%%%

\latexProblemContent{
\ifVerboseLocation This is Derivative Compute Question 0050. \\ \fi
\begin{problem}

Find the limit.  Use L'H$\hat{o}$pital's rule where appropriate.

\input{Derivative-Compute-0050.HELP.tex}

\[\lim\limits_{x\to\infty} {\left(-\frac{4}{x} + 1\right)}^{-3 \, x} - 8=\answer{e^{12} - 8}\]
\end{problem}}

%%%%%%%%%%%%%%%%%%%%%%

\latexProblemContent{
\ifVerboseLocation This is Derivative Compute Question 0050. \\ \fi
\begin{problem}

Find the limit.  Use L'H$\hat{o}$pital's rule where appropriate.

\input{Derivative-Compute-0050.HELP.tex}

\[\lim\limits_{x\to\infty} {\left(-\frac{7}{x} + 1\right)}^{5 \, x} - 18=\answer{e^{\left(-35\right)} - 18}\]
\end{problem}}

%%%%%%%%%%%%%%%%%%%%%%

\latexProblemContent{
\ifVerboseLocation This is Derivative Compute Question 0050. \\ \fi
\begin{problem}

Find the limit.  Use L'H$\hat{o}$pital's rule where appropriate.

\input{Derivative-Compute-0050.HELP.tex}

\[\lim\limits_{x\to\infty} {\left(-\frac{7}{x} + 1\right)}^{-x} + 11=\answer{e^{7} + 11}\]
\end{problem}}

%%%%%%%%%%%%%%%%%%%%%%

\latexProblemContent{
\ifVerboseLocation This is Derivative Compute Question 0050. \\ \fi
\begin{problem}

Find the limit.  Use L'H$\hat{o}$pital's rule where appropriate.

\input{Derivative-Compute-0050.HELP.tex}

\[\lim\limits_{x\to\infty} {\left(\frac{8}{x} + 1\right)}^{10 \, x} + 3=\answer{e^{80} + 3}\]
\end{problem}}

%%%%%%%%%%%%%%%%%%%%%%

\latexProblemContent{
\ifVerboseLocation This is Derivative Compute Question 0050. \\ \fi
\begin{problem}

Find the limit.  Use L'H$\hat{o}$pital's rule where appropriate.

\input{Derivative-Compute-0050.HELP.tex}

\[\lim\limits_{x\to\infty} {\left(-\frac{6}{x} + 1\right)}^{7 \, x} + 20=\answer{e^{\left(-42\right)} + 20}\]
\end{problem}}

%%%%%%%%%%%%%%%%%%%%%%

\latexProblemContent{
\ifVerboseLocation This is Derivative Compute Question 0050. \\ \fi
\begin{problem}

Find the limit.  Use L'H$\hat{o}$pital's rule where appropriate.

\input{Derivative-Compute-0050.HELP.tex}

\[\lim\limits_{x\to\infty} {\left(-\frac{4}{x} + 1\right)}^{3 \, x} + 7=\answer{e^{\left(-12\right)} + 7}\]
\end{problem}}

%%%%%%%%%%%%%%%%%%%%%%

\latexProblemContent{
\ifVerboseLocation This is Derivative Compute Question 0050. \\ \fi
\begin{problem}

Find the limit.  Use L'H$\hat{o}$pital's rule where appropriate.

\input{Derivative-Compute-0050.HELP.tex}

\[\lim\limits_{x\to\infty} {\left(\frac{3}{x} + 1\right)}^{9 \, x} - 5=\answer{e^{27} - 5}\]
\end{problem}}

%%%%%%%%%%%%%%%%%%%%%%

\latexProblemContent{
\ifVerboseLocation This is Derivative Compute Question 0050. \\ \fi
\begin{problem}

Find the limit.  Use L'H$\hat{o}$pital's rule where appropriate.

\input{Derivative-Compute-0050.HELP.tex}

\[\lim\limits_{x\to\infty} {\left(\frac{10}{x} + 1\right)}^{3 \, x} + 11=\answer{e^{30} + 11}\]
\end{problem}}

%%%%%%%%%%%%%%%%%%%%%%

\latexProblemContent{
\ifVerboseLocation This is Derivative Compute Question 0050. \\ \fi
\begin{problem}

Find the limit.  Use L'H$\hat{o}$pital's rule where appropriate.

\input{Derivative-Compute-0050.HELP.tex}

\[\lim\limits_{x\to\infty} {\left(-\frac{2}{x} + 1\right)}^{-5 \, x} + 18=\answer{e^{10} + 18}\]
\end{problem}}

%%%%%%%%%%%%%%%%%%%%%%

\latexProblemContent{
\ifVerboseLocation This is Derivative Compute Question 0050. \\ \fi
\begin{problem}

Find the limit.  Use L'H$\hat{o}$pital's rule where appropriate.

\input{Derivative-Compute-0050.HELP.tex}

\[\lim\limits_{x\to\infty} {\left(-\frac{8}{x} + 1\right)}^{5 \, x} - 7=\answer{e^{\left(-40\right)} - 7}\]
\end{problem}}

%%%%%%%%%%%%%%%%%%%%%%

\latexProblemContent{
\ifVerboseLocation This is Derivative Compute Question 0050. \\ \fi
\begin{problem}

Find the limit.  Use L'H$\hat{o}$pital's rule where appropriate.

\input{Derivative-Compute-0050.HELP.tex}

\[\lim\limits_{x\to\infty} {\left(-\frac{7}{x} + 1\right)}^{3 \, x} - 12=\answer{e^{\left(-21\right)} - 12}\]
\end{problem}}

%%%%%%%%%%%%%%%%%%%%%%

\latexProblemContent{
\ifVerboseLocation This is Derivative Compute Question 0050. \\ \fi
\begin{problem}

Find the limit.  Use L'H$\hat{o}$pital's rule where appropriate.

\input{Derivative-Compute-0050.HELP.tex}

\[\lim\limits_{x\to\infty} {\left(-\frac{3}{x} + 1\right)}^{-10 \, x} - 20=\answer{e^{30} - 20}\]
\end{problem}}

%%%%%%%%%%%%%%%%%%%%%%

\latexProblemContent{
\ifVerboseLocation This is Derivative Compute Question 0050. \\ \fi
\begin{problem}

Find the limit.  Use L'H$\hat{o}$pital's rule where appropriate.

\input{Derivative-Compute-0050.HELP.tex}

\[\lim\limits_{x\to\infty} {\left(\frac{4}{x} + 1\right)}^{7 \, x} - 6=\answer{e^{28} - 6}\]
\end{problem}}

%%%%%%%%%%%%%%%%%%%%%%

\latexProblemContent{
\ifVerboseLocation This is Derivative Compute Question 0050. \\ \fi
\begin{problem}

Find the limit.  Use L'H$\hat{o}$pital's rule where appropriate.

\input{Derivative-Compute-0050.HELP.tex}

\[\lim\limits_{x\to\infty} {\left(\frac{3}{x} + 1\right)}^{2 \, x} + 11=\answer{e^{6} + 11}\]
\end{problem}}

%%%%%%%%%%%%%%%%%%%%%%

\latexProblemContent{
\ifVerboseLocation This is Derivative Compute Question 0050. \\ \fi
\begin{problem}

Find the limit.  Use L'H$\hat{o}$pital's rule where appropriate.

\input{Derivative-Compute-0050.HELP.tex}

\[\lim\limits_{x\to\infty} {\left(\frac{3}{x} + 1\right)}^{9 \, x} + 14=\answer{e^{27} + 14}\]
\end{problem}}

%%%%%%%%%%%%%%%%%%%%%%

\latexProblemContent{
\ifVerboseLocation This is Derivative Compute Question 0050. \\ \fi
\begin{problem}

Find the limit.  Use L'H$\hat{o}$pital's rule where appropriate.

\input{Derivative-Compute-0050.HELP.tex}

\[\lim\limits_{x\to\infty} {\left(-\frac{10}{x} + 1\right)}^{5 \, x} - 14=\answer{e^{\left(-50\right)} - 14}\]
\end{problem}}

%%%%%%%%%%%%%%%%%%%%%%

\latexProblemContent{
\ifVerboseLocation This is Derivative Compute Question 0050. \\ \fi
\begin{problem}

Find the limit.  Use L'H$\hat{o}$pital's rule where appropriate.

\input{Derivative-Compute-0050.HELP.tex}

\[\lim\limits_{x\to\infty} {\left(\frac{10}{x} + 1\right)}^{-10 \, x} - 11=\answer{e^{\left(-100\right)} - 11}\]
\end{problem}}

%%%%%%%%%%%%%%%%%%%%%%

\latexProblemContent{
\ifVerboseLocation This is Derivative Compute Question 0050. \\ \fi
\begin{problem}

Find the limit.  Use L'H$\hat{o}$pital's rule where appropriate.

\input{Derivative-Compute-0050.HELP.tex}

\[\lim\limits_{x\to\infty} {\left(-\frac{3}{x} + 1\right)}^{-3 \, x} + 2=\answer{e^{9} + 2}\]
\end{problem}}

%%%%%%%%%%%%%%%%%%%%%%

\latexProblemContent{
\ifVerboseLocation This is Derivative Compute Question 0050. \\ \fi
\begin{problem}

Find the limit.  Use L'H$\hat{o}$pital's rule where appropriate.

\input{Derivative-Compute-0050.HELP.tex}

\[\lim\limits_{x\to\infty} {\left(\frac{10}{x} + 1\right)}^{-3 \, x} + 12=\answer{e^{\left(-30\right)} + 12}\]
\end{problem}}

%%%%%%%%%%%%%%%%%%%%%%

\latexProblemContent{
\ifVerboseLocation This is Derivative Compute Question 0050. \\ \fi
\begin{problem}

Find the limit.  Use L'H$\hat{o}$pital's rule where appropriate.

\input{Derivative-Compute-0050.HELP.tex}

\[\lim\limits_{x\to\infty} {\left(\frac{4}{x} + 1\right)}^{5 \, x} - 4=\answer{e^{20} - 4}\]
\end{problem}}

%%%%%%%%%%%%%%%%%%%%%%

\latexProblemContent{
\ifVerboseLocation This is Derivative Compute Question 0050. \\ \fi
\begin{problem}

Find the limit.  Use L'H$\hat{o}$pital's rule where appropriate.

\input{Derivative-Compute-0050.HELP.tex}

\[\lim\limits_{x\to\infty} {\left(\frac{8}{x} + 1\right)}^{2 \, x} + 6=\answer{e^{16} + 6}\]
\end{problem}}

%%%%%%%%%%%%%%%%%%%%%%

\latexProblemContent{
\ifVerboseLocation This is Derivative Compute Question 0050. \\ \fi
\begin{problem}

Find the limit.  Use L'H$\hat{o}$pital's rule where appropriate.

\input{Derivative-Compute-0050.HELP.tex}

\[\lim\limits_{x\to\infty} {\left(-\frac{4}{x} + 1\right)}^{-x} + 18=\answer{e^{4} + 18}\]
\end{problem}}

%%%%%%%%%%%%%%%%%%%%%%

\latexProblemContent{
\ifVerboseLocation This is Derivative Compute Question 0050. \\ \fi
\begin{problem}

Find the limit.  Use L'H$\hat{o}$pital's rule where appropriate.

\input{Derivative-Compute-0050.HELP.tex}

\[\lim\limits_{x\to\infty} {\left(\frac{4}{x} + 1\right)}^{x} - 7=\answer{e^{4} - 7}\]
\end{problem}}

%%%%%%%%%%%%%%%%%%%%%%

\latexProblemContent{
\ifVerboseLocation This is Derivative Compute Question 0050. \\ \fi
\begin{problem}

Find the limit.  Use L'H$\hat{o}$pital's rule where appropriate.

\input{Derivative-Compute-0050.HELP.tex}

\[\lim\limits_{x\to\infty} {\left(\frac{3}{x} + 1\right)}^{-10 \, x} - 19=\answer{e^{\left(-30\right)} - 19}\]
\end{problem}}

%%%%%%%%%%%%%%%%%%%%%%

\latexProblemContent{
\ifVerboseLocation This is Derivative Compute Question 0050. \\ \fi
\begin{problem}

Find the limit.  Use L'H$\hat{o}$pital's rule where appropriate.

\input{Derivative-Compute-0050.HELP.tex}

\[\lim\limits_{x\to\infty} {\left(\frac{5}{x} + 1\right)}^{-6 \, x} + 14=\answer{e^{\left(-30\right)} + 14}\]
\end{problem}}

%%%%%%%%%%%%%%%%%%%%%%

\latexProblemContent{
\ifVerboseLocation This is Derivative Compute Question 0050. \\ \fi
\begin{problem}

Find the limit.  Use L'H$\hat{o}$pital's rule where appropriate.

\input{Derivative-Compute-0050.HELP.tex}

\[\lim\limits_{x\to\infty} {\left(\frac{8}{x} + 1\right)}^{-9 \, x} + 17=\answer{e^{\left(-72\right)} + 17}\]
\end{problem}}

%%%%%%%%%%%%%%%%%%%%%%

\latexProblemContent{
\ifVerboseLocation This is Derivative Compute Question 0050. \\ \fi
\begin{problem}

Find the limit.  Use L'H$\hat{o}$pital's rule where appropriate.

\input{Derivative-Compute-0050.HELP.tex}

\[\lim\limits_{x\to\infty} {\left(-\frac{6}{x} + 1\right)}^{2 \, x} + 17=\answer{e^{\left(-12\right)} + 17}\]
\end{problem}}

%%%%%%%%%%%%%%%%%%%%%%

\latexProblemContent{
\ifVerboseLocation This is Derivative Compute Question 0050. \\ \fi
\begin{problem}

Find the limit.  Use L'H$\hat{o}$pital's rule where appropriate.

\input{Derivative-Compute-0050.HELP.tex}

\[\lim\limits_{x\to\infty} {\left(\frac{5}{x} + 1\right)}^{-8 \, x}=\answer{e^{\left(-40\right)}}\]
\end{problem}}

%%%%%%%%%%%%%%%%%%%%%%

\latexProblemContent{
\ifVerboseLocation This is Derivative Compute Question 0050. \\ \fi
\begin{problem}

Find the limit.  Use L'H$\hat{o}$pital's rule where appropriate.

\input{Derivative-Compute-0050.HELP.tex}

\[\lim\limits_{x\to\infty} {\left(-\frac{8}{x} + 1\right)}^{-x} - 8=\answer{e^{8} - 8}\]
\end{problem}}

%%%%%%%%%%%%%%%%%%%%%%

\latexProblemContent{
\ifVerboseLocation This is Derivative Compute Question 0050. \\ \fi
\begin{problem}

Find the limit.  Use L'H$\hat{o}$pital's rule where appropriate.

\input{Derivative-Compute-0050.HELP.tex}

\[\lim\limits_{x\to\infty} {\left(\frac{10}{x} + 1\right)}^{-7 \, x} + 4=\answer{e^{\left(-70\right)} + 4}\]
\end{problem}}

%%%%%%%%%%%%%%%%%%%%%%

\latexProblemContent{
\ifVerboseLocation This is Derivative Compute Question 0050. \\ \fi
\begin{problem}

Find the limit.  Use L'H$\hat{o}$pital's rule where appropriate.

\input{Derivative-Compute-0050.HELP.tex}

\[\lim\limits_{x\to\infty} {\left(\frac{9}{x} + 1\right)}^{9 \, x} + 1=\answer{e^{81} + 1}\]
\end{problem}}

%%%%%%%%%%%%%%%%%%%%%%

\latexProblemContent{
\ifVerboseLocation This is Derivative Compute Question 0050. \\ \fi
\begin{problem}

Find the limit.  Use L'H$\hat{o}$pital's rule where appropriate.

\input{Derivative-Compute-0050.HELP.tex}

\[\lim\limits_{x\to\infty} {\left(-\frac{10}{x} + 1\right)}^{-4 \, x} - 16=\answer{e^{40} - 16}\]
\end{problem}}

%%%%%%%%%%%%%%%%%%%%%%

\latexProblemContent{
\ifVerboseLocation This is Derivative Compute Question 0050. \\ \fi
\begin{problem}

Find the limit.  Use L'H$\hat{o}$pital's rule where appropriate.

\input{Derivative-Compute-0050.HELP.tex}

\[\lim\limits_{x\to\infty} {\left(\frac{1}{x} + 1\right)}^{-5 \, x} + 19=\answer{e^{\left(-5\right)} + 19}\]
\end{problem}}

%%%%%%%%%%%%%%%%%%%%%%

\latexProblemContent{
\ifVerboseLocation This is Derivative Compute Question 0050. \\ \fi
\begin{problem}

Find the limit.  Use L'H$\hat{o}$pital's rule where appropriate.

\input{Derivative-Compute-0050.HELP.tex}

\[\lim\limits_{x\to\infty} {\left(-\frac{4}{x} + 1\right)}^{-2 \, x} + 1=\answer{e^{8} + 1}\]
\end{problem}}

%%%%%%%%%%%%%%%%%%%%%%

\latexProblemContent{
\ifVerboseLocation This is Derivative Compute Question 0050. \\ \fi
\begin{problem}

Find the limit.  Use L'H$\hat{o}$pital's rule where appropriate.

\input{Derivative-Compute-0050.HELP.tex}

\[\lim\limits_{x\to\infty} {\left(-\frac{7}{x} + 1\right)}^{10 \, x} + 2=\answer{e^{\left(-70\right)} + 2}\]
\end{problem}}

%%%%%%%%%%%%%%%%%%%%%%

\latexProblemContent{
\ifVerboseLocation This is Derivative Compute Question 0050. \\ \fi
\begin{problem}

Find the limit.  Use L'H$\hat{o}$pital's rule where appropriate.

\input{Derivative-Compute-0050.HELP.tex}

\[\lim\limits_{x\to\infty} {\left(-\frac{5}{x} + 1\right)}^{7 \, x} - 18=\answer{e^{\left(-35\right)} - 18}\]
\end{problem}}

%%%%%%%%%%%%%%%%%%%%%%

\latexProblemContent{
\ifVerboseLocation This is Derivative Compute Question 0050. \\ \fi
\begin{problem}

Find the limit.  Use L'H$\hat{o}$pital's rule where appropriate.

\input{Derivative-Compute-0050.HELP.tex}

\[\lim\limits_{x\to\infty} {\left(\frac{9}{x} + 1\right)}^{4 \, x} + 19=\answer{e^{36} + 19}\]
\end{problem}}

%%%%%%%%%%%%%%%%%%%%%%

\latexProblemContent{
\ifVerboseLocation This is Derivative Compute Question 0050. \\ \fi
\begin{problem}

Find the limit.  Use L'H$\hat{o}$pital's rule where appropriate.

\input{Derivative-Compute-0050.HELP.tex}

\[\lim\limits_{x\to\infty} {\left(\frac{10}{x} + 1\right)}^{8 \, x} + 5=\answer{e^{80} + 5}\]
\end{problem}}

%%%%%%%%%%%%%%%%%%%%%%

\latexProblemContent{
\ifVerboseLocation This is Derivative Compute Question 0050. \\ \fi
\begin{problem}

Find the limit.  Use L'H$\hat{o}$pital's rule where appropriate.

\input{Derivative-Compute-0050.HELP.tex}

\[\lim\limits_{x\to\infty} {\left(-\frac{10}{x} + 1\right)}^{-9 \, x} - 20=\answer{e^{90} - 20}\]
\end{problem}}

%%%%%%%%%%%%%%%%%%%%%%

\latexProblemContent{
\ifVerboseLocation This is Derivative Compute Question 0050. \\ \fi
\begin{problem}

Find the limit.  Use L'H$\hat{o}$pital's rule where appropriate.

\input{Derivative-Compute-0050.HELP.tex}

\[\lim\limits_{x\to\infty} {\left(-\frac{10}{x} + 1\right)}^{4 \, x} + 10=\answer{e^{\left(-40\right)} + 10}\]
\end{problem}}

%%%%%%%%%%%%%%%%%%%%%%

\latexProblemContent{
\ifVerboseLocation This is Derivative Compute Question 0050. \\ \fi
\begin{problem}

Find the limit.  Use L'H$\hat{o}$pital's rule where appropriate.

\input{Derivative-Compute-0050.HELP.tex}

\[\lim\limits_{x\to\infty} {\left(-\frac{2}{x} + 1\right)}^{5 \, x} - 1=\answer{e^{\left(-10\right)} - 1}\]
\end{problem}}

%%%%%%%%%%%%%%%%%%%%%%

\latexProblemContent{
\ifVerboseLocation This is Derivative Compute Question 0050. \\ \fi
\begin{problem}

Find the limit.  Use L'H$\hat{o}$pital's rule where appropriate.

\input{Derivative-Compute-0050.HELP.tex}

\[\lim\limits_{x\to\infty} {\left(\frac{6}{x} + 1\right)}^{-5 \, x} + 14=\answer{e^{\left(-30\right)} + 14}\]
\end{problem}}

%%%%%%%%%%%%%%%%%%%%%%

\latexProblemContent{
\ifVerboseLocation This is Derivative Compute Question 0050. \\ \fi
\begin{problem}

Find the limit.  Use L'H$\hat{o}$pital's rule where appropriate.

\input{Derivative-Compute-0050.HELP.tex}

\[\lim\limits_{x\to\infty} {\left(\frac{2}{x} + 1\right)}^{8 \, x} - 1=\answer{e^{16} - 1}\]
\end{problem}}

%%%%%%%%%%%%%%%%%%%%%%

\latexProblemContent{
\ifVerboseLocation This is Derivative Compute Question 0050. \\ \fi
\begin{problem}

Find the limit.  Use L'H$\hat{o}$pital's rule where appropriate.

\input{Derivative-Compute-0050.HELP.tex}

\[\lim\limits_{x\to\infty} {\left(-\frac{10}{x} + 1\right)}^{-6 \, x} + 14=\answer{e^{60} + 14}\]
\end{problem}}

%%%%%%%%%%%%%%%%%%%%%%

\latexProblemContent{
\ifVerboseLocation This is Derivative Compute Question 0050. \\ \fi
\begin{problem}

Find the limit.  Use L'H$\hat{o}$pital's rule where appropriate.

\input{Derivative-Compute-0050.HELP.tex}

\[\lim\limits_{x\to\infty} {\left(\frac{1}{x} + 1\right)}^{-6 \, x} - 1=\answer{e^{\left(-6\right)} - 1}\]
\end{problem}}

%%%%%%%%%%%%%%%%%%%%%%

\latexProblemContent{
\ifVerboseLocation This is Derivative Compute Question 0050. \\ \fi
\begin{problem}

Find the limit.  Use L'H$\hat{o}$pital's rule where appropriate.

\input{Derivative-Compute-0050.HELP.tex}

\[\lim\limits_{x\to\infty} {\left(-\frac{1}{x} + 1\right)}^{-7 \, x} + 15=\answer{e^{7} + 15}\]
\end{problem}}

%%%%%%%%%%%%%%%%%%%%%%

\latexProblemContent{
\ifVerboseLocation This is Derivative Compute Question 0050. \\ \fi
\begin{problem}

Find the limit.  Use L'H$\hat{o}$pital's rule where appropriate.

\input{Derivative-Compute-0050.HELP.tex}

\[\lim\limits_{x\to\infty} {\left(\frac{5}{x} + 1\right)}^{-6 \, x} - 9=\answer{e^{\left(-30\right)} - 9}\]
\end{problem}}

%%%%%%%%%%%%%%%%%%%%%%

\latexProblemContent{
\ifVerboseLocation This is Derivative Compute Question 0050. \\ \fi
\begin{problem}

Find the limit.  Use L'H$\hat{o}$pital's rule where appropriate.

\input{Derivative-Compute-0050.HELP.tex}

\[\lim\limits_{x\to\infty} {\left(-\frac{4}{x} + 1\right)}^{-9 \, x} + 5=\answer{e^{36} + 5}\]
\end{problem}}

%%%%%%%%%%%%%%%%%%%%%%

\latexProblemContent{
\ifVerboseLocation This is Derivative Compute Question 0050. \\ \fi
\begin{problem}

Find the limit.  Use L'H$\hat{o}$pital's rule where appropriate.

\input{Derivative-Compute-0050.HELP.tex}

\[\lim\limits_{x\to\infty} {\left(-\frac{1}{x} + 1\right)}^{-3 \, x} - 17=\answer{e^{3} - 17}\]
\end{problem}}

%%%%%%%%%%%%%%%%%%%%%%

\latexProblemContent{
\ifVerboseLocation This is Derivative Compute Question 0050. \\ \fi
\begin{problem}

Find the limit.  Use L'H$\hat{o}$pital's rule where appropriate.

\input{Derivative-Compute-0050.HELP.tex}

\[\lim\limits_{x\to\infty} {\left(-\frac{6}{x} + 1\right)}^{-10 \, x} + 12=\answer{e^{60} + 12}\]
\end{problem}}

%%%%%%%%%%%%%%%%%%%%%%

\latexProblemContent{
\ifVerboseLocation This is Derivative Compute Question 0050. \\ \fi
\begin{problem}

Find the limit.  Use L'H$\hat{o}$pital's rule where appropriate.

\input{Derivative-Compute-0050.HELP.tex}

\[\lim\limits_{x\to\infty} {\left(\frac{2}{x} + 1\right)}^{-x} + 3=\answer{e^{\left(-2\right)} + 3}\]
\end{problem}}

%%%%%%%%%%%%%%%%%%%%%%

\latexProblemContent{
\ifVerboseLocation This is Derivative Compute Question 0050. \\ \fi
\begin{problem}

Find the limit.  Use L'H$\hat{o}$pital's rule where appropriate.

\input{Derivative-Compute-0050.HELP.tex}

\[\lim\limits_{x\to\infty} {\left(-\frac{5}{x} + 1\right)}^{6 \, x} + 11=\answer{e^{\left(-30\right)} + 11}\]
\end{problem}}

%%%%%%%%%%%%%%%%%%%%%%

\latexProblemContent{
\ifVerboseLocation This is Derivative Compute Question 0050. \\ \fi
\begin{problem}

Find the limit.  Use L'H$\hat{o}$pital's rule where appropriate.

\input{Derivative-Compute-0050.HELP.tex}

\[\lim\limits_{x\to\infty} {\left(-\frac{10}{x} + 1\right)}^{-2 \, x} - 4=\answer{e^{20} - 4}\]
\end{problem}}\fi             %% end of \ifproblemToFind near top of file
\fi             %% end of \ifquestionCount near top of file
\ProblemFileFooter