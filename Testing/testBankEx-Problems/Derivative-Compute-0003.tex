% Ans        : ShortAns
% File       : 0003
% Sub        : Trig, Product-Rule
% Topic      : Derivative
% Type       : Compute

\ProblemFileHeader{500}
\ifquestionPull
\ifproblemToFind\latexProblemContent{
\ifVerboseLocation This is Derivative Compute Question 0003. \\ \fi
\begin{problem}

Compute the following derivative:

\input{Derivative-Compute-0003.HELP.tex}

\[\dfrac{d}{dx}\left(-6 \, \cos\left(-\frac{2}{3} \, \pi x\right) \cos\left(-2 \, \pi x\right)\right)=\answer{12 \, \pi \cos\left(\frac{2}{3} \, \pi x\right) \sin\left(2 \, \pi x\right) + 4 \, \pi \cos\left(2 \, \pi x\right) \sin\left(\frac{2}{3} \, \pi x\right)}\]

\end{problem}}

%%%%%%%%%%%%%%%%%%%%%%

\latexProblemContent{
\ifVerboseLocation This is Derivative Compute Question 0003. \\ \fi
\begin{problem}

Compute the following derivative:

\input{Derivative-Compute-0003.HELP.tex}

\[\dfrac{d}{dx}\left(-3 \, \sin\left(\frac{5}{3} \, \pi x\right) \tan\left(-\frac{2}{3} \, \pi x\right)\right)=\answer{2 \, \pi \sec\left(\frac{2}{3} \, \pi x\right)^{2} \sin\left(\frac{5}{3} \, \pi x\right) + 5 \, \pi \cos\left(\frac{5}{3} \, \pi x\right) \tan\left(\frac{2}{3} \, \pi x\right)}\]

\end{problem}}

%%%%%%%%%%%%%%%%%%%%%%

\latexProblemContent{
\ifVerboseLocation This is Derivative Compute Question 0003. \\ \fi
\begin{problem}

Compute the following derivative:

\input{Derivative-Compute-0003.HELP.tex}

\[\dfrac{d}{dx}\left(3 \, \cos\left(\frac{1}{2} \, \pi x\right) \sin\left(3 \, \pi x\right)\right)=\answer{9 \, \pi \cos\left(3 \, \pi x\right) \cos\left(\frac{1}{2} \, \pi x\right) - \frac{3}{2} \, \pi \sin\left(3 \, \pi x\right) \sin\left(\frac{1}{2} \, \pi x\right)}\]

\end{problem}}

%%%%%%%%%%%%%%%%%%%%%%

\latexProblemContent{
\ifVerboseLocation This is Derivative Compute Question 0003. \\ \fi
\begin{problem}

Compute the following derivative:

\input{Derivative-Compute-0003.HELP.tex}

\[\dfrac{d}{dx}\left(-8 \, \cos\left(\frac{5}{6} \, \pi x\right) \tan\left(-5 \, \pi x\right)\right)=\answer{40 \, \pi \cos\left(\frac{5}{6} \, \pi x\right) \sec\left(5 \, \pi x\right)^{2} - \frac{20}{3} \, \pi \sin\left(\frac{5}{6} \, \pi x\right) \tan\left(5 \, \pi x\right)}\]

\end{problem}}

%%%%%%%%%%%%%%%%%%%%%%

\latexProblemContent{
\ifVerboseLocation This is Derivative Compute Question 0003. \\ \fi
\begin{problem}

Compute the following derivative:

\input{Derivative-Compute-0003.HELP.tex}

\[\dfrac{d}{dx}\left(-12 \, \cos\left(\frac{2}{3} \, \pi x\right) \tan\left(\pi x\right)\right)=\answer{-12 \, \pi \cos\left(\frac{2}{3} \, \pi x\right) \sec\left(\pi x\right)^{2} + 8 \, \pi \sin\left(\frac{2}{3} \, \pi x\right) \tan\left(\pi x\right)}\]

\end{problem}}

%%%%%%%%%%%%%%%%%%%%%%

\latexProblemContent{
\ifVerboseLocation This is Derivative Compute Question 0003. \\ \fi
\begin{problem}

Compute the following derivative:

\input{Derivative-Compute-0003.HELP.tex}

\[\dfrac{d}{dx}\left(6 \, \cos\left(\frac{1}{3} \, \pi x\right) \cos\left(-\frac{5}{2} \, \pi x\right)\right)=\answer{-15 \, \pi \cos\left(\frac{1}{3} \, \pi x\right) \sin\left(\frac{5}{2} \, \pi x\right) - 2 \, \pi \cos\left(\frac{5}{2} \, \pi x\right) \sin\left(\frac{1}{3} \, \pi x\right)}\]

\end{problem}}

%%%%%%%%%%%%%%%%%%%%%%

\latexProblemContent{
\ifVerboseLocation This is Derivative Compute Question 0003. \\ \fi
\begin{problem}

Compute the following derivative:

\input{Derivative-Compute-0003.HELP.tex}

\[\dfrac{d}{dx}\left(-3 \, \sin\left(-\frac{3}{2} \, \pi x\right) \sin\left(-2 \, \pi x\right)\right)=\answer{-\frac{9}{2} \, \pi \cos\left(\frac{3}{2} \, \pi x\right) \sin\left(2 \, \pi x\right) - 6 \, \pi \cos\left(2 \, \pi x\right) \sin\left(\frac{3}{2} \, \pi x\right)}\]

\end{problem}}

%%%%%%%%%%%%%%%%%%%%%%

\latexProblemContent{
\ifVerboseLocation This is Derivative Compute Question 0003. \\ \fi
\begin{problem}

Compute the following derivative:

\input{Derivative-Compute-0003.HELP.tex}

\[\dfrac{d}{dx}\left(-15 \, \cos\left(-5 \, \pi x\right) \sin\left(-\frac{1}{3} \, \pi x\right)\right)=\answer{5 \, \pi \cos\left(5 \, \pi x\right) \cos\left(\frac{1}{3} \, \pi x\right) - 75 \, \pi \sin\left(5 \, \pi x\right) \sin\left(\frac{1}{3} \, \pi x\right)}\]

\end{problem}}

%%%%%%%%%%%%%%%%%%%%%%

\latexProblemContent{
\ifVerboseLocation This is Derivative Compute Question 0003. \\ \fi
\begin{problem}

Compute the following derivative:

\input{Derivative-Compute-0003.HELP.tex}

\[\dfrac{d}{dx}\left(3 \, \sin\left(-2 \, \pi x\right) \tan\left(\frac{1}{3} \, \pi x\right)\right)=\answer{-\pi \sec\left(\frac{1}{3} \, \pi x\right)^{2} \sin\left(2 \, \pi x\right) - 6 \, \pi \cos\left(2 \, \pi x\right) \tan\left(\frac{1}{3} \, \pi x\right)}\]

\end{problem}}

%%%%%%%%%%%%%%%%%%%%%%

\latexProblemContent{
\ifVerboseLocation This is Derivative Compute Question 0003. \\ \fi
\begin{problem}

Compute the following derivative:

\input{Derivative-Compute-0003.HELP.tex}

\[\dfrac{d}{dx}\left(10 \, \tan\left(\pi x\right) \tan\left(-\frac{1}{3} \, \pi x\right)\right)=\answer{-\frac{10}{3} \, \pi \sec\left(\frac{1}{3} \, \pi x\right)^{2} \tan\left(\pi x\right) - 10 \, \pi \sec\left(\pi x\right)^{2} \tan\left(\frac{1}{3} \, \pi x\right)}\]

\end{problem}}

%%%%%%%%%%%%%%%%%%%%%%

\latexProblemContent{
\ifVerboseLocation This is Derivative Compute Question 0003. \\ \fi
\begin{problem}

Compute the following derivative:

\input{Derivative-Compute-0003.HELP.tex}

\[\dfrac{d}{dx}\left(4 \, \cos\left(-\frac{1}{2} \, \pi x\right) \sin\left(\frac{5}{3} \, \pi x\right)\right)=\answer{\frac{20}{3} \, \pi \cos\left(\frac{5}{3} \, \pi x\right) \cos\left(\frac{1}{2} \, \pi x\right) - 2 \, \pi \sin\left(\frac{5}{3} \, \pi x\right) \sin\left(\frac{1}{2} \, \pi x\right)}\]

\end{problem}}

%%%%%%%%%%%%%%%%%%%%%%

\latexProblemContent{
\ifVerboseLocation This is Derivative Compute Question 0003. \\ \fi
\begin{problem}

Compute the following derivative:

\input{Derivative-Compute-0003.HELP.tex}

\[\dfrac{d}{dx}\left(10 \, \sin\left(\frac{1}{2} \, \pi x\right) \tan\left(\frac{1}{2} \, \pi x\right)\right)=\answer{5 \, \pi \sec\left(\frac{1}{2} \, \pi x\right)^{2} \sin\left(\frac{1}{2} \, \pi x\right) + 5 \, \pi \cos\left(\frac{1}{2} \, \pi x\right) \tan\left(\frac{1}{2} \, \pi x\right)}\]

\end{problem}}

%%%%%%%%%%%%%%%%%%%%%%

\latexProblemContent{
\ifVerboseLocation This is Derivative Compute Question 0003. \\ \fi
\begin{problem}

Compute the following derivative:

\input{Derivative-Compute-0003.HELP.tex}

\[\dfrac{d}{dx}\left(8 \, \tan\left(\frac{5}{2} \, \pi x\right) \tan\left(-\frac{2}{3} \, \pi x\right)\right)=\answer{-\frac{16}{3} \, \pi \sec\left(\frac{2}{3} \, \pi x\right)^{2} \tan\left(\frac{5}{2} \, \pi x\right) - 20 \, \pi \sec\left(\frac{5}{2} \, \pi x\right)^{2} \tan\left(\frac{2}{3} \, \pi x\right)}\]

\end{problem}}

%%%%%%%%%%%%%%%%%%%%%%

\latexProblemContent{
\ifVerboseLocation This is Derivative Compute Question 0003. \\ \fi
\begin{problem}

Compute the following derivative:

\input{Derivative-Compute-0003.HELP.tex}

\[\dfrac{d}{dx}\left(-4 \, \cos\left(-4 \, \pi x\right) \tan\left(-\frac{2}{3} \, \pi x\right)\right)=\answer{\frac{8}{3} \, \pi \cos\left(4 \, \pi x\right) \sec\left(\frac{2}{3} \, \pi x\right)^{2} - 16 \, \pi \sin\left(4 \, \pi x\right) \tan\left(\frac{2}{3} \, \pi x\right)}\]

\end{problem}}

%%%%%%%%%%%%%%%%%%%%%%

\latexProblemContent{
\ifVerboseLocation This is Derivative Compute Question 0003. \\ \fi
\begin{problem}

Compute the following derivative:

\input{Derivative-Compute-0003.HELP.tex}

\[\dfrac{d}{dx}\left(-10 \, \sin\left(\frac{5}{3} \, \pi x\right) \tan\left(-3 \, \pi x\right)\right)=\answer{30 \, \pi \sec\left(3 \, \pi x\right)^{2} \sin\left(\frac{5}{3} \, \pi x\right) + \frac{50}{3} \, \pi \cos\left(\frac{5}{3} \, \pi x\right) \tan\left(3 \, \pi x\right)}\]

\end{problem}}

%%%%%%%%%%%%%%%%%%%%%%

\latexProblemContent{
\ifVerboseLocation This is Derivative Compute Question 0003. \\ \fi
\begin{problem}

Compute the following derivative:

\input{Derivative-Compute-0003.HELP.tex}

\[\dfrac{d}{dx}\left(-20 \, \cos\left(\pi x\right) \tan\left(4 \, \pi x\right)\right)=\answer{-80 \, \pi \cos\left(\pi x\right) \sec\left(4 \, \pi x\right)^{2} + 20 \, \pi \sin\left(\pi x\right) \tan\left(4 \, \pi x\right)}\]

\end{problem}}

%%%%%%%%%%%%%%%%%%%%%%

\latexProblemContent{
\ifVerboseLocation This is Derivative Compute Question 0003. \\ \fi
\begin{problem}

Compute the following derivative:

\input{Derivative-Compute-0003.HELP.tex}

\[\dfrac{d}{dx}\left(-8 \, \sin\left(2 \, \pi x\right) \tan\left(\frac{3}{2} \, \pi x\right)\right)=\answer{-12 \, \pi \sec\left(\frac{3}{2} \, \pi x\right)^{2} \sin\left(2 \, \pi x\right) - 16 \, \pi \cos\left(2 \, \pi x\right) \tan\left(\frac{3}{2} \, \pi x\right)}\]

\end{problem}}

%%%%%%%%%%%%%%%%%%%%%%

\latexProblemContent{
\ifVerboseLocation This is Derivative Compute Question 0003. \\ \fi
\begin{problem}

Compute the following derivative:

\input{Derivative-Compute-0003.HELP.tex}

\[\dfrac{d}{dx}\left(-4 \, \sin\left(2 \, \pi x\right) \sin\left(\pi x\right)\right)=\answer{-4 \, \pi \cos\left(\pi x\right) \sin\left(2 \, \pi x\right) - 8 \, \pi \cos\left(2 \, \pi x\right) \sin\left(\pi x\right)}\]

\end{problem}}

%%%%%%%%%%%%%%%%%%%%%%

\latexProblemContent{
\ifVerboseLocation This is Derivative Compute Question 0003. \\ \fi
\begin{problem}

Compute the following derivative:

\input{Derivative-Compute-0003.HELP.tex}

\[\dfrac{d}{dx}\left(-3 \, \sin\left(\frac{4}{3} \, \pi x\right) \sin\left(\frac{2}{3} \, \pi x\right)\right)=\answer{-2 \, \pi \cos\left(\frac{2}{3} \, \pi x\right) \sin\left(\frac{4}{3} \, \pi x\right) - 4 \, \pi \cos\left(\frac{4}{3} \, \pi x\right) \sin\left(\frac{2}{3} \, \pi x\right)}\]

\end{problem}}

%%%%%%%%%%%%%%%%%%%%%%

\latexProblemContent{
\ifVerboseLocation This is Derivative Compute Question 0003. \\ \fi
\begin{problem}

Compute the following derivative:

\input{Derivative-Compute-0003.HELP.tex}

\[\dfrac{d}{dx}\left(-8 \, \cos\left(-\frac{1}{2} \, \pi x\right) \sin\left(-\frac{2}{3} \, \pi x\right)\right)=\answer{\frac{16}{3} \, \pi \cos\left(\frac{2}{3} \, \pi x\right) \cos\left(\frac{1}{2} \, \pi x\right) - 4 \, \pi \sin\left(\frac{2}{3} \, \pi x\right) \sin\left(\frac{1}{2} \, \pi x\right)}\]

\end{problem}}

%%%%%%%%%%%%%%%%%%%%%%

\latexProblemContent{
\ifVerboseLocation This is Derivative Compute Question 0003. \\ \fi
\begin{problem}

Compute the following derivative:

\input{Derivative-Compute-0003.HELP.tex}

\[\dfrac{d}{dx}\left(6 \, \sin\left(-\frac{3}{2} \, \pi x\right) \tan\left(-\frac{2}{3} \, \pi x\right)\right)=\answer{4 \, \pi \sec\left(\frac{2}{3} \, \pi x\right)^{2} \sin\left(\frac{3}{2} \, \pi x\right) + 9 \, \pi \cos\left(\frac{3}{2} \, \pi x\right) \tan\left(\frac{2}{3} \, \pi x\right)}\]

\end{problem}}

%%%%%%%%%%%%%%%%%%%%%%

\latexProblemContent{
\ifVerboseLocation This is Derivative Compute Question 0003. \\ \fi
\begin{problem}

Compute the following derivative:

\input{Derivative-Compute-0003.HELP.tex}

\[\dfrac{d}{dx}\left(25 \, \sin\left(-\frac{1}{2} \, \pi x\right) \tan\left(-\pi x\right)\right)=\answer{25 \, \pi \sec\left(\pi x\right)^{2} \sin\left(\frac{1}{2} \, \pi x\right) + \frac{25}{2} \, \pi \cos\left(\frac{1}{2} \, \pi x\right) \tan\left(\pi x\right)}\]

\end{problem}}

%%%%%%%%%%%%%%%%%%%%%%

\latexProblemContent{
\ifVerboseLocation This is Derivative Compute Question 0003. \\ \fi
\begin{problem}

Compute the following derivative:

\input{Derivative-Compute-0003.HELP.tex}

\[\dfrac{d}{dx}\left(12 \, \cos\left(-\frac{5}{2} \, \pi x\right) \sin\left(\frac{4}{3} \, \pi x\right)\right)=\answer{16 \, \pi \cos\left(\frac{5}{2} \, \pi x\right) \cos\left(\frac{4}{3} \, \pi x\right) - 30 \, \pi \sin\left(\frac{5}{2} \, \pi x\right) \sin\left(\frac{4}{3} \, \pi x\right)}\]

\end{problem}}

%%%%%%%%%%%%%%%%%%%%%%

\latexProblemContent{
\ifVerboseLocation This is Derivative Compute Question 0003. \\ \fi
\begin{problem}

Compute the following derivative:

\input{Derivative-Compute-0003.HELP.tex}

\[\dfrac{d}{dx}\left(-10 \, \cos\left(\frac{1}{3} \, \pi x\right) \sin\left(\frac{3}{2} \, \pi x\right)\right)=\answer{-15 \, \pi \cos\left(\frac{3}{2} \, \pi x\right) \cos\left(\frac{1}{3} \, \pi x\right) + \frac{10}{3} \, \pi \sin\left(\frac{3}{2} \, \pi x\right) \sin\left(\frac{1}{3} \, \pi x\right)}\]

\end{problem}}

%%%%%%%%%%%%%%%%%%%%%%

\latexProblemContent{
\ifVerboseLocation This is Derivative Compute Question 0003. \\ \fi
\begin{problem}

Compute the following derivative:

\input{Derivative-Compute-0003.HELP.tex}

\[\dfrac{d}{dx}\left(-9 \, \cos\left(-3 \, \pi x\right) \tan\left(-\frac{1}{2} \, \pi x\right)\right)=\answer{\frac{9}{2} \, \pi \cos\left(3 \, \pi x\right) \sec\left(\frac{1}{2} \, \pi x\right)^{2} - 27 \, \pi \sin\left(3 \, \pi x\right) \tan\left(\frac{1}{2} \, \pi x\right)}\]

\end{problem}}

%%%%%%%%%%%%%%%%%%%%%%

\latexProblemContent{
\ifVerboseLocation This is Derivative Compute Question 0003. \\ \fi
\begin{problem}

Compute the following derivative:

\input{Derivative-Compute-0003.HELP.tex}

\[\dfrac{d}{dx}\left(9 \, \tan\left(-\frac{2}{3} \, \pi x\right) \tan\left(-\pi x\right)\right)=\answer{6 \, \pi \sec\left(\frac{2}{3} \, \pi x\right)^{2} \tan\left(\pi x\right) + 9 \, \pi \sec\left(\pi x\right)^{2} \tan\left(\frac{2}{3} \, \pi x\right)}\]

\end{problem}}

%%%%%%%%%%%%%%%%%%%%%%

\latexProblemContent{
\ifVerboseLocation This is Derivative Compute Question 0003. \\ \fi
\begin{problem}

Compute the following derivative:

\input{Derivative-Compute-0003.HELP.tex}

\[\dfrac{d}{dx}\left(-6 \, \cos\left(\pi x\right) \sin\left(-\frac{1}{2} \, \pi x\right)\right)=\answer{3 \, \pi \cos\left(\pi x\right) \cos\left(\frac{1}{2} \, \pi x\right) - 6 \, \pi \sin\left(\pi x\right) \sin\left(\frac{1}{2} \, \pi x\right)}\]

\end{problem}}

%%%%%%%%%%%%%%%%%%%%%%

\latexProblemContent{
\ifVerboseLocation This is Derivative Compute Question 0003. \\ \fi
\begin{problem}

Compute the following derivative:

\input{Derivative-Compute-0003.HELP.tex}

\[\dfrac{d}{dx}\left(-2 \, \cos\left(-\frac{5}{2} \, \pi x\right) \tan\left(\frac{2}{3} \, \pi x\right)\right)=\answer{-\frac{4}{3} \, \pi \cos\left(\frac{5}{2} \, \pi x\right) \sec\left(\frac{2}{3} \, \pi x\right)^{2} + 5 \, \pi \sin\left(\frac{5}{2} \, \pi x\right) \tan\left(\frac{2}{3} \, \pi x\right)}\]

\end{problem}}

%%%%%%%%%%%%%%%%%%%%%%

\latexProblemContent{
\ifVerboseLocation This is Derivative Compute Question 0003. \\ \fi
\begin{problem}

Compute the following derivative:

\input{Derivative-Compute-0003.HELP.tex}

\[\dfrac{d}{dx}\left(16 \, \cos\left(\frac{1}{3} \, \pi x\right) \sin\left(-\frac{1}{2} \, \pi x\right)\right)=\answer{-8 \, \pi \cos\left(\frac{1}{2} \, \pi x\right) \cos\left(\frac{1}{3} \, \pi x\right) + \frac{16}{3} \, \pi \sin\left(\frac{1}{2} \, \pi x\right) \sin\left(\frac{1}{3} \, \pi x\right)}\]

\end{problem}}

%%%%%%%%%%%%%%%%%%%%%%

\latexProblemContent{
\ifVerboseLocation This is Derivative Compute Question 0003. \\ \fi
\begin{problem}

Compute the following derivative:

\input{Derivative-Compute-0003.HELP.tex}

\[\dfrac{d}{dx}\left(3 \, \cos\left(\frac{1}{3} \, \pi x\right) \sin\left(4 \, \pi x\right)\right)=\answer{12 \, \pi \cos\left(4 \, \pi x\right) \cos\left(\frac{1}{3} \, \pi x\right) - \pi \sin\left(4 \, \pi x\right) \sin\left(\frac{1}{3} \, \pi x\right)}\]

\end{problem}}

%%%%%%%%%%%%%%%%%%%%%%

\latexProblemContent{
\ifVerboseLocation This is Derivative Compute Question 0003. \\ \fi
\begin{problem}

Compute the following derivative:

\input{Derivative-Compute-0003.HELP.tex}

\[\dfrac{d}{dx}\left(10 \, \cos\left(-\pi x\right) \tan\left(-\frac{3}{2} \, \pi x\right)\right)=\answer{-15 \, \pi \cos\left(\pi x\right) \sec\left(\frac{3}{2} \, \pi x\right)^{2} + 10 \, \pi \sin\left(\pi x\right) \tan\left(\frac{3}{2} \, \pi x\right)}\]

\end{problem}}

%%%%%%%%%%%%%%%%%%%%%%

\latexProblemContent{
\ifVerboseLocation This is Derivative Compute Question 0003. \\ \fi
\begin{problem}

Compute the following derivative:

\input{Derivative-Compute-0003.HELP.tex}

\[\dfrac{d}{dx}\left(6 \, \cos\left(\pi x\right) \sin\left(\frac{2}{3} \, \pi x\right)\right)=\answer{4 \, \pi \cos\left(\pi x\right) \cos\left(\frac{2}{3} \, \pi x\right) - 6 \, \pi \sin\left(\pi x\right) \sin\left(\frac{2}{3} \, \pi x\right)}\]

\end{problem}}

%%%%%%%%%%%%%%%%%%%%%%

\latexProblemContent{
\ifVerboseLocation This is Derivative Compute Question 0003. \\ \fi
\begin{problem}

Compute the following derivative:

\input{Derivative-Compute-0003.HELP.tex}

\[\dfrac{d}{dx}\left(\sin\left(-\frac{2}{3} \, \pi x\right) \tan\left(\frac{2}{3} \, \pi x\right)\right)=\answer{-\frac{2}{3} \, \pi \sec\left(\frac{2}{3} \, \pi x\right)^{2} \sin\left(\frac{2}{3} \, \pi x\right) - \frac{2}{3} \, \pi \cos\left(\frac{2}{3} \, \pi x\right) \tan\left(\frac{2}{3} \, \pi x\right)}\]

\end{problem}}

%%%%%%%%%%%%%%%%%%%%%%

\latexProblemContent{
\ifVerboseLocation This is Derivative Compute Question 0003. \\ \fi
\begin{problem}

Compute the following derivative:

\input{Derivative-Compute-0003.HELP.tex}

\[\dfrac{d}{dx}\left(-15 \, \sin\left(2 \, \pi x\right) \sin\left(\pi x\right)\right)=\answer{-15 \, \pi \cos\left(\pi x\right) \sin\left(2 \, \pi x\right) - 30 \, \pi \cos\left(2 \, \pi x\right) \sin\left(\pi x\right)}\]

\end{problem}}

%%%%%%%%%%%%%%%%%%%%%%

\latexProblemContent{
\ifVerboseLocation This is Derivative Compute Question 0003. \\ \fi
\begin{problem}

Compute the following derivative:

\input{Derivative-Compute-0003.HELP.tex}

\[\dfrac{d}{dx}\left(-4 \, \cos\left(-\frac{1}{2} \, \pi x\right) \cos\left(-\frac{5}{2} \, \pi x\right)\right)=\answer{10 \, \pi \cos\left(\frac{1}{2} \, \pi x\right) \sin\left(\frac{5}{2} \, \pi x\right) + 2 \, \pi \cos\left(\frac{5}{2} \, \pi x\right) \sin\left(\frac{1}{2} \, \pi x\right)}\]

\end{problem}}

%%%%%%%%%%%%%%%%%%%%%%

\latexProblemContent{
\ifVerboseLocation This is Derivative Compute Question 0003. \\ \fi
\begin{problem}

Compute the following derivative:

\input{Derivative-Compute-0003.HELP.tex}

\[\dfrac{d}{dx}\left(9 \, \sin\left(-\frac{1}{3} \, \pi x\right) \tan\left(-\frac{1}{6} \, \pi x\right)\right)=\answer{\frac{3}{2} \, \pi \sec\left(\frac{1}{6} \, \pi x\right)^{2} \sin\left(\frac{1}{3} \, \pi x\right) + 3 \, \pi \cos\left(\frac{1}{3} \, \pi x\right) \tan\left(\frac{1}{6} \, \pi x\right)}\]

\end{problem}}

%%%%%%%%%%%%%%%%%%%%%%

\latexProblemContent{
\ifVerboseLocation This is Derivative Compute Question 0003. \\ \fi
\begin{problem}

Compute the following derivative:

\input{Derivative-Compute-0003.HELP.tex}

\[\dfrac{d}{dx}\left(-15 \, \cos\left(\frac{2}{3} \, \pi x\right) \cos\left(\frac{1}{3} \, \pi x\right)\right)=\answer{10 \, \pi \cos\left(\frac{1}{3} \, \pi x\right) \sin\left(\frac{2}{3} \, \pi x\right) + 5 \, \pi \cos\left(\frac{2}{3} \, \pi x\right) \sin\left(\frac{1}{3} \, \pi x\right)}\]

\end{problem}}

%%%%%%%%%%%%%%%%%%%%%%

\latexProblemContent{
\ifVerboseLocation This is Derivative Compute Question 0003. \\ \fi
\begin{problem}

Compute the following derivative:

\input{Derivative-Compute-0003.HELP.tex}

\[\dfrac{d}{dx}\left(-12 \, \cos\left(\frac{1}{3} \, \pi x\right) \tan\left(-\frac{5}{6} \, \pi x\right)\right)=\answer{10 \, \pi \cos\left(\frac{1}{3} \, \pi x\right) \sec\left(\frac{5}{6} \, \pi x\right)^{2} - 4 \, \pi \sin\left(\frac{1}{3} \, \pi x\right) \tan\left(\frac{5}{6} \, \pi x\right)}\]

\end{problem}}

%%%%%%%%%%%%%%%%%%%%%%

\latexProblemContent{
\ifVerboseLocation This is Derivative Compute Question 0003. \\ \fi
\begin{problem}

Compute the following derivative:

\input{Derivative-Compute-0003.HELP.tex}

\[\dfrac{d}{dx}\left(-4 \, \cos\left(\frac{2}{3} \, \pi x\right) \tan\left(\frac{3}{2} \, \pi x\right)\right)=\answer{-6 \, \pi \cos\left(\frac{2}{3} \, \pi x\right) \sec\left(\frac{3}{2} \, \pi x\right)^{2} + \frac{8}{3} \, \pi \sin\left(\frac{2}{3} \, \pi x\right) \tan\left(\frac{3}{2} \, \pi x\right)}\]

\end{problem}}

%%%%%%%%%%%%%%%%%%%%%%

\latexProblemContent{
\ifVerboseLocation This is Derivative Compute Question 0003. \\ \fi
\begin{problem}

Compute the following derivative:

\input{Derivative-Compute-0003.HELP.tex}

\[\dfrac{d}{dx}\left(16 \, \cos\left(-\frac{1}{2} \, \pi x\right) \tan\left(\frac{5}{6} \, \pi x\right)\right)=\answer{\frac{40}{3} \, \pi \cos\left(\frac{1}{2} \, \pi x\right) \sec\left(\frac{5}{6} \, \pi x\right)^{2} - 8 \, \pi \sin\left(\frac{1}{2} \, \pi x\right) \tan\left(\frac{5}{6} \, \pi x\right)}\]

\end{problem}}

%%%%%%%%%%%%%%%%%%%%%%

\latexProblemContent{
\ifVerboseLocation This is Derivative Compute Question 0003. \\ \fi
\begin{problem}

Compute the following derivative:

\input{Derivative-Compute-0003.HELP.tex}

\[\dfrac{d}{dx}\left(-10 \, \cos\left(3 \, \pi x\right) \cos\left(\frac{4}{3} \, \pi x\right)\right)=\answer{30 \, \pi \cos\left(\frac{4}{3} \, \pi x\right) \sin\left(3 \, \pi x\right) + \frac{40}{3} \, \pi \cos\left(3 \, \pi x\right) \sin\left(\frac{4}{3} \, \pi x\right)}\]

\end{problem}}

%%%%%%%%%%%%%%%%%%%%%%

\latexProblemContent{
\ifVerboseLocation This is Derivative Compute Question 0003. \\ \fi
\begin{problem}

Compute the following derivative:

\input{Derivative-Compute-0003.HELP.tex}

\[\dfrac{d}{dx}\left(4 \, \cos\left(\frac{5}{2} \, \pi x\right) \sin\left(-2 \, \pi x\right)\right)=\answer{-8 \, \pi \cos\left(\frac{5}{2} \, \pi x\right) \cos\left(2 \, \pi x\right) + 10 \, \pi \sin\left(\frac{5}{2} \, \pi x\right) \sin\left(2 \, \pi x\right)}\]

\end{problem}}

%%%%%%%%%%%%%%%%%%%%%%

\latexProblemContent{
\ifVerboseLocation This is Derivative Compute Question 0003. \\ \fi
\begin{problem}

Compute the following derivative:

\input{Derivative-Compute-0003.HELP.tex}

\[\dfrac{d}{dx}\left(-4 \, \sin\left(-\frac{1}{6} \, \pi x\right) \sin\left(-\frac{5}{6} \, \pi x\right)\right)=\answer{-\frac{2}{3} \, \pi \cos\left(\frac{1}{6} \, \pi x\right) \sin\left(\frac{5}{6} \, \pi x\right) - \frac{10}{3} \, \pi \cos\left(\frac{5}{6} \, \pi x\right) \sin\left(\frac{1}{6} \, \pi x\right)}\]

\end{problem}}

%%%%%%%%%%%%%%%%%%%%%%

\latexProblemContent{
\ifVerboseLocation This is Derivative Compute Question 0003. \\ \fi
\begin{problem}

Compute the following derivative:

\input{Derivative-Compute-0003.HELP.tex}

\[\dfrac{d}{dx}\left(4 \, \tan\left(4 \, \pi x\right) \tan\left(-\frac{5}{2} \, \pi x\right)\right)=\answer{-10 \, \pi \sec\left(\frac{5}{2} \, \pi x\right)^{2} \tan\left(4 \, \pi x\right) - 16 \, \pi \sec\left(4 \, \pi x\right)^{2} \tan\left(\frac{5}{2} \, \pi x\right)}\]

\end{problem}}

%%%%%%%%%%%%%%%%%%%%%%

\latexProblemContent{
\ifVerboseLocation This is Derivative Compute Question 0003. \\ \fi
\begin{problem}

Compute the following derivative:

\input{Derivative-Compute-0003.HELP.tex}

\[\dfrac{d}{dx}\left(-2 \, \cos\left(\frac{5}{6} \, \pi x\right) \tan\left(-4 \, \pi x\right)\right)=\answer{8 \, \pi \cos\left(\frac{5}{6} \, \pi x\right) \sec\left(4 \, \pi x\right)^{2} - \frac{5}{3} \, \pi \sin\left(\frac{5}{6} \, \pi x\right) \tan\left(4 \, \pi x\right)}\]

\end{problem}}

%%%%%%%%%%%%%%%%%%%%%%

\latexProblemContent{
\ifVerboseLocation This is Derivative Compute Question 0003. \\ \fi
\begin{problem}

Compute the following derivative:

\input{Derivative-Compute-0003.HELP.tex}

\[\dfrac{d}{dx}\left(25 \, \cos\left(-2 \, \pi x\right) \tan\left(\frac{2}{3} \, \pi x\right)\right)=\answer{\frac{50}{3} \, \pi \cos\left(2 \, \pi x\right) \sec\left(\frac{2}{3} \, \pi x\right)^{2} - 50 \, \pi \sin\left(2 \, \pi x\right) \tan\left(\frac{2}{3} \, \pi x\right)}\]

\end{problem}}

%%%%%%%%%%%%%%%%%%%%%%

\latexProblemContent{
\ifVerboseLocation This is Derivative Compute Question 0003. \\ \fi
\begin{problem}

Compute the following derivative:

\input{Derivative-Compute-0003.HELP.tex}

\[\dfrac{d}{dx}\left(10 \, \sin\left(\frac{2}{3} \, \pi x\right) \sin\left(\frac{1}{3} \, \pi x\right)\right)=\answer{\frac{10}{3} \, \pi \cos\left(\frac{1}{3} \, \pi x\right) \sin\left(\frac{2}{3} \, \pi x\right) + \frac{20}{3} \, \pi \cos\left(\frac{2}{3} \, \pi x\right) \sin\left(\frac{1}{3} \, \pi x\right)}\]

\end{problem}}

%%%%%%%%%%%%%%%%%%%%%%

\latexProblemContent{
\ifVerboseLocation This is Derivative Compute Question 0003. \\ \fi
\begin{problem}

Compute the following derivative:

\input{Derivative-Compute-0003.HELP.tex}

\[\dfrac{d}{dx}\left(-16 \, \cos\left(\frac{1}{3} \, \pi x\right) \cos\left(-\pi x\right)\right)=\answer{16 \, \pi \cos\left(\frac{1}{3} \, \pi x\right) \sin\left(\pi x\right) + \frac{16}{3} \, \pi \cos\left(\pi x\right) \sin\left(\frac{1}{3} \, \pi x\right)}\]

\end{problem}}

%%%%%%%%%%%%%%%%%%%%%%

\latexProblemContent{
\ifVerboseLocation This is Derivative Compute Question 0003. \\ \fi
\begin{problem}

Compute the following derivative:

\input{Derivative-Compute-0003.HELP.tex}

\[\dfrac{d}{dx}\left(-4 \, \sin\left(-\frac{1}{3} \, \pi x\right) \tan\left(-4 \, \pi x\right)\right)=\answer{-16 \, \pi \sec\left(4 \, \pi x\right)^{2} \sin\left(\frac{1}{3} \, \pi x\right) - \frac{4}{3} \, \pi \cos\left(\frac{1}{3} \, \pi x\right) \tan\left(4 \, \pi x\right)}\]

\end{problem}}

%%%%%%%%%%%%%%%%%%%%%%

\latexProblemContent{
\ifVerboseLocation This is Derivative Compute Question 0003. \\ \fi
\begin{problem}

Compute the following derivative:

\input{Derivative-Compute-0003.HELP.tex}

\[\dfrac{d}{dx}\left(-12 \, \cos\left(\frac{1}{2} \, \pi x\right)^{2}\right)=\answer{12 \, \pi \cos\left(\frac{1}{2} \, \pi x\right) \sin\left(\frac{1}{2} \, \pi x\right)}\]

\end{problem}}

%%%%%%%%%%%%%%%%%%%%%%

\latexProblemContent{
\ifVerboseLocation This is Derivative Compute Question 0003. \\ \fi
\begin{problem}

Compute the following derivative:

\input{Derivative-Compute-0003.HELP.tex}

\[\dfrac{d}{dx}\left(-8 \, \cos\left(-\frac{5}{2} \, \pi x\right) \sin\left(5 \, \pi x\right)\right)=\answer{-40 \, \pi \cos\left(5 \, \pi x\right) \cos\left(\frac{5}{2} \, \pi x\right) + 20 \, \pi \sin\left(5 \, \pi x\right) \sin\left(\frac{5}{2} \, \pi x\right)}\]

\end{problem}}

%%%%%%%%%%%%%%%%%%%%%%

\latexProblemContent{
\ifVerboseLocation This is Derivative Compute Question 0003. \\ \fi
\begin{problem}

Compute the following derivative:

\input{Derivative-Compute-0003.HELP.tex}

\[\dfrac{d}{dx}\left(-15 \, \tan\left(\frac{1}{3} \, \pi x\right)^{2}\right)=\answer{-10 \, \pi \sec\left(\frac{1}{3} \, \pi x\right)^{2} \tan\left(\frac{1}{3} \, \pi x\right)}\]

\end{problem}}

%%%%%%%%%%%%%%%%%%%%%%

\latexProblemContent{
\ifVerboseLocation This is Derivative Compute Question 0003. \\ \fi
\begin{problem}

Compute the following derivative:

\input{Derivative-Compute-0003.HELP.tex}

\[\dfrac{d}{dx}\left(-16 \, \cos\left(\frac{2}{3} \, \pi x\right) \sin\left(\frac{2}{3} \, \pi x\right)\right)=\answer{-\frac{32}{3} \, \pi \cos\left(\frac{2}{3} \, \pi x\right)^{2} + \frac{32}{3} \, \pi \sin\left(\frac{2}{3} \, \pi x\right)^{2}}\]

\end{problem}}

%%%%%%%%%%%%%%%%%%%%%%

\latexProblemContent{
\ifVerboseLocation This is Derivative Compute Question 0003. \\ \fi
\begin{problem}

Compute the following derivative:

\input{Derivative-Compute-0003.HELP.tex}

\[\dfrac{d}{dx}\left(12 \, \cos\left(\pi x\right) \tan\left(-\frac{5}{2} \, \pi x\right)\right)=\answer{-30 \, \pi \cos\left(\pi x\right) \sec\left(\frac{5}{2} \, \pi x\right)^{2} + 12 \, \pi \sin\left(\pi x\right) \tan\left(\frac{5}{2} \, \pi x\right)}\]

\end{problem}}

%%%%%%%%%%%%%%%%%%%%%%

\latexProblemContent{
\ifVerboseLocation This is Derivative Compute Question 0003. \\ \fi
\begin{problem}

Compute the following derivative:

\input{Derivative-Compute-0003.HELP.tex}

\[\dfrac{d}{dx}\left(-10 \, \cos\left(\frac{1}{2} \, \pi x\right) \tan\left(\pi x\right)\right)=\answer{-10 \, \pi \cos\left(\frac{1}{2} \, \pi x\right) \sec\left(\pi x\right)^{2} + 5 \, \pi \sin\left(\frac{1}{2} \, \pi x\right) \tan\left(\pi x\right)}\]

\end{problem}}

%%%%%%%%%%%%%%%%%%%%%%

\latexProblemContent{
\ifVerboseLocation This is Derivative Compute Question 0003. \\ \fi
\begin{problem}

Compute the following derivative:

\input{Derivative-Compute-0003.HELP.tex}

\[\dfrac{d}{dx}\left(-20 \, \cos\left(-\frac{3}{2} \, \pi x\right) \tan\left(-\pi x\right)\right)=\answer{20 \, \pi \cos\left(\frac{3}{2} \, \pi x\right) \sec\left(\pi x\right)^{2} - 30 \, \pi \sin\left(\frac{3}{2} \, \pi x\right) \tan\left(\pi x\right)}\]

\end{problem}}

%%%%%%%%%%%%%%%%%%%%%%

\latexProblemContent{
\ifVerboseLocation This is Derivative Compute Question 0003. \\ \fi
\begin{problem}

Compute the following derivative:

\input{Derivative-Compute-0003.HELP.tex}

\[\dfrac{d}{dx}\left(-6 \, \cos\left(4 \, \pi x\right) \tan\left(-3 \, \pi x\right)\right)=\answer{18 \, \pi \cos\left(4 \, \pi x\right) \sec\left(3 \, \pi x\right)^{2} - 24 \, \pi \sin\left(4 \, \pi x\right) \tan\left(3 \, \pi x\right)}\]

\end{problem}}

%%%%%%%%%%%%%%%%%%%%%%

\latexProblemContent{
\ifVerboseLocation This is Derivative Compute Question 0003. \\ \fi
\begin{problem}

Compute the following derivative:

\input{Derivative-Compute-0003.HELP.tex}

\[\dfrac{d}{dx}\left(-10 \, \cos\left(4 \, \pi x\right) \tan\left(-\frac{3}{2} \, \pi x\right)\right)=\answer{15 \, \pi \cos\left(4 \, \pi x\right) \sec\left(\frac{3}{2} \, \pi x\right)^{2} - 40 \, \pi \sin\left(4 \, \pi x\right) \tan\left(\frac{3}{2} \, \pi x\right)}\]

\end{problem}}

%%%%%%%%%%%%%%%%%%%%%%

\latexProblemContent{
\ifVerboseLocation This is Derivative Compute Question 0003. \\ \fi
\begin{problem}

Compute the following derivative:

\input{Derivative-Compute-0003.HELP.tex}

\[\dfrac{d}{dx}\left(-4 \, \sin\left(\pi x\right) \sin\left(\frac{1}{6} \, \pi x\right)\right)=\answer{-\frac{2}{3} \, \pi \cos\left(\frac{1}{6} \, \pi x\right) \sin\left(\pi x\right) - 4 \, \pi \cos\left(\pi x\right) \sin\left(\frac{1}{6} \, \pi x\right)}\]

\end{problem}}

%%%%%%%%%%%%%%%%%%%%%%

\latexProblemContent{
\ifVerboseLocation This is Derivative Compute Question 0003. \\ \fi
\begin{problem}

Compute the following derivative:

\input{Derivative-Compute-0003.HELP.tex}

\[\dfrac{d}{dx}\left(-2 \, \cos\left(\pi x\right) \sin\left(-\frac{5}{3} \, \pi x\right)\right)=\answer{\frac{10}{3} \, \pi \cos\left(\frac{5}{3} \, \pi x\right) \cos\left(\pi x\right) - 2 \, \pi \sin\left(\frac{5}{3} \, \pi x\right) \sin\left(\pi x\right)}\]

\end{problem}}

%%%%%%%%%%%%%%%%%%%%%%

\latexProblemContent{
\ifVerboseLocation This is Derivative Compute Question 0003. \\ \fi
\begin{problem}

Compute the following derivative:

\input{Derivative-Compute-0003.HELP.tex}

\[\dfrac{d}{dx}\left(20 \, \sin\left(\frac{5}{6} \, \pi x\right) \sin\left(-\frac{2}{3} \, \pi x\right)\right)=\answer{-\frac{40}{3} \, \pi \cos\left(\frac{2}{3} \, \pi x\right) \sin\left(\frac{5}{6} \, \pi x\right) - \frac{50}{3} \, \pi \cos\left(\frac{5}{6} \, \pi x\right) \sin\left(\frac{2}{3} \, \pi x\right)}\]

\end{problem}}

%%%%%%%%%%%%%%%%%%%%%%

\latexProblemContent{
\ifVerboseLocation This is Derivative Compute Question 0003. \\ \fi
\begin{problem}

Compute the following derivative:

\input{Derivative-Compute-0003.HELP.tex}

\[\dfrac{d}{dx}\left(6 \, \sin\left(-2 \, \pi x\right) \tan\left(-\frac{5}{6} \, \pi x\right)\right)=\answer{5 \, \pi \sec\left(\frac{5}{6} \, \pi x\right)^{2} \sin\left(2 \, \pi x\right) + 12 \, \pi \cos\left(2 \, \pi x\right) \tan\left(\frac{5}{6} \, \pi x\right)}\]

\end{problem}}

%%%%%%%%%%%%%%%%%%%%%%

\latexProblemContent{
\ifVerboseLocation This is Derivative Compute Question 0003. \\ \fi
\begin{problem}

Compute the following derivative:

\input{Derivative-Compute-0003.HELP.tex}

\[\dfrac{d}{dx}\left(-9 \, \cos\left(-\frac{5}{3} \, \pi x\right) \cos\left(-5 \, \pi x\right)\right)=\answer{45 \, \pi \cos\left(\frac{5}{3} \, \pi x\right) \sin\left(5 \, \pi x\right) + 15 \, \pi \cos\left(5 \, \pi x\right) \sin\left(\frac{5}{3} \, \pi x\right)}\]

\end{problem}}

%%%%%%%%%%%%%%%%%%%%%%

\latexProblemContent{
\ifVerboseLocation This is Derivative Compute Question 0003. \\ \fi
\begin{problem}

Compute the following derivative:

\input{Derivative-Compute-0003.HELP.tex}

\[\dfrac{d}{dx}\left(12 \, \cos\left(\frac{2}{3} \, \pi x\right) \tan\left(-\pi x\right)\right)=\answer{-12 \, \pi \cos\left(\frac{2}{3} \, \pi x\right) \sec\left(\pi x\right)^{2} + 8 \, \pi \sin\left(\frac{2}{3} \, \pi x\right) \tan\left(\pi x\right)}\]

\end{problem}}

%%%%%%%%%%%%%%%%%%%%%%

\latexProblemContent{
\ifVerboseLocation This is Derivative Compute Question 0003. \\ \fi
\begin{problem}

Compute the following derivative:

\input{Derivative-Compute-0003.HELP.tex}

\[\dfrac{d}{dx}\left(10 \, \tan\left(-3 \, \pi x\right) \tan\left(-4 \, \pi x\right)\right)=\answer{30 \, \pi \sec\left(3 \, \pi x\right)^{2} \tan\left(4 \, \pi x\right) + 40 \, \pi \sec\left(4 \, \pi x\right)^{2} \tan\left(3 \, \pi x\right)}\]

\end{problem}}

%%%%%%%%%%%%%%%%%%%%%%

\latexProblemContent{
\ifVerboseLocation This is Derivative Compute Question 0003. \\ \fi
\begin{problem}

Compute the following derivative:

\input{Derivative-Compute-0003.HELP.tex}

\[\dfrac{d}{dx}\left(-3 \, \tan\left(\pi x\right) \tan\left(-\frac{5}{6} \, \pi x\right)\right)=\answer{\frac{5}{2} \, \pi \sec\left(\frac{5}{6} \, \pi x\right)^{2} \tan\left(\pi x\right) + 3 \, \pi \sec\left(\pi x\right)^{2} \tan\left(\frac{5}{6} \, \pi x\right)}\]

\end{problem}}

%%%%%%%%%%%%%%%%%%%%%%

\latexProblemContent{
\ifVerboseLocation This is Derivative Compute Question 0003. \\ \fi
\begin{problem}

Compute the following derivative:

\input{Derivative-Compute-0003.HELP.tex}

\[\dfrac{d}{dx}\left(10 \, \cos\left(-2 \, \pi x\right) \tan\left(\frac{1}{2} \, \pi x\right)\right)=\answer{5 \, \pi \cos\left(2 \, \pi x\right) \sec\left(\frac{1}{2} \, \pi x\right)^{2} - 20 \, \pi \sin\left(2 \, \pi x\right) \tan\left(\frac{1}{2} \, \pi x\right)}\]

\end{problem}}

%%%%%%%%%%%%%%%%%%%%%%

\latexProblemContent{
\ifVerboseLocation This is Derivative Compute Question 0003. \\ \fi
\begin{problem}

Compute the following derivative:

\input{Derivative-Compute-0003.HELP.tex}

\[\dfrac{d}{dx}\left(-8 \, \tan\left(-\frac{1}{3} \, \pi x\right) \tan\left(-\pi x\right)\right)=\answer{-\frac{8}{3} \, \pi \sec\left(\frac{1}{3} \, \pi x\right)^{2} \tan\left(\pi x\right) - 8 \, \pi \sec\left(\pi x\right)^{2} \tan\left(\frac{1}{3} \, \pi x\right)}\]

\end{problem}}

%%%%%%%%%%%%%%%%%%%%%%

\latexProblemContent{
\ifVerboseLocation This is Derivative Compute Question 0003. \\ \fi
\begin{problem}

Compute the following derivative:

\input{Derivative-Compute-0003.HELP.tex}

\[\dfrac{d}{dx}\left(-6 \, \cos\left(\frac{5}{6} \, \pi x\right) \cos\left(-\frac{2}{3} \, \pi x\right)\right)=\answer{5 \, \pi \cos\left(\frac{2}{3} \, \pi x\right) \sin\left(\frac{5}{6} \, \pi x\right) + 4 \, \pi \cos\left(\frac{5}{6} \, \pi x\right) \sin\left(\frac{2}{3} \, \pi x\right)}\]

\end{problem}}

%%%%%%%%%%%%%%%%%%%%%%

\latexProblemContent{
\ifVerboseLocation This is Derivative Compute Question 0003. \\ \fi
\begin{problem}

Compute the following derivative:

\input{Derivative-Compute-0003.HELP.tex}

\[\dfrac{d}{dx}\left(25 \, \sin\left(-\frac{3}{2} \, \pi x\right) \tan\left(-\frac{5}{3} \, \pi x\right)\right)=\answer{\frac{125}{3} \, \pi \sec\left(\frac{5}{3} \, \pi x\right)^{2} \sin\left(\frac{3}{2} \, \pi x\right) + \frac{75}{2} \, \pi \cos\left(\frac{3}{2} \, \pi x\right) \tan\left(\frac{5}{3} \, \pi x\right)}\]

\end{problem}}

%%%%%%%%%%%%%%%%%%%%%%

\latexProblemContent{
\ifVerboseLocation This is Derivative Compute Question 0003. \\ \fi
\begin{problem}

Compute the following derivative:

\input{Derivative-Compute-0003.HELP.tex}

\[\dfrac{d}{dx}\left(12 \, \cos\left(3 \, \pi x\right) \sin\left(3 \, \pi x\right)\right)=\answer{36 \, \pi \cos\left(3 \, \pi x\right)^{2} - 36 \, \pi \sin\left(3 \, \pi x\right)^{2}}\]

\end{problem}}

%%%%%%%%%%%%%%%%%%%%%%

\latexProblemContent{
\ifVerboseLocation This is Derivative Compute Question 0003. \\ \fi
\begin{problem}

Compute the following derivative:

\input{Derivative-Compute-0003.HELP.tex}

\[\dfrac{d}{dx}\left(-25 \, \sin\left(\frac{5}{3} \, \pi x\right) \tan\left(\frac{1}{2} \, \pi x\right)\right)=\answer{-\frac{25}{2} \, \pi \sec\left(\frac{1}{2} \, \pi x\right)^{2} \sin\left(\frac{5}{3} \, \pi x\right) - \frac{125}{3} \, \pi \cos\left(\frac{5}{3} \, \pi x\right) \tan\left(\frac{1}{2} \, \pi x\right)}\]

\end{problem}}

%%%%%%%%%%%%%%%%%%%%%%

\latexProblemContent{
\ifVerboseLocation This is Derivative Compute Question 0003. \\ \fi
\begin{problem}

Compute the following derivative:

\input{Derivative-Compute-0003.HELP.tex}

\[\dfrac{d}{dx}\left(-3 \, \sin\left(\pi x\right) \tan\left(-4 \, \pi x\right)\right)=\answer{12 \, \pi \sec\left(4 \, \pi x\right)^{2} \sin\left(\pi x\right) + 3 \, \pi \cos\left(\pi x\right) \tan\left(4 \, \pi x\right)}\]

\end{problem}}

%%%%%%%%%%%%%%%%%%%%%%

\latexProblemContent{
\ifVerboseLocation This is Derivative Compute Question 0003. \\ \fi
\begin{problem}

Compute the following derivative:

\input{Derivative-Compute-0003.HELP.tex}

\[\dfrac{d}{dx}\left(-8 \, \cos\left(-\frac{3}{2} \, \pi x\right) \tan\left(2 \, \pi x\right)\right)=\answer{-16 \, \pi \cos\left(\frac{3}{2} \, \pi x\right) \sec\left(2 \, \pi x\right)^{2} + 12 \, \pi \sin\left(\frac{3}{2} \, \pi x\right) \tan\left(2 \, \pi x\right)}\]

\end{problem}}

%%%%%%%%%%%%%%%%%%%%%%

\latexProblemContent{
\ifVerboseLocation This is Derivative Compute Question 0003. \\ \fi
\begin{problem}

Compute the following derivative:

\input{Derivative-Compute-0003.HELP.tex}

\[\dfrac{d}{dx}\left(2 \, \cos\left(\frac{5}{3} \, \pi x\right) \tan\left(-\pi x\right)\right)=\answer{-2 \, \pi \cos\left(\frac{5}{3} \, \pi x\right) \sec\left(\pi x\right)^{2} + \frac{10}{3} \, \pi \sin\left(\frac{5}{3} \, \pi x\right) \tan\left(\pi x\right)}\]

\end{problem}}

%%%%%%%%%%%%%%%%%%%%%%

\latexProblemContent{
\ifVerboseLocation This is Derivative Compute Question 0003. \\ \fi
\begin{problem}

Compute the following derivative:

\input{Derivative-Compute-0003.HELP.tex}

\[\dfrac{d}{dx}\left(20 \, \cos\left(-2 \, \pi x\right) \tan\left(\frac{4}{3} \, \pi x\right)\right)=\answer{\frac{80}{3} \, \pi \cos\left(2 \, \pi x\right) \sec\left(\frac{4}{3} \, \pi x\right)^{2} - 40 \, \pi \sin\left(2 \, \pi x\right) \tan\left(\frac{4}{3} \, \pi x\right)}\]

\end{problem}}

%%%%%%%%%%%%%%%%%%%%%%

\latexProblemContent{
\ifVerboseLocation This is Derivative Compute Question 0003. \\ \fi
\begin{problem}

Compute the following derivative:

\input{Derivative-Compute-0003.HELP.tex}

\[\dfrac{d}{dx}\left(-3 \, \sin\left(-\frac{1}{3} \, \pi x\right) \sin\left(-\frac{5}{2} \, \pi x\right)\right)=\answer{-\pi \cos\left(\frac{1}{3} \, \pi x\right) \sin\left(\frac{5}{2} \, \pi x\right) - \frac{15}{2} \, \pi \cos\left(\frac{5}{2} \, \pi x\right) \sin\left(\frac{1}{3} \, \pi x\right)}\]

\end{problem}}

%%%%%%%%%%%%%%%%%%%%%%

\latexProblemContent{
\ifVerboseLocation This is Derivative Compute Question 0003. \\ \fi
\begin{problem}

Compute the following derivative:

\input{Derivative-Compute-0003.HELP.tex}

\[\dfrac{d}{dx}\left(-2 \, \cos\left(-5 \, \pi x\right) \sin\left(3 \, \pi x\right)\right)=\answer{-6 \, \pi \cos\left(5 \, \pi x\right) \cos\left(3 \, \pi x\right) + 10 \, \pi \sin\left(5 \, \pi x\right) \sin\left(3 \, \pi x\right)}\]

\end{problem}}

%%%%%%%%%%%%%%%%%%%%%%

\latexProblemContent{
\ifVerboseLocation This is Derivative Compute Question 0003. \\ \fi
\begin{problem}

Compute the following derivative:

\input{Derivative-Compute-0003.HELP.tex}

\[\dfrac{d}{dx}\left(2 \, \cos\left(\pi x\right) \cos\left(-5 \, \pi x\right)\right)=\answer{-10 \, \pi \cos\left(\pi x\right) \sin\left(5 \, \pi x\right) - 2 \, \pi \cos\left(5 \, \pi x\right) \sin\left(\pi x\right)}\]

\end{problem}}

%%%%%%%%%%%%%%%%%%%%%%

\latexProblemContent{
\ifVerboseLocation This is Derivative Compute Question 0003. \\ \fi
\begin{problem}

Compute the following derivative:

\input{Derivative-Compute-0003.HELP.tex}

\[\dfrac{d}{dx}\left(-10 \, \cos\left(-\frac{1}{3} \, \pi x\right) \tan\left(\frac{1}{6} \, \pi x\right)\right)=\answer{-\frac{5}{3} \, \pi \cos\left(\frac{1}{3} \, \pi x\right) \sec\left(\frac{1}{6} \, \pi x\right)^{2} + \frac{10}{3} \, \pi \sin\left(\frac{1}{3} \, \pi x\right) \tan\left(\frac{1}{6} \, \pi x\right)}\]

\end{problem}}

%%%%%%%%%%%%%%%%%%%%%%

\latexProblemContent{
\ifVerboseLocation This is Derivative Compute Question 0003. \\ \fi
\begin{problem}

Compute the following derivative:

\input{Derivative-Compute-0003.HELP.tex}

\[\dfrac{d}{dx}\left(6 \, \sin\left(\frac{1}{3} \, \pi x\right) \tan\left(\frac{2}{3} \, \pi x\right)\right)=\answer{4 \, \pi \sec\left(\frac{2}{3} \, \pi x\right)^{2} \sin\left(\frac{1}{3} \, \pi x\right) + 2 \, \pi \cos\left(\frac{1}{3} \, \pi x\right) \tan\left(\frac{2}{3} \, \pi x\right)}\]

\end{problem}}

%%%%%%%%%%%%%%%%%%%%%%

\latexProblemContent{
\ifVerboseLocation This is Derivative Compute Question 0003. \\ \fi
\begin{problem}

Compute the following derivative:

\input{Derivative-Compute-0003.HELP.tex}

\[\dfrac{d}{dx}\left(20 \, \cos\left(-\frac{1}{2} \, \pi x\right) \tan\left(-\frac{5}{2} \, \pi x\right)\right)=\answer{-50 \, \pi \cos\left(\frac{1}{2} \, \pi x\right) \sec\left(\frac{5}{2} \, \pi x\right)^{2} + 10 \, \pi \sin\left(\frac{1}{2} \, \pi x\right) \tan\left(\frac{5}{2} \, \pi x\right)}\]

\end{problem}}

%%%%%%%%%%%%%%%%%%%%%%

\latexProblemContent{
\ifVerboseLocation This is Derivative Compute Question 0003. \\ \fi
\begin{problem}

Compute the following derivative:

\input{Derivative-Compute-0003.HELP.tex}

\[\dfrac{d}{dx}\left(-6 \, \cos\left(2 \, \pi x\right) \sin\left(-\frac{1}{3} \, \pi x\right)\right)=\answer{2 \, \pi \cos\left(2 \, \pi x\right) \cos\left(\frac{1}{3} \, \pi x\right) - 12 \, \pi \sin\left(2 \, \pi x\right) \sin\left(\frac{1}{3} \, \pi x\right)}\]

\end{problem}}

%%%%%%%%%%%%%%%%%%%%%%

\latexProblemContent{
\ifVerboseLocation This is Derivative Compute Question 0003. \\ \fi
\begin{problem}

Compute the following derivative:

\input{Derivative-Compute-0003.HELP.tex}

\[\dfrac{d}{dx}\left(3 \, \cos\left(\frac{1}{2} \, \pi x\right) \sin\left(\frac{5}{6} \, \pi x\right)\right)=\answer{\frac{5}{2} \, \pi \cos\left(\frac{5}{6} \, \pi x\right) \cos\left(\frac{1}{2} \, \pi x\right) - \frac{3}{2} \, \pi \sin\left(\frac{5}{6} \, \pi x\right) \sin\left(\frac{1}{2} \, \pi x\right)}\]

\end{problem}}

%%%%%%%%%%%%%%%%%%%%%%

\latexProblemContent{
\ifVerboseLocation This is Derivative Compute Question 0003. \\ \fi
\begin{problem}

Compute the following derivative:

\input{Derivative-Compute-0003.HELP.tex}

\[\dfrac{d}{dx}\left(-4 \, \cos\left(3 \, \pi x\right) \cos\left(\frac{5}{2} \, \pi x\right)\right)=\answer{12 \, \pi \cos\left(\frac{5}{2} \, \pi x\right) \sin\left(3 \, \pi x\right) + 10 \, \pi \cos\left(3 \, \pi x\right) \sin\left(\frac{5}{2} \, \pi x\right)}\]

\end{problem}}

%%%%%%%%%%%%%%%%%%%%%%

\latexProblemContent{
\ifVerboseLocation This is Derivative Compute Question 0003. \\ \fi
\begin{problem}

Compute the following derivative:

\input{Derivative-Compute-0003.HELP.tex}

\[\dfrac{d}{dx}\left(-15 \, \sin\left(4 \, \pi x\right) \tan\left(-\frac{1}{3} \, \pi x\right)\right)=\answer{5 \, \pi \sec\left(\frac{1}{3} \, \pi x\right)^{2} \sin\left(4 \, \pi x\right) + 60 \, \pi \cos\left(4 \, \pi x\right) \tan\left(\frac{1}{3} \, \pi x\right)}\]

\end{problem}}

%%%%%%%%%%%%%%%%%%%%%%

\latexProblemContent{
\ifVerboseLocation This is Derivative Compute Question 0003. \\ \fi
\begin{problem}

Compute the following derivative:

\input{Derivative-Compute-0003.HELP.tex}

\[\dfrac{d}{dx}\left(-12 \, \cos\left(-5 \, \pi x\right) \tan\left(5 \, \pi x\right)\right)=\answer{-60 \, \pi \cos\left(5 \, \pi x\right) \sec\left(5 \, \pi x\right)^{2} + 60 \, \pi \sin\left(5 \, \pi x\right) \tan\left(5 \, \pi x\right)}\]

\end{problem}}

%%%%%%%%%%%%%%%%%%%%%%

\latexProblemContent{
\ifVerboseLocation This is Derivative Compute Question 0003. \\ \fi
\begin{problem}

Compute the following derivative:

\input{Derivative-Compute-0003.HELP.tex}

\[\dfrac{d}{dx}\left(-12 \, \cos\left(\pi x\right) \cos\left(\frac{2}{3} \, \pi x\right)\right)=\answer{12 \, \pi \cos\left(\frac{2}{3} \, \pi x\right) \sin\left(\pi x\right) + 8 \, \pi \cos\left(\pi x\right) \sin\left(\frac{2}{3} \, \pi x\right)}\]

\end{problem}}

%%%%%%%%%%%%%%%%%%%%%%

\latexProblemContent{
\ifVerboseLocation This is Derivative Compute Question 0003. \\ \fi
\begin{problem}

Compute the following derivative:

\input{Derivative-Compute-0003.HELP.tex}

\[\dfrac{d}{dx}\left(-12 \, \cos\left(\frac{1}{3} \, \pi x\right) \sin\left(-4 \, \pi x\right)\right)=\answer{48 \, \pi \cos\left(4 \, \pi x\right) \cos\left(\frac{1}{3} \, \pi x\right) - 4 \, \pi \sin\left(4 \, \pi x\right) \sin\left(\frac{1}{3} \, \pi x\right)}\]

\end{problem}}

%%%%%%%%%%%%%%%%%%%%%%

\latexProblemContent{
\ifVerboseLocation This is Derivative Compute Question 0003. \\ \fi
\begin{problem}

Compute the following derivative:

\input{Derivative-Compute-0003.HELP.tex}

\[\dfrac{d}{dx}\left(-8 \, \cos\left(-\frac{5}{2} \, \pi x\right) \tan\left(-\frac{1}{2} \, \pi x\right)\right)=\answer{4 \, \pi \cos\left(\frac{5}{2} \, \pi x\right) \sec\left(\frac{1}{2} \, \pi x\right)^{2} - 20 \, \pi \sin\left(\frac{5}{2} \, \pi x\right) \tan\left(\frac{1}{2} \, \pi x\right)}\]

\end{problem}}

%%%%%%%%%%%%%%%%%%%%%%

\latexProblemContent{
\ifVerboseLocation This is Derivative Compute Question 0003. \\ \fi
\begin{problem}

Compute the following derivative:

\input{Derivative-Compute-0003.HELP.tex}

\[\dfrac{d}{dx}\left(2 \, \tan\left(\frac{1}{3} \, \pi x\right) \tan\left(-\frac{1}{3} \, \pi x\right)\right)=\answer{-\frac{4}{3} \, \pi \sec\left(\frac{1}{3} \, \pi x\right)^{2} \tan\left(\frac{1}{3} \, \pi x\right)}\]

\end{problem}}

%%%%%%%%%%%%%%%%%%%%%%

\latexProblemContent{
\ifVerboseLocation This is Derivative Compute Question 0003. \\ \fi
\begin{problem}

Compute the following derivative:

\input{Derivative-Compute-0003.HELP.tex}

\[\dfrac{d}{dx}\left(-20 \, \cos\left(\frac{2}{3} \, \pi x\right) \cos\left(-\pi x\right)\right)=\answer{20 \, \pi \cos\left(\frac{2}{3} \, \pi x\right) \sin\left(\pi x\right) + \frac{40}{3} \, \pi \cos\left(\pi x\right) \sin\left(\frac{2}{3} \, \pi x\right)}\]

\end{problem}}

%%%%%%%%%%%%%%%%%%%%%%

\latexProblemContent{
\ifVerboseLocation This is Derivative Compute Question 0003. \\ \fi
\begin{problem}

Compute the following derivative:

\input{Derivative-Compute-0003.HELP.tex}

\[\dfrac{d}{dx}\left(12 \, \sin\left(-\frac{1}{3} \, \pi x\right) \tan\left(\frac{1}{3} \, \pi x\right)\right)=\answer{-4 \, \pi \sec\left(\frac{1}{3} \, \pi x\right)^{2} \sin\left(\frac{1}{3} \, \pi x\right) - 4 \, \pi \cos\left(\frac{1}{3} \, \pi x\right) \tan\left(\frac{1}{3} \, \pi x\right)}\]

\end{problem}}

%%%%%%%%%%%%%%%%%%%%%%

\latexProblemContent{
\ifVerboseLocation This is Derivative Compute Question 0003. \\ \fi
\begin{problem}

Compute the following derivative:

\input{Derivative-Compute-0003.HELP.tex}

\[\dfrac{d}{dx}\left(16 \, \sin\left(-\frac{1}{3} \, \pi x\right) \sin\left(-\frac{2}{3} \, \pi x\right)\right)=\answer{\frac{16}{3} \, \pi \cos\left(\frac{1}{3} \, \pi x\right) \sin\left(\frac{2}{3} \, \pi x\right) + \frac{32}{3} \, \pi \cos\left(\frac{2}{3} \, \pi x\right) \sin\left(\frac{1}{3} \, \pi x\right)}\]

\end{problem}}

%%%%%%%%%%%%%%%%%%%%%%

\latexProblemContent{
\ifVerboseLocation This is Derivative Compute Question 0003. \\ \fi
\begin{problem}

Compute the following derivative:

\input{Derivative-Compute-0003.HELP.tex}

\[\dfrac{d}{dx}\left(2 \, \sin\left(2 \, \pi x\right) \tan\left(-\frac{2}{3} \, \pi x\right)\right)=\answer{-\frac{4}{3} \, \pi \sec\left(\frac{2}{3} \, \pi x\right)^{2} \sin\left(2 \, \pi x\right) - 4 \, \pi \cos\left(2 \, \pi x\right) \tan\left(\frac{2}{3} \, \pi x\right)}\]

\end{problem}}

%%%%%%%%%%%%%%%%%%%%%%

\latexProblemContent{
\ifVerboseLocation This is Derivative Compute Question 0003. \\ \fi
\begin{problem}

Compute the following derivative:

\input{Derivative-Compute-0003.HELP.tex}

\[\dfrac{d}{dx}\left(10 \, \sin\left(\frac{2}{3} \, \pi x\right) \tan\left(-\pi x\right)\right)=\answer{-10 \, \pi \sec\left(\pi x\right)^{2} \sin\left(\frac{2}{3} \, \pi x\right) - \frac{20}{3} \, \pi \cos\left(\frac{2}{3} \, \pi x\right) \tan\left(\pi x\right)}\]

\end{problem}}

%%%%%%%%%%%%%%%%%%%%%%

\latexProblemContent{
\ifVerboseLocation This is Derivative Compute Question 0003. \\ \fi
\begin{problem}

Compute the following derivative:

\input{Derivative-Compute-0003.HELP.tex}

\[\dfrac{d}{dx}\left(4 \, \cos\left(-\frac{3}{2} \, \pi x\right) \sin\left(-\frac{1}{3} \, \pi x\right)\right)=\answer{-\frac{4}{3} \, \pi \cos\left(\frac{3}{2} \, \pi x\right) \cos\left(\frac{1}{3} \, \pi x\right) + 6 \, \pi \sin\left(\frac{3}{2} \, \pi x\right) \sin\left(\frac{1}{3} \, \pi x\right)}\]

\end{problem}}

%%%%%%%%%%%%%%%%%%%%%%

\latexProblemContent{
\ifVerboseLocation This is Derivative Compute Question 0003. \\ \fi
\begin{problem}

Compute the following derivative:

\input{Derivative-Compute-0003.HELP.tex}

\[\dfrac{d}{dx}\left(8 \, \cos\left(\frac{5}{3} \, \pi x\right) \tan\left(\frac{1}{2} \, \pi x\right)\right)=\answer{4 \, \pi \cos\left(\frac{5}{3} \, \pi x\right) \sec\left(\frac{1}{2} \, \pi x\right)^{2} - \frac{40}{3} \, \pi \sin\left(\frac{5}{3} \, \pi x\right) \tan\left(\frac{1}{2} \, \pi x\right)}\]

\end{problem}}

%%%%%%%%%%%%%%%%%%%%%%

\latexProblemContent{
\ifVerboseLocation This is Derivative Compute Question 0003. \\ \fi
\begin{problem}

Compute the following derivative:

\input{Derivative-Compute-0003.HELP.tex}

\[\dfrac{d}{dx}\left(-20 \, \sin\left(\frac{2}{3} \, \pi x\right) \tan\left(-\frac{5}{6} \, \pi x\right)\right)=\answer{\frac{50}{3} \, \pi \sec\left(\frac{5}{6} \, \pi x\right)^{2} \sin\left(\frac{2}{3} \, \pi x\right) + \frac{40}{3} \, \pi \cos\left(\frac{2}{3} \, \pi x\right) \tan\left(\frac{5}{6} \, \pi x\right)}\]

\end{problem}}

%%%%%%%%%%%%%%%%%%%%%%

\latexProblemContent{
\ifVerboseLocation This is Derivative Compute Question 0003. \\ \fi
\begin{problem}

Compute the following derivative:

\input{Derivative-Compute-0003.HELP.tex}

\[\dfrac{d}{dx}\left(15 \, \tan\left(-\frac{1}{3} \, \pi x\right) \tan\left(-\pi x\right)\right)=\answer{5 \, \pi \sec\left(\frac{1}{3} \, \pi x\right)^{2} \tan\left(\pi x\right) + 15 \, \pi \sec\left(\pi x\right)^{2} \tan\left(\frac{1}{3} \, \pi x\right)}\]

\end{problem}}

%%%%%%%%%%%%%%%%%%%%%%

\latexProblemContent{
\ifVerboseLocation This is Derivative Compute Question 0003. \\ \fi
\begin{problem}

Compute the following derivative:

\input{Derivative-Compute-0003.HELP.tex}

\[\dfrac{d}{dx}\left(8 \, \sin\left(-\frac{1}{2} \, \pi x\right) \sin\left(-\frac{2}{3} \, \pi x\right)\right)=\answer{4 \, \pi \cos\left(\frac{1}{2} \, \pi x\right) \sin\left(\frac{2}{3} \, \pi x\right) + \frac{16}{3} \, \pi \cos\left(\frac{2}{3} \, \pi x\right) \sin\left(\frac{1}{2} \, \pi x\right)}\]

\end{problem}}

%%%%%%%%%%%%%%%%%%%%%%

\latexProblemContent{
\ifVerboseLocation This is Derivative Compute Question 0003. \\ \fi
\begin{problem}

Compute the following derivative:

\input{Derivative-Compute-0003.HELP.tex}

\[\dfrac{d}{dx}\left(4 \, \sin\left(-\frac{5}{6} \, \pi x\right) \sin\left(-\pi x\right)\right)=\answer{\frac{10}{3} \, \pi \cos\left(\frac{5}{6} \, \pi x\right) \sin\left(\pi x\right) + 4 \, \pi \cos\left(\pi x\right) \sin\left(\frac{5}{6} \, \pi x\right)}\]

\end{problem}}

%%%%%%%%%%%%%%%%%%%%%%

\latexProblemContent{
\ifVerboseLocation This is Derivative Compute Question 0003. \\ \fi
\begin{problem}

Compute the following derivative:

\input{Derivative-Compute-0003.HELP.tex}

\[\dfrac{d}{dx}\left(15 \, \cos\left(-\pi x\right) \sin\left(3 \, \pi x\right)\right)=\answer{45 \, \pi \cos\left(3 \, \pi x\right) \cos\left(\pi x\right) - 15 \, \pi \sin\left(3 \, \pi x\right) \sin\left(\pi x\right)}\]

\end{problem}}

%%%%%%%%%%%%%%%%%%%%%%

\latexProblemContent{
\ifVerboseLocation This is Derivative Compute Question 0003. \\ \fi
\begin{problem}

Compute the following derivative:

\input{Derivative-Compute-0003.HELP.tex}

\[\dfrac{d}{dx}\left(10 \, \cos\left(-\frac{1}{6} \, \pi x\right) \tan\left(\pi x\right)\right)=\answer{10 \, \pi \cos\left(\frac{1}{6} \, \pi x\right) \sec\left(\pi x\right)^{2} - \frac{5}{3} \, \pi \sin\left(\frac{1}{6} \, \pi x\right) \tan\left(\pi x\right)}\]

\end{problem}}

%%%%%%%%%%%%%%%%%%%%%%

\latexProblemContent{
\ifVerboseLocation This is Derivative Compute Question 0003. \\ \fi
\begin{problem}

Compute the following derivative:

\input{Derivative-Compute-0003.HELP.tex}

\[\dfrac{d}{dx}\left(15 \, \sin\left(\pi x\right) \sin\left(-5 \, \pi x\right)\right)=\answer{-15 \, \pi \cos\left(\pi x\right) \sin\left(5 \, \pi x\right) - 75 \, \pi \cos\left(5 \, \pi x\right) \sin\left(\pi x\right)}\]

\end{problem}}

%%%%%%%%%%%%%%%%%%%%%%

\latexProblemContent{
\ifVerboseLocation This is Derivative Compute Question 0003. \\ \fi
\begin{problem}

Compute the following derivative:

\input{Derivative-Compute-0003.HELP.tex}

\[\dfrac{d}{dx}\left(-5 \, \cos\left(-3 \, \pi x\right) \tan\left(\frac{4}{3} \, \pi x\right)\right)=\answer{-\frac{20}{3} \, \pi \cos\left(3 \, \pi x\right) \sec\left(\frac{4}{3} \, \pi x\right)^{2} + 15 \, \pi \sin\left(3 \, \pi x\right) \tan\left(\frac{4}{3} \, \pi x\right)}\]

\end{problem}}

%%%%%%%%%%%%%%%%%%%%%%

\latexProblemContent{
\ifVerboseLocation This is Derivative Compute Question 0003. \\ \fi
\begin{problem}

Compute the following derivative:

\input{Derivative-Compute-0003.HELP.tex}

\[\dfrac{d}{dx}\left(6 \, \cos\left(2 \, \pi x\right) \sin\left(-\pi x\right)\right)=\answer{-6 \, \pi \cos\left(2 \, \pi x\right) \cos\left(\pi x\right) + 12 \, \pi \sin\left(2 \, \pi x\right) \sin\left(\pi x\right)}\]

\end{problem}}

%%%%%%%%%%%%%%%%%%%%%%

\latexProblemContent{
\ifVerboseLocation This is Derivative Compute Question 0003. \\ \fi
\begin{problem}

Compute the following derivative:

\input{Derivative-Compute-0003.HELP.tex}

\[\dfrac{d}{dx}\left(-6 \, \cos\left(-4 \, \pi x\right) \sin\left(\frac{5}{3} \, \pi x\right)\right)=\answer{-10 \, \pi \cos\left(4 \, \pi x\right) \cos\left(\frac{5}{3} \, \pi x\right) + 24 \, \pi \sin\left(4 \, \pi x\right) \sin\left(\frac{5}{3} \, \pi x\right)}\]

\end{problem}}

%%%%%%%%%%%%%%%%%%%%%%

\latexProblemContent{
\ifVerboseLocation This is Derivative Compute Question 0003. \\ \fi
\begin{problem}

Compute the following derivative:

\input{Derivative-Compute-0003.HELP.tex}

\[\dfrac{d}{dx}\left(-9 \, \cos\left(\frac{5}{6} \, \pi x\right) \sin\left(-\pi x\right)\right)=\answer{9 \, \pi \cos\left(\pi x\right) \cos\left(\frac{5}{6} \, \pi x\right) - \frac{15}{2} \, \pi \sin\left(\pi x\right) \sin\left(\frac{5}{6} \, \pi x\right)}\]

\end{problem}}

%%%%%%%%%%%%%%%%%%%%%%

\latexProblemContent{
\ifVerboseLocation This is Derivative Compute Question 0003. \\ \fi
\begin{problem}

Compute the following derivative:

\input{Derivative-Compute-0003.HELP.tex}

\[\dfrac{d}{dx}\left(-6 \, \tan\left(-\frac{5}{6} \, \pi x\right) \tan\left(-5 \, \pi x\right)\right)=\answer{-5 \, \pi \sec\left(\frac{5}{6} \, \pi x\right)^{2} \tan\left(5 \, \pi x\right) - 30 \, \pi \sec\left(5 \, \pi x\right)^{2} \tan\left(\frac{5}{6} \, \pi x\right)}\]

\end{problem}}

%%%%%%%%%%%%%%%%%%%%%%

\latexProblemContent{
\ifVerboseLocation This is Derivative Compute Question 0003. \\ \fi
\begin{problem}

Compute the following derivative:

\input{Derivative-Compute-0003.HELP.tex}

\[\dfrac{d}{dx}\left(15 \, \sin\left(4 \, \pi x\right) \tan\left(5 \, \pi x\right)\right)=\answer{75 \, \pi \sec\left(5 \, \pi x\right)^{2} \sin\left(4 \, \pi x\right) + 60 \, \pi \cos\left(4 \, \pi x\right) \tan\left(5 \, \pi x\right)}\]

\end{problem}}

%%%%%%%%%%%%%%%%%%%%%%

\latexProblemContent{
\ifVerboseLocation This is Derivative Compute Question 0003. \\ \fi
\begin{problem}

Compute the following derivative:

\input{Derivative-Compute-0003.HELP.tex}

\[\dfrac{d}{dx}\left(8 \, \sin\left(\pi x\right) \tan\left(-\pi x\right)\right)=\answer{-8 \, \pi \sec\left(\pi x\right)^{2} \sin\left(\pi x\right) - 8 \, \pi \cos\left(\pi x\right) \tan\left(\pi x\right)}\]

\end{problem}}

%%%%%%%%%%%%%%%%%%%%%%

\latexProblemContent{
\ifVerboseLocation This is Derivative Compute Question 0003. \\ \fi
\begin{problem}

Compute the following derivative:

\input{Derivative-Compute-0003.HELP.tex}

\[\dfrac{d}{dx}\left(-6 \, \sin\left(\pi x\right) \tan\left(\frac{5}{6} \, \pi x\right)\right)=\answer{-5 \, \pi \sec\left(\frac{5}{6} \, \pi x\right)^{2} \sin\left(\pi x\right) - 6 \, \pi \cos\left(\pi x\right) \tan\left(\frac{5}{6} \, \pi x\right)}\]

\end{problem}}

%%%%%%%%%%%%%%%%%%%%%%

\latexProblemContent{
\ifVerboseLocation This is Derivative Compute Question 0003. \\ \fi
\begin{problem}

Compute the following derivative:

\input{Derivative-Compute-0003.HELP.tex}

\[\dfrac{d}{dx}\left(-2 \, \cos\left(-\frac{5}{2} \, \pi x\right) \tan\left(\frac{1}{2} \, \pi x\right)\right)=\answer{-\pi \cos\left(\frac{5}{2} \, \pi x\right) \sec\left(\frac{1}{2} \, \pi x\right)^{2} + 5 \, \pi \sin\left(\frac{5}{2} \, \pi x\right) \tan\left(\frac{1}{2} \, \pi x\right)}\]

\end{problem}}

%%%%%%%%%%%%%%%%%%%%%%

\latexProblemContent{
\ifVerboseLocation This is Derivative Compute Question 0003. \\ \fi
\begin{problem}

Compute the following derivative:

\input{Derivative-Compute-0003.HELP.tex}

\[\dfrac{d}{dx}\left(-25 \, \cos\left(-2 \, \pi x\right) \tan\left(\frac{1}{3} \, \pi x\right)\right)=\answer{-\frac{25}{3} \, \pi \cos\left(2 \, \pi x\right) \sec\left(\frac{1}{3} \, \pi x\right)^{2} + 50 \, \pi \sin\left(2 \, \pi x\right) \tan\left(\frac{1}{3} \, \pi x\right)}\]

\end{problem}}

%%%%%%%%%%%%%%%%%%%%%%

\latexProblemContent{
\ifVerboseLocation This is Derivative Compute Question 0003. \\ \fi
\begin{problem}

Compute the following derivative:

\input{Derivative-Compute-0003.HELP.tex}

\[\dfrac{d}{dx}\left(-3 \, \cos\left(\frac{1}{3} \, \pi x\right) \sin\left(\frac{1}{6} \, \pi x\right)\right)=\answer{-\frac{1}{2} \, \pi \cos\left(\frac{1}{3} \, \pi x\right) \cos\left(\frac{1}{6} \, \pi x\right) + \pi \sin\left(\frac{1}{3} \, \pi x\right) \sin\left(\frac{1}{6} \, \pi x\right)}\]

\end{problem}}

%%%%%%%%%%%%%%%%%%%%%%

\latexProblemContent{
\ifVerboseLocation This is Derivative Compute Question 0003. \\ \fi
\begin{problem}

Compute the following derivative:

\input{Derivative-Compute-0003.HELP.tex}

\[\dfrac{d}{dx}\left(3 \, \cos\left(-\pi x\right) \tan\left(5 \, \pi x\right)\right)=\answer{15 \, \pi \cos\left(\pi x\right) \sec\left(5 \, \pi x\right)^{2} - 3 \, \pi \sin\left(\pi x\right) \tan\left(5 \, \pi x\right)}\]

\end{problem}}

%%%%%%%%%%%%%%%%%%%%%%

\latexProblemContent{
\ifVerboseLocation This is Derivative Compute Question 0003. \\ \fi
\begin{problem}

Compute the following derivative:

\input{Derivative-Compute-0003.HELP.tex}

\[\dfrac{d}{dx}\left(-4 \, \cos\left(-\frac{1}{3} \, \pi x\right) \sin\left(-\pi x\right)\right)=\answer{4 \, \pi \cos\left(\pi x\right) \cos\left(\frac{1}{3} \, \pi x\right) - \frac{4}{3} \, \pi \sin\left(\pi x\right) \sin\left(\frac{1}{3} \, \pi x\right)}\]

\end{problem}}

%%%%%%%%%%%%%%%%%%%%%%

\latexProblemContent{
\ifVerboseLocation This is Derivative Compute Question 0003. \\ \fi
\begin{problem}

Compute the following derivative:

\input{Derivative-Compute-0003.HELP.tex}

\[\dfrac{d}{dx}\left(-4 \, \cos\left(-5 \, \pi x\right) \sin\left(-\frac{1}{3} \, \pi x\right)\right)=\answer{\frac{4}{3} \, \pi \cos\left(5 \, \pi x\right) \cos\left(\frac{1}{3} \, \pi x\right) - 20 \, \pi \sin\left(5 \, \pi x\right) \sin\left(\frac{1}{3} \, \pi x\right)}\]

\end{problem}}

%%%%%%%%%%%%%%%%%%%%%%

\latexProblemContent{
\ifVerboseLocation This is Derivative Compute Question 0003. \\ \fi
\begin{problem}

Compute the following derivative:

\input{Derivative-Compute-0003.HELP.tex}

\[\dfrac{d}{dx}\left(-12 \, \sin\left(-\pi x\right) \tan\left(-\frac{4}{3} \, \pi x\right)\right)=\answer{-16 \, \pi \sec\left(\frac{4}{3} \, \pi x\right)^{2} \sin\left(\pi x\right) - 12 \, \pi \cos\left(\pi x\right) \tan\left(\frac{4}{3} \, \pi x\right)}\]

\end{problem}}

%%%%%%%%%%%%%%%%%%%%%%

\latexProblemContent{
\ifVerboseLocation This is Derivative Compute Question 0003. \\ \fi
\begin{problem}

Compute the following derivative:

\input{Derivative-Compute-0003.HELP.tex}

\[\dfrac{d}{dx}\left(6 \, \sin\left(-2 \, \pi x\right) \sin\left(-5 \, \pi x\right)\right)=\answer{12 \, \pi \cos\left(2 \, \pi x\right) \sin\left(5 \, \pi x\right) + 30 \, \pi \cos\left(5 \, \pi x\right) \sin\left(2 \, \pi x\right)}\]

\end{problem}}

%%%%%%%%%%%%%%%%%%%%%%

\latexProblemContent{
\ifVerboseLocation This is Derivative Compute Question 0003. \\ \fi
\begin{problem}

Compute the following derivative:

\input{Derivative-Compute-0003.HELP.tex}

\[\dfrac{d}{dx}\left(-8 \, \cos\left(-\frac{3}{2} \, \pi x\right) \tan\left(-\frac{4}{3} \, \pi x\right)\right)=\answer{\frac{32}{3} \, \pi \cos\left(\frac{3}{2} \, \pi x\right) \sec\left(\frac{4}{3} \, \pi x\right)^{2} - 12 \, \pi \sin\left(\frac{3}{2} \, \pi x\right) \tan\left(\frac{4}{3} \, \pi x\right)}\]

\end{problem}}

%%%%%%%%%%%%%%%%%%%%%%

\latexProblemContent{
\ifVerboseLocation This is Derivative Compute Question 0003. \\ \fi
\begin{problem}

Compute the following derivative:

\input{Derivative-Compute-0003.HELP.tex}

\[\dfrac{d}{dx}\left(-5 \, \sin\left(2 \, \pi x\right) \sin\left(-4 \, \pi x\right)\right)=\answer{10 \, \pi \cos\left(2 \, \pi x\right) \sin\left(4 \, \pi x\right) + 20 \, \pi \cos\left(4 \, \pi x\right) \sin\left(2 \, \pi x\right)}\]

\end{problem}}

%%%%%%%%%%%%%%%%%%%%%%

\latexProblemContent{
\ifVerboseLocation This is Derivative Compute Question 0003. \\ \fi
\begin{problem}

Compute the following derivative:

\input{Derivative-Compute-0003.HELP.tex}

\[\dfrac{d}{dx}\left(-20 \, \cos\left(-\frac{1}{2} \, \pi x\right) \tan\left(5 \, \pi x\right)\right)=\answer{-100 \, \pi \cos\left(\frac{1}{2} \, \pi x\right) \sec\left(5 \, \pi x\right)^{2} + 10 \, \pi \sin\left(\frac{1}{2} \, \pi x\right) \tan\left(5 \, \pi x\right)}\]

\end{problem}}

%%%%%%%%%%%%%%%%%%%%%%

\latexProblemContent{
\ifVerboseLocation This is Derivative Compute Question 0003. \\ \fi
\begin{problem}

Compute the following derivative:

\input{Derivative-Compute-0003.HELP.tex}

\[\dfrac{d}{dx}\left(16 \, \sin\left(-\frac{2}{3} \, \pi x\right) \tan\left(-\frac{5}{3} \, \pi x\right)\right)=\answer{\frac{80}{3} \, \pi \sec\left(\frac{5}{3} \, \pi x\right)^{2} \sin\left(\frac{2}{3} \, \pi x\right) + \frac{32}{3} \, \pi \cos\left(\frac{2}{3} \, \pi x\right) \tan\left(\frac{5}{3} \, \pi x\right)}\]

\end{problem}}

%%%%%%%%%%%%%%%%%%%%%%

\latexProblemContent{
\ifVerboseLocation This is Derivative Compute Question 0003. \\ \fi
\begin{problem}

Compute the following derivative:

\input{Derivative-Compute-0003.HELP.tex}

\[\dfrac{d}{dx}\left(-3 \, \cos\left(3 \, \pi x\right) \cos\left(\frac{2}{3} \, \pi x\right)\right)=\answer{9 \, \pi \cos\left(\frac{2}{3} \, \pi x\right) \sin\left(3 \, \pi x\right) + 2 \, \pi \cos\left(3 \, \pi x\right) \sin\left(\frac{2}{3} \, \pi x\right)}\]

\end{problem}}

%%%%%%%%%%%%%%%%%%%%%%

\latexProblemContent{
\ifVerboseLocation This is Derivative Compute Question 0003. \\ \fi
\begin{problem}

Compute the following derivative:

\input{Derivative-Compute-0003.HELP.tex}

\[\dfrac{d}{dx}\left(6 \, \tan\left(\pi x\right) \tan\left(-\frac{1}{3} \, \pi x\right)\right)=\answer{-2 \, \pi \sec\left(\frac{1}{3} \, \pi x\right)^{2} \tan\left(\pi x\right) - 6 \, \pi \sec\left(\pi x\right)^{2} \tan\left(\frac{1}{3} \, \pi x\right)}\]

\end{problem}}

%%%%%%%%%%%%%%%%%%%%%%

\latexProblemContent{
\ifVerboseLocation This is Derivative Compute Question 0003. \\ \fi
\begin{problem}

Compute the following derivative:

\input{Derivative-Compute-0003.HELP.tex}

\[\dfrac{d}{dx}\left(2 \, \cos\left(\frac{1}{6} \, \pi x\right) \sin\left(\pi x\right)\right)=\answer{2 \, \pi \cos\left(\pi x\right) \cos\left(\frac{1}{6} \, \pi x\right) - \frac{1}{3} \, \pi \sin\left(\pi x\right) \sin\left(\frac{1}{6} \, \pi x\right)}\]

\end{problem}}

%%%%%%%%%%%%%%%%%%%%%%

\latexProblemContent{
\ifVerboseLocation This is Derivative Compute Question 0003. \\ \fi
\begin{problem}

Compute the following derivative:

\input{Derivative-Compute-0003.HELP.tex}

\[\dfrac{d}{dx}\left(-12 \, \cos\left(-\frac{1}{6} \, \pi x\right) \cos\left(-\frac{1}{3} \, \pi x\right)\right)=\answer{4 \, \pi \cos\left(\frac{1}{6} \, \pi x\right) \sin\left(\frac{1}{3} \, \pi x\right) + 2 \, \pi \cos\left(\frac{1}{3} \, \pi x\right) \sin\left(\frac{1}{6} \, \pi x\right)}\]

\end{problem}}

%%%%%%%%%%%%%%%%%%%%%%

\latexProblemContent{
\ifVerboseLocation This is Derivative Compute Question 0003. \\ \fi
\begin{problem}

Compute the following derivative:

\input{Derivative-Compute-0003.HELP.tex}

\[\dfrac{d}{dx}\left(9 \, \cos\left(-\frac{1}{3} \, \pi x\right) \tan\left(\frac{5}{6} \, \pi x\right)\right)=\answer{\frac{15}{2} \, \pi \cos\left(\frac{1}{3} \, \pi x\right) \sec\left(\frac{5}{6} \, \pi x\right)^{2} - 3 \, \pi \sin\left(\frac{1}{3} \, \pi x\right) \tan\left(\frac{5}{6} \, \pi x\right)}\]

\end{problem}}

%%%%%%%%%%%%%%%%%%%%%%

\latexProblemContent{
\ifVerboseLocation This is Derivative Compute Question 0003. \\ \fi
\begin{problem}

Compute the following derivative:

\input{Derivative-Compute-0003.HELP.tex}

\[\dfrac{d}{dx}\left(12 \, \sin\left(-\frac{3}{2} \, \pi x\right) \sin\left(-4 \, \pi x\right)\right)=\answer{18 \, \pi \cos\left(\frac{3}{2} \, \pi x\right) \sin\left(4 \, \pi x\right) + 48 \, \pi \cos\left(4 \, \pi x\right) \sin\left(\frac{3}{2} \, \pi x\right)}\]

\end{problem}}

%%%%%%%%%%%%%%%%%%%%%%

\latexProblemContent{
\ifVerboseLocation This is Derivative Compute Question 0003. \\ \fi
\begin{problem}

Compute the following derivative:

\input{Derivative-Compute-0003.HELP.tex}

\[\dfrac{d}{dx}\left(-5 \, \tan\left(\frac{2}{3} \, \pi x\right) \tan\left(-2 \, \pi x\right)\right)=\answer{\frac{10}{3} \, \pi \sec\left(\frac{2}{3} \, \pi x\right)^{2} \tan\left(2 \, \pi x\right) + 10 \, \pi \sec\left(2 \, \pi x\right)^{2} \tan\left(\frac{2}{3} \, \pi x\right)}\]

\end{problem}}

%%%%%%%%%%%%%%%%%%%%%%

\latexProblemContent{
\ifVerboseLocation This is Derivative Compute Question 0003. \\ \fi
\begin{problem}

Compute the following derivative:

\input{Derivative-Compute-0003.HELP.tex}

\[\dfrac{d}{dx}\left(\tan\left(\pi x\right) \tan\left(\frac{1}{2} \, \pi x\right)\right)=\answer{\frac{1}{2} \, \pi \sec\left(\frac{1}{2} \, \pi x\right)^{2} \tan\left(\pi x\right) + \pi \sec\left(\pi x\right)^{2} \tan\left(\frac{1}{2} \, \pi x\right)}\]

\end{problem}}

%%%%%%%%%%%%%%%%%%%%%%

\latexProblemContent{
\ifVerboseLocation This is Derivative Compute Question 0003. \\ \fi
\begin{problem}

Compute the following derivative:

\input{Derivative-Compute-0003.HELP.tex}

\[\dfrac{d}{dx}\left(-12 \, \cos\left(\frac{2}{3} \, \pi x\right) \sin\left(-5 \, \pi x\right)\right)=\answer{60 \, \pi \cos\left(5 \, \pi x\right) \cos\left(\frac{2}{3} \, \pi x\right) - 8 \, \pi \sin\left(5 \, \pi x\right) \sin\left(\frac{2}{3} \, \pi x\right)}\]

\end{problem}}

%%%%%%%%%%%%%%%%%%%%%%

\latexProblemContent{
\ifVerboseLocation This is Derivative Compute Question 0003. \\ \fi
\begin{problem}

Compute the following derivative:

\input{Derivative-Compute-0003.HELP.tex}

\[\dfrac{d}{dx}\left(5 \, \sin\left(-\pi x\right) \tan\left(-\frac{1}{3} \, \pi x\right)\right)=\answer{\frac{5}{3} \, \pi \sec\left(\frac{1}{3} \, \pi x\right)^{2} \sin\left(\pi x\right) + 5 \, \pi \cos\left(\pi x\right) \tan\left(\frac{1}{3} \, \pi x\right)}\]

\end{problem}}

%%%%%%%%%%%%%%%%%%%%%%

\latexProblemContent{
\ifVerboseLocation This is Derivative Compute Question 0003. \\ \fi
\begin{problem}

Compute the following derivative:

\input{Derivative-Compute-0003.HELP.tex}

\[\dfrac{d}{dx}\left(-8 \, \cos\left(-\frac{1}{3} \, \pi x\right) \tan\left(-\frac{1}{6} \, \pi x\right)\right)=\answer{\frac{4}{3} \, \pi \cos\left(\frac{1}{3} \, \pi x\right) \sec\left(\frac{1}{6} \, \pi x\right)^{2} - \frac{8}{3} \, \pi \sin\left(\frac{1}{3} \, \pi x\right) \tan\left(\frac{1}{6} \, \pi x\right)}\]

\end{problem}}

%%%%%%%%%%%%%%%%%%%%%%

\latexProblemContent{
\ifVerboseLocation This is Derivative Compute Question 0003. \\ \fi
\begin{problem}

Compute the following derivative:

\input{Derivative-Compute-0003.HELP.tex}

\[\dfrac{d}{dx}\left(15 \, \tan\left(2 \, \pi x\right) \tan\left(-3 \, \pi x\right)\right)=\answer{-30 \, \pi \sec\left(2 \, \pi x\right)^{2} \tan\left(3 \, \pi x\right) - 45 \, \pi \sec\left(3 \, \pi x\right)^{2} \tan\left(2 \, \pi x\right)}\]

\end{problem}}

%%%%%%%%%%%%%%%%%%%%%%

\latexProblemContent{
\ifVerboseLocation This is Derivative Compute Question 0003. \\ \fi
\begin{problem}

Compute the following derivative:

\input{Derivative-Compute-0003.HELP.tex}

\[\dfrac{d}{dx}\left(10 \, \cos\left(-\pi x\right) \tan\left(\frac{3}{2} \, \pi x\right)\right)=\answer{15 \, \pi \cos\left(\pi x\right) \sec\left(\frac{3}{2} \, \pi x\right)^{2} - 10 \, \pi \sin\left(\pi x\right) \tan\left(\frac{3}{2} \, \pi x\right)}\]

\end{problem}}

%%%%%%%%%%%%%%%%%%%%%%

\latexProblemContent{
\ifVerboseLocation This is Derivative Compute Question 0003. \\ \fi
\begin{problem}

Compute the following derivative:

\input{Derivative-Compute-0003.HELP.tex}

\[\dfrac{d}{dx}\left(15 \, \cos\left(-4 \, \pi x\right) \sin\left(\frac{1}{2} \, \pi x\right)\right)=\answer{\frac{15}{2} \, \pi \cos\left(4 \, \pi x\right) \cos\left(\frac{1}{2} \, \pi x\right) - 60 \, \pi \sin\left(4 \, \pi x\right) \sin\left(\frac{1}{2} \, \pi x\right)}\]

\end{problem}}

%%%%%%%%%%%%%%%%%%%%%%

\latexProblemContent{
\ifVerboseLocation This is Derivative Compute Question 0003. \\ \fi
\begin{problem}

Compute the following derivative:

\input{Derivative-Compute-0003.HELP.tex}

\[\dfrac{d}{dx}\left(3 \, \cos\left(\frac{4}{3} \, \pi x\right) \tan\left(\frac{3}{2} \, \pi x\right)\right)=\answer{\frac{9}{2} \, \pi \cos\left(\frac{4}{3} \, \pi x\right) \sec\left(\frac{3}{2} \, \pi x\right)^{2} - 4 \, \pi \sin\left(\frac{4}{3} \, \pi x\right) \tan\left(\frac{3}{2} \, \pi x\right)}\]

\end{problem}}

%%%%%%%%%%%%%%%%%%%%%%

\latexProblemContent{
\ifVerboseLocation This is Derivative Compute Question 0003. \\ \fi
\begin{problem}

Compute the following derivative:

\input{Derivative-Compute-0003.HELP.tex}

\[\dfrac{d}{dx}\left(-4 \, \cos\left(\frac{1}{2} \, \pi x\right) \sin\left(-\frac{5}{2} \, \pi x\right)\right)=\answer{10 \, \pi \cos\left(\frac{5}{2} \, \pi x\right) \cos\left(\frac{1}{2} \, \pi x\right) - 2 \, \pi \sin\left(\frac{5}{2} \, \pi x\right) \sin\left(\frac{1}{2} \, \pi x\right)}\]

\end{problem}}

%%%%%%%%%%%%%%%%%%%%%%

\latexProblemContent{
\ifVerboseLocation This is Derivative Compute Question 0003. \\ \fi
\begin{problem}

Compute the following derivative:

\input{Derivative-Compute-0003.HELP.tex}

\[\dfrac{d}{dx}\left(4 \, \sin\left(2 \, \pi x\right) \tan\left(-\pi x\right)\right)=\answer{-4 \, \pi \sec\left(\pi x\right)^{2} \sin\left(2 \, \pi x\right) - 8 \, \pi \cos\left(2 \, \pi x\right) \tan\left(\pi x\right)}\]

\end{problem}}

%%%%%%%%%%%%%%%%%%%%%%

\latexProblemContent{
\ifVerboseLocation This is Derivative Compute Question 0003. \\ \fi
\begin{problem}

Compute the following derivative:

\input{Derivative-Compute-0003.HELP.tex}

\[\dfrac{d}{dx}\left(6 \, \tan\left(-\frac{1}{3} \, \pi x\right) \tan\left(-2 \, \pi x\right)\right)=\answer{2 \, \pi \sec\left(\frac{1}{3} \, \pi x\right)^{2} \tan\left(2 \, \pi x\right) + 12 \, \pi \sec\left(2 \, \pi x\right)^{2} \tan\left(\frac{1}{3} \, \pi x\right)}\]

\end{problem}}

%%%%%%%%%%%%%%%%%%%%%%

\latexProblemContent{
\ifVerboseLocation This is Derivative Compute Question 0003. \\ \fi
\begin{problem}

Compute the following derivative:

\input{Derivative-Compute-0003.HELP.tex}

\[\dfrac{d}{dx}\left(12 \, \sin\left(\frac{3}{2} \, \pi x\right) \sin\left(\pi x\right)\right)=\answer{12 \, \pi \cos\left(\pi x\right) \sin\left(\frac{3}{2} \, \pi x\right) + 18 \, \pi \cos\left(\frac{3}{2} \, \pi x\right) \sin\left(\pi x\right)}\]

\end{problem}}

%%%%%%%%%%%%%%%%%%%%%%

\latexProblemContent{
\ifVerboseLocation This is Derivative Compute Question 0003. \\ \fi
\begin{problem}

Compute the following derivative:

\input{Derivative-Compute-0003.HELP.tex}

\[\dfrac{d}{dx}\left(4 \, \sin\left(-\frac{1}{6} \, \pi x\right) \tan\left(-\frac{1}{2} \, \pi x\right)\right)=\answer{2 \, \pi \sec\left(\frac{1}{2} \, \pi x\right)^{2} \sin\left(\frac{1}{6} \, \pi x\right) + \frac{2}{3} \, \pi \cos\left(\frac{1}{6} \, \pi x\right) \tan\left(\frac{1}{2} \, \pi x\right)}\]

\end{problem}}

%%%%%%%%%%%%%%%%%%%%%%

\latexProblemContent{
\ifVerboseLocation This is Derivative Compute Question 0003. \\ \fi
\begin{problem}

Compute the following derivative:

\input{Derivative-Compute-0003.HELP.tex}

\[\dfrac{d}{dx}\left(20 \, \cos\left(\frac{2}{3} \, \pi x\right) \tan\left(\frac{1}{6} \, \pi x\right)\right)=\answer{\frac{10}{3} \, \pi \cos\left(\frac{2}{3} \, \pi x\right) \sec\left(\frac{1}{6} \, \pi x\right)^{2} - \frac{40}{3} \, \pi \sin\left(\frac{2}{3} \, \pi x\right) \tan\left(\frac{1}{6} \, \pi x\right)}\]

\end{problem}}

%%%%%%%%%%%%%%%%%%%%%%

\latexProblemContent{
\ifVerboseLocation This is Derivative Compute Question 0003. \\ \fi
\begin{problem}

Compute the following derivative:

\input{Derivative-Compute-0003.HELP.tex}

\[\dfrac{d}{dx}\left(8 \, \cos\left(-\frac{1}{3} \, \pi x\right) \sin\left(3 \, \pi x\right)\right)=\answer{24 \, \pi \cos\left(3 \, \pi x\right) \cos\left(\frac{1}{3} \, \pi x\right) - \frac{8}{3} \, \pi \sin\left(3 \, \pi x\right) \sin\left(\frac{1}{3} \, \pi x\right)}\]

\end{problem}}

%%%%%%%%%%%%%%%%%%%%%%

\latexProblemContent{
\ifVerboseLocation This is Derivative Compute Question 0003. \\ \fi
\begin{problem}

Compute the following derivative:

\input{Derivative-Compute-0003.HELP.tex}

\[\dfrac{d}{dx}\left(15 \, \cos\left(5 \, \pi x\right) \cos\left(-\pi x\right)\right)=\answer{-75 \, \pi \cos\left(\pi x\right) \sin\left(5 \, \pi x\right) - 15 \, \pi \cos\left(5 \, \pi x\right) \sin\left(\pi x\right)}\]

\end{problem}}

%%%%%%%%%%%%%%%%%%%%%%

\latexProblemContent{
\ifVerboseLocation This is Derivative Compute Question 0003. \\ \fi
\begin{problem}

Compute the following derivative:

\input{Derivative-Compute-0003.HELP.tex}

\[\dfrac{d}{dx}\left(20 \, \sin\left(\frac{5}{2} \, \pi x\right) \tan\left(-\frac{1}{3} \, \pi x\right)\right)=\answer{-\frac{20}{3} \, \pi \sec\left(\frac{1}{3} \, \pi x\right)^{2} \sin\left(\frac{5}{2} \, \pi x\right) - 50 \, \pi \cos\left(\frac{5}{2} \, \pi x\right) \tan\left(\frac{1}{3} \, \pi x\right)}\]

\end{problem}}

%%%%%%%%%%%%%%%%%%%%%%

\latexProblemContent{
\ifVerboseLocation This is Derivative Compute Question 0003. \\ \fi
\begin{problem}

Compute the following derivative:

\input{Derivative-Compute-0003.HELP.tex}

\[\dfrac{d}{dx}\left(\tan\left(4 \, \pi x\right) \tan\left(-\frac{5}{3} \, \pi x\right)\right)=\answer{-\frac{5}{3} \, \pi \sec\left(\frac{5}{3} \, \pi x\right)^{2} \tan\left(4 \, \pi x\right) - 4 \, \pi \sec\left(4 \, \pi x\right)^{2} \tan\left(\frac{5}{3} \, \pi x\right)}\]

\end{problem}}

%%%%%%%%%%%%%%%%%%%%%%

\latexProblemContent{
\ifVerboseLocation This is Derivative Compute Question 0003. \\ \fi
\begin{problem}

Compute the following derivative:

\input{Derivative-Compute-0003.HELP.tex}

\[\dfrac{d}{dx}\left(9 \, \sin\left(-\frac{1}{3} \, \pi x\right) \tan\left(-4 \, \pi x\right)\right)=\answer{36 \, \pi \sec\left(4 \, \pi x\right)^{2} \sin\left(\frac{1}{3} \, \pi x\right) + 3 \, \pi \cos\left(\frac{1}{3} \, \pi x\right) \tan\left(4 \, \pi x\right)}\]

\end{problem}}

%%%%%%%%%%%%%%%%%%%%%%

\latexProblemContent{
\ifVerboseLocation This is Derivative Compute Question 0003. \\ \fi
\begin{problem}

Compute the following derivative:

\input{Derivative-Compute-0003.HELP.tex}

\[\dfrac{d}{dx}\left(10 \, \sin\left(\frac{5}{6} \, \pi x\right) \tan\left(4 \, \pi x\right)\right)=\answer{40 \, \pi \sec\left(4 \, \pi x\right)^{2} \sin\left(\frac{5}{6} \, \pi x\right) + \frac{25}{3} \, \pi \cos\left(\frac{5}{6} \, \pi x\right) \tan\left(4 \, \pi x\right)}\]

\end{problem}}

%%%%%%%%%%%%%%%%%%%%%%

\latexProblemContent{
\ifVerboseLocation This is Derivative Compute Question 0003. \\ \fi
\begin{problem}

Compute the following derivative:

\input{Derivative-Compute-0003.HELP.tex}

\[\dfrac{d}{dx}\left(10 \, \cos\left(\frac{5}{3} \, \pi x\right) \cos\left(-\frac{1}{3} \, \pi x\right)\right)=\answer{-\frac{50}{3} \, \pi \cos\left(\frac{1}{3} \, \pi x\right) \sin\left(\frac{5}{3} \, \pi x\right) - \frac{10}{3} \, \pi \cos\left(\frac{5}{3} \, \pi x\right) \sin\left(\frac{1}{3} \, \pi x\right)}\]

\end{problem}}

%%%%%%%%%%%%%%%%%%%%%%

\latexProblemContent{
\ifVerboseLocation This is Derivative Compute Question 0003. \\ \fi
\begin{problem}

Compute the following derivative:

\input{Derivative-Compute-0003.HELP.tex}

\[\dfrac{d}{dx}\left(-8 \, \cos\left(-\frac{1}{2} \, \pi x\right) \tan\left(-5 \, \pi x\right)\right)=\answer{40 \, \pi \cos\left(\frac{1}{2} \, \pi x\right) \sec\left(5 \, \pi x\right)^{2} - 4 \, \pi \sin\left(\frac{1}{2} \, \pi x\right) \tan\left(5 \, \pi x\right)}\]

\end{problem}}

%%%%%%%%%%%%%%%%%%%%%%

\latexProblemContent{
\ifVerboseLocation This is Derivative Compute Question 0003. \\ \fi
\begin{problem}

Compute the following derivative:

\input{Derivative-Compute-0003.HELP.tex}

\[\dfrac{d}{dx}\left(-5 \, \cos\left(\frac{3}{2} \, \pi x\right) \sin\left(\frac{2}{3} \, \pi x\right)\right)=\answer{-\frac{10}{3} \, \pi \cos\left(\frac{3}{2} \, \pi x\right) \cos\left(\frac{2}{3} \, \pi x\right) + \frac{15}{2} \, \pi \sin\left(\frac{3}{2} \, \pi x\right) \sin\left(\frac{2}{3} \, \pi x\right)}\]

\end{problem}}

%%%%%%%%%%%%%%%%%%%%%%

\latexProblemContent{
\ifVerboseLocation This is Derivative Compute Question 0003. \\ \fi
\begin{problem}

Compute the following derivative:

\input{Derivative-Compute-0003.HELP.tex}

\[\dfrac{d}{dx}\left(-2 \, \cos\left(\pi x\right) \cos\left(-\frac{1}{6} \, \pi x\right)\right)=\answer{2 \, \pi \cos\left(\frac{1}{6} \, \pi x\right) \sin\left(\pi x\right) + \frac{1}{3} \, \pi \cos\left(\pi x\right) \sin\left(\frac{1}{6} \, \pi x\right)}\]

\end{problem}}

%%%%%%%%%%%%%%%%%%%%%%

\latexProblemContent{
\ifVerboseLocation This is Derivative Compute Question 0003. \\ \fi
\begin{problem}

Compute the following derivative:

\input{Derivative-Compute-0003.HELP.tex}

\[\dfrac{d}{dx}\left(-20 \, \sin\left(\pi x\right) \tan\left(\pi x\right)\right)=\answer{-20 \, \pi \sec\left(\pi x\right)^{2} \sin\left(\pi x\right) - 20 \, \pi \cos\left(\pi x\right) \tan\left(\pi x\right)}\]

\end{problem}}

%%%%%%%%%%%%%%%%%%%%%%

\latexProblemContent{
\ifVerboseLocation This is Derivative Compute Question 0003. \\ \fi
\begin{problem}

Compute the following derivative:

\input{Derivative-Compute-0003.HELP.tex}

\[\dfrac{d}{dx}\left(10 \, \cos\left(\frac{3}{2} \, \pi x\right) \sin\left(-3 \, \pi x\right)\right)=\answer{-30 \, \pi \cos\left(3 \, \pi x\right) \cos\left(\frac{3}{2} \, \pi x\right) + 15 \, \pi \sin\left(3 \, \pi x\right) \sin\left(\frac{3}{2} \, \pi x\right)}\]

\end{problem}}

%%%%%%%%%%%%%%%%%%%%%%

\latexProblemContent{
\ifVerboseLocation This is Derivative Compute Question 0003. \\ \fi
\begin{problem}

Compute the following derivative:

\input{Derivative-Compute-0003.HELP.tex}

\[\dfrac{d}{dx}\left(-9 \, \cos\left(\frac{1}{2} \, \pi x\right) \sin\left(-\frac{5}{3} \, \pi x\right)\right)=\answer{15 \, \pi \cos\left(\frac{5}{3} \, \pi x\right) \cos\left(\frac{1}{2} \, \pi x\right) - \frac{9}{2} \, \pi \sin\left(\frac{5}{3} \, \pi x\right) \sin\left(\frac{1}{2} \, \pi x\right)}\]

\end{problem}}

%%%%%%%%%%%%%%%%%%%%%%

\latexProblemContent{
\ifVerboseLocation This is Derivative Compute Question 0003. \\ \fi
\begin{problem}

Compute the following derivative:

\input{Derivative-Compute-0003.HELP.tex}

\[\dfrac{d}{dx}\left(2 \, \tan\left(2 \, \pi x\right)^{2}\right)=\answer{8 \, \pi \sec\left(2 \, \pi x\right)^{2} \tan\left(2 \, \pi x\right)}\]

\end{problem}}

%%%%%%%%%%%%%%%%%%%%%%

\latexProblemContent{
\ifVerboseLocation This is Derivative Compute Question 0003. \\ \fi
\begin{problem}

Compute the following derivative:

\input{Derivative-Compute-0003.HELP.tex}

\[\dfrac{d}{dx}\left(6 \, \sin\left(4 \, \pi x\right) \sin\left(\frac{1}{3} \, \pi x\right)\right)=\answer{2 \, \pi \cos\left(\frac{1}{3} \, \pi x\right) \sin\left(4 \, \pi x\right) + 24 \, \pi \cos\left(4 \, \pi x\right) \sin\left(\frac{1}{3} \, \pi x\right)}\]

\end{problem}}

%%%%%%%%%%%%%%%%%%%%%%

\latexProblemContent{
\ifVerboseLocation This is Derivative Compute Question 0003. \\ \fi
\begin{problem}

Compute the following derivative:

\input{Derivative-Compute-0003.HELP.tex}

\[\dfrac{d}{dx}\left(20 \, \cos\left(-3 \, \pi x\right) \sin\left(\frac{1}{3} \, \pi x\right)\right)=\answer{\frac{20}{3} \, \pi \cos\left(3 \, \pi x\right) \cos\left(\frac{1}{3} \, \pi x\right) - 60 \, \pi \sin\left(3 \, \pi x\right) \sin\left(\frac{1}{3} \, \pi x\right)}\]

\end{problem}}

%%%%%%%%%%%%%%%%%%%%%%

\latexProblemContent{
\ifVerboseLocation This is Derivative Compute Question 0003. \\ \fi
\begin{problem}

Compute the following derivative:

\input{Derivative-Compute-0003.HELP.tex}

\[\dfrac{d}{dx}\left(-16 \, \sin\left(\frac{1}{3} \, \pi x\right) \tan\left(-5 \, \pi x\right)\right)=\answer{80 \, \pi \sec\left(5 \, \pi x\right)^{2} \sin\left(\frac{1}{3} \, \pi x\right) + \frac{16}{3} \, \pi \cos\left(\frac{1}{3} \, \pi x\right) \tan\left(5 \, \pi x\right)}\]

\end{problem}}

%%%%%%%%%%%%%%%%%%%%%%

\latexProblemContent{
\ifVerboseLocation This is Derivative Compute Question 0003. \\ \fi
\begin{problem}

Compute the following derivative:

\input{Derivative-Compute-0003.HELP.tex}

\[\dfrac{d}{dx}\left(-3 \, \cos\left(2 \, \pi x\right) \sin\left(-\frac{1}{6} \, \pi x\right)\right)=\answer{\frac{1}{2} \, \pi \cos\left(2 \, \pi x\right) \cos\left(\frac{1}{6} \, \pi x\right) - 6 \, \pi \sin\left(2 \, \pi x\right) \sin\left(\frac{1}{6} \, \pi x\right)}\]

\end{problem}}

%%%%%%%%%%%%%%%%%%%%%%

\latexProblemContent{
\ifVerboseLocation This is Derivative Compute Question 0003. \\ \fi
\begin{problem}

Compute the following derivative:

\input{Derivative-Compute-0003.HELP.tex}

\[\dfrac{d}{dx}\left(6 \, \cos\left(-\pi x\right) \tan\left(-\frac{1}{6} \, \pi x\right)\right)=\answer{-\pi \cos\left(\pi x\right) \sec\left(\frac{1}{6} \, \pi x\right)^{2} + 6 \, \pi \sin\left(\pi x\right) \tan\left(\frac{1}{6} \, \pi x\right)}\]

\end{problem}}

%%%%%%%%%%%%%%%%%%%%%%

\latexProblemContent{
\ifVerboseLocation This is Derivative Compute Question 0003. \\ \fi
\begin{problem}

Compute the following derivative:

\input{Derivative-Compute-0003.HELP.tex}

\[\dfrac{d}{dx}\left(-8 \, \sin\left(\pi x\right) \tan\left(\frac{3}{2} \, \pi x\right)\right)=\answer{-12 \, \pi \sec\left(\frac{3}{2} \, \pi x\right)^{2} \sin\left(\pi x\right) - 8 \, \pi \cos\left(\pi x\right) \tan\left(\frac{3}{2} \, \pi x\right)}\]

\end{problem}}

%%%%%%%%%%%%%%%%%%%%%%

\latexProblemContent{
\ifVerboseLocation This is Derivative Compute Question 0003. \\ \fi
\begin{problem}

Compute the following derivative:

\input{Derivative-Compute-0003.HELP.tex}

\[\dfrac{d}{dx}\left(8 \, \sin\left(-\frac{1}{6} \, \pi x\right) \tan\left(\frac{5}{3} \, \pi x\right)\right)=\answer{-\frac{40}{3} \, \pi \sec\left(\frac{5}{3} \, \pi x\right)^{2} \sin\left(\frac{1}{6} \, \pi x\right) - \frac{4}{3} \, \pi \cos\left(\frac{1}{6} \, \pi x\right) \tan\left(\frac{5}{3} \, \pi x\right)}\]

\end{problem}}

%%%%%%%%%%%%%%%%%%%%%%

\latexProblemContent{
\ifVerboseLocation This is Derivative Compute Question 0003. \\ \fi
\begin{problem}

Compute the following derivative:

\input{Derivative-Compute-0003.HELP.tex}

\[\dfrac{d}{dx}\left(5 \, \sin\left(3 \, \pi x\right) \sin\left(-\frac{1}{2} \, \pi x\right)\right)=\answer{-\frac{5}{2} \, \pi \cos\left(\frac{1}{2} \, \pi x\right) \sin\left(3 \, \pi x\right) - 15 \, \pi \cos\left(3 \, \pi x\right) \sin\left(\frac{1}{2} \, \pi x\right)}\]

\end{problem}}

%%%%%%%%%%%%%%%%%%%%%%

\latexProblemContent{
\ifVerboseLocation This is Derivative Compute Question 0003. \\ \fi
\begin{problem}

Compute the following derivative:

\input{Derivative-Compute-0003.HELP.tex}

\[\dfrac{d}{dx}\left(-10 \, \cos\left(4 \, \pi x\right) \tan\left(-\frac{2}{3} \, \pi x\right)\right)=\answer{\frac{20}{3} \, \pi \cos\left(4 \, \pi x\right) \sec\left(\frac{2}{3} \, \pi x\right)^{2} - 40 \, \pi \sin\left(4 \, \pi x\right) \tan\left(\frac{2}{3} \, \pi x\right)}\]

\end{problem}}

%%%%%%%%%%%%%%%%%%%%%%

\latexProblemContent{
\ifVerboseLocation This is Derivative Compute Question 0003. \\ \fi
\begin{problem}

Compute the following derivative:

\input{Derivative-Compute-0003.HELP.tex}

\[\dfrac{d}{dx}\left(-8 \, \cos\left(-\frac{5}{2} \, \pi x\right) \sin\left(\pi x\right)\right)=\answer{-8 \, \pi \cos\left(\frac{5}{2} \, \pi x\right) \cos\left(\pi x\right) + 20 \, \pi \sin\left(\frac{5}{2} \, \pi x\right) \sin\left(\pi x\right)}\]

\end{problem}}

%%%%%%%%%%%%%%%%%%%%%%

\latexProblemContent{
\ifVerboseLocation This is Derivative Compute Question 0003. \\ \fi
\begin{problem}

Compute the following derivative:

\input{Derivative-Compute-0003.HELP.tex}

\[\dfrac{d}{dx}\left(8 \, \cos\left(-\pi x\right) \tan\left(\frac{5}{6} \, \pi x\right)\right)=\answer{\frac{20}{3} \, \pi \cos\left(\pi x\right) \sec\left(\frac{5}{6} \, \pi x\right)^{2} - 8 \, \pi \sin\left(\pi x\right) \tan\left(\frac{5}{6} \, \pi x\right)}\]

\end{problem}}

%%%%%%%%%%%%%%%%%%%%%%

\latexProblemContent{
\ifVerboseLocation This is Derivative Compute Question 0003. \\ \fi
\begin{problem}

Compute the following derivative:

\input{Derivative-Compute-0003.HELP.tex}

\[\dfrac{d}{dx}\left(20 \, \cos\left(\frac{3}{2} \, \pi x\right) \sin\left(-\frac{1}{3} \, \pi x\right)\right)=\answer{-\frac{20}{3} \, \pi \cos\left(\frac{3}{2} \, \pi x\right) \cos\left(\frac{1}{3} \, \pi x\right) + 30 \, \pi \sin\left(\frac{3}{2} \, \pi x\right) \sin\left(\frac{1}{3} \, \pi x\right)}\]

\end{problem}}

%%%%%%%%%%%%%%%%%%%%%%

\latexProblemContent{
\ifVerboseLocation This is Derivative Compute Question 0003. \\ \fi
\begin{problem}

Compute the following derivative:

\input{Derivative-Compute-0003.HELP.tex}

\[\dfrac{d}{dx}\left(20 \, \sin\left(3 \, \pi x\right) \sin\left(-\frac{1}{2} \, \pi x\right)\right)=\answer{-10 \, \pi \cos\left(\frac{1}{2} \, \pi x\right) \sin\left(3 \, \pi x\right) - 60 \, \pi \cos\left(3 \, \pi x\right) \sin\left(\frac{1}{2} \, \pi x\right)}\]

\end{problem}}

%%%%%%%%%%%%%%%%%%%%%%

\latexProblemContent{
\ifVerboseLocation This is Derivative Compute Question 0003. \\ \fi
\begin{problem}

Compute the following derivative:

\input{Derivative-Compute-0003.HELP.tex}

\[\dfrac{d}{dx}\left(-4 \, \cos\left(\pi x\right) \tan\left(-\frac{1}{3} \, \pi x\right)\right)=\answer{\frac{4}{3} \, \pi \cos\left(\pi x\right) \sec\left(\frac{1}{3} \, \pi x\right)^{2} - 4 \, \pi \sin\left(\pi x\right) \tan\left(\frac{1}{3} \, \pi x\right)}\]

\end{problem}}

%%%%%%%%%%%%%%%%%%%%%%

\latexProblemContent{
\ifVerboseLocation This is Derivative Compute Question 0003. \\ \fi
\begin{problem}

Compute the following derivative:

\input{Derivative-Compute-0003.HELP.tex}

\[\dfrac{d}{dx}\left(20 \, \tan\left(\frac{4}{3} \, \pi x\right) \tan\left(-\frac{2}{3} \, \pi x\right)\right)=\answer{-\frac{40}{3} \, \pi \sec\left(\frac{2}{3} \, \pi x\right)^{2} \tan\left(\frac{4}{3} \, \pi x\right) - \frac{80}{3} \, \pi \sec\left(\frac{4}{3} \, \pi x\right)^{2} \tan\left(\frac{2}{3} \, \pi x\right)}\]

\end{problem}}

%%%%%%%%%%%%%%%%%%%%%%

\latexProblemContent{
\ifVerboseLocation This is Derivative Compute Question 0003. \\ \fi
\begin{problem}

Compute the following derivative:

\input{Derivative-Compute-0003.HELP.tex}

\[\dfrac{d}{dx}\left(5 \, \sin\left(\pi x\right) \tan\left(-\frac{1}{2} \, \pi x\right)\right)=\answer{-\frac{5}{2} \, \pi \sec\left(\frac{1}{2} \, \pi x\right)^{2} \sin\left(\pi x\right) - 5 \, \pi \cos\left(\pi x\right) \tan\left(\frac{1}{2} \, \pi x\right)}\]

\end{problem}}

%%%%%%%%%%%%%%%%%%%%%%

\latexProblemContent{
\ifVerboseLocation This is Derivative Compute Question 0003. \\ \fi
\begin{problem}

Compute the following derivative:

\input{Derivative-Compute-0003.HELP.tex}

\[\dfrac{d}{dx}\left(-15 \, \cos\left(-\frac{2}{3} \, \pi x\right) \sin\left(2 \, \pi x\right)\right)=\answer{-30 \, \pi \cos\left(2 \, \pi x\right) \cos\left(\frac{2}{3} \, \pi x\right) + 10 \, \pi \sin\left(2 \, \pi x\right) \sin\left(\frac{2}{3} \, \pi x\right)}\]

\end{problem}}

%%%%%%%%%%%%%%%%%%%%%%

\latexProblemContent{
\ifVerboseLocation This is Derivative Compute Question 0003. \\ \fi
\begin{problem}

Compute the following derivative:

\input{Derivative-Compute-0003.HELP.tex}

\[\dfrac{d}{dx}\left(8 \, \cos\left(-2 \, \pi x\right) \tan\left(-\frac{1}{6} \, \pi x\right)\right)=\answer{-\frac{4}{3} \, \pi \cos\left(2 \, \pi x\right) \sec\left(\frac{1}{6} \, \pi x\right)^{2} + 16 \, \pi \sin\left(2 \, \pi x\right) \tan\left(\frac{1}{6} \, \pi x\right)}\]

\end{problem}}

%%%%%%%%%%%%%%%%%%%%%%

\latexProblemContent{
\ifVerboseLocation This is Derivative Compute Question 0003. \\ \fi
\begin{problem}

Compute the following derivative:

\input{Derivative-Compute-0003.HELP.tex}

\[\dfrac{d}{dx}\left(2 \, \cos\left(\frac{1}{2} \, \pi x\right) \cos\left(-4 \, \pi x\right)\right)=\answer{-8 \, \pi \cos\left(\frac{1}{2} \, \pi x\right) \sin\left(4 \, \pi x\right) - \pi \cos\left(4 \, \pi x\right) \sin\left(\frac{1}{2} \, \pi x\right)}\]

\end{problem}}

%%%%%%%%%%%%%%%%%%%%%%

\latexProblemContent{
\ifVerboseLocation This is Derivative Compute Question 0003. \\ \fi
\begin{problem}

Compute the following derivative:

\input{Derivative-Compute-0003.HELP.tex}

\[\dfrac{d}{dx}\left(-25 \, \cos\left(2 \, \pi x\right) \tan\left(\pi x\right)\right)=\answer{-25 \, \pi \cos\left(2 \, \pi x\right) \sec\left(\pi x\right)^{2} + 50 \, \pi \sin\left(2 \, \pi x\right) \tan\left(\pi x\right)}\]

\end{problem}}

%%%%%%%%%%%%%%%%%%%%%%

\latexProblemContent{
\ifVerboseLocation This is Derivative Compute Question 0003. \\ \fi
\begin{problem}

Compute the following derivative:

\input{Derivative-Compute-0003.HELP.tex}

\[\dfrac{d}{dx}\left(-3 \, \sin\left(-5 \, \pi x\right) \tan\left(-3 \, \pi x\right)\right)=\answer{-9 \, \pi \sec\left(3 \, \pi x\right)^{2} \sin\left(5 \, \pi x\right) - 15 \, \pi \cos\left(5 \, \pi x\right) \tan\left(3 \, \pi x\right)}\]

\end{problem}}

%%%%%%%%%%%%%%%%%%%%%%

\latexProblemContent{
\ifVerboseLocation This is Derivative Compute Question 0003. \\ \fi
\begin{problem}

Compute the following derivative:

\input{Derivative-Compute-0003.HELP.tex}

\[\dfrac{d}{dx}\left(-10 \, \sin\left(5 \, \pi x\right) \tan\left(-3 \, \pi x\right)\right)=\answer{30 \, \pi \sec\left(3 \, \pi x\right)^{2} \sin\left(5 \, \pi x\right) + 50 \, \pi \cos\left(5 \, \pi x\right) \tan\left(3 \, \pi x\right)}\]

\end{problem}}

%%%%%%%%%%%%%%%%%%%%%%

\latexProblemContent{
\ifVerboseLocation This is Derivative Compute Question 0003. \\ \fi
\begin{problem}

Compute the following derivative:

\input{Derivative-Compute-0003.HELP.tex}

\[\dfrac{d}{dx}\left(9 \, \cos\left(\frac{1}{3} \, \pi x\right) \tan\left(5 \, \pi x\right)\right)=\answer{45 \, \pi \cos\left(\frac{1}{3} \, \pi x\right) \sec\left(5 \, \pi x\right)^{2} - 3 \, \pi \sin\left(\frac{1}{3} \, \pi x\right) \tan\left(5 \, \pi x\right)}\]

\end{problem}}

%%%%%%%%%%%%%%%%%%%%%%

\latexProblemContent{
\ifVerboseLocation This is Derivative Compute Question 0003. \\ \fi
\begin{problem}

Compute the following derivative:

\input{Derivative-Compute-0003.HELP.tex}

\[\dfrac{d}{dx}\left(6 \, \sin\left(4 \, \pi x\right) \sin\left(-\pi x\right)\right)=\answer{-6 \, \pi \cos\left(\pi x\right) \sin\left(4 \, \pi x\right) - 24 \, \pi \cos\left(4 \, \pi x\right) \sin\left(\pi x\right)}\]

\end{problem}}

%%%%%%%%%%%%%%%%%%%%%%

\latexProblemContent{
\ifVerboseLocation This is Derivative Compute Question 0003. \\ \fi
\begin{problem}

Compute the following derivative:

\input{Derivative-Compute-0003.HELP.tex}

\[\dfrac{d}{dx}\left(4 \, \cos\left(\frac{2}{3} \, \pi x\right) \tan\left(-5 \, \pi x\right)\right)=\answer{-20 \, \pi \cos\left(\frac{2}{3} \, \pi x\right) \sec\left(5 \, \pi x\right)^{2} + \frac{8}{3} \, \pi \sin\left(\frac{2}{3} \, \pi x\right) \tan\left(5 \, \pi x\right)}\]

\end{problem}}

%%%%%%%%%%%%%%%%%%%%%%

\latexProblemContent{
\ifVerboseLocation This is Derivative Compute Question 0003. \\ \fi
\begin{problem}

Compute the following derivative:

\input{Derivative-Compute-0003.HELP.tex}

\[\dfrac{d}{dx}\left(10 \, \cos\left(-\frac{4}{3} \, \pi x\right) \sin\left(2 \, \pi x\right)\right)=\answer{20 \, \pi \cos\left(2 \, \pi x\right) \cos\left(\frac{4}{3} \, \pi x\right) - \frac{40}{3} \, \pi \sin\left(2 \, \pi x\right) \sin\left(\frac{4}{3} \, \pi x\right)}\]

\end{problem}}

%%%%%%%%%%%%%%%%%%%%%%

\latexProblemContent{
\ifVerboseLocation This is Derivative Compute Question 0003. \\ \fi
\begin{problem}

Compute the following derivative:

\input{Derivative-Compute-0003.HELP.tex}

\[\dfrac{d}{dx}\left(-6 \, \sin\left(-\pi x\right) \tan\left(-\frac{1}{2} \, \pi x\right)\right)=\answer{-3 \, \pi \sec\left(\frac{1}{2} \, \pi x\right)^{2} \sin\left(\pi x\right) - 6 \, \pi \cos\left(\pi x\right) \tan\left(\frac{1}{2} \, \pi x\right)}\]

\end{problem}}

%%%%%%%%%%%%%%%%%%%%%%

\latexProblemContent{
\ifVerboseLocation This is Derivative Compute Question 0003. \\ \fi
\begin{problem}

Compute the following derivative:

\input{Derivative-Compute-0003.HELP.tex}

\[\dfrac{d}{dx}\left(-15 \, \tan\left(\frac{1}{2} \, \pi x\right) \tan\left(-\frac{5}{6} \, \pi x\right)\right)=\answer{\frac{15}{2} \, \pi \sec\left(\frac{1}{2} \, \pi x\right)^{2} \tan\left(\frac{5}{6} \, \pi x\right) + \frac{25}{2} \, \pi \sec\left(\frac{5}{6} \, \pi x\right)^{2} \tan\left(\frac{1}{2} \, \pi x\right)}\]

\end{problem}}

%%%%%%%%%%%%%%%%%%%%%%

\latexProblemContent{
\ifVerboseLocation This is Derivative Compute Question 0003. \\ \fi
\begin{problem}

Compute the following derivative:

\input{Derivative-Compute-0003.HELP.tex}

\[\dfrac{d}{dx}\left(15 \, \cos\left(\frac{2}{3} \, \pi x\right) \sin\left(-\frac{1}{3} \, \pi x\right)\right)=\answer{-5 \, \pi \cos\left(\frac{2}{3} \, \pi x\right) \cos\left(\frac{1}{3} \, \pi x\right) + 10 \, \pi \sin\left(\frac{2}{3} \, \pi x\right) \sin\left(\frac{1}{3} \, \pi x\right)}\]

\end{problem}}

%%%%%%%%%%%%%%%%%%%%%%

\latexProblemContent{
\ifVerboseLocation This is Derivative Compute Question 0003. \\ \fi
\begin{problem}

Compute the following derivative:

\input{Derivative-Compute-0003.HELP.tex}

\[\dfrac{d}{dx}\left(-5 \, \tan\left(3 \, \pi x\right) \tan\left(\frac{1}{3} \, \pi x\right)\right)=\answer{-\frac{5}{3} \, \pi \sec\left(\frac{1}{3} \, \pi x\right)^{2} \tan\left(3 \, \pi x\right) - 15 \, \pi \sec\left(3 \, \pi x\right)^{2} \tan\left(\frac{1}{3} \, \pi x\right)}\]

\end{problem}}

%%%%%%%%%%%%%%%%%%%%%%

\latexProblemContent{
\ifVerboseLocation This is Derivative Compute Question 0003. \\ \fi
\begin{problem}

Compute the following derivative:

\input{Derivative-Compute-0003.HELP.tex}

\[\dfrac{d}{dx}\left(20 \, \sin\left(-2 \, \pi x\right) \sin\left(-4 \, \pi x\right)\right)=\answer{40 \, \pi \cos\left(2 \, \pi x\right) \sin\left(4 \, \pi x\right) + 80 \, \pi \cos\left(4 \, \pi x\right) \sin\left(2 \, \pi x\right)}\]

\end{problem}}

%%%%%%%%%%%%%%%%%%%%%%

\latexProblemContent{
\ifVerboseLocation This is Derivative Compute Question 0003. \\ \fi
\begin{problem}

Compute the following derivative:

\input{Derivative-Compute-0003.HELP.tex}

\[\dfrac{d}{dx}\left(6 \, \cos\left(-\frac{2}{3} \, \pi x\right) \sin\left(5 \, \pi x\right)\right)=\answer{30 \, \pi \cos\left(5 \, \pi x\right) \cos\left(\frac{2}{3} \, \pi x\right) - 4 \, \pi \sin\left(5 \, \pi x\right) \sin\left(\frac{2}{3} \, \pi x\right)}\]

\end{problem}}

%%%%%%%%%%%%%%%%%%%%%%

\latexProblemContent{
\ifVerboseLocation This is Derivative Compute Question 0003. \\ \fi
\begin{problem}

Compute the following derivative:

\input{Derivative-Compute-0003.HELP.tex}

\[\dfrac{d}{dx}\left(-6 \, \cos\left(-\frac{4}{3} \, \pi x\right) \tan\left(-\frac{4}{3} \, \pi x\right)\right)=\answer{8 \, \pi \cos\left(\frac{4}{3} \, \pi x\right) \sec\left(\frac{4}{3} \, \pi x\right)^{2} - 8 \, \pi \sin\left(\frac{4}{3} \, \pi x\right) \tan\left(\frac{4}{3} \, \pi x\right)}\]

\end{problem}}

%%%%%%%%%%%%%%%%%%%%%%

\latexProblemContent{
\ifVerboseLocation This is Derivative Compute Question 0003. \\ \fi
\begin{problem}

Compute the following derivative:

\input{Derivative-Compute-0003.HELP.tex}

\[\dfrac{d}{dx}\left(-3 \, \sin\left(-\frac{4}{3} \, \pi x\right) \tan\left(-\frac{2}{3} \, \pi x\right)\right)=\answer{-2 \, \pi \sec\left(\frac{2}{3} \, \pi x\right)^{2} \sin\left(\frac{4}{3} \, \pi x\right) - 4 \, \pi \cos\left(\frac{4}{3} \, \pi x\right) \tan\left(\frac{2}{3} \, \pi x\right)}\]

\end{problem}}

%%%%%%%%%%%%%%%%%%%%%%

\latexProblemContent{
\ifVerboseLocation This is Derivative Compute Question 0003. \\ \fi
\begin{problem}

Compute the following derivative:

\input{Derivative-Compute-0003.HELP.tex}

\[\dfrac{d}{dx}\left(25 \, \sin\left(\pi x\right) \tan\left(\frac{1}{3} \, \pi x\right)\right)=\answer{\frac{25}{3} \, \pi \sec\left(\frac{1}{3} \, \pi x\right)^{2} \sin\left(\pi x\right) + 25 \, \pi \cos\left(\pi x\right) \tan\left(\frac{1}{3} \, \pi x\right)}\]

\end{problem}}

%%%%%%%%%%%%%%%%%%%%%%

\latexProblemContent{
\ifVerboseLocation This is Derivative Compute Question 0003. \\ \fi
\begin{problem}

Compute the following derivative:

\input{Derivative-Compute-0003.HELP.tex}

\[\dfrac{d}{dx}\left(5 \, \cos\left(2 \, \pi x\right) \sin\left(4 \, \pi x\right)\right)=\answer{20 \, \pi \cos\left(4 \, \pi x\right) \cos\left(2 \, \pi x\right) - 10 \, \pi \sin\left(4 \, \pi x\right) \sin\left(2 \, \pi x\right)}\]

\end{problem}}

%%%%%%%%%%%%%%%%%%%%%%

\latexProblemContent{
\ifVerboseLocation This is Derivative Compute Question 0003. \\ \fi
\begin{problem}

Compute the following derivative:

\input{Derivative-Compute-0003.HELP.tex}

\[\dfrac{d}{dx}\left(16 \, \sin\left(\frac{1}{3} \, \pi x\right) \sin\left(-\frac{5}{3} \, \pi x\right)\right)=\answer{-\frac{16}{3} \, \pi \cos\left(\frac{1}{3} \, \pi x\right) \sin\left(\frac{5}{3} \, \pi x\right) - \frac{80}{3} \, \pi \cos\left(\frac{5}{3} \, \pi x\right) \sin\left(\frac{1}{3} \, \pi x\right)}\]

\end{problem}}

%%%%%%%%%%%%%%%%%%%%%%

\latexProblemContent{
\ifVerboseLocation This is Derivative Compute Question 0003. \\ \fi
\begin{problem}

Compute the following derivative:

\input{Derivative-Compute-0003.HELP.tex}

\[\dfrac{d}{dx}\left(-3 \, \sin\left(\pi x\right) \sin\left(-\pi x\right)\right)=\answer{6 \, \pi \cos\left(\pi x\right) \sin\left(\pi x\right)}\]

\end{problem}}

%%%%%%%%%%%%%%%%%%%%%%

\latexProblemContent{
\ifVerboseLocation This is Derivative Compute Question 0003. \\ \fi
\begin{problem}

Compute the following derivative:

\input{Derivative-Compute-0003.HELP.tex}

\[\dfrac{d}{dx}\left(-2 \, \cos\left(-\frac{1}{2} \, \pi x\right) \sin\left(-\pi x\right)\right)=\answer{2 \, \pi \cos\left(\pi x\right) \cos\left(\frac{1}{2} \, \pi x\right) - \pi \sin\left(\pi x\right) \sin\left(\frac{1}{2} \, \pi x\right)}\]

\end{problem}}

%%%%%%%%%%%%%%%%%%%%%%

\latexProblemContent{
\ifVerboseLocation This is Derivative Compute Question 0003. \\ \fi
\begin{problem}

Compute the following derivative:

\input{Derivative-Compute-0003.HELP.tex}

\[\dfrac{d}{dx}\left(-\tan\left(\frac{3}{2} \, \pi x\right) \tan\left(-\frac{2}{3} \, \pi x\right)\right)=\answer{\frac{2}{3} \, \pi \sec\left(\frac{2}{3} \, \pi x\right)^{2} \tan\left(\frac{3}{2} \, \pi x\right) + \frac{3}{2} \, \pi \sec\left(\frac{3}{2} \, \pi x\right)^{2} \tan\left(\frac{2}{3} \, \pi x\right)}\]

\end{problem}}

%%%%%%%%%%%%%%%%%%%%%%

\latexProblemContent{
\ifVerboseLocation This is Derivative Compute Question 0003. \\ \fi
\begin{problem}

Compute the following derivative:

\input{Derivative-Compute-0003.HELP.tex}

\[\dfrac{d}{dx}\left(-3 \, \tan\left(2 \, \pi x\right) \tan\left(\frac{4}{3} \, \pi x\right)\right)=\answer{-4 \, \pi \sec\left(\frac{4}{3} \, \pi x\right)^{2} \tan\left(2 \, \pi x\right) - 6 \, \pi \sec\left(2 \, \pi x\right)^{2} \tan\left(\frac{4}{3} \, \pi x\right)}\]

\end{problem}}

%%%%%%%%%%%%%%%%%%%%%%

\latexProblemContent{
\ifVerboseLocation This is Derivative Compute Question 0003. \\ \fi
\begin{problem}

Compute the following derivative:

\input{Derivative-Compute-0003.HELP.tex}

\[\dfrac{d}{dx}\left(-10 \, \sin\left(\frac{1}{2} \, \pi x\right) \sin\left(-\pi x\right)\right)=\answer{5 \, \pi \cos\left(\frac{1}{2} \, \pi x\right) \sin\left(\pi x\right) + 10 \, \pi \cos\left(\pi x\right) \sin\left(\frac{1}{2} \, \pi x\right)}\]

\end{problem}}

%%%%%%%%%%%%%%%%%%%%%%

\latexProblemContent{
\ifVerboseLocation This is Derivative Compute Question 0003. \\ \fi
\begin{problem}

Compute the following derivative:

\input{Derivative-Compute-0003.HELP.tex}

\[\dfrac{d}{dx}\left(3 \, \cos\left(\frac{2}{3} \, \pi x\right) \tan\left(-\frac{1}{3} \, \pi x\right)\right)=\answer{-\pi \cos\left(\frac{2}{3} \, \pi x\right) \sec\left(\frac{1}{3} \, \pi x\right)^{2} + 2 \, \pi \sin\left(\frac{2}{3} \, \pi x\right) \tan\left(\frac{1}{3} \, \pi x\right)}\]

\end{problem}}

%%%%%%%%%%%%%%%%%%%%%%

\latexProblemContent{
\ifVerboseLocation This is Derivative Compute Question 0003. \\ \fi
\begin{problem}

Compute the following derivative:

\input{Derivative-Compute-0003.HELP.tex}

\[\dfrac{d}{dx}\left(-2 \, \cos\left(-\frac{1}{6} \, \pi x\right) \cos\left(-\frac{1}{2} \, \pi x\right)\right)=\answer{\pi \cos\left(\frac{1}{6} \, \pi x\right) \sin\left(\frac{1}{2} \, \pi x\right) + \frac{1}{3} \, \pi \cos\left(\frac{1}{2} \, \pi x\right) \sin\left(\frac{1}{6} \, \pi x\right)}\]

\end{problem}}

%%%%%%%%%%%%%%%%%%%%%%

\latexProblemContent{
\ifVerboseLocation This is Derivative Compute Question 0003. \\ \fi
\begin{problem}

Compute the following derivative:

\input{Derivative-Compute-0003.HELP.tex}

\[\dfrac{d}{dx}\left(-\sin\left(-\pi x\right) \tan\left(\frac{3}{2} \, \pi x\right)\right)=\answer{\frac{3}{2} \, \pi \sec\left(\frac{3}{2} \, \pi x\right)^{2} \sin\left(\pi x\right) + \pi \cos\left(\pi x\right) \tan\left(\frac{3}{2} \, \pi x\right)}\]

\end{problem}}

%%%%%%%%%%%%%%%%%%%%%%

\latexProblemContent{
\ifVerboseLocation This is Derivative Compute Question 0003. \\ \fi
\begin{problem}

Compute the following derivative:

\input{Derivative-Compute-0003.HELP.tex}

\[\dfrac{d}{dx}\left(-5 \, \sin\left(\frac{5}{3} \, \pi x\right) \sin\left(-\pi x\right)\right)=\answer{5 \, \pi \cos\left(\pi x\right) \sin\left(\frac{5}{3} \, \pi x\right) + \frac{25}{3} \, \pi \cos\left(\frac{5}{3} \, \pi x\right) \sin\left(\pi x\right)}\]

\end{problem}}

%%%%%%%%%%%%%%%%%%%%%%

\latexProblemContent{
\ifVerboseLocation This is Derivative Compute Question 0003. \\ \fi
\begin{problem}

Compute the following derivative:

\input{Derivative-Compute-0003.HELP.tex}

\[\dfrac{d}{dx}\left(-10 \, \sin\left(\pi x\right) \sin\left(\frac{1}{2} \, \pi x\right)\right)=\answer{-5 \, \pi \cos\left(\frac{1}{2} \, \pi x\right) \sin\left(\pi x\right) - 10 \, \pi \cos\left(\pi x\right) \sin\left(\frac{1}{2} \, \pi x\right)}\]

\end{problem}}

%%%%%%%%%%%%%%%%%%%%%%

\latexProblemContent{
\ifVerboseLocation This is Derivative Compute Question 0003. \\ \fi
\begin{problem}

Compute the following derivative:

\input{Derivative-Compute-0003.HELP.tex}

\[\dfrac{d}{dx}\left(4 \, \tan\left(\pi x\right) \tan\left(-\frac{5}{3} \, \pi x\right)\right)=\answer{-4 \, \pi \sec\left(\pi x\right)^{2} \tan\left(\frac{5}{3} \, \pi x\right) - \frac{20}{3} \, \pi \sec\left(\frac{5}{3} \, \pi x\right)^{2} \tan\left(\pi x\right)}\]

\end{problem}}

%%%%%%%%%%%%%%%%%%%%%%

\latexProblemContent{
\ifVerboseLocation This is Derivative Compute Question 0003. \\ \fi
\begin{problem}

Compute the following derivative:

\input{Derivative-Compute-0003.HELP.tex}

\[\dfrac{d}{dx}\left(25 \, \cos\left(2 \, \pi x\right) \tan\left(\frac{3}{2} \, \pi x\right)\right)=\answer{\frac{75}{2} \, \pi \cos\left(2 \, \pi x\right) \sec\left(\frac{3}{2} \, \pi x\right)^{2} - 50 \, \pi \sin\left(2 \, \pi x\right) \tan\left(\frac{3}{2} \, \pi x\right)}\]

\end{problem}}

%%%%%%%%%%%%%%%%%%%%%%

\latexProblemContent{
\ifVerboseLocation This is Derivative Compute Question 0003. \\ \fi
\begin{problem}

Compute the following derivative:

\input{Derivative-Compute-0003.HELP.tex}

\[\dfrac{d}{dx}\left(8 \, \cos\left(\frac{5}{6} \, \pi x\right) \cos\left(-\frac{3}{2} \, \pi x\right)\right)=\answer{-12 \, \pi \cos\left(\frac{5}{6} \, \pi x\right) \sin\left(\frac{3}{2} \, \pi x\right) - \frac{20}{3} \, \pi \cos\left(\frac{3}{2} \, \pi x\right) \sin\left(\frac{5}{6} \, \pi x\right)}\]

\end{problem}}

%%%%%%%%%%%%%%%%%%%%%%

\latexProblemContent{
\ifVerboseLocation This is Derivative Compute Question 0003. \\ \fi
\begin{problem}

Compute the following derivative:

\input{Derivative-Compute-0003.HELP.tex}

\[\dfrac{d}{dx}\left(-6 \, \sin\left(\frac{5}{6} \, \pi x\right) \sin\left(-3 \, \pi x\right)\right)=\answer{5 \, \pi \cos\left(\frac{5}{6} \, \pi x\right) \sin\left(3 \, \pi x\right) + 18 \, \pi \cos\left(3 \, \pi x\right) \sin\left(\frac{5}{6} \, \pi x\right)}\]

\end{problem}}

%%%%%%%%%%%%%%%%%%%%%%

\latexProblemContent{
\ifVerboseLocation This is Derivative Compute Question 0003. \\ \fi
\begin{problem}

Compute the following derivative:

\input{Derivative-Compute-0003.HELP.tex}

\[\dfrac{d}{dx}\left(5 \, \cos\left(-5 \, \pi x\right) \sin\left(-\pi x\right)\right)=\answer{-5 \, \pi \cos\left(5 \, \pi x\right) \cos\left(\pi x\right) + 25 \, \pi \sin\left(5 \, \pi x\right) \sin\left(\pi x\right)}\]

\end{problem}}

%%%%%%%%%%%%%%%%%%%%%%

\latexProblemContent{
\ifVerboseLocation This is Derivative Compute Question 0003. \\ \fi
\begin{problem}

Compute the following derivative:

\input{Derivative-Compute-0003.HELP.tex}

\[\dfrac{d}{dx}\left(5 \, \cos\left(\frac{1}{6} \, \pi x\right) \tan\left(-3 \, \pi x\right)\right)=\answer{-15 \, \pi \cos\left(\frac{1}{6} \, \pi x\right) \sec\left(3 \, \pi x\right)^{2} + \frac{5}{6} \, \pi \sin\left(\frac{1}{6} \, \pi x\right) \tan\left(3 \, \pi x\right)}\]

\end{problem}}

%%%%%%%%%%%%%%%%%%%%%%

\latexProblemContent{
\ifVerboseLocation This is Derivative Compute Question 0003. \\ \fi
\begin{problem}

Compute the following derivative:

\input{Derivative-Compute-0003.HELP.tex}

\[\dfrac{d}{dx}\left(8 \, \cos\left(-\pi x\right) \tan\left(\frac{5}{3} \, \pi x\right)\right)=\answer{\frac{40}{3} \, \pi \cos\left(\pi x\right) \sec\left(\frac{5}{3} \, \pi x\right)^{2} - 8 \, \pi \sin\left(\pi x\right) \tan\left(\frac{5}{3} \, \pi x\right)}\]

\end{problem}}

%%%%%%%%%%%%%%%%%%%%%%

\latexProblemContent{
\ifVerboseLocation This is Derivative Compute Question 0003. \\ \fi
\begin{problem}

Compute the following derivative:

\input{Derivative-Compute-0003.HELP.tex}

\[\dfrac{d}{dx}\left(-4 \, \cos\left(\frac{2}{3} \, \pi x\right) \tan\left(\frac{5}{6} \, \pi x\right)\right)=\answer{-\frac{10}{3} \, \pi \cos\left(\frac{2}{3} \, \pi x\right) \sec\left(\frac{5}{6} \, \pi x\right)^{2} + \frac{8}{3} \, \pi \sin\left(\frac{2}{3} \, \pi x\right) \tan\left(\frac{5}{6} \, \pi x\right)}\]

\end{problem}}

%%%%%%%%%%%%%%%%%%%%%%

\latexProblemContent{
\ifVerboseLocation This is Derivative Compute Question 0003. \\ \fi
\begin{problem}

Compute the following derivative:

\input{Derivative-Compute-0003.HELP.tex}

\[\dfrac{d}{dx}\left(4 \, \tan\left(-\frac{4}{3} \, \pi x\right) \tan\left(-2 \, \pi x\right)\right)=\answer{\frac{16}{3} \, \pi \sec\left(\frac{4}{3} \, \pi x\right)^{2} \tan\left(2 \, \pi x\right) + 8 \, \pi \sec\left(2 \, \pi x\right)^{2} \tan\left(\frac{4}{3} \, \pi x\right)}\]

\end{problem}}

%%%%%%%%%%%%%%%%%%%%%%

\latexProblemContent{
\ifVerboseLocation This is Derivative Compute Question 0003. \\ \fi
\begin{problem}

Compute the following derivative:

\input{Derivative-Compute-0003.HELP.tex}

\[\dfrac{d}{dx}\left(-5 \, \cos\left(-\frac{1}{2} \, \pi x\right)^{2}\right)=\answer{5 \, \pi \cos\left(\frac{1}{2} \, \pi x\right) \sin\left(\frac{1}{2} \, \pi x\right)}\]

\end{problem}}

%%%%%%%%%%%%%%%%%%%%%%

\latexProblemContent{
\ifVerboseLocation This is Derivative Compute Question 0003. \\ \fi
\begin{problem}

Compute the following derivative:

\input{Derivative-Compute-0003.HELP.tex}

\[\dfrac{d}{dx}\left(4 \, \cos\left(-\frac{3}{2} \, \pi x\right) \tan\left(-\pi x\right)\right)=\answer{-4 \, \pi \cos\left(\frac{3}{2} \, \pi x\right) \sec\left(\pi x\right)^{2} + 6 \, \pi \sin\left(\frac{3}{2} \, \pi x\right) \tan\left(\pi x\right)}\]

\end{problem}}

%%%%%%%%%%%%%%%%%%%%%%

\latexProblemContent{
\ifVerboseLocation This is Derivative Compute Question 0003. \\ \fi
\begin{problem}

Compute the following derivative:

\input{Derivative-Compute-0003.HELP.tex}

\[\dfrac{d}{dx}\left(10 \, \sin\left(\frac{1}{3} \, \pi x\right) \sin\left(-\frac{1}{6} \, \pi x\right)\right)=\answer{-\frac{5}{3} \, \pi \cos\left(\frac{1}{6} \, \pi x\right) \sin\left(\frac{1}{3} \, \pi x\right) - \frac{10}{3} \, \pi \cos\left(\frac{1}{3} \, \pi x\right) \sin\left(\frac{1}{6} \, \pi x\right)}\]

\end{problem}}

%%%%%%%%%%%%%%%%%%%%%%

\latexProblemContent{
\ifVerboseLocation This is Derivative Compute Question 0003. \\ \fi
\begin{problem}

Compute the following derivative:

\input{Derivative-Compute-0003.HELP.tex}

\[\dfrac{d}{dx}\left(\cos\left(\pi x\right) \sin\left(\frac{1}{3} \, \pi x\right)\right)=\answer{\frac{1}{3} \, \pi \cos\left(\pi x\right) \cos\left(\frac{1}{3} \, \pi x\right) - \pi \sin\left(\pi x\right) \sin\left(\frac{1}{3} \, \pi x\right)}\]

\end{problem}}

%%%%%%%%%%%%%%%%%%%%%%

\latexProblemContent{
\ifVerboseLocation This is Derivative Compute Question 0003. \\ \fi
\begin{problem}

Compute the following derivative:

\input{Derivative-Compute-0003.HELP.tex}

\[\dfrac{d}{dx}\left(-10 \, \cos\left(-\pi x\right) \sin\left(-\frac{1}{6} \, \pi x\right)\right)=\answer{\frac{5}{3} \, \pi \cos\left(\pi x\right) \cos\left(\frac{1}{6} \, \pi x\right) - 10 \, \pi \sin\left(\pi x\right) \sin\left(\frac{1}{6} \, \pi x\right)}\]

\end{problem}}

%%%%%%%%%%%%%%%%%%%%%%

\latexProblemContent{
\ifVerboseLocation This is Derivative Compute Question 0003. \\ \fi
\begin{problem}

Compute the following derivative:

\input{Derivative-Compute-0003.HELP.tex}

\[\dfrac{d}{dx}\left(2 \, \tan\left(\pi x\right) \tan\left(-\frac{1}{2} \, \pi x\right)\right)=\answer{-\pi \sec\left(\frac{1}{2} \, \pi x\right)^{2} \tan\left(\pi x\right) - 2 \, \pi \sec\left(\pi x\right)^{2} \tan\left(\frac{1}{2} \, \pi x\right)}\]

\end{problem}}

%%%%%%%%%%%%%%%%%%%%%%

\latexProblemContent{
\ifVerboseLocation This is Derivative Compute Question 0003. \\ \fi
\begin{problem}

Compute the following derivative:

\input{Derivative-Compute-0003.HELP.tex}

\[\dfrac{d}{dx}\left(6 \, \sin\left(2 \, \pi x\right) \tan\left(4 \, \pi x\right)\right)=\answer{24 \, \pi \sec\left(4 \, \pi x\right)^{2} \sin\left(2 \, \pi x\right) + 12 \, \pi \cos\left(2 \, \pi x\right) \tan\left(4 \, \pi x\right)}\]

\end{problem}}

%%%%%%%%%%%%%%%%%%%%%%

\latexProblemContent{
\ifVerboseLocation This is Derivative Compute Question 0003. \\ \fi
\begin{problem}

Compute the following derivative:

\input{Derivative-Compute-0003.HELP.tex}

\[\dfrac{d}{dx}\left(4 \, \cos\left(-\frac{1}{3} \, \pi x\right) \sin\left(-\pi x\right)\right)=\answer{-4 \, \pi \cos\left(\pi x\right) \cos\left(\frac{1}{3} \, \pi x\right) + \frac{4}{3} \, \pi \sin\left(\pi x\right) \sin\left(\frac{1}{3} \, \pi x\right)}\]

\end{problem}}

%%%%%%%%%%%%%%%%%%%%%%

\latexProblemContent{
\ifVerboseLocation This is Derivative Compute Question 0003. \\ \fi
\begin{problem}

Compute the following derivative:

\input{Derivative-Compute-0003.HELP.tex}

\[\dfrac{d}{dx}\left(-12 \, \cos\left(\frac{3}{2} \, \pi x\right) \tan\left(-\frac{1}{2} \, \pi x\right)\right)=\answer{6 \, \pi \cos\left(\frac{3}{2} \, \pi x\right) \sec\left(\frac{1}{2} \, \pi x\right)^{2} - 18 \, \pi \sin\left(\frac{3}{2} \, \pi x\right) \tan\left(\frac{1}{2} \, \pi x\right)}\]

\end{problem}}

%%%%%%%%%%%%%%%%%%%%%%

\latexProblemContent{
\ifVerboseLocation This is Derivative Compute Question 0003. \\ \fi
\begin{problem}

Compute the following derivative:

\input{Derivative-Compute-0003.HELP.tex}

\[\dfrac{d}{dx}\left(-20 \, \sin\left(\pi x\right) \sin\left(-\frac{2}{3} \, \pi x\right)\right)=\answer{\frac{40}{3} \, \pi \cos\left(\frac{2}{3} \, \pi x\right) \sin\left(\pi x\right) + 20 \, \pi \cos\left(\pi x\right) \sin\left(\frac{2}{3} \, \pi x\right)}\]

\end{problem}}

%%%%%%%%%%%%%%%%%%%%%%

\latexProblemContent{
\ifVerboseLocation This is Derivative Compute Question 0003. \\ \fi
\begin{problem}

Compute the following derivative:

\input{Derivative-Compute-0003.HELP.tex}

\[\dfrac{d}{dx}\left(-6 \, \cos\left(\frac{1}{2} \, \pi x\right) \sin\left(\frac{1}{6} \, \pi x\right)\right)=\answer{-\pi \cos\left(\frac{1}{2} \, \pi x\right) \cos\left(\frac{1}{6} \, \pi x\right) + 3 \, \pi \sin\left(\frac{1}{2} \, \pi x\right) \sin\left(\frac{1}{6} \, \pi x\right)}\]

\end{problem}}

%%%%%%%%%%%%%%%%%%%%%%

\latexProblemContent{
\ifVerboseLocation This is Derivative Compute Question 0003. \\ \fi
\begin{problem}

Compute the following derivative:

\input{Derivative-Compute-0003.HELP.tex}

\[\dfrac{d}{dx}\left(-6 \, \cos\left(\frac{3}{2} \, \pi x\right) \cos\left(-\frac{2}{3} \, \pi x\right)\right)=\answer{9 \, \pi \cos\left(\frac{2}{3} \, \pi x\right) \sin\left(\frac{3}{2} \, \pi x\right) + 4 \, \pi \cos\left(\frac{3}{2} \, \pi x\right) \sin\left(\frac{2}{3} \, \pi x\right)}\]

\end{problem}}

%%%%%%%%%%%%%%%%%%%%%%

\latexProblemContent{
\ifVerboseLocation This is Derivative Compute Question 0003. \\ \fi
\begin{problem}

Compute the following derivative:

\input{Derivative-Compute-0003.HELP.tex}

\[\dfrac{d}{dx}\left(-3 \, \sin\left(\pi x\right) \tan\left(\frac{2}{3} \, \pi x\right)\right)=\answer{-2 \, \pi \sec\left(\frac{2}{3} \, \pi x\right)^{2} \sin\left(\pi x\right) - 3 \, \pi \cos\left(\pi x\right) \tan\left(\frac{2}{3} \, \pi x\right)}\]

\end{problem}}

%%%%%%%%%%%%%%%%%%%%%%

\latexProblemContent{
\ifVerboseLocation This is Derivative Compute Question 0003. \\ \fi
\begin{problem}

Compute the following derivative:

\input{Derivative-Compute-0003.HELP.tex}

\[\dfrac{d}{dx}\left(-3 \, \sin\left(\pi x\right) \sin\left(-\frac{1}{2} \, \pi x\right)\right)=\answer{\frac{3}{2} \, \pi \cos\left(\frac{1}{2} \, \pi x\right) \sin\left(\pi x\right) + 3 \, \pi \cos\left(\pi x\right) \sin\left(\frac{1}{2} \, \pi x\right)}\]

\end{problem}}

%%%%%%%%%%%%%%%%%%%%%%

\latexProblemContent{
\ifVerboseLocation This is Derivative Compute Question 0003. \\ \fi
\begin{problem}

Compute the following derivative:

\input{Derivative-Compute-0003.HELP.tex}

\[\dfrac{d}{dx}\left(-5 \, \cos\left(-\frac{4}{3} \, \pi x\right) \sin\left(\frac{2}{3} \, \pi x\right)\right)=\answer{-\frac{10}{3} \, \pi \cos\left(\frac{4}{3} \, \pi x\right) \cos\left(\frac{2}{3} \, \pi x\right) + \frac{20}{3} \, \pi \sin\left(\frac{4}{3} \, \pi x\right) \sin\left(\frac{2}{3} \, \pi x\right)}\]

\end{problem}}

%%%%%%%%%%%%%%%%%%%%%%

\latexProblemContent{
\ifVerboseLocation This is Derivative Compute Question 0003. \\ \fi
\begin{problem}

Compute the following derivative:

\input{Derivative-Compute-0003.HELP.tex}

\[\dfrac{d}{dx}\left(-16 \, \cos\left(-2 \, \pi x\right) \tan\left(\frac{5}{6} \, \pi x\right)\right)=\answer{-\frac{40}{3} \, \pi \cos\left(2 \, \pi x\right) \sec\left(\frac{5}{6} \, \pi x\right)^{2} + 32 \, \pi \sin\left(2 \, \pi x\right) \tan\left(\frac{5}{6} \, \pi x\right)}\]

\end{problem}}

%%%%%%%%%%%%%%%%%%%%%%

\latexProblemContent{
\ifVerboseLocation This is Derivative Compute Question 0003. \\ \fi
\begin{problem}

Compute the following derivative:

\input{Derivative-Compute-0003.HELP.tex}

\[\dfrac{d}{dx}\left(-4 \, \cos\left(\frac{5}{3} \, \pi x\right) \tan\left(4 \, \pi x\right)\right)=\answer{-16 \, \pi \cos\left(\frac{5}{3} \, \pi x\right) \sec\left(4 \, \pi x\right)^{2} + \frac{20}{3} \, \pi \sin\left(\frac{5}{3} \, \pi x\right) \tan\left(4 \, \pi x\right)}\]

\end{problem}}

%%%%%%%%%%%%%%%%%%%%%%

\latexProblemContent{
\ifVerboseLocation This is Derivative Compute Question 0003. \\ \fi
\begin{problem}

Compute the following derivative:

\input{Derivative-Compute-0003.HELP.tex}

\[\dfrac{d}{dx}\left(-3 \, \cos\left(\frac{3}{2} \, \pi x\right) \tan\left(-\frac{2}{3} \, \pi x\right)\right)=\answer{2 \, \pi \cos\left(\frac{3}{2} \, \pi x\right) \sec\left(\frac{2}{3} \, \pi x\right)^{2} - \frac{9}{2} \, \pi \sin\left(\frac{3}{2} \, \pi x\right) \tan\left(\frac{2}{3} \, \pi x\right)}\]

\end{problem}}

%%%%%%%%%%%%%%%%%%%%%%

\latexProblemContent{
\ifVerboseLocation This is Derivative Compute Question 0003. \\ \fi
\begin{problem}

Compute the following derivative:

\input{Derivative-Compute-0003.HELP.tex}

\[\dfrac{d}{dx}\left(-25 \, \cos\left(-\frac{3}{2} \, \pi x\right) \tan\left(\frac{1}{3} \, \pi x\right)\right)=\answer{-\frac{25}{3} \, \pi \cos\left(\frac{3}{2} \, \pi x\right) \sec\left(\frac{1}{3} \, \pi x\right)^{2} + \frac{75}{2} \, \pi \sin\left(\frac{3}{2} \, \pi x\right) \tan\left(\frac{1}{3} \, \pi x\right)}\]

\end{problem}}

%%%%%%%%%%%%%%%%%%%%%%

\latexProblemContent{
\ifVerboseLocation This is Derivative Compute Question 0003. \\ \fi
\begin{problem}

Compute the following derivative:

\input{Derivative-Compute-0003.HELP.tex}

\[\dfrac{d}{dx}\left(\sin\left(\frac{2}{3} \, \pi x\right)^{2}\right)=\answer{\frac{4}{3} \, \pi \cos\left(\frac{2}{3} \, \pi x\right) \sin\left(\frac{2}{3} \, \pi x\right)}\]

\end{problem}}

%%%%%%%%%%%%%%%%%%%%%%

\latexProblemContent{
\ifVerboseLocation This is Derivative Compute Question 0003. \\ \fi
\begin{problem}

Compute the following derivative:

\input{Derivative-Compute-0003.HELP.tex}

\[\dfrac{d}{dx}\left(-4 \, \cos\left(-\pi x\right) \tan\left(\pi x\right)\right)=\answer{-4 \, \pi \cos\left(\pi x\right) \sec\left(\pi x\right)^{2} + 4 \, \pi \sin\left(\pi x\right) \tan\left(\pi x\right)}\]

\end{problem}}

%%%%%%%%%%%%%%%%%%%%%%

\latexProblemContent{
\ifVerboseLocation This is Derivative Compute Question 0003. \\ \fi
\begin{problem}

Compute the following derivative:

\input{Derivative-Compute-0003.HELP.tex}

\[\dfrac{d}{dx}\left(-6 \, \tan\left(\frac{4}{3} \, \pi x\right) \tan\left(-2 \, \pi x\right)\right)=\answer{8 \, \pi \sec\left(\frac{4}{3} \, \pi x\right)^{2} \tan\left(2 \, \pi x\right) + 12 \, \pi \sec\left(2 \, \pi x\right)^{2} \tan\left(\frac{4}{3} \, \pi x\right)}\]

\end{problem}}

%%%%%%%%%%%%%%%%%%%%%%

\latexProblemContent{
\ifVerboseLocation This is Derivative Compute Question 0003. \\ \fi
\begin{problem}

Compute the following derivative:

\input{Derivative-Compute-0003.HELP.tex}

\[\dfrac{d}{dx}\left(-15 \, \sin\left(\frac{4}{3} \, \pi x\right) \tan\left(\frac{1}{2} \, \pi x\right)\right)=\answer{-\frac{15}{2} \, \pi \sec\left(\frac{1}{2} \, \pi x\right)^{2} \sin\left(\frac{4}{3} \, \pi x\right) - 20 \, \pi \cos\left(\frac{4}{3} \, \pi x\right) \tan\left(\frac{1}{2} \, \pi x\right)}\]

\end{problem}}

%%%%%%%%%%%%%%%%%%%%%%

\latexProblemContent{
\ifVerboseLocation This is Derivative Compute Question 0003. \\ \fi
\begin{problem}

Compute the following derivative:

\input{Derivative-Compute-0003.HELP.tex}

\[\dfrac{d}{dx}\left(8 \, \cos\left(\frac{2}{3} \, \pi x\right)^{2}\right)=\answer{-\frac{32}{3} \, \pi \cos\left(\frac{2}{3} \, \pi x\right) \sin\left(\frac{2}{3} \, \pi x\right)}\]

\end{problem}}

%%%%%%%%%%%%%%%%%%%%%%

\latexProblemContent{
\ifVerboseLocation This is Derivative Compute Question 0003. \\ \fi
\begin{problem}

Compute the following derivative:

\input{Derivative-Compute-0003.HELP.tex}

\[\dfrac{d}{dx}\left(-5 \, \sin\left(-\frac{2}{3} \, \pi x\right) \sin\left(-4 \, \pi x\right)\right)=\answer{-\frac{10}{3} \, \pi \cos\left(\frac{2}{3} \, \pi x\right) \sin\left(4 \, \pi x\right) - 20 \, \pi \cos\left(4 \, \pi x\right) \sin\left(\frac{2}{3} \, \pi x\right)}\]

\end{problem}}

%%%%%%%%%%%%%%%%%%%%%%

\latexProblemContent{
\ifVerboseLocation This is Derivative Compute Question 0003. \\ \fi
\begin{problem}

Compute the following derivative:

\input{Derivative-Compute-0003.HELP.tex}

\[\dfrac{d}{dx}\left(-10 \, \sin\left(\frac{1}{2} \, \pi x\right) \tan\left(-3 \, \pi x\right)\right)=\answer{30 \, \pi \sec\left(3 \, \pi x\right)^{2} \sin\left(\frac{1}{2} \, \pi x\right) + 5 \, \pi \cos\left(\frac{1}{2} \, \pi x\right) \tan\left(3 \, \pi x\right)}\]

\end{problem}}

%%%%%%%%%%%%%%%%%%%%%%

\latexProblemContent{
\ifVerboseLocation This is Derivative Compute Question 0003. \\ \fi
\begin{problem}

Compute the following derivative:

\input{Derivative-Compute-0003.HELP.tex}

\[\dfrac{d}{dx}\left(20 \, \cos\left(\frac{5}{6} \, \pi x\right) \tan\left(\frac{3}{2} \, \pi x\right)\right)=\answer{30 \, \pi \cos\left(\frac{5}{6} \, \pi x\right) \sec\left(\frac{3}{2} \, \pi x\right)^{2} - \frac{50}{3} \, \pi \sin\left(\frac{5}{6} \, \pi x\right) \tan\left(\frac{3}{2} \, \pi x\right)}\]

\end{problem}}

%%%%%%%%%%%%%%%%%%%%%%

\latexProblemContent{
\ifVerboseLocation This is Derivative Compute Question 0003. \\ \fi
\begin{problem}

Compute the following derivative:

\input{Derivative-Compute-0003.HELP.tex}

\[\dfrac{d}{dx}\left(-3 \, \cos\left(5 \, \pi x\right) \cos\left(-4 \, \pi x\right)\right)=\answer{15 \, \pi \cos\left(4 \, \pi x\right) \sin\left(5 \, \pi x\right) + 12 \, \pi \cos\left(5 \, \pi x\right) \sin\left(4 \, \pi x\right)}\]

\end{problem}}

%%%%%%%%%%%%%%%%%%%%%%

\latexProblemContent{
\ifVerboseLocation This is Derivative Compute Question 0003. \\ \fi
\begin{problem}

Compute the following derivative:

\input{Derivative-Compute-0003.HELP.tex}

\[\dfrac{d}{dx}\left(-4 \, \cos\left(-\frac{5}{2} \, \pi x\right) \tan\left(-4 \, \pi x\right)\right)=\answer{16 \, \pi \cos\left(\frac{5}{2} \, \pi x\right) \sec\left(4 \, \pi x\right)^{2} - 10 \, \pi \sin\left(\frac{5}{2} \, \pi x\right) \tan\left(4 \, \pi x\right)}\]

\end{problem}}

%%%%%%%%%%%%%%%%%%%%%%

\latexProblemContent{
\ifVerboseLocation This is Derivative Compute Question 0003. \\ \fi
\begin{problem}

Compute the following derivative:

\input{Derivative-Compute-0003.HELP.tex}

\[\dfrac{d}{dx}\left(12 \, \cos\left(\frac{5}{3} \, \pi x\right) \sin\left(-\frac{2}{3} \, \pi x\right)\right)=\answer{-8 \, \pi \cos\left(\frac{5}{3} \, \pi x\right) \cos\left(\frac{2}{3} \, \pi x\right) + 20 \, \pi \sin\left(\frac{5}{3} \, \pi x\right) \sin\left(\frac{2}{3} \, \pi x\right)}\]

\end{problem}}

%%%%%%%%%%%%%%%%%%%%%%

\latexProblemContent{
\ifVerboseLocation This is Derivative Compute Question 0003. \\ \fi
\begin{problem}

Compute the following derivative:

\input{Derivative-Compute-0003.HELP.tex}

\[\dfrac{d}{dx}\left(-20 \, \tan\left(-\frac{1}{6} \, \pi x\right) \tan\left(-\frac{2}{3} \, \pi x\right)\right)=\answer{-\frac{10}{3} \, \pi \sec\left(\frac{1}{6} \, \pi x\right)^{2} \tan\left(\frac{2}{3} \, \pi x\right) - \frac{40}{3} \, \pi \sec\left(\frac{2}{3} \, \pi x\right)^{2} \tan\left(\frac{1}{6} \, \pi x\right)}\]

\end{problem}}

%%%%%%%%%%%%%%%%%%%%%%

\latexProblemContent{
\ifVerboseLocation This is Derivative Compute Question 0003. \\ \fi
\begin{problem}

Compute the following derivative:

\input{Derivative-Compute-0003.HELP.tex}

\[\dfrac{d}{dx}\left(-\sin\left(\frac{1}{2} \, \pi x\right) \tan\left(\frac{1}{6} \, \pi x\right)\right)=\answer{-\frac{1}{6} \, \pi \sec\left(\frac{1}{6} \, \pi x\right)^{2} \sin\left(\frac{1}{2} \, \pi x\right) - \frac{1}{2} \, \pi \cos\left(\frac{1}{2} \, \pi x\right) \tan\left(\frac{1}{6} \, \pi x\right)}\]

\end{problem}}

%%%%%%%%%%%%%%%%%%%%%%

\latexProblemContent{
\ifVerboseLocation This is Derivative Compute Question 0003. \\ \fi
\begin{problem}

Compute the following derivative:

\input{Derivative-Compute-0003.HELP.tex}

\[\dfrac{d}{dx}\left(-4 \, \cos\left(-\frac{5}{3} \, \pi x\right) \tan\left(-\frac{5}{6} \, \pi x\right)\right)=\answer{\frac{10}{3} \, \pi \cos\left(\frac{5}{3} \, \pi x\right) \sec\left(\frac{5}{6} \, \pi x\right)^{2} - \frac{20}{3} \, \pi \sin\left(\frac{5}{3} \, \pi x\right) \tan\left(\frac{5}{6} \, \pi x\right)}\]

\end{problem}}

%%%%%%%%%%%%%%%%%%%%%%

\latexProblemContent{
\ifVerboseLocation This is Derivative Compute Question 0003. \\ \fi
\begin{problem}

Compute the following derivative:

\input{Derivative-Compute-0003.HELP.tex}

\[\dfrac{d}{dx}\left(-2 \, \sin\left(-\frac{1}{2} \, \pi x\right) \tan\left(3 \, \pi x\right)\right)=\answer{6 \, \pi \sec\left(3 \, \pi x\right)^{2} \sin\left(\frac{1}{2} \, \pi x\right) + \pi \cos\left(\frac{1}{2} \, \pi x\right) \tan\left(3 \, \pi x\right)}\]

\end{problem}}

%%%%%%%%%%%%%%%%%%%%%%

\latexProblemContent{
\ifVerboseLocation This is Derivative Compute Question 0003. \\ \fi
\begin{problem}

Compute the following derivative:

\input{Derivative-Compute-0003.HELP.tex}

\[\dfrac{d}{dx}\left(-4 \, \sin\left(\frac{5}{6} \, \pi x\right)^{2}\right)=\answer{-\frac{20}{3} \, \pi \cos\left(\frac{5}{6} \, \pi x\right) \sin\left(\frac{5}{6} \, \pi x\right)}\]

\end{problem}}

%%%%%%%%%%%%%%%%%%%%%%

\latexProblemContent{
\ifVerboseLocation This is Derivative Compute Question 0003. \\ \fi
\begin{problem}

Compute the following derivative:

\input{Derivative-Compute-0003.HELP.tex}

\[\dfrac{d}{dx}\left(10 \, \cos\left(\frac{5}{3} \, \pi x\right) \tan\left(2 \, \pi x\right)\right)=\answer{20 \, \pi \cos\left(\frac{5}{3} \, \pi x\right) \sec\left(2 \, \pi x\right)^{2} - \frac{50}{3} \, \pi \sin\left(\frac{5}{3} \, \pi x\right) \tan\left(2 \, \pi x\right)}\]

\end{problem}}

%%%%%%%%%%%%%%%%%%%%%%

\latexProblemContent{
\ifVerboseLocation This is Derivative Compute Question 0003. \\ \fi
\begin{problem}

Compute the following derivative:

\input{Derivative-Compute-0003.HELP.tex}

\[\dfrac{d}{dx}\left(12 \, \sin\left(-5 \, \pi x\right) \tan\left(-3 \, \pi x\right)\right)=\answer{36 \, \pi \sec\left(3 \, \pi x\right)^{2} \sin\left(5 \, \pi x\right) + 60 \, \pi \cos\left(5 \, \pi x\right) \tan\left(3 \, \pi x\right)}\]

\end{problem}}

%%%%%%%%%%%%%%%%%%%%%%

\latexProblemContent{
\ifVerboseLocation This is Derivative Compute Question 0003. \\ \fi
\begin{problem}

Compute the following derivative:

\input{Derivative-Compute-0003.HELP.tex}

\[\dfrac{d}{dx}\left(10 \, \sin\left(\frac{1}{3} \, \pi x\right) \tan\left(5 \, \pi x\right)\right)=\answer{50 \, \pi \sec\left(5 \, \pi x\right)^{2} \sin\left(\frac{1}{3} \, \pi x\right) + \frac{10}{3} \, \pi \cos\left(\frac{1}{3} \, \pi x\right) \tan\left(5 \, \pi x\right)}\]

\end{problem}}

%%%%%%%%%%%%%%%%%%%%%%

\latexProblemContent{
\ifVerboseLocation This is Derivative Compute Question 0003. \\ \fi
\begin{problem}

Compute the following derivative:

\input{Derivative-Compute-0003.HELP.tex}

\[\dfrac{d}{dx}\left(-25 \, \tan\left(-\frac{5}{6} \, \pi x\right) \tan\left(-\pi x\right)\right)=\answer{-\frac{125}{6} \, \pi \sec\left(\frac{5}{6} \, \pi x\right)^{2} \tan\left(\pi x\right) - 25 \, \pi \sec\left(\pi x\right)^{2} \tan\left(\frac{5}{6} \, \pi x\right)}\]

\end{problem}}

%%%%%%%%%%%%%%%%%%%%%%

\latexProblemContent{
\ifVerboseLocation This is Derivative Compute Question 0003. \\ \fi
\begin{problem}

Compute the following derivative:

\input{Derivative-Compute-0003.HELP.tex}

\[\dfrac{d}{dx}\left(8 \, \sin\left(-\pi x\right) \sin\left(-\frac{5}{2} \, \pi x\right)\right)=\answer{8 \, \pi \cos\left(\pi x\right) \sin\left(\frac{5}{2} \, \pi x\right) + 20 \, \pi \cos\left(\frac{5}{2} \, \pi x\right) \sin\left(\pi x\right)}\]

\end{problem}}

%%%%%%%%%%%%%%%%%%%%%%

\latexProblemContent{
\ifVerboseLocation This is Derivative Compute Question 0003. \\ \fi
\begin{problem}

Compute the following derivative:

\input{Derivative-Compute-0003.HELP.tex}

\[\dfrac{d}{dx}\left(25 \, \cos\left(\pi x\right) \cos\left(-\frac{1}{3} \, \pi x\right)\right)=\answer{-25 \, \pi \cos\left(\frac{1}{3} \, \pi x\right) \sin\left(\pi x\right) - \frac{25}{3} \, \pi \cos\left(\pi x\right) \sin\left(\frac{1}{3} \, \pi x\right)}\]

\end{problem}}

%%%%%%%%%%%%%%%%%%%%%%

\latexProblemContent{
\ifVerboseLocation This is Derivative Compute Question 0003. \\ \fi
\begin{problem}

Compute the following derivative:

\input{Derivative-Compute-0003.HELP.tex}

\[\dfrac{d}{dx}\left(-6 \, \cos\left(-\frac{1}{2} \, \pi x\right) \sin\left(-\frac{1}{2} \, \pi x\right)\right)=\answer{3 \, \pi \cos\left(\frac{1}{2} \, \pi x\right)^{2} - 3 \, \pi \sin\left(\frac{1}{2} \, \pi x\right)^{2}}\]

\end{problem}}

%%%%%%%%%%%%%%%%%%%%%%

\latexProblemContent{
\ifVerboseLocation This is Derivative Compute Question 0003. \\ \fi
\begin{problem}

Compute the following derivative:

\input{Derivative-Compute-0003.HELP.tex}

\[\dfrac{d}{dx}\left(10 \, \sin\left(-3 \, \pi x\right) \tan\left(\frac{1}{3} \, \pi x\right)\right)=\answer{-\frac{10}{3} \, \pi \sec\left(\frac{1}{3} \, \pi x\right)^{2} \sin\left(3 \, \pi x\right) - 30 \, \pi \cos\left(3 \, \pi x\right) \tan\left(\frac{1}{3} \, \pi x\right)}\]

\end{problem}}

%%%%%%%%%%%%%%%%%%%%%%

\latexProblemContent{
\ifVerboseLocation This is Derivative Compute Question 0003. \\ \fi
\begin{problem}

Compute the following derivative:

\input{Derivative-Compute-0003.HELP.tex}

\[\dfrac{d}{dx}\left(10 \, \sin\left(3 \, \pi x\right) \sin\left(\frac{1}{2} \, \pi x\right)\right)=\answer{5 \, \pi \cos\left(\frac{1}{2} \, \pi x\right) \sin\left(3 \, \pi x\right) + 30 \, \pi \cos\left(3 \, \pi x\right) \sin\left(\frac{1}{2} \, \pi x\right)}\]

\end{problem}}

%%%%%%%%%%%%%%%%%%%%%%

\latexProblemContent{
\ifVerboseLocation This is Derivative Compute Question 0003. \\ \fi
\begin{problem}

Compute the following derivative:

\input{Derivative-Compute-0003.HELP.tex}

\[\dfrac{d}{dx}\left(-9 \, \sin\left(\frac{4}{3} \, \pi x\right) \sin\left(\frac{1}{3} \, \pi x\right)\right)=\answer{-3 \, \pi \cos\left(\frac{1}{3} \, \pi x\right) \sin\left(\frac{4}{3} \, \pi x\right) - 12 \, \pi \cos\left(\frac{4}{3} \, \pi x\right) \sin\left(\frac{1}{3} \, \pi x\right)}\]

\end{problem}}

%%%%%%%%%%%%%%%%%%%%%%

\latexProblemContent{
\ifVerboseLocation This is Derivative Compute Question 0003. \\ \fi
\begin{problem}

Compute the following derivative:

\input{Derivative-Compute-0003.HELP.tex}

\[\dfrac{d}{dx}\left(-\cos\left(-\frac{2}{3} \, \pi x\right) \sin\left(\frac{2}{3} \, \pi x\right)\right)=\answer{-\frac{2}{3} \, \pi \cos\left(\frac{2}{3} \, \pi x\right)^{2} + \frac{2}{3} \, \pi \sin\left(\frac{2}{3} \, \pi x\right)^{2}}\]

\end{problem}}

%%%%%%%%%%%%%%%%%%%%%%

\latexProblemContent{
\ifVerboseLocation This is Derivative Compute Question 0003. \\ \fi
\begin{problem}

Compute the following derivative:

\input{Derivative-Compute-0003.HELP.tex}

\[\dfrac{d}{dx}\left(-12 \, \cos\left(-\pi x\right) \tan\left(-2 \, \pi x\right)\right)=\answer{24 \, \pi \cos\left(\pi x\right) \sec\left(2 \, \pi x\right)^{2} - 12 \, \pi \sin\left(\pi x\right) \tan\left(2 \, \pi x\right)}\]

\end{problem}}

%%%%%%%%%%%%%%%%%%%%%%

\latexProblemContent{
\ifVerboseLocation This is Derivative Compute Question 0003. \\ \fi
\begin{problem}

Compute the following derivative:

\input{Derivative-Compute-0003.HELP.tex}

\[\dfrac{d}{dx}\left(\sin\left(-\frac{2}{3} \, \pi x\right) \tan\left(5 \, \pi x\right)\right)=\answer{-5 \, \pi \sec\left(5 \, \pi x\right)^{2} \sin\left(\frac{2}{3} \, \pi x\right) - \frac{2}{3} \, \pi \cos\left(\frac{2}{3} \, \pi x\right) \tan\left(5 \, \pi x\right)}\]

\end{problem}}

%%%%%%%%%%%%%%%%%%%%%%

\latexProblemContent{
\ifVerboseLocation This is Derivative Compute Question 0003. \\ \fi
\begin{problem}

Compute the following derivative:

\input{Derivative-Compute-0003.HELP.tex}

\[\dfrac{d}{dx}\left(-4 \, \cos\left(\frac{2}{3} \, \pi x\right) \sin\left(\frac{5}{3} \, \pi x\right)\right)=\answer{-\frac{20}{3} \, \pi \cos\left(\frac{5}{3} \, \pi x\right) \cos\left(\frac{2}{3} \, \pi x\right) + \frac{8}{3} \, \pi \sin\left(\frac{5}{3} \, \pi x\right) \sin\left(\frac{2}{3} \, \pi x\right)}\]

\end{problem}}

%%%%%%%%%%%%%%%%%%%%%%

\latexProblemContent{
\ifVerboseLocation This is Derivative Compute Question 0003. \\ \fi
\begin{problem}

Compute the following derivative:

\input{Derivative-Compute-0003.HELP.tex}

\[\dfrac{d}{dx}\left(-10 \, \cos\left(\pi x\right) \tan\left(\frac{5}{2} \, \pi x\right)\right)=\answer{-25 \, \pi \cos\left(\pi x\right) \sec\left(\frac{5}{2} \, \pi x\right)^{2} + 10 \, \pi \sin\left(\pi x\right) \tan\left(\frac{5}{2} \, \pi x\right)}\]

\end{problem}}

%%%%%%%%%%%%%%%%%%%%%%

\latexProblemContent{
\ifVerboseLocation This is Derivative Compute Question 0003. \\ \fi
\begin{problem}

Compute the following derivative:

\input{Derivative-Compute-0003.HELP.tex}

\[\dfrac{d}{dx}\left(12 \, \sin\left(-\frac{5}{2} \, \pi x\right) \tan\left(\frac{1}{6} \, \pi x\right)\right)=\answer{-2 \, \pi \sec\left(\frac{1}{6} \, \pi x\right)^{2} \sin\left(\frac{5}{2} \, \pi x\right) - 30 \, \pi \cos\left(\frac{5}{2} \, \pi x\right) \tan\left(\frac{1}{6} \, \pi x\right)}\]

\end{problem}}

%%%%%%%%%%%%%%%%%%%%%%

\latexProblemContent{
\ifVerboseLocation This is Derivative Compute Question 0003. \\ \fi
\begin{problem}

Compute the following derivative:

\input{Derivative-Compute-0003.HELP.tex}

\[\dfrac{d}{dx}\left(8 \, \tan\left(3 \, \pi x\right) \tan\left(-\frac{5}{3} \, \pi x\right)\right)=\answer{-\frac{40}{3} \, \pi \sec\left(\frac{5}{3} \, \pi x\right)^{2} \tan\left(3 \, \pi x\right) - 24 \, \pi \sec\left(3 \, \pi x\right)^{2} \tan\left(\frac{5}{3} \, \pi x\right)}\]

\end{problem}}

%%%%%%%%%%%%%%%%%%%%%%

\latexProblemContent{
\ifVerboseLocation This is Derivative Compute Question 0003. \\ \fi
\begin{problem}

Compute the following derivative:

\input{Derivative-Compute-0003.HELP.tex}

\[\dfrac{d}{dx}\left(20 \, \cos\left(\frac{5}{2} \, \pi x\right) \sin\left(\pi x\right)\right)=\answer{20 \, \pi \cos\left(\frac{5}{2} \, \pi x\right) \cos\left(\pi x\right) - 50 \, \pi \sin\left(\frac{5}{2} \, \pi x\right) \sin\left(\pi x\right)}\]

\end{problem}}

%%%%%%%%%%%%%%%%%%%%%%

\latexProblemContent{
\ifVerboseLocation This is Derivative Compute Question 0003. \\ \fi
\begin{problem}

Compute the following derivative:

\input{Derivative-Compute-0003.HELP.tex}

\[\dfrac{d}{dx}\left(-16 \, \tan\left(\pi x\right) \tan\left(\frac{5}{6} \, \pi x\right)\right)=\answer{-\frac{40}{3} \, \pi \sec\left(\frac{5}{6} \, \pi x\right)^{2} \tan\left(\pi x\right) - 16 \, \pi \sec\left(\pi x\right)^{2} \tan\left(\frac{5}{6} \, \pi x\right)}\]

\end{problem}}

%%%%%%%%%%%%%%%%%%%%%%

\latexProblemContent{
\ifVerboseLocation This is Derivative Compute Question 0003. \\ \fi
\begin{problem}

Compute the following derivative:

\input{Derivative-Compute-0003.HELP.tex}

\[\dfrac{d}{dx}\left(-12 \, \cos\left(\frac{1}{6} \, \pi x\right) \cos\left(-2 \, \pi x\right)\right)=\answer{24 \, \pi \cos\left(\frac{1}{6} \, \pi x\right) \sin\left(2 \, \pi x\right) + 2 \, \pi \cos\left(2 \, \pi x\right) \sin\left(\frac{1}{6} \, \pi x\right)}\]

\end{problem}}

%%%%%%%%%%%%%%%%%%%%%%

\latexProblemContent{
\ifVerboseLocation This is Derivative Compute Question 0003. \\ \fi
\begin{problem}

Compute the following derivative:

\input{Derivative-Compute-0003.HELP.tex}

\[\dfrac{d}{dx}\left(4 \, \sin\left(\frac{4}{3} \, \pi x\right) \sin\left(\frac{2}{3} \, \pi x\right)\right)=\answer{\frac{8}{3} \, \pi \cos\left(\frac{2}{3} \, \pi x\right) \sin\left(\frac{4}{3} \, \pi x\right) + \frac{16}{3} \, \pi \cos\left(\frac{4}{3} \, \pi x\right) \sin\left(\frac{2}{3} \, \pi x\right)}\]

\end{problem}}

%%%%%%%%%%%%%%%%%%%%%%

\latexProblemContent{
\ifVerboseLocation This is Derivative Compute Question 0003. \\ \fi
\begin{problem}

Compute the following derivative:

\input{Derivative-Compute-0003.HELP.tex}

\[\dfrac{d}{dx}\left(20 \, \cos\left(-2 \, \pi x\right) \sin\left(-\frac{1}{2} \, \pi x\right)\right)=\answer{-10 \, \pi \cos\left(2 \, \pi x\right) \cos\left(\frac{1}{2} \, \pi x\right) + 40 \, \pi \sin\left(2 \, \pi x\right) \sin\left(\frac{1}{2} \, \pi x\right)}\]

\end{problem}}

%%%%%%%%%%%%%%%%%%%%%%

\latexProblemContent{
\ifVerboseLocation This is Derivative Compute Question 0003. \\ \fi
\begin{problem}

Compute the following derivative:

\input{Derivative-Compute-0003.HELP.tex}

\[\dfrac{d}{dx}\left(3 \, \cos\left(5 \, \pi x\right) \tan\left(5 \, \pi x\right)\right)=\answer{15 \, \pi \cos\left(5 \, \pi x\right) \sec\left(5 \, \pi x\right)^{2} - 15 \, \pi \sin\left(5 \, \pi x\right) \tan\left(5 \, \pi x\right)}\]

\end{problem}}

%%%%%%%%%%%%%%%%%%%%%%

\latexProblemContent{
\ifVerboseLocation This is Derivative Compute Question 0003. \\ \fi
\begin{problem}

Compute the following derivative:

\input{Derivative-Compute-0003.HELP.tex}

\[\dfrac{d}{dx}\left(-10 \, \sin\left(\frac{2}{3} \, \pi x\right) \tan\left(\pi x\right)\right)=\answer{-10 \, \pi \sec\left(\pi x\right)^{2} \sin\left(\frac{2}{3} \, \pi x\right) - \frac{20}{3} \, \pi \cos\left(\frac{2}{3} \, \pi x\right) \tan\left(\pi x\right)}\]

\end{problem}}

%%%%%%%%%%%%%%%%%%%%%%

\latexProblemContent{
\ifVerboseLocation This is Derivative Compute Question 0003. \\ \fi
\begin{problem}

Compute the following derivative:

\input{Derivative-Compute-0003.HELP.tex}

\[\dfrac{d}{dx}\left(15 \, \cos\left(\frac{2}{3} \, \pi x\right) \sin\left(-\pi x\right)\right)=\answer{-15 \, \pi \cos\left(\pi x\right) \cos\left(\frac{2}{3} \, \pi x\right) + 10 \, \pi \sin\left(\pi x\right) \sin\left(\frac{2}{3} \, \pi x\right)}\]

\end{problem}}

%%%%%%%%%%%%%%%%%%%%%%

\latexProblemContent{
\ifVerboseLocation This is Derivative Compute Question 0003. \\ \fi
\begin{problem}

Compute the following derivative:

\input{Derivative-Compute-0003.HELP.tex}

\[\dfrac{d}{dx}\left(-25 \, \sin\left(\frac{4}{3} \, \pi x\right) \sin\left(-\frac{1}{2} \, \pi x\right)\right)=\answer{\frac{25}{2} \, \pi \cos\left(\frac{1}{2} \, \pi x\right) \sin\left(\frac{4}{3} \, \pi x\right) + \frac{100}{3} \, \pi \cos\left(\frac{4}{3} \, \pi x\right) \sin\left(\frac{1}{2} \, \pi x\right)}\]

\end{problem}}

%%%%%%%%%%%%%%%%%%%%%%

\latexProblemContent{
\ifVerboseLocation This is Derivative Compute Question 0003. \\ \fi
\begin{problem}

Compute the following derivative:

\input{Derivative-Compute-0003.HELP.tex}

\[\dfrac{d}{dx}\left(-20 \, \cos\left(-\frac{5}{3} \, \pi x\right) \sin\left(\frac{1}{3} \, \pi x\right)\right)=\answer{-\frac{20}{3} \, \pi \cos\left(\frac{5}{3} \, \pi x\right) \cos\left(\frac{1}{3} \, \pi x\right) + \frac{100}{3} \, \pi \sin\left(\frac{5}{3} \, \pi x\right) \sin\left(\frac{1}{3} \, \pi x\right)}\]

\end{problem}}

%%%%%%%%%%%%%%%%%%%%%%

\latexProblemContent{
\ifVerboseLocation This is Derivative Compute Question 0003. \\ \fi
\begin{problem}

Compute the following derivative:

\input{Derivative-Compute-0003.HELP.tex}

\[\dfrac{d}{dx}\left(15 \, \cos\left(-\pi x\right) \sin\left(-\frac{1}{3} \, \pi x\right)\right)=\answer{-5 \, \pi \cos\left(\pi x\right) \cos\left(\frac{1}{3} \, \pi x\right) + 15 \, \pi \sin\left(\pi x\right) \sin\left(\frac{1}{3} \, \pi x\right)}\]

\end{problem}}

%%%%%%%%%%%%%%%%%%%%%%

\latexProblemContent{
\ifVerboseLocation This is Derivative Compute Question 0003. \\ \fi
\begin{problem}

Compute the following derivative:

\input{Derivative-Compute-0003.HELP.tex}

\[\dfrac{d}{dx}\left(-2 \, \cos\left(-\frac{1}{3} \, \pi x\right) \tan\left(-\frac{1}{6} \, \pi x\right)\right)=\answer{\frac{1}{3} \, \pi \cos\left(\frac{1}{3} \, \pi x\right) \sec\left(\frac{1}{6} \, \pi x\right)^{2} - \frac{2}{3} \, \pi \sin\left(\frac{1}{3} \, \pi x\right) \tan\left(\frac{1}{6} \, \pi x\right)}\]

\end{problem}}

%%%%%%%%%%%%%%%%%%%%%%

\latexProblemContent{
\ifVerboseLocation This is Derivative Compute Question 0003. \\ \fi
\begin{problem}

Compute the following derivative:

\input{Derivative-Compute-0003.HELP.tex}

\[\dfrac{d}{dx}\left(-10 \, \sin\left(-\pi x\right) \tan\left(\frac{5}{2} \, \pi x\right)\right)=\answer{25 \, \pi \sec\left(\frac{5}{2} \, \pi x\right)^{2} \sin\left(\pi x\right) + 10 \, \pi \cos\left(\pi x\right) \tan\left(\frac{5}{2} \, \pi x\right)}\]

\end{problem}}

%%%%%%%%%%%%%%%%%%%%%%

\latexProblemContent{
\ifVerboseLocation This is Derivative Compute Question 0003. \\ \fi
\begin{problem}

Compute the following derivative:

\input{Derivative-Compute-0003.HELP.tex}

\[\dfrac{d}{dx}\left(25 \, \cos\left(\frac{1}{2} \, \pi x\right)^{2}\right)=\answer{-25 \, \pi \cos\left(\frac{1}{2} \, \pi x\right) \sin\left(\frac{1}{2} \, \pi x\right)}\]

\end{problem}}

%%%%%%%%%%%%%%%%%%%%%%

\latexProblemContent{
\ifVerboseLocation This is Derivative Compute Question 0003. \\ \fi
\begin{problem}

Compute the following derivative:

\input{Derivative-Compute-0003.HELP.tex}

\[\dfrac{d}{dx}\left(-25 \, \sin\left(2 \, \pi x\right) \tan\left(-5 \, \pi x\right)\right)=\answer{125 \, \pi \sec\left(5 \, \pi x\right)^{2} \sin\left(2 \, \pi x\right) + 50 \, \pi \cos\left(2 \, \pi x\right) \tan\left(5 \, \pi x\right)}\]

\end{problem}}

%%%%%%%%%%%%%%%%%%%%%%

\latexProblemContent{
\ifVerboseLocation This is Derivative Compute Question 0003. \\ \fi
\begin{problem}

Compute the following derivative:

\input{Derivative-Compute-0003.HELP.tex}

\[\dfrac{d}{dx}\left(-6 \, \cos\left(-\frac{2}{3} \, \pi x\right) \sin\left(-\pi x\right)\right)=\answer{6 \, \pi \cos\left(\pi x\right) \cos\left(\frac{2}{3} \, \pi x\right) - 4 \, \pi \sin\left(\pi x\right) \sin\left(\frac{2}{3} \, \pi x\right)}\]

\end{problem}}

%%%%%%%%%%%%%%%%%%%%%%

\latexProblemContent{
\ifVerboseLocation This is Derivative Compute Question 0003. \\ \fi
\begin{problem}

Compute the following derivative:

\input{Derivative-Compute-0003.HELP.tex}

\[\dfrac{d}{dx}\left(6 \, \cos\left(\frac{1}{2} \, \pi x\right) \cos\left(-\frac{5}{6} \, \pi x\right)\right)=\answer{-5 \, \pi \cos\left(\frac{1}{2} \, \pi x\right) \sin\left(\frac{5}{6} \, \pi x\right) - 3 \, \pi \cos\left(\frac{5}{6} \, \pi x\right) \sin\left(\frac{1}{2} \, \pi x\right)}\]

\end{problem}}

%%%%%%%%%%%%%%%%%%%%%%

\latexProblemContent{
\ifVerboseLocation This is Derivative Compute Question 0003. \\ \fi
\begin{problem}

Compute the following derivative:

\input{Derivative-Compute-0003.HELP.tex}

\[\dfrac{d}{dx}\left(9 \, \cos\left(-\frac{1}{3} \, \pi x\right) \sin\left(\pi x\right)\right)=\answer{9 \, \pi \cos\left(\pi x\right) \cos\left(\frac{1}{3} \, \pi x\right) - 3 \, \pi \sin\left(\pi x\right) \sin\left(\frac{1}{3} \, \pi x\right)}\]

\end{problem}}

%%%%%%%%%%%%%%%%%%%%%%

\latexProblemContent{
\ifVerboseLocation This is Derivative Compute Question 0003. \\ \fi
\begin{problem}

Compute the following derivative:

\input{Derivative-Compute-0003.HELP.tex}

\[\dfrac{d}{dx}\left(12 \, \cos\left(\frac{1}{2} \, \pi x\right) \cos\left(-\frac{1}{2} \, \pi x\right)\right)=\answer{-12 \, \pi \cos\left(\frac{1}{2} \, \pi x\right) \sin\left(\frac{1}{2} \, \pi x\right)}\]

\end{problem}}

%%%%%%%%%%%%%%%%%%%%%%

\latexProblemContent{
\ifVerboseLocation This is Derivative Compute Question 0003. \\ \fi
\begin{problem}

Compute the following derivative:

\input{Derivative-Compute-0003.HELP.tex}

\[\dfrac{d}{dx}\left(8 \, \sin\left(\pi x\right) \tan\left(\frac{5}{3} \, \pi x\right)\right)=\answer{\frac{40}{3} \, \pi \sec\left(\frac{5}{3} \, \pi x\right)^{2} \sin\left(\pi x\right) + 8 \, \pi \cos\left(\pi x\right) \tan\left(\frac{5}{3} \, \pi x\right)}\]

\end{problem}}

%%%%%%%%%%%%%%%%%%%%%%

\latexProblemContent{
\ifVerboseLocation This is Derivative Compute Question 0003. \\ \fi
\begin{problem}

Compute the following derivative:

\input{Derivative-Compute-0003.HELP.tex}

\[\dfrac{d}{dx}\left(-12 \, \sin\left(-\frac{2}{3} \, \pi x\right) \tan\left(-\pi x\right)\right)=\answer{-12 \, \pi \sec\left(\pi x\right)^{2} \sin\left(\frac{2}{3} \, \pi x\right) - 8 \, \pi \cos\left(\frac{2}{3} \, \pi x\right) \tan\left(\pi x\right)}\]

\end{problem}}

%%%%%%%%%%%%%%%%%%%%%%

\latexProblemContent{
\ifVerboseLocation This is Derivative Compute Question 0003. \\ \fi
\begin{problem}

Compute the following derivative:

\input{Derivative-Compute-0003.HELP.tex}

\[\dfrac{d}{dx}\left(-25 \, \sin\left(\frac{1}{3} \, \pi x\right) \tan\left(-\frac{1}{2} \, \pi x\right)\right)=\answer{\frac{25}{2} \, \pi \sec\left(\frac{1}{2} \, \pi x\right)^{2} \sin\left(\frac{1}{3} \, \pi x\right) + \frac{25}{3} \, \pi \cos\left(\frac{1}{3} \, \pi x\right) \tan\left(\frac{1}{2} \, \pi x\right)}\]

\end{problem}}

%%%%%%%%%%%%%%%%%%%%%%

\latexProblemContent{
\ifVerboseLocation This is Derivative Compute Question 0003. \\ \fi
\begin{problem}

Compute the following derivative:

\input{Derivative-Compute-0003.HELP.tex}

\[\dfrac{d}{dx}\left(-\sin\left(-\frac{5}{6} \, \pi x\right) \tan\left(5 \, \pi x\right)\right)=\answer{5 \, \pi \sec\left(5 \, \pi x\right)^{2} \sin\left(\frac{5}{6} \, \pi x\right) + \frac{5}{6} \, \pi \cos\left(\frac{5}{6} \, \pi x\right) \tan\left(5 \, \pi x\right)}\]

\end{problem}}

%%%%%%%%%%%%%%%%%%%%%%

\latexProblemContent{
\ifVerboseLocation This is Derivative Compute Question 0003. \\ \fi
\begin{problem}

Compute the following derivative:

\input{Derivative-Compute-0003.HELP.tex}

\[\dfrac{d}{dx}\left(-8 \, \cos\left(-\pi x\right) \cos\left(-3 \, \pi x\right)\right)=\answer{24 \, \pi \cos\left(\pi x\right) \sin\left(3 \, \pi x\right) + 8 \, \pi \cos\left(3 \, \pi x\right) \sin\left(\pi x\right)}\]

\end{problem}}

%%%%%%%%%%%%%%%%%%%%%%

\latexProblemContent{
\ifVerboseLocation This is Derivative Compute Question 0003. \\ \fi
\begin{problem}

Compute the following derivative:

\input{Derivative-Compute-0003.HELP.tex}

\[\dfrac{d}{dx}\left(-2 \, \sin\left(\frac{3}{2} \, \pi x\right) \sin\left(-\frac{5}{6} \, \pi x\right)\right)=\answer{\frac{5}{3} \, \pi \cos\left(\frac{5}{6} \, \pi x\right) \sin\left(\frac{3}{2} \, \pi x\right) + 3 \, \pi \cos\left(\frac{3}{2} \, \pi x\right) \sin\left(\frac{5}{6} \, \pi x\right)}\]

\end{problem}}

%%%%%%%%%%%%%%%%%%%%%%

\latexProblemContent{
\ifVerboseLocation This is Derivative Compute Question 0003. \\ \fi
\begin{problem}

Compute the following derivative:

\input{Derivative-Compute-0003.HELP.tex}

\[\dfrac{d}{dx}\left(\cos\left(\frac{1}{3} \, \pi x\right) \cos\left(-\frac{5}{2} \, \pi x\right)\right)=\answer{-\frac{5}{2} \, \pi \cos\left(\frac{1}{3} \, \pi x\right) \sin\left(\frac{5}{2} \, \pi x\right) - \frac{1}{3} \, \pi \cos\left(\frac{5}{2} \, \pi x\right) \sin\left(\frac{1}{3} \, \pi x\right)}\]

\end{problem}}

%%%%%%%%%%%%%%%%%%%%%%

\latexProblemContent{
\ifVerboseLocation This is Derivative Compute Question 0003. \\ \fi
\begin{problem}

Compute the following derivative:

\input{Derivative-Compute-0003.HELP.tex}

\[\dfrac{d}{dx}\left(5 \, \cos\left(5 \, \pi x\right) \sin\left(-\frac{1}{2} \, \pi x\right)\right)=\answer{-\frac{5}{2} \, \pi \cos\left(5 \, \pi x\right) \cos\left(\frac{1}{2} \, \pi x\right) + 25 \, \pi \sin\left(5 \, \pi x\right) \sin\left(\frac{1}{2} \, \pi x\right)}\]

\end{problem}}

%%%%%%%%%%%%%%%%%%%%%%

\latexProblemContent{
\ifVerboseLocation This is Derivative Compute Question 0003. \\ \fi
\begin{problem}

Compute the following derivative:

\input{Derivative-Compute-0003.HELP.tex}

\[\dfrac{d}{dx}\left(3 \, \sin\left(2 \, \pi x\right) \tan\left(2 \, \pi x\right)\right)=\answer{6 \, \pi \sec\left(2 \, \pi x\right)^{2} \sin\left(2 \, \pi x\right) + 6 \, \pi \cos\left(2 \, \pi x\right) \tan\left(2 \, \pi x\right)}\]

\end{problem}}

%%%%%%%%%%%%%%%%%%%%%%

\latexProblemContent{
\ifVerboseLocation This is Derivative Compute Question 0003. \\ \fi
\begin{problem}

Compute the following derivative:

\input{Derivative-Compute-0003.HELP.tex}

\[\dfrac{d}{dx}\left(-16 \, \cos\left(\frac{2}{3} \, \pi x\right) \cos\left(-\frac{2}{3} \, \pi x\right)\right)=\answer{\frac{64}{3} \, \pi \cos\left(\frac{2}{3} \, \pi x\right) \sin\left(\frac{2}{3} \, \pi x\right)}\]

\end{problem}}

%%%%%%%%%%%%%%%%%%%%%%

\latexProblemContent{
\ifVerboseLocation This is Derivative Compute Question 0003. \\ \fi
\begin{problem}

Compute the following derivative:

\input{Derivative-Compute-0003.HELP.tex}

\[\dfrac{d}{dx}\left(4 \, \cos\left(\frac{5}{2} \, \pi x\right) \tan\left(\frac{2}{3} \, \pi x\right)\right)=\answer{\frac{8}{3} \, \pi \cos\left(\frac{5}{2} \, \pi x\right) \sec\left(\frac{2}{3} \, \pi x\right)^{2} - 10 \, \pi \sin\left(\frac{5}{2} \, \pi x\right) \tan\left(\frac{2}{3} \, \pi x\right)}\]

\end{problem}}

%%%%%%%%%%%%%%%%%%%%%%

\latexProblemContent{
\ifVerboseLocation This is Derivative Compute Question 0003. \\ \fi
\begin{problem}

Compute the following derivative:

\input{Derivative-Compute-0003.HELP.tex}

\[\dfrac{d}{dx}\left(-\sin\left(5 \, \pi x\right) \sin\left(-\frac{2}{3} \, \pi x\right)\right)=\answer{\frac{2}{3} \, \pi \cos\left(\frac{2}{3} \, \pi x\right) \sin\left(5 \, \pi x\right) + 5 \, \pi \cos\left(5 \, \pi x\right) \sin\left(\frac{2}{3} \, \pi x\right)}\]

\end{problem}}

%%%%%%%%%%%%%%%%%%%%%%

\latexProblemContent{
\ifVerboseLocation This is Derivative Compute Question 0003. \\ \fi
\begin{problem}

Compute the following derivative:

\input{Derivative-Compute-0003.HELP.tex}

\[\dfrac{d}{dx}\left(-2 \, \sin\left(\frac{1}{6} \, \pi x\right) \tan\left(\frac{2}{3} \, \pi x\right)\right)=\answer{-\frac{4}{3} \, \pi \sec\left(\frac{2}{3} \, \pi x\right)^{2} \sin\left(\frac{1}{6} \, \pi x\right) - \frac{1}{3} \, \pi \cos\left(\frac{1}{6} \, \pi x\right) \tan\left(\frac{2}{3} \, \pi x\right)}\]

\end{problem}}

%%%%%%%%%%%%%%%%%%%%%%

\latexProblemContent{
\ifVerboseLocation This is Derivative Compute Question 0003. \\ \fi
\begin{problem}

Compute the following derivative:

\input{Derivative-Compute-0003.HELP.tex}

\[\dfrac{d}{dx}\left(-9 \, \cos\left(\pi x\right) \sin\left(-\frac{5}{6} \, \pi x\right)\right)=\answer{\frac{15}{2} \, \pi \cos\left(\pi x\right) \cos\left(\frac{5}{6} \, \pi x\right) - 9 \, \pi \sin\left(\pi x\right) \sin\left(\frac{5}{6} \, \pi x\right)}\]

\end{problem}}

%%%%%%%%%%%%%%%%%%%%%%

\latexProblemContent{
\ifVerboseLocation This is Derivative Compute Question 0003. \\ \fi
\begin{problem}

Compute the following derivative:

\input{Derivative-Compute-0003.HELP.tex}

\[\dfrac{d}{dx}\left(-4 \, \sin\left(4 \, \pi x\right) \tan\left(\frac{5}{3} \, \pi x\right)\right)=\answer{-\frac{20}{3} \, \pi \sec\left(\frac{5}{3} \, \pi x\right)^{2} \sin\left(4 \, \pi x\right) - 16 \, \pi \cos\left(4 \, \pi x\right) \tan\left(\frac{5}{3} \, \pi x\right)}\]

\end{problem}}

%%%%%%%%%%%%%%%%%%%%%%

\latexProblemContent{
\ifVerboseLocation This is Derivative Compute Question 0003. \\ \fi
\begin{problem}

Compute the following derivative:

\input{Derivative-Compute-0003.HELP.tex}

\[\dfrac{d}{dx}\left(15 \, \cos\left(2 \, \pi x\right) \cos\left(\pi x\right)\right)=\answer{-30 \, \pi \cos\left(\pi x\right) \sin\left(2 \, \pi x\right) - 15 \, \pi \cos\left(2 \, \pi x\right) \sin\left(\pi x\right)}\]

\end{problem}}

%%%%%%%%%%%%%%%%%%%%%%

\latexProblemContent{
\ifVerboseLocation This is Derivative Compute Question 0003. \\ \fi
\begin{problem}

Compute the following derivative:

\input{Derivative-Compute-0003.HELP.tex}

\[\dfrac{d}{dx}\left(15 \, \cos\left(\frac{1}{3} \, \pi x\right) \tan\left(2 \, \pi x\right)\right)=\answer{30 \, \pi \cos\left(\frac{1}{3} \, \pi x\right) \sec\left(2 \, \pi x\right)^{2} - 5 \, \pi \sin\left(\frac{1}{3} \, \pi x\right) \tan\left(2 \, \pi x\right)}\]

\end{problem}}

%%%%%%%%%%%%%%%%%%%%%%

\latexProblemContent{
\ifVerboseLocation This is Derivative Compute Question 0003. \\ \fi
\begin{problem}

Compute the following derivative:

\input{Derivative-Compute-0003.HELP.tex}

\[\dfrac{d}{dx}\left(-15 \, \sin\left(-3 \, \pi x\right) \tan\left(-\frac{1}{6} \, \pi x\right)\right)=\answer{-\frac{5}{2} \, \pi \sec\left(\frac{1}{6} \, \pi x\right)^{2} \sin\left(3 \, \pi x\right) - 45 \, \pi \cos\left(3 \, \pi x\right) \tan\left(\frac{1}{6} \, \pi x\right)}\]

\end{problem}}

%%%%%%%%%%%%%%%%%%%%%%

\latexProblemContent{
\ifVerboseLocation This is Derivative Compute Question 0003. \\ \fi
\begin{problem}

Compute the following derivative:

\input{Derivative-Compute-0003.HELP.tex}

\[\dfrac{d}{dx}\left(-12 \, \sin\left(3 \, \pi x\right) \tan\left(5 \, \pi x\right)\right)=\answer{-60 \, \pi \sec\left(5 \, \pi x\right)^{2} \sin\left(3 \, \pi x\right) - 36 \, \pi \cos\left(3 \, \pi x\right) \tan\left(5 \, \pi x\right)}\]

\end{problem}}

%%%%%%%%%%%%%%%%%%%%%%

\latexProblemContent{
\ifVerboseLocation This is Derivative Compute Question 0003. \\ \fi
\begin{problem}

Compute the following derivative:

\input{Derivative-Compute-0003.HELP.tex}

\[\dfrac{d}{dx}\left(12 \, \tan\left(\frac{1}{2} \, \pi x\right) \tan\left(-\frac{2}{3} \, \pi x\right)\right)=\answer{-6 \, \pi \sec\left(\frac{1}{2} \, \pi x\right)^{2} \tan\left(\frac{2}{3} \, \pi x\right) - 8 \, \pi \sec\left(\frac{2}{3} \, \pi x\right)^{2} \tan\left(\frac{1}{2} \, \pi x\right)}\]

\end{problem}}

%%%%%%%%%%%%%%%%%%%%%%

\latexProblemContent{
\ifVerboseLocation This is Derivative Compute Question 0003. \\ \fi
\begin{problem}

Compute the following derivative:

\input{Derivative-Compute-0003.HELP.tex}

\[\dfrac{d}{dx}\left(-2 \, \sin\left(5 \, \pi x\right) \sin\left(\frac{1}{3} \, \pi x\right)\right)=\answer{-\frac{2}{3} \, \pi \cos\left(\frac{1}{3} \, \pi x\right) \sin\left(5 \, \pi x\right) - 10 \, \pi \cos\left(5 \, \pi x\right) \sin\left(\frac{1}{3} \, \pi x\right)}\]

\end{problem}}

%%%%%%%%%%%%%%%%%%%%%%

\latexProblemContent{
\ifVerboseLocation This is Derivative Compute Question 0003. \\ \fi
\begin{problem}

Compute the following derivative:

\input{Derivative-Compute-0003.HELP.tex}

\[\dfrac{d}{dx}\left(-\sin\left(-\frac{5}{3} \, \pi x\right) \tan\left(\frac{5}{2} \, \pi x\right)\right)=\answer{\frac{5}{2} \, \pi \sec\left(\frac{5}{2} \, \pi x\right)^{2} \sin\left(\frac{5}{3} \, \pi x\right) + \frac{5}{3} \, \pi \cos\left(\frac{5}{3} \, \pi x\right) \tan\left(\frac{5}{2} \, \pi x\right)}\]

\end{problem}}

%%%%%%%%%%%%%%%%%%%%%%

\latexProblemContent{
\ifVerboseLocation This is Derivative Compute Question 0003. \\ \fi
\begin{problem}

Compute the following derivative:

\input{Derivative-Compute-0003.HELP.tex}

\[\dfrac{d}{dx}\left(-6 \, \cos\left(\frac{2}{3} \, \pi x\right) \tan\left(-\frac{1}{6} \, \pi x\right)\right)=\answer{\pi \cos\left(\frac{2}{3} \, \pi x\right) \sec\left(\frac{1}{6} \, \pi x\right)^{2} - 4 \, \pi \sin\left(\frac{2}{3} \, \pi x\right) \tan\left(\frac{1}{6} \, \pi x\right)}\]

\end{problem}}

%%%%%%%%%%%%%%%%%%%%%%

\latexProblemContent{
\ifVerboseLocation This is Derivative Compute Question 0003. \\ \fi
\begin{problem}

Compute the following derivative:

\input{Derivative-Compute-0003.HELP.tex}

\[\dfrac{d}{dx}\left(25 \, \tan\left(5 \, \pi x\right) \tan\left(-\frac{1}{3} \, \pi x\right)\right)=\answer{-\frac{25}{3} \, \pi \sec\left(\frac{1}{3} \, \pi x\right)^{2} \tan\left(5 \, \pi x\right) - 125 \, \pi \sec\left(5 \, \pi x\right)^{2} \tan\left(\frac{1}{3} \, \pi x\right)}\]

\end{problem}}

%%%%%%%%%%%%%%%%%%%%%%

\latexProblemContent{
\ifVerboseLocation This is Derivative Compute Question 0003. \\ \fi
\begin{problem}

Compute the following derivative:

\input{Derivative-Compute-0003.HELP.tex}

\[\dfrac{d}{dx}\left(8 \, \cos\left(\frac{1}{2} \, \pi x\right) \tan\left(-\frac{5}{3} \, \pi x\right)\right)=\answer{-\frac{40}{3} \, \pi \cos\left(\frac{1}{2} \, \pi x\right) \sec\left(\frac{5}{3} \, \pi x\right)^{2} + 4 \, \pi \sin\left(\frac{1}{2} \, \pi x\right) \tan\left(\frac{5}{3} \, \pi x\right)}\]

\end{problem}}

%%%%%%%%%%%%%%%%%%%%%%

\latexProblemContent{
\ifVerboseLocation This is Derivative Compute Question 0003. \\ \fi
\begin{problem}

Compute the following derivative:

\input{Derivative-Compute-0003.HELP.tex}

\[\dfrac{d}{dx}\left(-10 \, \tan\left(\frac{5}{3} \, \pi x\right) \tan\left(-2 \, \pi x\right)\right)=\answer{\frac{50}{3} \, \pi \sec\left(\frac{5}{3} \, \pi x\right)^{2} \tan\left(2 \, \pi x\right) + 20 \, \pi \sec\left(2 \, \pi x\right)^{2} \tan\left(\frac{5}{3} \, \pi x\right)}\]

\end{problem}}

%%%%%%%%%%%%%%%%%%%%%%

\latexProblemContent{
\ifVerboseLocation This is Derivative Compute Question 0003. \\ \fi
\begin{problem}

Compute the following derivative:

\input{Derivative-Compute-0003.HELP.tex}

\[\dfrac{d}{dx}\left(25 \, \cos\left(2 \, \pi x\right) \cos\left(-\pi x\right)\right)=\answer{-50 \, \pi \cos\left(\pi x\right) \sin\left(2 \, \pi x\right) - 25 \, \pi \cos\left(2 \, \pi x\right) \sin\left(\pi x\right)}\]

\end{problem}}

%%%%%%%%%%%%%%%%%%%%%%

\latexProblemContent{
\ifVerboseLocation This is Derivative Compute Question 0003. \\ \fi
\begin{problem}

Compute the following derivative:

\input{Derivative-Compute-0003.HELP.tex}

\[\dfrac{d}{dx}\left(9 \, \cos\left(-\frac{4}{3} \, \pi x\right) \sin\left(\pi x\right)\right)=\answer{9 \, \pi \cos\left(\frac{4}{3} \, \pi x\right) \cos\left(\pi x\right) - 12 \, \pi \sin\left(\frac{4}{3} \, \pi x\right) \sin\left(\pi x\right)}\]

\end{problem}}

%%%%%%%%%%%%%%%%%%%%%%

\latexProblemContent{
\ifVerboseLocation This is Derivative Compute Question 0003. \\ \fi
\begin{problem}

Compute the following derivative:

\input{Derivative-Compute-0003.HELP.tex}

\[\dfrac{d}{dx}\left(-6 \, \cos\left(\frac{4}{3} \, \pi x\right) \tan\left(5 \, \pi x\right)\right)=\answer{-30 \, \pi \cos\left(\frac{4}{3} \, \pi x\right) \sec\left(5 \, \pi x\right)^{2} + 8 \, \pi \sin\left(\frac{4}{3} \, \pi x\right) \tan\left(5 \, \pi x\right)}\]

\end{problem}}

%%%%%%%%%%%%%%%%%%%%%%

\latexProblemContent{
\ifVerboseLocation This is Derivative Compute Question 0003. \\ \fi
\begin{problem}

Compute the following derivative:

\input{Derivative-Compute-0003.HELP.tex}

\[\dfrac{d}{dx}\left(-25 \, \tan\left(\frac{5}{3} \, \pi x\right) \tan\left(-\frac{5}{3} \, \pi x\right)\right)=\answer{\frac{250}{3} \, \pi \sec\left(\frac{5}{3} \, \pi x\right)^{2} \tan\left(\frac{5}{3} \, \pi x\right)}\]

\end{problem}}

%%%%%%%%%%%%%%%%%%%%%%

\latexProblemContent{
\ifVerboseLocation This is Derivative Compute Question 0003. \\ \fi
\begin{problem}

Compute the following derivative:

\input{Derivative-Compute-0003.HELP.tex}

\[\dfrac{d}{dx}\left(20 \, \cos\left(-\pi x\right) \tan\left(-\frac{5}{6} \, \pi x\right)\right)=\answer{-\frac{50}{3} \, \pi \cos\left(\pi x\right) \sec\left(\frac{5}{6} \, \pi x\right)^{2} + 20 \, \pi \sin\left(\pi x\right) \tan\left(\frac{5}{6} \, \pi x\right)}\]

\end{problem}}

%%%%%%%%%%%%%%%%%%%%%%

\latexProblemContent{
\ifVerboseLocation This is Derivative Compute Question 0003. \\ \fi
\begin{problem}

Compute the following derivative:

\input{Derivative-Compute-0003.HELP.tex}

\[\dfrac{d}{dx}\left(3 \, \cos\left(\frac{5}{3} \, \pi x\right) \tan\left(-\frac{2}{3} \, \pi x\right)\right)=\answer{-2 \, \pi \cos\left(\frac{5}{3} \, \pi x\right) \sec\left(\frac{2}{3} \, \pi x\right)^{2} + 5 \, \pi \sin\left(\frac{5}{3} \, \pi x\right) \tan\left(\frac{2}{3} \, \pi x\right)}\]

\end{problem}}

%%%%%%%%%%%%%%%%%%%%%%

\latexProblemContent{
\ifVerboseLocation This is Derivative Compute Question 0003. \\ \fi
\begin{problem}

Compute the following derivative:

\input{Derivative-Compute-0003.HELP.tex}

\[\dfrac{d}{dx}\left(-4 \, \cos\left(-\pi x\right) \tan\left(-4 \, \pi x\right)\right)=\answer{16 \, \pi \cos\left(\pi x\right) \sec\left(4 \, \pi x\right)^{2} - 4 \, \pi \sin\left(\pi x\right) \tan\left(4 \, \pi x\right)}\]

\end{problem}}

%%%%%%%%%%%%%%%%%%%%%%

\latexProblemContent{
\ifVerboseLocation This is Derivative Compute Question 0003. \\ \fi
\begin{problem}

Compute the following derivative:

\input{Derivative-Compute-0003.HELP.tex}

\[\dfrac{d}{dx}\left(16 \, \cos\left(\frac{2}{3} \, \pi x\right) \tan\left(-\pi x\right)\right)=\answer{-16 \, \pi \cos\left(\frac{2}{3} \, \pi x\right) \sec\left(\pi x\right)^{2} + \frac{32}{3} \, \pi \sin\left(\frac{2}{3} \, \pi x\right) \tan\left(\pi x\right)}\]

\end{problem}}

%%%%%%%%%%%%%%%%%%%%%%

\latexProblemContent{
\ifVerboseLocation This is Derivative Compute Question 0003. \\ \fi
\begin{problem}

Compute the following derivative:

\input{Derivative-Compute-0003.HELP.tex}

\[\dfrac{d}{dx}\left(-5 \, \tan\left(\frac{5}{2} \, \pi x\right) \tan\left(-4 \, \pi x\right)\right)=\answer{\frac{25}{2} \, \pi \sec\left(\frac{5}{2} \, \pi x\right)^{2} \tan\left(4 \, \pi x\right) + 20 \, \pi \sec\left(4 \, \pi x\right)^{2} \tan\left(\frac{5}{2} \, \pi x\right)}\]

\end{problem}}

%%%%%%%%%%%%%%%%%%%%%%

\latexProblemContent{
\ifVerboseLocation This is Derivative Compute Question 0003. \\ \fi
\begin{problem}

Compute the following derivative:

\input{Derivative-Compute-0003.HELP.tex}

\[\dfrac{d}{dx}\left(2 \, \cos\left(-\frac{1}{2} \, \pi x\right) \sin\left(\frac{5}{6} \, \pi x\right)\right)=\answer{\frac{5}{3} \, \pi \cos\left(\frac{5}{6} \, \pi x\right) \cos\left(\frac{1}{2} \, \pi x\right) - \pi \sin\left(\frac{5}{6} \, \pi x\right) \sin\left(\frac{1}{2} \, \pi x\right)}\]

\end{problem}}

%%%%%%%%%%%%%%%%%%%%%%

\latexProblemContent{
\ifVerboseLocation This is Derivative Compute Question 0003. \\ \fi
\begin{problem}

Compute the following derivative:

\input{Derivative-Compute-0003.HELP.tex}

\[\dfrac{d}{dx}\left(-12 \, \cos\left(-\pi x\right) \tan\left(\pi x\right)\right)=\answer{-12 \, \pi \cos\left(\pi x\right) \sec\left(\pi x\right)^{2} + 12 \, \pi \sin\left(\pi x\right) \tan\left(\pi x\right)}\]

\end{problem}}

%%%%%%%%%%%%%%%%%%%%%%

\latexProblemContent{
\ifVerboseLocation This is Derivative Compute Question 0003. \\ \fi
\begin{problem}

Compute the following derivative:

\input{Derivative-Compute-0003.HELP.tex}

\[\dfrac{d}{dx}\left(-12 \, \sin\left(-\frac{5}{6} \, \pi x\right) \sin\left(-5 \, \pi x\right)\right)=\answer{-10 \, \pi \cos\left(\frac{5}{6} \, \pi x\right) \sin\left(5 \, \pi x\right) - 60 \, \pi \cos\left(5 \, \pi x\right) \sin\left(\frac{5}{6} \, \pi x\right)}\]

\end{problem}}

%%%%%%%%%%%%%%%%%%%%%%

\latexProblemContent{
\ifVerboseLocation This is Derivative Compute Question 0003. \\ \fi
\begin{problem}

Compute the following derivative:

\input{Derivative-Compute-0003.HELP.tex}

\[\dfrac{d}{dx}\left(5 \, \cos\left(-2 \, \pi x\right) \sin\left(-\frac{1}{2} \, \pi x\right)\right)=\answer{-\frac{5}{2} \, \pi \cos\left(2 \, \pi x\right) \cos\left(\frac{1}{2} \, \pi x\right) + 10 \, \pi \sin\left(2 \, \pi x\right) \sin\left(\frac{1}{2} \, \pi x\right)}\]

\end{problem}}

%%%%%%%%%%%%%%%%%%%%%%

\latexProblemContent{
\ifVerboseLocation This is Derivative Compute Question 0003. \\ \fi
\begin{problem}

Compute the following derivative:

\input{Derivative-Compute-0003.HELP.tex}

\[\dfrac{d}{dx}\left(12 \, \tan\left(\pi x\right) \tan\left(-\frac{5}{6} \, \pi x\right)\right)=\answer{-10 \, \pi \sec\left(\frac{5}{6} \, \pi x\right)^{2} \tan\left(\pi x\right) - 12 \, \pi \sec\left(\pi x\right)^{2} \tan\left(\frac{5}{6} \, \pi x\right)}\]

\end{problem}}

%%%%%%%%%%%%%%%%%%%%%%

\latexProblemContent{
\ifVerboseLocation This is Derivative Compute Question 0003. \\ \fi
\begin{problem}

Compute the following derivative:

\input{Derivative-Compute-0003.HELP.tex}

\[\dfrac{d}{dx}\left(-4 \, \cos\left(\frac{3}{2} \, \pi x\right) \sin\left(-\frac{1}{3} \, \pi x\right)\right)=\answer{\frac{4}{3} \, \pi \cos\left(\frac{3}{2} \, \pi x\right) \cos\left(\frac{1}{3} \, \pi x\right) - 6 \, \pi \sin\left(\frac{3}{2} \, \pi x\right) \sin\left(\frac{1}{3} \, \pi x\right)}\]

\end{problem}}

%%%%%%%%%%%%%%%%%%%%%%

\latexProblemContent{
\ifVerboseLocation This is Derivative Compute Question 0003. \\ \fi
\begin{problem}

Compute the following derivative:

\input{Derivative-Compute-0003.HELP.tex}

\[\dfrac{d}{dx}\left(-5 \, \tan\left(2 \, \pi x\right) \tan\left(-2 \, \pi x\right)\right)=\answer{20 \, \pi \sec\left(2 \, \pi x\right)^{2} \tan\left(2 \, \pi x\right)}\]

\end{problem}}

%%%%%%%%%%%%%%%%%%%%%%

\latexProblemContent{
\ifVerboseLocation This is Derivative Compute Question 0003. \\ \fi
\begin{problem}

Compute the following derivative:

\input{Derivative-Compute-0003.HELP.tex}

\[\dfrac{d}{dx}\left(2 \, \tan\left(-\frac{1}{6} \, \pi x\right) \tan\left(-\pi x\right)\right)=\answer{\frac{1}{3} \, \pi \sec\left(\frac{1}{6} \, \pi x\right)^{2} \tan\left(\pi x\right) + 2 \, \pi \sec\left(\pi x\right)^{2} \tan\left(\frac{1}{6} \, \pi x\right)}\]

\end{problem}}

%%%%%%%%%%%%%%%%%%%%%%

\latexProblemContent{
\ifVerboseLocation This is Derivative Compute Question 0003. \\ \fi
\begin{problem}

Compute the following derivative:

\input{Derivative-Compute-0003.HELP.tex}

\[\dfrac{d}{dx}\left(16 \, \sin\left(-\frac{5}{2} \, \pi x\right) \tan\left(\frac{1}{6} \, \pi x\right)\right)=\answer{-\frac{8}{3} \, \pi \sec\left(\frac{1}{6} \, \pi x\right)^{2} \sin\left(\frac{5}{2} \, \pi x\right) - 40 \, \pi \cos\left(\frac{5}{2} \, \pi x\right) \tan\left(\frac{1}{6} \, \pi x\right)}\]

\end{problem}}

%%%%%%%%%%%%%%%%%%%%%%

\latexProblemContent{
\ifVerboseLocation This is Derivative Compute Question 0003. \\ \fi
\begin{problem}

Compute the following derivative:

\input{Derivative-Compute-0003.HELP.tex}

\[\dfrac{d}{dx}\left(15 \, \cos\left(\frac{4}{3} \, \pi x\right) \cos\left(-\frac{3}{2} \, \pi x\right)\right)=\answer{-\frac{45}{2} \, \pi \cos\left(\frac{4}{3} \, \pi x\right) \sin\left(\frac{3}{2} \, \pi x\right) - 20 \, \pi \cos\left(\frac{3}{2} \, \pi x\right) \sin\left(\frac{4}{3} \, \pi x\right)}\]

\end{problem}}

%%%%%%%%%%%%%%%%%%%%%%

\latexProblemContent{
\ifVerboseLocation This is Derivative Compute Question 0003. \\ \fi
\begin{problem}

Compute the following derivative:

\input{Derivative-Compute-0003.HELP.tex}

\[\dfrac{d}{dx}\left(-\cos\left(\frac{1}{2} \, \pi x\right) \cos\left(-\frac{3}{2} \, \pi x\right)\right)=\answer{\frac{3}{2} \, \pi \cos\left(\frac{1}{2} \, \pi x\right) \sin\left(\frac{3}{2} \, \pi x\right) + \frac{1}{2} \, \pi \cos\left(\frac{3}{2} \, \pi x\right) \sin\left(\frac{1}{2} \, \pi x\right)}\]

\end{problem}}

%%%%%%%%%%%%%%%%%%%%%%

\latexProblemContent{
\ifVerboseLocation This is Derivative Compute Question 0003. \\ \fi
\begin{problem}

Compute the following derivative:

\input{Derivative-Compute-0003.HELP.tex}

\[\dfrac{d}{dx}\left(5 \, \cos\left(\frac{1}{6} \, \pi x\right) \tan\left(2 \, \pi x\right)\right)=\answer{10 \, \pi \cos\left(\frac{1}{6} \, \pi x\right) \sec\left(2 \, \pi x\right)^{2} - \frac{5}{6} \, \pi \sin\left(\frac{1}{6} \, \pi x\right) \tan\left(2 \, \pi x\right)}\]

\end{problem}}

%%%%%%%%%%%%%%%%%%%%%%

\latexProblemContent{
\ifVerboseLocation This is Derivative Compute Question 0003. \\ \fi
\begin{problem}

Compute the following derivative:

\input{Derivative-Compute-0003.HELP.tex}

\[\dfrac{d}{dx}\left(-2 \, \tan\left(-\frac{1}{3} \, \pi x\right) \tan\left(-\frac{5}{2} \, \pi x\right)\right)=\answer{-\frac{2}{3} \, \pi \sec\left(\frac{1}{3} \, \pi x\right)^{2} \tan\left(\frac{5}{2} \, \pi x\right) - 5 \, \pi \sec\left(\frac{5}{2} \, \pi x\right)^{2} \tan\left(\frac{1}{3} \, \pi x\right)}\]

\end{problem}}

%%%%%%%%%%%%%%%%%%%%%%

\latexProblemContent{
\ifVerboseLocation This is Derivative Compute Question 0003. \\ \fi
\begin{problem}

Compute the following derivative:

\input{Derivative-Compute-0003.HELP.tex}

\[\dfrac{d}{dx}\left(-3 \, \cos\left(\frac{5}{6} \, \pi x\right) \cos\left(-5 \, \pi x\right)\right)=\answer{15 \, \pi \cos\left(\frac{5}{6} \, \pi x\right) \sin\left(5 \, \pi x\right) + \frac{5}{2} \, \pi \cos\left(5 \, \pi x\right) \sin\left(\frac{5}{6} \, \pi x\right)}\]

\end{problem}}

%%%%%%%%%%%%%%%%%%%%%%

\latexProblemContent{
\ifVerboseLocation This is Derivative Compute Question 0003. \\ \fi
\begin{problem}

Compute the following derivative:

\input{Derivative-Compute-0003.HELP.tex}

\[\dfrac{d}{dx}\left(-25 \, \cos\left(\frac{3}{2} \, \pi x\right) \cos\left(-\frac{3}{2} \, \pi x\right)\right)=\answer{75 \, \pi \cos\left(\frac{3}{2} \, \pi x\right) \sin\left(\frac{3}{2} \, \pi x\right)}\]

\end{problem}}

%%%%%%%%%%%%%%%%%%%%%%

\latexProblemContent{
\ifVerboseLocation This is Derivative Compute Question 0003. \\ \fi
\begin{problem}

Compute the following derivative:

\input{Derivative-Compute-0003.HELP.tex}

\[\dfrac{d}{dx}\left(-9 \, \sin\left(2 \, \pi x\right) \sin\left(-\frac{1}{3} \, \pi x\right)\right)=\answer{3 \, \pi \cos\left(\frac{1}{3} \, \pi x\right) \sin\left(2 \, \pi x\right) + 18 \, \pi \cos\left(2 \, \pi x\right) \sin\left(\frac{1}{3} \, \pi x\right)}\]

\end{problem}}

%%%%%%%%%%%%%%%%%%%%%%

\latexProblemContent{
\ifVerboseLocation This is Derivative Compute Question 0003. \\ \fi
\begin{problem}

Compute the following derivative:

\input{Derivative-Compute-0003.HELP.tex}

\[\dfrac{d}{dx}\left(-20 \, \sin\left(-\pi x\right) \sin\left(-\frac{3}{2} \, \pi x\right)\right)=\answer{-20 \, \pi \cos\left(\pi x\right) \sin\left(\frac{3}{2} \, \pi x\right) - 30 \, \pi \cos\left(\frac{3}{2} \, \pi x\right) \sin\left(\pi x\right)}\]

\end{problem}}

%%%%%%%%%%%%%%%%%%%%%%

\latexProblemContent{
\ifVerboseLocation This is Derivative Compute Question 0003. \\ \fi
\begin{problem}

Compute the following derivative:

\input{Derivative-Compute-0003.HELP.tex}

\[\dfrac{d}{dx}\left(\tan\left(\frac{1}{2} \, \pi x\right) \tan\left(\frac{1}{6} \, \pi x\right)\right)=\answer{\frac{1}{6} \, \pi \sec\left(\frac{1}{6} \, \pi x\right)^{2} \tan\left(\frac{1}{2} \, \pi x\right) + \frac{1}{2} \, \pi \sec\left(\frac{1}{2} \, \pi x\right)^{2} \tan\left(\frac{1}{6} \, \pi x\right)}\]

\end{problem}}

%%%%%%%%%%%%%%%%%%%%%%

\latexProblemContent{
\ifVerboseLocation This is Derivative Compute Question 0003. \\ \fi
\begin{problem}

Compute the following derivative:

\input{Derivative-Compute-0003.HELP.tex}

\[\dfrac{d}{dx}\left(5 \, \cos\left(-\frac{5}{6} \, \pi x\right) \sin\left(-\pi x\right)\right)=\answer{-5 \, \pi \cos\left(\pi x\right) \cos\left(\frac{5}{6} \, \pi x\right) + \frac{25}{6} \, \pi \sin\left(\pi x\right) \sin\left(\frac{5}{6} \, \pi x\right)}\]

\end{problem}}

%%%%%%%%%%%%%%%%%%%%%%

\latexProblemContent{
\ifVerboseLocation This is Derivative Compute Question 0003. \\ \fi
\begin{problem}

Compute the following derivative:

\input{Derivative-Compute-0003.HELP.tex}

\[\dfrac{d}{dx}\left(-15 \, \cos\left(\pi x\right) \sin\left(-\frac{2}{3} \, \pi x\right)\right)=\answer{10 \, \pi \cos\left(\pi x\right) \cos\left(\frac{2}{3} \, \pi x\right) - 15 \, \pi \sin\left(\pi x\right) \sin\left(\frac{2}{3} \, \pi x\right)}\]

\end{problem}}

%%%%%%%%%%%%%%%%%%%%%%

\latexProblemContent{
\ifVerboseLocation This is Derivative Compute Question 0003. \\ \fi
\begin{problem}

Compute the following derivative:

\input{Derivative-Compute-0003.HELP.tex}

\[\dfrac{d}{dx}\left(16 \, \cos\left(\frac{1}{2} \, \pi x\right) \tan\left(-3 \, \pi x\right)\right)=\answer{-48 \, \pi \cos\left(\frac{1}{2} \, \pi x\right) \sec\left(3 \, \pi x\right)^{2} + 8 \, \pi \sin\left(\frac{1}{2} \, \pi x\right) \tan\left(3 \, \pi x\right)}\]

\end{problem}}

%%%%%%%%%%%%%%%%%%%%%%

\latexProblemContent{
\ifVerboseLocation This is Derivative Compute Question 0003. \\ \fi
\begin{problem}

Compute the following derivative:

\input{Derivative-Compute-0003.HELP.tex}

\[\dfrac{d}{dx}\left(-16 \, \cos\left(-\frac{1}{2} \, \pi x\right) \tan\left(\frac{1}{3} \, \pi x\right)\right)=\answer{-\frac{16}{3} \, \pi \cos\left(\frac{1}{2} \, \pi x\right) \sec\left(\frac{1}{3} \, \pi x\right)^{2} + 8 \, \pi \sin\left(\frac{1}{2} \, \pi x\right) \tan\left(\frac{1}{3} \, \pi x\right)}\]

\end{problem}}

%%%%%%%%%%%%%%%%%%%%%%

\latexProblemContent{
\ifVerboseLocation This is Derivative Compute Question 0003. \\ \fi
\begin{problem}

Compute the following derivative:

\input{Derivative-Compute-0003.HELP.tex}

\[\dfrac{d}{dx}\left(3 \, \cos\left(-\frac{4}{3} \, \pi x\right) \sin\left(-\frac{5}{3} \, \pi x\right)\right)=\answer{-5 \, \pi \cos\left(\frac{5}{3} \, \pi x\right) \cos\left(\frac{4}{3} \, \pi x\right) + 4 \, \pi \sin\left(\frac{5}{3} \, \pi x\right) \sin\left(\frac{4}{3} \, \pi x\right)}\]

\end{problem}}

%%%%%%%%%%%%%%%%%%%%%%

\latexProblemContent{
\ifVerboseLocation This is Derivative Compute Question 0003. \\ \fi
\begin{problem}

Compute the following derivative:

\input{Derivative-Compute-0003.HELP.tex}

\[\dfrac{d}{dx}\left(-12 \, \sin\left(-\pi x\right) \tan\left(3 \, \pi x\right)\right)=\answer{36 \, \pi \sec\left(3 \, \pi x\right)^{2} \sin\left(\pi x\right) + 12 \, \pi \cos\left(\pi x\right) \tan\left(3 \, \pi x\right)}\]

\end{problem}}

%%%%%%%%%%%%%%%%%%%%%%

\latexProblemContent{
\ifVerboseLocation This is Derivative Compute Question 0003. \\ \fi
\begin{problem}

Compute the following derivative:

\input{Derivative-Compute-0003.HELP.tex}

\[\dfrac{d}{dx}\left(6 \, \cos\left(\frac{2}{3} \, \pi x\right) \sin\left(-2 \, \pi x\right)\right)=\answer{-12 \, \pi \cos\left(2 \, \pi x\right) \cos\left(\frac{2}{3} \, \pi x\right) + 4 \, \pi \sin\left(2 \, \pi x\right) \sin\left(\frac{2}{3} \, \pi x\right)}\]

\end{problem}}

%%%%%%%%%%%%%%%%%%%%%%

\latexProblemContent{
\ifVerboseLocation This is Derivative Compute Question 0003. \\ \fi
\begin{problem}

Compute the following derivative:

\input{Derivative-Compute-0003.HELP.tex}

\[\dfrac{d}{dx}\left(12 \, \sin\left(-\frac{5}{6} \, \pi x\right) \tan\left(-\frac{2}{3} \, \pi x\right)\right)=\answer{8 \, \pi \sec\left(\frac{2}{3} \, \pi x\right)^{2} \sin\left(\frac{5}{6} \, \pi x\right) + 10 \, \pi \cos\left(\frac{5}{6} \, \pi x\right) \tan\left(\frac{2}{3} \, \pi x\right)}\]

\end{problem}}

%%%%%%%%%%%%%%%%%%%%%%

\latexProblemContent{
\ifVerboseLocation This is Derivative Compute Question 0003. \\ \fi
\begin{problem}

Compute the following derivative:

\input{Derivative-Compute-0003.HELP.tex}

\[\dfrac{d}{dx}\left(9 \, \cos\left(\frac{2}{3} \, \pi x\right) \cos\left(-\frac{1}{3} \, \pi x\right)\right)=\answer{-6 \, \pi \cos\left(\frac{1}{3} \, \pi x\right) \sin\left(\frac{2}{3} \, \pi x\right) - 3 \, \pi \cos\left(\frac{2}{3} \, \pi x\right) \sin\left(\frac{1}{3} \, \pi x\right)}\]

\end{problem}}

%%%%%%%%%%%%%%%%%%%%%%

\latexProblemContent{
\ifVerboseLocation This is Derivative Compute Question 0003. \\ \fi
\begin{problem}

Compute the following derivative:

\input{Derivative-Compute-0003.HELP.tex}

\[\dfrac{d}{dx}\left(4 \, \sin\left(\frac{1}{2} \, \pi x\right) \tan\left(\frac{5}{2} \, \pi x\right)\right)=\answer{10 \, \pi \sec\left(\frac{5}{2} \, \pi x\right)^{2} \sin\left(\frac{1}{2} \, \pi x\right) + 2 \, \pi \cos\left(\frac{1}{2} \, \pi x\right) \tan\left(\frac{5}{2} \, \pi x\right)}\]

\end{problem}}

%%%%%%%%%%%%%%%%%%%%%%

\latexProblemContent{
\ifVerboseLocation This is Derivative Compute Question 0003. \\ \fi
\begin{problem}

Compute the following derivative:

\input{Derivative-Compute-0003.HELP.tex}

\[\dfrac{d}{dx}\left(10 \, \cos\left(-\frac{4}{3} \, \pi x\right) \sin\left(\frac{2}{3} \, \pi x\right)\right)=\answer{\frac{20}{3} \, \pi \cos\left(\frac{4}{3} \, \pi x\right) \cos\left(\frac{2}{3} \, \pi x\right) - \frac{40}{3} \, \pi \sin\left(\frac{4}{3} \, \pi x\right) \sin\left(\frac{2}{3} \, \pi x\right)}\]

\end{problem}}

%%%%%%%%%%%%%%%%%%%%%%

\latexProblemContent{
\ifVerboseLocation This is Derivative Compute Question 0003. \\ \fi
\begin{problem}

Compute the following derivative:

\input{Derivative-Compute-0003.HELP.tex}

\[\dfrac{d}{dx}\left(9 \, \cos\left(\pi x\right) \tan\left(2 \, \pi x\right)\right)=\answer{18 \, \pi \cos\left(\pi x\right) \sec\left(2 \, \pi x\right)^{2} - 9 \, \pi \sin\left(\pi x\right) \tan\left(2 \, \pi x\right)}\]

\end{problem}}

%%%%%%%%%%%%%%%%%%%%%%

\latexProblemContent{
\ifVerboseLocation This is Derivative Compute Question 0003. \\ \fi
\begin{problem}

Compute the following derivative:

\input{Derivative-Compute-0003.HELP.tex}

\[\dfrac{d}{dx}\left(-20 \, \cos\left(-\frac{1}{3} \, \pi x\right) \tan\left(-\frac{5}{6} \, \pi x\right)\right)=\answer{\frac{50}{3} \, \pi \cos\left(\frac{1}{3} \, \pi x\right) \sec\left(\frac{5}{6} \, \pi x\right)^{2} - \frac{20}{3} \, \pi \sin\left(\frac{1}{3} \, \pi x\right) \tan\left(\frac{5}{6} \, \pi x\right)}\]

\end{problem}}

%%%%%%%%%%%%%%%%%%%%%%

\latexProblemContent{
\ifVerboseLocation This is Derivative Compute Question 0003. \\ \fi
\begin{problem}

Compute the following derivative:

\input{Derivative-Compute-0003.HELP.tex}

\[\dfrac{d}{dx}\left(5 \, \sin\left(\frac{1}{2} \, \pi x\right) \tan\left(\frac{5}{2} \, \pi x\right)\right)=\answer{\frac{25}{2} \, \pi \sec\left(\frac{5}{2} \, \pi x\right)^{2} \sin\left(\frac{1}{2} \, \pi x\right) + \frac{5}{2} \, \pi \cos\left(\frac{1}{2} \, \pi x\right) \tan\left(\frac{5}{2} \, \pi x\right)}\]

\end{problem}}

%%%%%%%%%%%%%%%%%%%%%%

\latexProblemContent{
\ifVerboseLocation This is Derivative Compute Question 0003. \\ \fi
\begin{problem}

Compute the following derivative:

\input{Derivative-Compute-0003.HELP.tex}

\[\dfrac{d}{dx}\left(12 \, \cos\left(-\frac{2}{3} \, \pi x\right) \tan\left(\frac{1}{6} \, \pi x\right)\right)=\answer{2 \, \pi \cos\left(\frac{2}{3} \, \pi x\right) \sec\left(\frac{1}{6} \, \pi x\right)^{2} - 8 \, \pi \sin\left(\frac{2}{3} \, \pi x\right) \tan\left(\frac{1}{6} \, \pi x\right)}\]

\end{problem}}

%%%%%%%%%%%%%%%%%%%%%%

\latexProblemContent{
\ifVerboseLocation This is Derivative Compute Question 0003. \\ \fi
\begin{problem}

Compute the following derivative:

\input{Derivative-Compute-0003.HELP.tex}

\[\dfrac{d}{dx}\left(-16 \, \cos\left(\pi x\right) \tan\left(-\frac{5}{3} \, \pi x\right)\right)=\answer{\frac{80}{3} \, \pi \cos\left(\pi x\right) \sec\left(\frac{5}{3} \, \pi x\right)^{2} - 16 \, \pi \sin\left(\pi x\right) \tan\left(\frac{5}{3} \, \pi x\right)}\]

\end{problem}}

%%%%%%%%%%%%%%%%%%%%%%

\latexProblemContent{
\ifVerboseLocation This is Derivative Compute Question 0003. \\ \fi
\begin{problem}

Compute the following derivative:

\input{Derivative-Compute-0003.HELP.tex}

\[\dfrac{d}{dx}\left(16 \, \sin\left(-\pi x\right) \tan\left(-5 \, \pi x\right)\right)=\answer{80 \, \pi \sec\left(5 \, \pi x\right)^{2} \sin\left(\pi x\right) + 16 \, \pi \cos\left(\pi x\right) \tan\left(5 \, \pi x\right)}\]

\end{problem}}

%%%%%%%%%%%%%%%%%%%%%%

\latexProblemContent{
\ifVerboseLocation This is Derivative Compute Question 0003. \\ \fi
\begin{problem}

Compute the following derivative:

\input{Derivative-Compute-0003.HELP.tex}

\[\dfrac{d}{dx}\left(2 \, \cos\left(2 \, \pi x\right) \cos\left(-\frac{1}{6} \, \pi x\right)\right)=\answer{-4 \, \pi \cos\left(\frac{1}{6} \, \pi x\right) \sin\left(2 \, \pi x\right) - \frac{1}{3} \, \pi \cos\left(2 \, \pi x\right) \sin\left(\frac{1}{6} \, \pi x\right)}\]

\end{problem}}

%%%%%%%%%%%%%%%%%%%%%%

\latexProblemContent{
\ifVerboseLocation This is Derivative Compute Question 0003. \\ \fi
\begin{problem}

Compute the following derivative:

\input{Derivative-Compute-0003.HELP.tex}

\[\dfrac{d}{dx}\left(2 \, \cos\left(\frac{3}{2} \, \pi x\right) \tan\left(4 \, \pi x\right)\right)=\answer{8 \, \pi \cos\left(\frac{3}{2} \, \pi x\right) \sec\left(4 \, \pi x\right)^{2} - 3 \, \pi \sin\left(\frac{3}{2} \, \pi x\right) \tan\left(4 \, \pi x\right)}\]

\end{problem}}

%%%%%%%%%%%%%%%%%%%%%%

\latexProblemContent{
\ifVerboseLocation This is Derivative Compute Question 0003. \\ \fi
\begin{problem}

Compute the following derivative:

\input{Derivative-Compute-0003.HELP.tex}

\[\dfrac{d}{dx}\left(-3 \, \tan\left(\frac{1}{3} \, \pi x\right) \tan\left(-\frac{5}{2} \, \pi x\right)\right)=\answer{\pi \sec\left(\frac{1}{3} \, \pi x\right)^{2} \tan\left(\frac{5}{2} \, \pi x\right) + \frac{15}{2} \, \pi \sec\left(\frac{5}{2} \, \pi x\right)^{2} \tan\left(\frac{1}{3} \, \pi x\right)}\]

\end{problem}}

%%%%%%%%%%%%%%%%%%%%%%

\latexProblemContent{
\ifVerboseLocation This is Derivative Compute Question 0003. \\ \fi
\begin{problem}

Compute the following derivative:

\input{Derivative-Compute-0003.HELP.tex}

\[\dfrac{d}{dx}\left(5 \, \cos\left(4 \, \pi x\right) \tan\left(-\frac{1}{3} \, \pi x\right)\right)=\answer{-\frac{5}{3} \, \pi \cos\left(4 \, \pi x\right) \sec\left(\frac{1}{3} \, \pi x\right)^{2} + 20 \, \pi \sin\left(4 \, \pi x\right) \tan\left(\frac{1}{3} \, \pi x\right)}\]

\end{problem}}

%%%%%%%%%%%%%%%%%%%%%%

\latexProblemContent{
\ifVerboseLocation This is Derivative Compute Question 0003. \\ \fi
\begin{problem}

Compute the following derivative:

\input{Derivative-Compute-0003.HELP.tex}

\[\dfrac{d}{dx}\left(-12 \, \sin\left(\frac{5}{3} \, \pi x\right) \tan\left(-\pi x\right)\right)=\answer{12 \, \pi \sec\left(\pi x\right)^{2} \sin\left(\frac{5}{3} \, \pi x\right) + 20 \, \pi \cos\left(\frac{5}{3} \, \pi x\right) \tan\left(\pi x\right)}\]

\end{problem}}

%%%%%%%%%%%%%%%%%%%%%%

\latexProblemContent{
\ifVerboseLocation This is Derivative Compute Question 0003. \\ \fi
\begin{problem}

Compute the following derivative:

\input{Derivative-Compute-0003.HELP.tex}

\[\dfrac{d}{dx}\left(15 \, \sin\left(\pi x\right) \tan\left(-4 \, \pi x\right)\right)=\answer{-60 \, \pi \sec\left(4 \, \pi x\right)^{2} \sin\left(\pi x\right) - 15 \, \pi \cos\left(\pi x\right) \tan\left(4 \, \pi x\right)}\]

\end{problem}}

%%%%%%%%%%%%%%%%%%%%%%

\latexProblemContent{
\ifVerboseLocation This is Derivative Compute Question 0003. \\ \fi
\begin{problem}

Compute the following derivative:

\input{Derivative-Compute-0003.HELP.tex}

\[\dfrac{d}{dx}\left(5 \, \cos\left(\pi x\right) \cos\left(-3 \, \pi x\right)\right)=\answer{-15 \, \pi \cos\left(\pi x\right) \sin\left(3 \, \pi x\right) - 5 \, \pi \cos\left(3 \, \pi x\right) \sin\left(\pi x\right)}\]

\end{problem}}

%%%%%%%%%%%%%%%%%%%%%%

\latexProblemContent{
\ifVerboseLocation This is Derivative Compute Question 0003. \\ \fi
\begin{problem}

Compute the following derivative:

\input{Derivative-Compute-0003.HELP.tex}

\[\dfrac{d}{dx}\left(10 \, \sin\left(\pi x\right) \sin\left(\frac{1}{3} \, \pi x\right)\right)=\answer{\frac{10}{3} \, \pi \cos\left(\frac{1}{3} \, \pi x\right) \sin\left(\pi x\right) + 10 \, \pi \cos\left(\pi x\right) \sin\left(\frac{1}{3} \, \pi x\right)}\]

\end{problem}}

%%%%%%%%%%%%%%%%%%%%%%

\latexProblemContent{
\ifVerboseLocation This is Derivative Compute Question 0003. \\ \fi
\begin{problem}

Compute the following derivative:

\input{Derivative-Compute-0003.HELP.tex}

\[\dfrac{d}{dx}\left(-16 \, \cos\left(\frac{1}{6} \, \pi x\right) \tan\left(\frac{3}{2} \, \pi x\right)\right)=\answer{-24 \, \pi \cos\left(\frac{1}{6} \, \pi x\right) \sec\left(\frac{3}{2} \, \pi x\right)^{2} + \frac{8}{3} \, \pi \sin\left(\frac{1}{6} \, \pi x\right) \tan\left(\frac{3}{2} \, \pi x\right)}\]

\end{problem}}

%%%%%%%%%%%%%%%%%%%%%%

\latexProblemContent{
\ifVerboseLocation This is Derivative Compute Question 0003. \\ \fi
\begin{problem}

Compute the following derivative:

\input{Derivative-Compute-0003.HELP.tex}

\[\dfrac{d}{dx}\left(-15 \, \tan\left(\frac{5}{3} \, \pi x\right) \tan\left(-2 \, \pi x\right)\right)=\answer{25 \, \pi \sec\left(\frac{5}{3} \, \pi x\right)^{2} \tan\left(2 \, \pi x\right) + 30 \, \pi \sec\left(2 \, \pi x\right)^{2} \tan\left(\frac{5}{3} \, \pi x\right)}\]

\end{problem}}

%%%%%%%%%%%%%%%%%%%%%%

\latexProblemContent{
\ifVerboseLocation This is Derivative Compute Question 0003. \\ \fi
\begin{problem}

Compute the following derivative:

\input{Derivative-Compute-0003.HELP.tex}

\[\dfrac{d}{dx}\left(3 \, \sin\left(5 \, \pi x\right) \sin\left(\frac{1}{2} \, \pi x\right)\right)=\answer{\frac{3}{2} \, \pi \cos\left(\frac{1}{2} \, \pi x\right) \sin\left(5 \, \pi x\right) + 15 \, \pi \cos\left(5 \, \pi x\right) \sin\left(\frac{1}{2} \, \pi x\right)}\]

\end{problem}}

%%%%%%%%%%%%%%%%%%%%%%

\latexProblemContent{
\ifVerboseLocation This is Derivative Compute Question 0003. \\ \fi
\begin{problem}

Compute the following derivative:

\input{Derivative-Compute-0003.HELP.tex}

\[\dfrac{d}{dx}\left(3 \, \sin\left(\frac{5}{2} \, \pi x\right) \tan\left(-\pi x\right)\right)=\answer{-3 \, \pi \sec\left(\pi x\right)^{2} \sin\left(\frac{5}{2} \, \pi x\right) - \frac{15}{2} \, \pi \cos\left(\frac{5}{2} \, \pi x\right) \tan\left(\pi x\right)}\]

\end{problem}}

%%%%%%%%%%%%%%%%%%%%%%

\latexProblemContent{
\ifVerboseLocation This is Derivative Compute Question 0003. \\ \fi
\begin{problem}

Compute the following derivative:

\input{Derivative-Compute-0003.HELP.tex}

\[\dfrac{d}{dx}\left(-5 \, \cos\left(\frac{2}{3} \, \pi x\right) \cos\left(-\frac{2}{3} \, \pi x\right)\right)=\answer{\frac{20}{3} \, \pi \cos\left(\frac{2}{3} \, \pi x\right) \sin\left(\frac{2}{3} \, \pi x\right)}\]

\end{problem}}

%%%%%%%%%%%%%%%%%%%%%%

\latexProblemContent{
\ifVerboseLocation This is Derivative Compute Question 0003. \\ \fi
\begin{problem}

Compute the following derivative:

\input{Derivative-Compute-0003.HELP.tex}

\[\dfrac{d}{dx}\left(10 \, \tan\left(\frac{2}{3} \, \pi x\right) \tan\left(-5 \, \pi x\right)\right)=\answer{-\frac{20}{3} \, \pi \sec\left(\frac{2}{3} \, \pi x\right)^{2} \tan\left(5 \, \pi x\right) - 50 \, \pi \sec\left(5 \, \pi x\right)^{2} \tan\left(\frac{2}{3} \, \pi x\right)}\]

\end{problem}}

%%%%%%%%%%%%%%%%%%%%%%

\latexProblemContent{
\ifVerboseLocation This is Derivative Compute Question 0003. \\ \fi
\begin{problem}

Compute the following derivative:

\input{Derivative-Compute-0003.HELP.tex}

\[\dfrac{d}{dx}\left(-25 \, \cos\left(\pi x\right) \cos\left(-\frac{3}{2} \, \pi x\right)\right)=\answer{\frac{75}{2} \, \pi \cos\left(\pi x\right) \sin\left(\frac{3}{2} \, \pi x\right) + 25 \, \pi \cos\left(\frac{3}{2} \, \pi x\right) \sin\left(\pi x\right)}\]

\end{problem}}

%%%%%%%%%%%%%%%%%%%%%%

\latexProblemContent{
\ifVerboseLocation This is Derivative Compute Question 0003. \\ \fi
\begin{problem}

Compute the following derivative:

\input{Derivative-Compute-0003.HELP.tex}

\[\dfrac{d}{dx}\left(-8 \, \cos\left(\frac{2}{3} \, \pi x\right) \cos\left(-\frac{2}{3} \, \pi x\right)\right)=\answer{\frac{32}{3} \, \pi \cos\left(\frac{2}{3} \, \pi x\right) \sin\left(\frac{2}{3} \, \pi x\right)}\]

\end{problem}}

%%%%%%%%%%%%%%%%%%%%%%

\latexProblemContent{
\ifVerboseLocation This is Derivative Compute Question 0003. \\ \fi
\begin{problem}

Compute the following derivative:

\input{Derivative-Compute-0003.HELP.tex}

\[\dfrac{d}{dx}\left(8 \, \cos\left(3 \, \pi x\right) \tan\left(\pi x\right)\right)=\answer{8 \, \pi \cos\left(3 \, \pi x\right) \sec\left(\pi x\right)^{2} - 24 \, \pi \sin\left(3 \, \pi x\right) \tan\left(\pi x\right)}\]

\end{problem}}

%%%%%%%%%%%%%%%%%%%%%%

\latexProblemContent{
\ifVerboseLocation This is Derivative Compute Question 0003. \\ \fi
\begin{problem}

Compute the following derivative:

\input{Derivative-Compute-0003.HELP.tex}

\[\dfrac{d}{dx}\left(\cos\left(\frac{2}{3} \, \pi x\right) \tan\left(-\frac{5}{2} \, \pi x\right)\right)=\answer{-\frac{5}{2} \, \pi \cos\left(\frac{2}{3} \, \pi x\right) \sec\left(\frac{5}{2} \, \pi x\right)^{2} + \frac{2}{3} \, \pi \sin\left(\frac{2}{3} \, \pi x\right) \tan\left(\frac{5}{2} \, \pi x\right)}\]

\end{problem}}

%%%%%%%%%%%%%%%%%%%%%%

\latexProblemContent{
\ifVerboseLocation This is Derivative Compute Question 0003. \\ \fi
\begin{problem}

Compute the following derivative:

\input{Derivative-Compute-0003.HELP.tex}

\[\dfrac{d}{dx}\left(16 \, \sin\left(\frac{1}{2} \, \pi x\right) \sin\left(-\frac{1}{2} \, \pi x\right)\right)=\answer{-16 \, \pi \cos\left(\frac{1}{2} \, \pi x\right) \sin\left(\frac{1}{2} \, \pi x\right)}\]

\end{problem}}

%%%%%%%%%%%%%%%%%%%%%%

\latexProblemContent{
\ifVerboseLocation This is Derivative Compute Question 0003. \\ \fi
\begin{problem}

Compute the following derivative:

\input{Derivative-Compute-0003.HELP.tex}

\[\dfrac{d}{dx}\left(3 \, \sin\left(5 \, \pi x\right) \tan\left(\frac{1}{2} \, \pi x\right)\right)=\answer{\frac{3}{2} \, \pi \sec\left(\frac{1}{2} \, \pi x\right)^{2} \sin\left(5 \, \pi x\right) + 15 \, \pi \cos\left(5 \, \pi x\right) \tan\left(\frac{1}{2} \, \pi x\right)}\]

\end{problem}}

%%%%%%%%%%%%%%%%%%%%%%

\latexProblemContent{
\ifVerboseLocation This is Derivative Compute Question 0003. \\ \fi
\begin{problem}

Compute the following derivative:

\input{Derivative-Compute-0003.HELP.tex}

\[\dfrac{d}{dx}\left(5 \, \cos\left(-\frac{2}{3} \, \pi x\right) \tan\left(\frac{2}{3} \, \pi x\right)\right)=\answer{\frac{10}{3} \, \pi \cos\left(\frac{2}{3} \, \pi x\right) \sec\left(\frac{2}{3} \, \pi x\right)^{2} - \frac{10}{3} \, \pi \sin\left(\frac{2}{3} \, \pi x\right) \tan\left(\frac{2}{3} \, \pi x\right)}\]

\end{problem}}

%%%%%%%%%%%%%%%%%%%%%%

\latexProblemContent{
\ifVerboseLocation This is Derivative Compute Question 0003. \\ \fi
\begin{problem}

Compute the following derivative:

\input{Derivative-Compute-0003.HELP.tex}

\[\dfrac{d}{dx}\left(-25 \, \sin\left(-\frac{1}{2} \, \pi x\right) \sin\left(-\pi x\right)\right)=\answer{-\frac{25}{2} \, \pi \cos\left(\frac{1}{2} \, \pi x\right) \sin\left(\pi x\right) - 25 \, \pi \cos\left(\pi x\right) \sin\left(\frac{1}{2} \, \pi x\right)}\]

\end{problem}}

%%%%%%%%%%%%%%%%%%%%%%

\latexProblemContent{
\ifVerboseLocation This is Derivative Compute Question 0003. \\ \fi
\begin{problem}

Compute the following derivative:

\input{Derivative-Compute-0003.HELP.tex}

\[\dfrac{d}{dx}\left(-5 \, \tan\left(\pi x\right) \tan\left(-\frac{2}{3} \, \pi x\right)\right)=\answer{\frac{10}{3} \, \pi \sec\left(\frac{2}{3} \, \pi x\right)^{2} \tan\left(\pi x\right) + 5 \, \pi \sec\left(\pi x\right)^{2} \tan\left(\frac{2}{3} \, \pi x\right)}\]

\end{problem}}

%%%%%%%%%%%%%%%%%%%%%%

\latexProblemContent{
\ifVerboseLocation This is Derivative Compute Question 0003. \\ \fi
\begin{problem}

Compute the following derivative:

\input{Derivative-Compute-0003.HELP.tex}

\[\dfrac{d}{dx}\left(-15 \, \cos\left(-\pi x\right) \cos\left(-2 \, \pi x\right)\right)=\answer{30 \, \pi \cos\left(\pi x\right) \sin\left(2 \, \pi x\right) + 15 \, \pi \cos\left(2 \, \pi x\right) \sin\left(\pi x\right)}\]

\end{problem}}

%%%%%%%%%%%%%%%%%%%%%%

\latexProblemContent{
\ifVerboseLocation This is Derivative Compute Question 0003. \\ \fi
\begin{problem}

Compute the following derivative:

\input{Derivative-Compute-0003.HELP.tex}

\[\dfrac{d}{dx}\left(-2 \, \sin\left(\pi x\right) \sin\left(-\frac{2}{3} \, \pi x\right)\right)=\answer{\frac{4}{3} \, \pi \cos\left(\frac{2}{3} \, \pi x\right) \sin\left(\pi x\right) + 2 \, \pi \cos\left(\pi x\right) \sin\left(\frac{2}{3} \, \pi x\right)}\]

\end{problem}}

%%%%%%%%%%%%%%%%%%%%%%

\latexProblemContent{
\ifVerboseLocation This is Derivative Compute Question 0003. \\ \fi
\begin{problem}

Compute the following derivative:

\input{Derivative-Compute-0003.HELP.tex}

\[\dfrac{d}{dx}\left(20 \, \cos\left(-\frac{1}{2} \, \pi x\right) \sin\left(\frac{1}{3} \, \pi x\right)\right)=\answer{\frac{20}{3} \, \pi \cos\left(\frac{1}{2} \, \pi x\right) \cos\left(\frac{1}{3} \, \pi x\right) - 10 \, \pi \sin\left(\frac{1}{2} \, \pi x\right) \sin\left(\frac{1}{3} \, \pi x\right)}\]

\end{problem}}

%%%%%%%%%%%%%%%%%%%%%%

\latexProblemContent{
\ifVerboseLocation This is Derivative Compute Question 0003. \\ \fi
\begin{problem}

Compute the following derivative:

\input{Derivative-Compute-0003.HELP.tex}

\[\dfrac{d}{dx}\left(-4 \, \cos\left(\frac{1}{3} \, \pi x\right) \tan\left(-\frac{1}{6} \, \pi x\right)\right)=\answer{\frac{2}{3} \, \pi \cos\left(\frac{1}{3} \, \pi x\right) \sec\left(\frac{1}{6} \, \pi x\right)^{2} - \frac{4}{3} \, \pi \sin\left(\frac{1}{3} \, \pi x\right) \tan\left(\frac{1}{6} \, \pi x\right)}\]

\end{problem}}

%%%%%%%%%%%%%%%%%%%%%%

\latexProblemContent{
\ifVerboseLocation This is Derivative Compute Question 0003. \\ \fi
\begin{problem}

Compute the following derivative:

\input{Derivative-Compute-0003.HELP.tex}

\[\dfrac{d}{dx}\left(-25 \, \sin\left(\pi x\right) \tan\left(\pi x\right)\right)=\answer{-25 \, \pi \sec\left(\pi x\right)^{2} \sin\left(\pi x\right) - 25 \, \pi \cos\left(\pi x\right) \tan\left(\pi x\right)}\]

\end{problem}}

%%%%%%%%%%%%%%%%%%%%%%

\latexProblemContent{
\ifVerboseLocation This is Derivative Compute Question 0003. \\ \fi
\begin{problem}

Compute the following derivative:

\input{Derivative-Compute-0003.HELP.tex}

\[\dfrac{d}{dx}\left(-5 \, \cos\left(-\frac{5}{2} \, \pi x\right) \tan\left(\frac{1}{3} \, \pi x\right)\right)=\answer{-\frac{5}{3} \, \pi \cos\left(\frac{5}{2} \, \pi x\right) \sec\left(\frac{1}{3} \, \pi x\right)^{2} + \frac{25}{2} \, \pi \sin\left(\frac{5}{2} \, \pi x\right) \tan\left(\frac{1}{3} \, \pi x\right)}\]

\end{problem}}

%%%%%%%%%%%%%%%%%%%%%%

\latexProblemContent{
\ifVerboseLocation This is Derivative Compute Question 0003. \\ \fi
\begin{problem}

Compute the following derivative:

\input{Derivative-Compute-0003.HELP.tex}

\[\dfrac{d}{dx}\left(10 \, \cos\left(-2 \, \pi x\right) \tan\left(-\pi x\right)\right)=\answer{-10 \, \pi \cos\left(2 \, \pi x\right) \sec\left(\pi x\right)^{2} + 20 \, \pi \sin\left(2 \, \pi x\right) \tan\left(\pi x\right)}\]

\end{problem}}

%%%%%%%%%%%%%%%%%%%%%%

\latexProblemContent{
\ifVerboseLocation This is Derivative Compute Question 0003. \\ \fi
\begin{problem}

Compute the following derivative:

\input{Derivative-Compute-0003.HELP.tex}

\[\dfrac{d}{dx}\left(-\sin\left(-\frac{2}{3} \, \pi x\right) \tan\left(\frac{5}{2} \, \pi x\right)\right)=\answer{\frac{5}{2} \, \pi \sec\left(\frac{5}{2} \, \pi x\right)^{2} \sin\left(\frac{2}{3} \, \pi x\right) + \frac{2}{3} \, \pi \cos\left(\frac{2}{3} \, \pi x\right) \tan\left(\frac{5}{2} \, \pi x\right)}\]

\end{problem}}

%%%%%%%%%%%%%%%%%%%%%%

\latexProblemContent{
\ifVerboseLocation This is Derivative Compute Question 0003. \\ \fi
\begin{problem}

Compute the following derivative:

\input{Derivative-Compute-0003.HELP.tex}

\[\dfrac{d}{dx}\left(3 \, \cos\left(-\pi x\right) \tan\left(-\frac{1}{6} \, \pi x\right)\right)=\answer{-\frac{1}{2} \, \pi \cos\left(\pi x\right) \sec\left(\frac{1}{6} \, \pi x\right)^{2} + 3 \, \pi \sin\left(\pi x\right) \tan\left(\frac{1}{6} \, \pi x\right)}\]

\end{problem}}

%%%%%%%%%%%%%%%%%%%%%%

\latexProblemContent{
\ifVerboseLocation This is Derivative Compute Question 0003. \\ \fi
\begin{problem}

Compute the following derivative:

\input{Derivative-Compute-0003.HELP.tex}

\[\dfrac{d}{dx}\left(16 \, \sin\left(2 \, \pi x\right) \sin\left(\frac{3}{2} \, \pi x\right)\right)=\answer{24 \, \pi \cos\left(\frac{3}{2} \, \pi x\right) \sin\left(2 \, \pi x\right) + 32 \, \pi \cos\left(2 \, \pi x\right) \sin\left(\frac{3}{2} \, \pi x\right)}\]

\end{problem}}

%%%%%%%%%%%%%%%%%%%%%%

\latexProblemContent{
\ifVerboseLocation This is Derivative Compute Question 0003. \\ \fi
\begin{problem}

Compute the following derivative:

\input{Derivative-Compute-0003.HELP.tex}

\[\dfrac{d}{dx}\left(-12 \, \cos\left(-\frac{5}{2} \, \pi x\right) \tan\left(-\frac{1}{3} \, \pi x\right)\right)=\answer{4 \, \pi \cos\left(\frac{5}{2} \, \pi x\right) \sec\left(\frac{1}{3} \, \pi x\right)^{2} - 30 \, \pi \sin\left(\frac{5}{2} \, \pi x\right) \tan\left(\frac{1}{3} \, \pi x\right)}\]

\end{problem}}

%%%%%%%%%%%%%%%%%%%%%%

\latexProblemContent{
\ifVerboseLocation This is Derivative Compute Question 0003. \\ \fi
\begin{problem}

Compute the following derivative:

\input{Derivative-Compute-0003.HELP.tex}

\[\dfrac{d}{dx}\left(-8 \, \sin\left(-3 \, \pi x\right) \tan\left(-\pi x\right)\right)=\answer{-8 \, \pi \sec\left(\pi x\right)^{2} \sin\left(3 \, \pi x\right) - 24 \, \pi \cos\left(3 \, \pi x\right) \tan\left(\pi x\right)}\]

\end{problem}}

%%%%%%%%%%%%%%%%%%%%%%

\latexProblemContent{
\ifVerboseLocation This is Derivative Compute Question 0003. \\ \fi
\begin{problem}

Compute the following derivative:

\input{Derivative-Compute-0003.HELP.tex}

\[\dfrac{d}{dx}\left(-10 \, \sin\left(\frac{4}{3} \, \pi x\right) \sin\left(-\frac{4}{3} \, \pi x\right)\right)=\answer{\frac{80}{3} \, \pi \cos\left(\frac{4}{3} \, \pi x\right) \sin\left(\frac{4}{3} \, \pi x\right)}\]

\end{problem}}

%%%%%%%%%%%%%%%%%%%%%%

\latexProblemContent{
\ifVerboseLocation This is Derivative Compute Question 0003. \\ \fi
\begin{problem}

Compute the following derivative:

\input{Derivative-Compute-0003.HELP.tex}

\[\dfrac{d}{dx}\left(-6 \, \cos\left(\frac{1}{6} \, \pi x\right) \tan\left(-\frac{2}{3} \, \pi x\right)\right)=\answer{4 \, \pi \cos\left(\frac{1}{6} \, \pi x\right) \sec\left(\frac{2}{3} \, \pi x\right)^{2} - \pi \sin\left(\frac{1}{6} \, \pi x\right) \tan\left(\frac{2}{3} \, \pi x\right)}\]

\end{problem}}

%%%%%%%%%%%%%%%%%%%%%%

\latexProblemContent{
\ifVerboseLocation This is Derivative Compute Question 0003. \\ \fi
\begin{problem}

Compute the following derivative:

\input{Derivative-Compute-0003.HELP.tex}

\[\dfrac{d}{dx}\left(8 \, \sin\left(\frac{1}{6} \, \pi x\right) \tan\left(\frac{5}{3} \, \pi x\right)\right)=\answer{\frac{40}{3} \, \pi \sec\left(\frac{5}{3} \, \pi x\right)^{2} \sin\left(\frac{1}{6} \, \pi x\right) + \frac{4}{3} \, \pi \cos\left(\frac{1}{6} \, \pi x\right) \tan\left(\frac{5}{3} \, \pi x\right)}\]

\end{problem}}

%%%%%%%%%%%%%%%%%%%%%%

\latexProblemContent{
\ifVerboseLocation This is Derivative Compute Question 0003. \\ \fi
\begin{problem}

Compute the following derivative:

\input{Derivative-Compute-0003.HELP.tex}

\[\dfrac{d}{dx}\left(3 \, \cos\left(\frac{1}{6} \, \pi x\right) \cos\left(-\frac{1}{2} \, \pi x\right)\right)=\answer{-\frac{3}{2} \, \pi \cos\left(\frac{1}{6} \, \pi x\right) \sin\left(\frac{1}{2} \, \pi x\right) - \frac{1}{2} \, \pi \cos\left(\frac{1}{2} \, \pi x\right) \sin\left(\frac{1}{6} \, \pi x\right)}\]

\end{problem}}

%%%%%%%%%%%%%%%%%%%%%%

\latexProblemContent{
\ifVerboseLocation This is Derivative Compute Question 0003. \\ \fi
\begin{problem}

Compute the following derivative:

\input{Derivative-Compute-0003.HELP.tex}

\[\dfrac{d}{dx}\left(15 \, \cos\left(\pi x\right) \tan\left(-2 \, \pi x\right)\right)=\answer{-30 \, \pi \cos\left(\pi x\right) \sec\left(2 \, \pi x\right)^{2} + 15 \, \pi \sin\left(\pi x\right) \tan\left(2 \, \pi x\right)}\]

\end{problem}}

%%%%%%%%%%%%%%%%%%%%%%

\latexProblemContent{
\ifVerboseLocation This is Derivative Compute Question 0003. \\ \fi
\begin{problem}

Compute the following derivative:

\input{Derivative-Compute-0003.HELP.tex}

\[\dfrac{d}{dx}\left(-10 \, \cos\left(-\frac{2}{3} \, \pi x\right) \sin\left(\pi x\right)\right)=\answer{-10 \, \pi \cos\left(\pi x\right) \cos\left(\frac{2}{3} \, \pi x\right) + \frac{20}{3} \, \pi \sin\left(\pi x\right) \sin\left(\frac{2}{3} \, \pi x\right)}\]

\end{problem}}

%%%%%%%%%%%%%%%%%%%%%%

\latexProblemContent{
\ifVerboseLocation This is Derivative Compute Question 0003. \\ \fi
\begin{problem}

Compute the following derivative:

\input{Derivative-Compute-0003.HELP.tex}

\[\dfrac{d}{dx}\left(4 \, \cos\left(\frac{4}{3} \, \pi x\right) \sin\left(-\frac{2}{3} \, \pi x\right)\right)=\answer{-\frac{8}{3} \, \pi \cos\left(\frac{4}{3} \, \pi x\right) \cos\left(\frac{2}{3} \, \pi x\right) + \frac{16}{3} \, \pi \sin\left(\frac{4}{3} \, \pi x\right) \sin\left(\frac{2}{3} \, \pi x\right)}\]

\end{problem}}

%%%%%%%%%%%%%%%%%%%%%%

\latexProblemContent{
\ifVerboseLocation This is Derivative Compute Question 0003. \\ \fi
\begin{problem}

Compute the following derivative:

\input{Derivative-Compute-0003.HELP.tex}

\[\dfrac{d}{dx}\left(3 \, \cos\left(-\frac{3}{2} \, \pi x\right) \sin\left(-\frac{2}{3} \, \pi x\right)\right)=\answer{-2 \, \pi \cos\left(\frac{3}{2} \, \pi x\right) \cos\left(\frac{2}{3} \, \pi x\right) + \frac{9}{2} \, \pi \sin\left(\frac{3}{2} \, \pi x\right) \sin\left(\frac{2}{3} \, \pi x\right)}\]

\end{problem}}

%%%%%%%%%%%%%%%%%%%%%%

\latexProblemContent{
\ifVerboseLocation This is Derivative Compute Question 0003. \\ \fi
\begin{problem}

Compute the following derivative:

\input{Derivative-Compute-0003.HELP.tex}

\[\dfrac{d}{dx}\left(2 \, \sin\left(\frac{1}{2} \, \pi x\right) \sin\left(\frac{1}{3} \, \pi x\right)\right)=\answer{\frac{2}{3} \, \pi \cos\left(\frac{1}{3} \, \pi x\right) \sin\left(\frac{1}{2} \, \pi x\right) + \pi \cos\left(\frac{1}{2} \, \pi x\right) \sin\left(\frac{1}{3} \, \pi x\right)}\]

\end{problem}}

%%%%%%%%%%%%%%%%%%%%%%

\latexProblemContent{
\ifVerboseLocation This is Derivative Compute Question 0003. \\ \fi
\begin{problem}

Compute the following derivative:

\input{Derivative-Compute-0003.HELP.tex}

\[\dfrac{d}{dx}\left(-3 \, \cos\left(\pi x\right) \sin\left(-\frac{4}{3} \, \pi x\right)\right)=\answer{4 \, \pi \cos\left(\frac{4}{3} \, \pi x\right) \cos\left(\pi x\right) - 3 \, \pi \sin\left(\frac{4}{3} \, \pi x\right) \sin\left(\pi x\right)}\]

\end{problem}}

%%%%%%%%%%%%%%%%%%%%%%

\latexProblemContent{
\ifVerboseLocation This is Derivative Compute Question 0003. \\ \fi
\begin{problem}

Compute the following derivative:

\input{Derivative-Compute-0003.HELP.tex}

\[\dfrac{d}{dx}\left(-16 \, \tan\left(2 \, \pi x\right) \tan\left(-\pi x\right)\right)=\answer{16 \, \pi \sec\left(\pi x\right)^{2} \tan\left(2 \, \pi x\right) + 32 \, \pi \sec\left(2 \, \pi x\right)^{2} \tan\left(\pi x\right)}\]

\end{problem}}

%%%%%%%%%%%%%%%%%%%%%%

\latexProblemContent{
\ifVerboseLocation This is Derivative Compute Question 0003. \\ \fi
\begin{problem}

Compute the following derivative:

\input{Derivative-Compute-0003.HELP.tex}

\[\dfrac{d}{dx}\left(8 \, \cos\left(\pi x\right) \tan\left(\pi x\right)\right)=\answer{8 \, \pi \cos\left(\pi x\right) \sec\left(\pi x\right)^{2} - 8 \, \pi \sin\left(\pi x\right) \tan\left(\pi x\right)}\]

\end{problem}}

%%%%%%%%%%%%%%%%%%%%%%

\latexProblemContent{
\ifVerboseLocation This is Derivative Compute Question 0003. \\ \fi
\begin{problem}

Compute the following derivative:

\input{Derivative-Compute-0003.HELP.tex}

\[\dfrac{d}{dx}\left(5 \, \cos\left(\pi x\right) \tan\left(-\frac{1}{2} \, \pi x\right)\right)=\answer{-\frac{5}{2} \, \pi \cos\left(\pi x\right) \sec\left(\frac{1}{2} \, \pi x\right)^{2} + 5 \, \pi \sin\left(\pi x\right) \tan\left(\frac{1}{2} \, \pi x\right)}\]

\end{problem}}

%%%%%%%%%%%%%%%%%%%%%%

\latexProblemContent{
\ifVerboseLocation This is Derivative Compute Question 0003. \\ \fi
\begin{problem}

Compute the following derivative:

\input{Derivative-Compute-0003.HELP.tex}

\[\dfrac{d}{dx}\left(6 \, \sin\left(5 \, \pi x\right) \sin\left(\pi x\right)\right)=\answer{6 \, \pi \cos\left(\pi x\right) \sin\left(5 \, \pi x\right) + 30 \, \pi \cos\left(5 \, \pi x\right) \sin\left(\pi x\right)}\]

\end{problem}}

%%%%%%%%%%%%%%%%%%%%%%

\latexProblemContent{
\ifVerboseLocation This is Derivative Compute Question 0003. \\ \fi
\begin{problem}

Compute the following derivative:

\input{Derivative-Compute-0003.HELP.tex}

\[\dfrac{d}{dx}\left(-\sin\left(-5 \, \pi x\right) \tan\left(-\frac{5}{3} \, \pi x\right)\right)=\answer{-\frac{5}{3} \, \pi \sec\left(\frac{5}{3} \, \pi x\right)^{2} \sin\left(5 \, \pi x\right) - 5 \, \pi \cos\left(5 \, \pi x\right) \tan\left(\frac{5}{3} \, \pi x\right)}\]

\end{problem}}

%%%%%%%%%%%%%%%%%%%%%%

\latexProblemContent{
\ifVerboseLocation This is Derivative Compute Question 0003. \\ \fi
\begin{problem}

Compute the following derivative:

\input{Derivative-Compute-0003.HELP.tex}

\[\dfrac{d}{dx}\left(-9 \, \cos\left(\frac{5}{6} \, \pi x\right) \sin\left(-2 \, \pi x\right)\right)=\answer{18 \, \pi \cos\left(2 \, \pi x\right) \cos\left(\frac{5}{6} \, \pi x\right) - \frac{15}{2} \, \pi \sin\left(2 \, \pi x\right) \sin\left(\frac{5}{6} \, \pi x\right)}\]

\end{problem}}

%%%%%%%%%%%%%%%%%%%%%%

\latexProblemContent{
\ifVerboseLocation This is Derivative Compute Question 0003. \\ \fi
\begin{problem}

Compute the following derivative:

\input{Derivative-Compute-0003.HELP.tex}

\[\dfrac{d}{dx}\left(-6 \, \tan\left(3 \, \pi x\right) \tan\left(\frac{1}{3} \, \pi x\right)\right)=\answer{-2 \, \pi \sec\left(\frac{1}{3} \, \pi x\right)^{2} \tan\left(3 \, \pi x\right) - 18 \, \pi \sec\left(3 \, \pi x\right)^{2} \tan\left(\frac{1}{3} \, \pi x\right)}\]

\end{problem}}

%%%%%%%%%%%%%%%%%%%%%%

\latexProblemContent{
\ifVerboseLocation This is Derivative Compute Question 0003. \\ \fi
\begin{problem}

Compute the following derivative:

\input{Derivative-Compute-0003.HELP.tex}

\[\dfrac{d}{dx}\left(10 \, \tan\left(2 \, \pi x\right) \tan\left(-\frac{1}{3} \, \pi x\right)\right)=\answer{-\frac{10}{3} \, \pi \sec\left(\frac{1}{3} \, \pi x\right)^{2} \tan\left(2 \, \pi x\right) - 20 \, \pi \sec\left(2 \, \pi x\right)^{2} \tan\left(\frac{1}{3} \, \pi x\right)}\]

\end{problem}}

%%%%%%%%%%%%%%%%%%%%%%

\latexProblemContent{
\ifVerboseLocation This is Derivative Compute Question 0003. \\ \fi
\begin{problem}

Compute the following derivative:

\input{Derivative-Compute-0003.HELP.tex}

\[\dfrac{d}{dx}\left(20 \, \cos\left(-4 \, \pi x\right) \tan\left(2 \, \pi x\right)\right)=\answer{40 \, \pi \cos\left(4 \, \pi x\right) \sec\left(2 \, \pi x\right)^{2} - 80 \, \pi \sin\left(4 \, \pi x\right) \tan\left(2 \, \pi x\right)}\]

\end{problem}}

%%%%%%%%%%%%%%%%%%%%%%

\latexProblemContent{
\ifVerboseLocation This is Derivative Compute Question 0003. \\ \fi
\begin{problem}

Compute the following derivative:

\input{Derivative-Compute-0003.HELP.tex}

\[\dfrac{d}{dx}\left(-10 \, \sin\left(-\pi x\right) \tan\left(\frac{5}{3} \, \pi x\right)\right)=\answer{\frac{50}{3} \, \pi \sec\left(\frac{5}{3} \, \pi x\right)^{2} \sin\left(\pi x\right) + 10 \, \pi \cos\left(\pi x\right) \tan\left(\frac{5}{3} \, \pi x\right)}\]

\end{problem}}

%%%%%%%%%%%%%%%%%%%%%%

\latexProblemContent{
\ifVerboseLocation This is Derivative Compute Question 0003. \\ \fi
\begin{problem}

Compute the following derivative:

\input{Derivative-Compute-0003.HELP.tex}

\[\dfrac{d}{dx}\left(9 \, \tan\left(2 \, \pi x\right) \tan\left(-\pi x\right)\right)=\answer{-9 \, \pi \sec\left(\pi x\right)^{2} \tan\left(2 \, \pi x\right) - 18 \, \pi \sec\left(2 \, \pi x\right)^{2} \tan\left(\pi x\right)}\]

\end{problem}}

%%%%%%%%%%%%%%%%%%%%%%

\latexProblemContent{
\ifVerboseLocation This is Derivative Compute Question 0003. \\ \fi
\begin{problem}

Compute the following derivative:

\input{Derivative-Compute-0003.HELP.tex}

\[\dfrac{d}{dx}\left(-8 \, \tan\left(-\pi x\right) \tan\left(-2 \, \pi x\right)\right)=\answer{-8 \, \pi \sec\left(\pi x\right)^{2} \tan\left(2 \, \pi x\right) - 16 \, \pi \sec\left(2 \, \pi x\right)^{2} \tan\left(\pi x\right)}\]

\end{problem}}

%%%%%%%%%%%%%%%%%%%%%%

\latexProblemContent{
\ifVerboseLocation This is Derivative Compute Question 0003. \\ \fi
\begin{problem}

Compute the following derivative:

\input{Derivative-Compute-0003.HELP.tex}

\[\dfrac{d}{dx}\left(-6 \, \cos\left(4 \, \pi x\right) \sin\left(-\frac{1}{3} \, \pi x\right)\right)=\answer{2 \, \pi \cos\left(4 \, \pi x\right) \cos\left(\frac{1}{3} \, \pi x\right) - 24 \, \pi \sin\left(4 \, \pi x\right) \sin\left(\frac{1}{3} \, \pi x\right)}\]

\end{problem}}

%%%%%%%%%%%%%%%%%%%%%%

\latexProblemContent{
\ifVerboseLocation This is Derivative Compute Question 0003. \\ \fi
\begin{problem}

Compute the following derivative:

\input{Derivative-Compute-0003.HELP.tex}

\[\dfrac{d}{dx}\left(10 \, \sin\left(\frac{4}{3} \, \pi x\right) \sin\left(\pi x\right)\right)=\answer{10 \, \pi \cos\left(\pi x\right) \sin\left(\frac{4}{3} \, \pi x\right) + \frac{40}{3} \, \pi \cos\left(\frac{4}{3} \, \pi x\right) \sin\left(\pi x\right)}\]

\end{problem}}

%%%%%%%%%%%%%%%%%%%%%%

\latexProblemContent{
\ifVerboseLocation This is Derivative Compute Question 0003. \\ \fi
\begin{problem}

Compute the following derivative:

\input{Derivative-Compute-0003.HELP.tex}

\[\dfrac{d}{dx}\left(\cos\left(2 \, \pi x\right) \tan\left(\frac{5}{6} \, \pi x\right)\right)=\answer{\frac{5}{6} \, \pi \cos\left(2 \, \pi x\right) \sec\left(\frac{5}{6} \, \pi x\right)^{2} - 2 \, \pi \sin\left(2 \, \pi x\right) \tan\left(\frac{5}{6} \, \pi x\right)}\]

\end{problem}}

%%%%%%%%%%%%%%%%%%%%%%

\latexProblemContent{
\ifVerboseLocation This is Derivative Compute Question 0003. \\ \fi
\begin{problem}

Compute the following derivative:

\input{Derivative-Compute-0003.HELP.tex}

\[\dfrac{d}{dx}\left(8 \, \cos\left(\frac{5}{6} \, \pi x\right) \tan\left(\frac{2}{3} \, \pi x\right)\right)=\answer{\frac{16}{3} \, \pi \cos\left(\frac{5}{6} \, \pi x\right) \sec\left(\frac{2}{3} \, \pi x\right)^{2} - \frac{20}{3} \, \pi \sin\left(\frac{5}{6} \, \pi x\right) \tan\left(\frac{2}{3} \, \pi x\right)}\]

\end{problem}}

%%%%%%%%%%%%%%%%%%%%%%

\latexProblemContent{
\ifVerboseLocation This is Derivative Compute Question 0003. \\ \fi
\begin{problem}

Compute the following derivative:

\input{Derivative-Compute-0003.HELP.tex}

\[\dfrac{d}{dx}\left(20 \, \cos\left(\frac{1}{2} \, \pi x\right) \tan\left(2 \, \pi x\right)\right)=\answer{40 \, \pi \cos\left(\frac{1}{2} \, \pi x\right) \sec\left(2 \, \pi x\right)^{2} - 10 \, \pi \sin\left(\frac{1}{2} \, \pi x\right) \tan\left(2 \, \pi x\right)}\]

\end{problem}}

%%%%%%%%%%%%%%%%%%%%%%

\latexProblemContent{
\ifVerboseLocation This is Derivative Compute Question 0003. \\ \fi
\begin{problem}

Compute the following derivative:

\input{Derivative-Compute-0003.HELP.tex}

\[\dfrac{d}{dx}\left(-16 \, \sin\left(\frac{5}{2} \, \pi x\right) \tan\left(\frac{5}{3} \, \pi x\right)\right)=\answer{-\frac{80}{3} \, \pi \sec\left(\frac{5}{3} \, \pi x\right)^{2} \sin\left(\frac{5}{2} \, \pi x\right) - 40 \, \pi \cos\left(\frac{5}{2} \, \pi x\right) \tan\left(\frac{5}{3} \, \pi x\right)}\]

\end{problem}}

%%%%%%%%%%%%%%%%%%%%%%

\latexProblemContent{
\ifVerboseLocation This is Derivative Compute Question 0003. \\ \fi
\begin{problem}

Compute the following derivative:

\input{Derivative-Compute-0003.HELP.tex}

\[\dfrac{d}{dx}\left(-2 \, \sin\left(\frac{2}{3} \, \pi x\right) \tan\left(-\frac{5}{6} \, \pi x\right)\right)=\answer{\frac{5}{3} \, \pi \sec\left(\frac{5}{6} \, \pi x\right)^{2} \sin\left(\frac{2}{3} \, \pi x\right) + \frac{4}{3} \, \pi \cos\left(\frac{2}{3} \, \pi x\right) \tan\left(\frac{5}{6} \, \pi x\right)}\]

\end{problem}}

%%%%%%%%%%%%%%%%%%%%%%

\latexProblemContent{
\ifVerboseLocation This is Derivative Compute Question 0003. \\ \fi
\begin{problem}

Compute the following derivative:

\input{Derivative-Compute-0003.HELP.tex}

\[\dfrac{d}{dx}\left(3 \, \cos\left(-\frac{1}{3} \, \pi x\right) \sin\left(-\frac{3}{2} \, \pi x\right)\right)=\answer{-\frac{9}{2} \, \pi \cos\left(\frac{3}{2} \, \pi x\right) \cos\left(\frac{1}{3} \, \pi x\right) + \pi \sin\left(\frac{3}{2} \, \pi x\right) \sin\left(\frac{1}{3} \, \pi x\right)}\]

\end{problem}}

%%%%%%%%%%%%%%%%%%%%%%

\latexProblemContent{
\ifVerboseLocation This is Derivative Compute Question 0003. \\ \fi
\begin{problem}

Compute the following derivative:

\input{Derivative-Compute-0003.HELP.tex}

\[\dfrac{d}{dx}\left(-20 \, \tan\left(-\frac{2}{3} \, \pi x\right) \tan\left(-3 \, \pi x\right)\right)=\answer{-\frac{40}{3} \, \pi \sec\left(\frac{2}{3} \, \pi x\right)^{2} \tan\left(3 \, \pi x\right) - 60 \, \pi \sec\left(3 \, \pi x\right)^{2} \tan\left(\frac{2}{3} \, \pi x\right)}\]

\end{problem}}

%%%%%%%%%%%%%%%%%%%%%%

\latexProblemContent{
\ifVerboseLocation This is Derivative Compute Question 0003. \\ \fi
\begin{problem}

Compute the following derivative:

\input{Derivative-Compute-0003.HELP.tex}

\[\dfrac{d}{dx}\left(-8 \, \cos\left(\frac{1}{6} \, \pi x\right) \sin\left(\frac{2}{3} \, \pi x\right)\right)=\answer{-\frac{16}{3} \, \pi \cos\left(\frac{2}{3} \, \pi x\right) \cos\left(\frac{1}{6} \, \pi x\right) + \frac{4}{3} \, \pi \sin\left(\frac{2}{3} \, \pi x\right) \sin\left(\frac{1}{6} \, \pi x\right)}\]

\end{problem}}

%%%%%%%%%%%%%%%%%%%%%%

\latexProblemContent{
\ifVerboseLocation This is Derivative Compute Question 0003. \\ \fi
\begin{problem}

Compute the following derivative:

\input{Derivative-Compute-0003.HELP.tex}

\[\dfrac{d}{dx}\left(25 \, \sin\left(\frac{3}{2} \, \pi x\right) \tan\left(-\frac{3}{2} \, \pi x\right)\right)=\answer{-\frac{75}{2} \, \pi \sec\left(\frac{3}{2} \, \pi x\right)^{2} \sin\left(\frac{3}{2} \, \pi x\right) - \frac{75}{2} \, \pi \cos\left(\frac{3}{2} \, \pi x\right) \tan\left(\frac{3}{2} \, \pi x\right)}\]

\end{problem}}

%%%%%%%%%%%%%%%%%%%%%%

\latexProblemContent{
\ifVerboseLocation This is Derivative Compute Question 0003. \\ \fi
\begin{problem}

Compute the following derivative:

\input{Derivative-Compute-0003.HELP.tex}

\[\dfrac{d}{dx}\left(-5 \, \cos\left(\frac{1}{3} \, \pi x\right) \tan\left(\pi x\right)\right)=\answer{-5 \, \pi \cos\left(\frac{1}{3} \, \pi x\right) \sec\left(\pi x\right)^{2} + \frac{5}{3} \, \pi \sin\left(\frac{1}{3} \, \pi x\right) \tan\left(\pi x\right)}\]

\end{problem}}

%%%%%%%%%%%%%%%%%%%%%%

\latexProblemContent{
\ifVerboseLocation This is Derivative Compute Question 0003. \\ \fi
\begin{problem}

Compute the following derivative:

\input{Derivative-Compute-0003.HELP.tex}

\[\dfrac{d}{dx}\left(-15 \, \cos\left(-\frac{1}{6} \, \pi x\right) \cos\left(-\frac{3}{2} \, \pi x\right)\right)=\answer{\frac{45}{2} \, \pi \cos\left(\frac{1}{6} \, \pi x\right) \sin\left(\frac{3}{2} \, \pi x\right) + \frac{5}{2} \, \pi \cos\left(\frac{3}{2} \, \pi x\right) \sin\left(\frac{1}{6} \, \pi x\right)}\]

\end{problem}}

%%%%%%%%%%%%%%%%%%%%%%

\latexProblemContent{
\ifVerboseLocation This is Derivative Compute Question 0003. \\ \fi
\begin{problem}

Compute the following derivative:

\input{Derivative-Compute-0003.HELP.tex}

\[\dfrac{d}{dx}\left(5 \, \cos\left(-\frac{1}{3} \, \pi x\right) \cos\left(-2 \, \pi x\right)\right)=\answer{-10 \, \pi \cos\left(\frac{1}{3} \, \pi x\right) \sin\left(2 \, \pi x\right) - \frac{5}{3} \, \pi \cos\left(2 \, \pi x\right) \sin\left(\frac{1}{3} \, \pi x\right)}\]

\end{problem}}

%%%%%%%%%%%%%%%%%%%%%%

\latexProblemContent{
\ifVerboseLocation This is Derivative Compute Question 0003. \\ \fi
\begin{problem}

Compute the following derivative:

\input{Derivative-Compute-0003.HELP.tex}

\[\dfrac{d}{dx}\left(-4 \, \cos\left(\frac{2}{3} \, \pi x\right) \sin\left(3 \, \pi x\right)\right)=\answer{-12 \, \pi \cos\left(3 \, \pi x\right) \cos\left(\frac{2}{3} \, \pi x\right) + \frac{8}{3} \, \pi \sin\left(3 \, \pi x\right) \sin\left(\frac{2}{3} \, \pi x\right)}\]

\end{problem}}

%%%%%%%%%%%%%%%%%%%%%%

\latexProblemContent{
\ifVerboseLocation This is Derivative Compute Question 0003. \\ \fi
\begin{problem}

Compute the following derivative:

\input{Derivative-Compute-0003.HELP.tex}

\[\dfrac{d}{dx}\left(2 \, \tan\left(\frac{1}{3} \, \pi x\right) \tan\left(-3 \, \pi x\right)\right)=\answer{-\frac{2}{3} \, \pi \sec\left(\frac{1}{3} \, \pi x\right)^{2} \tan\left(3 \, \pi x\right) - 6 \, \pi \sec\left(3 \, \pi x\right)^{2} \tan\left(\frac{1}{3} \, \pi x\right)}\]

\end{problem}}

%%%%%%%%%%%%%%%%%%%%%%

\latexProblemContent{
\ifVerboseLocation This is Derivative Compute Question 0003. \\ \fi
\begin{problem}

Compute the following derivative:

\input{Derivative-Compute-0003.HELP.tex}

\[\dfrac{d}{dx}\left(4 \, \sin\left(-\frac{5}{6} \, \pi x\right) \tan\left(\frac{1}{2} \, \pi x\right)\right)=\answer{-2 \, \pi \sec\left(\frac{1}{2} \, \pi x\right)^{2} \sin\left(\frac{5}{6} \, \pi x\right) - \frac{10}{3} \, \pi \cos\left(\frac{5}{6} \, \pi x\right) \tan\left(\frac{1}{2} \, \pi x\right)}\]

\end{problem}}

%%%%%%%%%%%%%%%%%%%%%%

\latexProblemContent{
\ifVerboseLocation This is Derivative Compute Question 0003. \\ \fi
\begin{problem}

Compute the following derivative:

\input{Derivative-Compute-0003.HELP.tex}

\[\dfrac{d}{dx}\left(4 \, \sin\left(-\frac{5}{3} \, \pi x\right) \tan\left(\frac{2}{3} \, \pi x\right)\right)=\answer{-\frac{8}{3} \, \pi \sec\left(\frac{2}{3} \, \pi x\right)^{2} \sin\left(\frac{5}{3} \, \pi x\right) - \frac{20}{3} \, \pi \cos\left(\frac{5}{3} \, \pi x\right) \tan\left(\frac{2}{3} \, \pi x\right)}\]

\end{problem}}

%%%%%%%%%%%%%%%%%%%%%%

\latexProblemContent{
\ifVerboseLocation This is Derivative Compute Question 0003. \\ \fi
\begin{problem}

Compute the following derivative:

\input{Derivative-Compute-0003.HELP.tex}

\[\dfrac{d}{dx}\left(10 \, \tan\left(\frac{5}{2} \, \pi x\right) \tan\left(\frac{3}{2} \, \pi x\right)\right)=\answer{15 \, \pi \sec\left(\frac{3}{2} \, \pi x\right)^{2} \tan\left(\frac{5}{2} \, \pi x\right) + 25 \, \pi \sec\left(\frac{5}{2} \, \pi x\right)^{2} \tan\left(\frac{3}{2} \, \pi x\right)}\]

\end{problem}}

%%%%%%%%%%%%%%%%%%%%%%

\latexProblemContent{
\ifVerboseLocation This is Derivative Compute Question 0003. \\ \fi
\begin{problem}

Compute the following derivative:

\input{Derivative-Compute-0003.HELP.tex}

\[\dfrac{d}{dx}\left(-2 \, \sin\left(5 \, \pi x\right) \tan\left(\pi x\right)\right)=\answer{-2 \, \pi \sec\left(\pi x\right)^{2} \sin\left(5 \, \pi x\right) - 10 \, \pi \cos\left(5 \, \pi x\right) \tan\left(\pi x\right)}\]

\end{problem}}

%%%%%%%%%%%%%%%%%%%%%%

\latexProblemContent{
\ifVerboseLocation This is Derivative Compute Question 0003. \\ \fi
\begin{problem}

Compute the following derivative:

\input{Derivative-Compute-0003.HELP.tex}

\[\dfrac{d}{dx}\left(-16 \, \tan\left(5 \, \pi x\right) \tan\left(\pi x\right)\right)=\answer{-16 \, \pi \sec\left(\pi x\right)^{2} \tan\left(5 \, \pi x\right) - 80 \, \pi \sec\left(5 \, \pi x\right)^{2} \tan\left(\pi x\right)}\]

\end{problem}}

%%%%%%%%%%%%%%%%%%%%%%

\latexProblemContent{
\ifVerboseLocation This is Derivative Compute Question 0003. \\ \fi
\begin{problem}

Compute the following derivative:

\input{Derivative-Compute-0003.HELP.tex}

\[\dfrac{d}{dx}\left(3 \, \sin\left(\frac{5}{3} \, \pi x\right) \sin\left(-\frac{5}{6} \, \pi x\right)\right)=\answer{-\frac{5}{2} \, \pi \cos\left(\frac{5}{6} \, \pi x\right) \sin\left(\frac{5}{3} \, \pi x\right) - 5 \, \pi \cos\left(\frac{5}{3} \, \pi x\right) \sin\left(\frac{5}{6} \, \pi x\right)}\]

\end{problem}}

%%%%%%%%%%%%%%%%%%%%%%

\latexProblemContent{
\ifVerboseLocation This is Derivative Compute Question 0003. \\ \fi
\begin{problem}

Compute the following derivative:

\input{Derivative-Compute-0003.HELP.tex}

\[\dfrac{d}{dx}\left(5 \, \cos\left(\frac{1}{2} \, \pi x\right) \sin\left(-\pi x\right)\right)=\answer{-5 \, \pi \cos\left(\pi x\right) \cos\left(\frac{1}{2} \, \pi x\right) + \frac{5}{2} \, \pi \sin\left(\pi x\right) \sin\left(\frac{1}{2} \, \pi x\right)}\]

\end{problem}}

%%%%%%%%%%%%%%%%%%%%%%

\latexProblemContent{
\ifVerboseLocation This is Derivative Compute Question 0003. \\ \fi
\begin{problem}

Compute the following derivative:

\input{Derivative-Compute-0003.HELP.tex}

\[\dfrac{d}{dx}\left(5 \, \sin\left(\frac{1}{2} \, \pi x\right) \sin\left(-2 \, \pi x\right)\right)=\answer{-\frac{5}{2} \, \pi \cos\left(\frac{1}{2} \, \pi x\right) \sin\left(2 \, \pi x\right) - 10 \, \pi \cos\left(2 \, \pi x\right) \sin\left(\frac{1}{2} \, \pi x\right)}\]

\end{problem}}

%%%%%%%%%%%%%%%%%%%%%%

\latexProblemContent{
\ifVerboseLocation This is Derivative Compute Question 0003. \\ \fi
\begin{problem}

Compute the following derivative:

\input{Derivative-Compute-0003.HELP.tex}

\[\dfrac{d}{dx}\left(-25 \, \cos\left(-3 \, \pi x\right) \tan\left(\frac{2}{3} \, \pi x\right)\right)=\answer{-\frac{50}{3} \, \pi \cos\left(3 \, \pi x\right) \sec\left(\frac{2}{3} \, \pi x\right)^{2} + 75 \, \pi \sin\left(3 \, \pi x\right) \tan\left(\frac{2}{3} \, \pi x\right)}\]

\end{problem}}

%%%%%%%%%%%%%%%%%%%%%%

\latexProblemContent{
\ifVerboseLocation This is Derivative Compute Question 0003. \\ \fi
\begin{problem}

Compute the following derivative:

\input{Derivative-Compute-0003.HELP.tex}

\[\dfrac{d}{dx}\left(10 \, \sin\left(\frac{3}{2} \, \pi x\right) \sin\left(\pi x\right)\right)=\answer{10 \, \pi \cos\left(\pi x\right) \sin\left(\frac{3}{2} \, \pi x\right) + 15 \, \pi \cos\left(\frac{3}{2} \, \pi x\right) \sin\left(\pi x\right)}\]

\end{problem}}

%%%%%%%%%%%%%%%%%%%%%%

\latexProblemContent{
\ifVerboseLocation This is Derivative Compute Question 0003. \\ \fi
\begin{problem}

Compute the following derivative:

\input{Derivative-Compute-0003.HELP.tex}

\[\dfrac{d}{dx}\left(3 \, \cos\left(2 \, \pi x\right) \tan\left(\frac{4}{3} \, \pi x\right)\right)=\answer{4 \, \pi \cos\left(2 \, \pi x\right) \sec\left(\frac{4}{3} \, \pi x\right)^{2} - 6 \, \pi \sin\left(2 \, \pi x\right) \tan\left(\frac{4}{3} \, \pi x\right)}\]

\end{problem}}

%%%%%%%%%%%%%%%%%%%%%%

\latexProblemContent{
\ifVerboseLocation This is Derivative Compute Question 0003. \\ \fi
\begin{problem}

Compute the following derivative:

\input{Derivative-Compute-0003.HELP.tex}

\[\dfrac{d}{dx}\left(-25 \, \cos\left(-3 \, \pi x\right) \sin\left(\frac{2}{3} \, \pi x\right)\right)=\answer{-\frac{50}{3} \, \pi \cos\left(3 \, \pi x\right) \cos\left(\frac{2}{3} \, \pi x\right) + 75 \, \pi \sin\left(3 \, \pi x\right) \sin\left(\frac{2}{3} \, \pi x\right)}\]

\end{problem}}

%%%%%%%%%%%%%%%%%%%%%%

\latexProblemContent{
\ifVerboseLocation This is Derivative Compute Question 0003. \\ \fi
\begin{problem}

Compute the following derivative:

\input{Derivative-Compute-0003.HELP.tex}

\[\dfrac{d}{dx}\left(9 \, \tan\left(-\frac{2}{3} \, \pi x\right) \tan\left(-3 \, \pi x\right)\right)=\answer{6 \, \pi \sec\left(\frac{2}{3} \, \pi x\right)^{2} \tan\left(3 \, \pi x\right) + 27 \, \pi \sec\left(3 \, \pi x\right)^{2} \tan\left(\frac{2}{3} \, \pi x\right)}\]

\end{problem}}

%%%%%%%%%%%%%%%%%%%%%%

\latexProblemContent{
\ifVerboseLocation This is Derivative Compute Question 0003. \\ \fi
\begin{problem}

Compute the following derivative:

\input{Derivative-Compute-0003.HELP.tex}

\[\dfrac{d}{dx}\left(-5 \, \cos\left(\pi x\right) \sin\left(-\pi x\right)\right)=\answer{5 \, \pi \cos\left(\pi x\right)^{2} - 5 \, \pi \sin\left(\pi x\right)^{2}}\]

\end{problem}}

%%%%%%%%%%%%%%%%%%%%%%

\latexProblemContent{
\ifVerboseLocation This is Derivative Compute Question 0003. \\ \fi
\begin{problem}

Compute the following derivative:

\input{Derivative-Compute-0003.HELP.tex}

\[\dfrac{d}{dx}\left(-6 \, \tan\left(4 \, \pi x\right)^{2}\right)=\answer{-48 \, \pi \sec\left(4 \, \pi x\right)^{2} \tan\left(4 \, \pi x\right)}\]

\end{problem}}

%%%%%%%%%%%%%%%%%%%%%%

\latexProblemContent{
\ifVerboseLocation This is Derivative Compute Question 0003. \\ \fi
\begin{problem}

Compute the following derivative:

\input{Derivative-Compute-0003.HELP.tex}

\[\dfrac{d}{dx}\left(-\cos\left(-\frac{5}{2} \, \pi x\right) \sin\left(-\frac{2}{3} \, \pi x\right)\right)=\answer{\frac{2}{3} \, \pi \cos\left(\frac{5}{2} \, \pi x\right) \cos\left(\frac{2}{3} \, \pi x\right) - \frac{5}{2} \, \pi \sin\left(\frac{5}{2} \, \pi x\right) \sin\left(\frac{2}{3} \, \pi x\right)}\]

\end{problem}}

%%%%%%%%%%%%%%%%%%%%%%

\latexProblemContent{
\ifVerboseLocation This is Derivative Compute Question 0003. \\ \fi
\begin{problem}

Compute the following derivative:

\input{Derivative-Compute-0003.HELP.tex}

\[\dfrac{d}{dx}\left(20 \, \cos\left(-\frac{4}{3} \, \pi x\right) \cos\left(-2 \, \pi x\right)\right)=\answer{-40 \, \pi \cos\left(\frac{4}{3} \, \pi x\right) \sin\left(2 \, \pi x\right) - \frac{80}{3} \, \pi \cos\left(2 \, \pi x\right) \sin\left(\frac{4}{3} \, \pi x\right)}\]

\end{problem}}

%%%%%%%%%%%%%%%%%%%%%%

\latexProblemContent{
\ifVerboseLocation This is Derivative Compute Question 0003. \\ \fi
\begin{problem}

Compute the following derivative:

\input{Derivative-Compute-0003.HELP.tex}

\[\dfrac{d}{dx}\left(-25 \, \sin\left(\frac{4}{3} \, \pi x\right) \sin\left(-\frac{4}{3} \, \pi x\right)\right)=\answer{\frac{200}{3} \, \pi \cos\left(\frac{4}{3} \, \pi x\right) \sin\left(\frac{4}{3} \, \pi x\right)}\]

\end{problem}}

%%%%%%%%%%%%%%%%%%%%%%

\latexProblemContent{
\ifVerboseLocation This is Derivative Compute Question 0003. \\ \fi
\begin{problem}

Compute the following derivative:

\input{Derivative-Compute-0003.HELP.tex}

\[\dfrac{d}{dx}\left(-10 \, \cos\left(\pi x\right) \sin\left(\pi x\right)\right)=\answer{-10 \, \pi \cos\left(\pi x\right)^{2} + 10 \, \pi \sin\left(\pi x\right)^{2}}\]

\end{problem}}

%%%%%%%%%%%%%%%%%%%%%%

\latexProblemContent{
\ifVerboseLocation This is Derivative Compute Question 0003. \\ \fi
\begin{problem}

Compute the following derivative:

\input{Derivative-Compute-0003.HELP.tex}

\[\dfrac{d}{dx}\left(-\sin\left(-5 \, \pi x\right) \tan\left(\frac{5}{3} \, \pi x\right)\right)=\answer{\frac{5}{3} \, \pi \sec\left(\frac{5}{3} \, \pi x\right)^{2} \sin\left(5 \, \pi x\right) + 5 \, \pi \cos\left(5 \, \pi x\right) \tan\left(\frac{5}{3} \, \pi x\right)}\]

\end{problem}}

%%%%%%%%%%%%%%%%%%%%%%

\latexProblemContent{
\ifVerboseLocation This is Derivative Compute Question 0003. \\ \fi
\begin{problem}

Compute the following derivative:

\input{Derivative-Compute-0003.HELP.tex}

\[\dfrac{d}{dx}\left(-15 \, \cos\left(\frac{1}{3} \, \pi x\right) \tan\left(-\frac{2}{3} \, \pi x\right)\right)=\answer{10 \, \pi \cos\left(\frac{1}{3} \, \pi x\right) \sec\left(\frac{2}{3} \, \pi x\right)^{2} - 5 \, \pi \sin\left(\frac{1}{3} \, \pi x\right) \tan\left(\frac{2}{3} \, \pi x\right)}\]

\end{problem}}

%%%%%%%%%%%%%%%%%%%%%%

\latexProblemContent{
\ifVerboseLocation This is Derivative Compute Question 0003. \\ \fi
\begin{problem}

Compute the following derivative:

\input{Derivative-Compute-0003.HELP.tex}

\[\dfrac{d}{dx}\left(-25 \, \sin\left(-2 \, \pi x\right) \sin\left(-\frac{5}{2} \, \pi x\right)\right)=\answer{-50 \, \pi \cos\left(2 \, \pi x\right) \sin\left(\frac{5}{2} \, \pi x\right) - \frac{125}{2} \, \pi \cos\left(\frac{5}{2} \, \pi x\right) \sin\left(2 \, \pi x\right)}\]

\end{problem}}

%%%%%%%%%%%%%%%%%%%%%%

\latexProblemContent{
\ifVerboseLocation This is Derivative Compute Question 0003. \\ \fi
\begin{problem}

Compute the following derivative:

\input{Derivative-Compute-0003.HELP.tex}

\[\dfrac{d}{dx}\left(4 \, \cos\left(-\frac{1}{3} \, \pi x\right) \sin\left(-2 \, \pi x\right)\right)=\answer{-8 \, \pi \cos\left(2 \, \pi x\right) \cos\left(\frac{1}{3} \, \pi x\right) + \frac{4}{3} \, \pi \sin\left(2 \, \pi x\right) \sin\left(\frac{1}{3} \, \pi x\right)}\]

\end{problem}}

%%%%%%%%%%%%%%%%%%%%%%

\latexProblemContent{
\ifVerboseLocation This is Derivative Compute Question 0003. \\ \fi
\begin{problem}

Compute the following derivative:

\input{Derivative-Compute-0003.HELP.tex}

\[\dfrac{d}{dx}\left(5 \, \sin\left(\pi x\right) \tan\left(-\frac{5}{3} \, \pi x\right)\right)=\answer{-\frac{25}{3} \, \pi \sec\left(\frac{5}{3} \, \pi x\right)^{2} \sin\left(\pi x\right) - 5 \, \pi \cos\left(\pi x\right) \tan\left(\frac{5}{3} \, \pi x\right)}\]

\end{problem}}

%%%%%%%%%%%%%%%%%%%%%%

\latexProblemContent{
\ifVerboseLocation This is Derivative Compute Question 0003. \\ \fi
\begin{problem}

Compute the following derivative:

\input{Derivative-Compute-0003.HELP.tex}

\[\dfrac{d}{dx}\left(6 \, \sin\left(5 \, \pi x\right) \sin\left(\frac{1}{2} \, \pi x\right)\right)=\answer{3 \, \pi \cos\left(\frac{1}{2} \, \pi x\right) \sin\left(5 \, \pi x\right) + 30 \, \pi \cos\left(5 \, \pi x\right) \sin\left(\frac{1}{2} \, \pi x\right)}\]

\end{problem}}

%%%%%%%%%%%%%%%%%%%%%%

\latexProblemContent{
\ifVerboseLocation This is Derivative Compute Question 0003. \\ \fi
\begin{problem}

Compute the following derivative:

\input{Derivative-Compute-0003.HELP.tex}

\[\dfrac{d}{dx}\left(5 \, \tan\left(\frac{1}{2} \, \pi x\right) \tan\left(-\pi x\right)\right)=\answer{-\frac{5}{2} \, \pi \sec\left(\frac{1}{2} \, \pi x\right)^{2} \tan\left(\pi x\right) - 5 \, \pi \sec\left(\pi x\right)^{2} \tan\left(\frac{1}{2} \, \pi x\right)}\]

\end{problem}}

%%%%%%%%%%%%%%%%%%%%%%

\latexProblemContent{
\ifVerboseLocation This is Derivative Compute Question 0003. \\ \fi
\begin{problem}

Compute the following derivative:

\input{Derivative-Compute-0003.HELP.tex}

\[\dfrac{d}{dx}\left(-12 \, \cos\left(5 \, \pi x\right) \cos\left(-\frac{1}{6} \, \pi x\right)\right)=\answer{60 \, \pi \cos\left(\frac{1}{6} \, \pi x\right) \sin\left(5 \, \pi x\right) + 2 \, \pi \cos\left(5 \, \pi x\right) \sin\left(\frac{1}{6} \, \pi x\right)}\]

\end{problem}}

%%%%%%%%%%%%%%%%%%%%%%

\latexProblemContent{
\ifVerboseLocation This is Derivative Compute Question 0003. \\ \fi
\begin{problem}

Compute the following derivative:

\input{Derivative-Compute-0003.HELP.tex}

\[\dfrac{d}{dx}\left(6 \, \sin\left(\frac{2}{3} \, \pi x\right) \sin\left(-\frac{1}{2} \, \pi x\right)\right)=\answer{-3 \, \pi \cos\left(\frac{1}{2} \, \pi x\right) \sin\left(\frac{2}{3} \, \pi x\right) - 4 \, \pi \cos\left(\frac{2}{3} \, \pi x\right) \sin\left(\frac{1}{2} \, \pi x\right)}\]

\end{problem}}

%%%%%%%%%%%%%%%%%%%%%%

\latexProblemContent{
\ifVerboseLocation This is Derivative Compute Question 0003. \\ \fi
\begin{problem}

Compute the following derivative:

\input{Derivative-Compute-0003.HELP.tex}

\[\dfrac{d}{dx}\left(5 \, \sin\left(4 \, \pi x\right) \tan\left(\frac{5}{3} \, \pi x\right)\right)=\answer{\frac{25}{3} \, \pi \sec\left(\frac{5}{3} \, \pi x\right)^{2} \sin\left(4 \, \pi x\right) + 20 \, \pi \cos\left(4 \, \pi x\right) \tan\left(\frac{5}{3} \, \pi x\right)}\]

\end{problem}}

%%%%%%%%%%%%%%%%%%%%%%

\latexProblemContent{
\ifVerboseLocation This is Derivative Compute Question 0003. \\ \fi
\begin{problem}

Compute the following derivative:

\input{Derivative-Compute-0003.HELP.tex}

\[\dfrac{d}{dx}\left(5 \, \cos\left(\frac{5}{6} \, \pi x\right) \sin\left(\pi x\right)\right)=\answer{5 \, \pi \cos\left(\pi x\right) \cos\left(\frac{5}{6} \, \pi x\right) - \frac{25}{6} \, \pi \sin\left(\pi x\right) \sin\left(\frac{5}{6} \, \pi x\right)}\]

\end{problem}}

%%%%%%%%%%%%%%%%%%%%%%

\latexProblemContent{
\ifVerboseLocation This is Derivative Compute Question 0003. \\ \fi
\begin{problem}

Compute the following derivative:

\input{Derivative-Compute-0003.HELP.tex}

\[\dfrac{d}{dx}\left(-2 \, \cos\left(-4 \, \pi x\right) \sin\left(\pi x\right)\right)=\answer{-2 \, \pi \cos\left(4 \, \pi x\right) \cos\left(\pi x\right) + 8 \, \pi \sin\left(4 \, \pi x\right) \sin\left(\pi x\right)}\]

\end{problem}}

%%%%%%%%%%%%%%%%%%%%%%

\latexProblemContent{
\ifVerboseLocation This is Derivative Compute Question 0003. \\ \fi
\begin{problem}

Compute the following derivative:

\input{Derivative-Compute-0003.HELP.tex}

\[\dfrac{d}{dx}\left(4 \, \tan\left(-\pi x\right) \tan\left(-\frac{4}{3} \, \pi x\right)\right)=\answer{4 \, \pi \sec\left(\pi x\right)^{2} \tan\left(\frac{4}{3} \, \pi x\right) + \frac{16}{3} \, \pi \sec\left(\frac{4}{3} \, \pi x\right)^{2} \tan\left(\pi x\right)}\]

\end{problem}}

%%%%%%%%%%%%%%%%%%%%%%

\latexProblemContent{
\ifVerboseLocation This is Derivative Compute Question 0003. \\ \fi
\begin{problem}

Compute the following derivative:

\input{Derivative-Compute-0003.HELP.tex}

\[\dfrac{d}{dx}\left(3 \, \cos\left(-\frac{4}{3} \, \pi x\right) \tan\left(-3 \, \pi x\right)\right)=\answer{-9 \, \pi \cos\left(\frac{4}{3} \, \pi x\right) \sec\left(3 \, \pi x\right)^{2} + 4 \, \pi \sin\left(\frac{4}{3} \, \pi x\right) \tan\left(3 \, \pi x\right)}\]

\end{problem}}

%%%%%%%%%%%%%%%%%%%%%%

\latexProblemContent{
\ifVerboseLocation This is Derivative Compute Question 0003. \\ \fi
\begin{problem}

Compute the following derivative:

\input{Derivative-Compute-0003.HELP.tex}

\[\dfrac{d}{dx}\left(-16 \, \cos\left(5 \, \pi x\right) \cos\left(\frac{1}{6} \, \pi x\right)\right)=\answer{80 \, \pi \cos\left(\frac{1}{6} \, \pi x\right) \sin\left(5 \, \pi x\right) + \frac{8}{3} \, \pi \cos\left(5 \, \pi x\right) \sin\left(\frac{1}{6} \, \pi x\right)}\]

\end{problem}}

%%%%%%%%%%%%%%%%%%%%%%

\latexProblemContent{
\ifVerboseLocation This is Derivative Compute Question 0003. \\ \fi
\begin{problem}

Compute the following derivative:

\input{Derivative-Compute-0003.HELP.tex}

\[\dfrac{d}{dx}\left(10 \, \cos\left(\frac{1}{6} \, \pi x\right) \tan\left(\frac{1}{2} \, \pi x\right)\right)=\answer{5 \, \pi \cos\left(\frac{1}{6} \, \pi x\right) \sec\left(\frac{1}{2} \, \pi x\right)^{2} - \frac{5}{3} \, \pi \sin\left(\frac{1}{6} \, \pi x\right) \tan\left(\frac{1}{2} \, \pi x\right)}\]

\end{problem}}

%%%%%%%%%%%%%%%%%%%%%%

\latexProblemContent{
\ifVerboseLocation This is Derivative Compute Question 0003. \\ \fi
\begin{problem}

Compute the following derivative:

\input{Derivative-Compute-0003.HELP.tex}

\[\dfrac{d}{dx}\left(15 \, \cos\left(\frac{1}{2} \, \pi x\right) \tan\left(-\frac{5}{3} \, \pi x\right)\right)=\answer{-25 \, \pi \cos\left(\frac{1}{2} \, \pi x\right) \sec\left(\frac{5}{3} \, \pi x\right)^{2} + \frac{15}{2} \, \pi \sin\left(\frac{1}{2} \, \pi x\right) \tan\left(\frac{5}{3} \, \pi x\right)}\]

\end{problem}}

%%%%%%%%%%%%%%%%%%%%%%

\latexProblemContent{
\ifVerboseLocation This is Derivative Compute Question 0003. \\ \fi
\begin{problem}

Compute the following derivative:

\input{Derivative-Compute-0003.HELP.tex}

\[\dfrac{d}{dx}\left(-4 \, \cos\left(-\frac{5}{6} \, \pi x\right) \tan\left(\frac{5}{3} \, \pi x\right)\right)=\answer{-\frac{20}{3} \, \pi \cos\left(\frac{5}{6} \, \pi x\right) \sec\left(\frac{5}{3} \, \pi x\right)^{2} + \frac{10}{3} \, \pi \sin\left(\frac{5}{6} \, \pi x\right) \tan\left(\frac{5}{3} \, \pi x\right)}\]

\end{problem}}

%%%%%%%%%%%%%%%%%%%%%%

\latexProblemContent{
\ifVerboseLocation This is Derivative Compute Question 0003. \\ \fi
\begin{problem}

Compute the following derivative:

\input{Derivative-Compute-0003.HELP.tex}

\[\dfrac{d}{dx}\left(3 \, \tan\left(4 \, \pi x\right) \tan\left(\frac{4}{3} \, \pi x\right)\right)=\answer{4 \, \pi \sec\left(\frac{4}{3} \, \pi x\right)^{2} \tan\left(4 \, \pi x\right) + 12 \, \pi \sec\left(4 \, \pi x\right)^{2} \tan\left(\frac{4}{3} \, \pi x\right)}\]

\end{problem}}

%%%%%%%%%%%%%%%%%%%%%%

\latexProblemContent{
\ifVerboseLocation This is Derivative Compute Question 0003. \\ \fi
\begin{problem}

Compute the following derivative:

\input{Derivative-Compute-0003.HELP.tex}

\[\dfrac{d}{dx}\left(10 \, \cos\left(-\frac{4}{3} \, \pi x\right) \tan\left(2 \, \pi x\right)\right)=\answer{20 \, \pi \cos\left(\frac{4}{3} \, \pi x\right) \sec\left(2 \, \pi x\right)^{2} - \frac{40}{3} \, \pi \sin\left(\frac{4}{3} \, \pi x\right) \tan\left(2 \, \pi x\right)}\]

\end{problem}}

%%%%%%%%%%%%%%%%%%%%%%

\latexProblemContent{
\ifVerboseLocation This is Derivative Compute Question 0003. \\ \fi
\begin{problem}

Compute the following derivative:

\input{Derivative-Compute-0003.HELP.tex}

\[\dfrac{d}{dx}\left(-20 \, \cos\left(2 \, \pi x\right) \sin\left(\pi x\right)\right)=\answer{-20 \, \pi \cos\left(2 \, \pi x\right) \cos\left(\pi x\right) + 40 \, \pi \sin\left(2 \, \pi x\right) \sin\left(\pi x\right)}\]

\end{problem}}

%%%%%%%%%%%%%%%%%%%%%%

\latexProblemContent{
\ifVerboseLocation This is Derivative Compute Question 0003. \\ \fi
\begin{problem}

Compute the following derivative:

\input{Derivative-Compute-0003.HELP.tex}

\[\dfrac{d}{dx}\left(4 \, \sin\left(-5 \, \pi x\right) \tan\left(-\frac{5}{2} \, \pi x\right)\right)=\answer{10 \, \pi \sec\left(\frac{5}{2} \, \pi x\right)^{2} \sin\left(5 \, \pi x\right) + 20 \, \pi \cos\left(5 \, \pi x\right) \tan\left(\frac{5}{2} \, \pi x\right)}\]

\end{problem}}

%%%%%%%%%%%%%%%%%%%%%%

\latexProblemContent{
\ifVerboseLocation This is Derivative Compute Question 0003. \\ \fi
\begin{problem}

Compute the following derivative:

\input{Derivative-Compute-0003.HELP.tex}

\[\dfrac{d}{dx}\left(-20 \, \cos\left(2 \, \pi x\right) \cos\left(-\pi x\right)\right)=\answer{40 \, \pi \cos\left(\pi x\right) \sin\left(2 \, \pi x\right) + 20 \, \pi \cos\left(2 \, \pi x\right) \sin\left(\pi x\right)}\]

\end{problem}}

%%%%%%%%%%%%%%%%%%%%%%

\latexProblemContent{
\ifVerboseLocation This is Derivative Compute Question 0003. \\ \fi
\begin{problem}

Compute the following derivative:

\input{Derivative-Compute-0003.HELP.tex}

\[\dfrac{d}{dx}\left(-15 \, \sin\left(\frac{2}{3} \, \pi x\right) \tan\left(-2 \, \pi x\right)\right)=\answer{30 \, \pi \sec\left(2 \, \pi x\right)^{2} \sin\left(\frac{2}{3} \, \pi x\right) + 10 \, \pi \cos\left(\frac{2}{3} \, \pi x\right) \tan\left(2 \, \pi x\right)}\]

\end{problem}}

%%%%%%%%%%%%%%%%%%%%%%

\latexProblemContent{
\ifVerboseLocation This is Derivative Compute Question 0003. \\ \fi
\begin{problem}

Compute the following derivative:

\input{Derivative-Compute-0003.HELP.tex}

\[\dfrac{d}{dx}\left(-20 \, \cos\left(-3 \, \pi x\right) \tan\left(\pi x\right)\right)=\answer{-20 \, \pi \cos\left(3 \, \pi x\right) \sec\left(\pi x\right)^{2} + 60 \, \pi \sin\left(3 \, \pi x\right) \tan\left(\pi x\right)}\]

\end{problem}}

%%%%%%%%%%%%%%%%%%%%%%

\latexProblemContent{
\ifVerboseLocation This is Derivative Compute Question 0003. \\ \fi
\begin{problem}

Compute the following derivative:

\input{Derivative-Compute-0003.HELP.tex}

\[\dfrac{d}{dx}\left(4 \, \sin\left(-\frac{2}{3} \, \pi x\right) \tan\left(\frac{5}{6} \, \pi x\right)\right)=\answer{-\frac{10}{3} \, \pi \sec\left(\frac{5}{6} \, \pi x\right)^{2} \sin\left(\frac{2}{3} \, \pi x\right) - \frac{8}{3} \, \pi \cos\left(\frac{2}{3} \, \pi x\right) \tan\left(\frac{5}{6} \, \pi x\right)}\]

\end{problem}}

%%%%%%%%%%%%%%%%%%%%%%

\latexProblemContent{
\ifVerboseLocation This is Derivative Compute Question 0003. \\ \fi
\begin{problem}

Compute the following derivative:

\input{Derivative-Compute-0003.HELP.tex}

\[\dfrac{d}{dx}\left(5 \, \cos\left(\pi x\right) \sin\left(-\frac{2}{3} \, \pi x\right)\right)=\answer{-\frac{10}{3} \, \pi \cos\left(\pi x\right) \cos\left(\frac{2}{3} \, \pi x\right) + 5 \, \pi \sin\left(\pi x\right) \sin\left(\frac{2}{3} \, \pi x\right)}\]

\end{problem}}

%%%%%%%%%%%%%%%%%%%%%%

\latexProblemContent{
\ifVerboseLocation This is Derivative Compute Question 0003. \\ \fi
\begin{problem}

Compute the following derivative:

\input{Derivative-Compute-0003.HELP.tex}

\[\dfrac{d}{dx}\left(5 \, \sin\left(-3 \, \pi x\right) \tan\left(-\frac{5}{3} \, \pi x\right)\right)=\answer{\frac{25}{3} \, \pi \sec\left(\frac{5}{3} \, \pi x\right)^{2} \sin\left(3 \, \pi x\right) + 15 \, \pi \cos\left(3 \, \pi x\right) \tan\left(\frac{5}{3} \, \pi x\right)}\]

\end{problem}}

%%%%%%%%%%%%%%%%%%%%%%

\latexProblemContent{
\ifVerboseLocation This is Derivative Compute Question 0003. \\ \fi
\begin{problem}

Compute the following derivative:

\input{Derivative-Compute-0003.HELP.tex}

\[\dfrac{d}{dx}\left(25 \, \cos\left(-\frac{2}{3} \, \pi x\right) \sin\left(2 \, \pi x\right)\right)=\answer{50 \, \pi \cos\left(2 \, \pi x\right) \cos\left(\frac{2}{3} \, \pi x\right) - \frac{50}{3} \, \pi \sin\left(2 \, \pi x\right) \sin\left(\frac{2}{3} \, \pi x\right)}\]

\end{problem}}

%%%%%%%%%%%%%%%%%%%%%%

\latexProblemContent{
\ifVerboseLocation This is Derivative Compute Question 0003. \\ \fi
\begin{problem}

Compute the following derivative:

\input{Derivative-Compute-0003.HELP.tex}

\[\dfrac{d}{dx}\left(-15 \, \sin\left(-\frac{5}{3} \, \pi x\right) \sin\left(-4 \, \pi x\right)\right)=\answer{-25 \, \pi \cos\left(\frac{5}{3} \, \pi x\right) \sin\left(4 \, \pi x\right) - 60 \, \pi \cos\left(4 \, \pi x\right) \sin\left(\frac{5}{3} \, \pi x\right)}\]

\end{problem}}

%%%%%%%%%%%%%%%%%%%%%%

\latexProblemContent{
\ifVerboseLocation This is Derivative Compute Question 0003. \\ \fi
\begin{problem}

Compute the following derivative:

\input{Derivative-Compute-0003.HELP.tex}

\[\dfrac{d}{dx}\left(-6 \, \sin\left(-\pi x\right) \sin\left(-2 \, \pi x\right)\right)=\answer{-6 \, \pi \cos\left(\pi x\right) \sin\left(2 \, \pi x\right) - 12 \, \pi \cos\left(2 \, \pi x\right) \sin\left(\pi x\right)}\]

\end{problem}}

%%%%%%%%%%%%%%%%%%%%%%

\latexProblemContent{
\ifVerboseLocation This is Derivative Compute Question 0003. \\ \fi
\begin{problem}

Compute the following derivative:

\input{Derivative-Compute-0003.HELP.tex}

\[\dfrac{d}{dx}\left(12 \, \cos\left(\pi x\right) \cos\left(-\frac{5}{6} \, \pi x\right)\right)=\answer{-12 \, \pi \cos\left(\frac{5}{6} \, \pi x\right) \sin\left(\pi x\right) - 10 \, \pi \cos\left(\pi x\right) \sin\left(\frac{5}{6} \, \pi x\right)}\]

\end{problem}}

%%%%%%%%%%%%%%%%%%%%%%

\latexProblemContent{
\ifVerboseLocation This is Derivative Compute Question 0003. \\ \fi
\begin{problem}

Compute the following derivative:

\input{Derivative-Compute-0003.HELP.tex}

\[\dfrac{d}{dx}\left(16 \, \cos\left(5 \, \pi x\right) \tan\left(-\frac{2}{3} \, \pi x\right)\right)=\answer{-\frac{32}{3} \, \pi \cos\left(5 \, \pi x\right) \sec\left(\frac{2}{3} \, \pi x\right)^{2} + 80 \, \pi \sin\left(5 \, \pi x\right) \tan\left(\frac{2}{3} \, \pi x\right)}\]

\end{problem}}

%%%%%%%%%%%%%%%%%%%%%%

\latexProblemContent{
\ifVerboseLocation This is Derivative Compute Question 0003. \\ \fi
\begin{problem}

Compute the following derivative:

\input{Derivative-Compute-0003.HELP.tex}

\[\dfrac{d}{dx}\left(-4 \, \cos\left(\frac{2}{3} \, \pi x\right) \tan\left(-5 \, \pi x\right)\right)=\answer{20 \, \pi \cos\left(\frac{2}{3} \, \pi x\right) \sec\left(5 \, \pi x\right)^{2} - \frac{8}{3} \, \pi \sin\left(\frac{2}{3} \, \pi x\right) \tan\left(5 \, \pi x\right)}\]

\end{problem}}

%%%%%%%%%%%%%%%%%%%%%%

\latexProblemContent{
\ifVerboseLocation This is Derivative Compute Question 0003. \\ \fi
\begin{problem}

Compute the following derivative:

\input{Derivative-Compute-0003.HELP.tex}

\[\dfrac{d}{dx}\left(2 \, \cos\left(\frac{1}{2} \, \pi x\right) \cos\left(-3 \, \pi x\right)\right)=\answer{-6 \, \pi \cos\left(\frac{1}{2} \, \pi x\right) \sin\left(3 \, \pi x\right) - \pi \cos\left(3 \, \pi x\right) \sin\left(\frac{1}{2} \, \pi x\right)}\]

\end{problem}}

%%%%%%%%%%%%%%%%%%%%%%

\latexProblemContent{
\ifVerboseLocation This is Derivative Compute Question 0003. \\ \fi
\begin{problem}

Compute the following derivative:

\input{Derivative-Compute-0003.HELP.tex}

\[\dfrac{d}{dx}\left(-5 \, \cos\left(-\frac{5}{3} \, \pi x\right) \sin\left(\frac{1}{2} \, \pi x\right)\right)=\answer{-\frac{5}{2} \, \pi \cos\left(\frac{5}{3} \, \pi x\right) \cos\left(\frac{1}{2} \, \pi x\right) + \frac{25}{3} \, \pi \sin\left(\frac{5}{3} \, \pi x\right) \sin\left(\frac{1}{2} \, \pi x\right)}\]

\end{problem}}

%%%%%%%%%%%%%%%%%%%%%%

\latexProblemContent{
\ifVerboseLocation This is Derivative Compute Question 0003. \\ \fi
\begin{problem}

Compute the following derivative:

\input{Derivative-Compute-0003.HELP.tex}

\[\dfrac{d}{dx}\left(-2 \, \cos\left(-5 \, \pi x\right) \sin\left(\frac{2}{3} \, \pi x\right)\right)=\answer{-\frac{4}{3} \, \pi \cos\left(5 \, \pi x\right) \cos\left(\frac{2}{3} \, \pi x\right) + 10 \, \pi \sin\left(5 \, \pi x\right) \sin\left(\frac{2}{3} \, \pi x\right)}\]

\end{problem}}

%%%%%%%%%%%%%%%%%%%%%%

\latexProblemContent{
\ifVerboseLocation This is Derivative Compute Question 0003. \\ \fi
\begin{problem}

Compute the following derivative:

\input{Derivative-Compute-0003.HELP.tex}

\[\dfrac{d}{dx}\left(12 \, \cos\left(\frac{3}{2} \, \pi x\right) \cos\left(\frac{5}{6} \, \pi x\right)\right)=\answer{-18 \, \pi \cos\left(\frac{5}{6} \, \pi x\right) \sin\left(\frac{3}{2} \, \pi x\right) - 10 \, \pi \cos\left(\frac{3}{2} \, \pi x\right) \sin\left(\frac{5}{6} \, \pi x\right)}\]

\end{problem}}

%%%%%%%%%%%%%%%%%%%%%%

\latexProblemContent{
\ifVerboseLocation This is Derivative Compute Question 0003. \\ \fi
\begin{problem}

Compute the following derivative:

\input{Derivative-Compute-0003.HELP.tex}

\[\dfrac{d}{dx}\left(10 \, \cos\left(\frac{2}{3} \, \pi x\right) \tan\left(\frac{1}{3} \, \pi x\right)\right)=\answer{\frac{10}{3} \, \pi \cos\left(\frac{2}{3} \, \pi x\right) \sec\left(\frac{1}{3} \, \pi x\right)^{2} - \frac{20}{3} \, \pi \sin\left(\frac{2}{3} \, \pi x\right) \tan\left(\frac{1}{3} \, \pi x\right)}\]

\end{problem}}

%%%%%%%%%%%%%%%%%%%%%%

\latexProblemContent{
\ifVerboseLocation This is Derivative Compute Question 0003. \\ \fi
\begin{problem}

Compute the following derivative:

\input{Derivative-Compute-0003.HELP.tex}

\[\dfrac{d}{dx}\left(3 \, \sin\left(\frac{1}{6} \, \pi x\right) \tan\left(4 \, \pi x\right)\right)=\answer{12 \, \pi \sec\left(4 \, \pi x\right)^{2} \sin\left(\frac{1}{6} \, \pi x\right) + \frac{1}{2} \, \pi \cos\left(\frac{1}{6} \, \pi x\right) \tan\left(4 \, \pi x\right)}\]

\end{problem}}

%%%%%%%%%%%%%%%%%%%%%%

\latexProblemContent{
\ifVerboseLocation This is Derivative Compute Question 0003. \\ \fi
\begin{problem}

Compute the following derivative:

\input{Derivative-Compute-0003.HELP.tex}

\[\dfrac{d}{dx}\left(-25 \, \cos\left(-\frac{1}{2} \, \pi x\right) \sin\left(-\frac{1}{2} \, \pi x\right)\right)=\answer{\frac{25}{2} \, \pi \cos\left(\frac{1}{2} \, \pi x\right)^{2} - \frac{25}{2} \, \pi \sin\left(\frac{1}{2} \, \pi x\right)^{2}}\]

\end{problem}}

%%%%%%%%%%%%%%%%%%%%%%

\latexProblemContent{
\ifVerboseLocation This is Derivative Compute Question 0003. \\ \fi
\begin{problem}

Compute the following derivative:

\input{Derivative-Compute-0003.HELP.tex}

\[\dfrac{d}{dx}\left(4 \, \sin\left(\frac{5}{3} \, \pi x\right) \sin\left(\frac{5}{6} \, \pi x\right)\right)=\answer{\frac{10}{3} \, \pi \cos\left(\frac{5}{6} \, \pi x\right) \sin\left(\frac{5}{3} \, \pi x\right) + \frac{20}{3} \, \pi \cos\left(\frac{5}{3} \, \pi x\right) \sin\left(\frac{5}{6} \, \pi x\right)}\]

\end{problem}}

%%%%%%%%%%%%%%%%%%%%%%

\latexProblemContent{
\ifVerboseLocation This is Derivative Compute Question 0003. \\ \fi
\begin{problem}

Compute the following derivative:

\input{Derivative-Compute-0003.HELP.tex}

\[\dfrac{d}{dx}\left(-3 \, \sin\left(-2 \, \pi x\right) \tan\left(-\pi x\right)\right)=\answer{-3 \, \pi \sec\left(\pi x\right)^{2} \sin\left(2 \, \pi x\right) - 6 \, \pi \cos\left(2 \, \pi x\right) \tan\left(\pi x\right)}\]

\end{problem}}

%%%%%%%%%%%%%%%%%%%%%%

\latexProblemContent{
\ifVerboseLocation This is Derivative Compute Question 0003. \\ \fi
\begin{problem}

Compute the following derivative:

\input{Derivative-Compute-0003.HELP.tex}

\[\dfrac{d}{dx}\left(2 \, \cos\left(-\frac{2}{3} \, \pi x\right) \cos\left(-\pi x\right)\right)=\answer{-2 \, \pi \cos\left(\frac{2}{3} \, \pi x\right) \sin\left(\pi x\right) - \frac{4}{3} \, \pi \cos\left(\pi x\right) \sin\left(\frac{2}{3} \, \pi x\right)}\]

\end{problem}}

%%%%%%%%%%%%%%%%%%%%%%

\latexProblemContent{
\ifVerboseLocation This is Derivative Compute Question 0003. \\ \fi
\begin{problem}

Compute the following derivative:

\input{Derivative-Compute-0003.HELP.tex}

\[\dfrac{d}{dx}\left(20 \, \cos\left(-3 \, \pi x\right) \tan\left(\frac{1}{2} \, \pi x\right)\right)=\answer{10 \, \pi \cos\left(3 \, \pi x\right) \sec\left(\frac{1}{2} \, \pi x\right)^{2} - 60 \, \pi \sin\left(3 \, \pi x\right) \tan\left(\frac{1}{2} \, \pi x\right)}\]

\end{problem}}

%%%%%%%%%%%%%%%%%%%%%%

\latexProblemContent{
\ifVerboseLocation This is Derivative Compute Question 0003. \\ \fi
\begin{problem}

Compute the following derivative:

\input{Derivative-Compute-0003.HELP.tex}

\[\dfrac{d}{dx}\left(20 \, \cos\left(-\frac{3}{2} \, \pi x\right) \sin\left(-5 \, \pi x\right)\right)=\answer{-100 \, \pi \cos\left(5 \, \pi x\right) \cos\left(\frac{3}{2} \, \pi x\right) + 30 \, \pi \sin\left(5 \, \pi x\right) \sin\left(\frac{3}{2} \, \pi x\right)}\]

\end{problem}}

%%%%%%%%%%%%%%%%%%%%%%

\latexProblemContent{
\ifVerboseLocation This is Derivative Compute Question 0003. \\ \fi
\begin{problem}

Compute the following derivative:

\input{Derivative-Compute-0003.HELP.tex}

\[\dfrac{d}{dx}\left(-9 \, \cos\left(-\pi x\right) \sin\left(\frac{5}{2} \, \pi x\right)\right)=\answer{-\frac{45}{2} \, \pi \cos\left(\frac{5}{2} \, \pi x\right) \cos\left(\pi x\right) + 9 \, \pi \sin\left(\frac{5}{2} \, \pi x\right) \sin\left(\pi x\right)}\]

\end{problem}}

%%%%%%%%%%%%%%%%%%%%%%

\latexProblemContent{
\ifVerboseLocation This is Derivative Compute Question 0003. \\ \fi
\begin{problem}

Compute the following derivative:

\input{Derivative-Compute-0003.HELP.tex}

\[\dfrac{d}{dx}\left(-15 \, \sin\left(\frac{1}{2} \, \pi x\right) \sin\left(-\pi x\right)\right)=\answer{\frac{15}{2} \, \pi \cos\left(\frac{1}{2} \, \pi x\right) \sin\left(\pi x\right) + 15 \, \pi \cos\left(\pi x\right) \sin\left(\frac{1}{2} \, \pi x\right)}\]

\end{problem}}

%%%%%%%%%%%%%%%%%%%%%%

\latexProblemContent{
\ifVerboseLocation This is Derivative Compute Question 0003. \\ \fi
\begin{problem}

Compute the following derivative:

\input{Derivative-Compute-0003.HELP.tex}

\[\dfrac{d}{dx}\left(6 \, \sin\left(-3 \, \pi x\right) \tan\left(\frac{1}{6} \, \pi x\right)\right)=\answer{-\pi \sec\left(\frac{1}{6} \, \pi x\right)^{2} \sin\left(3 \, \pi x\right) - 18 \, \pi \cos\left(3 \, \pi x\right) \tan\left(\frac{1}{6} \, \pi x\right)}\]

\end{problem}}

%%%%%%%%%%%%%%%%%%%%%%

\latexProblemContent{
\ifVerboseLocation This is Derivative Compute Question 0003. \\ \fi
\begin{problem}

Compute the following derivative:

\input{Derivative-Compute-0003.HELP.tex}

\[\dfrac{d}{dx}\left(\sin\left(5 \, \pi x\right) \tan\left(\frac{2}{3} \, \pi x\right)\right)=\answer{\frac{2}{3} \, \pi \sec\left(\frac{2}{3} \, \pi x\right)^{2} \sin\left(5 \, \pi x\right) + 5 \, \pi \cos\left(5 \, \pi x\right) \tan\left(\frac{2}{3} \, \pi x\right)}\]

\end{problem}}

%%%%%%%%%%%%%%%%%%%%%%

\latexProblemContent{
\ifVerboseLocation This is Derivative Compute Question 0003. \\ \fi
\begin{problem}

Compute the following derivative:

\input{Derivative-Compute-0003.HELP.tex}

\[\dfrac{d}{dx}\left(2 \, \cos\left(-\frac{1}{2} \, \pi x\right) \cos\left(-2 \, \pi x\right)\right)=\answer{-4 \, \pi \cos\left(\frac{1}{2} \, \pi x\right) \sin\left(2 \, \pi x\right) - \pi \cos\left(2 \, \pi x\right) \sin\left(\frac{1}{2} \, \pi x\right)}\]

\end{problem}}\fi             %% end of \ifproblemToFind near top of file
\fi             %% end of \ifquestionCount near top of file
\ProblemFileFooter