% Ans        : ShortAns
% File       : 0004
% Sub        : Rational, Quotient-Rule
% Topic      : Derivative
% Type       : Compute

\ProblemFileHeader{500}
\ifquestionPull
\ifproblemToFind\latexProblemContent{
\ifVerboseLocation This is Derivative Compute Question 0004. \\ \fi
\begin{problem}

Compute the following derivative:

\input{Derivative-Compute-0004.HELP.tex}

\[\dfrac{d}{dx}\left(\frac{x^{3} + 4 \, x^{2} - 11 \, x - 30}{x^{2} - 16}\right)=\answer{\frac{x^{4} - 37 \, x^{2} - 68 \, x + 176}{{\left(x^{2} - 16\right)}^{2}}}\]
\end{problem}}

%%%%%%%%%%%%%%%%%%%%%%

\latexProblemContent{
\ifVerboseLocation This is Derivative Compute Question 0004. \\ \fi
\begin{problem}

Compute the following derivative:

\input{Derivative-Compute-0004.HELP.tex}

\[\dfrac{d}{dx}\left(\frac{x^{2} - 7 \, x + 12}{x - 3}\right)=\answer{\frac{x^{2} - 6 \, x + 9}{{\left(x - 3\right)}^{2}}}\]
\end{problem}}

%%%%%%%%%%%%%%%%%%%%%%

\latexProblemContent{
\ifVerboseLocation This is Derivative Compute Question 0004. \\ \fi
\begin{problem}

Compute the following derivative:

\input{Derivative-Compute-0004.HELP.tex}

\[\dfrac{d}{dx}\left(\frac{x^{2} + 5 \, x + 6}{x^{3} - 14 \, x^{2} + 65 \, x - 100}\right)=\answer{-\frac{x^{4} + 10 \, x^{3} - 117 \, x^{2} + 32 \, x + 890}{{\left(x^{3} - 14 \, x^{2} + 65 \, x - 100\right)}^{2}}}\]
\end{problem}}

%%%%%%%%%%%%%%%%%%%%%%

\latexProblemContent{
\ifVerboseLocation This is Derivative Compute Question 0004. \\ \fi
\begin{problem}

Compute the following derivative:

\input{Derivative-Compute-0004.HELP.tex}

\[\dfrac{d}{dx}\left(\frac{x + 1}{x^{3} - x^{2} - 16 \, x + 16}\right)=\answer{-\frac{2 \, {\left(x^{3} + x^{2} - x - 16\right)}}{{\left(x^{3} - x^{2} - 16 \, x + 16\right)}^{2}}}\]
\end{problem}}

%%%%%%%%%%%%%%%%%%%%%%

\latexProblemContent{
\ifVerboseLocation This is Derivative Compute Question 0004. \\ \fi
\begin{problem}

Compute the following derivative:

\input{Derivative-Compute-0004.HELP.tex}

\[\dfrac{d}{dx}\left(\frac{x^{2} + x - 12}{x + 3}\right)=\answer{\frac{x^{2} + 6 \, x + 15}{{\left(x + 3\right)}^{2}}}\]
\end{problem}}

%%%%%%%%%%%%%%%%%%%%%%

\latexProblemContent{
\ifVerboseLocation This is Derivative Compute Question 0004. \\ \fi
\begin{problem}

Compute the following derivative:

\input{Derivative-Compute-0004.HELP.tex}

\[\dfrac{d}{dx}\left(\frac{x - 5}{x + 3}\right)=\answer{\frac{8}{{\left(x + 3\right)}^{2}}}\]
\end{problem}}

%%%%%%%%%%%%%%%%%%%%%%

\latexProblemContent{
\ifVerboseLocation This is Derivative Compute Question 0004. \\ \fi
\begin{problem}

Compute the following derivative:

\input{Derivative-Compute-0004.HELP.tex}

\[\dfrac{d}{dx}\left(\frac{x - 1}{x^{3} + 2 \, x^{2} - 23 \, x - 60}\right)=\answer{-\frac{2 \, x^{3} - x^{2} - 4 \, x + 83}{{\left(x^{3} + 2 \, x^{2} - 23 \, x - 60\right)}^{2}}}\]
\end{problem}}

%%%%%%%%%%%%%%%%%%%%%%

\latexProblemContent{
\ifVerboseLocation This is Derivative Compute Question 0004. \\ \fi
\begin{problem}

Compute the following derivative:

\input{Derivative-Compute-0004.HELP.tex}

\[\dfrac{d}{dx}\left(\frac{x^{2} - x - 20}{x^{3} - 8 \, x^{2} + 11 \, x + 20}\right)=\answer{-\frac{x^{4} - 2 \, x^{3} - 63 \, x^{2} + 280 \, x - 200}{{\left(x^{3} - 8 \, x^{2} + 11 \, x + 20\right)}^{2}}}\]
\end{problem}}

%%%%%%%%%%%%%%%%%%%%%%

\latexProblemContent{
\ifVerboseLocation This is Derivative Compute Question 0004. \\ \fi
\begin{problem}

Compute the following derivative:

\input{Derivative-Compute-0004.HELP.tex}

\[\dfrac{d}{dx}\left(\frac{1}{x^{2} - 3 \, x - 10}\right)=\answer{-\frac{2 \, x - 3}{{\left(x^{2} - 3 \, x - 10\right)}^{2}}}\]
\end{problem}}

%%%%%%%%%%%%%%%%%%%%%%

\latexProblemContent{
\ifVerboseLocation This is Derivative Compute Question 0004. \\ \fi
\begin{problem}

Compute the following derivative:

\input{Derivative-Compute-0004.HELP.tex}

\[\dfrac{d}{dx}\left(\frac{x^{3} - 19 \, x + 30}{x^{2} + 6 \, x + 5}\right)=\answer{\frac{x^{4} + 12 \, x^{3} + 34 \, x^{2} - 60 \, x - 275}{{\left(x^{2} + 6 \, x + 5\right)}^{2}}}\]
\end{problem}}

%%%%%%%%%%%%%%%%%%%%%%

\latexProblemContent{
\ifVerboseLocation This is Derivative Compute Question 0004. \\ \fi
\begin{problem}

Compute the following derivative:

\input{Derivative-Compute-0004.HELP.tex}

\[\dfrac{d}{dx}\left(\frac{1}{x^{3} + 2 \, x^{2} - 25 \, x - 50}\right)=\answer{-\frac{3 \, x^{2} + 4 \, x - 25}{{\left(x^{3} + 2 \, x^{2} - 25 \, x - 50\right)}^{2}}}\]
\end{problem}}

%%%%%%%%%%%%%%%%%%%%%%

\latexProblemContent{
\ifVerboseLocation This is Derivative Compute Question 0004. \\ \fi
\begin{problem}

Compute the following derivative:

\input{Derivative-Compute-0004.HELP.tex}

\[\dfrac{d}{dx}\left(\frac{x - 5}{x^{2} + 4 \, x - 5}\right)=\answer{-\frac{x^{2} - 10 \, x - 15}{{\left(x^{2} + 4 \, x - 5\right)}^{2}}}\]
\end{problem}}

%%%%%%%%%%%%%%%%%%%%%%

\latexProblemContent{
\ifVerboseLocation This is Derivative Compute Question 0004. \\ \fi
\begin{problem}

Compute the following derivative:

\input{Derivative-Compute-0004.HELP.tex}

\[\dfrac{d}{dx}\left(\frac{x^{2} - 3 \, x - 10}{x^{3} - 6 \, x^{2} - 15 \, x + 100}\right)=\answer{-\frac{x^{4} - 6 \, x^{3} + 3 \, x^{2} - 80 \, x + 450}{{\left(x^{3} - 6 \, x^{2} - 15 \, x + 100\right)}^{2}}}\]
\end{problem}}

%%%%%%%%%%%%%%%%%%%%%%

\latexProblemContent{
\ifVerboseLocation This is Derivative Compute Question 0004. \\ \fi
\begin{problem}

Compute the following derivative:

\input{Derivative-Compute-0004.HELP.tex}

\[\dfrac{d}{dx}\left(\frac{x^{2} - 5 \, x + 4}{x^{3} + 2 \, x^{2} - 4 \, x - 8}\right)=\answer{-\frac{x^{4} - 10 \, x^{3} + 6 \, x^{2} + 32 \, x - 56}{{\left(x^{3} + 2 \, x^{2} - 4 \, x - 8\right)}^{2}}}\]
\end{problem}}

%%%%%%%%%%%%%%%%%%%%%%

\latexProblemContent{
\ifVerboseLocation This is Derivative Compute Question 0004. \\ \fi
\begin{problem}

Compute the following derivative:

\input{Derivative-Compute-0004.HELP.tex}

\[\dfrac{d}{dx}\left(\frac{x - 4}{x + 2}\right)=\answer{\frac{6}{{\left(x + 2\right)}^{2}}}\]
\end{problem}}

%%%%%%%%%%%%%%%%%%%%%%

\latexProblemContent{
\ifVerboseLocation This is Derivative Compute Question 0004. \\ \fi
\begin{problem}

Compute the following derivative:

\input{Derivative-Compute-0004.HELP.tex}

\[\dfrac{d}{dx}\left(\frac{x^{2} - 3 \, x + 2}{x^{3} + 12 \, x^{2} + 45 \, x + 50}\right)=\answer{-\frac{x^{4} - 6 \, x^{3} - 75 \, x^{2} - 52 \, x + 240}{{\left(x^{3} + 12 \, x^{2} + 45 \, x + 50\right)}^{2}}}\]
\end{problem}}

%%%%%%%%%%%%%%%%%%%%%%

\latexProblemContent{
\ifVerboseLocation This is Derivative Compute Question 0004. \\ \fi
\begin{problem}

Compute the following derivative:

\input{Derivative-Compute-0004.HELP.tex}

\[\dfrac{d}{dx}\left(\frac{x + 5}{x^{3} + 5 \, x^{2} - 8 \, x - 48}\right)=\answer{-\frac{2 \, {\left(x^{3} + 10 \, x^{2} + 25 \, x + 4\right)}}{{\left(x^{3} + 5 \, x^{2} - 8 \, x - 48\right)}^{2}}}\]
\end{problem}}

%%%%%%%%%%%%%%%%%%%%%%

\latexProblemContent{
\ifVerboseLocation This is Derivative Compute Question 0004. \\ \fi
\begin{problem}

Compute the following derivative:

\input{Derivative-Compute-0004.HELP.tex}

\[\dfrac{d}{dx}\left(\frac{x^{3} - x^{2} - 9 \, x + 9}{x^{3} + 8 \, x^{2} + 20 \, x + 16}\right)=\answer{\frac{9 \, x^{4} + 58 \, x^{3} + 73 \, x^{2} - 176 \, x - 324}{{\left(x^{3} + 8 \, x^{2} + 20 \, x + 16\right)}^{2}}}\]
\end{problem}}

%%%%%%%%%%%%%%%%%%%%%%

\latexProblemContent{
\ifVerboseLocation This is Derivative Compute Question 0004. \\ \fi
\begin{problem}

Compute the following derivative:

\input{Derivative-Compute-0004.HELP.tex}

\[\dfrac{d}{dx}\left(\frac{x + 2}{x^{3} - x^{2} - 9 \, x + 9}\right)=\answer{-\frac{2 \, x^{3} + 5 \, x^{2} - 4 \, x - 27}{{\left(x^{3} - x^{2} - 9 \, x + 9\right)}^{2}}}\]
\end{problem}}

%%%%%%%%%%%%%%%%%%%%%%

\latexProblemContent{
\ifVerboseLocation This is Derivative Compute Question 0004. \\ \fi
\begin{problem}

Compute the following derivative:

\input{Derivative-Compute-0004.HELP.tex}

\[\dfrac{d}{dx}\left(\frac{x^{2} - 1}{x - 2}\right)=\answer{\frac{x^{2} - 4 \, x + 1}{{\left(x - 2\right)}^{2}}}\]
\end{problem}}

%%%%%%%%%%%%%%%%%%%%%%

\latexProblemContent{
\ifVerboseLocation This is Derivative Compute Question 0004. \\ \fi
\begin{problem}

Compute the following derivative:

\input{Derivative-Compute-0004.HELP.tex}

\[\dfrac{d}{dx}\left(\frac{x + 2}{x^{2} - 5 \, x + 4}\right)=\answer{-\frac{x^{2} + 4 \, x - 14}{{\left(x^{2} - 5 \, x + 4\right)}^{2}}}\]
\end{problem}}

%%%%%%%%%%%%%%%%%%%%%%

\latexProblemContent{
\ifVerboseLocation This is Derivative Compute Question 0004. \\ \fi
\begin{problem}

Compute the following derivative:

\input{Derivative-Compute-0004.HELP.tex}

\[\dfrac{d}{dx}\left(\frac{x - 3}{x^{3} - 5 \, x^{2} + 2 \, x + 8}\right)=\answer{-\frac{2 \, {\left(x^{3} - 7 \, x^{2} + 15 \, x - 7\right)}}{{\left(x^{3} - 5 \, x^{2} + 2 \, x + 8\right)}^{2}}}\]
\end{problem}}

%%%%%%%%%%%%%%%%%%%%%%

\latexProblemContent{
\ifVerboseLocation This is Derivative Compute Question 0004. \\ \fi
\begin{problem}

Compute the following derivative:

\input{Derivative-Compute-0004.HELP.tex}

\[\dfrac{d}{dx}\left(\frac{x^{3} + 5 \, x^{2} + 8 \, x + 4}{x - 5}\right)=\answer{\frac{2 \, {\left(x^{3} - 5 \, x^{2} - 25 \, x - 22\right)}}{{\left(x - 5\right)}^{2}}}\]
\end{problem}}

%%%%%%%%%%%%%%%%%%%%%%

\latexProblemContent{
\ifVerboseLocation This is Derivative Compute Question 0004. \\ \fi
\begin{problem}

Compute the following derivative:

\input{Derivative-Compute-0004.HELP.tex}

\[\dfrac{d}{dx}\left(\frac{x^{2} - 7 \, x + 12}{x^{2} - 7 \, x + 10}\right)=\answer{-\frac{2 \, {\left(2 \, x - 7\right)}}{{\left(x^{2} - 7 \, x + 10\right)}^{2}}}\]
\end{problem}}

%%%%%%%%%%%%%%%%%%%%%%

\latexProblemContent{
\ifVerboseLocation This is Derivative Compute Question 0004. \\ \fi
\begin{problem}

Compute the following derivative:

\input{Derivative-Compute-0004.HELP.tex}

\[\dfrac{d}{dx}\left(\frac{x + 5}{x + 4}\right)=\answer{-\frac{1}{{\left(x + 4\right)}^{2}}}\]
\end{problem}}

%%%%%%%%%%%%%%%%%%%%%%

\latexProblemContent{
\ifVerboseLocation This is Derivative Compute Question 0004. \\ \fi
\begin{problem}

Compute the following derivative:

\input{Derivative-Compute-0004.HELP.tex}

\[\dfrac{d}{dx}\left(\frac{x - 3}{x^{2} + 7 \, x + 10}\right)=\answer{-\frac{x^{2} - 6 \, x - 31}{{\left(x^{2} + 7 \, x + 10\right)}^{2}}}\]
\end{problem}}

%%%%%%%%%%%%%%%%%%%%%%

\latexProblemContent{
\ifVerboseLocation This is Derivative Compute Question 0004. \\ \fi
\begin{problem}

Compute the following derivative:

\input{Derivative-Compute-0004.HELP.tex}

\[\dfrac{d}{dx}\left(\frac{x^{2} - 9 \, x + 20}{x^{3} + 6 \, x^{2} - x - 30}\right)=\answer{-\frac{x^{4} - 18 \, x^{3} + 7 \, x^{2} + 300 \, x - 290}{{\left(x^{3} + 6 \, x^{2} - x - 30\right)}^{2}}}\]
\end{problem}}

%%%%%%%%%%%%%%%%%%%%%%

\latexProblemContent{
\ifVerboseLocation This is Derivative Compute Question 0004. \\ \fi
\begin{problem}

Compute the following derivative:

\input{Derivative-Compute-0004.HELP.tex}

\[\dfrac{d}{dx}\left(\frac{1}{x^{3} - 3 \, x^{2} - 9 \, x + 27}\right)=\answer{-\frac{3 \, {\left(x^{2} - 2 \, x - 3\right)}}{{\left(x^{3} - 3 \, x^{2} - 9 \, x + 27\right)}^{2}}}\]
\end{problem}}

%%%%%%%%%%%%%%%%%%%%%%

\latexProblemContent{
\ifVerboseLocation This is Derivative Compute Question 0004. \\ \fi
\begin{problem}

Compute the following derivative:

\input{Derivative-Compute-0004.HELP.tex}

\[\dfrac{d}{dx}\left(\frac{x + 3}{x^{2} - 4 \, x + 3}\right)=\answer{-\frac{x^{2} + 6 \, x - 15}{{\left(x^{2} - 4 \, x + 3\right)}^{2}}}\]
\end{problem}}

%%%%%%%%%%%%%%%%%%%%%%

\latexProblemContent{
\ifVerboseLocation This is Derivative Compute Question 0004. \\ \fi
\begin{problem}

Compute the following derivative:

\input{Derivative-Compute-0004.HELP.tex}

\[\dfrac{d}{dx}\left(\frac{1}{x^{2} + 8 \, x + 16}\right)=\answer{-\frac{2 \, {\left(x + 4\right)}}{{\left(x^{2} + 8 \, x + 16\right)}^{2}}}\]
\end{problem}}

%%%%%%%%%%%%%%%%%%%%%%

\latexProblemContent{
\ifVerboseLocation This is Derivative Compute Question 0004. \\ \fi
\begin{problem}

Compute the following derivative:

\input{Derivative-Compute-0004.HELP.tex}

\[\dfrac{d}{dx}\left(\frac{x + 3}{x^{3} + 5 \, x^{2} - x - 5}\right)=\answer{-\frac{2 \, {\left(x^{3} + 7 \, x^{2} + 15 \, x + 1\right)}}{{\left(x^{3} + 5 \, x^{2} - x - 5\right)}^{2}}}\]
\end{problem}}

%%%%%%%%%%%%%%%%%%%%%%

\latexProblemContent{
\ifVerboseLocation This is Derivative Compute Question 0004. \\ \fi
\begin{problem}

Compute the following derivative:

\input{Derivative-Compute-0004.HELP.tex}

\[\dfrac{d}{dx}\left(\frac{x + 3}{x - 4}\right)=\answer{-\frac{7}{{\left(x - 4\right)}^{2}}}\]
\end{problem}}

%%%%%%%%%%%%%%%%%%%%%%

\latexProblemContent{
\ifVerboseLocation This is Derivative Compute Question 0004. \\ \fi
\begin{problem}

Compute the following derivative:

\input{Derivative-Compute-0004.HELP.tex}

\[\dfrac{d}{dx}\left(\frac{x^{3} + 2 \, x^{2} - 19 \, x - 20}{x^{3} + 8 \, x^{2} + 17 \, x + 10}\right)=\answer{\frac{6 \, {\left(x^{4} + 12 \, x^{3} + 46 \, x^{2} + 60 \, x + 25\right)}}{{\left(x^{3} + 8 \, x^{2} + 17 \, x + 10\right)}^{2}}}\]
\end{problem}}

%%%%%%%%%%%%%%%%%%%%%%

\latexProblemContent{
\ifVerboseLocation This is Derivative Compute Question 0004. \\ \fi
\begin{problem}

Compute the following derivative:

\input{Derivative-Compute-0004.HELP.tex}

\[\dfrac{d}{dx}\left(\frac{x^{3} - 6 \, x^{2} + 5 \, x + 12}{x^{3} + 9 \, x^{2} + 26 \, x + 24}\right)=\answer{\frac{3 \, {\left(5 \, x^{4} + 14 \, x^{3} - 55 \, x^{2} - 168 \, x - 64\right)}}{{\left(x^{3} + 9 \, x^{2} + 26 \, x + 24\right)}^{2}}}\]
\end{problem}}

%%%%%%%%%%%%%%%%%%%%%%

\latexProblemContent{
\ifVerboseLocation This is Derivative Compute Question 0004. \\ \fi
\begin{problem}

Compute the following derivative:

\input{Derivative-Compute-0004.HELP.tex}

\[\dfrac{d}{dx}\left(\frac{x^{2} - 6 \, x + 8}{x - 3}\right)=\answer{\frac{x^{2} - 6 \, x + 10}{{\left(x - 3\right)}^{2}}}\]
\end{problem}}

%%%%%%%%%%%%%%%%%%%%%%

\latexProblemContent{
\ifVerboseLocation This is Derivative Compute Question 0004. \\ \fi
\begin{problem}

Compute the following derivative:

\input{Derivative-Compute-0004.HELP.tex}

\[\dfrac{d}{dx}\left(\frac{x^{3} + 8 \, x^{2} + 11 \, x - 20}{x^{3} - x^{2} - x + 1}\right)=\answer{-\frac{3 \, {\left(3 \, x^{4} + 8 \, x^{3} - 22 \, x^{2} + 8 \, x + 3\right)}}{{\left(x^{3} - x^{2} - x + 1\right)}^{2}}}\]
\end{problem}}

%%%%%%%%%%%%%%%%%%%%%%

\latexProblemContent{
\ifVerboseLocation This is Derivative Compute Question 0004. \\ \fi
\begin{problem}

Compute the following derivative:

\input{Derivative-Compute-0004.HELP.tex}

\[\dfrac{d}{dx}\left(\frac{1}{x^{3} - 2 \, x^{2} - 13 \, x - 10}\right)=\answer{-\frac{3 \, x^{2} - 4 \, x - 13}{{\left(x^{3} - 2 \, x^{2} - 13 \, x - 10\right)}^{2}}}\]
\end{problem}}

%%%%%%%%%%%%%%%%%%%%%%

\latexProblemContent{
\ifVerboseLocation This is Derivative Compute Question 0004. \\ \fi
\begin{problem}

Compute the following derivative:

\input{Derivative-Compute-0004.HELP.tex}

\[\dfrac{d}{dx}\left(\frac{x - 5}{x^{3} + 11 \, x^{2} + 40 \, x + 48}\right)=\answer{-\frac{2 \, {\left(x^{3} - 2 \, x^{2} - 55 \, x - 124\right)}}{{\left(x^{3} + 11 \, x^{2} + 40 \, x + 48\right)}^{2}}}\]
\end{problem}}

%%%%%%%%%%%%%%%%%%%%%%

\latexProblemContent{
\ifVerboseLocation This is Derivative Compute Question 0004. \\ \fi
\begin{problem}

Compute the following derivative:

\input{Derivative-Compute-0004.HELP.tex}

\[\dfrac{d}{dx}\left(\frac{x + 4}{x^{3} - 4 \, x^{2} - 9 \, x + 36}\right)=\answer{-\frac{2 \, {\left(x^{3} + 4 \, x^{2} - 16 \, x - 36\right)}}{{\left(x^{3} - 4 \, x^{2} - 9 \, x + 36\right)}^{2}}}\]
\end{problem}}

%%%%%%%%%%%%%%%%%%%%%%

\latexProblemContent{
\ifVerboseLocation This is Derivative Compute Question 0004. \\ \fi
\begin{problem}

Compute the following derivative:

\input{Derivative-Compute-0004.HELP.tex}

\[\dfrac{d}{dx}\left(\frac{x^{2} + 5 \, x + 4}{x^{2} - 4}\right)=\answer{-\frac{5 \, x^{2} + 16 \, x + 20}{{\left(x^{2} - 4\right)}^{2}}}\]
\end{problem}}

%%%%%%%%%%%%%%%%%%%%%%

\latexProblemContent{
\ifVerboseLocation This is Derivative Compute Question 0004. \\ \fi
\begin{problem}

Compute the following derivative:

\input{Derivative-Compute-0004.HELP.tex}

\[\dfrac{d}{dx}\left(\frac{x^{2} - 3 \, x - 10}{x^{2} + 3 \, x - 4}\right)=\answer{\frac{6 \, {\left(x^{2} + 2 \, x + 7\right)}}{{\left(x^{2} + 3 \, x - 4\right)}^{2}}}\]
\end{problem}}

%%%%%%%%%%%%%%%%%%%%%%

\latexProblemContent{
\ifVerboseLocation This is Derivative Compute Question 0004. \\ \fi
\begin{problem}

Compute the following derivative:

\input{Derivative-Compute-0004.HELP.tex}

\[\dfrac{d}{dx}\left(\frac{x + 2}{x + 3}\right)=\answer{\frac{1}{{\left(x + 3\right)}^{2}}}\]
\end{problem}}

%%%%%%%%%%%%%%%%%%%%%%

\latexProblemContent{
\ifVerboseLocation This is Derivative Compute Question 0004. \\ \fi
\begin{problem}

Compute the following derivative:

\input{Derivative-Compute-0004.HELP.tex}

\[\dfrac{d}{dx}\left(\frac{x - 4}{x^{3} + 11 \, x^{2} + 35 \, x + 25}\right)=\answer{-\frac{2 \, x^{3} - x^{2} - 88 \, x - 165}{{\left(x^{3} + 11 \, x^{2} + 35 \, x + 25\right)}^{2}}}\]
\end{problem}}

%%%%%%%%%%%%%%%%%%%%%%

\latexProblemContent{
\ifVerboseLocation This is Derivative Compute Question 0004. \\ \fi
\begin{problem}

Compute the following derivative:

\input{Derivative-Compute-0004.HELP.tex}

\[\dfrac{d}{dx}\left(\frac{x - 1}{x^{2} - x - 2}\right)=\answer{-\frac{x^{2} - 2 \, x + 3}{{\left(x^{2} - x - 2\right)}^{2}}}\]
\end{problem}}

%%%%%%%%%%%%%%%%%%%%%%

\latexProblemContent{
\ifVerboseLocation This is Derivative Compute Question 0004. \\ \fi
\begin{problem}

Compute the following derivative:

\input{Derivative-Compute-0004.HELP.tex}

\[\dfrac{d}{dx}\left(\frac{x^{2} - 9}{x + 5}\right)=\answer{\frac{x^{2} + 10 \, x + 9}{{\left(x + 5\right)}^{2}}}\]
\end{problem}}

%%%%%%%%%%%%%%%%%%%%%%

\latexProblemContent{
\ifVerboseLocation This is Derivative Compute Question 0004. \\ \fi
\begin{problem}

Compute the following derivative:

\input{Derivative-Compute-0004.HELP.tex}

\[\dfrac{d}{dx}\left(\frac{x + 3}{x^{2} - 5 \, x + 6}\right)=\answer{-\frac{x^{2} + 6 \, x - 21}{{\left(x^{2} - 5 \, x + 6\right)}^{2}}}\]
\end{problem}}

%%%%%%%%%%%%%%%%%%%%%%

\latexProblemContent{
\ifVerboseLocation This is Derivative Compute Question 0004. \\ \fi
\begin{problem}

Compute the following derivative:

\input{Derivative-Compute-0004.HELP.tex}

\[\dfrac{d}{dx}\left(\frac{x^{3} - 5 \, x^{2} - 9 \, x + 45}{x^{3} + 12 \, x^{2} + 45 \, x + 50}\right)=\answer{\frac{17 \, x^{4} + 108 \, x^{3} - 102 \, x^{2} - 1580 \, x - 2475}{{\left(x^{3} + 12 \, x^{2} + 45 \, x + 50\right)}^{2}}}\]
\end{problem}}

%%%%%%%%%%%%%%%%%%%%%%

\latexProblemContent{
\ifVerboseLocation This is Derivative Compute Question 0004. \\ \fi
\begin{problem}

Compute the following derivative:

\input{Derivative-Compute-0004.HELP.tex}

\[\dfrac{d}{dx}\left(\frac{x^{3} - 2 \, x^{2} - 5 \, x + 6}{x^{3} - 3 \, x + 2}\right)=\answer{\frac{2 \, {\left(x^{4} + 2 \, x^{3} - 3 \, x^{2} - 4 \, x + 4\right)}}{{\left(x^{3} - 3 \, x + 2\right)}^{2}}}\]
\end{problem}}

%%%%%%%%%%%%%%%%%%%%%%

\latexProblemContent{
\ifVerboseLocation This is Derivative Compute Question 0004. \\ \fi
\begin{problem}

Compute the following derivative:

\input{Derivative-Compute-0004.HELP.tex}

\[\dfrac{d}{dx}\left(\frac{x^{3} + 6 \, x^{2} + 5 \, x - 12}{x^{3} - 3 \, x^{2} - 9 \, x + 27}\right)=\answer{-\frac{9 \, x^{4} + 28 \, x^{3} - 78 \, x^{2} - 252 \, x - 27}{{\left(x^{3} - 3 \, x^{2} - 9 \, x + 27\right)}^{2}}}\]
\end{problem}}

%%%%%%%%%%%%%%%%%%%%%%

\latexProblemContent{
\ifVerboseLocation This is Derivative Compute Question 0004. \\ \fi
\begin{problem}

Compute the following derivative:

\input{Derivative-Compute-0004.HELP.tex}

\[\dfrac{d}{dx}\left(\frac{x - 1}{x^{2} + 4 \, x + 4}\right)=\answer{-\frac{x^{2} - 2 \, x - 8}{{\left(x^{2} + 4 \, x + 4\right)}^{2}}}\]
\end{problem}}

%%%%%%%%%%%%%%%%%%%%%%

\latexProblemContent{
\ifVerboseLocation This is Derivative Compute Question 0004. \\ \fi
\begin{problem}

Compute the following derivative:

\input{Derivative-Compute-0004.HELP.tex}

\[\dfrac{d}{dx}\left(\frac{x^{3} + 5 \, x^{2} - 2 \, x - 24}{x^{2} + 2 \, x - 15}\right)=\answer{\frac{x^{4} + 4 \, x^{3} - 33 \, x^{2} - 102 \, x + 78}{{\left(x^{2} + 2 \, x - 15\right)}^{2}}}\]
\end{problem}}

%%%%%%%%%%%%%%%%%%%%%%

\latexProblemContent{
\ifVerboseLocation This is Derivative Compute Question 0004. \\ \fi
\begin{problem}

Compute the following derivative:

\input{Derivative-Compute-0004.HELP.tex}

\[\dfrac{d}{dx}\left(\frac{x^{3} + x^{2} - 4 \, x - 4}{x + 4}\right)=\answer{\frac{2 \, x^{3} + 13 \, x^{2} + 8 \, x - 12}{{\left(x + 4\right)}^{2}}}\]
\end{problem}}

%%%%%%%%%%%%%%%%%%%%%%

\latexProblemContent{
\ifVerboseLocation This is Derivative Compute Question 0004. \\ \fi
\begin{problem}

Compute the following derivative:

\input{Derivative-Compute-0004.HELP.tex}

\[\dfrac{d}{dx}\left(\frac{x^{3} - x^{2} - 16 \, x + 16}{x^{2} - 9}\right)=\answer{\frac{x^{4} - 11 \, x^{2} - 14 \, x + 144}{{\left(x^{2} - 9\right)}^{2}}}\]
\end{problem}}

%%%%%%%%%%%%%%%%%%%%%%

\latexProblemContent{
\ifVerboseLocation This is Derivative Compute Question 0004. \\ \fi
\begin{problem}

Compute the following derivative:

\input{Derivative-Compute-0004.HELP.tex}

\[\dfrac{d}{dx}\left(\frac{x^{2} - 6 \, x + 8}{x^{2} - 6 \, x + 5}\right)=\answer{-\frac{6 \, {\left(x - 3\right)}}{{\left(x^{2} - 6 \, x + 5\right)}^{2}}}\]
\end{problem}}

%%%%%%%%%%%%%%%%%%%%%%

\latexProblemContent{
\ifVerboseLocation This is Derivative Compute Question 0004. \\ \fi
\begin{problem}

Compute the following derivative:

\input{Derivative-Compute-0004.HELP.tex}

\[\dfrac{d}{dx}\left(\frac{x^{3} - 7 \, x + 6}{x^{3} + 2 \, x^{2} - 5 \, x - 6}\right)=\answer{\frac{2 \, {\left(x^{4} + 2 \, x^{3} - 11 \, x^{2} - 12 \, x + 36\right)}}{{\left(x^{3} + 2 \, x^{2} - 5 \, x - 6\right)}^{2}}}\]
\end{problem}}

%%%%%%%%%%%%%%%%%%%%%%

\latexProblemContent{
\ifVerboseLocation This is Derivative Compute Question 0004. \\ \fi
\begin{problem}

Compute the following derivative:

\input{Derivative-Compute-0004.HELP.tex}

\[\dfrac{d}{dx}\left(\frac{x^{2} + 2 \, x - 15}{x^{3} + 4 \, x^{2} - 4 \, x - 16}\right)=\answer{-\frac{x^{4} + 4 \, x^{3} - 33 \, x^{2} - 88 \, x + 92}{{\left(x^{3} + 4 \, x^{2} - 4 \, x - 16\right)}^{2}}}\]
\end{problem}}

%%%%%%%%%%%%%%%%%%%%%%

\latexProblemContent{
\ifVerboseLocation This is Derivative Compute Question 0004. \\ \fi
\begin{problem}

Compute the following derivative:

\input{Derivative-Compute-0004.HELP.tex}

\[\dfrac{d}{dx}\left(\frac{x^{2} - 2 \, x - 3}{x^{3} - 7 \, x + 6}\right)=\answer{-\frac{x^{4} - 4 \, x^{3} - 2 \, x^{2} - 12 \, x + 33}{{\left(x^{3} - 7 \, x + 6\right)}^{2}}}\]
\end{problem}}

%%%%%%%%%%%%%%%%%%%%%%

\latexProblemContent{
\ifVerboseLocation This is Derivative Compute Question 0004. \\ \fi
\begin{problem}

Compute the following derivative:

\input{Derivative-Compute-0004.HELP.tex}

\[\dfrac{d}{dx}\left(\frac{x^{3} + 4 \, x^{2} - 17 \, x - 60}{x^{3} - 2 \, x^{2} - 16 \, x + 32}\right)=\answer{-\frac{2 \, {\left(3 \, x^{4} - x^{3} - 89 \, x^{2} - 8 \, x + 752\right)}}{{\left(x^{3} - 2 \, x^{2} - 16 \, x + 32\right)}^{2}}}\]
\end{problem}}

%%%%%%%%%%%%%%%%%%%%%%

\latexProblemContent{
\ifVerboseLocation This is Derivative Compute Question 0004. \\ \fi
\begin{problem}

Compute the following derivative:

\input{Derivative-Compute-0004.HELP.tex}

\[\dfrac{d}{dx}\left(\frac{x^{2} - 9}{x^{3} - 7 \, x^{2} + 8 \, x + 16}\right)=\answer{-\frac{x^{4} - 35 \, x^{2} + 94 \, x - 72}{{\left(x^{3} - 7 \, x^{2} + 8 \, x + 16\right)}^{2}}}\]
\end{problem}}

%%%%%%%%%%%%%%%%%%%%%%

\latexProblemContent{
\ifVerboseLocation This is Derivative Compute Question 0004. \\ \fi
\begin{problem}

Compute the following derivative:

\input{Derivative-Compute-0004.HELP.tex}

\[\dfrac{d}{dx}\left(\frac{1}{x^{2} + 4 \, x + 3}\right)=\answer{-\frac{2 \, {\left(x + 2\right)}}{{\left(x^{2} + 4 \, x + 3\right)}^{2}}}\]
\end{problem}}

%%%%%%%%%%%%%%%%%%%%%%

\latexProblemContent{
\ifVerboseLocation This is Derivative Compute Question 0004. \\ \fi
\begin{problem}

Compute the following derivative:

\input{Derivative-Compute-0004.HELP.tex}

\[\dfrac{d}{dx}\left(\frac{x^{2} - 1}{x + 3}\right)=\answer{\frac{x^{2} + 6 \, x + 1}{{\left(x + 3\right)}^{2}}}\]
\end{problem}}

%%%%%%%%%%%%%%%%%%%%%%

\latexProblemContent{
\ifVerboseLocation This is Derivative Compute Question 0004. \\ \fi
\begin{problem}

Compute the following derivative:

\input{Derivative-Compute-0004.HELP.tex}

\[\dfrac{d}{dx}\left(\frac{x - 1}{x^{2} + 7 \, x + 12}\right)=\answer{-\frac{x^{2} - 2 \, x - 19}{{\left(x^{2} + 7 \, x + 12\right)}^{2}}}\]
\end{problem}}

%%%%%%%%%%%%%%%%%%%%%%

\latexProblemContent{
\ifVerboseLocation This is Derivative Compute Question 0004. \\ \fi
\begin{problem}

Compute the following derivative:

\input{Derivative-Compute-0004.HELP.tex}

\[\dfrac{d}{dx}\left(\frac{x^{2} + 4 \, x + 3}{x - 2}\right)=\answer{\frac{x^{2} - 4 \, x - 11}{{\left(x - 2\right)}^{2}}}\]
\end{problem}}

%%%%%%%%%%%%%%%%%%%%%%

\latexProblemContent{
\ifVerboseLocation This is Derivative Compute Question 0004. \\ \fi
\begin{problem}

Compute the following derivative:

\input{Derivative-Compute-0004.HELP.tex}

\[\dfrac{d}{dx}\left(\frac{x - 3}{x - 2}\right)=\answer{\frac{1}{{\left(x - 2\right)}^{2}}}\]
\end{problem}}

%%%%%%%%%%%%%%%%%%%%%%

\latexProblemContent{
\ifVerboseLocation This is Derivative Compute Question 0004. \\ \fi
\begin{problem}

Compute the following derivative:

\input{Derivative-Compute-0004.HELP.tex}

\[\dfrac{d}{dx}\left(\frac{x^{3} - 7 \, x^{2} + 14 \, x - 8}{x^{3} - 3 \, x^{2} - 16 \, x + 48}\right)=\answer{\frac{2 \, {\left(2 \, x^{4} - 30 \, x^{3} + 161 \, x^{2} - 360 \, x + 272\right)}}{{\left(x^{3} - 3 \, x^{2} - 16 \, x + 48\right)}^{2}}}\]
\end{problem}}

%%%%%%%%%%%%%%%%%%%%%%

\latexProblemContent{
\ifVerboseLocation This is Derivative Compute Question 0004. \\ \fi
\begin{problem}

Compute the following derivative:

\input{Derivative-Compute-0004.HELP.tex}

\[\dfrac{d}{dx}\left(\frac{1}{x^{2} + 2 \, x - 8}\right)=\answer{-\frac{2 \, {\left(x + 1\right)}}{{\left(x^{2} + 2 \, x - 8\right)}^{2}}}\]
\end{problem}}

%%%%%%%%%%%%%%%%%%%%%%

\latexProblemContent{
\ifVerboseLocation This is Derivative Compute Question 0004. \\ \fi
\begin{problem}

Compute the following derivative:

\input{Derivative-Compute-0004.HELP.tex}

\[\dfrac{d}{dx}\left(\frac{1}{x^{3} + 2 \, x^{2} - 13 \, x + 10}\right)=\answer{-\frac{3 \, x^{2} + 4 \, x - 13}{{\left(x^{3} + 2 \, x^{2} - 13 \, x + 10\right)}^{2}}}\]
\end{problem}}

%%%%%%%%%%%%%%%%%%%%%%

\latexProblemContent{
\ifVerboseLocation This is Derivative Compute Question 0004. \\ \fi
\begin{problem}

Compute the following derivative:

\input{Derivative-Compute-0004.HELP.tex}

\[\dfrac{d}{dx}\left(\frac{x^{3} - x^{2} - 14 \, x + 24}{x^{3} + 4 \, x^{2} - 25 \, x - 100}\right)=\answer{\frac{5 \, x^{4} - 22 \, x^{3} - 291 \, x^{2} + 8 \, x + 2000}{{\left(x^{3} + 4 \, x^{2} - 25 \, x - 100\right)}^{2}}}\]
\end{problem}}

%%%%%%%%%%%%%%%%%%%%%%

\latexProblemContent{
\ifVerboseLocation This is Derivative Compute Question 0004. \\ \fi
\begin{problem}

Compute the following derivative:

\input{Derivative-Compute-0004.HELP.tex}

\[\dfrac{d}{dx}\left(\frac{x + 4}{x - 2}\right)=\answer{-\frac{6}{{\left(x - 2\right)}^{2}}}\]
\end{problem}}

%%%%%%%%%%%%%%%%%%%%%%

\latexProblemContent{
\ifVerboseLocation This is Derivative Compute Question 0004. \\ \fi
\begin{problem}

Compute the following derivative:

\input{Derivative-Compute-0004.HELP.tex}

\[\dfrac{d}{dx}\left(\frac{x^{2} - 9 \, x + 20}{x - 2}\right)=\answer{\frac{x^{2} - 4 \, x - 2}{{\left(x - 2\right)}^{2}}}\]
\end{problem}}

%%%%%%%%%%%%%%%%%%%%%%

\latexProblemContent{
\ifVerboseLocation This is Derivative Compute Question 0004. \\ \fi
\begin{problem}

Compute the following derivative:

\input{Derivative-Compute-0004.HELP.tex}

\[\dfrac{d}{dx}\left(\frac{x^{2} - 2 \, x - 3}{x - 3}\right)=\answer{\frac{x^{2} - 6 \, x + 9}{{\left(x - 3\right)}^{2}}}\]
\end{problem}}

%%%%%%%%%%%%%%%%%%%%%%

\latexProblemContent{
\ifVerboseLocation This is Derivative Compute Question 0004. \\ \fi
\begin{problem}

Compute the following derivative:

\input{Derivative-Compute-0004.HELP.tex}

\[\dfrac{d}{dx}\left(\frac{1}{x^{2} + 2 \, x + 1}\right)=\answer{-\frac{2 \, {\left(x + 1\right)}}{{\left(x^{2} + 2 \, x + 1\right)}^{2}}}\]
\end{problem}}

%%%%%%%%%%%%%%%%%%%%%%

\latexProblemContent{
\ifVerboseLocation This is Derivative Compute Question 0004. \\ \fi
\begin{problem}

Compute the following derivative:

\input{Derivative-Compute-0004.HELP.tex}

\[\dfrac{d}{dx}\left(\frac{x^{3} - 7 \, x^{2} + 14 \, x - 8}{x^{3} + 9 \, x^{2} + 15 \, x - 25}\right)=\answer{\frac{2 \, {\left(8 \, x^{4} + x^{3} - 141 \, x^{2} + 247 \, x - 115\right)}}{{\left(x^{3} + 9 \, x^{2} + 15 \, x - 25\right)}^{2}}}\]
\end{problem}}

%%%%%%%%%%%%%%%%%%%%%%

\latexProblemContent{
\ifVerboseLocation This is Derivative Compute Question 0004. \\ \fi
\begin{problem}

Compute the following derivative:

\input{Derivative-Compute-0004.HELP.tex}

\[\dfrac{d}{dx}\left(\frac{x - 2}{x + 1}\right)=\answer{\frac{3}{{\left(x + 1\right)}^{2}}}\]
\end{problem}}

%%%%%%%%%%%%%%%%%%%%%%

\latexProblemContent{
\ifVerboseLocation This is Derivative Compute Question 0004. \\ \fi
\begin{problem}

Compute the following derivative:

\input{Derivative-Compute-0004.HELP.tex}

\[\dfrac{d}{dx}\left(\frac{x + 4}{x^{3} - 8 \, x^{2} + 5 \, x + 50}\right)=\answer{-\frac{2 \, {\left(x^{3} + 2 \, x^{2} - 32 \, x - 15\right)}}{{\left(x^{3} - 8 \, x^{2} + 5 \, x + 50\right)}^{2}}}\]
\end{problem}}

%%%%%%%%%%%%%%%%%%%%%%

\latexProblemContent{
\ifVerboseLocation This is Derivative Compute Question 0004. \\ \fi
\begin{problem}

Compute the following derivative:

\input{Derivative-Compute-0004.HELP.tex}

\[\dfrac{d}{dx}\left(\frac{x + 3}{x^{3} - 9 \, x^{2} + 26 \, x - 24}\right)=\answer{-\frac{2 \, {\left(x^{3} - 27 \, x + 51\right)}}{{\left(x^{3} - 9 \, x^{2} + 26 \, x - 24\right)}^{2}}}\]
\end{problem}}

%%%%%%%%%%%%%%%%%%%%%%

\latexProblemContent{
\ifVerboseLocation This is Derivative Compute Question 0004. \\ \fi
\begin{problem}

Compute the following derivative:

\input{Derivative-Compute-0004.HELP.tex}

\[\dfrac{d}{dx}\left(\frac{1}{x^{3} - 5 \, x^{2} - x + 5}\right)=\answer{-\frac{3 \, x^{2} - 10 \, x - 1}{{\left(x^{3} - 5 \, x^{2} - x + 5\right)}^{2}}}\]
\end{problem}}

%%%%%%%%%%%%%%%%%%%%%%

\latexProblemContent{
\ifVerboseLocation This is Derivative Compute Question 0004. \\ \fi
\begin{problem}

Compute the following derivative:

\input{Derivative-Compute-0004.HELP.tex}

\[\dfrac{d}{dx}\left(\frac{x^{2} + 2 \, x - 3}{x^{3} - 3 \, x^{2} - 6 \, x + 8}\right)=\answer{-\frac{x^{4} + 4 \, x^{3} - 9 \, x^{2} + 2 \, x + 2}{{\left(x^{3} - 3 \, x^{2} - 6 \, x + 8\right)}^{2}}}\]
\end{problem}}

%%%%%%%%%%%%%%%%%%%%%%

\latexProblemContent{
\ifVerboseLocation This is Derivative Compute Question 0004. \\ \fi
\begin{problem}

Compute the following derivative:

\input{Derivative-Compute-0004.HELP.tex}

\[\dfrac{d}{dx}\left(\frac{x^{2} + x - 20}{x^{2} + 5 \, x + 4}\right)=\answer{\frac{4 \, {\left(x^{2} + 12 \, x + 26\right)}}{{\left(x^{2} + 5 \, x + 4\right)}^{2}}}\]
\end{problem}}

%%%%%%%%%%%%%%%%%%%%%%

\latexProblemContent{
\ifVerboseLocation This is Derivative Compute Question 0004. \\ \fi
\begin{problem}

Compute the following derivative:

\input{Derivative-Compute-0004.HELP.tex}

\[\dfrac{d}{dx}\left(\frac{x^{3} + 6 \, x^{2} - 32}{x + 1}\right)=\answer{\frac{2 \, x^{3} + 9 \, x^{2} + 12 \, x + 32}{{\left(x + 1\right)}^{2}}}\]
\end{problem}}

%%%%%%%%%%%%%%%%%%%%%%

\latexProblemContent{
\ifVerboseLocation This is Derivative Compute Question 0004. \\ \fi
\begin{problem}

Compute the following derivative:

\input{Derivative-Compute-0004.HELP.tex}

\[\dfrac{d}{dx}\left(\frac{x^{3} + 12 \, x^{2} + 47 \, x + 60}{x^{2} - 6 \, x + 8}\right)=\answer{\frac{x^{4} - 12 \, x^{3} - 95 \, x^{2} + 72 \, x + 736}{{\left(x^{2} - 6 \, x + 8\right)}^{2}}}\]
\end{problem}}

%%%%%%%%%%%%%%%%%%%%%%

\latexProblemContent{
\ifVerboseLocation This is Derivative Compute Question 0004. \\ \fi
\begin{problem}

Compute the following derivative:

\input{Derivative-Compute-0004.HELP.tex}

\[\dfrac{d}{dx}\left(\frac{x^{2} - 4 \, x - 5}{x^{3} - 4 \, x^{2} - 3 \, x + 18}\right)=\answer{-\frac{x^{4} - 8 \, x^{3} + 4 \, x^{2} + 4 \, x + 87}{{\left(x^{3} - 4 \, x^{2} - 3 \, x + 18\right)}^{2}}}\]
\end{problem}}

%%%%%%%%%%%%%%%%%%%%%%

\latexProblemContent{
\ifVerboseLocation This is Derivative Compute Question 0004. \\ \fi
\begin{problem}

Compute the following derivative:

\input{Derivative-Compute-0004.HELP.tex}

\[\dfrac{d}{dx}\left(\frac{x^{3} + 3 \, x^{2} - 13 \, x - 15}{x^{3} + 3 \, x^{2} - 6 \, x - 8}\right)=\answer{\frac{14 \, {\left(x^{3} + 3 \, x^{2} + 3 \, x + 1\right)}}{{\left(x^{3} + 3 \, x^{2} - 6 \, x - 8\right)}^{2}}}\]
\end{problem}}

%%%%%%%%%%%%%%%%%%%%%%

\latexProblemContent{
\ifVerboseLocation This is Derivative Compute Question 0004. \\ \fi
\begin{problem}

Compute the following derivative:

\input{Derivative-Compute-0004.HELP.tex}

\[\dfrac{d}{dx}\left(\frac{x^{3} - 21 \, x - 20}{x^{3} - 9 \, x^{2} + 26 \, x - 24}\right)=\answer{-\frac{9 \, x^{4} - 94 \, x^{3} + 201 \, x^{2} + 360 \, x - 1024}{{\left(x^{3} - 9 \, x^{2} + 26 \, x - 24\right)}^{2}}}\]
\end{problem}}

%%%%%%%%%%%%%%%%%%%%%%

\latexProblemContent{
\ifVerboseLocation This is Derivative Compute Question 0004. \\ \fi
\begin{problem}

Compute the following derivative:

\input{Derivative-Compute-0004.HELP.tex}

\[\dfrac{d}{dx}\left(\frac{1}{x^{3} - 8 \, x^{2} + 17 \, x - 10}\right)=\answer{-\frac{3 \, x^{2} - 16 \, x + 17}{{\left(x^{3} - 8 \, x^{2} + 17 \, x - 10\right)}^{2}}}\]
\end{problem}}

%%%%%%%%%%%%%%%%%%%%%%

\latexProblemContent{
\ifVerboseLocation This is Derivative Compute Question 0004. \\ \fi
\begin{problem}

Compute the following derivative:

\input{Derivative-Compute-0004.HELP.tex}

\[\dfrac{d}{dx}\left(\frac{x + 3}{x^{3} - 11 \, x^{2} + 38 \, x - 40}\right)=\answer{-\frac{2 \, {\left(x^{3} - x^{2} - 33 \, x + 77\right)}}{{\left(x^{3} - 11 \, x^{2} + 38 \, x - 40\right)}^{2}}}\]
\end{problem}}

%%%%%%%%%%%%%%%%%%%%%%

\latexProblemContent{
\ifVerboseLocation This is Derivative Compute Question 0004. \\ \fi
\begin{problem}

Compute the following derivative:

\input{Derivative-Compute-0004.HELP.tex}

\[\dfrac{d}{dx}\left(\frac{x^{2} - x - 6}{x^{2} + 2 \, x - 3}\right)=\answer{\frac{3 \, {\left(x^{2} + 2 \, x + 5\right)}}{{\left(x^{2} + 2 \, x - 3\right)}^{2}}}\]
\end{problem}}

%%%%%%%%%%%%%%%%%%%%%%

\latexProblemContent{
\ifVerboseLocation This is Derivative Compute Question 0004. \\ \fi
\begin{problem}

Compute the following derivative:

\input{Derivative-Compute-0004.HELP.tex}

\[\dfrac{d}{dx}\left(\frac{x - 5}{x^{3} + 3 \, x^{2} - 13 \, x - 15}\right)=\answer{-\frac{2 \, {\left(x^{3} - 6 \, x^{2} - 15 \, x + 40\right)}}{{\left(x^{3} + 3 \, x^{2} - 13 \, x - 15\right)}^{2}}}\]
\end{problem}}

%%%%%%%%%%%%%%%%%%%%%%

\latexProblemContent{
\ifVerboseLocation This is Derivative Compute Question 0004. \\ \fi
\begin{problem}

Compute the following derivative:

\input{Derivative-Compute-0004.HELP.tex}

\[\dfrac{d}{dx}\left(\frac{1}{x^{3} - 21 \, x - 20}\right)=\answer{-\frac{3 \, {\left(x^{2} - 7\right)}}{{\left(x^{3} - 21 \, x - 20\right)}^{2}}}\]
\end{problem}}

%%%%%%%%%%%%%%%%%%%%%%

\latexProblemContent{
\ifVerboseLocation This is Derivative Compute Question 0004. \\ \fi
\begin{problem}

Compute the following derivative:

\input{Derivative-Compute-0004.HELP.tex}

\[\dfrac{d}{dx}\left(\frac{x^{2} + 4 \, x + 3}{x^{2} - 3 \, x - 4}\right)=\answer{-\frac{7 \, {\left(x^{2} + 2 \, x + 1\right)}}{{\left(x^{2} - 3 \, x - 4\right)}^{2}}}\]
\end{problem}}

%%%%%%%%%%%%%%%%%%%%%%

\latexProblemContent{
\ifVerboseLocation This is Derivative Compute Question 0004. \\ \fi
\begin{problem}

Compute the following derivative:

\input{Derivative-Compute-0004.HELP.tex}

\[\dfrac{d}{dx}\left(\frac{x^{3} - 3 \, x^{2} - 24 \, x + 80}{x + 3}\right)=\answer{\frac{2 \, {\left(x^{3} + 3 \, x^{2} - 9 \, x - 76\right)}}{{\left(x + 3\right)}^{2}}}\]
\end{problem}}

%%%%%%%%%%%%%%%%%%%%%%

\latexProblemContent{
\ifVerboseLocation This is Derivative Compute Question 0004. \\ \fi
\begin{problem}

Compute the following derivative:

\input{Derivative-Compute-0004.HELP.tex}

\[\dfrac{d}{dx}\left(\frac{x^{2} - x - 6}{x^{3} - 11 \, x^{2} + 38 \, x - 40}\right)=\answer{-\frac{x^{4} - 2 \, x^{3} - 45 \, x^{2} + 212 \, x - 268}{{\left(x^{3} - 11 \, x^{2} + 38 \, x - 40\right)}^{2}}}\]
\end{problem}}

%%%%%%%%%%%%%%%%%%%%%%

\latexProblemContent{
\ifVerboseLocation This is Derivative Compute Question 0004. \\ \fi
\begin{problem}

Compute the following derivative:

\input{Derivative-Compute-0004.HELP.tex}

\[\dfrac{d}{dx}\left(\frac{x^{3} - 2 \, x^{2} - 9 \, x + 18}{x^{2} + x - 6}\right)=\answer{\frac{x^{4} + 2 \, x^{3} - 11 \, x^{2} - 12 \, x + 36}{{\left(x^{2} + x - 6\right)}^{2}}}\]
\end{problem}}

%%%%%%%%%%%%%%%%%%%%%%

\latexProblemContent{
\ifVerboseLocation This is Derivative Compute Question 0004. \\ \fi
\begin{problem}

Compute the following derivative:

\input{Derivative-Compute-0004.HELP.tex}

\[\dfrac{d}{dx}\left(\frac{x^{2} + 6 \, x + 8}{x - 4}\right)=\answer{\frac{x^{2} - 8 \, x - 32}{{\left(x - 4\right)}^{2}}}\]
\end{problem}}

%%%%%%%%%%%%%%%%%%%%%%

\latexProblemContent{
\ifVerboseLocation This is Derivative Compute Question 0004. \\ \fi
\begin{problem}

Compute the following derivative:

\input{Derivative-Compute-0004.HELP.tex}

\[\dfrac{d}{dx}\left(\frac{x^{3} + 13 \, x^{2} + 56 \, x + 80}{x + 4}\right)=\answer{\frac{2 \, x^{3} + 25 \, x^{2} + 104 \, x + 144}{{\left(x + 4\right)}^{2}}}\]
\end{problem}}

%%%%%%%%%%%%%%%%%%%%%%

\latexProblemContent{
\ifVerboseLocation This is Derivative Compute Question 0004. \\ \fi
\begin{problem}

Compute the following derivative:

\input{Derivative-Compute-0004.HELP.tex}

\[\dfrac{d}{dx}\left(\frac{x^{3} - 2 \, x^{2} - 16 \, x + 32}{x^{2} - 25}\right)=\answer{\frac{x^{4} - 59 \, x^{2} + 36 \, x + 400}{{\left(x^{2} - 25\right)}^{2}}}\]
\end{problem}}

%%%%%%%%%%%%%%%%%%%%%%

\latexProblemContent{
\ifVerboseLocation This is Derivative Compute Question 0004. \\ \fi
\begin{problem}

Compute the following derivative:

\input{Derivative-Compute-0004.HELP.tex}

\[\dfrac{d}{dx}\left(\frac{1}{x^{2} - 7 \, x + 12}\right)=\answer{-\frac{2 \, x - 7}{{\left(x^{2} - 7 \, x + 12\right)}^{2}}}\]
\end{problem}}

%%%%%%%%%%%%%%%%%%%%%%

\latexProblemContent{
\ifVerboseLocation This is Derivative Compute Question 0004. \\ \fi
\begin{problem}

Compute the following derivative:

\input{Derivative-Compute-0004.HELP.tex}

\[\dfrac{d}{dx}\left(\frac{x^{3} - 3 \, x^{2} - 4 \, x + 12}{x^{3} + 14 \, x^{2} + 65 \, x + 100}\right)=\answer{\frac{17 \, x^{4} + 138 \, x^{3} + 125 \, x^{2} - 936 \, x - 1180}{{\left(x^{3} + 14 \, x^{2} + 65 \, x + 100\right)}^{2}}}\]
\end{problem}}

%%%%%%%%%%%%%%%%%%%%%%

\latexProblemContent{
\ifVerboseLocation This is Derivative Compute Question 0004. \\ \fi
\begin{problem}

Compute the following derivative:

\input{Derivative-Compute-0004.HELP.tex}

\[\dfrac{d}{dx}\left(\frac{x^{2} + 6 \, x + 5}{x - 1}\right)=\answer{\frac{x^{2} - 2 \, x - 11}{{\left(x - 1\right)}^{2}}}\]
\end{problem}}

%%%%%%%%%%%%%%%%%%%%%%

\latexProblemContent{
\ifVerboseLocation This is Derivative Compute Question 0004. \\ \fi
\begin{problem}

Compute the following derivative:

\input{Derivative-Compute-0004.HELP.tex}

\[\dfrac{d}{dx}\left(\frac{x^{3} - 2 \, x^{2} - 7 \, x - 4}{x + 4}\right)=\answer{\frac{2 \, {\left(x^{3} + 5 \, x^{2} - 8 \, x - 12\right)}}{{\left(x + 4\right)}^{2}}}\]
\end{problem}}

%%%%%%%%%%%%%%%%%%%%%%

\latexProblemContent{
\ifVerboseLocation This is Derivative Compute Question 0004. \\ \fi
\begin{problem}

Compute the following derivative:

\input{Derivative-Compute-0004.HELP.tex}

\[\dfrac{d}{dx}\left(\frac{x - 2}{x^{2} + 6 \, x + 8}\right)=\answer{-\frac{x^{2} - 4 \, x - 20}{{\left(x^{2} + 6 \, x + 8\right)}^{2}}}\]
\end{problem}}

%%%%%%%%%%%%%%%%%%%%%%

\latexProblemContent{
\ifVerboseLocation This is Derivative Compute Question 0004. \\ \fi
\begin{problem}

Compute the following derivative:

\input{Derivative-Compute-0004.HELP.tex}

\[\dfrac{d}{dx}\left(\frac{x^{2} - 8 \, x + 15}{x^{2} + 3 \, x + 2}\right)=\answer{\frac{11 \, x^{2} - 26 \, x - 61}{{\left(x^{2} + 3 \, x + 2\right)}^{2}}}\]
\end{problem}}

%%%%%%%%%%%%%%%%%%%%%%

\latexProblemContent{
\ifVerboseLocation This is Derivative Compute Question 0004. \\ \fi
\begin{problem}

Compute the following derivative:

\input{Derivative-Compute-0004.HELP.tex}

\[\dfrac{d}{dx}\left(\frac{x^{3} + x^{2} - 22 \, x - 40}{x - 4}\right)=\answer{\frac{2 \, x^{3} - 11 \, x^{2} - 8 \, x + 128}{{\left(x - 4\right)}^{2}}}\]
\end{problem}}

%%%%%%%%%%%%%%%%%%%%%%

\latexProblemContent{
\ifVerboseLocation This is Derivative Compute Question 0004. \\ \fi
\begin{problem}

Compute the following derivative:

\input{Derivative-Compute-0004.HELP.tex}

\[\dfrac{d}{dx}\left(\frac{x^{2} - 6 \, x + 8}{x^{3} - 2 \, x^{2} - 13 \, x - 10}\right)=\answer{-\frac{x^{4} - 12 \, x^{3} + 49 \, x^{2} - 12 \, x - 164}{{\left(x^{3} - 2 \, x^{2} - 13 \, x - 10\right)}^{2}}}\]
\end{problem}}

%%%%%%%%%%%%%%%%%%%%%%

\latexProblemContent{
\ifVerboseLocation This is Derivative Compute Question 0004. \\ \fi
\begin{problem}

Compute the following derivative:

\input{Derivative-Compute-0004.HELP.tex}

\[\dfrac{d}{dx}\left(\frac{x^{3} - 7 \, x^{2} + 7 \, x + 15}{x + 4}\right)=\answer{\frac{2 \, x^{3} + 5 \, x^{2} - 56 \, x + 13}{{\left(x + 4\right)}^{2}}}\]
\end{problem}}

%%%%%%%%%%%%%%%%%%%%%%

\latexProblemContent{
\ifVerboseLocation This is Derivative Compute Question 0004. \\ \fi
\begin{problem}

Compute the following derivative:

\input{Derivative-Compute-0004.HELP.tex}

\[\dfrac{d}{dx}\left(\frac{1}{x^{3} - 8 \, x^{2} + 20 \, x - 16}\right)=\answer{-\frac{3 \, x^{2} - 16 \, x + 20}{{\left(x^{3} - 8 \, x^{2} + 20 \, x - 16\right)}^{2}}}\]
\end{problem}}

%%%%%%%%%%%%%%%%%%%%%%

\latexProblemContent{
\ifVerboseLocation This is Derivative Compute Question 0004. \\ \fi
\begin{problem}

Compute the following derivative:

\input{Derivative-Compute-0004.HELP.tex}

\[\dfrac{d}{dx}\left(\frac{x^{3} + 3 \, x^{2} - x - 3}{x - 4}\right)=\answer{\frac{2 \, x^{3} - 9 \, x^{2} - 24 \, x + 7}{{\left(x - 4\right)}^{2}}}\]
\end{problem}}

%%%%%%%%%%%%%%%%%%%%%%

\latexProblemContent{
\ifVerboseLocation This is Derivative Compute Question 0004. \\ \fi
\begin{problem}

Compute the following derivative:

\input{Derivative-Compute-0004.HELP.tex}

\[\dfrac{d}{dx}\left(\frac{x^{2} - x - 12}{x - 1}\right)=\answer{\frac{x^{2} - 2 \, x + 13}{{\left(x - 1\right)}^{2}}}\]
\end{problem}}

%%%%%%%%%%%%%%%%%%%%%%

\latexProblemContent{
\ifVerboseLocation This is Derivative Compute Question 0004. \\ \fi
\begin{problem}

Compute the following derivative:

\input{Derivative-Compute-0004.HELP.tex}

\[\dfrac{d}{dx}\left(\frac{x^{2} + 2 \, x - 3}{x^{2} - 2 \, x - 8}\right)=\answer{-\frac{2 \, {\left(2 \, x^{2} + 5 \, x + 11\right)}}{{\left(x^{2} - 2 \, x - 8\right)}^{2}}}\]
\end{problem}}

%%%%%%%%%%%%%%%%%%%%%%

\latexProblemContent{
\ifVerboseLocation This is Derivative Compute Question 0004. \\ \fi
\begin{problem}

Compute the following derivative:

\input{Derivative-Compute-0004.HELP.tex}

\[\dfrac{d}{dx}\left(\frac{x^{2} + 7 \, x + 12}{x^{3} - 9 \, x^{2} + 26 \, x - 24}\right)=\answer{-\frac{x^{4} + 14 \, x^{3} - 53 \, x^{2} - 168 \, x + 480}{{\left(x^{3} - 9 \, x^{2} + 26 \, x - 24\right)}^{2}}}\]
\end{problem}}

%%%%%%%%%%%%%%%%%%%%%%

\latexProblemContent{
\ifVerboseLocation This is Derivative Compute Question 0004. \\ \fi
\begin{problem}

Compute the following derivative:

\input{Derivative-Compute-0004.HELP.tex}

\[\dfrac{d}{dx}\left(\frac{x + 5}{x + 2}\right)=\answer{-\frac{3}{{\left(x + 2\right)}^{2}}}\]
\end{problem}}

%%%%%%%%%%%%%%%%%%%%%%

\latexProblemContent{
\ifVerboseLocation This is Derivative Compute Question 0004. \\ \fi
\begin{problem}

Compute the following derivative:

\input{Derivative-Compute-0004.HELP.tex}

\[\dfrac{d}{dx}\left(\frac{1}{x^{3} + 2 \, x^{2} - 19 \, x - 20}\right)=\answer{-\frac{3 \, x^{2} + 4 \, x - 19}{{\left(x^{3} + 2 \, x^{2} - 19 \, x - 20\right)}^{2}}}\]
\end{problem}}

%%%%%%%%%%%%%%%%%%%%%%

\latexProblemContent{
\ifVerboseLocation This is Derivative Compute Question 0004. \\ \fi
\begin{problem}

Compute the following derivative:

\input{Derivative-Compute-0004.HELP.tex}

\[\dfrac{d}{dx}\left(\frac{1}{x^{2} + 6 \, x + 5}\right)=\answer{-\frac{2 \, {\left(x + 3\right)}}{{\left(x^{2} + 6 \, x + 5\right)}^{2}}}\]
\end{problem}}

%%%%%%%%%%%%%%%%%%%%%%

\latexProblemContent{
\ifVerboseLocation This is Derivative Compute Question 0004. \\ \fi
\begin{problem}

Compute the following derivative:

\input{Derivative-Compute-0004.HELP.tex}

\[\dfrac{d}{dx}\left(\frac{x^{2} - 25}{x + 2}\right)=\answer{\frac{x^{2} + 4 \, x + 25}{{\left(x + 2\right)}^{2}}}\]
\end{problem}}

%%%%%%%%%%%%%%%%%%%%%%

\latexProblemContent{
\ifVerboseLocation This is Derivative Compute Question 0004. \\ \fi
\begin{problem}

Compute the following derivative:

\input{Derivative-Compute-0004.HELP.tex}

\[\dfrac{d}{dx}\left(\frac{x^{3} - 4 \, x^{2} - 11 \, x + 30}{x^{2} + x - 20}\right)=\answer{\frac{x^{4} + 2 \, x^{3} - 53 \, x^{2} + 100 \, x + 190}{{\left(x^{2} + x - 20\right)}^{2}}}\]
\end{problem}}

%%%%%%%%%%%%%%%%%%%%%%

\latexProblemContent{
\ifVerboseLocation This is Derivative Compute Question 0004. \\ \fi
\begin{problem}

Compute the following derivative:

\input{Derivative-Compute-0004.HELP.tex}

\[\dfrac{d}{dx}\left(\frac{x - 5}{x^{3} + x^{2} - 4 \, x - 4}\right)=\answer{-\frac{2 \, {\left(x^{3} - 7 \, x^{2} - 5 \, x + 12\right)}}{{\left(x^{3} + x^{2} - 4 \, x - 4\right)}^{2}}}\]
\end{problem}}

%%%%%%%%%%%%%%%%%%%%%%

\latexProblemContent{
\ifVerboseLocation This is Derivative Compute Question 0004. \\ \fi
\begin{problem}

Compute the following derivative:

\input{Derivative-Compute-0004.HELP.tex}

\[\dfrac{d}{dx}\left(\frac{x^{3} - 7 \, x + 6}{x + 1}\right)=\answer{\frac{2 \, x^{3} + 3 \, x^{2} - 13}{{\left(x + 1\right)}^{2}}}\]
\end{problem}}

%%%%%%%%%%%%%%%%%%%%%%

\latexProblemContent{
\ifVerboseLocation This is Derivative Compute Question 0004. \\ \fi
\begin{problem}

Compute the following derivative:

\input{Derivative-Compute-0004.HELP.tex}

\[\dfrac{d}{dx}\left(\frac{x^{2} - x - 12}{x^{3} + x^{2} - 14 \, x - 24}\right)=\answer{-\frac{x^{4} - 2 \, x^{3} - 23 \, x^{2} + 24 \, x + 144}{{\left(x^{3} + x^{2} - 14 \, x - 24\right)}^{2}}}\]
\end{problem}}

%%%%%%%%%%%%%%%%%%%%%%

\latexProblemContent{
\ifVerboseLocation This is Derivative Compute Question 0004. \\ \fi
\begin{problem}

Compute the following derivative:

\input{Derivative-Compute-0004.HELP.tex}

\[\dfrac{d}{dx}\left(\frac{x^{2} - 7 \, x + 12}{x^{3} - x^{2} - 9 \, x + 9}\right)=\answer{-\frac{x^{4} - 14 \, x^{3} + 52 \, x^{2} - 42 \, x - 45}{{\left(x^{3} - x^{2} - 9 \, x + 9\right)}^{2}}}\]
\end{problem}}

%%%%%%%%%%%%%%%%%%%%%%

\latexProblemContent{
\ifVerboseLocation This is Derivative Compute Question 0004. \\ \fi
\begin{problem}

Compute the following derivative:

\input{Derivative-Compute-0004.HELP.tex}

\[\dfrac{d}{dx}\left(\frac{x + 3}{x^{2} - 2 \, x - 3}\right)=\answer{-\frac{x^{2} + 6 \, x - 3}{{\left(x^{2} - 2 \, x - 3\right)}^{2}}}\]
\end{problem}}

%%%%%%%%%%%%%%%%%%%%%%

\latexProblemContent{
\ifVerboseLocation This is Derivative Compute Question 0004. \\ \fi
\begin{problem}

Compute the following derivative:

\input{Derivative-Compute-0004.HELP.tex}

\[\dfrac{d}{dx}\left(\frac{x^{2} - 1}{x^{3} + 2 \, x^{2} - 11 \, x - 12}\right)=\answer{-\frac{x^{4} + 8 \, x^{2} + 20 \, x + 11}{{\left(x^{3} + 2 \, x^{2} - 11 \, x - 12\right)}^{2}}}\]
\end{problem}}

%%%%%%%%%%%%%%%%%%%%%%

\latexProblemContent{
\ifVerboseLocation This is Derivative Compute Question 0004. \\ \fi
\begin{problem}

Compute the following derivative:

\input{Derivative-Compute-0004.HELP.tex}

\[\dfrac{d}{dx}\left(\frac{x^{3} + 9 \, x^{2} + 15 \, x - 25}{x^{3} - 4 \, x^{2} - x + 4}\right)=\answer{-\frac{13 \, x^{4} + 32 \, x^{3} - 138 \, x^{2} + 128 \, x - 35}{{\left(x^{3} - 4 \, x^{2} - x + 4\right)}^{2}}}\]
\end{problem}}

%%%%%%%%%%%%%%%%%%%%%%

\latexProblemContent{
\ifVerboseLocation This is Derivative Compute Question 0004. \\ \fi
\begin{problem}

Compute the following derivative:

\input{Derivative-Compute-0004.HELP.tex}

\[\dfrac{d}{dx}\left(\frac{x + 4}{x^{2} - 6 \, x + 9}\right)=\answer{-\frac{x^{2} + 8 \, x - 33}{{\left(x^{2} - 6 \, x + 9\right)}^{2}}}\]
\end{problem}}

%%%%%%%%%%%%%%%%%%%%%%

\latexProblemContent{
\ifVerboseLocation This is Derivative Compute Question 0004. \\ \fi
\begin{problem}

Compute the following derivative:

\input{Derivative-Compute-0004.HELP.tex}

\[\dfrac{d}{dx}\left(\frac{x^{3} + 5 \, x^{2} - x - 5}{x^{3} + 2 \, x^{2} - 16 \, x - 32}\right)=\answer{-\frac{3 \, {\left(x^{4} + 10 \, x^{3} + 53 \, x^{2} + 100 \, x + 16\right)}}{{\left(x^{3} + 2 \, x^{2} - 16 \, x - 32\right)}^{2}}}\]
\end{problem}}

%%%%%%%%%%%%%%%%%%%%%%

\latexProblemContent{
\ifVerboseLocation This is Derivative Compute Question 0004. \\ \fi
\begin{problem}

Compute the following derivative:

\input{Derivative-Compute-0004.HELP.tex}

\[\dfrac{d}{dx}\left(\frac{x^{2} + 3 \, x + 2}{x^{2} - 2 \, x - 8}\right)=\answer{-\frac{5 \, {\left(x^{2} + 4 \, x + 4\right)}}{{\left(x^{2} - 2 \, x - 8\right)}^{2}}}\]
\end{problem}}

%%%%%%%%%%%%%%%%%%%%%%

\latexProblemContent{
\ifVerboseLocation This is Derivative Compute Question 0004. \\ \fi
\begin{problem}

Compute the following derivative:

\input{Derivative-Compute-0004.HELP.tex}

\[\dfrac{d}{dx}\left(\frac{x^{3} + 8 \, x^{2} + 11 \, x - 20}{x^{2} - 6 \, x + 8}\right)=\answer{\frac{x^{4} - 12 \, x^{3} - 35 \, x^{2} + 168 \, x - 32}{{\left(x^{2} - 6 \, x + 8\right)}^{2}}}\]
\end{problem}}

%%%%%%%%%%%%%%%%%%%%%%

\latexProblemContent{
\ifVerboseLocation This is Derivative Compute Question 0004. \\ \fi
\begin{problem}

Compute the following derivative:

\input{Derivative-Compute-0004.HELP.tex}

\[\dfrac{d}{dx}\left(\frac{x^{3} + 2 \, x^{2} - 19 \, x - 20}{x^{3} + 6 \, x^{2} - 15 \, x - 100}\right)=\answer{\frac{4 \, {\left(x^{4} + 2 \, x^{3} - 39 \, x^{2} - 40 \, x + 400\right)}}{{\left(x^{3} + 6 \, x^{2} - 15 \, x - 100\right)}^{2}}}\]
\end{problem}}

%%%%%%%%%%%%%%%%%%%%%%

\latexProblemContent{
\ifVerboseLocation This is Derivative Compute Question 0004. \\ \fi
\begin{problem}

Compute the following derivative:

\input{Derivative-Compute-0004.HELP.tex}

\[\dfrac{d}{dx}\left(\frac{1}{x^{3} - 12 \, x^{2} + 47 \, x - 60}\right)=\answer{-\frac{3 \, x^{2} - 24 \, x + 47}{{\left(x^{3} - 12 \, x^{2} + 47 \, x - 60\right)}^{2}}}\]
\end{problem}}

%%%%%%%%%%%%%%%%%%%%%%

\latexProblemContent{
\ifVerboseLocation This is Derivative Compute Question 0004. \\ \fi
\begin{problem}

Compute the following derivative:

\input{Derivative-Compute-0004.HELP.tex}

\[\dfrac{d}{dx}\left(\frac{x^{2} - 16}{x - 4}\right)=\answer{\frac{x^{2} - 8 \, x + 16}{{\left(x - 4\right)}^{2}}}\]
\end{problem}}

%%%%%%%%%%%%%%%%%%%%%%

\latexProblemContent{
\ifVerboseLocation This is Derivative Compute Question 0004. \\ \fi
\begin{problem}

Compute the following derivative:

\input{Derivative-Compute-0004.HELP.tex}

\[\dfrac{d}{dx}\left(\frac{1}{x^{2} + 3 \, x + 2}\right)=\answer{-\frac{2 \, x + 3}{{\left(x^{2} + 3 \, x + 2\right)}^{2}}}\]
\end{problem}}

%%%%%%%%%%%%%%%%%%%%%%

\latexProblemContent{
\ifVerboseLocation This is Derivative Compute Question 0004. \\ \fi
\begin{problem}

Compute the following derivative:

\input{Derivative-Compute-0004.HELP.tex}

\[\dfrac{d}{dx}\left(\frac{1}{x^{3} - 9 \, x^{2} + 24 \, x - 20}\right)=\answer{-\frac{3 \, {\left(x^{2} - 6 \, x + 8\right)}}{{\left(x^{3} - 9 \, x^{2} + 24 \, x - 20\right)}^{2}}}\]
\end{problem}}

%%%%%%%%%%%%%%%%%%%%%%

\latexProblemContent{
\ifVerboseLocation This is Derivative Compute Question 0004. \\ \fi
\begin{problem}

Compute the following derivative:

\input{Derivative-Compute-0004.HELP.tex}

\[\dfrac{d}{dx}\left(\frac{x^{3} - 7 \, x^{2} - 5 \, x + 75}{x^{3} + 2 \, x^{2} - 13 \, x + 10}\right)=\answer{\frac{9 \, x^{4} - 16 \, x^{3} - 94 \, x^{2} - 440 \, x + 925}{{\left(x^{3} + 2 \, x^{2} - 13 \, x + 10\right)}^{2}}}\]
\end{problem}}

%%%%%%%%%%%%%%%%%%%%%%

\latexProblemContent{
\ifVerboseLocation This is Derivative Compute Question 0004. \\ \fi
\begin{problem}

Compute the following derivative:

\input{Derivative-Compute-0004.HELP.tex}

\[\dfrac{d}{dx}\left(\frac{x - 1}{x - 2}\right)=\answer{-\frac{1}{{\left(x - 2\right)}^{2}}}\]
\end{problem}}

%%%%%%%%%%%%%%%%%%%%%%

\latexProblemContent{
\ifVerboseLocation This is Derivative Compute Question 0004. \\ \fi
\begin{problem}

Compute the following derivative:

\input{Derivative-Compute-0004.HELP.tex}

\[\dfrac{d}{dx}\left(\frac{x^{3} - x^{2} - 8 \, x + 12}{x^{3} - 4 \, x^{2} - 7 \, x + 10}\right)=\answer{-\frac{3 \, x^{4} - 2 \, x^{3} + 31 \, x^{2} - 76 \, x - 4}{{\left(x^{3} - 4 \, x^{2} - 7 \, x + 10\right)}^{2}}}\]
\end{problem}}

%%%%%%%%%%%%%%%%%%%%%%

\latexProblemContent{
\ifVerboseLocation This is Derivative Compute Question 0004. \\ \fi
\begin{problem}

Compute the following derivative:

\input{Derivative-Compute-0004.HELP.tex}

\[\dfrac{d}{dx}\left(\frac{x^{2} - x - 6}{x^{3} + 4 \, x^{2} - 17 \, x - 60}\right)=\answer{-\frac{x^{4} - 2 \, x^{3} - 5 \, x^{2} + 72 \, x + 42}{{\left(x^{3} + 4 \, x^{2} - 17 \, x - 60\right)}^{2}}}\]
\end{problem}}

%%%%%%%%%%%%%%%%%%%%%%

\latexProblemContent{
\ifVerboseLocation This is Derivative Compute Question 0004. \\ \fi
\begin{problem}

Compute the following derivative:

\input{Derivative-Compute-0004.HELP.tex}

\[\dfrac{d}{dx}\left(\frac{x^{3} + 5 \, x^{2} + 2 \, x - 8}{x^{2} - 5 \, x + 4}\right)=\answer{\frac{x^{4} - 10 \, x^{3} - 15 \, x^{2} + 56 \, x - 32}{{\left(x^{2} - 5 \, x + 4\right)}^{2}}}\]
\end{problem}}

%%%%%%%%%%%%%%%%%%%%%%

\latexProblemContent{
\ifVerboseLocation This is Derivative Compute Question 0004. \\ \fi
\begin{problem}

Compute the following derivative:

\input{Derivative-Compute-0004.HELP.tex}

\[\dfrac{d}{dx}\left(\frac{x^{2} - 5 \, x + 4}{x^{2} - 9}\right)=\answer{\frac{5 \, x^{2} - 26 \, x + 45}{{\left(x^{2} - 9\right)}^{2}}}\]
\end{problem}}

%%%%%%%%%%%%%%%%%%%%%%

\latexProblemContent{
\ifVerboseLocation This is Derivative Compute Question 0004. \\ \fi
\begin{problem}

Compute the following derivative:

\input{Derivative-Compute-0004.HELP.tex}

\[\dfrac{d}{dx}\left(\frac{x^{3} - 4 \, x^{2} + x + 6}{x^{3} - 9 \, x^{2} + 24 \, x - 20}\right)=\answer{-\frac{5 \, x^{4} - 46 \, x^{3} + 165 \, x^{2} - 268 \, x + 164}{{\left(x^{3} - 9 \, x^{2} + 24 \, x - 20\right)}^{2}}}\]
\end{problem}}

%%%%%%%%%%%%%%%%%%%%%%

\latexProblemContent{
\ifVerboseLocation This is Derivative Compute Question 0004. \\ \fi
\begin{problem}

Compute the following derivative:

\input{Derivative-Compute-0004.HELP.tex}

\[\dfrac{d}{dx}\left(\frac{1}{x^{3} + 7 \, x^{2} + 2 \, x - 40}\right)=\answer{-\frac{3 \, x^{2} + 14 \, x + 2}{{\left(x^{3} + 7 \, x^{2} + 2 \, x - 40\right)}^{2}}}\]
\end{problem}}

%%%%%%%%%%%%%%%%%%%%%%

\latexProblemContent{
\ifVerboseLocation This is Derivative Compute Question 0004. \\ \fi
\begin{problem}

Compute the following derivative:

\input{Derivative-Compute-0004.HELP.tex}

\[\dfrac{d}{dx}\left(\frac{x - 4}{x^{2} + 6 \, x + 9}\right)=\answer{-\frac{x^{2} - 8 \, x - 33}{{\left(x^{2} + 6 \, x + 9\right)}^{2}}}\]
\end{problem}}

%%%%%%%%%%%%%%%%%%%%%%

\latexProblemContent{
\ifVerboseLocation This is Derivative Compute Question 0004. \\ \fi
\begin{problem}

Compute the following derivative:

\input{Derivative-Compute-0004.HELP.tex}

\[\dfrac{d}{dx}\left(\frac{x + 3}{x^{2} + 2 \, x - 15}\right)=\answer{-\frac{x^{2} + 6 \, x + 21}{{\left(x^{2} + 2 \, x - 15\right)}^{2}}}\]
\end{problem}}

%%%%%%%%%%%%%%%%%%%%%%

\latexProblemContent{
\ifVerboseLocation This is Derivative Compute Question 0004. \\ \fi
\begin{problem}

Compute the following derivative:

\input{Derivative-Compute-0004.HELP.tex}

\[\dfrac{d}{dx}\left(\frac{x^{2} - x - 20}{x + 3}\right)=\answer{\frac{x^{2} + 6 \, x + 17}{{\left(x + 3\right)}^{2}}}\]
\end{problem}}

%%%%%%%%%%%%%%%%%%%%%%

\latexProblemContent{
\ifVerboseLocation This is Derivative Compute Question 0004. \\ \fi
\begin{problem}

Compute the following derivative:

\input{Derivative-Compute-0004.HELP.tex}

\[\dfrac{d}{dx}\left(\frac{x^{3} - 5 \, x^{2} + 2 \, x + 8}{x^{2} + 2 \, x + 1}\right)=\answer{\frac{x^{4} + 4 \, x^{3} - 9 \, x^{2} - 26 \, x - 14}{{\left(x^{2} + 2 \, x + 1\right)}^{2}}}\]
\end{problem}}

%%%%%%%%%%%%%%%%%%%%%%

\latexProblemContent{
\ifVerboseLocation This is Derivative Compute Question 0004. \\ \fi
\begin{problem}

Compute the following derivative:

\input{Derivative-Compute-0004.HELP.tex}

\[\dfrac{d}{dx}\left(\frac{1}{x^{3} + 3 \, x^{2} - 6 \, x - 8}\right)=\answer{-\frac{3 \, {\left(x^{2} + 2 \, x - 2\right)}}{{\left(x^{3} + 3 \, x^{2} - 6 \, x - 8\right)}^{2}}}\]
\end{problem}}

%%%%%%%%%%%%%%%%%%%%%%

\latexProblemContent{
\ifVerboseLocation This is Derivative Compute Question 0004. \\ \fi
\begin{problem}

Compute the following derivative:

\input{Derivative-Compute-0004.HELP.tex}

\[\dfrac{d}{dx}\left(\frac{x - 5}{x^{3} - 7 \, x - 6}\right)=\answer{-\frac{2 \, x^{3} - 15 \, x^{2} + 41}{{\left(x^{3} - 7 \, x - 6\right)}^{2}}}\]
\end{problem}}

%%%%%%%%%%%%%%%%%%%%%%

\latexProblemContent{
\ifVerboseLocation This is Derivative Compute Question 0004. \\ \fi
\begin{problem}

Compute the following derivative:

\input{Derivative-Compute-0004.HELP.tex}

\[\dfrac{d}{dx}\left(\frac{x^{3} - 5 \, x^{2} - x + 5}{x^{3} + 9 \, x^{2} + 24 \, x + 16}\right)=\answer{\frac{2 \, {\left(7 \, x^{4} + 25 \, x^{3} - 39 \, x^{2} - 125 \, x - 68\right)}}{{\left(x^{3} + 9 \, x^{2} + 24 \, x + 16\right)}^{2}}}\]
\end{problem}}

%%%%%%%%%%%%%%%%%%%%%%

\latexProblemContent{
\ifVerboseLocation This is Derivative Compute Question 0004. \\ \fi
\begin{problem}

Compute the following derivative:

\input{Derivative-Compute-0004.HELP.tex}

\[\dfrac{d}{dx}\left(\frac{1}{x^{3} - 8 \, x^{2} + 21 \, x - 18}\right)=\answer{-\frac{3 \, x^{2} - 16 \, x + 21}{{\left(x^{3} - 8 \, x^{2} + 21 \, x - 18\right)}^{2}}}\]
\end{problem}}

%%%%%%%%%%%%%%%%%%%%%%

\latexProblemContent{
\ifVerboseLocation This is Derivative Compute Question 0004. \\ \fi
\begin{problem}

Compute the following derivative:

\input{Derivative-Compute-0004.HELP.tex}

\[\dfrac{d}{dx}\left(\frac{1}{x^{3} - 19 \, x - 30}\right)=\answer{-\frac{3 \, x^{2} - 19}{{\left(x^{3} - 19 \, x - 30\right)}^{2}}}\]
\end{problem}}

%%%%%%%%%%%%%%%%%%%%%%

\latexProblemContent{
\ifVerboseLocation This is Derivative Compute Question 0004. \\ \fi
\begin{problem}

Compute the following derivative:

\input{Derivative-Compute-0004.HELP.tex}

\[\dfrac{d}{dx}\left(\frac{x^{2} + x - 2}{x^{2} - 9 \, x + 20}\right)=\answer{-\frac{2 \, {\left(5 \, x^{2} - 22 \, x - 1\right)}}{{\left(x^{2} - 9 \, x + 20\right)}^{2}}}\]
\end{problem}}

%%%%%%%%%%%%%%%%%%%%%%

\latexProblemContent{
\ifVerboseLocation This is Derivative Compute Question 0004. \\ \fi
\begin{problem}

Compute the following derivative:

\input{Derivative-Compute-0004.HELP.tex}

\[\dfrac{d}{dx}\left(\frac{x + 5}{x + 1}\right)=\answer{-\frac{4}{{\left(x + 1\right)}^{2}}}\]
\end{problem}}

%%%%%%%%%%%%%%%%%%%%%%

\latexProblemContent{
\ifVerboseLocation This is Derivative Compute Question 0004. \\ \fi
\begin{problem}

Compute the following derivative:

\input{Derivative-Compute-0004.HELP.tex}

\[\dfrac{d}{dx}\left(\frac{x + 3}{x^{3} - 3 \, x^{2} - 6 \, x + 8}\right)=\answer{-\frac{2 \, {\left(x^{3} + 3 \, x^{2} - 9 \, x - 13\right)}}{{\left(x^{3} - 3 \, x^{2} - 6 \, x + 8\right)}^{2}}}\]
\end{problem}}

%%%%%%%%%%%%%%%%%%%%%%

\latexProblemContent{
\ifVerboseLocation This is Derivative Compute Question 0004. \\ \fi
\begin{problem}

Compute the following derivative:

\input{Derivative-Compute-0004.HELP.tex}

\[\dfrac{d}{dx}\left(\frac{1}{x^{2} + x - 12}\right)=\answer{-\frac{2 \, x + 1}{{\left(x^{2} + x - 12\right)}^{2}}}\]
\end{problem}}

%%%%%%%%%%%%%%%%%%%%%%

\latexProblemContent{
\ifVerboseLocation This is Derivative Compute Question 0004. \\ \fi
\begin{problem}

Compute the following derivative:

\input{Derivative-Compute-0004.HELP.tex}

\[\dfrac{d}{dx}\left(\frac{x^{2} - 6 \, x + 5}{x - 2}\right)=\answer{\frac{x^{2} - 4 \, x + 7}{{\left(x - 2\right)}^{2}}}\]
\end{problem}}

%%%%%%%%%%%%%%%%%%%%%%

\latexProblemContent{
\ifVerboseLocation This is Derivative Compute Question 0004. \\ \fi
\begin{problem}

Compute the following derivative:

\input{Derivative-Compute-0004.HELP.tex}

\[\dfrac{d}{dx}\left(\frac{x - 3}{x^{3} + 9 \, x^{2} + 27 \, x + 27}\right)=\answer{-\frac{2 \, {\left(x^{3} - 27 \, x - 54\right)}}{{\left(x^{3} + 9 \, x^{2} + 27 \, x + 27\right)}^{2}}}\]
\end{problem}}

%%%%%%%%%%%%%%%%%%%%%%

\latexProblemContent{
\ifVerboseLocation This is Derivative Compute Question 0004. \\ \fi
\begin{problem}

Compute the following derivative:

\input{Derivative-Compute-0004.HELP.tex}

\[\dfrac{d}{dx}\left(\frac{x^{2} - 2 \, x - 3}{x^{3} + 2 \, x^{2} - 25 \, x - 50}\right)=\answer{-\frac{x^{4} - 4 \, x^{3} + 12 \, x^{2} + 88 \, x - 25}{{\left(x^{3} + 2 \, x^{2} - 25 \, x - 50\right)}^{2}}}\]
\end{problem}}

%%%%%%%%%%%%%%%%%%%%%%

\latexProblemContent{
\ifVerboseLocation This is Derivative Compute Question 0004. \\ \fi
\begin{problem}

Compute the following derivative:

\input{Derivative-Compute-0004.HELP.tex}

\[\dfrac{d}{dx}\left(\frac{x^{2} - 8 \, x + 15}{x^{2} + x - 6}\right)=\answer{\frac{3 \, {\left(3 \, x^{2} - 14 \, x + 11\right)}}{{\left(x^{2} + x - 6\right)}^{2}}}\]
\end{problem}}

%%%%%%%%%%%%%%%%%%%%%%

\latexProblemContent{
\ifVerboseLocation This is Derivative Compute Question 0004. \\ \fi
\begin{problem}

Compute the following derivative:

\input{Derivative-Compute-0004.HELP.tex}

\[\dfrac{d}{dx}\left(\frac{1}{x^{2} - 25}\right)=\answer{-\frac{2 \, x}{{\left(x^{2} - 25\right)}^{2}}}\]
\end{problem}}

%%%%%%%%%%%%%%%%%%%%%%

\latexProblemContent{
\ifVerboseLocation This is Derivative Compute Question 0004. \\ \fi
\begin{problem}

Compute the following derivative:

\input{Derivative-Compute-0004.HELP.tex}

\[\dfrac{d}{dx}\left(\frac{x - 2}{x^{2} + 7 \, x + 10}\right)=\answer{-\frac{x^{2} - 4 \, x - 24}{{\left(x^{2} + 7 \, x + 10\right)}^{2}}}\]
\end{problem}}

%%%%%%%%%%%%%%%%%%%%%%

\latexProblemContent{
\ifVerboseLocation This is Derivative Compute Question 0004. \\ \fi
\begin{problem}

Compute the following derivative:

\input{Derivative-Compute-0004.HELP.tex}

\[\dfrac{d}{dx}\left(\frac{1}{x^{3} + 6 \, x^{2} + 3 \, x - 10}\right)=\answer{-\frac{3 \, {\left(x^{2} + 4 \, x + 1\right)}}{{\left(x^{3} + 6 \, x^{2} + 3 \, x - 10\right)}^{2}}}\]
\end{problem}}

%%%%%%%%%%%%%%%%%%%%%%

\latexProblemContent{
\ifVerboseLocation This is Derivative Compute Question 0004. \\ \fi
\begin{problem}

Compute the following derivative:

\input{Derivative-Compute-0004.HELP.tex}

\[\dfrac{d}{dx}\left(\frac{1}{x^{3} - 3 \, x^{2} + 3 \, x - 1}\right)=\answer{-\frac{3 \, {\left(x^{2} - 2 \, x + 1\right)}}{{\left(x^{3} - 3 \, x^{2} + 3 \, x - 1\right)}^{2}}}\]
\end{problem}}

%%%%%%%%%%%%%%%%%%%%%%

\latexProblemContent{
\ifVerboseLocation This is Derivative Compute Question 0004. \\ \fi
\begin{problem}

Compute the following derivative:

\input{Derivative-Compute-0004.HELP.tex}

\[\dfrac{d}{dx}\left(\frac{x^{3} - 2 \, x^{2} - x + 2}{x + 2}\right)=\answer{\frac{2 \, {\left(x^{3} + 2 \, x^{2} - 4 \, x - 2\right)}}{{\left(x + 2\right)}^{2}}}\]
\end{problem}}

%%%%%%%%%%%%%%%%%%%%%%

\latexProblemContent{
\ifVerboseLocation This is Derivative Compute Question 0004. \\ \fi
\begin{problem}

Compute the following derivative:

\input{Derivative-Compute-0004.HELP.tex}

\[\dfrac{d}{dx}\left(\frac{1}{x^{2} + 4 \, x - 5}\right)=\answer{-\frac{2 \, {\left(x + 2\right)}}{{\left(x^{2} + 4 \, x - 5\right)}^{2}}}\]
\end{problem}}

%%%%%%%%%%%%%%%%%%%%%%

\latexProblemContent{
\ifVerboseLocation This is Derivative Compute Question 0004. \\ \fi
\begin{problem}

Compute the following derivative:

\input{Derivative-Compute-0004.HELP.tex}

\[\dfrac{d}{dx}\left(\frac{x + 1}{x^{3} - 14 \, x^{2} + 65 \, x - 100}\right)=\answer{-\frac{2 \, x^{3} - 11 \, x^{2} - 28 \, x + 165}{{\left(x^{3} - 14 \, x^{2} + 65 \, x - 100\right)}^{2}}}\]
\end{problem}}

%%%%%%%%%%%%%%%%%%%%%%

\latexProblemContent{
\ifVerboseLocation This is Derivative Compute Question 0004. \\ \fi
\begin{problem}

Compute the following derivative:

\input{Derivative-Compute-0004.HELP.tex}

\[\dfrac{d}{dx}\left(\frac{x^{2} - x - 20}{x^{2} + 3 \, x - 10}\right)=\answer{\frac{2 \, {\left(2 \, x^{2} + 10 \, x + 35\right)}}{{\left(x^{2} + 3 \, x - 10\right)}^{2}}}\]
\end{problem}}

%%%%%%%%%%%%%%%%%%%%%%

\latexProblemContent{
\ifVerboseLocation This is Derivative Compute Question 0004. \\ \fi
\begin{problem}

Compute the following derivative:

\input{Derivative-Compute-0004.HELP.tex}

\[\dfrac{d}{dx}\left(\frac{1}{x^{3} + 5 \, x^{2} + 2 \, x - 8}\right)=\answer{-\frac{3 \, x^{2} + 10 \, x + 2}{{\left(x^{3} + 5 \, x^{2} + 2 \, x - 8\right)}^{2}}}\]
\end{problem}}

%%%%%%%%%%%%%%%%%%%%%%

\latexProblemContent{
\ifVerboseLocation This is Derivative Compute Question 0004. \\ \fi
\begin{problem}

Compute the following derivative:

\input{Derivative-Compute-0004.HELP.tex}

\[\dfrac{d}{dx}\left(\frac{1}{x^{2} - x - 2}\right)=\answer{-\frac{2 \, x - 1}{{\left(x^{2} - x - 2\right)}^{2}}}\]
\end{problem}}

%%%%%%%%%%%%%%%%%%%%%%

\latexProblemContent{
\ifVerboseLocation This is Derivative Compute Question 0004. \\ \fi
\begin{problem}

Compute the following derivative:

\input{Derivative-Compute-0004.HELP.tex}

\[\dfrac{d}{dx}\left(\frac{x^{2} + 5 \, x + 4}{x - 4}\right)=\answer{\frac{x^{2} - 8 \, x - 24}{{\left(x - 4\right)}^{2}}}\]
\end{problem}}

%%%%%%%%%%%%%%%%%%%%%%

\latexProblemContent{
\ifVerboseLocation This is Derivative Compute Question 0004. \\ \fi
\begin{problem}

Compute the following derivative:

\input{Derivative-Compute-0004.HELP.tex}

\[\dfrac{d}{dx}\left(\frac{x + 2}{x - 3}\right)=\answer{-\frac{5}{{\left(x - 3\right)}^{2}}}\]
\end{problem}}

%%%%%%%%%%%%%%%%%%%%%%

\latexProblemContent{
\ifVerboseLocation This is Derivative Compute Question 0004. \\ \fi
\begin{problem}

Compute the following derivative:

\input{Derivative-Compute-0004.HELP.tex}

\[\dfrac{d}{dx}\left(\frac{x^{2} - x - 2}{x^{2} - 6 \, x + 5}\right)=\answer{-\frac{5 \, x^{2} - 14 \, x + 17}{{\left(x^{2} - 6 \, x + 5\right)}^{2}}}\]
\end{problem}}

%%%%%%%%%%%%%%%%%%%%%%

\latexProblemContent{
\ifVerboseLocation This is Derivative Compute Question 0004. \\ \fi
\begin{problem}

Compute the following derivative:

\input{Derivative-Compute-0004.HELP.tex}

\[\dfrac{d}{dx}\left(\frac{x + 2}{x^{3} + x^{2} - 9 \, x - 9}\right)=\answer{-\frac{2 \, x^{3} + 7 \, x^{2} + 4 \, x - 9}{{\left(x^{3} + x^{2} - 9 \, x - 9\right)}^{2}}}\]
\end{problem}}

%%%%%%%%%%%%%%%%%%%%%%

\latexProblemContent{
\ifVerboseLocation This is Derivative Compute Question 0004. \\ \fi
\begin{problem}

Compute the following derivative:

\input{Derivative-Compute-0004.HELP.tex}

\[\dfrac{d}{dx}\left(\frac{x + 1}{x^{2} - 8 \, x + 16}\right)=\answer{-\frac{x^{2} + 2 \, x - 24}{{\left(x^{2} - 8 \, x + 16\right)}^{2}}}\]
\end{problem}}

%%%%%%%%%%%%%%%%%%%%%%

\latexProblemContent{
\ifVerboseLocation This is Derivative Compute Question 0004. \\ \fi
\begin{problem}

Compute the following derivative:

\input{Derivative-Compute-0004.HELP.tex}

\[\dfrac{d}{dx}\left(\frac{1}{x^{2} - 8 \, x + 15}\right)=\answer{-\frac{2 \, {\left(x - 4\right)}}{{\left(x^{2} - 8 \, x + 15\right)}^{2}}}\]
\end{problem}}

%%%%%%%%%%%%%%%%%%%%%%

\latexProblemContent{
\ifVerboseLocation This is Derivative Compute Question 0004. \\ \fi
\begin{problem}

Compute the following derivative:

\input{Derivative-Compute-0004.HELP.tex}

\[\dfrac{d}{dx}\left(\frac{x^{2} - 7 \, x + 12}{x^{2} + 6 \, x + 8}\right)=\answer{\frac{13 \, x^{2} - 8 \, x - 128}{{\left(x^{2} + 6 \, x + 8\right)}^{2}}}\]
\end{problem}}

%%%%%%%%%%%%%%%%%%%%%%

\latexProblemContent{
\ifVerboseLocation This is Derivative Compute Question 0004. \\ \fi
\begin{problem}

Compute the following derivative:

\input{Derivative-Compute-0004.HELP.tex}

\[\dfrac{d}{dx}\left(\frac{x + 1}{x^{3} - 5 \, x^{2} - 25 \, x + 125}\right)=\answer{-\frac{2 \, {\left(x^{3} - x^{2} - 5 \, x - 75\right)}}{{\left(x^{3} - 5 \, x^{2} - 25 \, x + 125\right)}^{2}}}\]
\end{problem}}

%%%%%%%%%%%%%%%%%%%%%%

\latexProblemContent{
\ifVerboseLocation This is Derivative Compute Question 0004. \\ \fi
\begin{problem}

Compute the following derivative:

\input{Derivative-Compute-0004.HELP.tex}

\[\dfrac{d}{dx}\left(\frac{1}{x^{2} + 6 \, x + 8}\right)=\answer{-\frac{2 \, {\left(x + 3\right)}}{{\left(x^{2} + 6 \, x + 8\right)}^{2}}}\]
\end{problem}}

%%%%%%%%%%%%%%%%%%%%%%

\latexProblemContent{
\ifVerboseLocation This is Derivative Compute Question 0004. \\ \fi
\begin{problem}

Compute the following derivative:

\input{Derivative-Compute-0004.HELP.tex}

\[\dfrac{d}{dx}\left(\frac{x^{3} - 10 \, x^{2} + 29 \, x - 20}{x^{3} - 5 \, x^{2} + 2 \, x + 8}\right)=\answer{\frac{5 \, x^{4} - 54 \, x^{3} + 209 \, x^{2} - 360 \, x + 272}{{\left(x^{3} - 5 \, x^{2} + 2 \, x + 8\right)}^{2}}}\]
\end{problem}}

%%%%%%%%%%%%%%%%%%%%%%

\latexProblemContent{
\ifVerboseLocation This is Derivative Compute Question 0004. \\ \fi
\begin{problem}

Compute the following derivative:

\input{Derivative-Compute-0004.HELP.tex}

\[\dfrac{d}{dx}\left(\frac{x^{2} - 7 \, x + 12}{x^{2} - 16}\right)=\answer{\frac{7 \, {\left(x^{2} - 8 \, x + 16\right)}}{{\left(x^{2} - 16\right)}^{2}}}\]
\end{problem}}

%%%%%%%%%%%%%%%%%%%%%%

\latexProblemContent{
\ifVerboseLocation This is Derivative Compute Question 0004. \\ \fi
\begin{problem}

Compute the following derivative:

\input{Derivative-Compute-0004.HELP.tex}

\[\dfrac{d}{dx}\left(\frac{1}{x^{3} - 13 \, x - 12}\right)=\answer{-\frac{3 \, x^{2} - 13}{{\left(x^{3} - 13 \, x - 12\right)}^{2}}}\]
\end{problem}}

%%%%%%%%%%%%%%%%%%%%%%

\latexProblemContent{
\ifVerboseLocation This is Derivative Compute Question 0004. \\ \fi
\begin{problem}

Compute the following derivative:

\input{Derivative-Compute-0004.HELP.tex}

\[\dfrac{d}{dx}\left(\frac{x + 2}{x^{2} - 16}\right)=\answer{-\frac{x^{2} + 4 \, x + 16}{{\left(x^{2} - 16\right)}^{2}}}\]
\end{problem}}

%%%%%%%%%%%%%%%%%%%%%%

\latexProblemContent{
\ifVerboseLocation This is Derivative Compute Question 0004. \\ \fi
\begin{problem}

Compute the following derivative:

\input{Derivative-Compute-0004.HELP.tex}

\[\dfrac{d}{dx}\left(\frac{x^{2} - 5 \, x + 6}{x^{2} - 9 \, x + 20}\right)=\answer{-\frac{2 \, {\left(2 \, x^{2} - 14 \, x + 23\right)}}{{\left(x^{2} - 9 \, x + 20\right)}^{2}}}\]
\end{problem}}

%%%%%%%%%%%%%%%%%%%%%%

\latexProblemContent{
\ifVerboseLocation This is Derivative Compute Question 0004. \\ \fi
\begin{problem}

Compute the following derivative:

\input{Derivative-Compute-0004.HELP.tex}

\[\dfrac{d}{dx}\left(\frac{1}{x^{3} - 5 \, x^{2} - 9 \, x + 45}\right)=\answer{-\frac{3 \, x^{2} - 10 \, x - 9}{{\left(x^{3} - 5 \, x^{2} - 9 \, x + 45\right)}^{2}}}\]
\end{problem}}

%%%%%%%%%%%%%%%%%%%%%%

\latexProblemContent{
\ifVerboseLocation This is Derivative Compute Question 0004. \\ \fi
\begin{problem}

Compute the following derivative:

\input{Derivative-Compute-0004.HELP.tex}

\[\dfrac{d}{dx}\left(\frac{x^{3} - 9 \, x^{2} + 23 \, x - 15}{x^{3} - 9 \, x^{2} + 26 \, x - 24}\right)=\answer{\frac{6 \, {\left(x^{3} - 9 \, x^{2} + 27 \, x - 27\right)}}{{\left(x^{3} - 9 \, x^{2} + 26 \, x - 24\right)}^{2}}}\]
\end{problem}}

%%%%%%%%%%%%%%%%%%%%%%

\latexProblemContent{
\ifVerboseLocation This is Derivative Compute Question 0004. \\ \fi
\begin{problem}

Compute the following derivative:

\input{Derivative-Compute-0004.HELP.tex}

\[\dfrac{d}{dx}\left(\frac{x^{2} - x - 2}{x + 4}\right)=\answer{\frac{x^{2} + 8 \, x - 2}{{\left(x + 4\right)}^{2}}}\]
\end{problem}}

%%%%%%%%%%%%%%%%%%%%%%

\latexProblemContent{
\ifVerboseLocation This is Derivative Compute Question 0004. \\ \fi
\begin{problem}

Compute the following derivative:

\input{Derivative-Compute-0004.HELP.tex}

\[\dfrac{d}{dx}\left(\frac{x^{2} + x - 2}{x - 2}\right)=\answer{\frac{x^{2} - 4 \, x}{{\left(x - 2\right)}^{2}}}\]
\end{problem}}

%%%%%%%%%%%%%%%%%%%%%%

\latexProblemContent{
\ifVerboseLocation This is Derivative Compute Question 0004. \\ \fi
\begin{problem}

Compute the following derivative:

\input{Derivative-Compute-0004.HELP.tex}

\[\dfrac{d}{dx}\left(\frac{x + 1}{x^{3} - 2 \, x^{2} - 25 \, x + 50}\right)=\answer{-\frac{2 \, x^{3} + x^{2} - 4 \, x - 75}{{\left(x^{3} - 2 \, x^{2} - 25 \, x + 50\right)}^{2}}}\]
\end{problem}}

%%%%%%%%%%%%%%%%%%%%%%

\latexProblemContent{
\ifVerboseLocation This is Derivative Compute Question 0004. \\ \fi
\begin{problem}

Compute the following derivative:

\input{Derivative-Compute-0004.HELP.tex}

\[\dfrac{d}{dx}\left(\frac{1}{x - 2}\right)=\answer{-\frac{1}{{\left(x - 2\right)}^{2}}}\]
\end{problem}}

%%%%%%%%%%%%%%%%%%%%%%

\latexProblemContent{
\ifVerboseLocation This is Derivative Compute Question 0004. \\ \fi
\begin{problem}

Compute the following derivative:

\input{Derivative-Compute-0004.HELP.tex}

\[\dfrac{d}{dx}\left(\frac{1}{x^{3} + x^{2} - 5 \, x + 3}\right)=\answer{-\frac{3 \, x^{2} + 2 \, x - 5}{{\left(x^{3} + x^{2} - 5 \, x + 3\right)}^{2}}}\]
\end{problem}}

%%%%%%%%%%%%%%%%%%%%%%

\latexProblemContent{
\ifVerboseLocation This is Derivative Compute Question 0004. \\ \fi
\begin{problem}

Compute the following derivative:

\input{Derivative-Compute-0004.HELP.tex}

\[\dfrac{d}{dx}\left(\frac{x^{2} - 5 \, x + 6}{x + 3}\right)=\answer{\frac{x^{2} + 6 \, x - 21}{{\left(x + 3\right)}^{2}}}\]
\end{problem}}

%%%%%%%%%%%%%%%%%%%%%%

\latexProblemContent{
\ifVerboseLocation This is Derivative Compute Question 0004. \\ \fi
\begin{problem}

Compute the following derivative:

\input{Derivative-Compute-0004.HELP.tex}

\[\dfrac{d}{dx}\left(\frac{1}{x^{3} + 7 \, x^{2} + 11 \, x + 5}\right)=\answer{-\frac{3 \, x^{2} + 14 \, x + 11}{{\left(x^{3} + 7 \, x^{2} + 11 \, x + 5\right)}^{2}}}\]
\end{problem}}

%%%%%%%%%%%%%%%%%%%%%%

\latexProblemContent{
\ifVerboseLocation This is Derivative Compute Question 0004. \\ \fi
\begin{problem}

Compute the following derivative:

\input{Derivative-Compute-0004.HELP.tex}

\[\dfrac{d}{dx}\left(\frac{x - 5}{x^{2} + 2 \, x - 15}\right)=\answer{-\frac{x^{2} - 10 \, x + 5}{{\left(x^{2} + 2 \, x - 15\right)}^{2}}}\]
\end{problem}}

%%%%%%%%%%%%%%%%%%%%%%

\latexProblemContent{
\ifVerboseLocation This is Derivative Compute Question 0004. \\ \fi
\begin{problem}

Compute the following derivative:

\input{Derivative-Compute-0004.HELP.tex}

\[\dfrac{d}{dx}\left(\frac{x^{3} + 8 \, x^{2} + 17 \, x + 10}{x - 5}\right)=\answer{\frac{2 \, x^{3} - 7 \, x^{2} - 80 \, x - 95}{{\left(x - 5\right)}^{2}}}\]
\end{problem}}

%%%%%%%%%%%%%%%%%%%%%%

\latexProblemContent{
\ifVerboseLocation This is Derivative Compute Question 0004. \\ \fi
\begin{problem}

Compute the following derivative:

\input{Derivative-Compute-0004.HELP.tex}

\[\dfrac{d}{dx}\left(\frac{1}{x^{2} - x - 6}\right)=\answer{-\frac{2 \, x - 1}{{\left(x^{2} - x - 6\right)}^{2}}}\]
\end{problem}}

%%%%%%%%%%%%%%%%%%%%%%

\latexProblemContent{
\ifVerboseLocation This is Derivative Compute Question 0004. \\ \fi
\begin{problem}

Compute the following derivative:

\input{Derivative-Compute-0004.HELP.tex}

\[\dfrac{d}{dx}\left(\frac{x^{2} - 6 \, x + 5}{x^{3} - 3 \, x - 2}\right)=\answer{-\frac{x^{4} - 12 \, x^{3} + 18 \, x^{2} + 4 \, x - 27}{{\left(x^{3} - 3 \, x - 2\right)}^{2}}}\]
\end{problem}}

%%%%%%%%%%%%%%%%%%%%%%

\latexProblemContent{
\ifVerboseLocation This is Derivative Compute Question 0004. \\ \fi
\begin{problem}

Compute the following derivative:

\input{Derivative-Compute-0004.HELP.tex}

\[\dfrac{d}{dx}\left(\frac{x^{3} - 11 \, x^{2} + 40 \, x - 48}{x^{3} - 7 \, x^{2} + 14 \, x - 8}\right)=\answer{\frac{2 \, {\left(2 \, x^{4} - 26 \, x^{3} + 123 \, x^{2} - 248 \, x + 176\right)}}{{\left(x^{3} - 7 \, x^{2} + 14 \, x - 8\right)}^{2}}}\]
\end{problem}}

%%%%%%%%%%%%%%%%%%%%%%

\latexProblemContent{
\ifVerboseLocation This is Derivative Compute Question 0004. \\ \fi
\begin{problem}

Compute the following derivative:

\input{Derivative-Compute-0004.HELP.tex}

\[\dfrac{d}{dx}\left(\frac{x + 4}{x - 5}\right)=\answer{-\frac{9}{{\left(x - 5\right)}^{2}}}\]
\end{problem}}

%%%%%%%%%%%%%%%%%%%%%%

\latexProblemContent{
\ifVerboseLocation This is Derivative Compute Question 0004. \\ \fi
\begin{problem}

Compute the following derivative:

\input{Derivative-Compute-0004.HELP.tex}

\[\dfrac{d}{dx}\left(\frac{x^{3} - x^{2} - 14 \, x + 24}{x^{3} - 7 \, x^{2} + 15 \, x - 9}\right)=\answer{-\frac{2 \, {\left(3 \, x^{4} - 29 \, x^{3} + 106 \, x^{2} - 177 \, x + 117\right)}}{{\left(x^{3} - 7 \, x^{2} + 15 \, x - 9\right)}^{2}}}\]
\end{problem}}

%%%%%%%%%%%%%%%%%%%%%%

\latexProblemContent{
\ifVerboseLocation This is Derivative Compute Question 0004. \\ \fi
\begin{problem}

Compute the following derivative:

\input{Derivative-Compute-0004.HELP.tex}

\[\dfrac{d}{dx}\left(\frac{x^{2} + 3 \, x - 4}{x + 4}\right)=\answer{\frac{x^{2} + 8 \, x + 16}{{\left(x + 4\right)}^{2}}}\]
\end{problem}}

%%%%%%%%%%%%%%%%%%%%%%

\latexProblemContent{
\ifVerboseLocation This is Derivative Compute Question 0004. \\ \fi
\begin{problem}

Compute the following derivative:

\input{Derivative-Compute-0004.HELP.tex}

\[\dfrac{d}{dx}\left(\frac{x^{2} - x - 6}{x^{2} + 4 \, x + 3}\right)=\answer{\frac{5 \, x^{2} + 18 \, x + 21}{{\left(x^{2} + 4 \, x + 3\right)}^{2}}}\]
\end{problem}}

%%%%%%%%%%%%%%%%%%%%%%

\latexProblemContent{
\ifVerboseLocation This is Derivative Compute Question 0004. \\ \fi
\begin{problem}

Compute the following derivative:

\input{Derivative-Compute-0004.HELP.tex}

\[\dfrac{d}{dx}\left(\frac{x^{2} - x - 2}{x^{3} + 5 \, x^{2} + 2 \, x - 8}\right)=\answer{-\frac{x^{4} - 2 \, x^{3} - 13 \, x^{2} - 4 \, x - 12}{{\left(x^{3} + 5 \, x^{2} + 2 \, x - 8\right)}^{2}}}\]
\end{problem}}

%%%%%%%%%%%%%%%%%%%%%%

\latexProblemContent{
\ifVerboseLocation This is Derivative Compute Question 0004. \\ \fi
\begin{problem}

Compute the following derivative:

\input{Derivative-Compute-0004.HELP.tex}

\[\dfrac{d}{dx}\left(\frac{1}{x^{2} - 9 \, x + 20}\right)=\answer{-\frac{2 \, x - 9}{{\left(x^{2} - 9 \, x + 20\right)}^{2}}}\]
\end{problem}}

%%%%%%%%%%%%%%%%%%%%%%

\latexProblemContent{
\ifVerboseLocation This is Derivative Compute Question 0004. \\ \fi
\begin{problem}

Compute the following derivative:

\input{Derivative-Compute-0004.HELP.tex}

\[\dfrac{d}{dx}\left(\frac{x^{2} + 6 \, x + 8}{x^{2} + 6 \, x + 5}\right)=\answer{-\frac{6 \, {\left(x + 3\right)}}{{\left(x^{2} + 6 \, x + 5\right)}^{2}}}\]
\end{problem}}

%%%%%%%%%%%%%%%%%%%%%%

\latexProblemContent{
\ifVerboseLocation This is Derivative Compute Question 0004. \\ \fi
\begin{problem}

Compute the following derivative:

\input{Derivative-Compute-0004.HELP.tex}

\[\dfrac{d}{dx}\left(\frac{x^{2} - 7 \, x + 10}{x^{2} - 2 \, x - 8}\right)=\answer{\frac{5 \, x^{2} - 36 \, x + 76}{{\left(x^{2} - 2 \, x - 8\right)}^{2}}}\]
\end{problem}}

%%%%%%%%%%%%%%%%%%%%%%

\latexProblemContent{
\ifVerboseLocation This is Derivative Compute Question 0004. \\ \fi
\begin{problem}

Compute the following derivative:

\input{Derivative-Compute-0004.HELP.tex}

\[\dfrac{d}{dx}\left(\frac{x^{2} + 5 \, x + 6}{x^{3} - 7 \, x^{2} + 8 \, x + 16}\right)=\answer{-\frac{x^{4} + 10 \, x^{3} - 25 \, x^{2} - 116 \, x - 32}{{\left(x^{3} - 7 \, x^{2} + 8 \, x + 16\right)}^{2}}}\]
\end{problem}}

%%%%%%%%%%%%%%%%%%%%%%

\latexProblemContent{
\ifVerboseLocation This is Derivative Compute Question 0004. \\ \fi
\begin{problem}

Compute the following derivative:

\input{Derivative-Compute-0004.HELP.tex}

\[\dfrac{d}{dx}\left(\frac{x^{2} + x - 12}{x^{2} + 2 \, x - 8}\right)=\answer{\frac{x^{2} + 8 \, x + 16}{{\left(x^{2} + 2 \, x - 8\right)}^{2}}}\]
\end{problem}}

%%%%%%%%%%%%%%%%%%%%%%

\latexProblemContent{
\ifVerboseLocation This is Derivative Compute Question 0004. \\ \fi
\begin{problem}

Compute the following derivative:

\input{Derivative-Compute-0004.HELP.tex}

\[\dfrac{d}{dx}\left(\frac{x^{2} + 2 \, x - 3}{x^{2} - 1}\right)=\answer{-\frac{2 \, {\left(x^{2} - 2 \, x + 1\right)}}{{\left(x^{2} - 1\right)}^{2}}}\]
\end{problem}}

%%%%%%%%%%%%%%%%%%%%%%

\latexProblemContent{
\ifVerboseLocation This is Derivative Compute Question 0004. \\ \fi
\begin{problem}

Compute the following derivative:

\input{Derivative-Compute-0004.HELP.tex}

\[\dfrac{d}{dx}\left(\frac{x + 5}{x^{2} + 7 \, x + 12}\right)=\answer{-\frac{x^{2} + 10 \, x + 23}{{\left(x^{2} + 7 \, x + 12\right)}^{2}}}\]
\end{problem}}

%%%%%%%%%%%%%%%%%%%%%%

\latexProblemContent{
\ifVerboseLocation This is Derivative Compute Question 0004. \\ \fi
\begin{problem}

Compute the following derivative:

\input{Derivative-Compute-0004.HELP.tex}

\[\dfrac{d}{dx}\left(\frac{x^{2} - 4 \, x + 3}{x - 4}\right)=\answer{\frac{x^{2} - 8 \, x + 13}{{\left(x - 4\right)}^{2}}}\]
\end{problem}}

%%%%%%%%%%%%%%%%%%%%%%

\latexProblemContent{
\ifVerboseLocation This is Derivative Compute Question 0004. \\ \fi
\begin{problem}

Compute the following derivative:

\input{Derivative-Compute-0004.HELP.tex}

\[\dfrac{d}{dx}\left(\frac{x - 5}{x^{3} + 5 \, x^{2} - 16 \, x - 80}\right)=\answer{-\frac{2 \, {\left(x^{3} - 5 \, x^{2} - 25 \, x + 80\right)}}{{\left(x^{3} + 5 \, x^{2} - 16 \, x - 80\right)}^{2}}}\]
\end{problem}}

%%%%%%%%%%%%%%%%%%%%%%

\latexProblemContent{
\ifVerboseLocation This is Derivative Compute Question 0004. \\ \fi
\begin{problem}

Compute the following derivative:

\input{Derivative-Compute-0004.HELP.tex}

\[\dfrac{d}{dx}\left(\frac{x^{3} - 4 \, x^{2} - 25 \, x + 100}{x - 4}\right)=\answer{\frac{2 \, {\left(x^{3} - 8 \, x^{2} + 16 \, x\right)}}{{\left(x - 4\right)}^{2}}}\]
\end{problem}}

%%%%%%%%%%%%%%%%%%%%%%

\latexProblemContent{
\ifVerboseLocation This is Derivative Compute Question 0004. \\ \fi
\begin{problem}

Compute the following derivative:

\input{Derivative-Compute-0004.HELP.tex}

\[\dfrac{d}{dx}\left(\frac{x + 5}{x + 3}\right)=\answer{-\frac{2}{{\left(x + 3\right)}^{2}}}\]
\end{problem}}

%%%%%%%%%%%%%%%%%%%%%%

\latexProblemContent{
\ifVerboseLocation This is Derivative Compute Question 0004. \\ \fi
\begin{problem}

Compute the following derivative:

\input{Derivative-Compute-0004.HELP.tex}

\[\dfrac{d}{dx}\left(\frac{x - 3}{x^{3} + 7 \, x^{2} + 15 \, x + 9}\right)=\answer{-\frac{2 \, {\left(x^{3} - x^{2} - 21 \, x - 27\right)}}{{\left(x^{3} + 7 \, x^{2} + 15 \, x + 9\right)}^{2}}}\]
\end{problem}}

%%%%%%%%%%%%%%%%%%%%%%

\latexProblemContent{
\ifVerboseLocation This is Derivative Compute Question 0004. \\ \fi
\begin{problem}

Compute the following derivative:

\input{Derivative-Compute-0004.HELP.tex}

\[\dfrac{d}{dx}\left(\frac{x^{2} + 3 \, x - 4}{x^{3} + 2 \, x^{2} - 16 \, x - 32}\right)=\answer{-\frac{x^{4} + 6 \, x^{3} + 10 \, x^{2} + 48 \, x + 160}{{\left(x^{3} + 2 \, x^{2} - 16 \, x - 32\right)}^{2}}}\]
\end{problem}}

%%%%%%%%%%%%%%%%%%%%%%

\latexProblemContent{
\ifVerboseLocation This is Derivative Compute Question 0004. \\ \fi
\begin{problem}

Compute the following derivative:

\input{Derivative-Compute-0004.HELP.tex}

\[\dfrac{d}{dx}\left(\frac{x^{2} + x - 12}{x^{3} - 8 \, x^{2} + 11 \, x + 20}\right)=\answer{-\frac{x^{4} + 2 \, x^{3} - 55 \, x^{2} + 152 \, x - 152}{{\left(x^{3} - 8 \, x^{2} + 11 \, x + 20\right)}^{2}}}\]
\end{problem}}

%%%%%%%%%%%%%%%%%%%%%%

\latexProblemContent{
\ifVerboseLocation This is Derivative Compute Question 0004. \\ \fi
\begin{problem}

Compute the following derivative:

\input{Derivative-Compute-0004.HELP.tex}

\[\dfrac{d}{dx}\left(\frac{x - 3}{x^{3} + 5 \, x^{2} + 7 \, x + 3}\right)=\answer{-\frac{2 \, {\left(x^{3} - 2 \, x^{2} - 15 \, x - 12\right)}}{{\left(x^{3} + 5 \, x^{2} + 7 \, x + 3\right)}^{2}}}\]
\end{problem}}

%%%%%%%%%%%%%%%%%%%%%%

\latexProblemContent{
\ifVerboseLocation This is Derivative Compute Question 0004. \\ \fi
\begin{problem}

Compute the following derivative:

\input{Derivative-Compute-0004.HELP.tex}

\[\dfrac{d}{dx}\left(\frac{x^{2} - 3 \, x - 10}{x^{2} + 2 \, x - 8}\right)=\answer{\frac{5 \, x^{2} + 4 \, x + 44}{{\left(x^{2} + 2 \, x - 8\right)}^{2}}}\]
\end{problem}}

%%%%%%%%%%%%%%%%%%%%%%

\latexProblemContent{
\ifVerboseLocation This is Derivative Compute Question 0004. \\ \fi
\begin{problem}

Compute the following derivative:

\input{Derivative-Compute-0004.HELP.tex}

\[\dfrac{d}{dx}\left(\frac{x^{2} - 1}{x^{3} - x^{2} - 5 \, x - 3}\right)=\answer{-\frac{x^{4} + 2 \, x^{2} + 8 \, x + 5}{{\left(x^{3} - x^{2} - 5 \, x - 3\right)}^{2}}}\]
\end{problem}}

%%%%%%%%%%%%%%%%%%%%%%

\latexProblemContent{
\ifVerboseLocation This is Derivative Compute Question 0004. \\ \fi
\begin{problem}

Compute the following derivative:

\input{Derivative-Compute-0004.HELP.tex}

\[\dfrac{d}{dx}\left(\frac{x - 2}{x^{3} - 5 \, x^{2} - 2 \, x + 24}\right)=\answer{-\frac{2 \, x^{3} - 11 \, x^{2} + 20 \, x - 20}{{\left(x^{3} - 5 \, x^{2} - 2 \, x + 24\right)}^{2}}}\]
\end{problem}}

%%%%%%%%%%%%%%%%%%%%%%

\latexProblemContent{
\ifVerboseLocation This is Derivative Compute Question 0004. \\ \fi
\begin{problem}

Compute the following derivative:

\input{Derivative-Compute-0004.HELP.tex}

\[\dfrac{d}{dx}\left(\frac{x^{3} - 2 \, x^{2} - 5 \, x + 6}{x^{2} + 3 \, x - 10}\right)=\answer{\frac{x^{4} + 6 \, x^{3} - 31 \, x^{2} + 28 \, x + 32}{{\left(x^{2} + 3 \, x - 10\right)}^{2}}}\]
\end{problem}}

%%%%%%%%%%%%%%%%%%%%%%

\latexProblemContent{
\ifVerboseLocation This is Derivative Compute Question 0004. \\ \fi
\begin{problem}

Compute the following derivative:

\input{Derivative-Compute-0004.HELP.tex}

\[\dfrac{d}{dx}\left(\frac{x + 4}{x^{2} + x - 20}\right)=\answer{-\frac{x^{2} + 8 \, x + 24}{{\left(x^{2} + x - 20\right)}^{2}}}\]
\end{problem}}

%%%%%%%%%%%%%%%%%%%%%%

\latexProblemContent{
\ifVerboseLocation This is Derivative Compute Question 0004. \\ \fi
\begin{problem}

Compute the following derivative:

\input{Derivative-Compute-0004.HELP.tex}

\[\dfrac{d}{dx}\left(\frac{x^{3} + 5 \, x^{2} + 2 \, x - 8}{x^{3} + 7 \, x^{2} - 5 \, x - 75}\right)=\answer{\frac{2 \, {\left(x^{4} - 7 \, x^{3} - 120 \, x^{2} - 319 \, x - 95\right)}}{{\left(x^{3} + 7 \, x^{2} - 5 \, x - 75\right)}^{2}}}\]
\end{problem}}

%%%%%%%%%%%%%%%%%%%%%%

\latexProblemContent{
\ifVerboseLocation This is Derivative Compute Question 0004. \\ \fi
\begin{problem}

Compute the following derivative:

\input{Derivative-Compute-0004.HELP.tex}

\[\dfrac{d}{dx}\left(\frac{1}{x^{3} - 9 \, x^{2} + 23 \, x - 15}\right)=\answer{-\frac{3 \, x^{2} - 18 \, x + 23}{{\left(x^{3} - 9 \, x^{2} + 23 \, x - 15\right)}^{2}}}\]
\end{problem}}

%%%%%%%%%%%%%%%%%%%%%%

\latexProblemContent{
\ifVerboseLocation This is Derivative Compute Question 0004. \\ \fi
\begin{problem}

Compute the following derivative:

\input{Derivative-Compute-0004.HELP.tex}

\[\dfrac{d}{dx}\left(\frac{x^{2} + 3 \, x + 2}{x^{2} - 3 \, x - 4}\right)=\answer{-\frac{6 \, {\left(x^{2} + 2 \, x + 1\right)}}{{\left(x^{2} - 3 \, x - 4\right)}^{2}}}\]
\end{problem}}

%%%%%%%%%%%%%%%%%%%%%%

\latexProblemContent{
\ifVerboseLocation This is Derivative Compute Question 0004. \\ \fi
\begin{problem}

Compute the following derivative:

\input{Derivative-Compute-0004.HELP.tex}

\[\dfrac{d}{dx}\left(\frac{x^{3} + 5 \, x^{2} - 16 \, x - 80}{x^{2} + 6 \, x + 9}\right)=\answer{\frac{x^{4} + 12 \, x^{3} + 73 \, x^{2} + 250 \, x + 336}{{\left(x^{2} + 6 \, x + 9\right)}^{2}}}\]
\end{problem}}

%%%%%%%%%%%%%%%%%%%%%%

\latexProblemContent{
\ifVerboseLocation This is Derivative Compute Question 0004. \\ \fi
\begin{problem}

Compute the following derivative:

\input{Derivative-Compute-0004.HELP.tex}

\[\dfrac{d}{dx}\left(\frac{x^{2} - 3 \, x - 10}{x^{3} + 7 \, x^{2} + 2 \, x - 40}\right)=\answer{-\frac{x^{4} - 6 \, x^{3} - 53 \, x^{2} - 60 \, x - 140}{{\left(x^{3} + 7 \, x^{2} + 2 \, x - 40\right)}^{2}}}\]
\end{problem}}

%%%%%%%%%%%%%%%%%%%%%%

\latexProblemContent{
\ifVerboseLocation This is Derivative Compute Question 0004. \\ \fi
\begin{problem}

Compute the following derivative:

\input{Derivative-Compute-0004.HELP.tex}

\[\dfrac{d}{dx}\left(\frac{1}{x^{3} + 11 \, x^{2} + 40 \, x + 48}\right)=\answer{-\frac{3 \, x^{2} + 22 \, x + 40}{{\left(x^{3} + 11 \, x^{2} + 40 \, x + 48\right)}^{2}}}\]
\end{problem}}

%%%%%%%%%%%%%%%%%%%%%%

\latexProblemContent{
\ifVerboseLocation This is Derivative Compute Question 0004. \\ \fi
\begin{problem}

Compute the following derivative:

\input{Derivative-Compute-0004.HELP.tex}

\[\dfrac{d}{dx}\left(\frac{x + 2}{x^{2} + 2 \, x - 8}\right)=\answer{-\frac{x^{2} + 4 \, x + 12}{{\left(x^{2} + 2 \, x - 8\right)}^{2}}}\]
\end{problem}}

%%%%%%%%%%%%%%%%%%%%%%

\latexProblemContent{
\ifVerboseLocation This is Derivative Compute Question 0004. \\ \fi
\begin{problem}

Compute the following derivative:

\input{Derivative-Compute-0004.HELP.tex}

\[\dfrac{d}{dx}\left(\frac{x^{2} + x - 2}{x^{3} + 6 \, x^{2} + 12 \, x + 8}\right)=\answer{-\frac{x^{4} + 2 \, x^{3} - 12 \, x^{2} - 40 \, x - 32}{{\left(x^{3} + 6 \, x^{2} + 12 \, x + 8\right)}^{2}}}\]
\end{problem}}

%%%%%%%%%%%%%%%%%%%%%%

\latexProblemContent{
\ifVerboseLocation This is Derivative Compute Question 0004. \\ \fi
\begin{problem}

Compute the following derivative:

\input{Derivative-Compute-0004.HELP.tex}

\[\dfrac{d}{dx}\left(\frac{x^{2} - 16}{x - 1}\right)=\answer{\frac{x^{2} - 2 \, x + 16}{{\left(x - 1\right)}^{2}}}\]
\end{problem}}

%%%%%%%%%%%%%%%%%%%%%%

\latexProblemContent{
\ifVerboseLocation This is Derivative Compute Question 0004. \\ \fi
\begin{problem}

Compute the following derivative:

\input{Derivative-Compute-0004.HELP.tex}

\[\dfrac{d}{dx}\left(\frac{x^{3} + 7 \, x^{2} + 7 \, x - 15}{x^{3} - 3 \, x^{2} - x + 3}\right)=\answer{-\frac{2 \, {\left(5 \, x^{4} + 8 \, x^{3} - 34 \, x^{2} + 24 \, x - 3\right)}}{{\left(x^{3} - 3 \, x^{2} - x + 3\right)}^{2}}}\]
\end{problem}}

%%%%%%%%%%%%%%%%%%%%%%

\latexProblemContent{
\ifVerboseLocation This is Derivative Compute Question 0004. \\ \fi
\begin{problem}

Compute the following derivative:

\input{Derivative-Compute-0004.HELP.tex}

\[\dfrac{d}{dx}\left(\frac{x^{2} - 4 \, x - 5}{x + 5}\right)=\answer{\frac{x^{2} + 10 \, x - 15}{{\left(x + 5\right)}^{2}}}\]
\end{problem}}

%%%%%%%%%%%%%%%%%%%%%%

\latexProblemContent{
\ifVerboseLocation This is Derivative Compute Question 0004. \\ \fi
\begin{problem}

Compute the following derivative:

\input{Derivative-Compute-0004.HELP.tex}

\[\dfrac{d}{dx}\left(\frac{x^{2} - 6 \, x + 5}{x^{2} + 5 \, x + 6}\right)=\answer{\frac{11 \, x^{2} + 2 \, x - 61}{{\left(x^{2} + 5 \, x + 6\right)}^{2}}}\]
\end{problem}}

%%%%%%%%%%%%%%%%%%%%%%

\latexProblemContent{
\ifVerboseLocation This is Derivative Compute Question 0004. \\ \fi
\begin{problem}

Compute the following derivative:

\input{Derivative-Compute-0004.HELP.tex}

\[\dfrac{d}{dx}\left(\frac{x^{2} + 2 \, x - 8}{x + 4}\right)=\answer{\frac{x^{2} + 8 \, x + 16}{{\left(x + 4\right)}^{2}}}\]
\end{problem}}

%%%%%%%%%%%%%%%%%%%%%%

\latexProblemContent{
\ifVerboseLocation This is Derivative Compute Question 0004. \\ \fi
\begin{problem}

Compute the following derivative:

\input{Derivative-Compute-0004.HELP.tex}

\[\dfrac{d}{dx}\left(\frac{x^{2} - 25}{x^{2} - 16}\right)=\answer{\frac{18 \, x}{{\left(x^{2} - 16\right)}^{2}}}\]
\end{problem}}

%%%%%%%%%%%%%%%%%%%%%%

\latexProblemContent{
\ifVerboseLocation This is Derivative Compute Question 0004. \\ \fi
\begin{problem}

Compute the following derivative:

\input{Derivative-Compute-0004.HELP.tex}

\[\dfrac{d}{dx}\left(\frac{x^{2} - 3 \, x + 2}{x^{3} - 3 \, x^{2} - 16 \, x + 48}\right)=\answer{-\frac{x^{4} - 6 \, x^{3} + 31 \, x^{2} - 108 \, x + 112}{{\left(x^{3} - 3 \, x^{2} - 16 \, x + 48\right)}^{2}}}\]
\end{problem}}

%%%%%%%%%%%%%%%%%%%%%%

\latexProblemContent{
\ifVerboseLocation This is Derivative Compute Question 0004. \\ \fi
\begin{problem}

Compute the following derivative:

\input{Derivative-Compute-0004.HELP.tex}

\[\dfrac{d}{dx}\left(\frac{x^{3} + 3 \, x^{2} - 25 \, x - 75}{x^{2} - 25}\right)=\answer{\frac{x^{4} - 50 \, x^{2} + 625}{{\left(x^{2} - 25\right)}^{2}}}\]
\end{problem}}

%%%%%%%%%%%%%%%%%%%%%%

\latexProblemContent{
\ifVerboseLocation This is Derivative Compute Question 0004. \\ \fi
\begin{problem}

Compute the following derivative:

\input{Derivative-Compute-0004.HELP.tex}

\[\dfrac{d}{dx}\left(\frac{1}{x^{2} + 7 \, x + 12}\right)=\answer{-\frac{2 \, x + 7}{{\left(x^{2} + 7 \, x + 12\right)}^{2}}}\]
\end{problem}}

%%%%%%%%%%%%%%%%%%%%%%

\latexProblemContent{
\ifVerboseLocation This is Derivative Compute Question 0004. \\ \fi
\begin{problem}

Compute the following derivative:

\input{Derivative-Compute-0004.HELP.tex}

\[\dfrac{d}{dx}\left(\frac{x^{2} - 9}{x^{3} - 5 \, x^{2} - 8 \, x + 48}\right)=\answer{-\frac{x^{4} - 19 \, x^{2} - 6 \, x + 72}{{\left(x^{3} - 5 \, x^{2} - 8 \, x + 48\right)}^{2}}}\]
\end{problem}}

%%%%%%%%%%%%%%%%%%%%%%

\latexProblemContent{
\ifVerboseLocation This is Derivative Compute Question 0004. \\ \fi
\begin{problem}

Compute the following derivative:

\input{Derivative-Compute-0004.HELP.tex}

\[\dfrac{d}{dx}\left(\frac{1}{x^{3} + 6 \, x^{2} - x - 30}\right)=\answer{-\frac{3 \, x^{2} + 12 \, x - 1}{{\left(x^{3} + 6 \, x^{2} - x - 30\right)}^{2}}}\]
\end{problem}}

%%%%%%%%%%%%%%%%%%%%%%

\latexProblemContent{
\ifVerboseLocation This is Derivative Compute Question 0004. \\ \fi
\begin{problem}

Compute the following derivative:

\input{Derivative-Compute-0004.HELP.tex}

\[\dfrac{d}{dx}\left(\frac{x + 3}{x^{2} + 4 \, x - 5}\right)=\answer{-\frac{x^{2} + 6 \, x + 17}{{\left(x^{2} + 4 \, x - 5\right)}^{2}}}\]
\end{problem}}

%%%%%%%%%%%%%%%%%%%%%%

\latexProblemContent{
\ifVerboseLocation This is Derivative Compute Question 0004. \\ \fi
\begin{problem}

Compute the following derivative:

\input{Derivative-Compute-0004.HELP.tex}

\[\dfrac{d}{dx}\left(\frac{x + 2}{x + 5}\right)=\answer{\frac{3}{{\left(x + 5\right)}^{2}}}\]
\end{problem}}

%%%%%%%%%%%%%%%%%%%%%%

\latexProblemContent{
\ifVerboseLocation This is Derivative Compute Question 0004. \\ \fi
\begin{problem}

Compute the following derivative:

\input{Derivative-Compute-0004.HELP.tex}

\[\dfrac{d}{dx}\left(\frac{x + 4}{x^{3} - 12 \, x - 16}\right)=\answer{-\frac{2 \, {\left(x^{3} + 6 \, x^{2} - 16\right)}}{{\left(x^{3} - 12 \, x - 16\right)}^{2}}}\]
\end{problem}}

%%%%%%%%%%%%%%%%%%%%%%

\latexProblemContent{
\ifVerboseLocation This is Derivative Compute Question 0004. \\ \fi
\begin{problem}

Compute the following derivative:

\input{Derivative-Compute-0004.HELP.tex}

\[\dfrac{d}{dx}\left(\frac{x + 2}{x^{2} - 2 \, x - 15}\right)=\answer{-\frac{x^{2} + 4 \, x + 11}{{\left(x^{2} - 2 \, x - 15\right)}^{2}}}\]
\end{problem}}

%%%%%%%%%%%%%%%%%%%%%%

\latexProblemContent{
\ifVerboseLocation This is Derivative Compute Question 0004. \\ \fi
\begin{problem}

Compute the following derivative:

\input{Derivative-Compute-0004.HELP.tex}

\[\dfrac{d}{dx}\left(\frac{x^{2} + 3 \, x - 4}{x^{3} + 8 \, x^{2} + 11 \, x - 20}\right)=\answer{-\frac{x^{4} + 6 \, x^{3} + x^{2} - 24 \, x + 16}{{\left(x^{3} + 8 \, x^{2} + 11 \, x - 20\right)}^{2}}}\]
\end{problem}}

%%%%%%%%%%%%%%%%%%%%%%

\latexProblemContent{
\ifVerboseLocation This is Derivative Compute Question 0004. \\ \fi
\begin{problem}

Compute the following derivative:

\input{Derivative-Compute-0004.HELP.tex}

\[\dfrac{d}{dx}\left(\frac{x^{2} + 6 \, x + 8}{x^{2} - 3 \, x - 4}\right)=\answer{-\frac{3 \, {\left(3 \, x^{2} + 8 \, x\right)}}{{\left(x^{2} - 3 \, x - 4\right)}^{2}}}\]
\end{problem}}

%%%%%%%%%%%%%%%%%%%%%%

\latexProblemContent{
\ifVerboseLocation This is Derivative Compute Question 0004. \\ \fi
\begin{problem}

Compute the following derivative:

\input{Derivative-Compute-0004.HELP.tex}

\[\dfrac{d}{dx}\left(\frac{1}{x^{2} + 9 \, x + 20}\right)=\answer{-\frac{2 \, x + 9}{{\left(x^{2} + 9 \, x + 20\right)}^{2}}}\]
\end{problem}}

%%%%%%%%%%%%%%%%%%%%%%

\latexProblemContent{
\ifVerboseLocation This is Derivative Compute Question 0004. \\ \fi
\begin{problem}

Compute the following derivative:

\input{Derivative-Compute-0004.HELP.tex}

\[\dfrac{d}{dx}\left(\frac{x - 3}{x + 2}\right)=\answer{\frac{5}{{\left(x + 2\right)}^{2}}}\]
\end{problem}}

%%%%%%%%%%%%%%%%%%%%%%

\latexProblemContent{
\ifVerboseLocation This is Derivative Compute Question 0004. \\ \fi
\begin{problem}

Compute the following derivative:

\input{Derivative-Compute-0004.HELP.tex}

\[\dfrac{d}{dx}\left(\frac{x^{2} + 2 \, x - 8}{x + 5}\right)=\answer{\frac{x^{2} + 10 \, x + 18}{{\left(x + 5\right)}^{2}}}\]
\end{problem}}

%%%%%%%%%%%%%%%%%%%%%%

\latexProblemContent{
\ifVerboseLocation This is Derivative Compute Question 0004. \\ \fi
\begin{problem}

Compute the following derivative:

\input{Derivative-Compute-0004.HELP.tex}

\[\dfrac{d}{dx}\left(\frac{x^{3} + 7 \, x^{2} + 7 \, x - 15}{x^{2} + 3 \, x + 2}\right)=\answer{\frac{x^{4} + 6 \, x^{3} + 20 \, x^{2} + 58 \, x + 59}{{\left(x^{2} + 3 \, x + 2\right)}^{2}}}\]
\end{problem}}

%%%%%%%%%%%%%%%%%%%%%%

\latexProblemContent{
\ifVerboseLocation This is Derivative Compute Question 0004. \\ \fi
\begin{problem}

Compute the following derivative:

\input{Derivative-Compute-0004.HELP.tex}

\[\dfrac{d}{dx}\left(\frac{x - 2}{x^{2} - 8 \, x + 15}\right)=\answer{-\frac{x^{2} - 4 \, x + 1}{{\left(x^{2} - 8 \, x + 15\right)}^{2}}}\]
\end{problem}}

%%%%%%%%%%%%%%%%%%%%%%

\latexProblemContent{
\ifVerboseLocation This is Derivative Compute Question 0004. \\ \fi
\begin{problem}

Compute the following derivative:

\input{Derivative-Compute-0004.HELP.tex}

\[\dfrac{d}{dx}\left(\frac{x^{3} - 3 \, x^{2} - 25 \, x + 75}{x + 2}\right)=\answer{\frac{2 \, x^{3} + 3 \, x^{2} - 12 \, x - 125}{{\left(x + 2\right)}^{2}}}\]
\end{problem}}

%%%%%%%%%%%%%%%%%%%%%%

\latexProblemContent{
\ifVerboseLocation This is Derivative Compute Question 0004. \\ \fi
\begin{problem}

Compute the following derivative:

\input{Derivative-Compute-0004.HELP.tex}

\[\dfrac{d}{dx}\left(\frac{1}{x^{3} - 4 \, x^{2} - 4 \, x + 16}\right)=\answer{-\frac{3 \, x^{2} - 8 \, x - 4}{{\left(x^{3} - 4 \, x^{2} - 4 \, x + 16\right)}^{2}}}\]
\end{problem}}

%%%%%%%%%%%%%%%%%%%%%%

\latexProblemContent{
\ifVerboseLocation This is Derivative Compute Question 0004. \\ \fi
\begin{problem}

Compute the following derivative:

\input{Derivative-Compute-0004.HELP.tex}

\[\dfrac{d}{dx}\left(\frac{x^{3} + 10 \, x^{2} + 31 \, x + 30}{x^{3} - 5 \, x^{2} - 4 \, x + 20}\right)=\answer{-\frac{5 \, {\left(3 \, x^{4} + 14 \, x^{3} - 17 \, x^{2} - 140 \, x - 148\right)}}{{\left(x^{3} - 5 \, x^{2} - 4 \, x + 20\right)}^{2}}}\]
\end{problem}}

%%%%%%%%%%%%%%%%%%%%%%

\latexProblemContent{
\ifVerboseLocation This is Derivative Compute Question 0004. \\ \fi
\begin{problem}

Compute the following derivative:

\input{Derivative-Compute-0004.HELP.tex}

\[\dfrac{d}{dx}\left(\frac{x^{3} + 4 \, x^{2} - 11 \, x - 30}{x + 3}\right)=\answer{\frac{2 \, x^{3} + 13 \, x^{2} + 24 \, x - 3}{{\left(x + 3\right)}^{2}}}\]
\end{problem}}

%%%%%%%%%%%%%%%%%%%%%%

\latexProblemContent{
\ifVerboseLocation This is Derivative Compute Question 0004. \\ \fi
\begin{problem}

Compute the following derivative:

\input{Derivative-Compute-0004.HELP.tex}

\[\dfrac{d}{dx}\left(\frac{x^{2} - 25}{x^{3} - 2 \, x^{2} - 11 \, x + 12}\right)=\answer{-\frac{x^{4} - 64 \, x^{2} + 76 \, x + 275}{{\left(x^{3} - 2 \, x^{2} - 11 \, x + 12\right)}^{2}}}\]
\end{problem}}

%%%%%%%%%%%%%%%%%%%%%%

\latexProblemContent{
\ifVerboseLocation This is Derivative Compute Question 0004. \\ \fi
\begin{problem}

Compute the following derivative:

\input{Derivative-Compute-0004.HELP.tex}

\[\dfrac{d}{dx}\left(\frac{x^{2} + 4 \, x - 5}{x^{3} + 2 \, x^{2} - 15 \, x - 36}\right)=\answer{-\frac{x^{4} + 8 \, x^{3} + 8 \, x^{2} + 52 \, x + 219}{{\left(x^{3} + 2 \, x^{2} - 15 \, x - 36\right)}^{2}}}\]
\end{problem}}

%%%%%%%%%%%%%%%%%%%%%%

\latexProblemContent{
\ifVerboseLocation This is Derivative Compute Question 0004. \\ \fi
\begin{problem}

Compute the following derivative:

\input{Derivative-Compute-0004.HELP.tex}

\[\dfrac{d}{dx}\left(\frac{x^{3} - 7 \, x^{2} + 14 \, x - 8}{x^{2} + 2 \, x - 15}\right)=\answer{\frac{x^{4} + 4 \, x^{3} - 73 \, x^{2} + 226 \, x - 194}{{\left(x^{2} + 2 \, x - 15\right)}^{2}}}\]
\end{problem}}

%%%%%%%%%%%%%%%%%%%%%%

\latexProblemContent{
\ifVerboseLocation This is Derivative Compute Question 0004. \\ \fi
\begin{problem}

Compute the following derivative:

\input{Derivative-Compute-0004.HELP.tex}

\[\dfrac{d}{dx}\left(\frac{x^{2} - 7 \, x + 12}{x^{3} - 4 \, x^{2} - 7 \, x + 10}\right)=\answer{-\frac{x^{4} - 14 \, x^{3} + 71 \, x^{2} - 116 \, x - 14}{{\left(x^{3} - 4 \, x^{2} - 7 \, x + 10\right)}^{2}}}\]
\end{problem}}

%%%%%%%%%%%%%%%%%%%%%%

\latexProblemContent{
\ifVerboseLocation This is Derivative Compute Question 0004. \\ \fi
\begin{problem}

Compute the following derivative:

\input{Derivative-Compute-0004.HELP.tex}

\[\dfrac{d}{dx}\left(\frac{x^{3} + 8 \, x^{2} + 19 \, x + 12}{x - 2}\right)=\answer{\frac{2 \, {\left(x^{3} + x^{2} - 16 \, x - 25\right)}}{{\left(x - 2\right)}^{2}}}\]
\end{problem}}

%%%%%%%%%%%%%%%%%%%%%%

\latexProblemContent{
\ifVerboseLocation This is Derivative Compute Question 0004. \\ \fi
\begin{problem}

Compute the following derivative:

\input{Derivative-Compute-0004.HELP.tex}

\[\dfrac{d}{dx}\left(\frac{x^{3} - x^{2} - 25 \, x + 25}{x^{2} + 2 \, x - 3}\right)=\answer{\frac{x^{4} + 4 \, x^{3} + 14 \, x^{2} - 44 \, x + 25}{{\left(x^{2} + 2 \, x - 3\right)}^{2}}}\]
\end{problem}}

%%%%%%%%%%%%%%%%%%%%%%

\latexProblemContent{
\ifVerboseLocation This is Derivative Compute Question 0004. \\ \fi
\begin{problem}

Compute the following derivative:

\input{Derivative-Compute-0004.HELP.tex}

\[\dfrac{d}{dx}\left(\frac{x - 3}{x + 4}\right)=\answer{\frac{7}{{\left(x + 4\right)}^{2}}}\]
\end{problem}}

%%%%%%%%%%%%%%%%%%%%%%

\latexProblemContent{
\ifVerboseLocation This is Derivative Compute Question 0004. \\ \fi
\begin{problem}

Compute the following derivative:

\input{Derivative-Compute-0004.HELP.tex}

\[\dfrac{d}{dx}\left(\frac{x^{3} + 3 \, x^{2} - 18 \, x - 40}{x^{3} + 13 \, x^{2} + 55 \, x + 75}\right)=\answer{\frac{2 \, {\left(5 \, x^{4} + 73 \, x^{3} + 372 \, x^{2} + 745 \, x + 425\right)}}{{\left(x^{3} + 13 \, x^{2} + 55 \, x + 75\right)}^{2}}}\]
\end{problem}}

%%%%%%%%%%%%%%%%%%%%%%

\latexProblemContent{
\ifVerboseLocation This is Derivative Compute Question 0004. \\ \fi
\begin{problem}

Compute the following derivative:

\input{Derivative-Compute-0004.HELP.tex}

\[\dfrac{d}{dx}\left(\frac{x^{2} - 2 \, x - 3}{x^{2} - 4}\right)=\answer{\frac{2 \, {\left(x^{2} - x + 4\right)}}{{\left(x^{2} - 4\right)}^{2}}}\]
\end{problem}}

%%%%%%%%%%%%%%%%%%%%%%

\latexProblemContent{
\ifVerboseLocation This is Derivative Compute Question 0004. \\ \fi
\begin{problem}

Compute the following derivative:

\input{Derivative-Compute-0004.HELP.tex}

\[\dfrac{d}{dx}\left(\frac{1}{x^{3} - x^{2} - 4 \, x + 4}\right)=\answer{-\frac{3 \, x^{2} - 2 \, x - 4}{{\left(x^{3} - x^{2} - 4 \, x + 4\right)}^{2}}}\]
\end{problem}}

%%%%%%%%%%%%%%%%%%%%%%

\latexProblemContent{
\ifVerboseLocation This is Derivative Compute Question 0004. \\ \fi
\begin{problem}

Compute the following derivative:

\input{Derivative-Compute-0004.HELP.tex}

\[\dfrac{d}{dx}\left(\frac{1}{x^{2} - 6 \, x + 5}\right)=\answer{-\frac{2 \, {\left(x - 3\right)}}{{\left(x^{2} - 6 \, x + 5\right)}^{2}}}\]
\end{problem}}

%%%%%%%%%%%%%%%%%%%%%%

\latexProblemContent{
\ifVerboseLocation This is Derivative Compute Question 0004. \\ \fi
\begin{problem}

Compute the following derivative:

\input{Derivative-Compute-0004.HELP.tex}

\[\dfrac{d}{dx}\left(\frac{x^{2} - 9}{x^{3} - 9 \, x^{2} + 15 \, x + 25}\right)=\answer{-\frac{x^{4} - 42 \, x^{2} + 112 \, x - 135}{{\left(x^{3} - 9 \, x^{2} + 15 \, x + 25\right)}^{2}}}\]
\end{problem}}

%%%%%%%%%%%%%%%%%%%%%%

\latexProblemContent{
\ifVerboseLocation This is Derivative Compute Question 0004. \\ \fi
\begin{problem}

Compute the following derivative:

\input{Derivative-Compute-0004.HELP.tex}

\[\dfrac{d}{dx}\left(\frac{x^{3} - 7 \, x^{2} + 8 \, x + 16}{x - 4}\right)=\answer{\frac{2 \, x^{3} - 19 \, x^{2} + 56 \, x - 48}{{\left(x - 4\right)}^{2}}}\]
\end{problem}}

%%%%%%%%%%%%%%%%%%%%%%

\latexProblemContent{
\ifVerboseLocation This is Derivative Compute Question 0004. \\ \fi
\begin{problem}

Compute the following derivative:

\input{Derivative-Compute-0004.HELP.tex}

\[\dfrac{d}{dx}\left(\frac{x - 4}{x^{2} - 3 \, x - 10}\right)=\answer{-\frac{x^{2} - 8 \, x + 22}{{\left(x^{2} - 3 \, x - 10\right)}^{2}}}\]
\end{problem}}

%%%%%%%%%%%%%%%%%%%%%%

\latexProblemContent{
\ifVerboseLocation This is Derivative Compute Question 0004. \\ \fi
\begin{problem}

Compute the following derivative:

\input{Derivative-Compute-0004.HELP.tex}

\[\dfrac{d}{dx}\left(\frac{1}{x^{2} + 3 \, x - 10}\right)=\answer{-\frac{2 \, x + 3}{{\left(x^{2} + 3 \, x - 10\right)}^{2}}}\]
\end{problem}}

%%%%%%%%%%%%%%%%%%%%%%

\latexProblemContent{
\ifVerboseLocation This is Derivative Compute Question 0004. \\ \fi
\begin{problem}

Compute the following derivative:

\input{Derivative-Compute-0004.HELP.tex}

\[\dfrac{d}{dx}\left(\frac{x^{3} + 5 \, x^{2} - 9 \, x - 45}{x + 4}\right)=\answer{\frac{2 \, x^{3} + 17 \, x^{2} + 40 \, x + 9}{{\left(x + 4\right)}^{2}}}\]
\end{problem}}

%%%%%%%%%%%%%%%%%%%%%%

\latexProblemContent{
\ifVerboseLocation This is Derivative Compute Question 0004. \\ \fi
\begin{problem}

Compute the following derivative:

\input{Derivative-Compute-0004.HELP.tex}

\[\dfrac{d}{dx}\left(\frac{x^{2} + 2 \, x - 3}{x^{2} + 6 \, x + 5}\right)=\answer{\frac{4 \, {\left(x^{2} + 4 \, x + 7\right)}}{{\left(x^{2} + 6 \, x + 5\right)}^{2}}}\]
\end{problem}}

%%%%%%%%%%%%%%%%%%%%%%

\latexProblemContent{
\ifVerboseLocation This is Derivative Compute Question 0004. \\ \fi
\begin{problem}

Compute the following derivative:

\input{Derivative-Compute-0004.HELP.tex}

\[\dfrac{d}{dx}\left(\frac{1}{x - 3}\right)=\answer{-\frac{1}{{\left(x - 3\right)}^{2}}}\]
\end{problem}}

%%%%%%%%%%%%%%%%%%%%%%

\latexProblemContent{
\ifVerboseLocation This is Derivative Compute Question 0004. \\ \fi
\begin{problem}

Compute the following derivative:

\input{Derivative-Compute-0004.HELP.tex}

\[\dfrac{d}{dx}\left(\frac{1}{x - 5}\right)=\answer{-\frac{1}{{\left(x - 5\right)}^{2}}}\]
\end{problem}}

%%%%%%%%%%%%%%%%%%%%%%

\latexProblemContent{
\ifVerboseLocation This is Derivative Compute Question 0004. \\ \fi
\begin{problem}

Compute the following derivative:

\input{Derivative-Compute-0004.HELP.tex}

\[\dfrac{d}{dx}\left(\frac{x^{3} + 3 \, x^{2} - 4 \, x - 12}{x^{3} - 5 \, x^{2} - 4 \, x + 20}\right)=\answer{-\frac{8 \, {\left(x^{4} - 8 \, x^{2} + 16\right)}}{{\left(x^{3} - 5 \, x^{2} - 4 \, x + 20\right)}^{2}}}\]
\end{problem}}

%%%%%%%%%%%%%%%%%%%%%%

\latexProblemContent{
\ifVerboseLocation This is Derivative Compute Question 0004. \\ \fi
\begin{problem}

Compute the following derivative:

\input{Derivative-Compute-0004.HELP.tex}

\[\dfrac{d}{dx}\left(\frac{x^{3} + 4 \, x^{2} - 9 \, x - 36}{x^{2} - 1}\right)=\answer{\frac{x^{4} + 6 \, x^{2} + 64 \, x + 9}{{\left(x^{2} - 1\right)}^{2}}}\]
\end{problem}}

%%%%%%%%%%%%%%%%%%%%%%

\latexProblemContent{
\ifVerboseLocation This is Derivative Compute Question 0004. \\ \fi
\begin{problem}

Compute the following derivative:

\input{Derivative-Compute-0004.HELP.tex}

\[\dfrac{d}{dx}\left(\frac{x^{2} - 4 \, x + 3}{x - 2}\right)=\answer{\frac{x^{2} - 4 \, x + 5}{{\left(x - 2\right)}^{2}}}\]
\end{problem}}

%%%%%%%%%%%%%%%%%%%%%%

\latexProblemContent{
\ifVerboseLocation This is Derivative Compute Question 0004. \\ \fi
\begin{problem}

Compute the following derivative:

\input{Derivative-Compute-0004.HELP.tex}

\[\dfrac{d}{dx}\left(\frac{x^{2} - x - 20}{x + 2}\right)=\answer{\frac{x^{2} + 4 \, x + 18}{{\left(x + 2\right)}^{2}}}\]
\end{problem}}

%%%%%%%%%%%%%%%%%%%%%%

\latexProblemContent{
\ifVerboseLocation This is Derivative Compute Question 0004. \\ \fi
\begin{problem}

Compute the following derivative:

\input{Derivative-Compute-0004.HELP.tex}

\[\dfrac{d}{dx}\left(\frac{x^{3} - 4 \, x^{2} - 7 \, x + 10}{x - 3}\right)=\answer{\frac{2 \, x^{3} - 13 \, x^{2} + 24 \, x + 11}{{\left(x - 3\right)}^{2}}}\]
\end{problem}}

%%%%%%%%%%%%%%%%%%%%%%

\latexProblemContent{
\ifVerboseLocation This is Derivative Compute Question 0004. \\ \fi
\begin{problem}

Compute the following derivative:

\input{Derivative-Compute-0004.HELP.tex}

\[\dfrac{d}{dx}\left(\frac{x^{3} - 6 \, x^{2} + 11 \, x - 6}{x^{2} + 3 \, x - 4}\right)=\answer{\frac{x^{4} + 6 \, x^{3} - 41 \, x^{2} + 60 \, x - 26}{{\left(x^{2} + 3 \, x - 4\right)}^{2}}}\]
\end{problem}}

%%%%%%%%%%%%%%%%%%%%%%

\latexProblemContent{
\ifVerboseLocation This is Derivative Compute Question 0004. \\ \fi
\begin{problem}

Compute the following derivative:

\input{Derivative-Compute-0004.HELP.tex}

\[\dfrac{d}{dx}\left(\frac{x - 4}{x - 2}\right)=\answer{\frac{2}{{\left(x - 2\right)}^{2}}}\]
\end{problem}}

%%%%%%%%%%%%%%%%%%%%%%

\latexProblemContent{
\ifVerboseLocation This is Derivative Compute Question 0004. \\ \fi
\begin{problem}

Compute the following derivative:

\input{Derivative-Compute-0004.HELP.tex}

\[\dfrac{d}{dx}\left(\frac{x^{2} - x - 2}{x + 2}\right)=\answer{\frac{x^{2} + 4 \, x}{{\left(x + 2\right)}^{2}}}\]
\end{problem}}

%%%%%%%%%%%%%%%%%%%%%%

\latexProblemContent{
\ifVerboseLocation This is Derivative Compute Question 0004. \\ \fi
\begin{problem}

Compute the following derivative:

\input{Derivative-Compute-0004.HELP.tex}

\[\dfrac{d}{dx}\left(\frac{x^{2} + 2 \, x - 8}{x^{3} + 7 \, x^{2} + 7 \, x - 15}\right)=\answer{-\frac{x^{4} + 4 \, x^{3} - 17 \, x^{2} - 82 \, x - 26}{{\left(x^{3} + 7 \, x^{2} + 7 \, x - 15\right)}^{2}}}\]
\end{problem}}

%%%%%%%%%%%%%%%%%%%%%%

\latexProblemContent{
\ifVerboseLocation This is Derivative Compute Question 0004. \\ \fi
\begin{problem}

Compute the following derivative:

\input{Derivative-Compute-0004.HELP.tex}

\[\dfrac{d}{dx}\left(\frac{x^{2} - 2 \, x - 3}{x - 4}\right)=\answer{\frac{x^{2} - 8 \, x + 11}{{\left(x - 4\right)}^{2}}}\]
\end{problem}}

%%%%%%%%%%%%%%%%%%%%%%

\latexProblemContent{
\ifVerboseLocation This is Derivative Compute Question 0004. \\ \fi
\begin{problem}

Compute the following derivative:

\input{Derivative-Compute-0004.HELP.tex}

\[\dfrac{d}{dx}\left(\frac{x^{2} + 2 \, x - 8}{x^{2} - 4 \, x + 3}\right)=\answer{-\frac{2 \, {\left(3 \, x^{2} - 11 \, x + 13\right)}}{{\left(x^{2} - 4 \, x + 3\right)}^{2}}}\]
\end{problem}}

%%%%%%%%%%%%%%%%%%%%%%

\latexProblemContent{
\ifVerboseLocation This is Derivative Compute Question 0004. \\ \fi
\begin{problem}

Compute the following derivative:

\input{Derivative-Compute-0004.HELP.tex}

\[\dfrac{d}{dx}\left(\frac{x - 4}{x + 1}\right)=\answer{\frac{5}{{\left(x + 1\right)}^{2}}}\]
\end{problem}}

%%%%%%%%%%%%%%%%%%%%%%

\latexProblemContent{
\ifVerboseLocation This is Derivative Compute Question 0004. \\ \fi
\begin{problem}

Compute the following derivative:

\input{Derivative-Compute-0004.HELP.tex}

\[\dfrac{d}{dx}\left(\frac{x^{3} + 2 \, x^{2} - 13 \, x + 10}{x^{2} - 3 \, x - 10}\right)=\answer{\frac{x^{4} - 6 \, x^{3} - 23 \, x^{2} - 60 \, x + 160}{{\left(x^{2} - 3 \, x - 10\right)}^{2}}}\]
\end{problem}}

%%%%%%%%%%%%%%%%%%%%%%

\latexProblemContent{
\ifVerboseLocation This is Derivative Compute Question 0004. \\ \fi
\begin{problem}

Compute the following derivative:

\input{Derivative-Compute-0004.HELP.tex}

\[\dfrac{d}{dx}\left(\frac{1}{x^{2} - 7 \, x + 10}\right)=\answer{-\frac{2 \, x - 7}{{\left(x^{2} - 7 \, x + 10\right)}^{2}}}\]
\end{problem}}

%%%%%%%%%%%%%%%%%%%%%%

\latexProblemContent{
\ifVerboseLocation This is Derivative Compute Question 0004. \\ \fi
\begin{problem}

Compute the following derivative:

\input{Derivative-Compute-0004.HELP.tex}

\[\dfrac{d}{dx}\left(\frac{x + 5}{x^{3} - 5 \, x^{2} - 16 \, x + 80}\right)=\answer{-\frac{2 \, {\left(x^{3} + 5 \, x^{2} - 25 \, x - 80\right)}}{{\left(x^{3} - 5 \, x^{2} - 16 \, x + 80\right)}^{2}}}\]
\end{problem}}

%%%%%%%%%%%%%%%%%%%%%%

\latexProblemContent{
\ifVerboseLocation This is Derivative Compute Question 0004. \\ \fi
\begin{problem}

Compute the following derivative:

\input{Derivative-Compute-0004.HELP.tex}

\[\dfrac{d}{dx}\left(\frac{x - 1}{x^{3} - 2 \, x^{2} - 16 \, x + 32}\right)=\answer{-\frac{2 \, x^{3} - 5 \, x^{2} + 4 \, x - 16}{{\left(x^{3} - 2 \, x^{2} - 16 \, x + 32\right)}^{2}}}\]
\end{problem}}

%%%%%%%%%%%%%%%%%%%%%%

\latexProblemContent{
\ifVerboseLocation This is Derivative Compute Question 0004. \\ \fi
\begin{problem}

Compute the following derivative:

\input{Derivative-Compute-0004.HELP.tex}

\[\dfrac{d}{dx}\left(\frac{x^{2} + 3 \, x - 10}{x^{2} - 3 \, x - 10}\right)=\answer{-\frac{6 \, {\left(x^{2} + 10\right)}}{{\left(x^{2} - 3 \, x - 10\right)}^{2}}}\]
\end{problem}}

%%%%%%%%%%%%%%%%%%%%%%

\latexProblemContent{
\ifVerboseLocation This is Derivative Compute Question 0004. \\ \fi
\begin{problem}

Compute the following derivative:

\input{Derivative-Compute-0004.HELP.tex}

\[\dfrac{d}{dx}\left(\frac{x^{2} - x - 6}{x^{2} + 3 \, x - 4}\right)=\answer{\frac{2 \, {\left(2 \, x^{2} + 2 \, x + 11\right)}}{{\left(x^{2} + 3 \, x - 4\right)}^{2}}}\]
\end{problem}}

%%%%%%%%%%%%%%%%%%%%%%

\latexProblemContent{
\ifVerboseLocation This is Derivative Compute Question 0004. \\ \fi
\begin{problem}

Compute the following derivative:

\input{Derivative-Compute-0004.HELP.tex}

\[\dfrac{d}{dx}\left(\frac{x^{3} + x^{2} - 22 \, x - 40}{x^{2} + 2 \, x - 3}\right)=\answer{\frac{x^{4} + 4 \, x^{3} + 15 \, x^{2} + 74 \, x + 146}{{\left(x^{2} + 2 \, x - 3\right)}^{2}}}\]
\end{problem}}

%%%%%%%%%%%%%%%%%%%%%%

\latexProblemContent{
\ifVerboseLocation This is Derivative Compute Question 0004. \\ \fi
\begin{problem}

Compute the following derivative:

\input{Derivative-Compute-0004.HELP.tex}

\[\dfrac{d}{dx}\left(\frac{x^{3} - 6 \, x^{2} - 7 \, x + 60}{x - 3}\right)=\answer{\frac{2 \, x^{3} - 15 \, x^{2} + 36 \, x - 39}{{\left(x - 3\right)}^{2}}}\]
\end{problem}}

%%%%%%%%%%%%%%%%%%%%%%

\latexProblemContent{
\ifVerboseLocation This is Derivative Compute Question 0004. \\ \fi
\begin{problem}

Compute the following derivative:

\input{Derivative-Compute-0004.HELP.tex}

\[\dfrac{d}{dx}\left(\frac{x - 5}{x^{3} - 12 \, x + 16}\right)=\answer{-\frac{2 \, x^{3} - 15 \, x^{2} + 44}{{\left(x^{3} - 12 \, x + 16\right)}^{2}}}\]
\end{problem}}

%%%%%%%%%%%%%%%%%%%%%%

\latexProblemContent{
\ifVerboseLocation This is Derivative Compute Question 0004. \\ \fi
\begin{problem}

Compute the following derivative:

\input{Derivative-Compute-0004.HELP.tex}

\[\dfrac{d}{dx}\left(\frac{1}{x^{3} - 7 \, x^{2} + 7 \, x + 15}\right)=\answer{-\frac{3 \, x^{2} - 14 \, x + 7}{{\left(x^{3} - 7 \, x^{2} + 7 \, x + 15\right)}^{2}}}\]
\end{problem}}

%%%%%%%%%%%%%%%%%%%%%%

\latexProblemContent{
\ifVerboseLocation This is Derivative Compute Question 0004. \\ \fi
\begin{problem}

Compute the following derivative:

\input{Derivative-Compute-0004.HELP.tex}

\[\dfrac{d}{dx}\left(\frac{x^{2} + x - 2}{x - 5}\right)=\answer{\frac{x^{2} - 10 \, x - 3}{{\left(x - 5\right)}^{2}}}\]
\end{problem}}

%%%%%%%%%%%%%%%%%%%%%%

\latexProblemContent{
\ifVerboseLocation This is Derivative Compute Question 0004. \\ \fi
\begin{problem}

Compute the following derivative:

\input{Derivative-Compute-0004.HELP.tex}

\[\dfrac{d}{dx}\left(\frac{x - 5}{x^{3} + 6 \, x^{2} + 5 \, x - 12}\right)=\answer{-\frac{2 \, x^{3} - 9 \, x^{2} - 60 \, x - 13}{{\left(x^{3} + 6 \, x^{2} + 5 \, x - 12\right)}^{2}}}\]
\end{problem}}

%%%%%%%%%%%%%%%%%%%%%%

\latexProblemContent{
\ifVerboseLocation This is Derivative Compute Question 0004. \\ \fi
\begin{problem}

Compute the following derivative:

\input{Derivative-Compute-0004.HELP.tex}

\[\dfrac{d}{dx}\left(\frac{x^{3} - 4 \, x^{2} - 9 \, x + 36}{x^{3} - 6 \, x^{2} - 15 \, x + 100}\right)=\answer{-\frac{2 \, {\left(x^{4} + 6 \, x^{3} - 99 \, x^{2} + 184 \, x + 180\right)}}{{\left(x^{3} - 6 \, x^{2} - 15 \, x + 100\right)}^{2}}}\]
\end{problem}}

%%%%%%%%%%%%%%%%%%%%%%

\latexProblemContent{
\ifVerboseLocation This is Derivative Compute Question 0004. \\ \fi
\begin{problem}

Compute the following derivative:

\input{Derivative-Compute-0004.HELP.tex}

\[\dfrac{d}{dx}\left(\frac{x - 5}{x^{3} - 4 \, x^{2} + x + 6}\right)=\answer{-\frac{2 \, x^{3} - 19 \, x^{2} + 40 \, x - 11}{{\left(x^{3} - 4 \, x^{2} + x + 6\right)}^{2}}}\]
\end{problem}}

%%%%%%%%%%%%%%%%%%%%%%

\latexProblemContent{
\ifVerboseLocation This is Derivative Compute Question 0004. \\ \fi
\begin{problem}

Compute the following derivative:

\input{Derivative-Compute-0004.HELP.tex}

\[\dfrac{d}{dx}\left(\frac{x^{2} - 3 \, x - 4}{x + 2}\right)=\answer{\frac{x^{2} + 4 \, x - 2}{{\left(x + 2\right)}^{2}}}\]
\end{problem}}

%%%%%%%%%%%%%%%%%%%%%%

\latexProblemContent{
\ifVerboseLocation This is Derivative Compute Question 0004. \\ \fi
\begin{problem}

Compute the following derivative:

\input{Derivative-Compute-0004.HELP.tex}

\[\dfrac{d}{dx}\left(\frac{x + 4}{x^{3} - 7 \, x^{2} + 16 \, x - 12}\right)=\answer{-\frac{2 \, x^{3} + 5 \, x^{2} - 56 \, x + 76}{{\left(x^{3} - 7 \, x^{2} + 16 \, x - 12\right)}^{2}}}\]
\end{problem}}

%%%%%%%%%%%%%%%%%%%%%%

\latexProblemContent{
\ifVerboseLocation This is Derivative Compute Question 0004. \\ \fi
\begin{problem}

Compute the following derivative:

\input{Derivative-Compute-0004.HELP.tex}

\[\dfrac{d}{dx}\left(\frac{x^{2} - 9 \, x + 20}{x + 5}\right)=\answer{\frac{x^{2} + 10 \, x - 65}{{\left(x + 5\right)}^{2}}}\]
\end{problem}}

%%%%%%%%%%%%%%%%%%%%%%

\latexProblemContent{
\ifVerboseLocation This is Derivative Compute Question 0004. \\ \fi
\begin{problem}

Compute the following derivative:

\input{Derivative-Compute-0004.HELP.tex}

\[\dfrac{d}{dx}\left(\frac{1}{x + 4}\right)=\answer{-\frac{1}{{\left(x + 4\right)}^{2}}}\]
\end{problem}}

%%%%%%%%%%%%%%%%%%%%%%

\latexProblemContent{
\ifVerboseLocation This is Derivative Compute Question 0004. \\ \fi
\begin{problem}

Compute the following derivative:

\input{Derivative-Compute-0004.HELP.tex}

\[\dfrac{d}{dx}\left(\frac{x + 4}{x^{3} - 12 \, x^{2} + 48 \, x - 64}\right)=\answer{-\frac{2 \, {\left(x^{3} - 48 \, x + 128\right)}}{{\left(x^{3} - 12 \, x^{2} + 48 \, x - 64\right)}^{2}}}\]
\end{problem}}

%%%%%%%%%%%%%%%%%%%%%%

\latexProblemContent{
\ifVerboseLocation This is Derivative Compute Question 0004. \\ \fi
\begin{problem}

Compute the following derivative:

\input{Derivative-Compute-0004.HELP.tex}

\[\dfrac{d}{dx}\left(\frac{x^{2} + 5 \, x + 6}{x^{3} + 7 \, x^{2} + 7 \, x - 15}\right)=\answer{-\frac{x^{4} + 10 \, x^{3} + 46 \, x^{2} + 114 \, x + 117}{{\left(x^{3} + 7 \, x^{2} + 7 \, x - 15\right)}^{2}}}\]
\end{problem}}

%%%%%%%%%%%%%%%%%%%%%%

\latexProblemContent{
\ifVerboseLocation This is Derivative Compute Question 0004. \\ \fi
\begin{problem}

Compute the following derivative:

\input{Derivative-Compute-0004.HELP.tex}

\[\dfrac{d}{dx}\left(\frac{x^{3} - 7 \, x - 6}{x - 2}\right)=\answer{\frac{2 \, {\left(x^{3} - 3 \, x^{2} + 10\right)}}{{\left(x - 2\right)}^{2}}}\]
\end{problem}}

%%%%%%%%%%%%%%%%%%%%%%

\latexProblemContent{
\ifVerboseLocation This is Derivative Compute Question 0004. \\ \fi
\begin{problem}

Compute the following derivative:

\input{Derivative-Compute-0004.HELP.tex}

\[\dfrac{d}{dx}\left(\frac{x^{3} + 8 \, x^{2} + 19 \, x + 12}{x - 4}\right)=\answer{\frac{2 \, {\left(x^{3} - 2 \, x^{2} - 32 \, x - 44\right)}}{{\left(x - 4\right)}^{2}}}\]
\end{problem}}

%%%%%%%%%%%%%%%%%%%%%%

\latexProblemContent{
\ifVerboseLocation This is Derivative Compute Question 0004. \\ \fi
\begin{problem}

Compute the following derivative:

\input{Derivative-Compute-0004.HELP.tex}

\[\dfrac{d}{dx}\left(\frac{x^{2} + 6 \, x + 8}{x^{2} + 2 \, x - 15}\right)=\answer{-\frac{2 \, {\left(2 \, x^{2} + 23 \, x + 53\right)}}{{\left(x^{2} + 2 \, x - 15\right)}^{2}}}\]
\end{problem}}

%%%%%%%%%%%%%%%%%%%%%%

\latexProblemContent{
\ifVerboseLocation This is Derivative Compute Question 0004. \\ \fi
\begin{problem}

Compute the following derivative:

\input{Derivative-Compute-0004.HELP.tex}

\[\dfrac{d}{dx}\left(\frac{x - 3}{x + 1}\right)=\answer{\frac{4}{{\left(x + 1\right)}^{2}}}\]
\end{problem}}

%%%%%%%%%%%%%%%%%%%%%%

\latexProblemContent{
\ifVerboseLocation This is Derivative Compute Question 0004. \\ \fi
\begin{problem}

Compute the following derivative:

\input{Derivative-Compute-0004.HELP.tex}

\[\dfrac{d}{dx}\left(\frac{x^{3} - 3 \, x^{2} - 6 \, x + 8}{x - 3}\right)=\answer{\frac{2 \, {\left(x^{3} - 6 \, x^{2} + 9 \, x + 5\right)}}{{\left(x - 3\right)}^{2}}}\]
\end{problem}}

%%%%%%%%%%%%%%%%%%%%%%

\latexProblemContent{
\ifVerboseLocation This is Derivative Compute Question 0004. \\ \fi
\begin{problem}

Compute the following derivative:

\input{Derivative-Compute-0004.HELP.tex}

\[\dfrac{d}{dx}\left(\frac{x^{2} - 4}{x - 1}\right)=\answer{\frac{x^{2} - 2 \, x + 4}{{\left(x - 1\right)}^{2}}}\]
\end{problem}}

%%%%%%%%%%%%%%%%%%%%%%

\latexProblemContent{
\ifVerboseLocation This is Derivative Compute Question 0004. \\ \fi
\begin{problem}

Compute the following derivative:

\input{Derivative-Compute-0004.HELP.tex}

\[\dfrac{d}{dx}\left(\frac{x^{2} + x - 6}{x^{2} + 9 \, x + 20}\right)=\answer{\frac{2 \, {\left(4 \, x^{2} + 26 \, x + 37\right)}}{{\left(x^{2} + 9 \, x + 20\right)}^{2}}}\]
\end{problem}}

%%%%%%%%%%%%%%%%%%%%%%

\latexProblemContent{
\ifVerboseLocation This is Derivative Compute Question 0004. \\ \fi
\begin{problem}

Compute the following derivative:

\input{Derivative-Compute-0004.HELP.tex}

\[\dfrac{d}{dx}\left(\frac{x - 1}{x^{2} - 8 \, x + 16}\right)=\answer{-\frac{x^{2} - 2 \, x - 8}{{\left(x^{2} - 8 \, x + 16\right)}^{2}}}\]
\end{problem}}

%%%%%%%%%%%%%%%%%%%%%%

\latexProblemContent{
\ifVerboseLocation This is Derivative Compute Question 0004. \\ \fi
\begin{problem}

Compute the following derivative:

\input{Derivative-Compute-0004.HELP.tex}

\[\dfrac{d}{dx}\left(\frac{1}{x^{3} - 5 \, x^{2} - 4 \, x + 20}\right)=\answer{-\frac{3 \, x^{2} - 10 \, x - 4}{{\left(x^{3} - 5 \, x^{2} - 4 \, x + 20\right)}^{2}}}\]
\end{problem}}

%%%%%%%%%%%%%%%%%%%%%%

\latexProblemContent{
\ifVerboseLocation This is Derivative Compute Question 0004. \\ \fi
\begin{problem}

Compute the following derivative:

\input{Derivative-Compute-0004.HELP.tex}

\[\dfrac{d}{dx}\left(\frac{x^{3} + 6 \, x^{2} + 3 \, x - 10}{x - 4}\right)=\answer{\frac{2 \, {\left(x^{3} - 3 \, x^{2} - 24 \, x - 1\right)}}{{\left(x - 4\right)}^{2}}}\]
\end{problem}}

%%%%%%%%%%%%%%%%%%%%%%

\latexProblemContent{
\ifVerboseLocation This is Derivative Compute Question 0004. \\ \fi
\begin{problem}

Compute the following derivative:

\input{Derivative-Compute-0004.HELP.tex}

\[\dfrac{d}{dx}\left(\frac{x^{3} - x^{2} - 4 \, x + 4}{x^{3} + x^{2} - x - 1}\right)=\answer{\frac{2 \, {\left(x^{4} + 3 \, x^{3} - 5 \, x^{2} - 3 \, x + 4\right)}}{{\left(x^{3} + x^{2} - x - 1\right)}^{2}}}\]
\end{problem}}

%%%%%%%%%%%%%%%%%%%%%%

\latexProblemContent{
\ifVerboseLocation This is Derivative Compute Question 0004. \\ \fi
\begin{problem}

Compute the following derivative:

\input{Derivative-Compute-0004.HELP.tex}

\[\dfrac{d}{dx}\left(\frac{1}{x^{3} + 4 \, x^{2} - 4 \, x - 16}\right)=\answer{-\frac{3 \, x^{2} + 8 \, x - 4}{{\left(x^{3} + 4 \, x^{2} - 4 \, x - 16\right)}^{2}}}\]
\end{problem}}

%%%%%%%%%%%%%%%%%%%%%%

\latexProblemContent{
\ifVerboseLocation This is Derivative Compute Question 0004. \\ \fi
\begin{problem}

Compute the following derivative:

\input{Derivative-Compute-0004.HELP.tex}

\[\dfrac{d}{dx}\left(\frac{x^{3} + 5 \, x^{2} - 2 \, x - 24}{x^{3} - 2 \, x^{2} - 4 \, x + 8}\right)=\answer{-\frac{7 \, x^{4} + 4 \, x^{3} - 72 \, x^{2} + 16 \, x + 112}{{\left(x^{3} - 2 \, x^{2} - 4 \, x + 8\right)}^{2}}}\]
\end{problem}}

%%%%%%%%%%%%%%%%%%%%%%

\latexProblemContent{
\ifVerboseLocation This is Derivative Compute Question 0004. \\ \fi
\begin{problem}

Compute the following derivative:

\input{Derivative-Compute-0004.HELP.tex}

\[\dfrac{d}{dx}\left(\frac{x^{2} + 6 \, x + 8}{x + 3}\right)=\answer{\frac{x^{2} + 6 \, x + 10}{{\left(x + 3\right)}^{2}}}\]
\end{problem}}

%%%%%%%%%%%%%%%%%%%%%%

\latexProblemContent{
\ifVerboseLocation This is Derivative Compute Question 0004. \\ \fi
\begin{problem}

Compute the following derivative:

\input{Derivative-Compute-0004.HELP.tex}

\[\dfrac{d}{dx}\left(\frac{x^{2} - x - 20}{x^{2} - 5 \, x + 4}\right)=\answer{-\frac{4 \, {\left(x^{2} - 12 \, x + 26\right)}}{{\left(x^{2} - 5 \, x + 4\right)}^{2}}}\]
\end{problem}}

%%%%%%%%%%%%%%%%%%%%%%

\latexProblemContent{
\ifVerboseLocation This is Derivative Compute Question 0004. \\ \fi
\begin{problem}

Compute the following derivative:

\input{Derivative-Compute-0004.HELP.tex}

\[\dfrac{d}{dx}\left(\frac{x + 2}{x^{3} + x^{2} - 16 \, x + 20}\right)=\answer{-\frac{2 \, x^{3} + 7 \, x^{2} + 4 \, x - 52}{{\left(x^{3} + x^{2} - 16 \, x + 20\right)}^{2}}}\]
\end{problem}}

%%%%%%%%%%%%%%%%%%%%%%

\latexProblemContent{
\ifVerboseLocation This is Derivative Compute Question 0004. \\ \fi
\begin{problem}

Compute the following derivative:

\input{Derivative-Compute-0004.HELP.tex}

\[\dfrac{d}{dx}\left(\frac{1}{x^{2} - 16}\right)=\answer{-\frac{2 \, x}{{\left(x^{2} - 16\right)}^{2}}}\]
\end{problem}}

%%%%%%%%%%%%%%%%%%%%%%

\latexProblemContent{
\ifVerboseLocation This is Derivative Compute Question 0004. \\ \fi
\begin{problem}

Compute the following derivative:

\input{Derivative-Compute-0004.HELP.tex}

\[\dfrac{d}{dx}\left(\frac{x^{3} + x^{2} - 17 \, x + 15}{x - 3}\right)=\answer{\frac{2 \, {\left(x^{3} - 4 \, x^{2} - 3 \, x + 18\right)}}{{\left(x - 3\right)}^{2}}}\]
\end{problem}}

%%%%%%%%%%%%%%%%%%%%%%

\latexProblemContent{
\ifVerboseLocation This is Derivative Compute Question 0004. \\ \fi
\begin{problem}

Compute the following derivative:

\input{Derivative-Compute-0004.HELP.tex}

\[\dfrac{d}{dx}\left(\frac{1}{x^{3} + x^{2} - 16 \, x + 20}\right)=\answer{-\frac{3 \, x^{2} + 2 \, x - 16}{{\left(x^{3} + x^{2} - 16 \, x + 20\right)}^{2}}}\]
\end{problem}}

%%%%%%%%%%%%%%%%%%%%%%

\latexProblemContent{
\ifVerboseLocation This is Derivative Compute Question 0004. \\ \fi
\begin{problem}

Compute the following derivative:

\input{Derivative-Compute-0004.HELP.tex}

\[\dfrac{d}{dx}\left(\frac{x^{2} + 3 \, x + 2}{x - 1}\right)=\answer{\frac{x^{2} - 2 \, x - 5}{{\left(x - 1\right)}^{2}}}\]
\end{problem}}

%%%%%%%%%%%%%%%%%%%%%%

\latexProblemContent{
\ifVerboseLocation This is Derivative Compute Question 0004. \\ \fi
\begin{problem}

Compute the following derivative:

\input{Derivative-Compute-0004.HELP.tex}

\[\dfrac{d}{dx}\left(\frac{x^{2} + 5 \, x + 4}{x^{3} - 6 \, x^{2} + 32}\right)=\answer{-\frac{x^{4} + 10 \, x^{3} - 18 \, x^{2} - 112 \, x - 160}{{\left(x^{3} - 6 \, x^{2} + 32\right)}^{2}}}\]
\end{problem}}

%%%%%%%%%%%%%%%%%%%%%%

\latexProblemContent{
\ifVerboseLocation This is Derivative Compute Question 0004. \\ \fi
\begin{problem}

Compute the following derivative:

\input{Derivative-Compute-0004.HELP.tex}

\[\dfrac{d}{dx}\left(\frac{x^{2} + 2 \, x - 3}{x^{3} - 5 \, x^{2} - 16 \, x + 80}\right)=\answer{-\frac{x^{4} + 4 \, x^{3} - 3 \, x^{2} - 130 \, x - 112}{{\left(x^{3} - 5 \, x^{2} - 16 \, x + 80\right)}^{2}}}\]
\end{problem}}

%%%%%%%%%%%%%%%%%%%%%%

\latexProblemContent{
\ifVerboseLocation This is Derivative Compute Question 0004. \\ \fi
\begin{problem}

Compute the following derivative:

\input{Derivative-Compute-0004.HELP.tex}

\[\dfrac{d}{dx}\left(\frac{1}{x + 1}\right)=\answer{-\frac{1}{{\left(x + 1\right)}^{2}}}\]
\end{problem}}

%%%%%%%%%%%%%%%%%%%%%%

\latexProblemContent{
\ifVerboseLocation This is Derivative Compute Question 0004. \\ \fi
\begin{problem}

Compute the following derivative:

\input{Derivative-Compute-0004.HELP.tex}

\[\dfrac{d}{dx}\left(\frac{x^{2} - x - 2}{x^{2} + 4 \, x + 4}\right)=\answer{\frac{5 \, x^{2} + 12 \, x + 4}{{\left(x^{2} + 4 \, x + 4\right)}^{2}}}\]
\end{problem}}

%%%%%%%%%%%%%%%%%%%%%%

\latexProblemContent{
\ifVerboseLocation This is Derivative Compute Question 0004. \\ \fi
\begin{problem}

Compute the following derivative:

\input{Derivative-Compute-0004.HELP.tex}

\[\dfrac{d}{dx}\left(\frac{x^{2} - x - 2}{x^{2} - 2 \, x - 3}\right)=\answer{-\frac{x^{2} + 2 \, x + 1}{{\left(x^{2} - 2 \, x - 3\right)}^{2}}}\]
\end{problem}}

%%%%%%%%%%%%%%%%%%%%%%

\latexProblemContent{
\ifVerboseLocation This is Derivative Compute Question 0004. \\ \fi
\begin{problem}

Compute the following derivative:

\input{Derivative-Compute-0004.HELP.tex}

\[\dfrac{d}{dx}\left(\frac{x - 5}{x^{3} + 4 \, x^{2} - 16 \, x - 64}\right)=\answer{-\frac{2 \, x^{3} - 11 \, x^{2} - 40 \, x + 144}{{\left(x^{3} + 4 \, x^{2} - 16 \, x - 64\right)}^{2}}}\]
\end{problem}}

%%%%%%%%%%%%%%%%%%%%%%

\latexProblemContent{
\ifVerboseLocation This is Derivative Compute Question 0004. \\ \fi
\begin{problem}

Compute the following derivative:

\input{Derivative-Compute-0004.HELP.tex}

\[\dfrac{d}{dx}\left(\frac{x + 1}{x + 4}\right)=\answer{\frac{3}{{\left(x + 4\right)}^{2}}}\]
\end{problem}}

%%%%%%%%%%%%%%%%%%%%%%

\latexProblemContent{
\ifVerboseLocation This is Derivative Compute Question 0004. \\ \fi
\begin{problem}

Compute the following derivative:

\input{Derivative-Compute-0004.HELP.tex}

\[\dfrac{d}{dx}\left(\frac{x^{2} + 3 \, x + 2}{x^{2} - 3 \, x - 10}\right)=\answer{-\frac{6 \, {\left(x^{2} + 4 \, x + 4\right)}}{{\left(x^{2} - 3 \, x - 10\right)}^{2}}}\]
\end{problem}}

%%%%%%%%%%%%%%%%%%%%%%

\latexProblemContent{
\ifVerboseLocation This is Derivative Compute Question 0004. \\ \fi
\begin{problem}

Compute the following derivative:

\input{Derivative-Compute-0004.HELP.tex}

\[\dfrac{d}{dx}\left(\frac{1}{x + 3}\right)=\answer{-\frac{1}{{\left(x + 3\right)}^{2}}}\]
\end{problem}}

%%%%%%%%%%%%%%%%%%%%%%

\latexProblemContent{
\ifVerboseLocation This is Derivative Compute Question 0004. \\ \fi
\begin{problem}

Compute the following derivative:

\input{Derivative-Compute-0004.HELP.tex}

\[\dfrac{d}{dx}\left(\frac{x^{3} - 7 \, x^{2} + 7 \, x + 15}{x^{2} + 2 \, x - 8}\right)=\answer{\frac{x^{4} + 4 \, x^{3} - 45 \, x^{2} + 82 \, x - 86}{{\left(x^{2} + 2 \, x - 8\right)}^{2}}}\]
\end{problem}}

%%%%%%%%%%%%%%%%%%%%%%

\latexProblemContent{
\ifVerboseLocation This is Derivative Compute Question 0004. \\ \fi
\begin{problem}

Compute the following derivative:

\input{Derivative-Compute-0004.HELP.tex}

\[\dfrac{d}{dx}\left(\frac{1}{x^{2} + 7 \, x + 10}\right)=\answer{-\frac{2 \, x + 7}{{\left(x^{2} + 7 \, x + 10\right)}^{2}}}\]
\end{problem}}

%%%%%%%%%%%%%%%%%%%%%%

\latexProblemContent{
\ifVerboseLocation This is Derivative Compute Question 0004. \\ \fi
\begin{problem}

Compute the following derivative:

\input{Derivative-Compute-0004.HELP.tex}

\[\dfrac{d}{dx}\left(\frac{1}{x^{3} + 3 \, x^{2} - 18 \, x - 40}\right)=\answer{-\frac{3 \, {\left(x^{2} + 2 \, x - 6\right)}}{{\left(x^{3} + 3 \, x^{2} - 18 \, x - 40\right)}^{2}}}\]
\end{problem}}

%%%%%%%%%%%%%%%%%%%%%%

\latexProblemContent{
\ifVerboseLocation This is Derivative Compute Question 0004. \\ \fi
\begin{problem}

Compute the following derivative:

\input{Derivative-Compute-0004.HELP.tex}

\[\dfrac{d}{dx}\left(\frac{x + 2}{x^{3} - 4 \, x^{2} - x + 4}\right)=\answer{-\frac{2 \, {\left(x^{3} + x^{2} - 8 \, x - 3\right)}}{{\left(x^{3} - 4 \, x^{2} - x + 4\right)}^{2}}}\]
\end{problem}}

%%%%%%%%%%%%%%%%%%%%%%

\latexProblemContent{
\ifVerboseLocation This is Derivative Compute Question 0004. \\ \fi
\begin{problem}

Compute the following derivative:

\input{Derivative-Compute-0004.HELP.tex}

\[\dfrac{d}{dx}\left(\frac{x^{2} - 5 \, x + 6}{x^{3} + 4 \, x^{2} - 4 \, x - 16}\right)=\answer{-\frac{x^{4} - 10 \, x^{3} + 2 \, x^{2} + 80 \, x - 104}{{\left(x^{3} + 4 \, x^{2} - 4 \, x - 16\right)}^{2}}}\]
\end{problem}}

%%%%%%%%%%%%%%%%%%%%%%

\latexProblemContent{
\ifVerboseLocation This is Derivative Compute Question 0004. \\ \fi
\begin{problem}

Compute the following derivative:

\input{Derivative-Compute-0004.HELP.tex}

\[\dfrac{d}{dx}\left(\frac{x + 1}{x - 5}\right)=\answer{-\frac{6}{{\left(x - 5\right)}^{2}}}\]
\end{problem}}

%%%%%%%%%%%%%%%%%%%%%%

\latexProblemContent{
\ifVerboseLocation This is Derivative Compute Question 0004. \\ \fi
\begin{problem}

Compute the following derivative:

\input{Derivative-Compute-0004.HELP.tex}

\[\dfrac{d}{dx}\left(\frac{x^{3} - 5 \, x^{2} - 4 \, x + 20}{x^{2} - 5 \, x + 4}\right)=\answer{\frac{x^{4} - 10 \, x^{3} + 41 \, x^{2} - 80 \, x + 84}{{\left(x^{2} - 5 \, x + 4\right)}^{2}}}\]
\end{problem}}

%%%%%%%%%%%%%%%%%%%%%%

\latexProblemContent{
\ifVerboseLocation This is Derivative Compute Question 0004. \\ \fi
\begin{problem}

Compute the following derivative:

\input{Derivative-Compute-0004.HELP.tex}

\[\dfrac{d}{dx}\left(\frac{x^{2} - 2 \, x - 15}{x^{3} + x^{2} - 8 \, x - 12}\right)=\answer{-\frac{x^{4} - 4 \, x^{3} - 39 \, x^{2} - 6 \, x + 96}{{\left(x^{3} + x^{2} - 8 \, x - 12\right)}^{2}}}\]
\end{problem}}

%%%%%%%%%%%%%%%%%%%%%%

\latexProblemContent{
\ifVerboseLocation This is Derivative Compute Question 0004. \\ \fi
\begin{problem}

Compute the following derivative:

\input{Derivative-Compute-0004.HELP.tex}

\[\dfrac{d}{dx}\left(\frac{1}{x - 4}\right)=\answer{-\frac{1}{{\left(x - 4\right)}^{2}}}\]
\end{problem}}

%%%%%%%%%%%%%%%%%%%%%%

\latexProblemContent{
\ifVerboseLocation This is Derivative Compute Question 0004. \\ \fi
\begin{problem}

Compute the following derivative:

\input{Derivative-Compute-0004.HELP.tex}

\[\dfrac{d}{dx}\left(\frac{x^{3} + 2 \, x^{2} - 25 \, x - 50}{x + 1}\right)=\answer{\frac{2 \, x^{3} + 5 \, x^{2} + 4 \, x + 25}{{\left(x + 1\right)}^{2}}}\]
\end{problem}}

%%%%%%%%%%%%%%%%%%%%%%

\latexProblemContent{
\ifVerboseLocation This is Derivative Compute Question 0004. \\ \fi
\begin{problem}

Compute the following derivative:

\input{Derivative-Compute-0004.HELP.tex}

\[\dfrac{d}{dx}\left(\frac{x^{2} + x - 12}{x^{2} - 4 \, x - 5}\right)=\answer{-\frac{5 \, x^{2} - 14 \, x + 53}{{\left(x^{2} - 4 \, x - 5\right)}^{2}}}\]
\end{problem}}

%%%%%%%%%%%%%%%%%%%%%%

\latexProblemContent{
\ifVerboseLocation This is Derivative Compute Question 0004. \\ \fi
\begin{problem}

Compute the following derivative:

\input{Derivative-Compute-0004.HELP.tex}

\[\dfrac{d}{dx}\left(\frac{x - 1}{x - 4}\right)=\answer{-\frac{3}{{\left(x - 4\right)}^{2}}}\]
\end{problem}}

%%%%%%%%%%%%%%%%%%%%%%

\latexProblemContent{
\ifVerboseLocation This is Derivative Compute Question 0004. \\ \fi
\begin{problem}

Compute the following derivative:

\input{Derivative-Compute-0004.HELP.tex}

\[\dfrac{d}{dx}\left(\frac{x^{3} - 13 \, x - 12}{x - 3}\right)=\answer{\frac{2 \, x^{3} - 9 \, x^{2} + 51}{{\left(x - 3\right)}^{2}}}\]
\end{problem}}

%%%%%%%%%%%%%%%%%%%%%%

\latexProblemContent{
\ifVerboseLocation This is Derivative Compute Question 0004. \\ \fi
\begin{problem}

Compute the following derivative:

\input{Derivative-Compute-0004.HELP.tex}

\[\dfrac{d}{dx}\left(\frac{x + 5}{x - 3}\right)=\answer{-\frac{8}{{\left(x - 3\right)}^{2}}}\]
\end{problem}}

%%%%%%%%%%%%%%%%%%%%%%

\latexProblemContent{
\ifVerboseLocation This is Derivative Compute Question 0004. \\ \fi
\begin{problem}

Compute the following derivative:

\input{Derivative-Compute-0004.HELP.tex}

\[\dfrac{d}{dx}\left(\frac{1}{x + 5}\right)=\answer{-\frac{1}{{\left(x + 5\right)}^{2}}}\]
\end{problem}}

%%%%%%%%%%%%%%%%%%%%%%

\latexProblemContent{
\ifVerboseLocation This is Derivative Compute Question 0004. \\ \fi
\begin{problem}

Compute the following derivative:

\input{Derivative-Compute-0004.HELP.tex}

\[\dfrac{d}{dx}\left(\frac{x + 3}{x^{2} + 3 \, x - 4}\right)=\answer{-\frac{x^{2} + 6 \, x + 13}{{\left(x^{2} + 3 \, x - 4\right)}^{2}}}\]
\end{problem}}

%%%%%%%%%%%%%%%%%%%%%%

\latexProblemContent{
\ifVerboseLocation This is Derivative Compute Question 0004. \\ \fi
\begin{problem}

Compute the following derivative:

\input{Derivative-Compute-0004.HELP.tex}

\[\dfrac{d}{dx}\left(\frac{x^{2} - 7 \, x + 12}{x^{3} - 4 \, x^{2} - 25 \, x + 100}\right)=\answer{-\frac{x^{4} - 14 \, x^{3} + 89 \, x^{2} - 296 \, x + 400}{{\left(x^{3} - 4 \, x^{2} - 25 \, x + 100\right)}^{2}}}\]
\end{problem}}

%%%%%%%%%%%%%%%%%%%%%%

\latexProblemContent{
\ifVerboseLocation This is Derivative Compute Question 0004. \\ \fi
\begin{problem}

Compute the following derivative:

\input{Derivative-Compute-0004.HELP.tex}

\[\dfrac{d}{dx}\left(\frac{x^{2} - x - 12}{x^{3} - 4 \, x^{2} + x + 6}\right)=\answer{-\frac{x^{4} - 2 \, x^{3} - 33 \, x^{2} + 84 \, x - 6}{{\left(x^{3} - 4 \, x^{2} + x + 6\right)}^{2}}}\]
\end{problem}}

%%%%%%%%%%%%%%%%%%%%%%

\latexProblemContent{
\ifVerboseLocation This is Derivative Compute Question 0004. \\ \fi
\begin{problem}

Compute the following derivative:

\input{Derivative-Compute-0004.HELP.tex}

\[\dfrac{d}{dx}\left(\frac{1}{x^{3} + x^{2} - 16 \, x - 16}\right)=\answer{-\frac{3 \, x^{2} + 2 \, x - 16}{{\left(x^{3} + x^{2} - 16 \, x - 16\right)}^{2}}}\]
\end{problem}}

%%%%%%%%%%%%%%%%%%%%%%

\latexProblemContent{
\ifVerboseLocation This is Derivative Compute Question 0004. \\ \fi
\begin{problem}

Compute the following derivative:

\input{Derivative-Compute-0004.HELP.tex}

\[\dfrac{d}{dx}\left(\frac{x^{3} + 4 \, x^{2} - 16 \, x - 64}{x^{3} + 2 \, x^{2} - 25 \, x - 50}\right)=\answer{-\frac{2 \, {\left(x^{4} + 9 \, x^{3} + 13 \, x^{2} + 72 \, x + 400\right)}}{{\left(x^{3} + 2 \, x^{2} - 25 \, x - 50\right)}^{2}}}\]
\end{problem}}

%%%%%%%%%%%%%%%%%%%%%%

\latexProblemContent{
\ifVerboseLocation This is Derivative Compute Question 0004. \\ \fi
\begin{problem}

Compute the following derivative:

\input{Derivative-Compute-0004.HELP.tex}

\[\dfrac{d}{dx}\left(\frac{x^{3} + x^{2} - 16 \, x - 16}{x^{3} + x^{2} - 22 \, x - 40}\right)=\answer{-\frac{6 \, {\left(2 \, x^{3} + 13 \, x^{2} + 8 \, x - 48\right)}}{{\left(x^{3} + x^{2} - 22 \, x - 40\right)}^{2}}}\]
\end{problem}}

%%%%%%%%%%%%%%%%%%%%%%

\latexProblemContent{
\ifVerboseLocation This is Derivative Compute Question 0004. \\ \fi
\begin{problem}

Compute the following derivative:

\input{Derivative-Compute-0004.HELP.tex}

\[\dfrac{d}{dx}\left(\frac{x^{3} - 9 \, x^{2} + 23 \, x - 15}{x + 3}\right)=\answer{\frac{2 \, {\left(x^{3} - 27 \, x + 42\right)}}{{\left(x + 3\right)}^{2}}}\]
\end{problem}}

%%%%%%%%%%%%%%%%%%%%%%

\latexProblemContent{
\ifVerboseLocation This is Derivative Compute Question 0004. \\ \fi
\begin{problem}

Compute the following derivative:

\input{Derivative-Compute-0004.HELP.tex}

\[\dfrac{d}{dx}\left(\frac{x - 5}{x^{3} - x^{2} - 9 \, x + 9}\right)=\answer{-\frac{2 \, {\left(x^{3} - 8 \, x^{2} + 5 \, x + 18\right)}}{{\left(x^{3} - x^{2} - 9 \, x + 9\right)}^{2}}}\]
\end{problem}}

%%%%%%%%%%%%%%%%%%%%%%

\latexProblemContent{
\ifVerboseLocation This is Derivative Compute Question 0004. \\ \fi
\begin{problem}

Compute the following derivative:

\input{Derivative-Compute-0004.HELP.tex}

\[\dfrac{d}{dx}\left(\frac{x^{3} - 7 \, x^{2} + 14 \, x - 8}{x + 1}\right)=\answer{\frac{2 \, {\left(x^{3} - 2 \, x^{2} - 7 \, x + 11\right)}}{{\left(x + 1\right)}^{2}}}\]
\end{problem}}

%%%%%%%%%%%%%%%%%%%%%%

\latexProblemContent{
\ifVerboseLocation This is Derivative Compute Question 0004. \\ \fi
\begin{problem}

Compute the following derivative:

\input{Derivative-Compute-0004.HELP.tex}

\[\dfrac{d}{dx}\left(\frac{x^{3} + 3 \, x^{2} - 18 \, x - 40}{x^{3} + 10 \, x^{2} + 32 \, x + 32}\right)=\answer{\frac{7 \, x^{4} + 100 \, x^{3} + 492 \, x^{2} + 992 \, x + 704}{{\left(x^{3} + 10 \, x^{2} + 32 \, x + 32\right)}^{2}}}\]
\end{problem}}

%%%%%%%%%%%%%%%%%%%%%%

\latexProblemContent{
\ifVerboseLocation This is Derivative Compute Question 0004. \\ \fi
\begin{problem}

Compute the following derivative:

\input{Derivative-Compute-0004.HELP.tex}

\[\dfrac{d}{dx}\left(\frac{1}{x^{3} + x^{2} - 10 \, x + 8}\right)=\answer{-\frac{3 \, x^{2} + 2 \, x - 10}{{\left(x^{3} + x^{2} - 10 \, x + 8\right)}^{2}}}\]
\end{problem}}

%%%%%%%%%%%%%%%%%%%%%%

\latexProblemContent{
\ifVerboseLocation This is Derivative Compute Question 0004. \\ \fi
\begin{problem}

Compute the following derivative:

\input{Derivative-Compute-0004.HELP.tex}

\[\dfrac{d}{dx}\left(\frac{x^{3} - 8 \, x^{2} + 17 \, x - 10}{x - 4}\right)=\answer{\frac{2 \, {\left(x^{3} - 10 \, x^{2} + 32 \, x - 29\right)}}{{\left(x - 4\right)}^{2}}}\]
\end{problem}}

%%%%%%%%%%%%%%%%%%%%%%

\latexProblemContent{
\ifVerboseLocation This is Derivative Compute Question 0004. \\ \fi
\begin{problem}

Compute the following derivative:

\input{Derivative-Compute-0004.HELP.tex}

\[\dfrac{d}{dx}\left(\frac{x^{3} + 7 \, x^{2} + 14 \, x + 8}{x^{2} + 7 \, x + 10}\right)=\answer{\frac{x^{4} + 14 \, x^{3} + 65 \, x^{2} + 124 \, x + 84}{{\left(x^{2} + 7 \, x + 10\right)}^{2}}}\]
\end{problem}}

%%%%%%%%%%%%%%%%%%%%%%

\latexProblemContent{
\ifVerboseLocation This is Derivative Compute Question 0004. \\ \fi
\begin{problem}

Compute the following derivative:

\input{Derivative-Compute-0004.HELP.tex}

\[\dfrac{d}{dx}\left(\frac{x^{2} - 16}{x + 3}\right)=\answer{\frac{x^{2} + 6 \, x + 16}{{\left(x + 3\right)}^{2}}}\]
\end{problem}}

%%%%%%%%%%%%%%%%%%%%%%

\latexProblemContent{
\ifVerboseLocation This is Derivative Compute Question 0004. \\ \fi
\begin{problem}

Compute the following derivative:

\input{Derivative-Compute-0004.HELP.tex}

\[\dfrac{d}{dx}\left(\frac{x^{3} + 8 \, x^{2} + 19 \, x + 12}{x^{3} - 19 \, x - 30}\right)=\answer{-\frac{2 \, {\left(4 \, x^{4} + 38 \, x^{3} + 139 \, x^{2} + 240 \, x + 171\right)}}{{\left(x^{3} - 19 \, x - 30\right)}^{2}}}\]
\end{problem}}

%%%%%%%%%%%%%%%%%%%%%%

\latexProblemContent{
\ifVerboseLocation This is Derivative Compute Question 0004. \\ \fi
\begin{problem}

Compute the following derivative:

\input{Derivative-Compute-0004.HELP.tex}

\[\dfrac{d}{dx}\left(\frac{x - 4}{x^{2} - 4}\right)=\answer{-\frac{x^{2} - 8 \, x + 4}{{\left(x^{2} - 4\right)}^{2}}}\]
\end{problem}}

%%%%%%%%%%%%%%%%%%%%%%

\latexProblemContent{
\ifVerboseLocation This is Derivative Compute Question 0004. \\ \fi
\begin{problem}

Compute the following derivative:

\input{Derivative-Compute-0004.HELP.tex}

\[\dfrac{d}{dx}\left(\frac{x^{2} + 7 \, x + 12}{x^{2} - 4}\right)=\answer{-\frac{7 \, x^{2} + 32 \, x + 28}{{\left(x^{2} - 4\right)}^{2}}}\]
\end{problem}}

%%%%%%%%%%%%%%%%%%%%%%

\latexProblemContent{
\ifVerboseLocation This is Derivative Compute Question 0004. \\ \fi
\begin{problem}

Compute the following derivative:

\input{Derivative-Compute-0004.HELP.tex}

\[\dfrac{d}{dx}\left(\frac{x^{3} - 10 \, x^{2} + 29 \, x - 20}{x^{3} - 3 \, x^{2} - 6 \, x + 8}\right)=\answer{\frac{7 \, {\left(x^{4} - 10 \, x^{3} + 33 \, x^{2} - 40 \, x + 16\right)}}{{\left(x^{3} - 3 \, x^{2} - 6 \, x + 8\right)}^{2}}}\]
\end{problem}}

%%%%%%%%%%%%%%%%%%%%%%

\latexProblemContent{
\ifVerboseLocation This is Derivative Compute Question 0004. \\ \fi
\begin{problem}

Compute the following derivative:

\input{Derivative-Compute-0004.HELP.tex}

\[\dfrac{d}{dx}\left(\frac{x + 1}{x^{2} - 4}\right)=\answer{-\frac{x^{2} + 2 \, x + 4}{{\left(x^{2} - 4\right)}^{2}}}\]
\end{problem}}

%%%%%%%%%%%%%%%%%%%%%%

\latexProblemContent{
\ifVerboseLocation This is Derivative Compute Question 0004. \\ \fi
\begin{problem}

Compute the following derivative:

\input{Derivative-Compute-0004.HELP.tex}

\[\dfrac{d}{dx}\left(\frac{x^{3} + 2 \, x^{2} - 23 \, x - 60}{x^{3} + 4 \, x^{2} - x - 4}\right)=\answer{\frac{2 \, {\left(x^{4} + 22 \, x^{3} + 129 \, x^{2} + 232 \, x + 16\right)}}{{\left(x^{3} + 4 \, x^{2} - x - 4\right)}^{2}}}\]
\end{problem}}

%%%%%%%%%%%%%%%%%%%%%%

\latexProblemContent{
\ifVerboseLocation This is Derivative Compute Question 0004. \\ \fi
\begin{problem}

Compute the following derivative:

\input{Derivative-Compute-0004.HELP.tex}

\[\dfrac{d}{dx}\left(\frac{x - 2}{x^{3} + 4 \, x^{2} - 4 \, x - 16}\right)=\answer{-\frac{2 \, {\left(x^{3} - x^{2} - 8 \, x + 12\right)}}{{\left(x^{3} + 4 \, x^{2} - 4 \, x - 16\right)}^{2}}}\]
\end{problem}}

%%%%%%%%%%%%%%%%%%%%%%

\latexProblemContent{
\ifVerboseLocation This is Derivative Compute Question 0004. \\ \fi
\begin{problem}

Compute the following derivative:

\input{Derivative-Compute-0004.HELP.tex}

\[\dfrac{d}{dx}\left(\frac{x^{2} - 2 \, x - 15}{x + 2}\right)=\answer{\frac{x^{2} + 4 \, x + 11}{{\left(x + 2\right)}^{2}}}\]
\end{problem}}

%%%%%%%%%%%%%%%%%%%%%%

\latexProblemContent{
\ifVerboseLocation This is Derivative Compute Question 0004. \\ \fi
\begin{problem}

Compute the following derivative:

\input{Derivative-Compute-0004.HELP.tex}

\[\dfrac{d}{dx}\left(\frac{x^{2} - x - 20}{x^{3} + 3 \, x^{2} - 18 \, x - 40}\right)=\answer{-\frac{x^{4} - 2 \, x^{3} - 45 \, x^{2} - 40 \, x + 320}{{\left(x^{3} + 3 \, x^{2} - 18 \, x - 40\right)}^{2}}}\]
\end{problem}}

%%%%%%%%%%%%%%%%%%%%%%

\latexProblemContent{
\ifVerboseLocation This is Derivative Compute Question 0004. \\ \fi
\begin{problem}

Compute the following derivative:

\input{Derivative-Compute-0004.HELP.tex}

\[\dfrac{d}{dx}\left(\frac{x^{2} + 3 \, x - 10}{x^{3} - 4 \, x^{2} - 7 \, x + 10}\right)=\answer{-\frac{x^{4} + 6 \, x^{3} - 35 \, x^{2} + 60 \, x + 40}{{\left(x^{3} - 4 \, x^{2} - 7 \, x + 10\right)}^{2}}}\]
\end{problem}}

%%%%%%%%%%%%%%%%%%%%%%

\latexProblemContent{
\ifVerboseLocation This is Derivative Compute Question 0004. \\ \fi
\begin{problem}

Compute the following derivative:

\input{Derivative-Compute-0004.HELP.tex}

\[\dfrac{d}{dx}\left(\frac{x + 2}{x^{2} + 6 \, x + 5}\right)=\answer{-\frac{x^{2} + 4 \, x + 7}{{\left(x^{2} + 6 \, x + 5\right)}^{2}}}\]
\end{problem}}

%%%%%%%%%%%%%%%%%%%%%%

\latexProblemContent{
\ifVerboseLocation This is Derivative Compute Question 0004. \\ \fi
\begin{problem}

Compute the following derivative:

\input{Derivative-Compute-0004.HELP.tex}

\[\dfrac{d}{dx}\left(\frac{x + 1}{x + 2}\right)=\answer{\frac{1}{{\left(x + 2\right)}^{2}}}\]
\end{problem}}

%%%%%%%%%%%%%%%%%%%%%%

\latexProblemContent{
\ifVerboseLocation This is Derivative Compute Question 0004. \\ \fi
\begin{problem}

Compute the following derivative:

\input{Derivative-Compute-0004.HELP.tex}

\[\dfrac{d}{dx}\left(\frac{x^{3} - x^{2} - 16 \, x + 16}{x + 2}\right)=\answer{\frac{2 \, x^{3} + 5 \, x^{2} - 4 \, x - 48}{{\left(x + 2\right)}^{2}}}\]
\end{problem}}

%%%%%%%%%%%%%%%%%%%%%%

\latexProblemContent{
\ifVerboseLocation This is Derivative Compute Question 0004. \\ \fi
\begin{problem}

Compute the following derivative:

\input{Derivative-Compute-0004.HELP.tex}

\[\dfrac{d}{dx}\left(\frac{x^{2} - 3 \, x + 2}{x + 2}\right)=\answer{\frac{x^{2} + 4 \, x - 8}{{\left(x + 2\right)}^{2}}}\]
\end{problem}}

%%%%%%%%%%%%%%%%%%%%%%

\latexProblemContent{
\ifVerboseLocation This is Derivative Compute Question 0004. \\ \fi
\begin{problem}

Compute the following derivative:

\input{Derivative-Compute-0004.HELP.tex}

\[\dfrac{d}{dx}\left(\frac{1}{x^{3} - 19 \, x + 30}\right)=\answer{-\frac{3 \, x^{2} - 19}{{\left(x^{3} - 19 \, x + 30\right)}^{2}}}\]
\end{problem}}

%%%%%%%%%%%%%%%%%%%%%%

\latexProblemContent{
\ifVerboseLocation This is Derivative Compute Question 0004. \\ \fi
\begin{problem}

Compute the following derivative:

\input{Derivative-Compute-0004.HELP.tex}

\[\dfrac{d}{dx}\left(\frac{x^{2} - 5 \, x + 4}{x^{3} + 3 \, x^{2} - 4}\right)=\answer{-\frac{x^{4} - 10 \, x^{3} - 3 \, x^{2} + 32 \, x - 20}{{\left(x^{3} + 3 \, x^{2} - 4\right)}^{2}}}\]
\end{problem}}

%%%%%%%%%%%%%%%%%%%%%%

\latexProblemContent{
\ifVerboseLocation This is Derivative Compute Question 0004. \\ \fi
\begin{problem}

Compute the following derivative:

\input{Derivative-Compute-0004.HELP.tex}

\[\dfrac{d}{dx}\left(\frac{x^{2} + 3 \, x + 2}{x^{2} + 2 \, x - 15}\right)=\answer{-\frac{x^{2} + 34 \, x + 49}{{\left(x^{2} + 2 \, x - 15\right)}^{2}}}\]
\end{problem}}

%%%%%%%%%%%%%%%%%%%%%%

\latexProblemContent{
\ifVerboseLocation This is Derivative Compute Question 0004. \\ \fi
\begin{problem}

Compute the following derivative:

\input{Derivative-Compute-0004.HELP.tex}

\[\dfrac{d}{dx}\left(\frac{x - 3}{x^{3} - 11 \, x^{2} + 35 \, x - 25}\right)=\answer{-\frac{2 \, {\left(x^{3} - 10 \, x^{2} + 33 \, x - 40\right)}}{{\left(x^{3} - 11 \, x^{2} + 35 \, x - 25\right)}^{2}}}\]
\end{problem}}

%%%%%%%%%%%%%%%%%%%%%%

\latexProblemContent{
\ifVerboseLocation This is Derivative Compute Question 0004. \\ \fi
\begin{problem}

Compute the following derivative:

\input{Derivative-Compute-0004.HELP.tex}

\[\dfrac{d}{dx}\left(\frac{x - 4}{x + 5}\right)=\answer{\frac{9}{{\left(x + 5\right)}^{2}}}\]
\end{problem}}

%%%%%%%%%%%%%%%%%%%%%%

\latexProblemContent{
\ifVerboseLocation This is Derivative Compute Question 0004. \\ \fi
\begin{problem}

Compute the following derivative:

\input{Derivative-Compute-0004.HELP.tex}

\[\dfrac{d}{dx}\left(\frac{x^{3} + 7 \, x^{2} + 14 \, x + 8}{x + 1}\right)=\answer{\frac{2 \, {\left(x^{3} + 5 \, x^{2} + 7 \, x + 3\right)}}{{\left(x + 1\right)}^{2}}}\]
\end{problem}}

%%%%%%%%%%%%%%%%%%%%%%

\latexProblemContent{
\ifVerboseLocation This is Derivative Compute Question 0004. \\ \fi
\begin{problem}

Compute the following derivative:

\input{Derivative-Compute-0004.HELP.tex}

\[\dfrac{d}{dx}\left(\frac{x^{2} - 6 \, x + 5}{x + 1}\right)=\answer{\frac{x^{2} + 2 \, x - 11}{{\left(x + 1\right)}^{2}}}\]
\end{problem}}

%%%%%%%%%%%%%%%%%%%%%%

\latexProblemContent{
\ifVerboseLocation This is Derivative Compute Question 0004. \\ \fi
\begin{problem}

Compute the following derivative:

\input{Derivative-Compute-0004.HELP.tex}

\[\dfrac{d}{dx}\left(\frac{x - 4}{x^{2} - 4 \, x + 3}\right)=\answer{-\frac{x^{2} - 8 \, x + 13}{{\left(x^{2} - 4 \, x + 3\right)}^{2}}}\]
\end{problem}}

%%%%%%%%%%%%%%%%%%%%%%

\latexProblemContent{
\ifVerboseLocation This is Derivative Compute Question 0004. \\ \fi
\begin{problem}

Compute the following derivative:

\input{Derivative-Compute-0004.HELP.tex}

\[\dfrac{d}{dx}\left(\frac{x^{2} - 4 \, x - 5}{x^{3} + 6 \, x^{2} - x - 30}\right)=\answer{-\frac{x^{4} - 8 \, x^{3} - 38 \, x^{2} - 115}{{\left(x^{3} + 6 \, x^{2} - x - 30\right)}^{2}}}\]
\end{problem}}

%%%%%%%%%%%%%%%%%%%%%%

\latexProblemContent{
\ifVerboseLocation This is Derivative Compute Question 0004. \\ \fi
\begin{problem}

Compute the following derivative:

\input{Derivative-Compute-0004.HELP.tex}

\[\dfrac{d}{dx}\left(\frac{1}{x^{3} - 5 \, x^{2} - 16 \, x + 80}\right)=\answer{-\frac{3 \, x^{2} - 10 \, x - 16}{{\left(x^{3} - 5 \, x^{2} - 16 \, x + 80\right)}^{2}}}\]
\end{problem}}

%%%%%%%%%%%%%%%%%%%%%%

\latexProblemContent{
\ifVerboseLocation This is Derivative Compute Question 0004. \\ \fi
\begin{problem}

Compute the following derivative:

\input{Derivative-Compute-0004.HELP.tex}

\[\dfrac{d}{dx}\left(\frac{x - 2}{x - 4}\right)=\answer{-\frac{2}{{\left(x - 4\right)}^{2}}}\]
\end{problem}}

%%%%%%%%%%%%%%%%%%%%%%

\latexProblemContent{
\ifVerboseLocation This is Derivative Compute Question 0004. \\ \fi
\begin{problem}

Compute the following derivative:

\input{Derivative-Compute-0004.HELP.tex}

\[\dfrac{d}{dx}\left(\frac{x^{2} - 3 \, x - 4}{x^{3} - x^{2} - 14 \, x + 24}\right)=\answer{-\frac{x^{4} - 6 \, x^{3} + 5 \, x^{2} - 40 \, x + 128}{{\left(x^{3} - x^{2} - 14 \, x + 24\right)}^{2}}}\]
\end{problem}}

%%%%%%%%%%%%%%%%%%%%%%

\latexProblemContent{
\ifVerboseLocation This is Derivative Compute Question 0004. \\ \fi
\begin{problem}

Compute the following derivative:

\input{Derivative-Compute-0004.HELP.tex}

\[\dfrac{d}{dx}\left(\frac{x^{2} + 2 \, x - 3}{x^{3} + x^{2} - 22 \, x - 40}\right)=\answer{-\frac{x^{4} + 4 \, x^{3} + 15 \, x^{2} + 74 \, x + 146}{{\left(x^{3} + x^{2} - 22 \, x - 40\right)}^{2}}}\]
\end{problem}}

%%%%%%%%%%%%%%%%%%%%%%

\latexProblemContent{
\ifVerboseLocation This is Derivative Compute Question 0004. \\ \fi
\begin{problem}

Compute the following derivative:

\input{Derivative-Compute-0004.HELP.tex}

\[\dfrac{d}{dx}\left(\frac{x - 5}{x^{2} + 8 \, x + 16}\right)=\answer{-\frac{x^{2} - 10 \, x - 56}{{\left(x^{2} + 8 \, x + 16\right)}^{2}}}\]
\end{problem}}

%%%%%%%%%%%%%%%%%%%%%%

\latexProblemContent{
\ifVerboseLocation This is Derivative Compute Question 0004. \\ \fi
\begin{problem}

Compute the following derivative:

\input{Derivative-Compute-0004.HELP.tex}

\[\dfrac{d}{dx}\left(\frac{x^{3} + 2 \, x^{2} - x - 2}{x - 2}\right)=\answer{\frac{2 \, {\left(x^{3} - 2 \, x^{2} - 4 \, x + 2\right)}}{{\left(x - 2\right)}^{2}}}\]
\end{problem}}

%%%%%%%%%%%%%%%%%%%%%%

\latexProblemContent{
\ifVerboseLocation This is Derivative Compute Question 0004. \\ \fi
\begin{problem}

Compute the following derivative:

\input{Derivative-Compute-0004.HELP.tex}

\[\dfrac{d}{dx}\left(\frac{x - 4}{x^{3} - 13 \, x^{2} + 55 \, x - 75}\right)=\answer{-\frac{2 \, x^{3} - 25 \, x^{2} + 104 \, x - 145}{{\left(x^{3} - 13 \, x^{2} + 55 \, x - 75\right)}^{2}}}\]
\end{problem}}

%%%%%%%%%%%%%%%%%%%%%%

\latexProblemContent{
\ifVerboseLocation This is Derivative Compute Question 0004. \\ \fi
\begin{problem}

Compute the following derivative:

\input{Derivative-Compute-0004.HELP.tex}

\[\dfrac{d}{dx}\left(\frac{x + 3}{x^{2} + 7 \, x + 10}\right)=\answer{-\frac{x^{2} + 6 \, x + 11}{{\left(x^{2} + 7 \, x + 10\right)}^{2}}}\]
\end{problem}}

%%%%%%%%%%%%%%%%%%%%%%

\latexProblemContent{
\ifVerboseLocation This is Derivative Compute Question 0004. \\ \fi
\begin{problem}

Compute the following derivative:

\input{Derivative-Compute-0004.HELP.tex}

\[\dfrac{d}{dx}\left(\frac{x + 2}{x - 2}\right)=\answer{-\frac{4}{{\left(x - 2\right)}^{2}}}\]
\end{problem}}

%%%%%%%%%%%%%%%%%%%%%%

\latexProblemContent{
\ifVerboseLocation This is Derivative Compute Question 0004. \\ \fi
\begin{problem}

Compute the following derivative:

\input{Derivative-Compute-0004.HELP.tex}

\[\dfrac{d}{dx}\left(\frac{x^{2} + 2 \, x - 3}{x^{2} - 9}\right)=\answer{-\frac{2 \, {\left(x^{2} + 6 \, x + 9\right)}}{{\left(x^{2} - 9\right)}^{2}}}\]
\end{problem}}

%%%%%%%%%%%%%%%%%%%%%%

\latexProblemContent{
\ifVerboseLocation This is Derivative Compute Question 0004. \\ \fi
\begin{problem}

Compute the following derivative:

\input{Derivative-Compute-0004.HELP.tex}

\[\dfrac{d}{dx}\left(\frac{x - 5}{x^{2} + 2 \, x + 1}\right)=\answer{-\frac{x^{2} - 10 \, x - 11}{{\left(x^{2} + 2 \, x + 1\right)}^{2}}}\]
\end{problem}}

%%%%%%%%%%%%%%%%%%%%%%

\latexProblemContent{
\ifVerboseLocation This is Derivative Compute Question 0004. \\ \fi
\begin{problem}

Compute the following derivative:

\input{Derivative-Compute-0004.HELP.tex}

\[\dfrac{d}{dx}\left(\frac{x^{2} - 3 \, x + 2}{x^{3} - 21 \, x - 20}\right)=\answer{-\frac{x^{4} - 6 \, x^{3} + 27 \, x^{2} + 40 \, x - 102}{{\left(x^{3} - 21 \, x - 20\right)}^{2}}}\]
\end{problem}}

%%%%%%%%%%%%%%%%%%%%%%

\latexProblemContent{
\ifVerboseLocation This is Derivative Compute Question 0004. \\ \fi
\begin{problem}

Compute the following derivative:

\input{Derivative-Compute-0004.HELP.tex}

\[\dfrac{d}{dx}\left(\frac{1}{x^{2} + 2 \, x - 15}\right)=\answer{-\frac{2 \, {\left(x + 1\right)}}{{\left(x^{2} + 2 \, x - 15\right)}^{2}}}\]
\end{problem}}

%%%%%%%%%%%%%%%%%%%%%%

\latexProblemContent{
\ifVerboseLocation This is Derivative Compute Question 0004. \\ \fi
\begin{problem}

Compute the following derivative:

\input{Derivative-Compute-0004.HELP.tex}

\[\dfrac{d}{dx}\left(\frac{x^{2} + x - 12}{x + 2}\right)=\answer{\frac{x^{2} + 4 \, x + 14}{{\left(x + 2\right)}^{2}}}\]
\end{problem}}

%%%%%%%%%%%%%%%%%%%%%%

\latexProblemContent{
\ifVerboseLocation This is Derivative Compute Question 0004. \\ \fi
\begin{problem}

Compute the following derivative:

\input{Derivative-Compute-0004.HELP.tex}

\[\dfrac{d}{dx}\left(\frac{x^{2} + 7 \, x + 10}{x^{3} - 14 \, x^{2} + 65 \, x - 100}\right)=\answer{-\frac{x^{4} + 14 \, x^{3} - 133 \, x^{2} - 80 \, x + 1350}{{\left(x^{3} - 14 \, x^{2} + 65 \, x - 100\right)}^{2}}}\]
\end{problem}}

%%%%%%%%%%%%%%%%%%%%%%

\latexProblemContent{
\ifVerboseLocation This is Derivative Compute Question 0004. \\ \fi
\begin{problem}

Compute the following derivative:

\input{Derivative-Compute-0004.HELP.tex}

\[\dfrac{d}{dx}\left(\frac{x + 3}{x^{2} + 6 \, x + 8}\right)=\answer{-\frac{x^{2} + 6 \, x + 10}{{\left(x^{2} + 6 \, x + 8\right)}^{2}}}\]
\end{problem}}

%%%%%%%%%%%%%%%%%%%%%%

\latexProblemContent{
\ifVerboseLocation This is Derivative Compute Question 0004. \\ \fi
\begin{problem}

Compute the following derivative:

\input{Derivative-Compute-0004.HELP.tex}

\[\dfrac{d}{dx}\left(\frac{x + 5}{x^{3} - 5 \, x^{2} - 2 \, x + 24}\right)=\answer{-\frac{2 \, {\left(x^{3} + 5 \, x^{2} - 25 \, x - 17\right)}}{{\left(x^{3} - 5 \, x^{2} - 2 \, x + 24\right)}^{2}}}\]
\end{problem}}

%%%%%%%%%%%%%%%%%%%%%%

\latexProblemContent{
\ifVerboseLocation This is Derivative Compute Question 0004. \\ \fi
\begin{problem}

Compute the following derivative:

\input{Derivative-Compute-0004.HELP.tex}

\[\dfrac{d}{dx}\left(\frac{x^{3} + 2 \, x^{2} - 15 \, x - 36}{x^{3} - 5 \, x^{2} - x + 5}\right)=\answer{-\frac{7 \, x^{4} - 28 \, x^{3} - 46 \, x^{2} + 340 \, x + 111}{{\left(x^{3} - 5 \, x^{2} - x + 5\right)}^{2}}}\]
\end{problem}}

%%%%%%%%%%%%%%%%%%%%%%

\latexProblemContent{
\ifVerboseLocation This is Derivative Compute Question 0004. \\ \fi
\begin{problem}

Compute the following derivative:

\input{Derivative-Compute-0004.HELP.tex}

\[\dfrac{d}{dx}\left(\frac{x + 2}{x^{2} + 6 \, x + 9}\right)=\answer{-\frac{x^{2} + 4 \, x + 3}{{\left(x^{2} + 6 \, x + 9\right)}^{2}}}\]
\end{problem}}

%%%%%%%%%%%%%%%%%%%%%%

\latexProblemContent{
\ifVerboseLocation This is Derivative Compute Question 0004. \\ \fi
\begin{problem}

Compute the following derivative:

\input{Derivative-Compute-0004.HELP.tex}

\[\dfrac{d}{dx}\left(\frac{x^{2} + 3 \, x - 4}{x^{2} - 5 \, x + 4}\right)=\answer{-\frac{8 \, {\left(x^{2} - 2 \, x + 1\right)}}{{\left(x^{2} - 5 \, x + 4\right)}^{2}}}\]
\end{problem}}

%%%%%%%%%%%%%%%%%%%%%%

\latexProblemContent{
\ifVerboseLocation This is Derivative Compute Question 0004. \\ \fi
\begin{problem}

Compute the following derivative:

\input{Derivative-Compute-0004.HELP.tex}

\[\dfrac{d}{dx}\left(\frac{x^{3} - 10 \, x^{2} + 29 \, x - 20}{x + 4}\right)=\answer{\frac{2 \, {\left(x^{3} + x^{2} - 40 \, x + 68\right)}}{{\left(x + 4\right)}^{2}}}\]
\end{problem}}

%%%%%%%%%%%%%%%%%%%%%%

\latexProblemContent{
\ifVerboseLocation This is Derivative Compute Question 0004. \\ \fi
\begin{problem}

Compute the following derivative:

\input{Derivative-Compute-0004.HELP.tex}

\[\dfrac{d}{dx}\left(\frac{1}{x^{2} - 3 \, x + 2}\right)=\answer{-\frac{2 \, x - 3}{{\left(x^{2} - 3 \, x + 2\right)}^{2}}}\]
\end{problem}}

%%%%%%%%%%%%%%%%%%%%%%

\latexProblemContent{
\ifVerboseLocation This is Derivative Compute Question 0004. \\ \fi
\begin{problem}

Compute the following derivative:

\input{Derivative-Compute-0004.HELP.tex}

\[\dfrac{d}{dx}\left(\frac{x + 3}{x^{3} - 10 \, x^{2} + 29 \, x - 20}\right)=\answer{-\frac{2 \, x^{3} - x^{2} - 60 \, x + 107}{{\left(x^{3} - 10 \, x^{2} + 29 \, x - 20\right)}^{2}}}\]
\end{problem}}

%%%%%%%%%%%%%%%%%%%%%%

\latexProblemContent{
\ifVerboseLocation This is Derivative Compute Question 0004. \\ \fi
\begin{problem}

Compute the following derivative:

\input{Derivative-Compute-0004.HELP.tex}

\[\dfrac{d}{dx}\left(\frac{x^{2} - x - 20}{x^{3} + 5 \, x^{2} - 4 \, x - 20}\right)=\answer{-\frac{x^{4} - 2 \, x^{3} - 61 \, x^{2} - 160 \, x + 60}{{\left(x^{3} + 5 \, x^{2} - 4 \, x - 20\right)}^{2}}}\]
\end{problem}}

%%%%%%%%%%%%%%%%%%%%%%

\latexProblemContent{
\ifVerboseLocation This is Derivative Compute Question 0004. \\ \fi
\begin{problem}

Compute the following derivative:

\input{Derivative-Compute-0004.HELP.tex}

\[\dfrac{d}{dx}\left(\frac{x + 5}{x^{3} + 4 \, x^{2} - 4 \, x - 16}\right)=\answer{-\frac{2 \, x^{3} + 19 \, x^{2} + 40 \, x - 4}{{\left(x^{3} + 4 \, x^{2} - 4 \, x - 16\right)}^{2}}}\]
\end{problem}}

%%%%%%%%%%%%%%%%%%%%%%

\latexProblemContent{
\ifVerboseLocation This is Derivative Compute Question 0004. \\ \fi
\begin{problem}

Compute the following derivative:

\input{Derivative-Compute-0004.HELP.tex}

\[\dfrac{d}{dx}\left(\frac{x^{2} - 8 \, x + 15}{x^{2} + x - 12}\right)=\answer{\frac{9 \, {\left(x^{2} - 6 \, x + 9\right)}}{{\left(x^{2} + x - 12\right)}^{2}}}\]
\end{problem}}

%%%%%%%%%%%%%%%%%%%%%%

\latexProblemContent{
\ifVerboseLocation This is Derivative Compute Question 0004. \\ \fi
\begin{problem}

Compute the following derivative:

\input{Derivative-Compute-0004.HELP.tex}

\[\dfrac{d}{dx}\left(\frac{x^{3} + 3 \, x^{2} - 6 \, x - 8}{x^{2} - x - 12}\right)=\answer{\frac{x^{4} - 2 \, x^{3} - 33 \, x^{2} - 56 \, x + 64}{{\left(x^{2} - x - 12\right)}^{2}}}\]
\end{problem}}

%%%%%%%%%%%%%%%%%%%%%%

\latexProblemContent{
\ifVerboseLocation This is Derivative Compute Question 0004. \\ \fi
\begin{problem}

Compute the following derivative:

\input{Derivative-Compute-0004.HELP.tex}

\[\dfrac{d}{dx}\left(\frac{x^{2} - 5 \, x + 6}{x^{3} + 5 \, x^{2} + 3 \, x - 9}\right)=\answer{-\frac{x^{4} - 10 \, x^{3} - 10 \, x^{2} + 78 \, x - 27}{{\left(x^{3} + 5 \, x^{2} + 3 \, x - 9\right)}^{2}}}\]
\end{problem}}

%%%%%%%%%%%%%%%%%%%%%%

\latexProblemContent{
\ifVerboseLocation This is Derivative Compute Question 0004. \\ \fi
\begin{problem}

Compute the following derivative:

\input{Derivative-Compute-0004.HELP.tex}

\[\dfrac{d}{dx}\left(\frac{x^{2} + 3 \, x + 2}{x^{2} + 10 \, x + 25}\right)=\answer{\frac{7 \, x^{2} + 46 \, x + 55}{{\left(x^{2} + 10 \, x + 25\right)}^{2}}}\]
\end{problem}}

%%%%%%%%%%%%%%%%%%%%%%

\latexProblemContent{
\ifVerboseLocation This is Derivative Compute Question 0004. \\ \fi
\begin{problem}

Compute the following derivative:

\input{Derivative-Compute-0004.HELP.tex}

\[\dfrac{d}{dx}\left(\frac{x^{2} + 3 \, x + 2}{x^{2} - 8 \, x + 16}\right)=\answer{-\frac{11 \, x^{2} - 28 \, x - 64}{{\left(x^{2} - 8 \, x + 16\right)}^{2}}}\]
\end{problem}}

%%%%%%%%%%%%%%%%%%%%%%

\latexProblemContent{
\ifVerboseLocation This is Derivative Compute Question 0004. \\ \fi
\begin{problem}

Compute the following derivative:

\input{Derivative-Compute-0004.HELP.tex}

\[\dfrac{d}{dx}\left(\frac{1}{x^{3} - 10 \, x^{2} + 29 \, x - 20}\right)=\answer{-\frac{3 \, x^{2} - 20 \, x + 29}{{\left(x^{3} - 10 \, x^{2} + 29 \, x - 20\right)}^{2}}}\]
\end{problem}}

%%%%%%%%%%%%%%%%%%%%%%

\latexProblemContent{
\ifVerboseLocation This is Derivative Compute Question 0004. \\ \fi
\begin{problem}

Compute the following derivative:

\input{Derivative-Compute-0004.HELP.tex}

\[\dfrac{d}{dx}\left(\frac{1}{x^{2} - 5 \, x + 6}\right)=\answer{-\frac{2 \, x - 5}{{\left(x^{2} - 5 \, x + 6\right)}^{2}}}\]
\end{problem}}

%%%%%%%%%%%%%%%%%%%%%%

\latexProblemContent{
\ifVerboseLocation This is Derivative Compute Question 0004. \\ \fi
\begin{problem}

Compute the following derivative:

\input{Derivative-Compute-0004.HELP.tex}

\[\dfrac{d}{dx}\left(\frac{x^{2} + 3 \, x + 2}{x^{2} + 4 \, x + 4}\right)=\answer{\frac{1}{x^{2} + 4 \, x + 4}}\]
\end{problem}}

%%%%%%%%%%%%%%%%%%%%%%

\latexProblemContent{
\ifVerboseLocation This is Derivative Compute Question 0004. \\ \fi
\begin{problem}

Compute the following derivative:

\input{Derivative-Compute-0004.HELP.tex}

\[\dfrac{d}{dx}\left(\frac{x + 3}{x^{3} - 9 \, x^{2} + 24 \, x - 16}\right)=\answer{-\frac{2 \, {\left(x^{3} - 27 \, x + 44\right)}}{{\left(x^{3} - 9 \, x^{2} + 24 \, x - 16\right)}^{2}}}\]
\end{problem}}

%%%%%%%%%%%%%%%%%%%%%%

\latexProblemContent{
\ifVerboseLocation This is Derivative Compute Question 0004. \\ \fi
\begin{problem}

Compute the following derivative:

\input{Derivative-Compute-0004.HELP.tex}

\[\dfrac{d}{dx}\left(\frac{x^{2} - 7 \, x + 10}{x^{3} - 7 \, x - 6}\right)=\answer{-\frac{x^{4} - 14 \, x^{3} + 37 \, x^{2} + 12 \, x - 112}{{\left(x^{3} - 7 \, x - 6\right)}^{2}}}\]
\end{problem}}

%%%%%%%%%%%%%%%%%%%%%%

\latexProblemContent{
\ifVerboseLocation This is Derivative Compute Question 0004. \\ \fi
\begin{problem}

Compute the following derivative:

\input{Derivative-Compute-0004.HELP.tex}

\[\dfrac{d}{dx}\left(\frac{1}{x^{3} - x^{2} - 25 \, x + 25}\right)=\answer{-\frac{3 \, x^{2} - 2 \, x - 25}{{\left(x^{3} - x^{2} - 25 \, x + 25\right)}^{2}}}\]
\end{problem}}

%%%%%%%%%%%%%%%%%%%%%%

\latexProblemContent{
\ifVerboseLocation This is Derivative Compute Question 0004. \\ \fi
\begin{problem}

Compute the following derivative:

\input{Derivative-Compute-0004.HELP.tex}

\[\dfrac{d}{dx}\left(\frac{x^{2} - 8 \, x + 15}{x^{2} - 4 \, x + 4}\right)=\answer{\frac{2 \, {\left(2 \, x^{2} - 11 \, x + 14\right)}}{{\left(x^{2} - 4 \, x + 4\right)}^{2}}}\]
\end{problem}}

%%%%%%%%%%%%%%%%%%%%%%

\latexProblemContent{
\ifVerboseLocation This is Derivative Compute Question 0004. \\ \fi
\begin{problem}

Compute the following derivative:

\input{Derivative-Compute-0004.HELP.tex}

\[\dfrac{d}{dx}\left(\frac{1}{x^{2} + 4 \, x + 4}\right)=\answer{-\frac{2 \, {\left(x + 2\right)}}{{\left(x^{2} + 4 \, x + 4\right)}^{2}}}\]
\end{problem}}

%%%%%%%%%%%%%%%%%%%%%%

\latexProblemContent{
\ifVerboseLocation This is Derivative Compute Question 0004. \\ \fi
\begin{problem}

Compute the following derivative:

\input{Derivative-Compute-0004.HELP.tex}

\[\dfrac{d}{dx}\left(\frac{x^{3} - x^{2} - 14 \, x + 24}{x^{3} - 7 \, x^{2} + 7 \, x + 15}\right)=\answer{-\frac{6 \, {\left(x^{4} - 7 \, x^{3} + 22 \, x^{2} - 51 \, x + 63\right)}}{{\left(x^{3} - 7 \, x^{2} + 7 \, x + 15\right)}^{2}}}\]
\end{problem}}

%%%%%%%%%%%%%%%%%%%%%%

\latexProblemContent{
\ifVerboseLocation This is Derivative Compute Question 0004. \\ \fi
\begin{problem}

Compute the following derivative:

\input{Derivative-Compute-0004.HELP.tex}

\[\dfrac{d}{dx}\left(\frac{x - 5}{x^{3} - 7 \, x^{2} + 8 \, x + 16}\right)=\answer{-\frac{2 \, {\left(x^{3} - 11 \, x^{2} + 35 \, x - 28\right)}}{{\left(x^{3} - 7 \, x^{2} + 8 \, x + 16\right)}^{2}}}\]
\end{problem}}

%%%%%%%%%%%%%%%%%%%%%%

\latexProblemContent{
\ifVerboseLocation This is Derivative Compute Question 0004. \\ \fi
\begin{problem}

Compute the following derivative:

\input{Derivative-Compute-0004.HELP.tex}

\[\dfrac{d}{dx}\left(\frac{x + 5}{x^{2} - 7 \, x + 12}\right)=\answer{-\frac{x^{2} + 10 \, x - 47}{{\left(x^{2} - 7 \, x + 12\right)}^{2}}}\]
\end{problem}}

%%%%%%%%%%%%%%%%%%%%%%

\latexProblemContent{
\ifVerboseLocation This is Derivative Compute Question 0004. \\ \fi
\begin{problem}

Compute the following derivative:

\input{Derivative-Compute-0004.HELP.tex}

\[\dfrac{d}{dx}\left(\frac{x^{3} - x^{2} - x + 1}{x - 5}\right)=\answer{\frac{2 \, {\left(x^{3} - 8 \, x^{2} + 5 \, x + 2\right)}}{{\left(x - 5\right)}^{2}}}\]
\end{problem}}

%%%%%%%%%%%%%%%%%%%%%%

\latexProblemContent{
\ifVerboseLocation This is Derivative Compute Question 0004. \\ \fi
\begin{problem}

Compute the following derivative:

\input{Derivative-Compute-0004.HELP.tex}

\[\dfrac{d}{dx}\left(\frac{x - 3}{x^{3} + 2 \, x^{2} - 5 \, x - 6}\right)=\answer{-\frac{2 \, x^{3} - 7 \, x^{2} - 12 \, x + 21}{{\left(x^{3} + 2 \, x^{2} - 5 \, x - 6\right)}^{2}}}\]
\end{problem}}

%%%%%%%%%%%%%%%%%%%%%%

\latexProblemContent{
\ifVerboseLocation This is Derivative Compute Question 0004. \\ \fi
\begin{problem}

Compute the following derivative:

\input{Derivative-Compute-0004.HELP.tex}

\[\dfrac{d}{dx}\left(\frac{x^{3} - 4 \, x^{2} - 9 \, x + 36}{x + 4}\right)=\answer{\frac{2 \, {\left(x^{3} + 4 \, x^{2} - 16 \, x - 36\right)}}{{\left(x + 4\right)}^{2}}}\]
\end{problem}}

%%%%%%%%%%%%%%%%%%%%%%

\latexProblemContent{
\ifVerboseLocation This is Derivative Compute Question 0004. \\ \fi
\begin{problem}

Compute the following derivative:

\input{Derivative-Compute-0004.HELP.tex}

\[\dfrac{d}{dx}\left(\frac{x - 2}{x + 3}\right)=\answer{\frac{5}{{\left(x + 3\right)}^{2}}}\]
\end{problem}}

%%%%%%%%%%%%%%%%%%%%%%

\latexProblemContent{
\ifVerboseLocation This is Derivative Compute Question 0004. \\ \fi
\begin{problem}

Compute the following derivative:

\input{Derivative-Compute-0004.HELP.tex}

\[\dfrac{d}{dx}\left(\frac{x - 4}{x^{2} + 6 \, x + 5}\right)=\answer{-\frac{x^{2} - 8 \, x - 29}{{\left(x^{2} + 6 \, x + 5\right)}^{2}}}\]
\end{problem}}

%%%%%%%%%%%%%%%%%%%%%%

\latexProblemContent{
\ifVerboseLocation This is Derivative Compute Question 0004. \\ \fi
\begin{problem}

Compute the following derivative:

\input{Derivative-Compute-0004.HELP.tex}

\[\dfrac{d}{dx}\left(\frac{x^{2} + 9 \, x + 20}{x^{2} + 2 \, x + 1}\right)=\answer{-\frac{7 \, x^{2} + 38 \, x + 31}{{\left(x^{2} + 2 \, x + 1\right)}^{2}}}\]
\end{problem}}

%%%%%%%%%%%%%%%%%%%%%%

\latexProblemContent{
\ifVerboseLocation This is Derivative Compute Question 0004. \\ \fi
\begin{problem}

Compute the following derivative:

\input{Derivative-Compute-0004.HELP.tex}

\[\dfrac{d}{dx}\left(\frac{x - 3}{x^{3} - 2 \, x^{2} - 19 \, x + 20}\right)=\answer{-\frac{2 \, x^{3} - 11 \, x^{2} + 12 \, x + 37}{{\left(x^{3} - 2 \, x^{2} - 19 \, x + 20\right)}^{2}}}\]
\end{problem}}

%%%%%%%%%%%%%%%%%%%%%%

\latexProblemContent{
\ifVerboseLocation This is Derivative Compute Question 0004. \\ \fi
\begin{problem}

Compute the following derivative:

\input{Derivative-Compute-0004.HELP.tex}

\[\dfrac{d}{dx}\left(\frac{x^{3} + 8 \, x^{2} + 11 \, x - 20}{x^{2} + x - 12}\right)=\answer{\frac{x^{4} + 2 \, x^{3} - 39 \, x^{2} - 152 \, x - 112}{{\left(x^{2} + x - 12\right)}^{2}}}\]
\end{problem}}

%%%%%%%%%%%%%%%%%%%%%%

\latexProblemContent{
\ifVerboseLocation This is Derivative Compute Question 0004. \\ \fi
\begin{problem}

Compute the following derivative:

\input{Derivative-Compute-0004.HELP.tex}

\[\dfrac{d}{dx}\left(\frac{x^{3} + 8 \, x^{2} + 11 \, x - 20}{x + 2}\right)=\answer{\frac{2 \, {\left(x^{3} + 7 \, x^{2} + 16 \, x + 21\right)}}{{\left(x + 2\right)}^{2}}}\]
\end{problem}}

%%%%%%%%%%%%%%%%%%%%%%

\latexProblemContent{
\ifVerboseLocation This is Derivative Compute Question 0004. \\ \fi
\begin{problem}

Compute the following derivative:

\input{Derivative-Compute-0004.HELP.tex}

\[\dfrac{d}{dx}\left(\frac{x^{2} - 4 \, x + 3}{x^{2} - 3 \, x - 4}\right)=\answer{\frac{x^{2} - 14 \, x + 25}{{\left(x^{2} - 3 \, x - 4\right)}^{2}}}\]
\end{problem}}

%%%%%%%%%%%%%%%%%%%%%%

\latexProblemContent{
\ifVerboseLocation This is Derivative Compute Question 0004. \\ \fi
\begin{problem}

Compute the following derivative:

\input{Derivative-Compute-0004.HELP.tex}

\[\dfrac{d}{dx}\left(\frac{x^{3} + 3 \, x^{2} - 6 \, x - 8}{x^{3} + x^{2} - 4 \, x - 4}\right)=\answer{-\frac{2 \, {\left(x^{4} - 2 \, x^{3} - 3 \, x^{2} + 4 \, x + 4\right)}}{{\left(x^{3} + x^{2} - 4 \, x - 4\right)}^{2}}}\]
\end{problem}}

%%%%%%%%%%%%%%%%%%%%%%

\latexProblemContent{
\ifVerboseLocation This is Derivative Compute Question 0004. \\ \fi
\begin{problem}

Compute the following derivative:

\input{Derivative-Compute-0004.HELP.tex}

\[\dfrac{d}{dx}\left(\frac{x^{2} + 4 \, x - 5}{x - 4}\right)=\answer{\frac{x^{2} - 8 \, x - 11}{{\left(x - 4\right)}^{2}}}\]
\end{problem}}

%%%%%%%%%%%%%%%%%%%%%%

\latexProblemContent{
\ifVerboseLocation This is Derivative Compute Question 0004. \\ \fi
\begin{problem}

Compute the following derivative:

\input{Derivative-Compute-0004.HELP.tex}

\[\dfrac{d}{dx}\left(\frac{x + 4}{x + 1}\right)=\answer{-\frac{3}{{\left(x + 1\right)}^{2}}}\]
\end{problem}}

%%%%%%%%%%%%%%%%%%%%%%

\latexProblemContent{
\ifVerboseLocation This is Derivative Compute Question 0004. \\ \fi
\begin{problem}

Compute the following derivative:

\input{Derivative-Compute-0004.HELP.tex}

\[\dfrac{d}{dx}\left(\frac{x^{2} + x - 2}{x + 3}\right)=\answer{\frac{x^{2} + 6 \, x + 5}{{\left(x + 3\right)}^{2}}}\]
\end{problem}}

%%%%%%%%%%%%%%%%%%%%%%

\latexProblemContent{
\ifVerboseLocation This is Derivative Compute Question 0004. \\ \fi
\begin{problem}

Compute the following derivative:

\input{Derivative-Compute-0004.HELP.tex}

\[\dfrac{d}{dx}\left(\frac{x^{2} - 9 \, x + 20}{x^{3} + 8 \, x^{2} + 11 \, x - 20}\right)=\answer{-\frac{x^{4} - 18 \, x^{3} - 23 \, x^{2} + 360 \, x + 40}{{\left(x^{3} + 8 \, x^{2} + 11 \, x - 20\right)}^{2}}}\]
\end{problem}}

%%%%%%%%%%%%%%%%%%%%%%

\latexProblemContent{
\ifVerboseLocation This is Derivative Compute Question 0004. \\ \fi
\begin{problem}

Compute the following derivative:

\input{Derivative-Compute-0004.HELP.tex}

\[\dfrac{d}{dx}\left(\frac{1}{x - 1}\right)=\answer{-\frac{1}{{\left(x - 1\right)}^{2}}}\]
\end{problem}}

%%%%%%%%%%%%%%%%%%%%%%

\latexProblemContent{
\ifVerboseLocation This is Derivative Compute Question 0004. \\ \fi
\begin{problem}

Compute the following derivative:

\input{Derivative-Compute-0004.HELP.tex}

\[\dfrac{d}{dx}\left(\frac{x - 4}{x^{2} - 8 \, x + 15}\right)=\answer{-\frac{x^{2} - 8 \, x + 17}{{\left(x^{2} - 8 \, x + 15\right)}^{2}}}\]
\end{problem}}

%%%%%%%%%%%%%%%%%%%%%%

\latexProblemContent{
\ifVerboseLocation This is Derivative Compute Question 0004. \\ \fi
\begin{problem}

Compute the following derivative:

\input{Derivative-Compute-0004.HELP.tex}

\[\dfrac{d}{dx}\left(\frac{x^{2} + 7 \, x + 12}{x^{3} + 5 \, x^{2} - 4 \, x - 20}\right)=\answer{-\frac{x^{4} + 14 \, x^{3} + 75 \, x^{2} + 160 \, x + 92}{{\left(x^{3} + 5 \, x^{2} - 4 \, x - 20\right)}^{2}}}\]
\end{problem}}

%%%%%%%%%%%%%%%%%%%%%%

\latexProblemContent{
\ifVerboseLocation This is Derivative Compute Question 0004. \\ \fi
\begin{problem}

Compute the following derivative:

\input{Derivative-Compute-0004.HELP.tex}

\[\dfrac{d}{dx}\left(\frac{x^{3} - 5 \, x^{2} - 9 \, x + 45}{x + 5}\right)=\answer{\frac{2 \, {\left(x^{3} + 5 \, x^{2} - 25 \, x - 45\right)}}{{\left(x + 5\right)}^{2}}}\]
\end{problem}}

%%%%%%%%%%%%%%%%%%%%%%

\latexProblemContent{
\ifVerboseLocation This is Derivative Compute Question 0004. \\ \fi
\begin{problem}

Compute the following derivative:

\input{Derivative-Compute-0004.HELP.tex}

\[\dfrac{d}{dx}\left(\frac{x^{3} + 6 \, x^{2} - x - 30}{x^{2} + 6 \, x + 9}\right)=\answer{\frac{x^{4} + 12 \, x^{3} + 64 \, x^{2} + 168 \, x + 171}{{\left(x^{2} + 6 \, x + 9\right)}^{2}}}\]
\end{problem}}

%%%%%%%%%%%%%%%%%%%%%%

\latexProblemContent{
\ifVerboseLocation This is Derivative Compute Question 0004. \\ \fi
\begin{problem}

Compute the following derivative:

\input{Derivative-Compute-0004.HELP.tex}

\[\dfrac{d}{dx}\left(\frac{1}{x^{3} - x^{2} - x + 1}\right)=\answer{-\frac{3 \, x^{2} - 2 \, x - 1}{{\left(x^{3} - x^{2} - x + 1\right)}^{2}}}\]
\end{problem}}

%%%%%%%%%%%%%%%%%%%%%%

\latexProblemContent{
\ifVerboseLocation This is Derivative Compute Question 0004. \\ \fi
\begin{problem}

Compute the following derivative:

\input{Derivative-Compute-0004.HELP.tex}

\[\dfrac{d}{dx}\left(\frac{x^{2} + 2 \, x - 8}{x^{3} - 2 \, x^{2} - 19 \, x + 20}\right)=\answer{-\frac{x^{4} + 4 \, x^{3} - 9 \, x^{2} - 8 \, x + 112}{{\left(x^{3} - 2 \, x^{2} - 19 \, x + 20\right)}^{2}}}\]
\end{problem}}

%%%%%%%%%%%%%%%%%%%%%%

\latexProblemContent{
\ifVerboseLocation This is Derivative Compute Question 0004. \\ \fi
\begin{problem}

Compute the following derivative:

\input{Derivative-Compute-0004.HELP.tex}

\[\dfrac{d}{dx}\left(\frac{x^{3} + 3 \, x^{2} - 4 \, x - 12}{x^{2} - 5 \, x + 6}\right)=\answer{\frac{x^{4} - 10 \, x^{3} + 7 \, x^{2} + 60 \, x - 84}{{\left(x^{2} - 5 \, x + 6\right)}^{2}}}\]
\end{problem}}

%%%%%%%%%%%%%%%%%%%%%%

\latexProblemContent{
\ifVerboseLocation This is Derivative Compute Question 0004. \\ \fi
\begin{problem}

Compute the following derivative:

\input{Derivative-Compute-0004.HELP.tex}

\[\dfrac{d}{dx}\left(\frac{x^{3} - 8 \, x^{2} + 11 \, x + 20}{x^{3} + 8 \, x^{2} + 17 \, x + 10}\right)=\answer{\frac{2 \, {\left(8 \, x^{4} + 6 \, x^{3} - 127 \, x^{2} - 240 \, x - 115\right)}}{{\left(x^{3} + 8 \, x^{2} + 17 \, x + 10\right)}^{2}}}\]
\end{problem}}

%%%%%%%%%%%%%%%%%%%%%%

\latexProblemContent{
\ifVerboseLocation This is Derivative Compute Question 0004. \\ \fi
\begin{problem}

Compute the following derivative:

\input{Derivative-Compute-0004.HELP.tex}

\[\dfrac{d}{dx}\left(\frac{x^{3} + 2 \, x^{2} - 19 \, x - 20}{x^{3} + 4 \, x^{2} - 17 \, x - 60}\right)=\answer{\frac{2 \, {\left(x^{4} + 2 \, x^{3} - 39 \, x^{2} - 40 \, x + 400\right)}}{{\left(x^{3} + 4 \, x^{2} - 17 \, x - 60\right)}^{2}}}\]
\end{problem}}

%%%%%%%%%%%%%%%%%%%%%%

\latexProblemContent{
\ifVerboseLocation This is Derivative Compute Question 0004. \\ \fi
\begin{problem}

Compute the following derivative:

\input{Derivative-Compute-0004.HELP.tex}

\[\dfrac{d}{dx}\left(\frac{x^{3} - 2 \, x^{2} - 23 \, x + 60}{x^{2} + 3 \, x - 4}\right)=\answer{\frac{x^{4} + 6 \, x^{3} + 5 \, x^{2} - 104 \, x - 88}{{\left(x^{2} + 3 \, x - 4\right)}^{2}}}\]
\end{problem}}

%%%%%%%%%%%%%%%%%%%%%%

\latexProblemContent{
\ifVerboseLocation This is Derivative Compute Question 0004. \\ \fi
\begin{problem}

Compute the following derivative:

\input{Derivative-Compute-0004.HELP.tex}

\[\dfrac{d}{dx}\left(\frac{x - 3}{x^{2} - 1}\right)=\answer{-\frac{x^{2} - 6 \, x + 1}{{\left(x^{2} - 1\right)}^{2}}}\]
\end{problem}}

%%%%%%%%%%%%%%%%%%%%%%

\latexProblemContent{
\ifVerboseLocation This is Derivative Compute Question 0004. \\ \fi
\begin{problem}

Compute the following derivative:

\input{Derivative-Compute-0004.HELP.tex}

\[\dfrac{d}{dx}\left(\frac{1}{x^{3} + 5 \, x^{2} - 4 \, x - 20}\right)=\answer{-\frac{3 \, x^{2} + 10 \, x - 4}{{\left(x^{3} + 5 \, x^{2} - 4 \, x - 20\right)}^{2}}}\]
\end{problem}}

%%%%%%%%%%%%%%%%%%%%%%

\latexProblemContent{
\ifVerboseLocation This is Derivative Compute Question 0004. \\ \fi
\begin{problem}

Compute the following derivative:

\input{Derivative-Compute-0004.HELP.tex}

\[\dfrac{d}{dx}\left(\frac{x^{3} - 4 \, x^{2} - 7 \, x + 10}{x^{2} - 4 \, x + 4}\right)=\answer{\frac{x^{4} - 8 \, x^{3} + 35 \, x^{2} - 52 \, x + 12}{{\left(x^{2} - 4 \, x + 4\right)}^{2}}}\]
\end{problem}}

%%%%%%%%%%%%%%%%%%%%%%

\latexProblemContent{
\ifVerboseLocation This is Derivative Compute Question 0004. \\ \fi
\begin{problem}

Compute the following derivative:

\input{Derivative-Compute-0004.HELP.tex}

\[\dfrac{d}{dx}\left(\frac{x^{2} - 6 \, x + 5}{x^{3} - 3 \, x^{2} - 18 \, x + 40}\right)=\answer{-\frac{x^{4} - 12 \, x^{3} + 51 \, x^{2} - 110 \, x + 150}{{\left(x^{3} - 3 \, x^{2} - 18 \, x + 40\right)}^{2}}}\]
\end{problem}}

%%%%%%%%%%%%%%%%%%%%%%

\latexProblemContent{
\ifVerboseLocation This is Derivative Compute Question 0004. \\ \fi
\begin{problem}

Compute the following derivative:

\input{Derivative-Compute-0004.HELP.tex}

\[\dfrac{d}{dx}\left(\frac{x - 1}{x^{2} - 25}\right)=\answer{-\frac{x^{2} - 2 \, x + 25}{{\left(x^{2} - 25\right)}^{2}}}\]
\end{problem}}

%%%%%%%%%%%%%%%%%%%%%%

\latexProblemContent{
\ifVerboseLocation This is Derivative Compute Question 0004. \\ \fi
\begin{problem}

Compute the following derivative:

\input{Derivative-Compute-0004.HELP.tex}

\[\dfrac{d}{dx}\left(\frac{x^{2} + 5 \, x + 6}{x + 1}\right)=\answer{\frac{x^{2} + 2 \, x - 1}{{\left(x + 1\right)}^{2}}}\]
\end{problem}}

%%%%%%%%%%%%%%%%%%%%%%

\latexProblemContent{
\ifVerboseLocation This is Derivative Compute Question 0004. \\ \fi
\begin{problem}

Compute the following derivative:

\input{Derivative-Compute-0004.HELP.tex}

\[\dfrac{d}{dx}\left(\frac{x^{3} - 13 \, x + 12}{x - 4}\right)=\answer{\frac{2 \, {\left(x^{3} - 6 \, x^{2} + 20\right)}}{{\left(x - 4\right)}^{2}}}\]
\end{problem}}

%%%%%%%%%%%%%%%%%%%%%%

\latexProblemContent{
\ifVerboseLocation This is Derivative Compute Question 0004. \\ \fi
\begin{problem}

Compute the following derivative:

\input{Derivative-Compute-0004.HELP.tex}

\[\dfrac{d}{dx}\left(\frac{x^{2} - 2 \, x - 8}{x^{2} + 6 \, x + 5}\right)=\answer{\frac{2 \, {\left(4 \, x^{2} + 13 \, x + 19\right)}}{{\left(x^{2} + 6 \, x + 5\right)}^{2}}}\]
\end{problem}}

%%%%%%%%%%%%%%%%%%%%%%

\latexProblemContent{
\ifVerboseLocation This is Derivative Compute Question 0004. \\ \fi
\begin{problem}

Compute the following derivative:

\input{Derivative-Compute-0004.HELP.tex}

\[\dfrac{d}{dx}\left(\frac{x^{3} + 2 \, x^{2} - 23 \, x - 60}{x^{3} - 13 \, x^{2} + 56 \, x - 80}\right)=\answer{-\frac{15 \, x^{4} - 158 \, x^{3} + 247 \, x^{2} + 1880 \, x - 5200}{{\left(x^{3} - 13 \, x^{2} + 56 \, x - 80\right)}^{2}}}\]
\end{problem}}

%%%%%%%%%%%%%%%%%%%%%%

\latexProblemContent{
\ifVerboseLocation This is Derivative Compute Question 0004. \\ \fi
\begin{problem}

Compute the following derivative:

\input{Derivative-Compute-0004.HELP.tex}

\[\dfrac{d}{dx}\left(\frac{x^{2} + x - 20}{x^{2} - 1}\right)=\answer{-\frac{x^{2} - 38 \, x + 1}{{\left(x^{2} - 1\right)}^{2}}}\]
\end{problem}}

%%%%%%%%%%%%%%%%%%%%%%

\latexProblemContent{
\ifVerboseLocation This is Derivative Compute Question 0004. \\ \fi
\begin{problem}

Compute the following derivative:

\input{Derivative-Compute-0004.HELP.tex}

\[\dfrac{d}{dx}\left(\frac{x^{2} + x - 12}{x^{2} - 2 \, x - 15}\right)=\answer{-\frac{3 \, {\left(x^{2} + 2 \, x + 13\right)}}{{\left(x^{2} - 2 \, x - 15\right)}^{2}}}\]
\end{problem}}

%%%%%%%%%%%%%%%%%%%%%%

\latexProblemContent{
\ifVerboseLocation This is Derivative Compute Question 0004. \\ \fi
\begin{problem}

Compute the following derivative:

\input{Derivative-Compute-0004.HELP.tex}

\[\dfrac{d}{dx}\left(\frac{x^{2} + 8 \, x + 15}{x + 3}\right)=\answer{\frac{x^{2} + 6 \, x + 9}{{\left(x + 3\right)}^{2}}}\]
\end{problem}}

%%%%%%%%%%%%%%%%%%%%%%

\latexProblemContent{
\ifVerboseLocation This is Derivative Compute Question 0004. \\ \fi
\begin{problem}

Compute the following derivative:

\input{Derivative-Compute-0004.HELP.tex}

\[\dfrac{d}{dx}\left(\frac{x^{2} + 2 \, x - 3}{x^{3} - x^{2} - 16 \, x + 16}\right)=\answer{-\frac{x^{4} + 4 \, x^{3} + 5 \, x^{2} - 26 \, x + 16}{{\left(x^{3} - x^{2} - 16 \, x + 16\right)}^{2}}}\]
\end{problem}}

%%%%%%%%%%%%%%%%%%%%%%

\latexProblemContent{
\ifVerboseLocation This is Derivative Compute Question 0004. \\ \fi
\begin{problem}

Compute the following derivative:

\input{Derivative-Compute-0004.HELP.tex}

\[\dfrac{d}{dx}\left(\frac{x^{2} + 7 \, x + 12}{x - 3}\right)=\answer{\frac{x^{2} - 6 \, x - 33}{{\left(x - 3\right)}^{2}}}\]
\end{problem}}

%%%%%%%%%%%%%%%%%%%%%%

\latexProblemContent{
\ifVerboseLocation This is Derivative Compute Question 0004. \\ \fi
\begin{problem}

Compute the following derivative:

\input{Derivative-Compute-0004.HELP.tex}

\[\dfrac{d}{dx}\left(\frac{x + 3}{x^{2} - 8 \, x + 15}\right)=\answer{-\frac{x^{2} + 6 \, x - 39}{{\left(x^{2} - 8 \, x + 15\right)}^{2}}}\]
\end{problem}}

%%%%%%%%%%%%%%%%%%%%%%

\latexProblemContent{
\ifVerboseLocation This is Derivative Compute Question 0004. \\ \fi
\begin{problem}

Compute the following derivative:

\input{Derivative-Compute-0004.HELP.tex}

\[\dfrac{d}{dx}\left(\frac{1}{x + 2}\right)=\answer{-\frac{1}{{\left(x + 2\right)}^{2}}}\]
\end{problem}}

%%%%%%%%%%%%%%%%%%%%%%

\latexProblemContent{
\ifVerboseLocation This is Derivative Compute Question 0004. \\ \fi
\begin{problem}

Compute the following derivative:

\input{Derivative-Compute-0004.HELP.tex}

\[\dfrac{d}{dx}\left(\frac{x^{2} + 8 \, x + 15}{x^{2} - x - 6}\right)=\answer{-\frac{3 \, {\left(3 \, x^{2} + 14 \, x + 11\right)}}{{\left(x^{2} - x - 6\right)}^{2}}}\]
\end{problem}}

%%%%%%%%%%%%%%%%%%%%%%

\latexProblemContent{
\ifVerboseLocation This is Derivative Compute Question 0004. \\ \fi
\begin{problem}

Compute the following derivative:

\input{Derivative-Compute-0004.HELP.tex}

\[\dfrac{d}{dx}\left(\frac{x - 4}{x - 1}\right)=\answer{\frac{3}{{\left(x - 1\right)}^{2}}}\]
\end{problem}}

%%%%%%%%%%%%%%%%%%%%%%

\latexProblemContent{
\ifVerboseLocation This is Derivative Compute Question 0004. \\ \fi
\begin{problem}

Compute the following derivative:

\input{Derivative-Compute-0004.HELP.tex}

\[\dfrac{d}{dx}\left(\frac{x^{3} + 3 \, x^{2} - 25 \, x - 75}{x^{2} - x - 2}\right)=\answer{\frac{x^{4} - 2 \, x^{3} + 16 \, x^{2} + 138 \, x - 25}{{\left(x^{2} - x - 2\right)}^{2}}}\]
\end{problem}}

%%%%%%%%%%%%%%%%%%%%%%

\latexProblemContent{
\ifVerboseLocation This is Derivative Compute Question 0004. \\ \fi
\begin{problem}

Compute the following derivative:

\input{Derivative-Compute-0004.HELP.tex}

\[\dfrac{d}{dx}\left(\frac{x^{3} + 4 \, x^{2} - 17 \, x - 60}{x^{3} + 5 \, x^{2} - 16 \, x - 80}\right)=\answer{\frac{x^{4} + 2 \, x^{3} - 39 \, x^{2} - 40 \, x + 400}{{\left(x^{3} + 5 \, x^{2} - 16 \, x - 80\right)}^{2}}}\]
\end{problem}}

%%%%%%%%%%%%%%%%%%%%%%

\latexProblemContent{
\ifVerboseLocation This is Derivative Compute Question 0004. \\ \fi
\begin{problem}

Compute the following derivative:

\input{Derivative-Compute-0004.HELP.tex}

\[\dfrac{d}{dx}\left(\frac{x - 5}{x^{2} + 3 \, x + 2}\right)=\answer{-\frac{x^{2} - 10 \, x - 17}{{\left(x^{2} + 3 \, x + 2\right)}^{2}}}\]
\end{problem}}

%%%%%%%%%%%%%%%%%%%%%%

\latexProblemContent{
\ifVerboseLocation This is Derivative Compute Question 0004. \\ \fi
\begin{problem}

Compute the following derivative:

\input{Derivative-Compute-0004.HELP.tex}

\[\dfrac{d}{dx}\left(\frac{x^{3} - 5 \, x^{2} - x + 5}{x + 5}\right)=\answer{\frac{2 \, {\left(x^{3} + 5 \, x^{2} - 25 \, x - 5\right)}}{{\left(x + 5\right)}^{2}}}\]
\end{problem}}

%%%%%%%%%%%%%%%%%%%%%%

\latexProblemContent{
\ifVerboseLocation This is Derivative Compute Question 0004. \\ \fi
\begin{problem}

Compute the following derivative:

\input{Derivative-Compute-0004.HELP.tex}

\[\dfrac{d}{dx}\left(\frac{x^{3} - 4 \, x^{2} + x + 6}{x + 5}\right)=\answer{\frac{2 \, x^{3} + 11 \, x^{2} - 40 \, x - 1}{{\left(x + 5\right)}^{2}}}\]
\end{problem}}

%%%%%%%%%%%%%%%%%%%%%%

\latexProblemContent{
\ifVerboseLocation This is Derivative Compute Question 0004. \\ \fi
\begin{problem}

Compute the following derivative:

\input{Derivative-Compute-0004.HELP.tex}

\[\dfrac{d}{dx}\left(\frac{x^{3} - 12 \, x^{2} + 47 \, x - 60}{x - 3}\right)=\answer{\frac{2 \, x^{3} - 21 \, x^{2} + 72 \, x - 81}{{\left(x - 3\right)}^{2}}}\]
\end{problem}}

%%%%%%%%%%%%%%%%%%%%%%

\latexProblemContent{
\ifVerboseLocation This is Derivative Compute Question 0004. \\ \fi
\begin{problem}

Compute the following derivative:

\input{Derivative-Compute-0004.HELP.tex}

\[\dfrac{d}{dx}\left(\frac{1}{x^{2} - 4 \, x + 4}\right)=\answer{-\frac{2 \, {\left(x - 2\right)}}{{\left(x^{2} - 4 \, x + 4\right)}^{2}}}\]
\end{problem}}

%%%%%%%%%%%%%%%%%%%%%%

\latexProblemContent{
\ifVerboseLocation This is Derivative Compute Question 0004. \\ \fi
\begin{problem}

Compute the following derivative:

\input{Derivative-Compute-0004.HELP.tex}

\[\dfrac{d}{dx}\left(\frac{1}{x^{3} + 4 \, x^{2} - 7 \, x - 10}\right)=\answer{-\frac{3 \, x^{2} + 8 \, x - 7}{{\left(x^{3} + 4 \, x^{2} - 7 \, x - 10\right)}^{2}}}\]
\end{problem}}

%%%%%%%%%%%%%%%%%%%%%%

\latexProblemContent{
\ifVerboseLocation This is Derivative Compute Question 0004. \\ \fi
\begin{problem}

Compute the following derivative:

\input{Derivative-Compute-0004.HELP.tex}

\[\dfrac{d}{dx}\left(\frac{x^{3} - 4 \, x^{2} - 7 \, x + 10}{x + 4}\right)=\answer{\frac{2 \, {\left(x^{3} + 4 \, x^{2} - 16 \, x - 19\right)}}{{\left(x + 4\right)}^{2}}}\]
\end{problem}}

%%%%%%%%%%%%%%%%%%%%%%

\latexProblemContent{
\ifVerboseLocation This is Derivative Compute Question 0004. \\ \fi
\begin{problem}

Compute the following derivative:

\input{Derivative-Compute-0004.HELP.tex}

\[\dfrac{d}{dx}\left(\frac{x + 2}{x^{3} + x^{2} - 17 \, x + 15}\right)=\answer{-\frac{2 \, x^{3} + 7 \, x^{2} + 4 \, x - 49}{{\left(x^{3} + x^{2} - 17 \, x + 15\right)}^{2}}}\]
\end{problem}}

%%%%%%%%%%%%%%%%%%%%%%

\latexProblemContent{
\ifVerboseLocation This is Derivative Compute Question 0004. \\ \fi
\begin{problem}

Compute the following derivative:

\input{Derivative-Compute-0004.HELP.tex}

\[\dfrac{d}{dx}\left(\frac{x^{2} + 2 \, x - 3}{x^{2} - 8 \, x + 15}\right)=\answer{-\frac{2 \, {\left(5 \, x^{2} - 18 \, x - 3\right)}}{{\left(x^{2} - 8 \, x + 15\right)}^{2}}}\]
\end{problem}}

%%%%%%%%%%%%%%%%%%%%%%

\latexProblemContent{
\ifVerboseLocation This is Derivative Compute Question 0004. \\ \fi
\begin{problem}

Compute the following derivative:

\input{Derivative-Compute-0004.HELP.tex}

\[\dfrac{d}{dx}\left(\frac{1}{x^{3} - 7 \, x^{2} + 15 \, x - 9}\right)=\answer{-\frac{3 \, x^{2} - 14 \, x + 15}{{\left(x^{3} - 7 \, x^{2} + 15 \, x - 9\right)}^{2}}}\]
\end{problem}}

%%%%%%%%%%%%%%%%%%%%%%

\latexProblemContent{
\ifVerboseLocation This is Derivative Compute Question 0004. \\ \fi
\begin{problem}

Compute the following derivative:

\input{Derivative-Compute-0004.HELP.tex}

\[\dfrac{d}{dx}\left(\frac{x^{2} - 9}{x - 4}\right)=\answer{\frac{x^{2} - 8 \, x + 9}{{\left(x - 4\right)}^{2}}}\]
\end{problem}}

%%%%%%%%%%%%%%%%%%%%%%

\latexProblemContent{
\ifVerboseLocation This is Derivative Compute Question 0004. \\ \fi
\begin{problem}

Compute the following derivative:

\input{Derivative-Compute-0004.HELP.tex}

\[\dfrac{d}{dx}\left(\frac{x^{3} + 7 \, x^{2} + 7 \, x - 15}{x^{3} + 3 \, x^{2} - 24 \, x - 80}\right)=\answer{-\frac{2 \, {\left(2 \, x^{4} + 31 \, x^{3} + 192 \, x^{2} + 515 \, x + 460\right)}}{{\left(x^{3} + 3 \, x^{2} - 24 \, x - 80\right)}^{2}}}\]
\end{problem}}

%%%%%%%%%%%%%%%%%%%%%%

\latexProblemContent{
\ifVerboseLocation This is Derivative Compute Question 0004. \\ \fi
\begin{problem}

Compute the following derivative:

\input{Derivative-Compute-0004.HELP.tex}

\[\dfrac{d}{dx}\left(\frac{1}{x^{3} - 5 \, x^{2} - 2 \, x + 24}\right)=\answer{-\frac{3 \, x^{2} - 10 \, x - 2}{{\left(x^{3} - 5 \, x^{2} - 2 \, x + 24\right)}^{2}}}\]
\end{problem}}

%%%%%%%%%%%%%%%%%%%%%%

\latexProblemContent{
\ifVerboseLocation This is Derivative Compute Question 0004. \\ \fi
\begin{problem}

Compute the following derivative:

\input{Derivative-Compute-0004.HELP.tex}

\[\dfrac{d}{dx}\left(\frac{x + 5}{x^{2} + x - 6}\right)=\answer{-\frac{x^{2} + 10 \, x + 11}{{\left(x^{2} + x - 6\right)}^{2}}}\]
\end{problem}}

%%%%%%%%%%%%%%%%%%%%%%

\latexProblemContent{
\ifVerboseLocation This is Derivative Compute Question 0004. \\ \fi
\begin{problem}

Compute the following derivative:

\input{Derivative-Compute-0004.HELP.tex}

\[\dfrac{d}{dx}\left(\frac{x^{3} - 3 \, x^{2} - 6 \, x + 8}{x^{3} - 4 \, x^{2} + x + 6}\right)=\answer{-\frac{x^{4} - 14 \, x^{3} + 33 \, x^{2} - 28 \, x + 44}{{\left(x^{3} - 4 \, x^{2} + x + 6\right)}^{2}}}\]
\end{problem}}

%%%%%%%%%%%%%%%%%%%%%%

\latexProblemContent{
\ifVerboseLocation This is Derivative Compute Question 0004. \\ \fi
\begin{problem}

Compute the following derivative:

\input{Derivative-Compute-0004.HELP.tex}

\[\dfrac{d}{dx}\left(\frac{x^{3} - 3 \, x^{2} - 10 \, x + 24}{x + 5}\right)=\answer{\frac{2 \, {\left(x^{3} + 6 \, x^{2} - 15 \, x - 37\right)}}{{\left(x + 5\right)}^{2}}}\]
\end{problem}}

%%%%%%%%%%%%%%%%%%%%%%

\latexProblemContent{
\ifVerboseLocation This is Derivative Compute Question 0004. \\ \fi
\begin{problem}

Compute the following derivative:

\input{Derivative-Compute-0004.HELP.tex}

\[\dfrac{d}{dx}\left(\frac{x^{2} - 25}{x^{3} - x^{2} - 14 \, x + 24}\right)=\answer{-\frac{x^{4} - 61 \, x^{2} + 2 \, x + 350}{{\left(x^{3} - x^{2} - 14 \, x + 24\right)}^{2}}}\]
\end{problem}}

%%%%%%%%%%%%%%%%%%%%%%

\latexProblemContent{
\ifVerboseLocation This is Derivative Compute Question 0004. \\ \fi
\begin{problem}

Compute the following derivative:

\input{Derivative-Compute-0004.HELP.tex}

\[\dfrac{d}{dx}\left(\frac{1}{x^{2} - 3 \, x - 4}\right)=\answer{-\frac{2 \, x - 3}{{\left(x^{2} - 3 \, x - 4\right)}^{2}}}\]
\end{problem}}

%%%%%%%%%%%%%%%%%%%%%%

\latexProblemContent{
\ifVerboseLocation This is Derivative Compute Question 0004. \\ \fi
\begin{problem}

Compute the following derivative:

\input{Derivative-Compute-0004.HELP.tex}

\[\dfrac{d}{dx}\left(\frac{x - 1}{x + 5}\right)=\answer{\frac{6}{{\left(x + 5\right)}^{2}}}\]
\end{problem}}

%%%%%%%%%%%%%%%%%%%%%%

\latexProblemContent{
\ifVerboseLocation This is Derivative Compute Question 0004. \\ \fi
\begin{problem}

Compute the following derivative:

\input{Derivative-Compute-0004.HELP.tex}

\[\dfrac{d}{dx}\left(\frac{1}{x^{3} + 3 \, x^{2} - x - 3}\right)=\answer{-\frac{3 \, x^{2} + 6 \, x - 1}{{\left(x^{3} + 3 \, x^{2} - x - 3\right)}^{2}}}\]
\end{problem}}

%%%%%%%%%%%%%%%%%%%%%%

\latexProblemContent{
\ifVerboseLocation This is Derivative Compute Question 0004. \\ \fi
\begin{problem}

Compute the following derivative:

\input{Derivative-Compute-0004.HELP.tex}

\[\dfrac{d}{dx}\left(\frac{1}{x^{3} + 2 \, x^{2} - 4 \, x - 8}\right)=\answer{-\frac{3 \, x^{2} + 4 \, x - 4}{{\left(x^{3} + 2 \, x^{2} - 4 \, x - 8\right)}^{2}}}\]
\end{problem}}

%%%%%%%%%%%%%%%%%%%%%%

\latexProblemContent{
\ifVerboseLocation This is Derivative Compute Question 0004. \\ \fi
\begin{problem}

Compute the following derivative:

\input{Derivative-Compute-0004.HELP.tex}

\[\dfrac{d}{dx}\left(\frac{x^{3} + 2 \, x^{2} - 16 \, x - 32}{x^{2} - x - 2}\right)=\answer{\frac{x^{4} - 2 \, x^{3} + 8 \, x^{2} + 56 \, x}{{\left(x^{2} - x - 2\right)}^{2}}}\]
\end{problem}}

%%%%%%%%%%%%%%%%%%%%%%

\latexProblemContent{
\ifVerboseLocation This is Derivative Compute Question 0004. \\ \fi
\begin{problem}

Compute the following derivative:

\input{Derivative-Compute-0004.HELP.tex}

\[\dfrac{d}{dx}\left(\frac{x - 2}{x^{3} - 19 \, x + 30}\right)=\answer{-\frac{2 \, {\left(x^{3} - 3 \, x^{2} + 4\right)}}{{\left(x^{3} - 19 \, x + 30\right)}^{2}}}\]
\end{problem}}

%%%%%%%%%%%%%%%%%%%%%%

\latexProblemContent{
\ifVerboseLocation This is Derivative Compute Question 0004. \\ \fi
\begin{problem}

Compute the following derivative:

\input{Derivative-Compute-0004.HELP.tex}

\[\dfrac{d}{dx}\left(\frac{x^{2} + 9 \, x + 20}{x^{2} - 3 \, x + 2}\right)=\answer{-\frac{6 \, {\left(2 \, x^{2} + 6 \, x - 13\right)}}{{\left(x^{2} - 3 \, x + 2\right)}^{2}}}\]
\end{problem}}

%%%%%%%%%%%%%%%%%%%%%%

\latexProblemContent{
\ifVerboseLocation This is Derivative Compute Question 0004. \\ \fi
\begin{problem}

Compute the following derivative:

\input{Derivative-Compute-0004.HELP.tex}

\[\dfrac{d}{dx}\left(\frac{x^{3} + 2 \, x^{2} - 16 \, x - 32}{x + 3}\right)=\answer{\frac{2 \, x^{3} + 11 \, x^{2} + 12 \, x - 16}{{\left(x + 3\right)}^{2}}}\]
\end{problem}}

%%%%%%%%%%%%%%%%%%%%%%

\latexProblemContent{
\ifVerboseLocation This is Derivative Compute Question 0004. \\ \fi
\begin{problem}

Compute the following derivative:

\input{Derivative-Compute-0004.HELP.tex}

\[\dfrac{d}{dx}\left(\frac{x^{2} + 2 \, x - 3}{x^{3} - 4 \, x^{2} - 25 \, x + 100}\right)=\answer{-\frac{x^{4} + 4 \, x^{3} + 8 \, x^{2} - 176 \, x - 125}{{\left(x^{3} - 4 \, x^{2} - 25 \, x + 100\right)}^{2}}}\]
\end{problem}}

%%%%%%%%%%%%%%%%%%%%%%

\latexProblemContent{
\ifVerboseLocation This is Derivative Compute Question 0004. \\ \fi
\begin{problem}

Compute the following derivative:

\input{Derivative-Compute-0004.HELP.tex}

\[\dfrac{d}{dx}\left(\frac{x^{3} - 3 \, x^{2} - 25 \, x + 75}{x^{2} + 4 \, x + 3}\right)=\answer{\frac{x^{4} + 8 \, x^{3} + 22 \, x^{2} - 168 \, x - 375}{{\left(x^{2} + 4 \, x + 3\right)}^{2}}}\]
\end{problem}}

%%%%%%%%%%%%%%%%%%%%%%

\latexProblemContent{
\ifVerboseLocation This is Derivative Compute Question 0004. \\ \fi
\begin{problem}

Compute the following derivative:

\input{Derivative-Compute-0004.HELP.tex}

\[\dfrac{d}{dx}\left(\frac{1}{x^{2} + x - 6}\right)=\answer{-\frac{2 \, x + 1}{{\left(x^{2} + x - 6\right)}^{2}}}\]
\end{problem}}\fi             %% end of \ifproblemToFind near top of file
\fi             %% end of \ifquestionCount near top of file
\ProblemFileFooter