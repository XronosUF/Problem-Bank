%%%%%%  This is the raw list of questions before processing %%%%%%

%%%%%%%%%%%%%%%%%%%%%%%%%%%%%%%%%%%%%%%%%%%%%%%%%%%%%%%%%%%%%%%%%%%%%%%%%%%%%%%
%%%%%%%%%%%%%%%%%%%%%%%%%%%%%%%%%%%%%%%%%%%%%%%%%%%%%%%%%%%%%%%%%%%%%%%%%%%%%%%
%%%%%%%%%%%%%%%%%%%										 %%%%%%%%%%%%%%%%%%%%%%%
%%%%%%%%%%%%%%%%%%%				Computation				%%%%%%%%%%%%%%%%%%%%%%%
%%%%%%%%%%%%%%%%%%%										%%%%%%%%%%%%%%%%%%%%%%%
%%%%%%%%%%%%%%%%%%%%%%%%%%%%%%%%%%%%%%%%%%%%%%%%%%%%%%%%%%%%%%%%%%%%%%%%%%%%%%%
%%%%%%%%%%%%%%%%%%%%%%%%%%%%%%%%%%%%%%%%%%%%%%%%%%%%%%%%%%%%%%%%%%%%%%%%%%%%%%%



%%%%%%%%%%%%%%%%%%%%%%%%%%%%%%%%%%%%%%%%%%%%%%%%%%%%%%%%%%%%%%%%%%%%%%%%%%%%%%%
%%%%%%%%%%%%%%%%%%%			MAC2311: Calculus 1			%%%%%%%%%%%%%%%%%%%%%%%
%%%%%%%%%%%%%%%%%%%%%%%%%%%%%%%%%%%%%%%%%%%%%%%%%%%%%%%%%%%%%%%%%%%%%%%%%%%%%%%




%%%%%%%%%%%%%%%%%%%%%
%\tagged{Ans@ShortAns, Type@Compute, Topic@Limit, Func@Rational, Sub@Disc-Remove, Sub@Asymptote, File@0001}{
\begin{sagesilent}
a = NonZeroInt(-10,10)
b = NonZeroInt(-10,10, [0,a])
c = RandInt(1,3)

fNum=expand(c*(x-a)*(x-b))
fDen=(x-a)
ans=c*(a-b)# The limit is simplified and computable by design.
\end{sagesilent}

\latexProblemContent{
\ifVerboseLocation This is Limit Compute Question 0001. \\ \fi
\begin{problem}
Determine if the limit approaches a finite number, $\infty$, $-\infty$, or does not exist. (If the limit does not exist, write DNE)

\input{Limit-Compute-0001.HELP.tex}

\[\lim_{x\to\sage{a}}\dfrac{\sage{fNum}}{\sage{fDen}}=\answer{\sage{ans}}\]
\end{problem}}%}
%%%%%%%%%%%%%%%%%%%%%


%%%%%%%%%%%%%%%%%%%%%
%\tagged{Ans@ShortAns, Type@Compute, Topic@Limit, Sub@Rational, File@0002}{
\begin{sagesilent}
a = NonZeroInt(-10,10)
b = NonZeroInt(-10,10, [0,a])
c = RandInt(1,3)

fDen=expand((x-a)*(x-b))
fNum=c*(x-a)
Ans=c/(a-b)# Limit is simplified to be computable by design.
\end{sagesilent}

\latexProblemContent{
\ifVerboseLocation This is Limit Compute Question 0002. \\ \fi
\begin{problem}

Determine if the limit approaches a finite number, $\infty$, $-\infty$, or does not exist. (If the limit does not exist, write DNE)

\input{Limit-Compute-0002.HELP.tex}

\[\lim_{x\to\sage{a}}\dfrac{\sage{fNum}}{\sage{fDen}}=\answer{\sage{Ans}}\]
\end{problem}}%}
%%%%%%%%%%%%%%%%%%%%%%



%%%%%%%%%%%%%%%%%%%%%%
%\tagged{Ans@ShortAns, Type@Compute, Topic@Limit, Sub@Rational, File@0003}{
\begin{sagesilent}
a = NonZeroInt(-10,10)
b = NonZeroInt(-10,10, [0,a])
c = RandInt(1,3)

fNum=expand(c*(x-a)*(x-b))
fDen=(x-a)
Ans = 0# by simplification; limit(fNum/fDen, x=b)
\end{sagesilent}

\latexProblemContent{
\ifVerboseLocation This is Limit Compute Question 0003. \\ \fi
\begin{problem}

Determine if the limit approaches a finite number, $\infty$, $-\infty$, or does not exist. (If the limit does not exist, write DNE)

\input{Limit-Compute-0003.HELP.tex}

\[
   \lim_{x\to\sage{b}}\dfrac{\sage{fNum}}{\sage{fDen}}
   =
   \answer{0}
\]
\end{problem}}%}
%%%%%%%%%%%%%%%%%%%%%%


%%%%%%%%%%%%%%%%%%%%%%
%\tagged{Ans@ShortAns, Type@Compute, Topic@Limit, Sub@Rational, Sub@Asymptote, File@0004}{
\begin{sagesilent}
a = NonZeroInt(-10,10)
b = NonZeroInt(-10,10, [0,a])
c = RandInt(1,3)

fDen=expand((x-a)*(x-b))
fNum=c*(x-a)
\end{sagesilent}

\latexProblemContent{
\ifVerboseLocation This is Limit Compute Question 0004. \\ \fi
\begin{problem}

Determine if the limit approaches a finite number, $\infty$, $-\infty$, or does not exist. (If the limit does not exist, write DNE)

\input{Limit-Compute-0004.HELP.tex}

\[
   \lim_{x\to\sage{b}}\dfrac{\sage{fNum}}{\sage{fDen}}
   =
   \answer[format=string]{DNE}
\]
\end{problem}}%}
%%%%%%%%%%%%%%%%%%%%%


%%%%%%%%%%%%%%%%%%%%%%
%\tagged{Ans@ShortAns, Type@Compute, Topic@Limit, Sub@Rational, File@0005}{
\begin{sagesilent}
a = NonZeroInt(-6,6)
b = NonZeroInt(-6,6, [0,a])
c = NonZeroInt(-5,5)

fNum=expand((x-a)*(x-c))
fDen=expand((x-a)*(x-b))
#Ans=limit(fNum/fDen, x=a)
Ans=(a-c)/(a-b)
\end{sagesilent}

\latexProblemContent{
\ifVerboseLocation This is Limit Compute Question 0005. \\ \fi
\begin{problem}

Determine if the limit approaches a finite number, $\infty$, $-\infty$, or does not exist. (If the limit does not exist, write DNE)

\input{Limit-Compute-0005.HELP.tex}

\[
   \lim_{x\to\sage{a}}\dfrac{\sage{fNum}}{\sage{fDen}}
   =
   \answer{\sage{Ans}}
\]
\end{problem}}%}
%%%%%%%%%%%%%%%%%%%%%


%%%%%%%%%%%%%%%%%%%%%%
%\tagged{Ans@ShortAns, Type@Compute, Topic@Limit, Sub@Rational, File@0006}{
\begin{sagesilent}
a = NonZeroInt(-6,6)
b = NonZeroInt(-6,6, [0,a])
c = NonZeroInt(-5,5)

fNum=expand(c*(x^3-a^3))# = c*(x-a)*(x^2+a*x+a^2)
fDen=expand((x-a)*(x-b))
Ans=(c*(3*a^2))/(a-b)# Answer is computable by simplification by design; limit(fNum/fDen, x=a)
\end{sagesilent}

\latexProblemContent{
\ifVerboseLocation This is Limit Compute Question 0006. \\ \fi
\begin{problem}

Determine if the limit approaches a finite number, $\infty$, $-\infty$, or does not exist. (If the limit does not exist, write DNE)

\input{Limit-Compute-0006.HELP.tex}

\[
   \lim_{x\to\sage{a}}\dfrac{\sage{fNum}}{\sage{fDen}}
   =
   \answer{\sage{Ans}}
\]
\end{problem}}%}
%%%%%%%%%%%%%%%%%%%%%



%%%%%%%%%%%%%%%%%%%%%%
%\tagged{Ans@ShortAns, Type@Compute, Topic@Limit, Sub@Rational, Sub@Asymptote, File@0007}{
\begin{sagesilent}
a = RandInt(-5,5)
b = NonZeroInt(-5,5, [a])
c = NonZeroInt(-5,5, [a, b])
d = NonZeroInt(-5,5, [a,b,c])
e = NonZeroInt(-5,5, [a,b,c,d])
   
v = [
   (x-a),
   (x-b),
   (x-c),
   (x-d),
   (x-e)
]

p = RandInt(1,3)
q = RandInt(p,4)

f = 1
for j in [1..p]:
   k = RandInt(0,4)
   f = v[k] * f

g = 1
for j in [1..q]:
   k = RandInt(0,4)
   g = v[k] * g

fNum = expand(f)
fDen = expand(g)
#Ans = limit(fNum/fDen, x=infinity)
if p == q:
   Ans = 1 # If the degrees are equal, then the limit is 1
else:
   Ans = 0 # Otherwise, by design the bottom is higher degree than top so the answer is 0.

Inf = choice([infinity, -infinity])

\end{sagesilent}

\latexProblemContent{
\ifVerboseLocation This is Limit Compute Question 0007. \\ \fi
\begin{problem}

Determine if the limit approaches a finite number, $\infty$, $-\infty$, or does not exist. (If the limit does not exist, write DNE)

\input{Limit-Compute-0007.HELP.tex}

\[
   \lim\limits_{x\to\sage{Inf}}\frac{\sage{fNum}}{\sage{fDen}}
   =
   \answer{\sage{Ans}}
\]

\end{problem}}%}
%%%%%%%%%%%%%%%%%%%%%


%%%%%%%%%%%%%%%%%%%%%%%
%%\tagged{Ans@FRQ, Type@Compute, Topic@Limit, Sub@Rational, Sub@Continuity, File@0008}{
\begin{sagesilent}
a = RandInt(1,5)
b = NonZeroInt(-10,10, [0,a])
r = a-b   
c = NonZeroInt(-5,5, [0,r])

vleft = [
   b/2 * (x+a)/a,
   expand((x-c)*(x-r)) / (x-c),
   sqrt(x^2-a^2) + b
]
vright = [
   b/2 * (x+a)/a,
   expand((x-c)*(x-r)) / (x-c)
]

first = RandInt(0,2)
second = NonZeroInt(0,1, [first])

Fone=vleft[first]
Ftwo=vright[second]
\end{sagesilent}

\latexProblemContent{
\ifVerboseLocation This is Limit Compute Question 0008. \\ \fi
\begin{problem}


Use the function to answer the following questions.
\[
   f(x)
   =
   \left\lbrace\begin{array}{ll}\sage{Fone}& x<\sage{a}\\[5pt] 
   \sage{Ftwo}& x\geq\sage{a} 
   \end{array}\right.
\]

\input{Limit-Compute-0008.HELP.tex}


The function is ...
\begin{multipleChoice}
    \choice[correct]{continuous at $x=\sage{a}$.}
    \choice{discontinuous at $x=\sage{a}$.}
\end{multipleChoice}


\begin{problem}
Compute $\lim\limits_{x\to\sage{a}^-}f(x)=\answer{\sage{b}}$\\[1in]

Compute $\lim\limits_{x\to\sage{a}^+}f(x)=\answer{\sage{b}}$\\[1in]

Compute $f(\sage{a})=\answer{\sage{b}}$\\[1in]

\end{problem}

\end{problem}}%}
%%%%%%%%%%%%%%%%%%%%%%


%%%%%%%%%%%%%%%%%%%%%%%
%%\tagged{Ans@ShortAns, Type@Compute, Topic@Limit, Sub@Poly, Sub@Trig, Sub@Continuity, File@0009}{
\begin{sagesilent}
a = NonZeroInt(-5,5)
b = NonZeroInt(-5,5, [0,a])
c = NonZeroInt(-5,5)
d = NonZeroInt(-5,5, [0,a])
   
vleft=[
   a*sin(b*x*pi), 
   a*cos(b*x*pi), 
   a*tan(b*x*pi), 
   a*sin(b*x*pi/2), 
   a*cos(b*x*pi/2), 
   a*tan(b*x*pi/2), 
   a*sin(b*x*pi/6), 
   a*cos(b*x*pi/6), 
   a*tan(b*x*pi/6), 
   a*sin(b*x*pi/3), 
   a*cos(b * x * pi/3), 
   a*tan(b * x * pi/3)
]

vright=[
   (x-a),
   (x-a)*(x-b),
   (x-a)*(x-b)*(x-c),
   expand((x-a)*(x-b)),
   expand((x-a)*(x-b)*(x-c))
]

first = Integer(randint(0,11))
second = Integer(randint(0,4))

Fone = vleft[first]
Ftwo = vright[second]
F = Fone * Ftwo

Ans = limit(F,x=d)

\end{sagesilent}

\latexProblemContent{
\ifVerboseLocation This is Limit Compute Question 0009. \\ \fi
\begin{problem}

Determine if the limit approaches a finite number, $\infty$, $-\infty$, or does not exist. (If the limit does not exist, write DNE)

\input{Limit-Compute-0009.HELP.tex}

\[
   \lim_{x\to\sage{d}}\sage{F}
   =
   \answer{\sage{Ans}}
\]

\end{problem}}%}
%%%%%%%%%%%%%%%%%%%%%%

%%%%%%%%%%%%%%%%%%%%%%%
%%\tagged{Ans@ShortAns, Type@Compute, Topic@Limit, Sub@Rational, Sub@DifferenceQuotient, File@0010}{
\begin{sagesilent}
a = NonZeroInt(-8,8)
b = NonZeroInt(-15,15)
Fone=(b/a)+(b/x)
Ftwo=a+x
F=Fone/Ftwo

Ans=limit(F,x=-a)
\end{sagesilent}

\latexProblemContent{
\ifVerboseLocation This is Limit Compute Question 0010. \\ \fi
\begin{problem}

Compute the limit of the following difference quotient:

\input{Limit-Compute-0010.HELP.tex}

\[
   \lim_{x\to\sage{-a}}\dfrac{\sage{Fone}}{\sage{Ftwo}}
   =
   \answer{\sage{Ans}}
\]
\end{problem}}%}
%%%%%%%%%%%%%%%%%%%%%%


%%%%%%%%%%%%%%%%%%%%%%%
%%\tagged{Ans@ShortAns, Type@Compute, Topic@Limit, Sub@Asymptote, Sub@OneSided, File@0011}{
\begin{sagesilent}
a = NonZeroInt(-30,30)
b = RandInt(1,9)
v = [
   a*x*ln(x),
   a/(x^b)
]
pick1 = min(1,RandInt(0,9)) # Use min to do a clever weighting so we don't end up with half of all problems are log.
F = v[pick1]

if pick1==0:
   Ans=0# By definition
else:# The bottom is going to 0 from the right regardless of other choice
   if a > 0:# Top is positive
      Ans=infinity
   else:# Top is negative
      Ans=-infinity

###  Ans=limit(F,x=0,dir='plus') # Answer is computable directly.
\end{sagesilent}

\latexProblemContent{
\ifVerboseLocation This is Limit Compute Question 0011. \\ \fi
\begin{problem}

Determine if the limit approaches a finite number, $\infty$, $-\infty$, or does not exist. (If the limit does not exist, write DNE)

\input{Limit-Compute-0011.HELP.tex}

\[\lim_{x\to 0^+}\sage{F}=\answer{\sage{Ans}}\]

\end{problem}}%}
%%%%%%%%%%%%%%%%%%%%%%


%%%%%%%%%%%%%%%%%%%%%%%
%%\tagged{Ans@ShortAns, Type@Compute, Topic@Limit, Sub@Asymptote, Sub@Radical, File@0012}{
\begin{sagesilent}
a = NonZeroInt(-5,5)
b = NonZeroInt(-5,5)
c = NonZeroInt(-5,5)   
d1 = NonZeroInt(1,10, [a])
d = d1^2 # For a nice answer
e = NonZeroInt(-5,5) 
m = RandInt(1,3)

F = (sqrt(d*x^(2*m)+a)-b)/(e*x^m+c)

Ans = d1 / e # Computable by simplification. limit(F,x=infinity)
\end{sagesilent}

\latexProblemContent{
\ifVerboseLocation This is Limit Compute Question 0012. \\ \fi
\begin{problem}

Determine if the limit approaches a finite number, $\infty$, $-\infty$, or does not exist. (If the limit does not exist, write DNE)

\input{Limit-Compute-0012.HELP.tex}

\[
   \lim_{x\to\infty}\sage{F}
   =
   \answer{\sage{Ans}}
\]

\end{problem}}%}
%%%%%%%%%%%%%%%%%%%%%%

%%%%%%%%%%%%%%%%%%%%%%%
%%\tagged{Ans@ShortAns, Type@Compute, Topic@Limit, Sub@DifferenceQuotient, Sub@Poly, File@0013}{
\begin{sagesilent}
a = NonZeroInt(-15,15)
b = NonZeroInt(-4,4)
p=RandInt(0,3)
if p==0:
   b = RandInt(1,4)

v = [
   a*sqrt(x),
   a/x, 
   a*x^2, 
   a*x^3
]

F = v[p]
Fone = F(b+h)
Ftwo = F(b)
DQ = (F(b+h)-F(b))/h

Ans=limit(DQ,h=0)
\end{sagesilent}

\latexProblemContent{
\ifVerboseLocation This is Limit Compute Question 0013. \\ \fi
\begin{problem}

Determine if the limit approaches a finite number, $\infty$, $-\infty$, or does not exist. (If the limit does not exist, write DNE)

\input{Limit-Compute-0013.HELP.tex}

\[
   \lim_{h\to0}\frac{\sage{Fone-Ftwo}}{h}
   =
   \answer{\sage{Ans}}
\]

\end{problem}}%}
%%%%%%%%%%%%%%%%%%%%%%

%%%%%%%%%%%%%%%%%%%%%  
%\tagged{Ans@ShortAns, Type@Compute, Topic@Limit, Sub@Trig, File@0014}{
\begin{sagesilent}
a = NonZeroInt(-15,15)
b = NonZeroInt(1,5,[a])

p=RandInt(0,4)
v=[sin(a*x)/(b*x), (a*x)/sin(b*x), tan(a*x)/(b*x), (a*x)/tan(b*x), b*(1-cos(a*x))/(a*x)]

F=v[p]
Ans=limit(F,x=0)
\end{sagesilent}

\latexProblemContent{
\ifVerboseLocation This is Limit Compute Question 0014. \\ \fi
\begin{problem}

Determine if the limit approaches a finite number, $\infty$, $-\infty$, or does not exist. (If the limit does not exist, write DNE)

\input{Limit-Compute-0014.HELP.tex}

\[\lim_{x\to0}\sage{F}=\answer{\sage{Ans}}\]
\end{problem}}%}
%%%%%%%%%%%%%%%%%%%%%

%%%%%%%%%%%%%%%%%%%%%
%\tagged{Ans@ShortAns, Type@Compute, Topic@Limit, Sub@Trig, File@0015}{
\begin{sagesilent}
a = NonZeroInt(-15,15)
b = NonZeroInt(1,5,[a])

funcVec = [
   sin(a*x)   / (b*x), 
   sin(a*x)^2 / (b*x^2), 
   sin(a*x)^3 / (b*x^3), 
   tan(a*x)   / (b*x), 
   tan(a*x)^2 / (b*x^2), 
   tan(a*x)^3 / (b*x^3)
]

F = choice(funcVec)
ans = limit(F,x=0)
\end{sagesilent}

\latexProblemContent{
\ifVerboseLocation This is Limit Compute Question 0015. \\ \fi
\begin{problem}

Determine if the limit approaches a finite number, $\infty$, $-\infty$, or does not exist. (If the limit does not exist, write DNE)

\input{Limit-Compute-0015.HELP.tex}

\[
   \lim_{x\to0}\sage{F}
   =
   \answer{\sage{ans}}
\]
\end{problem}}%}
%%%%%%%%%%%%%%%%%%%%%


%\tagged{Ans@ShortAns, Type@Compute, Topic@Limit, Sub@Rational, File@0016}{

\begin{sagesilent}
funcVec = [
   x, 
   x^(1/3), 
   x^2
] 
assume(x>0)
f1 = choice(funcVec)

c1 = RandInt(1,6)
c2 = NonZeroInt(-10,10)

p1 = RandInt(2,3)

f2top(x) = expand(c2*(f1^p1 - c1^p1))
f2bot(x) = f1 - c1
f2(x) = f2top(x) / f2bot(x)

finv = solve(x==f1(x=y), y)[0].rhs()

Ans1 = limit(f2(x), x=finv(c1))
Ans = Ans1.simplify_rational()
\end{sagesilent}

\latexProblemContent{
\ifVerboseLocation This is Limit Compute Question 0016. \\ \fi
\begin{problem}
Calculate the following limit:

\input{Limit-Compute-0016.HELP.tex}

\[
   \lim_{x \rightarrow \sage{finv(c1)}} \sage{f2(x)} 
   = 
   \answer{\sage{Ans}}
\]
\end{problem}}%}


%\tagged{Ans@ShortAns, Type@Compute, Topic@Limit, Sub@Rational, File@0017}{

\begin{sagesilent}
# Our goal here is to pick a function `g(x)` for the numerator, and then set our 
# actual numerator to be `f(g(h(x)))`, where both `f` and `h` are linear functions.

# Then repeat for the denominator.

Check = 0

while Check == 0:
   funcVec = [
      x^(-1), 
      x^(1/3), 
      x^1, 
      x^2, 
      exp(x), 
      log(x)
   ]
   numMiddleFunc = choice(funcVec) # This is `g` in the description above.
   denMiddleFunc = choice(funcVec) # This is `g` in the description above.

   chosenFuncs = [numMiddleFunc, denMiddleFunc]
   
   if log(x) in chosenFuncs:
      minVal = 1 # Avoid negative targets if the function is logarithmic.
   else:
      minVal = -8

   if exp(x) in chosenFuncs:
      maxVal = 2 # Avoid larget targets if the function is exponential.
   else:
      maxVal = 8

   target = RandInt(minVal, maxVal)
   
   #Setup for while loop.
   numerator   = x + 1000 # Just so that the while loop runs at least once.
   denominator = x + 1000 # Just so that the while loop runs at least once.
      
   #First we do the numerator
   while abs(numerator(x = target)) > 400:
      inner_coef  = RandInt(1, 10)
      inner_const = RandInt(1, 10)
      inner = inner_coef*x + inner_const # This is a candidate for `h` in the description above.
      middle = numMiddleFunc(x = inner)

      while inner(x = target) == 0:
         inner_coef  = RandInt(1, 10)
         inner_const = RandInt(1, 10)
         inner = inner_coef*x + inner_const # This is a candidate for `h` in the description above.
         middle = numMiddleFunc(x = inner)
      
      outer_coef  = NonZeroInt(-5,5)
      outer_const = RandInt(-10,10)
      numerator = outer_coef*middle + outer_coef
   
   # Now we work on the denominator
   while abs(denominator(x = target)) > 400 or denominator(x = target) == 0:
      inner_coef  = RandInt(1, 10)
      inner_const = RandInt(1, 10)
      inner = inner_coef*x + inner_const # This is a candidate for `h` in the description above.
      middle = denMiddleFunc(x = inner)
      
      while inner(x = target) == 0:
         inner_coef  = RandInt(1, 10)
         inner_const = RandInt(1, 10)
         inner = inner_coef*x + inner_const # This is a candidate for `h` in the description above.
         middle = denMiddleFunc(x = inner)

      outer_coef  = NonZeroInt(-5,5)
      outer_const = RandInt(-10,10)
      denominator = outer_coef*middle + outer_coef

   # Now we have a nice answer, so we just need to clean up.
   func = numerator/denominator
   ans = func(x = target)
   Check = saneCheck(ans)

   
\end{sagesilent}

\latexProblemContent{
\ifVerboseLocation This is Limit Compute Question 0017. \\ \fi
\begin{problem}


Compute the following limit:

\input{Limit-Compute-0017.HELP.tex}

$\lim\limits_{x \rightarrow \sage{target}} \sage{func} = \answer{\sage{ans}}$

\end{problem}}%}


% Squeeze theorem problem. Currently the answer is always zero, should mix it up a little more.

%\tagged{Ans@ShortAns, Type@Compute, Topic@Limit, Sub@Squeeze, File@0018}{

\begin{sagesilent}

# These functions go to 0 as x goes to 0.
funcVec1 = [
   sin(x), 
   cos((pi/2)-(x)), 
   sqrt(x), 
   x^1, 
   x^2, 
   x^3
]
f1 = choice(funcVec1) # This is the part that will let us use the Squeeze theorem.

# These functions "blow up" as x goes to 0.
funcVec2 = [
   1/x, 
   log(abs(x))
]
f2 = choice(funcVec2)

# These functions oscillate as x tends to +infinity or -infinity.
funcVec3 = [
   sin(x),
   cos(x)
]
f3 = choice(funcVec3)
f4 = f3.subs(x = f2) # This function will now not have a limit as x goes to 0

c = RandInt(-10,10) # This will be our shift.
fFinal(x) = (f1 * f4).subs(x = (x-c))

\end{sagesilent}

\latexProblemContent{
\ifVerboseLocation This is Limit Compute Question 0018. \\ \fi
\begin{problem}

Calculate the following limit:

\input{Limit-Compute-0018.HELP.tex}

$\lim\limits_{x\rightarrow\sage{c}} \sage{fFinal(x)} = \answer{0}$

\end{problem}}%}


%%Special Limits, sinx/x and tanx/x

%\tagged{Ans@ShortAns, Type@Compute, Topic@Limit, Sub@squeeze, File@0019}{

\begin{sagesilent}

#This problem will involve sin(x) and tan(x) (to a -1,0 or 1 power)
f1(x) = sin(x)
f2(x) = tan(x)

pwr1 = RandInt(-1,1)
if pwr1 == 0: # If pwr1 is 0, we need to make sure the other one isn't.
   pwr2 = NonZeroInt(-1,1)
else: # Otherwise it doesn't matter.
   pwr2 = RandInt(-1,1)

c1 = RandInt(1,10)
c2 = RandInt(1,10)
pwr3 = pwr1 + pwr2 # This is the power of x we will end up needing.
f3(x) = f1(c1*x)^pwr1*f2(c2*x)^pwr2

pwr4 = (-1)^(1 - RandInt(0,1))# The power of the constant
c4 = NonZeroInt(-10,10)

f4(x) = c4^pwr4 * f3(x) / (x^pwr3) 

ans = limit(f4(x), x=0)

\end{sagesilent}

\latexProblemContent{
\ifVerboseLocation This is Limit Compute Question 0019. \\ \fi
\begin{problem}

Calculate the following limit:

\input{Limit-Compute-0019.HELP.tex}

$\lim\limits_{x\rightarrow 0} \sage{f4(x)} = \answer{\sage{ans}}$
\end{problem}}%}



%\tagged{Ans@ShortAns, Type@Compute, Topic@Limit, Sub@Rational, File@0020}{

\begin{sagesilent}

cp1 = RandInt(3,7)
c1 = RandInt(-10,10)
c2 = RandInt(-10,10)
c3 = RandInt(-10,10)
c4 = RandInt(-10,10)

cp2 = RandInt(3,7)
c5 = RandInt(-10,10)
c6 = RandInt(-10,10)
c7 = RandInt(-10,10)
c8 = RandInt(-10,10)

f1(x) = c1*x^cp1+c2*x^2+c3*x+c4
f2(x) = c5*x^cp2+c6*x^2+c7*x+c8

tog = (-1)^RandInt(0,1)

fFinal(x) = f1(x)/f2(x)

lim = tog*infinity

ans = limit(fFinal(x),x=lim)

\end{sagesilent}

\latexProblemContent{
\ifVerboseLocation This is Limit Compute Question 0020. \\ \fi
\begin{problem}

Compute the following limit:

\input{Limit-Compute-0020.HELP.tex}

$\lim\limits_{x \rightarrow \sage{lim}} \sage{fFinal(x)} = \answer{\sage{ans}}$.
\end{problem}}%}

%\tagged{Ans@ShortAns, Type@Compute, Topic@Limit, Sub@Rational, Sub@Indeterminant, File@0021}{

\begin{sagesilent}

num_pwr = RandInt(3,7)
num_coef = NonZeroInt(-10,10)
num_poly = num_coef * x^num_pwr

num_terms = 4
for _ in range(num_terms-1):
   pwr = choice(range(num_pwr))
   coef = NonZeroInt(-10,10)
   num_poly += coef * x^pwr


root = RandInt(2,5)
den_pwr = RandInt(3,7) * root
den_coef = NonZeroInt(-10,10)
den_poly = den_coef * x^den_pwr

den_terms = 4
for _ in range(den_terms-1):
   pwr = choice(range(den_pwr))
   coef = NonZeroInt(-10,10)
   den_poly += coef * x^pwr

fCompute(x) = num_poly/(den_poly^(1/root))

sign = choice([-1,1])
target = sign * infinity

ans = limit(fCompute(x),x=target)

\end{sagesilent}

\latexProblemContent{
\ifVerboseLocation This is Limit Compute Question 0021. \\ \fi
\begin{problem}

Compute the following limit:

\input{Limit-Compute-0021.HELP.tex}

\[
\lim\limits_{x \rightarrow \sage{target}} 
\dfrac{ \sage{num_poly} }{ \sqrt[\sage{root}]{\sage{den_poly}} } 
= 
\answer{\sage{ans}}.
\]
\end{problem}}%}



%%%%%%%%%%%%%%%%%%%%%
%\tagged{Ans@ShortAns, Type@Compute, Topic@Limit, Sub@Poly, Sub@DifferenceQuotient, File@0022}{
\begin{sagesilent}
# This is a ballistic arc problem
a = -16 # This is acceleration in ft/sec
b = RandInt(1,40)# Positive for initial velocity ft/sec
c = RandInt(0,10)# Starting position in ft.

start = RandInt(1,4)# Choose a time to consider location

f(t) = a*t^2+b*t+c

ans1 = 2*( f(start+(1/2))-f(start) )# Difference quotient for h=1/2

ans2 = ( f(start+h)-f(start) )/h# Difference quotient

ans3 = limit(ans2(h),h=0)
\end{sagesilent}

\latexProblemContent{
\ifVerboseLocation This is Limit Compute Question 0022. \\ \fi
\begin{problem}
The position function of a ball thrown into the air with a velocity of $\sage{b}$ ft/sec and initial height of $\sage{c}$ feet is given by the function $s(t) = \sage{f(t)}$ where $s(t)$ is the height of the ball above the ground after $t$ seconds.  

\input{Limit-Compute-0022.HELP.tex}

Find the average velocity of the ball on the interval starting with $t = \sage{start}$ to the time $0.5$ seconds later. $V_{ave}=\answer[tolerance=0.1]{\sage{ans1}}$

\begin{problem}
Now write an expression for the average velocity of the ball on the time interval from $t=\sage{start}$ to $h$ seconds later.  That is, find the expression for the average velocity on the interval $[\sage{start},\sage{start}+h]$ where $h\neq 0$.

$\answer{\sage{ans2}}$

\begin{problem}
How can we (mathematically) determine what happens as $h$ gets closer and closer to 0? 

\begin{multipleChoice}
\choice[correct]{We can use limits to find what happens when $h$ goes to 0.}
\choice{We can use limits to find what happens when $h$ goes to $\sage{start}$.}
\choice{We can plug in very small values of $h$ to figure out what the value of $s(t)$ is heading to.}
\choice{We can plug in $h=0$ and then simplify to find an answer.}
\end{multipleChoice}
%average velocity approaches -24 ft/sec as h goes to 0
\begin{problem}
Calculate the limit:

$\lim\limits_{h\rightarrow 0} \sage{ans2} = \answer{\sage{ans3}}$

\end{problem}
\end{problem}
\end{problem}
\end{problem}}%}



%%%%%%%%%%%%%%%%%%%%%
%\tagged{Ans@ShortAns, Type@Compute, Topic@Limit, Sub@Poly, Sub@DifferenceQuotient, File@0023}{
\begin{sagesilent}
a = RandInt(-20,-1)# This is negative to account for acceleration
b = RandInt(1,10)# Positive for initial velocity
c = RandInt(-5,5)# Starting position

start = RandInt(1,4)# Choose a time to consider location

f(t) = a*t^2+b*t+c

ans = ( f(start+h)-f(start) )/h# Difference quotient

\end{sagesilent}
\latexProblemContent{
\ifVerboseLocation This is Limit Compute Question 0023. \\ \fi
\begin{problem}
Find the average velocity of $s(t) = \sage{f(t)}$ from $t=\sage{start}$ to $t=\sage{start}+h$. 

\input{Limit-Compute-0023.HELP.tex}

$V_{ave}=\answer{\sage{ans}}$

\end{problem}}%}


%%%%%%%%%%%%%%%%%%%%%
%\tagged{Ans@ShortAns, Type@Compute, Topic@Limit, Sub@Poly, Sub@DifferenceQuotient, File@0024}{
\begin{sagesilent}
a = NonZeroInt(-15,15)
c = NonZeroInt(-8,9)

f(x) = a*x^2

dif(h) = (f(c+h)-f(c))/(h)

ans = 2*a*c

\end{sagesilent}

\latexProblemContent{
\begin{problem}
\ifVerboseLocation This is Limit Compute Question 0024. \\ \fi
Use algebra to find the following limit

\input{Limit-Compute-0024.HELP.tex}

\[
\lim\limits_{h\rightarrow 0} \sage{dif(h)} = \answer{\sage{ans}}
\]
\end{problem}}%}




%%%%%%%%%%%%%%%%%%%%%%%%%%%%%%%%%%%%%%%%%%%%%%%%%%%%%%%%%%%%%%%%%%%%%%%%%%%%%%%
%
%
%
%
%
%
%
%
%%%%%%%%%%%%%%%%%%%%%%%%%%%%%%%%%%%%%%%%%%%%%%%%%%%%%%%%%%%%%%%%%%%%%%%%%%%%%%%
%%%%%%%%%%%%%%%%%%%%%%%%%%%%%%%%%%%%%%%%%%%%%%%%%%%%%%%%%%%%%%%%%%%%%%%%%%%%%%%
%%%%%%%%%%%%%%%%%%%										%%%%%%%%%%%%%%%%%%%%%%%
%%%%%%%%%%%%%%%%%%%				Concept					%%%%%%%%%%%%%%%%%%%%%%%
%%%%%%%%%%%%%%%%%%%										%%%%%%%%%%%%%%%%%%%%%%%
%%%%%%%%%%%%%%%%%%%%%%%%%%%%%%%%%%%%%%%%%%%%%%%%%%%%%%%%%%%%%%%%%%%%%%%%%%%%%%%
%%%%%%%%%%%%%%%%%%%%%%%%%%%%%%%%%%%%%%%%%%%%%%%%%%%%%%%%%%%%%%%%%%%%%%%%%%%%%%%
%
%
%
%
%
%
%
%
%
%%%%%%%%%%%%%%%%%%%%%%%%%%%%%%%%%%%%%%%%%%%%%%%%%%%%%%%%%%%%%%%%%%%%%%%%%%%%%%%
%
%
%
%
%%%%%%%%%%%%%%%%%%%%%%%%%%%%%%%%%%%%%%%%%%%%%%%%%%%%%%%%%%%%%%%%%%%%%%%%%%%%%%%
%%%%%%%%%%%%%%%%%%%				Calc 1 Concept			%%%%%%%%%%%%%%%%%%%%%%%
%%%%%%%%%%%%%%%%%%%%%%%%%%%%%%%%%%%%%%%%%%%%%%%%%%%%%%%%%%%%%%%%%%%%%%%%%%%%%%%

%%%%%%%%%%%%%%%%%%%%%%%
%%\tagged{Ans@ShortAns, Type@Concept, Topic@Limit, Sub@LimitLaws, File@0001}{
\begin{sagesilent}
a = NonZeroInt(-5,5)
b = NonZeroInt(-5,5)
c = NonZeroInt(-5,5)

f=x+a
ans=b*c
\end{sagesilent}

\latexProblemContent{
\ifVerboseLocation This is Limit Concept Question 0001. \\ \fi
\begin{problem}

If you know that $\lim\limits_{x\to\sage{a}}f(x)=\sage{b}$ and $\lim\limits_{x\to0}g(x)=\sage{c}$, then evaluate the following limit:

\input{Limit-Concept-0001.HELP.tex}

\[\lim_{x\to0}f(\sage{f})g(x)=\answer{\sage{ans}}\]
\end{problem}}%}
%%%%%%%%%%%%%%%%%%%%%%

%%%%%%%%%%%%%%%%%%%%%%%
%%\tagged{Ans@ShortAns, Type@Concept, Topic@Limit, Sub@LimitLaws, File@0002}{
\begin{sagesilent}
a = NonZeroInt(-5,5)
b = NonZeroInt(-5,5)
c = NonZeroInt(-5,5)

f=x+a
ans=b+c
\end{sagesilent}

\latexProblemContent{
\ifVerboseLocation This is Limit Concept Question 0002. \\ \fi
\begin{problem}

If you know that $\lim\limits_{x\to\sage{a}}f(x)=\sage{b}$ and $\lim\limits_{x\to0}g(x)=\sage{c}$, then evaluate the following limit:

\input{Limit-Concept-0002.HELP.tex}

\[\lim_{x\to0}f(\sage{f})+g(x)=\answer{\sage{ans}}\]
\end{problem}}%}
%%%%%%%%%%%%%%%%%%%%%%

%%%%%%%%%%%%%%%%%%%%%%%
%%\tagged{Ans@MultiAns, Type@Concept, Topic@Limit, Sub@LimitLaws, File@0003}{
\begin{sagesilent}
a = NonZeroInt(-5,5)
b = NonZeroInt(-5,5)
c = NonZeroInt(-5,5)

f=x+a
g=x-a
ans=b*c
ans2=b+c
ans3=c-b
ans4=b/c

\end{sagesilent}

\latexProblemContent{
\ifVerboseLocation This is Limit Concept Question 0003. \\ \fi
\begin{problem}

If you know that $\lim\limits_{x\to\sage{a}}f(x)=\sage{b}$ and $\lim\limits_{x\to0}g(x)=\sage{c}$, then evaluate the following limits:

\input{Limit-Concept-0003.HELP.tex}


\[\lim_{x\to0}f(\sage{f})g(x)=\answer{\sage{ans}}\]

\[\lim_{x\to0}f(\sage{f})+g(x)=\answer{\sage{ans2}}\]

\[\lim_{x\to0}g(x)-f(\sage{f})=\answer{\sage{ans3}}\]

\[\lim_{x\to\sage{a}}\dfrac{f(x)}{g(\sage{g})}=\answer{\sage{ans4}}\]
\end{problem}}%}
%%%%%%%%%%%%%%%%%%%%%%

%%%%%%%%%%%%%%%%%%%%%%%
%%\tagged{Ans@MultiAns, Type@Concept, Topic@Limit, Sub@Continuity, Sub@Disc-Jump, Sub@Piecewise, File@0004}{
\begin{sagesilent}
a = RandInt(-3,3)
b = NonZeroInt(a+2,a+5)

c = RandInt(-5,5)
c2 = RandInt(-5,5)
c3 = RandInt(-5,5)

func1=[x-c, c-x, (x-c)^2, expand((x-c)*(x-c2))]
func2=[x, x^2-c3*x, x^2+c3*x, (x-c3)^2]

p1 = RandInt(0,3)
p2 = RandInt(0,3)
p3 = RandInt(0,3)

f1 = func1[p1]
F2 = func2[p2]
F3 = func2[p3]

A = f1(a) - F3(a) + 1
f3 = F3 + A
B = f1(b)- F2(b)
f2 = F2 + B

\end{sagesilent}

\latexProblemContent{
\ifVerboseLocation This is Limit Concept Question 0004. \\ \fi
\begin{problem}

Let $f(x)=\left\lbrace\begin{array}{clc}
\sage{f3} & , & x\leq \sage{a}\\
\sage{f1} & , & \sage{a}<x\leq\sage{b}\\
\sage{f2} & , & x>\sage{b}
\end{array}\right.$.  

Find the numbers at which $f$ is discontinuous: $x=\answer{\sage{a}}$\\ \qquad (If no such numbers exist, enter ``None'')

\input{Limit-Concept-0004.HELP.tex}

At which of these points of discontinuity is $f$ continuous from the right?\\  $x=\answer[format=string]{None}$ \qquad (If no such numbers exist, enter ``None'')

At which of these points of discontinuity is $f$ continuous from the left?\\  $x=\answer{\sage{a}}$ \qquad (If no such numbers exist, enter ``None'')
\end{problem}}%}
%%%%%%%%%%%%%%%%%%%%%%



%%%%%%%%%%%%%%%%%%%%%%%
%%\tagged{Ans@MultiAns, Type@Concept, Topic@Limit, Sub@Continuity, Sub@Disc-Jump, Sub@Piecewise, File@0005}{
\begin{sagesilent}
a = Integer(randint(-3,3))
b = NonZeroInt(a+2,a+5)

c = RandInt(-6,6)
d = RandInt(-6,6)

intp1 = RandInt(0,3)
intp2 = NonZeroInt(0,3,[intp1])
nonintp=RandInt(0,3)

integerfunc=[x-c, (x-c)^2, expand((x-c)*(x-d)), c-x]
nonintfunc=[e^x, e^(x-b), sin(x), cos(x)]

f1 = integerfunc[intp1]
f2 = nonintfunc[nonintp]
f3 = integerfunc[intp2]


\end{sagesilent}

\latexProblemContent{
\ifVerboseLocation This is Limit Concept Question 0005. \\ \fi
\begin{problem}

Let $f(x)=\left\lbrace\begin{array}{ll}
\sage{f1} , & x< \sage{a}\\
\sage{f2} , & \sage{a}\leq x\leq\sage{b}\\
\sage{f3} , & x>\sage{b}
\end{array}\right.$.  

Find the numbers at which $f$ is discontinuous (list your answers from lowest to highest and enter "None" in any box remaining after all answers are entered): 

$x=\answer{\sage{a}}$ \qquad $x=\answer{\sage{b}}$ 


\input{Limit-Concept-0005.HELP.tex}

At which of these points of discontinuity is $f$ continuous from the right? (List your answers from lowest to highest and enter "None" in any box remaining after all answers are entered): 

$x=\answer{\sage{a}}$ \qquad $x=\answer[format=string]{None}$

At which of these points of discontinuity is $f$ continuous from the left? (List your answers from lowest to highest and enter "None" in any box remaining after all answers are entered): 

$x=\answer{\sage{b}}$ \qquad $x=\answer[format=string]{None}$

\end{problem}}%}
%%%%%%%%%%%%%%%%%%%%%%


%%%%%%%%%%%%%%%%%%%%%%%
%%\tagged{Ans@MC, Type@Concept, Topic@Limit, Sub@Continuity, Sub@Theorems-IVT, Sub@Poly, File@0006}{
\begin{sagesilent}
check = 0;
while check == 0:
    a = RandInt(1,3)
    r1 = RandInt(-5,5)
    shift1 = RandInt(2,4)
    shift2 = RandInt(2,4)
    r2 = r1 + shift1
    r3 = r1 - shift2
    pwr1 = RandInt(0,2)
    
    start = r1-RandInt(1,shift2-1)
    end = r1+RandInt(1,shift1-1)
    
    F=expand(a*(x-r3)^pwr1*(x-r1)*(x-r2))
    checktmp = 1
    for coef in F.list():
        if coef  > 400:
            checktmp = 0
    check = checktmp
        
\end{sagesilent}

\latexProblemContent{
\ifVerboseLocation This is Limit Concept Question 0006. \\ \fi
\begin{problem}

Does $\sage{F}$ have a root in the interval $(\sage{start},\sage{end})$?  

\input{Limit-Concept-0006.HELP.tex}

\begin{multipleChoice}
\choice[correct]{Yes}
\choice{No}
\choice{Inconclusive}
\end{multipleChoice}

What is the reason for this answer?\begin{multipleChoice}
\choice{$f(\sage{start})<0$ and $f(\sage{end})>0$, so $f$ has a root by IVT.}
\choice[correct]{$f(\sage{start})>0$ and $f(\sage{end})<0$, so $f$ has a root by IVT.}
\choice{$f(\sage{start})<0$ and $f(\sage{end})<0$, so $f$ does not have a root by IVT.}
\choice{$f(\sage{start})>0$ and $f(\sage{end})>0$, so $f$ does not have a root by IVT.}
\choice{The answer could not be determined.}
\end{multipleChoice}
\end{problem}}%}
%%%%%%%%%%%%%%%%%%%%%%

%%%%%%%%%%%%%%%%%%%%%%%
%%\tagged{Ans@MC, Type@Concept, Topic@Limit, Sub@Continuity, Sub@Theorems-IVT, Sub@Poly, Sub@Trig, File@0007}{
\begin{sagesilent}
a = RandInt(-8,8)
b = RandInt(2,18)

trigVec = [
   sin(x), 
   cos(x)
]
polyVec = [
   a+b-x, 
   x-a
]

p = choice([0, 1])

F = trigVec[p].subs(x = x-a)
G = polyVec[p]
\end{sagesilent}

\latexProblemContent{
\ifVerboseLocation This is Limit Concept Question 0007. \\ \fi
\begin{problem}

Does the equation $\sage{F}=\sage{G}$ have a solution in the interval $(\sage{a},\sage{a+b})$?  

\input{Limit-Concept-0007.HELP.tex}

\begin{multipleChoice}
\choice[correct]{Yes}
\choice{No}
\choice{Inconclusive}
\end{multipleChoice}
\end{problem}}%}
%%%%%%%%%%%%%%%%%%%%%%

%% SO FAR 13 Compute, 7 Concept


%%%%%%%%%%%%%%%%%%%%%
%\tagged{Ans@MultiAns, Type@Concept, Topic@Limit, Sub@Trig, Sub@Squeeze, File@0008}{
\begin{sagesilent}
a = NonZeroInt(-25,25)
b = RandInt(-15,15)
p=Integer(randint(0,5))
v=[(x-b)*cos(a/(x-b)), (x-b)^2*cos(a/(x-b)), (x-b)^3*cos(a/(x-b)), (x-b)*cos(a/(x-b)^2), (x-b)^2*cos(a/(x-b)^2), (x-b)^3*cos(a/(x-b)^2)]
F=v[p]

Ans=limit(F,x=0)
\end{sagesilent}

\latexProblemContent{
\ifVerboseLocation This is Limit Concept Question 0008. \\ \fi
\begin{problem}

The limit as $x$ approaches $\sage{b}$ of $f(x)=\sage{F}$ is $0$.  What is the reason why this is true?

\input{Limit-Concept-0008.HELP.tex}

\begin{multipleChoice}
\choice{The statement is in fact false: $\lim\limits_{x\to\sage{b}}\sage{F}\neq0$.}
\choice{The cosine factor decreases to $0$ faster than the polynomial.}
\choice[correct]{The cosine factor is bounded between $-1$ and $1$, so the polynomial forces the function to $0$.}
\choice{The cosine factor directly cancels out the polynomial factor.}
\end{multipleChoice}


What is the name of the theorem that applies to this problem? \qquad \\
The $\underline{\answer[format=string]{Squeeze}}$ Theorem
\end{problem}}%}
%%%%%%%%%%%%%%%%%%%%%
%%%%%%%%%%%%%%%%%%%%%%%
%%\tagged{Ans@ShortAns, Type@Compute, Topic@Limit, Func@Rational, File@1000}{
\begin{sagesilent}
numCons = NonZeroInt(-10,10)
num = x - numCons

denCons = NonZeroInt(-10,10, [0, numCons])
den = x - denCons

target = denCons

dir = choice(["-", "+"])

if dir == "-":
    # Approaching target from the left, so denominator will go to 0^-
    denSign = -1
else:
    # Approaching target from the right, so denominator will go to 0^+
    denSign = +1

numLimit = num.subs(x = target)
assert numLimit != 0, "Something went wrong!"
if numLimit > 0:
    ans = denSign * infinity
else:
    ans = denSign * infinity * -1

\end{sagesilent}

\latexProblemContent{
\ifVerboseLocation This is Limit Compute Question 1000. \\ \fi
\begin{problem}

Determine the limit.

\input{Limit-Compute-1000.HELP.tex}

\[
    \lim_{x \to \sage{target}^{\sage{dir}}} \frac{\sage{num}}{\sage{den}}
    =
    \answer{\sage{ans}}
\]

\end{problem}}%}
%%%%%%%%%%%%%%%%%%%%%%

%%%%%%%%%%%%%%%%%%%%%%%
%%\tagged{Ans@ShortAns, Type@Compute, Topic@Limit, Func@Rational, File@1001}{
\begin{sagesilent}
numCons = NonZeroInt(-10,10)
num = numCons - x

denCons = NonZeroInt(-10,10, [0, numCons])
den = (x - denCons)^2

target = denCons

numLimit = num.subs(x = target)
assert numLimit != 0, "Something went wrong!"
if numLimit > 0:
    ans = infinity
else:
    ans = infinity * -1

\end{sagesilent}

\latexProblemContent{
\ifVerboseLocation This is Limit Compute Question 1001. \\ \fi
\begin{problem}

Determine the limit.

\input{Limit-Compute-1001.HELP.tex}

\[
    \lim_{x \to \sage{target}} \frac{\sage{numCons} - x}{\sage{den}}
    =
    \answer{\sage{ans}}
\]

\end{problem}}%}
%%%%%%%%%%%%%%%%%%%%%%

%%%%%%%%%%%%%%%%%%%%%%%
%%\tagged{Ans@ShortAns, Type@Compute, Topic@Limit, Func@Log, File@1002}{
\begin{sagesilent}

smallRoot = RandInt(-10,10)
bigRoot   = RandInt(smallRoot,10)

assert smallRoot <= bigRoot

inner = (x - smallRoot)*(x - bigRoot)
target = bigRoot

func = log(inner)

ans = -infinity

\end{sagesilent}

\latexProblemContent{
\ifVerboseLocation This is Limit Compute Question 1002. \\ \fi
\begin{problem}

Determine the limit.

\input{Limit-Compute-1002.HELP.tex}

\[
    \lim_{x \to \sage{target}^{+}} \sage{func}
    =
    \answer{\sage{ans}}
\]

\end{problem}}%}
%%%%%%%%%%%%%%%%%%%%%%

%%%%%%%%%%%%%%%%%%%%%%%
%%\tagged{Ans@ShortAns, Type@Compute, Topic@Limit, Func@Trig, File@1003}{
\begin{sagesilent}

trigVec = [
    tan(x),
    sec(x),
    csc(x),
    cot(x)
]
trig = choice(trigVec)

pwr = RandInt(1,5)
func = x^pwr * trig

multiple = RandInt(1,5)
if trig in [tan(x), sec(x)]:
    target = latex(2 * multiple) + latex(pi)
elif trig in [csc(x), cot(x)]:
    target = latex(4 * multiple + 1) + latex(pi) + "/2"
else:
    raise Exception

dirVec = ["-", "+"]
dir = choice(dirVec)

if dir == "-":
    ans = -infinity
elif dir == "+":
    ans = +infinity
else:
    raise Exception

\end{sagesilent}

\latexProblemContent{
\ifVerboseLocation This is Limit Compute Question 1003. \\ \fi
\begin{problem}

Determine the limit.

\input{Limit-Compute-1003.HELP.tex}

\[
    \lim_{x \to \sage{target}^{\sage{dir}}} \sage{func}
    =
    \answer{\sage{ans}}
\]

\end{problem}}%}
%%%%%%%%%%%%%%%%%%%%%%

%%%%%%%%%%%%%%%%%%%%%%%
%%\tagged{Ans@ShortAns, Type@Compute, Topic@Limit, Func@Rational, Func@Log, Sub@Asymptote, File@1004}{
\begin{sagesilent}

shift = RandInt(0,9)
inner = x - shift
target = shift

sign = choice([-1, 1])
coef = sign * RandInt(1,9)

func = coef / inner - sign * log(inner)

ans = sign * infinity

\end{sagesilent}

\latexProblemContent{
\ifVerboseLocation This is Limit Compute Question 1004. \\ \fi
\begin{problem}

Determine the limit.

\input{Limit-Compute-1004.HELP.tex}

\[
    \lim_{x \to \sage{target}^{+}} \sage{func}
    =
    \answer{\sage{ans}}
\]

\end{problem}}%}
%%%%%%%%%%%%%%%%%%%%%%

%%%%%%%%%%%%%%%%%%%%%%%
%%\tagged{Ans@ShortAns, Type@Compute, Topic@Limit, Func@Rational, Sub@Asymptote, File@1005}{
\begin{sagesilent}

sign = choice([-1,1])
coef = RandInt(1,10)
target = RandInt(1,5)

num = coef * sign
den = x^3 - target^3

\end{sagesilent}

\latexProblemContent{
\ifVerboseLocation This is Limit Compute Question 1005. \\ \fi
\begin{problem}

Evaluate the function for values of $x$ that approach $\sage{target}$ from the left and from the right.

\input{Limit-Compute-1005.HELP.tex}

\[
    \lim_{x \to \sage{target}^{-}} \frac{\sage{num}}{\sage{den}}
    =
    \answer{\sage{-sign * infinity}}
\]
\[
    \lim_{x \to \sage{target}^{+}} \frac{\sage{num}}{\sage{den}}
    =
    \answer{\sage{sign * infinity}}
\]

\end{problem}}%}
%%%%%%%%%%%%%%%%%%%%%%

%%%%%%%%%%%%%%%%%%%%%%%
%%\tagged{Ans@ShortAns, Type@Compute, Topic@Limit, Func@Exp, Func@Rational, Sub@Asymptote, File@1006}{
\begin{sagesilent}

numCoef = RandInt(1,9)
numCons = RandInt(1,9)
num = numCoef * e^x + numCons

denRoot = RandInt(1,9)
pwr = 2 * RandInt(0,4) + 1
den = (x - denRoot)^pwr

target = denRoot

dir = choice(["-", "+"])
if dir == "-":
    ans = -infinity
elif dir == "+":
    ans = infinity
else:
    raise Exception

\end{sagesilent}

\latexProblemContent{
\ifVerboseLocation This is Limit Compute Question 1006. \\ \fi
\begin{problem}

Determine the limit.

\input{Limit-Compute-1006.HELP.tex}

\[
    \lim_{x \to \sage{target}^{\sage{dir}}} \frac{\sage{num}}{\sage{den}}
    =
    \answer{\sage{ans}}
\]

\end{problem}}%}
%%%%%%%%%%%%%%%%%%%%%%

%%%%%%%%%%%%%%%%%%%%%%%
%%\tagged{Ans@ShortAns, Type@Compute, Topic@Limit, Func@Poly, File@1007}{
\begin{sagesilent}

deg = RandInt(2,5)
ans = 100

while abs(ans) >= 100:
    func = RandPoly(deg)
    target = RandInt(-5,5)
    ans = func.subs(x = target)

\end{sagesilent}

\latexProblemContent{
\ifVerboseLocation This is Limit Compute Question 1007. \\ \fi
\begin{problem}

Evaluate the limit using the appropriate Limit Law(s). (If an answer does not exist, enter DNE.)

\input{Limit-Compute-1007.HELP.tex}

\[
    \lim_{x \to \sage{target}} \sage{func}
    =
    \answer{\sage{ans}}
\]

\end{problem}}%}
%%%%%%%%%%%%%%%%%%%%%%

%%%%%%%%%%%%%%%%%%%%%%%
%%\tagged{Ans@ShortAns, Type@Compute, Topic@Limit, Func@Poly, Func@Rational, File@1008}{
\begin{sagesilent}

numDeg = RandInt(2,3)
denDeg = RandInt(2,3)

numLim = 100
numDen = 100

while abs(numLim) >= 100 or abs(denLim) >= 100:
    target = RandInt(-5,5)
    num = RandPoly(numDeg, avoid=[target], expanded=True)
    den = RandPoly(denDeg, avoid=[target], expanded=True)

    numLim = num.subs(x = target)
    denLim = den.subs(x = target)

ans = numLim / denLim

\end{sagesilent}

\latexProblemContent{
\ifVerboseLocation This is Limit Compute Question 1008. \\ \fi
\begin{problem}

Evaluate the limit using the appropriate Limit Law(s). (If an answer does not exist, enter DNE.)

\input{Limit-Compute-1008.HELP.tex}

\[
    \lim_{x \to \sage{target}} \frac{\sage{num}}{\sage{den}}
    =
    \answer{\sage{ans}}
\]

\end{problem}}%}
%%%%%%%%%%%%%%%%%%%%%%

%%%%%%%%%%%%%%%%%%%%%%%
%%\tagged{Ans@ShortAns, Type@Compute, Topic@Limit, Func@Radical, Func@Poly, File@1009}{
\begin{sagesilent}

deg = RandInt(2,4)
ans = 10

while abs(ans) >= 10:
    target = RandInt(-9,9)
    func = RandPoly(deg)
    ans2 = func.subs(x = target)
    if ans2 < 0:
        ans = 100
    else:
        ans = sqrt(ans2)

\end{sagesilent}

\latexProblemContent{
\ifVerboseLocation This is Limit Compute Question 1009. \\ \fi
\begin{problem}

Evaluate the limit using the appropriate Limit Law(s). (If an answer does not exist, enter DNE.)

\input{Limit-Compute-1009.HELP.tex}

\[
    \lim_{x \to \sage{target}} \sqrt{\sage{func}}
    =
    \answer{\sage{ans}}
\]

\end{problem}}%}
%%%%%%%%%%%%%%%%%%%%%%

%%%%%%%%%%%%%%%%%%%%%%%
%%\tagged{Ans@ShortAns, Type@Compute, Topic@Limit, Func@Radical, Func@Poly, File@1010}{
\begin{sagesilent}
ans = 1000

while abs(ans) >= 1000:
    target = choice([-8, -1, 1, 8])

    deg = RandInt(1,3)
    poly = RandPoly(deg)

    cons = RandInt(-9,9)
    func = (cons + x^(1/3)) * poly

    ans = func.subs(x = target)

\end{sagesilent}

\latexProblemContent{
\ifVerboseLocation This is Limit Compute Question 1010. \\ \fi
\begin{problem}

Evaluate the limit using the appropriate Limit Law(s). (If an answer does not exist, enter DNE.)

\input{Limit-Compute-1010.HELP.tex}

\[
    \lim_{x \to \sage{target}} \sage{func}
    =
    \answer{\sage{ans}}
\]

\end{problem}}%}
%%%%%%%%%%%%%%%%%%%%%%

%%%%%%%%%%%%%%%%%%%%%%%
%%\tagged{Ans@ShortAns, Type@Compute, Topic@Limit, Func@Rational, Func@Abs, File@1011}{
\begin{sagesilent}

target = RandInt(-9,9)

coef = NonZeroInt(-5,5)

num = coef * (x - target)
den = abs(x - target)

# ans = "DNE"
\end{sagesilent}

\latexProblemContent{
\ifVerboseLocation This is Limit Compute Question 1011. \\ \fi
\begin{problem}

Evaluate the limit using the appropriate Limit Law(s). (If an answer does not exist, enter DNE.)

\input{Limit-Compute-1011.HELP.tex}

\[
    \lim_{x \to \sage{target}} \frac{\sage{num}}{\sage{den}}
    =
    \answer[format=string]{DNE}
\]

\end{problem}}%}
%%%%%%%%%%%%%%%%%%%%%%

%%%%%%%%%%%%%%%%%%%%%%%
%%\tagged{Ans@ShortAns, Type@Compute, Topic@Limit, Func@Rational, Func@Abs, File@1012}{
\begin{sagesilent}

target = RandInt(-10,-1)

numCoef = NonZeroInt(1,9)
denCoef = NonZeroInt(1,9)

num = numCoef * (-target) - abs(numCoef * x)
den = denCoef * (-target + x)

ans = numCoef / denCoef
\end{sagesilent}

\latexProblemContent{
\ifVerboseLocation This is Limit Compute Question 1012. \\ \fi
\begin{problem}

Evaluate the limit using the appropriate Limit Law(s). (If an answer does not exist, enter DNE.)

\input{Limit-Compute-1012.HELP.tex}

\[
    \lim_{x \to \sage{target}} \frac{\sage{num}}{\sage{den}}
    =
    \answer{\sage{ans}}
\]

\end{problem}}%}
%%%%%%%%%%%%%%%%%%%%%%

%%%%%%%%%%%%%%%%%%%%%%%
%%\tagged{Ans@ShortAns, Type@Compute, Topic@Limit, Func@Rational, Func@Poly, Sub@Disc-Remove, File@1013}{
\begin{sagesilent}

disc = RandInt(-9,9) # Discontinuity
other = NonZeroInt(-9,9,[disc])

den = x - disc
num = expand((x - disc)*(x - other))

ans = other - disc

\end{sagesilent}

\latexProblemContent{
\ifVerboseLocation This is Limit Compute Question 1013. \\ \fi
\begin{problem}

How would you "remove the discontinuity" of $f$? In other words, how would you define $f(\sage{disc})$ in order to make $f$ continuous at $\sage{disc}$?

\input{Limit-Compute-1013.HELP.tex}

\[
   f(x)
   =
   \frac{\sage{num}}{\sage{den}}
\]

$
f(\sage{disc})
=
\answer{\sage{ans}}
$

\end{problem}}%}
%%%%%%%%%%%%%%%%%%%%%%

%%%%%%%%%%%%%%%%%%%%%%%
%%\tagged{Ans@ShortAns, Type@Compute, Topic@Limit, Func@Radical, Func@Poly, File@1014}{
\begin{sagesilent}

ans = 100

while abs(ans) >= 100:
    target = RandInt(-9,9)

    innerRoot = RandInt(-5,5)^2 # The inside of the radical will equal this.
    innerCons = innerRoot + target^2
    inner = innerCons - x^2

    outerCons = RandInt(-9,9)
    outerCoef = NonZeroInt(-9,9)
    outer = outerCons + outerCoef * x

    func = outer * inner^(1/2)
    ans = func(x = target)

\end{sagesilent}

\latexProblemContent{
\ifVerboseLocation This is Limit Compute Question 1014. \\ \fi
\begin{problem}

Use continuity to evaluate the limit.

\input{Limit-Compute-1014.HELP.tex}

\[
    \lim_{x \to \sage{target}} \left(\sage{outer}\right) \sqrt{\sage{innerCons}-x^2}
    =
    \answer{\sage{ans}}
\]

\end{problem}}%}
%%%%%%%%%%%%%%%%%%%%%%

%%%%%%%%%%%%%%%%%%%%%%%
%%\tagged{Ans@ShortAns, Type@Compute, Topic@Limit, Func@Exp, Func@Poly, File@1015}{
\begin{sagesilent}

ans = 100

while abs(ans) >= 100:
    target = 1

    poly = 0
    for _ in range(3):
        pwr = RandInt(0,5)
        coef = NonZeroInt(-9,9)
        poly += coef * x^pwr

    func = e^poly
    ans = func.subs(x = 1)

\end{sagesilent}

\latexProblemContent{
\ifVerboseLocation This is Limit Compute Question 1015. \\ \fi
\begin{problem}

Use continuity to evaluate the limit.

\input{Limit-Compute-1015.HELP.tex}

\[
    \lim_{x \to 1} \sage{func}
    =
    \answer{\sage{ans}}
\]

\end{problem}}%}
%%%%%%%%%%%%%%%%%%%%%%                                                                                                                  