%%%%%%  This is the raw list of questions before processing %%%%%%




%%%%%%%%%%%%%%%%%%%%%%%%%%%%%%%%%%%%%%%%%%%%%%%%%%%%%%%%%%%%%%%%%%%%%%%%%%%%%%%
%%%%%%%%%%%%%%%%%%%%%%%%%%%%%%%%%%%%%%%%%%%%%%%%%%%%%%%%%%%%%%%%%%%%%%%%%%%%%%%
%%%%%%%%%%%%%%%%%%%										%%%%%%%%%%%%%%%%%%%%%%%
%%%%%%%%%%%%%%%%%%%				Computation				%%%%%%%%%%%%%%%%%%%%%%%
%%%%%%%%%%%%%%%%%%%										%%%%%%%%%%%%%%%%%%%%%%%
%%%%%%%%%%%%%%%%%%%%%%%%%%%%%%%%%%%%%%%%%%%%%%%%%%%%%%%%%%%%%%%%%%%%%%%%%%%%%%%
%%%%%%%%%%%%%%%%%%%%%%%%%%%%%%%%%%%%%%%%%%%%%%%%%%%%%%%%%%%%%%%%%%%%%%%%%%%%%%%



%%%%%%%%%%%%%%%%%%%%%%%%%%%%%%%%%%%%%%%%%%%%%%%%%%%%%%%%%%%%%%%%%%%%%%%%%%%%%%%
%%%%%%%%%%%%%%%%%%%			MAC2311: Calculus 1			%%%%%%%%%%%%%%%%%%%%%%%
%%%%%%%%%%%%%%%%%%%%%%%%%%%%%%%%%%%%%%%%%%%%%%%%%%%%%%%%%%%%%%%%%%%%%%%%%%%%%%%




%%%%%%%%%%%%%%%%%%%%%
%\tagged{Ans@ShortAns, Type@Compute, Topic@Limit, Sub@Rational, Sub@Asymptote, File@0001}{
\begin{sagesilent}
a = NonZeroInt(-4,4)
b = NonZeroInt(-4,4, [0,a])

fNum=expand((x-a)*(x-b))
fDen=(x-a)
ans=limit(fNum/fDen, x=a)
\end{sagesilent}

\latexProblemContent{
\begin{problem}
Determine if the limit approaches a finite number, $\pm\infty$, or does not exist. (If the limit does not exist, write DNE)

\input{Limit-Compute-0001.HELP.tex}

\[\lim_{x\to\sage{a}}\dfrac{\sage{fNum}}{\sage{fDen}}=\answer{\sage{ans}}\]
\end{problem}}%}
%%%%%%%%%%%%%%%%%%%%%


%%%%%%%%%%%%%%%%%%%%%
%\tagged{Ans@ShortAns, Type@Compute, Topic@Limit, Sub@Rational, File@0002}{
\begin{sagesilent}
a = NonZeroInt(-4,4)
b = NonZeroInt(-4,4, [0,a])

fDen=expand((x-a)*(x-b))
fNum=(x-a)
Ans=limit(fNum/fDen, x=a)
#fans=1/(a-b)
\end{sagesilent}

\latexProblemContent{
\begin{problem}

Determine if the limit approaches a finite number, $\pm\infty$, or does not exist. (If the limit does not exist, write DNE)

\input{Limit-Compute-0002.HELP.tex}

\[\lim_{x\to\sage{a}}\dfrac{\sage{fNum}}{\sage{fDen}}=\answer{\sage{Ans}}\]
\end{problem}}%}
%%%%%%%%%%%%%%%%%%%%%%



%%%%%%%%%%%%%%%%%%%%%%
%\tagged{Ans@ShortAns, Type@Compute, Topic@Limit, Sub@Rational, File@0003}{
\begin{sagesilent}
a = NonZeroInt(-4,4)
b = NonZeroInt(-4,4, [0,a])

fNum=expand((x-a)*(x-b))
fDen=(x-a)
Ans = limit(fNum/fDen, x=b)
\end{sagesilent}

\latexProblemContent{
\begin{problem}

Determine if the limit approaches a finite number, $\pm\infty$, or does not exist. (If the limit does not exist, write DNE)

\input{Limit-Compute-0003.HELP.tex}

\[\lim_{x\to\sage{b}}\dfrac{\sage{fNum}}{\sage{fDen}}=\answer{\sage{Ans}}\]
\end{problem}}%}
%%%%%%%%%%%%%%%%%%%%%%


%%%%%%%%%%%%%%%%%%%%%%
%\tagged{Ans@ShortAns, Type@Compute, Topic@Limit, Sub@Rational, Sub@Asymptote, File@0004}{
\begin{sagesilent}
a = NonZeroInt(-4,4)
b = NonZeroInt(-4,4, [0,a])

fDen=expand((x-a)*(x-b))
fNum=(x-a)
\end{sagesilent}

\latexProblemContent{
\begin{problem}

Determine if the limit approaches a finite number, $\pm\infty$, or does not exist. (If the limit does not exist, write DNE)

\input{Limit-Compute-0004.HELP.tex}

\[\lim_{x\to\sage{b}}\dfrac{\sage{fNum}}{\sage{fDen}}=\answer{DNE}\]
\end{problem}}%}
%%%%%%%%%%%%%%%%%%%%%


%%%%%%%%%%%%%%%%%%%%%%
%\tagged{Ans@ShortAns, Type@Compute, Topic@Limit, Sub@Rational, File@0005}{
\begin{sagesilent}
a = NonZeroInt(-4,4)
b = NonZeroInt(-4,4, [0,a])
c = NonZeroInt(-5,5)

fNum=expand((x-a)*(x-c))
fDen=expand((x-a)*(x-b))
Ans=limit(fNum/fDen, x=a)
#fans=(a-c)/(a-b)
\end{sagesilent}

\latexProblemContent{
\begin{problem}

Determine if the limit approaches a finite number, $\pm\infty$, or does not exist. (If the limit does not exist, write DNE)

\input{Limit-Compute-0005.HELP.tex}

\[\lim_{x\to\sage{a}}\dfrac{\sage{fNum}}{\sage{fDen}}=\answer{\sage{Ans}}\]
\end{problem}}%}
%%%%%%%%%%%%%%%%%%%%%


%%%%%%%%%%%%%%%%%%%%%%
%\tagged{Ans@ShortAns, Type@Compute, Topic@Limit, Sub@Rational, File@0006}{
\begin{sagesilent}
a = NonZeroInt(-5,5)
b = NonZeroInt(-5,5, [a])

fNum=expand((x-a)*(x^2+a*x+a^2))
fDen=expand((x-a)*(x-b))
Ans=limit(fNum/fDen, x=a)
#fans=(3*a^2)/(a-b)
\end{sagesilent}

\latexProblemContent{
\begin{problem}

Determine if the limit approaches a finite number, $\pm\infty$, or does not exist. (If the limit does not exist, write DNE)

\input{Limit-Compute-0006.HELP.tex}

\[\lim_{x\to\sage{a}}\dfrac{\sage{fNum}}{\sage{fDen}}=\answer{\sage{Ans}}\]
\end{problem}}%}
%%%%%%%%%%%%%%%%%%%%%



%%%%%%%%%%%%%%%%%%%%%%
%\tagged{Ans@ShortAns, Type@Compute, Topic@Limit, Sub@Rational, \Sub@Asymptote, File@0007}{
\begin{sagesilent}
a = NonZeroInt(-5,5)
b = NonZeroInt(-5,5, [0,a])
c = NonZeroInt(-5,5)
d = NonZeroInt(-5,5, [0,c])
e = NonZeroInt(-5,5, [0,b])
   
v=[(x-a),(x-b),(x-c),(x-d),(x-e)]
p=NonZeroInt(1,3)
q=NonZeroInt(1,4)
while p>>q:
   q=NonZeroInt(1,4)
   
f=1
for j in [1..p]:
   k=Integer(randint(0,4))
   f=v[k]*f

g=1
for j in [1..q]:
   k=Integer(randint(0,4))
   g=v[k]*g

fNum=expand(f)
fDen=expand(g)
Ans=limit(fNum/fDen,x=infinity)

I=NonZeroInt(-1,1)
Inf=I*infinity

\end{sagesilent}

\latexProblemContent{
\begin{problem}

Determine if the limit approaches a finite number, $\pm\infty$, or does not exist. (If the limit does not exist, write DNE)

\input{Limit-Compute-0007.HELP.tex}

\[\lim_{x\to\sage{Inf}}\dfrac{\sage{fNum}}{\sage{fDen}}=\answer{\sage{Ans}}\]

\end{problem}}%}
%%%%%%%%%%%%%%%%%%%%%


%%%%%%%%%%%%%%%%%%%%%%%
%%\tagged{Ans@FRQ, Type@Compute, Topic@Limit, Sub@Rational, Sub@Continuity, File@0008}{
\begin{sagesilent}
a = RandInt(1,5)
b = NonZeroInt(-10,10, [0,a])
r=a-b   
c=NonZeroInt(-5,5, [0,r])

vleft=[b/2*(x+a)/a,expand((x-c)*(x-r))/(x-c),sqrt(x^2-a^2)+b]
vright=[b/2*(x+a)/a,expand((x-c)*(x-r))/(x-c)]

first = RandInt(0,2)
second = NonZeroInt(0,1, [first])

Fone=vleft[first]
Ftwo=vright[second]
\end{sagesilent}

\latexProblemContent{
\begin{problem}


Use the function to answer the following questions.
\[f(x)=\left\{\begin{array}{ll}\sage{Fone}&,x<\sage{a}\\[5pt] 
\sage{Ftwo}&,x\geq\sage{a} 
\end{array}\right.\]

\input{Limit-Compute-0008.HELP.tex}


Compute $\lim\limits_{x\to\sage{a}^-}f(x)=\answer{\sage{b}}$\\[1in]

Compute $\lim\limits_{x\to\sage{a}^+}f(x)=\answer{\sage{b}}$\\[1in]

Compute $f(\sage{a})=\answer{\sage{b}}$\\[1in]

The function is ...
\begin{multipleChoice}
\choice[correct]{continuous at $x=\sage{a}$.}
\choice{discontinuous at $x=\sage{a}$.}
\end{multipleChoice}

\end{problem}}%}
%%%%%%%%%%%%%%%%%%%%%%


%%%%%%%%%%%%%%%%%%%%%%%
%%\tagged{Ans@ShortAns, Type@Compute, Topic@Limit, Sub@Poly, Sub@Trig, Sub@Continuity, File@0009}{
\begin{sagesilent}
a = NonZeroInt(-5,5)
b = NonZeroInt(-5,5, [0,a])
c = NonZeroInt(-5,5)
d=NonZeroInt(-5,5, [0,a])
   
vleft=[a*sin(b*x*pi), a*cos(b*x*pi), a*tan(b*x*pi), a*sin(b*x*pi/2), a*cos(b*x*pi/2), a*tan(b*x*pi/2), a*sin(b*x*pi/6), a*cos(b*x*pi/6), a*tan(b*x*pi/6), a*sin(b*x*pi/3), a*cos(b*x*pi/3), a*tan(b*x*pi/3)]

vright=[(x-a),(x-a)*(x-b),(x-a)*(x-b)*(x-c),expand((x-a)*(x-b)),expand((x-a)*(x-b)*(x-c))]

first = Integer(randint(0,11))
second = Integer(randint(0,4))

Fone=vleft[first]
Ftwo=vright[second]
F=Fone*Ftwo

Ans=limit(F,x=d)

\end{sagesilent}

\latexProblemContent{
\begin{problem}

Determine if the limit approaches a finite number, $\pm\infty$, or does not exist. (If the limit does not exist, write DNE)

\input{Limit-Compute-0009.HELP.tex}

\[\lim_{x\to\sage{d}}\sage{F}=\answer{\sage{Ans}}\]

\end{problem}}%}
%%%%%%%%%%%%%%%%%%%%%%

%%%%%%%%%%%%%%%%%%%%%%%
%%\tagged{Ans@ShortAns, Type@Compute, Topic@Limit, Sub@Rational, File@0010}{
\begin{sagesilent}
a = NonZeroInt(-5,5)
Fone=(1/a)+(1/x)
Ftwo=a+x
F=Fone/Ftwo

Ans=limit(F,x=-a)
\end{sagesilent}

\latexProblemContent{
\begin{problem}

Compute the following derivative:

\input{Limit-Compute-0010.HELP.tex}

\[\lim_{x\to\sage{-a}}\dfrac{\sage{Fone}}{\sage{Ftwo}}=\answer{\sage{Ans}}\]
\end{problem}}%}
%%%%%%%%%%%%%%%%%%%%%%


%%%%%%%%%%%%%%%%%%%%%%%
%%\tagged{Ans@ShortAns, Type@Compute, Topic@Limit, Sub@Asymptote, Sub@OneSided, File@0011}{
\begin{sagesilent}
a = NonZeroInt(-5,5)
p=Integer(randint(1,9))
v=[a*x*ln(x),a/x, a/x^2, a/x^3, a/x^p]
first = Integer(randint(0,4))
F=v[first]
Ans=limit(F,x=0,dir='plus')
\end{sagesilent}

\latexProblemContent{
\begin{problem}

Determine if the limit approaches a finite number, $\pm\infty$, or does not exist. (If the limit does not exist, write DNE)

\input{Limit-Compute-0011.HELP.tex}

\[\lim_{x\to 0^+}\sage{F}=\answer{\sage{Ans}}\]

\end{problem}}%}
%%%%%%%%%%%%%%%%%%%%%%


%%%%%%%%%%%%%%%%%%%%%%%
%%\tagged{Ans@ShortAns, Type@Compute, Topic@Limit, Sub@Asymptote, Sub@Radical, File@0012}{
\begin{sagesilent}
a = NonZeroInt(-5,5)
b = NonZeroInt(-5,5)
c = NonZeroInt(-5,5)   
d=NonZeroInt(1,5, [a])
e=NonZeroInt(-5,5) 
m=NonZeroInt(2,6,[3,5])  

F=(sqrt(d*x^m+a)-b)/(e*x^(m/2)+c)

Ans=limit(F,x=infinity)
\end{sagesilent}

\latexProblemContent{
\begin{problem}

Determine if the limit approaches a finite number, $\pm\infty$, or does not exist. (If the limit does not exist, write DNE)

\input{Limit-Compute-0012.HELP.tex}

\[\lim_{x\to\infty}\sage{F}=\answer{\sage{Ans}}\]

\end{problem}}%}
%%%%%%%%%%%%%%%%%%%%%%

%%%%%%%%%%%%%%%%%%%%%%%
%%\tagged{Ans@ShortAns, Type@Compute, Topic@Limit, Sub@DifferenceQuotient, Sub@Poly, File@0013}{
\begin{sagesilent}
a = NonZeroInt(-5,5)
b = NonZeroInt(-4,4)
p=Integer(randint(0,3))
if p==0:
   b = NonZeroInt(1,4)

v=[a*sqrt(x),a/x, a*x^2, a*x^3]

F=v[p]
Fone=F(b+h)
Ftwo=F(b)
DQ=(F(b+h)-F(b))/h

Ans=limit(DQ,h=0)
\end{sagesilent}

\latexProblemContent{
\begin{problem}

Determine if the limit approaches a finite number, $\pm\infty$, or does not exist. (If the limit does not exist, write DNE)

\input{Limit-Compute-0013.HELP.tex}

\[\lim_{h\to0}\frac{\sage{Fone-Ftwo}}{h}=\answer{\sage{Ans}}\]

\end{problem}}%}
%%%%%%%%%%%%%%%%%%%%%%

%%%%%%%%%%%%%%%%%%%%%
%\tagged{Ans@ShortAns, Type@Compute, Topic@Limit, Sub@Trig, File@0014}{
\begin{sagesilent}
a = NonZeroInt(-5,5)
b = NonZeroInt(1,5,[0,a])

p=Integer(randint(0,5))
v=[sin(a*x)/(b*x), (a*x)/sin(b*x), tan(a*x)/(b*x), (a*x)/tan(b*x), (1-cos(a*x))/(b*x), (a*x)/(1-cos(b*x))]

F=v[p]
Ans=limit(F,x=0)
\end{sagesilent}

\latexProblemContent{
\begin{problem}

Determine if the limit approaches a finite number, $\pm\infty$, or does not exist. (If the limit does not exist, write DNE)

\input{Limit-Compute-0014.HELP.tex}

\[\lim_{x\to0}\sage{F}=\answer{\sage{Ans}}\]
\end{problem}}%}
%%%%%%%%%%%%%%%%%%%%%

%%%%%%%%%%%%%%%%%%%%%
%\tagged{Ans@ShortAns, Type@Compute, Topic@Limit, Sub@Trig, File@0015}{
\begin{sagesilent}
a = NonZeroInt(-5,5)
b = NonZeroInt(1,5,[0,a])

p=Integer(randint(0,5))
v=[sin(a*x)/(b*x), sin(a*x)^2/(b*x^2), sin(a*x)^3/(b*x^3), tan(a*x)/(b*x), tan(a*x)^2/(b*x^2), tan(a*x)^3/(b*x^3)]

F=v[p]
Ans=limit(F,x=0)
\end{sagesilent}

\latexProblemContent{
\begin{problem}

Determine if the limit approaches a finite number, $\pm\infty$, or does not exist. (If the limit does not exist, write DNE)

\input{Limit-Compute-0015.HELP.tex}

\[\lim_{x\to0}\sage{F}=\answer{\sage{Ans}}\]
\end{problem}}%}
%%%%%%%%%%%%%%%%%%%%%


%\tagged{Ans@ShortAns, Type@Compute, Topic@Limit, Sub@Rational, Sub@algebraic, File@0016}{

\begin{sagesilent}
v1=[x, x^(1/3), x^2] 
assume(x>0)
pick1 = RandInt(0,2)
f1(x) = v1[pick1](x)

c1 = RandInt(1,6)
c2 = NonZeroInt(-10,10)

p1 = RandInt(2,3)

f2top(x) = expand(c2*(f1(x)^p1 - c1^p1))
f2bot(x) = (f1(x)-c1)
f2(x) = f2top(x)/f2bot(x)

finv = solve(x==f1(y), y)[0].rhs()

Ans1=limit(f2(x), x=finv(c1))
Ans=Ans1.simplify_rational()
\end{sagesilent}

\latexProblemContent{
\begin{problem}
Calculate the following limit:

\input{Limit-Compute-0016.HELP.tex}

$\lim\limits_{x \rightarrow \sage{finv(c1)}} \sage{f2(x)} = \answer{\sage{Ans}}$
\end{problem}}






%\tagged{Ans@ShortAns, Type@Compute, Topic@Limit, Sub@Rational, File@0017}{

\begin{sagesilent}
funcvec=[x, x^(1/3), 1/x, x^2, exp(x), log(x)]
pick1 =RandInt(0,5)
fpick = funcvec[pick1]

c1 = RandInt(1, 10)
c2 = RandInt(1, 10)
c3 = NonZeroInt(-5,5)
c4 = RandInt(-10, 10)
c5 = RandInt(1,10)

f1(x) = fpick(c1*x+c2)
f2(x) = c3*f1(x)+c4

c21 = RandInt(1, 10)
c22 = RandInt(1, 10)
c23 = NonZeroInt(-5,5)
c24 = RandInt(-10, 10)

f3(x) = c23*fpick(c21*x+c22)+c24

if f3(c5)==0:
    f4(x) = f3(x) + 1
else:
    f4(x) = f3(x)

fFinal(x) = f2(x)/f4(x)
ans = fFinal(c5)


\end{sagesilent}

\latexProblemContent{
\begin{problem}

Compute the following limit:

\input{Limit-Compute-0017.HELP.tex}

$\lim\limits_{x \rightarrow \sage{c5}} \sage{fFinal(x)} = \answer{\sage{ans}}$

\end{problem}}

%% Squeeze theorem problem. Currently the answer is always zero, should mix it up a little more.

%\tagged{Ans@ShortAns, Type@Compute, Topic@Limit, Sub@squeeze, File@0018}{

\begin{sagesilent}

funcvec1=[(x), sin(x), cos((pi/2)-(x)), sqrt(x), x^2, x^3]
funcvec2=[1/x, log(abs(x))]
pick1 = RandInt(0,5)
pick2 = RandInt(0,1)

funcvec3=[sin(x),cos(x)]
pick3 = RandInt(0,1)

f1 = funcvec1[pick1]
f2 = funcvec2[pick2]

c1 = RandInt(-10,10)

f3(x) = f2(x-c1)

pick4 = RandInt(0,5)
f4(x) = funcvec3[pick3](f3(x))

f5 = funcvec1[pick4]
fFinal(x) = f5(x-c1)*f4(x)

\end{sagesilent}

\latexProblemContent{
\begin{problem}

Calculate the following limit:

\input{Limit-Compute-0018.HELP.tex}

$\lim\limits_{x\rightarrow\sage{c1}} \sage{fFinal(x)} = \answer{0}$

\end{problem}}


%%Special Limits, sinx/x and tanx/x

%\tagged{Ans@ShortAns, Type@Compute, Topic@Limit, Sub@squeeze, File@0019}{

\begin{sagesilent}
funcvec = [sin(x), tan(x)]
pick1 = RandInt(0,1)
f1(x) = funcvec[pick1](x)
f2(x) = funcvec[1-pick1](x) - 1

c1 = RandInt(1,10)
c2 = RandInt(1,10)
c3 = RandInt(0,1)
f3(x) = f1(c1*x)*(1 + f2(c2*x)*c3)

f4(x) = f3(x)/(x*(1 + (x-1)*c3))

c4 = NonZeroInt(-10,10)
c5 = RandInt(0,1)
fFinal(x) = c4*(f4(x))*(1 + (1/c4^2 - 1)*c5)

ans = c4/(c1*c2)*(1 + (1/(c4)^2 - 1)*c5)

\end{sagesilent}

\latexProblemContent{
\begin{problem}

Calculate the following limit:

\input{Limit-Compute-0019.HELP.tex}

$\lim\limits_{x\rightarrow 0} \sage{fFinal(x)} = \answer{\sage{ans}}$
\end{problem}}

%\tagged{Ans@ShortAns, Type@Compute, Topic@Limit, Sub@Rational, Sub@Indeterminant, File@0020}{

\begin{sagesilent}

cp1 = RandInt(3,7)
c1 = RandInt(-10,10)
c2 = RandInt(-10,10)
c3 = RandInt(-10,10)
c4 = RandInt(-10,10)

cp2 = RandInt(3,7)
c5 = RandInt(-10,10)
c6 = RandInt(-10,10)
c7 = RandInt(-10,10)
c8 = RandInt(-10,10)

f1(x) = c1*x^cp1+c2*x^2+c3*x+c4
f2(x) = c5*x^cp2+c6*x^2+c7*x+c8

tog = (-1)^RandInt(0,1)

fFinal(x) = f1(x)/f2(x)

lim=tog*infinity

ans = limit(fFinal(x),x=lim)

\end{sagesilent}

\latexProblemContent{
\begin{problem}

Compute the following limit:

\input{Limit-Compute-0020.HELP.tex}

$\lim\limits_{x \rightarrow \sage{lim}} \sage{fFinal(x)} = \answer{\sage{ans}}$.
\end{problem}}

%\tagged{Ans@ShortAns, Type@Compute, Topic@Limit, Sub@Rational, Sub@Indeterminant, File@0021}{

\begin{sagesilent}

cp1 = RandInt(3,7)
c1 = RandInt(-10,10)
c2 = RandInt(-10,10)
c3 = RandInt(-10,10)
c4 = RandInt(-10,10)

cprt = RandInt(2,5)
cp2 = RandInt(3,7)*cprt
c5 = RandInt(-10,10)
c6 = RandInt(-10,10)
c7 = RandInt(-10,10)
c8 = RandInt(-10,10)

f1(x) = c1*x^cp1+c2*x^2+c3*x+c4
f2(x) = c5*x^cp2+c6*x^2+c7*x+c8

tog = (-1)^RandInt(0,1)

fCompute(x) = f1(x)/((f2(x))^(1/cprt))
lim=tog*infinity

ans = limit(fCompute(x),x=lim)

\end{sagesilent}

\latexProblemContent{
\begin{problem}

Compute the following limit:

\input{Limit-Compute-0021.HELP.tex}

$\lim\limits_{x \rightarrow \sage{lim}} \dfrac{\sage{f1(x)}}{\sqrt[\sage{cprt}]{\sage{f2(x)}}} = \answer{\sage{ans}}$.
\end{problem}}



%%%%%%%%%%%%%%%%%%%%%%%%%%%%%%%%%%%%%%%%%%%%%%%%%%%%%%%%%%%%%%%%%%%%%%%%%%%%%%%
%
%
%
%
%
%
%
%
%%%%%%%%%%%%%%%%%%%%%%%%%%%%%%%%%%%%%%%%%%%%%%%%%%%%%%%%%%%%%%%%%%%%%%%%%%%%%%%
%%%%%%%%%%%%%%%%%%%%%%%%%%%%%%%%%%%%%%%%%%%%%%%%%%%%%%%%%%%%%%%%%%%%%%%%%%%%%%%
%%%%%%%%%%%%%%%%%%%										%%%%%%%%%%%%%%%%%%%%%%%
%%%%%%%%%%%%%%%%%%%				Concept					%%%%%%%%%%%%%%%%%%%%%%%
%%%%%%%%%%%%%%%%%%%										%%%%%%%%%%%%%%%%%%%%%%%
%%%%%%%%%%%%%%%%%%%%%%%%%%%%%%%%%%%%%%%%%%%%%%%%%%%%%%%%%%%%%%%%%%%%%%%%%%%%%%%
%%%%%%%%%%%%%%%%%%%%%%%%%%%%%%%%%%%%%%%%%%%%%%%%%%%%%%%%%%%%%%%%%%%%%%%%%%%%%%%
%
%
%
%
%
%
%
%
%
%%%%%%%%%%%%%%%%%%%%%%%%%%%%%%%%%%%%%%%%%%%%%%%%%%%%%%%%%%%%%%%%%%%%%%%%%%%%%%%
%
%
%
%
%%%%%%%%%%%%%%%%%%%%%%%%%%%%%%%%%%%%%%%%%%%%%%%%%%%%%%%%%%%%%%%%%%%%%%%%%%%%%%%
%%%%%%%%%%%%%%%%%%%				Calc 1 Concept			%%%%%%%%%%%%%%%%%%%%%%%
%%%%%%%%%%%%%%%%%%%%%%%%%%%%%%%%%%%%%%%%%%%%%%%%%%%%%%%%%%%%%%%%%%%%%%%%%%%%%%%

%%%%%%%%%%%%%%%%%%%%%%%
%%\tagged{Ans@ShortAns, Type@Concept, Topic@Limit, Sub@LimitLaws, File@0001}{
\begin{sagesilent}
a = NonZeroInt(-5,5)
b = NonZeroInt(-5,5)
c = NonZeroInt(-5,5)

f=x+a
d=b*c
\end{sagesilent}

\latexProblemContent{
\begin{problem}

If you know that $\lim\limits_{x\to\sage{a}}f(x)=\sage{b}$ and $\lim\limits_{x\to0}g(x)=\sage{c}$, then evaluate the following limit:

\input{Limit-Concept-0001.HELP.tex}

\[\lim_{x\to0}f(\sage{f})g(x)=\answer{\sage{d}}\]
\end{problem}}%}
%%%%%%%%%%%%%%%%%%%%%%

%%%%%%%%%%%%%%%%%%%%%%%
%%\tagged{Ans@ShortAns, Type@Concept, Topic@Limit, Sub@LimitLaws, File@0002}{
\begin{sagesilent}
a = NonZeroInt(-5,5)
b = NonZeroInt(-5,5)
c = NonZeroInt(-5,5)

f=x+a
d=b+c
\end{sagesilent}

\latexProblemContent{
\begin{problem}

If you know that $\lim\limits_{x\to\sage{a}}f(x)=\sage{b}$ and $\lim\limits_{x\to0}g(x)=\sage{c}$, then evaluate the following limit:

\input{Limit-Concept-0002.HELP.tex}

\[\lim_{x\to0}f(\sage{f})+g(x)=\answer{\sage{d}}\]
\end{problem}}%}
%%%%%%%%%%%%%%%%%%%%%%

%%%%%%%%%%%%%%%%%%%%%%%
%%\tagged{Ans@MultiAns, Type@Concept, Topic@Limit, Sub@LimitLaws, File@0003}{
\begin{sagesilent}
a = NonZeroInt(-5,5)
b = NonZeroInt(-5,5)
c = NonZeroInt(-5,5)

f=x+a
g=x-a
d=b*c
d2=b+c
d3=c-b
d4=b/c

\end{sagesilent}

\latexProblemContent{
\begin{problem}

If you know that $\lim\limits_{x\to\sage{a}}f(x)=\sage{b}$ and $\lim\limits_{x\to0}g(x)=\sage{c}$, then evaluate the following limits:

\input{Limit-Concept-0003.HELP.tex}


\[\lim_{x\to0}f(\sage{f})g(x)=\answer{\sage{d}}\]

\[\lim_{x\to0}f(\sage{f})+g(x)=\answer{\sage{d2}}\]

\[\lim_{x\to0}g(x)-f(\sage{f})=\answer{\sage{d3}}\]

\[\lim_{x\to\sage{a}}\dfrac{f(x)}{g(\sage{g})}=\answer{\sage{d4}}\]
\end{problem}}%}
%%%%%%%%%%%%%%%%%%%%%%

%%%%%%%%%%%%%%%%%%%%%%%
%%\tagged{Ans@MultiAns, Type@Concept, Topic@Limit, Sub@Continuity, Sub@Disc-Jump, Sub@Piecewise, File@0004}{
\begin{sagesilent}
a = Integer(randint(-3,3))
b = NonZeroInt(a+2,a+5)

c = RandInt(-5,5)
c2 = RandInt(-5,5)
c3 = RandInt(-5,5)

func1=[x-c, c-x, (x-c)^2, expand((x-c)*(x-c2))]
func2=[x, x^2-c3*x, x^2+c3*x, (x-c3)^2]

p1 = RandInt(0,3)
p2 = RandInt(0,3)
p3 = RandInt(0,3)

f1 = func1[p1]
F2 = func2[p2]
F3 = func2[p3]

A = f1(a) - F3(a) + 1
f3 = F3 + A
B = f1(b)- F2(b)
f2 = F2 + B

\end{sagesilent}

\latexProblemContent{
\begin{problem}

Let $f(x)=\left\{\begin{array}{clc}
\sage{f3} & , & x\leq \sage{a}\\
\sage{f1} & , & \sage{a}<x\leq\sage{b}\\
\sage{f2} & , & x>\sage{b}
\end{array}\right.$.  

Find the numbers at which $f$ is discontinuous: $x=\answer{\sage{a}}$\\ \qquad (If no such numbers exist, enter ``None'')

\input{Limit-Concept-0004.HELP.tex}

At which of these points of discontinuity is $f$ continuous from the right?\\  $x=\answer{None}$ \qquad (If no such numbers exist, enter ``None'')

At which of these points of discontinuity is $f$ continuous from the left?\\  $x=\answer{\sage{a}}$ \qquad (If no such numbers exist, enter ``None'')
\end{problem}}%}
%%%%%%%%%%%%%%%%%%%%%%



%%%%%%%%%%%%%%%%%%%%%%%
%%\tagged{Ans@MultiAns, Type@Concept, Topic@Limit, Sub@Continuity, Sub@Disc-Jump, Sub@Piecewise, File@0005}{
\begin{sagesilent}
a = Integer(randint(-3,3))
b = NonZeroInt(a+2,a+5)

c = RandInt(-6,6)
d = RandInt(-6,6)

intp1 = RandInt(0,3)
intp2 = NonZeroInt(0,3,[intp1])
nonintp=RandInt(0,3)

integerfunc=[x-c, (x-c)^2, expand((x-c)*(x-d)), c-x]
nonintfunc=[e^x, e^(x-b), sin(x), cos(x)]

f1 = integerfunc[intp1]
f2 = nonintfunc[nonintp]
f3 = integerfunc[intp2]


\end{sagesilent}

\latexProblemContent{
\begin{problem}

Let $f(x)=\left\{\begin{array}{clc}
\sage{f1} & , & x< \sage{a}\\
\sage{f2} & , & \sage{a}\leq x\leq\sage{b}\\
\sage{f3} & , & x>\sage{b}
\end{array}\right.$.  

Find the numbers at which $f$ is discontinuous: $x=\answer{\sage{a},\sage{b}}$\\ \qquad (If no such numbers exist, enter ``None'')

\input{Limit-Concept-0005.HELP.tex}

At which of these points of discontinuity is $f$ continuous from the right?\\  $x=\answer{\sage{a}}$ \qquad (If no such numbers exist, enter ``None'')

At which of these points of discontinuity is $f$ continuous from the left?\\  $x=\answer{\sage{b}}$ \qquad (If no such numbers exist, enter ``None'')

\end{problem}}%}
%%%%%%%%%%%%%%%%%%%%%%


%%%%%%%%%%%%%%%%%%%%%%%
%%\tagged{Ans@MC, Type@Concept, Topic@Limit, Sub@Continuity, Sub@Theorems-IVT, Sub@Poly, File@0006}{
\begin{sagesilent}
a = Integer(randint(-3,3))
b = NonZeroInt(a+2,a+5)
left=a+1
right=b-1
while left==right:
   b = NonZeroInt(a+2,a+5)
   right=b-1

c = b+1
d=a-1

m=NonZeroInt(a+1,b-1)
p=Integer(randint(0,2))
v=[expand((x-m)*(x-c)), expand((x-m)*(x-c)*(x-d)), expand((x-m)*(x-c)*(x-d)^2)]

F=v[p]
\end{sagesilent}

\latexProblemContent{
\begin{problem}

Does $\sage{F}$ have a root in the interval $(\sage{a},\sage{b})$?  

\input{Limit-Concept-0006.HELP.tex}

\begin{multipleChoice}
\choice[correct]{Yes}
\choice{No}
\choice{Inconclusive}
\end{multipleChoice}

What is the reason for this answer?\begin{multipleChoice}
\choice{$f(\sage{a})<0$ and $f(\sage{b})>0$, so $f$ has a root by IVT.}
\choice[correct]{$f(\sage{a})>0$ and $f(\sage{b})<0$, so $f$ has a root by IVT.}
\choice{$f(\sage{a})<0$ and $f(\sage{b})<0$, so $f$ does not have a root by IVT.}
\choice{$f(\sage{a})>0$ and $f(\sage{b})>0$, so $f$ does not have a root by IVT.}
\choice{The answer could not be determined.}
\end{multipleChoice}
\end{problem}}%}
%%%%%%%%%%%%%%%%%%%%%%

%%%%%%%%%%%%%%%%%%%%%%%
%%\tagged{Ans@MC, Type@Concept, Topic@Limit, Sub@Continuity, Sub@Theorems-IVT, Sub@Poly, Sub@Trig, File@0007}{
\begin{sagesilent}
a = Integer(randint(-3,3))
p=Integer(randint(0,1))

trig=[sin(x-a), cos(x-a)]
poly=[a+1-x, x-a]

F=trig[p]
G=poly[p]
\end{sagesilent}

\latexProblemContent{
\begin{problem}

Does the equation $\sage{F}=\sage{G}$ have a solution in the interval $(\sage{a},\sage{a+1})$?  

\input{Limit-Concept-0007.HELP.tex}

\begin{multipleChoice}
\choice[correct]{Yes}
\choice{No}
\choice{Inconclusive}
\end{multipleChoice}
\end{problem}}%}
%%%%%%%%%%%%%%%%%%%%%%

%% SO FAR 13 Compute, 7 Concept


%%%%%%%%%%%%%%%%%%%%%
%\tagged{Ans@MultiAns, Type@Concept, Topic@Limit, Sub@Trig, Sub@Squeeze, File@0008}{
\begin{sagesilent}
a = NonZeroInt(-25,25)
b = RandInt(-15,15)
p=Integer(randint(0,5))
v=[(x-b)*cos(a/(x-b)), (x-b)^2*cos(a/(x-b)), (x-b)^3*cos(a/(x-b)), (x-b)*cos(a/(x-b)^2), (x-b)^2*cos(a/(x-b)^2), (x-b)^3*cos(a/(x-b)^2)]
F=v[p]

Ans=limit(F,x=0)
\end{sagesilent}

\latexProblemContent{
\begin{problem}

The limit as $x\to\sage{b}$ of $f(x)=\sage{F}$ is $0$.  What is the reason why this is true?

\input{Limit-Concept-0008.HELP.tex}

\begin{multipleChoice}
\choice{The statement is in fact false: $\lim\limits_{x\to\sage{b}}\sage{F}\neq0$.}
\choice{The cosine factor decreases to $0$ faster than the polynomial.}
\choice[correct]{The cosine factor is bounded between $-1$ and $1$, so the polynomial forces the function to $0$.}
\choice{The cosine factor directly cancels out the polynomial factor.}
\end{multipleChoice}


What is the name of the theorem that applies to this problem? \qquad \\
The $\underline{\answer{Squeeze}}$ Theorem
\end{problem}}%}
%%%%%%%%%%%%%%%%%%%%%


